
\documentclass{article} % For LaTeX2e
\usepackage[dvipsnames, table, xcdraw]{xcolor}
\usepackage{iclr2025_conference,times}

% Optional math commands from https://github.com/goodfeli/dlbook_notation.
% %%%%% NEW MATH DEFINITIONS %%%%%

\usepackage{amsmath,amsfonts,bm}
\usepackage{derivative}
% Mark sections of captions for referring to divisions of figures
\newcommand{\figleft}{{\em (Left)}}
\newcommand{\figcenter}{{\em (Center)}}
\newcommand{\figright}{{\em (Right)}}
\newcommand{\figtop}{{\em (Top)}}
\newcommand{\figbottom}{{\em (Bottom)}}
\newcommand{\captiona}{{\em (a)}}
\newcommand{\captionb}{{\em (b)}}
\newcommand{\captionc}{{\em (c)}}
\newcommand{\captiond}{{\em (d)}}

% Highlight a newly defined term
\newcommand{\newterm}[1]{{\bf #1}}

% Derivative d 
\newcommand{\deriv}{{\mathrm{d}}}

% Figure reference, lower-case.
\def\figref#1{figure~\ref{#1}}
% Figure reference, capital. For start of sentence
\def\Figref#1{Figure~\ref{#1}}
\def\twofigref#1#2{figures \ref{#1} and \ref{#2}}
\def\quadfigref#1#2#3#4{figures \ref{#1}, \ref{#2}, \ref{#3} and \ref{#4}}
% Section reference, lower-case.
\def\secref#1{section~\ref{#1}}
% Section reference, capital.
\def\Secref#1{Section~\ref{#1}}
% Reference to two sections.
\def\twosecrefs#1#2{sections \ref{#1} and \ref{#2}}
% Reference to three sections.
\def\secrefs#1#2#3{sections \ref{#1}, \ref{#2} and \ref{#3}}
% Reference to an equation, lower-case.
\def\eqref#1{equation~\ref{#1}}
% Reference to an equation, upper case
\def\Eqref#1{Equation~\ref{#1}}
% A raw reference to an equation---avoid using if possible
\def\plaineqref#1{\ref{#1}}
% Reference to a chapter, lower-case.
\def\chapref#1{chapter~\ref{#1}}
% Reference to an equation, upper case.
\def\Chapref#1{Chapter~\ref{#1}}
% Reference to a range of chapters
\def\rangechapref#1#2{chapters\ref{#1}--\ref{#2}}
% Reference to an algorithm, lower-case.
\def\algref#1{algorithm~\ref{#1}}
% Reference to an algorithm, upper case.
\def\Algref#1{Algorithm~\ref{#1}}
\def\twoalgref#1#2{algorithms \ref{#1} and \ref{#2}}
\def\Twoalgref#1#2{Algorithms \ref{#1} and \ref{#2}}
% Reference to a part, lower case
\def\partref#1{part~\ref{#1}}
% Reference to a part, upper case
\def\Partref#1{Part~\ref{#1}}
\def\twopartref#1#2{parts \ref{#1} and \ref{#2}}

\def\ceil#1{\lceil #1 \rceil}
\def\floor#1{\lfloor #1 \rfloor}
\def\1{\bm{1}}
\newcommand{\train}{\mathcal{D}}
\newcommand{\valid}{\mathcal{D_{\mathrm{valid}}}}
\newcommand{\test}{\mathcal{D_{\mathrm{test}}}}

\def\eps{{\epsilon}}


% Random variables
\def\reta{{\textnormal{$\eta$}}}
\def\ra{{\textnormal{a}}}
\def\rb{{\textnormal{b}}}
\def\rc{{\textnormal{c}}}
\def\rd{{\textnormal{d}}}
\def\re{{\textnormal{e}}}
\def\rf{{\textnormal{f}}}
\def\rg{{\textnormal{g}}}
\def\rh{{\textnormal{h}}}
\def\ri{{\textnormal{i}}}
\def\rj{{\textnormal{j}}}
\def\rk{{\textnormal{k}}}
\def\rl{{\textnormal{l}}}
% rm is already a command, just don't name any random variables m
\def\rn{{\textnormal{n}}}
\def\ro{{\textnormal{o}}}
\def\rp{{\textnormal{p}}}
\def\rq{{\textnormal{q}}}
\def\rr{{\textnormal{r}}}
\def\rs{{\textnormal{s}}}
\def\rt{{\textnormal{t}}}
\def\ru{{\textnormal{u}}}
\def\rv{{\textnormal{v}}}
\def\rw{{\textnormal{w}}}
\def\rx{{\textnormal{x}}}
\def\ry{{\textnormal{y}}}
\def\rz{{\textnormal{z}}}

% Random vectors
\def\rvepsilon{{\mathbf{\epsilon}}}
\def\rvphi{{\mathbf{\phi}}}
\def\rvtheta{{\mathbf{\theta}}}
\def\rva{{\mathbf{a}}}
\def\rvb{{\mathbf{b}}}
\def\rvc{{\mathbf{c}}}
\def\rvd{{\mathbf{d}}}
\def\rve{{\mathbf{e}}}
\def\rvf{{\mathbf{f}}}
\def\rvg{{\mathbf{g}}}
\def\rvh{{\mathbf{h}}}
\def\rvu{{\mathbf{i}}}
\def\rvj{{\mathbf{j}}}
\def\rvk{{\mathbf{k}}}
\def\rvl{{\mathbf{l}}}
\def\rvm{{\mathbf{m}}}
\def\rvn{{\mathbf{n}}}
\def\rvo{{\mathbf{o}}}
\def\rvp{{\mathbf{p}}}
\def\rvq{{\mathbf{q}}}
\def\rvr{{\mathbf{r}}}
\def\rvs{{\mathbf{s}}}
\def\rvt{{\mathbf{t}}}
\def\rvu{{\mathbf{u}}}
\def\rvv{{\mathbf{v}}}
\def\rvw{{\mathbf{w}}}
\def\rvx{{\mathbf{x}}}
\def\rvy{{\mathbf{y}}}
\def\rvz{{\mathbf{z}}}

% Elements of random vectors
\def\erva{{\textnormal{a}}}
\def\ervb{{\textnormal{b}}}
\def\ervc{{\textnormal{c}}}
\def\ervd{{\textnormal{d}}}
\def\erve{{\textnormal{e}}}
\def\ervf{{\textnormal{f}}}
\def\ervg{{\textnormal{g}}}
\def\ervh{{\textnormal{h}}}
\def\ervi{{\textnormal{i}}}
\def\ervj{{\textnormal{j}}}
\def\ervk{{\textnormal{k}}}
\def\ervl{{\textnormal{l}}}
\def\ervm{{\textnormal{m}}}
\def\ervn{{\textnormal{n}}}
\def\ervo{{\textnormal{o}}}
\def\ervp{{\textnormal{p}}}
\def\ervq{{\textnormal{q}}}
\def\ervr{{\textnormal{r}}}
\def\ervs{{\textnormal{s}}}
\def\ervt{{\textnormal{t}}}
\def\ervu{{\textnormal{u}}}
\def\ervv{{\textnormal{v}}}
\def\ervw{{\textnormal{w}}}
\def\ervx{{\textnormal{x}}}
\def\ervy{{\textnormal{y}}}
\def\ervz{{\textnormal{z}}}

% Random matrices
\def\rmA{{\mathbf{A}}}
\def\rmB{{\mathbf{B}}}
\def\rmC{{\mathbf{C}}}
\def\rmD{{\mathbf{D}}}
\def\rmE{{\mathbf{E}}}
\def\rmF{{\mathbf{F}}}
\def\rmG{{\mathbf{G}}}
\def\rmH{{\mathbf{H}}}
\def\rmI{{\mathbf{I}}}
\def\rmJ{{\mathbf{J}}}
\def\rmK{{\mathbf{K}}}
\def\rmL{{\mathbf{L}}}
\def\rmM{{\mathbf{M}}}
\def\rmN{{\mathbf{N}}}
\def\rmO{{\mathbf{O}}}
\def\rmP{{\mathbf{P}}}
\def\rmQ{{\mathbf{Q}}}
\def\rmR{{\mathbf{R}}}
\def\rmS{{\mathbf{S}}}
\def\rmT{{\mathbf{T}}}
\def\rmU{{\mathbf{U}}}
\def\rmV{{\mathbf{V}}}
\def\rmW{{\mathbf{W}}}
\def\rmX{{\mathbf{X}}}
\def\rmY{{\mathbf{Y}}}
\def\rmZ{{\mathbf{Z}}}

% Elements of random matrices
\def\ermA{{\textnormal{A}}}
\def\ermB{{\textnormal{B}}}
\def\ermC{{\textnormal{C}}}
\def\ermD{{\textnormal{D}}}
\def\ermE{{\textnormal{E}}}
\def\ermF{{\textnormal{F}}}
\def\ermG{{\textnormal{G}}}
\def\ermH{{\textnormal{H}}}
\def\ermI{{\textnormal{I}}}
\def\ermJ{{\textnormal{J}}}
\def\ermK{{\textnormal{K}}}
\def\ermL{{\textnormal{L}}}
\def\ermM{{\textnormal{M}}}
\def\ermN{{\textnormal{N}}}
\def\ermO{{\textnormal{O}}}
\def\ermP{{\textnormal{P}}}
\def\ermQ{{\textnormal{Q}}}
\def\ermR{{\textnormal{R}}}
\def\ermS{{\textnormal{S}}}
\def\ermT{{\textnormal{T}}}
\def\ermU{{\textnormal{U}}}
\def\ermV{{\textnormal{V}}}
\def\ermW{{\textnormal{W}}}
\def\ermX{{\textnormal{X}}}
\def\ermY{{\textnormal{Y}}}
\def\ermZ{{\textnormal{Z}}}

% Vectors
\def\vzero{{\bm{0}}}
\def\vone{{\bm{1}}}
\def\vmu{{\bm{\mu}}}
\def\vtheta{{\bm{\theta}}}
\def\vphi{{\bm{\phi}}}
\def\va{{\bm{a}}}
\def\vb{{\bm{b}}}
\def\vc{{\bm{c}}}
\def\vd{{\bm{d}}}
\def\ve{{\bm{e}}}
\def\vf{{\bm{f}}}
\def\vg{{\bm{g}}}
\def\vh{{\bm{h}}}
\def\vi{{\bm{i}}}
\def\vj{{\bm{j}}}
\def\vk{{\bm{k}}}
\def\vl{{\bm{l}}}
\def\vm{{\bm{m}}}
\def\vn{{\bm{n}}}
\def\vo{{\bm{o}}}
\def\vp{{\bm{p}}}
\def\vq{{\bm{q}}}
\def\vr{{\bm{r}}}
\def\vs{{\bm{s}}}
\def\vt{{\bm{t}}}
\def\vu{{\bm{u}}}
\def\vv{{\bm{v}}}
\def\vw{{\bm{w}}}
\def\vx{{\bm{x}}}
\def\vy{{\bm{y}}}
\def\vz{{\bm{z}}}

% Elements of vectors
\def\evalpha{{\alpha}}
\def\evbeta{{\beta}}
\def\evepsilon{{\epsilon}}
\def\evlambda{{\lambda}}
\def\evomega{{\omega}}
\def\evmu{{\mu}}
\def\evpsi{{\psi}}
\def\evsigma{{\sigma}}
\def\evtheta{{\theta}}
\def\eva{{a}}
\def\evb{{b}}
\def\evc{{c}}
\def\evd{{d}}
\def\eve{{e}}
\def\evf{{f}}
\def\evg{{g}}
\def\evh{{h}}
\def\evi{{i}}
\def\evj{{j}}
\def\evk{{k}}
\def\evl{{l}}
\def\evm{{m}}
\def\evn{{n}}
\def\evo{{o}}
\def\evp{{p}}
\def\evq{{q}}
\def\evr{{r}}
\def\evs{{s}}
\def\evt{{t}}
\def\evu{{u}}
\def\evv{{v}}
\def\evw{{w}}
\def\evx{{x}}
\def\evy{{y}}
\def\evz{{z}}

% Matrix
\def\mA{{\bm{A}}}
\def\mB{{\bm{B}}}
\def\mC{{\bm{C}}}
\def\mD{{\bm{D}}}
\def\mE{{\bm{E}}}
\def\mF{{\bm{F}}}
\def\mG{{\bm{G}}}
\def\mH{{\bm{H}}}
\def\mI{{\bm{I}}}
\def\mJ{{\bm{J}}}
\def\mK{{\bm{K}}}
\def\mL{{\bm{L}}}
\def\mM{{\bm{M}}}
\def\mN{{\bm{N}}}
\def\mO{{\bm{O}}}
\def\mP{{\bm{P}}}
\def\mQ{{\bm{Q}}}
\def\mR{{\bm{R}}}
\def\mS{{\bm{S}}}
\def\mT{{\bm{T}}}
\def\mU{{\bm{U}}}
\def\mV{{\bm{V}}}
\def\mW{{\bm{W}}}
\def\mX{{\bm{X}}}
\def\mY{{\bm{Y}}}
\def\mZ{{\bm{Z}}}
\def\mBeta{{\bm{\beta}}}
\def\mPhi{{\bm{\Phi}}}
\def\mLambda{{\bm{\Lambda}}}
\def\mSigma{{\bm{\Sigma}}}

% Tensor
\DeclareMathAlphabet{\mathsfit}{\encodingdefault}{\sfdefault}{m}{sl}
\SetMathAlphabet{\mathsfit}{bold}{\encodingdefault}{\sfdefault}{bx}{n}
\newcommand{\tens}[1]{\bm{\mathsfit{#1}}}
\def\tA{{\tens{A}}}
\def\tB{{\tens{B}}}
\def\tC{{\tens{C}}}
\def\tD{{\tens{D}}}
\def\tE{{\tens{E}}}
\def\tF{{\tens{F}}}
\def\tG{{\tens{G}}}
\def\tH{{\tens{H}}}
\def\tI{{\tens{I}}}
\def\tJ{{\tens{J}}}
\def\tK{{\tens{K}}}
\def\tL{{\tens{L}}}
\def\tM{{\tens{M}}}
\def\tN{{\tens{N}}}
\def\tO{{\tens{O}}}
\def\tP{{\tens{P}}}
\def\tQ{{\tens{Q}}}
\def\tR{{\tens{R}}}
\def\tS{{\tens{S}}}
\def\tT{{\tens{T}}}
\def\tU{{\tens{U}}}
\def\tV{{\tens{V}}}
\def\tW{{\tens{W}}}
\def\tX{{\tens{X}}}
\def\tY{{\tens{Y}}}
\def\tZ{{\tens{Z}}}


% Graph
\def\gA{{\mathcal{A}}}
\def\gB{{\mathcal{B}}}
\def\gC{{\mathcal{C}}}
\def\gD{{\mathcal{D}}}
\def\gE{{\mathcal{E}}}
\def\gF{{\mathcal{F}}}
\def\gG{{\mathcal{G}}}
\def\gH{{\mathcal{H}}}
\def\gI{{\mathcal{I}}}
\def\gJ{{\mathcal{J}}}
\def\gK{{\mathcal{K}}}
\def\gL{{\mathcal{L}}}
\def\gM{{\mathcal{M}}}
\def\gN{{\mathcal{N}}}
\def\gO{{\mathcal{O}}}
\def\gP{{\mathcal{P}}}
\def\gQ{{\mathcal{Q}}}
\def\gR{{\mathcal{R}}}
\def\gS{{\mathcal{S}}}
\def\gT{{\mathcal{T}}}
\def\gU{{\mathcal{U}}}
\def\gV{{\mathcal{V}}}
\def\gW{{\mathcal{W}}}
\def\gX{{\mathcal{X}}}
\def\gY{{\mathcal{Y}}}
\def\gZ{{\mathcal{Z}}}

% Sets
\def\sA{{\mathbb{A}}}
\def\sB{{\mathbb{B}}}
\def\sC{{\mathbb{C}}}
\def\sD{{\mathbb{D}}}
% Don't use a set called E, because this would be the same as our symbol
% for expectation.
\def\sF{{\mathbb{F}}}
\def\sG{{\mathbb{G}}}
\def\sH{{\mathbb{H}}}
\def\sI{{\mathbb{I}}}
\def\sJ{{\mathbb{J}}}
\def\sK{{\mathbb{K}}}
\def\sL{{\mathbb{L}}}
\def\sM{{\mathbb{M}}}
\def\sN{{\mathbb{N}}}
\def\sO{{\mathbb{O}}}
\def\sP{{\mathbb{P}}}
\def\sQ{{\mathbb{Q}}}
\def\sR{{\mathbb{R}}}
\def\sS{{\mathbb{S}}}
\def\sT{{\mathbb{T}}}
\def\sU{{\mathbb{U}}}
\def\sV{{\mathbb{V}}}
\def\sW{{\mathbb{W}}}
\def\sX{{\mathbb{X}}}
\def\sY{{\mathbb{Y}}}
\def\sZ{{\mathbb{Z}}}

% Entries of a matrix
\def\emLambda{{\Lambda}}
\def\emA{{A}}
\def\emB{{B}}
\def\emC{{C}}
\def\emD{{D}}
\def\emE{{E}}
\def\emF{{F}}
\def\emG{{G}}
\def\emH{{H}}
\def\emI{{I}}
\def\emJ{{J}}
\def\emK{{K}}
\def\emL{{L}}
\def\emM{{M}}
\def\emN{{N}}
\def\emO{{O}}
\def\emP{{P}}
\def\emQ{{Q}}
\def\emR{{R}}
\def\emS{{S}}
\def\emT{{T}}
\def\emU{{U}}
\def\emV{{V}}
\def\emW{{W}}
\def\emX{{X}}
\def\emY{{Y}}
\def\emZ{{Z}}
\def\emSigma{{\Sigma}}

% entries of a tensor
% Same font as tensor, without \bm wrapper
\newcommand{\etens}[1]{\mathsfit{#1}}
\def\etLambda{{\etens{\Lambda}}}
\def\etA{{\etens{A}}}
\def\etB{{\etens{B}}}
\def\etC{{\etens{C}}}
\def\etD{{\etens{D}}}
\def\etE{{\etens{E}}}
\def\etF{{\etens{F}}}
\def\etG{{\etens{G}}}
\def\etH{{\etens{H}}}
\def\etI{{\etens{I}}}
\def\etJ{{\etens{J}}}
\def\etK{{\etens{K}}}
\def\etL{{\etens{L}}}
\def\etM{{\etens{M}}}
\def\etN{{\etens{N}}}
\def\etO{{\etens{O}}}
\def\etP{{\etens{P}}}
\def\etQ{{\etens{Q}}}
\def\etR{{\etens{R}}}
\def\etS{{\etens{S}}}
\def\etT{{\etens{T}}}
\def\etU{{\etens{U}}}
\def\etV{{\etens{V}}}
\def\etW{{\etens{W}}}
\def\etX{{\etens{X}}}
\def\etY{{\etens{Y}}}
\def\etZ{{\etens{Z}}}

% The true underlying data generating distribution
\newcommand{\pdata}{p_{\rm{data}}}
\newcommand{\ptarget}{p_{\rm{target}}}
\newcommand{\pprior}{p_{\rm{prior}}}
\newcommand{\pbase}{p_{\rm{base}}}
\newcommand{\pref}{p_{\rm{ref}}}

% The empirical distribution defined by the training set
\newcommand{\ptrain}{\hat{p}_{\rm{data}}}
\newcommand{\Ptrain}{\hat{P}_{\rm{data}}}
% The model distribution
\newcommand{\pmodel}{p_{\rm{model}}}
\newcommand{\Pmodel}{P_{\rm{model}}}
\newcommand{\ptildemodel}{\tilde{p}_{\rm{model}}}
% Stochastic autoencoder distributions
\newcommand{\pencode}{p_{\rm{encoder}}}
\newcommand{\pdecode}{p_{\rm{decoder}}}
\newcommand{\precons}{p_{\rm{reconstruct}}}

\newcommand{\laplace}{\mathrm{Laplace}} % Laplace distribution

\newcommand{\E}{\mathbb{E}}
\newcommand{\Ls}{\mathcal{L}}
\newcommand{\R}{\mathbb{R}}
\newcommand{\emp}{\tilde{p}}
\newcommand{\lr}{\alpha}
\newcommand{\reg}{\lambda}
\newcommand{\rect}{\mathrm{rectifier}}
\newcommand{\softmax}{\mathrm{softmax}}
\newcommand{\sigmoid}{\sigma}
\newcommand{\softplus}{\zeta}
\newcommand{\KL}{D_{\mathrm{KL}}}
\newcommand{\Var}{\mathrm{Var}}
\newcommand{\standarderror}{\mathrm{SE}}
\newcommand{\Cov}{\mathrm{Cov}}
% Wolfram Mathworld says $L^2$ is for function spaces and $\ell^2$ is for vectors
% But then they seem to use $L^2$ for vectors throughout the site, and so does
% wikipedia.
\newcommand{\normlzero}{L^0}
\newcommand{\normlone}{L^1}
\newcommand{\normltwo}{L^2}
\newcommand{\normlp}{L^p}
\newcommand{\normmax}{L^\infty}

\newcommand{\parents}{Pa} % See usage in notation.tex. Chosen to match Daphne's book.

\DeclareMathOperator*{\argmax}{arg\,max}
\DeclareMathOperator*{\argmin}{arg\,min}

\DeclareMathOperator{\sign}{sign}
\DeclareMathOperator{\Tr}{Tr}
\let\ab\allowbreak


% \usepackage{hyperref}
\usepackage[hidelinks, breaklinks, colorlinks, linkcolor=Red, citecolor=RoyalBlue]{hyperref}
\usepackage{url}

\usepackage{multirow}
\usepackage{makecell}
\usepackage{booktabs}
\usepackage{graphicx}
\usepackage{algpseudocode}
\usepackage{eqparbox}
\usepackage{algorithmicx}
\usepackage{algorithm}
\usepackage{array}
\usepackage{pifont}
\usepackage{wrapfig}
\usepackage{amsmath}
\usepackage{amsfonts}
\usepackage{colortbl}
\usepackage{fontenc}
\usepackage{cleveref}
\usepackage{tikz}
\usepackage{xspace}
%\usepackage{caption}

\newcommand{\figref}[1]{Fig.~\ref{#1}}
\newcommand{\tabref}[1]{Tab.~\ref{#1}}
\newcommand{\Figref}[1]{Figure~\ref{#1}}
\newcommand{\Tabref}[1]{Table~\ref{#1}}
\newcommand{\eqnref}[1]{Eq.~(\ref{#1})}
\newcommand{\secref}[1]{Sec.~\ref{#1}}
\newcommand{\Secref}[1]{Section~\ref{#1}}
\newcommand{\Appref}[1]{Appendix~\ref{#1}}

\definecolor{baselinecolor}{rgb}{0.9, 0.9, 1.}
\definecolor{graycolor}{gray}{0.9}
\definecolor{Green}{rgb}{0.0, 0.5, 0.0}
\definecolor{Green}{rgb}{0.0, 0.5, 0.0}
\definecolor{rebuttal}{rgb}{0.6, 0.6, 1.}
\newcommand{\baseline}[1]{\cellcolor{baselinecolor}{#1}}
\newcommand{\graybase}[1]{\cellcolor{graycolor}{#1}}


\newcommand{\plus}[1]{\scriptsize\bf\textcolor{Green}{#1}}
\newcommand{\tinyplus}[1]{\tiny\bf\textcolor{Green}{#1}}

\newcommand{\cmark}{\ding{51}}%
\newcommand{\xmark}{\ding{55}}%

\makeatletter
\DeclareRobustCommand\onedot{\futurelet\@let@token\@onedot}
\def\@onedot{\ifx\@let@token.\else.\null\fi\xspace}
\def\eg{\emph{e.g}\onedot} \def\Eg{\emph{E.g}\onedot}
\def\ie{\emph{i.e}\onedot} \def\Ie{\emph{I.e}\onedot}
\def\cf{\emph{c.f}\onedot} \def\Cf{\emph{C.f}\onedot}
\def\etc{\emph{etc}\onedot} \def\vs{\emph{vs}\onedot}
\def\wrt{w.r.t\onedot} \def\dof{d.o.f\onedot}
\def\etal{\emph{et al}\onedot}
\makeatother

% \title{Formatting Instructions for ICLR 2025 \\ Conference Submissions}
\title{UNIP: Rethinking Pre-trained Attention Patterns for Infrared Semantic Segmentation}
% Authors must not appear in the submitted version. They should be hidden
% as long as the \iclrfinalcopy macro remains commented out below.
% Non-anonymous submissions will be rejected without review.

% \author{Antiquus S.~Hippocampus, Natalia Cerebro \& Amelie P. Amygdale \thanks{ Use footnote for providing further information
% about author (webpage, alternative address)---\emph{not} for acknowledging
% funding agencies.  Funding acknowledgements go at the end of the paper.} \\
% Department of Computer Science\\
% Cranberry-Lemon University\\
% Pittsburgh, PA 15213, USA \\
% \texttt{\{hippo,brain,jen\}@cs.cranberry-lemon.edu} \\
% \And
% Ji Q. Ren \& Yevgeny LeNet \\
% Department of Computational Neuroscience \\
% University of the Witwatersrand \\
% Joburg, South Africa \\
% \texttt{\{robot,net\}@wits.ac.za} \\
% \AND
% Coauthor \\
% Affiliation \\
% Address \\
% \texttt{email}
% }

\author{Tao Zhang\textsuperscript{1,2} \ \
Jinyong Wen\textsuperscript{1,2} \ \
Zhen Chen\textsuperscript{3} \ \
Kun Ding\textsuperscript{1}\thanks{Corresponding author} \ \
Shiming Xiang\textsuperscript{1,2} \ \
Chunhong Pan\textsuperscript{1} \\
\textsuperscript{1}MAIS, Institute of Automation, Chinese Academy of Sciences, China \\
\textsuperscript{2}School of Artificial Intelligence, University of Chinese Academy of Sciences, China \\
\textsuperscript{3}School of Automation Science and Electrical Engineering, Beihang University, China \\ 
\texttt{\{zhangtao2021,wenjinyong2019,kun.ding\}@ia.ac.cn;} \\ \texttt{\{czhen\}@buaa.edu.cn;\{smxiang, chpan\}@nlpr.ia.ac.cn}\\
}

% The \author macro works with any number of authors. There are two commands
% used to separate the names and addresses of multiple authors: \And and \AND.
%
% Using \And between authors leaves it to \LaTeX{} to determine where to break
% the lines. Using \AND forces a linebreak at that point. So, if \LaTeX{}
% puts 3 of 4 authors names on the first line, and the last on the second
% line, try using \AND instead of \And before the third author name.

\newcommand{\fix}{\marginpar{FIX}}
\newcommand{\new}{\marginpar{NEW}}


\iclrfinalcopy % Uncomment for camera-ready version, but NOT for submission.
\begin{document}


\maketitle


\begin{abstract}
Pre-training techniques significantly enhance the performance of semantic segmentation tasks with limited training data. However, the efficacy under a large domain gap between pre-training (\eg RGB) and fine-tuning (\eg infrared) remains underexplored. In this study, we first benchmark the infrared semantic segmentation performance of various pre-training methods and reveal several phenomena distinct from the RGB domain. Next, our layerwise analysis of pre-trained attention maps uncovers that: (1) There are three typical attention patterns (local, hybrid, and global); (2) Pre-training tasks notably influence the pattern distribution across layers; (3) The hybrid pattern is crucial for semantic segmentation as it attends to both nearby and foreground elements; (4) The texture bias impedes model generalization in infrared tasks. Building on these insights, we propose \textbf{UNIP}, a \textbf{UN}ified \textbf{I}nfrared \textbf{P}re-training framework, to enhance the pre-trained model performance. This framework uses the hybrid-attention distillation NMI-HAD as the pre-training target, a large-scale mixed dataset InfMix for pre-training, and a last-layer feature pyramid network LL-FPN for fine-tuning. Experimental results show that UNIP outperforms various pre-training methods by up to \textbf{13.5\%} in average mIoU on three infrared segmentation tasks, evaluated using fine-tuning and linear probing metrics. UNIP-S\footnote{We use the term \textit{method-size} to denote the vision transformer (ViT) of a specific \textit{size} pre-trained by a specific \textit{method}. T, S, B, and L refer to the ViT-Tiny, ViT-Small, ViT-Base, and ViT-Large, respectively.} achieves performance on par with MAE-L while requiring only \textbf{1/10} of the computational cost. Furthermore, UNIP significantly surpasses state-of-the-art (SOTA) infrared or RGB segmentation methods and demonstrates broad potential for application in other modalities, such as RGB and depth. Our code is available at \href{https://github.com/casiatao/UNIP}{https://github.com/casiatao/UNIP}.

\end{abstract}

\section{Introduction}
\label{sec:introduction}
The business processes of organizations are experiencing ever-increasing complexity due to the large amount of data, high number of users, and high-tech devices involved \cite{martin2021pmopportunitieschallenges, beerepoot2023biggestbpmproblems}. This complexity may cause business processes to deviate from normal control flow due to unforeseen and disruptive anomalies \cite{adams2023proceddsriftdetection}. These control-flow anomalies manifest as unknown, skipped, and wrongly-ordered activities in the traces of event logs monitored from the execution of business processes \cite{ko2023adsystematicreview}. For the sake of clarity, let us consider an illustrative example of such anomalies. Figure \ref{FP_ANOMALIES} shows a so-called event log footprint, which captures the control flow relations of four activities of a hypothetical event log. In particular, this footprint captures the control-flow relations between activities \texttt{a}, \texttt{b}, \texttt{c} and \texttt{d}. These are the causal ($\rightarrow$) relation, concurrent ($\parallel$) relation, and other ($\#$) relations such as exclusivity or non-local dependency \cite{aalst2022pmhandbook}. In addition, on the right are six traces, of which five exhibit skipped, wrongly-ordered and unknown control-flow anomalies. For example, $\langle$\texttt{a b d}$\rangle$ has a skipped activity, which is \texttt{c}. Because of this skipped activity, the control-flow relation \texttt{b}$\,\#\,$\texttt{d} is violated, since \texttt{d} directly follows \texttt{b} in the anomalous trace.
\begin{figure}[!t]
\centering
\includegraphics[width=0.9\columnwidth]{images/FP_ANOMALIES.png}
\caption{An example event log footprint with six traces, of which five exhibit control-flow anomalies.}
\label{FP_ANOMALIES}
\end{figure}

\subsection{Control-flow anomaly detection}
Control-flow anomaly detection techniques aim to characterize the normal control flow from event logs and verify whether these deviations occur in new event logs \cite{ko2023adsystematicreview}. To develop control-flow anomaly detection techniques, \revision{process mining} has seen widespread adoption owing to process discovery and \revision{conformance checking}. On the one hand, process discovery is a set of algorithms that encode control-flow relations as a set of model elements and constraints according to a given modeling formalism \cite{aalst2022pmhandbook}; hereafter, we refer to the Petri net, a widespread modeling formalism. On the other hand, \revision{conformance checking} is an explainable set of algorithms that allows linking any deviations with the reference Petri net and providing the fitness measure, namely a measure of how much the Petri net fits the new event log \cite{aalst2022pmhandbook}. Many control-flow anomaly detection techniques based on \revision{conformance checking} (hereafter, \revision{conformance checking}-based techniques) use the fitness measure to determine whether an event log is anomalous \cite{bezerra2009pmad, bezerra2013adlogspais, myers2018icsadpm, pecchia2020applicationfailuresanalysispm}. 

The scientific literature also includes many \revision{conformance checking}-independent techniques for control-flow anomaly detection that combine specific types of trace encodings with machine/deep learning \cite{ko2023adsystematicreview, tavares2023pmtraceencoding}. Whereas these techniques are very effective, their explainability is challenging due to both the type of trace encoding employed and the machine/deep learning model used \cite{rawal2022trustworthyaiadvances,li2023explainablead}. Hence, in the following, we focus on the shortcomings of \revision{conformance checking}-based techniques to investigate whether it is possible to support the development of competitive control-flow anomaly detection techniques while maintaining the explainable nature of \revision{conformance checking}.
\begin{figure}[!t]
\centering
\includegraphics[width=\columnwidth]{images/HIGH_LEVEL_VIEW.png}
\caption{A high-level view of the proposed framework for combining \revision{process mining}-based feature extraction with dimensionality reduction for control-flow anomaly detection.}
\label{HIGH_LEVEL_VIEW}
\end{figure}

\subsection{Shortcomings of \revision{conformance checking}-based techniques}
Unfortunately, the detection effectiveness of \revision{conformance checking}-based techniques is affected by noisy data and low-quality Petri nets, which may be due to human errors in the modeling process or representational bias of process discovery algorithms \cite{bezerra2013adlogspais, pecchia2020applicationfailuresanalysispm, aalst2016pm}. Specifically, on the one hand, noisy data may introduce infrequent and deceptive control-flow relations that may result in inconsistent fitness measures, whereas, on the other hand, checking event logs against a low-quality Petri net could lead to an unreliable distribution of fitness measures. Nonetheless, such Petri nets can still be used as references to obtain insightful information for \revision{process mining}-based feature extraction, supporting the development of competitive and explainable \revision{conformance checking}-based techniques for control-flow anomaly detection despite the problems above. For example, a few works outline that token-based \revision{conformance checking} can be used for \revision{process mining}-based feature extraction to build tabular data and develop effective \revision{conformance checking}-based techniques for control-flow anomaly detection \cite{singh2022lapmsh, debenedictis2023dtadiiot}. However, to the best of our knowledge, the scientific literature lacks a structured proposal for \revision{process mining}-based feature extraction using the state-of-the-art \revision{conformance checking} variant, namely alignment-based \revision{conformance checking}.

\subsection{Contributions}
We propose a novel \revision{process mining}-based feature extraction approach with alignment-based \revision{conformance checking}. This variant aligns the deviating control flow with a reference Petri net; the resulting alignment can be inspected to extract additional statistics such as the number of times a given activity caused mismatches \cite{aalst2022pmhandbook}. We integrate this approach into a flexible and explainable framework for developing techniques for control-flow anomaly detection. The framework combines \revision{process mining}-based feature extraction and dimensionality reduction to handle high-dimensional feature sets, achieve detection effectiveness, and support explainability. Notably, in addition to our proposed \revision{process mining}-based feature extraction approach, the framework allows employing other approaches, enabling a fair comparison of multiple \revision{conformance checking}-based and \revision{conformance checking}-independent techniques for control-flow anomaly detection. Figure \ref{HIGH_LEVEL_VIEW} shows a high-level view of the framework. Business processes are monitored, and event logs obtained from the database of information systems. Subsequently, \revision{process mining}-based feature extraction is applied to these event logs and tabular data input to dimensionality reduction to identify control-flow anomalies. We apply several \revision{conformance checking}-based and \revision{conformance checking}-independent framework techniques to publicly available datasets, simulated data of a case study from railways, and real-world data of a case study from healthcare. We show that the framework techniques implementing our approach outperform the baseline \revision{conformance checking}-based techniques while maintaining the explainable nature of \revision{conformance checking}.

In summary, the contributions of this paper are as follows.
\begin{itemize}
    \item{
        A novel \revision{process mining}-based feature extraction approach to support the development of competitive and explainable \revision{conformance checking}-based techniques for control-flow anomaly detection.
    }
    \item{
        A flexible and explainable framework for developing techniques for control-flow anomaly detection using \revision{process mining}-based feature extraction and dimensionality reduction.
    }
    \item{
        Application to synthetic and real-world datasets of several \revision{conformance checking}-based and \revision{conformance checking}-independent framework techniques, evaluating their detection effectiveness and explainability.
    }
\end{itemize}

The rest of the paper is organized as follows.
\begin{itemize}
    \item Section \ref{sec:related_work} reviews the existing techniques for control-flow anomaly detection, categorizing them into \revision{conformance checking}-based and \revision{conformance checking}-independent techniques.
    \item Section \ref{sec:abccfe} provides the preliminaries of \revision{process mining} to establish the notation used throughout the paper, and delves into the details of the proposed \revision{process mining}-based feature extraction approach with alignment-based \revision{conformance checking}.
    \item Section \ref{sec:framework} describes the framework for developing \revision{conformance checking}-based and \revision{conformance checking}-independent techniques for control-flow anomaly detection that combine \revision{process mining}-based feature extraction and dimensionality reduction.
    \item Section \ref{sec:evaluation} presents the experiments conducted with multiple framework and baseline techniques using data from publicly available datasets and case studies.
    \item Section \ref{sec:conclusions} draws the conclusions and presents future work.
\end{itemize}
\begin{figure}[t]
    \centering
    \begin{tikzpicture}
       \node at (0,0) {
        \includegraphics[width=\linewidth]{figures/benchmark_v6.pdf}
      };
      % Draw the table
      \node[anchor=south west, xshift=-5.4cm, yshift=-1.85cm] at (current page.south west) {
        \scriptsize
        \setlength{\tabcolsep}{0.8mm}{
        \scalebox{0.9}{
        \begin{tabular}{lcc}
          \hline
          Model & Param & FLOPs \\
          \hline
          ViT-T & 11.0M & 36.7G \\
          ViT-S & 41.9M & 146.2G \\
          ViT-B & 163.7M & 584.0G \\
          ViT-L & 441.3M & 1192.6G \\
          \hline
        \end{tabular}}}
      };
    \end{tikzpicture}
    \vspace{-20pt}
    \caption{The performance of pre-trained models across various methods and sizes. \textit{Left}: The average fine-tuning (FT) performance on three infrared semantic segmentation datasets, along with the associated computational cost. \textit{Middle}: The average linear probing (LP) performance on three infrared datasets. \textit{Right}: The fine-tuning performance on ImageNet \citep{imagenet}. The \textcolor{gray}{gray} dotted lines and corresponding values highlight the performance gains of UNIP over other methods. Detailed results for each dataset are presented in \tabref{tab:benchmark}.}
    \label{fig:benchmark}
    \vspace{-4pt}
\end{figure}

\section{How do pre-training methods perform on infrared tasks?}
\label{sec:benchmark}
In this section, we benchmark six pre-training methods on three infrared semantic segmentation datasets and discuss several key phenomena.

\subsection{Infrared Segmentation Benchmark of RGB Pre-trained Models}
\label{sec:benchmark_setup}
\textbf{Pre-trained Backbone.} Pre-training of the Vision Transformer (ViT) \citep{vit} has gained widespread attention and demonstrated powerful performance in various fields. Many recent pre-training methods \citep{deit,mae,iBOT,dinov2} use ViT for experiments, making pre-trained ViT models readily available. Therefore, ViT models of various sizes are set as the evaluation backbone.

\textbf{Pre-training Methods.} Both supervised and self-supervised methods are investigated. For supervised approaches, we use \textbf{DeiT} \citep{deit} and \textbf{DeiT III} \citep{deit3}, which perform image classification on ImageNet for pre-training. In self-supervised methods, we study contrastive learning (CL) and masked image modeling (MIM). CL methods like \textbf{DINO} \citep{dino} encourage features from different views of the same image to be close, while keeping features from different images distinct. MIM methods like \textbf{MAE} \citep{mae} and \textbf{CrossMAE} \citep{crossmae} focus on reconstructing masked image patches by learning context relations. Although \textbf{iBOT} \citep{iBOT} combines CL with masked feature prediction, we classify it as a CL method due to its similar characteristics to DINO. \textbf{The above methods are selected because they all pre-train vanilla ViT models on ImageNet}, without additional pre-trained tokenizers like BeiT \citep{beit} or MILAN \citep{milan}, or larger datasets like EVA \citep{eva} and DINOv2 \citep{dinov2}. This allows us to focus on the impact of the pre-training tasks alone.

\textbf{Evaluation Datasets.} The evaluation is conducted on three infrared semantic segmentation datasets: SODA \citep{soda}, MFNet-T \citep{mfnet}, and SCUT-Seg \citep{mcnet}. Notably, MFNet is an RGB-Thermal paired dataset. The thermal part MFNet-T is used for benchmarks while the RGB part MFNet-RGB is employed in further investigations. Additionally, RGB datasets like ImageNet-1K \citep{imagenet} and ADE20K \citep{ade20k} are also used for comparison. Details about these datasets can be found in \Appref{app:evaluation_datasets}.

\textbf{Evaluation Metrics.} We employ two metrics: \textit{fine-tuning} (FT) and \textit{linear probing} (LP). FT (\figref{fig:ft_lp}\textcolor{red}{a}) is the primary metric, where both the pre-trained model and the decoder are tuned with the labeled datasets. In LP (\figref{fig:ft_lp}\textcolor{red}{b}), only a linear head is updated while all other parameters remain frozen. \textbf{Average (Avg) FT or LP performance in subsequent sections denotes the mean mIoU across three infrared semantic segmentation datasets.} More details are available in \Appref{app:appendix_evaluation}.

\textbf{Benchmark Results.} In the benchmark, all models are trained for 100 epochs for both evaluation metrics. Typical results are illustrated in \figref{fig:benchmark}, with ImageNet \citep{imagenet} fine-tuning performance included for comparison. The complete results of each dataset are detailed in \tabref{tab:benchmark}. 



\subsection{What insights can we gain from this benchmark?}
\label{sec:benchmark_multi_layer}

\begin{wraptable}{r}{0.36\textwidth}
    \vspace{-7.5mm}
    \caption{Pearson coefficients between average FT and other metrics.}
    \label{tab:correlation_coeff}
    \vspace{1mm}
    \centering
    \scriptsize
    \setlength{\tabcolsep}{0.8mm}{
    \scalebox{1.0}{
    \begin{tabular}{l c c c c}
        \toprule
        Metric & Small & Base & Large & Mean \\
        \midrule
        Avg FT \& Avg LP & \textbf{0.89} & \textbf{0.93} & \textbf{0.81} & \textbf{0.88}  \\
        Avg FT \& IN1K FT & 0.78 & 0.12 & -0.66 & 0.08 \\
        \bottomrule
    \end{tabular}}}
    \vspace{-4mm}
\end{wraptable}

\textbf{The infrared FT performance is strongly positively correlated with LP, but has no clear relationship with ImageNet FT.} \tabref{tab:correlation_coeff} presents the Pearson \citep{pearson} correlation coefficients between different metrics. For each metric pair, the coefficients are calculated across six pre-training methods in \figref{fig:benchmark}. Notably, the coefficients between average infrared LP and FT are close to 1, indicating that models with better LP performance generally exhibit better FT performance. Conversely, the ImageNet FT performance does not consistently correlate with infrared FT results across various model sizes, likely due to domain and task differences. Therefore, using ImageNet accuracy to predict transfer performance on infrared segmentation datasets is not reliable, underscoring the importance of benchmarking on infrared segmentation datasets.

\textbf{Supervised and CL methods outperform MIM methods, especially for small models.} As depicted in \figref{fig:benchmark} and \tabref{tab:benchmark}, the performance of supervised and CL methods like DeiT, DeiT III, DINO, and iBOT is similar across both metrics, except for the LP of DeiT-S. For the LP metric, MIM methods of various sizes consistently lag behind supervised and CL methods by a significant margin, matching observations in the RGB domain \citep{mae} that MIM representations are less linearly separable. In terms of FT, smaller MIM models (ViT-T, S, and B) still underperform supervised and CL methods, while larger models (ViT-L) are more competitive. For instance, MAE-S is far behind iBOT-S (55.39\% vs 62.09\%), but MAE-L performs comparably to iBOT-L (64.35\% vs 64.97\%). As we will discuss in \secref{sec:investigation}, the discrepancy in the attention pattern distribution and texture bias between different models accounts for their distinct infrared segmentation performance.

\textbf{Larger models perform better, but their computational cost increases sharply.}
As illustrated in \figref{fig:benchmark}, larger MIM models bring considerable performance gains over smaller models. However, for supervised and CL methods, small models are already well-trained, and the performance improvement from larger models is marginal compared to the significant increase in computational cost. For example, iBOT-L surpasses iBOT-S by only 2.85\% (64.97\% vs 62.09\%), while the parameter count and FLOPs increase by 10.5$\times$ (441M vs 42M) and 8.2$\times$ (1193G vs 146G), respectively. Given that infrared images are often processed on edge devices with limited computing budgets, using large models to pursue better performance is not cost-effective. Therefore, we believe improving small models is a more effective approach. In \secref{sec:distill}, we propose several strategies to elevate the performance of small models to be on par with larger models.

\section{What matters for infrared semantic segmentation?}
\label{sec:investigation}
To determine which characteristics of the pre-trained models are critical for infrared semantic segmentation, we analyze different models from multiple perspectives.

\subsection{The Pre-training tasks influence attention patterns}
\label{sec:attention_pattern}
The self-attention mechanism is a key component of ViT. In semantic segmentation tasks, the spatial interactions between tokens are crucial. Thus, we visualize the attention maps of pre-trained models. 

\textbf{Attention maps of supervised/CL and MIM methods differ significantly.} As shown in \figref{fig:query_attn}\textcolor{red}{a}, DINO-S exhibits three distinct attention patterns: (1) \textit{Local}: In shallow layers (Layer 5), different query tokens focus only on their spatially nearby key tokens; (2) \textit{Hybrid}: In middle layers (Layer 9), query tokens attend to both nearby tokens and foreground tokens; (3) \textit{Global}: In deep layers (Layer 12), different query tokens all focus on foreground tokens with nearly identical attention maps, a phenomenon known as \textit{attention collapse} \citep{ssl_vit}. This attention pattern distribution is consistent across different sizes in CL and supervised methods, as shown in \figref{fig:query_attn_cl}. However, in MIM methods like MAE, the distribution varies. In MAE-S (\figref{fig:query_attn}\textcolor{red}{b}), attention maps are mainly \textit{local}, with slight \textit{hybrid} patterns emerging in deep layers. Conversely, in MAE-L (\figref{fig:query_attn}\textcolor{red}{c}), shallow and deep layers exhibit \textit{local} patterns, while middle layers show \textit{hybrid} patterns. CKA \citep{CKA} analysis in \Appref{app:CKA} reveals similar phenomena regarding feature representation.

\begin{figure}[t]
\centering
\includegraphics[width=\linewidth]{figures/query_attn_v9.pdf}
\vspace{-20pt}
\caption{Attention maps for different query tokens in three representative layers. Each query token's attention map corresponds to a row in the attention matrix, averaged over different heads.}
\label{fig:query_attn}
\vspace{-8pt}
\end{figure}


\textbf{Differences in attention patterns stem from the pre-training tasks.} CL methods, similar to supervised approaches, treat views from the same image as belonging to the same class. This setup encourages models to focus on foreground tokens, as images in the same class often share similar foreground objects but may differ in background. Consequently, attention maps in later layers present \textit{global} patterns. The \textit{local} and \textit{hybrid} patterns can be regarded as the intermediate states in forming the \textit{global} pattern. This high-level pre-training task causes models of different sizes and methods to have similar pattern distributions across layers.
In contrast, pre-training tasks of MIM methods focus on reconstructing features or raw pixels of masked tokens, which is a relatively low-level task relying heavily on spatially nearby tokens. Consequently, models are not compelled to capture global image information, leading small models to primarily exhibit \textit{local} patterns. In larger models like MAE-L, the increased representation capacity allows \textit{hybrid} patterns to spontaneously emerge in the middle layers to capture broader context. In deep layers near the decoder, \textit{local} patterns reappear to support the pre-training task of reconstructing nearby masked tokens.

As a supplement, iBOT exhibits similar patterns with DINO in shallow and middle layers but shows less \textit{attention collapse} in deep layers (see \figref{fig:query_attn_cl}). This can be attributed to that iBOT combines DINO with masked feature prediction \citep{iBOT}, which encourages the later layers to leverage spatial information to predict features of masked tokens.


\subsection{How to quantitively identify different attention patterns?}
\label{sec:nmi}


Three distinct attention patterns are qualitatively summarized in \secref{sec:attention_pattern}, prompting the question of whether a metric can quantitatively measure them. Attention distance measures the average distance 
\begin{wrapfigure}{r}{0.32\textwidth}
\vspace{-1pt}
\centering
\includegraphics[width=1.0\linewidth]{figures/nmi_v9.pdf}
\vspace{-23pt}
\caption{NMI on ImageNet.}
\label{fig:nmi}
\vspace{-11pt}
\end{wrapfigure}
between the query and key tokens, while attention entropy implies the concentration of the attention distribution. However, both metrics depict the relationship between one query and multiple key tokens and are unable to reflect differences in the attention maps of various queries. \textbf{We find that the normalized mutual information (NMI) between query and key tokens is an effective indicator.} The calculation process is elaborated in \Appref{app:NMI}. Let $A\in \mathbb{R}^{N\times N}$ denote the attention matrix, where $N$ is the number of tokens. The NMI is a function of $A$, ranging from 0 to 1.  We highlight two special cases to clarify NMI: (1) When query tokens focus solely on their spatially corresponding key tokens (an extreme \textit{local} pattern), $A$ becomes an identity matrix. Thus the joint probability of query and key tokens is equivalent to their marginal probability and the NMI reaches its maximum value of 1; (2) When all query tokens attend to the same key tokens (an extreme \textit{global} pattern), each row of $A$ is identical. Consequently, the probability distribution of query and key tokens are independent, leading to the minimum NMI of 0.


\textbf{Therefore, \textit{local} patterns have larger NMI values, while \textit{global} patterns exhibit lower ones.} As illustrated in \figref{fig:nmi}, the NMI of DINO-S and iBOT-S decreases with depth and consistently stays below that of MAE models. Especially, the NMI of DINO-S approaches 0 in later layers, revealing its \textit{attention collapse}. In contrast, the NMI of MAE-L first decreases and then increases, due to the \textit{hybrid} patterns in middle layers and \textit{local} patterns in later layers. 


\subsection{The Hybrid pattern matters for semantic segmentation}
\label{sec:lg_matter}
Semantic segmentation is a dense prediction task where all pixels in an image are classified into different semantic classes. Local information is crucial as nearby pixels usually belong to the same class, while global cues are also essential since instances of the same class may appear in different positions within the image. Therefore, we hypothesize that \textbf{\textit{hybrid} patterns, which capture both local and global information, are more important for semantic segmentation than purely \textit{local} or \textit{global} patterns}. To demonstrate this, we conduct the \textit{layerwise linear probing} (LLP) experiments, where frozen features of only one layer are passed to the linear head, as shown in \figref{fig:ft_lp}\textcolor{red}{d}.

\textbf{The LLP performance peaks where \textit{hybrid} attention patterns emerge.} As shown in \figref{fig:layerwise_lp}, supervised and CL methods peak at about three-quarters of the model's depth. Large MIM models (ViT-L) perform better in the middle layers. These peaks commonly occur near the \textit{hybrid} patterns. In contrast, the performance of small MIM models (ViT-S and ViT-B) gradually increases with depth, peaking in the last two layers. This is because, although all layers exhibit \textit{local} patterns, deep layers focus more on foreground tokens (\figref{fig:query_attn}\textcolor{red}{b}) and have smaller NMI values (\figref{fig:nmi}), leading to better LLP performance. Additionally, the performance degradation in iBOT's deep layers is less pronounced than that in DINO and supervised methods. This aligns with observations that iBOT's deep-layer attention maps contain more local information (\secref{sec:attention_pattern}) and have larger NMI values than those of DINO (\figref{fig:nmi}), underscoring the importance of the \textit{hybrid} attention pattern.



\textbf{This hypothesis can explain the phenomena in \secref{sec:benchmark}.} Small MIM models struggle to learn the \textit{hybrid} pattern, resulting in a notable performance gap compared to supervised and CL methods. Conversely, large MIM models successfully develop the \textit{hybrid} pattern, making their fine-tuning performance comparable to other methods. iBOT performs best across different model sizes and evaluation metrics because the \textit{hybrid} pattern occurs more frequently than in other methods.



\subsection{The Texture bias hinders the model's generalization on infrared images}
\label{sec:texture}


\begin{wraptable}{r}{0.51\textwidth}
    \vspace{-7.5mm}
    \caption{The FT performance on RGB and infrared semantic segmentation datasets.}
    \label{tab:rgb_infrared}
    \centering
    \scriptsize
    \setlength{\tabcolsep}{1.0mm}{
    \scalebox{1.0}{
    \begin{tabular}{l c c c c c}
        \toprule
        \multirow{2}{*}{Methods} & \multicolumn{2}{c}{RGB} & \multicolumn{3}{c}{Infrared} \\
        \cmidrule(lr){2-3} \cmidrule(lr){4-6}
        & ADE20K & MFNet-RGB & MFNet-T & SODA & SCUT-Seg \\
        \midrule
        DeiT-B & 47.4 & 57.07 & \textbf{48.59} & 69.73 & 69.35 \\
        DINO-B & 46.8 & 55.20 & 48.54 & \textbf{69.79} & \textbf{69.82} \\
        MAE-B & \textbf{48.1} & \textbf{57.29} & 46.78 & 68.18 & 67.86 \\
        \bottomrule
    \end{tabular}}}
    \vspace{-4mm}
\end{wraptable}

When transferring RGB pre-trained models to infrared tasks, the distribution shift between these modalities significantly impacts performance. A major difference between RGB and infrared images is that RGB images can capture fine-grained textures, which are scarce in infrared images. Therefore, we assume that \textbf{the model's bias towards texture would exacerbate the distribution shift, thereby impairing the transfer performance on infrared tasks.}

\begin{figure}[t]
\centering
\includegraphics[width=\linewidth]{figures/layer_lp_v9.pdf}
\vspace{-20pt}
\caption{The layerwise linear probing performance of different methods on SODA \citep{soda}.}
\label{fig:layerwise_lp}
\vspace{-8pt}
\end{figure}

According to \citet{ssl_vit}, MIM methods are texture-biased while CL and supervised methods are shape-biased. This bias is evident in attention maps in \figref{fig:query_attn}, where MAE models focus on textures while DINO emphasizes edges. We conduct experiments to investigate the bias's impact on infrared segmentation. As shown in \tabref{tab:rgb_infrared}, MAE-B outperforms DeiT-B and DINO-B on RGB datasets like ADE20K \citep{ade20k} and MFNet-RGB \citep{mfnet}, but consistently underperforms on infrared datasets. Notably, the paired MFNet-RGB and MFNet-T share the same scenario and image counts, differing only in modality. This indicates that MAE models pre-trained on ImageNet rely on low-level texture information to reconstruct masked patches, leading to poor generalization on texture-less infrared images. Therefore, \textbf{reducing the texture bias is a promising way to enhance the transfer performance of RGB-pre-trained models on infrared tasks.}

\section{How to improve the performance on infrared segmentation?}
\label{sec:distill}
As discussed in \secref{sec:benchmark_multi_layer}, scaling up model sizes for better performance is impractical for resource-constrained scenarios. Therefore, we focus on enhancing small pre-trained models by introducing a comprehensive framework, UNIP, and validating its effectiveness through extensive experiments.

\subsection{UNIP: a unified infrared pre-training framework}
\label{sec:unip}
UNIP improves small pre-trained models by optimizing the pre-training task, constructing an appropriate pre-training dataset, and refining the fine-tuning architecture, as depicted in \figref{fig:architecture}.

\textbf{NMI-Guided Hybrid Attention Pattern Distillation (NMI-HAD).} Compressing knowledge from large models into smaller ones is an effective strategy to enhance performance without increasing parameter count. Previous works use various distillation targets like logits \citep{dino} and features \citep{efficient_sam}. However, they often overlook the relationship between distillation targets and attention patterns. As revealed in \secref{sec:lg_matter}, the \textit{hybrid} attention pattern is crucial for semantic segmentation, with NMI values linked to attention patterns. Therefore, we propose using \textit{hybird} patterns as the distillation target and introduce the NMI-guided \textit{hybrid} attention pattern distillation. First, the NMI value $\text{NMI}(A_l)$ of each teacher model's layer is calculated on ImageNet-1K:
\begin{equation}
    \text{NMI}(A_l)=\frac{1}{M}\sum_{m=1}^M\text{NMI}(A_l^m),  \quad A_l^m=\text{softmax}\left(\frac{Q_l^m (K_l^m)^T}{\sqrt{d}}\right), \quad l=\frac{L}{2}+1,...,L,
    \label{eq:nmi_cal}
\end{equation}
where $A_l^m$ denotes the $m$-th head attention matrix in the $l$-th layer. $L$ and $d$ are the number of layers and the dimension of the teacher model. The method for calculating NMI is detailed in \Appref{app:NMI}. Note that we only consider layers in the latter half of the model, as shallow layers do not capture sufficient knowledge. Next, NMI values are used to identify the location of the \textit{hybrid} attention pattern. The attention map of the layer whose NMI is closest to an empirical value $s$ is utilized as the distillation target $A_T$. Finally, the attention map $A_S$ in the last layer of the student model is forced to imitate $A_T$ by employing the Kullback-Leibler (KL) divergence constraints:
\begin{equation}
    A_T = \mathop{\arg\max}_{A_l}\Delta \text{NMI}(A_l), \quad \Delta \text{NMI}(A_l) =-\left\vert \text{NMI}(A_l)-s \right\vert, \quad
    \mathcal{L}=\frac{1}{M}\sum_{m=1}^M \text{KL}(A_{T}^m||A_{S}^m), 
    \label{eq:distill}
\end{equation}
Empirically, the NMI values of \textit{hybrid} patterns range between 0.06 and 0.12. We find that setting $s$ within this range yields good results. In all our experiments, we set it to 0.09 by default.


\begin{wraptable}{r}{0.47\textwidth}
    \vspace{-3.5mm}
    \centering
    \caption{Comparisons of infrared pre-training datasets. \#Subset denotes the number of datasets from which the images are collected.}
    \label{tab:dataset_comparison}
    \vspace{1mm}
    \scriptsize
    \setlength{\tabcolsep}{1.0mm}{
    \scalebox{1.0}{
    \begin{tabular}{l c c c c}
        \toprule
        Dataset & \#Image & \#Subset & Width & Height \\
        \midrule
        MSIP \citep{pad} & 178,756 & 8 & 844 & 596 \\
        Inf30 \citep{infmae} & 305,241 & - & 700 & 562 \\
        InfPre (ours) & \textbf{541,088} & \textbf{23} & \textbf{1,075} & \textbf{686} \\
        % InfMix (ours) & 859,375 & 25 & & \\
        \bottomrule
    \end{tabular}}}
    \vspace{-3.5mm}
\end{wraptable}

\textbf{InfMix Dataset.} To alleviate the distribution shift and reduce texture bias when distilling RGB pre-trained models for infrared tasks, we develop InfMix, a mixed dataset for distillation. InfMix comprises \textbf{859,375} images from both RGB and infrared modalities, constructed through four steps. (1) Infrared images play a key role in mitigating the distribution shift. However, existing datasets often lack diversity and sufficient images, so we collect a large and unlabelled infrared pre-training dataset called \textbf{InfPre}. It consists of \textbf{541,088} images from \textbf{23} infrared-related datasets. Compared to the other two datasets in \tabref{tab:dataset_comparison}, InfPre offers a larger number of higher-resolution images sourced from more diverse datasets. Importantly, three segmentation datasets used in the benchmark are excluded from InfPre for fair comparison. Details on data collection and deduplication can be found in \Appref{app:infpre}. (2) A subset of ImageNet-1K \citep{imagenet} is used, comprising \textbf{200,000} images evenly sampled from 1,000 classes. Since these images are part of the teacher model's pre-training data, they can anchor the student representation space close to the teacher's, thereby aiding in transferring the teacher's general feature extraction capabilities to the student. (3) The training set of COCO \citep{coco}, with \textbf{118,287} images, is also included to further enrich the pre-training dataset. Unlike single-object-centric images in ImageNet, COCO images typically depict larger scenes with multiple objects, making them more similar to infrared images, as indicated in \tabref{tab:domain_gap} in the appendix. (4) Images from ImageNet and COCO are converted to grayscale (three identical channels) to resemble infrared images more closely, as noted in \tabref{tab:domain_gap}.

\textbf{Last-Layer Feature Pyramid Network (LL-FPN).} To adapt the non-hierarchical ViT to multi-scale decoders in dense prediction tasks, previous works \citep{mae,iBOT} typically generate multi-scale feature maps from different layers of ViT, as shown in \figref{fig:ft_lp}\textcolor{red}{a}. \textbf{However, we find this multi-layer design unnecessary for our distilled models.} In these models, the \textit{hybrid} patterns in later layers equip the final features with both local and global information, making them suitable for multi-scale feature map generation. Inspired by ViTDet \citep{vitdet}, we propose using the last-layer feature pyramid network during fine-tuning. It constructs all feature maps of different scales upon the last layer's features, as illustrated in \figref{fig:architecture} and \figref{fig:ft_lp}\textcolor{red}{c}. As a bonus, this approach enhances the representation capacity of each scale branch compared to the configuration in \figref{fig:ft_lp}\textcolor{red}{a}, since they go through the entire backbone, leading to improved fine-tuning performance.

\begin{table}[t]
    \centering
    \caption{The infrared semantic segmentation performance of different models. Across various pre-trained methods, UNIP models significantly surpass pre-trained models of the same size. Remarkably, they even outperform their teacher models, despite the latter having more parameters.}
    \label{tab:main_distill}
    % \vspace{-2mm}
    \centering
    \scriptsize
    \setlength{\tabcolsep}{1.2mm}{
    \scalebox{1.0}{
    \begin{tabular}{l c c c c c c c c c}
        \toprule
        \multirow{2}{*}{Methods} & \multirow{2}{*}{Params(M)} & \multicolumn{4}{c}{Fine-tuning (FT)} & \multicolumn{4}{c}{Linear Probing (LP)} \\ 
        \cmidrule(lr){3-6} \cmidrule(lr){7-10}
        & & SODA & MFNet-T & SCUT-Seg & Average FT & SODA & MFNet-T & SCUT-Seg & Average LP \\
        \midrule
        \textcolor{gray}{MAE-L (Teacher)} & \textcolor{gray}{441.3} & \textcolor{gray}{71.04} & \textcolor{gray}{51.17} & \textcolor{gray}{70.83} & \textcolor{gray}{64.35} & \textcolor{gray}{52.20} & \textcolor{gray}{31.21} & \textcolor{gray}{43.71} & \textcolor{gray}{42.37} \\
        MAE-T & 11.0 & 52.85 & 35.93 & 51.31 & 46.70 & 23.75 & 15.79 & 27.18 & 22.24 \\
        \rowcolor{cyan!15} UNIP-T & 11.0 & \textbf{64.83} & \textbf{48.77} & \textbf{67.22} & \textbf{60.27} (\plus{+13.57}) & \textbf{44.12} & \textbf{28.26} & \textbf{35.09} & \textbf{35.82} (\plus{+13.58}) \\
        MAE-S & 41.9 & 63.36 & 42.44 & 60.38 & 55.39 & 38.17 & 21.14 & 34.15 & 31.15 \\
        \rowcolor{cyan!15} UNIP-S & 41.9 & \textbf{70.99} & \textbf{51.32} & \textbf{70.79} & \textbf{64.37} (\plus{+8.98}) & \textbf{55.25} & \textbf{33.49} & \textbf{43.37} & \textbf{44.04} (\plus{+12.89}) \\
        MAE-B & 163.7 & 68.18 & 46.78 & 67.86 & 60.94 & 43.01 & 23.42 & 37.48 & 34.64 \\
        \rowcolor{cyan!15} UNIP-B & 163.7 & \textbf{71.47} & \textbf{52.55} & \textbf{71.82} & \textbf{65.28} (\plus{+4.34}) & \textbf{58.82} & \textbf{34.75} & \textbf{48.74} & \textbf{47.43} (\plus{+12.79}) \\
        \midrule
        \textcolor{gray}{DINO-B (Teacher)} & \textcolor{gray}{163.7} & \textcolor{gray}{69.79} & \textcolor{gray}{48.54} & \textcolor{gray}{69.82} & \textcolor{gray}{62.72} & \textcolor{gray}{59.33} & \textcolor{gray}{34.86} & \textcolor{gray}{47.23} & \textcolor{gray}{47.14} \\
        DINO-S & 41.9 & 68.56 & 47.98 & 68.74 & 61.76 & 56.02 & 32.94 & 45.94 & 44.97 \\
        \rowcolor{cyan!15} UNIP-S & 41.9 & \textbf{69.35} & \textbf{49.95} & \textbf{69.70} & \textbf{63.00} (\plus{+1.24}) & \textbf{57.76} & \textbf{34.15} & \textbf{46.37} & \textbf{46.09} (\plus{+1.12}) \\
        \midrule
        \textcolor{gray}{iBOT-L (Teacher)} & \textcolor{gray}{441.3} & \textcolor{gray}{71.75} & \textcolor{gray}{51.66} & \textcolor{gray}{71.49} & \textcolor{gray}{64.97} & \textcolor{gray}{61.73} & \textcolor{gray}{36.68} & \textcolor{gray}{50.12} & \textcolor{gray}{49.51} \\
        iBOT-S & 41.9 & 69.33 & 47.15 & 69.80 & 62.09 & 57.10 & 33.87 & 45.82 & 45.60 \\
        \rowcolor{cyan!15} UNIP-S & 41.9 & \textbf{70.75} & \textbf{51.81} & \textbf{71.55} & \textbf{64.70} (\plus{+2.61}) & \textbf{60.28} & \textbf{37.16} & \textbf{47.68} & \textbf{48.37} (\plus{+2.77}) \\
        iBOT-B & 163.7 & 71.15 & 48.98 & 71.26 & 63.80 & 60.05 & 34.34 & 49.12 & 47.84 \\
        \rowcolor{cyan!15} UNIP-B & 163.7 & \textbf{71.75} & \textbf{51.46} & \textbf{72.00} & \textbf{65.07} (\plus{+1.27}) & \textbf{63.14} & \textbf{39.08} & \textbf{52.53} & \textbf{51.58} (\plus{+3.74}) \\    
        \bottomrule
    \end{tabular}}}
    \vspace{-4mm}
\end{table} 


\subsection{Experiments}
\label{sec:unip_experiments}

The MAE-L, DINO-B, and iBOT-L are utilized as teacher models for distillation, and the 18th, 9th, and 21st layers are used as the target layer, according to \eqnref{eq:nmi_cal} and \eqnref{eq:distill}. Unless otherwise specified, the distillation, fine-tuning, and linear probing processes are each conducted for 100 epochs. For ablation studies, we mainly focus on the fine-tuning metric as it reflects the model's highest achievable performance. More details about experimental settings can be found in \Appref{app:pre-training}.

\begin{table}[t]
  \begin{minipage}{0.49\linewidth}
  \caption{Comparisons with other segmentation methods (FT). All the compared results except PAD are borrowed from TINN. Training epochs of SODA and MFNet-T are 200 and 300.}
    \label{tab:sota_comparison}
    \vspace{1mm}
    \centering
    \scriptsize
    \setlength{\tabcolsep}{0.8mm}{
    \scalebox{1.0}{
    \begin{tabular}{l c c c c c c}
        \toprule
        Methods & Params(M) & SODA & MFNet-T \\
        \midrule
        DeepLab V3+ \citep{deeplabv3+} & 62.7  & 68.73 & 49.80  \\
        PSPNet \citep{pspnet} & 68.1 & 68.68 & 45.24 \\
        UPerNet \citep{upernet} & 72.3 & 67.45 & 48.56 \\
        SegFormer \citep{segformer} & 84.7 & 67.86 & 50.68 \\
        ViT-Adapter \citep{vitadapter} & 99.8 & 68.12 & 50.62 \\
        Mask2Former \citep{mask2former} & 216.0 & 67.58 & 51.30 \\
        MaskDINO \citep{maskdino} & 223.0 & 66.32 & 51.03 \\
        \midrule
        EC-CNN \citep{soda} & 54.5 & 65.87 & 47.56 \\
        MCNet \citep{mcnet} & 35.7 & 63.89 & 43.15 \\
        PAD (MAE-B) \citep{pad} & 164.9 & 69.71 & 50.14 \\
        TINN \citep{tinn} & 85.3 & 69.45 & 51.93  \\
        \midrule
        \rowcolor{cyan!15} UNIP-T (MAE-L) & 11.0 & 67.29 & 50.39 \\
        \rowcolor{cyan!15} UNIP-S (MAE-L) & 41.9 & \underline{71.35} & \underline{53.76} \\
        \rowcolor{cyan!15} UNIP-B (MAE-L) & 163.7 & \textbf{72.19} & \textbf{54.35} \\
        \bottomrule
    \end{tabular}}}
    \vspace{-4mm}
  \end{minipage}
  \hfill
  \begin{minipage}{0.5\linewidth}
    \centering
    \caption{Impact of distillation targets (UNIP-S).}
    \label{tab:target_ablation}
    \scriptsize
    \setlength{\tabcolsep}{1.0mm}{
    \scalebox{0.97}{
    \begin{tabular}{c c c c c c}
        \toprule
        Target (Teacher) & Layer & SODA & MFNet-T & SCUT-Seg & Avg FT \\
        \midrule
        \multirow{2}{*}{\makecell[c]{Feature \\ (MAE-L)}} & 18 & 65.66 & 48.44 & 66.55 & 60.22  \\
        & 24 & 66.86 & 49.00 & 66.61 & 60.82  \\
        \midrule
        \multirow{2}{*}{\makecell[c]{Attention \\ (MAE-L)}} & \cellcolor{cyan!15}18 & \cellcolor{cyan!15}\textbf{70.99} & \cellcolor{cyan!15}\textbf{51.32} & \cellcolor{cyan!15}\textbf{70.79} & \cellcolor{cyan!15}\textbf{64.37}  \\
        & 24 & 67.74 & 50.39 & 69.00 & 62.38  \\
        \bottomrule
    \end{tabular}}}
    \vspace{-2mm}
    \caption{Impact for the LL-FPN (UNIP-S). \xmark\, and \cmark\, denote the ones in \figref{fig:ft_lp}\textcolor{red}{a} and \figref{fig:ft_lp}\textcolor{red}{c}.}
    \label{tab:fpn_ablation}
    \centering
    \scriptsize
    \setlength{\tabcolsep}{0.8mm}{
    \scalebox{0.95}{
    \begin{tabular}{l c c c c c}
        \toprule
        Teacher (Layer) & LL-FPN & SODA & MFNet-T & SCUT-Seg & Avg FT \\
        \midrule
        \multirow{2}{*}{\makecell[c]{\textcolor{gray}{\textit{Hybrid} Pattern} \\ MAE-L (Layer 18)}} & \xmark & 69.54 & 50.18 & 70.63 & 63.45 \\
        & \cellcolor{cyan!15}\cmark & \cellcolor{cyan!15}\textbf{70.99} & \cellcolor{cyan!15}\textbf{51.32} & \cellcolor{cyan!15}\textbf{70.79} & \cellcolor{cyan!15}\textbf{64.37} \\
        \midrule
        \multirow{2}{*}{\makecell[c]{\textcolor{gray}{\textit{Local} Pattern} \\ MAE-L (Layer 24)}} & \xmark & 67.64 & 49.93 & 68.68 & 62.08 \\
        & \cmark & \textbf{67.74} & \textbf{50.39} & \textbf{69.00} & \textbf{62.38} \\
        \midrule
        \multirow{2}{*}{\makecell[c]{\textcolor{gray}{\textit{Hybrid} Pattern} \\ DINO-B (Layer 9)}} & \xmark & 68.56 & 49.44 & 68.49 & 62.16 \\
        & \cellcolor{cyan!15}\cmark & \cellcolor{cyan!15}\textbf{69.35} & \cellcolor{cyan!15}\textbf{49.95} & \cellcolor{cyan!15}\textbf{69.70} & \cellcolor{cyan!15}\textbf{63.00} \\
        \midrule
        \multirow{2}{*}{\makecell[c]{\textcolor{gray}{\textit{Global} Pattern} \\ DINO-B (Layer 12)}} & \xmark & \textbf{68.62} & 47.36 & \textbf{69.71} & 61.90 \\
        & \cmark & 68.50 & \textbf{48.40} & 69.67 & \textbf{62.19} \\
        \bottomrule
    \end{tabular}}}
    \vspace{-4mm}
  \end{minipage}
\end{table}




\textbf{Improvements of UNIP.} As shown in \tabref{tab:main_distill}, UNIP significantly enhances the performance of small models across both metrics, often exhibiting comparable or even better performance than teacher models. With MAE-L as the teacher, UNIP-T, UNIP-S, and UNIP-B achieve average mIoU gains of \textbf{13.57\%}, \textbf{8.98\%}, and \textbf{4.34\%} in fine-tuning, and \textbf{13.58\%}, \textbf{12.98\%}, and \textbf{12.79\%} in linear probing. Notably, UNIP-S performs comparably to MAE-L with only \textbf{1/10} of the computational cost. UNIP-B even outperforms MAE-L by \textbf{0.93\%} in FT and \textbf{5.06\%} in LP. Using iBOT-L as the teacher, UNIP-S transcends iBOT-S by \textbf{2.61\%} in FT and \textbf{2.77\%} in LP. Meanwhile, UNIP-B shows gains of \textbf{1.27\%} in FT and \textbf{3.74\%} in LP, exceeding its teacher iBOT-L. Even with a smaller teacher like DINO-B, UNIP-S still enhances performance by as least \textbf{1.12\%}. \tabref{tab:sota_comparison} compares the fine-tuning performance of UNIP with other RGB or infrared segmentation methods. With fewer than half the parameters, UNIP-S, distilled from MAE-L, surpasses the universal segmentation method Mask2Former \citep{mask2former} by \textbf{3.77\%} on SODA and \textbf{2.46\%} on MFNet-T. It also outperforms TINN \citep{tinn}, specially designed for infrared semantic segmentation, by \textbf{1.9\%} on SODA and \textbf{1.83\%} on MFNet-T. A larger model UNIP-B further widens this performance gap, indicating that UNIP can greatly unleash the potential of the vanilla ViT for infrared semantic segmentation.


\begin{wrapfigure}{r}{0.51\textwidth}
\vspace{-10pt}
\centering
\includegraphics[width=1.0\linewidth]{figures/ft_nmi_v3.pdf}
\vspace{-22pt}
\caption{The average FT and NMI of each target layer. Each model is distilled for 20 epochs.}
\label{fig:ft_nmi}
\vspace{-11pt}
\end{wrapfigure}

\textbf{Impact of Distillation Target Layers.} \figref{fig:ft_nmi} displays the average fine-tuning performance using different layers of MAE-L and DINO-B as the distillation target layer. Notably, both models exhibit a strong positive correlation between average FT performance and $\Delta$NMI in \eqnref{eq:distill}, as indicated by a large Pearson coefficient. Furthermore, the peaks of FT and $\Delta$NMI occur in the same layer, highlighting the effectiveness of the NMI-HAD.


\begin{wraptable}{r}{0.51\textwidth}
    \vspace{-7.5mm}
    \caption{Ablations for components of the InfMix dataset. The teacher and student models are MAE-L and UNIP-S. All datasets are distilled for the same number of iterations for fair comparison.}
    \label{tab:dataset_composition_ablation}
    \centering
    \scriptsize
    \setlength{\tabcolsep}{0.8mm}{
    \scalebox{1.0}{
    \begin{tabular}{l c c c c c}
        \toprule
        Dataset & \#Images & SODA & MFNet-T & SCUT-Seg & Avg FT \\
        \midrule
        \rowcolor{cyan!15} InfMix & 859,375 & \textbf{70.99} & \textbf{51.32} & 70.79 & \textbf{64.37} \\
        -- w/o IN1K & 659,375 & 69.41 & 51.13 & 70.21 & 63.58 \\
        -- w/o COCO & 741,088 & 69.62 & 51.29 & 69.58 & 63.50 \\
        -- w/o Grayscale & 859,375 & 69.73 & 50.71 & \textbf{71.09} & 63.84 \\
        \midrule
        ImageNet-1K & 1,281,167 & 69.39	& 49.11 & 69.63 & 62.71 \\ 
        InfPre & 541,088 & 68.45 & 51.27 & 67.87 & 62.53 \\
        \bottomrule
    \end{tabular}}}
    \vspace{-4mm}
\end{wraptable}
\textbf{Impact of Pre-training Datasets.} \tabref{tab:dataset_composition_ablation} illustrates the performance of different datasets. As anticipated, all components of the InfMix dataset are necessary, including the infrared dataset InfPre, the ImageNet subset \citep{imagenet}, the COCO training set  \citep{coco}, and the grayscale operation. Remarkably, InfMix significantly outperforms single-modality datasets like ImageNet and InfPre. This improvement can be attributed to the complementary strengths of both modalities: infrared images help mitigate the distribution shift issue, while RGB images enhance general feature extraction capabilities. The mixed dataset effectively balances these two aspects. Moreover, \tabref{tab:data_ratio} in the appendix displays the scaling characteristics of pre-training data, demonstrating the necessity of constructing the larger InfMix dataset.


\begin{wrapfigure}{r}{0.25\textwidth}
\vspace{-11pt}
\centering
\includegraphics[width=1.0\linewidth]{figures/layer_select_s_v2.pdf}
\vspace{-22pt}
\caption{The average FT when employing different $s$ in \eqnref{eq:distill}.}
\label{fig:para_s}
\vspace{-15pt}
\end{wrapfigure}
\textbf{Impact of the Hyperparameter $s$.} The parameter $s$ in \eqnref{eq:distill} determines the layer chosen for distillation. As presented in \figref{fig:para_s}, when $s$ ranges from 0.06 to 0.12, the selected layer remains nearly constant: the 18th layer for MAE-L and the 9th layer for DINO-B. Therefore, the performance of UNIP is relatively stable with respect to $s$.


\textbf{Comparison with the Feature Distillation.} As compared in \tabref{tab:target_ablation}, the performance of feature distillation consistently lags behind attention distillation across different layers, implying the latter's superiority. We believe this is because attention distillation only restricts the relationship between tokens, whereas feature distillation imposes direct constraints on each token's features. Excessive constraints on features may intensify the distribution shift and hinder the generalization of distilled models. Additionally, the performance of feature distillation across different layers is similar, likely due to the skip connections in ViT, which enhance feature similarities between layers. In contrast, the attention maps of different layers differ significantly, as revealed in \secref{sec:attention_pattern}.


\begin{wraptable}{r}{0.48\textwidth}
    \vspace{-7.5mm}
    \centering
    \caption{Comparisons of pre-training methods.}
    \label{tab:other_methods}
    \scriptsize
    \setlength{\tabcolsep}{1.0mm}{
    \scalebox{1.0}{
    \begin{tabular}{l c c}
        \toprule
        Method & Avg FT & Training Time (h) \\
        \midrule
        Continual Pre-trained (MAE-S) & 58.53 & 75.0 (1x RTX3090) \\
        UNIP-S (MAE-L distilled) & \textbf{64.37} & \textbf{72.5} (1x RTX3090)  \\
        \bottomrule
    \end{tabular}}}
    \vspace{-4mm}
\end{wraptable}
\textbf{Comparison with Contiunal Pre-training on Target Domain.}
We initialize MAE-S with RGB pre-trained weights and further pre-train it on InfMix for 100 epochs. As shown in \tabref{tab:other_methods}, this continually pre-trained MAE-S (58.53\%) exceeds the RGB pre-trained one (55.39\% in \tabref{tab:main_distill}). However, it still underperforms UNIP-S by 5.84\% and requires more training time, highlighting the efficiency of UNIP over continual pre-training.

\textbf{Impact of the LL-FPN.} \tabref{tab:fpn_ablation} shows the performance of models distilled from various teachers. While LL-FPN enhances performance for all models, the improvements are much greater when using \textit{hybrid} patterns as distillation targets than \textit{local} or \textit{global} patterns. This demonstrates LL-FPN's superiority and good compatibility with the \textit{hybrid} pattern, supporting the analysis in \secref{sec:unip}.


\begin{wraptable}{r}{0.6\textwidth}
    \vspace{-8mm}
    \caption{The LLP performance on RGB and depth datasets. Training epochs are 30 for ADE20K and 100 for others.}
    \label{tab:rgb_llp}
    \vspace{1mm}
    \centering
    \scriptsize
    \setlength{\tabcolsep}{0.8mm}{
    \scalebox{0.93}{
    \begin{tabular}{l c c c c c}
        \toprule
        \multirow{3}{*}{\textcolor{gray}{(Modality)} Dataset} & \multicolumn{2}{c}{DINO-S} & \multicolumn{2}{c}{DeiT-S} \\
        \cmidrule(lr){2-3} \cmidrule(lr){4-5}
        & Layer 9 & Layer 12  & Layer 9 & Layer 12 \\
        & \textcolor{gray}{(\textit{Hybrid})} & \textcolor{gray}{(\textit{Global})} & \textcolor{gray}{(\textit{Hybrid})} & \textcolor{gray}{(\textit{Global})} \\
        \midrule
        \textcolor{gray}{(RGB)} ADE20K \citep{ade20k} & \textbf{26.11} & 23.15 & \textbf{24.35} & 22.68  \\
        \textcolor{gray}{(RGB)} MFNet-RGB \citep{mfnet} & \textbf{38.94} & 37.53  & \textbf{30.43} & 29.44 \\
        \midrule
        \textcolor{gray}{(Depth)} NYUDepthv2 \citep{nyuv2} & \textbf{17.25} & 15.29 & \textbf{5.55} & 5.15 \\
        \textcolor{gray}{(Depth)} SUN-RGBD \citep{sunrgbd} & \textbf{13.17} & 11.41 & \textbf{5.61} & 4.94 \\
        \bottomrule
    \end{tabular}}}
    \vspace{-4mm}
\end{wraptable}

\textbf{Applicability to Other Modalities.} We extend the LLP experiments in \secref{sec:lg_matter} to the RGB and depth modalities. As shown in \tabref{tab:rgb_llp}, for both DINO-S and DeiT-S, the LLP performance of middle layers (the \textit{local} pattern) surpasses that of deep layers (the \textit{global} pattern) across all RGB and depth semantic segmentation datasets. This mirrors the phenomenon in the infrared domain discussed in \secref{sec:lg_matter}, underscoring the importance of \textit{hybrid} patterns for semantic segmentation tasks, regardless of dataset size or modality. Therefore, we believe that UNIP can be effectively extended to other modalities.

\textbf{Visualizations.} We present visualizations of distilled models in the appendix. As shown in \figref{fig:distill_query_attn}\textcolor{red}{c}, the deep layers of UNIP-S exhibit \textit{hybrid} patterns, indicating that UNIP effectively transfers these patterns from the teacher to the student. The CKA alignment between DION-S and UNIP-S in shallow and middle layers, shown in \figref{fig:cka}, further demonstrates this from a feature representation perspective. Additionally, compared to MAE-L, attention maps in UNIP-S focus more on shape information than textures, as evident in \figref{fig:texture_query_attn}. The emergence of \textit{hybrid} patterns in deep layers and the reduced bias towards texture both contribute to the excellent performance of UNIP.




\section{Conclusion and Discussion}
In this work, we comprehensively benchmark the infrared segmentation performance of different pre-training methods and uncover several valuable insights. We further analyze the pre-trained attention maps and identify the importance of \textit{hybrid} patterns for semantic segmentation. Finally, we propose the UNIP framework to improve the performance of small ViT models. Extensive experimental results demonstrate the effectiveness of our dataset and method. UNIP presents a viable approach for selective knowledge distillation in domain transfer settings. We hope our analysis can provide meaningful insights into the characteristics and differences among pre-training methods, ultimately contributing to the advancements of visual pre-training and downstream transfer learning.

\textbf{Limitations and Future Work.} Due to limited computing resources, we validate UNIP's effectiveness only in the infrared domain for semantic segmentation. However, we believe UNIP can be effectively extended to other modalities, such as RGB and depth images, as the superiority of \textit{hybrid} patterns in these modalities is demonstrated in \tabref{tab:rgb_llp}. Exploring its potential in other dense prediction tasks, like object detection and depth estimation, is also worthwhile. Moreover, combining \textit{hybrid} patterns from different pre-trained methods could 
be a promising avenue.




% \subsubsection*{Acknowledgments}
% Use unnumbered third level headings for the acknowledgments. All
% acknowledgments, including those to funding agencies, go at the end of the paper.

\section*{Reproducibility Statement}
Reproducibility is a priority in our research. In this statement, we outline the measures taken to ensure our work can be reproduced.

\textbf{Source Code.} The source code of our work is available at this \href{https://github.com/casiatao/UNIP}{link}. Researchers can access and utilize our code to reproduce the experimental results in this paper. The source code and pre-trained model weights will be made publicly available.

\textbf{Experimental Setup and Details.} In the main paper, the basic experimental configurations are presented in \secref{sec:benchmark_setup} (benchmark) and in \secref{sec:unip_experiments} (UNIP). In \Appref{app:experimental_details}, we provide the detailed settings, including the implementation details of the benchmark (\Appref{app:benchmark_details}) and UNIP (\Appref{app:pre-training}), the comparisons of different evaluation metrics and their hyperparameter settings (\Appref{app:appendix_evaluation}), and the evaluation datasets usage (\Appref{app:evaluation_datasets}).

\textbf{Datasets.} We outline the construction steps of our InfMix dataset in \secref{sec:unip}. In \Appref{app:pretraining_dataset}, we further present more details about the dataset collection and preprocessing.

By highlighting these references, we intend to improve the reproducibility of our work, helping other researchers verify and build on our findings. We're open to any questions or requests for more information about our methods, as we aspire to ensure our research is transparent and reliable.



\bibliography{iclr2025_conference}
\bibliographystyle{iclr2025_conference}

\newpage
\appendix
\section*{Appendix}
In \secref{app:related_work}, we discuss the related works. In \secref{app:relationship}, we review the relationship between our motivations and proposed methods. In \secref{app:experimental_details}, we provide detailed descriptions of the experimental settings, including the complete benchmark results, evaluation metrics and datasets, and the experimental specifics of UNIP. Further analysis is conducted in \secref{app:appendix_analysis}, covering (1) the relationship between NMI and attention patterns, and (2) the CKA analysis of feature representation. Additional experimental results are presented in \secref{app:more_experiments}. Finally, we provide more details of the pre-training dataset in \secref{app:pretraining_dataset} and offer additional visualization results in \secref{app:more_visualization}.

\section{RELATED WORK}
\label{sec:relatedwork}
In this section, we describe the previous works related to our proposal, which are divided into two parts. In Section~\ref{sec:relatedwork_exoplanet}, we present a review of approaches based on machine learning techniques for the detection of planetary transit signals. Section~\ref{sec:relatedwork_attention} provides an account of the approaches based on attention mechanisms applied in Astronomy.\par

\subsection{Exoplanet detection}
\label{sec:relatedwork_exoplanet}
Machine learning methods have achieved great performance for the automatic selection of exoplanet transit signals. One of the earliest applications of machine learning is a model named Autovetter \citep{MCcauliff}, which is a random forest (RF) model based on characteristics derived from Kepler pipeline statistics to classify exoplanet and false positive signals. Then, other studies emerged that also used supervised learning. \cite{mislis2016sidra} also used a RF, but unlike the work by \citet{MCcauliff}, they used simulated light curves and a box least square \citep[BLS;][]{kovacs2002box}-based periodogram to search for transiting exoplanets. \citet{thompson2015machine} proposed a k-nearest neighbors model for Kepler data to determine if a given signal has similarity to known transits. Unsupervised learning techniques were also applied, such as self-organizing maps (SOM), proposed \citet{armstrong2016transit}; which implements an architecture to segment similar light curves. In the same way, \citet{armstrong2018automatic} developed a combination of supervised and unsupervised learning, including RF and SOM models. In general, these approaches require a previous phase of feature engineering for each light curve. \par

%DL is a modern data-driven technology that automatically extracts characteristics, and that has been successful in classification problems from a variety of application domains. The architecture relies on several layers of NNs of simple interconnected units and uses layers to build increasingly complex and useful features by means of linear and non-linear transformation. This family of models is capable of generating increasingly high-level representations \citep{lecun2015deep}.

The application of DL for exoplanetary signal detection has evolved rapidly in recent years and has become very popular in planetary science.  \citet{pearson2018} and \citet{zucker2018shallow} developed CNN-based algorithms that learn from synthetic data to search for exoplanets. Perhaps one of the most successful applications of the DL models in transit detection was that of \citet{Shallue_2018}; who, in collaboration with Google, proposed a CNN named AstroNet that recognizes exoplanet signals in real data from Kepler. AstroNet uses the training set of labelled TCEs from the Autovetter planet candidate catalog of Q1–Q17 data release 24 (DR24) of the Kepler mission \citep{catanzarite2015autovetter}. AstroNet analyses the data in two views: a ``global view'', and ``local view'' \citep{Shallue_2018}. \par


% The global view shows the characteristics of the light curve over an orbital period, and a local view shows the moment at occurring the transit in detail

%different = space-based

Based on AstroNet, researchers have modified the original AstroNet model to rank candidates from different surveys, specifically for Kepler and TESS missions. \citet{ansdell2018scientific} developed a CNN trained on Kepler data, and included for the first time the information on the centroids, showing that the model improves performance considerably. Then, \citet{osborn2020rapid} and \citet{yu2019identifying} also included the centroids information, but in addition, \citet{osborn2020rapid} included information of the stellar and transit parameters. Finally, \citet{rao2021nigraha} proposed a pipeline that includes a new ``half-phase'' view of the transit signal. This half-phase view represents a transit view with a different time and phase. The purpose of this view is to recover any possible secondary eclipse (the object hiding behind the disk of the primary star).


%last pipeline applies a procedure after the prediction of the model to obtain new candidates, this process is carried out through a series of steps that include the evaluation with Discovery and Validation of Exoplanets (DAVE) \citet{kostov2019discovery} that was adapted for the TESS telescope.\par
%



\subsection{Attention mechanisms in astronomy}
\label{sec:relatedwork_attention}
Despite the remarkable success of attention mechanisms in sequential data, few papers have exploited their advantages in astronomy. In particular, there are no models based on attention mechanisms for detecting planets. Below we present a summary of the main applications of this modeling approach to astronomy, based on two points of view; performance and interpretability of the model.\par
%Attention mechanisms have not yet been explored in all sub-areas of astronomy. However, recent works show a successful application of the mechanism.
%performance

The application of attention mechanisms has shown improvements in the performance of some regression and classification tasks compared to previous approaches. One of the first implementations of the attention mechanism was to find gravitational lenses proposed by \citet{thuruthipilly2021finding}. They designed 21 self-attention-based encoder models, where each model was trained separately with 18,000 simulated images, demonstrating that the model based on the Transformer has a better performance and uses fewer trainable parameters compared to CNN. A novel application was proposed by \citet{lin2021galaxy} for the morphological classification of galaxies, who used an architecture derived from the Transformer, named Vision Transformer (VIT) \citep{dosovitskiy2020image}. \citet{lin2021galaxy} demonstrated competitive results compared to CNNs. Another application with successful results was proposed by \citet{zerveas2021transformer}; which first proposed a transformer-based framework for learning unsupervised representations of multivariate time series. Their methodology takes advantage of unlabeled data to train an encoder and extract dense vector representations of time series. Subsequently, they evaluate the model for regression and classification tasks, demonstrating better performance than other state-of-the-art supervised methods, even with data sets with limited samples.

%interpretation
Regarding the interpretability of the model, a recent contribution that analyses the attention maps was presented by \citet{bowles20212}, which explored the use of group-equivariant self-attention for radio astronomy classification. Compared to other approaches, this model analysed the attention maps of the predictions and showed that the mechanism extracts the brightest spots and jets of the radio source more clearly. This indicates that attention maps for prediction interpretation could help experts see patterns that the human eye often misses. \par

In the field of variable stars, \citet{allam2021paying} employed the mechanism for classifying multivariate time series in variable stars. And additionally, \citet{allam2021paying} showed that the activation weights are accommodated according to the variation in brightness of the star, achieving a more interpretable model. And finally, related to the TESS telescope, \citet{morvan2022don} proposed a model that removes the noise from the light curves through the distribution of attention weights. \citet{morvan2022don} showed that the use of the attention mechanism is excellent for removing noise and outliers in time series datasets compared with other approaches. In addition, the use of attention maps allowed them to show the representations learned from the model. \par

Recent attention mechanism approaches in astronomy demonstrate comparable results with earlier approaches, such as CNNs. At the same time, they offer interpretability of their results, which allows a post-prediction analysis. \par


\section{Relationships between motivations and methods}
\label{app:relationship}

Our primary motivation is to enhance the performance of models on infrared semantic segmentation tasks. From the model perspective, factors affecting the performance on specific tasks include not only the design of the model architecture but also the quality of the model's pre-training. Previous works \citep{soda, tinn} have aimed to improve performance by designing specific network architectures for infrared semantic segmentation tasks. However, in the infrared domain, where labeled data is limited, the quality of the pre-trained model is also crucial. \textbf{Therefore, our work explores an alternative approach by emphasizing the optimization of pre-trained models specifically for infrared semantic segmentation tasks to enhance performance.} To facilitate this exploration, our work is organized into three stages: benchmark establishment, cause analysis, and method proposal, as illustrated in \figref{fig:architecture}.

\textbf{Benchmark Establishment (\secref{sec:benchmark}).} We establish a benchmark for the transfer performance of six RGB pre-training methods, encompassing a total of 18 pre-trained models, on three infrared semantic segmentation datasets (\secref{sec:benchmark_setup}). Our findings reveal several key phenomena (\secref{sec:benchmark_multi_layer}), such as the lack of correlation between model performance on ImageNet and infrared segmentation datasets (\tabref{tab:correlation_coeff}), and the superior generalization of supervised and contrastive learning methods over masked image modeling methods in the context of infrared segmentation tasks (\figref{fig:benchmark}). 

\textbf{Cause Analysis (\secref{sec:investigation}).} To analyze the performance discrepancies among various pre-training methods in infrared segmentation tasks, we conduct an in-depth analysis of the attention maps from the pre-trained models. Our findings indicate that the degree of focus on local and global information (\secref{sec:attention_pattern} - \secref{sec:lg_matter}), as well as on shape and texture information (\secref{sec:texture}), significantly impacts the performance of infrared semantic segmentation tasks. We further validate through corresponding experiments that the existence of hybrid attention patterns (\figref{fig:query_attn} - \figref{fig:layerwise_lp}) and the reduced bias towards texture (\tabref{tab:rgb_infrared}) both play crucial roles in enhancing the performance of pre-trained models in infrared segmentation tasks.

\textbf{Method Proposal (\secref{sec:unip}).} Based on the observations and analyses from the previous two sections, we propose UNIP, a framework designed to improve the infrared segmentation performance of pre-trained models through three key aspects: the pre-training objective (NMI-HAD), the pre-training data (InfMix), and the fine-tuning architecture (LL-FPN). Both NMI-HAD and LL-FPN enhance performance by effectively leveraging hybrid attention patterns, while InfMix enhances performance by reducing the pre-trained model's bias toward texture information. \textbf{Importantly, our approach does not alter the structure of the backbone model or the decoder; instead, we focus on targeted pre-training specifically designed for infrared segmentation tasks. We believe this is one way in which the proposed method is specific to infrared semantic segmentation tasks.} As a result, our pre-trained models significantly outperform RGB pre-trained models of comparable or even larger sizes in infrared semantic segmentation tasks (\tabref{tab:main_distill}), and also achieve superior performance compared to other models specifically designed for infrared segmentation (\tabref{tab:sota_comparison}).
\section{Experimental Details}
\label{app:experimental_details}
\subsection{Benchmark details}
\label{app:benchmark_details}
\begin{table}[t]
    \centering
    \caption{The performance of different pre-trained models on ImageNet and infrared semantic segmentation datasets. The \textit{Scratch} means the performance of randomly initialized models. The \textit{PT Epochs} denotes the pre-training epochs while the \textit{IN1K FT epochs} represents the fine-tuning epochs on ImageNet \citep{imagenet}. $^\dag$ denotes models reproduced using official codes. $^\star$ refers to the effective epochs used in \citet{iBOT}. The top two results are marked in \textbf{bold} and \underline{underlined} format. Supervised and CL methods, MIM methods, and UNIP models are colored in \colorbox{orange!15}{\rule[-0.2ex]{0pt}{1.5ex}orange}, \colorbox{gray!15}{\rule[-0.2ex]{0pt}{1.5ex}gray}, and \colorbox{cyan!15}{\rule[-0.2ex]{0pt}{1.5ex}cyan}, respectively.}
    \label{tab:benchmark}
    \centering
    \scriptsize
    \setlength{\tabcolsep}{1.0mm}{
    \scalebox{1.0}{
    \begin{tabular}{l c c c c  c c c c c c c c}
        \toprule
         \multirow{2}{*}{Methods} & \multirow{2}{*}{\makecell[c]{PT \\ Epochs}} & \multicolumn{2}{c}{IN1K FT} & \multicolumn{4}{c}{Fine-tuning (FT)} & \multicolumn{4}{c}{Linear Probing (LP)} \\
         \cmidrule{3-4} \cmidrule(lr){5-8} \cmidrule(lr){9-12} 
         & & Epochs & Acc & SODA & MFNet-T & SCUT-Seg & Mean & SODA & MFNet-T & SCUT-Seg & Mean \\
         \midrule
         \textcolor{gray}{ViT-Tiny/16} & & &  & & & & & & & & \\
         Scratch & - & - & - & 31.34 & 19.50 & 41.09 & 30.64 & - & - & - & - \\
         \rowcolor{gray!15} MAE$^\dag$ \citep{mae} & 800 & 200 & \underline{71.8} & 52.85 & 35.93 & 51.31 & 46.70 & 23.75 & 15.79 & 27.18 & 22.24 \\
         \rowcolor{orange!15} DeiT \citep{deit} & 300 & - & \textbf{72.2} & 63.14 & 44.60 & 61.36 & 56.37 & 42.29 & 21.78 & 31.96 & 32.01 \\
         \rowcolor{cyan!15} UNIP (MAE-L) & 100 & - & - & \underline{64.83} & \textbf{48.77} & \underline{67.22} & \underline{60.27} & \underline{44.12} & \underline{28.26} & \underline{35.09} & \underline{35.82} \\
         \rowcolor{cyan!15} UNIP (iBOT-L) & 100 & - & - & \textbf{65.54} & \underline{48.45} & \textbf{67.73} & \textbf{60.57} & \textbf{52.95} & \textbf{30.10} & \textbf{40.12} & \textbf{41.06}  \\
         \midrule
         \textcolor{gray}{ViT-Small/16} & & & & & & & & & & & \\
         Scratch & - & - & - & 41.70 & 22.49 & 46.28 & 36.82 & - & - & - & - \\
         \rowcolor{gray!15} MAE$^\dag$ \citep{mae} & 800 & 200 & 80.0 & 63.36 & 42.44 & 60.38 & 55.39 & 38.17 & 21.14 & 34.15 & 31.15 \\
         \rowcolor{gray!15} CrossMAE \citep{crossmae} & 800 & 200 & 80.5 & 63.95 & 43.99 & 63.53 & 57.16 & 39.40 & 23.87 & 34.01 & 32.43 \\
         \rowcolor{orange!15} DeiT \citep{deit} & 300 & - & 79.9 & 68.08 & 45.91 & 66.17 & 60.05 & 44.88 & 28.53 & 38.92 & 37.44 \\
         \rowcolor{orange!15} DeiT III \citep{deit3} & 800 & - & 81.4 & 69.35 & 47.73 & 67.32 & 61.47 & 54.17 & 32.01 & 43.54 & 43.24 \\
         \rowcolor{orange!15} DINO \citep{dino} & 3200$^\star$ & 200 & \underline{82.0} & 68.56 & 47.98 & 68.74 & 61.76 & 56.02 & 32.94 & 45.94 & 44.97 \\
         \rowcolor{orange!15} iBOT \citep{iBOT} & 3200$^\star$ & 200 & \textbf{82.3} & 69.33 & 47.15 & 69.80 & 62.09 & 57.10 & 33.87 & 45.82 & 45.60 \\
         \rowcolor{cyan!15} UNIP (DINO-B) & 100 & - & - & 69.35 & 49.95 & 69.70 & 63.00 & \underline{57.76} & \underline{34.15} & \underline{46.37} & \underline{46.09} \\
         \rowcolor{cyan!15} UNIP (MAE-L) & 100 & - & - & \textbf{70.99} & \underline{51.32} & \underline{70.79} & \underline{64.37} & 55.25 & 33.49 & 43.37 & 44.04 \\
         \rowcolor{cyan!15} UNIP (iBOT-L) & 100 & - & - & \underline{70.75} & \textbf{51.81} & \textbf{71.55} & \textbf{64.70} & \textbf{60.28} & \textbf{37.16} & \textbf{47.68} & \textbf{48.37} \\ 
        \midrule
        \textcolor{gray}{ViT-Base/16} & & & & & & & & & & & \\
        Scratch & - & - & - & 44.25 & 23.72 & 49.44 & 39.14 & - & - & - & - \\
        \rowcolor{gray!15} MAE \citep{mae} & 1600 & 100 & 83.6 & 68.18 & 46.78 & 67.86 & 60.94 & 43.01 & 23.42 & 37.48 & 34.64 \\
        \rowcolor{gray!15} CrossMAE \citep{crossmae} & 800 & 100 & 83.7 & 68.29 & 47.85 & 68.39 & 61.51 & 43.35 & 26.03 & 38.36 & 35.91 \\
        \rowcolor{orange!15} DeiT \citep{deit} & 300 & - & 81.8 & 69.73 & 48.59 & 69.35 & 62.56 & 57.40 & 34.82 & 46.44 & 46.22 \\
        \rowcolor{orange!15} DeiT III \citep{deit3} & 800 & 20 & \underline{83.8} & 71.09 & 49.62 & 70.19 & 63.63 & 59.01 & \underline{35.34} & 48.01 & 47.45 \\
        \rowcolor{orange!15} DINO \citep{dino} & 1600$^\star$ & 100 & 83.6 & 69.79 & 48.54 & 69.82 & 62.72 & 59.33 & 34.86 & 47.23 & 47.14 \\
        \rowcolor{orange!15} iBOT \citep{iBOT} & 1600$^\star$ & 100 & \textbf{84.0} & 71.15 & 48.98 & 71.26 & 63.80 & \underline{60.05} & 34.34 & \underline{49.12} & \underline{47.84} \\
        \rowcolor{cyan!15} UNIP (MAE-L) & 100 & - & - & \underline{71.47} & \textbf{52.55} & \underline{71.82} & \textbf{65.28} & 58.82 & 34.75 & 48.74 & 47.43 \\
        \rowcolor{cyan!15} UNIP (iBOT-L) & 100 & - & - & \textbf{71.75} & \underline{51.46} & \textbf{72.00} & \underline{65.07} & \textbf{63.14} & \textbf{39.08} & \textbf{52.53} & \textbf{51.58} \\
        \midrule
        \textcolor{gray}{ViT-Large/16} & & & & & & & & & & & \\
        Scratch & - & - & - & 44.70 & 23.68 & 49.55 & 39.31 & - & - & - & - \\
        \rowcolor{gray!15} MAE \citep{mae} & 1600 & 50 & \textbf{85.9} & 71.04 & \underline{51.17} & 70.83 & 64.35 & 52.20 & 31.21 & 43.71 & 42.37 \\
        \rowcolor{gray!15} CrossMAE \citep{crossmae} & 800 & 50 & 85.4 & 70.48 & 50.97 & 70.24 & 63.90 & 53.29 & 33.09 & 45.01 & 43.80 \\
        \rowcolor{orange!15} DeiT3 \citep{deit3} & 800 & 20 & \underline{84.9} & \underline{71.67} & 50.78 & \textbf{71.54} & \underline{64.66} & \underline{59.42} & \textbf{37.57} & \textbf{50.27} & \underline{49.09} \\
        \rowcolor{orange!15} iBOT \citep{iBOT} & 1000$^\star$ & 50 & 84.8 & \textbf{71.75} & \textbf{51.66} & \underline{71.49} & \textbf{64.97} & \textbf{61.73} & \underline{36.68} & \underline{50.12} & \textbf{49.51} \\
        \bottomrule
    \end{tabular}}}
    \vspace{-2mm}
\end{table}

\textbf{Reproduction of small MAE models.} The MAE-T and MAE-S are reproduced following the settings in \citet{mae}. We make several adjustments to the decoder to make it suitable for small encoders. For both MAE-S and MAE-T, the decoder includes 8 transformer blocks, each with 8 attention heads. The decoder dimensions in MAE-S and MAE-T are 256 and 192, respectively.

\textbf{Implementation details.} The weights of all pre-trained models are downloaded from corresponding official repositories. The models are trained for 100 epochs using MMSegmentation \citep{mmseg}. For different methods and model sizes, we keep the learning rate constant and sweep the layerwise decay rate across \{0.5, 0.65, 0.75, 0.85, 1.0\}. To adapt models pre-trained on three-channel RGB images for single-channel infrared images, we duplicate the infrared images three times to create pseudo-three-channel images.

\subsection{Evaluation metrics}
\label{app:appendix_evaluation}

\textbf{Fine-tuning.} \textit{Fine-tuning} is the default evaluation metric in this work, which utilizes the pre-trained model as the backbone of existing semantic segmentation models. Following previous works \citep{mae, iBOT}, we employ UperNet \citep{upernet} as the semantic segmentation model. As illustrated in \figref{fig:ft_lp}\textcolor{red}{a}, to build the feature pyramid based on the non-hierarchical ViT model, features from different layers are passed through the MaxPooling layers or DeConv layers, to obtain features of different resolutions. These multi-scale features are then input into the decoder for segmentation results. Following \citet{mae} and \citet{iBOT}, we use features of the \{4, 6, 8, 12\} layers in ViT-T, ViT-S, and ViT-B, and the features of the \{8, 12, 16, 24\} layers in ViT-L, to build the feature pyramid. Remarkably, in fine-tuning, all parameters including the pre-trained model, the feature pyramid, and the decoder, are tuned with the labeled downstream datasets. Hyperparameters are listed in \tabref{tab:setting_seg}.

\textbf{Linear Probing.} As mentioned above, \textit{fine-tuning} introduces additional learnable parameters and alters the pre-trained feature representation. Its performance may not fully reflect the characteristics of the pre-trained features. Therefore, \textit{linear probing} is also employed as an evaluation metric. As shown in \figref{fig:ft_lp}\textcolor{red}{b}, features from different layers are resized to $1/4$ of the input resolution and then concatenated together. Finally, a linear head ($1\times1$ conv) utilizes these concatenated features to predict segmentation results. Notably, only the linear head is trainable, while all other parameters are frozen. The layer settings of output features are the same as \textit{fine-tuning}.

\textbf{Fine-tuning (LL-FPN).} This metric is discussed in \secref{sec:distill}, which aims to enhance the fine-tuning performance of UNIP models by using the last layer to obtain features of different resolutions, as depicted in \figref{fig:ft_lp}\textcolor{red}{c}. Specifically, we employ the features of the \{12, 12, 12, 12\} layers in ViT-T, ViT-S, and ViT-B, and the features of the \{24, 24, 24, 24\} layers in ViT-L, to build the feature pyramid. Other settings remain the same as \textit{fine-tuning}.

\begin{figure}[t]
\centering
\includegraphics[width=\linewidth]{figures/ft_lp_v7.pdf}
\vspace{-20pt}
\caption{Illustrations of different transfer architectures for semantic segmentation tasks.}
\label{fig:ft_lp}
\vspace{-8pt}
\end{figure}

\textbf{Layerwise Linear Probing.} This metric is a layerwise version of the \textit{linear probing} metric. It is designed to assess the pre-trained feature representation at each layer. As shown in \figref{fig:ft_lp}\textcolor{red}{d}, only the features of a single layer are forwarded to the linear head following the resize operation. Other settings are the same as \textit{linear probing}.

\begin{table}[t]
    \caption{Settings of semantic segmentation.}
    \label{tab:setting_seg}
    \centering
    \footnotesize
    \setlength{\tabcolsep}{1.5mm}{
    \scalebox{1.0}{
    \begin{tabular}{l  c c c}
         \toprule
         Hyperparameters & SODA & MFNet-T & SCUT-seg \\
         \midrule
         Input resolution & 512 $\times$ 512 & 512 $\times$ 512 & 512 $\times$ 512 \\
         Training epochs & 100 / 200 & 100 / 300 & 100 \\
         Training iterations  &  14400 / 28800  &  9800 / 29400  &  16800 \\
         Peak learning rate  & 1e-4 & 1e-4 & 1e-4 \\
         Batch size  & 8 & 8 & 8  \\
         Optimizer  & AdamW & AdamW & AdamW \\
         Weight decay  & 0.05 & 0.05 & 0.05  \\
         Optimizer momentum  & $\beta_1, \beta_2=$ 0.9, 0.999 & $\beta_1, \beta_2=$ 0.9, 0.999 & $\beta_1, \beta_2=$ 0.9, 0.999 \\
         Learning rate schedule  & Linear decay & Linear decay & Linear decay \\
         Minimal learning rate & 0 & 0 & 0 \\
         Warmup steps  & 1500 / 3000 & 1000 / 3000 & 1700 \\
         \bottomrule
    \end{tabular}}}
    \vspace{-2mm}
\end{table}


\subsection{Evaluation datasets}
\label{app:evaluation_datasets}

\textbf{SODA} \citep{soda}. This dataset features a variety of indoor and outdoor scenes. It comprises 1,168 training images and 1,000 test images, spanning 20 distinct semantic categories, including road, building, car, chair, lamp, table, monitor, and others.

\textbf{MFNet} \citep{mfnet}. This dataset focuses on RGBT semantic segmentation for automotive driving scenarios and includes 1,569 image pairs of infrared and RGB images. It is divided into 784 training images, 392 validation images, and 393 test images, covering 8 semantic categories such as car, person, bike, curve, and others. When benchmarking the performance of different pre-training methods, we combine the validation set with the test set, resulting in a larger test set of 785 images. When comparing UNIP models with other SOTA semantic segmentation models, we follow their settings, \ie, using the original 393 test images for evaluation.

\textbf{SCUT-Seg} \citep{mcnet}. This dataset includes 1345 training images and 665 test images in nighttime driving scenes. It has 10 classes including road, person, fence, pole, and others.

\textbf{ADE20K} \citep{ade20k}. ADE20K is a large-scale RGB semantic segmentation dataset, covering a variety of scenes from indoor to outdoor and nature to urban. It consists of 20,210 training images and 2,000 test images, with 150 different semantic categories.

\textbf{ImageNet-1K} \citep{imagenet}. ImageNet-1K is a subset of the ImageNet database, consisting of 1,000 categories with roughly 1.2 million training images, 50,000 validation images, and 100,000 test images. It is widely used in computer vision research like image classification and pre-training.

\subsection{UNIP.}
\label{app:pre-training}

\begin{table}[t]
  \begin{minipage}{0.47\linewidth}
  \centering
    \caption{Configurations of ViT for semantic segmentation tasks.}
    \label{tab:vit_config}
    \centering
    \footnotesize
    \setlength{\tabcolsep}{1.2mm}{
    \scalebox{1.0}{
    \begin{tabular}{l c c c}
        \toprule
        Model & Dimension & Head Num & Depth \\
        \midrule
        ViT-T & 192 & 3 & 12 \\
        ViT-S & 384 & 6 & 12 \\
        ViT-B & 768 & 12 & 12 \\
        ViT-L & 1024 & 16 & 24 \\
        \bottomrule
    \end{tabular}}}
    \vspace{-4mm}
  \end{minipage}
  \hfill
  \begin{minipage}{0.52\linewidth}
  \caption{Settings of pre-training.}
    \label{tab:distill_setting}
    \centering
    \footnotesize
    \setlength{\tabcolsep}{1.5mm}{
    \scalebox{1.0}{
    \begin{tabular}{l c}
        \toprule
        Hyperparameters  &  Value \\
        \midrule
        Input resolution  &  224 $\times$  224 \\
        Training epochs &  100 \\
        Warmup epochs  &  5  \\
        Optimizer  &  AdamW  \\
        Base learning rate  & 1e-4  \\
        Weight decay  & 0.05  \\
        Optimizer momentum  &  $\beta_1, \beta_2=$ 0.9, 0.95 \\
        Batch size  &  4096 \\
        Learning rate schedule  &  Cosine decay  \\
        Augmentation & Random resized cropping \& \\
        & Random horizontal flipping \\
        \bottomrule
    \end{tabular}}}
    \vspace{-4mm}
  \end{minipage}
\end{table}





\textbf{Head Misalignment.} To solve the head misalignment between teacher and student models during distillation, we experiment with two methods. (1) The first method is the adaptive block proposed in  \citet{tinymim}. Specifically, during distillation, the number of attention heads in the student model's last layer is adjusted to be the same as that of the teacher model by changing the head dimension while keeping the overall dimension constant. When performing fine-tuning or linear probing on downstream tasks, the number of attention heads is reverted to the standard setting in \tabref{tab:vit_config}. (2) The second method involves adding a self-attention layer at the end of the student model during distillation. The number of attention heads in the extra attention layer is equivalent to the teacher model's. This layer is removed when transferring to downstream tasks. These two methods achieve similar performance, but the latter consumes slightly more training time. Therefore, we use the first method in practice.

\textbf{Feature Distillation.} For the feature distillation in \tabref{tab:target_ablation}, we employ a linear projection layer to match the dimension of the student model to that of the teacher model. The distillation and fine-tuning settings are the same as UNIP. The loss function is the cosine similarity loss between the $L_2$ normalized student feature $l_2(F_T)$ and teacher feature $l_2(F_S)$: 
\begin{align}
    L=1-\cos(l_2(F_T)\cdot l_2(F_S)).
\end{align}

\textbf{Implementation Details.} All experiments are conducted using the PyTorch toolkit \citep{pytorch} on 8 NVIDIA RTX 3090 GPUs. The default settings are shown in \tabref{tab:distill_setting}. We use the linear \textit{learning rate} scaling rule: $lr=base\_lr \times$ batchsize / 256, following \citet{mae}. The semantic segmentation settings of UNIP models are the same as those in \Appref{app:appendix_evaluation}.






\section{Additional analysis}
\label{app:appendix_analysis}

\subsection{Normalized Mutual Information}
\label{app:NMI}
The Normalized Mutual Information (NMI) is employed in \secref{sec:nmi} to measure the attention patterns. Let $p(q_i)$ denote the marginal probability of the $i$-th query token and $p(k_j)$ denote the marginal probability of the $j$-th key token. Since query tokens are evenly distributed across every spatial coordinate, $p(q_i)$ can be formulated as:
\begin{equation}
    p(q_i)=\frac{1}{N}, \quad i=1,2,...,N.
    \label{eq:marginal_q}
\end{equation}
Assume $A^m\in \mathbb{R}^{N\times N}$ represents the $m$-th head of the attention matrix after the softmax operation without the \textit{class} token, where $N$ is the number of spatial tokens. The attention scores from each query token to all key tokens sum to 1, \ie, $\sum_{j=1}^N A_{i,j}^m=1, i=1,2,...,N. $
Thus, each row of $A$ can be viewed as the conditional probability distribution of key tokens given the query token:
\begin{equation}
    p(k_j\vert q_i)=A_{i,j}^m.
\end{equation}
Then the joint probability of $q_i$ and $k_j$ can be calculated as:
\begin{equation}
    p(q_i,k_j)=p(k_j\vert q_i)p(q_i)=\frac{1}{N}A_{i,j}^m.
    \label{eq:joint_qk}
\end{equation}
The marginal probability of $k_j$ is:
\begin{equation}
    p(k_j)=\sum_{i=1}^Np(q_i,k_j)=\frac{1}{N}\sum_{i=1}^NA_{i,j}^m.
    \label{eq:marginal_k}
\end{equation}
The mutual information of query and key tokens can be formulated as:
\begin{equation}
\begin{aligned}
    I^m(Q;K)&=\sum_{i=1}^N\sum_{j=1}^Np(q_i,k_j)\log{\frac{p(q_i,k_j)}{p(q_i)p(k_j)}} \\
    &=\sum_{i=1}^N\sum_{j=1}^N\frac{1}{N}A_{i,j}^m\log{\frac{NA_{i,j}^m}{\sum_{i=1}^NA_{i,j}^m}}.
\end{aligned}
\end{equation}
The entropy of query and key tokens can be calculated as:
\begin{gather}
    H^m(Q)=-\sum_{i=1}^N p(q_i)\log{p(q_i)}=-\sum_{i=1}^N\frac{1}{N}\log{\frac{1}{N}}, \\
    H^m(K)=-\sum_{i=1}^N p(k_j)\log{p(k_j)}=-\sum_{j=1}^N\left(\frac{1}{N}\sum_{i=1}^NA_{i,j}^m\log{\frac{1}{N}\sum_{i=1}^NA_{i,j}^m}\right).
\end{gather}
Therefore, the NMI of the $m$ head is:
\begin{equation}
    \text{NMI}^m(Q;K)=\frac{I^m(Q;K)}{\sqrt{H^m(Q)H^m(K)}}.
\end{equation}
The final NMI is calculated by averaging on all heads:
\begin{equation}
    \text{NMI}(Q;K)=\frac{1}{M}\sum_{m=1}^M \text{NMI}^m(Q;K).
\end{equation}

The value of NMI ranges from 0 to 1. It reaches the maximum value of 1 when the joint probability of the query and key tokens is the same as their marginal probability:
\begin{equation}
    p(q_i,k_i)=p(q_i)=p(k_i), \quad i=1,2,...,N.
    \label{eq:nmi_1}
\end{equation}
According to \eqnref{eq:marginal_q}, \eqnref{eq:joint_qk}, \eqnref{eq:marginal_k}, and \eqnref{eq:nmi_1}, it can be derived that
\begin{equation}
A^m_{i,j}=
\begin{cases} 
1, & \mbox{if }i=j, \\
0, & \mbox{if }i\neq j,
\end{cases}
\end{equation}
which implies that the attention matrix of each head is an identity matrix. This indicates that each query token focuses only on the key token at the same spatial position, which is a particular case of the \textit{local} attention pattern.

On the other hand, the NMI has a value of 0 when the query and key tokens are independent:
\begin{equation}
    p(q_i,k_j)=p(q_i)p(k_j), \quad i=1,2,...,N, j=1,2,...,N.
    \label{eq:nmi_0}
\end{equation}
According to \eqnref{eq:marginal_q}, \eqnref{eq:joint_qk}, \eqnref{eq:marginal_k}, and \eqnref{eq:nmi_0}, we can derive that
\begin{equation}
    A^m_{i,j}=A^m_{k,j}, \quad i=1,2,...,N, k=1,2,...,N, j=1,2,...N,
\end{equation}
which indicates that every row of the attention matrix is the same. This means that each query token has the same attention maps for all key tokens, which is a particular case of the \textit{global} attention pattern.
\textbf{Therefore, a higher NMI value indicates a stronger relationship between the query and key tokens and a more local attention pattern. Conversely, a lower NMI value means that different query tokens have more similar and global attention patterns for key tokens.}




\subsection{Centered Kernel Alignment}
\label{app:CKA}
In this section, we extend the analysis in \secref{sec:attention_pattern} from the attention pattern to the feature representation. Let $x^l$ denote the input features of the $l$-th block of the ViT model. The features of the next block $x^{l+1}$ can be formulated as:
\begin{align}
    &x_{tmp} = x^l + \text{Attention}(\text{LN}(x^l), \\
    &x^{l+1} = x_{tmp} + \text{FFN}(\text{LN}(x_{tmp}),
\end{align}
where $\text{Attention},\text{FFN},\text{LN}$ refer to the self-attention module, the feedforward module, and the LayerNorm layer, respectively. Obviously, the self-attention module plays a crucial role in transforming the feature representation. The \textit{global} attention pattern will bring the features of different tokens closer since different query tokens interact similarly with all key tokens. In contrast, the \textit{local} attention pattern will make the features of different tokens further apart.

\begin{figure}[t]
\centering
\includegraphics[width=\linewidth]{figures/cka_v3.pdf}
\vspace{-22pt}
\caption{CKA representation analysis of different models. UNIP-S aligns well with DINO-S in the shallow and middle layers, indicating that the \textit{hybrid} patterns are effectively distilled from MAE-L.}
\label{fig:cka}
\vspace{-15pt}
\end{figure}

To investigate the relationships between features of different layers and models, we use the centered kernel alignment (CKA), a metric that measures the similarity between two feature maps. The details of CKA can refer to \citet{CKA}. As shown in \figref{fig:cka}, features in the later layers of MAE-S, \eg, the 10th and 11th layers, are similar to features in the shallow layers of DINO-S, \eg, the 4th, 5th, and 6th layer, implying that the features of MAE-S are relatively lower level compared to DINO-S. \textbf{This is consistent with observations in \secref{sec:attention_pattern} that the \textit{local} attention patterns are distributed in the shallow layers of DINO-S, but are present in all layers of MAE.}

For MAE-L, the features in the middle layers (13th to 20th) exhibit high similarity with the middle-layer features of DINO-S (9th and 10th), due to the \textit{hybrid} patterns in these layers. On the contrary, the features in the later layers (the 22nd and 23rd layers) gradually resemble the shallow layers of DINO-S, which can be attributed to the \textit{local} patterns in the later layers of MAE-L.

It is noteworthy that the UNIP model effectively imitates the features in the middle layers of MAE-L. Its features align more closely with DINO-S than those of MAE-S, especially in the shallow and middle layers, demonstrating that attention distillation can implicitly change the features of distilled models like what feature distillation explicitly does.


\section{Overhead of \ourSystem.}
We report the size and inference time of the model for NeRF and \ourSystem in Table~\ref{table_overhead}.  
The results indicate that \ourSystem has a larger model size than NeRF, \ie 27.1 \vs 8.0\,MB.  
Correspondingly, \ourSystem exhibits a longer inference time, \ie 1.79 \vs 0.43\,s.  
Unlike NeRF, \ourSystem requires neighboring spectra as input. 
During inference, the target transmitter's neighbors are extracted from the training dataset, so \ourSystem does not incur additional data burdens.  
Moreover, since \ourSystem can operate in unseen scenes, it significantly reduces the requirement for a time-consuming training process.



\begin{table}[h]
\centering
\caption{Comparison of model size and inference time.}

\begin{tabular}{lC{0.8in}C{0.8in}}
\toprule
     & \nerft    & \ourSystem    
     \\ \midrule
Model size (MB) & 8.0  & 27.1   \\
Inference time (s) & 0.43    & 1.79  \\
\bottomrule
\end{tabular}
\label{table_overhead}
\end{table}





\section{Pre-training Dataset.}
\label{app:pretraining_dataset}
\subsection{The InfPre dataset.}
\label{app:infpre}

The InfPre dataset is constructed by collecting images from 23 infrared-related visual datasets. The details of the extracted datasets are presented in \tabref{tab:infpre}. To reduce the redundancy in images with similar backgrounds, we employ two sampling methods: fixed-interval sampling and similarity-based sampling. For datasets containing diverse image sequences with different backgrounds, frames are sampled at fixed intervals (\eg 2, 5, and 10) within each sequence. For datasets captured in the same location, we only sample frames that are less similar to each other. The cosine similarity of image embeddings extracted by DINO-B is used as the similarity metric. Images with high similarity to those already sampled images will be discarded. The Faiss \citep{faiss} library is utilized to accelerate the sampling process.

\begin{table}[t]
    \centering
    \caption{Details of the InfPre dataset. \#Image and \#Extraced image represent the number of original and extracted images from the dataset. Interval and Similarity denote the fixed-interval and similarity-based sampling methods, respectively. The value after the slash indicates the fixed interval or similarity threshold.}
    \label{tab:infpre}
    \scriptsize
    \setlength{\tabcolsep}{1.0mm}{
    \scalebox{0.97}{
    \begin{tabular}{l c c c c c c c}
        \toprule
        Dataset & Task & Scenario & \#Image & \makecell[c]{\#Extracted \\ Image} & \makecell[c]{Average \\ Width} & \makecell[c]{Average \\ Height} & Sampling \\
        \midrule
        RGBT-CC \citep{rgbt_cc} & Crowd Counting & Urban & 2,030 & 2,030 & 636 & 484 & - \\
        KAIST \citep{kaist} & Object Detection & Driving & 95,328 & 9,546 & 640 & 512 & Interval / 10 \\
        Infrared City \citep{infrared_city} & Video Translation & Driving & 200,000+ & 20,187 & 256 & 256 & Interval / 10\\
        CVC-09 \citep{cvc9} & Object Detection & Driving & 13,184 & 13,184 & 640 & 480 & - \\
        CVC-14 \citep{cvc14} & Object Detection & Driving & 8518 & 8518 & 640 & 471 & - \\
        VAP \citep{vap} & Semantic Segmentation & Indoors & 23,080 & 2,309 & 640 & 480 & Interval / 10 \\
        RGBT-234 \citep{rgbt-234} & Object Tracking & Surveillance & 117,612 & 11,762 & 628 & 459 & Interval / 10 \\
        LTD \citep{ltd} & Concept Drift & Surveillance & 26,820,000 & 15,749 & 384 & 288 & Similarity / 0.95 \\
        Rain \citep{rainsnow} & Semantic Segmentation & Surveillance & 130,800 & 25,920 & 640 & 480 & Interval / 5 \\
        Infrared Security \citep{security} & Object Detection & Surveillance & 8,999 & 8,999 & 495 & 386 & - \\
        LLVIP \citep{llvip} & Object Detection & Surveillance & 15,488 & 15,488 & 1,280 & 1,024 & - \\
        LSOTB-TIR \citep{lsotb} & Object Tracking & Diverse & 600,000+ & 61,154 & 925 & 623 & Interval / 10 \\
        Dual-Sensor \citep{dual_sensor} & - & Driving & 73,638 & 14,728 & 384 & 288 & Interval / 5 \\
        LasHeR \citep{lasher} & Object Tracking & Diverse & 740,000+ & 74,035 & 879 & 554 & Interval / 10 \\
        VT5000 \citep{vt5000} & Salient Object Detection & Diverse & 5,000 & 5,000 & 640 & 480 & - \\
        Infrared Vehicle \citep{infrared_vehicle} & Object Detection & Driving & 13166 & 13,166 & 815 & 613 & - \\
        Infrared Ship \citep{ship} & Object Detection & Marine & 9,402 & 9,402 & 772 & 591 & - \\
        DroneVehicle \citep{dronevehicle} & Object Detection & Aerial & 28,439 & 28,439 & 640 & 512 & - \\
        Infrared Aerial \citep{aerial} & Object Detection & Aerial & 11,045 & 11,045 & 627 & 502 & - \\
        VTUAV \citep{vtuav} & Object Tracking & Aerial & 1,700,000 & 166,986 & 1,920 & 1,080 & Interval / 10 \\
        M3FD \citep{m3fd} & Object Detection & Driving & 4,200 & 4,200 & 1,00,1 & 744 & - \\
        OTCBVS IRIS Face \citep{face} & - & Human Face & 4,199 & 4,199 & 320 & 240 & - \\
        Multispectral \citep{multispectral_det} & Object Detection & Driving & 15,042 & 15,042 & 480 & 368 & - \\
        \midrule
        InfPre & Pre-training & Diverse & - & 541,088 & 1,075 & 686 & - \\ 
        \bottomrule
    \end{tabular}}}
    \vspace{-3mm}
\end{table}

\subsection{The InfMix dataset.}


\begin{table}[t]
    \centering
    \caption{The cosine similarity between pre-training and infrared segmentation datasets. The embeddings of images are extracted by DINO-B. The similarity is averaged over all pairwise images from different datasets.}
    \label{tab:domain_gap}
    \vspace{1mm}
    \scriptsize
    \setlength{\tabcolsep}{1.5mm}{
    \scalebox{1.1}{
    \begin{tabular}{l c c c c}
        \toprule
        \multirow{2}{*}{Pre-training dataset} & \multicolumn{4}{c}{Downstream dataset} \\
        \cmidrule(lr){2-5} 
        & SODA & MFNet-T & SCUT-Seg & Mean \\
        \midrule
        ImageNet-1K \citep{imagenet} & 0.083 & 0.074 & 0.081 & 0.079 \\
        COCO \citep{coco} & 0.111 & 0.101 & 0.106 & 0.106 \\
        InfMix & 0.200 & 0.227 & 0.236 & 0.221 \\
        InfMix (gray) & \textbf{0.216} & \textbf{0.246} & \textbf{0.254} & \textbf{0.239} \\
        \bottomrule
    \end{tabular}}}
    \vspace{-3mm}
\end{table}

The InfMix dataset combines the InfPre, the subset of ImageNet-1k \citep{imagenet}, and the training set of COCO \citep{coco}, totaling 859,375 images. \tabref{tab:domain_gap} compares the similarity between various pre-training datasets and three infrared segmentation datasets used in our benchmark. Notably, compared to RGB datasets like ImageNet-1k and COCO, the mixed dataset exhibits higher similarity with infrared downstream tasks, thereby mitigating the representation shift between pre-training and downstream data. Moreover, converting RGB images to grayscale further enhances this similarity, resulting in better fine-tuning performance, as shown in \tabref{tab:dataset_composition_ablation}.


\section{More visualizations.}
\label{app:more_visualization}

We provide additional visualization results in this section. \figref{fig:query_attn_cl} shows the attention maps of different supervised and CL methods of various sizes. The comparison of attention maps between MAE and UNIP is displayed in \figref{fig:distill_query_attn} and \figref{fig:texture_query_attn}. The attention maps of RGB image inputs are visualized in \figref{fig:query_attn_rgb}, exhibiting nearly identical attention pattern distribution with infrared images in \figref{fig:query_attn}.


\begin{figure}[ht]
\centering
\includegraphics[width=\linewidth]{figures/add_query_attn_v4.pdf}
\vspace{-22pt}
\caption{Visualizations of attention maps in supervised and CL models. The attention maps are averaged over different heads. All CL and supervised methods share similar attention pattern distribution across layers.}
\label{fig:query_attn_cl}
\end{figure}

\begin{figure}[ht]
\centering
\includegraphics[width=\linewidth]{figures/distill_query_attn_v2.pdf}
\vspace{-22pt}
\caption{Visualizations of layerwise attention maps in MAE and UNIP-S distilled from MAE-L. The \textit{hybrid} patterns emerge in the later layers of UNIP-S but in the middle layers of MAE-L.}
\label{fig:distill_query_attn}
\end{figure}


\begin{figure}[ht]
\centering
\includegraphics[width=\linewidth]{figures/texture_query_attn_v1.pdf}
\vspace{-22pt}
\caption{Visualizations of attention maps in MAE and UNIP distilled from MAE-L. Attention maps from the 12th layer of MAE-S, the 18th layer of MAE-L, and the 12th layer of UNIP-S are displayed, respectively. Compared to MAE-S and MAE-L, UNIP-S exhibits reduced texture bias, emphasizing shape information over textures.}
\label{fig:texture_query_attn}
\vspace{-15pt}
\end{figure}


\begin{figure}[t]
\centering
\includegraphics[width=\linewidth]{figures/query_attn_rgb.pdf}
\vspace{-24pt}
\caption{Attention maps of RGB image inputs for different query tokens in three representative layers. Each query token's attention map corresponds to a row in the attention matrix, averaged over different heads. Images are from ImageNet \citep{imagenet}.}
\label{fig:query_attn_rgb}
\vspace{-14pt}
\end{figure}


\end{document}
