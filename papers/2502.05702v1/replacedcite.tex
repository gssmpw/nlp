\section{Related Work}
\label{gen_inst}

\begin{figure}[htbp]
  \centering
  \includegraphics[width=0.5\textwidth]{Pictures/1.png}
  \caption{Proteins ____, social networks ____, and electrical power grids ____ are all graphs}
  \label{fig:sample-figure}
\end{figure}

In recent years, machine learning (ML) approaches have been introduced to approximate or solve OPF more efficiently. Among these, multi-layer perceptrons (MLPs) were some of the earliest ML models applied to power system problems. Studies such as ____ applied fully connected network (MLP) to imitate the output of ACOPF and predict power flow variables and generator outputs, However, MLPs face significant challenges with local minima and overfitting, especially when applied to large-scale power systems with non-linearities. They also struggle to model the graph structure of the power grid, where buses (nodes) and transmission lines (edges) are interconnected in a highly structured manner. The failure to leverage this topological information leads to inaccurate results and poor performance, especially in generalization across different grid topologies.

To address these shortcomings, Graph Neural Networks (GNNs) have emerged as a more suitable architecture for power system applications. GNNs excel in handling graph-structured data which make them suitable for power grids, where buses and transmission lines naturally form a graph. GNNs use a message-passing mechanism, allowing each node to aggregate information from neighboring nodes, which mirrors the flow of electrical voltages and currents between interconnected buses in the power system.

Authors in ____ were among the first to apply GNNs to the OPF problem. They developed a GNN-based model that uses imitation learning to predict OPF solutions based on the outputs of traditional solvers like IPOPT. The model demonstrated significant improvements in computation time compared to traditional solvers while maintaining high accuracy, particularly for the IEEE-30 and IEEE-118 bus systems. This approach successfully exploits the grid topology to improve efficiency and provides a foundation for future research in using GNNs for OPF.

____ proposed a Topology-Aware Graph Neural Network that incorporates both the spatial structure of the grid and physical constraints into the learning process. Their model introduces AC-feasibility regularization and ensures that the GNN’s predictions commit to the physical laws of power flow, Kirchhoff’s laws, and generator limits. This method enhanced the generalizability of GNN models across different grid topologies and more importantly ensured that solutions were physically feasible, a critical requirement in real-world power systems.

Further extending the capabilities of GNNs, ____ introduced a Typed Graph Neural Network (TGNN) approach in their work on power flow analysis. TGNNs differentiate between different types of buses (e.g., generator buses, load buses) in the power grid, allowing the model to treat each node type according to its operational role. This added complexity in node classification improves the accuracy of power flow predictions, especially in cases with diverse bus types and heterogeneous grid configurations. Their experiments with typed GNNs showed better performance compared to standard GNN models, particularly in grids with mixed bus types.

Authors in ____ provide a comprehensive review of the application of GNNs in power systems, highlighting their advantages in terms of scalability, generalizability, and performance when applied to complex, non-linear power system problems like OPF. The authors emphasize the ability of GNNs to handle dynamic grid configurations and temporal variations which make them effective in environments with high renewable penetration and fluctuating load conditions. Their review also covers various GNN architectures, including ChebNet and spectral-based GCNs, which have been applied to solve power flow and fault detection problems. The study’s key contribution is its focus on task analysis, where GNNs outperform conventional deep learning models by leveraging the intrinsic graph structure of the power grid. They also discuss the critical challenges of data availability and the need for advanced regularization techniques to ensure physical feasibility in GNN predictions.

____ introduced a model that combines Proximal Policy Optimization (PPO) with GNNs for OPF. Their work leverages the decision-making capabilities of PPO to control generator outputs in a power grid while using GNNs to model the spatial relationships within the grid. Their experiments showed that this approach outperforms traditional solvers like DCOPF in both cost minimization and constraint satisfaction, particularly in dynamic environments where grid topology or load conditions change.

Additionally, ____ explored the application of Graph Neural Solvers (GNS) to directly solve power flow equations by minimizing violations of Kirchhoff’s laws. Their work focused on developing a graph-based solver that could scale efficiently with grid size while ensuring commitment to physical laws, laying the groundwork for GNN-based models that solve power flow equations directly without the need for traditional optimization solvers.

More recently, ____ introduced a comparative study between Koopman Operator-based models and GNNs for learning power grid dynamics. This work highlighted the advantages of GNNs in capturing the spatio-temporal dynamics of power grids, demonstrating their utility not only for static optimization problems like OPF but also for transient analysis and fault detection. The study showed that GNNs could outperform Koopman models in terms of generalizability and adaptability to changing grid conditions.

Incorporating more recent work, ____ proposed a Graph Attention Network (GAT) that integrates physical constraints such as Kirchhoff's laws. This hybrid model focuses on incorporating domain-specific physics into the learning process, ensuring both scalability and physical accuracy. The GAT model allows for efficient propagation of information across nodes (buses), addressing scalability concerns while preserving physical feasibility. This makes it highly suitable for real-time OPF applications in complex, renewable-integrated grids.

____ demonstrate the use of GNNs in a probabilistic context. Their focus is on using attention mechanisms to prioritize critical nodes in a grid, especially under uncertain conditions caused by renewable energy sources. This marks a shift from deterministic to probabilistic approaches in power flow and addresses the challenges posed by stochastic variations in renewable energy generation.

The progression from traditional solvers and MLPs to GNNs represents a shift towards more scalable and efficient OPF solutions. The key advantage of GNNs lies in their ability to incorporate the grid topology which improves both prediction accuracy and computational efficiency. However, challenges remain, particularly in ensuring that GNNs can handle real-time grid operations and adapt to dynamic topologies in a computationally feasible manner.