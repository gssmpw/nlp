%%%%%%%%%%%%%%%%%%%%%%%%%%%%%%%%%%%%%%%%%%%%%%%%%%%%%%%%%%%%%%%%%%%%%%%%

%%% LaTeX Template for AAMAS-2025 (based on sample-sigconf.tex)
%%% Prepared by the AAMAS-2025 Program Chairs based on the version from AAMAS-2025. 

%%%%%%%%%%%%%%%%%%%%%%%%%%%%%%%%%%%%%%%%%%%%%%%%%%%%%%%%%%%%%%%%%%%%%%%%

%%% Start your document with the \documentclass command.


%%% == IMPORTANT ==
%%% Use the first variant below for the final paper (including auithor information).
%%% Use the second variant below to anonymize your submission (no authoir information shown).
%%% For further information on anonymity and double-blind reviewing, 
%%% please consult the call for paper information
%%% https://aamas2025.org/index.php/conference/calls/submission-instructions-main-technical-track/

%%%% For anonymized submission, use this
%\documentclass[sigconf,anonymous]{aamas} 

%%%% For camera-ready, use this
\documentclass[sigconf]{aamas} 


%%% Load required packages here (note that many are included already).

\usepackage{balance} % for balancing columns on the final page
\usepackage{graphicx}
\usepackage{subfigure}
\usepackage{hyperref}
\usepackage{dsfont}
\usepackage{multicol}
\newcommand{\argmax}{\operatornamewithlimits{argmax}}
\newcommand{\argmin}{\operatornamewithlimits{argmin}}
\newcommand{\E}{\mathbb{E}}
\newcommand{\I}{\mathbb{I}}

\newcommand{\Com}{\mathrm{Com}}
\newcommand{\Cost}{\mathrm{Cost}}
\newcommand{\rad}{\mathrm{rad}}

\newcommand{\siwei}[1]{\textcolor{blue}{[Siwei: #1]}}
\newcommand{\junning}[1]{\textcolor{red}{[Junning: #1]}}
\newcommand{\Fang}[1]{\textcolor{green}{[Fang: #1]}}

\usepackage{amsmath}
% \usepackage{amssymb}
\usepackage{mathtools}
\usepackage{amsthm}
\usepackage{algorithm}
\usepackage{algorithmic}

\theoremstyle{plain}
\newtheorem{theorem}{Theorem}[section]
\newtheorem{proposition}[theorem]{Proposition}
\newtheorem{lemma}[theorem]{Lemma}
\newtheorem{corollary}[theorem]{Corollary}
\theoremstyle{definition}
\newtheorem{definition}[theorem]{Definition}
\newtheorem{assumption}[theorem]{Assumption}
\theoremstyle{remark}
\newtheorem{remark}[theorem]{Remark}
\usepackage[toc,page]{appendix}
%%%%%%%%%%%%%%%%%%%%%%%%%%%%%%%%%%%%%%%%%%%%%%%%%%%%%%%%%%%%%%%%%%%%%%%%

%%% AAMAS-2025 copyright block (do not change!)

\makeatletter
\gdef\@copyrightpermission{
  \begin{minipage}{0.2\columnwidth}
   \href{https://creativecommons.org/licenses/by/4.0/}{\includegraphics[width=0.90\textwidth]{by}}
  \end{minipage}\hfill
  \begin{minipage}{0.8\columnwidth}
   \href{https://creativecommons.org/licenses/by/4.0/}{This work is licensed under a Creative Commons Attribution International 4.0 License.}
  \end{minipage}
  \vspace{5pt}
}
\makeatother

\setcopyright{ifaamas}
\acmConference[AAMAS '25]{Proc.\@ of the 24th International Conference
on Autonomous Agents and Multiagent Systems (AAMAS 2025)}{May 19 -- 23, 2025}
{Detroit, Michigan, USA}{Y.~Vorobeychik, S.~Das, A.~Nowé  (eds.)}
\copyrightyear{2025}
\acmYear{2025}
\acmDOI{}
\acmPrice{}
\acmISBN{}

%%%%%%%%%%%%%%%%%%%%%%%%%%%%%%%%%%%%%%%%%%%%%%%%%%%%%%%%%%%%%%%%%%%%%%%%

%%% == IMPORTANT ==
%%% Use this command to specify your OpenReview submission number.
%%% In anonymous mode, it will be printed on the first page.

\acmSubmissionID{468}

%%% Use this command to specify the title of your paper.

\title[AAMAS-2025 Formatting Instructions]{Learning with Limited Shared Information in Multi-agent Multi-armed Bandit}

% Add the subtitle below for an extended abstract
%\subtitle{Extended Abstract}

%%% Provide names, affiliations, and email addresses for all authors.

\author{Junning Shao}
\affiliation{
  \institution{Tsinghua University}
  \city{Beijing}
  \country{China}}
\affiliation{
  \institution{Shanghai Qi Zhi Institute}
  \city{Shanghai}
  \country{China}}
\email{sjn21@mails.tsinghua.edu.cn}

\author{Siwei Wang}
\affiliation{
  \institution{Microsoft Research}
  \city{Beijing}
  \country{China}}
\email{siweiwang@microsoft.com}

\author{Zhixuan Fang}\thanks{Corresponding authors: Siwei Wang (\texttt{siweiwang@microsoft.com}), Zhixuan Fang (\texttt{zfang@mail.tsinghua.edu.cn}).}
\affiliation{
  \institution{Tsinghua University}
  \city{Beijing}
  \country{China}}
\affiliation{
  \institution{Shanghai Qi Zhi Institute}
  \city{Shanghai}
  \country{China}}
\email{zfang@mail.tsinghua.edu.cn}

%%% Use this environment to specify a short abstract for your paper.

\begin{abstract}
Multi-agent multi-armed bandit (MAMAB) is a classic collaborative learning model and has gained much attention in recent years. However, existing studies do not consider the case where an agent may refuse to share all her information with others, e.g., when some of the data contains personal privacy. 
%
In this paper, we propose a novel limited shared information multi-agent multi-armed bandit (LSI-MAMAB) model in which each agent only shares the information that she is willing to share, and propose the Balanced-ETC algorithm to help multiple agents collaborate efficiently with limited shared information. 
%
Our analysis shows that Balanced-ETC is asymptotically optimal and its average regret (on each agent) approaches a constant when there are sufficient agents involved. 
%
Moreover, to encourage agents to participate in this collaborative learning, an incentive mechanism is proposed to make sure each agent can benefit from the collaboration system. Finally, we present experimental results to validate our theoretical results.
\end{abstract}

%%% The code below was generated by the tool at http://dl.acm.org/ccs.cfm.
%%% Please replace this example with code appropriate for your own paper.


%%% Use this command to specify a few keywords describing your work.
%%% Keywords should be separated by commas.

\keywords{Multi-armed bandit, Multi-agent collaboration, Incentive mechanism}


%%%%%%%%%%%%%%%%%%%%%%%%%%%%%%%%%%%%%%%%%%%%%%%%%%%%%%%%%%%%%%%%%%%%%%%%

%%% Include any author-defined commands here.
         
\newcommand{\BibTeX}{\rm B\kern-.05em{\sc i\kern-.025em b}\kern-.08em\TeX}

%%%%%%%%%%%%%%%%%%%%%%%%%%%%%%%%%%%%%%%%%%%%%%%%%%%%%%%%%%%%%%%%%%%%%%%%

\begin{document}

%%% The following commands remove the headers in your paper. For final 
%%% papers, these will be inserted during the pagination process.

\pagestyle{fancy}
\fancyhead{}

%%% The next command prints the information defined in the preamble.

\maketitle 

%%%%%%%%%%%%%%%%%%%%%%%%%%%%%%%%%%%%%%%%%%%%%%%%%%%%%%%%%%%%%%%%%%%%%%%%

\section{Introduction}
Multi-armed bandit (MAB) problem is a fundamental theoretical model and has been studied for decades.
There are many practical applications of multi-armed bandit algorithms in industry, especially in on-line platforms \cite{lattimore2020bandit}. 
In a standard MAB game with $N$ arms and time horizon $T$, a single player needs to choose one arm to pull in each time slot, and pulling different arms results in different expected rewards.
% After pulling arm $i$, the player receives a random reward $X_i(t)$, which is drawn independently from a corresponding reward distribution $D_i$. %Each arm $i$ has a reward distribution denoted by $D_i$ with mean $\mu_i$. 
% The key question of all bandit problems is finding the right balance between exploration and exploitation: should the player explore an arm which looks inferior or exploit the arm which looks best currently?
Numerous algorithms have been proposed to solve bandit problems, e.g., UCB \cite{auer2002finite, gittins2011multi}, which is a classic algorithm that guarantees $O(N\log T)$ regret upper bound (which matches the $\Omega(N\log T)$ regret lower bound).
% , Thompson Sampling (TS)\cite{thompson1933likelihood} and $\epsilon$-greedy\cite{watkins1989learning}. 
% The UCB algorithm is one of the most commonly asymptotically optimal algorithm which guarantee $O(\log T)$ regret upper bound(match the regret lower bound $\Omega(\log T)$ \cite{lai1985asymptotically}).
% Thus we consider the single-player UCB algorithm as the baseline in this paper. 
% \junning{I have a feeling that deleting this paragraph makes no difference to reader.
% I suggest delete this para, so that we can enter multi agent setting more quickly.}
% \siwei{What is the trade-off between exploration and exploitation in bandit problems? need to explain it} \siwei{list some algorithms here (e.g., UCB), and briefly describe them. Besides, mention the optimality of UCB, so that we can say that the individual regret of our algorithm is lower than UCB implies everyone can benefit from our collaboration framework.} 

However, the standard MAB setting only focuses on the game with a \emph{single} agent, while many real-world applications face the challenge of \emph{multiple} agents making decisions. 
%
As a concrete example, on an online shopping platform, when a buyer chooses to purchase a certain product, she will receive a corresponding reward. Hence, the buyer can be regarded as an agent and the product can be regarded as an arm. 
%
Different from the standard MAB setting with a single player, there are always a large number of buyers choosing products to purchase on the online shopping platform, and collaboration between them could help them learn much faster. 
%(agents) choose which products(arms) to purchase. 
Therefore, in this paper, we consider the multi-agent multi-armed bandit (MAMAB) model (also called multi-player multi-armed bandit in some literature), which features multiple players playing (and collaborating in) the same instance of an MAB problem together. 
%
Following previous literature, we focus on the common objective of maximizing the total reward of all agents, i.e., social welfare maximization. This is equivalent to minimizing the total regret of agents in the language of bandit problems. %Thus, the multi-agent decision process is essentially a collaborative learning.
% In this model, there exist $M$ independent agents $\{1,2,\cdots, M\}$, and each agent interacts with the same set of arms $\{1,2,\cdots,N\}$.
% \siwei{More detailed explanations about the relation between this example and the bandit problem, e.g., what are the arms, why we can use information sharing to improve their experiences, etc}.
%These arms are homogeneous for every agent. In other words, for agent $m$, the rewards $X_{i,m}(t)$ of arm $i \in [N]$ sampled independently from a unknown distribution $D_i$ with mean $\mu_i$. 

Several variants of the MAMAB model have been studied in the existing literature, e.g., letting the agents collaborate to speed up the learning procedure with limited communication (e.g., \cite{agarwal2021multi,shahrampour2017multi,sankararaman2019social}), considering decentralized matching market with multi-armed bandit (e.g., \cite{liu2020competing,zhangmatching,kong2023player}). 
%
%. In some studies, the agents can collaborate to speed up learning with limited communication\cite{agarwal2021multi,shahrampour2017multi,sankararaman2019social}. Some studies \cite{liu2020competing}\cite{zhangmatching} considers variants of multi-agent multi-armed bandits framework with the added feature. 
%
%
However, these studies do not consider the case where an agent may refuse to share her private information with others,
%In many situations, the agent is reluctant to share her information for many reasons. The agent may 
and may even decide to withdraw from learning if she is forced to share some data that she is not willing to share. 
%
For example, on an \emph{online shopping platform}, the shared information would be the users' comments %(e.g., ratings) 
about the products, which may help other users  make their decisions.
% \siwei{try to explain liu2020 and zhangmatching in simple words.} 
%
%
%
%consider the One typical example is the buyers(agents) on the online shopping platform. These buyers decide which products(arms) to purchase and leave a comment to share their experiences. Sometimes the buyer may give the product a rating, which can help other buyers to make their decision. We can consider the comment or the rating as the communication between agents. 
However, a user may not comment on every product she purchases in reality, and she can be reluctant to make comments for many reasons, e.g., because the user regards the comments as her privacy, or because commenting on these products does not directly improve her experiences. 
%
% \junning{survey}
A survey presented in \cite{tsai2006s} indicates that customers are likely to refrain from purchasing certain types of items on online shopping platforms due to privacy concerns. 
%
This survey reveals that different items often elicit varying levels of sensitivity among users, indicating that there are certain products that are widely accepted for online purchases by the majority of individuals, while there are other products that only a small fraction of people are willing to purchase online.
%
For example, the study shows that for common products such as office supplies, there is little hesitation about buying them online. However, as the items became more personal or related to sensitive topics such as sex, depression, or adult diapers, hesitation increased. When the items were indicative of behavior that could be associated with criminals or terrorists, such as a book on making bombs and bullets, there was significant reservations and reluctance to purchase online. Furthermore, some studies \cite{tsai2006s, bellman2004international} also find that individuals may have varying privacy considerations for the same product. Certain products are more likely to raise privacy concerns among individuals, but the specific privacy concerns of each person are contingent upon factors such as their values, cultural background, personal experiences, and other related factors. These objective survey findings have motivated us to consider a framework in which users selectively share only a portion of the information they are willing to disclose, while choosing to withhold information they perceive as private.
%it could be the agent has no motivation to spend too much time in commenting or rating. 
%
Another example is \emph{federated learning}. In a federated learning framework, it is obligatory to preserve data privacy, and a massive amount of data cannot be shared because of the legal concern. For instance, the EU adopted the General Data 
Protection Regulation (GDPR) \cite{european2016regulation}, which states that any institutions or organizations do not have the authority to share users' data unless they have the permission. 
To the best of our knowledge, we are the first to consider %takes into account 
this particular structural reality. Under our model, agents are more  willing to participate since he can hold his sensitive information private. 
% there is a greater willingness among users to engage in collaboration. \siwei{Do not really understand the last sentence... Do you mean agents are more  willing to participate since he can hold his sensitive information private?}


% While many frameworks have been investigated to federated learning in MAB setting, we recognize two major limitations. First, some studies \cite{kremer2014implementing}\cite{frazier2014incentivizing}\cite{shi2021federated} assume every agent in their framework must share the full history of pulling every arm to other agents or the principle.  It is unpractical to order players to upload all the data. In reality, the same arm always play a different role for different agents.  As a concrete example, some agents are willing to share their comments on clothes, however some agents hesitate to share their comments on clothes but hope to show their electronic devices. Second, some studies\cite{shi2021federated} cannot ensure every agent benefit from their framework.  As a rational individual, agent will not participate the framework if she cannot make a profit on the framework. 

Given the structure of limited information sharing, the first challenge arises as the shared data may be imbalanced, causing serious bias in data \cite{mehrabi2021survey}. For example, there could be a lot of observed rewards on arm $i$, while only limited observed rewards on arm $j$, since many users are willing to share the information of arm $i$, but only few of them are willing to share the information of arm $j$ (image the case where the data related to arm $j$ is more sensitive).
%
This may result in a serious over-exploration in bandit problems, leading to an inefficient learning.
%
%if a large amount of  data of these arms is shared by the agents. At the same time, the information of some arms may be shared by only a few agents and  the lack of some data will make learning difficult. In MAB setting,  data bias can result in serious over exploitation problem. 

The second challenge for multi-agent collaboration is how to guarantee individual rationality (IR), i,e, agents can always get non-negative utility from the mechanism \cite{shoham2008multiagent}. For example, consider the extreme case that one agent is willing to share all her information, while all the other agents do not share anything. In this case, the agent that shares information helps the others, but cannot directly benefit from sharing. In this case, she may refuse to join the collaboration, leading to a failure of cooperation.
%
Thus, it is also important to design mechanisms to ensure that all  agents who join this collaboration system can benefit (i.e., earn more reward than learning alone).
%otherwise any rational agent will leave the framework if she cannot benefit. 
%\siwei{Need to be more detailed}

In this paper, we introduce a novel multi-agent multi-armed bandit model named limited shared information multi-agent multi-armed bandit (LSI-MAMAB) to characterize the structure of collaborative learning in reality, 
% \siwei{Is this our new model name? Why do we want a "federated" in this name?}\junning{claiming we introduce a framework here is redundant to the paragraphs below, we can just point out the design goal here with the two challenges }
 in which each agent only shares the information that she is willing to share (e.g., only her received rewards on \emph{some} arms) with the other agents, and only decides to participate in this collaboration when she can benefit from it. 
% \junning{we didn't talk about this IR in previous motivation?} 
We solve the above two challenges, and design an algorithm \emph{Balanced-ETC} for the case that each agent only shares the information of some specific arms with the other agents.
%
On the one hand, the total regret of Balanced-ETC is upper bounded by $O(N\log T)$,  %($N$ is the number of arms), 
which is asymptotically the same as the best possible regret that one can do in the trivial case where all the information is shared. This reflects the asymptotic optimality of our algorithm.
%
On the other hand, with the additional incentive design, Balanced-ETC makes sure that the individual regret of each agent is upper bounded by the regret of running a UCB algorithm in the single-agent setting. This means that our algorithm satisfies individual rationality (IR), i.e., all agents have the motivation to join this collaboration system.

We summarize the main contributions of this paper below:
%\vspace{-0.1in}
\begin{itemize}
    \item We propose LSI-MAMAB, a novel multi-agent multi-armed bandit model, in which each agent takes part in the collaboration system by only sharing the information that she is willing to share. %This design lowers the barrier for agents to entry the framework.
    This lowers the barrier for agents to enter collaborative learning in practices.
    % \vspace{-0.05in}
    \item We design the Balanced-ETC algorithm to help multiple agents collaborate efficiently under the constraint that they only share partial information, and prove that its overall regret is $O(N\log T + MN^2)$ ($M$ is the number of agents, $N$ is the number of arms and $T$ is time horizon). This means that the average regret (on each agent) %is inversely proportional to the number of agents, and
    almost approaches a constant when there are sufficiently many agents.
    % \vspace{-0.05in}
    \item We also design an incentive mechanism in the Balanced-ETC algorithm to encourage agents to participate in collaboration. Analysis shows that the incentive mechanism satisfies IR, i.e.,  for any agent, her individual regret of joining this collaboration system is better than learning as a single agent. 
    
    %\item    A high-probability finite-time upper bound of total regret is proved, which demonstrates the regret will be inversely proportional to the number of agents. So the regret of Balanced-ETC algorithm will be almost free if there are sufficient many agents.
    
\end{itemize}

\section{Related Works}


%\begin{itemize}
    %\item 
    \textbf{Multi-agent multi-armed bandit.} Numerous frameworks and algorithms have been proposed to solve various multi-agent multi-armed bandit problems. 
    %
    There are many prior works on different multi-agent multi-armed variants.
    %
    For example, 
    \cite{vial2021robust} considers the setting of including honest and malicious agents who recommend best-arm estimates and arbitrary arms, respectively. \cite{liu2020competing,zhangmatching,kong2023player} 
    add the feature that the agents may have collisions with each other when they are pulling the same arm in each time step. \cite{baek2021fair,yang2022distributed} consider the user or group's set of available arms as a subset of the complete arm set, and the regret for each user or group is based on the choices made within their own arm set.
    % arms have preferences over players. 
    Compared with our model, these existing works do not consider the case where an agent may refuse to share all her information with others, and even decide to withdraw if she is forced to share some data that she is not willing to share.
    
    % \junning{communication related works}
    Another strand of literature studies multi-agent multi-armed bandit problems with limited communication. For example, \cite{agarwal2021multi} considers a multi-agent multi-armed bandit framework with limited communication rounds and limits bits in each communication rounds; \cite{madhushani2021call} proposes ComEx, a novel and cost-effective communication protocol for cooperative bandits; in the work of \cite{shahrampour2017multi}, players can only exchange information locally to estimate the global reward confined to a network structure; and \cite{sankararaman2019social} proposes a pairwise asynchronous gossip-based protocol that only needs to exchange a limited number of bits to finish communication. 
%
Compared with these models, we are more interested in which data users are willing to share, rather than the form in which the data is shared. Therefore, we did not consider the adoption of more efficient sharing methods by users, and regard the communication costs as zero. %Hence, we considered the scenario where communication costs are zero.

     %\item 
     \textbf{Federated Learning.}
     % \junning{foundations of FL}
     Federated learning (FL), a decentralized machine learning approach, has gained significant attention in recent years. This emerging paradigm enables training of machine learning models on distributed data sources while preserving data privacy. Several related works have explored the foundations of federated learning. \cite{mcmahan2017communication} introduced the concept of federated learning and proposed a practical framework for training models on decentralized data, which enables collaborative model training across multiple devices. There are numerous foundational works dedicated to addressing the various challenges of federated learning and designing federated learning algorithms from multiple aspects, such as privacy preservation, robustness, efficiency, security, scalability, and performance \cite{kamp2019efficient, pillutla2022robust, mcmahan2017communication, jeong2018communication, sattler2019robust}.  Federated learning also expands to encompass a broad range of applications in healthcare \citep{kaissis2020secure}, manufacturing \citep{qu2020blockchained}, agriculture \citep{durrant2022role}, energy \citep{saputra2019energy}, and other fields. 
     
     There are also some prior works on applying FL in bandit problems. 
     For example, in \cite{shi2021federated,shi2021federated1}, an MAB framework of multiple heterogeneous agents and a global principal is proposed. In this framework, agents' local bandit models are not the same (i.e., different agents may have different expected rewards on the same arm) and the goal of the principal is to find the arm with the largest global mean.
     %
    \cite{zhu2021federated} proposes a framework where agents can only communicate their local data with neighbors in a connected graph.  They propose the FedUCB policy, in which the agents preserve differential privacy of their local data. 
    Compared with our model, \cite{shi2021federated,shi2021federated1} require that all agents must follow the global instructions unconditionally and send all their information to the principal. \cite{zhu2021federated} do not consider the case where agents could refuse to share their private information even with differential privacy. In our model, each agent can limit her shared information as her wish. 
    % \siwei{\citet{zhu2021federated} also need to send all the data?}
    %In this work, agents only need to send the partial information she is willing to share instead of the full information which contains private information.
    
    %\item 
    \textbf{Incentivized Learning.} 
    % \junning{Incentive}
    Since its proposal by \cite{frazier2014incentivizing, kremer2014implementing}, significant progress has been made in the field of incentivized learning in multi-armed bandit (MAB) problems. Specifically, there are two distinct lines of research in this area. 

    The first line of research \cite{kremer2014implementing, mansour2015bayesian, mansour2016bayesian, immorlica2018incentivizing} assumes that the principal has access to the complete history of actions and rewards, while the agents do not. In this setting, the principal can incentivize them to learn by leveraging proper information to them.
    
    The second line of research considers a publicly available history of actions and rewards, and the incentives are implemented through compensations. This concept was initially introduced by \cite{frazier2014incentivizing} and further generalized by \cite{han2015incentivizing} in Bayesian settings. In the non-Bayesian case,  \cite{wang2018multi} first studied this approach, and it has been recently extended by \cite{wang2021incentivizing}.
        
    
    % Many progresses have been made in incentivized learning in MAB instances. 
    % For example, \cite{frazier2014incentivizing, han2015incentivizing, wang2018multi, wang2021incentivizing} consider the case that the full history information is publicly available and the agents are motivated through compensations, and \cite{kremer2014implementing,mansour2015bayesian} assume that the agents cannot observe the full history information, so the principal can incentivize them to learn by leveraging proper information to them. 
    %

    %Compared with our results, 
    All the aforementioned works assume that every agent is myopic (i.e., they only do exploitation to maximize their short-term rewards), while we assume that every agent is considering her long-term rewards. 
    %
    Hence, the incentive mechanism in this paper can be very different from theirs. 
    %
  


   
    
    
%\end{itemize}
%%%%%%%%%%%%%%%%%%%%%%%%%%%%%%%%%%%%%%%%%%%%%%%%%%%%%%%%%%%%%%%%%%%%%%%%

\section{Problem Formulation}
\begin{figure*}[t]
    % \vspace{-0.2in}
    \centering 
    \includegraphics[width=0.8\linewidth]{framework.png}
    %\vspace{-0.2in}
    \caption{Illustration of the LSI-MAMAB model. In each time step $t$, the agents choose arms sequentially. After agent $m$ pulls an arm $\pi_m(t)$, she then gets a random reward $X_{\pi_m(t), m}(t) \sim D_{\pi_2(t)}$ from the arm $\pi_m(t)$. If $\pi_m(t) \in A_2$, then she could broadcast the reward, so that other agents can use this information. Otherwise, agent $m$ keeps this reward information to herself. 
    %
    The central controller wants to design an algorithm to minimize the overall regret. Besides, he also uses an incentive mechanism to achieve IR: after an agent pulls an arm, he provides some compensation to her, and charges other agents for the shared information (if the arm-reward pair is shared).}
    \Description{Illustration of the LSI-MAMAB model. In each time step $t$, the agents choose arms sequentially. After agent $m$ pulls an arm $\pi_m(t)$, she then gets a random reward $X_{\pi_m(t), m}(t) \sim D_{\pi_2(t)}$ from the arm $\pi_m(t)$. If $\pi_m(t) \in A_2$, then she could broadcast the reward, so that other agents can use this information. Otherwise, agent $m$ keeps this reward information to herself. 
    %
    The central controller wants to design an algorithm to minimize the overall regret. Besides, he also uses an incentive mechanism to achieve IR: after an agent pulls an arm, he provides some compensation to her, and charges other agents for the shared information (if the arm-reward pair is shared).}
    %Every agent will receive some compensation at the end of the game because of the information she shares with others, and she also needs to pay some cost because of the information others share with her. }
    \label{model_figure} 
    % \vspace{-0.2in}
\end{figure*}
% \vspace{-0.8cm}

In the the limited shared information multi-agent multi-armed bandit (LSI-MAMAB) model, there exist $M$ independent agents $\{1,2,\cdots, M\}$ and each agent interacts with the same set of arms $A = \{1,2,\cdots,N\}$ for $T$ time steps. 
%
These arms are homogeneous for different agents, i.e., for any agent $m$, the random reward $X_{i,m}(t)$ of pulling arm $i \in [N]$ is sampled independently from a fixed (but unknown) distribution $D_i$, which is a bounded distribution on $[0,1]$.
%
We denote $\mu_i$ as the mean of $D_i$, and 
assume that $1\ge \mu_1> \mu_2\ge\cdots \ge \mu_N\ge 0$. We also let $\Delta_i \vcentcolon= \mu_1 - \mu_i$ be the expected reward gap between arm $i$ and the optimal arm and denote $\Delta_{\min} \vcentcolon= \Delta_2$. 
%
%\junning{assumption}
Although we assume that there only exists one optimal arm in the model, it is not a necessary condition for obtaining theoretical results in this paper. In fact, all the analysis works in the case that there are multiple optimal arms. 
Moreover, to simplify the analysis of cooperation, we assume that the agents always choose to share truthful information.


To model our assumption that each agent may only share partial information with the others, for any agent $m$, we denote $A_m \subseteq A$ as the set of arms that agent $m$ is willing to share information. We call $A_m$ the non-sensitive arm set and refer to the remaining arm set $A \setminus A_m$ as the sensitive arm set. Note that we allow $A_m$ to be an empty set, i.e., some agents may share nothing with others. 
%
We also let $S_i = \{m: i \in A_m\}$ be the set of agents who are willing to share the information of arm $i$, and we assume that for any arm $i$, $|S_i| \ge 1$, i.e., there is at least one agent who is willing to share her history information on arm $i$ with others. This assumption is to ensure feasible collaboration, because if the information of some arms is not shared by any agents, no collaboration mechanism can work to reduce the times of exploring these arms. 


%To make the collaboration feasible, we make such assumption. 

% \siwei{Maer the case some armybe we need to explain why this assumption makes sense.}

%We suppose any agent $m \in [M]$  share the information of some specific arms with other agents. Denote $A = \{1,..., N\}$ is the set of all the arms and $A_m$ is the set of arms whose information can be shared by agent $m$, and $|A_m|$ is the size of $A_m$. Denote $S_i$ is the set of agents who share the information of arm $i$. In this paper, we assume each arm is shared by at least one agents. At time $t$, agent $m$ chooses an arm $\pi_m(t)$ and receive the reward $X_{\pi_m(t), m}$.  Each agent aims to maximize the highest expected cumulative reward in $T$ rounds by choosing a proper arm to pull. The common metric for evaluating the performance of a policy is regret, which is the difference between the total expected reward using optimal policy and the total expected reward collected by the agent. The reward is often characterized by minimizing the regret. 
We illustrate how the LSI-MAMAB model works in Figure \ref{model_figure}. In each time step $t$, the agents choose arms sequentially, i.e., after agent $1$ pulls an arm (and broadcasts the arm-reward pair if she wants to), agent $2$ then chooses an arm to pull. 
%
%\siwei{
Note that the order of pulling arms in each time step does not influence any theoretical results in this paper, the assumption that the order being $1,2,\cdots,M$ is only for simplicity. %} 
% \junning{protocol}
After agent $m$ pulls an arm $\pi_m(t)$, she then gets a random reward $X_{\pi_m(t), m}(t) \sim D_{\pi_m(t)}$ from the arm $\pi_m(t)$. If arm $\pi_m(t) \in A_m$ (i.e., results from which agent $m$ feels as non-sensitive information), then she could broadcast the arm-reward pair $(\pi_m(t), X_{\pi_m(t), m}(t))$, so that other agents can use this information. %to help them choose arms. 
%
Otherwise, agent $m$ keeps this arm-reward pair information private to herself. 
%
Our goal is to design a collaboration protocol from the perspective of the central controller. On the one hand, we want an algorithm to minimize the overall regret when all the agents follow this protocol (i.e., each agent pulls arms according to the protocol, and share the information as long as the pulled arm is in her non-sensitive arm set), leading to an efficient collaboration. On the other hand, we expect the protocol guarantee a basic incentive of individual rationality, i.e., every agent can benefit from the collaboration compared to individual learning. To achieve this goal, once an agent pulls an arm and share the arm-reward information, we provide some compensation to her, and charge other agents for the shared information.

%after an agent shares her information, we provide some compensation to her, and charge other agents for this information.

%we design a protocol that every agent should comply with. Under the requirements of the protocol, every agent pull the arm in each round according to the protocol.

%is in $A_m$, she will broadcast which arm she choose and the reward of the arm with other agents. In the meantime,    other agents can receive the information from the agent who shares her information in this time slot. The extra information from others will help each agent explore each arm. 

\subsection{The Overall Regret}
% \siwei{what is $X_{*,m}(t)$? Should it be just $X_{1,m}(t)$? Or can we write the regret as $\E\left[\sum_{m=1}^M(\sum_{t=1}^T\mu_{1} - \sum_{t=1}^T \mu_{\pi_m(t)})\right]$}

First, for agent $m$, we define her individual regret $R_m(T)$ of a policy as the expected gap between her total reward using the policy and the total reward by pulling the optimal arm for $T$ times, i.e., 
\begin{equation*}
    R_m(T) \triangleq \E\left[\sum_{t=1}^T\mu_{1} - \sum_{t=1}^T \mu_{\pi_m(t)}\right].
\end{equation*}
%$R_m(T)$ can be viewed as agent $m$'s loss of pulling suboptimal arms. 
%For different agents, their individual regrets can also be different. % from each other.

For different agents, their individual regrets can also be different. 

Next, we define the overall regret as the sum of all agents' individual regrets, i.e.,
\begin{equation*}
    R(T) \triangleq \sum_{m=1}^M R_m(T). %\E\left[\sum_{m=1}^M R_m(T)\right].
\end{equation*}

As we can see, $R(T)$ is the overall loss of pulling sub-optimal arms by any agents, which reflects the efficiency of the collaboration. Our goal is to minimize the overall regret $R(T)$.
%
On the one hand, if everyone shares all her information with others, then it is easy to design algorithms with $R(T) = O(\sum_i {\log T\over \Delta_i})$.
%
On the other hand, if no one shares her information with others, then the overall regret is at least $\Omega(M\sum_i {\log T\over \Delta_i})$.
%
Hence, in our setting in which agents only share partial information with others, 
the overall regret should be in between this two bounds, and we want it to be as close to $O(\sum_i {\log T\over \Delta_i})$ as possible.

%we also want the overall regret $R(T)$ to be close to $$

%there is no collaboration, then
%$$R(T) = E[\sum_{m=1}^M(\sum_{t=1}^TX_{*,m}(t) - \sum_{t=1}^TX_{\pi(t),m}(t))]$$

\subsection{The Individual Rationality} 
% \siwei{I change the name to "individual regret". If you prefer "personal regret", you can just change them back.}
From the description above, every agent $m$ will receive some compensation $\Com_m$ at the end of the game because of the information she shares with others, and she also needs to pay some cost $\Cost_m$ because of the information others share with her. 
% \siwei{change the notation $C_m$ and $c_m$ to $\Com_m$ and $\Cost_m$ to avoid potential misunderstanding.}
%
Therefore, the individual regret with incentive mechanism of each agent $m$ consists of three parts: the regret comes from pulling sub-optimal arms ($R_m(T)$); the (minus) compensation she receives ($-\Com_m$); and the cost she needs to pay ($\Cost_m$).

In this paper, we assume that each agent applies the 2-UCB (``2'' represents the factor in confidence radius) policy as her single-player policy, which is one of the most commonly studied asymptotically optimal algorithms. Specifically, in the 2-UCB policy, agents are assumed to pull the arm $\pi(t) = \argmax_{i\in [N]} \{ \hat{\mu}_i(t) + \sqrt{\frac{2 \log T}{N_i(t)}}\}$ at time step $t$,
%\begin{equation*}
 %   \pi(t) \gets \argmax_{i\in [N]}\left\{ \hat{\mu}_i(t) + \sqrt{\frac{2 \log T}{N_i(t)}}\right\},
%\end{equation*}
where $\hat{\mu}_i(t)$ is the empirical mean of arm $i$ and $N_i(t)$ is the number of times that arm $i$ has
been pulled. 
%
We denote $R_{UCB}(T)$ the expected regret of this 2-UCB policy. 


To ensure each agent can benefit from collaboration, we call an incentive mechanism satisfies individual rationality (IR), if for each agent $m$, the following inequality holds: 
\begin{equation*}
 R_m(T) - \Com_m + \Cost_m -R_{UCB}(T) \le 0.
\end{equation*}
As we can see, $R_m(T) - \Com_m + \Cost_m -R_{UCB}(T)$ is the relative regret of joining the collaboration system, and we could say that all agents are willing to participate in the collaboration if the incentive mechanism achieves IR.

%We incorporate the 2-UCB algorithm as the baseline algorithm in our definition of IR. 
The choice of incorporating the 2-UCB algorithm as the baseline algorithm in our definition of IR is motivated by the fact that the UCB algorithm is one of the most classic and well-known bandit algorithms in academic research. The algorithm is theoretically proven to be asymptotically optimal and is often considered as a benchmark algorithm in single-agent scenarios. Hence, we believe that using the UCB algorithm as baseline is highly appropriate.
% Since $R_m(T) - \Com_m + \Cost_m -R_{UCB}(T)$ is the relative regret of joining the collaboration system, we could say that all agents are willing to join the collaboration if IR is achieved.
% As we can see, $R_m(T) - \Com_m + \Cost_m -R_{UCB}(T)$ is the relative regret of joining the collaboration system, and we could say that all agents are willing to participate the collaboration if the incentive mechanism achieves IR.

%We incorporate the 2-UCB algorithm as the baseline algorithm in our definition of IR. 

% The choice of incorporating the 2-UCB algorithm as the baseline algorithm in our definition of IR is motivated by the fact that the UCB algorithm is one of the most classic and well-known bandit algorithms in academic research. The algorithm is theoretically proven to be asymptotically optimal and is often considered as a benchmark algorithm in single-agent scenarios. Hence, we believe that using the UCB algorithm as baseline is highly appropriate.


% \siwei{similarly, what is $X_{*,m}(t)$? Should it be just $X_{1,m}(t)$? Or can we just write the regret as $\E\left[(\sum_{t=1}^T\mu_{1} - \sum_{t=1}^T \mu_{\pi_m(t)})\right]$}:

%of the algorithm if she shares her information with other agents. The amount of compensation depends on how much information the agent shares. The agents also should pay for $c_m$ shared information from other agents. 

% $R_m(T)$ can be viewed as the real loss of agent $m$, and we could say that all agents are willing to join 
 % our collaboration system if their individual regret $R_m(T)$ is less than the regret of playing the same MAB instance alone.
%is the overall loss of pulling sub-optimal arms by any agents, and can be viewed as the efficiency of the collaboration. 


%The reward consists of three parts, the first part is reward from arm $\pi_m(t)$ and the second part is the compensations, the third is the cost of other agents' shared information.
%For individual agent, the goal is to maximize her profits. The personal regret of agent $m$ is 

%$$R_m(T) = E[\sum_{t=1}^TX_{*,m}(t) - \sum_{t=1}^TX_{\pi(t),m}(t)] - C_m + c_m$$






%%%%%%%%%%%%%%%%%%%%%%%%%%%%%%%%%%%%%%%%%%%%%%%%%%%%%%%%%%%%%%%%%%%%%%%%

\section{Minimizing Overall Regret}
\subsection{Balanced-ETC Algorithm}

\begin{algorithm}[t]
\caption{Balanced-ETC (for agent $m$)}\label{algorithm}
\begin{algorithmic}[1]
 \STATE \textbf{Input}: the set of arms $A_m$ that agent $m$ is willing to share, the balance level threshold $B$. %the set of arms agent $m$ is willing to share$A_m$
 \STATE \textbf{Init}: $A(t) = A$
\FOR{$t=1,\dots,T$}
    \STATE $\forall i \in A(t)$, updating their $N_{i}(t)$ and $\hat{\mu}_{i}(t)$.
    \STATE Eliminate arms from $A(t)$ based on Eq. \eqref{Eq_1}.%which are identified as suboptimal arms from $A(t)$ 
    \IF {$|A(t)| > 1$ and $|A(t) \cap A_m | > 0$ and $B_m(t) \le B$} 
        %\IF{$|A(t) \cap A_m | > 0$}
         %   \IF {$B_m(t) \le B$} 
                \STATE //Explore step 
                % \siwei{I change the name to explore, which may be easier to understand}
                \STATE Pull arm $\pi_m(t) = \argmin_{i\in A(t) \cap A_m} N_i(t)$ and receive random reward $X_{\pi_m(t), m}(t)$.
                \STATE Broadcast arm-reward pair $(\pi_m(t), X_{\pi_m(t), m}(t))$.
            \ELSE 
                \STATE//Commit step
                \STATE $\pi_m(t)=\argmax_{i\in A(t)} \hat{\mu}_i(t)$ 
            \ENDIF


\ENDFOR
% \STATE \textbf{Incentive mechanism} Receive compensation $Com_m$ and pay cost $Cost_m$.

\end{algorithmic}
\end{algorithm}

In this section, we propose our Balanced-ETC algorithm, which achieves an overall regret upper bound of $O(\sum_i {\log T\over \Delta_i})$ and is asymptotically optimal.
%

Our Balanced-ETC algorithm follows the elimination framework, i.e., it maintains an active arm set $A(t) \subseteq A$, and eliminates arms from this set if they are sub-optimal with high probability.
% \junning{need to explain explore-then-commit framework?}
% \siwei{refs. BTW, is this an explore-then-commit framework or an elimination framework?}
%
To guarantee the synchronicity among different agents, the elimination only depends on the publicly shared information, i.e., the arm-reward pairs $(\pi_m(t), X_{\pi_m(t), m}(t))$ that have been broadcast. 
%We proposed Balanced-ETC algorithm to solve the federated multi-armed bandit problem. 
%
Specifically, let $N_i(t)$ be the number of times that arm $i$ has been broadcast, and $\hat{\mu}_i(t)$ be the empirical mean of the expected reward (in the broadcast information). 
%
Then by concentration analysis, one can see that with high probability, $|\mu_i -\hat{\mu}_i(t) | \le \rad_i(t) \triangleq \sqrt{2\log T/N_i(t)}$. 
% \siwei{I change $CB_i$ to $\rad_i$. If your prefer $CB_i$, just change the command at the beginning.}
%
Therefore, after receiving new information $(\pi_m(t), X_{\pi_m(t), m}(t))$, every agent can update  $N_i(t)$'s and $\hat{\mu}_i(t)$'s for all the arms, and eliminate arm $i$ from the active arm set $A(t)$ if 

\begin{equation}\label{Eq_1}
  \hat{\mu}_i(t) + \rad_i(t) \le \max_{j\in A} \hat{\mu}_j(t) - \rad_j(t).
\end{equation}
%since when Eq. \eqref{Eq_1} holds, arm $i$ is sub-optimal with high probability.

% \vspace{-0.1in}



%

As mentioned in the introduction, the main challenge here is that the imbalanced distribution of the shared data can result in severe over-exploration. 
%
For example, if all the agents are willing to share the information of arm $i$, but only one agent is willing to share the information of the optimal arm, the number of broadcast arm-reward pair of sub-optimal arm $i$ will be $M$ times larger than the number of broadcast arm-reward pairs of the optimal arm. In this case,  we need $\Omega({M\log T\over \Delta_i^2})$ explorations to eliminate arm $i$ from the active arm set, which leads to a regret of $\Omega({M\log T\over \Delta_i})$ and is far from optimal.
%
To tackle this problem, the Balanced-ETC algorithm sets a threshold to restrain over-exploration, i.e., we let the ratio between the maximum $N_i(t)$ and the minimum $N_i(t)$ in the active arm set to be less than the threshold. 
%
Specifically, when there are still some active arms that agent $m$ is willing to share (i.e., $A(t) \cap A_m \ne \emptyset$), we use $B_m(t) \triangleq {\min_{i\in A(t) \cap A_m} N_i(t)\over \min_{j \in A(t)}N_j(t)}$  to denote the balance level. %when agent $m$ pulls the arm $\argmin_{j\in A(t) \cap A_m} N_j(t)$ and share the information.
% \junning{pull arm $\pi_m$ only if $B_m(t) \le B$ before pulling }
% \siwei{$K$ is the number of arms, so I change here to be $B$}
%
As we can see, %$B_m(t)$ reflects the balance level of explorations between different arms, and 
the explorations are more imbalanced when $B_m(t)$ is larger.
%
Hence, only if the balance level $B_m(t)$ is smaller than some fixed constant $B \ge 1$ (which is given as an input of our algorithm), we let agent $m$ to pull the arm $\argmin_{j\in A(t) \cap A_m} N_j(t)$ and then broadcast the information.
%
Otherwise, we do not let agent $m$ to explore in this time step, since this can result in severe over-exploration.
%
%In this case, we just let her commit, i.e., choose the arm with the largest empirical mean in the active arm set $A(t)$. 


%The Balanced-ETC algorithm follows an explore-then-commit framework. Specifically, there are three kinds of steps: contributing step, waiting step and committing step. 

%Let $N_i(t)$ be the total number of times arm $i$ has been shared in the contributing step. Let $CB_i(t) = \sqrt{2\log T/N_i(t)}$ be the confidencd bound of arm $i$ at time slot $t$ and $\hat{\mu}_i(t) = M_i(t)/N_i(t)$ be the global empirical mean reward of arm $i$ , where $M_i(t) = \sum_{m \in S_i}\sum_{\tau <  t}\mathds{I}\{a_m(\tau)=i\}X_{i,m}(t)$.  
%Then, if 
%\begin{equation}
%  \hat{\mu}_i(t) + CB_i(t) < \max_{j\in A} \hat{\mu}_j(t) - CB_j(t)  
%\end{equation}
%, we call this arm $i$ inactive. This is because that arm $i$ is a sub-optimal arm with high probability if the above inequality holds.  Denote $A(t)$ is the set of active arms at time slot $t$ and $|A(t)|$ is the size of $A(t)$. The agents collect shared information and update estimations for active arms which have not been declared as suboptimal arms.

%In high-level ideas, the Balanced-ETC algorithm will limit the proportion of explorations of different arms to avoid overexploration problem caused by data bias.  Let $K(t) = \max_{i \in A}N_i(t)/\min_{j\in A}N_j(t)$ be imbalanced rate, where $N_i(t)$ is the the number of times arm $i$ has been played by the agents who are willing to share arm $i$ at time $t$. $K(t)$ reflects the  imbalance of exploration between different arms.  To avoid excessive imbalanced rate, the Balanced-ETC algorithm set a constant $K$ as the upper bound of $K(t)$. The algorithm can ensure $K(t) \le K$ for any time slot $t \in [T]$.

%
The pseudo code of Balanced-ETC is shown in Algorithm \ref{algorithm}.
%
At the beginning of time step $t$, each agent collects the information shared by other agents and updates $N_i(t)$'s and $\hat{\mu}_i(t)$'s for all the arms. 
%
Then they use Eq. \eqref{Eq_1} to update the active arm set $A(t)$.
%With the global estimations and the confidence bound of the arms, $A(t)$ will be updated  while some arms in $A(t-1)$ may be eliminated by Eqn1. The update of $A(t)$ applies to every agent because the estimations are based on shared information which is the same to each agent.
%
%\begin{equation}
%   A(t) = {i \in A(t-1)| \hat{\mu}_i(t) + CB_i(t) < \max_{j\in A} \hat{\mu}_j(t) - CB_j(t) }
%\end{equation}
%
Only if i) $|A(t)| > 2$ (there still require explorations); ii) $|A(t) \cap A_m | > 0$ (there are still arms that agent $m$ is willing to share and that need explorations); and iii) $B_m(t) \le B$ (exploring this arm will not result in severe over-exploration), agent $m$ will do one explore step, i.e., she chooses to pull arm $\pi_m(t) = \argmin_{i\in A(t) \cap A_m} N_i(t)$ and broadcast arm-reward pair $(\pi_m(t), X_{\pi_m(t), m}(t))$.
%
Otherwise, she will do one commit step, i.e., she chooses to pull the best active arm $\argmax_{i\in A(t)} \hat{\mu}_i(t)$ and do not share anything with others. 

%If there remains only one arm in $A(t)$, the remaining arm in $A(t)$ can be identified as optimal arm by all the agents. Thus, each agent will exploit the optimal arm in the remaining time steps. The explorations keep going if there remains multiple active arms in $A(t)$. 
%In order to increase the efficiency of collaboration, each agent $m$ will pull the active arms in set $A_m$ as much as possible. To achieve the balance of explorations between different arms, agent $m$ should choose the arm $i$ in $A_m$ with the minimum number of explorations as the candidate of $\pi_{m}(t)$. Then agent $m$ should calculate the imbalanced rate $K(t+1)$ after pulling the arm $i$. If there exist active arms in $A_m$ and $K(t+1) < K$ after pulling the candidate arm, the agent $m$ will enter contributing step. Otherwise, the agent $m$ enters the waiting phase. 


%In the contributing step, the agent $m$ will receive  the random reward $X_{\pi_m(t),m}(t) \sim D_{\pi_m(t)}$ by pulling the candidate arm of $\pi_m(t)$. Then the arm-reward pair $(\pi_m(t), X_{\pi_m(t), m}(t))$ will be broadcast to other agents. 

%In the waiting step, the agent $m$ is waiting for other agents' contributions and do not contribute at this time slot. Specifically, she will choose the arm with maximum empirical mean $\hat{\mu}$, and do not broadcast data with other agents at this time slot. The empirical mean is calculated by shared information in the contributing step.  








%%%%%%%%%%%%%%%%%%%%%%%%%%%%%%%%%%%%%%%%%%%%%%%%%%%%%%%%%%%%%%%%%%%%%%%%

\subsection{Regret Analysis}

In this section, we provide the overall regret upper bound of our Balanced-ETC algorithm, as well as its proof sketch. Due to space limit, 
detailed proofs
% some detailed proofs of the lemmas 
are deferred to the supplementary material. 

%\subsubsection{Regret Analysis}


% \siwei{First state our theorem, discuss about our regret bound (e.g., compare it with existing results), and then give a proof sketch. Move the two lemmas to proof sketch.}


\begin{theorem}\label{Th1}
The overall regret of Balanced-ETC can be upper bounded by:
\begin{equation*}
R(T) < \sum_{i = 2}^N  \frac{8(1+\sqrt{B})^2 \log T}{\Delta_{i}} + \frac{4eMN^2}{\Delta_{\min}} + 2MN.
\end{equation*}
%where $B$ is the constant we set as the upper bound of imbalanced rate. 
\end{theorem}


%In Theorem \ref{Th1}, $\sum_{i = 2}^N  \frac{8(1+\sqrt{B})^2 \log T}{\Delta_{i}} + 2N$ is the regret bound caused by explore step and $\frac{4eMN^2}{\Delta_{min}}$ is the regret bound caused by committing step.


Note that $B$ is a constant that does not depend on $N,M$ (one can simply choose $B = 1$ in practice). Hence, Theorem \ref{Th1} states that when $T$ is large enough, the overall regret of our algorithm is $O(\sum_{i=2}^N\frac{\log T}{\Delta_i})$. 
%
This is indeed the best one can do, since the overall regret lower bound is $\Omega(\sum_{i=2}^N\frac{\log T}{\Delta_i})$ even if everyone shares all her information with each other \cite{lai1985asymptotically}.
%
On the other hand, the average regret of each agent is $O(\sum_{i=2}^N\frac{\log T}{M\Delta_i} + {4eN^2\over \Delta_{\min}})$, which is $M$ times smaller than the regret of the single-agent case, and almost becomes a constant when there are sufficiently many agents in the collaboration.
%
Hence, our collaboration system is efficient in terms of the scale of players.




%which is the asymptotically optimal result. Let us consider an easier case: everyone shares all her information with each others. The case is equivalent to the standard single-player MAB case with time horizon $MT$ whose regret is $O(\sum_{i=2}^N\frac{\log T}{\Delta_i})$ for any consistent agent strategy\cite{lai1985asymptotically}. Considering the single-player case without any collaboration, the regret of any consistent agent strategy,e.g. UCB algorithm, is $O(\sum_{i=2}^N\frac{\log T}{\Delta_i})$. So the average regret of Balanced-ETC algorithm is  $O(1/M)$ of any strategy without collaboration. It means the collaboration of our algorithm is quite efficient. If there are sufficiently many agents involved, the exploration process of the algorithm can be (almost) free.

\begin{proof}[Proof Sketch of Theorem \ref{Th1}.] We first define a good event
\begin{align*}
 \mathcal{E} = \left\{ \forall t \le T, \forall i \in A, | \hat{\mu}_i(t) - \mu_i| \le \sqrt{3\log T\over 2N_i(t)}\right\},
 \end{align*}
 % \junning{delete}
 i.e., for any time step $t$ and any arm $i$, the gap between the empirical mean and the real mean is less than $\sqrt{3\log T\over 2N_i(t)}$. 
 
% And note that here the confidence radius is not the same as that we used ($\rad_i(t) = \sqrt{2\log T\over N_i(t)}$).
%

\begin{remark}
    Note that here the confidence radius in event $\mathcal{E}$ is not the same as that we used in Balanced-ETC ($\rad_i(t) = \sqrt{2\log T\over N_i(t)}$). The reason that we choose $\rad_i(t)$ to be larger than $\sqrt{3\log T\over 2N_i(t)}$ is we want to use $N_i(t)$ to obtain both an upper bound and a lower bound for $\Delta_i$. 
    %
    This is crucial for our incentive mechanism (please see details in Section \ref{Section_Incentive}).
\end{remark}

 After applying some concentration inequalities, we have: 
 \begin{lemma}\label{event1}
    The probability of $\mathcal{E}$ happens satisfies
%\begin{equation*}
    $\Pr(\mathcal{E}) \ge 1 - \frac{2N}{T}$.
%\end{equation*}
%where $N$ is the number of arms and $T$ is the horizon. 
\end{lemma}
Based on Lemma \ref{event1}, we can bound the regret when $\mathcal{E}$ does not happen as %$\E[R(T) \I[\neg \mathcal{E}]] \le MT \cdot \Pr(\neg\mathcal{E}) \le 2MN$.
\begin{equation*}%\label{Eq_2}
    \E[R(T) \I[\neg \mathcal{E}]] \le MT \cdot \Pr(\neg\mathcal{E}) \le 2MN.
\end{equation*}
Then we come to bound the regret when $\mathcal{E}$ happens. Conditioned on event $\mathcal{E}$,  for any sub-optimal arm $i \in [2,N]$, the number of explore steps pulling arm $i$ could be upper bounded by the following lemma. 
\begin{lemma}\label{suboptimal}
When event $\mathcal{E}$ happens, the optimal arm $1$ will never be eliminated. Moreover, $\forall i \in [2, N]$,  the number of explore steps pulling arm $i$ could be bounded by:
    $$ N_i(T) \le  \lceil \frac{8 \log T(1+\sqrt{B})^2}{\Delta_i^2} \rceil.$$
\end{lemma}
% \junning{delete}
%\begin{remark}
    The $(1+\sqrt{B})^2$ factor in Lemma \ref{suboptimal} is because that under our Balanced-ETC algorithm, the number of explorations on one active arm can be at most $B$ times larger than the number of explorations on another active arm.
    %
    Therefore, if there are $\Theta\left({\log T \over \Delta_i^2}\right)$ number of explorations on arm $i$, the sum of two confidence radius (sub-optimal arm $i$ and optimal arm $1$) is $\Theta(\Delta_i + \sqrt{B}\Delta_i) = \Theta((1+\sqrt{B})\Delta_i)$, and is larger than $\Theta(\Delta_i)$.
    % %
    Only if there are $\Theta\left({(1+\sqrt{B})^2\log T \over \Delta_i^2}\right)$ number of explorations, one could prove that the sum of two confidence radius is $\Theta\left({\Delta_i \over {1+\sqrt{B}}} + {\sqrt{B}\Delta_i \over {1+\sqrt{B}}}\right) = \Theta(\Delta_i)$.
%\end{remark}

%\begin{remark}
%    In fact, using $\rad_i(t) = \sqrt{2\log T\over N_i(t)}$ in Balanced-ETC and $\sqrt{3\log T\over 2N_i(t)}$ in event $\mathcal{E}$ leads to a factor of $(\sqrt{2} + \sqrt{3/2})^2$ in the upper bound of $N_i(T)$.
%    %
%    To simplify the notations, we use $(\sqrt{2} + \sqrt{2})^2 = 8$ as the upper bound in Lemma \ref{suboptimal}.
%\end{remark}

Based on Lemma \ref{suboptimal}, one can easily prove that under event $\mathcal{E}$, the expected regret in the explore steps can be upper bounded by 
    $\sum_{i = 2}^N  \frac{8(1+\sqrt{B})^2 \log T}{\Delta_{i}}$.
%$\sum_{i = 2}^N  \frac{8(1+\sqrt{B})^2 \log T}{\Delta_{i}} + N + \frac{2NM}{T}$

Finally, we consider the expected regret of commit steps under event $\mathcal{E}$. In a commit step, the agent pulls the active arm with the highest empirical mean, i.e., $\argmax_{i\in A(t)} \hat{\mu}_i(t)$. To guarantee the accuracy of empirical means $\hat{\mu}_i(t)$'s, we want to first prove that each active arm $i$ is pulled (and shared) for a sufficient number of times. This is stated in the following lemma.
\begin{lemma}\label{lemma_number_explore}
For $\forall t \in [T]$ and $\forall i \in A(t)$, we have 
%\begin{equation*}
    $N_i(t) \ge \frac{t}{N} - 1$.
%\end{equation*}
\end{lemma}
% \junning{delete}
Roughly speaking, since there is at least one agent that is willing to share the information of arm $i$ for each arm $i$, in each time step, the active arm $i$ with the least number of $N_i(t)$ must be shared once.
Hence, after $t$ time steps, each active arm must be shared for at least $\frac{t}{N}$ times. 

%With the sufficient explorations for every arm in the commit step, the regret caused by the commit step can be bounded by the following lemma. 
Based on Lemma \ref{lemma_number_explore}, we know that for any active arm $i$, $N_i(t) = \Theta(t/N)$. Along with the fact that the optimal arm is never eliminated, for any sub-optimal arm $i$, concentration inequalities tell us that its empirical mean $\hat{\mu}_i(t)$ is higher than the empirical mean of the optimal arm $\hat{\mu}_1(t)$ with probability at most $O(\exp(-c\Delta_i^2 t/N^2))$.
%
Because of this, we have the following lemma.
\begin{lemma}\label{Lemma_Commit}
When event $\mathcal{E}$ happens, the expected regret in commit steps can be bounded by
%$$
$\frac{4eMN^2}{\Delta_{\min}}$.
%$$
\end{lemma}

Summing over the expected regret in the explore steps and in commit steps when event $\mathcal{E}$ happens, as long as the expected regret when when event $\mathcal{E}$ does not happen, we finally get the regret upper bound in Theorem \ref{Th1}. %The details of the proof are deferred to Appendix due to space limit.
% \vspace{-0.2in}
\end{proof}

Although we assume that for any arm $i$, there is at least one agent who is willing to share her history information on arm $i$ with
others, our algorithm can still function properly even if there are arms that have not been shared by any user. 
%
In this case, each agent would need to independently explore these arms, and the cooperation among users would not accelerate the exploration of these arms.

% \siwei{Move the "not all arms have agents to share" part here as a remark?}

\subsection{How Much Shared Information Do We Need?}
%\siwei{I think this section name is not very good, but I do not get a good one too.}


In this section, we discuss how much shared information (i.e., the number of shared arm-reward pairs) is needed to ensure that the overall regret is close to $O(\sum_i\frac{\log T}{\Delta_i})$.  %\junning{not consider communication cost}
In the context of limited information sharing, our primary concern lies in the amount of data that users specifically share. We are more interested in which data users are willing to share, rather than the format in which the shared data is presented. Therefore, we have not taken into consideration the adoption of more efficient sharing methods by users or the compression of data to reduce communication costs.

\begin{theorem}\label{Theorem_Min_Info}
    For any algorithm in the LSI-MAMAB model, if the number of shared arm-reward pairs is $o(\log T)$, then the overall regret is lower bounded by $\Omega(MN \log T)$.
\end{theorem}

The key idea of the proof is that with only $o(\log T)$ shared arm-reward pairs, no one can make sure that which arm is the optimal one, and has to pull all the arms for $\Omega(\log T)$ times by themselves. The detailed proof is deferred to Appendix \ref{appendix E}. 
% \begin{proof}
%     Existing analysis \cite{lai1985asymptotically} proves the following asymptotic lower bound in classic bandit case: 
% \begin{equation}
%     \lim_{T\to \infty}\inf\frac{\E[N_i(T)]}{\log(T)} \ge \frac{1}{KL(D_i, D_1)},
% \end{equation}
% where $N_i(T)$ is the number of times that we pull arm $i$ until time step $T$, and $KL(D_i, D_1)$ is the KL-divergence between two reward distributions. 

% Thus, in our LSI-MAMAB model, each agent must observe (either by pulling the arm herself or by receiving the shared information from the others) the information of each arm for at least $\Omega(\log T)$ times. 
% %
% If the number of shared arm-reward pairs is $o(\log T)$, then each agent $m$ needs to pull every sub-optimal arm $i$ for at least $\Omega(\log T)$ times.
% %
% Hence her individual regret is at least $\Omega(N \log T)$, and the overall regret is at least $\Omega(MN \log T)$.

% In this framework, the observations for each arm is divided into two parts: shared information or the agent's own observations.  It means each agent has to pull each arm for $\Omega(\log T)$ times which leads to the $M\log T$ overall regret bound. 
% \end{proof}

Theorem \ref{Theorem_Min_Info} tells us that if we want the overall regret to be $O(\sum_i\frac{\log T}{\Delta_i})$, there must be at least $\Omega(\log T)$ number of shared arm-reward pairs.
%
On the other hand, based on Lemma \ref{suboptimal}, we know that with high probability, the number of shared arm-reward pairs in Balanced-ETC is upper bounded by $O(\sum_i\frac{\log T}{\Delta_i^2})$.
%
Hence there do not exist algorithms that achieve a similar overall regret upper bound as Balanced-ETC but with a much smaller number of shared arm-reward pairs.
%
This means that our Balanced-ETC algorithm is very cost-effective. 

% \siwei{Maybe emphasize that we do not care about communication cost, but only the number of shared data in the above paragraph?}



\section{Incentive Mechanism in Balanced-ETC}\label{Section_Incentive}

% \siwei{First state our theorem, discuss about our results, and then give a proof sketch. Move the lemma to proof sketch.}

Although the overall regret of Balanced-ETC is asymptotically optimal, the algorithm itself cannot ensure everyone benefits, i.e. , 
some agents may suffer from a higher individual regret (compared with not attending the collaboration system and running a 2-UCB policy herself).
%
For example, if only one agent is willing to share the information of arm $2$, then in our LSI-MAMAB model, she needs to pull (and share) arm $2$ for more times than running the 2-UCB algorithm alone, since the number of pulls on sub-optimal arms in elimination-based algorithms is always larger than in UCB-based algorithms (see a detailed example in Section \ref{Section_Experiment}). % details in our experiments in Section \ref{Section_Experiment})
%
Hence, it is necessary to apply some incentive mechanisms to achieve IR. 
%receive some compensation to make up the regret of explore steps.
%
%
%, her individual regret may be larger than the regret of single-player strategy without collaboration. In order to share information in explore step, the number of  exploring the suboptimal arm may be larger than  that in single-player case. Thus it is necessary for the agents to receive some compensation to make up the regret of explore steps.

% \junning{ex ante expectation}\siwei{Also move this paragraph to the end of this section as a remark?}
% %Although the UCB algorithm is mentioned in the definition of Information Retrieval (IR), we do not require observations of its outcomes at every decision-making step. 
% Note that the conclusion that our incentive mechanism achieves IR is an ex-ante expectation, which is reasonable in the field of algorithm design. \siwei{some refs here}
% Specifically, it is an anticipation of the outcomes or results based on available information and analysis prior to the actual occurrence. We focus on this ex-ante expectation since rational agents only participate in cooperation when they anticipate benefiting from it. %, rather than making decisions based on observing previous results at each decision-making moment.

% \junning{Practical Application Scenarios}\siwei{move this to remark 5.1? or just remove remark 5.1?}

In our incentive mechanism, the center controller is responsible for compensating the agents for sharing their data, and collecting the costs from them for reading the shared data from other agents. The specific amount of cost and compensation are given in Section 5.1. The proposed incentive mechanisms in this paper is objectively present in the real world. For example, in the current existing federal framework, it is common practice for the federated learning platform to provide compensation to the data providers and charge fees to the data users \cite{zeng2021comprehensive}. In our discussion, each agent can simultaneously act as both a data provider and a data user, receiving compensation and incurring charges from the platform. Our algorithms and experiments demonstrate that both agents and the platform can benefit from collaboration, indicating that the platform is viable and agents are willing to participate in the cooperation.


% \junning{information sharing}Notably, our incentive mechanism does not change agents' information-sharing patterns. In fact, regardless of the presence of incentive mechanisms, agents are only willing to share information about certain specific arms, e.g. $A_m$, while withholding information about other arms entirely. 
\subsection{Incentive Mechanism}

In our incentive mechanism, at the end of the game, agent $m$ receives $\Com_m$ to compensate her individual regret, where
%\begin{eqnarray}
%    \nonumber\Com_m &=& \sum_{i\in A_m\setminus A(T)} \frac{N_{i,m}(T)}{N_i(T)}\sqrt{8(1+\sqrt{B})^2N_i(T)\log{T}}  \\
%    \label{Eq_5}&&+\sum_{i\in A\setminus A(T)} N_{i,m}'(T)\sqrt{8(1+\sqrt{B})^2}\sqrt{\frac{\log T}{N_i(T)}}.
%\end{eqnarray}
% \begin{small}
\begin{equation}\label{Eq_5}
    \Com_m = \sum_{i\in A\setminus A(T)} (N_{i,m}^e(T) + N_{i,m}^c(T))\sqrt{\frac{8(1+\sqrt{B})^2\log T}{N_i(T)}}.
\end{equation}
% \end{small}
Here $A(T)$ is the active arm set at time step $T$, $N_{i,m}^e(T)$ is the number of times that agent $m$ shares the information of arm $i$ (i.e., the number of times agent $m$ pulls arm $i$ in an explore step), and $N_{i,m}^c(T)$ is the number of times agent $m$ pulls arm $i$ in a commit step, and $N_i(T)$ is the number of times that arm $i$ has been shared (by all the agents). 
%
That is, for each time agent $m$ pulls a sub-optimal arm $i \in A\setminus A(T)$, she receives $\sqrt{\frac{8(1+\sqrt{B})^2\log T}{N_i(T)}}$ for compensation.

%
However, this compensation is much higher than necessary.
%
In fact, with high probability, this compensation $\Com_m$ is higher than the individual regret $R_m(T)$ of agent $m$.
%
To make ends meet, we also let all the agents to pay some cost for receiving the shared information, since this information does help them learn and avoid some potential regret.
%
Specifically, at the end of the game, agent $m$ also needs to pay $\Cost_m$ for the shared information, where
% \begin{small}
    \begin{equation}\label{Eq_6}
    \Cost_m = \sum_{i \in A\setminus A(T)}N_i(T) \sqrt{\frac{(\sqrt{2}-\sqrt{3/2})^4 \log T}{128(1+\sqrt{B})^2 N_i(T)}}.
\end{equation}
% \end{small}

% \begin{equation}\label{Eq_6}
%     \Cost_m = \sum_{i \in A\setminus A(T)}N_i(T)\sqrt{(\sqrt{2}-\sqrt{3/2})^2\log T \over 4N_i(T)}.
% \end{equation}
That is, for each time agent $m$ receives a shared arm-reward pair from sub-optimal arm $i \in A\setminus A(T)$, she needs to pay $\sqrt{\frac{(\sqrt{2}-\sqrt{3/2})^4 \log T}{128(1+\sqrt{B})^2 N_i(T)}}$ for this information.
%$$Cost_m = \sum_{i \in A/A(T)}\frac{\sqrt{2}-\sqrt{3/2}}{2} \sqrt{N_i(T)\log T}$$

%\begin{remark}
%    To realize this incentive mechanism, one could assume that there is a central controller (a principal), who gives these compensations to all the agents and charges them for the corresponding costs at the end of the game.
%\end{remark}


%Assume that there is a central controller (a principal) to give these compensations to all the agents, then to make ends meet, the central controller also want all the agents to pay some cost for receiving the o

%Our analysis shows that with high probability, 



%The amount of compensation is based on the information the agent shares with others. Each agent also needs to pay some cost:
%$$Cost_m = \sum_{i \in A/A(T)}\frac{\sqrt{2}-\sqrt{3/2}}{2} \sqrt{N_i(T)\log T}$$
%at the end of the game because of the information others shares with her. 

%\begin{align*}
 %  \Com_m =& \sum_{i\in A_m/A(T)} \frac{N_{i,m}(T)}{N_i(T)}\sqrt{8(1+\sqrt{B})^2N_i(T)\log{T}}  \\
%   &+\sum_{i\in A/A(T)} N_{i,m}'(T)\sqrt{8(1+\sqrt{B})^2}\sqrt{\frac{\log T}{N_i(T)}}
%\end{align*}

%To ensure that every agent can benefit from the collaboration, every agent $m$ will receive some compensation:
%\begin{align*}
%   Com_m =& \sum_{i\in A_m/A(T)} \frac{N_{i,m}(T)}{N_i(T)}\sqrt{8(1+\sqrt{B})^2N_i(T)\log{T}}  \\
 %  &+\sum_{i\in A/A(T)} N_{i,m}'(T)\sqrt{8(1+\sqrt{B})^2}\sqrt{\frac{\log T}{N_i(T)}}
%\end{align*}

%at the end of the game, 

% \begin{figure*}[t]
% \vspace{-0.2in}
% \centering 
% \subfigure[Balanced Setting]{ 
% \label{Figure_1} 
% \includegraphics[width=0.31\linewidth]{regret in balanced case.pdf}
% %\vspace{-0.2in}
% }
% \subfigure[Imbalanced Setting]{ 
% \label{Figure_2} 
% \includegraphics[width=0.31\linewidth]{regret in imbalanced case.pdf}}
% \subfigure[Central Controller's Profits]{ 
% \label{Figure_3} 
% \includegraphics[width=0.31\linewidth]{profits in balanced case.pdf}
% %\vspace{-0.2in}
% }
% \caption{Experimental results of Balanced-ETC
% }\label{fig:image2}
% \vspace{-0.2in}
% \end{figure*}

\subsection{Theoretical Analysis}

To understand the meaning of our incentive mechanism, we first introduce the following lemma (detailed proof of which is deferred to supplementary material).

\begin{lemma}\label{range}
When event $\mathcal{E}$ happens and time horizon $T$ is large enough such that $\frac{T - 2N}{\log T} > \frac{8(1+\sqrt{B})^2N}{\Delta_{\min}^2}$ (which means that all the sub-optimal arms must be eliminated), for all sub-optimal arms $i \in [2,N], \Delta_i$ can be bounded by:
\begin{equation*}
    \sqrt{(\sqrt{2}-\sqrt{3/2})^2\log T \over N_i(T)} \le \Delta_i \le \sqrt{\frac{8(1+\sqrt{B})^2\log T}{N_i(T)}}.
\end{equation*}
\end{lemma}

% \junning{delete}
The proof of Lemma \ref{range} is quite straightforward: the second inequality could be obtained by Lemma \ref{suboptimal} directly. 
%
As for the first inequality, note that under event $\mathcal{E}$, 
\begin{equation*}
    \max_{j\in A(t)} \hat{\mu}_j(t) - \rad_j(t) \le \max_{j\in A(t)} \mu_j \le \mu_1.
\end{equation*}
Hence, we cannot eliminate arm $i$ from the active arm set if $\hat{\mu}_i(t) + \rad_i(t) > \mu_1$.
%
When $\sqrt{(\sqrt{2}-\sqrt{3/2})^2\log T \over N_i(t)} > \Delta_i$, we know that 
% \siwei{BTW, why we need the "4" in the denominator? It seems that the next equation works even if there is no such "4" in the denominator.}
\begin{equation*}
    \hat{\mu}_i(t) + \rad_i(t) \ge \mu_i + (\sqrt{2} - \sqrt{3/2})\sqrt{\log T\over N_i(t)} \ge \mu_i + \Delta_i = \mu_1,
\end{equation*}
which means that arm $i$ cannot be eliminated.
%
Therefore, since we eliminate arm $i$ from the active arm set at the end of the game, we must have that$\sqrt{(\sqrt{2}-\sqrt{3/2})^2\log T \over N_i(T)} \le \Delta_i$.

\begin{figure*}[ht]
\centering 
\subfigure[Balanced Setting]{ 
\label{Figure_1} 
\includegraphics[width=0.32\linewidth]{regret_in_balanced_case.pdf}
}
\subfigure[Imbalanced Setting]{ 
\label{Figure_2} 
\includegraphics[width=0.32\linewidth]{regret_in_imbalanced_case.pdf}
}
\subfigure[Central Controller's Profits]{ 
\label{Figure_3} 
\includegraphics[width=0.32\linewidth]{profits_in_balanced_case.pdf}
}
\caption{Experimental results of Balanced-ETC}
\label{fig:image2}
\Description{Experimental results of Balanced-ETC}
\end{figure*}

Lemma \ref{range} tells us a range for the values of $\Delta_i$'s. Hence our incentive mechanism is to let agent $m$ receive compensation as if $\Delta_i$ equals the upper bound (every time she pulls arm $i$) and pay cost as if $\Delta_i$ equals the lower bound (every time an arm-reward pair of arm $i$ is shared to her). 
%
This makes sense since if we want all the agents to benefit from this collaboration, we must compensate them by the upper bound of their loss, and charge them by the lower bound of their loss.
%
Besides, when calculating the value of $\Cost_m$, we must consider the ratio between the number of explorations of Balanced-ETC and 2-UCB (in single-agent system).
%
Since there could be more explorations in Balanced-ETC, the unit cost (of receiving one arm-reward pair of arm $i$) should also be smaller than the lower bound of $\Delta_i$ (as we set in Eq. \eqref{Eq_6}).
%
%
%Hence, we have the following theorem.

\begin{theorem}\label{Theorem_IR}
    When event $\mathcal{E}$ happens and time horizon $T$ is large enough such that $\frac{T}{\log^2 T} > \frac{N}{4\Delta_{\min}^4}$, applying Eq. \eqref{Eq_5} and Eq. \eqref{Eq_6} as the incentive mechanism in our Balanced-ETC algorithm achieves IR, i.e., for any agent $m$,
    \begin{equation*}
        R_m(T) - \Com_m + \Cost_m - R_{UCB}(T) \le 0. 
    \end{equation*}
\end{theorem}
%
The proof of Theorem \ref{Theorem_IR} is deferred to Appendix \ref{appendix F}. This result indicates that that every rational agent will join the %collaboration
collaborative learning. %, where her real gain is
%$T\mu_1 - R_m(T) + \Com_m - \Cost_m$, higher than $T\mu_1 - R_{UCB}(T)$ (her real gain by running  2-UCB). 
%

Under this incentive mechanism, the compensation that the central controller needs to pay is
\begin{align*}
    \sum_m \Com_m 
    =& \sum_m \sum_{i\in A\setminus A(T)} (N_{i,m}^e(T) + N_{i,m}^c(T))\sqrt{\frac{8(1+\sqrt{B})^2\log T}{N_i(T)}}\\
    % &=& \sum_{i=2}^N \sqrt{\frac{8(1+\sqrt{B})^2\log T}{N_i(T)}} \left(\sum_m N_{i,m}^e(T) + N_{i,m}^c(T)\right)\\
    =& \sum_{i=2}^N \sqrt{\frac{8(1+\sqrt{B})^2\log T}{N_i(T)}} \left(N_i(T) + \sum_m N_{i,m}^c(T)\right)\\
    =& O\left(\sum_{i=2}^N \sqrt{\log TN_i(T)} + MN^2\sqrt{\frac{\max_{i\ge 2}N_i(T)}{\log T}}\right),
\end{align*}
% \begin{small}
%     \begin{align*}
%         \sum_m \Com_m &=& O\left(\sum_{i=2}^N \sqrt{\frac{\log T}{N_i(T)}} \left(N_i(T) + \sum_m N_{i,m}^c(T)\right)\right) = O\left(\sum_{i=2}^N \sqrt{\log TN_i(T)}\right)
%     % \sum_m \Com_m     % =& \sum_m \sum_{i\in A\setminus A(T)} (N_{i,m}^e(T) + N_{i,m}^c(T))\sqrt{\frac{8(1+\sqrt{B})^2\log T}{N_i(T)}}\\
%     % % &=& \sum_{i=2}^N \sqrt{\frac{8(1+\sqrt{B})^2\log T}{N_i(T)}} \left(\sum_m N_{i,m}^e(T) + N_{i,m}^c(T)\right)\\
%     % =& \sum_{i=2}^N \sqrt{\frac{8(1+\sqrt{B})^2\log T}{N_i(T)}} \left(N_i(T) + \sum_m N_{i,m}^c(T)\right)\\
%     % =& O\left(\sum_{i=2}^N \sqrt{\log TN_i(T)} + MN^2\sqrt{\frac{\max_{i\ge 2}N_i(T)}{\log T}}\right),
% \end{align*}
% \end{small}


% \siwei{write the order of the second term.} \junning{need proof?}
% where the last equation is based on Lemma \ref{Lemma_Commit}.
%
The last equation holds because Lemma \ref{range} tells us $\Delta_i$ is asymptotic to $\sqrt{\frac{\log T}{N_i(T)}}$, which means that the compensation caused by commit steps has the same order as the regret caused by commit steps (Lemma \ref{Lemma_Commit}).
%
%above analysis holds because Lemma \ref{range} tells us $\Delta_i$ is asymptotic to $\sqrt{\frac{\log T}{N_i(T)}}$, which can help us bound the compensation caused by an commit step with the same upper bound order of Lemma \ref{Lemma_Commit}. 

On the other hand, the total cost received by the central controller is
\begin{align*}
    \sum_m \Cost_m 
    =& \sum_m \sum_{i \in A\setminus A(T)}N_i(T) \sqrt{\frac{(\sqrt{2}-\sqrt{3/2})^4 \log T}{128(1+\sqrt{B})^2 N_i(T)}}\\
    =& \sum_{i=2}^N M N_i(T) \sqrt{\frac{(\sqrt{2}-\sqrt{3/2})^4 \log T}{128(1+\sqrt{B})^2 N_i(T)}}\\
    =& \Omega\left(M\sum_{i=2}^N \sqrt{\log TN_i(T)}\right).
\end{align*}
% \begin{small}

% \end{small}

%
Although with fewer agents, as seen in Figure \ref{Figure_3}, the central controller might struggle to profit, potentially limiting our mechanism's practicality. However, the overall compensation will be smaller than the overall cost when there are a sufficient number of agents and $T$ is large enough. 
%
In this case, the central controller can also benefit from making the platform practical in reality.

% (see the detailed example in Section \ref{Section_Experiment}).

% Now we provide the detailed proof of Theorem \ref{Theorem_IR}.
% \begin{proof}[Proof of Theorem \ref{Theorem_IR}.]
% Firstly, for any agent $m$, once she pulls a sub-optimal arm $i$ (no matter in an explore step or in a commit step), she will suffer from a regret of $\Delta_i$, and receive a compensation of $\sqrt{\frac{8(1+\sqrt{B})^2\log T}{N_i(T)}}$.
% %
% Hence, by Lemma \ref{range}, under event $\mathcal{E}$, we could obtain that 
% \begin{equation}\label{Eq_10}
%     R_m(T) \le \Com_m.
% \end{equation}

% Secondly, we can prove the following lemma.
% % by applying a similar analysis as Lemma \ref{range}.\junning{the proof is different from Lemma \ref{range}, still corollary?}
% \begin{lemma}\label{range_ucb}
% Consider we are running the 2-UCB policy in a single-agent system.
% %
% Let $N_i'(t)$ be the number of pulls on arm $i$ until time step $t$, $\hat{\mu}_i'(t)$ be the empirical mean and $\mathcal{E}'$ be the event that
% \begin{align*}
%  \mathcal{E}' = \left\{ \forall t \le T, \forall i \in A, | \hat{\mu}_i'(t) - \mu_i| \le \sqrt{3\log T\over 2N_i'(t)}\right\}.
%  \end{align*}
% %
% Then if event $\mathcal{E}'$ happens and time horizon $T$ is large enough such that $\frac{T}{\log^2 T} > \frac{N}{4\Delta_{\min}^4}$, for all sub-optimal arms $i \in [2,N]$, we have that 
% \begin{equation*}
%     \Delta_i \ge \sqrt{(\sqrt{2}-\sqrt{3/2})^2\log T \over 4N_i'(T)}. %\le \Delta_i \le \sqrt{\frac{8(1+\sqrt{B})^2\log T}{N_i(T)}}.
% \end{equation*}
% \end{lemma}
% % \siwei{Lemma G.1? I guess the ref is not correct?}
% By Lemma \ref{range_ucb}, we know that with high probability, $N_i'(T) \ge {(\sqrt{2}-\sqrt{3/2})^2\log T \over 4\Delta_i^2}$. Hence
% \begin{eqnarray*}
%     R_{UCB}(T) = \sum_{i=2}^N N_i'(T)\Delta_i \ge \sum_{i=2}^N {(\sqrt{2}-\sqrt{3/2})^2\log T \over 4\Delta_i}.
% \end{eqnarray*}
% %
% On the other hand, 
% \begin{eqnarray*}
%     \Cost_m &=& \sum_{i=2}^N N_i(T) \sqrt{\frac{(\sqrt{2}-\sqrt{3/2})^4 \log T}{128(1+\sqrt{B})^2 N_i(T)}}\\
%     % &\le& \sum_{i=2}^N {(\sqrt{2}-\sqrt{3/2})^2\log T \over 4}\sqrt{\frac{N_i(T)}{8(1+\sqrt{B})^2\log T}}\\
%     &\le& 
%     \sum_{i=2}^N {(\sqrt{2}-\sqrt{3/2})^2\log T \over 4\Delta_i},
% \end{eqnarray*}
% where the last inequality holds because of Lemma \ref{range}. 
% % \siwei{It seems that here is a problem, i.e., we need to use the upper bound of $N_i(T)$, but not the lower bound of $N_i(T)$. This may cause that $\Cost_m$ is higher than $R_{UCB}(T)$. Please fix this problem.}
% %
% This means that 
% \begin{equation}\label{Eq_11}
%     \Cost_m \le R_{UCB}(T).
% \end{equation}
% Along with Eq. \eqref{Eq_10}, we finish the proof of Theorem \ref{Theorem_IR}.
% \end{proof}

% % \begin{remark}
%     % \junning{ex ante expectation}\siwei{Also move this paragraph to the end of this section as a remark?}
% %Although the UCB algorithm is mentioned in the definition of Information Retrieval (IR), we do not require observations of its outcomes at every decision-making step. 
Note that the conclusion that our incentive mechanism achieves IR is an ex-ante expectation, which is reasonable in the field of algorithm design \cite{hammond1981ex,farina2018ex,babichenko2021bayesian}. 
%

Specifically, it is an anticipation of the outcomes or results based on available information and analysis prior to the actual occurrence. We focus on this ex-ante expectation since rational agents only participate in cooperation when they anticipate benefiting from it.
% \end{remark}



\section{Experiments}\label{Section_Experiment}

\balance

%\begin{figure*}[t]
%\begin{multicols}{3}
%    \includegraphics[width=\linewidth]{ICML23/regret with agents in balanced case.png}\par 
%    \includegraphics[width=\linewidth]{regret with compensation in balanced case.png}\par 
%    \includegraphics[width=\linewidth]{regret with agents in imbalanced case.png}\par 
%    \includegraphics[width=\linewidth]{regret with compensation in imbalanced case.png}\par     
%    \includegraphics[width=\linewidth]{regret with different B.png}\par 
 %   \includegraphics[width=\linewidth]{ICML23/regret in different cases.png}\par 
%\end{multicols}

% \begin{subfigure}
%     \centering
 %    \includegraphics[scale = 0.3]{regret with agents in balanced case.png}
%     \caption{regret of multiple agents}
%     \label{fig:my_label}
% \end{subfigure}

% \begin{subfigure}
 %    \centering
 %    \includegraphics[scale = 0.3]{regret with agents in imbalanced case.png}
%     \caption{regret with compensation}
 %    \label{fig:my_label}
% \end{subfigure}

% \begin{subfigure}
%     \centering
%     \includegraphics[scale = 0.3]{ICML/regret with different K.png}
%     \caption{regret with different K}
%     \label{fig:my_label}
% \end{subfigure}


%\caption{Experimental results}
%\label{fig:image2}
%\end{figure*}



In this section, we present experimental results for our Balanced-ETC algorithm and incentive mechanism.
% (some results are deferred to Appendix \ref{appendix I}). 
Specifically, in our experiments in main text, there are 10 arms with an expected reward vector
$\mu = [0.9, 0.8, 0.7, 0.6, 0.5, 0.4, 0.3, 0.2, 0.1, 0]$.
%
As for the information sharing structure, we assume that every agent is only willing to share the information of \emph{one} arm, and consider two settings: the balanced setting and the imbalanced setting. 
%
In the balanced setting (Figure \ref{Figure_1} and \ref{Figure_3}), $|S_i|$ is the same for all arm $i$, i.e., the number of agents who are willing to share the information of any arm is the same. %We set $T = 10^6$, $B = 1$.
%
In the imbalanced setting (Figure \ref{Figure_2}), $|S_1| = |S_2| = 1$, while the other arms' $|S_i|$ is the same, i.e., only few agents are willing to share the good arms. %We also set $T = 10^6$, $B = 1$.
%
All these results take an average over 100 independent runs.

In Figure \ref{Figure_1} and Figure \ref{Figure_2}, we set $T = 10^6$, $B = 1$, and compare the performance of Balanced-ETC under different number of agents. 
%
We can observe that in both the balanced and imbalanced settings, the overall regret (green line) doesn't increase significantly as the number of agents grows. Hence, the \emph{average} individual regret (blue line) keeps decreasing and tends to be zero when there are more agents involved. This accords with our analysis, and demonstrates the effectiveness of our collaboration system.
%
However, as we can see, if we do not apply any incentive mechanism, then the max original individual regret (red line) can be larger than the regret of running the 2-UCB policy alone (black line), especially when there are few agents involved or when the sharing structure is imbalanced. 
%
After adding our incentive mechanism (with compensation $\Com_m$ and cost $\Cost_m$), the maximum incentive individual regret (red imaginary line) is always lower than the regret of running the 2-UCB policy alone (black line), and becomes almost $0$. 
% (by receiving compensation $\Com_m$ and paying cost $\Cost_m$)
%
This also accords with our analysis, and demonstrates that our incentive mechanism can achieve IR, i.e., 
it makes sure that every agent who joins this collaboration system can benefit. 
%
In Figure \ref{Figure_3}, we set $T = 10^6$, $B = 1$, and compare the profits of the central controller (i.e., $\sum_m \Cost_m - \sum_m \Com_m$) under different number of agents. 
%
We can see that the profit increases linearly in the number of agents, and 
 surpasses 0 at around 4k agents.
 % becomes higher than 0 when there are about 4k agents. 
%
This also accords with our analysis, and empirically shows that the central controller can also benefit from making the platform practical in reality, as long as there are sufficient agents participating.
%


%it can become profitable when the number of agents is sufficiently large according to Theorem \ref{Theorem_Profits}. In Figure \ref{Figure_3}, the result empirically shows that the central controller can also benefit from making the platform practical in reality.
%since with high probability, our $\Com_m$ is larger than the original individual regret, and $\Cost_m$ is much smaller than $\Com_m$.
%
%Hence, our incentive mechanism



%Because of this, we further use Figure \ref{Figure_3} and Figure \ref{Figure_4} to demonstrate the effectiveness of our incentive mechanism. We can see that after adding our incentive mechanism (by receiving compensation $\Com_m$ and paying cost $\Cost_m$), the maximum incentive individual regret (red 
%imaginary line) is always lower than the regret of running the UCB policy alone (black line), and becomes almost $0$. 
%
%This also accords with our analysis, since with high probability, our $\Com_m$ is larger than the original individual regret, and $\Cost_m$ is much smaller than $\Com_m$.
%
%These figures 


%In fact, it becomes almost $0$, since with high probability, our $\Com_m$ is larger than the 

%We also show the performances of our algorithm with different parameters $B$ and analyze how to select $B$. All the results in Figure[12345] are averaged over 100 runs of horizon $T = 10^6$. In our experiments, there are 10 arms with expected reward vector $\mu = [0.9, 0.8, 0.7, 0.6, 0.5, 0.4, 0.3, 0.2, 0.1, 0]$.  


%Figure 1 and Figure 2  demonstrate the performance of the overall regret of the Balanced-ETC algorithm in balanced case and imbalanced case. The balanced case is the case that active arms are explored with similar times. The imbalanced case is the case that the information of arm $2$ is shared by only one agent. As we analyze in Theorem\ref{Th1}, the overall regret is almost independent with the number of agents $M$. The average regret shows the Balanced-ETC algorithm performs better than single-player algorithm and tends to zero when $M$ goes to infinity.  We also find that the maximum individual regret of FLUCB algorithm in imbalanced case is always larger than single-player regret, which means some agent cannot benefit from the collaboration. 

%Figure 3 and Figure 4 demonstrate Balanced-ETC algorithm with incentive mechanism satisfies IR, i,e, every agent's individual regret with incentive is upper bounded by the regret in single-player setting.  Thus, all agents have the motivation to join in this collaboration system. 

%Figure5 demonstrates the average regret in imbalanced case with different parameter $B$, which is designed to upper bound the balance level $B_m(t)$.  As analyzed in Theorem \ref{Th1}, when the constant $B$ changes from $1$ to $2$, the average regret increase. Thus, we should select $B$ as small as possible to achieve better regret, e.g., simply let $B = 1$. 

%In Figure 6, performance of three concrete instances are presented. There is no difference between the case with different value of $B$ in balanced case. Therefore,  we only compare the imbalanced case with $B=1$ and $B=2$. The figure shows the distribution of shared data is not important. If the distribution is imbalanced, what we need to do is to select a proper constant $B$ to bound the balance level.

% \junning{discussion}
In addition to the balanced and imbalanced settings, we also designed experiments under a random setting, with randomized reward distributions and information-sharing structures, to test the robustness of our algorithm. The results demonstrate that our algorithm continues to perform effectively in random setting. Due to space limitations, these experiments are detailed in Appendix \ref{appendix I}.

\section{Conclusion}



In this paper, we propose the LSI-MAMAB model, and design the Balanced-ETC algorithm and a corresponding incentive mechanism. This is the first work on MAMAB
% multi-agent multi-armed bandit 
problem with limited information sharing, which sheds light on many collaborative learning scenarios with data sharing constraints.
%
We show that our algorithm's overall regret is asymptotically optimal and our incentive mechanism achieves individual rationality both theoretically and empirically.

We believe that further research can build on the current work to inspire deeper exploration of limited shared information structures. For example, considering the trade-offs between the costs of information exchange and the effectiveness of cooperation, or modeling heterogeneous user preferences. There remain significant challenges in developing learning frameworks that provide users with greater autonomy and better privacy protection.

\begin{acks}
The work of Junning Shao and Zhixuan Fang is supported by Tsinghua University Dushi Program and Shanghai Qi Zhi Institute Innovation Program SQZ202312.
The work of Siwei Wang is supported in part by the National Natural Science Foundation of China Grant 62106122.
\end{acks}
%%%%%%%%%%%%%%%%%%%%%%%%%%%%%%%%%%%%%%%%%%%%%%%%%%%%%%%%%%%%%%%%%%%%%%%%

%%% The next two lines define, first, the bibliography style to be 
%%% applied, and, second, the bibliography file to be used.

\bibliographystyle{ACM-Reference-Format} 
\bibliography{sample}

%%%%%%%%%%%%%%%%%%%%%%%%%%%%%%%%%%%%%%%%%%%%%%%%%%%%%%%%%%%%%%%%%%%%%%%%
\newpage
% \appendix
\onecolumn
\begin{appendices}
% \onecolumn
\section{Proof of Lemma \ref{event1}} \label{appendix A}

We first define the good event as
\begin{align*}
 \mathcal{E} = \left\{ \forall t \le T, \forall i \in A, | \hat{\mu}_i(t) - \mu_i| \le \sqrt{3\log T\over 2N_i(t)}\right\},
 \end{align*}
 i.e., for any time step $t$ and any arm $i$, the gap between its empirical mean and its real mean is less than $\sqrt{3\log T\over 2N_i(t)}$. 
 %
 %
 %Note that here the confidence radius is not the same as that we used ($\rad_i(t) = \sqrt{2\log T\over N_i(t)}$)
%
 %We obtain the following lemma. 
 
\noindent\textbf{Lemma \ref{event1}.}
    The probability of $\mathcal{E}$ happens satisfies
\begin{equation*}
    \Pr(\mathcal{E}) \ge 1 - \frac{2N}{T}.
\end{equation*}
%where $N$ is the number of arms and $T$ is the horizon. 

\begin{proof}
Using the Hoeffding inequality \cite{hoeffding1994probability}, we obtain
\begin{align*}
    \Pr( \neg \mathcal{E}) &= \Pr \left [ \exists t \le T,\exists i \in A, | \hat{\mu}_i(t) - \mu_i| > \sqrt{3\log T\over 2N_i(t)} \right]\\
    &\le \sum_{i\in A}\sum_{t=1}^T \Pr\left[ | \hat{\mu}_i(t) - \mu_i| > \sqrt{3\log T\over 2N_i(t)} \right]\\
    &\le \sum_{i\in A}\sum_{t=1}^T \sum_{\tau = 1}^{t-1} \Pr\left[ | \hat{\mu}_i(t) - \mu_i| > \sqrt{\frac{3\log T}{2\tau}}, N_i(t) = \tau \right]\\
    &\le \sum_{i\in A}\sum_{t=1}^T \sum_{\tau = 1}^{t-1} \frac{2}{T^3} \\
    &\le \frac{2N}{T}.
\end{align*}
    
\end{proof}

% Conditioned on event $\mathcal{E}$, we obtain the following lemma.

% \begin{lemma}
%     When event $\mathcal{E}$ happens, $\forall t \in [T]$, we have the optimal arm $1 \in A(t)$. In other words, the optimal arm $1$ would never be eliminated. 
% \end{lemma}
% \begin{proof}
%     When event $\mathcal{E}$ happens, $\forall i \in [2, N]$, $\forall t \in [T]$, it holds that
%     \begin{align*}
%         \hat{\mu}_1(t) + rad_1(t) \ge \mu_1 \ge \mu_i > \hat{\mu}_i(t) - rad_i(t)
%     \end{align*}
%     which means the optimal arm would not be eliminated.
% \end{proof}

\section{Proof of Lemma \ref{suboptimal}}\label{appendix B}
% When event $\mathcal{E}$ happens, the optimal arm $1$ will never be eliminated (by Eq. \eqref{Eq_1}). Moreover, for any sub-optimal arm $i \in [2,N]$, the number of explore steps pulling arm $i$ could be upper bounded by the following lemma.

%Conditioned on event $\mathcal{E}$,  for any sub-optimal arm $i \in [2,N]$, the number of explore steps pulling arm $i$ could be upper bounded by the following lemma. 

\textbf{Lemma \ref{suboptimal}.} 
When event $\mathcal{E}$ happens, the optimal arm $1$ will never be eliminated. Moreover, $\forall i \in [2, N]$,  the number of explore steps pulling arm $i$ could be bounded by:
    $$ N_i(T) \le  \lceil \frac{8 \log T(1+\sqrt{B})^2}{\Delta_i^2} \rceil.$$


\begin{proof}
When event $\mathcal{E}$ happens, for any sub-optimal arm $i$, it holds that 
\begin{eqnarray*}
\hat{\mu}_1(t) + \rad_1(t) &\ge & \hat{\mu}_1(t) + \sqrt{3\log T\over 2N_i(t)}\\
&\ge& \mu_1\\
&\ge& \mu_i\\
&\ge& \hat{\mu}_i(t) - \sqrt{3\log T\over 2N_i(t)}\\
&\ge& \hat{\mu}_i(t) - \rad_i(t).
\end{eqnarray*}
Thus, the optimal arm $1$ will never be eliminated.


Then for sub-optimal arm $i$, we prove the upper bound of $N_i(T)$ by contradiction.  Further, when 
    $$N_i(T) > \lceil \frac{8 \log T(1+\sqrt{B})^2}{\Delta_i^2} \rceil,$$
    it means $\exists t\in[T], \exists m \in [M], N_i(t) = \lceil \frac{8 \log T(1+\sqrt{B})^2}{\Delta_i^2} \rceil, \pi_m(t) = i$. According to our algorithm, we have $N_i(t)/N_1(t)\le B_m(t) \le B$. 
    %
    Hence, under event $\mathcal{E}$, it holds that
    \begin{align*}
        \hat{\mu}_i(t) + \rad_i(t) & \le \mu_i + 2\rad_i(t) \\
        & \le \mu_1 -\Delta_i + 2\rad_i(t) \\
        & \le \hat{\mu}_1(t) +\rad_1(t) +2\rad_i(t) - \Delta_i\\
        & \le \hat{\mu}_1 - \rad_1(t) + 2\rad_1(t) +2\rad_i(t) - \Delta_i.
    \end{align*}
    However, when $N_i(t) \ge \lceil \frac{8 \log T(1+\sqrt{B})^2}{\Delta_i^2} \rceil$, we know that $N_1(t) \ge N_i(t) / B$, which implies $2\rad_1(t) +2\rad_i(t) - \Delta_i \le 0$. 
    Therefore, it holds that
    $$\hat{\mu}_i(t) + \rad_i(t) \le \hat{\mu}_1(t) - \rad_1(t),$$
    which means arm $i$ is eliminated by Eq. (\ref{Eq_1}).
    %
    This leads to a contradiction and thus proves the lemma.
\end{proof}

\section{Proof of Lemma \ref{lemma_number_explore}} \label{appendix C}

\textbf{Lemma \ref{lemma_number_explore}.}
For $\forall t \in [T]$ and $\forall i \in A(t)$, we have 
$$N_i(t) \ge \frac{t}{N} - 1.$$


\begin{proof}
We use induction to prove this lemma.

Base case: For $t = 1, ..., N$, the statement holds.

Induction step: Assume the statement holds for $\forall t < t'$. It means $\min_{j\in [N]}{N_j(t)} \ge \frac{t}{N} - 1$.

In our algorithm, at least one active arm with the smallest number $N_j(t)$ will be pulled in each time step. It means that for $\forall t \in [t', t'+N]$, we have
\begin{align*}
\min_{j\in [N]}{N_j(t)} &\ge \min_{j\in [N]}{N_j(t-N)} + 1 \\
&\ge \frac{t-N}{N} - 1 + 1 \ge \frac{t}{N} - 1   
\end{align*}


That is, the statement also holds for $\forall t \in [t', t'+N]$, establishing the induction step.
%
%Conclusion: The statement holds for every $t \in [T]$.
\end{proof}


\section{Proof of Lemma \ref{Lemma_Commit}}\label{appendix D}

\textbf{Lemma \ref{Lemma_Commit}.} When event $\mathcal{E}$ happens, the expected regret in commit steps can be bounded by
$\frac{4eMN^2}{\Delta_{\min}} $.


\begin{proof}
Let $R^c(T)$ be the expected regret in commit steps, we have
\begin{align*}
    \E[R^c(T)] & \le \E\left[\sum_{m=1}^M\sum_{t=1}^T \sum_{i=2}^N \Delta_i \I\{\pi_{m}(t) = i\}\right]\\
            & \le \E\left[\sum_{m=1}^M\sum_{t=1}^T \sum_{i=2}^N \Delta_i \I\{\hat{\mu}_i(t) > \hat{\mu}_1(t)\}\right] \\
            & \le \E\left[\sum_{i=m}^M\sum_{t=1}^T \sum_{i=2}^N \Delta_i \left(\I\{\hat{\mu}_1(t) - \mu_1 \le -\Delta_i/2\} + \I\{\hat{\mu}_i(t) - \mu_i > \Delta_i/2\}\right) \right].
\end{align*}
Using the Hoeffding inequality \cite{hoeffding1994probability}, we have
$$
\Pr(\hat{\mu}_1(t) - \mu_1 \le -\Delta_i/2) \le e^{-N_i(t)\Delta_i^2/2}  \le e^{-(\frac{t}{N} - 1) \Delta_i^2/2};
$$
$$
\Pr(\hat{\mu}_i(t) - \mu_i > -\Delta_i/2) \le e^{-N_i(t)\Delta_i^2/2} \le e^{-(\frac{t}{N} - 1) \Delta_i^2/2}.
$$
The we can obtain that
\begin{align*}
        \E[R^c(T)] & \le \E\left[\sum_{m=1}^M\sum_{t=1}^T \sum_{i=2}^N \Delta_i (\I\{\hat{\mu}_1(t) - \mu_1 \le -\Delta_i/2\} + \I\{\hat{\mu}_i(t) - \mu_i > \Delta_i/2\})\right]\\
        & \le \E\left[\sum_{m=1}^M \sum_{i=2}^N \sum_{t=1}^T  \Delta_i 2e^{-(\frac{t}{N} - 1) \Delta_i^2/2} \right]\\
        & \le \E\left[\sum_{m=1}^M\sum_{i=2}^N \Delta_i \frac{4eN}{\Delta_i^2} \right] \le \frac{4eMN^2}{\Delta_{\min}}.
\end{align*}
\end{proof}

\section{Proof of Theorem \ref{Th1}} \label{appendix Th1}
\textbf{Theorem \ref{Th1}.}
The overall regret of Balanced-ETC can be upper bounded by:
\begin{equation*}
R(T) < \sum_{i = 2}^N  \frac{8(1+\sqrt{B})^2 \log T}{\Delta_{i}} + \frac{4eMN^2}{\Delta_{\min}} + 2MN.
\end{equation*}
\begin{proof} We first define a good event
\begin{align*}
 \mathcal{E} = \left\{ \forall t \le T, \forall i \in A, | \hat{\mu}_i(t) - \mu_i| \le \sqrt{3\log T\over 2N_i(t)}\right\},
 \end{align*}

 i.e., for any time step $t$ and any arm $i$, the gap between the empirical mean and the real mean is less than $\sqrt{3\log T\over 2N_i(t)}$. 
 
 
%


 
Based on Lemma \ref{event1}, we can bound the regret when $\mathcal{E}$ does not happen as 
\begin{equation*}%\label{Eq_2}
    \E[R(T) \I[\neg \mathcal{E}]] \le MT \cdot \Pr(\neg\mathcal{E}) \le 2MN.
\end{equation*}
Then we come to bound the regret when $\mathcal{E}$ happens. 


Based on Lemma \ref{suboptimal}, we prove that under event $\mathcal{E}$, the expected regret in the explore steps can be upper bounded by 
    $\sum_{i = 2}^N  \frac{8(1+\sqrt{B})^2 \log T}{\Delta_{i}}$.
%$\sum_{i = 2}^N  \frac{8(1+\sqrt{B})^2 \log T}{\Delta_{i}} + N + \frac{2NM}{T}$

Based on Lemma \ref{Lemma_Commit}, we can upper bound the expected regret of commit steps under event $\mathcal{E}$ by $\frac{4eMN^2}{\Delta_{\min}}$. 
%With the sufficient explorations for every arm in the commit step, the regret caused by the commit step can be bounded by the following lemma. 



Summing over the expected regret in the explore steps and in commit steps when event $\mathcal{E}$ happens, as long as the expected regret when when event $\mathcal{E}$ does not happen, we finally get the regret upper bound in Theorem \ref{Th1}. %The details of the proof are deferred to Appendix due to space limit.
% \vspace{-0.2in}
\end{proof}

\section{Proof of Theorem \ref{Theorem_Min_Info}}\label{appendix E}
\textbf{Theorem \ref{Theorem_Min_Info}.} For any algorithm in the LSI-MAMAB model, if the number of shared arm-reward pairs is $o(\log T)$, then the overall regret is lower bounded by $\Omega(MN \log T)$.
\begin{proof}
    Existing analysis \cite{lai1985asymptotically} proves the following asymptotic lower bound in classic bandit case: 
\begin{equation}
    \lim_{T\to \infty}\inf\frac{\E[N_i(T)]}{\log(T)} \ge \frac{1}{KL(D_i, D_1)},
\end{equation}
where $N_i(T)$ is the number of times that we pull arm $i$ until time step $T$, and $KL(D_i, D_1)$ is the KL-divergence between two reward distributions. 

Thus, in our LSI-MAMAB model, each agent must observe (either by pulling the arm herself or by receiving the shared information from the others) the information of each arm for at least $\Omega(\log T)$ times. 
%
If the number of shared arm-reward pairs is $o(\log T)$, then each agent $m$ needs to pull every sub-optimal arm $i$ for at least $\Omega(\log T)$ times.
%
Hence her individual regret is at least $\Omega(N \log T)$, and the overall regret is at least $\Omega(MN \log T)$.

\end{proof}


\section{Proof of Lemma \ref{range}} \label{appendix F}
\textbf{Lemma \ref{range}.} When event $\mathcal{E}$ happens and time horizon $T$ is large enough such that $\frac{T - 2N}{\log T} > \frac{8(1+\sqrt{B})^2N}{\Delta_{\min}^2}$ (which means that all the sub-optimal arms must be eliminated), for all sub-optimal arms $i \in [2,N], \Delta_i$ can be bounded by:
\begin{equation*}
    \sqrt{(\sqrt{2}-\sqrt{3/2})^2\log T \over N_i(T)} \le \Delta_i \le \sqrt{\frac{8(1+\sqrt{B})^2\log T}{N_i(T)}}.
\end{equation*}
\begin{proof}
 The proof of Lemma \ref{range} is quite straightforward: the second inequality could be obtained by Lemma \ref{suboptimal} directly. 
%
As for the first inequality, note that under event $\mathcal{E}$, 
\begin{equation*}
    \max_{j\in A(t)} \hat{\mu}_j(t) - \rad_j(t) \le \max_{j\in A(t)} \mu_j \le \mu_1.
\end{equation*}
Hence, we cannot eliminate arm $i$ from the active arm set if $\hat{\mu}_i(t) + \rad_i(t) > \mu_1$.
%
When $\sqrt{(\sqrt{2}-\sqrt{3/2})^2\log T \over N_i(t)} > \Delta_i$, we know that 

\begin{equation*}
    \hat{\mu}_i(t) + \rad_i(t) \ge \mu_i + (\sqrt{2} - \sqrt{3/2})\sqrt{\log T\over N_i(t)} \ge \mu_i + \Delta_i = \mu_1,
\end{equation*}
which means that arm $i$ cannot be eliminated.
%
Therefore, since we eliminate arm $i$ from the active arm set at the end of the game, we must have that$\sqrt{(\sqrt{2}-\sqrt{3/2})^2\log T \over N_i(T)} \le \Delta_i$.   
\end{proof}

\section{Proof of Lemma \ref{range_ucb}}\label{appendix G}

\begin{lemma}\label{range_ucb}
Consider we are running the 2-UCB policy in a single-agent system.
%
Let $N_i'(t)$ be the number of pulls on arm $i$ until time step $t$, $\hat{\mu}_i'(t)$ be the empirical mean and $\mathcal{E}'$ be the event that
\begin{align*}
 \mathcal{E}' = \left\{ \forall t \le T, \forall i \in A, | \hat{\mu}_i'(t) - \mu_i| \le \sqrt{3\log T\over 2N_i'(t)}\right\}.
 \end{align*}
%
Then if event $\mathcal{E}'$ happens and time horizon $T$ is large enough such that $\frac{T}{\log^2 T} > \frac{N}{4\Delta_{\min}^4}$, for all sub-optimal arms $i \in [2,N]$, we have that 
\begin{equation*}
    \Delta_i \ge \sqrt{(\sqrt{2}-\sqrt{3/2})^2\log T \over 4N_i'(T)}. %\le \Delta_i \le \sqrt{\frac{8(1+\sqrt{B})^2\log T}{N_i(T)}}.
\end{equation*}
\end{lemma}



\begin{proof}
    We prove the lemma by contradiction. For any arm $i$, denote $L_i = {(\sqrt{2}-\sqrt{3/2})^2\log T \over 4\Delta_i^2}$. 
    %
    
    If event $\mathcal{E}'$ happens, but 
    $\exists k \in [N], N_k'(T) < L_k$. 
    Then, we divide $[0, T]$ into $L_k$ blocks with length $\frac{T}{L_k}$. By the pigeonhole principle, there must exist one block $[t_1, t_2]$, in which arm $k$ is not pulled, i.e. , $N_k'(t_1) = N_k'(t_2) < L_k$.

When event $\mathcal{E}'$ happens, we have $\forall t \in [t_1, t_2]$
\begin{align*}
    \hat{\mu}_k(t) + \rad_k(t) &= \hat{\mu}_k(t) + \sqrt{\frac{3\log T}{2N_k'(t)}} + (\sqrt{2} - \sqrt{3/2})\sqrt{\frac{\log T}{N_k'(t)}}\\
    &\ge \mu_k + (\sqrt{2} - \sqrt{3/2})\sqrt{\frac{\log T}{N_k'(t)}}\\
    &\ge \mu_k + (\sqrt{2} - \sqrt{3/2})\sqrt{\frac{\log T}{L_k}}\\
    &\ge \mu_k + 2\Delta_k \\
    &\ge \mu_1 + \Delta_k.
\end{align*}

When $\frac{T}{\log^2 T} > \frac{N}{4\Delta_{\min}^4}$, it holds that 
$$\exists i \in [N], N_i'(t_2) - N_i'(t_1) > \frac{4(\sqrt{3/2}+\sqrt{2})^2\log T}{\Delta_k^2}.$$
Otherwise, we have 
\begin{align*}
    t_2 - t_1 &= \sum_{j=1}^N N_j'(t_2) - N_j'(t_1)\\
    &\le \sum_{j=1}^N \frac{4(\sqrt{3/2}+\sqrt{2})^2\log T}{\Delta_k^2} \\
    & \le   \frac{4N(\sqrt{3/2}+\sqrt{2})^2\log T}{\Delta_{\min}^2}\\
    & < \frac{4T\Delta_{\min}^2}{(\sqrt{2}-\sqrt{3/2})^2 \log T}\\
    & < \frac{4T\Delta_{k}^2}{(\sqrt{2}-\sqrt{3/2})^2 \log T}\\
    & < \frac{T}{L_k},
\end{align*}
which contradicts with $t_2 - t_1 = \frac{T}{L_k}$.

Then for this arm $i$, it holds that $\exists t_3 \in [t_1, t_2], N_i'(t_3) - N_i'(t_1) = \frac{4(\sqrt{3/2}+\sqrt{2})^2\log T}{\Delta_k^2}$ and arm $i$ is pulled at time $t_3$. For arm $i$ and time $t_3$, we have
\begin{align*}
    \hat{\mu}_i(t_3) + \rad_i(t_3) &\le \mu_i(t_3) + \sqrt{\frac{3\log T}{2N_i'(t_3)}} + \sqrt{\frac{2\log T}{N_i'(t_3)}}\\
    &\le \mu_1 + (\sqrt{2} + \sqrt{3/2})\sqrt{\frac{\log T}{N_i'(t_3)}}\\
    &\le \mu_1 + (\sqrt{2} + \sqrt{3/2})\sqrt{\frac{\log T}{N_i'(t_3) - N_i'(t_1)}}\\
    &\le \mu_1 + \frac{\Delta_k}{2}.
\end{align*}

However, this makes a contradiction with $\hat{\mu}_k(t) + \rad_k(t) \ge \mu_1 + \Delta_k$, since the 2-UCB algorithm always chooses the arm with the largest UCB to pull. 
%%
%We observe that arm $i$ cannot be pulled at time $t_3$, i.e.,
%$$\hat{\mu}_k(t_3) + \rad_k(t_3) \ge \mu_1 + \Delta_k \ge \mu_1 + \frac{\Delta_k}{2} \ge \hat{\mu}_i(t_3) + \rad_i(t_3),$$
%which leads to a contradiction and proves the lemma.
%
%
%
\end{proof}

\section{Proof of Theorem \ref{Theorem_IR}} \label{appendix H}
\textbf{Theorem \ref{Theorem_IR}.}
    When event $\mathcal{E}$ happens and time horizon $T$ is large enough such that $\frac{T}{\log^2 T} > \frac{N}{4\Delta_{\min}^4}$, applying Eq. \eqref{Eq_5} and Eq. \eqref{Eq_6} as the incentive mechanism in our Balanced-ETC algorithm achieves IR, i.e., for any agent $m$,
    \begin{equation*}
        R_m(T) - \Com_m + \Cost_m - R_{UCB}(T) \le 0. 
    \end{equation*}

\begin{proof}
Firstly, for any agent $m$, once she pulls a sub-optimal arm $i$ (no matter in an explore step or in a commit step), she will suffer from a regret of $\Delta_i$, and receive a compensation of $\sqrt{\frac{8(1+\sqrt{B})^2\log T}{N_i(T)}}$.
%
Hence, by Lemma \ref{range}, under event $\mathcal{E}$, we could obtain that 
\begin{equation}\label{Eq_10}
    R_m(T) \le \Com_m.
\end{equation}

Secondly, by Lemma \ref{range_ucb}, we know that with high probability, $N_i'(T) \ge {(\sqrt{2}-\sqrt{3/2})^2\log T \over 4\Delta_i^2}$. Hence
\begin{eqnarray*}
    R_{UCB}(T) = \sum_{i=2}^N N_i'(T)\Delta_i \ge \sum_{i=2}^N {(\sqrt{2}-\sqrt{3/2})^2\log T \over 4\Delta_i}.
\end{eqnarray*}
%
On the other hand, 
\begin{eqnarray*}
    \Cost_m &=& \sum_{i=2}^N N_i(T) \sqrt{\frac{(\sqrt{2}-\sqrt{3/2})^4 \log T}{128(1+\sqrt{B})^2 N_i(T)}}\\
    % &\le& \sum_{i=2}^N {(\sqrt{2}-\sqrt{3/2})^2\log T \over 4}\sqrt{\frac{N_i(T)}{8(1+\sqrt{B})^2\log T}}\\
    &\le& 
    \sum_{i=2}^N {(\sqrt{2}-\sqrt{3/2})^2\log T \over 4\Delta_i},
\end{eqnarray*}
where the last inequality holds because of Lemma \ref{range}. 
% \siwei{It seems that here is a problem, i.e., we need to use the upper bound of $N_i(T)$, but not the lower bound of $N_i(T)$. This may cause that $\Cost_m$ is higher than $R_{UCB}(T)$. Please fix this problem.}
%
This means that 
\begin{equation}\label{Eq_11}
    \Cost_m \le R_{UCB}(T).
\end{equation}
Along with Eq. \eqref{Eq_10}, we finish the proof of Theorem \ref{Theorem_IR}.
\end{proof}

\section{Experimental details.} \label{appendix I}
The type of compute worker used in our experiment is CPU. Our experimental environment is a machine with 96 Intel(R) Xeon(R) Gold 5220R CPUs @ 2.20GHz, with an x86\_64 architecture.
\subsection{Regret/Time experiments} 
\begin{figure}[h]
    \centering
    \includegraphics[scale = 0.5]{regret_with_T_in_balanced_case.pdf}
    \caption{regret with T}
    \label{fig:enter-label}
\end{figure}
To demonstrate that our algorithm can still function properly when one person shares multiple arms and to show the relationship between regret and T, we conduct the following Regret/Time experiments. In this experiments, there are 20 agents and 10 arms with an expected reward vector $\mu = [0.95, 0.85, 0.75, 0.65, 0.55, 0.45, 0.35, 0.25, 0.15, 0.05]$. we set $T = 10^5$, $B = 1$ and each agent shares multiple arms (more than one arm). Specifically, For agent m, her non-sensitive arm set is $\{m \bmod 10, (m + 1) \bmod 10\}$. All these results take an average over 100 independent runs.
We can see that the average regret of Balanced-ETC is significantly less than the regret of 2-UCB, which means our algorithm can effectively reduce the overall regret.

\subsection{Experimental results of random setting}
To enhance the experimental section, we design experiments involving randomized reward distributions and randomized information-sharing structures. These experiments are more practical in nature and aim to validate our theoretical findings. In the setting we considered, the expected reward for each arm is a \textbf{random} value within the interval [0, 1], and the arm that each agent is willing to share is \textbf{randomly} selected. Under the condition that each arm has at least one agent willing to share, we conducted 100 experiments and averaged the results.

\begin{figure}[h]
    \centering
    \includegraphics[scale = 0.5]{regret_in_random_case.pdf}
    \caption{experimental results of random setting}
    \label{fig:enter-label}
\end{figure}

Our experiments confirm our previous findings, showing that the average individual regret (blue
line) keeps decreasing and tends to be zero when there are more
agents involved. This indicates that our algorithm continues to perform effectively in a more practical environment.
\end{appendices}
\end{document}

%%%%%%%%%%%%%%%%%%%%%%%%%%%%%%%%%%%%%%%%%%%%%%%%%%%%%%%%%%%%%%%%%%%%%%%%

