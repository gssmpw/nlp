The primary contribution of this research is the systematic evaluation of key parameters that affect the accuracy of the Coloured ICP algorithm to measure checkerboard centres in unordered point clouds. We explored parameters such as Gaussian noise, translation, and rotation in a controlled setup using synthetic data. The finalised settings are validated using real scanned datasets, demonstrating that the approach effectively measures target centres. When compared to reference methods the approach delivers the expected accuracies. The approach is capable of handling unordered point clouds in the presence of significant noise both in range and intensity.

The experiments have also raised some new questions around the quality of the low-cost scanner data. At the moment, we do not know enough about the physical or optical properties of the sensor. For example, the 'bleeding' that is visible in Figure \ref{fig:binarisation} (a) could be caused by the spot size of the scanner or a cross-talk on the detector between subsequent point acquisitions. Likewise, the interaction of reflectance and range measurement is a continuing issue. Further investigations would be beneficial to create a better model of the sensor's characteristics. This would also aid the simulation of the sensor and could provide more realistic synthetic data.