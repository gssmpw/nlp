
\begin{figure*}[ht]
\minipage{0.32\textwidth}
  \includegraphics[width=\linewidth]{visuals/LIDAR_Synthetic_crop.png}
\endminipage\hfill
\minipage{0.32\textwidth}
  \includegraphics[width=\linewidth]{visuals/LIDAR_RTC360_crop.png}
\endminipage\hfill
\minipage{0.32\textwidth}%
  \includegraphics[width=\linewidth]{visuals/LIDAR_MID360_crop.png}
\endminipage
\caption{Visual comparison of a synthetic target template, survey-grade instrument scan of a target and low-cost scan of a target.}
\label{fig:target_visual}
\end{figure*}

It is a common workflow scenario for terrestrial laser scanning that several separate stations are necessary to capture a scene. The alignment of the stations into a common reference frame is usually referred to as registration. Several alternative approaches exist to solve the alignment problem. Often, approaches for terrestrial laser scanning can be traced back to classical surveying, photogrammetry, or computer vision. We can categorise the approaches into marker based, sensor based or data driven approaches \citep{pfeifer_early_2008}. The marker-based approaches are still expected to deliver the highest accuracy. 

Similar use cases for high accuracy targets arise for easy integration of terrestrial laser scans with other measurement technologies such as total station or laser tracker. Also, the long-term observation of single qualified points for example in monitoring or metrology applications can be directly accomplished with targets. For this reason, commercial laser scanning systems provide vendor specific software solutions to detect markers in laser scans from their systems. Although different target designs exist \citep{jansen_decreasing_2019}, checkerboard style targets have been universally adopted across vendors for reasons of cost and versatility.

Due to the raster-wise sampling of most common terrestrial laser scanning systems the back-scattered intensity can be represented in a matrix structure \citep{bohm_automatic_2007,sanchez_castillo_semantic_2021}. This allows for the adoption of robust and widely available image processing algorithms to perform target detection and measurement in 2D. Unfortunately, for non-standard laser scanners, this is not always possible. Some newer generation scanners deliver unordered point clouds \citep{arteaga2019}. It is not possible to apply image processing algorithms directly to this data type. One possible approach is to project (a portion of) the unordered point cloud into a plane and use a 2D approach in the projection plane \citep{ge_target_2015, goo2024zero}. This would then allow to revert to standard image processing. 

In this work, we want to follow another approach. We formulate the problem of measuring a target's centre as a template matching problem in 3D. We distinguish the measurement problem from the detection problem. Detection tries to find the target in the overall scene. Here we are concerned with finding the exact centre of the target when its approximate location is already known. We solve the template matching directly on the raw 3D point cloud data and thus avoid the projection of the data to 2D described above. 

In unordered 3D data there is no direct equivalent to a cross-correlation. Instead, the Iterative Closest Point (ICP) category of algorithms serves a similar purpose for template matching \citep{besl_method_1992}. It is well-known that ICP does not perform well on planar surfaces. Therefore, additional information needs to be taken into account. Using per-point intensity for ICP is a common approach and has been successfully applied to this problem \citep{liang_fast_2024}. 

Naively, we can extend the 3D geometry information to form a 4D vector by adding intensity \citep{feldmar_extension_1997}. However, the more recent literature of \cite{park2017} suggests a different approach. Two error terms separating geometry and intensity (or colour) are formulated and optimised for. This solution is reported to have superior precision over previous approaches. An open implementation of this algorithm is available in the widely adopted Open3D library \citep{Zhou2018}. Based on this implementation, we perform several experiments both using synthetic and real data to test the precision of the target measurement using a checkerboard target template. 


%***similar to above ***


%Accurately identifying the centres of chequerboard targets in point clouds is a critical step in industrial metrology. This process is significant for evaluating the performance of sensors under challenging conditions. LIDAR sensors, which generate unordered point clouds of variable quality, present significant challenges to precise measurement \cite{arteaga2019}. Modern automotive LIDAR systems, such as those developed by LIVOX, employ advanced scanning methods but still produce disordered point clouds, complicating traditional measurement approaches \cite{zhang2023}. Noise, inconsistent point densities, and the lack of an ordered data structure exacerbate the difficulty of detecting targets in these datasets.

%Accurate detection of chequerboard centres in point clouds has broad industrial applications, including robotic calibration, quality control, and precision assembly tasks \cite{Liang2024}. Ensuring precise alignment reduces errors and improves operational efficiency in manufacturing. Furthermore, monitoring the reliability of the sensor over time is critical, as deviations in the detected centre from the actual centre may signal sensor errors that could impact the dependent systems \cite{arteaga2019}.

%Traditional image processing techniques, designed for ordered data grids, struggle to handle the unordered nature of LIDAR-generated point clouds. This limitation poses significant challenges for accurately detecting chequerboard targets, particularly in noisy datasets or those with varying point densities, such as those from LIVOX Mid-40 and LIVOX Mid-360 sensors \cite{fryskowska2019}. Consequently, alternative methods are necessary to process such unstructured data effectively. One promising approach is the Coloured ICP algorithm, which offers robustness in point cloud registration. Optimising its parameters, this research aims to improve the precision of centre detection in unordered point clouds, ensuring reliable sensor performance in industrial environments \cite{park2017}.

%Early attempts to address this challenge involved developing a custom algorithm inspired by the work of Williams and Bennamoun \cite{williams2001}. The approach involved capturing multiple low-quality point clouds of the same target with slight variations in position or angle, aligning them globally to create a composite checkerboard with richer information. After denoising and filtering, the goal was to compute the accurate centre using analytical methods or ICP. However, practical implementation revealed significant complexity, and even synthetic data with minimal displacement yielded poor registration accuracy. After extensive experimentation and discussions with my supervisor, it became clear that improving an existing algorithm, such as Coloured ICP \cite{park2017}, was a more viable approach than developing a new one from scratch.

%The primary contribution of this research is the systematic evaluation of key parameters that affect the accuracy of the Coloured ICP algorithm to detect checkerboard centres in unordered point clouds. Parameters such as Gaussian noise for synthetic target checkerboard, translation, and rotation. The optimised settings were validated using actual datasets, demonstrating that carefully tuned parameters significantly enhance detection accuracy. This work provides a reliable framework for industrial metrology, offering practical guidance for selecting reference checkerboards and adjusting algorithm parameters based on the characteristics of the input point clouds.