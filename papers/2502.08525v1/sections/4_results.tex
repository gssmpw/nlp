\begin{figure}[t]
    \subfigure[RMSE of Coloured ICP ]{
        \includegraphics[width=1\columnwidth]{final_results/RMSE_noise_shift_rot.png}
    }
    \subfigure[Success Rate of Coloured ICP ]{
        \includegraphics[width=1\columnwidth]{final_results/SR_noise_shift_rot.png}
    }

    \caption{Results of Coloured ICP with noise in the target data. (a) RMSE across different shifts and rotations.(lower values indicate better performance) (b) Success Rate across different shifts and rotations.(higher values indicate better performance)}
    \label{fig:both_shift_rotation_noise_results}
\end{figure}

\begin{figure}[!ht]
    \subfigure[RMSE of ICP and Color ICP]{
        \includegraphics[width=1\columnwidth]{final_results/RMSE_ICP_COLOR_ICP_only_shift.png}
    }
    \subfigure[Success Rate of ICP and Color ICP]{
        \includegraphics[width=1\columnwidth]{final_results/success_ICP_COLOR_ICP_only_shift.png}
    }

    \caption{(a) RMSE of ICP and Coloured ICP, with numbers next to each point representing the standard deviation from 100 execution counts. (b) Success Rate of ICP and Coloured ICP, comparing their performance across different shift percentages.}
    \label{fig:only_shift_results}
\end{figure}

We start by conducting experiments on synthetic data. We create the source template as a regular grid of points with the colouring of a chequerboard. In the initial experiments, the same data act as the target point cloud to be matched against. Figure \ref{fig:only_shift_results} shows that the coloured ICP is able to maintain a high success rate and a low RMSE for shifts up to 90\% when only shifts between the source point cloud and target point cloud are considered. We show pure ICP as a baseline here to reiterate that pure ICP cannot reliably find a match for planar targets.

In a second round of experiments for synthetic data, we combine shift and rotation. Figure \ref{fig:both_shift_rotation_results} shows that adding rotation reduces the success rate. For rotations up to 30 degrees, shifts up to 60\% still provide a better success rate than 70\%. Figure \ref{fig:both_shift_rotation_noise_results} shows the performance when noise is added to the target point cloud. The range for successful convergence is slightly reduced.

% \begin{figure}[!ht]
%    \subfigure[RMSE and Success Rate of ICP]{
%        \includegraphics[width=1\columnwidth]{final_results/ICP_shift_rotation.png}
%    }
%    \subfigure[]{
%        \includegraphics[width=1\columnwidth]{final_results/color_ICP_shift_rotation.png}
%    }%
%
%    \caption{Comparison of ICP and Coloured ICP results across different shifts and rotations. The results highlight the success rate differences between the two methods under varying conditions.}
%    \label{fig:both_shift_rotation_results}
%\end{figure}






\begin{figure}[h]
    \centering
    \subfigure[Without pre-processing]{
        \includegraphics[width=0.45\linewidth]{visuals/without_processing.png}
    }
    \hfill
    \subfigure[With pre-processing]{
        \includegraphics[width=0.45\linewidth]{visuals/with_pre-processing.png}
    }
    \caption{Effect of pre-processing on Coloured ICP registration.}
    \label{fig:pre_processing_effect}
\end{figure}


\begin{figure}[h]
    \centering
    \subfigure[front]{
        \includegraphics[width=0.45\linewidth]{final_results/aligned_pseude_colour top.png}
    }
    \hfill
    \subfigure[side]{
        \includegraphics[width=0.45\linewidth]{final_results/aligned_pseude_colour side.png}
    }
    \caption{Final alignment between template and scanned target from a low-cost sensor. We use pseudo colour for visualising the sensor data to better see the intensity alignment. }
    \label{fig:final_alignment_psudo_colour}
\end{figure}


The next set of experiments are conducted on real point clouds that were acquired over a test field with different checkerboard targets using both a survey-grade terrestrial laser scanner and a low-cost automotive scanner. For the low-cost system, we have no ground truth data, so we will provide qualitative results. Figure \ref{fig:ransac_before_after} shows that the RANSAC filter has effectively removed the outliers while preserving most of the checkerboard points. Figure \ref{fig:binarisation} (a) shows the 'bleeding' across the black and white edges from the low-cost system. Binarisation shown in Figure \ref{fig:binarisation} (b) generates sharper edges which has improved results in our experiments. Figure \ref{fig:pre_processing_effect} illustrates how pre-processing impacts the outcome. Without preprocessing, the results are noticeably more susceptible to noise and exhibit a tilt. In  Figure \ref{fig:final_alignment_psudo_colour}, we show the final alignment that is achieved which gives us the centre of the target via the transformation parameters obtained with the Coloured ICP.

The experiments on the survey-grade data are performed to validate the approach and allow us to compare the results to vendor-calculated reference values which we consider as ground truth values. We use a Leica RTC 360 laser scanner. This scanner is specified by the vendor with a 3D accuracy of 1.9 millimetre at 10 metre range. We used the vendor provided software Leica Register 360 to calculate reference values for the target centres. We achieved an average accuracy on the target centres of 1.1 millimetre with our approach.   





