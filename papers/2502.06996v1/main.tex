%\documentclass[preprint,review,12pt]{elsarticle}


%% Use the options 1p,twocolumn; 3p; 3p,twocolumn; 5p; or 5p,twocolumn
%% for a journal layout: 
% \documentclass[final,1p,times]{elsarticle}
 \documentclass[final,1p,times,twocolumn]{elsarticle}
% \documentclass[final,3p,times]{elsarticle}
% \documentclass[final,3p,times,twocolumn]{elsarticle}
% \documentclass[final,5p,times]{elsarticle}
% \documentclass[final,5p,times,twocolumn]{elsarticle}
 
  


\usepackage[autonum,colorhypersetup]{shortex} % PDL - trying without this creates 432 errors. Maybe not. 

\usepackage{graphicx}
\graphicspath{{figures/}}

\usepackage{acro}
%\acsetup{single,
%uppercase/list}
\acsetup{single={1},	
	uppercase/list}
	
\newacronym{rl}{RL}{Reinforcement Learning}
\newacronym{drl}{DRL}{Deep Reinforcement Learning}
\newacronym{mdp}{MDP}{Markov Decision Process}
\newacronym{ppo}{PPO}{Proximal Policy Optimization}
\newacronym{sac}{SAC}{Soft Actor-Critic}
\newacronym{epvf}{EPVF}{Explicit Policy-conditioned Value Function}
\newacronym{unf}{UNF}{Universal Neural Functional} % (this is just to make my life easier when transferring content)

%\usepackage[
%style=numeric,
%sorting=none
%]{biblatex}
%\addbibresource{2024_RL_MPC.bib} %Import the bibliography file
%\mathtoolsset{showonlyrefs}

\usepackage{lineno}

\journal{arXiv}

\begin{document}
%\maketitle



\begin{frontmatter}

%% placeholder-think of a better title
%\title{Learning robust goal-conditioned value functions through MPC and reinforcement learning}
\title{A view on learning robust goal-conditioned value functions: Interplay between RL and MPC\tnoteref{label1}}
\tnotetext[label1]{This work has
been submitted to IFAC for possible publication.}

%\tnoteref{label1}
%\tnotetext[label1]{This paper is based on a tutorial given at Upper Bound 2024 in Edmonton: \url{https://github.com/NPLawrence/RL-MPC-tutorial}}

\author[ucb]{Nathan P. Lawrence}
\ead{input@nplawrence.com}
\author[ubc_math]{Philip D. Loewen}
\author[honeywell]{Michael G. Forbes}
\author[ubc_chbe]{R. Bhushan Gopaluni}
\author[ucb]{Ali Mesbah}
\ead{mesbah@berkeley.edu}


\address[ucb]{Department of Chemical and Biomolecular Engineering, University of California, Berkeley, CA 94720, USA}
\address[ubc_math]{Department of Mathematics, University of British Columbia, Vancouver, BC V6T 1Z2, Canada}
\address[ubc_chbe]{Department of Chemical and Biological Engineering, University of British Columbia,
Vancouver, BC V6T 1Z3, Canada}
\address[honeywell]{Honeywell Process Solutions, North Vancouver, BC V7J 3S4, Canada}

\begin{keyword}
Reinforcement learning \sep model predictive control \sep goal-conditioned learning \sep robust learning and control
\end{keyword}



\begin{abstract}
\Ac{RL} and \ac{MPC} offer a wealth of distinct approaches for automatic decision-making.
Given the impact both fields have had independently across numerous domains, there is growing interest in combining the general-purpose learning capability of \ac{RL} with the safety and robustness features of \ac{MPC}.
To this end, this paper presents a tutorial-style treatment of \ac{RL} and \ac{MPC}, treating them as alternative approaches to solving \aclp{MDP}.
In our formulation, \ac{RL} aims to learn a \emph{global} value function through offline exploration in an uncertain environment, whereas \ac{MPC} constructs a \emph{local} value function through online optimization.
This local-global perspective suggests new ways to design policies that combine robustness and goal-conditioned learning.
%We then build on this local-global perspective with an emphasis on robustness and goal-conditioned learning.
Robustness is incorporated into the \ac{RL} and \ac{MPC} pipelines through a scenario-based approach.
Goal-conditioned learning aims to alleviate the burden of engineering a reward function for \ac{RL}.
%Taken together, a single policy is devised, comprising a robust, high-level \ac{RL} terminal value function in conjunction with short-term, scenario-based \ac{MPC} planning for reliable constraint satisfaction.
Combining the two leads to a single policy that unites a robust, high-level \ac{RL} terminal value function with short-term, scenario-based \ac{MPC} planning for reliable constraint satisfaction.
This approach leverages the benefits of both \ac{RL} and \ac{MPC}, the effectiveness of which is demonstrated on classical control benchmarks.
%Overall, this paper serves as a balanced treatment of core concepts in \ac{RL} and \ac{MPC}, with the goal of enabling researchers in each field to develop safe, flexible, and data-driven control policies. 
\end{abstract}
\end{frontmatter}

\section{Introduction}
%\acbarrier
\acresetall
%\linenumbers
%\modulolinenumbers[5]

%%%% high-level description about interest in RL and success of MPC, where they struggle and how they can complement each other


%% draw attention to: RL & MPC aimed at the same problem; different approaches
\Ac{RL} and \ac{MPC} are optimization-based frameworks for decision-making \citep{bertsekas2022lessons}.
Model-free \ac{RL} represents a \emph{sample-based} approach in which a control policy is improved through trial and error in an uncertain environment \citep{sutton2018ReinforcementLearning}.
On the other hand, \ac{MPC} is a \emph{systems-based} approach in which forecasts are used to select appropriate control actions \citep{borrelli2017predictive}.
Both can be understood in the context of \acp{MDP}, but have enjoyed practical success in vastly different domains \citep{forbes2015ModelPredictive,busoniu2018ReinforcementLearning,lawrence2024MachineLearning}. 


Within the setting of \acp{MDP}, \ac{RL} and \ac{MPC} can be connected through the idea of value functions \citep{bertsekas2022lessons, bertsekas1996neuro}, a mechanism for predicting future performance; see \cref{fig:mdp}\footnote{Readers familiar with \ac{MPC} can recover the usual minimization problem by thinking of $-\hat{r}$ as the stage cost. We cast \ac{MPC} as a maximization problem for consistency within the overall framework and because $\hat{r}$ can be more general than a traditional stage cost.} for a conceptual diagram and \cref{sec:related} for an overview of prior work.
However, two challenges emerge in acquiring such a value function:
\begin{enumerate}
	\item {\bf Unknown cost.}\quad Desirable performance is often difficult to quantify precisely. \Ac{MPC} typically uses quadratic cost functions because they are tractable and a stability theory is available, but the parameters in the objective are only indirectly linked with appropriate closed-loop behavior. Fine-tuning is often required. \Ac{RL}, on the other hand, can learn from reward signals that express operational goals quite succinctly, such as a ``yes/no'' stimulus, but may require many trials to capture the designer's intent.  
	\item {\bf Unknown dynamics.}\quad A hallmark of \ac{RL} is its model-free learning capability, enabling it to generate a high-performing policy without a model of the system being controlled. \ac{MPC}, of course, requires a reasonably accurate system model. Since real-world environments are never truly stationary or precisely known, robust approaches to learning and modeling are essential.
\end{enumerate}



\begin{figure}
	\includegraphics[width=\textwidth]{mdp_rl_mpc.pdf}
	\caption{\Ac{RL} and \ac{MPC} can be seen as alternative approaches to solving \acp{MDP}. However, they both leverage the idea of selecting actions by maximizing a value function $Q$. The \ac{RL} agent learns a global value function offline, while the \ac{MPC} constructs a local value function for online control. $\hat{r}$ represents a tractable reward for the \ac{MPC} agent, possibly different from the true reward signal $r$.}
	\label{fig:mdp}
\end{figure}



This paper addresses these two challenges by taking a fresh look at both \ac{RL} and \ac{MPC} from the vantage point of value function estimation.\footnote{Throughout, we refer to \emph{model-free \ac{RL}} and \emph{robust \ac{MPC}}, but simply state \ac{RL} and \ac{MPC} for brevity and generality.}
\Cref{sec:globalRL,sec:localMPC} present a classical overview of \ac{RL} and \ac{MPC} ideas:
\Iac{RL} agent explores its environment to synthesize a high-level value function from a reward signal.
This is a \emph{global} approach wherein, at deployment, the \ac{RL} agent simply queries its value-maximizing policy. 
On the other hand, \ac{MPC} represents a modular strategy in which prior physical knowledge and safety specifications are directly embedded in the form of equality and inequality constraints.
This results in a \emph{local} value structure characterized by the agent continually replanning online.



Taken together, we present a unified framework wherein the \ac{MPC} architecture can take advantage of an \ac{RL}-learned value function to calibrate its long-term cost predictions to the system of interest.
Conversely, the \ac{RL} agent benefits from the exact, constrained optimization of the \ac{MPC} module to produce safe actions.
Specialized techniques from both the \ac{RL} and \ac{MPC} literature are embraced to make training more efficient and action selection more robust.
In particular, this paper builds on the classical local-global view by bringing together scenario-based planning and goal-conditioned learning into a single agent.
%% contributions
The contributions of this paper are summarized as follows:
\begin{enumerate}
	\item We use methods from robust \ac{MPC} to design actions guided by constraints and \iac{RL} terminal value function; and a scenario-based setup inspired by \ac{MPC} to train the robust \ac{RL} agent from simulated experience data.
	\item We use goal-conditioned \ac{RL} techniques to learn a high-level value function to augment the \ac{MPC} agent.
	\item We give a tutorial-style treatment of the local-global value function perspectives of \ac{MPC} and \ac{RL}. Moreover, we elaborate on the implementation of robust \ac{MPC} methods in \iac{RL} ecosystem.
\end{enumerate}



%Briefly, \ac{RL} is concerned with distilling data into a high-level objective for \ac{MPC} to optimize.
%Robust \ac{MPC} techniques are incorporated both into the learning and action selection processes.
%This enables efficient training through a robust simulation environment in conjunction with real-time data, as well as robust performance during deployment.
%A more precise overview is given in the next section.



%\Iac{RL} agent uses its prior experience to synthesize a value function, which predicts future performance from the current state alone.
%\Ac{MPC} directly simulates future trajectories using an internal model of the environment, accounting for constraints and tabulating costs along the way.
%In this way, \ac{MPC} represents a modular control strategy in which prior physical knowledge and safety specifications are directly embedded.
%The \ac{MPC} architecture can take advantage of an \ac{RL}-learned value function to calibrate its long-term cost predictions to the system of interest.
%Conversely, the \ac{RL} agent benefits from the exact, constrained optimization of the \ac{MPC} module to produce safe actions.

%However, critical aspects regarding safety are challenging to incorporate; doing so is certainly less natural than in \ac{MPC}.





%Broadly speaking, \ac{RL} excels at synthesizing performance-driven policies from large amounts of data collected from an environment.
%However, critical aspects regarding safety are challenging to incorporate; doing so is certainly less natural than in \ac{MPC}.
%Nonetheless, from an engineering perspective, \ac{MPC} can be difficult to ``align'' with the true environment due to the complex interactions among plant-model mismatch, constraints, and cost design.
%These distinct approaches and challenges invite a complementary control framework.





%A common challenge in both \ac{RL} and \ac{MPC} is designing a suitable cost function.
%Therefore, we utilize techniques from \emph{goal-conditioned} \ac{RL} that aim to learn a meaningful policy from high-level commands.\footnote{See \cref{subsec:goalconditioned}. Briefly, goal-conditioned \ac{RL} aims to avoid shaping a reward function, for example, through various metrics and penalty terms; rather, it is aimed at learning from minimal, sparse rewards.}
%We use a simple relabeling strategy called \ac{HER} \citep{andrychowicz2017HindsightExperience} that enables efficient learning from sparse rewards with off-policy \ac{RL} methods.
%This approach is independent of the proposed interface between \ac{RL} and \ac{MPC}; rather, it is a way of easing the burden of specifying a suitable reward function.




%Moreover, the internal model and parameter distribution is used to instantiate the aforementioned robust simulation environment.
%As the value function-informed \ac{MPC} policy interacts with the true environment, the collected data is mixed with the simulation data, enabling better calibration to the system of interest in the case of distributional mismatch.




%The corresponding goal-conditioned value function is then used as the objective function in \iac{MPC} solver.
%The \ac{MPC} solver minimizes the \ac{RL}-learned value function subject to constraints and prior system knowledge.
%The system description serves two functions:
%



%%%% more specific about the key features of both we find appealing
%\Iac{RL} agent uses its prior experience to synthesize a value function, which predicts future performance from the current state alone.
%\Ac{MPC} directly simulates future trajectories using an internal model of the environment, accounting for constraints and tabulating costs along the way.
%In this way, \ac{MPC} represents a modular control strategy in which prior physical knowledge and safety specifications are directly embedded.
%The \ac{MPC} architecture can take advantage of an \ac{RL}-learned value function to calibrate its long-term cost predictions to the system of interest.
%Conversely, the \ac{RL} agent benefits from the exact, constrained optimization of the \ac{MPC} module to produce safe actions.


%%%% more specific about the key features of both we find appealing
%This paper stems from a classical foundation but incorporates more recent tools in deep \ac{RL} and robust \ac{MPC}.
%\Iac{RL} agent is able distill its prior experience into a value function. 
%In essence, this function caches all the planning required for decision making.
%%In essence, it caches all the planning into a function of the current state. 
%\Ac{MPC} directly simulates future trajectories using a dynamic model of the environment.
%Consequently, constraints and long-term costs can be incorporated into this modular structure.



%%%% more technical details and modern aspects we are brining into a classical framework
%There is a natural interface between value function learning on the \ac{RL} side and optimization on the \ac{MPC} side.






%This tutorial-style paper does not argue in favor of \ac{RL} nor \ac{MPC}.
%
%Rather, it walks through the standard building blocks of each and their complementary nature.
%More specifically, \ac{MPC} can be augmented with a terminal value function.
%This has the effect of embedding infinite-horizon behavior into a finite-dimensional optimization problem.
%We show how to learn a value function through \ac{RL}, which can be directly ported into \iac{MPC} solver.
%This augmented control law handles system dynamics and constraints (\ac{MPC}) as well as high-level objectives (\ac{RL}) embedded through a learnable value function.
%
%
%Incorporating niche methodologies is desirable and demonstrated here.
%scenario-based \ac{MPC} may be incorporated into the control law to help mitigate the effect of parameter uncertainty.
%On the \ac{RL} side, goal-conditioned learning techniques are a promising approach towards mitigating the bottleneck of reward engineering.
%Taken together, a more robust and high-level control law is obtained.


\section{Learning a global value function through RL}
\label{sec:globalRL}

%\Ac{RL} deals with the problems of learning to achieve a goal through interactions with an environment \citep{sutton2018ReinforcementLearning}.
%Later sections will draw from the tools in this section and \cref{sec:RL} to develop learning and optimization-based control schemes.


This section introduces \acp{MDP} and the \ac{RL} perspective for solving them.
Our key point is that \ac{RL} aims to produce a \emph{global} value function over the state-action space.
This contrasts with \ac{MPC}, detailed in \cref{sec:localMPC}, which builds a \emph{local} value function through the combination of constraints, costs, and replanning.
%\Cref{eq:rlmpc} 


\subsection{Markov decision processes}
\label{sec:mdp}


We consider an optimization problem of the form:
\begin{equation}
\begin{aligned}
    &\underset{\pi}{\text{maximize}} && \mathbb{E}_{s_0 \sim p\left(s_0\right)}\left[ V^{\pi}_g (s_0) \right].
\end{aligned}
\label{eq:abstractObj}
\end{equation}
The idea behind \cref{eq:abstractObj} is the following:
Starting from some \emph{state} $s_0$ in a dynamic environment, design a \emph{policy} that brings future states $s_1, s_{2}, \ldots$ to a desired \emph{goal} $g$.
The function $V^{\pi}_g$ indicates the \emph{value} of the policy $\pi$; naturally, the ``best'' policy should act as efficiently as possible.


%The \ac{RL} framework consists of two disjoint entities: the \emph{agent} and the \emph{environment}.
%The agent is responsible for selecting actions and learning from its decisions.
%It makes decisions on the basis of the \emph{state} of the environment.
%The environment is essentially everything outside of the agent.
%The bridge between the agent-environment interaction is the scalar \emph{reward} signal.
%The reward is computed based on the current state of the environment and the action selected by the agent.
%It is feedback to the agent characterizing how effective its action is at achieving some goal.
%Naturally, the agent wants to maximize the amount of reward is receives now and in the future.


The construction and implementation of a policy are carried out by an \emph{agent}; the environment can be viewed as everything outside of the agent.
The environment involves a state space $\state$, while the agent selects actions in action space $\action$ according to the policy.
In the goal-conditioned setting, we also consider a goal space $\mathcal{G}$.
For a fixed goal $g \in \mathcal{G}$ and a given state $s \in \state$, an action $a \in \action$ is applied to the environment, which produces a new state $s' \in \state$.
Successive states should eventually arrive at $g$.
We often index the states and actions in discrete time steps.
Starting from an initial state $s_0 \in \state$, we obtain a \emph{trajectory}
\[
\{ s_0, a_0, s_1, a_1, \ldots, s_t, a_t, s_{t+1}, \ldots \}.
\label{eq:traj}
\]
Crucially, we assume the state-action tuple $(s_t,a_t)$ completely characterizes the probability density for the next state $s_{t+1}$.
Informally, the predictability of $s_{t+1}$ based only on $(s_t,a_t)$ cannot be improved by knowing the entire history of the trajectory up to index $t$:
\[
\pp{p}{s_{t+1}}{s_t,a_t} = \pp{p}{s_{t+1}}{s_0, a_0, \ldots, s_t, a_t}.
\label{eq:markov}
\]
This is the so-called \emph{Markov property};
here
$p: \state \times \action \times \state \to [0,1]$ is a conditional probability density function that defines the dynamics of the environment.



The environment dynamics encompass a large set of possible trajectories of the form shown in \cref{eq:traj}.
The desirability of each transition along a trajectory is summarized by a scalar-valued function $r_g: \state \times \action \to \reals$, known as the \emph{reward}.
Writing $r_t = r_g(s_t, a_t)$ produces a reward-annotated trajectory:
\[
\{ s_0, a_0, r_0, s_1, a_1, r_1 \ldots, s_t, a_t, r_t, s_{t+1}, \ldots \}.
\label{eq:sarsa}
\]
A scalar value for a given trajectory can be assigned by specifying a constant $\gamma \in [ 0, 1 ]$ called the \emph{discount factor} and forming the discounted sum of future rewards:
\[
\sum_{t=0}^\infty \gamma^t r_g(s_t,a_t).
\label{eq:returnt0}
\]
If $\gamma = 0$, we interpret the series as $r_g(s_0,a_0)$; 
choosing $0<\gamma<1$ guarantees that the series converges (assuming the reward function is bounded) and assigns a higher value to immediate rewards than to future rewards.


The link from states to actions is captured
in the \emph{policy}, a probability density over the set of actions that depends on the current state and the selected goal. For each state-goal pair $(s,g)$ in $\state \times \mathcal{G}$, $\pp{\pi}{a}{s,g}$ defines a distribution over actions $a \in \action$.
Each policy induces a probability on the set of trajectories mentioned above: Operationally, we focus on trajectories where the sample value of $s_{t+1}$ is determined by the density $\pp{p}{s_{t+1}}{s_t,a_t}$ after $a_t$ is drawn from the density $\pp{\pi}{a_t}{s_t,g}$.


Every policy $\pi$ assigns a scalar value to each point in the state space as follows:
\begin{equation}
	V^{\pi}_g \left( s \right) = \mathbb{E}_{\pi}\left[ \sum_{t=0}^{\infty} \gamma^{t} r_g (s_t,a_t) \middle| s_0 = s \right].
\label{eq:statevalue}
\end{equation} 
Here $V^\pi_g$ is a \emph{value function}: It returns the expected long-term reward accumulated under policy $\pi$ as a function of the trajectory's starting point.
%$\mathbb{E}_\pi\left[ \cdot \right]$ is shorthand for taking the expectation over trajectories of the form \cref{eq:sarsa} induced by $\pi$.



In the context detailed above, the problem of determining a policy $\pi$ that maximizes the expected return over all possible trajectories from all possible starting points $s_0 \sim p\left(s_0\right)$ is a \emph{\acf{MDP}}.
In terse mathematical notation, our MDP is:
\begin{equation}
\begin{aligned}
    &\text{maximize} && J(\pi) = \mathbb{E}_{\pi}\left[ \sum_{t=0}^{\infty} \gamma^{t}r_g (s_t,a_t) \right]\\
    &\text{over all} && \text{policies } \pi \colon \state\times\mathcal{G} \to \mathcal{P}(\mathcal{A}).
\end{aligned}
\label{eq:mdpobjective}
\end{equation}
Here $\mathcal{P}(\action)$ is the set of probability measures on $\action$.


Problem~(\ref{eq:mdpobjective}) is easy to state, but hard to solve.
Indeed, we cannot even evaluate the objective directly, as the infinite sum already restricts us to special cases or approximations.
Moreover, the transition density $p$, which governs the system's dynamics, is generally treated as unknown.
Thus, the expectation is unavailable in closed form and must be estimated, for example, with empirical observations of the form in \cref{eq:sarsa}.
Finally, the space of all competing policies is intractable, meaning that some simply parameterized subset, such as that provided by a neural network, will have to suffice.
In what follows, we outline the \ac{RL} perspective for approximating $V^{\pi}_g$.



%\cref{eq:traj} encompasses all possible trajectories under the environment dynamics.
%Implementing a policy $\pi$ focuses the actions towards the underlying goal.
%A policy is a state and goal-dependent probability density $\pi$ over actions.
%That is, $\pp{\pi}{a}{s,g}$ defines a distribution over actions $a \in \action$ at the state-goal pair $(s, g) \in \state \times \mathcal{G}$.
%The desirability of these actions is summarized by a scalar, deterministic function $r_g: \state \times \action \to \reals$, known as the \emph{reward}.
%Writing $r_t = r_g(s_t, a_t)$, a reward-guided trajectory is produced:
%\[
%\{ s_0, a_0, r_0, s_1, a_1, r_1 \ldots, s_t, a_t, r_t, s_{t+1}, \ldots \}.
%\label{eq:sarsa}
%\]
%The combination of dynamics $p$ and reward $r_g$ compose \emph{\iac{MDP}}.




%The sequential nature of the decision-making process in \cref{eq:sarsa} indicates that the objective in \cref{eq:abstractObj} should accumulate rewards over time.
%Consider the value function
%\begin{equation}
%	V^{\pi}_g \left( s \right) = \mathbb{E}_{\pi}\left[ \sum_{t=0}^{\infty} \gamma^{t}r_g (s_t,a_t) \middle| s_0 = s \right].
%\label{eq:statevalue}
%\end{equation} 
%$V^{\pi}_g$ is a seemingly magical function that takes the present state as input and returns the future cost of implementing policy $\pi$ in the environment.
%The constant $\gamma \in [ 0, 1 ]$ is a \emph{discount factor}: If $\gamma = 0$, then the value function is shortsighted; if $\gamma = 1$, then the sum inside \cref{eq:statevalue} is likely to diverge.
%Fixing $\gamma \in (0,1)$ resolves the divergence issue (assuming the reward function is bounded) and also creates a sense of urgency for the agent.
%Revisiting \cref{eq:abstractObj}, the objective of is:
%\begin{equation}
%\begin{aligned}
%    &\text{maximize} && J(\pi) = \mathbb{E}_{\pi}\left[ \sum_{t=0}^{\infty} \gamma^{t}r_g (s_t,a_t) \right]\\
%    &\text{over all} && \text{policies } \pi \colon \mathcal{S} \to \mathcal{P}(\mathcal{A}),
%\end{aligned}
%\label{eq:mdpobjective}
%\end{equation}
%where $\mathcal{P}(\action)$ is the set of probability measures on $\action$.


%\Cref{eq:mdpobjective} is a lofty goal and it is unclear how to get started.
%For one, we cannot evaluate the objective directly, as the infinite horizon already restricts us to special cases or approximations. 
%Moreover, the dynamics are generally unknown. 
%Similarly, the expectation must be estimated, for example, with empirical data of the form in \cref{eq:sarsa}.
%Also, the space of policies is intractable, meaning some restricted parameterization, such as neural networks, will have to suffice.
%In what follows, we outline the \ac{RL} perspective for approximating $V^{\pi}_g$.


%These aspects will actually turn out to be key strengths of \ac{RL}: Solving complex decision making problems with some high-powered function approximator directly from data.
%Nonetheless, we simply point out the conceptual challenges with the objective itself.
%Finally, this is not to mention the intricacies in deploying practical algorithms.


%The space of possible policies is vast. It can be completely random, or highly expressive, such as a neural network, or highly specialized, such as an industrial controller. The reward and policy architecture often require considerable care to formulate, but we are content leaving them as abstract objects for now.


\subsection{The reinforcement learning approach}
\label{sec:RL}
%\acbarrier


%This level of generality definitely requires a precise formulation, which we will get to shortly.
%Supervised and unsupervised learning are generally performed in isolation on a given dataset, whereas \ac{RL} systems are embodied in some environment, collect data, and leverage that data to make better decisions in the future.
\Ac{RL} offers an iterative, data-driven, and flexible framework for solving dynamic tasks:
\begin{itemize}
	\item \textbf{Iterative.}\quad Exact, analytical solutions are scarce. However, general optimality conditions, based on dynamic programming, inform elegant, iterative update schemes that improve decision-making performance over time. 
	\item \textbf{Data-driven.}\quad A model of the environment is not required (although one is welcome, if available). Instead, sequential data can be used in place of a dynamic model.
	\item \textbf{Flexible.}\quad The two aspects above mean \ac{RL} can be applied in many domains. Moreover, the training process is governed by a ``reward'' signal, which is a remarkably simple way of imposing goal-directed behavior.
\end{itemize}
This paper does not dwell on the minute details of individual algorithms, but rather looks to convey some general principles and structures that guide practical \ac{RL} solution methods.


\subsubsection{Evaluate, improve, and repeat...}


It is useful to define the state-action value function:
\begin{equation}
	Q^{\pi}_g \left( s, a \right) = \mathbb{E}_{\pi}\left[ \sum_{t=0}^{\infty} \gamma^{t} r_g (s_t, a_t) \middle| s_0 = s, a_0 = a \right].
\label{eq:stateactionvalue}
\end{equation} 
Given $Q^{\pi}_g$, one can obtain $V^{\pi}_g$ as $V^{\pi}_g (s) = \mathbb{E}_{a \sim \pp{\pi}{a}{s,g}} \left[ Q^{\pi}_g (s, a) \right]$.\footnote{We often drop super/subscripts (or both) when we do not need to reference a specific policy or goal.}
Therefore, focusing on $Q$ is sufficient in light of our objective in \cref{eq:abstractObj}, which is beneficial due to the the additional degree of freedom in the action component.
Indeed, if we had some oracle mapping $\pi \to Q^\pi$, then an even better policy $\pi^{+}$ could be derived as follows:
\[
\pi^{+} (s, g) = \argmax_{a} Q^{\pi}_g (s,a).
\label{eq:improvepolicy}
\]
This is the general recipe going forward: Acquire $Q$, maximize it, and repeat.


Although we can never access $Q$ precisely, it can be estimated with samples from the environment.
Based on \cref{eq:returnt0}, the \emph{discounted return} accumulates rewards starting at some time index $t$:
\[
G_t = r_t + \gamma r_{t+1} + \gamma^2 r_{t+2} + \ldots = \sum_{k=0}^\infty \gamma^k r_{t+k}.
\label{eq:discountreturn}
\]
By averaging over trajectories, we find that 
\[
Q^{\pi}_g (s,a) = \EE_\pi \left[ G_0 \middle| s_0 = s, a_0 = a \right].
\label{eq:Qfunc}
\]
However, there is a rich structure we can exploit:
The discounted returns satisfy the recursion
\[
\begin{aligned}
G_t &= r_t + \gamma r_{t+1} + \gamma^2 r_{t+2} + \ldots \\
& = r_t + \gamma \left( r_{t+1} + \gamma r_{t+2} + \ldots \right) \\
& = r_t + \gamma G_{t+1},
\end{aligned}
\label{eq:tdreturn}
\]
which in turn implies (along with the Markov property) that $Q$ itself satisfies a tidy self-consistency relationship:
\[
Q^{\pi}_g (s,a) = r_g (s,a) + \gamma \EE_{s' \sim \pp{p}{s'}{s,a}, a' \sim \pp{\pi}{a'}{s', g}} \left[ Q^{\pi}_g (s', a') \right]. 
\label{eq:Qfixed}
\]

\Cref{eq:Qfixed} holds for any policy.
Naturally, define $Q^\star_g (s,a) = \underset{\pi}{\max}\ Q^{\pi}_g (s,a)$; indeed, if $Q^\star_g$ is available, then the optimal policy is obtained by
\[
\pi^\star (s, g) = \argmax_{a} Q^\star_g (s, a).
\label{eq:optpolicy}
\]
When we apply the recursion in \cref{eq:Qfixed}, the optimization is offset to the next time step:
\begin{equation}
\begin{aligned}
Q^\star_g(s,a) &= r_g(s,a) + \max_{\pi} \gamma \EE_{s' \sim \pp{p}{s'}{s,a}, a' \sim \pp{\pi}{a'}{s', g}} \left[ Q^{\pi}_g(s', a') \right] \\
&= r_g (s,a) + \gamma \EE_{s' \sim \pp{p}{s'}{s,a}} \left[ \max_{a' \in \action} \max_{\pi} Q^{\pi}_g (s',a') \right] \\
&= r_g(s,a) + \gamma \EE_{s' \sim \pp{p}{s'}{s,a}} \left[ \max_{a' \in \action} Q^\star_g (s',a') \right].	
\end{aligned}
\label{eq:bellmanQ}
\end{equation}
\Cref{eq:bellmanQ} is known as the \emph{Bellman optimality equation} \citep{sutton2018ReinforcementLearning, bertsekas1996neuro}.
Although we do not directly have access to $Q^{\pi}$, much less $Q^\star$, the beauty of \cref{eq:bellmanQ} is that it distills all the complexity of the original problem in \cref{eq:mdpobjective} into a one-step relation.
Essentially, the Bellman equation provides a principled theoretical target around which \ac{RL} algorithms are built.


The practical algorithms aimed at solving \cref{eq:bellmanQ} are intricate and vast.
We briefly mention two principles that pertain to future sections.

{\bf Learning from past experience.}\quad
Fix some policy $\pi$ and let it acquire experience in the form of \cref{eq:sarsa}.
Now let $\tilde{Q}$ be a tractable approximation of $Q^\star$. 
In the simplest case, $\tilde{Q}$ is a large table containing value estimates corresponding to a discrete set of state-goal-action pairs.
%For example, $Q_\theta$ could be represented by a neural network.
Importantly, $\tilde{Q}$ is some function we can evaluate at any $(s,a, g) \in \state \times \action \times \mathcal{G}$, unlike the theoretical target $Q^\star$.
With our observed data $\{ s_t, a_t, r_t, s_{t+1}, \ldots \}$, $\tilde{Q}$ can be updated to encourage its predictions to satisfy \cref{eq:bellmanQ}:
\[
\tilde{Q}(s_t, a_t, g) \leftarrow \left( 1 - \alpha \right)\tilde{Q}(s_t, a_t, g) + \alpha \left( r_t + \gamma \max_{a' \in \action} \tilde{Q}(s_{t+1}, a', g) \right),
\label{eq:Qlearning}
\]
where $\alpha > 0$ is a step size.
Note that the policy $\pi$ that collected the data samples does not appear in this update equation, hence, \cref{eq:Qlearning} is emblematic of \emph{off-policy learning} methods.\footnote{\emph{On-policy} refers to the problem of learning $Q^\pi$.}
\Cref{eq:Qlearning} comes from $Q$-learning \citep{watkins1992Qlearning} and acts as inspiration for many deep \ac{RL} algorithms, popularized by \citet{mnih2013PlayingAtari} and \citet{silver2014DeterministicPolicy}.

More practically, consider a parameterized function approximator $Q_\phi$, such as a neural network, where $\phi$ represents the trainable weights.
$Q_\phi$ is an easy-to-evaluate function that we want to satisfy the Bellman optimality equation.
Given a tuple of data $(s, a, r_g, s')$, define the target value:
\begin{equation}
q = r_g + \gamma \max_{a' \in \action} Q_\phi (s', a', g).
\label{eq:target}
\end{equation}
We can compare $Q_\phi (s,a,g)$ to $q$ and penalize $\phi$ for any mismatch.
In particular, we formulate the loss:
\[
\mathcal{L}(\phi) = \frac{1}{\abs{\mathcal{D}}} \sum_{(s, a, r_g, s') \in \mathcal{D}} \left( Q_\phi (s,a,g) - q \right)^2,
\]
where each $q$ is a tuple-dependent target defined in \cref{eq:target}, treated as training data independent of $\phi$ .
The parameters $\phi$ can then be updated using some form of gradient descent:
\[
\phi \leftarrow \phi - \alpha \grad \mathcal{L}(\phi).
\label{eq:updateQ}
\]
While these ideas give a template for learning complex policies from past experience, the underlying optimization procedure required to compute the targets in \cref{eq:target} can limit this approach in its nominal form.



{\bf Approximating the optimization process.}\quad
Based on \cref{eq:Qlearning}, a promising new policy can be designed as 
\[
\mu^{+} (s,g) = \argmax_{a} Q_\phi (s,a,g).
\label{eq:maxpolicy}
\]
However, the maximization can be expensive. Not only does does the maximization step appear in this new policy definition during rollouts, but also in the update step in \cref{eq:updateQ} for each sample.
Instead, we can define a policy $\mu_\theta$ with some ``nice'' parameterization.
While $\mu_\theta$ represents a deterministic policy, in practice, a noisy version of $\mu_\theta$ is deployed for exploration:
\[
\pp{\pi}{a}{s,g} \sim \mathcal{N} \left( \mu_\theta (s,g), \Sigma \right).
\label{eq:noisepi}
\]

With both $\mu_\theta$ and $Q_\phi$ taking on some parameterization side-by-side, they are referred to as the \emph{actor} and \emph{critic}, respectively \citep{konda1999ActorcriticAlgorithms, silver2014DeterministicPolicy}.
The idea is to use the policy to approximate the maximization operation in \cref{eq:maxpolicy}, and to use the critic to approximate the $Q$-learning target based on \cref{eq:target}.
That is,
\begin{equation}
\begin{aligned}
	q &= r_g + \gamma  Q_\phi (s', \mu_\theta(s', g), g)\\
	\phi &\leftarrow \phi - \alpha \grad_\phi \frac{1}{\abs{\mathcal{D}}} \sum_{(s, a, r_g, s') \in \mathcal{D}} \left( Q_\phi (s,a,g) - q \right)^2 \\
	\theta &\leftarrow \theta + \alpha \grad_\theta \frac{1}{\abs{\mathcal{D}}}  \sum_{(s, a, r_g, s') \in \mathcal{D}} Q_\phi (s, \mu_\theta (s, g), g).
\end{aligned}
\label{eq:dpgalg}
\end{equation}
The targets $q$ are now very simple to compute, only requiring function evaluation, rather than exact optimization:
\[
Q_\phi (s, \mu_\theta(s), g) \approx \max_{a \in \action} Q_\phi (s, a, g).
\]
This streamlines the rest of the updates, possible over large datasets.
%However, these equations are mostly emblematic of the true machinery; see seminal works by \citep{mnih2013PlayingAtari, silver2014DeterministicPolicy, lillicrap2015ContinuousControl}.



\subsection{Goal-conditioned learning}
\label{subsec:goalconditioned}


The \ac{RL} agent is tasked with achieving some goal efficiently.
However, the notion of efficiency is characterized by the reward function, which is often defined and fine-tuned by a user through various metrics and penalty terms \citep{andrychowicz2017HindsightExperience}.
Effectively designing a reward or stage cost is a common challenge in both \ac{RL} \citep{liu2022GoalConditionedReinforcement} and \ac{MPC}  \citep{forbes2015ModelPredictive}.
Ideally, one would only need to set a target goal $g$ and the agent would learn from the reward:
\begin{equation}
r_g (s,a) = 
\begin{cases}
	1 \quad \text{Goal is achieved} \\
	0 \quad \text{Otherwise}
\end{cases}
\label{eq:sparsereward}
\end{equation}
Naturally, a goal-conditioned policy produces actions $a \sim \pp{\pi}{a}{s,g}$ aimed at bringing the environment to goal $g$ and staying there.


A reward like \cref{eq:sparsereward} benefits from a great deal of flexibility.
Its minimal structure imposes no restrictions on the agent that affect \emph{how} it reaches its goal; rather, the agent only knows \emph{what} to achieve.
However, the signal produced by such a reward is extremely sparse.
Newly initialized policies are likely to accumulate a large cache of zeros.
Moreover, two different but suboptimal policies can fail in very different ways and yet receive the same feedback.


There are two paths forward:
\begin{enumerate}
	\item \textbf{Use dense rewards.}\quad Rewards, for example, of the form
		\[-r_g(s,a) = \left( s - g \right)\transpose M \left( s - g \right) + \left(\Delta a\right)\transpose R \left( \Delta a \right)\]
		provide a continuous signal to the agent that makes it easier to distinguish the utility of different actions. 
		While the meaning of the weight terms is straightforward, they are nuisance parameters that can dramatically affect how an ``optimal'' policy looks; see  \citet{forbes2015ModelPredictive} for a simple illustration.
	\item \textbf{Use hindsight.}\quad Learning through hindsight follows the simple premise that unsuccessful trials towards a task are informative \citep{andrychowicz2017HindsightExperience,eysenbach2020RewritingHistory}.
	Given an unsuccessful trajectory $\{ s_0, a_0, \ldots, s_T \}$ aimed at achieving some goal $g$, the sequence of rewards would be all zeros. However, one things is for certain: Had $s_T$ been the goal, then the policy would have been successful. 
\end{enumerate}


\citet{andrychowicz2017HindsightExperience} first proposed \ac{HER}, that is, the  use of hindsight to learn goal-conditioned policies.
\Ac{HER} is not \iac{RL} algorithm, but rather a type of replay buffer that any off-policy algorithm can sample from.
For example, all the transition tuples in a goal-conditioned trajectory are kept:
\[
\{ (s_0, g), a_0, r_0, (s_1, g), a_1, r_1, \ldots (s_T, g), a_T, r_T \}
\]
Additionally, define $s_T$ to be a fictitious goal.\footnote{We use the terminal state for simplicity. One may also sample future states from the trajectory.}
Then for each time step $i$ in a given trajectory, add the following tuple to replay memory:
\[
\Big( (s_i, s_T), a_i, \underbrace{r_{s_T}(s_i, a_i)}_{\text{New reward}}, (s_{i+1}, s_T) \Big).
\]
The resulting dataset contains both the original and hindsight-relabeled tuples, resulting in additional ``excitation'' to the goal axis in the policy and value networks.
Over time, the agent learns a better correspondence between goals and actions, making it able to reliably reach the desired targets.







%{\color{red} (Rewrite to emphasize the significance of this wrt batches of data)}
%\begin{align}
%\tilde{Q}(s_t, a_t) &\leftarrow \left( 1 - \alpha \right)\tilde{Q}(s_t, a_t) + \alpha \left( r_t + \gamma \tilde{Q}(s_{t+1}, \mu(s_{t+1})) \right)\\
%\mu(s_t) &\leftarrow \underset{\mu}{\argmax}\ \tilde{Q}(s_t, \mu(s_t)).
%\label{eq:actorcritic}
%\end{align}
%Again, these equations are mostly emblematic of the true machinery.
%For instance, the second line is usually performed using one step of gradient ascent where $\tilde{Q}$ serves as a loss with respect to the parameters of $\pi$.



%Intuitively, in light of our abstract objective in \cref{eq:abstractObj}, the ideal situation would be to have a mapping $\pi \to Q^\pi$.
%Given such a mapping, 


%Based on \cref{eq:abstractObj}, the ideal situation would be to have a mapping $\pi \to V^\pi$.\footnote{We drop super or subscripts, such as $\pi$ or $g$, when they are not needed for exposition.}
%Such a mapping is intractable, but can be estimated.
%Based on the sequence of rewards \cref{eq:sarsa}, consider the \emph{discounted return} 
%\[
%G_t = r_t + \gamma r_{t+1} + \gamma^2 r_{t+2} + \ldots = \sum_{k=0}^\infty \gamma^k r_{t+k}.
%\label{eq:discountreturn}
%\]
%By averaging over trajectories, we find that 
%\[
%Q^{\pi}(s,a) = \EE_\pi \left[ G_0 \middle| s_0 = s, a_0 = a \right].
%\label{eq:Qfunc}
%\]


%\Ac{RL} deals with the problem of \emph{learning} to make \emph{good} decisions in a \emph{complex} environment \cite{sutton2018ReinforcementLearning}.
%Iterative schemes over analytical solutions; approximations over optimality; objectives where 
%``Learning'' generally implies iterative schemes over analytical solutions to some optimization problem.
%``Good'' means our learning scheme drives the decision maker towards some optimality condition but may not actually achieve it.
%We are in such a situation of iteratively improving the decision maker because the task has some confounding factors, such as:
%\begin{itemize}
%	\item Uncertainty.
%	\item Nonlinearity.
%	\item Long-term consequences.
%\end{itemize}


%\subsection{The reinforcement learning problem}

%The \ac{RL} framework consists of two disjoint entities: the \emph{agent} and the \emph{environment}.
%The agent is responsible for selecting actions and learning from its decisions.
%It makes decisions on the basis of the \emph{state} of the environment.
%The environment is essentially everything outside of the agent.
%The bridge between the agent-environment interaction is the scalar \emph{reward} signal.
%The reward is computed based on the current state of the environment and the action selected by the agent.
%It is feedback to the agent characterizing how effective its action is at achieving some goal.
%Naturally, the agent wants to maximize the amount of reward is receives now and in the future.

%The agent selects actions according to a \emph{policy}.
%A policy can be completely random, or highly expressive, such as a neural network, or highly specialized, such as an industrial controller.
%The reward and policy architecture often require considerable care to formulate, but we are content leaving them as abstract objects for now.

%Of course, the agent is not satisfied with an arbitrary policy; it wants to figure out the ``best'' one.
%Considering the sequence of rewards arising in \cref{eq:sarsa}, a simple strategy might be to find an action that optimizes the reward at state $s$: $a \in \argmax_{a \in \action}{ r(s,a) }$.
%This is a shortsighted strategy and is generally undesirable.
%Instead, it is more likely that the cumulative reward $r_0 + r_1 + r_2 + \ldots$ will inform more intelligent decisions.
%However, over the space of all policies and environments, it is unlikely that this sum will always converge, making it a poor quantity for optimization.

%Instead, we consider the notion of \emph{discounted return} at time:
%\[
%G_t = r_t + \gamma r_{t+1} + \gamma^2 r_{t+2} + \ldots = \sum_{k=0}^\infty \gamma^k r_{t+k},
%\label{eq:discountreturn}
%\]
%where $\gamma \in [0,1]$ is a fixed discount parameter.
%%Intuitively, discounting creates a sense of urgency for the agent. 
%More precisely, if the reward function is bounded and $\gamma \in (0,1)$, then $\abs{G_t} < \infty$, independent of the policy or environment.

%The objective of the agent is to solve the following optimization problem:
%\[
%&\text{maximize} && J(\pi) = \EE_{\pi}\left[ G_0 \right] \\
%&\text{over all} && \text{policies } \pi
%\]
%\begin{equation}
%\begin{aligned}
%    &\text{maximize} && J(\pi) = \mathbb{E}_{\pi}\left[ \sum_{t=0}^{\infty} \gamma^{t}r(s_t,a_t) \right]\\
%    &\text{over all} && \text{policies } \pi \colon \mathcal{S} \to \mathcal{P}(\mathcal{A}),
%\end{aligned}
%\label{eq:mdpobjective2}
%\end{equation}
%where $\mathcal{P}(\action)$ is the set of probability measures on $\action$.


% discussion: why is the RL objective challenging?
%\Cref{eq:mdpobjective} is a lofty goal and it is unclear how to get started.
%For one, we cannot evaluate the objective directly:
%The infinite horizon already restricts us to special cases or approximations. Moreover, the dynamics are generally unknown. Similarly, the expectation must be estimated, for example, with empirical data of the form in \cref{eq:sarsa}.
%Also, the space of policies is intractable, meaning some restricted subset will have to do.
%These aspects will actually turn out to be key strengths of \ac{RL}: Solving complex decision making problems with some high-powered function approximator directly from data.
%Nonetheless, we simply point out the conceptual challenges with the objective itself.
%Finally, this is not to mention the intricacies in deploying practical algorithms.


%\subsection{Abstracting the objective through value functions}
%
%
%As discussed, the objective in \cref{eq:mdpobjective} is unwieldy.
%Before we try to solve it, let us first extract some structure from the problem, which will guide the ensuing solution methods.
%
%Start with the infinite sequence in \cref{eq:sarsa}.
%It contains enough information to construct a sequence of returns:
%\[
%G_0, G_1, \ldots, G_t, G_{t+1}, \ldots
%\]
%However, constructing $G_0$, then $G_1$, and so on makes little sense because $G_1$ contains ``less'' information than $G_0$.
%Instead, notice that any successive terms $G_t, G_{t+1}$ only differ by $r_t$:
%\[
%\begin{aligned}
%G_t &= r_t + \gamma r_{t+1} + \gamma^2 r_{t+2} + \ldots \\
%& = r_t + \gamma \left( r_{t+1} + \gamma r_{t+2} + \ldots \right) \\
%& = r_t + \gamma G_{t+1}.
%\end{aligned}
%\label{eq:tdreturn}
%\]
%Suppose an estimate for some $G_T$ is available.
%Then \cref{eq:tdreturn} says we can work backwards to fill in the rest $G_{T-1}, \ldots, G_0$.
%
%
%\Cref{eq:tdreturn} does something amazing: it compresses an infinite number of actions taken by the agent into a tidy scalar recursion based on observed reward.
%However, our original objective deals with an \emph{expected} return, not an empirical sample trajectory.
%Therefore, define the \emph{state-action value function} to be the expected return starting from state $s$, action $a$, and following policy $\pi$:
%\[
%Q^{\pi}(s,a) = \EE_\pi \left[ G_0 \middle| s_0 = s, a_0 = a \right].
%\label{eq:Qfunc}
%\]
%Note by the Markov property and \cref{eq:tdreturn} we obtain the following result:
%\[
%Q^{\pi}(s,a) = r_g (s,a) + \gamma \EE_{s' \sim \pp{p}{s'}{s,a}, a' \sim \pp{\pi}{a'}{s'}} \left[ Q^{\pi}(s', a' \right]. 
%\label{eq:Qfixed}
%\]




%We have done two things:
%\begin{itemize}
%	\item Defined a seemingly magical function $Q^{\pi}$ that only takes in present information--the state and action--and tells us the long-term consequences of following some policy. 
%	\item Established a self-consistency relation for this function $Q^{\pi}$. 	
%\end{itemize}
%Taken together, we can establish optimality conditions.
%Notice that the original objective in \cref{eq:mdpobjective} is captured by the $Q$-function: $$J(\pi) = \EE_{s_0 \sim p(s_0), a_0 \sim \pp{\pi}{a_0}{s_0}} \left[ Q^{\pi} (s_0, a_0) \right].$$
%Therefore, our goal is to figure the best $Q$-function, which then solves the original problem.
%Naturally, define $Q^\star(s,a) = \max_{\pi} Q^{\pi}(s,a)$; indeed, if $Q^\star$ is available, then the optimal policy is obtained by
%\[
%\pi^\star (s) = \argmax_{a} Q^\star (s,a).
%\label{eq:optpolicy}
%\]
%
%When we apply the recursion in \cref{eq:Qfixed}, the optimization is offset to the next time step:
%\begin{equation}
%\begin{aligned}
%Q^\star(s,a) &= r_g(s,a) + \max_{\pi} \gamma \EE_{s' \sim \pp{p}{s'}{s,a}, a' \sim \pp{\pi}{a'}{s'}} \left[ Q^{\pi}(s', a' \right] \\
%&= r_g (s,a) + \gamma \EE_{s' \sim \pp{p}{s'}{s,a}} \left[ \max_{a' \in \action} \max_{\pi} Q^{\pi}(s',a') \right] \\
%&= r_g(s,a) + \gamma \EE_{s' \sim \pp{p}{s'}{s,a}} \left[ \max_{a' \in \action} Q^\star(s',a') \right].	
%\end{aligned}
%\label{eq:bellmanQ}
%\end{equation}
%\Cref{eq:bellmanQ} is known as the \emph{Bellman optimality equation}.
%Although we do not directly have access to $Q^{\pi}$, much less $Q^\star$, the beauty of \cref{eq:bellmanQ} is that it distills all the complexity of the original problem in \cref{eq:mdpobjective} into a one-step relation.
%Essentially, the Bellman equation provides a principled theoretical target around which \ac{RL} algorithms are built.


%\subsection{Fixed point aspirations}
%
%
%\subsection{Evaluate, improve, repeat}





%If we had access to $Q^\star$, we could select actions deterministically via $a \in \argmax_{a \in \action} Q^\star(s,a)$ at any state $s \in \state$.
%Unfortunately, we do not have $Q^\star$.

% discussion: discuss GPI and why evaluation/exploitation only stabilize at the optimal policy/value


%\subsection{Example: Learning to balance}




\section{Building a local value function through MPC}
\label{sec:localMPC}

%\subsection{What if we know something about the environment?}


\Ac{RL} hinges on the idea that an optimal policy can be discovered through a continuous cycle of exploration and improvement.
The subtext of this paradigm is that such a policy should be learned from scratch.
However, many control applications entail some prior physical understanding of the system, opening up opportunities to warm start the policy search \citep{venkatasubramanian2019PromiseArtificial}.\footnote{In this section, we do \emph{not} refer to model-based \ac{RL} wherein a dynamical model is learned or made available to aid in the training of the \ac{RL} agent with otherwise model-free algorithms \citep{jafferjee2020HallucinatingValue,janner2019WhenTrust}.}
Here, we provide an outline of nominal \ac{MPC}.
That is, an exact model is provided so as to emphasize the complete opposite of the \ac{RL} approach of the previous section.
%Our key point is that \ac{MPC} 



\Ac{MPC} is the most successful advanced control method \citep{qin2003Surveyindustrial,lee2011Modelpredictive,schwenzer2021Reviewmodel}.
It is ``safe'', modular, and interpretable:
\begin{itemize}
	\item \textbf{Safe.}\footnote{\Ac{MPC} is \emph{not} a magic bullet. Our point is that the \ac{MPC} literature provides a theoretical blueprint for formulating safe policies comprising technical conditions regarding stability, robustness, optimality, and constraint handling \citep{borrelli2017predictive}.}\quad Model knowledge and other constraints help compose an objective whose optimal solution leads to safe and stable operations.
	\item \textbf{Modular.}\quad Individual components of the controller can in principle be modified on the fly to reflect new knowledge or objectives.
	\item \textbf{Interpretable.}\quad The combination of constraints and modularity makes \ac{MPC} an intuitive approach for control (notwithstanding the underlying technical requirements).
\end{itemize}


%% RL for global value via approximate optimization, MPC for approximate local value via knowledge/planning


\subsection{An analytical foundation for MPC}


%\subsection{LQR: An analytical example}
%\label{sec:lqr}

Rather than using samples from the environment to learn a value function, this section focuses on constructing a value function.
This is done by combining a dynamic model and a cost function.
We begin with the \ac{LQR} problem:
\begin{equation}
\begin{aligned}
    &\underset{\mu(\cdot)}{\text{minimize}} && \sum_{t=0}^{\infty} \gamma^{t} \left( x_t\transpose M x_{t} + u_{t}\transpose R u_t \right) \\
    &\text{subject to } && u_t = \mu(x_t)\\
    &	&& x_{t+1} = A x_t + B u_t.
\end{aligned}
\label{eq:LQRobjective}
\end{equation}
This is the simplest case of the global objective in \cref{eq:mdpobjective} for which there is an analytical solution \citep{bertsekas2022lessons}.
This additional structure makes the new problem in \cref{eq:LQRobjective} seem more palpable than the original:
It considers a linear, time-invariant environment and a global objective that can be characterized by a quadratic cost around the origin.
Moreover, the optimization is over deterministic policies $\mu$.



The optimal solution to the \ac{LQR} problem is a static linear controller $\mu(x_t) = -K x_t$.
A key step in the solution is the use of the Bellman equation in tandem with a quadratic value function $V^\star (x) = x\transpose P x$ (see \cref{app:lqr} for more details):
\[
x\transpose P x = \min_{u} \left\{ x \transpose M x + u\transpose R u + \gamma \left(A x + B u\right)\transpose P \left(A x + B u\right) \right\},
\]
wherein solving for $u$ leads to an explicit formula for $K$.
This is a powerful result.
The \ac{LQR} problem not only yields a quadratic \emph{global} value function, but its simple structure lends itself to a tractable solution.
This means we are now equipped with a formula that takes system and cost parameters and maps them to an optimal set of controller parameters.\footnote{``Optimal'' is in the context of \cref{eq:LQRobjective}; generally, $M$ and $R$ need to be designed to give good performance. Moreover, we assume $M$ and $R$ are positive definite, which leads to the basic requirement of closed-loop stability.}



\subsection{MPC as an implicit control law}



In light of the \ac{LQR} objective in \cref{eq:LQRobjective}, it is natural to wonder about the possibility of additional constraints:
\begin{equation}
\begin{aligned}
    &\underset{\mu(\cdot)}{\text{minimize}} && J(\mu) = \sum_{t=0}^{\infty} \gamma^{t}\left( x_t\transpose M x_{t} + u_{t}\transpose R u_t \right)\\
    &\text{subject to } && u_t = \mu(x_t)\\
    & && x_{t+1} = A x_t + B u_t\\
    & && x_t \in \mathcal{X}, u_t \in \mathcal{U}.
\end{aligned}
\label{eq:feedbackobjective}
\end{equation}
This builds on \cref{eq:LQRobjective} by asserting that system behavior requirements are captured by state-input constraint sets $\mathcal{X} \times \mathcal{U}$, often box constraints. 


A controller resulting from the constrained problem in \cref{eq:feedbackobjective} is inherently nonlinear.
Indeed, control actions are state-dependent, as they account for proximity to the constraints.
This contrasts with the \ac{LQR} solution, which applies the same operation to the state no matter what.
Thus, the \ac{LQR} solution is not the best solution to the constrained problem, as it may only remain feasible inside a ``small'' portion of the state space \citep{borrelli2017predictive}.
In \cref{eq:feedbackobjective}, one could consider a parameterized class of policies---state feedback controllers---$\mu_\theta$ and proceed in a similar fashion to the \ac{RL} approach.
The result would be an explicit mapping $\mu_\theta: \state \to \action$ acting on the true environment.
However, this mapping introduces a degree of separation from the prior knowledge embedded in \cref{eq:feedbackobjective}, such as the system dynamics and cost structure.
In contrast, \ac{MPC} offers an implicit formulation aimed at retaining the design elements given in \cref{eq:feedbackobjective}. 
We outline two core features of the \ac{MPC} approach.






%This additional structure makes the new problem seem more palpable than the original:
%it assumes the environment is approximately linear, time-invariant; the global objective in \cref{eq:mdpobjective} can be characterized by a quadratic cost around the origin; and other behavior requirements are captured by state-input constraint sets $\mathcal{X} \times \mathcal{U}$, often box constraints. 
%This structure is used for simplicity, but a more general formulation is introduced in \cref{subsec:multiMPC}.








{\bf Preserving prior knowledge and requirements.}\quad
\Cref{eq:feedbackobjective} can equivalently be cast in terms of a sequence of inputs $u_0, u_1, u_1, \ldots$:
\begin{equation}
\begin{aligned}
    &\underset{u_0, u_1, u_2, \ldots}{\text{minimize}} && \sum_{t=0}^{\infty} \gamma^{t} \left( x_t\transpose M x_{t} + u_{t}\transpose R u_t \right) \\
    &\text{subject to } && x_{t+1} = A x_t + B u_t \\
    & && x_t \in \mathcal{X}, u_t \in \mathcal{U}.
\end{aligned}
\label{eq:desiredMPCobj}
\end{equation}
However, this problem contains an infinite number of decision variables.
A pragmatic idea is to formulate a hybrid between \cref{eq:desiredMPCobj,eq:LQRobjective}.
Consider the new objective, defined at some state $s$:
\begin{equation}
\begin{aligned}
    &\underset{u_0, u_1, \ldots, u_{N_c-1}}{\text{minimize}} && \sum_{t=0}^{N-1} \gamma^t \left( x_t\transpose M x_{t} + u_{t}\transpose R u_t \right) + x_{N}\transpose P x_N \\
    &\text{subject to } && x_0 = s \\
    & && x_{t+1} = A x_t + B u_t \\
    & && x_t \in \mathcal{X}, u_t \in \mathcal{U} \\
    & && u_t = -K x_t,\quad N_c \leq t \leq N-1.
\end{aligned}
\label{eq:nominalMPCobj}
\end{equation}
This new problem considers a finite number of decision variables, enabling reasonable command over constraints and system knowledge, while embedding infinite-horizon behavior cached in the \ac{LQR} value function \citep{lee2011Modelpredictive}.



%To make progress, we temporarily ignore the state-action constraints.
%Doing so leads to the \ac{LQR}, where an analytical solution is possible. 
%In fact, the optimal solution to the problem
%\begin{equation}
%\begin{aligned}
%    &\underset{u_0, u_1, u_2, \ldots}{\text{minimize}} && \sum_{t=0}^{\infty} \gamma^{t} \left( x_t\transpose M x_{t} + u_{t}\transpose R u_t \right) \\
%    &\text{subject to } && x_{t+1} = A x_t + B u_t \\
%\end{aligned}
%\label{eq:LQRobjective}
%\end{equation}
%is a static linear controller $u_t = -K x_t$.
%A key step in the solution is the use of the Bellman equation in tandem with a quadratic value function $V^\star (x) = x\transpose P x$ (see \cref{app:lqr} for more details):
%\[
%x\transpose P x = \min_{u} \left\{ x \transpose M x + u\transpose R u + \gamma \left(A x + B u\right)\transpose P \left(A x + B u\right) \right\},
%\]
%wherein solving for $u$ leads to an explicit formula for $K$.
%Unfortunately, the \ac{LQR} solution is is not the best solution to the constrained problem, as it may only remain feasible inside a ``small'' portion of the state space.\footnote{Controllers resulting from constrained problems such as \cref{eq:desiredMPCobj} are inherently nonlinear. Rather than applying the same operation to the state no matter what, control actions become state-dependent, as they depend on proximity to the constraints.}





{\bf Building a local value function.}\quad
The standard \ac{MPC} algorithm implements a \emph{receding horizon} strategy:
After solving \cref{eq:nominalMPCobj} at some state $s$ for optimal inputs $u_{0}^{\star}, \ldots, u_{N_{c}-1}^{\star}$, the action $ a = u_{0}^{\star}$ is applied to the true system.
The system transitions to some next state $s'$, at which point the problem in \cref{eq:nominalMPCobj} is reinitialized and solved again.


The use of $K$ as a ``fictitious'' controller in \cref{eq:nominalMPCobj} and $P$ as a terminal cost enable feasibility and stability guarantees \citep{borrelli2017predictive}.
Without them, perhaps by truncating the objective, the repeated application of solutions to \cref{eq:nominalMPCobj} is not guaranteed to always be feasible, much less stable.
Essentially, without incorporating infinite-horizon knowledge into the problem, anything beyond $N_c$ steps comes as a ``surprise'' to the controller.


All taken together, the receding horizon idea in tandem with the structure in \cref{eq:nominalMPCobj} represent an implicit, \emph{local} value function approximation \citep{mayne2000Constrainedmodel}.
Costs and actions are computed online as new state information is made available.
Crucially, the practical and theoretical success of \ac{MPC} is driven by this interplay between a global \ac{LQR} value function and local replanning.
The global \ac{LQR} solution uses principles of dynamic programming to cache all the planning into an explicit policy.
In turn, this alleviates the intractability of infinite-horizon planning as in \cref{eq:desiredMPCobj}.


%These ideas are in contrast to the \ac{RL} approach or the \ac{LQR} solution, which cache any prior knowledge and structure into an explicit, global value estimate.







%\Ac{LQR} is a tidy and globally optimal solution for controlling multivariate systems.
%However, let us tweak the objective in \cref{eq:LQRobjective}:
%\begin{equation}
%\begin{aligned}
%    &\underset{u_0, u_1, u_2, \ldots}{\text{minimize}} && \sum_{t=0}^{\infty} \gamma^{t} \left( x_t\transpose M x_{t} + u_{t}\transpose R u_t \right) \\
%    &\text{where} && x_{t+1} = A x_t + B u_t \\
%    & && x_t \in \mathcal{X} \\
%    & && u_t \in \mathcal{U}
%\end{aligned}
%\label{eq:desiredMPCobj}
%\end{equation}







%For a simple illustration, let us make the objective in \cref{eq:mdpobjective} more palpable by furnishing it with more structure:
%\begin{equation}
%\begin{aligned}
%    &\underset{u_0, u_1, u_2, \ldots}{\text{minimize}} && \sum_{t=0}^{\infty} \gamma^{t} \left( x_t\transpose M x_{t} + u_{t}\transpose R u_t \right) \\
%    &\text{where} && x_{t+1} = A x_t + B u_t \\
%\end{aligned}
%\label{eq:LQRobjective}
%\end{equation}

%Here we draw a distinction in notation: states $s$ come from the true environment, while states $x$ are used in an internal model; similarly, actions $a$ are inputs applied to the environment, while inputs $u$ are decision variables.
%The optimization problem \cref{eq:LQRobjective} as a whole can be thought of as a model for an intractable problem in the true \ac{MDP} seen in \cref{eq:abstractObj}. (While \cref{eq:LQRobjective} is cast as a \emph{minimization} problem, we view it as an approximation to \cref{eq:abstractObj} based on the value of the resulting policy in the true environment. One may also simply flip the signs, which we implicitly do throughout.)

%
%Inside this confined space, the goal in \cref{eq:LQRobjective} is to \emph{regulate} a \emph{linear system} $x' = A x + B u$ as efficiently as possible according to a \emph{quadratic cost}. 
%Naturally, \cref{eq:LQRobjective} is referred to as the \ac{LQR}.
%Although the \ac{LQR} problem contains an infinite number of decision variables, it turns out that the optimal solution is a static linear controller $u = -K x$.
%This can be shown by combining the structure of the problem with Bellman's optimality equation.
%
%
%\begin{enumerate}
%	\item Repurpose \cref{eq:bellmanQ}: Flipping signs, removing the expectation, plugging in the cost and dynamics equations, and finally minimizing both sides, we arrive at:
%		\[\min_{u} Q^\star (x, u) = \min_{u} \left\{ x \transpose M x + u\transpose R u + \gamma \min_{u'} Q^\star (A x + B u, u') \right\} \]
%	\item Quadratic optimal cost: ``Guess'' $\underset{u}{\min}\ Q^\star(x, u) = x\transpose P x$ for some symmetric $P$. (See \cref{app:lqr} for an intuitive argument.) We then have 
%		\[ x\transpose P x = \min_{u} \left\{ x \transpose M x + u\transpose R u + \gamma \left(A x + B u\right)\transpose P \left(A x + B u\right) \right\} \label{eq:bellmanLQR}\]
%	\item Solve for $u$: The righthandside above can be solved by setting the gradient of the inside term equal to zero to find $u = -K x$, where
%		\[K = \gamma \left(R + \gamma B\transpose P B\right)^{-1} B\transpose P A \]
%	\item Back-substitute: $K$ is expressed in terms of $P$. By plugging the solution $u = -K x$ back into \cref{eq:bellmanLQR} we arrive at the \ac{DARE}:
%		\[P = M + \gamma A\transpose P A - \gamma^2 A\transpose P B \left(R + \gamma B\transpose P B \right)^{-1} B\transpose P A\]
%\end{enumerate}
%
%
%%% TODO 1-D illustration, cite numerical solvers, etc
%% also talk about why \gamma < 1 might be useful, such as convergence / numerical stability
%The \ac{DARE} is a tractable form of Bellman's optimality equation for \ac{LQR}.
%Like Bellman's equation, the desirability of the optimal solution in the \ac{DARE} depends on the discount factor.
%For instance, as $\gamma \to 0$, the controller becomes degenerate, resulting in no control actions.
%For an open-loop unstable system, this is clearly problematic.


%Due to its simplicity, \ac{LQR} is a fundamental building block for constructing more complex policies. 
%We highlight two key features of \ac{MPC} 


%{\bf Preserving prior knowledge and requirements.}\quad
%
%\Ac{LQR} is a tidy and globally optimal solution for controlling multivariate systems.
%However, let us tweak the objective in \cref{eq:LQRobjective}:
%\begin{equation}
%\begin{aligned}
%    &\underset{u_0, u_1, u_2, \ldots}{\text{minimize}} && \sum_{t=0}^{\infty} \gamma^{t} \left( x_t\transpose M x_{t} + u_{t}\transpose R u_t \right) \\
%    &\text{where} && x_{t+1} = A x_t + B u_t \\
%    & && x_t \in \mathcal{X} \\
%    & && u_t \in \mathcal{U}
%\end{aligned}
%\label{eq:desiredMPCobj}
%\end{equation}
%$\mathcal{X}$ and $\mathcal{U}$ are constraint sets, often box constraints, for the states and inputs, respectively.
%The linear controller obtained from solving the \ac{LQR} is only feasible in a ``small'' portion of the state space because applying it to the system can quickly lead to infeasible states or actions later on.
%Therefore, the \ac{LQR} solution is not the best solution to the constrained problem.
%Unfortunately, the solution to \cref{eq:desiredMPCobj} is no longer tidy (that is, a closed-form linear controller).\footnote{Controllers resulting from constrained problems such as \cref{eq:desiredMPCobj} are inherently nonlinear. Rather than applying the same operation to the state no matter what, control actions become state-dependent, as they depend on proximity to the constraints.}


%We highlight two principles that lead to the practical deployment of \cref{eq:desiredMPCobj}.


%{\bf Embedding infinite-horizon behavior into finitely many decisions.}
%(terminal cost)

%\Ac{MPC} improves upon \ac{LQR} by accounting for constraints in its solution.
%This leads to a more complicated but practical controller.
%First, we abandon the idea of explicitly solving \cref{eq:desiredMPCobj} due to the infinite number of decision variables.
%Solving \cref{eq:desiredMPCobj} over a finite horizon $N$ can work in practice, but is not guaranteed to lead to stable operations.
%Essentially, this would be a finite-horizon solution applied to an infinite-horizon problem, meaning everything after $N$ time steps would be a ``surprise'' to the controller.

%Consider the following objective initialized at state $s$:
%\begin{equation}
%\begin{aligned}
%    &\underset{u_0, u_1, \ldots, u_{N_c-1}}{\text{minimize}} && \sum_{t=0}^{N-1} \left( x_t\transpose M x_{t} + u_{t}\transpose R u_t \right) + x_{N}\transpose P x_N \\
%    &\text{where} && x_0 = s \\
%    & && x_{t+1} = A x_t + B u_t \\
%    & && x_t \in \mathcal{X}, u_t \in \mathcal{U} \\
%    & && u_t = -K x_t,\quad N_c \leq t \leq N.
%\end{aligned}
%\label{eq:nominalMPCobj}
%\end{equation}
%In \cref{eq:nominalMPCobj}, the \emph{terminal cost} is the optimal \ac{LQR} cost; similarly, the resulting feedback controller $u = -K x$ is embedded into the \ac{MPC} objective after $N_c$ prediction steps.
%The idea behind embedding the \ac{LQR} solution into the \ac{MPC} objective is to incorporate infinite-horizon knowledge into the finite-dimensional problem.  


%{\bf Building a local value function.}
%(Receding horizon idea)



%The standard \ac{MPC} algorithm implements a \emph{receding horizon} strategy:
%After solving \cref{eq:nominalMPCobj} at some state $s$ for optimal inputs $u_{0}^{\star}, \ldots, u_{N_{c}-1}^{\star}$, the action $ a = u_{0}^{\star}$ is applied to the true system.
%The system transitions to some next state $s'$, at which point the problem in \cref{eq:nominalMPCobj} is reinitialized and solved again.




% TODO add algorithm


%\section{Industrial casestudy}
%
%\subsection{State estimation}
%
%\subsection{Offset-free tracking}
%
%\subsection{Cascade control}


%\subsection{Goal-conditioned learning and robust MPC}




%\subsection{MPC as a value-based policy}


\section{A brief survey of the RL-MPC interface}
\label{sec:related}


We now discuss related studies, limiting our discussion to works at the intersection of \ac{RL} and \ac{MPC}.
Value functions are fundamental to \ac{MPC} theory to derive stability and recursive feasibility conditions \citep{mayne2000Constrainedmodel, borrelli2017predictive, abdufattokhov2024LearningLyapunov}, which is not the focus of this paper.
Moreover, we do not survey filtering approaches in which \iac{MPC} is designed to safely modify or initialize the actions of a data-driven controller \citep{wabersich2021Predictivesafety, bejarano2024SafetyFiltering, hosseinionari2024IntegrationModel}.
Instead, we focus on works that take advantage of the conceptual similarities between \ac{RL} and \ac{MPC}.


{\bf Value function-augmented \ac{MPC}.}\quad 
Our approach falls into this category because we use \iac{MPC} agent to design actions using a learned value function.
However, the basic idea of \iac{RL}-based value function-augmented \ac{MPC} law is not new. 
This idea is based on dynamic programming, but made practical through \ac{RL} approaches.\footnote{Some of the referenced works use terms like \emph{\ac{ADP}} or \emph{neuro-dynamic programming}. We use \emph{\acl{RL}} for simplicity.}
Foundational works by \citet{bertsekas1996neuro} provide a rigorous treatment of \ac{RL}, while \citet{bertsekas2022lessons} gives a more recent account with emphasis on value function approximation and \ac{MPC}. 
 
 %% Mainly works using only simulated data (from MPC model) to boost the MPC in some way. That is, MPC is is assumed to be (near) optimal
Early works by \citet{lee2001NeurodynamicProgramming, lee2004SimulationbasedLearning} demonstrated the utility of embedding a learned value function into \ac{MPC} for process control applications.
\Citet{zhong2013Valuefunction} apply similar ideas in the context of classic control problems with an emphasis on data collection and value function parameterizations.
%In their framework, \ac{MPC} is viewed as a local approximation of the optimal value function, 
Similarly, the works of \citet{lowrey2019PlanOnline} consider \ac{MPC} as a trajectory optimizer that can aid in value function estimation, but with emphases on exploration.
So far, these works assume \ac{MPC} uses a locally optimal value function, meaning the cost and internal model accurately represent the true objective and environment. 
Nonetheless, a key benefit of value function-augmented \ac{MPC} via \ac{RL} is the ability to effectively shrink the planning horizon. Instead, a significant amount of planning and uncertainty can be cached into the value function representation, which lends itself nicely to stochastic systems.


%% Works that consider feedback from a ``true'' environment - some kind of mismatch in the dynamics or cost
\Citet{farshidian2019DeepValue} consider the case where an external, possibly sparse, reward signal is used to update the stage cost and value function in \ac{MPC}, but still assume an accurate model.
\Citet{arroyo2022ReinforcedModel} train \iac{RL} agent offline in simulation with an identified model, then deploy a value function-augmented \ac{MPC} scheme on the true, more complex system.
However, the agent remains static in the online phase, not taking into account information from the environment.
On the other hand, \citet{bhardwaj2020BlendingMPC} devise a time-weighted averaging strategy that blends together the \ac{MPC} and \iac{RL}-learned value estimate, taking advantage of prior information while enabling feedback from the true environment.


{\bf \ac{MPC} as a function approximator.}\quad
Another line of work takes the view that \ac{MPC}---its model, stage cost, constraints, and terminal value function---represents a set of parameters that can be steered towards optimality using \ac{RL}.
A common approach is to differentiate through the \ac{MPC} action with respect to its parameters.
\Citet{amos2019DifferentiableMPC} propose this idea, but applied it for imitation learning tasks. In a similar vein, \citet{tamar2017LearningHindsight} iteratively refine the \ac{MPC} cost based on offline replanning.
\Citet{gros2020DataDrivenEconomica, gros2022LearningMPC} further develop this line of work with an emphasis on safety and stability under \ac{RL}-based updates to the \ac{MPC} parameters.
In the context of deep \ac{RL}, \citet{romero2024ActorCriticModel} propose an actor-critic setup in which the actor feeds cost coefficients to a differentiable \ac{MPC} module.
\Citet{hansenTemporalDifferenceLearning2022} take a different perspective wherein they propose a temporal difference-based approach to learning the reward, dynamics, and value, which are then combined to construct the online \ac{MPC} agent.

Broadly speaking, these approaches place less trust in prior system knowledge than value function-augmented approaches and instead aim to find the best model for control, inspired by the notion of identification for control \citep{gevers2005identification}.
This leads to an all-in-one approach in which the \ac{MPC} is intimately tied to the \ac{RL} agent.
On the other hand, value function-augmented schemes allow for more structural separation, meaning the value function can be trained in a deep \ac{RL} pipeline, possibly offline based on prior system knowledge, and ported to the online \ac{MPC} agent.






\section{Robust goal-conditioned control policies}
\label{sec:rlmpc}

This section builds on the local-global interface through robust \ac{MPC} and goal-conditioned \ac{RL}. 
We first extend the discussion of nominal \ac{MPC} to robust \ac{MPC}.
This then inspires a robust training scheme for goal-conditioned \ac{RL}.
Finally, we show how to combine these agents such that the \ac{RL} policy benefits from replanning and constraint handling, while the \ac{MPC} policy benefits from high-level goal-conditioned objectives. 
See \cref{fig:concept} for an illustration of the proposed framework.




%Robustness is incorporated through the \ac{MPC} internal system description:


%%%% more specific about the key features of both we find appealing
%We revisit this classical interface between value function learning and optimization.
Our framework is model-based in nature.
Specifically, we assume an \emph{uncertain} dynamic model of the environment is available: Some prior knowledge is available, but not so much that we have a perfect representation of the underlying dynamics.
Robustness is incorporated into our framework through the uncertain system description:
\begin{enumerate}
	\item \textbf{Robustness of the online agent.}\quad We use a robust scenario-based \ac{MPC} agent, which incorporates a distribution of system uncertainties into its predictions. Specifically, the \ac{MPC} agent constructs a scenario tree, illustrated in \cref{fig:scenario}, to tabulate costs and account for constraints over different situations.  
	\item \textbf{Robustness of the offline agent.}\quad We formulate a scenario-based value function based on the distribution of system uncertainties. This leads to a robust Bellman equation, which serves as a target for the \ac{RL} agent to learn simply through a branching process during offline rollouts; see the left-hand portion of \cref{fig:concept}. General off-policy actor-critic algorithms are applicable for this portion of the framework \citep{konda1999ActorcriticAlgorithms}.
%	\item Under this formulation, the agent is able to learn from both the true system data and simulated rollouts in a unified way.
\end{enumerate}
Essentially, this scenario-based approach to robustness aligns the robust \ac{RL}-learned value function with the short-term, uncertain \ac{MPC} predictions.


\begin{figure*}
	\includegraphics[width=\textwidth]{concept.pdf}
	\caption{An actor-critic agent interacts with a branching simulation environment offline to learn a robust global value function. The critic is used in the usual fashion to inform parameter updates, but also to construct a robust local \ac{MPC} agent for online control of the ``true'' system.}
	\label{fig:concept}
\end{figure*}


\begin{figure}
	\includegraphics[width=0.75\textwidth]{scenario.pdf}
	\caption{A scenario tree branches at some state $x$, applying the same control action $u$ for each of the three cases in the uncertainty set $\{\psi^1, \psi^2, \psi^3\}$. Three successive states are computed using the model $f$, after which each scenario remains constant. However, it is possible to keep branching at each node.}
	\label{fig:scenario}
\end{figure}



Although a model is available, we target complicated objectives where a straightforward implementation of \ac{MPC} may not be suitable.
Consequently, we leverage model-free \ac{RL} techniques to directly learn the optimal value function from offline exploration.
%More specifically, we formulate a robust Bellman equation, 
%which enables the use of simulation and real data in parallel.
%which enjoys purely offline training, given a suitable range of uncertainty scenarios, rather than a nominal model.
%General off-policy actor-critic algorithms are applicable for this portion of the framework \citep{haarnoja2018Softactorcritic}.
%The possible sources of system uncertainty are general, such as structural model uncertainty, parameter model uncertainty, or time-varying systems.
In particular, the \ac{RL} agent is trained in a \emph{goal-conditioned} manner. 
After training such \iac{RL} agent offline, \iac{MPC} agent generates actions using short-term, scenario-based predictions and the \ac{RL} value function as a terminal cost.
Specifically, the \ac{MPC} agent makes these predictions subject to constraints and system uncertainty.
Together, this combination of \ac{RL} and \ac{MPC} produces a robust and safe goal-conditioned policy.




%Note that rewards of the form in \cref{eq:goalreward} do not inform an agent (\ac{RL} or \ac{MPC}) \emph{how} to achieve a goal. In that vein, we demonstrate in \cref{sec:dip} the prohibitively long planning horizon for \ac{MPC} alone, motivating the utility in learning a value function.




\subsection{Scenario-based MPC}
\label{subsec:multiMPC}




The \ac{MPC} formulation given in \cref{eq:nominalMPCobj} is often referred to as \emph{nominal MPC} wherein one assumes the system model reflects the true dynamics being controlled.
%There is little use in belaboring over this detail, as it is always true in practice.
Yet, the basic idea of continually replanning endows the basic \ac{MPC} structure with some inherent robustness to plant-model mismatch.
Nonetheless, the risk of violating constraints when deploying \iac{MPC} scheme should not be overlooked.
Our proposed framework employs a scenario-based approach to robustness.
This approach has its backbone in dynamic programming, making it a unified target for approximate solutions through both \ac{MPC} and \ac{RL} \citep{delapenad2005StochasticProgramming,bernardini2009Scenariobasedmodel,lucia2013MultistageNonlinear}.
In addition to this structural connection, scenario-based \ac{MPC} does not require a precomputed ancillary controller, as in tube-based \ac{MPC} \citep{mayne2005robust}, nor does it generally lead to a very conservative solution, as in min-max \ac{MPC} \citep{campo1987RobustModel}.

Scenario-based \ac{MPC} considers a scenario tree in its planning to help cope with uncertainty \citep{lucia2013MultistageNonlinear}. 
A scenario is essentially a realization of the system model under some uncertainty specification.
System uncertainty is general under our framework, but some possible sources include structural model uncertainty, model parameter uncertainty, or time-varying components \citep{paulson2018NonlinearModel}.
Each scenario is subject to the same constraints, which means actions that are otherwise reasonable under nominal MPC may get pruned from consideration.
This strategy results in more robust actions.



Mathematically, consider a general system model $f$ whose successive state $x'$ evolves as follows:
\begin{equation}
	x' = f(x, u, \psi).
\label{eq:paramsys}
\end{equation}
In addition to states $x$ and control actions $u$, $f$ also takes in scenarios $\psi$.
We assume $f$ is given through prior physical understanding of the process, but $\psi$ represents system uncertainty due to structural and parametric model uncertainty or exogenous disturbances.
Consider $N_s$ scenarios, each of which is a realization of the uncertainties $\psi$ in a system model $f$, branching from the start state then remaining constant; \cref{fig:scenario} illustrates the basic concept. 
It is possible to branch out the uncertainty scenarios at each time step, but this is discouraged due to the exponential growth in scenarios.






Controlling the growth in the number of scenarios is a practical innovation of scenario-based \ac{MPC} \citep{lucia2013MultistageNonlinear}, which considers the following objective at some initial state $s$:\footnote{The form given by \citet{lucia2013MultistageNonlinear} includes a \emph{robustness horizon} parameter, which controls how many time steps branch out in the prediction horizon. We present the case where the robustness horizon is $1$.}
\begin{equation}
\begin{aligned}
    &\underset{\substack{u_0^1, u_1^1, \ldots, u_{N-1}^1\\\vdots\\ u_0^{N_s}, u_1^{N_s}, \ldots, u_{N-1}^{N_s}}}{\text{minimize}} && \frac{1}{N_s} \sum_{i=1}^{N_s} J\left( x_0^i,\ldots x_{N}^i, u_0^i, \ldots, u_{N-1}^i \right) \\
    &\text{subject to } && x_0^i = s \\
    & && x_{t+1}^i = f(x_t^i, u_t^i, \psi^i) \\
    & && u_{0}^i = u_{0}^j \\
    & && x_t^i \in \mathcal{X}, u_t^i \in \mathcal{U} \\
    & && x_{N}^i \in \mathcal{X}_{\text{terminal}}
\end{aligned}
\label{eq:multiMPCobj}
\end{equation}
where $J$ is an $N$-step cost function:
\[
J(x_0,\ldots x_{N}, u_0, \ldots, u_{N-1}) = \sum_{t=0}^{N-1} l(x_t, u_t) + m(x_N).
\]
This formulation considers a general \emph{stage cost} $l$ and \emph{terminal cost} $m$.
In the context of \cref{eq:nominalMPCobj}, the stage cost is quadratic and the terminal cost is the \ac{LQR} value function.
Note the constraint $u^i_0 = u^j_0$ ensures that the agent selects actions only according to current information $s$.
Successfully solving \cref{eq:multiMPCobj} provides a certificate that the optimal solution satisfies the constraints even under the worst-case scenario.
This provides some extra assurance that the endorsed action will keep the true system operating safely.




\subsection{Offline MDP based on uncertain knowledge}



Beyond scenario-based \ac{MPC}, we utilize the idea of a scenario tree to formulate a branching \ac{MDP}, which can be used to train a robust \ac{RL} agent.
This is in contrast to other approaches to robustness in \ac{RL}.
A min-max formulation is a common strategy for training robust, although conservative, policies \citep{zouitine2024Solvingrobust, nilim2003RobustnessMarkov}.
Another algorithmic approach is to train conservative agents with respect to static, \emph{offline} datasets, leading to robustness in \emph{online} performance \citep{kumar2020ConservativeQlearning}.
Other approaches focus on imposing structural constraints on the policy architecture based on \acp{IQC} to achieve robustness \citep{jin2020StabilitycertifiedReinforcement, revay2021RecurrentEquilibrium}.
The proposed scenario-based approach is both simple and congruous with the overarching \ac{MDP} framework, as discussed next.





In the context of \iac{MDP}, structural knowledge of the model $f$ can be combined with the uncertainty in $\psi$ to formulate an environment.
That is,
\begin{equation}
\begin{split}
	\psi &\sim p(\psi)	\\
	s' &\sim \pp{p_\psi}{s'}{s,a}
\label{eq:offlineMDP}
\end{split}
\end{equation}
where $\pp{p_\psi}{s'}{s,a} = \delta\left( s' - f(x,u,\psi) \mid s = x, a = u \right)$ is the Dirac delta function conditioned on the current state-action pair.
%{\color{red} (p is a density so need to make this more precise)}
%\begin{align}
%\pp{p_\psi}{s'}{s,a} = & 
%\begin{cases}
%	1 & \text{if}\ x' = f(x, u, \psi)\\
%	0 & \text{otherwise}
%\end{cases}\\
%&( x' = s', x = s, u = a ).
%\end{align}
The dynamics in \cref{eq:offlineMDP} define the transitions for an offline, simulation environment.
Finally, a reward, such as in \cref{eq:sparsereward}, completes the \ac{MDP}.


\textbf{A robust value function}\quad
We consider a set of $N_s$ possible realizations of system uncertainty:
\[
\{ \psi^0, \ldots, \psi^{N_s-1} \}.
\label{eq:paramset}
\]
No preference is given to any one of them, meaning they are uniformly distributed.
In the context of \cref{eq:offlineMDP} and the Bellman equation in \cref{eq:Qfixed}, we have the following relationship:
\begin{align}
\begin{split}
	Q^{\pi}_g (s,a) &= r_g (s,a) + \gamma \EE_{\psi \sim p(\psi), s' \sim \pp{p_\psi}{s'}{s,a}, a' \sim \pp{\pi} {a'}{s'}} \left[ Q^{\pi}_g (s', a') \right] \\
	&= r_g (s,a) + \gamma \frac{1}{N_s} \sum_{i = 0}^{N_s-1} \EE_{s' \sim \pp{p_{\psi^{i}}}{s'}{s,a}, a' \sim \pp{\pi}{a'}{s'}}\left[ Q^{\pi}_g (s', a') \right].
\label{eq:robustQ}
\end{split}
\end{align}


This theoretical target is in competition with some straightforward options regarding robustness:
\begin{enumerate}
	\item One could opt for a single uncertainty instance and hope that the resulting policy generalizes well to other scenarios.
	\item Going further, one could create multiple scenarios in parallel, sharing the same policy, then pool together the respective value functions $Q_{0},\ldots, Q_{N_s-1}$ through averaging $\frac{1}{N_s} \sum_{i = 0}^{N_s-1} Q_{i}$. This forms an approximation to $Q^{\pi}_g$ in \cref{eq:robustQ}, but ultimately does not result in a value function itself for the \ac{MDP} in \cref{eq:offlineMDP}. 
\end{enumerate}
%{\bf Offline data.}\quad 

Instead, in the spirit of option (1), the branching \ac{MDP} is only a single (but specialized) environment, but experience from all scenarios informs the value estimation, like option (2).
However, unlike these options, a policy satisfying \cref{eq:robustQ} directly incorporates uncertainty into the decision-making process in a state-dependent fashion. 
This means it has to operate with enough margin to elevate the next-step return across several scenarios.

%{\bf Online data.}\quad Under this framework of ``branching'' the parameters during offline environment rollouts, it becomes natural to incorporate ``true'' system data into the training process.
%Essentially, if the real system of interest is not exactly captured in \cref{eq:paramset}, its data can augment the dataset used to train an agent to satisfy \cref{eq:robustQ}.
%This has the effect of creating an implicit scenario from the agent's perspective, modifying the parameter distribution and overall \ac{MDP} composition in the first line of \cref{eq:robustQ} based on the balance between simulated and true data.


%% (some info about a strategy that involves multiple instances and averaging)
%The degree of margin is state-dependent






%In this environment, a goal-conditioned policy $\pi$ can be trained under sparse rewards, such as in \cref{eq:sparsereward}, using hindsight relabeling as discussed in \cref{subsec:goalconditioned}.



\subsection{Offline training and online deployment}


The simulated \ac{MDP} in \cref{eq:offlineMDP} enables an agent to learn a goal-conditioned policy under uncertainty.
%By ``efficiently'' we simply mean \iac{MPC} solver is not needed at this stage.
%Instead, \iac{DNN} policy $\pi$ 
For our general formulation, any off-policy actor-critic algorithm can be used wherein a policy $\pi$ is learned alongside a value function $Q$ \citep{konda1999ActorcriticAlgorithms}.
Both are represented by \iac{DNN}.
Briefly, the policy network $\pi$ is used in the simulation environment to enable ``fast'' decision-making and streamlined implementation.
We then deploy the critic $Q$ on the ``true'' system, where \iac{MPC} agent designs actions subject to constraints and uncertainty intervals.
A concept diagram summarizing this section is shown in \cref{fig:concept}.


{\bf Actor-critic training.}\quad
Based on \cref{eq:robustQ}, \iac{RL} algorithm seeks to learn $\pi$ and $Q$ such that:
\[
Q^{\pi}_g (s,a) \approx r_g (s,a) + \gamma \frac{1}{N_s} \sum_{i = 0}^{N_s-1} \EE_{s' \sim \pp{p_{\psi^{i}}}{s'}{s,a}}\left[\max_{a'} Q^{\pi}_g (s', a') \right].
\label{eq:approxQ}
\]
First, a dynamic model class is created based on \cref{eq:paramsys}.
This model structure is the basis for the environment in \cref{eq:offlineMDP}.
Such an environment has two key elements:
\begin{enumerate}
	\item \textbf{Branched rollouts.}\quad Sampling from the scenario set in \cref{eq:paramset} at each time step to create branched rollouts. Note that the scenario set used for offline \ac{RL} training may be larger than the one in scenario-based \ac{MPC} because the learned value function in \cref{eq:approxQ} does not perform explicit planning upon deployment.
	\item \textbf{Goal-augmented state.}\quad For goal-conditioned learning, the state definition used in the environment contains the goal itself\footnote{Equivalently, we use the error signal $g-s$, rather than the goal, as input to the actor-critic networks.}, the observed state, in the spirit of \cref{eq:paramsys}, as well as the ``achieved goal.'' The achieved goal could be the state itself, or some transformed version of the state, for instance, if the goal is an output value rather than a state value. All the information is necessary in order to implement the \ac{HER} strategy from a replay buffer.
\end{enumerate}
With the environment ready, an off-the-shelf off-policy, deep \ac{RL} algorithm can be deployed aimed at learning $\pi$ and $Q$ in \cref{eq:approxQ}.
The correct state formulation allows for the \ac{HER} strategy to be used to relabel training samples drawn from the replay buffer and used for updating the actor-critic weights.





{\bf Critic-informed \ac{MPC} deployment.}\quad 
A key innovation of deep \ac{RL} algorithms is the ability to train complex policies while avoiding exact optimization.
Specifically, the policy $\pi$ is trained to optimize $Q$, but only approximately, as discussed around \cref{eq:noisepi,eq:dpgalg}.
%For example,
%\begin{align}
%	a &\sim \pp{\pi}{a}{s}\\
%	a &\approx \underset{a}{\argmax}\ Q^\pi (s,a)
%\end{align}
This amounts to using $Q$ as a loss function in training, and then the fast-to-evaluate $\pi$ for decision-making.

While the agent explores and learns in the offline MDP in \cref{eq:offlineMDP} through the policy $\pi$, the corresponding value approximation $Q$ is used in conjunction with \iac{MPC} agent to create a refined policy for online deployment.
Like the \ac{RL} policy $\pi$, this new, refined policy is also goal-conditioned.
It uses a Gaussian reward $\hat{r}$ with a fixed variance  $\sigma^2$:
\begin{equation}
	\hat{r} (s,a) = e^{-\frac{\norm{g-s}^2}{2 \sigma^2}} \approx 
	\begin{cases}
	1 & \text{Goal $g$ is achieved}\\
	0 & \text{Otherwise}.	
	\end{cases}
\label{eq:goalreward}
\end{equation}
Such a reward aligns the short-term costs with the terminal, goal-conditioned value function $Q$.
Our experimental evaluation examines the variance parameter; briefly, a small variance is not necessary, as the short-term predictions are primarily concerned with the constraints, while the terminal value function provides more fine-grained guidance toward the goal.
Now, define the unified \ac{RL} and \ac{MPC} policy based on the following objective:
%\begin{equation}
%\begin{aligned}
%\mu (s) = &\underset{a}{\argmax} && Q^\pi (s,a) \\
%&\text{such that} && x_0 = s,\ u_0 = a\\
%&				&& x^{(i)}_{1} = f(x_0, u_0, \hat{\psi}_i)\ \forall i \\
%&				&& u_0 \in \mathcal{U}\\
%&				&& x^{(i)}_{1} \in \mathcal{X}\ \forall i
%\end{aligned}
%\label{eq:QMPC}
%\end{equation}
\begin{equation}
\begin{aligned}
&\underset{\substack{u_0^1, u_1^1, \ldots, u_{N-1}^1\\\vdots\\ u_0^{\hat{N}_s}, u_1^{\hat{N}_s}, \ldots, u_{N-1}^{\hat{N}_s}}}{\text{minimize}} && \frac{1}{\hat{N}_s} \sum_{i=1}^{\hat{N}_s} \sum_{t=0}^{N-1} \left[ \norm{\epsilon_t^i}_1 - e^{-\frac{\norm{g - x_{t}^{i}}^2}{2 \sigma^2}} \right] - V^{\pi}_g (x_{N}^{i}) \\
    &\text{subject to } && x_0^i = s \\
    & && x_{t+1}^i = f(x_t^i, u_t^i, \hat{\psi}^i) \\
    & && u_{0}^i = u_{0}^j \\
    & && x_t^i - \epsilon_t^i \in \mathcal{X}, \epsilon_t^i \in \mathcal{E}, u_t^i \in \mathcal{U}\\
    & && x_{N}^i \in \mathcal{X}_{\text{terminal}},
\end{aligned}
\label{eq:QMPC}
\end{equation}
where $\hat{\psi}^i$ represents scenarios from a restricted subset of those used for \ac{RL} training and $\hat{N}_s$ is the corresponding number of scenarios.
As with any \ac{MPC}-based policy, only the first action in \cref{eq:QMPC} is deployed.
There are several pieces to unpack:
\begin{itemize}
	\item \textbf{Terminal cost.}\quad We use the learned value function $V^{\pi}_g (s) = Q^{\pi}_g (s, \mu_{\text{actor}}(s))$ as a terminal cost, where $\mu_{\text{actor}}$ is the mean of the policy $\pi$. Rather than implementing the \ac{RL} actions directly, the value function informs the constrained loss landscape.
	\item \textbf{Soft constraints.}\quad When the state is very far from the goal, we have $$e^{-\frac{\norm{g - x_{t}^{i}}^2}{2 \sigma^2}} \approx 0 \quad \forall t=0,\ldots,N-1, i=1,\ldots,\hat{N}_s$$ meaning the $N$-step cost only accounts for constraint penalties $\epsilon$. This directly enables the agent to focus on short-term constraint satisfaction, and then consider long-term cost through the terminal value function. In contrast, weighing constraint violations can be cumbersome when using, for example, a quadratic stage cost; on the other hand, hard constraints may be used, which can result in the controller getting ``stuck'' trying to avoid violations. This is illustrated in \cref{fig:cstr_rlmpc_profile} in \cref{subsec:cstr}.
	\item \textbf{Scenario tree.}\quad $\hat{\psi}^i$ encompasses a set of scenarios, possibly different from those seen in the offline \ac{MDP}. For example, the policy \cref{eq:QMPC} might only factor in the extreme uncertainty realizations. As in \cref{eq:multiMPCobj}, we assume branching occurs only at the initial state $x^i_0 = s$, then predictions are performed over fixed scenarios for a short time horizon, leading to the terminal cost. 
\end{itemize}
To summarize, \cref{eq:QMPC} includes the core feature of scenario-based \ac{MPC}, which is planning over different realizations of a system model.
Moreover, it incorporates a high-level cost through the goal-conditioned \ac{RL} state value function, and a local Gaussian stage cost.
This stage cost enables the proposed value function-augmented \ac{MPC} agent to prioritize constraints at states far away from the goal.

%{\color{red} (Fit in some discussion about why only using data from the critic-MPC policy is bad. (Robustness, efficiency, data, exploration))}





\section{Case studies}

We present two case studies.
The first illustrates the goal-conditioned reward in \cref{eq:goalreward} in a nominal \ac{MPC} setup, compared against more traditional objectives.
The second brings together all the elements discussed in this paper: robustness, goal conditioning, and the combination of local-global values.
The corresponding code is available here: \url{https://github.com/NPLawrence/RL-MPC}.


\subsection{Example 1: Nominal goal-conditioned MPC}
\label{sec:dip}


This example focuses on a simplified version of the proposed policy structure in \cref{eq:QMPC}.
We consider nominal goal-conditioned \ac{MPC} applied to a double inverted pendulum.
That is, the \ac{MPC} policy is given a nominal physics model and does not consider system uncertainty, nor does it include \iac{RL} value function.
The purpose of this demonstration is to isolate the Gaussian reward in a planning context and to compare it against other \ac{MPC} agents.

The task is to apply force to a cart in order to bring the double inverted pendulum from its natural resting state to the upright position.
The goal can be formulated in terms of the angle of each link relative to the upright position.
Below is a summary of each \ac{MPC} agent.
\begin{itemize}
	\item \textbf{Expert.}\quad This is the formulation and implementation used in the benchmark example by \citet{fiedler2023DompcFAIR}, readily available in the authors' \texttt{\href{https://www.do-mpc.com}{do-mpc}} toolbox. The stage cost aims to maximize potential energy and minimize kinetic energy; it also includes a penalty term on changes to the actions to encourage ``smooth'' control.
	\item \textbf{Quadratic.}\quad The stage and terminal costs are $$\frac{1}{2}\left(\left( 1 - \cos(\theta_1)\right)^2 + \left( 1 - \cos(\theta_2)\right)^2\right),$$ where $\theta_1, \theta_2$ are the angles of the two links. The controller does not include a penalty term on the actions.
	\item \textbf{Goal-conditioned}.\quad The same setup as the quadratic formulation, but with the cost set to 
	\begin{equation}
		-e^{-\frac{1}{2\sigma^2}\left(\left( 1 - \cos(\theta_1)\right)^2 + \left( 1 - \cos(\theta_2)\right)^2\right)},
	\label{eq:smoothreward}
	\end{equation}
	with $\sigma^2=1$.
\end{itemize}


\begin{figure}
	\includegraphics[width=\textwidth]{dip_boxplot_performance.pdf}
	\caption{All three agents are able to solve the swing up task for prediction horizons $75$ and $300$. However, the goal-conditioned agent gives the most consistent performance in terms of time spent in the upright position.}
	\label{fig:dip_performance}
\end{figure}

\begin{figure}
	\includegraphics[width=\textwidth]{dip_boxplot_tv.pdf}
	\caption{The expert agent is very efficient at solving the the swing up task, whereas the quadratic agent is the most aggressive. The goal-conditioned agent becomes much more efficient with its actions as the prediction horizon increases.}
	\label{fig:dip_tv}
\end{figure}


We perform a sweep over three different prediction horizons.
\Cref{fig:dip_performance,fig:dip_tv} summarize the performance of each agent as follows: ``Time near goal'' is quantified using \cref{eq:smoothreward} with $\sigma^2 = 0.01$ (much more stringent than the goal-conditioned \ac{MPC} stage cost.); ``Action total variation'' reports $\sum_{t=0}^{99} \norm{a_{t} - a_{t-1}}$ over the course of each $100$-time step experiment.

 
Based on \cref{fig:dip_performance}, the quadratic and goal-conditioned \ac{MPC} agents are able to solve the swing up task under the three prediction horizons, while the expert formulation does not for $N=25$.
For $N=75$, all three agents are approximately aligned in terms of time spent near the goal, but the expert agent does so with at most $1/3$ the action variation of the other two policies, shown in \cref{fig:dip_tv}.
Planning very far into the future, the goal-conditioned agent creates slightly more separation from the other agents in time spent in the upright position.
However, its decrease in action variation is more noteworthy:
Across the three prediction horizons, the expert agent's action variation slowly increased, and the quadratic agent's slowly decreased.
In contrast, the goal-conditioned agent became roughly $45\%$ more efficient with its actions.
This illustrates the idea that a goal-conditioned objective does not react aggressively to large errors, like a quadratic objective.
Instead, a long-term view means the sensitivities of \cref{eq:smoothreward} ``light up'' most strongly to trajectories that bring the state to the goal.
We note, however, that this is also a limitation, as we were not able to solve the swing up task with a smaller variance of $\sigma^2 = 0.1$.


\begin{figure}
	\includegraphics[width=\textwidth]{DIP_timeprofile.pdf}
	\caption{Starting from rest, the goal-conditioned \ac{MPC} agent is given a sequence of three different unstable equilibria to reach. A corresponding animation can be found here: \url{https://github.com/NPLawrence/RL-MPC}. This experiment was performed with $\sigma^2 = 0.5$ and $N=35$.}
	\label{fig:dip_profile}
\end{figure}


Our final experiment for this example showcases the goal-conditioned \ac{MPC} agent on the other two unstable equilibria of the double inverted pendulum.
While the expert agent is able to solve the swing up task, this is a secondary effect of its objective. 
In other words, it cannot readily be applied to the other equilibria.
\Cref{fig:dip_profile} shows a time profile of the angle trajectories as the goal-conditioned agent is directed to achieve different configurations.
This isolates and validates the use of a nonstandard goal-conditioned stage cost, independent of all the other machinery discussed in this paper.
The resulting \ac{MPC} agent is able to solve a complicated control problem efficiently.
However, we have also demonstrated the challenge of deploying \ac{MPC} alone with such an objective, namely, the inability to solve the task with a small variance value, which more accurately characterizes the goal as in \cref{eq:goalreward}.
This motivates the use of derivative-free optimization frameworks, such as \ac{RL}, for long-term goal-conditioned objectives, demonstrated in \cref{subsec:cstr}.






%\begin{table*}[tbp]
%\caption{The swing up task was solved using \iac{MPC} scheme with the Gaussian reward in \cref{eq:smoothreward} for $\sigma^2 = 1.0$ and long prediction horizons. However, this configuration did not work for the other two unstable equilibria of the double inverted pendulum, further underscoring the challenge of planning with a sparse objective.}
%\begin{center}
%\begin{tabular}{l|lll}
%\toprule
%\diagbox[height=1.35\line]{$\sigma^2$}{$N$} & $100$ & $250$ & $400$ \\
%\midrule
% $0.1$ & \ding{55} & \ding{55} & \ding{55}    \\
% $0.5$ & \ding{55} & \ding{55} & \ding{55}    \\
% $1.0$ & \ding{55} & \ding{51} & \ding{51}    \\
%\bottomrule
%\end{tabular}
%\end{center}
%\label{table:DIP}  
%\end{table*}




\subsection{Example 2: Robust goal-conditioned policies for process control}
\label{subsec:cstr}


We study \iac{CSTR}, a common benchmark in process control, particularly in \ac{MPC} and learning-based applications \citep{fiedler2023DompcFAIR, bloor2024PCGymBenchmark, lee2001NeurodynamicProgramming, nejatbakhshesfahani2023Learningbasedstate}.
In our example, we use the model and parameters given by \citet{klatt1998Gainschedulingtrajectory}.
This is also the formulation used in the robust scenario-based \ac{MPC} benchmark by \citet{fiedler2023DompcFAIR}, readily available in the authors' \texttt{\href{https://www.do-mpc.com}{do-mpc}} toolbox.
For completeness, a short summary is given below, with the accompanying equations given in \cref{app:cstr}.


%\subsubsection{Process and task description}



The \ac{CSTR} process is described by a fourth-order nonlinear ordinary differential equation.
The state variables are concentrations $c_A$ and $c_B$, reactor temperature $T_R$, and coolant temperature $T_K$.
The reaction $A \to B$ is controlled through the input variables $F$ (normalized inflow) and $\dot{Q}$ (heat removed by coolant).
Within this process are two additional reactions $B \to C$ and $A \to D$, forming byproducts $C$ and $D$.
Two key rate coefficients are considered uncertain. 
These rate terms depend exponentially on the reactor temperature $T_R$.
The uncertainty is characterized by two multipliers $\alpha$ and $\beta$: $\alpha$ characterizes uncertainty in the activation energy for reaction the $A \to D$, while $\beta$ characterizes uncertainty in the rate coefficient for the reaction $A \to B$.


\subsubsection{Robust offline training}


We train two agents:
\begin{itemize}
	\item \textbf{Nominal \ac{RL}.}\quad The environment does not contain any system uncertainty; only the true nominal parameter values are used.
	\item \textbf{Robust \ac{RL}.}\quad The environment is constructed with a branching process as described in \cref{eq:offlineMDP}. It assumes structural knowledge of the system dynamics and a range of possible values for $\alpha$ and $\beta$. This range is gridded and each value is considered equally likely, making \cref{eq:robustQ} the theoretical target.
\end{itemize}


The task for the agent is to control the concentration $c_B$ through the actions $F$ and $\dot{Q}$.
Therefore, for a desired concentration $c_{B}^{\text{goal}}$, the reward is defined as:
\begin{equation}
	r_g (s,a) = e^{-\frac{\left( c_{B}^{\text{goal}} - c_B \right)^2}{2 \sigma^2}}
\label{eq:cstrreward}
\end{equation}
with $\sigma^2 = 0.0001$.
The agents are trained using the soft actor-critic algorithm \citep{haarnoja2018Softactorcritic} and \ac{HER} for the replay buffer. See \cref{app:implementation} for further implementation details.

After training, we evaluate the agents based on how effectively they reach a novel goal.
This is measured as follows: $200$ initial states are randomly sampled; they are sampled from the constraint intervals used in \ac{MPC}, illustrated next in \cref{fig:cstr_rlmpc_profile} in \cref{subsec:cstr_rlmpc}. 
The agent is given $50$ time steps to reach the goal. 
The agent's effectiveness is quantified for the last $25$ time steps; this isolates steady state performance from the transient stage of each rollout.


\begin{figure}
	\includegraphics[width=\textwidth]{boxplot_rl_reward.pdf}
	\caption{A ``boxen'' plot illustrating the distribution of steady-state reward achieved by the nominal and robust \ac{RL} agents. The middle line is the median, the widest box contains $50\%$ of samples, then successively narrower boxes include additional $25\%$, $12.5\%, \ldots$ samples.}
	\label{fig:cstr_offlineagent}
\end{figure}

\begin{figure}
	\includegraphics[width=\textwidth]{boxplot_rl_error.pdf}
	\caption{A boxen plot based on the same data as in \cref{fig:cstr_offlineagent_error}, but showing the distribution of percentage error across \ac{RL} agents and experiments.}
	\label{fig:cstr_offlineagent_error}
\end{figure}


\Cref{fig:cstr_offlineagent} reports the average reward over these final $25$ time steps; similarly, but more interpretable, \cref{fig:cstr_offlineagent_error} reports the average absolute percentage error.
The nominal evaluation means the aforementioned experiment is performed on the nominal plant without uncertainty; the robust evaluation samples and fixes parameter values from a set not used for the robust \ac{RL} training.
\Cref{fig:cstr_offlineagent} suggests that the nominal \ac{RL} agent fails to track the goal.
This is partially true, as \cref{fig:cstr_offlineagent_error} shows both the nominal and robust evaluation experiments leave about $5\%$ median offset over the final $25$ time steps, in contrast to the roughly $2\%$ of the robust \ac{RL} agent.
While it may be possible to fine-tune the hyperparameters to improve the nominal evaluation, it mainly serves as a reference.
The main takeaway is the robust evaluation, which shows a more modest dip by the robust \ac{RL} agent.
Overall, our primary motivation is to validate the robust \ac{RL} scheme, as we will incorporate it into the scenario-based \ac{MPC} framework next.



\subsubsection{Combining robust RL and scenario-based MPC}
\label{subsec:cstr_rlmpc}



We evaluate the performance of the value function-augmented scenario-based \ac{MPC} scheme in \cref{eq:QMPC}.
We compare it to a benchmark scenario-based \ac{MPC} scheme as well as the trained policy $\pi$ used to construct the terminal value $V$.
In \cref{fig:cstr_rlmpc}, we report the amount of time each agent spends near the goal as well as outside the constraints:
\begin{itemize}
	\item \textbf{Time near goal.}\quad We use the reward function in \cref{eq:cstrreward} with $\sigma^2=0.01$. We do not evaluate with the $\sigma^2$ value used for training because it is too sparse for all the comparisons.
	\item \textbf{Time outside constraints.}\quad We use the function $$e^{-\frac{\norm{s - \prox{\mathcal{X}}{s}}^2}{2 \sigma^2}} - 1,$$ also with $\sigma^2=0.01$, where $\text{prox}(\cdot)$ returns the closest point in the constraint set to the point of interest.
\end{itemize}
\Cref{fig:cstr_rlmpc} shows the sum of these quantities over $100$ time step.
Like the previous section, we sample parameter configurations and initial states to collect these measurements for each agent.
\Cref{fig:full_cstr_rlmpc}, in \cref{app:cstr}, is a more comprehensive version of \cref{fig:cstr_rlmpc}, including nominal \ac{MPC} and other instances of the value function-augmented \ac{MPC} agent; we briefly note that the prediction horizon is the main tuning parameter in \cref{eq:QMPC}, which is a significant benefit of the proposed architecture.


\begin{figure}
	\includegraphics[width=\textwidth]{boxplot_rl_mpc.pdf}
	\caption{The robust \ac{MPC} agent shows excellent constraint satisfaction, but highly variable performance in terms of reaching the goal. The robust \ac{RL} agent has no knowledge of constraints, meaning it can quickly achieve its goals. The robust value function-augmented \ac{MPC} agent, dubbed ``RL+MPC'' balances the strengths of both.}
	\label{fig:cstr_rlmpc}
\end{figure}


\begin{figure}
	\includegraphics[width=\textwidth]{cstr_timeprofile.pdf}
	\caption{A time profile of each agent evaluated in \cref{fig:cstr_rlmpc_profile}. The \ac{MPC} agent gets stuck satisfying the constraints but never achieving the goal, whereas the \ac{RL} agent immediately violates the constraints, but ultimately reaches the goal. The RL+MPC agent tames the \ac{RL} agent's trajectory, satisfying the constraints, while also eventually reaching the goal.}
	\label{fig:cstr_rlmpc_profile}
\end{figure}


\cref{fig:cstr_rlmpc} clearly indicates that the trained policy $\pi$ used to construct $V$ has no knowledge of the state constraints.
Indeed, it was trained in a global goal-driven fashion, meaning it comes as no surprise to see it almost always violates the constraints, indicated by a large mass at the bottom of \cref{fig:cstr_rlmpc}.
However, this comes with a significant payoff in terms of eventually achieving the goal, as the \ac{RL} agent spends more time overall near the goal than the \ac{MPC}-based agents. 
The robust \ac{MPC} agent shows good constraint satisfaction with its time outside constraints tapering away from zero.
However, there is wide variation in its time near the goal.
This manifests in trajectories where the agent does an excellent job at avoiding constraint violations, but is never able to reach the goal as a result.





Finally, the \ac{RL} $+$ \ac{MPC} agent, deploying \cref{eq:QMPC}, has similar or slightly tighter constraint satisfaction to the robust \ac{MPC} agent, but without the same extreme lows.
Overall, the time spent near the goal is more consistent (like the \ac{RL} agent), but with excellent constraint satisfaction.
A time profile of all three agents is shown in \cref{fig:cstr_rlmpc_profile}.
As indicated, the unconstrained \ac{RL} agent represents the quickest path to the goal, while the \ac{MPC} agent is only able to stick to the state constraints.
The \ac{RL} $+$ \ac{MPC} agent tampers the \ac{RL} agent's trajectory, hitting the upper bound of concentration $c_A$ just long enough until it can regulate $c_B$ effectively.



\section{Conclusions}
\label{sec:conclusion}



This paper advocated for treating \ac{RL} and \ac{MPC} as complementary control frameworks for solving \acp{MDP}.
Broadly speaking, \ac{RL} thrives at learning complex policies when it is able to freely explore its environment.
This is most readily achieved in simulation environments where concerns of safety are secondary; the benefit, however, is the ability to distill \emph{exploration} into high-level policies from simple rewards.
\Ac{MPC} represents another extreme, in which safety is at the forefront and achieved through repeated, online \emph{exploitation} of prior system knowledge, costs, and constraints.
We have shown that these differing perspectives \citep{mesbah2018Stochasticmodel} enable a single agent to utilize the strengths of both frameworks: \Iac{RL}-based terminal value function working in tandem with short-term \ac{MPC} planning.


While we contributed to this classical local-global view of \ac{RL} and \ac{MPC} by incorporating scenario-based planning and goal-conditioned learning, this is far from the end of the story.
One pertinent issue pertains to potential distributional mismatch between the true environment and the system model.
While we did not assumed an exact model is available, we considered the environment is well characterized by the system model uncertainty.
In principle, the robust \ac{RL} training setup could compensate for structural or parametric mismatch by including ``true'' data from the environment into its replay buffer alongside simulation data.
However, the proposed policy in \cref{eq:QMPC} would still contain a mismatched internal model.
Nonetheless, our proposed framework makes an initial step towards bringing together niche techniques form the vastly different \ac{RL} and \ac{MPC} communities.



\section*{Acknowledgement}
\label{sec:acknowledgements}
This material is based upon work supported by the U.S. Department of Energy, Office of Science, Office of Fusion Energy Sciences under award number DE‐SC0024472.


\renewcommand*{\bibfont}{\footnotesize}
%\bibliographystyle{plain}        % Include this if you use bibtex 
\bibliographystyle{elsarticle-num-names}
\bibliography{2024_RL_MPC}           % and a bib file to produce the 
                                 % bibliography (preferred). The
                                 % correct style is generated by
                                 % Elsevier at the time of printing.
                
                

\appendix



\section{LQR value function}
\label{app:lqr}



%% TODO add robust value function to illustrate multistage idea


The goal in \cref{eq:LQRobjective} is to \emph{regulate} a \emph{linear system} $x' = A x + B u$ as efficiently as possible according to a \emph{quadratic cost}. 
%Naturally, \cref{eq:LQRobjective} is referred to as the \ac{LQR}.
Although the \ac{LQR} problem contains an infinite number of decision variables, it turns out that the optimal solution is a static linear controller $u = -K x$.
This can be shown by combining the structure of the problem with Bellman's optimality equation.


\begin{enumerate}
	\item \textbf{Repurpose \cref{eq:bellmanQ}.}\quad Flipping signs, removing the expectation, plugging in the cost and dynamics equations, and finally minimizing both sides, we arrive at:
		\[\min_{u} Q^\star (x, u) = \min_{u} \left\{ x \transpose M x + u\transpose R u + \gamma \min_{u'} Q^\star (A x + B u, u') \right\} \]
	\item \textbf{Quadratic optimal cost.} ``Guess'' $\underset{u}{\min}\ Q^\star(x, u) = x\transpose P x$ for some symmetric $P$. (See below for an intuitive argument.) We then have 
		\[ x\transpose P x = \min_{u} \left\{ x \transpose M x + u\transpose R u + \gamma \left(A x + B u\right)\transpose P \left(A x + B u\right) \right\} \label{eq:bellmanLQR}\]
	\item \textbf{Solve for $u$.}\quad The right-hand side above can be solved by setting the gradient of the inside term equal to zero to find $u = -K x$, where
		\[K = \gamma \left(R + \gamma B\transpose P B\right)^{-1} B\transpose P A \]
	\item \textbf{Back-substitute.}\quad $K$ is expressed in terms of $P$. By plugging the solution $u = -K x$ back into \cref{eq:bellmanLQR} we arrive at the \ac{DARE}:
		\[P = M + \gamma A\transpose P A - \gamma^2 A\transpose P B \left(R + \gamma B\transpose P B \right)^{-1} B\transpose P A\]
\end{enumerate}


%% TODO 1-D illustration, cite numerical solvers, etc
% also talk about why \gamma < 1 might be useful, such as convergence / numerical stability
The \ac{DARE} is a tractable form of Bellman's optimality equation for \ac{LQR}.
Like Bellman's equation, the desirability of the optimal solution in the \ac{DARE} depends on the discount factor.
For instance, as $\gamma \to 0$, the controller becomes degenerate, resulting in no control actions.
For an open-loop unstable system, this is clearly problematic.


Next, we illustrate why one would make the guess $\underset{u}{\min}\ Q^\star(x, u) = x\transpose P x$ in the first place (other than the simple fact that it works).
This is a two-step process:
\begin{enumerate}
	\item \textbf{Correspondence between linear controllers and quadratic value functions.}\quad Note that for any controller $\hat{K}$ with finite return, its value function is quadratic. Conversely, for any symmetric positive definite $\hat{P}$, solving
		\[
		\min_{u}\left\{ x\transpose M x + u\transpose R u + \left(A x + B u\right)\transpose \hat{P} \left(A x + B u\right) \right\}
		\]
		results in a linear controller.
	\item \textbf{The optimization problem in \cref{eq:LQRobjective} is lower bounded by the optimal linear controller.}\quad Define an surrogate objective over linear controllers:
\begin{equation}
\begin{aligned}
    &\underset{K}{\text{minimize}} && \sum_{t=0}^{\infty} \gamma^{t} \left( x_t\transpose M x_{t} + u_{t}\transpose R u_t \right) \\
    &\text{subject to } && x_{t+1} = A x_t + B u_t \\
    & && u_t = -K x_t
\end{aligned}
\label{eq:K-LQR}
\end{equation}
	Of course, \cref{eq:LQRobjective} lower bounds \cref{eq:K-LQR}.
	Showing the other direction starts with an auxiliary infinite-horizon objective:
\begin{equation}
\begin{aligned}
    &\underset{u_0, \ldots, u_{N-1}}{\text{minimize}} && \sum_{t=0}^{N-1} \gamma^{t} \left( x_t\transpose M x_{t} + u_{t}\transpose R u_t \right) + \text{cost}(\hat{K})\\
    &\text{subject to } && x_{t+1} = A x_t + B u_t
\end{aligned}
\label{eq:finitelqr}
\end{equation}
where $\text{cost}(\hat{K})$ is the cost of applying some linear controller $K$ after $N-1$ time steps.
Note that solving \cref{eq:finitelqr} results in a linear controller, meaning \cref{eq:K-LQR} lower bounds \cref{eq:finitelqr}.
Moreover, for $N = 1, 2, 3, \ldots$ the respective values in \cref{eq:finitelqr} decrease in $N$.
\end{enumerate}
Taken together, the \ac{LQR} objective is equivalent to optimizing over linear controllers, affirming the original choice of $\underset{u}{\min}\ Q^\star(x, u) = x\transpose P x$ in the Bellman equation.




\section{Additional CSTR results and details}
\label{app:cstr}



\textbf{CSTR system description.}\quad
Refer to \cref{tab:cstr} for parameter values in the following system model:
\begin{align}
	\dot{c}_{A} &= F \left( c_{A,0} - c_A \right) - k_1 c_A - k_3 c^2_A \\
	\dot{c}_{B} &= -F c_B + k_1 c_A - k_2 c_B \\
	\dot{T}_{R} &= \frac{k_1 c_A H_{R, ab} + k_2 c_B H_{R, bc} + k_3 c^2_A H_{R, ad}}{- \rho C_p} + F \left( T_{\text{in}} - T_R \right) \frac{K_w A_R \left( T_K - T_R \right)}{\rho C_p V_R} \\
	\dot{T}_{K} &= \frac{\dot{Q} + K_w A_R T_{\text{dif}}}{m_k C_{p,k}},
\end{align}
where
\begin{align}
	k_1 &= \beta k_{0,ab}\ \text{exp} \left( \frac{-E_{A, ab}}{T_R + 273.15} \right) \\
	k_2 &= k_{0, bc}\ \text{exp} \left( \frac{-E_{A,bc}}{T_R + 273.15} \right)\\
	k_3 &= k_{0, ad}\ \text{exp} \left( \frac{-\alpha E_{A,ad}}{T_R + 273.15} \right).
\end{align}
$\alpha, \beta$ are uncertainty parameters, with nominal values of $1.0$. For robust \ac{MPC}, the extreme values for $\alpha$ are $0.95$ and $1.05$; for $\beta$ they are $0.9$ and $1.1$.
For robust \ac{RL} training, we take these intervals and grid them into $10$ evenly spaced values.


\begin{table*}[tbp]
\caption{Certain parameters in the CSTR model.}
\begin{center}
\begin{tabular}{ll|ll}
\toprule
$k_{0,ab}$ & $1.287 \cdot 10^{12}\ \text{h}^{-1}$ & $\rho$ & $0.9342$ kg/l \\
$k_{0, bc}$ & $1.287 \cdot 10^{12}\ \text{h}^{-1}$ & $C_p$ & $3.01$ kJ/kg K \\
$k_{0, ad}$ & $9.043 \cdot 10^9$ l/mol h & $C_{p,k}$ & $2.0$ kJ/kg K \\
$R$ & $8.3144621 \cdot 10^{-3}$ & $A_R$ & $0.215$ $\text{m}^2$ \\
$E_{A, ab}$ & $9758.3 \cdot R$ kJ/mol & $V_R$ & $10.01$ l\\
$E_{A, bc}$ & $9758.3 \cdot R$ kJ/mol & $m_k$ & $5.0$ kg \\
$E_{A, ad}$ & $8560.0 \cdot R$ kJ/mol & $T_{\text{in}}$ & $130.0$ $^{\circ}$C \\
$H_{R, ab}$ & $4.2$ kJ/molA & $K_w$ & $4032.0$ kJ/h $\text{m}^2$ K \\
$H_{R, bc}$ & $-11.0$ kJ/molB & $c_{A,0}$ & $5.1$ mol/l \\
$H_{R, ad}$ & $-41.85$ kJ/molA & & \\
\bottomrule
\end{tabular}
\end{center}
\label{tab:cstr}  
\end{table*}


\textbf{Tuning the value function-augmented \ac{MPC} agent.}\quad
The main tuning parameters in \cref{eq:QMPC} are the prediction horizon and the variance in the Gaussian reward.
One could, in principle, also manipulate the uncertainty set and the state-action constraints. 
We consider these to be fixed.
Moreover, one could explore the possibility of fine-tuning the \ac{RL} agent online:
If the uncertainty set is not sufficiently accurate, then the \ac{RL} agent could adapt its policy to online data; however, this complicates the interplay between the global \ac{RL} value function and the now-inaccurate local replanning.
This was also discussed in \cref{sec:conclusion} and is a topic for future research.


\begin{figure}
	\includegraphics[width=\textwidth]{boxplot_rl_mpc_full.pdf}
	\caption{A comprehensive version of \cref{fig:cstr_rlmpc}.}
	\label{fig:full_cstr_rlmpc}
\end{figure}



We fixed the variance parameter to be $\sigma^2 = 0.25^2$.
This was fixed somewhat arbitrarily.
Mainly, it is larger than the value used by the \ac{RL} agent ($\sigma^2 = 0.0001$), but still small relative to the range of the target variable ($0.1-2.0$).
The more important parameter of the value function-augmented \ac{MPC} agent is the prediction horizon, as the terminal value function is concerned with performance, while the planning steps are concerned with constraint satisfaction.
\Cref{fig:full_cstr_rlmpc} shows the effect of varying the prediction horizon.
Overall, increasing the prediction horizon leads to better constraint satisfaction.
We only needed to set $N=5$ to get roughly the same result as the benchmark \ac{MPC} with $N=20$.
Finally, only having the prediction horizon as the main tuning knob is a significant advantage, especially when one can start from $N=0$ as the best option for pure performance.




\section{Implementation details}
\label{app:implementation}




The software implementation of the ideas presented in this paper relied heavily on \texttt{\href{https://www.do-mpc.com}{do-mpc}} \citep{fiedler2023DompcFAIR}, \texttt{\href{https://docs.cleanrl.dev}{CleanRL}} \citep{huang2022CleanRLHighquality}, and \texttt{\href{https://github.com/Tim-Salzmann/l4casadi}{L4CasADi}} \citep{salzmann2024LearningCasADi}.
We outline how these tools were used in this work.
The corresponding code is available here: \url{https://github.com/NPLawrence/RL-MPC}.


\begin{itemize}
	\item \textbf{do-mpc} is an all-in-one toolbox for nonlinear and scenario-based \ac{MPC}, including simulation, state estimation, and data management tools. It has a modular structure, which enabled us to integrate it into \iac{RL} pipeline. More specifically, we created a Gym environment for \ac{RL} training by using the do-mpc simulator to generate the state transitions under uncertain system parameters. At deployment time, we created the value function-augmented \ac{RL} agent in \cref{eq:QMPC} by specifying the value function as the terminal cost, specifying soft constraints, and defining a custom stage cost, all subject to a set of uncertainty parameters.
	\item \textbf{CleanRL} is a library of single-file implementations of deep \ac{RL} algorithms. This structure streamlines the process of building a custom pipeline involving the do-mpc-based Gym environment and the option to evaluate a nonstandard policy.
	\item \textbf{L4CasADi}\quad is a package that enables the integration of PyTorch models with CasADi. Since do-mpc is an API for CasADi \citep{Andersson2019}, L4CasADi was essential for using the \ac{RL}-learned value function as a terminal cost in \cref{eq:QMPC}. do-mpc offers some functionality for integrating ML models, but L4CasADi is a much more flexible option.
\end{itemize}




\end{document}