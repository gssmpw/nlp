\section{Conclusions}

In this work, we have investigated the conditions regarding round complexity for achieving $\approximateagreement$ on trees in the synchronous model. Our results extend previous findings from real-valued domains to tree-structured input spaces, providing novel insights for achieving efficient $\approximateagreement$ in discrete input spaces.
Our primary contribution is showing that $O\left( \frac{ \log \abs{ \vertices(\tree)} } {\log \log \abs{\vertices(\tree)}} \right)$ rounds are sufficient. This result is derived via a non-trivial reduction that transforms the problem of  $\approximateagreement$ on trees into its real-number counterpart, enabling the use of the protocol of \cite{BenDoHo10} as a building block. Additionally, we have established that $\Omega\left( \frac{ \log \diameter(T)} {\log \log \diameter(T)} \right)$ rounds are necessary when $t \in \Theta(n)$
%- assuming a relatively \textit{large} number of corrupted parties - 
by adapting the impossibility result of \cite{Fekete90} from real numbers to trees, thereby demonstrating the optimality of our protocol for trees with large diameter, i.e., when $\diameter(\tree) \in \Theta(|\vertices(\tree)|)$.

Our findings highlight promising directions for further investigation.
First, our protocol achieves optimal round complexity for large-diameter trees, up to constant factors. Improving the constants in the round complexity of 
$\approximateagreement$ for real numbers would directly refine our protocol's efficiency, making it more practical for real-world applications.

Second, it would be valuable to determine whether our lower bound on round complexity can be matched for trees $T$ with low diameter $\diameter(\tree) \in o(|\vertices(\tree)|)$. 
Finally, while our work focuses on trees, it remains an open question whether similar round-optimal guarantees can be achieved for synchronous $\approximateagreement$ on broader classes of graphs.
