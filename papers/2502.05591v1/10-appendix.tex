\appendix
\section*{Appendix}

\section{Lower Bound: Missing proofs} \label{appendix:lower-bound}
We present the proof of Theorem \ref{thm:fekete-lowerbound-trees}. This converts the bound of \cite{Fekete90} regarding how close the honest parties' values/vertices get after $R$ rounds into a lower bound on round complexity.

\FeketeCorollary*
\begin{proof}[Proof of \Cref{thm:fekete-lowerbound}]
    By \Cref{corollary:fekete-bound} we know that for any $\approximateagreement$ protocol, there exist two honest parties such that their outputs differ by at least $K(R, D)$ after $R$ rounds, where $D := D(T)$ is the diameter of the graph (note that at least diameter $D(T)$ many different input values exist in $T$). Due to Equation \ref{eq:fekete-performance} and our assumption $t = n/c$ for some constant $c$ (note that throughout the paper we assume $t < n/3$ and thus $c > 3$) we can simplify the equation as follows:
    \begin{align*}
        K(R, D) \geq D \cdot \frac{t^R}{R^R \cdot (c \cdot t+t)^R} \geq \frac{D}{R^R \cdot (c+1)^R}
    \end{align*}
    As any $\approximateagreement$ protocol can only terminate if the difference of all output values is at most $1$, we can never finish before
    \begin{align*}
        \frac{D}{R^R \cdot (c+1)^R} \leq 1.
    \end{align*}
    We will now show that for any $R \leq \frac{1}{2} \cdot \frac{\log_{c+1} D}{\log_{c+1} \log_{c+1} D}$ this is not true:
    \begin{align*}
        & \log_{c+1} \left(  \frac{D}{R^R \cdot (c+1)^R} \right) \\
        >& \log_{c+1} D - \frac{\log_{c+1} D}{2\log_{c+1} \log_{c+1} D} \cdot \left(\log_{c+1} \log_{c+1} D- \log_{c+1} \log_{c+1} \log_{c+1} D + 1 \right) \\
        \geq& \frac{\log_{c+1} D}{2} -  \frac{\log_{c+1} D}{2 \log_{c+1} \log_{c+1} D} \\
        >& 0
    \end{align*}
    Thus, exponentiation with base $c+1$ on both sides shows the claim. It follows that the number of rounds is at least $R = \Omega(\frac{\log D}{\log \log D})$.
\end{proof}


\section{Warm-Up: Additional Proofs} \label{appendix:realvalues}

As mentioned in Section \ref{section:warmup}, the analysis of \cite{BenDoHo10} regarding algorithm $\realAA$ proves that $\approximateagreement$ is achieved for $\varepsilon = 1/n$. In order to extend the analysis for any $\varepsilon$, we make use of Claim 12 of the full version  \cite{BDH10} of \cite{BenDoHo10}.


In the following, assume that $V_{R}$ denotes the multiset of values held by the honest parties in round $3 \cdot R$ (iteration $R$), and $V_0$ denotes the multiset of honest inputs.

\begin{lemma}[Claim 12 of \cite{BDH10}]
    \label{lmm:Claim12}
    $(\max V_R - \min V_{R}) \leq (\max V_0 - \min V_0) \cdot \frac{t^R}{R^R \cdot (n - 2t)^R}$. 
\end{lemma}

Additionally, we require the following validity property:
\begin{lemma}[Claim 8 of \cite{BDH10}] \label{lemma:claim8}
    $V_R \subseteq [\min V_0, \max V_0]$.
\end{lemma}

We are now ready to prove Theorem \ref{theorem:real-values-aa}, which we restate below.

\RealValuesAA*
\begin{proof}
    By the Lemma \ref{lmm:Claim12}, we know that the maximum absolute difference between the output values of any two honest parties is at most
    \begin{align*}
        \frac{(b-a) \cdot t^R}{(n-2t)^R \cdot R^R} < \frac{b-a}{R^R}
    \end{align*}
    after $R$ iterations. We will now show that for any $R \geq  \frac{20}{9} \cdot \frac{\log \delta}{\log \log \delta}$ where $\delta := (b-a)/\varepsilon$, we obtain $(b-a) / \varepsilon \leq R^R$, and thus $\varepsilon$-Agreement holds. We will now show this claim.
    \begin{align*}
        \log_2(R^R) &\geq R \log_2 R \\
        &\geq \frac{20}{9} \cdot \frac{\log_2 \delta}{\log_2 \log_2 \delta} \cdot (\log_2 \log_2 \delta - \log_2 \log_2 \log_2 \delta) \\
        &\geq \frac{20}{9} \cdot\frac{\log_2 \delta}{\log_2 \log_2 \delta} \cdot (\log_2 \log_2 \delta - \frac{11}{20} \cdot \log_2 \log_2 \delta) \\
        &=\log_2 \delta \\
        &= \log_2\left( \frac{a-b}{\varepsilon} \right)
    \end{align*}
    The claim thus follows by exponentiation on both sides (with base $2$).
    Note that in line 3 we use that the function $\log_2(x)/x$ has its global maximum (at $x = e$) of $\frac{1}{e \cdot \ln(2)} < 11/20$ and therefore also $\log_2(\log_2\log_2(x))/\log_2(\log_2(x)) < 11/20$. 
    
    To end the statement of the Theorem, note that any of the $R$ iterations require $3$ synchronous rounds (\cite{BDH10}), and thus, the protocol terminates after at most
    \begin{align*}
         R_{\realAA}(b - a, \varepsilon) \leq \lceil 3 \cdot R \rceil < \left\lceil 7 \cdot \frac{\log_2 ((b - a)/\varepsilon)}{\log_2 \log_2 ((b - a)/\varepsilon)} \right\rceil
    \end{align*}
    rounds. In addition, Validity holds due to Lemma \ref{lemma:claim8}, hence $\approximateagreement$ is achieved.
\end{proof}



We also include the proof of Remark \ref{remark:agreement-helper}, restated below.
\AgrementHelper*
\begin{proof}
    Assume without loss of generality that $j < j'$. Since $j' - j \le 1$, either there is an integer $z$ such that $z \leq j, j', \leq z + 1$ or there is an integer $z$ such that $z - 1 \leq j \leq z \leq j' \leq z + 1$. In the first case, both $\closestInt(j)$ and $\closestInt(j')$ are either $z$ or $z + 1$ by the definition of $\closestInt(\ )$. In the second case, assume that $\closestInt(j) = z - 1$, and $\closestInt(j') = z + 1$. Then, $j - (z - 1) < z - j$, hence $j < z - \frac{1}{2}$,
    while $(z + 1) - j' \leq j' - z$, and therefore $j' \geq z + \frac{1}{2}$. This leads to a contradiction, as it implies $j' - j > 1$. Hence, in both cases, $\closestInt(j') - \closestInt(j) \leq 1$.
\end{proof}












\section{Protocol For Trees: Missing Proofs} \label{appendix:trees}

We present the proof of Lemma \ref{lemma:list-construction}, describing the guarantees of algorithm $\listConstruction$. The parties run this algorithm locally to convert the input tree $T$ into a list representation that provides a few special properties.
\ListConstruction*
\begin{proof}
First, we prove that $\listConstruction(\tree, \roott)$ terminates in a finite amount of time. The algorithm follows a depth-first search (DFS) traversal, which has a time complexity of $\mathcal{O}(|V| + |E|)$. Since $T$ is a tree, it has $|E| = |V| - 1$ edges, leading to an overall complexity of $\mathcal{O}(|V|)$. Thus, the algorithm terminates within a finite number of steps. Then, we prove each property in order.

\paragraph{Property 1.} This follows directly from the definition of depth-first search (DFS). At each step of the DFS traversal, if a vertex has an unvisited child, we move to that child. Otherwise, we backtrack to the parent. This ensures that each successive pair of vertices in $L$ represents either a move to a child or a return to a parent, both of which are adjacent in $T$.

\paragraph{Property 2.} We proceed by induction on the number of vertices $m = |V(T)|$. 

\textbf{Base case:} If $T$ consists of a single vertex, then $L = [r]$, where $r$ is the root. Since $1 \leq 2|V(T)| = 2$, the claim holds.

\textbf{Inductive step:} Suppose the claim holds for all trees with at most $m-1$ vertices. Consider a tree $T$ with $m$ vertices. Let $T'$ be the tree obtained by removing a leaf $v$ and its edge $(u,v)$.

By the induction hypothesis, the DFS traversal on $T'$ results in a list $L'$ of length at most $2(m-1)$. When we add $v$ back, the DFS first visits $v$ once and then backtracks to $u$, adding two elements to $L'$. Therefore, $|L| = |L'| + 2 \leq 2(m-1) + 2 = 2m$, which completes the induction.

To show that every vertex appears in $L$, note that DFS visits each vertex at least once upon first exploration, ensuring that every vertex is included.

\paragraph{Property 3.} During DFS, the first occurrence of $v$ in $L$ happens before visiting any of its descendants. The last occurrence happens after all descendants have been visited. Hence, the indices corresponding to $v$ enclose all its descendants in $L$. If $u$ is in the subtree of $v$, all its occurrences in $L$ appear between $i_{\min}$ and $i_{\max}$.
Conversely, if $L(u) \subseteq [i_{\min}, i_{\max}]$, then DFS must have discovered $u$ while visiting $v$'s subtree, confirming that $u$ is in the subtree rooted at $v$.

\paragraph{Property 4.} If one of $v$ or $v'$ is an ancestor of the other, then the lowest common ancestor (LCA) of two nodes $v$ and $v'$ is trivially in this range. Otherwise, let $w = \operatorname{LCA}(v, v')$. If $w \notin {L_k : \min(i, i') \leq k \leq \max(i, i')}$, then DFS would have visited $v$'s subtree and then $v'$'s subtree without encountering $w$. However, this contradicts the DFS property that it does not leave a subtree before visiting all its nodes. Thus, $w$ must appear in the interval, completing the proof.


\end{proof}
