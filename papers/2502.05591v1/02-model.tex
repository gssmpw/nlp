\section{Preliminaries}

We describe the model and a few concepts and definitions.

\paragraph{Model.}
We consider $n$ parties $\party_1, \party_2, \dots, \party_n$, running a protocol over a network. The network is fully-connected: each pair of parties is connected by an authenticated channel.
We assume that the network is synchronous: the parties' clocks are synchronized, all messages get delivered within $\Delta$ time, and $\Delta$ is publicly known. 

\paragraph{Adversary.}
Our protocols consider a computationally-unbounded adaptive adversary that can permanently corrupt up to $t < n / 3$ parties at any point in the protocol's execution. Corrupted parties become byzantine: they may deviate arbitrarily (maliciously) from the protocol. Our impossibility arguments, however, hold even against a static adversary, that has to choose which parties to corrupt at the beginning of the protocol's execution.

\paragraph{Approximate Agreement on $\realvalues$.} We recall the definition of $\approximateagreement$, as presented in \cite{JACM:DLPSW86}.
\begin{definition} \label{def:aa-real}
Let $\Pi(\varepsilon)$ be an $n$-party protocol where each party holds a value in $\realvalues$ as input, and parties terminate upon generating an output in $\realvalues$. $\Pi$ achieves $\approximateagreement$ if the following properties hold for any $\varepsilon> 0$ even when $t$ of the $n$ parties are corrupted:
(Termination) All honest parties terminate; (Validity) Honest parties' outputs lie in the honest inputs' range; ($\varepsilon$-Agreement) If two honest parties output $v$ and $v'$, then $\abs{v - v'} \leq \varepsilon$.
\end{definition}
We add that we will use the term \emph{valid value} to refer to a value in the honest inputs' range.

\paragraph{Paths and trees.} Before presenting the definition of $\approximateagreement$ on trees, we need to describe a few notations and concepts. For a tree $\tree$, we use $\vertices(\tree)$ to denote its set of vertices, and $\diameter(\tree)$ to denote its diameter, i.e., the length of its longest path. We denote the (unique) path between two vertices $u, v \in \vertices(\tree)$ by $\pathh(u, v)$.
The distance $\distance(u, v)$ between two vertices $u, v \in \vertices(T)$ is defined as the length (i.e., number of edges) of $\pathh(u, v)$.
We also use the notation $(v_1, v_2, \ldots, v_k)$ to describe a path of $k$ vertices: here, vertices $v_{i}$ and $v_{i + 1}$ are adjacent. In addition, for some adjacent vertices $v$ and $w$, where $v$ is the endpoint of some path $\pathh(u, v)$ and $w \notin \vertices(\pathh(u, v))$, we write $\pathh(u, v) \oplus (v, w)$ to denote the path $(u, \ldots, v, w)$.

To adapt the validity requirement presented in \Cref{def:aa-real} to trees, we will need to define the \emph{convex hull} $\hull{S}$ of a set of vertices $S \subseteq \vertices(\tree)$: if $\tree'$ is the smallest connected subtree of $\tree$ that contains all the vertices in $S$, $\hull{S} = \vertices(\tree')$.
Note that $w \in \hull{S}$ if and only if there are some vertices $u, v \in S$ such that $w \in \vertices(\pathh(u, v))$. 

We may now define $\approximateagreement$ on trees, as presented in \cite{DISC:NoRy19}.
\begin{definition} \label{def:aa-trees}
Consider a (labeled) tree $\tree$, and let $\Pi$ be an $n$-party protocol where each party holds a vertex in $\tree$ as input, and parties terminate upon generating an output in $\vertices(\tree)$. $\Pi$ achieves $\approximateagreement$ if the following properties hold even when $t$ of the $n$ parties are corrupted:
(Termination) All honest parties terminate; (Validity) Honest parties' outputs lie in the honest inputs' convex hull; ($1$-Agreement) If two honest parties output $v$ and $v'$, then $\distance(v, v') \leq 1$.
\end{definition}
Similarly to \emph{valid values}, we will use the term \emph{valid vertex} to refer to a vertex in the honest inputs' convex hull.