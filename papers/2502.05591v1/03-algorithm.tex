\section{Warm-up: Protocol for Paths}\label{section:warmup}

Building towards our solution, we first describe a warm-up protocol for paths: the input tree is a path $P$.
We make use of the protocol of \cite{BenDoHo10}, denoted by $\realAA$, which achieves $\approximateagreement$ on $\realvalues$ with asymptotically-optimal round complexity.

Internally, protocol $\realAA$ roughly follows the general iterative framework described in our papers' introduction: in each iteration, the parties distribute their values, each party discards the lowest $t$ and the highest $t$ values received (ensuring that all remaining values are valid) and computes its new value as the average of the remaining values. The value distribution mechanism plays a crucial role in ensuring that protocol $\realAA$ matches the lower bound established in \cref{thm:fekete-lowerbound}. Roughly, this mechanism guarantees that, if a byzantine party $\tilde{\party}$ attempts to cause inconsistencies in iteration $i$, then honest parties detect such behavior and subsequently \emph{ignore} $\tilde{\party}$ in all future iterations. If $t_i$ of the $t$ byzantine parties cause inconsistencies in iteration $i$, the honest values' range shrinks by a (multiplicative) factor of $\frac{t_i}{n - 2t}$. This is a notable improvement over the factor of roughly $\frac{1}{2}$ observed in $\approximateagreement$ protocols that do not leverage information from prior rounds \cite{JACM:DLPSW86}. Consequently, after $R$ iterations, the honest values' range is reduced by a factor of at least $\frac{t^R}{R^R \cdot (n - 2t)^R}$. The protocol allows the honest parties to terminate once they observe that their values are $\varepsilon$-close (possibly in consecutive iterations).


It is important to note that the round complexity analysis presented in \cite{BenDoHo10} is based on the assumption that $\varepsilon = 1/n$. The theorem below extends their analysis to any $\varepsilon$. We attached the technical details of the proof to \Cref{appendix:realvalues}.
\begin{restatable}{theorem}{RealValuesAA}\label{theorem:real-values-aa}
    There is a protocol $\realAA(\varepsilon)$ achieving $\approximateagreement$ on real values when $t < n / 3$. If the honest inputs are in $[a, b]$, $\realAA(\varepsilon)$ ensures termination within $R_{\realAA}(b - a, \varepsilon) = \left\lceil 7 \cdot \frac{\log_2 ((b - a)/\varepsilon)}{\log_2 \log_2 ((b - a)/\varepsilon)} \right\rceil$ rounds.
\end{restatable}


Protocol $\realAA$ can be adapted to paths in a straightforward manner. The parties label the $k = \diameter(\pathh)$ vertices in the input path $(v_1, v_2, \ldots, v_k)$ (so that $v_i$, $v_{i + 1}$ are adjacent, and $v_1$ is the endpoint with the lowest label in lexicographic order). If a party's input is the vertex now labeled as $v_i$, it joins $\realAA(1)$ with input $i$. Each party obtains a value $j \in \realvalues$ from $\realAA$ and may output vertex $v_{\closestInt(j)}$, where $\closestInt(j)$ denotes the closest integer to $j$. That is, if $z \leq j < z + 1$ for $z \in \integers$, $\closestInt(j) := z$ if $j - z < (z + 1) - j$ and $\closestInt(j) := z + 1$ otherwise. Remark \ref{remark:validity-helper} follows immediately from the definition of $\closestInt$ and ensures that vertices $v_{\closestInt(j)}$ are valid. 
Remark \ref{remark:agreement-helper}, proven in Appendix \ref{appendix:realvalues}, allows us to discuss $1$-Agreement.
\begin{remark} \label{remark:validity-helper} 
If $i_{\min}, i_{\max} \in \integers$ and $j \in [i_{\min}, i_{\max}]$, then $\closestInt(j) \in [i_{\min}, i_{\max}]$.
\end{remark}

\begin{restatable}{remark}{AgrementHelper}\label{remark:agreement-helper}
    If $j, j' \in \realvalues$ satisfy $\abs{j - j'} \leq 1$, then $\abs{\closestInt(j) - \closestInt(j')} \leq 1$. 
\end{restatable}

As two honest parties obtain $1$-close values $j, j'$,  we obtain that $\abs{\closestInt(j) - \closestInt(j')} \leq 1$. This implies that the honest parties' vertices $v_{\closestInt(j)}$ and $v_{\closestInt(j')}$  have distance at most $1$, and $1$-Agreement holds. Therefore, we have achieved $\approximateagreement$ (with $\varepsilon = 1$) on the path $\pathh$ within  $O\left( \frac{\log \diameter(\pathh)}{\log \log \diameter(\pathh)} \right)$ rounds by \Cref{theorem:real-values-aa}, where $\diameter(\pathh)$ is the diameter (and thus also the length) of $\pathh$.


\section{Moving Towards Trees} \label{section:path-assumed}
We reduce the problem of solving $\approximateagreement$ on trees to $\approximateagreement$ on $\realvalues$ as well. This section presents a stepping stone towards our final solution: we assume that the parties \emph{know} a path $\pathh$ in the tree that passes through the honest input's convex hull. That is, there is a vertex $v$ in $\vertices(\pathh)$ such that $v$ is in the honest inputs' convex hull. 
Then, the parties may proceed as follows: each party with input $v \in \vertices(T)$ computes the projection of vertex $v$ onto the path $\pathh$, denoted by $\projection_{\pathh}(v)$. This is the vertex in $\pathh$ that has the shortest distance to $v$, as shown in Figure \ref{figure:path-trick}.

\begin{figure}[h]
\centering
\includegraphics[width=0.7\textwidth]{figures/path-trick1.png}
\caption{Let $\pathh$ be the assumed path, represented by the sequence of vertices $v_1, v_2, \dots, v_8$. The vertices $u_1, u_2, u_3$ correspond to the honest inputs, whose convex hull is highlighted in green. The projections of $u_1, u_2, u_3$ onto path $\pathh$ are vertices $v_3, v_4, v_6$ respectively. 
}\label{figure:path-trick}
\end{figure}

Note that $\projection_{\pathh}(v)$ is in the honest inputs' convex hull, as described by the remark below.
\begin{restatable}{lemma}{ObviousHullMaintained}\label{remark:convex-hull-maintained}
    Consider a set of vertices $S \subseteq \vertices(\tree)$ and a path $\pathh$ in $\tree$ such that $\vertices(\pathh) \cap \hull{S} \neq \emptyset$. Then, for any $v \in S$, $\projection_{\pathh}(v) \in \vertices(\pathh) \cap \hull{S}$.
\end{restatable}
\begin{proof}
    Let $v$ be an arbitrary vertex in $S$. If $v \in \vertices(P)$, then $v = \projection_{\pathh}(v)$ and the statement follows immediately.
    Otherwise, assume for contradiction that $v_{\pathh} := \projection_{\pathh}(v) \notin \hull{S}$.
    
    Let $v_h$ be a vertex in $\vertices(\pathh) \cap \hull{S}$.  Note that all vertices on the path $\pathh(v, v_h)$ are in $\hull{S}$. We label the vertices in this path as  $(v_1 := v, v_2, \ldots v_m := v_h)$.
    Afterwards, we label the the vertices on the path $\pathh(v, v_{\pathh})$ as $(v := u_1, u_2, \ldots, u_k := v_{\pathh})$. Since $\pathh(v, v_{\pathh})$ and $\pathh(v, v_h)$ have $u_1 = v_1 = v$, we may denote by $i \geq 1$ the highest index such that $u_i = v_i$.  Note that there is no $j, j' > i$ with $v_j = v_j'$, as this would imply the existence of a cycle in $\tree$. 

    
     This implies that the subpaths $(u_i, u_{i + 1} \ldots, u_k)$ and $(v_i, v_{i + 1}, \ldots, v_m)$ are non-empty and only intersect in $u_i = v_i$. Then, $(u_k = v_{\pathh}, u_{k - 1}, \ldots u_i, v_{i + 1}, \ldots, v_m = v_h)$ is a path connecting $v_{\pathh}$ and $v_h$ in the tree $T$.
    However, since $\tree$ is a tree, there is a unique path $\pathh(v_{\pathh}, v_h)$, and, since both $v_{\pathh}$ and $v_h$ are vertices of $\pathh$, then $\pathh(v_{\pathh}, v_h)$ must be a subpath of $\pathh$.
    This implies that vertex $u_i$ is on $\pathh$, and, since $v_{\pathh}$ is the vertex on path $\pathh$ that has the shortest distance to $v$, $v_{\pathh} = u_i = u_k$. This means that $(u_1 = v_1, u_2 = v_2, \ldots u_i = v_i, v_{i + 1} \dots v_m)$ is the unique path between $v$ and $v_h$. Since $v, v_h \in \hull{S}$, then all vertices on this path are in $\hull{S}$. Therefore,  $v_{\pathh}$ is in $\hull{S}$.
\end{proof}

Note that we can now use this statement to reduce the problem of trees to path, and proceed using the same approach as described in Section \ref{section:warmup}. We label the vertices in the path $\pathh$  as $(v_1, v_2, \ldots, v_m)$, where $v_1$ is the endpoint with the lower label lexicographically, and each party joins $\realAA$ with input $i$, where $v_i$ is the projection $\projection_{\pathh}(v)$ of the party's input vertex $v$. Each party obtains an output $j \in \realvalues$ and may output $v_{\closestInt(j)}$. These vertices are in the convex hull of the honest parties' projections, and therefore in the honest inputs' convex hull. 


\section{Finding a Path} \label{section:path}
In Section \ref{section:path-assumed}, we have discussed how $\approximateagreement$ can be achieved when the honest parties \emph{know} a path that intersects the convex hull. In this section, we build a subprotocol provides the honest parties with such \emph{paths}. Intuitively, it may seem that finding a path $\pathh$ that passes through the honest inputs' convex hull comes down to solving \emph{Byzantine Agreement}. This would require $t + 1 = O(n)$ communication rounds \cite{DolStr83}, which generally prevents us from achieving our round-complexity goal.

We will instead implement a subprotocol $\pathfinder$ that enables the honest parties to \emph{approximately} agree on such a path. Concretely, each honest party will obtain a subpath $\pathh$ of $\tree$ such that (i) $\pathh$ intersects the honest inputs' convex hull, and (ii) if two honest parties obtain two different paths $\pathh$ and $\pathh'$, then either $\pathh$ is $\pathh'$ with one additional edge, or $\pathh'$ is $\pathh$ with one additional edge. 
Formally, if $\pathh = (v_1, \ldots, v_k)$ and $\pathh' = (u_1, u_2, \ldots u_{k'})$, then either $\pathh' = \pathh \oplus (v_k, u_{k'})$ or $\pathh = \pathh' \oplus (u_{k'}, v_k)$.
This will suffice to use the approach of Section~\ref{section:path-assumed}.

In order to find such paths, we will fix some arbitrary $\roott$ for our tree $\tree$. This could be, for instance, the node with the lowest label lexicographically. Roughly, we will enable honest parties to obtain a $1$-Agreement on some vertices in a subtree rooted at a 
valid vertex. Then, if this approach provides an honest party with a vertex $v$, $\pathh(\roott, v)$ intersects the honest inputs' convex hull. In addition, the $1$-Agreement property will ensure that honest parties' paths $\pathh(\roott, v)$ are the same except for the potential additional edge.

\paragraph{List representation.} In order to provide the honest parties with vertices in the subtree rooted at a valid vertex, we will first convert the (now rooted) tree $\tree$ into a list $L$ with a few special properties. This list representation relies on a technique that is commonly known for finding \emph{lowest common ancestors} of given vertices in a rooted tree \cite{LCATrick}. Each party locally runs a depth-first-search starting from the fixed $\roott$, and writes down each vertex whenever visited in the depth-first-search. 
We present the code for this step below, and we present an example in Figure \ref{figure:small-tree}.
\begin{dianabox}{$\listconstruction(\tree, v)$}
\algoHeadNoBold{$\dfs(v)$, where $v \in \vertices(\tree)$, $L$ is a global list:}
	\begin{algorithmic}[1]
            \State Append $v$ to $L$.
            \State For each neighbor $v \notin L$ of $v$ (ordered by labels):
            \State \hspace{0.5cm} Run $\dfs(v')$. Afterwards, append $v$ to $L$.
    \end{algorithmic}
    \vspace{0.5cm}
    
	\algoHeadNoBold{Code for party $\party$, given tree $\tree$ and a vertex $\roott \in \vertices(\tree)$:}
	\begin{algorithmic}[1]
            \State $L = [\ ]$. Run $\dfs(\roott)$. Afterwards, return $L$.
    \end{algorithmic}
\end{dianabox}

%------------------------------------------
\begin{wrapfigure}{r}{5.3cm}
\centering
\includegraphics[width=5cm]{figures/small-tree-2}
\caption{An input tree.}\label{figure:small-tree}
\end{wrapfigure} 
%------------------------------------------



\Cref{figure:small-tree} shows an example of an input tree. Parties start the depth-first-search from the fixed root $v_1$. Afterwards, the search visits $v_2$, then $v_3$, $v_6$ returns to $v_3$, goes to $v_7$, returns to $v_3$, returns to $v_2$ and so on. The list we obtain at the end of the search is $L = [v_1, v_2, v_3, v_6, v_3, v_7, v_3, v_2, v_4, v_8, v_4, v_2, v_5, v_2, v_1]$.


Before describing the properties of the list obtained, we need to establish a few notations.
We use $L_i$ (with $1 \leq i \leq \abs{L}$) to denote the $i$-th element in list $L$, and, for a vertex $v \in \vertices(\tree)$, we use $L(v)$ to denote the set of indices $i$ such that $L_i = v$. Then, Lemma \ref{lemma:list-construction} formally establishes the guarantees of $\listconstruction$. The proof is provided in Appendix \ref{appendix:trees}.

\begin{restatable}{lemma}{ListConstruction} \label{lemma:list-construction}
    Consider a rooted tree $\tree$, and let $\roott$ denote its root.
    
    Then, $\listConstruction(\tree, \roott)$ returns within a finite amount of time a list of vertices $L$ with the following properties:
    \begin{enumerate}[nosep]
        \item If $\abs{\vertices(\tree)} > 1$, then, for any $i < \abs{L}$, $L_i$ and $L_{i+1}$ are adjacent vertices in $\tree$.
        \item The list $L$ contains $\abs{L} \leq 2 \cdot \abs{\vertices(T)}$ elements, and, for every vertex $v \in \vertices(\tree)$, $L(v) \neq \emptyset$. 
        \item Consider a vertex $v \in \vertices(\tree)$, and let $i_{\min} = \min L(v)$ and $i_{\max} = \max L(v)$. Then, a vertex $u$ is in the subtree rooted at $v$ if and only if $L(u) \subseteq [i_{\min}, i_{\max}]$.
        \item Consider two vertices $v, v' \in \vertices(\tree)$. Then, for any $i \in L(v)$ and $i' \in L(v')$, the lowest common ancestor of $v$ and $v'$ is in $\{L_k : \min(i, i') \leq k \leq \max(i, i')\}$.
    \end{enumerate}
\end{restatable}

Note that the $\listConstruction$ algorithm is deterministic, and therefore the honest parties obtain the same list $L$.

\paragraph{Finding vertices in the subtree of a valid vertex.}
The list $L$ enables us to use $\realAA$ to find vertices in a subtree rooted at a valid vertex: after computing the list $L$, 
each party $\party$ with input vertex $v_{\inputt}$ joins $\realAA(1)$ with input $i \in L(v_{\inputt})$.
This provides the parties with $1$-close real values $j$ within the honest range of values $i$, and each party outputs $L_{\closestInt(j)}$. 
We add that any two honest parties obtain $j, j'$ such that $\abs{\closestInt(j) - \closestInt(j')} \leq 1$, and $\closestInt(j)$ is also within the range of values $i$. 

%------------------------------------------
\begin{wrapfigure}{r}{5.3cm}
\centering
\includegraphics[width=5cm]{figures/small-tree3}
\caption{An input tree.}\label{figure:small-tree-extra}
\end{wrapfigure} 
%------------------------------------------
We note that this \textbf{does not} imply that $L_{\closestInt(j)}$ is a valid vertex.
For instance, recall the input tree example of  Figure \ref{figure:small-tree}, depicted again in Figure \ref{figure:small-tree-extra}.
If the honest inputs are $v_3$, $v_6$, and $v_5$, the honest inputs' convex hull is  $\{v_5, v_2, v_3, v_6\}$. The honest parties join $\realAA$ with indices $i$ in $L(v_3) = \{3, 5, 7\}$, $L(v_6) = \{4\}$, and $L(v_5) = \{13\}$. Note that the indices in $L(v_4) = \{9, 11\}$ and $L(v_8) = \{10\}$ are in the range of honest indices $i$. This means that the honest parties may obtain an output $j$ in $\realAA$ leading to $v_4$ or $v_8$: these are outside the honest input's convex hull $\{v_5, v_2, v_3, v_6\}$. On the other hand, $v_4$ and $v_8$ are in the subtree rooted at the valid vertex $v_2$. 



\paragraph{Paths intersecting the honest inputs' convex hull.} Note that, in the example above, since nodes $v_4$ and $v_8$ are in a subtree rooted at the valid node $v_2$, the paths $\pathh(v_1, v_4)$ and $\pathh(v_1, v_8)$ intersect the honest inputs' convex hull.
The lemma below shows that this is true in general: if an honest party obtains vertex $L_{\closestInt(j)}$ from our approach, then $\pathh(\roott, L_{\closestInt(j)})$ intersects the honest inputs' convex hull.
\begin{lemma} \label{lemma:path-crosses-convex-hull}
    Assume a rooted tree $T$ with root $\roott$ and let $L := \listconstruction(\tree, \roott)$.
    Consider non-empty set of vertices $S \subseteq \vertices(\tree)$, and let $i_{\min}$ and $i_{\max}$ denote the lowest and respectively highest indices in $L$ representing vertices in $V$: $i_{\min} := \min \bigcup_{v \in S} L(v)$ and $i_{\max} := \max \bigcup_{v \in S} L(v)$. 
    
    Then, for any $i$ satisfying $i_{\min} \leq i \leq i_{\max}$, $\pathh(\roott, L_i) \cap \hull{S} \neq \emptyset$.
\end{lemma}
\begin{proof}
    Assume that $\vertices(\pathh) \cap \hull{S} = \emptyset$.
    Let $\lca$ denote the lowest common ancestor of $L_{i_{\min}}$ and $L_{i_{\max}}$. Since both $L_{i_{\min}}$ and $L_{i_{\max}}$ belong to $S$, and since $\lca$ lies on the unique path connecting them, it follows that $\lca \in \hull{S}$. Given that $\vertices(\pathh) \cap \hull{S} = \emptyset$, we conclude that $\lca \notin \vertices(\pathh)$.  
    
    Since $\pathh$ is the path connecting the root vertex $\roott$ to $L_i$ and does not pass through $\lca$, it follows that $L_i$ is not in the subtree rooted at $\lca$. Otherwise, there would exist a path from $\roott$ to $L_i$ passing through $\lca$, contradicting the assumption that $\pathh$ does not intersect $\hull{S}$. Moreover, if $L_i$ were in the subtree of $\lca$, then $\tree$ would contain a cycle, which is impossible.  


    On the other hand, Lemma \ref{lemma:list-construction} implies that, since $L_i$ is not in the subtree rooted at $\lca$, $i \notin [\min L(\lca), \max L(\lca)]$. Assume without loss of generality that $i > \max L(\lca)$. However, $L_{i_{\max}}$ is in the subtree rooted at $\lca$, and Lemma \ref{lemma:list-construction} ensures that $i_{\max} \in[\min L(\lca), \max L(\lca)]$, hence $i_{\max} \leq \max L(\lca) < i$.  We have obtained a contradiction as $i \in [i_{\min}, i_{\max}]$.
\end{proof}

\paragraph{Subprotocol $\pathfinder$. } We may now present our subprotocol $\pathfinder$. The parties locally fix a root vertex $\roott$ of $\tree$ and compute the list representation $L$ of $\tree$ using the subprotocol $\listconstruction$. Afterwards, the parties run $\realAA$ on the list $L$ to obtain $1$-close vertices $v$ in $\tree$. Finally, each honest party returns the path $\pathh(\roott, v)$.
\begin{dianabox}{$\pathfinder(\tree, \roott, v_{\inputt})$}
	\algoHeadNoBold{Code for party $\party$, given the tree $\tree$ with root $\roott$ and input $v_{\inputt} \in \vertices(\tree)$}
	\begin{algorithmic}[1]
            \State $L := \listConstruction(\tree, \roott)$.
            \State Join $\realAA(1)$ with input $i := \min L(v_{\inputt})$. Obtain output $j \in \realvalues$.
            Return $\pathh := \pathh(\roott, L_{\closestInt(j)}).$ 
    \end{algorithmic}
\end{dianabox}

We establish the guarantees of $\pathfinder$ below. We recall that $R_{\realAA}$ denotes a sufficient number of rounds needed by $\realAA$ to achieve $\approximateagreement$ on real values, as described in \Cref{theorem:real-values-aa}.

\begin{lemma} \label{lemma:path-finder}
    Assume that the honest parties join $\pathfinder$ with the same tree $\tree$ and the same root vertex $\roott$. Then, $\pathfinder$ provides each honest party with a subpath $\pathh$ of $\tree$ such that:
    \begin{enumerate}[nosep]
        \item If $S$ denotes the set 
        of honest parties' inputs $v_{\inputt}$,  $\vertices(\pathh) \cap \hull{S} \neq \emptyset$.
        \item There is a subpath $\pathh^{\star} = (v_1, v_2, \ldots, v_{k^\star + 1})$ of $\tree$ such that, for every honest party, either $\pathh = (v_1, v_2, \ldots, v_{k^\star})$ or $\pathh = (v_1, v_2, \ldots, v_{k^\star + 1})$.
    \end{enumerate}
    Moreover, each honest party obtains its path within $R_{\pathfinder} := R_{\realAA}( 2\cdot\abs{\vertices(T)}, 1)$ communication rounds.
\end{lemma}
\begin{proof}
We first discuss Property 1.
Lemma \ref{lemma:list-construction} ensures that every vertex of $\tree$ appears in list $L$, and therefore every party obtains a well-defined index $i$.
Afterwards, Theorem \ref{theorem:real-values-aa} ensures that parties obtain $1$-close real values $j$ within the range of honest values $i$. According to Remark \ref{remark:validity-helper}, the values $\closestInt(j)$ are also within the range of honest values $i$,
Then, since the honest parties have computed the list $L$ identically, Lemma \ref{lemma:path-crosses-convex-hull} ensures that each honest party obtains a path $\pathh$ that intersects the honest inputs' convex hull, therefore Property 1 holds.


For Property 2, we note that, since the values $j$ obtained by the honest parties are $1$-close, Remark \ref{remark:agreement-helper} ensures that the values $\closestInt(j)$ are also $1$-close. Moreover, Lemma \ref{lemma:list-construction} ensures that consecutive elements in list $L$ are adjacent vertices in $\tree$. Then, since the parties have computed the list representation $L$ identically, the parties obtain vertices $v_{\closestInt(j)}$ that are $1$-close in $\tree$.
If the honest parties have obtained the same vertex $v_{\closestInt(j)}$, then all honest parties obtain $\pathh^{\star} = \pathh(\roott, v_\closestInt(j))$, and hence Property 2 holds. Otherwise, honest parties have obtained adjacent vertices $v, v'$. Assume without loss of generality that $v$ is the lowest common ancestor of $v$ and $v'$ in the tree $\tree$ with root vertex $\roott$. 

We may then define $\pathh^\star$ as the ``longer'' path obtained by the honest parties, i.e. $\pathh^\star := \pathh(\roott, v')$: parties that have obtained vertex $v'$ return path $\pathh^\star$, while parties that have obtained vertex $v$ return path $\pathh(\roott, v)$ such that $v' \notin \vertices(\pathh(\roott, v))$ (since $v$ is the lowest common ancestor of $v$ and $v'$). Consequently, Property 2 holds in this case as well.

Finally, the guarantee regarding the number of rounds follows from the fact that we run $\realAA$ with $\varepsilon := 1$ on inputs between $1$ and $\abs{L}$, and $\abs{L} \leq 2 \cdot \abs{\vertices(T)}$ according to Lemma \ref{lemma:list-construction}. 
\end{proof}

\section{Final protocol}
We may now describe our final protocol on trees $\treeAA$. The parties first run the subprotocol $\pathfinder$ described in Section \ref{section:path} to \emph{approximately} agree on paths that pass through the honest inputs' convex hull. 
Afterwards, each party $\party$ proceeds as described in Section \ref{section:path-assumed}: it labels the $k$ vertices on its own path $\pathh$ as $(v_1 := \roott, v_2, \ldots, v_k := v)$. It joins $\realAA(1)$ with input $i$, where $v_i := \projection_{\pathh}(v_{\inputt})$ and $v_{\inputt}$
is the input vertex. 
Afterwards, upon obtaining output $j$, it \emph{should} output the vertex labeled $v_{\closestInt(j)}$ in its path $\pathh$. This is where we need to be careful about honest parties holding different paths. 
%\zsnote{is j > k correct? can you please take a look at it?} 
If an honest party $\party$ obtains $j > k$, then $\party$ holds the ``shorter'' path, while other honest parties hold the ``longer'' path: $\party$ holds the path $(v_1, v_2, \ldots, v_{k^\star})$ described in Lemma~\ref{lemma:path-finder}. 
In this case, if $\closestInt(j) = k + 1$ and vertex $v_k$ has at least three neighbors (as depicted in Figure \ref{figure:two-paths}) $\party$ is unable to decide which neighbor is vertex $v_{k + 1}$, i.e. the last vertex of the ``longer path''. Instead, in this case, $\party$ will simply output $v_k = v_{k^\star}$. 

\begin{dianabox}{$\treeAA$}
	\algoHeadNoBold{Code for party $\party$ with input $v_{\inputt} \in \vertices(\tree)$}
	\begin{algorithmic}[1]
            \State $\roott := $the vertex in $\tree$ with the lowest label (in lexicographic order).
            \State  $\pathh := \pathfinder(\tree, \roott, v_{\inputt}).$ Label the $k = \abs{V(\pathh)}$ vertices in $\pathh$ as $(v_1 := \roott, v_2, \ldots, v_k)$.
            \State Wait until round $R_{\pathfinder}$ ends.
            \State Join $\realAA(1)$ with input $i$ such that $\projection_{\pathh}(v_{\inputt}) = v_i$. Obtain output $j$.
            \State If $\closestInt(j) > k$, output $v_{k} \in \vertices(\pathh).$ Otherwise, output $v_{\closestInt(j)} \in \vertices(\pathh)$.
    \end{algorithmic}
\end{dianabox}

Note that the parties wait until a specific number of rounds $R_{\pathfinder}$ before joining the execution of $\realAA$. This is because the protocol  $\realAA$ described in Theorem \ref{theorem:real-values-aa} and used in $\pathfinder$ achieves early termination. However, the honest parties are not guaranteed to obtain outputs in the same round. Afterwards, it is essential for the honest parties to start the second invocation simultaneously.

The theorem below describes the properties of $\treeAA$.
\begin{theorem}
    If $t < n/3$, protocol $\treeAA$ achieves $\approximateagreement$ on any input tree $\tree$ within $O \left(\frac{\log \abs{\vertices(\tree)}}{\log \log \abs{\vertices(\tree)}}\right)$ rounds of communication.
\end{theorem}

\begin{proof}
    According to \Cref{lemma:path-finder}, and as depicted in \Cref{figure:two-paths}, there are two adjacent vertices $v$ and $v'$ such that (i) $v$ is the lowest common ancestor of $v$ and $v'$ and (ii) honest parties either obtain path $\pathh(\roott, v)$ of $k^\star$ vertices or path $\pathh^\star = \pathh(\roott, v) \oplus (v, v')$ of $k^\star + 1$ vertices. We note that the remainder of the proof will use both notations $k$ and $k^\star$: $k$ will refer to the value $k$ held by a party (denoting the length of the party's path), while $k^{\star}$ to the length of $\pathh(\roott, v)$. Hence, some parties obtain $k = k^\star$, while others $k = k^\star + 1$. 
    
    Then, all honest parties assign the labels identically to the vertices of \(\pathh(\roott, v)\) as the following \((v_1 := \roott, v_2, \dots, v_{k^\star} := v)\). If honest parties hold \(\pathh^\star\), they additionally assign label \(v_{k^\star + 1} := v'\) identically.
    
    \begin{figure}[h]
    \centering
    \includegraphics[width=0.8\textwidth]{figures/new-path.png}
    \caption{Vertices $v_1, v_2, \ldots, v_6 = v$ form the path $\pathh(\roott, v)$.  Vertices $v_1, v_2, \ldots, v_7 = v'$ form the path $\pathh(\roott, v') = \pathh^\star$. Vertices $u_1, u_2, u_3$ represent the honest inputs,  hence vertices $v_3, v_5, v_5, v_6, v_7$ are in $\vertices{(\pathh^\star)} \cap \hull{S}$, where $\hull{S}$ denotes the honest inputs' convex hull.
    Note that, an honest party $\party$ holding the blue path $\pathh(v_1, \ldots v_6)$ might obtain $\closestInt(j) = 7$ in $\treeAA$. In this case, $\party$ does not know whether $v_7$ should be the actual vertex $v_7$ or the red vertex adjacent to $v_6$. The red vertex is, in fact, outside the honest inputs' convex hull. In this case, $\party$ decides on output $v_6$. We will show, that in this case, all parties output either $v_6$ or $v_7$.
    }\label{figure:two-paths}
    \end{figure}


    Each honest party joins $\realAA(1)$ with input $i$ such that $v_i = \projection_{\pathh}(v_{\inputt})$. Following \Cref{remark:convex-hull-maintained}, each of the vertices $v_i = \projection_{\pathh}(v_{\inputt})$ is in the honest inputs' convex hull (denoted by $\hull{S}$).

    Before discussing the guarantees $\realAA$ gives for the values $j$, we need to ensure that the honest parties are ready to start the execution of  $\realAA$ at the same time (this is an implicit assumption of synchronous protocols).  Since $\pathfinder$ provides each honest party with a path within $R_{\pathfinder}$ rounds, the honest parties join $\realAA$ at the same time. We then apply Theorem~\ref{theorem:real-values-aa} and conclude that the honest parties obtain $1$-close real values $j$ within the range of the honest values $i$. 
    
    Afterward, according to Remark \ref{remark:validity-helper} and Remark \ref{remark:agreement-helper}, the honest parties obtain values $\closestInt(j)$ that are also within the range of honest values $i$, and that are $1$-close. 
    This implies that every honest party obtains a valid vertex $v_{\closestInt(j)} \in \vertices(\pathh^\star)$. This is because there is some honest party that has joined $\realAA$ with $i \leq \closestInt(j)$, and there is another honest party that has joined $\realAA$ with $i' \geq \closestInt(j)$. Hence, these two honest parties have obtained $v_i \in \hull{S}$ and $v_{i'} \in \hull{S}$ as the projections of their input vertices onto path $P$. Therefore, $v_{\closestInt(j)} \in \hull{S}$ since it is  on the unique path connecting $v_i$ and $v_{i'}$. 

    Therefore we proved that all honest parties that output $v_{\closestInt(j)}$ indeed output vertices in $\hull{S}$. Before claiming that Validity holds, we still need to discuss the case where an honest party $\party$ obtains $\closestInt(j) > k$. Since $\closestInt(j)$ is in the range of honest values $i$, then there is an honest party $p'$ that has proposed value $i' > k$. Given the properties of the honest parties' paths, this means that party $p$ holds the shorter path $\pathh(\roott, v)$ while $p'$ holds the longer path $\pathh^\star$, and therefore $p'$ has obtained projection $v_{k + 1}$, i.e. vertex $v_{k^\star + 1}$ in path $\pathh^\star$. At the same time, $p$ has obtained projection $v_i$ with $i \leq k = k^\star$. Therefore, as vertex $v_k = v_{k^\star}$ is on the unique path between $v_i$ and $v_{k^\star + 1}$, and both $v_i$ and $v_{k^\star + 1}$ are in $\hull{S}$, then $v_k$ is also in $\hull{S}$. Hence, $\party$ outputs a valid vertex. Consequently, Validity holds.    


    We now discuss $1$-Agreement. Recall that the honest parties obtain $1$-close values $\closestInt(j)$. Vertices labeled $v_i, v_{i + 1}$ are adjacent, and therefore vertices labeled $v_{\closestInt(j)}$ are $1$-close. Hence, if every honest party obtains $\closestInt(j) \leq k$, $1$-Agreement is achieved. We still need to discuss the case where some honest party $\party$ obtains $\closestInt(j) > k$. We show that, in this case, all honest parties obtain values $j$ such that $k^\star \leq \closestInt(j) \leq k^\star + 1$.
    As described above, $\party$ holds the shorter path $\pathh(\roott, v)$ of $k = k^\star$ vertices, and there is some honest party $\party'$ that holds the longer path $\pathh(\roott, v')$ of $k^\star + 1 = k + 1$ vertices. Note that no honest party holds a path of $k^\star + 2$ vertices or more, therefore no honest party joins $\realAA$ with $i 
    > k^\star + 1$, so no honest party obtains $j > k^\star + 1$. Hence, all honest parties obtain $\closestInt(j) \leq k^\star + 1$. Then, as $\party$ has obtained $\closestInt(j) = k^\star + 1$, therefore $j \geq k^\star + \frac{1}{2}$. This implies that, since honest parties have obtained $1$-close values $j$, every honest party has obtained $j \geq (k^\star + \frac{1}{2}) - 1$ and $\closestInt(j) \geq k^\star$. Hence, as all honest parties hold either $k = k^\star$ or $k = k^\star + 1$, all honest parties output either $v_{k^\star}$ or $v_{k^\star + 1}$, which enables us to conclude that $1$-Agreement holds.

    For the round complexity, note that our protocol runs $\pathfinder$ afterwards runs $\Pi_{\realAA}(1)$ on inputs between $1$ and $k^\star + 1$, and $k^\star + 1 \leq \diameter(\tree)$. Hence, our protocol terminates within $R_{\pathfinder} + R_{\realAA}(\diameter(T), 1)$ rounds. The complexity in the theorem statement then follows from \Cref{lemma:path-finder} and \Cref{theorem:real-values-aa}.
\end{proof}


\paragraph{A note on the $t < n / 2$ case.} We have presented our protocol $\treeAA$ in the synchronous model, assuming $t < n / 3$: this resilience threshold is optimal in an \emph{unauthenticated setting} (no cryptographic setup) \cite{JACM:DLPSW86}. This setting was chosen solely for simplicity of presentation: our reduction, in fact, does not depend on the number of corrupted parties: whenever protocol $\realAA$ achieves $\approximateagreement$ on $[1, 2 \cdot \abs{\vertices(\tree)}]$, our protocol $\treeAA$ achieves $\approximateagreement$ on the input tree $T$, and the round complexity of $\treeAA$ is double that of $\realAA$. Hence, our reduction also enables asymptotically optimal round complexity for trees of diameter $\diameter(\tree) \in \Theta(\abs{\vertices(\tree)})$ in \emph{authenticated settings} (assuming digital signatures) up to $t < (1 - c)/2 \cdot n$ corruptions for any constant $c > 0$: we may simply replace $\realAA$ with the Proxcensus protocol of \cite{EUROCRYPT:GhGoLi22} with minor adjustments.