\subsection{Prompts for the Evolution agent}
\label{sec:prompts_4_evolution}


\begin{figure}[htbp!]
\begin{tcolorbox}[
    colback=black!5!white,
    colframe=black!60!white,
    title=\textbf{Prompt for hypothesis feasibility improvement},
    fonttitle=\bfseries,
    arc=3mm,
    boxrule=1pt,
    bottomrule=2pt,
]
\footnotesize
\begin{verbatim}
You are an expert in scientific research and technological feasibility analysis. 
Your task is to refine the provided conceptual idea, enhancing its practical implementability 
by leveraging contemporary technological capabilities. Ensure the revised concept retains 
its novelty, logical coherence, and specific articulation.

Goal: {goal}

Guidelines:
1. Begin with an introductory overview of the relevant scientific domain.
2. Provide a concise synopsis of recent pertinent research findings and related investigations, 
   highlighting successful methodologies and established precedents.
3. Articulate a reasoned argument for how current technological advancements can facilitate 
   the realization of the proposed concept.
4. CORE CONTRIBUTION: Develop a detailed, innovative, and technologically viable alternative 
   to achieve the objective, emphasizing simplicity and practicality.

Evaluation Criteria:
{preferences}

Original Conceptualization:
{hypothesis}

Response:
\end{verbatim}
\end{tcolorbox}
\vspace{0.1cm}
\caption{\textbf{Example Evolution agent prompt for hypothesis feasibility improvement.}}
\label{fig:FEASIBILITY_IDEA_PROMPT}
\end{figure}


\begin{figure}[htbp!]
\begin{tcolorbox}[
    colback=black!5!white,
    colframe=black!60!white,
    title=\textbf{Prompt for hypothesis generation through out-of-the-box thinking},
    fonttitle=\bfseries,
    arc=3mm,
    boxrule=1pt,
    bottomrule=2pt,
]
\footnotesize
\begin{verbatim}
You are an expert researcher tasked with generating a novel, singular hypothesis 
inspired by analogous elements from provided concepts.

Goal: {goal}

Instructions:
1. Provide a concise introduction to the relevant scientific domain.
2. Summarize recent findings and pertinent research, highlighting successful approaches.
3. Identify promising avenues for exploration that may yield innovative hypotheses.
4. CORE HYPOTHESIS: Develop a detailed, original, and specific single hypothesis 
   for achieving the stated goal, leveraging analogous principles from the provided 
   ideas. This should not be a mere aggregation of existing methods or entities. Think out-of-the-box.

Criteria for a robust hypothesis:
{preferences}

Inspiration may be drawn from the following concepts (utilize analogy and inspiration, 
not direct replication):
{hypotheses}

Response:
\end{verbatim}
\end{tcolorbox}
\vspace{0.1cm}
\caption{\textbf{Example Evolution agent prompt for hypothesis generation through out-of-the-box thinking.}}
\label{fig:OUT_OF_THE_BOX_SINGLE_IDEA_PROMPT}
\end{figure}
