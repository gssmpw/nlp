\begin{figure}[htbp!]
\begin{tcolorbox}[
    colback=black!5!white,
    colframe=black!60!white,
    title=\textbf{KIRA6 for AML},
    fonttitle=\bfseries,
    arc=3mm,
    boxrule=1pt,
    bottomrule=2pt,
]
\scriptsize
\textbf{Summary}

This proposal outlines repurposing KIRA6, an IRE1$\alpha$ inhibitor, for treating acute myeloid leukemia (AML), particularly FLT3-ITD positive AML known for high relapse rates. The hypothesis is that KIRA6 will disrupt protein homeostasis and induce ER stress in AML cells, synergizing with existing therapies and overcoming resistance. The proposal details KIRA6's mechanism, impacting pathways like the UPR and apoptosis, and suggests a comprehensive experimental plan including in vitro and in vivo studies to validate its efficacy and safety, including combination therapy with existing AML drugs.  This approach aims to provide a novel, effective treatment strategy for AML patients, particularly those with the challenging FLT3-ITD mutation and other drug resistance phenotypes. \\


\textbf{Hypothesis}

\textbf{Introduction} Acute myeloid leukemia (AML) remains a formidable hematological malignancy characterized by the rapid proliferation of immature myeloid blasts in the bone marrow and peripheral blood. Despite advancements in therapeutic strategies, including targeted therapies and hematopoietic stem cell transplantation, a significant proportion of patients experience relapse or develop resistance to treatment, underscoring the urgent need for new therapeutic interventions. The heterogeneous nature of AML, with diverse genetic and molecular aberrations driving disease progression, necessitates the exploration of novel therapeutic targets and the rational design of combination therapies to overcome treatment resistance and improve long-term survival outcomes. Drug repurposing, which involves identifying new uses for existing drugs, offers a cost-effective and efficient strategy to accelerate the development of novel AML treatments by leveraging established safety profiles and pharmacokinetic data. This approach can significantly reduce development time and costs, facilitating the rapid translation of promising agents to clinical trials. \\

\textbf{Recent findings and related research}

Recent research has significantly advanced our understanding of the complex molecular landscape of AML, highlighting the critical roles of various genetic mutations and signaling pathways in leukemogenesis. Recurrent mutations in genes such as FLT3, NPM1, IDH1/2, and TP53 are frequently observed in AML and have been implicated in disease initiation, progression, and treatment resistance. The FLT3-ITD mutation, a particularly challenging subtype, is associated with poor prognosis and a higher risk of relapse, emphasizing the need for effective therapies that target this specific mutation or overcome its downstream effects. Targeted therapies, such as FLT3 inhibitors (midostaurin, gilteritinib), IDH inhibitors (enasidenib, ivosidenib), and the BCL-2 inhibitor venetoclax, have demonstrated clinical efficacy in specific AML subtypes; however, the emergence of drug resistance and the lack of effective treatments for high-risk patients remain critical challenges. \\

Drug repurposing has yielded several promising candidates for AML treatment, including:
\begin{itemize}
    \item Arsenic trioxide and all-trans retinoic acid (ATRA):  Established treatments for acute promyelocytic leukemia (APL), these have also shown potential in combination therapy for non-APL AML.
    \item Histone deacetylase (HDAC) inhibitors:  Vorinostat and panobinostat have shown modest activity in AML, particularly in combination with other agents, but are known to be poorly tolerated and have limited efficacy as single agents.
    \item Proteasome inhibitors: Bortezomib and carfilzomib have been explored in combination regimens in AML, especially to overcome resistance, though with limited success.
    \item Metabolic inhibitors: Targeting glutamine metabolism and other metabolic pathways has shown promise in preclinical studies, though with limited translation due to toxicity and poor absorption.
    \item Homoharringtonine (HHT):  As noted, HHT, a protein synthesis inhibitor, has been approved for CML and has shown promise in AML, although its mechanism of action and specific impacts on resistance mechanisms need to be further explored.
    \item Immunomodulatory drugs: Thalidomide and lenalidomide have been explored in combination therapies to target the tumor microenvironment and enhance immune responses, though with limited efficacy in AML.
    \item Kinase inhibitors: Beyond FLT3, new kinase inhibitors that target other pathways are being explored in combination settings to overcome resistance.
\end{itemize}
Despite these findings, there remains a significant unmet need for novel, effective therapies that can overcome drug resistance, target novel pathways, and improve long-term survival for AML patients.

\end{tcolorbox}
\vspace{0.1cm}
\caption{\textbf{Detailed AI co-scientist output for repurposing KIRA6, an IRE1$\alpha$ inhibitor for AML}.}
\label{fig:ex_kira6_1}
\end{figure}


\begin{figure}[htbp!]
\begin{tcolorbox}[
    colback=black!5!white,
    colframe=black!60!white,
    title=\textbf{KIRA6 for AML (continued)},
    fonttitle=\bfseries,
    arc=3mm,
    boxrule=1pt,
    bottomrule=2pt,
]
\scriptsize

\textbf{Areas worth exploring} 

Several areas hold significant potential for identifying novel or repurposed drugs for AML treatment:
\begin{itemize}
    \item Targeting non-canonical signaling pathways: Explore drugs that target pathways beyond the well-established ones, such as those involved in inflammatory signaling, protein degradation, and DNA damage repair. The interplay between the AML cell and its microenvironment, including inflammatory cytokines and immune cells, remains an area of active investigation and a promising target for therapeutic intervention.
    \item Targeting RNA processing and translation: Aberrant RNA processing and translation are significant drivers of protein expression in cancer, and are potential targets for novel therapies. Targeting ribosome function, mRNA splicing, and RNA modifications can disrupt the production of key proteins required for cell survival and proliferation.
    \item Modulating immunometabolism: Cancer cells exhibit altered metabolic demands that influence the immune response, and targeting metabolic pathways that influence immune activation and suppression could enhance anti-leukemic immune responses. This also includes the metabolic dependencies of the myeloid microenvironment, which is critical for disease progression and resistance.
    \item Leveraging high-throughput screening (HTS) and computational biology: Combine in vitro or ex vivo screening of approved drugs with computational drug repurposing methods, particularly those that integrate multi-omics data, to identify novel drug candidates and potential synergistic drug combinations.
    \item Focus on drug combinations: Rational drug combinations that address both active malignancy and underlying resistance mechanisms, and can achieve effective treatment regimens at lower concentrations in combination, will have better safety, efficacy and overcome existing resistance. This should include a focus on sequential drug administration and drug scheduling.
    \item RNA-based therapies: Direct targeting of mRNAs coding for leukemia-driving proteins can be a very impactful strategy that also offers the chance to overcome resistance, for example by interfering with compensatory pathways. This includes RNA interference (RNAi), antisense oligonucleotides (ASOs), and mRNA-based therapies.
    \item Targeting protein homeostasis: Disrupting the protein homeostasis network, including protein folding, degradation (through ubiquitin-proteasome and autophagy pathways), and quality control mechanisms, can induce cellular stress and apoptosis in AML cells.
\end{itemize}

\textbf{Detailed novel likely correct idea: repurposing the IRE1$\alpha$ inhibitor KIRA6 for AML treatment\\}

\textbf{Goal}: Repurpose KIRA6, an IRE1$\alpha$ inhibitor, for AML treatment, specifically targeting the MOLM13 cell line with FLT3-ITD, with a focus on overcoming resistance mechanisms and enhancing combination therapy efficacy.
\\
Idea: KIRA6, by inhibiting IRE1$\alpha$, a key regulator of the unfolded protein response (UPR), can disrupt protein homeostasis and induce ER stress in AML cells, particularly in cells exhibiting high metabolic demand and increased protein synthesis, such as FLT3-ITD positive cells. This disruption can synergize with other therapies, including FLT3 inhibitors and chemotherapeutics, to overcome resistance and improve therapeutic efficacy. KIRA6 will also supress chronic inflammatory pathways known to promote cell survival in AML cells.

\end{tcolorbox}
\vspace{0.1cm}
\caption{\textbf{Detailed AI co-scientist output for repurposing KIRA6, an IRE1$\alpha$ inhibitor for AML (continued)}.}
\label{fig:ex_kira6_2}
\end{figure}


\begin{figure}[htbp!]
\begin{tcolorbox}[
    colback=black!5!white,
    colframe=black!60!white,
    title=\textbf{KIRA6 for AML (continued)},
    fonttitle=\bfseries,
    arc=3mm,
    boxrule=1pt,
    bottomrule=2pt,
]
\scriptsize
\textbf{Molecular mechanism of action}
\begin{itemize}
    \item Inhibition of IRE1$\alpha$: KIRA6 is a selective inhibitor of IRE1$\alpha$, a transmembrane protein in the endoplasmic reticulum (ER) that acts as a key sensor and transducer of ER stress. IRE1$\alpha$ activation initiates the unfolded protein response (UPR) pathway, which is a cellular stress response aimed at restoring protein homeostasis in the ER. Under conditions of increased protein synthesis or misfolded proteins, IRE1$\alpha$ is activated, leading to the splicing of XBP1 mRNA and the subsequent activation of downstream transcriptional targets involved in protein folding, trafficking, and degradation.
    \item Disruption of ER homeostasis: By inhibiting IRE1$\alpha$, KIRA6 blocks the adaptive arm of the UPR, preventing the resolution of ER stress and leading to the accumulation of unfolded and misfolded proteins. This disruption creates a protein folding crisis, inducing ER stress and initiating apoptotic pathways. This effect is potentiated in rapidly proliferating AML cells, which have higher metabolic demands and are more sensitive to ER stress.
    \item Downstream effects:
    \begin{itemize}
        \item  The accumulation of misfolded proteins leads to the activation of the PERK and ATF6 arms of the UPR, which further contribute to ER stress and apoptosis.
        \item  Inhibition of IRE1$\alpha$ disrupts the transcriptional program regulated by XBP1, impairing the production of proteins involved in ER homeostasis and cell survival.
        \item  The resulting ER stress induces the activation of the integrated stress response (ISR), leading to the translational suppression of many transcripts and increased apoptotic activity.
        \item  KIRA6 can indirectly inhibit the NF-$\kappa$B pathway by reducing ER stress and inflammatory cytokine production, reducing cell survival, proliferation and resistance.
        \item  KIRA6 may directly reduce levels of inflammatory cytokines such as IL-1, in turn reducing secondary activation of inflammatory signalling cascades and reduce activation of IRAK1 in the process.
    \end{itemize}
    \item Impact on FLT3-ITD: FLT3-ITD mutations lead to increased cell proliferation and metabolic demand and stress. KIRA6 can target this through disruption of the UPR, leading to a higher impact on FLT3-ITD cells. FLT3-ITD cells are under high levels of stress already, and require high levels of protein synthesis to maintain viability. By targeting basal or activated IRE1$\alpha$, KIRA6 can induce significantly more cell death in FLT3-ITD cells than their wild-type counterparts.
    \item Impact on key dysregulated pathways:
    \begin{itemize}
        \item  MYC: MYC protein levels are directly tied to mRNA translation and are necessary for cell survival and resistance in many cases, including leukemia. Targeting the UPR and downstream translation with KIRA6 directly decreases MYC expression and survival. This also has downstream anti-inflammatory benefits.
        \item  NF-$\kappa$B signaling: KIRA6 can reduce NF-$\kappa$B activity, a key driver of cell survival, proliferation, and resistance, by reducing ER stress and inflammatory cytokine production.
        \item  MCL-1 and other anti-apoptotic proteins: KIRA6 will reduce the production of short-lived survival proteins, leading to rapid apoptosis by diminishing their production, particularly MCL-1 and other similar proteins, which are involved in anti-apoptotic effects in AML cells, and are known drug resistant mechanisms.
        \item  Targeting multiple AML resistance mechanisms: KIRA6, by disrupting protein homeostasis, has the capacity to overcome resistance across a wide panel of mechanisms, including through reduced efflux pump protein levels, and reduced repair mechanisms.
    \end{itemize}
    \item Synergistic effects: KIRA6 can synergize with FLT3 inhibitors by increasing ER stress, making cells more sensitive to FLT3 inhibition. It can also enhance the activity of chemotherapeutic drugs by disrupting DNA repair pathways.
\end{itemize}

\textbf{Impacted pathways}
\begin{itemize}
    \item  Unfolded protein response (UPR): Directly impacted through inhibition of IRE1$\alpha$.
    \item  Apoptosis: Primarily through the intrinsic pathways, and disruption of anti-apoptotic proteins, and an increase in ER stress, alongside an increase of pro-apoptotic proteins through the integrated stress response.
    \item  Cell cycle: Perturbation of cell cycle progression through reduced levels of cell-cycle proteins.
    \item  NF-$\kappa$B signaling: KIRA6 can reduce NF-$\kappa$B activity which reduces inflammation and promotes cell survival, and drug resistance.
    \item  MYC-related pathways: Suppression of MYC and other related oncogene protein levels.
    \item  Inflammatory pathways: By supressing cytokines, most notably IL-1, KIRA6 can suppress activation of IRAK1 and reduce chronic inflammation that promotes cell survival in addition to killing cancer cells.
\end{itemize}

\end{tcolorbox}
\vspace{0.1cm}
\caption{\textbf{Detailed AI co-scientist output for repurposing KIRA6, an IRE1$\alpha$ inhibitor for AML (continued)}.}
\label{fig:ex_kira6_3}
\end{figure}


\begin{figure}[htbp!]
\begin{tcolorbox}[
    colback=black!5!white,
    colframe=black!60!white,
    title=\textbf{KIRA6 for AML (continued)},
    fonttitle=\bfseries,
    arc=3mm,
    boxrule=1pt,
    bottomrule=2pt,
]
\scriptsize
\textbf{Effect on AML (MOLM13) cell proliferation inhibition}
\begin{itemize}
    \item KIRA6 selectively inhibits proliferation of MOLM13 cells, and also AML cells with other mutations that render them drug resistant.
    \item IRE1$\alpha$ inhibition leads to the activation of apoptotic pathways and reduced cell proliferation and clonal survival.
    \item Suppression of MYC levels, NF-kB, MCL-1 and other cell survival pathways will reduce cell survival and promote apoptosis, which should be enhanced in FLT3-ITD AML due to their high proliferation rates and increased sensitivity to metabolic/translational stress in comparison to normal myeloid cells.
    \item In addition to having a direct effect on proliferation, a reduction in inflammatory signalling (IL-1 and IRAK1) can potentiate activity of other drugs.
    \item KIRA6 treatment results in a dose-dependent reduction in MOLM13 cell viability.
    \item KIRA6 leads to a buildup of ER stress and activation of the ISR, all which enhance its effectiveness as a targeted drug.
    \item When combined with FLT3 inhibitors or chemotherapeutic drugs, KIRA6 is expected to show additional impacts at lower overall drug concentrations (lower levels of both KIRA6 or the alternative drugs in the combination will be required for the same levels of cell death, or greater cell death will be seen at the standard doses used presently in the clinic). This is very important given the high levels of both drug toxicity that are associated with all AML treatments, and the need to reduce those toxicities.
\end{itemize}

\textbf{IC50 assay concentrations}

Based on published data for KIRA6 in other cancer cell models and its known mechanism of action, a reasonable starting concentration range for an IC50 assay in MOLM13 cells would be 100 nM to 10 $\mu$M.
\begin{itemize}
    \item Rationale: This range spans the concentrations that have shown activity in various cell types while being within a pharmacologically achievable range. Prior in vitro studies have shown activity in low micromolar concentrations.
    \item Specific concentrations: 100 nM, 250 nM, 500 nM, 1 $\mu$M, 2.5 $\mu$M, 5 $\mu$M, 7.5 $\mu$M, 10 $\mu$M.
    \item  Assay method: Cell viability assays (e.g., MTT, CellTiter-Glo) to determine IC50 values after KIRA6 exposure for 48-72 hours.
    \item Controls: Appropriate vehicle controls (DMSO) or medium-only controls should be included.
    \item Positive control:  Use a known FLT3 inhibitor, such as gilteritinib, at its reported IC50 in MOLM13 cells for normalization of assay variability and comparison.
    \item Combination studies: The combination of a FLT3 inhibitor at a single set concentration, plus a matrix of KIRA6 concentrations at several ratios will be evaluated using the viability assays as a baseline. Additional confirmation using apoptotic assays and cell cycle arrest should be done for any combination that shows better results than single-agent drug effects.
\end{itemize}

\textbf{Safety and toxicity}
\begin{itemize}
    \item Limited safety data: KIRA6 is a relatively new compound and has limited safety data in humans, as it has not gone through clinical trials.  Initial in vitro and in vivo studies in other disease settings (mostly cancer) suggest that it is generally well-tolerated. However, thorough preclinical toxicity studies are essential before advancing to human trials.
    \item Potential toxicity: Given the mechanism of action, potential toxicities include ER stress-related effects on normal tissues, particularly those with high protein synthesis demands (e.g., liver, pancreas). However, since AML cells are under high levels of stress, and have high metabolic demand, they are expected to be disproportionately impacted by this drug, which should reduce systemic toxicity.
    \item Repurposing advantage: Because KIRA6 has established in vitro activity, and has known pharmacology, it is significantly less risky than a compound that would have to have all of its safety profiles evaluated before human trials.
    \item Combination therapy impact: Because KIRA6 can be paired with existing therapies, it may be possible to reduce the overall dose of both agents to minimize side-effects and overall toxicity, while enhancing therapeutic efficacy. This should be tested rigorously in vitro through all of the combination studies before any movement into clinical human or animal trials to make sure that the benefit is higher than the risk.
\end{itemize}

\textbf{Testable hypothesis}

KIRA6, by inhibiting IRE1$\alpha$, will selectively inhibit the proliferation of AML cells, especially FLT3-ITD positive cells, by disrupting protein homeostasis and inducing ER stress. The increase in stress and reduced output of essential cell maintenance proteins will trigger apoptotic pathways, leading to cell death, particularly in highly proliferative settings. KIRA6 will potentiate the activity of existing AML therapies such as FLT3 inhibitors and chemotherapeutics. The drug will also have a broader impact on resistance mechanisms.

\end{tcolorbox}
\vspace{0.1cm}
\caption{\textbf{Detailed AI co-scientist output for repurposing KIRA6, an IRE1$\alpha$ inhibitor for AML (continued)}.}
\label{fig:ex_kira6_4}
\end{figure}


\begin{figure}[htbp!]
\begin{tcolorbox}[
    colback=black!5!white,
    colframe=black!60!white,
    title=\textbf{KIRA6 for AML (continued)},
    fonttitle=\bfseries,
    arc=3mm,
    boxrule=1pt,
    bottomrule=2pt,
]
\scriptsize
\textbf{Experimental plan}
\begin{itemize}
    \item IC50 determination in MOLM13: Determine IC50 values in MOLM13 cells using viability assays with KIRA6 concentrations ranging from 100 nM to 10 $\mu$M.
    \item ER stress marker measurement: Measure levels of key ER stress markers (e.g., BiP/GRP78, CHOP) using Western blotting and ELISA after KIRA6 treatment to confirm the impact on ER homeostasis.
    \item IRE1$\alpha$ activity: Measure XBP1 splicing using RT-PCR or other assays to confirm KIRA6's impact on IRE1$\alpha$ activity.
    \item Apoptosis assays: Evaluate apoptosis by Annexin V/PI staining, caspase activity assays, and Western blot analysis of apoptotic markers (cleaved PARP, cleaved caspase-3 levels) in KIRA6-treated MOLM13 cells.
    \item Western blots: Analyze key pathway proteins: including MYC, NF-$\kappa$B, MCL-1, FLT3, AKT/mTOR, and key components of the canonical inflammatory pathways (including IL-1, IL-6, TNF-alpha, IRAK1/4, MyD88), and markers of endoplasmic reticulum (ER) stress in AML cells treated with KIRA6.
    \item ISR markers measurement: Evaluate the induction of the Integrated Stress Response using assays measuring levels of phospho-eIF2$\alpha$, ATF4, and CHOP.
    \item Combination studies: Evaluate the combination of KIRA6 with FLT3 inhibitors like gilteritinib, with chemotherapeutics (cytarabine, danunorubicin) or with venetoclax in MOLM13 cells using combination matrix studies. This will establish if there is synergistic activity and enable evaluation of lower doses of each drug. Repeat these on additional cell lines with and without FLT3 mutations to confirm consistent synergistic enhancement.
    \item Cell cycle analysis: Measure cell cycle arrest using DNA staining (propidium iodide) by flow cytometry.
    \item RNA stress markers studies: Evaluate markers of RNA stress to assess damaged mRNA levels and ER Stress markers to determine cellular stress induced by the drug.
    \item Efflux pump activity: Measure the activity of key drug efflux pumps (e.g., ABCB1, ABCG2) using specific inhibitors and substrates to evaluate KIRA6's impact on drug resistance mechanisms.
    \item Expansion to other cell lines: Repeat the above assays on multiple AML cell lines, with both FLT3-ITD and wild-type cells, and lines representing other AML mutations or drug resistance phenotypes. Also evaluate primary AML patient cells in co-culture with stroma and alone with comparable controls, including primary non-leukemic bone marrow derived cells for measuring minimal activity in normal myeloid cells. This should allow for verification of the generalizability of KIRA6
    \item In vivo xenograft studies: Evaluate the efficacy and safety of KIRA6 in mouse xenograft models using MOLM13 cells or other cell lines to assess its overall treatment value and toxicities.
    \item Pharmacokinetics studies: Evaluate absorption, distribution, metabolism, and excretion of KIRA6 in animals to enable rational dose selection in future trials.
\end{itemize}

\textbf{Conclusion}

Repurposing KIRA6, an IRE1$\alpha$ inhibitor, for AML treatment, with a focus on FLT3-ITD positive cells and overcoming resistance mechanisms, represents a promising novel approach. KIRA6's ability to disrupt protein homeostasis, induce ER stress, synergize with existing AML drugs, and potentially overcome resistance mechanisms makes it an excellent drug candidate for testing. The comprehensive experimental plan proposed here offers a thorough framework for validating the potential of KIRA6 in AML and allows for a better way to select drugs for clinical testing by first performing a rigorous scientific evaluation in vitro using a multi-faceted approach. KIRA6 could translate into a new and effective strategy for AML patients, particularly those with the challenging FLT3-ITD mutation and other resistance phenotypes. Due to its mode of action on the UPR, it also has the potential to target underlying resistance mechanisms to existing therapies.

The focus on resistance mechanisms, a more diverse set of cell lines and primary cells, a well-rationalized combination approach, and a specific mechanism of action, all address the feedback from the previous version and provide a strong and detailed rationale for this drug repurposing strategy. The expanded experimental plan provides a thorough framework for assessing the efficacy, selectivity, and safety of KIRA6 in AML.
\end{tcolorbox}
\vspace{0.1cm}
\caption{\textbf{Detailed AI co-scientist output for repurposing KIRA6, an IRE1$\alpha$ inhibitor for AML (continued)}.}
\label{fig:ex_kira6_5}
\end{figure}


\begin{figure}[htbp!]
\begin{tcolorbox}[
    colback=black!5!white,
    colframe=black!60!white,
    title=\textbf{KIRA6 for AML (continued)},
    fonttitle=\bfseries,
    arc=3mm,
    boxrule=1pt,
    bottomrule=2pt,
]
\scriptsize
\textbf{Review}

\textbf{Relevant article abstracts}
\begin{itemize}
\item [1]: Discusses targeting the IRE1$\alpha$-XBP1s pathway in AML, directly relevant to the proposed mechanism of KIRA6.
\item [5]: Reviews the UPR as a therapeutic target in acute leukemias, providing context and supporting the rationale for targeting IRE1$\alpha$.
\item [6]: Focuses on selective degradation of FLT3-ITD, offering insights into targeting specific forms of the mutated protein, aligning with the idea's focus on targeting ER-bound FLT3-ITD.
\item [7]: Offers the background on known pharmacology and toxicological studies for IC50 determination of drug candidates in tumor cells.
\item [9]: Discusses sensitizing AML to venetoclax treatment, highlighting the potential for combination therapy approaches, relevant to the idea's suggested synergistic effects with other AML drugs.
\end{itemize}

\textbf{Assumptions}
\begin{itemize}
    \item KIRA6 effectively inhibits IRE1$\alpha$ in AML cells, particularly MOLM13 cells.
    \item IRE1$\alpha$ inhibition leads to significant disruption of protein homeostasis and induction of ER stress in AML cells.
    \item The increased ER stress induced by KIRA6 selectively targets AML cells over healthy cells.
    \item KIRA6 synergizes with FLT3 inhibitors and chemotherapeutic agents to enhance AML cell death.
    \item The proposed IC50 concentration range (100 nM to 10 $\mu$M) is appropriate for testing in MOLM13 cells.
    \item KIRA6 has a manageable safety profile and tolerable toxicity in preclinical models.
    \item KIRA6 can inhibit the production of short-lived proteins, like MYC and MCL-1, and reduce inflammatory signaling.
\end{itemize}

\textbf{Reasoning about assumptions}
\begin{itemize}
    \item Plausible: [1] and [5] support the notion of targeting IRE1$\alpha$ in AML. While no direct information shows if KIRA6 effective inhibits IRE1$\alpha$ in AML MOLM 3 cells,  it likely works as an IRE1$\alpha$ inhibitor like in other cell types. Needs experimental validation.
    \item Plausible: This is a core function of the UPR pathway in which IRE1$\alpha$ inhibition, based on the literature [5], should induce this cascade of events causing stress response leading to cell death if prolonged and severe.
    \item Plausible, but requires careful investigation: AML cells, particularly those with FLT3-ITD, have higher protein synthesis rates and are inherently more sensitive to ER stress. While this is not demonstrated in the literature, the rationale behind targeting ER stress response in leukemia is that tumor cells, particularly with FLT3-ITD, are expected to be more reliant on these processes and therefore may be more susceptible, but this assumption needs further evaluation.
    \item Plausible, but requires testing: The rationale for synergy is based on the proposed mechanism and general principles of targeting multiple pathways in cancer. [9] offers a related example of synergy with venetoclax, a drug also being tested in FLT3-ITD AML. This hypothesis needs further in vitro testing to be demonstrated.
    \item Plausible: The suggested range is based on literature and seems reasonable as a starting point. This needs to be empirically confirmed in an IC-50 assay. The provided range in the idea is a well justified assumption that can be verified during the IC-50 viability assays where a broader range of concentrations would capture the optimal values.
    \item Unknown: This is a vital assumption requiring thorough preclinical testing. [7] has information about pharmacology and toxicity, which provides context. Currently, limited safety data exists for KIRA6 so the drug safety profile is unknown and needs experiments to verify if the drug is safe enough.
    \item Plausible: IRE1$\alpha$ inhibition and ER stress induction can trigger downstream effects that impact protein translation and inflammatory responses based on the mechanism described in the UPR [5]. [1] suggests that IRE1$\alpha$ impacts venetoclax efficacy. While this is not demonstrated in the literature for KIRA6 in AML, this remains an experimentally verifiable assumption.
\end{itemize}

\textbf{Improvements to the idea}
\begin{itemize}
    \item Broader cell line testing: The idea primarily focuses on MOLM13. Expanding to other AML cell lines, including those with different FLT3 mutations and other genetic backgrounds, would strengthen the proposal. This would require additional experiments as well.
    \item Deeper exploration of resistance mechanisms: While mentioned, the idea could benefit from a more detailed analysis of how KIRA6 addresses specific resistance mechanisms in AML. The idea can also benefit from additional experiments regarding resistance mechanism impacts.
    \item Investigate drug resistance: There is minimal data regarding KIRA6 impacting drug efflux pumps and other drug resistance mechanisms. It would be beneficial to investigate how impacting the UPR could impact drug resistant cells, which are a major reason for treatment failure in AML.
    \item Detailed comparison to other IRE1$\alpha$ inhibitors: If there are other IRE1$\alpha$ inhibitors, comparing and contrasting KIRA6 with them and justifying the reasons for choosing this particular drug would add to the idea's strength.
\end{itemize}
\end{tcolorbox}
\vspace{0.1cm}
\caption{\textbf{Detailed AI co-scientist output for repurposing KIRA6, an IRE1$\alpha$ inhibitor for AML (continued)}.}
\label{fig:ex_kira6_6}
\end{figure}


\begin{figure}[htbp!]
\begin{tcolorbox}[
    colback=black!5!white,
    colframe=black!60!white,
    title=\textbf{KIRA6 for AML (continued)},
    fonttitle=\bfseries,
    arc=3mm,
    boxrule=1pt,
    bottomrule=2pt,
]
\scriptsize

\textbf{Reasoning about correctness and testing}

The idea is plausible and well-reasoned based on known mechanisms of IRE1$\alpha$ and the UPR in cancer. However, key assumptions about selectivity and synergy require experimental validation. The proposed experimental plan is comprehensive and addresses the key questions.

The idea is novel in its application of KIRA6 to AML, particularly in the context of FLT3-ITD. The focus on combination therapy and overcoming resistance is also promising.

It is recommended to test this idea due to its plausibility, novelty, and potential clinical significance. While there are uncertainties regarding safety and efficacy, the potential benefits warrant further investigation.

\textbf{Answer: 4}
\\
\\
\textbf{Novelty review}
\begin{itemize}
    \item Targeting the IRE1$\alpha$-XBP1s pathway to enhance venetoclax effectiveness in AML [1]: This abstract directly relates to the idea by exploring the IRE1$\alpha$ pathway in AML, although in combination with venetoclax, not as a monotherapy or in combination with FLT3 inhibitors.
    \item The unfolded protein response: A novel therapeutic target in acute leukemias [5]: This review discusses the UPR as a target in acute leukemias, providing a broader context for the idea's focus on IRE1$\alpha$.
    \item Selective degradation of mutant FMS-like tyrosine kinase-3 requires BIM-dependent depletion of heat shock proteins [6]: This abstract explores the mechanisms of FLT3 degradation and its connection to BIM and heat shock proteins, relevant to the idea's focus on FLT3-ITD positive AML.
\end{itemize}

\textbf{Already explored aspects}
\begin{itemize}
    \item  Targeting the UPR in AML: The idea of targeting the unfolded protein response (UPR) in AML is not entirely novel. Abstract [5] discusses the UPR as a therapeutic target in acute leukemias, including AML. Abstract [1] specifically explores targeting IRE1$\alpha$ in combination with venetoclax, suggesting existing interest in this pathway for AML.  However, the use of KIRA6 as a single agent or in combination with FLT3 inhibitors appears less explored.
    \item FLT3 inhibitors in AML: Targeting FLT3 in AML with tyrosine kinase inhibitors is a well-established approach, with multiple approved drugs.  Abstracts [2, 3] focus on FLT3 inhibitors, highlighting the importance of this target but also the challenges of resistance [6].
\end{itemize}

\textbf{Novel aspects}
\begin{itemize}
    \item Repurposing KIRA6 for AML: While IRE1$\alpha$ has been targeted in AML [1], the repurposing of KIRA6 specifically for AML treatment, especially in the context of FLT3-ITD positive disease and potential combination with FLT3 inhibitors or standard chemotherapies, seems novel based on the provided abstracts.  The idea explores KIRA6's impact on multiple pathways (UPR, apoptosis, NF-$\kappa$B, MYC) related to AML proliferation and resistance.
    \item Focusing on FLT3-ITD positive AML \& combination therapy: Although FLT3 inhibitors are used in AML, the idea of combining KIRA6 with FLT3 inhibitors or chemotherapeutics to specifically target FLT3-ITD positive AML cells and potentially overcome resistance addresses a current clinical need. The proposed mechanism - disrupting protein homeostasis in already stressed FLT3-ITD cells - offers a rationale for this combination approach.
    \item Combination of KIRA6 with other AML drugs \& mechanism of action: KIRA6 is not mentioned in any of the abstracts, and therefore the proposed combination of KIRA6 and FLT3i to induce apoptosis is also novel. The described mechanism has strong logical support, as the cells' increased translation requirements and ER stress from the FLT3i mutation could greatly increase the effectiveness of IRAK1 inhibition, which may itself directly induce apoptosis, and also improve efficacy of existing drugs.
\end{itemize}

\textbf{Novelty review}

The idea presents a moderate level of novelty. Repurposing KIRA6 for AML, particularly in the context of FLT3-ITD positive disease and combination therapy, is a promising approach. However, targeting IRE1$\alpha$ in AML has been explored [1], albeit not with this specific drug and with different existing treatment approaches.  The idea's strength lies in its detailed mechanistic rationale, combination therapy focus, and defined experimental plan.  It is crucial to validate the proposed mechanism and selectivity across different AML cell lines and patient samples before concluding its true novelty. \\

\textbf{Improvements to the idea}
\begin{itemize}
    \item Literature search: Conduct a thorough literature search beyond the provided abstracts to confirm the novelty of KIRA6 in AML. Investigate any existing research on KIRA6 in other hematological malignancies.
    \item Selectivity testing:  Expand selectivity testing to include other healthy cell types beyond normal myeloid cells, to address potential off-target effects and demonstrate greater safety margin.
    \item Combination exploration: Further explore rational combinations beyond FLT3 inhibitors, including chemotherapeutic agents and other targeted therapies.
    \item Resistance mechanisms:  Thoroughly investigate the impact of KIRA6 on various AML resistance mechanisms through in vitro testing, using resistant cell lines and patient-derived samples.
    \item In vivo efficacy and toxicity profile: Expand the in vivo studies with multiple cell lines and patient-derived xenograft models, with a focus on establishing the efficacy and toxicity profile of KIRA6, both as single-agent and in the proposed combination strategy.
\end{itemize}

\end{tcolorbox}
\vspace{0.1cm}
\caption{\textbf{Detailed AI co-scientist output for repurposing KIRA6, an IRE1$\alpha$ inhibitor for AML (continued)}.}
\label{fig:ex_kira6_7}
\end{figure}


\begin{figure}[htbp!]
\begin{tcolorbox}[
    colback=black!5!white,
    colframe=black!60!white,
    title=\textbf{KIRA6 for AML (continued)},
    fonttitle=\bfseries,
    arc=3mm,
    boxrule=1pt,
    bottomrule=2pt,
]
\scriptsize
\textbf{Reasoning about novelty and recommendation}

The idea is novel enough to warrant further exploration.  While targeting IRE1$\alpha$ isn't completely new, using KIRA6 in AML, especially in the context outlined, hasn't been extensively investigated, and it also has not been previously examined as a companion drug with other therapies used in AML.  The proposed combination with FLT3 inhibitors and the detailed rationale provide a strong foundation. The comprehensive experimental plan should be executed to confirm the preliminary findings and assess the true potential of KIRA6 in AML.  If the in vitro and in vivo findings are positive, the idea would be worthy of publication in a specialized journal focused on hematological malignancies or drug repurposing.


\textbf{Answer: 3}
\\

[1] Targeting the IRE1$\alpha$-XBP1s Pathway to Enhance Venetoclax Effectiveness in AML | Blood | American Society of Hematology
(https://ashpublications.org/blood/article/144/Supplement\%201/4144/533082/Targeting-the-IRE1-XBP1s-Pathway-to-Enhance)

[2] FLT3 inhibitors in acute myeloid leukemia: ten frequently asked questions | Leukemia
(https://www.nature.com/articles/s41375-019-0694-3)

[3] Inhibition of FLT3-ITD Kinase in Acute Myeloid Leukemia by New Imidazo[1,2-b]pyridazine Derivatives Identified by Scaffold Hopping | Journal of Medicinal Chemistry
(https://pubs.acs.org/doi/10.1021/acs.jmedchem.3c00575)

[4] Repurposing approved drugs on the pathway to novel therapies - PMC
(https://pmc.ncbi.nlm.nih.gov/articles/PMC7018532/)

[5] The Unfolded Protein Response: A Novel Therapeutic Target in Acute Leukemias - PMC
(https://pmc.ncbi.nlm.nih.gov/articles/PMC7072709/)

[6] Drugging the unfolded protein response in acute leukemias | Journal of Hematology \& Oncology | Full Text
(https://jhoonline.biomedcentral.com/articles/10.1186/s13045-015-0184-7)

[7] Selective degradation of mutant FMS-like tyrosine kinase-3 requires BIM-dependent depletion of heat shock proteins | Leukemia
(https://www.nature.com/articles/s41375-024-02405-5)

[8] Pharmacology and Toxicology Testing: IC50 for Tumor Cell Lines - Altogen Labs
(https://altogenlabs.com/pre-clinical-research-services/pharmacology-and-toxicology-testing-ic50-for-tumor-cell-lines/)

[9] Targeting Mitochondrial Structure Sensitizes Acute Myeloid Leukemia to Venetoclax Treatment | Cancer Discovery | American Association for Cancer Research
(https://aacrjournals.org/cancerdiscovery/article/9/7/890/41983/Targeting-Mitochondrial-Structure-Sensitizes-Acute)
\\
\\
\textbf{Critiques}

Here's a summary of the negative critiques from the reviews:
\begin{itemize}
    \item Limited safety data: The biggest concern is the lack of human safety data for KIRA6, requiring thorough preclinical toxicity studies.  The drug's safety profile is currently unknown and needs experimental validation.
    \item Limited scope of cell line testing:  Testing primarily focuses on MOLM13. Expanding to other AML cell lines (with different FLT3 mutations and other genetic backgrounds) and primary patient samples is necessary.
    \item Limited comparison to other drugs:  The idea would benefit from comparing KIRA6 to other IRE1$\alpha$ inhibitors or similar drugs to justify its selection and benchmark efficacy/selectivity. Minimal data exists regarding KIRA6 compared to other drugs.
    \item Lack of pre-clinical and clinical data:  While proposed, in vivo and toxicity data are currently lacking for KIRA6, despite the comprehensive experimental plan.
    \item Assumptions requiring validation: Several assumptions about KIRA6's selectivity for AML cells over healthy cells, synergy with other drugs, impact on drug resistance mechanisms, and the suggested IC50 concentration range need experimental validation.
    \item Limited detail in some areas: The rationale for the specific IC50 range could be stronger. The exploration of resistance mechanisms and the details of synergy studies could also be more in-depth. More detail on how KIRA6 impacts MYC, NF-kB and inflammatory pathways is needed.  The connection between UPR activation and a lower incidence of FLT3 mutations needs further investigation.
    \item Potential for resistance: While the idea aims to overcome resistance, there's still a need to investigate how KIRA6 itself might contribute to new resistance mechanisms over time.
\end{itemize}

Although the idea is considered plausible, novel, and promising, these weaknesses need to be addressed through further research and experimentation.

\end{tcolorbox}
\vspace{0.1cm}
\caption{\textbf{Detailed AI co-scientist output for repurposing KIRA6, an IRE1$\alpha$ inhibitor for AML (continued)}.}
\label{fig:ex_kira6_8}
\end{figure}