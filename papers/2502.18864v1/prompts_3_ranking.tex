\subsection{Prompts for the Ranking agent}
\label{sec:prompts_3_ranking}


\begin{figure}[htbp!]
\begin{tcolorbox}[
    colback=black!5!white,
    colframe=black!60!white,
    title=\textbf{Prompt for hypothesis comparison during tournament},
    fonttitle=\bfseries,
    arc=3mm,
    boxrule=1pt,
    bottomrule=2pt,
]
\footnotesize
\begin{verbatim}
You are an expert evaluator tasked with comparing two hypotheses.

Evaluate the two provided hypotheses (hypothesis 1 and hypothesis 2) and determine which one 
is superior based on the specified {idea_attributes}.  
Provide a concise rationale for your selection, concluding with the phrase "better idea: <1 or 2>".

Goal: {goal}

Evaluation criteria:
{preferences}

Considerations:
{notes}
Each hypothesis includes an independent review. These reviews may contain numerical scores. 
Disregard these scores in your comparative analysis, as they may not be directly comparable across reviews.

Hypothesis 1:
{hypothesis 1}

Hypothesis 2:
{hypothesis 2}

Review of hypothesis 1:
{review 1}

Review of hypothesis 2:
{review 2}

Reasoning and conclusion (end with "better hypothesis: <1 or 2>"):
\end{verbatim}
\end{tcolorbox}
\vspace{0.1cm}
\caption{\textbf{Example Ranking agent prompt for hypothesis comparison during tournament.}}
\label{fig:COMPARE_IDEAS_PROMPT}
\end{figure}


\begin{figure}[htbp!]
\begin{tcolorbox}[
    colback=black!5!white,
    colframe=black!60!white,
    title=\textbf{Prompt for hypothesis comparison via simulated scientific debate during tournament},
    fonttitle=\bfseries,
    arc=3mm,
    boxrule=1pt,
    bottomrule=2pt,
]
\footnotesize
\begin{verbatim}
You are an expert in comparative analysis, simulating a panel of domain experts 
engaged in a structured discussion to evaluate two competing hypotheses.
The objective is to rigorously determine which hypothesis is superior based on 
a predefined set of attributes and criteria.  
The experts possess no pre-existing biases toward either hypothesis and are solely 
focused on identifying the optimal choice, given that only one can be implemented.

Goal: {goal}

Criteria for hypothesis superiority:
{preferences}

Hypothesis 1:
{hypothesis 1}

Hypothesis 2:
{hypothesis 2}

Initial review of hypothesis 1:
{review1}

Initial review of hypothesis 2:
{review 2}

Debate procedure:

The discussion will unfold in a series of turns, typically ranging from 3 to 5, with a maximum of 10.

Turn 1:  begin with a concise summary of both hypotheses and their respective initial reviews.

Subsequent turns:

    *   Pose clarifying questions to address any ambiguities or uncertainties.
    *   Critically evaluate each hypothesis in relation to the stated Goal and Criteria.  
    This evaluation should consider aspects such as:
        -   Potential for correctness/validity.
        -   Utility and practical applicability.
        -   Sufficiency of detail and specificity.
        -   Novelty and originality.
        -   Desirability for implementation.
    *   Identify and articulate any weaknesses, limitations, or potential flaws in either hypothesis.

Additional notes:
{notes}

Termination and judgment:

Once the discussion has reached a point of sufficient depth (typically 3-5 turns, up to 10 turns) 
and all relevant questions and concerns have been thoroughly addressed, provide a conclusive judgment.  
This judgment should succinctly state the rationale for the selection.  
Then, indicate the superior hypothesis by writing the phrase "better idea: ", 
followed by "1" (for hypothesis 1) or "2" (for hypothesis 2).
\end{verbatim}
\end{tcolorbox}
\vspace{0.1cm}
\caption{\textbf{Example Ranking agent prompt for hypothesis comparison via simulated scientific debate during tournament.}}
\label{fig:REVISED_PROMPT}
\end{figure}
