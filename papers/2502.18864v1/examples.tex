\section{Examples of the AI co-scientist inputs, intermediate outputs and final results}
\label{sec:supp_ex}

\begin{figure}[htbp!]
\begin{tcolorbox}[
    colback=black!5!white,
    colframe=black!60!white,
    title=\textbf{From research goal to research plan configuration},
    fonttitle=\bfseries,
    arc=3mm,
    boxrule=1pt,
    bottomrule=2pt,
]
\footnotesize
\textbf{Scientist research goal}

Develop a novel hypothesis for the key factor or process which causes ALS related to phosphorylation of a Nuclear Pore Complex (NPC) nucleoporin. Explain mechanism of action in detail. Include also a feasible experiment to test the hypothesis.
\\
\\
\textbf{Parsed research plan configuration}
\begin{itemize}
    \item \textbf{Preferences:} Focus on providing a novel hypothesis, with detailed explanation of the mechanism of action.
    \item \textbf{Attributes:} Novelty, Feasibility
    \item \textbf{Constraints:} should be correct, should be novel.
\end{itemize}
\end{tcolorbox}
\vspace{0.1cm}
\caption{\textbf{An example of a research goal and a parsed research plan configuration.} This figure illustrates how the co-scientist system parses a scientist's natural language research goal into a research plan configuration, which guides subsequent reasoning and computation.}
\label{fig:ex_research_goal}
\end{figure}


\begin{figure}[htbp!]
\begin{tcolorbox}[
    colback=black!5!white,
    colframe=black!60!white,
    title=\textbf{Example output of the Generation agent},
    fonttitle=\bfseries,
    arc=3mm,
    boxrule=1pt,
    bottomrule=2pt,
]
\footnotesize
\textbf{Research goal} Develop a novel hypothesis for the key factor or process which contributes to ALS progression ... We can use iPSC cells for the in-vitro experiments. ...
\\
\\
\textbf{Introduction}
\\
Amyotrophic Lateral Sclerosis (ALS) is a fatal neurodegenerative disease characterized by the selective loss of motor neurons, leading to progressive muscle weakness, paralysis, and ultimately death. While the exact cause of ALS remains unknown, a complex interplay of genetic predisposition, environmental factors, and age-related cellular changes is believed to contribute to its pathogenesis. ...
\\
\\
\textbf{Recent findings and related research}
\\
Recent research has implicated several key players and processes in ALS. TDP-43, an RNA-binding protein, is found mislocalized and aggregated in the cytoplasm of affected neurons in the majority of ALS cases. This protein's ability to undergo liquid-liquid phase separation (LLPS) has come under intense scrutiny, with evidence suggesting that dysregulation of this process leads to the formation of pathological aggregates. Further, impairments in nucleocytoplasmic transport, the process by which molecules move between the nucleus and cytoplasm, have been observed in ALS and linked to TDP-43 pathology. Endoplasmic reticulum (ER) stress, a cellular response to protein misfolding, has also been implicated as a potential contributing factor.
\\
\\
\textbf{Hypothesis}
\\
Cellular stress (including but not limited to ER stress) induces post-translational modifications (PTMs) on specific FG-Nups (focusing initially on Nup98 and Nup62) at the nuclear pore complex (NPC). These Nup PTMs modify the interaction between these Nups and TDP-43, leading to altered TDP-43 dynamics and increased retention at the NPC. This increased retention at the NPC, in turn, triggers the disruption of nucleocytoplasmic transport.
\\
\\
\textbf{Rationale and specificity}
\\
Instead of making the hypothesis about LLPS, we are making it much more specific but with the same starting point. Cellular stress, a broad concept encompassing various cellular insults, can trigger a cascade of molecular events. A key consequence of stress response is altered protein structure and function, which can be mediated by PTMs. Nuclear pore proteins, particularly FG-Nups like Nup98 and Nup62, form a "hydrogel" or selective barrier at the NPC, regulating the movement of molecules between the nucleus and cytoplasm. We hypothesize that specific PTMs on these FG-Nups (e.g., phosphorylation, O-GlcNAcylation initially) directly alter their interaction with TDP-43. TDP-43, known to interact with the NPC and be involved in nucleocytoplasmic transport, may get inappropriately trapped or retained at the NPC when interacting with PTM-modified Nups. This trapping alters TDP-43 dynamics (even if not disrupting LLPS directly initially), meaning it cannot perform its normal functions and disrupting nucleocytoplasmic transport, which we know is happening in ALS. Since even a small change in transport could lead to downstream problems, this is a plausible initial event with testable consequences. The specific FG-Nups (Nup98/62) and chosen PTMs (phosphorylation/O-GlcNAcylation) provide a concrete starting point for experimentation. This mechanism directly links cellular stress to TDP-43 dysfunction and the key ALS pathology of defective nucleocytoplasmic transport by invoking a novel interaction mechanism at the NPC – all testable in vitro.
\\
\\
\textbf{Experimental design and validation}
\\
\textbf{Cell Types:} Human iPSC-derived motor neurons from healthy controls and ALS patients.
Stress Induction: Induce cellular stress using established pharmacological agents (e.g., tunicamycin for ER stress, arsenite for oxidative stress, oligomycin A for mitochondrial stress) and physical stressors (heat shock). Test different stressors individually and in combination. This broadened approach allows for a more robust investigation of the stress response.

\textbf{Nup PTM Analysis:} Assess PTMs of Nup98 and Nup62 specifically using mass spectrometry and immunoblotting with PTM-specific antibodies (specifically looking for phosphorylation and O-GlcNAcylation). Focus the analysis on these two.
\\
...
\end{tcolorbox}
\vspace{0.1cm}
\caption{\textbf{An example output of the Generation agent.} The abbreviated example showcases an initial hypothesis and research proposal generated by the co-scientist Reflection agent pertaining to the research goal of explaining the basic mechanism for Amyotrophic Lateral Sclerosis (ALS).}
\label{fig:ex_generation}
\end{figure}





\begin{figure}[htbp!]
\begin{tcolorbox}[
    colback=black!5!white,
    colframe=black!60!white,
    title=\textbf{Example output of a complete novelty review by the Reflection agent},
    fonttitle=\bfseries,
    arc=3mm,
    boxrule=1pt,
    bottomrule=2pt,
]
\footnotesize
\textbf{Aspects already explored:}
\begin{itemize}
    \item TDP-43 mislocalization and aggregation in ALS: Extensively documented in multiple abstracts [1, 3, 4, 5, 6].
    \item Nucleocytoplasmic transport defects in ALS: Demonstrated in connection with TDP-43 pathology and C9orf72 mutations [4, 5, 6].
    \item NPC disruption in neurodegenerative diseases: Identified in ALS/FTD, Alzheimer's, Huntington's, and others, suggesting a common mechanism [4, 5, 7, 8].
    \item The link between cellular stress and TDP-43 pathology: Explored in the context of stress granule formation and TDP-43 aggregation [5]. Note - stress is explored in general, but the idea of it leading to PTMs of Nups is new.
    \item Role of FG-Nups in NPC function: FG-Nups are known to form a selective barrier regulating transport, but the impact of their PTMs on TDP-43 interaction in ALS is novel [7].
    \item The implication of PTMs for TDP-43: TDP-43 PTMs are widely researched, especially phosphorylation and ubiquitination. However, our hypothesis doesn't speak about TDP-43 PTMs, it is about Nup PTMs [1, 3, 4].
\end{itemize}
\textbf{Novel Aspects:}
\begin{itemize}
    \item Cellular stress as an initiator of Nup PTMs: While stress is linked to TDP-43 pathology, the idea that it triggers specific PTMs on Nups (Nup98/Nup62) as an initiating event in ALS is novel, especially that these PTMs are phosphorylation and O-GlcNAcylation.
    \item Nup PTMs (phosphorylation and O-GlcNAcylation specifically) altering TDP-43 interaction at the NPC: The hypothesis proposes a novel mechanism where stress-induced PTMs on specific Nups (Nup98/Nup62) directly modify their interaction with TDP-43. This hasn't been explored before, particularly the focus on these specific PTMs and Nups.
    \item TDP-43 retention at the NPC as a key event: Instead of focusing on LLPS or aggregation directly, the hypothesis suggests that altered TDP-43 dynamics, specifically increased retention at the NPC due to modified Nup interactions, are a triggering factor. This specific mechanism at the NPC disrupting nucleocytoplasmic transport is novel.
    \item Focus on early events and potential for therapeutic intervention: By pinpointing early molecular events (Nup PTMs, altered TDP-43-Nup interaction), the hypothesis opens new avenues for therapeutic intervention targeting these specific processes, potentially before irreversible damage occurs.
    \item Broadening stress beyond ER stress: It is novel to consider multiple stressors in combination and it broadens the idea beyond ER stress (as suggested in the original provided idea).
\end{itemize}
\end{tcolorbox}
\vspace{0.1cm}
\caption{\textbf{An example output of a complete novelty review from the Reflection agent.} The abbreviated example showcases a complete novelty review by the co-scientist Reflection agent for an example hypothesis pertaining to explaining the basic mechanism for Amyotrophic Lateral Sclerosis (ALS).}
\label{fig:ex_reflection_novelty}
\end{figure}


\begin{figure}[htbp!]
\begin{tcolorbox}[
    colback=black!5!white,
    colframe=black!60!white,
    title=\textbf{Example output of review critiques by the Reflection agent},
    fonttitle=\bfseries,
    arc=3mm,
    boxrule=1pt,
    bottomrule=2pt,
]
\footnotesize
\begin{itemize}
    \item Lack of strong justification for motor neuron specificity: The initial hypothesis lacked a strong explanation for why the proposed mechanism would predominantly affect motor neurons, a critical aspect of ALS pathology.
    \item Limited exploration of downstream consequences: While the idea connects to nucleocytoplasmic transport defects, it doesn't fully elaborate on how these defects lead to other ALS-related pathologies like TDP-43 aggregation or RNA processing defects.
    \item Potential technical challenges: Identifying the specific PTMs responsible and demonstrating their direct impact on TDP-43 retention and transport might be technically difficult.
    \item Narrow focus on specific Nups and PTMs: Initially focusing only on Nup98/Nup62 and two types of PTMs (phosphorylation, O-GlcNAcylation) could be too restrictive, potentially overlooking other important factors.
    \item Less holistic approach: The focus on a specific mechanism at the NPC might neglect broader cellular processes involved in ALS pathogenesis.
    \item The temporal relationship with TDP-43 pathology is not clear: It's not clear whether the hypothesized mechanism occurs before or after other ALS hallmarks like cytoplasmic TDP-43 aggregation, which is crucial for establishing causality.
    \item ...
\end{itemize}
\end{tcolorbox}
\vspace{0.1cm}
\caption{\textbf{An example output of review critiques from the Reflection agent.}}
\label{fig:ex_review_critiques}
\end{figure}

\begin{figure}[htbp!]
\begin{tcolorbox}[
    colback=black!5!white,
    colframe=black!60!white,
    title=\textbf{Example output of a full review by the Reflection agent},
    fonttitle=\bfseries,
    arc=3mm,
    boxrule=1pt,
    bottomrule=2pt,
]
\footnotesize
\textbf{Related Articles:}\\
$[5]$ TDP-43 pathology disrupts nuclear pore complexes and nucleocytoplasmic transport in ALS/FTD - PMC: This article is highly relevant because it directly links TDP-43 pathology to nuclear pore complex (NPC) dysfunction and nucleocytoplasmic transport (NCT) defects in ALS/FTD, providing a strong foundation for the hypothesis.\\
$[4]$ Unraveling the impact of disrupted nucleocytoplasmic transport systems in C9orf72-associated ALS - PMC: This article provides context on NCT defects in ALS, particularly in the context of C9orf72 mutations, highlighting the broader relevance of NCT dysfunction in different forms of ALS.\\
$[7]$ Nuclear pore and nucleocytoplasmic transport impairment in oxidative stress-induced neurodegeneration: relevance to molecular mechanisms in Pathogenesis of Parkinson’s and other related neurodegenerative diseases - PMC: This article connects oxidative stress, a form of cellular stress, to NPC dysfunction and NCT impairment in neurodegenerative diseases, supporting the idea that cellular stress can impact NPCs.\\
$[3]$ Frontiers | Molecular Mechanisms of TDP-43 Misfolding and Pathology in Amyotrophic Lateral Sclerosis: This article reviews TDP-43 pathology in ALS, including its mislocalization and aggregation, which are central to the proposed hypothesis.\\
$[1]$ TDP-43 proteinopathies: a new wave of neurodegenerative diseases - PMC: Provides an overview of TDP-43 proteinopathies, establishing the importance of TDP-43 dysfunction in ALS and other neurodegenerative diseases.\\
$[8]$ A$\beta$-driven nuclear pore complex dysfunction alters activation of necroptosis proteins in a mouse model of Alzheimer’s Disease: Although focused on Alzheimer's, this article demonstrates that NPC dysfunction can occur in neurodegenerative diseases and highlights the potential impact on cellular processes like necroptosis. It supports the general idea that NPC defects can be a consequence of proteinopathy-associated stress.\\
$[9]$ Traumatic injury compromises nucleocytoplasmic transport and leads to TDP-43 pathology: This article provides another example of how cellular stress (traumatic injury) can lead to NCT defects and TDP-43 pathology, further supporting the link between stress, NCT, and TDP-43 dysfunction.\\
...\\
\\
\textbf{Assumptions of the Idea:}
\begin{itemize}
    \item Cellular stress induces PTMs on Nup98 and Nup62: The hypothesis assumes that various forms of cellular stress can trigger specific PTMs (phosphorylation and O-GlcNAcylation) on Nup98 and Nup62.
    \item PTMs on Nup98/Nup62 alter their interaction with TDP-43: It's assumed that these specific PTMs directly modify the binding affinity or interaction dynamics between Nup98/Nup62 and TDP-43 at the NPC.
    \item Altered Nup-TDP-43 interaction leads to increased TDP-43 retention at the NPC: This assumption posits that the modified interaction results in TDP-43 being trapped or retained at the NPC for longer durations.
    \item Increased TDP-43 retention at the NPC disrupts nucleocytoplasmic transport: The hypothesis assumes that this prolonged retention of TDP-43 at the NPC interferes with the normal transport of other molecules across the nuclear envelope.
    \item Nucleocytoplasmic transport defects contribute to ALS pathology: The idea assumes that the resulting NCT disruption is a significant contributor to the downstream pathological events in ALS, such as motor neuron degeneration.
    \item Motor neurons are more susceptible to this mechanism: It's assumed that motor neurons are particularly vulnerable to this stress-induced, Nup-mediated TDP-43 retention and subsequent NCT defects compared to other cell types.
\end{itemize}
\textbf{Reasoning about assumptions:}
\begin{itemize}
    \item Cellular stress induces PTMs on Nup98 and Nup62: This is plausible. Cellular stress is known to activate various signaling pathways that can lead to PTMs on numerous proteins. Nups, being long-lived proteins, are susceptible to modifications. However, whether Nup98 and Nup62 are specifically targeted by stress-induced phosphorylation and O-GlcNAcylation needs experimental verification.
    \item PTMs on Nup98/Nup62 alter their interaction with TDP-43: This is possible but requires experimental validation. PTMs can significantly alter protein-protein interactions by changing surface charge, conformation, or binding sites. While there's no direct evidence for this specific interaction, it's a reasonable hypothesis given the role of FG-Nups in interacting with transport factors.
    \item Nucleocytoplasmic transport defects contribute to ALS pathology: This is supported by existing literature [4, 5]. NCT defects have been observed in ALS and are thought to contribute to disease pathogenesis. However, whether this specific mechanism is a primary driver or a downstream consequence needs further investigation.
    \item Motor neurons are more susceptible to this mechanism: This is a crucial assumption but lacks strong justification in the current proposal. Motor neurons have unique characteristics (high metabolic demand, long axons) that might make them more vulnerable to stress and NCT defects. However, this needs to be explicitly tested by comparing motor neurons to other cell types.
    \item ...
\end{itemize}
\end{tcolorbox}
\vspace{0.1cm}
\caption{\textbf{An example output of a full review by the Reflection agent} The abbreviated example showcases a full initial review generated by the co-scientist Reflection agent for an example hypothesis pertaining to explaining the basic mechanism for Amyotrophic Lateral Sclerosis (ALS).}
\label{fig:ex_reflection_basic}
\end{figure}


\begin{figure}[htbp!]
\begin{tcolorbox}[
    colback=black!5!white,
    colframe=black!60!white,
    title=\textbf{Example output of deep verification by the Reflection agent},
    fonttitle=\bfseries,
    arc=3mm,
    boxrule=1pt,
    bottomrule=2pt,
]
\footnotesize
\textbf{Assumptions:}
\begin{itemize}
    \item Cellular stress induces PTMs like phosphorylation and O-GlcNAcylation.
    \item Nup98 and Nup62 interact with TDP-43 at the NPC.
    \begin{itemize}
        \item Nup98 can interact with TDP-43
        \item Nup62 can interact with TDP-43
    \end{itemize}
    \item Altered interaction between Nups and TDP-43 leads to increased TDP-43 retention at the NPC.
    \item Disrupted nucleocytoplasmic transport triggers downstream ALS pathology.
    \item Motor neurons are inherently more vulnerable to this mechanism.
    \item The level of cellular stress inducing these PTMs is sufficient to cause a significant impact on TDP-43 localization.
    \item TDP-43 localizes to the nuclear pore complex (NPC).
    \item ...
\end{itemize}

\textbf{1. Cellular stress induces PTMs like phosphorylation and O-GlcNAcylation.}\\
Cellular stress encompasses a range of conditions that disrupt cellular homeostasis, including nutrient deprivation, hypoxia, oxidative stress, and the accumulation of misfolded proteins. The endoplasmic reticulum (ER) is a central organelle involved in protein synthesis, folding, and modification. When the ER's capacity to handle protein folding is overwhelmed, a condition known as ER stress arises. This triggers a signaling pathway called the unfolded protein response (UPR) aimed at restoring ER homeostasis.

Post-translational modifications (PTMs) are covalent modifications to proteins that occur after translation. They play crucial roles in regulating protein function, localization, and interactions. Phosphorylation, the addition of a phosphate group, and O-GlcNAcylation, the addition of O-linked N-acetylglucosamine, are two common PTMs.

Numerous studies have demonstrated a strong link between cellular stress, including ER stress, and the alteration of PTMs. Here's a breakdown:
\begin{itemize}
    \item \textbf{ER Stress and Phosphorylation:} The UPR, activated during ER stress, involves multiple signaling pathways, many of which rely on phosphorylation. For example, PERK (PKR-like ER kinase), a key UPR sensor, undergoes autophosphorylation upon sensing misfolded proteins. This activated PERK then phosphorylates eIF2$\alpha$ (eukaryotic initiation factor 2 alpha), leading to a global attenuation of translation. Another UPR sensor, IRE1 (inositol-requiring enzyme 1), possesses both kinase and endoribonuclease activity. Upon activation, IRE1 autophosphorylates and splices XBP1 mRNA, leading to the production of a transcription factor that upregulates chaperones and other proteins involved in protein folding and degradation. These are just two examples; numerous other proteins are phosphorylated during ER stress, affecting various cellular processes.
    \item \textbf{ER Stress and O-GlcNAcylation:} O-GlcNAcylation is a dynamic PTM influenced by nutrient availability and cellular stress. Evidence suggests a complex interplay between ER stress and O-GlcNAcylation. Some studies report increased O-GlcNAcylation during ER stress, potentially as a protective mechanism. For example, O-GlcNAcylation of certain chaperones can enhance their activity and promote protein folding. Conversely, other studies suggest that ER stress can lead to decreased O-GlcNAcylation. The exact relationship likely depends on the specific stress conditions, cell type, and the targeted proteins.
    \item \textbf{Cellular Stress (beyond ER Stress) and PTMs:} Other forms of cellular stress also impact PTMs. Oxidative stress, for example, can lead to protein oxidation, which can affect protein function and stability. Nutrient deprivation can alter the availability of substrates for PTMs, such as UDP-GlcNAc for O-GlcNAcylation. Hypoxia can also induce changes in phosphorylation and other PTMs, influencing cellular responses to low oxygen levels.
\end{itemize}
In summary, the available evidence strongly supports the assertion that cellular stress, including ER stress, can induce PTMs like phosphorylation and O-GlcNAcylation on proteins. This effect is mediated by complex signaling pathways and serves to regulate various cellular processes in response to stress.\\
\end{tcolorbox}
\vspace{0.1cm}
\caption{\textbf{An example output of a deep verification review by the Reflection agent.} The abbreviated example showcases a deep verification review by the co-scientist Reflection agent for the Amyotrophic Lateral Sclerosis (ALS) basic mechanism research goal given the co-scientist's generated hypothesis and other reviews of it.}
\label{fig:ex_deep_verification}
\end{figure}

\begin{figure}[htbp!]
\begin{tcolorbox}[
    colback=black!5!white,
    colframe=black!60!white,
    title=\textbf{Example output of deep verification via probing questions by the Reflection agent},
    fonttitle=\bfseries,
    arc=3mm,
    boxrule=1pt,
    bottomrule=2pt,
]
\footnotesize
\textbf{[Research Goal]}

Repurposing Reparixin for AML treatment.

Suggest an existing drug that could be repurposed for Acute myelogenous leukemia (AML) treatment and provide experimentally testable concentrations for an IC50 assay. The drug should inhibit the proliferation of AML cell lines, particularly MOLM13.

The drug should have minimal cytotoxicity in healthy cell lines (have a good tradeoff for AML treatment). The drug should not have been experimentally tested for AML treatment before.

Reason about the drug's molecular mechanism, impacted pathways, and effect on AML cell proliferation inhibition. Describe the mechanism of action in detail. Reason about the safety and toxicity, stating if no safety data exists.

\textbf{[Output]}

\textbf{Question:} Sufficiency of CXCR1/2 Inhibition: What is the rationale for assuming that CXCR1/2 inhibition *alone* will be sufficient to overcome the established heterogeneity and multiple compensatory pathways known to drive AML progression and resistance, *even before* considering combination therapies?\\

This challenges the fundamental assumption that targeting a single pathway (even a dual-target one) will be effective in a complex disease like AML. AML is known for its ability to bypass single-target inhibition. Addressing this upfront is crucial to justify the entire approach, especially before investing heavily in combination studies.\\

\textbf{Answer:} The rationale for initially focusing on CXCR1/2 inhibition with Reparixin, before immediately jumping to combinations, stems from several interwoven, but not entirely definitive, points:
\begin{itemize}
    \item \textbf{Targeting a Key TME Node:} CXCR1/2 are major regulators of neutrophil and myeloid-derived suppressor cell (MDSC) recruitment to the tumor microenvironment. These cells, in turn, contribute significantly to immune suppression, angiogenesis, and therapy resistance in AML. By blocking these receptors, Reparixin aims to disrupt a central hub of TME-mediated support for AML cells. The idea is not necessarily that CXCR1/2 inhibition is sufficient in all cases, but that it addresses a significant and often overlooked contributor to AML progression.
    \item \textbf{Dual Receptor Blockade:} Reparixin's dual inhibition of CXCR1 and CXCR2 is hypothesized to be more effective than targeting either receptor individually. While not explicitly stated in the provided text, there could be redundancy or synergy between these two receptors in promoting AML progression. Blocking both might offer a more substantial impact on the TME than single-receptor antagonists.
    \item \textbf{Establishing a Baseline:} The proposal, while acknowledging the complexity of AML, strategically starts with single-agent Reparixin to establish a clear baseline of its activity and effects. This is crucial for several reasons:
    \begin{itemize}
        \item Understanding Single-Agent Effects: It's essential to dissect the specific contributions of Reparixin alone before introducing the complexity of combinations. This allows for a more rational design of combination strategies.
        \item Identifying Resistance Mechanisms: Studying single-agent resistance is simpler and more direct than untangling resistance mechanisms in a multi-drug scenario. The extensive resistance studies proposed are specifically geared towards understanding Reparixin-specific resistance.
        \item Patient Stratification: The comprehensive patient stratification plan is designed to identify patients who might respond to Reparixin as a single agent, or for whom it forms a critical foundation for combination therapy.
    \end{itemize}
    \item \textbf{TME Reprogramming, Not Just Cytotoxicity:} The proposal positions Reparixin not just as a cytotoxic agent, but as a TME modulator. The hypothesis is that by altering the TME, Reparixin may sensitize AML cells to other therapies or even trigger immune-mediated clearance. This "reprogramming" effect might not be fully captured by traditional cytotoxicity assays alone.
    \item \textbf{Sequential therapy:} The proposal includes sequential administration of drugs, where Reparixin can prime the tumor cells for killing and/or resensitize them too.
\end{itemize}
\textbf{Reasoning:} The idea isn't incoherent, but it requires careful consideration of the sufficiency assumption. While the points above provide a rationale, they don't guarantee that CXCR1/2 inhibition alone will be sufficient in all, or even most, cases of AML. The heterogeneity of AML, with its multitude of driver mutations and compensatory pathways, strongly suggests that combination therapies will ultimately be necessary for many patients.

The strength of the proposal lies in its methodical approach. It doesn't claim that Reparixin is a magic bullet. It sets out to systematically investigate its potential, both alone and in combination, and to identify the specific patient subsets most likely to benefit. The extensive in vivo and patient stratification studies are key to addressing the "sufficiency" question empirically. The proposal is essentially a very detailed roadmap for determining when and how CXCR1/2 inhibition can be most effectively used in AML treatment, rather than an assertion of universal single-agent efficacy. Therefore the idea may require combination approaches but the study plan described in the proposal is structured in a way to determine that.
\end{tcolorbox}
\vspace{0.1cm}
\caption{\textbf{An example of a deep verification review via probing questions by the Reflection agent.} The example demonstrates how the Reflection agent performs deep verification via probing question for a research hypothesis pertaining to drug repurposing for Acute Myeloid Leukemia (AML). The original research goal is provide for illustration only, it is not included in the review output.}
\label{fig:ex_deep_verification_probe}
\end{figure}


\begin{figure}[htbp!]
\begin{tcolorbox}[
    colback=black!5!white,
    colframe=black!60!white,
    title=\textbf{Example output of an observation review by the Reflection Agent},
    fonttitle=\bfseries,
    arc=3mm,
    boxrule=1pt,
    bottomrule=2pt,
]
\footnotesize
\textbf{[Research Goal]}

Unravel a specific and novel molecular mechanism explaining how the same cf-PICI can be found in different bacterial species.

...

Why can cf-PICIs, but not other types of PICIs or satellites, be easily found in different bacterial species? What is the mechanism that explains this?

\textbf{[Output]}

The "Modular Capsid-OMV Mediated Transfer" (MCOMT) hypothesis proposes that cf-PICIs achieve their broad host range through a combination of modular capsid interactions and OMV-mediated transfer, using both tail-dependent and tail-independent mechanisms. Several observations support this hypothesis:
\begin{itemize}
    \item Widespread presence of PICIs in Gram-negative bacteria, particularly Enterobacteriaceae and Pastuerellaceae: These bacteria readily produce and internalize OMVs, aligning with the MCOMT hypothesis that OMVs are crucial for cf-PICI dissemination. The modular capsid, with its broad bacterial surface binding capabilities, further enhances transfer even without specific tail receptors on target cells.
    \item Unique PICI att sites not used by temperate prophages: The MCOMT hypothesis, by reducing reliance on helper phage integration due to OMV-mediated transfer and diverse entry routes, explains the evolution of cf-PICI-specific att sites independent of phage integration sites. This allows wider integration across bacterial genomes.
    \item ...
\end{itemize}
\end{tcolorbox}
\vspace{0.1cm}
\caption{\textbf{An example output of an observation review by the Reflection agent.} The abbreviated example showcases an observation review by the co-scientist Reflection agent for a hypothesis related to explaining a mechanism of antimicrobial resistance (AMR). The original research goal is provide for illustration only, it is not included in the review output.}
\label{fig:ex_observation}
\end{figure}


\begin{figure}[htbp!]
\begin{tcolorbox}[
    colback=black!5!white,
    colframe=black!60!white,
    title=\textbf{Example of a scientific debate match in the Ranking agent tournament},
    fonttitle=\bfseries,
    arc=3mm,
    boxrule=1pt,
    bottomrule=2pt,
]
\footnotesize
\textbf{Expert 1:} I'm intrigued by both ideas, but I have a few clarifying questions. For Idea 1, while the focus on TDP-43 retention at the NPC is novel, I'm not entirely convinced that it's fundamentally different from the extensively studied TDP-43 mislocalization. Could the retention just be a consequence of mislocalization, or are there experiments that can distinguish these two phenomena? Another question I have: Is there evidence already that specific PTMs modulate the interaction of Nups and TDP-43? Not necessarily in the context of ALS, but are there any precedents?\\

\textbf{Expert 2:} For Idea 2, I appreciate the bold approach, but the ISR as a trigger is a significant departure from the conventional view. Can the proponents of this idea elaborate on the evidence supporting this? Also, what specific lipid species are they proposing to be most critical at the ER-mitochondria interface and the AIS? Is there any preliminary reason why those lipids would be disrupted in ALS?\\

\textbf{Expert 3:} I agree with Expert B regarding the ISR. Shifting it from a response to a trigger is a big claim that needs strong justification. However, I think there are some initial clues that might be worth investigating. For example, there are reports that chronic low-level stress responses can actually be detrimental in some contexts. Perhaps a low, persistent activation of the ISR could predispose cells to damage. This idea needs to be fleshed out though.\\

...\\

\textbf{Expert 3:} I also find Idea 1 more compelling at this point. It's a more focused hypothesis with a clear path for experimental validation. While Idea 2 is undoubtedly more "outside the box," I'm concerned about the lack of evidence for the ISR as a trigger and the technical challenges. It feels like a high-risk, high-reward scenario, and given that we can only choose one, I'd prefer the more grounded approach of Idea 1. I do agree with the idea that the technical challenges are significant, but I like that the experiments proposed are standard and there are many commercial antibodies available to start testing this idea.\\

...\\

\textbf{Expert 2:} Alright, I'm on board with Idea 1. Let's focus our efforts on testing this specific mechanism and address the motor neuron specificity question rigorously in the experimental design. Perhaps by comparing different cell types and focusing on motor neuron-specific RNAs, as suggested earlier. We should also consider investigating different types of stress and their combined effects.\\

\textbf{Better idea: 1}
\end{tcolorbox}
\vspace{0.1cm}
\caption{\textbf{An example of a scientific debate match between two hypotheses in the tournament conducted by the Ranking agent.} The abbreviated example showcases a scientific debate match orchestrated by the co-scientist Ranking agent as part of its tournament to compare two ideas related to explaining the mechanism of ALS.}
\label{fig:ex_ranking}
\end{figure}


\begin{figure}[htbp!]
\begin{tcolorbox}[
    colback=black!5!white,
    colframe=black!60!white,
    title=\textbf{Example output of a critique by the Meta-review agent},
    fonttitle=\bfseries,
    arc=3mm,
    boxrule=1pt,
    bottomrule=2pt,
]
\footnotesize
\textbf{I. Core Hypothesis and Mechanism:}
\begin{itemize}
    \item \textbf{Primary Driver vs. Consequence:} A very common critique across many ideas was the difficulty in proving that the proposed mechanism is a primary initiating event in ALS rather than a downstream consequence of other processes. Future ideas need to:
    \begin{itemize}
        \item Provide strong evidence for the temporal sequence of events using longitudinal experiments.
        \item Employ models where early-stage events can be studied and manipulated.
        \item Explicitly address whether the proposed mechanism is causative or correlative.
        \item Specifically knockdown the primary driver to prove it's necessary and sufficient to cause the pathology.
    \end{itemize}
    \item \textbf{Specificity:} Many hypotheses were criticized for a lack of specificity in various aspects:
    \begin{itemize}
        \item \textbf{Molecular Targets:} Imprecise identification of specific proteins, RNA species, lipids, PTMs, etc. involved. Ideas should name specific candidates.
        \item \textbf{Cellular Compartments:} Vague localization (e.g., "mitochondria" or "AIS"). Ideas should pinpoint specific subcellular locations whenever possible.
        \item \textbf{Disease Mechanisms:} Broad terms like "oxidative stress" or "autophagy impairment." Ideas need to define the precise molecular events involved.
        \item \textbf{Types of Stress:} If stress is involved, the specific type of stress needs to be defined (e.g., ER stress, nutrient deprivation, mitochondrial stress, ionic stress).
        \item \textbf{Temporality:} The timing of the intervention is critical. Early vs. late events, duration of processes, etc., should be clearly specified.
    \end{itemize}
    \item \textbf{Motor Neuron Specificity:} Given the focus on ALS, a strong explanation for why the proposed mechanism would specifically or preferentially affect motor neurons (and often specific subtypes of motor neurons) is crucial. Many ideas were critiqued for not adequately addressing this.
    \item \textbf{Over-Simplification vs. Complexity:} This is a balancing act. Some ideas were criticized for being too narrowly focused, potentially oversimplifying the complex, multifactorial nature of ALS. Others were criticized for being overly complex, making them difficult to test and validate experimentally. When in doubt, err on the side of testability in a reasonable timeframe and budget.
    \item \textbf{ALS Subtype Relevance:} Ideas should address whether the proposed mechanism is relevant to all ALS subtypes or specific to certain genetic or sporadic forms. Ideally, address how the proposed work will differentiate or be applicable to different subtypes.
\end{itemize}

\textbf{II. Experimental Design and Feasibility:}
\begin{itemize}
    \item \textbf{Model System Limitations:} Reviewers frequently pointed out the limitations of in vitro models, particularly iPSC-derived motor neurons. While valuable, these models may not fully capture the in vivo environment, cell-cell interactions, or the aging process. Future ideas need to:
    \begin{itemize}
        \item Acknowledge the limitations of the chosen model system.
        \item Propose validation in multiple model systems if possible, including eventually animal, ex-vivo and, if applicable, in-vivo models.
        \item Carefully consider and justify the choice of control cell types.
        \item Propose isogenic controls where relevant.
    \end{itemize}
    \item \textbf{Technical Challenges:} Many ideas proposed experiments that are technically very challenging. Reviewers often raised concerns about feasibility and the potential for ambiguous results. Future ideas should:
    \begin{itemize}
        \item Demonstrate awareness of the technical hurdles.
        \item Propose realistic solutions and alternative approaches.
        \item Prioritize experiments that are most likely to yield clear, interpretable data.
    \end{itemize}
    \item \textbf{Specificity of Tools:} When using inhibitors, antibodies, or other tools, their specificity needs to be carefully considered and validated to avoid off-target effects that can confound results. Appropriate controls and validation experiments must be included.
    \item \textbf{Quantitative Rigor:} Many critiques centered on the need for more rigorous quantification of experimental results. Future ideas need to:
    \begin{itemize}
        \item Clearly define measurable outcomes.
        \item Describe the specific assays and techniques that will be used for quantification.
        \item Include appropriate statistical analysis plans.
        \item Have a proposed plan for controls, replicates, and sample sizes.
    \end{itemize}
\end{itemize}
\end{tcolorbox}
\vspace{0.1cm}
\caption{\textbf{An example of a critique generated by the Meta-review agent.} This is an example of the meta-review critique generated by the the co-scientist Meta-review agent for the Amyotrophic Lateral Sclerosis (ALS) mechanism research hypothesis}
\label{fig:ex_meta_critique}
\end{figure}

\begin{figure}[htbp!]
\begin{tcolorbox}[
    colback=black!5!white,
    colframe=black!60!white,
    title=\textbf{Example output of a critique by the Meta-review agent (continued)},
    fonttitle=\bfseries,
    arc=3mm,
    boxrule=1pt,
    bottomrule=2pt,
]
\footnotesize
\begin{itemize}
    \item \textbf{Time-Course Experiments:} Given the dynamic nature of cellular processes, incorporating time-course experiments is often crucial to understanding the sequence of events and demonstrating causality.
    \item \textbf{Distinguishing Cause from Effect:} This requires careful experimental design, including:
    \begin{itemize}
        \item Longitudinal studies to observe the temporal order of events.
        \item Intervention experiments to manipulate the proposed cause and observe the effects downstream.
        \item Temporal resolution to demonstrate that the potential cause occurs before a downstream event.
    \end{itemize}
\end{itemize}

\textbf{III. Novelty and Impact:}
\begin{itemize}
    \item \textbf{Incremental vs. Groundbreaking:} Many ideas were criticized for being incrementally novel, building upon existing knowledge rather than proposing truly groundbreaking mechanisms. 
    \item \textbf{"Key Switch" Claims:} Several ideas were criticized for prematurely claiming that the proposed mechanism is the "key switch" for ALS. This type of strong claim requires very strong supporting evidence. It is often better to frame the hypothesis in terms of a significant contributing factor and test its requirement for the pathology.
    \item \textbf{Therapeutic Potential:} While not always explicitly stated, the potential for translating the findings into therapeutic strategies is an important consideration. Ideas that offer a clear path toward therapeutic development may be viewed more favorably, provided a strong rationale is presented for the specific therapeutic approach.
\end{itemize}

\textbf{IV. Assumptions and Validation:}
\begin{itemize}
    \item \textbf{Strong Assumptions:} Many hypotheses rely on strong assumptions that lack direct experimental support. Future ideas need to:
    \begin{itemize}
        \item Clearly state all assumptions.
        \item Provide a strong rationale for each assumption, citing relevant literature.
        \item Propose experiments to directly test the most critical assumptions.
    \end{itemize}
    \item \textbf{Validation of Assumptions:}
    \begin{itemize}
        \item Reviewers consistently emphasized the need for rigorous experimental validation of all assumptions.
        \item Assumptions should be addressed in the order of the proposed mechanism.
        \item If the initial steps are not validated, it may be unnecessary to proceed to later steps.
    \end{itemize}
\end{itemize}

\textbf{V. General Advice Based on Common Critiques:}
\begin{itemize}
    \item \textbf{Mechanistic Detail:} Provide as much specific mechanistic detail as possible. Don't just state that pathway X is affected; describe how it is affected at the molecular level.
    \item \textbf{Literature Review:} Demonstrate a thorough understanding of the existing literature, including contradictory findings. Address how the proposed idea fits into the current understanding of ALS and how it differs from previous work.
    \item \textbf{Focus on Causality:} Always design experiments that can distinguish between cause and effect.
    \item \textbf{Feasibility:} Prioritize experiments that are feasible with available resources and expertise. Consider using a phased experimental approach. Prioritize the experiments used to validate assumptions in the order they appear in the proposed mechanism.
    \item \textbf{Strong Rationale:} Provide a clear and compelling rationale for every aspect of the hypothesis and experimental design.
    \item \textbf{Specificity of Controls:} Explain why the choices of controls used in the experimental design are the best choice to test the effects of interest. Demonstrate that the effect is not due to a more general effect or mechanism.
    \item \textbf{Address Limitations:} Acknowledge the limitations of the proposed approach and discuss potential alternative explanations.
    \item \textbf{Quantitative data:} Emphasize quantitative data, consider including mathematical modeling of data where appropriate.
\end{itemize}
\end{tcolorbox}
\vspace{0.1cm}
\caption{\textbf{An example of a critique generated by the Meta-review agent (continued).} This is an example of the meta-review critique generated by the co-scientist Meta-review agent for the Amyotrophic Lateral Sclerosis (ALS) mechanism research hypothesis.}
\label{fig:ex_meta_critique_2}
\end{figure}




\begin{figure}[htbp!]
\begin{tcolorbox}[
    colback=black!5!white,
    colframe=black!60!white,
    title=\textbf{Example of a research overview by the Meta-review agent},
    fonttitle=\bfseries,
    arc=3mm,
    boxrule=1pt,
    bottomrule=2pt,
]
\footnotesize
\textbf{[Main Research Directions]}
\\
We will be focusing on these interrelated areas:
\begin{itemize}
    \item \textbf{Mitochondrial Dysfunction and Oxidative Stress:} This direction investigates defects in mitochondrial DNA repair, mitochondrial dynamics, calcium handling, and oxidative stress response, focusing on their role in initiating inflammatory signaling and neurodegeneration.
    \item \textbf{RNA Processing and Stress Granules:} This direction explores abnormalities in RNA processing, transport, and modification, as well as impaired stress granule dynamics, and their impact on protein homeostasis and neuronal survival. It includes examining RNA-binding proteins, RNA structure and modifications, translation defects, and the interplay between RNA and proteostatic stress responses.
    \item \textbf{Protein Homeostasis and the Nucleocytoplasmic Transport:} This direction is aimed at investigating defects in protein folding, trafficking, aggregation, and degradation, as well as nucleocytoplasmic transport (NCT) dysfunction, with a particular focus on the nuclear pore complex (NPC). This includes how those pathways and processes influence, and are influenced by, TDP-43 and FUS protein function and localization.
    \item ...
\end{itemize}

\textbf{Mitochondrial Dysfunction and Oxidative Stress}
\begin{itemize}
    \item \textbf{Rationale:} Mitochondrial dysfunction and oxidative stress are consistently implicated in ALS pathogenesis. Mitochondria are critical for cellular energy production, and their dysfunction can lead to a cascade of events including increased oxidative stress, impaired calcium handling, inflammation, and ultimately cell death. Given the high energy requirements of motor neurons, especially upper motor neurons, these deficits are expected to have a significant impact. We hypothesize that specific mitochondrial vulnerabilities, not just generalized dysfunction play a very early role in initiating disease and provide a specific cellular event which can be targeted for research and for potential future therapy.
    \item \textbf{Recent Findings:} Research has shown that mutations in mitochondrial DNA (mtDNA) maintenance genes are associated with ALS. Further there are links between impaired mitochondrial calcium handling and oxidative stress, with a particular focus on base excision repair (BER) pathway defects and a potential link to activation of the cGAS-STING pathway. These findings suggest that a deeper understanding of mtDNA integrity, repair mechanisms, and ROS dynamics is critical for uncovering the initial drivers of ALS.
    \item \textbf{Areas of Research:}
    \begin{itemize}
        \item \textbf{Mitochondrial DNA Repair Defects:}
        \begin{itemize}
            \item \textit{Why Research?} Explore the possibility that a deficiency in base excision repair (BER) enzymes for mitochondrial DNA is a primary driver of ALS.
            \item \textit{What to Research?} Investigate the activity and expression of specific BER enzymes (e.g., OGG1) in iPSC-derived motor neurons from ALS patients and controls. Assess whether deficiencies correlate with increased accumulation of oxidized mtDNA lesions (e.g., 8-oxo-dG) and whether these can be released into the cytoplasm after VDAC or MOMP activation. Determine if this release activates the cGAS-STING pathway.
            \item \textit{Example Idea:} Measure OGG1 enzyme activity, oxidized mtDNA levels (8-oxo-dG) and cGAS-STING activation in iPSC-derived motor neurons. Genetically manipulate (knock-out/overexpress) key components of the mitochondrial BER pathway or target specific pathways using small molecules. Use VDAC inhibitors to prevent release or modulate MOMP function to understand cause and effect of ox-mtDNA release and cGAS-STING activation on cell death and inflammatory outcomes.
        \end{itemize}
        \item \textbf{Mitochondrial Calcium Handling and Oxidative Stress:}
        \begin{itemize}
            \item \textit{Why Research?} Motor neurons have high calcium demands and are sensitive to oxidative damage. Imbalances in mitochondrial calcium handling and ROS production may trigger a cascade leading to neurodegeneration.
            \item \textit{What to Research?} Examine the role of mitochondrial calcium uniporter (MCU) and other calcium transporters in ALS. Assess if dysregulated calcium influx into mitochondria leads to increased ROS production and whether this is different in different types of motor neurons. Study the potential role of calcium handling defects as a primary trigger or modifier of the disease.
            \item \textit{Example Idea:} Use calcium indicators and ROS sensors in iPSC-derived motor neurons to monitor real-time calcium and ROS levels in mitochondria. Manipulate mitochondrial calcium uptake via pharmacological or genetic approaches (e.g., MCU inhibitors or knock-outs). Assess cell survival and the impact on inflammatory response markers and activation of the cGAS-STING pathway to assess direct causality.
        \end{itemize}
        \item ...
    \end{itemize}
\end{itemize}
\end{tcolorbox}
\vspace{0.1cm}
\caption{\textbf{An abbreviated example of a research overview generated by the Meta-review agent.} This is an example of a research overview generated by the co-scientist Meta-review agent for the research goal pertaining to explaining the basic mechanism of Amyotrophic Lateral Sclerosis (ALS). In addition to the the individual hypotheses and proposals, this the main output presented to the scientist by the AI co-scientist system.}
\label{fig:ex_overview}
\end{figure}


\begin{figure}[htbp!]
\begin{tcolorbox}[
    colback=black!5!white,
    colframe=black!60!white,
    title=\textbf{Example of a research overview by the Meta-review agent (continued)},
    fonttitle=\bfseries,
    arc=3mm,
    boxrule=1pt,
    bottomrule=2pt,
]
\footnotesize
\textbf{RNA Processing and Stress Granules:}
\begin{itemize}
    \item ...
\end{itemize}

\textbf{Protein Homeostasis and the Nucleocytoplasmic Transport:}
\begin{itemize}
    \item \textbf{Rationale:} The proper folding, trafficking, degradation, and localization of proteins are critical for cellular function. Defects in these processes can lead to the accumulation of toxic protein aggregates, disrupt cellular signaling, and initiate pathways leading to cell death in ALS. The nucleus is a key compartment, and transport into and out of this organelle is essential for cellular function.
    \item \textbf{Recent Findings:} Research has identified misfolded TDP-43 and FUS as key pathological features of ALS. Furthermore, the dysregulation of nucleocytoplasmic transport (NCT) and the nuclear pore complex (NPC) is a recurring feature, with recent findings that emphasize defects in early stages of assembly and dynamic function of the NPC as an early initiating step.
    \item \textbf{Areas of Research:}
    \begin{itemize}
        \item \textbf{Ubiquitin-Proteasome System (UPS) Dysfunction:}
        \begin{itemize}
            \item \textit{Why Research?} The UPS is a major cellular protein degradation machinery. Defects in the UPS can lead to an accumulation of misfolded proteins, which could lead to cellular stress and death. Exploring the potential of a functional deficiency in these system as an early trigger in disease.
            \item \textit{What to Research:} Investigate how UPS functionality is altered in ALS models, with particular attention to the role of E3 ligases in TDP-43 and FUS degradation. Determine whether specific modifications on ALS related proteins impair their degradation and whether specific pathways of protein quality control are affected in early stages of the disease. Assess if this is a primary cause or secondary outcome of other disease related pathways.
            \item \textit{Example Idea:} Use cell-based assays to measure UPS activity and substrate degradation rates in iPSC-derived motor neurons. Conduct proteomic analysis to identify specific ubiquitinated proteins that accumulate in ALS cells. Genetically manipulate UPS components or use small molecules to modulate their function and assess cellular responses (e.g. measuring protein aggregation and cell survival)
        \end{itemize}
        \item \textbf{Autophagy Dysfunction:}
        \begin{itemize}
            \item \textit{Why Research?} Autophagy is crucial for clearing cellular debris and aggregates. Dysfunctional autophagy can lead to the accumulation of toxic proteins and can lead to organelle defects.
            \item \textit{What to Research?} Explore how autophagy pathways are altered in ALS, focusing on the specific mechanisms of cargo recognition, autophagosome formation, and lysosome fusion. Assess if specific types of autophagy including mitophagy (mitochondrial specific) or lipophagy (lipid specific) are more vulnerable in specific cell types.
            \item \textit{Example Idea:} Track the formation and degradation of autophagosomes in real-time in hiPSC-derived motor neurons after different stress conditions. Manipulate key effectors of the autophagy pathway and observe cellular response, measuring protein aggregation, cell viability and inflammatory markers.
        \end{itemize}
        \item \textbf{Nucleocytoplasmic Transport (NCT) Defects:}
        \begin{itemize}
            \item ...
        \end{itemize}
        \item \textbf{Targeted Protein misfolding:}
        \begin{itemize}
            \item ...
        \end{itemize}
    \end{itemize}
    \item ...
\end{itemize}
\end{tcolorbox}
\vspace{0.1cm}
\caption{\textbf{An abbreviated example of a research overview generated by the Meta-review agent (continued).} This is an example of a research overview generated by the co-scientist Meta-review agent for the research goal pertaining to explaining the basic mechanism of Amyotrophic Lateral Sclerosis (ALS). In addition to the the individual hypotheses and proposals, this the main output presented to the scientist by the co-scientist system.}
\label{fig:ex_overview_2}
\end{figure}


\begin{figure}[htbp!]
\begin{tcolorbox}[
    colback=black!5!white,
    colframe=black!60!white,
    title=\textbf{Example of research contact identification by the Meta-review agent},
    fonttitle=\bfseries,
    arc=3mm,
    boxrule=1pt,
    bottomrule=2pt,
]
\footnotesize
\textbf{Research Direction: Oxidative DNA Damage \& Mitochondrial Base Excision Repair (BER) in ALS}
\begin{itemize}
    \item \textbf{[Researcher names]:} They directly study the role of mitochondrial OGG1 (a key BER enzyme) in controlling cytosolic mtDNA release and neuroinflammation. Their expertise is highly valuable for understanding the link between BER, mtDNA, and inflammation. Also, they have experience with experiments using mtOGG1 overexpressing mice which is relevant to the in-vitro experiments proposed.
    \item ...
\end{itemize}
\end{tcolorbox}
\vspace{0.1cm}
\caption{\textbf{An example of a research contact identified by the Meta-review agent as a potential domain expert in the research topic and hypothesis of interest.}}
\label{fig:ex_contact}
\end{figure}

