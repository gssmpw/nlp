
\clearpage
% \onecolumn


\renewcommand{\thesection}{A.\arabic{section}}
\renewcommand{\thefigure}{A.\arabic{figure}}
\renewcommand{\thetable}{A.\arabic{table}} 
\renewcommand{\theequation}{A.\arabic{equation}} 
\renewcommand{\theHsection}{A\arabic{section}}

\setcounter{section}{0}
\setcounter{figure}{0}
\setcounter{table}{0}
\setcounter{equation}{0}
% \setcounter{page}{1}

% \appendix

\noindent \textbf{\LARGE{Appendix}}\\


In the following sections, we report additional details on the following topics:
\begin{itemize}[leftmargin=1.5em,rightmargin=0em]
    \item Glossary of terminology and concepts (\cref{sec:glossary})
    \item Prompts for the specialized agents in the AI co-scientist system (\cref{sec:prompts})
    \begin{itemize}
        \item Prompts for the Generation agent (\cref{sec:prompts_1_generation})
        \item Prompt for the Reflection agent (\cref{sec:prompts_2_reflection})
        \item Prompts for the Ranking agent (\cref{sec:prompts_3_ranking})
        \item Prompts for the Evolution agent (\cref{sec:prompts_4_evolution})
        \item Prompt for the Meta-review agent (\cref{sec:prompts_5_meta})
    \end{itemize}
    \item Examples of the AI co-scientist system inputs, intermediate outputs, and final results (\cref{sec:supp_ex})
    \item Supplementary information for drug repurposing evaluation (\cref{sec:supp_drug_repurposing})
    \begin{itemize}
        \item Datasets (\cref{sec:eval_data})
        \item Computational biology analyses (\cref{sec:eval_compbio})
        \item \textit{In vitro} wet-lab validation (\cref{sec:eval_wetlab})
        \item Additional wet-lab results (\cref{sec:result_wetlab})
    \end{itemize}
    \item Supplementary information for the co-scientist drug repurposing Specific Aims evaluation (\cref{sec:supp_aims})
    \begin{itemize}
        \item Specific Aims count distribution (\cref{sec:specific_aims_count})
        \item Specific Aims evaluation rubric (\cref{sec:specific_aims_rubric})
        \item Examples of co-scientist generated Specific Aims with expert evaluation (\cref{sec:ex_specific_aims})
    \end{itemize}
    \item Detailed AI co-scientist output for a validated AML repurposing candidate (\cref{sec:drug_output})
    \item An example of tool use in the AI co-scientist with AlphaFold (\cref{sec:alphafold})
\end{itemize}


\section{Glossary of terminology and concepts}
\label{sec:glossary}

\begin{itemize}
    \item \textbf{Novel repurposing candidate.} An existing drug (a chemical compound that binds to a target), with an established safety profile, proposed for use in a disease or condition for which it is not currently approved or widely used. This approach of repurposing existing drugs differs significantly from traditional drug discovery, which involves identifying novel chemical compounds that can bind to specific targets implicated in a disease state.
    \item \textbf{Novel target.} A biological entity (e.g., gene, protein, pathway) not previously known as a therapeutic target for a specific disease.
    \item \textbf{Novel mechanistic explanation.} A newly proposed pathway, interaction, or biological process that explains a phenomenon (e.g., disease progression, antimicrobial resistance) in a way that is not explicitly described in prior literature.
\end{itemize}


\clearpage
\section{Prompts for the specialized agents in the AI co-scientist system}
\label{sec:prompts}
\subsection{Prompts for the Generation agent}
\label{sec:prompts_1_generation}

\begin{figure}[htbp!]
\begin{tcolorbox}[
    colback=black!5!white,
    colframe=black!60!white,
    title=\textbf{Prompt for hypothesis generation after literature review},
    fonttitle=\bfseries,
    arc=3mm,
    boxrule=1pt,
    bottomrule=2pt,
]
\footnotesize
\begin{verbatim}

You are an expert tasked with formulating a novel and robust hypothesis to address 
the following objective.

Describe the proposed hypothesis in detail, including specific entities, mechanisms, 
and anticipated outcomes.

This description is intended for an audience of domain experts.

You have conducted a thorough review of relevant literature and developed a logical framework 
for addressing the objective. The articles consulted, along with your analytical reasoning, 
are provided below.

Goal: {goal}

Criteria for a strong hypothesis:
{preferences}

Existing hypothesis (if applicable):
{source_hypothesis}

{instructions}

Literature review and analytical rationale (chronologically ordered, beginning 
with the most recent analysis):

{articles_with_reasoning}

Proposed hypothesis (detailed description for domain experts):

\end{verbatim}
\end{tcolorbox}
\vspace{0.1cm}
\caption{\textbf{Example Generation agent prompt for hypothesis generation after literature review and relevant article exploration.}}
\label{fig:EXPLORE_ARTICLES_PROMPT}
\end{figure}


\begin{figure}[htbp!]
\begin{tcolorbox}[
    colback=black!5!white,
    colframe=black!60!white,
    title=\textbf{Prompt for hypothesis generation after scientific debate},
    fonttitle=\bfseries,
    arc=3mm,
    boxrule=1pt,
    bottomrule=2pt,
]
\footnotesize
\begin{verbatim}
You are an expert participating in a collaborative discourse concerning the generation 
of a {idea_attributes} hypothesis. You will engage in a simulated discussion with other experts. 
The overarching objective of this discourse is to collaboratively develop a novel 
and robust {idea_attributes} hypothesis.

Goal: {goal}

Criteria for a high-quality hypothesis:
{preferences}

Instructions:
{instructions}

Review Overview:
{reviews_overview}

Procedure:

Initial contribution (if initiating the discussion):
    Propose three distinct {idea_attributes} hypotheses.

Subsequent contributions (continuing the discussion):
    * Pose clarifying questions if ambiguities or uncertainties arise.
    * Critically evaluate the hypotheses proposed thus far, addressing the following aspects:
        -  Adherence to {idea_attributes} criteria.
        -  Utility and practicality.
        -  Level of detail and specificity.
    * Identify any weaknesses or potential limitations.
    * Propose concrete improvements and refinements to address identified weaknesses.
    * Conclude your response with a refined iteration of the hypothesis.

General guidelines:
    * Exhibit boldness and creativity in your contributions.
    * Maintain a helpful and collaborative approach.
    * Prioritize the generation of a high-quality {idea_attributes} hypothesis.

Termination condition:
    When sufficient discussion has transpired (typically 3-5 conversational turns, 
    with a maximum of 10 turns) and all relevant questions and points have been 
    thoroughly addressed and clarified, conclude the process by writing "HYPOTHESIS" 
    (in all capital letters) followed by a concise and self-contained exposition of the finalized idea.

#BEGIN TRANSCRIPT#
{transcript}
#END TRANSCRIPT#

Your Turn:
\end{verbatim}
\end{tcolorbox}
\vspace{0.1cm}
\caption{\textbf{Example Generation agent prompt for hypothesis generation after simulated scientific debate.}}
\label{fig:SELF_PLAY_PROMPT}
\end{figure}

\subsection{Prompt for the Reflection agent}
\label{sec:prompts_2_reflection}


\begin{figure}[htbp!]
\begin{tcolorbox}[
    colback=black!5!white,
    colframe=black!60!white,
    title=\textbf{Prompt for generating observations which can be explained by the hypothesis },
    fonttitle=\bfseries,
    arc=3mm,
    boxrule=1pt,
    bottomrule=2pt,
]
\footnotesize
\begin{verbatim}
You are an expert in scientific hypothesis evaluation. Your task is to analyze the 
relationship between a provided hypothesis and observations from a scientific article.
Specifically, determine if the hypothesis provides a novel causal explanation 
for the observations, or if they contradict it.

Instructions:

1.  Observation extraction: list relevant observations from the article.
2.  Causal analysis (individual): for each observation:
    a.  State if its cause is already established.
    b.  Assess if the hypothesis could be a causal factor (hypothesis => observation).
    c.  Start with: "would we see this observation if the hypothesis was true:".
    d.  Explain if it's a novel explanation. If not, or if a better explanation exists, 
        state: "not a missing piece."
3.  Causal analysis (summary): determine if the hypothesis offers a novel explanation 
    for a subset of observations. Include reasoning. Start with: "would we see some of
    the observations if the hypothesis was true:".
4.  Disproof analysis: determine if any observations contradict the hypothesis. 
    Start with: "does some observations disprove the hypothesis:".
5.  Conclusion: state: "hypothesis: <already explained, other explanations more likely, 
    missing piece, neutral, or disproved>".

Scoring:
    *   Already explained: hypothesis consistent, but causes are known. No novel explanation.
    *   Other explanations more likely: hypothesis *could* explain, but better explanations exist.
    *   Missing piece: hypothesis offers a novel, plausible explanation.
    *   Neutral: hypothesis neither explains nor is contradicted.
    *   Disproved: observations contradict the hypothesis.

Important: if observations are expected regardless of the hypothesis, and don't disprove it, 
it's neutral.

Article:
{article}

Hypothesis:
{hypothesis}

Response {provide reasoning. end with: "hypothesis: <already explained, other explanations 
more likely, missing piece, neutral, or disproved>".)
\end{verbatim}
\end{tcolorbox}
\vspace{0.1cm}
\caption{\textbf{Example Reflection agent prompt for generating observations from prior experimental results which can be explained by the hypothesis under consideration.}}
\label{fig:GENERATE_OBSERVATIONS_PROMPT}
\end{figure}

\subsection{Prompts for the Ranking agent}
\label{sec:prompts_3_ranking}


\begin{figure}[htbp!]
\begin{tcolorbox}[
    colback=black!5!white,
    colframe=black!60!white,
    title=\textbf{Prompt for hypothesis comparison during tournament},
    fonttitle=\bfseries,
    arc=3mm,
    boxrule=1pt,
    bottomrule=2pt,
]
\footnotesize
\begin{verbatim}
You are an expert evaluator tasked with comparing two hypotheses.

Evaluate the two provided hypotheses (hypothesis 1 and hypothesis 2) and determine which one 
is superior based on the specified {idea_attributes}.  
Provide a concise rationale for your selection, concluding with the phrase "better idea: <1 or 2>".

Goal: {goal}

Evaluation criteria:
{preferences}

Considerations:
{notes}
Each hypothesis includes an independent review. These reviews may contain numerical scores. 
Disregard these scores in your comparative analysis, as they may not be directly comparable across reviews.

Hypothesis 1:
{hypothesis 1}

Hypothesis 2:
{hypothesis 2}

Review of hypothesis 1:
{review 1}

Review of hypothesis 2:
{review 2}

Reasoning and conclusion (end with "better hypothesis: <1 or 2>"):
\end{verbatim}
\end{tcolorbox}
\vspace{0.1cm}
\caption{\textbf{Example Ranking agent prompt for hypothesis comparison during tournament.}}
\label{fig:COMPARE_IDEAS_PROMPT}
\end{figure}


\begin{figure}[htbp!]
\begin{tcolorbox}[
    colback=black!5!white,
    colframe=black!60!white,
    title=\textbf{Prompt for hypothesis comparison via simulated scientific debate during tournament},
    fonttitle=\bfseries,
    arc=3mm,
    boxrule=1pt,
    bottomrule=2pt,
]
\footnotesize
\begin{verbatim}
You are an expert in comparative analysis, simulating a panel of domain experts 
engaged in a structured discussion to evaluate two competing hypotheses.
The objective is to rigorously determine which hypothesis is superior based on 
a predefined set of attributes and criteria.  
The experts possess no pre-existing biases toward either hypothesis and are solely 
focused on identifying the optimal choice, given that only one can be implemented.

Goal: {goal}

Criteria for hypothesis superiority:
{preferences}

Hypothesis 1:
{hypothesis 1}

Hypothesis 2:
{hypothesis 2}

Initial review of hypothesis 1:
{review1}

Initial review of hypothesis 2:
{review 2}

Debate procedure:

The discussion will unfold in a series of turns, typically ranging from 3 to 5, with a maximum of 10.

Turn 1:  begin with a concise summary of both hypotheses and their respective initial reviews.

Subsequent turns:

    *   Pose clarifying questions to address any ambiguities or uncertainties.
    *   Critically evaluate each hypothesis in relation to the stated Goal and Criteria.  
    This evaluation should consider aspects such as:
        -   Potential for correctness/validity.
        -   Utility and practical applicability.
        -   Sufficiency of detail and specificity.
        -   Novelty and originality.
        -   Desirability for implementation.
    *   Identify and articulate any weaknesses, limitations, or potential flaws in either hypothesis.

Additional notes:
{notes}

Termination and judgment:

Once the discussion has reached a point of sufficient depth (typically 3-5 turns, up to 10 turns) 
and all relevant questions and concerns have been thoroughly addressed, provide a conclusive judgment.  
This judgment should succinctly state the rationale for the selection.  
Then, indicate the superior hypothesis by writing the phrase "better idea: ", 
followed by "1" (for hypothesis 1) or "2" (for hypothesis 2).
\end{verbatim}
\end{tcolorbox}
\vspace{0.1cm}
\caption{\textbf{Example Ranking agent prompt for hypothesis comparison via simulated scientific debate during tournament.}}
\label{fig:REVISED_PROMPT}
\end{figure}

\subsection{Prompts for the Evolution agent}
\label{sec:prompts_4_evolution}


\begin{figure}[htbp!]
\begin{tcolorbox}[
    colback=black!5!white,
    colframe=black!60!white,
    title=\textbf{Prompt for hypothesis feasibility improvement},
    fonttitle=\bfseries,
    arc=3mm,
    boxrule=1pt,
    bottomrule=2pt,
]
\footnotesize
\begin{verbatim}
You are an expert in scientific research and technological feasibility analysis. 
Your task is to refine the provided conceptual idea, enhancing its practical implementability 
by leveraging contemporary technological capabilities. Ensure the revised concept retains 
its novelty, logical coherence, and specific articulation.

Goal: {goal}

Guidelines:
1. Begin with an introductory overview of the relevant scientific domain.
2. Provide a concise synopsis of recent pertinent research findings and related investigations, 
   highlighting successful methodologies and established precedents.
3. Articulate a reasoned argument for how current technological advancements can facilitate 
   the realization of the proposed concept.
4. CORE CONTRIBUTION: Develop a detailed, innovative, and technologically viable alternative 
   to achieve the objective, emphasizing simplicity and practicality.

Evaluation Criteria:
{preferences}

Original Conceptualization:
{hypothesis}

Response:
\end{verbatim}
\end{tcolorbox}
\vspace{0.1cm}
\caption{\textbf{Example Evolution agent prompt for hypothesis feasibility improvement.}}
\label{fig:FEASIBILITY_IDEA_PROMPT}
\end{figure}


\begin{figure}[htbp!]
\begin{tcolorbox}[
    colback=black!5!white,
    colframe=black!60!white,
    title=\textbf{Prompt for hypothesis generation through out-of-the-box thinking},
    fonttitle=\bfseries,
    arc=3mm,
    boxrule=1pt,
    bottomrule=2pt,
]
\footnotesize
\begin{verbatim}
You are an expert researcher tasked with generating a novel, singular hypothesis 
inspired by analogous elements from provided concepts.

Goal: {goal}

Instructions:
1. Provide a concise introduction to the relevant scientific domain.
2. Summarize recent findings and pertinent research, highlighting successful approaches.
3. Identify promising avenues for exploration that may yield innovative hypotheses.
4. CORE HYPOTHESIS: Develop a detailed, original, and specific single hypothesis 
   for achieving the stated goal, leveraging analogous principles from the provided 
   ideas. This should not be a mere aggregation of existing methods or entities. Think out-of-the-box.

Criteria for a robust hypothesis:
{preferences}

Inspiration may be drawn from the following concepts (utilize analogy and inspiration, 
not direct replication):
{hypotheses}

Response:
\end{verbatim}
\end{tcolorbox}
\vspace{0.1cm}
\caption{\textbf{Example Evolution agent prompt for hypothesis generation through out-of-the-box thinking.}}
\label{fig:OUT_OF_THE_BOX_SINGLE_IDEA_PROMPT}
\end{figure}

\subsection{Prompt for the Meta-review agent}
\label{sec:prompts_5_meta}


\begin{figure}[htbp!]
\begin{tcolorbox}[
    colback=black!5!white,
    colframe=black!60!white,
    title=\textbf{Prompt for meta-review generation},
    fonttitle=\bfseries,
    arc=3mm,
    boxrule=1pt,
    bottomrule=2pt,
]
\footnotesize
\begin{verbatim}
You are an expert in scientific research and meta-analysis. 
Synthesize a comprehensive meta-review of provided reviews 
pertaining to the following research goal:

Goal: {goal}

Preferences:
{preferences}

Additional instructions:
{instructions}

Provided reviews for meta-analysis:
{reviews}

Instructions:
    * Generate a structured meta-analysis report of the provided reviews. 
    * Focus on identifying recurring critique points and common issues raised by reviewers. 
    * The generated meta-analysis should provide actionable insights for researchers
      developing future proposals. 
    * Refrain from evaluating individual proposals or reviews;
      focus on producing a synthesized meta-analysis.

Response: 
\end{verbatim}
\end{tcolorbox}
\vspace{0.1cm}
\caption{\textbf{Example Meta-review agent prompt for meta-review generation from existing reviews.}}
\label{fig:REVIEWS_PROMPT_TEMPLATE}
\end{figure}


\clearpage

\begin{figure*}[ht!]
    \centering
    \includegraphics[width=0.99\linewidth]{plots/squares_p4_and_z1z2_d5000_10seeds_overlaps.pdf}
    \caption{Overlap $\normf{\bM}^2 / \Tr(\bQ)$ as a function of the sample complexity $\alpha$. The dots represent numerical simulation results, computed for $n = 5000$ (for the asymmetric method) or $d = 5000$ (for the symmetric method) and averaging over $10$ instances. (\textbf{Left}) Link function $g(z_1,z_2) = z_1z_2$. Solid lines are obtained from state evolution predictions eq. (\ref{eq:overlap_prod_zk},\ref{eq:examples_symmetric_general}). Dashed line at $\alpha_c \approx 0.59375$. (\textbf{Right}) Link function $g(\bz) = p^{-1}\norm{\bz}^2$, $p = 4$. Solid lines are obtained from state evolution predictions eq. (\ref{eq:overlap_asymmetric_squares},\ref{eq:examples_symmetric_general}). Dashed line at $\alpha_c = 2$.}
    \label{fig:z1z2_overlap}
\end{figure*}
In this section we illustrate the framework introduced in Section \ref{sec:main_results} to predict the asymptotic performance of the spectral estimators (\ref{eq:def:spectral_asymmetric},\ref{eq:def:spectral_symmetric}) for specific examples of link functions, providing a comparison between our asymptotic analytical results and finite size numerical simulations for the overlap between the spectral estimators and the weights $\mat{W}_\star$, defined as $m \coloneqq \nicefrac{\normf{\bM}}{\sqrt{\Tr(\bQ)}}$, where $\bM$ and $\bQ$ are the overlap matrices defined in eq. (\ref{eq:def:overlaps_amp}) correspondent to the fixed points in Lemmas \ref{result:1}, \ref{result:2}, \ref{result:3}, \ref{result:4}. In Figure \ref{fig:z1z2_overlap} we compare these theoretical predictions to numerical simulations at finite dimensions, respectively for the link functions $g(z_1,z_2) = z_1z_2$ and $g(\bz) = p^{-1}\norm{\bz}^2$. Additional numerical experiments are presented in Appendix \ref{app:example_details}.
\subsection{Asymmetric spectral method}\label{sec:examples_asymmetric}
%\subsection{$\dgout(y)$ jointly diagonalizable $\forall y$}
We provide closed-form expressions for the overlap parameter $m \coloneqq \nicefrac{\normf{\bM}}{\sqrt{\Tr(\bQ)}}$ of the spectral estimator $\matwhat_{\tens{L}}$ (\ref{eq:def:spectral_asymmetric}), for a selection of examples of link functions. The details of the derivation are given in Appendix \ref{app:example_details}.
\begin{itemize}[leftmargin=2em,wide=1pt]
    \item $g(z\in\R)$ (single-index model):
    \begin{align}
        \alpha_c &= \left(\E_{\rdm{y}\sim\Zout}\left[\left(\Var[z\big|\rdm{y}] - 1\right)^2\right]\right)^{-1},\\
        m^2 &= \Bigg(\frac{\alpha - \alpha_c}{\alpha + \alpha_c^2\E_{\rdm{y}\sim\Zout}\left[\left(\Var[z\big|\rdm{y}] - 1\right)^3\right]}\Bigg)_+
    \end{align}
    \item $g(\bz) = p^{-1}||\bz||^2$:
    \begin{equation}\label{eq:overlap_asymmetric_squares}
        \alpha_c = \frac{p}{2},\quad m^2 = \left(\frac{(2\alpha-p)}{2(\alpha+2)}\right)_+
    \end{equation}
    \item $g(\bz) = \operatorname{sign}(z_1z_2)$: 
    \begin{equation}
        \alpha_c = \frac{\pi^2}{4},\quad m^2 = \left(1-\frac{\pi^2}{4\alpha}\right)_+
    \end{equation}
    \item $g(\bz) = \prod_{k=1}^pz_k$:
   \begin{align}\label{eq:overlap_prod_zk}
        \alpha_c &= \left(\E_{\rdm{y}\sim\Zout}\left[\lambda(\rdm{y})^2\right]\right)^{-1},\\
        m^2 &= \left(\frac{\alpha - \alpha_c}{\alpha + \alpha_c^2\E_{\rdm{y}\sim\Zout}\left[\lambda(\rdm{y})^3\right]}\right)_+,
    \end{align}
    where
    \begin{equation}\label{eq:lambda_prod_zk}
        \lambda(y) = \begin{cases}
            \begin{array}{ll}
              |y|\frac{K_1(|y|)}{K_0(|y|)} +\rdm{y}- 1,   & p = 2 \\
              \frac{2G^{p, 0}_{0, p} \left( y^22^{-p}  \, \bigg| \, \begin{array}{c}
0 \\
\vect{e}_p
\end{array} \right)}{ G^{p, 0}_{0, p} \left( y^22^{-p} \, \bigg| \, \begin{array}{c}
0 \\
\bzero_p
\end{array} \right)} - 1,   & p\geq 3
            \end{array}
        \end{cases}
    \end{equation}
    and the previous expression are written in terms of the modified Bessel function of the second kind and Meijer $G$-function, with the notations $\bzero_p\in\R^p = (0,\ldots,0)^T$ and $\vect{e}_p\in\R^p=(0,\ldots,0,1)^T$.
    \item $g(z_1,z_2) = z_1z_2^{-1}$: $
        \alpha_c = 1,\quad m^2 = (1 -\alpha^{-1})_+$
\end{itemize}

\subsection{Symmetric spectral method}
We provide expressions for the overlap parameter $m \coloneqq \nicefrac{\normf{\bM}}{\sqrt{\Tr(\bQ)}}$ of the spectral estimator $\matwhat_{\tens{T}}$ (\ref{eq:def:spectral_symmetric}), for a selection of examples of link functions.
In all the following cases, the state evolution equations simplify, allowing to write the results as functionals of $\lambda:\R\to\R$, which is a specific to each problem:
\begin{itemize}
    \item $g(z\in\R)$ (single-index model): $\lambda(y) = \Var[z\big|y] - 1$;
    
    \item $g(\bz) = p^{-1}||\bz||^2$: $\lambda(y) = y - 1$;
    \item $g(\bz) = \operatorname{sign}(z_1z_2)$: $\lambda(y) = 2\pi^{-1}y$;
    \item $g(\bz) = \prod_{k=1}^pz_k$: $\lambda(y)$ defined in eq. (\ref{eq:lambda_prod_zk}).
\end{itemize}
For all these example, the value $\alpha_c$ can be found in Section \ref{sec:examples_asymmetric}.
For $\alpha>\alpha_c$, consider $a$ and $\gamma$ solutions of
\begin{align}\label{eq:examples_symmetric_a}
&\E_{\rdm{y}\sim\Zout}\left[\frac{\lambda(\rdm{y})^2}{a(1 + \lambda(\rdm{y})) - \lambda(\rdm{y}) }\right] = \frac{1}{\alpha}\\
\label{eq:examples_symmetric_gamma}
&\gamma = 1 + \alpha\E_{\rdm{y}\sim\Zout}\left[\frac{\lambda(\rdm{y})}{a(1 + \lambda(\rdm{y})) - \lambda(\rdm{y}) }\right].
\end{align}
Then, for any $\alpha$, the overlap $m \coloneqq \nicefrac{\normf{\bM}}{\sqrt{\Tr(\bQ)}} $ is given by
\begin{equation}\label{eq:examples_symmetric_general}
    m^2 = \Bigg(\frac{1 - \alpha\E_{\rdm{y}\sim\Zout}[\lambda^2(\rdm{y})\left(a(1 + \lambda(\rdm{y})) - \lambda(\rdm{y})\right)^{-2}]}{1 + \alpha\E_{\rdm{y}\sim\Zout}[\lambda^3(\rdm{y})\left(a(1+ \lambda(\rdm{y})) - \lambda(\rdm{y})\right)^{-2}] } \Bigg)_+,
\end{equation}
which is strictly positive $\forall \alpha > \alpha_c$. Additional details on the derivation of this result can be found in Appendix \ref{app:details_examples_symmetric}.



\section{Supplementary information for drug repurposing evaluation}
\label{sec:supp_drug_repurposing}

\subsection{Datasets}
\label{sec:eval_data}

\paragraph{Cancer type dataset}
A list of 33 cancer types, along with their corresponding abbreviations, was constructed based on a curated text string derived from the \href{https://www.cancer.gov/tcga}{TCGA project} (Appendix \cref{tab:cancer_type}). This list includes 10 rare cancers.

\begin{table}[ht!]
\centering
\begin{tabular}{cc}
\toprule
\textbf{Cancer Name Abbreviation} & \textbf{Cancer Name} \\
\midrule
LAML & Acute Myeloid Leukemia \\
ACC & Adrenocortical carcinoma \\
BLCA & Bladder Urothelial Carcinoma \\
LGG & Brain Lower Grade Glioma \\
BRCA & Breast invasive carcinoma \\
CESC & Cervical squamous cell carcinoma and endocervical adenocarcinoma \\
CHOL & Cholangiocarcinoma \\
LCML & Chronic Myelogenous Leukemia \\
COAD & Colon adenocarcinoma \\
CNTL* & Controls \\
ESCA & Esophageal carcinoma \\
FPPP & FFPE Pilot Phase II \\
GBM & Glioblastoma multiforme \\
HNSC & Head and Neck squamous cell carcinoma \\
KICH & Kidney Chromophobe \\
KIRC & Kidney renal clear cell carcinoma \\
KIRP & Kidney renal papillary cell carcinoma \\
LIHC & Liver hepatocellular carcinoma \\
LUAD & Lung adenocarcinoma \\
LUSC & Lung squamous cell carcinoma \\
DLBC & Lymphoid Neoplasm Diffuse Large B-cell Lymphoma \\
MESO & Mesothelioma \\
MISC* & Miscellaneous \\
OV & Ovarian serous cystadenocarcinoma \\
PAAD & Pancreatic adenocarcinoma \\
PCPG & Pheochromocytoma and Paraganglioma \\
PRAD & Prostate adenocarcinoma \\
READ & Rectum adenocarcinoma \\
SARC & Sarcoma \\
SKCM & Skin Cutaneous Melanoma \\
STAD & Stomach adenocarcinoma \\
TGCT & Testicular Germ Cell Tumors \\
THYM & Thymoma \\
THCA & Thyroid carcinoma \\
UCS & Uterine Carcinosarcoma \\
UCEC & Uterine Corpus Endometrial Carcinoma \\
UVM & Uveal Melanoma \\
\bottomrule
\end{tabular}
\vspace{0.1cm}
\caption{\textbf{TCGA cancer type.} We consider all cancer types except CNTL and MISC while exploring the drug repurposing candidates with preclinical evidences. *Not a cancer type.}
\label{tab:cancer_type}
\end{table}

\paragraph{Drug repurposing proposals.}
We used the co-scientist to generate drug repurposing proposals for different types of cancer from a subset of 2300 drugs curated with the \href{https://platform.opentargets.org/downloads}{Open Targets Platform}. The proposal for each drug candidate includes a generated hypothesis, a review of the hypothesis, and a description of the possible mechanism of action. For each drug, the co-scientist assigned a drug repurposing likelihood score indicating how likely this drug can be successfully repurposed for a given cancer across all 33 TCGA cancer types.


\subsection{Computational biology analyses}
\label{sec:eval_compbio}
To enhance our drug repurposing hypothesis generation, verification, and evaluation processes, we integrated a computational biology approach utilizing DepMap (Cancer Dependency Map). DepMap is a project focused on identifying genetic vulnerabilities in cancer cells~\citep{depmap2024}. By systematically perturbing genes in a large panel of cancer cell lines using techniques like CRISPR and observing the resulting phenotypic effects, researchers are able to construct a comprehensive map of genetic dependencies for different cancers. This resource can be leveraged to identify promising drug targets for cancer therapies.

Specifically, DepMap generates perturbational data from CRISPR screens that are subsequently used to calculate DepMap dependency scores. These scores represent the probability of gene essentiality for a given cancer cell line. A high DepMap score indicates a high probability that a gene is required for cell survival and proliferation, thereby highlighting it as a potential therapeutic target. A DepMap score of 1.0 indicates a very strong signal, and any novel drug repurposing hypotheses with a high DepMap score would be interesting targets to explore.

For our experiments, we utilized the Q2 2024 data release of DepMap and used the default DepMap dependency probability. We first observed a significant correlation between DepMap score and the co-scientist score. As shown in Appendix~\cref{fig:cos_depmap}, drug-cancer pairs scored highly by the co-scientist also have high average DepMap score. We then leveraged DepMap data to assess the novelty of high-scoring proposals generated by the co-scientist. Candidates selected for expert review were required to meet stringent criteria, exhibiting both co-scientist review score $\geq 4$ and a DepMap score $\geq 0.99$.

\begin{figure}[htbp!]
\centering
\includegraphics[width=0.6\textwidth,keepaspectratio]{figures/fig_cos_depmap.png}
\vspace{0.1cm}
\caption{\textbf{The co-scientist review score is concordant with the DepMap score.} We demonstrate the distribution of the DepMap score across five AI co-scientist review score groups. The group with the co-scientist score of 5 has the highest median DepMap score, and the group with co-scientist score of 1 has the lowest. All pairwise comparisons between the group of AI co-scientist score of 5 and each of the other groups are statistically significant (Two-sided Wilcoxon rank-sum test, $p < 0.001$).}
\label{fig:cos_depmap}
\end{figure}


\subsection{\textit{In vitro} wet-lab validation}
\label{sec:eval_wetlab}
We tested expert selected drug candidates for AML repurposing by measuring the half-maximal inhibitory concentration (IC50), the concentration of drug required to inhibit cell viability by 50\%, of each drug on representative AML cell lines for the respective cancer indication. IC50 is a value commonly used as a way to quantify the effectiveness of a drug at inhibiting cellular processes, and can be measured by treating cells across a broad range of drug concentrations and fitting the dose response to a sigmoidal curve to determine 50\% of maximal inhibition~\cite{sebaugh2011guidelines}. For our measurements, cells were plated in 96-well cell culture-treated plates, seeding 5000 cells per well, and treated with the respective drug concentrations for 48 hours. Subsequently, cell viability was assayed using an MTS assay (CellTiter 96 AQueous One Solution Cell Proliferation Assay, Promega). After 1 hour incubation with the MTS reagent at 37°C, absorbance at 490 nm was measured in each well using a plate reader. Each condition was performed in triplicate. Dose response curves were background corrected (subtracted the smallest response), normalized (divided by the largest response), fit, and plotted using the Quest Graph™ IC50 Calculator (AAT Bioquest, Inc). The cell lines used (MOLM13, HL60, and KG-1) were generous gifts from the lab of Dr. Ravi Majeti (Stanford University).
\begin{itemize}
    \item MOLM-13 cells are a human AML cell line that were derived from bone marrow samples of a 62-year-old woman with AML.
    \item HL60 cells are a human cell line from a patient with acute promyelocytic leukemia (APL), often used to study the development and proliferation of normal and leukemic cells. 
    \item KG-1 cells are a human cell line isolated from the bone marrow aspirate of a 59-year-old, white male with erythroleukemia that evolved into AML.
\end{itemize}


\subsection{Additional wet-lab results}
\label{sec:result_wetlab}
This section provides additional \textit{in vitro} laboratory results of drug repurposing. Appendix \cref{fig:fig_wetlab_appendix} shows dose-response curves of Binimetinib in KG-1 and HL-60 cell lines. Appendix \cref{fig:fig_wetlab_appendix_failed} shows the does response curves of drug repurposing candidates for AML suggested by the AI co-scientist with little to no effect on MOLM-13 cells.

\begin{figure}[htbp!]
\centering
\includegraphics[width=0.8\textwidth,keepaspectratio]{figures/fig_bini.pdf}
\vspace{0.1cm}
\caption{\textbf{Dose-response of drug repurposing candidate Binimetinib in other AML cell lines.} Binimetinib demonstrates activity inhibiting cell viability in KG-1 and HL-60 cell lines. X-axis is the drug concentration (nM), and Y-axis is normalized cell viability (arbitrary unit).}
\label{fig:fig_wetlab_appendix}
\end{figure}

\begin{figure}[ht!]
\centering
\includegraphics[width=\textwidth,keepaspectratio]{figures/fig_failed.pdf}
\vspace{0.1cm}
\caption{\textbf{Dose-response of the drug repurposing candidates with little to no effect on MOLM-13.} Of the expert-selected drug repurposing candidates, Pravastatin and DMF showed little to no effect on the MOLM-13 cell line across the concentrations tested. Of the novel drug repurposing candidates, Leflunomide and Nanvuralat showed little to no effect on the MOLM-13 cell line across the concentrations tested. X-axis is the drug concentration (nM), and Y-axis is normalized cell viability (arbitrary unit).}
\label{fig:fig_wetlab_appendix_failed}
\end{figure}



\clearpage
\section{Supplementary information for the co-scientist drug repurposing Specific Aims evaluation}
\label{sec:supp_aims}

\subsection{Specific Aims count distribution}
\label{sec:specific_aims_count}

\begin{table}[htbp!]
\centering
\begin{tabular}{cc}
\toprule
\textbf{Repurposed Cancer} & \textbf{Count} \\
\midrule
Acute Myeloid Leukemia (LAML) & 13 \\
Colon adenocarcinoma (COAD) & 10 \\
Breast invasive carcinoma (BRCA) & 10 \\
Skin Cutaneous Melanoma (SKCM) & 8 \\
Lung adenocarcinoma (LUAD) & 6 \\
Head and Neck squamous cell carcinoma (HNSC) & 5 \\
Bladder Urothelial Carcinoma (BLCA) & 5 \\
Pancreatic adenocarcinoma (PAAD) & 4 \\
Stomach adenocarcinoma (STAD) & 3 \\
Rectum adenocarcinoma (READ) & 3 \\
Esophageal carcinoma (ESCA) & 3 \\
Uveal Melanoma (UVM) & 2 \\
Uterine Corpus Endometrial Carcinoma (UCEC) & 2 \\
Prostate adenocarcinoma (PRAD) & 2 \\
Lung squamous cell carcinoma (LUSC) & 1 \\
Cervical squamous cell carcinoma and endocervical adenocarcinoma (CESC) & 1 \\
\midrule
Grand Total & 78 \\
\bottomrule
\end{tabular}
\vspace{0.1cm}
\caption{\textbf{Count distribution of the cancer types in the Specific Aims drug repurposing proposals generated by the AI co-scientist.} }
\end{table}


\subsection{Specific Aims evaluation rubric}
\label{sec:specific_aims_rubric}
A pilot evaluation framework was developed by oncologists at a US institute, inspired by the axes used in evaluating research proposals in their field (for example, the NIH Specific Aims review criteria). While such original criteria serve as a gold standard for the review of in-depth proposals written by human domain experts, a context-specific approach is needed for AI systems such as the AI co-scientist because there are AI-specific limitations to consider.

The evaluation framework therefore focused on enabling assessment of similar axes of proposal quality viewed to be important by collaborating oncologists, while additionally accounting for unique limitations and capabilities of LLM-based systems.

For instance, while precedent real-world criteria assume deep domain expertise and the ability to cite contemporary unpublished work, the pilot framework presented here emphasizes the evaluation of logical consistency and the appropriate use of publicly available scientific knowledge. This acknowledges that LLMs, unlike human experts, may struggle to draw upon tacit knowledge or recent unpublished findings in their field.

The resulting pilot framework explores two domains: (1) clinical significance and (2) scientific rigor and methodology, with 5 and 10 questions respectively, for a total of 15 evaluation criteria.

A 5-point Likert-type scale is used for each item (Strongly Disagree, Disagree, Neutral, Agree and Strongly Agree). The approach used here represents a pilot framework only and would require considerable further research if it were to be developed into a reproducible or valid ratings instrument, including assessment of validity and reliability, inter and intra-observer variance, and correlation of ratings with ground truths or well-established prior frameworks not designed to evaluate AI systems.

The ``clinical significance'' domain addressed some fundamental aspects of drug repurposing proposals, including unmet clinical needs, therapeutic landscape analysis, and scientific rationale. Items in this domain aimed to assess the clinical relevance and theoretical foundation of proposals, and how AI systems integrate and synthesize publicly available information.

The ``scientific rigor and methodology'' domain aimed to evaluate technical quality of proposals, including hypothesis formation, methodological clarity, translational potential, overall scientific accuracy as reflected by the quality of preclinical experimental design and the presence of well-defined clinical endpoints. Special attention was paid to evaluating LLM-specific concerns, such as the assessment of factual accuracy and the avoidance of hallucinated content. This pilot framework is not comprehensive and is not intended to mirror the full breadth and depth of grant design evaluations. For example, omitted axes of quality could be addressed by future works, including absence of detailed criteria for experimental design (for example, review of power calculation or study design methodologies).

\subsection*{Significance and innovation (5 questions)}
\begin{enumerate}
    \item The proposal adequately identifies significant unmet clinical needs that could be addressed through drug repurposing of this pharmacological agent.
    \item The proposal effectively bridges an important gap in the current therapeutic landscape for the target disease by repurposing this drug.
    \item The proposal presents a scientifically rigorous rationale for repurposing the drug, grounded in current evidence and literature.
    \item The scientific background integrates relevant prior studies and preliminary data to support the proposed drug repurposing.
    \item The proposal avoids over-extrapolation or speculative conclusions beyond the supporting evidence.
\end{enumerate}

\subsection*{Rigor and feasibility (10 questions)}
\begin{enumerate}
    \item Does each Specific Aim have a clear hypothesis, and specific methodological approaches?
    \item Are the Specific Aims clearly stated and logically organized?
    \item There is a clear path from the proposed research to clinical application, including consideration of necessary pre-clinical and clinical studies.
    \item The proposal includes well-defined and clinically relevant endpoints.
    \item The pre-clinical experiments are designed to yield meaningful insights into the drug's potential efficacy and safety in its new therapeutic role.
    \item The proposal includes a well-described translational component to assess the drug's efficacy in its new indication.
    \item The proposal avoids factual inaccuracies or hallucinations, presenting information that is accurate and reliable.
    \item The assumptions made within the proposal regarding the drug's efficacy, mechanism, and potential outcomes are grounded in current scientific evidence.
    \item The proposal demonstrates originality in thought and approach, employs precise and appropriate scientific terminology and avoids ambiguous or generic statements.
    \item The proposal is clearly written, with well-organized sections and a logical flow of ideas that enhances comprehension and ensures the aims are easily understood.
\end{enumerate}

\newpage

\vspace{-0.2cm}
\subsection{Examples of co-scientist generated Specific Aims with expert evaluation}
\label{sec:ex_specific_aims}

\vspace{-0.2cm}
\begin{figure}[htbp!]
\caption{\textbf{Example of co-scientist generated Specific Aims drug repurposing proposal for AML.}}
\label{fig:aim1_1}
\vspace{0.1cm}
\begin{tcolorbox}[title=Givosiran for AML]
\footnotesize
\textbf{Disease Description}\\
Acute myeloid leukemia (AML) is an aggressive hematological malignancy with an incidence rate of approximately 4.3 per 100,000 persons per year in the United States, predominantly affecting older adults.  AML arises from a complex interplay of genetic mutations, epigenetic alterations, and dysregulated signaling pathways, disrupting normal hematopoiesis by blocking myeloid differentiation and promoting uncontrolled blast proliferation.  This leads to bone marrow failure, infections, bleeding, and other life-threatening complications.  The current standard of care includes intensive chemotherapy, often combined with targeted therapies or hematopoietic stem cell transplantation.  However, these treatments are often associated with significant toxicity, high relapse rates, and limited efficacy in certain patient populations, particularly in relapsed/refractory or high-risk AML.\\

\textbf{Unmet Need}\\
Despite advancements, significant unmet needs persist in AML treatment.  Current therapies often cause severe side effects, particularly in older or frail patients, limiting their tolerability and effectiveness.  Relapse rates remain high, and treatment options for relapsed/refractory AML are limited and often less effective.  Many patients fail to achieve complete remission or experience only short-lived responses, underscoring the urgent need for novel, less toxic, and more effective therapies, especially for patients with relapsed/refractory or high-risk disease.\\

\textbf{Proposed Solution}\\
Givosiran sodium is an RNA interference (RNAi) therapeutic approved for acute hepatic porphyria (AHP). It targets aminolevulinate synthase 1 (ALAS1) mRNA, the rate-limiting enzyme in heme biosynthesis, reducing the production of heme precursors $\delta$-aminolevulinic acid (ALA) and porphobilinogen. This prevents the accumulation of neurotoxic heme intermediates in AHP.

Repurposing givosiran for AML stems from the crucial role of heme biosynthesis in AML pathogenesis.  Several studies suggest that disrupting heme biosynthesis offers a therapeutic advantage in AML. The proposed approach addresses the unmet need by targeting AML cells dependent on increased heme biosynthesis, particularly those with MYCN overexpression [1]. Modulating heme levels could influence oxidative stress [5, 6], apoptosis [1, 2], and drug sensitivity in AML [2]. Givosiran's ALAS1 inhibition aligns with this approach, offering a novel therapeutic strategy.

We hypothesize that givosiran, by inhibiting ALAS1 and reducing heme biosynthesis, will suppress AML cell growth and survival, particularly in those with upregulated heme biosynthesis. This is supported by preclinical evidence demonstrating that altering heme levels impacts AML cell proliferation, apoptosis, and drug sensitivity [2]. Abstract [1] suggests that elevated heme biosynthesis in MYCN-driven AML is a therapeutic vulnerability.  Reducing ALA and porphobilinogen accumulation via givosiran can mitigate oxidative stress [6], a factor implicated in AML progression.

Overall, our goal is to evaluate the efficacy and safety of givosiran as a novel therapeutic strategy for AML by exploiting the crucial role of heme biosynthesis in its pathogenesis and the drug's ability to modulate heme levels and downstream effects on AML cell proliferation, survival, and drug sensitivity.\\

\textbf{Specific Aims 1}\\
\textbf{Overarching goal:}\\
Determine the anti-leukemic activity of givosiran in MYCN-driven AML models.\\
\textbf{Hypothesis:}\\
Givosiran treatment will decrease the viability and proliferation of MYCN-overexpressing AML cells in vitro and reduce tumor growth in MYCN-driven AML xenograft mouse models.\\
\textbf{Reasoning:}\\
MYCN-driven AML frequently exhibits upregulated heme biosynthesis [1], creating a potential dependency on this pathway. Givosiran, by inhibiting ALAS1, could disrupt this dependency, leading to decreased heme and growth inhibition. This is supported by preclinical data showing that inhibiting heme biosynthesis impacts AML cell growth and survival [1, 2].\\
\end{tcolorbox}
\end{figure}

\begin{figure}[htbp!]
\caption{\textbf{Example of co-scientist generated Specific Aims drug repurposing proposal for AML (continued).}}
\label{fig:aim1_2}
\vspace{0.1cm}
\begin{tcolorbox}[title=Givosiran for AML (continued)]
\footnotesize
\textbf{Specific Aims 2}\\
\textbf{Overarching goal:}\\
Elucidate the impact of givosiran on oxidative stress and drug sensitivity in AML.\\
\textbf{Hypothesis:}\\
Givosiran treatment will modulate oxidative stress levels and enhance the cytotoxic effects of standard AML chemotherapeutics (e.g., cytarabine) in AML cell lines and primary patient samples.\\
\textbf{Reasoning:}\\
Heme plays a role in oxidative stress regulation, and its modulation by givosiran could influence AML cell chemosensitivity.  Abstracts [5, 6] highlight oxidative stress in AML and the potential for ALA accumulation to contribute to it. By reducing ALA and heme, givosiran could alter reactive oxygen species (ROS) levels and potentially sensitize AML cells to chemotherapy-induced death.\\

\textbf{Specific Aims 3}\\
\textbf{Overarching goal:}\\
Characterize the safety and tolerability of givosiran in preclinical AML models, focusing on its impact on liver function.\\
\textbf{Hypothesis:}\\
Givosiran treatment will be well-tolerated in AML mouse models, with minimal adverse effects on liver function and drug metabolism, at doses that effectively inhibit ALAS1 and reduce heme biosynthesis.\\
\textbf{Reasoning:}\\
Given givosiran's hepatic target (ALAS1), evaluating its safety profile in AML is crucial. Abstracts [7, 8] highlight the clinical significance of liver function in AML patients. This aim will assess potential hepatotoxicity and drug-drug interactions, ensuring safe translation to clinical trials. We will evaluate relevant liver function markers and givosiran's impact on standard AML chemotherapeutic metabolism.\\

\textbf{Pilot Evaluation}\\
A pilot study in a human AML xenograft mouse model will assess givosiran's in vivo efficacy and safety.  Givosiran will be administered at various doses, monitoring tumor growth, survival, and liver function. The primary endpoint will be tumor growth inhibition. Secondary endpoints include survival, changes in heme levels, oxidative stress markers, and liver function tests. Existing safety data from givosiran's use in AHP will inform dose selection and monitoring. While givosiran is approved for AHP, its use in AML requires an Investigational New Drug (IND) application to the FDA before clinical trials.  Existing safety data might facilitate a streamlined review process.\\

\textbf{Articles}

[1] Upregulated heme biosynthesis, an exploitable vulnerability in MYCN-driven leukemogenesis
Summary: This study demonstrates increased heme biosynthesis in MYCN-driven leukemias, suggesting it as a therapeutic target.
Relevance: Supports the rationale for targeting heme biosynthesis in MYCN-driven AML and provides a rationale for the proposed mechanism of action of givosiran in this context.

[2] Systematic Dissection of the Metabolic-Apoptotic Interface in AML Reveals Heme Biosynthesis to Be a Regulator of Drug Sensitivity
Summary: This study highlights the role of heme biosynthesis in regulating apoptosis and drug sensitivity in AML.
Relevance: Supports the potential for givosiran to impact AML cell survival and treatment response by modulating heme biosynthesis.

[3] 3113 – IDENTIFICATION OF HEME AS A CRITICAL REGULATOR OF LEUKEMOGENESIS
Summary: This abstract points to the importance of heme levels in AML, especially in leukemic stem cells, and suggests targeting heme biosynthesis.
Relevance: Reinforces the rationale for targeting heme biosynthesis as a therapeutic strategy in AML.

...

 [8] Acute myeloid leukemia with hepatic infiltration presenting as obstructive jaundice
Summary: This abstract shows liver dysfunction in AML and the need to consider liver health in treatment.
Relevance: Reinforces the importance of monitoring liver function in AML patients treated with givosiran, further supporting the rationale for Aim 3.
\end{tcolorbox}
\end{figure}


\begin{figure}[htbp!]
\caption{\textbf{Example of co-scientist generated Specific Aims drug repurposing proposal for colon cancer.}}
\label{fig:aim2_1}
\vspace{0.1cm}
\begin{tcolorbox}[title=Selinexor for colon cancer]
\footnotesize
\textbf{Disease Description}\\
Colon adenocarcinoma (COAD) is a significant public health concern, with an estimated incidence rate of over 1.9 million new cases and 935,000 deaths globally in 2020. COAD arises from the epithelial lining of the colon and is driven by a complex interplay of genetic mutations (e.g., APC, KRAS, BRAF, TP53) and epigenetic alterations, leading to uncontrolled cell proliferation, impaired apoptosis, and chronic inflammation. These molecular changes disrupt crucial cellular pathways like Wnt/$\beta$-catenin signaling, cell cycle control, and DNA damage repair, ultimately driving tumor progression. COAD typically progresses through stages, from localized polyps to invasive tumors with potential for metastasis. Current standard of care involves surgery, chemotherapy, radiation therapy, and targeted therapies, but treatment resistance and recurrence remain major challenges, highlighting the need for new therapeutic approaches.

\textbf{Unmet Need}\\
Current COAD treatments have limitations, including acquired resistance to chemotherapy, significant toxicities, and incomplete responses in advanced disease. Patients often experience diminished quality of life due to treatment side effects and disease burden. High recurrence rates and limited effective treatment options after progression contribute to poor long-term outcomes.  This unmet need underscores the urgency for novel therapeutic strategies that can overcome resistance, improve response rates, and minimize toxicity, ultimately extending survival and enhancing quality of life for COAD patients.

\textbf{Proposed Solution}\\
Selinexor, a first-in-class selective inhibitor of nuclear export (SINE), is currently approved for the treatment of multiple myeloma and diffuse large B-cell lymphoma. It specifically targets XPO1 (Exportin 1), a key protein responsible for the nuclear export of tumor suppressor proteins, oncoproteins, and RNA. By binding to XPO1, Selinexor blocks the nuclear export of these molecules, leading to their accumulation in the nucleus and restoration of tumor suppressor function, cell cycle arrest, and apoptosis induction.

Repurposing Selinexor for COAD is rationally supported by its mechanism of action and the molecular characteristics of the disease. Overexpression of XPO1 is common in various cancers, including COAD [1, 2, 6]. Selinexor inhibits XPO1, preventing the nuclear export and restoring the function of key tumor suppressors (p53, RB, FOXO, APC) frequently dysregulated in COAD [2, 3, 4, 5]. Furthermore, Selinexor can suppress constitutively activated NF-$\kappa$B signaling, a driver of chronic inflammation and tumor progression in COAD, by blocking I$\kappa$B export and increasing its nuclear accumulation [7, 8]. These mechanisms align with key aspects of COAD pathogenesis and offer opportunities for therapeutic intervention.

We hypothesize that Selinexor will effectively inhibit COAD cell growth and enhance apoptosis by restoring tumor suppressor function and modulating crucial signaling pathways. Selinexor has shown anti-tumor activity in solid tumors in a Phase I trial [3], with observations of nuclear accumulation of tumor suppressor proteins [3, 6]. While not specific to COAD, these findings, coupled with evidence of Selinexor's efficacy in other cancers driven by XPO1 overexpression [3, 6], suggest that a similar mechanism could be effective in COAD. Further supporting our hypothesis, XPO1 overexpression has been linked to NF-$\kappa$B activation and increased proliferation in COAD [7].
Overall, our goal is: to demonstrate that Selinexor’s XPO1 inhibitory activity can effectively target key oncogenic drivers and restore tumor suppressor functions in COAD, ultimately leading to tumor growth inhibition and improved patient outcomes.

\textbf{Specific Aims 1}\\
\textbf{Overarching goal:}\\
Determine the in vitro efficacy of Selinexor in inhibiting COAD cell growth and inducing apoptosis.\\
\textbf{Hypothesis:}\\
Selinexor treatment will significantly reduce the viability and increase apoptosis in a panel of COAD cell lines, including those with varying genetic backgrounds (e.g., APC, KRAS, TP53 mutations).\\
\textbf{Reasoning:}\\
Selinexor's inhibition of XPO1 leads to nuclear accumulation of tumor suppressors like p53, a key regulator of apoptosis [2, 6]. Restoration of p53 function and suppression of NF-$\kappa$B, a pro-survival pathway [7, 8], are expected to induce apoptosis in COAD cells. We will evaluate this hypothesis using cell viability and apoptosis assays in diverse COAD cell lines to assess the impact of genetic background on Selinexor's efficacy.\\
\end{tcolorbox}
\end{figure}

\begin{figure}[htbp!]
\caption{\textbf{Example of co-scientist generated Specific Aims drug repurposing proposal for colon cancer (continued).}}
\label{fig:aim2_2}
\vspace{0.1cm}
\begin{tcolorbox}[title=Selinexor for colon cancer (continued)]
\footnotesize
\textbf{Specific Aims 2}\\
\textbf{Overarching goal:}\\
Elucidate the mechanisms by which Selinexor inhibits COAD cell growth, focusing on XPO1-mediated restoration of tumor suppressor function.\\
\textbf{Hypothesis:}\\
Selinexor treatment will increase nuclear accumulation of p53, RB, and FOXO proteins and decrease nuclear export of I$\kappa$B, leading to cell cycle arrest and decreased NF-$\kappa$B activity in COAD cells.\\
\textbf{Reasoning:}\\
Selinexor blocks XPO1, preventing nuclear export of key tumor suppressors (p53, RB, FOXO) [2, 4, 5] and the NF-$\kappa$B inhibitor I$\kappa$B [7, 8]. Increased nuclear localization of these proteins should restore their growth regulatory functions, leading to cell cycle arrest and reduced NF-$\kappa$B-driven proliferation. We will evaluate this hypothesis using immunofluorescence and western blotting to assess protein localization and activity.

\textbf{Specific Aims 3}\\
\textbf{Overarching goal:}\\
Characterize the potential synergistic effects of Selinexor in combination with standard-of-care chemotherapies for COAD.\\
\textbf{Hypothesis:}\\
Combination treatment with Selinexor and 5-fluorouracil (5-FU) will synergistically reduce COAD cell viability compared to either treatment alone.\\
\textbf{Reasoning:}\\
XPO1 inhibition can sensitize cancer cells to chemotherapy [9]. Combining Selinexor with 5-FU, a common COAD chemotherapy, may enhance cellular stress and improve treatment response. We will test this hypothesis using cell viability assays and investigate the underlying mechanisms of synergy.

\textbf{Pilot Evaluation}\\
A pilot study will evaluate Selinexor's efficacy in a patient-derived xenograft (PDX) model of COAD. The primary objective is to determine the effect of Selinexor on tumor growth. The study will utilize an open-label, single-arm design with escalating Selinexor doses in established COAD PDX models. Inclusion criteria: established COAD PDX models. Exclusion criteria: none. Primary endpoint: tumor volume change. Secondary endpoints: changes in biomarkers (p53, Ki67, NF-$\kappa$B) within the tumor. Selinexor's established safety profile in other cancers provides a basis for evaluating its safety in this new indication. Potential for expedited review through existing regulatory pathways will be explored.\\
\\
\textbf{Expert rating}\\
1. Strongly Agree (unmet clinical needs)\\
2. Agree (bridges therapeutic gap)\\
3. Strongly Agree (scientifically rigorous rationale)\\
4. Agree (integrates prior studies)\\
5. Agree (avoids over-extrapolation)\\
6. Strongly Agree (clear hypotheses and methods)\\
7. Strongly Agree (clearly stated aims)\\
8. Agree (path to clinical application)\\
9. Agree (well-defined endpoints)\\
10. Agree (meaningful pre-clinical experiments)\\
11. Agree (translational component)\\
12. Strongly Agree (avoids inaccuracies)\\
13. Strongly Agree (evidence-based assumptions)\\
14. Agree (originality and terminology)\\
15. Strongly Agree (clear writing and organization)\\
\end{tcolorbox}
\end{figure}



\begin{figure}[htbp!]
\caption{\textbf{Example of co-scientist generated Specific Aims drug repurposing proposal for colon cancer.}}
\label{fig:aim3_1}
\vspace{0.1cm}
\begin{tcolorbox}[title=Lapatinib for colon cancer]
\footnotesize
\textbf{Disease Description}\\
Colon adenocarcinoma (COAD) is a significant public health concern, with an estimated incidence rate of over 150,000 new cases annually in the United States [19]. COAD arises from the epithelial lining of the colon, and its pathogenesis involves a complex interplay of genetic and environmental factors. Key molecular mechanisms include mutations in genes such as APC, KRAS, BRAF, and PIK3CA, as well as dysregulation of signaling pathways like Wnt, RAS/RAF/MEK/ERK, and PI3K/AKT/mTOR [8, 9, 12]. These alterations drive uncontrolled cell proliferation, evade apoptosis [14], promote angiogenesis [15], and ultimately lead to tumor growth, invasion, and metastasis [16]. The disease typically progresses through a series of stages, from localized tumors to regional lymph node involvement and distant metastasis. Current standard of care involves surgery, chemotherapy, and targeted therapies such as anti-EGFR antibodies. However, these treatments have limitations, including acquired resistance and significant toxicity.

\textbf{Unmet Need}\\
A major unmet need in COAD treatment is the development of effective therapies for patients who progress on or are refractory to standard treatments, particularly those with resistance to anti-EGFR therapy [2]. Despite available therapies, many patients experience disease recurrence and metastasis, leading to poor outcomes and diminished quality of life. There's a critical need for new therapeutic options that can overcome resistance mechanisms, improve response rates, and offer better tolerability profiles. Specifically, addressing resistance driven by KRAS mutations [10] and exploring alternative therapeutic targets remains crucial.

\textbf{Proposed Solution}\\
Lapatinib is an orally available small molecule tyrosine kinase inhibitor currently approved for use in combination with capecitabine for the treatment of HER2-positive metastatic breast cancer. It reversibly inhibits the intracellular tyrosine kinase domains of both EGFR (ErbB1) and HER2 (ErbB2), thereby blocking downstream signaling cascades, including RAS/RAF/MEK/ERK and PI3K/AKT/mTOR. This inhibition leads to decreased cell proliferation and increased apoptosis.

Repurposing lapatinib for COAD is rationalized by the shared ErbB signaling pathway between breast cancer and a subset of COAD. EGFR is commonly overexpressed in COAD [1], and while HER2 overexpression is less frequent than in breast cancer, it occurs in a clinically relevant subset [3, 17] and is associated with resistance to anti-EGFR therapy [2]. Lapatinib can directly inhibit both EGFR and HER2, potentially disrupting crucial oncogenic signaling [13, 14] including PLC$\gamma$ [7].

We hypothesize that lapatinib can effectively inhibit ErbB signaling in COAD, leading to decreased cell proliferation, increased apoptosis, and suppression of metastasis. Preclinical studies demonstrate synergistic antitumor activity of lapatinib with HDAC inhibitors in COAD models [6], and lapatinib has also been shown to sensitize COAD cells to TRAIL-induced apoptosis via an off-target mechanism [5]. Studies have also investigated HER2 as a therapeutic target in CRC, especially after failure of anti-EGFR therapy [2, 18]. Although lapatinib as a single agent has shown limited efficacy in unselected CRC populations, this could be attributed to the heterogeneity of COAD and the presence of resistance mechanisms like KRAS mutations. We anticipate that patient stratification based on molecular profiles will identify subgroups that derive greater benefit.

Overall, our goal is: to demonstrate the efficacy of lapatinib in specific molecular subtypes of COAD, potentially in combination with other targeted therapies, to address the unmet need for new treatment options, particularly in patients resistant to standard therapies.

\textbf{Specific Aims 1}\\
\textbf{Overarching goal:}\\
To determine the efficacy of lapatinib in inhibiting HER2/EGFR signaling and suppressing cell proliferation in a panel of well-characterized COAD cell lines.\\
\textbf{Hypothesis:}\\
Lapatinib will inhibit cell proliferation in HER2-amplified/overexpressing and/or EGFR-overexpressing COAD cell lines.\\
\textbf{Reasoning:}\\
Lapatinib directly inhibits both HER2 and EGFR, key drivers of cell proliferation in a subset of COAD. Inhibition of these receptors should lead to reduced downstream signaling through the RAS/RAF/MEK/ERK and PI3K/AKT/mTOR pathways, ultimately suppressing cell growth [6].\\
\end{tcolorbox}
\end{figure}

\begin{figure}[htbp!]
\caption{\textbf{Example of co-scientist generated Specific Aims drug repurposing proposal for colon cancer (continued).}}
\label{fig:aim3_2}
\vspace{0.1cm}
\begin{tcolorbox}[title=Lapatinib for colon cancer (continued)]
\footnotesize
\textbf{Specific Aims 2}\\
\textbf{Overarching goal:}\\
To elucidate the impact of lapatinib on apoptosis and key signaling pathways in COAD models.\\
\textbf{Hypothesis:}\\
Lapatinib will induce apoptosis and modulate key signaling pathways (RAS/RAF/MEK/ERK, PI3K/AKT/mTOR, and PLC$\gamma$) in COAD cell lines, especially those with HER2/EGFR alterations.\\
\textbf{Reasoning:}\\
Inhibition of HER2/EGFR by lapatinib disrupts pro-survival signaling, potentially promoting apoptosis [14].  Additionally, lapatinib can indirectly modulate downstream pathways [4, 7], even demonstrating synergistic effects with other agents [6]. This aim will explore the extent of this modulation in various COAD subtypes.

\textbf{Specific Aims 3}\\
\textbf{Overarching goal:}\\
To identify predictive biomarkers for lapatinib response and explore potential combination strategies in COAD.\\
\textbf{Hypothesis:}\\
A combination of lapatinib with a MEK inhibitor will enhance antitumor activity in KRAS-mutant COAD cell lines compared to lapatinib alone.\\
\textbf{Reasoning:}\\
KRAS mutations are a known mechanism of resistance to EGFR/HER2 targeted therapies [10]. Combining lapatinib with a MEK inhibitor may overcome this resistance by blocking the downstream MAPK pathway activation, leading to enhanced antitumor efficacy.

\textbf{Pilot Evaluation}\\
A pilot study will be conducted to evaluate the safety and preliminary efficacy of lapatinib in patients with HER2-positive metastatic COAD who have progressed on standard therapy. This open-label, single-arm study will enroll 15-20 patients with confirmed HER2 overexpression/amplification and KRAS wild-type status. Key exclusion criteria will include prior treatment with lapatinib and significant comorbidities. The primary endpoint will be progression-free survival, and secondary endpoints will include overall response rate, duration of response, and safety. Lapatinib's known safety profile from its use in breast cancer will inform the monitoring plan. Expedited regulatory review may be possible due to lapatinib's existing approval.

\textbf{Articles}

[1] Targeting the EGFR signalling pathway in metastatic colorectal cancer\\
Summary: Details the importance of EGFR signaling in CRC and how targeting it is a crucial therapeutic strategy.\\
Relevance: Provides context for lapatinib's action as it targets EGFR.

[2] HER2 as an Emerging Oncotarget for Colorectal Cancer Treatment After Failure of Anti-Epidermal Growth Factor Receptor Therapy\\
Summary: Discusses HER2 as a target in CRC, especially in the context of anti-EGFR therapy resistance.\\
Relevance: Directly relevant to lapatinib's mechanism and potential in COAD.

...

[19] United States Cancer Statistics\\
Summary: Provides official U.S. cancer incidence and mortality statistics.\\
Relevance: Source of epidemiological data for colon adenocarcinoma.\\
\\
\textbf{Expert rating}\\
1.	Strongly Agree (unmet clinical needs)\\
2.	Agree (bridges therapeutic gap)\\
3.	Strongly Agree (scientifically rigorous rationale)\\
4.	Strongly Agree (integrates prior studies)\\
5.	Strongly Agree (avoids over-extrapolation)\\
6.	Strongly Agree (clear hypotheses and methods)\\
7.	Strongly Agree (clearly stated aims)\\
8.	Strongly Agree (path to clinical application)\\
9.	Agree (well-defined endpoints)\\
10.	Neutral (meaningful pre-clinical experiments)\\
11.	Strongly Agree (translational component)\\
12.	Strongly Agree (avoids inaccuracies)\\
13.	Strongly Agree (evidence-based assumptions)\\
14.	Strongly Agree (originality and terminology)\\
15.	Agree (clear writing and organization)\\
\end{tcolorbox}
\end{figure}


\clearpage
\subsection{Detailed AI co-scientist output for a validated AML repurposing candidate}
\label{sec:drug_output}
\begin{figure}[htbp!]
\begin{tcolorbox}[
    colback=black!5!white,
    colframe=black!60!white,
    title=\textbf{KIRA6 for AML},
    fonttitle=\bfseries,
    arc=3mm,
    boxrule=1pt,
    bottomrule=2pt,
]
\scriptsize
\textbf{Summary}

This proposal outlines repurposing KIRA6, an IRE1$\alpha$ inhibitor, for treating acute myeloid leukemia (AML), particularly FLT3-ITD positive AML known for high relapse rates. The hypothesis is that KIRA6 will disrupt protein homeostasis and induce ER stress in AML cells, synergizing with existing therapies and overcoming resistance. The proposal details KIRA6's mechanism, impacting pathways like the UPR and apoptosis, and suggests a comprehensive experimental plan including in vitro and in vivo studies to validate its efficacy and safety, including combination therapy with existing AML drugs.  This approach aims to provide a novel, effective treatment strategy for AML patients, particularly those with the challenging FLT3-ITD mutation and other drug resistance phenotypes. \\


\textbf{Hypothesis}

\textbf{Introduction} Acute myeloid leukemia (AML) remains a formidable hematological malignancy characterized by the rapid proliferation of immature myeloid blasts in the bone marrow and peripheral blood. Despite advancements in therapeutic strategies, including targeted therapies and hematopoietic stem cell transplantation, a significant proportion of patients experience relapse or develop resistance to treatment, underscoring the urgent need for new therapeutic interventions. The heterogeneous nature of AML, with diverse genetic and molecular aberrations driving disease progression, necessitates the exploration of novel therapeutic targets and the rational design of combination therapies to overcome treatment resistance and improve long-term survival outcomes. Drug repurposing, which involves identifying new uses for existing drugs, offers a cost-effective and efficient strategy to accelerate the development of novel AML treatments by leveraging established safety profiles and pharmacokinetic data. This approach can significantly reduce development time and costs, facilitating the rapid translation of promising agents to clinical trials. \\

\textbf{Recent findings and related research}

Recent research has significantly advanced our understanding of the complex molecular landscape of AML, highlighting the critical roles of various genetic mutations and signaling pathways in leukemogenesis. Recurrent mutations in genes such as FLT3, NPM1, IDH1/2, and TP53 are frequently observed in AML and have been implicated in disease initiation, progression, and treatment resistance. The FLT3-ITD mutation, a particularly challenging subtype, is associated with poor prognosis and a higher risk of relapse, emphasizing the need for effective therapies that target this specific mutation or overcome its downstream effects. Targeted therapies, such as FLT3 inhibitors (midostaurin, gilteritinib), IDH inhibitors (enasidenib, ivosidenib), and the BCL-2 inhibitor venetoclax, have demonstrated clinical efficacy in specific AML subtypes; however, the emergence of drug resistance and the lack of effective treatments for high-risk patients remain critical challenges. \\

Drug repurposing has yielded several promising candidates for AML treatment, including:
\begin{itemize}
    \item Arsenic trioxide and all-trans retinoic acid (ATRA):  Established treatments for acute promyelocytic leukemia (APL), these have also shown potential in combination therapy for non-APL AML.
    \item Histone deacetylase (HDAC) inhibitors:  Vorinostat and panobinostat have shown modest activity in AML, particularly in combination with other agents, but are known to be poorly tolerated and have limited efficacy as single agents.
    \item Proteasome inhibitors: Bortezomib and carfilzomib have been explored in combination regimens in AML, especially to overcome resistance, though with limited success.
    \item Metabolic inhibitors: Targeting glutamine metabolism and other metabolic pathways has shown promise in preclinical studies, though with limited translation due to toxicity and poor absorption.
    \item Homoharringtonine (HHT):  As noted, HHT, a protein synthesis inhibitor, has been approved for CML and has shown promise in AML, although its mechanism of action and specific impacts on resistance mechanisms need to be further explored.
    \item Immunomodulatory drugs: Thalidomide and lenalidomide have been explored in combination therapies to target the tumor microenvironment and enhance immune responses, though with limited efficacy in AML.
    \item Kinase inhibitors: Beyond FLT3, new kinase inhibitors that target other pathways are being explored in combination settings to overcome resistance.
\end{itemize}
Despite these findings, there remains a significant unmet need for novel, effective therapies that can overcome drug resistance, target novel pathways, and improve long-term survival for AML patients.

\end{tcolorbox}
\vspace{0.1cm}
\caption{\textbf{Detailed AI co-scientist output for repurposing KIRA6, an IRE1$\alpha$ inhibitor for AML}.}
\label{fig:ex_kira6_1}
\end{figure}


\begin{figure}[htbp!]
\begin{tcolorbox}[
    colback=black!5!white,
    colframe=black!60!white,
    title=\textbf{KIRA6 for AML (continued)},
    fonttitle=\bfseries,
    arc=3mm,
    boxrule=1pt,
    bottomrule=2pt,
]
\scriptsize

\textbf{Areas worth exploring} 

Several areas hold significant potential for identifying novel or repurposed drugs for AML treatment:
\begin{itemize}
    \item Targeting non-canonical signaling pathways: Explore drugs that target pathways beyond the well-established ones, such as those involved in inflammatory signaling, protein degradation, and DNA damage repair. The interplay between the AML cell and its microenvironment, including inflammatory cytokines and immune cells, remains an area of active investigation and a promising target for therapeutic intervention.
    \item Targeting RNA processing and translation: Aberrant RNA processing and translation are significant drivers of protein expression in cancer, and are potential targets for novel therapies. Targeting ribosome function, mRNA splicing, and RNA modifications can disrupt the production of key proteins required for cell survival and proliferation.
    \item Modulating immunometabolism: Cancer cells exhibit altered metabolic demands that influence the immune response, and targeting metabolic pathways that influence immune activation and suppression could enhance anti-leukemic immune responses. This also includes the metabolic dependencies of the myeloid microenvironment, which is critical for disease progression and resistance.
    \item Leveraging high-throughput screening (HTS) and computational biology: Combine in vitro or ex vivo screening of approved drugs with computational drug repurposing methods, particularly those that integrate multi-omics data, to identify novel drug candidates and potential synergistic drug combinations.
    \item Focus on drug combinations: Rational drug combinations that address both active malignancy and underlying resistance mechanisms, and can achieve effective treatment regimens at lower concentrations in combination, will have better safety, efficacy and overcome existing resistance. This should include a focus on sequential drug administration and drug scheduling.
    \item RNA-based therapies: Direct targeting of mRNAs coding for leukemia-driving proteins can be a very impactful strategy that also offers the chance to overcome resistance, for example by interfering with compensatory pathways. This includes RNA interference (RNAi), antisense oligonucleotides (ASOs), and mRNA-based therapies.
    \item Targeting protein homeostasis: Disrupting the protein homeostasis network, including protein folding, degradation (through ubiquitin-proteasome and autophagy pathways), and quality control mechanisms, can induce cellular stress and apoptosis in AML cells.
\end{itemize}

\textbf{Detailed novel likely correct idea: repurposing the IRE1$\alpha$ inhibitor KIRA6 for AML treatment\\}

\textbf{Goal}: Repurpose KIRA6, an IRE1$\alpha$ inhibitor, for AML treatment, specifically targeting the MOLM13 cell line with FLT3-ITD, with a focus on overcoming resistance mechanisms and enhancing combination therapy efficacy.
\\
Idea: KIRA6, by inhibiting IRE1$\alpha$, a key regulator of the unfolded protein response (UPR), can disrupt protein homeostasis and induce ER stress in AML cells, particularly in cells exhibiting high metabolic demand and increased protein synthesis, such as FLT3-ITD positive cells. This disruption can synergize with other therapies, including FLT3 inhibitors and chemotherapeutics, to overcome resistance and improve therapeutic efficacy. KIRA6 will also supress chronic inflammatory pathways known to promote cell survival in AML cells.

\end{tcolorbox}
\vspace{0.1cm}
\caption{\textbf{Detailed AI co-scientist output for repurposing KIRA6, an IRE1$\alpha$ inhibitor for AML (continued)}.}
\label{fig:ex_kira6_2}
\end{figure}


\begin{figure}[htbp!]
\begin{tcolorbox}[
    colback=black!5!white,
    colframe=black!60!white,
    title=\textbf{KIRA6 for AML (continued)},
    fonttitle=\bfseries,
    arc=3mm,
    boxrule=1pt,
    bottomrule=2pt,
]
\scriptsize
\textbf{Molecular mechanism of action}
\begin{itemize}
    \item Inhibition of IRE1$\alpha$: KIRA6 is a selective inhibitor of IRE1$\alpha$, a transmembrane protein in the endoplasmic reticulum (ER) that acts as a key sensor and transducer of ER stress. IRE1$\alpha$ activation initiates the unfolded protein response (UPR) pathway, which is a cellular stress response aimed at restoring protein homeostasis in the ER. Under conditions of increased protein synthesis or misfolded proteins, IRE1$\alpha$ is activated, leading to the splicing of XBP1 mRNA and the subsequent activation of downstream transcriptional targets involved in protein folding, trafficking, and degradation.
    \item Disruption of ER homeostasis: By inhibiting IRE1$\alpha$, KIRA6 blocks the adaptive arm of the UPR, preventing the resolution of ER stress and leading to the accumulation of unfolded and misfolded proteins. This disruption creates a protein folding crisis, inducing ER stress and initiating apoptotic pathways. This effect is potentiated in rapidly proliferating AML cells, which have higher metabolic demands and are more sensitive to ER stress.
    \item Downstream effects:
    \begin{itemize}
        \item  The accumulation of misfolded proteins leads to the activation of the PERK and ATF6 arms of the UPR, which further contribute to ER stress and apoptosis.
        \item  Inhibition of IRE1$\alpha$ disrupts the transcriptional program regulated by XBP1, impairing the production of proteins involved in ER homeostasis and cell survival.
        \item  The resulting ER stress induces the activation of the integrated stress response (ISR), leading to the translational suppression of many transcripts and increased apoptotic activity.
        \item  KIRA6 can indirectly inhibit the NF-$\kappa$B pathway by reducing ER stress and inflammatory cytokine production, reducing cell survival, proliferation and resistance.
        \item  KIRA6 may directly reduce levels of inflammatory cytokines such as IL-1, in turn reducing secondary activation of inflammatory signalling cascades and reduce activation of IRAK1 in the process.
    \end{itemize}
    \item Impact on FLT3-ITD: FLT3-ITD mutations lead to increased cell proliferation and metabolic demand and stress. KIRA6 can target this through disruption of the UPR, leading to a higher impact on FLT3-ITD cells. FLT3-ITD cells are under high levels of stress already, and require high levels of protein synthesis to maintain viability. By targeting basal or activated IRE1$\alpha$, KIRA6 can induce significantly more cell death in FLT3-ITD cells than their wild-type counterparts.
    \item Impact on key dysregulated pathways:
    \begin{itemize}
        \item  MYC: MYC protein levels are directly tied to mRNA translation and are necessary for cell survival and resistance in many cases, including leukemia. Targeting the UPR and downstream translation with KIRA6 directly decreases MYC expression and survival. This also has downstream anti-inflammatory benefits.
        \item  NF-$\kappa$B signaling: KIRA6 can reduce NF-$\kappa$B activity, a key driver of cell survival, proliferation, and resistance, by reducing ER stress and inflammatory cytokine production.
        \item  MCL-1 and other anti-apoptotic proteins: KIRA6 will reduce the production of short-lived survival proteins, leading to rapid apoptosis by diminishing their production, particularly MCL-1 and other similar proteins, which are involved in anti-apoptotic effects in AML cells, and are known drug resistant mechanisms.
        \item  Targeting multiple AML resistance mechanisms: KIRA6, by disrupting protein homeostasis, has the capacity to overcome resistance across a wide panel of mechanisms, including through reduced efflux pump protein levels, and reduced repair mechanisms.
    \end{itemize}
    \item Synergistic effects: KIRA6 can synergize with FLT3 inhibitors by increasing ER stress, making cells more sensitive to FLT3 inhibition. It can also enhance the activity of chemotherapeutic drugs by disrupting DNA repair pathways.
\end{itemize}

\textbf{Impacted pathways}
\begin{itemize}
    \item  Unfolded protein response (UPR): Directly impacted through inhibition of IRE1$\alpha$.
    \item  Apoptosis: Primarily through the intrinsic pathways, and disruption of anti-apoptotic proteins, and an increase in ER stress, alongside an increase of pro-apoptotic proteins through the integrated stress response.
    \item  Cell cycle: Perturbation of cell cycle progression through reduced levels of cell-cycle proteins.
    \item  NF-$\kappa$B signaling: KIRA6 can reduce NF-$\kappa$B activity which reduces inflammation and promotes cell survival, and drug resistance.
    \item  MYC-related pathways: Suppression of MYC and other related oncogene protein levels.
    \item  Inflammatory pathways: By supressing cytokines, most notably IL-1, KIRA6 can suppress activation of IRAK1 and reduce chronic inflammation that promotes cell survival in addition to killing cancer cells.
\end{itemize}

\end{tcolorbox}
\vspace{0.1cm}
\caption{\textbf{Detailed AI co-scientist output for repurposing KIRA6, an IRE1$\alpha$ inhibitor for AML (continued)}.}
\label{fig:ex_kira6_3}
\end{figure}


\begin{figure}[htbp!]
\begin{tcolorbox}[
    colback=black!5!white,
    colframe=black!60!white,
    title=\textbf{KIRA6 for AML (continued)},
    fonttitle=\bfseries,
    arc=3mm,
    boxrule=1pt,
    bottomrule=2pt,
]
\scriptsize
\textbf{Effect on AML (MOLM13) cell proliferation inhibition}
\begin{itemize}
    \item KIRA6 selectively inhibits proliferation of MOLM13 cells, and also AML cells with other mutations that render them drug resistant.
    \item IRE1$\alpha$ inhibition leads to the activation of apoptotic pathways and reduced cell proliferation and clonal survival.
    \item Suppression of MYC levels, NF-kB, MCL-1 and other cell survival pathways will reduce cell survival and promote apoptosis, which should be enhanced in FLT3-ITD AML due to their high proliferation rates and increased sensitivity to metabolic/translational stress in comparison to normal myeloid cells.
    \item In addition to having a direct effect on proliferation, a reduction in inflammatory signalling (IL-1 and IRAK1) can potentiate activity of other drugs.
    \item KIRA6 treatment results in a dose-dependent reduction in MOLM13 cell viability.
    \item KIRA6 leads to a buildup of ER stress and activation of the ISR, all which enhance its effectiveness as a targeted drug.
    \item When combined with FLT3 inhibitors or chemotherapeutic drugs, KIRA6 is expected to show additional impacts at lower overall drug concentrations (lower levels of both KIRA6 or the alternative drugs in the combination will be required for the same levels of cell death, or greater cell death will be seen at the standard doses used presently in the clinic). This is very important given the high levels of both drug toxicity that are associated with all AML treatments, and the need to reduce those toxicities.
\end{itemize}

\textbf{IC50 assay concentrations}

Based on published data for KIRA6 in other cancer cell models and its known mechanism of action, a reasonable starting concentration range for an IC50 assay in MOLM13 cells would be 100 nM to 10 $\mu$M.
\begin{itemize}
    \item Rationale: This range spans the concentrations that have shown activity in various cell types while being within a pharmacologically achievable range. Prior in vitro studies have shown activity in low micromolar concentrations.
    \item Specific concentrations: 100 nM, 250 nM, 500 nM, 1 $\mu$M, 2.5 $\mu$M, 5 $\mu$M, 7.5 $\mu$M, 10 $\mu$M.
    \item  Assay method: Cell viability assays (e.g., MTT, CellTiter-Glo) to determine IC50 values after KIRA6 exposure for 48-72 hours.
    \item Controls: Appropriate vehicle controls (DMSO) or medium-only controls should be included.
    \item Positive control:  Use a known FLT3 inhibitor, such as gilteritinib, at its reported IC50 in MOLM13 cells for normalization of assay variability and comparison.
    \item Combination studies: The combination of a FLT3 inhibitor at a single set concentration, plus a matrix of KIRA6 concentrations at several ratios will be evaluated using the viability assays as a baseline. Additional confirmation using apoptotic assays and cell cycle arrest should be done for any combination that shows better results than single-agent drug effects.
\end{itemize}

\textbf{Safety and toxicity}
\begin{itemize}
    \item Limited safety data: KIRA6 is a relatively new compound and has limited safety data in humans, as it has not gone through clinical trials.  Initial in vitro and in vivo studies in other disease settings (mostly cancer) suggest that it is generally well-tolerated. However, thorough preclinical toxicity studies are essential before advancing to human trials.
    \item Potential toxicity: Given the mechanism of action, potential toxicities include ER stress-related effects on normal tissues, particularly those with high protein synthesis demands (e.g., liver, pancreas). However, since AML cells are under high levels of stress, and have high metabolic demand, they are expected to be disproportionately impacted by this drug, which should reduce systemic toxicity.
    \item Repurposing advantage: Because KIRA6 has established in vitro activity, and has known pharmacology, it is significantly less risky than a compound that would have to have all of its safety profiles evaluated before human trials.
    \item Combination therapy impact: Because KIRA6 can be paired with existing therapies, it may be possible to reduce the overall dose of both agents to minimize side-effects and overall toxicity, while enhancing therapeutic efficacy. This should be tested rigorously in vitro through all of the combination studies before any movement into clinical human or animal trials to make sure that the benefit is higher than the risk.
\end{itemize}

\textbf{Testable hypothesis}

KIRA6, by inhibiting IRE1$\alpha$, will selectively inhibit the proliferation of AML cells, especially FLT3-ITD positive cells, by disrupting protein homeostasis and inducing ER stress. The increase in stress and reduced output of essential cell maintenance proteins will trigger apoptotic pathways, leading to cell death, particularly in highly proliferative settings. KIRA6 will potentiate the activity of existing AML therapies such as FLT3 inhibitors and chemotherapeutics. The drug will also have a broader impact on resistance mechanisms.

\end{tcolorbox}
\vspace{0.1cm}
\caption{\textbf{Detailed AI co-scientist output for repurposing KIRA6, an IRE1$\alpha$ inhibitor for AML (continued)}.}
\label{fig:ex_kira6_4}
\end{figure}


\begin{figure}[htbp!]
\begin{tcolorbox}[
    colback=black!5!white,
    colframe=black!60!white,
    title=\textbf{KIRA6 for AML (continued)},
    fonttitle=\bfseries,
    arc=3mm,
    boxrule=1pt,
    bottomrule=2pt,
]
\scriptsize
\textbf{Experimental plan}
\begin{itemize}
    \item IC50 determination in MOLM13: Determine IC50 values in MOLM13 cells using viability assays with KIRA6 concentrations ranging from 100 nM to 10 $\mu$M.
    \item ER stress marker measurement: Measure levels of key ER stress markers (e.g., BiP/GRP78, CHOP) using Western blotting and ELISA after KIRA6 treatment to confirm the impact on ER homeostasis.
    \item IRE1$\alpha$ activity: Measure XBP1 splicing using RT-PCR or other assays to confirm KIRA6's impact on IRE1$\alpha$ activity.
    \item Apoptosis assays: Evaluate apoptosis by Annexin V/PI staining, caspase activity assays, and Western blot analysis of apoptotic markers (cleaved PARP, cleaved caspase-3 levels) in KIRA6-treated MOLM13 cells.
    \item Western blots: Analyze key pathway proteins: including MYC, NF-$\kappa$B, MCL-1, FLT3, AKT/mTOR, and key components of the canonical inflammatory pathways (including IL-1, IL-6, TNF-alpha, IRAK1/4, MyD88), and markers of endoplasmic reticulum (ER) stress in AML cells treated with KIRA6.
    \item ISR markers measurement: Evaluate the induction of the Integrated Stress Response using assays measuring levels of phospho-eIF2$\alpha$, ATF4, and CHOP.
    \item Combination studies: Evaluate the combination of KIRA6 with FLT3 inhibitors like gilteritinib, with chemotherapeutics (cytarabine, danunorubicin) or with venetoclax in MOLM13 cells using combination matrix studies. This will establish if there is synergistic activity and enable evaluation of lower doses of each drug. Repeat these on additional cell lines with and without FLT3 mutations to confirm consistent synergistic enhancement.
    \item Cell cycle analysis: Measure cell cycle arrest using DNA staining (propidium iodide) by flow cytometry.
    \item RNA stress markers studies: Evaluate markers of RNA stress to assess damaged mRNA levels and ER Stress markers to determine cellular stress induced by the drug.
    \item Efflux pump activity: Measure the activity of key drug efflux pumps (e.g., ABCB1, ABCG2) using specific inhibitors and substrates to evaluate KIRA6's impact on drug resistance mechanisms.
    \item Expansion to other cell lines: Repeat the above assays on multiple AML cell lines, with both FLT3-ITD and wild-type cells, and lines representing other AML mutations or drug resistance phenotypes. Also evaluate primary AML patient cells in co-culture with stroma and alone with comparable controls, including primary non-leukemic bone marrow derived cells for measuring minimal activity in normal myeloid cells. This should allow for verification of the generalizability of KIRA6
    \item In vivo xenograft studies: Evaluate the efficacy and safety of KIRA6 in mouse xenograft models using MOLM13 cells or other cell lines to assess its overall treatment value and toxicities.
    \item Pharmacokinetics studies: Evaluate absorption, distribution, metabolism, and excretion of KIRA6 in animals to enable rational dose selection in future trials.
\end{itemize}

\textbf{Conclusion}

Repurposing KIRA6, an IRE1$\alpha$ inhibitor, for AML treatment, with a focus on FLT3-ITD positive cells and overcoming resistance mechanisms, represents a promising novel approach. KIRA6's ability to disrupt protein homeostasis, induce ER stress, synergize with existing AML drugs, and potentially overcome resistance mechanisms makes it an excellent drug candidate for testing. The comprehensive experimental plan proposed here offers a thorough framework for validating the potential of KIRA6 in AML and allows for a better way to select drugs for clinical testing by first performing a rigorous scientific evaluation in vitro using a multi-faceted approach. KIRA6 could translate into a new and effective strategy for AML patients, particularly those with the challenging FLT3-ITD mutation and other resistance phenotypes. Due to its mode of action on the UPR, it also has the potential to target underlying resistance mechanisms to existing therapies.

The focus on resistance mechanisms, a more diverse set of cell lines and primary cells, a well-rationalized combination approach, and a specific mechanism of action, all address the feedback from the previous version and provide a strong and detailed rationale for this drug repurposing strategy. The expanded experimental plan provides a thorough framework for assessing the efficacy, selectivity, and safety of KIRA6 in AML.
\end{tcolorbox}
\vspace{0.1cm}
\caption{\textbf{Detailed AI co-scientist output for repurposing KIRA6, an IRE1$\alpha$ inhibitor for AML (continued)}.}
\label{fig:ex_kira6_5}
\end{figure}


\begin{figure}[htbp!]
\begin{tcolorbox}[
    colback=black!5!white,
    colframe=black!60!white,
    title=\textbf{KIRA6 for AML (continued)},
    fonttitle=\bfseries,
    arc=3mm,
    boxrule=1pt,
    bottomrule=2pt,
]
\scriptsize
\textbf{Review}

\textbf{Relevant article abstracts}
\begin{itemize}
\item [1]: Discusses targeting the IRE1$\alpha$-XBP1s pathway in AML, directly relevant to the proposed mechanism of KIRA6.
\item [5]: Reviews the UPR as a therapeutic target in acute leukemias, providing context and supporting the rationale for targeting IRE1$\alpha$.
\item [6]: Focuses on selective degradation of FLT3-ITD, offering insights into targeting specific forms of the mutated protein, aligning with the idea's focus on targeting ER-bound FLT3-ITD.
\item [7]: Offers the background on known pharmacology and toxicological studies for IC50 determination of drug candidates in tumor cells.
\item [9]: Discusses sensitizing AML to venetoclax treatment, highlighting the potential for combination therapy approaches, relevant to the idea's suggested synergistic effects with other AML drugs.
\end{itemize}

\textbf{Assumptions}
\begin{itemize}
    \item KIRA6 effectively inhibits IRE1$\alpha$ in AML cells, particularly MOLM13 cells.
    \item IRE1$\alpha$ inhibition leads to significant disruption of protein homeostasis and induction of ER stress in AML cells.
    \item The increased ER stress induced by KIRA6 selectively targets AML cells over healthy cells.
    \item KIRA6 synergizes with FLT3 inhibitors and chemotherapeutic agents to enhance AML cell death.
    \item The proposed IC50 concentration range (100 nM to 10 $\mu$M) is appropriate for testing in MOLM13 cells.
    \item KIRA6 has a manageable safety profile and tolerable toxicity in preclinical models.
    \item KIRA6 can inhibit the production of short-lived proteins, like MYC and MCL-1, and reduce inflammatory signaling.
\end{itemize}

\textbf{Reasoning about assumptions}
\begin{itemize}
    \item Plausible: [1] and [5] support the notion of targeting IRE1$\alpha$ in AML. While no direct information shows if KIRA6 effective inhibits IRE1$\alpha$ in AML MOLM 3 cells,  it likely works as an IRE1$\alpha$ inhibitor like in other cell types. Needs experimental validation.
    \item Plausible: This is a core function of the UPR pathway in which IRE1$\alpha$ inhibition, based on the literature [5], should induce this cascade of events causing stress response leading to cell death if prolonged and severe.
    \item Plausible, but requires careful investigation: AML cells, particularly those with FLT3-ITD, have higher protein synthesis rates and are inherently more sensitive to ER stress. While this is not demonstrated in the literature, the rationale behind targeting ER stress response in leukemia is that tumor cells, particularly with FLT3-ITD, are expected to be more reliant on these processes and therefore may be more susceptible, but this assumption needs further evaluation.
    \item Plausible, but requires testing: The rationale for synergy is based on the proposed mechanism and general principles of targeting multiple pathways in cancer. [9] offers a related example of synergy with venetoclax, a drug also being tested in FLT3-ITD AML. This hypothesis needs further in vitro testing to be demonstrated.
    \item Plausible: The suggested range is based on literature and seems reasonable as a starting point. This needs to be empirically confirmed in an IC-50 assay. The provided range in the idea is a well justified assumption that can be verified during the IC-50 viability assays where a broader range of concentrations would capture the optimal values.
    \item Unknown: This is a vital assumption requiring thorough preclinical testing. [7] has information about pharmacology and toxicity, which provides context. Currently, limited safety data exists for KIRA6 so the drug safety profile is unknown and needs experiments to verify if the drug is safe enough.
    \item Plausible: IRE1$\alpha$ inhibition and ER stress induction can trigger downstream effects that impact protein translation and inflammatory responses based on the mechanism described in the UPR [5]. [1] suggests that IRE1$\alpha$ impacts venetoclax efficacy. While this is not demonstrated in the literature for KIRA6 in AML, this remains an experimentally verifiable assumption.
\end{itemize}

\textbf{Improvements to the idea}
\begin{itemize}
    \item Broader cell line testing: The idea primarily focuses on MOLM13. Expanding to other AML cell lines, including those with different FLT3 mutations and other genetic backgrounds, would strengthen the proposal. This would require additional experiments as well.
    \item Deeper exploration of resistance mechanisms: While mentioned, the idea could benefit from a more detailed analysis of how KIRA6 addresses specific resistance mechanisms in AML. The idea can also benefit from additional experiments regarding resistance mechanism impacts.
    \item Investigate drug resistance: There is minimal data regarding KIRA6 impacting drug efflux pumps and other drug resistance mechanisms. It would be beneficial to investigate how impacting the UPR could impact drug resistant cells, which are a major reason for treatment failure in AML.
    \item Detailed comparison to other IRE1$\alpha$ inhibitors: If there are other IRE1$\alpha$ inhibitors, comparing and contrasting KIRA6 with them and justifying the reasons for choosing this particular drug would add to the idea's strength.
\end{itemize}
\end{tcolorbox}
\vspace{0.1cm}
\caption{\textbf{Detailed AI co-scientist output for repurposing KIRA6, an IRE1$\alpha$ inhibitor for AML (continued)}.}
\label{fig:ex_kira6_6}
\end{figure}


\begin{figure}[htbp!]
\begin{tcolorbox}[
    colback=black!5!white,
    colframe=black!60!white,
    title=\textbf{KIRA6 for AML (continued)},
    fonttitle=\bfseries,
    arc=3mm,
    boxrule=1pt,
    bottomrule=2pt,
]
\scriptsize

\textbf{Reasoning about correctness and testing}

The idea is plausible and well-reasoned based on known mechanisms of IRE1$\alpha$ and the UPR in cancer. However, key assumptions about selectivity and synergy require experimental validation. The proposed experimental plan is comprehensive and addresses the key questions.

The idea is novel in its application of KIRA6 to AML, particularly in the context of FLT3-ITD. The focus on combination therapy and overcoming resistance is also promising.

It is recommended to test this idea due to its plausibility, novelty, and potential clinical significance. While there are uncertainties regarding safety and efficacy, the potential benefits warrant further investigation.

\textbf{Answer: 4}
\\
\\
\textbf{Novelty review}
\begin{itemize}
    \item Targeting the IRE1$\alpha$-XBP1s pathway to enhance venetoclax effectiveness in AML [1]: This abstract directly relates to the idea by exploring the IRE1$\alpha$ pathway in AML, although in combination with venetoclax, not as a monotherapy or in combination with FLT3 inhibitors.
    \item The unfolded protein response: A novel therapeutic target in acute leukemias [5]: This review discusses the UPR as a target in acute leukemias, providing a broader context for the idea's focus on IRE1$\alpha$.
    \item Selective degradation of mutant FMS-like tyrosine kinase-3 requires BIM-dependent depletion of heat shock proteins [6]: This abstract explores the mechanisms of FLT3 degradation and its connection to BIM and heat shock proteins, relevant to the idea's focus on FLT3-ITD positive AML.
\end{itemize}

\textbf{Already explored aspects}
\begin{itemize}
    \item  Targeting the UPR in AML: The idea of targeting the unfolded protein response (UPR) in AML is not entirely novel. Abstract [5] discusses the UPR as a therapeutic target in acute leukemias, including AML. Abstract [1] specifically explores targeting IRE1$\alpha$ in combination with venetoclax, suggesting existing interest in this pathway for AML.  However, the use of KIRA6 as a single agent or in combination with FLT3 inhibitors appears less explored.
    \item FLT3 inhibitors in AML: Targeting FLT3 in AML with tyrosine kinase inhibitors is a well-established approach, with multiple approved drugs.  Abstracts [2, 3] focus on FLT3 inhibitors, highlighting the importance of this target but also the challenges of resistance [6].
\end{itemize}

\textbf{Novel aspects}
\begin{itemize}
    \item Repurposing KIRA6 for AML: While IRE1$\alpha$ has been targeted in AML [1], the repurposing of KIRA6 specifically for AML treatment, especially in the context of FLT3-ITD positive disease and potential combination with FLT3 inhibitors or standard chemotherapies, seems novel based on the provided abstracts.  The idea explores KIRA6's impact on multiple pathways (UPR, apoptosis, NF-$\kappa$B, MYC) related to AML proliferation and resistance.
    \item Focusing on FLT3-ITD positive AML \& combination therapy: Although FLT3 inhibitors are used in AML, the idea of combining KIRA6 with FLT3 inhibitors or chemotherapeutics to specifically target FLT3-ITD positive AML cells and potentially overcome resistance addresses a current clinical need. The proposed mechanism - disrupting protein homeostasis in already stressed FLT3-ITD cells - offers a rationale for this combination approach.
    \item Combination of KIRA6 with other AML drugs \& mechanism of action: KIRA6 is not mentioned in any of the abstracts, and therefore the proposed combination of KIRA6 and FLT3i to induce apoptosis is also novel. The described mechanism has strong logical support, as the cells' increased translation requirements and ER stress from the FLT3i mutation could greatly increase the effectiveness of IRAK1 inhibition, which may itself directly induce apoptosis, and also improve efficacy of existing drugs.
\end{itemize}

\textbf{Novelty review}

The idea presents a moderate level of novelty. Repurposing KIRA6 for AML, particularly in the context of FLT3-ITD positive disease and combination therapy, is a promising approach. However, targeting IRE1$\alpha$ in AML has been explored [1], albeit not with this specific drug and with different existing treatment approaches.  The idea's strength lies in its detailed mechanistic rationale, combination therapy focus, and defined experimental plan.  It is crucial to validate the proposed mechanism and selectivity across different AML cell lines and patient samples before concluding its true novelty. \\

\textbf{Improvements to the idea}
\begin{itemize}
    \item Literature search: Conduct a thorough literature search beyond the provided abstracts to confirm the novelty of KIRA6 in AML. Investigate any existing research on KIRA6 in other hematological malignancies.
    \item Selectivity testing:  Expand selectivity testing to include other healthy cell types beyond normal myeloid cells, to address potential off-target effects and demonstrate greater safety margin.
    \item Combination exploration: Further explore rational combinations beyond FLT3 inhibitors, including chemotherapeutic agents and other targeted therapies.
    \item Resistance mechanisms:  Thoroughly investigate the impact of KIRA6 on various AML resistance mechanisms through in vitro testing, using resistant cell lines and patient-derived samples.
    \item In vivo efficacy and toxicity profile: Expand the in vivo studies with multiple cell lines and patient-derived xenograft models, with a focus on establishing the efficacy and toxicity profile of KIRA6, both as single-agent and in the proposed combination strategy.
\end{itemize}

\end{tcolorbox}
\vspace{0.1cm}
\caption{\textbf{Detailed AI co-scientist output for repurposing KIRA6, an IRE1$\alpha$ inhibitor for AML (continued)}.}
\label{fig:ex_kira6_7}
\end{figure}


\begin{figure}[htbp!]
\begin{tcolorbox}[
    colback=black!5!white,
    colframe=black!60!white,
    title=\textbf{KIRA6 for AML (continued)},
    fonttitle=\bfseries,
    arc=3mm,
    boxrule=1pt,
    bottomrule=2pt,
]
\scriptsize
\textbf{Reasoning about novelty and recommendation}

The idea is novel enough to warrant further exploration.  While targeting IRE1$\alpha$ isn't completely new, using KIRA6 in AML, especially in the context outlined, hasn't been extensively investigated, and it also has not been previously examined as a companion drug with other therapies used in AML.  The proposed combination with FLT3 inhibitors and the detailed rationale provide a strong foundation. The comprehensive experimental plan should be executed to confirm the preliminary findings and assess the true potential of KIRA6 in AML.  If the in vitro and in vivo findings are positive, the idea would be worthy of publication in a specialized journal focused on hematological malignancies or drug repurposing.


\textbf{Answer: 3}
\\

[1] Targeting the IRE1$\alpha$-XBP1s Pathway to Enhance Venetoclax Effectiveness in AML | Blood | American Society of Hematology
(https://ashpublications.org/blood/article/144/Supplement\%201/4144/533082/Targeting-the-IRE1-XBP1s-Pathway-to-Enhance)

[2] FLT3 inhibitors in acute myeloid leukemia: ten frequently asked questions | Leukemia
(https://www.nature.com/articles/s41375-019-0694-3)

[3] Inhibition of FLT3-ITD Kinase in Acute Myeloid Leukemia by New Imidazo[1,2-b]pyridazine Derivatives Identified by Scaffold Hopping | Journal of Medicinal Chemistry
(https://pubs.acs.org/doi/10.1021/acs.jmedchem.3c00575)

[4] Repurposing approved drugs on the pathway to novel therapies - PMC
(https://pmc.ncbi.nlm.nih.gov/articles/PMC7018532/)

[5] The Unfolded Protein Response: A Novel Therapeutic Target in Acute Leukemias - PMC
(https://pmc.ncbi.nlm.nih.gov/articles/PMC7072709/)

[6] Drugging the unfolded protein response in acute leukemias | Journal of Hematology \& Oncology | Full Text
(https://jhoonline.biomedcentral.com/articles/10.1186/s13045-015-0184-7)

[7] Selective degradation of mutant FMS-like tyrosine kinase-3 requires BIM-dependent depletion of heat shock proteins | Leukemia
(https://www.nature.com/articles/s41375-024-02405-5)

[8] Pharmacology and Toxicology Testing: IC50 for Tumor Cell Lines - Altogen Labs
(https://altogenlabs.com/pre-clinical-research-services/pharmacology-and-toxicology-testing-ic50-for-tumor-cell-lines/)

[9] Targeting Mitochondrial Structure Sensitizes Acute Myeloid Leukemia to Venetoclax Treatment | Cancer Discovery | American Association for Cancer Research
(https://aacrjournals.org/cancerdiscovery/article/9/7/890/41983/Targeting-Mitochondrial-Structure-Sensitizes-Acute)
\\
\\
\textbf{Critiques}

Here's a summary of the negative critiques from the reviews:
\begin{itemize}
    \item Limited safety data: The biggest concern is the lack of human safety data for KIRA6, requiring thorough preclinical toxicity studies.  The drug's safety profile is currently unknown and needs experimental validation.
    \item Limited scope of cell line testing:  Testing primarily focuses on MOLM13. Expanding to other AML cell lines (with different FLT3 mutations and other genetic backgrounds) and primary patient samples is necessary.
    \item Limited comparison to other drugs:  The idea would benefit from comparing KIRA6 to other IRE1$\alpha$ inhibitors or similar drugs to justify its selection and benchmark efficacy/selectivity. Minimal data exists regarding KIRA6 compared to other drugs.
    \item Lack of pre-clinical and clinical data:  While proposed, in vivo and toxicity data are currently lacking for KIRA6, despite the comprehensive experimental plan.
    \item Assumptions requiring validation: Several assumptions about KIRA6's selectivity for AML cells over healthy cells, synergy with other drugs, impact on drug resistance mechanisms, and the suggested IC50 concentration range need experimental validation.
    \item Limited detail in some areas: The rationale for the specific IC50 range could be stronger. The exploration of resistance mechanisms and the details of synergy studies could also be more in-depth. More detail on how KIRA6 impacts MYC, NF-kB and inflammatory pathways is needed.  The connection between UPR activation and a lower incidence of FLT3 mutations needs further investigation.
    \item Potential for resistance: While the idea aims to overcome resistance, there's still a need to investigate how KIRA6 itself might contribute to new resistance mechanisms over time.
\end{itemize}

Although the idea is considered plausible, novel, and promising, these weaknesses need to be addressed through further research and experimentation.

\end{tcolorbox}
\vspace{0.1cm}
\caption{\textbf{Detailed AI co-scientist output for repurposing KIRA6, an IRE1$\alpha$ inhibitor for AML (continued)}.}
\label{fig:ex_kira6_8}
\end{figure}


\clearpage
\section{An example of tool use in the AI co-scientist with AlphaFold}
\label{sec:alphafold}
The AI co-scientist is a general purpose system broadly applicable across different areas of science and medicine. To better understand the capabilities and limitations of the system, we task it with the goal of suggesting protein sequences with specific properties. Determining the correct primary amino acid sequence with the desired properties is an essential part of protein engineering. While LLM-based systems can predict protein properties and suggest modifications~\citep{wang2025large}, they can sometimes generate incorrect sequences (i.e., hallucinations). To address this, we integrate AlphaFold~\citep{jumper2021highly}, a specialized AI system for predicting protein 3D structure, into our co-scientist. AlphaFold acts as a validation tool, evaluating the structural plausibility of sequences proposed by the co-scientist and provides feedback. This increases the reliability of the sequence design optimization process, which can be further validated with wet laboratory experiments. The approach to integrate tools highlights how specialized AI models can work in collaboration with more general AI systems like the AI co-scientist, facilitating the solution of complex challenges like protein design.

As an illustrative example, we used AlphaFold to assess a co-scientist's proposed modification to the OCT4 (octamer-binding transcription factor 4) protein (Appendix \cref{fig:fig_af}), one of the four Yamanaka factors~\citep{takahashi2007induction}, to increase binding affinity of its DNA binding domain. The co-scientist suggested adding a mechano-sensitive loop to the POU domain (a family of eukaryotic transcription factors) and a dynamic phosphorylation site outside of it. The co-scientist first verified the proposed sequence against the UniProt database via web-search. AlphaFold then predicted the 3D structure of the modified protein, suggesting that the modifications maintained structural stability. These predictions were used to refine the co-scientist's hypothesis, allowing it to improve its protein sequence design in subsequent iterations. We also independently validated the modification using ESM-2 \citep{lin2022language} and RoseTTAFold \citep{baek2021accurate}. ESM-2 predicted an increased log-likelihood ratio and a similar predicted local distance difference test (pLDDT), and RoseTTAFold predicted similar confidence score (GDT), compared to the original sequence. The insertion and modification did not seem to disrupt SOX2 and OCT4 interactions, indicated by the similar pLDDT scores between the original and modified OCT4 sequences. However, this example is for demonstration purposes only. Further \textit{in silico} analysis (e.g., predicting binding affinity and off-target effects), and thorough laboratory validation are necessary to confirm that the proposed modifications actually improve the complex roles of OCT4 binding, while maintaining SOX2 interaction integrity, during pluripotency.

\begin{figure}[htbp!]
\centering
\includegraphics[width=0.92\textwidth,keepaspectratio]{figures/fig_af_mechano.pdf}
\vspace{0.1cm}
\caption{\textbf{AlphaFold predicted protein 3D structure and metrics for original OCT4 and AI co-scientist suggested modifications.} (left panel) original OCT4 sequence with SOX2 and DNA binding (right panel) modified OCT4 sequence with SOX2 and DNA binding. The left 3D structure in each panel is the POU domain of the corresponding OCT4 sequence. The predicted template modeling (pTM) score, the interface predicted template modeling (ipTM), and predicted local distance difference test (pLDDT) are derived from the AlphaFold outputs.}
\label{fig:fig_af}
\end{figure}
\vspace{-0.3cm}

Combining AlphaFold with the co-scientist framework offers a powerful approach for both improving existing proteins and designing entirely new ones. This integrated system allows researchers to iteratively optimize protein sequences for enhanced properties (e.g., stability, binding affinity, or catalytic activity) or putatively to create proteins with novel functions. It enables exploration of protein design while ensuring structural feasibility. Future work will focus on experimentally validating these capabilities and applying them to targeted protein design efforts as well as expansion to integration of other specialized AI tools with the co-scientist.

\newpage
\setlength\bibitemsep{3pt}
\printbibliography
\balance
\clearpage

