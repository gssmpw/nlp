\section{Introduction}
\label{sec:introduction}

\begin{figure}[htbp!]
\centering
\includegraphics[width=\textwidth,keepaspectratio]{figures/fig1.pdf}
\vspace{0.1cm}
\caption{\textbf{The AI co-scientist system design and experimental validation summary.} (a) Here, we illustrate the different components of the the AI co-scientist multi-agent system, and its interaction paradigm with scientists. Given a research goal in natural language, the co-scientist generates novel research hypotheses and proposals. The system employs specialized agents — Generation, Reflection, Ranking, Evolution, Proximity (which evaluates relatedness), Meta-review (which provides high level analysis) — to continuously generate, debate, and evolve research hypotheses within a tournament framework. Feedback from the tournament enables iterative improvement, creating a self-improving loop towards novel and high-quality outputs. The co-scientist leverages tools, including web search and specialized AI models to improve the grounding and quality of generated research hypotheses. Scientists can converse with the co-scientist in natural language to specify research goals, incorporate constraints, provide feedback and suggest new directions for explorations via the designated user interface. (b) We perform end-to-end validation of the co-scientist generated hypotheses in three important topics of biomedicine with varied complexity--- suggesting novel drug repurposing candidates for acute myeloid leukemia (AML) (upper panel), discovering novel epigenetic targets for liver fibrosis treatment (middle panel), and recapitulating the discovery of novel mechanism of gene transfer evolution in bacteria key to anti-microbial resistance (lower panel). The co-scientist's hypotheses for these three settings are externally, independently validated by \textit{in vitro} laboratory experiments and detailed in separate preprints co-timed with this work. In the figure, blue denotes expert scientist inputs while red denotes the co-scientist agents or outputs.
}
\label{fig:system-overview}
\end{figure}

Human ingenuity and creativity propel the advancement of fundamental research in science and medicine. However, researchers, particularly in biomedicine, are faced with a breadth and depth conundrum. The complexity of biomedical topics require increasingly deep and specific subject matter expertise, while leaps in insight may still arise from broad knowledge bridging across disciplines. With the rapid rise in scientific publications and the availability of numerous technologies for specialized high-throughput assays, mastery of both discipline-specific depth and trans-disciplinary insight can be challenging.

Despite these challenges, many modern breakthroughs have emerged from trans-disciplinary endeavours. Emmanuelle Charpentier and Jennifer Doudna won the 2020 Nobel Prize in Chemistry for their work on CRISPR~\citep{jinek2012programmable}, which combined techniques and strategies ranging from microbiology to genetics to molecular biology. These benefits of synergy have also been seen beyond experimental biomedicine in numerous other areas of science. Notably, Geoffrey Hinton and John Hopfield combined ideas from physics and neuroscience~\citep{hopfield1982neural, hinton1986learning} to develop artificial intelligence (AI) systems, which were awarded the 2024 Nobel Prize in Physics.

There has been rapid technological progress in AI towards generally intelligent and collaborative systems, which might empower scientists in creatively traversing and expertly reasoning across disciplinary domains. Such systems are capable of advanced reasoning~\citep{guo2025deepseek, jaech2024openai, team2024gemini}, multimodal understanding~\citep{team2024gemini}, and agentic behaviors~\citep{wiesinger2024agents} such as the ability to use tools to solve complex tasks over long time horizons. Further, the trends with distillation~\cite{hinton2015distilling} and inference time compute costs~\citep{team2024gemini, team2024gemma} indicate that such intelligent and general AI systems are rapidly becoming more affordable and available. Motivated by the aforementioned unmet needs in the modern discovery process in science and medicine and building on the advancements in frontier AI~\citep{leslie2024frontier}, we develop and introduce an AI co-scientist system.

The co-scientist is designed to act as a helpful assistant and collaborator to scientists and to help accelerate the scientific discovery process. The system is a compound, multi-agent AI system~\cite{chen2024more} building on Gemini 2.0 and designed to mirror the reasoning process underpinning the scientific method~\citep{gower2012scientific}. Given a research goal specified in natural language, the system can search and reason over relevant literature to summarize and synthesize prior work and build on it to propose novel, original research hypotheses and experimental protocols for downstream validations~(\cref{fig:system-overview}a). The co-scientist provides grounding for its recommendations by citing relevant literature and explaining the reasoning behind its proposals. 

This work does not aim to completely automate the scientific process with AI. Instead, the co-scientist is purpose-built for a ``scientist-in-the-loop'' collaborative paradigm, to help domain experts augment their hypothesis generation process and guide the exploration that follows. Scientists can specify their research goals in simple natural language, including informing the system of desirable attributes for the hypotheses or research proposals it should create and the constraints that the synthesized outputs should satisfy. They can also collaborate and provide feedback in a variety of ways, including directly supplying their own ideas and hypotheses, refining those generated by the system, or using natural language chat to guide the system and ensure alignment with their expertise.

The co-scientist works through a significant scaling of the test-time compute paradigm~\citep{snell2024scaling, brown2019superhuman, silver2016mastering} to iteratively reason, evolve, and improve the outputs as it gathers more knowledge and understanding. Underpinning the system are thinking and reasoning steps---notably a self-play based scientific debate step for generating novel research hypotheses; tournaments that compare and rank hypotheses via the process of finding win and loss patterns, and a hypothesis evolution process to improve their quality. Finally, the agentic nature of the system enables it to recursively self-critique its output and use tools such as web-search to provide itself with feedback to iteratively refine its hypotheses and research proposals.

While the co-scientist system is general-purpose and applicable across multiple scientific disciplines, in this study we focus our development and validation of the system to biomedicine. We validate the co-scientist's capability in three impactful areas of biomedicine with varied complexity: (1) drug repurposing, (2) novel treatment targets discovery, and (3) new mechanistic explanations for antimicrobial resistance (\cref{fig:system-overview}b).

Drug development is an increasingly time-consuming and expensive process~\citep{ringel2020breaking} in which new therapeutics require restarting many aspects of the discovery and development process for each indication or disease (roughly 70\% of drug approvals are for new drugs). In contrast, drug repurposing—--identifying novel therapeutic indications for drugs beyond their original intended use--—has emerged as a compelling strategy to address these challenges~\citep{pushpakom2019drug}. Successful examples of repurposing include Humira (adalimumab) and Keytruda (pembrolizumab), both of which have become among the most successful drugs in history.~\citep{pushpakom2019drug}. The process typically involves analyzing molecular signatures, signaling pathways, drug interactions, clinical trial results, adverse event reports, and other literature-based information~\citep{xia2024drug}, along with off-label use data and, in some cases, patient experiences. However, drug repurposing is limited by several factors: (1) the need for extensive expertise across biomedical, molecular biology, and biochemical systems, (2) the inherent complexity of mammalian biological systems, and (3) the time-intensive nature of traditional computational biology analyses required. We leverage the co-scientist to generate predictions for large-scale drug repurposing, validating the generated predictions using a combination of computational biology, expert clinician feedback, and \textit{in vitro} wet-lab validation approaches. Notably, our system has proposed novel repurposing candidates for acute myeloid leukamia (AML) that inhibit tumor viability at clinically relevant concentrations \textit{in vitro} across multiple AML cell lines.

Unlike drug repurposing, which is a combinatorial search problem through a large but constrained set of drugs and diseases, identifying novel treatment targets for diseases presents a more significant challenge, traditionally requiring extensive literature review, deep biological understanding, sophisticated hypothesis generation and complex experimental validation strategies. The uncertainty of identifying novel treatment targets is significantly greater than in drug repurposing, as it involves not only repurposing existing compounds but also uncovering entirely new components and mechanisms within biological systems. This target discovery process can be inefficient, potentially leading to suboptimal hypothesis selection and prioritization for \textit{in vitro} and \textit{in vivo} experimentation. Given the high costs and time associated with experimental validation, a more effective approach is needed. We probe the capabilities of the co-scientist to propose, rank, and provide experimental protocols for novel research hypotheses pertaining to target discovery. To demonstrate this capability, we focus on liver fibrosis, a prevalent and serious disease, showcasing the co-scientist's potential to discover novel treatment targets amenable to experimental validation. In particular, the co-scientist has suggested novel epigenetic targets demonstrating significant anti-fibrotic activity in human hepatic organoids.

As a third validation of the capabilities of our system, we focus on generation of hypotheses to explain mechanisms related to gene transfer evolution in bacteria pertaining to antimicrobial resistance (AMR) - mechanisms developed by microbes to circumvent drug applications used to fight infections. This is arguably an even more complex challenge than drug repurposing and target discovery and involves understanding of not only the molecular mechanisms of gene transfer (conjugation, transduction, and transformation) but also the ecological and evolutionary pressures that drive the spread of AMR genes: a system-level problem with many interacting variables. This is also an important healthcare challenge with increasing rates of infections and deaths worldwide~\citep{keown2014antimicrobial}. In this validation, researchers instructed the AI co-scientist to explore a topic that had already been subject to novel discovery by their independent research group. Notably, at the time of instructing the AI co-scientist system, the researchers' novel experimental insights had not yet been published or revealed in the public domain. The system was instructed to hypothesize how capsid-forming phage-inducible chromosomal islands (cf-PICIs) exist across multiple bacterial species. The system independently proposed that cf-PICIs interact with diverse phage tails to expand their host range. This \textit{in silico} discovery mirrored the novel and experimentally validated results that expert researchers had already performed, as detailed in the co-timed report~\citep{he2025chimeric, penades2025ai}.

Overall, our key contributions are summarized as follows:
\begin{itemize}[leftmargin=1.5em,rightmargin=0em]
    \item\textbf{Introducing an AI co-scientist.} We develop and introduce an AI co-scientist that goes beyond literature summarization and ``deep research'' tools to assist scientists in uncovering new knowledge, novel hypothesis generation and experimental planning.
    \item\textbf{Significant scaling of test-time compute paradigm for scientific reasoning.} The co-scientist is built on a Gemini 2.0 multi-agent architecture, utilizing an asynchronous task execution framework. This framework allows the system to flexibly allocate computational resources to scientific reasoning, mirroring key aspects of the scientific method. Specifically, the system uses self-play strategies, including a scientific debate and a tournament-based evolution process, to iteratively refine hypotheses and research proposals creating a self-improving loop. Using automated evaluations across 15 complex expert curated open scientific goals, we demonstrate the benefits of scaling the test-time compute paradigm with the AI co-scientist outperforming other state-of-the-art (SOTA) agentic and reasoning models in generating high quality hypotheses for complex problems.
    \item\textbf{Expert-in-the-loop scientific workflow.} Our system is designed for collaboration with scientists. The system can flexibly incorporate conversational feedback in natural language from scientists and co-develop, evolve and refine outputs.
    \item \textbf{End-to-end validation of the co-scientist in important topics in biomedicine.} We present end-to-end validation of novel AI-generated hypotheses through new empirical findings in three distinct and increasingly complex areas of biomedicine: drug repurposing, novel target discovery, and antimicrobial resistance. The AI co-scientist predicts novel repurposing drugs for AML, identifies novel epigenetic treatment targets grounded in preclinical evidence for liver fibrosis, and proposes novel mechanisms for gene transfer in bacterial evolution and antimicrobial resistance. These discoveries from the AI co-scientist have been validated in wet-lab settings and are detailed in separate, co-timed technical reports.
\end{itemize}


\section{Related Works}
\subsection{Reasoning models and test-time compute scaling}
The modern revolution in foundation AI models~\citep{bommasani2021opportunities} and large language models (LLMs) has been largely driven by advances in pre-training techniques~\citep{erhan2010does, radford2018improving}, leading to breakthroughs in models like the GPT and Gemini family~\citep{team2023gemini, achiam2023gpt}. These models, trained on increasingly massive internet-scale and multimodal datasets, have demonstrated impressive abilities in language understanding and generation leading to breakthrough performance in a variety of benchmarks~\citep{chowdhery2022palm, google2023palm2}.  However, a key area of ongoing development is enhancing their \textit{reasoning} capabilities. This has led to the emergence of ``reasoning models'' which go beyond simply predicting the next word and instead attempt to mimic human thought processes~\citep{wei2022chain}. One promising direction in this pursuit is the test-time compute paradigm.  This approach moves beyond solely relying on the knowledge acquired during pre-training and allocates additional computational resources during inference to enable System-2 style thinking---slower deliberate reasoning to reduce uncertainty and progress optimally towards the goal~\citep{kahneman2011thinking}. This concept emerged with early successes such as AlphaGo~\citep{silver2016mastering}, which used Monte Carlo Tree Search (MCTS) to explore game states and strategically select moves, and Libratus~\citep{brown2019superhuman}, which employed similar techniques to achieve superhuman performance in poker. This paradigm has now found applications in LLMs, where increased compute at test-time allows for more thorough exploration of possible responses, leading to improved reasoning and accuracy~\citep{wei2022chain, yao2024tree, zelikman2022star, chen2024more, snell2024scaling,4928, muennighoff2025s1, tu2024towards}. Recent advancements, like the Deepseek-R1 model~\citep{guo2025deepseek}, further demonstrate the potential of test-time compute by leveraging reinforcement learning to refine the model's ``chain-of-thought'' and enhance complex reasoning abilities over longer horizons. In this work, we propose a significant scaling of the test-time compute paradigm using inductive biases derived from the scientific method to design a multi-agent framework for scientific reasoning and hypothesis generation without any additional learning techniques.

\subsection{AI-driven scientific discovery}
AI-driven scientific discovery represents a paradigm shift in how research is conducted across various scientific domains. Recent advancements, particularly the development of large deep learning and generative models, have cemented AI's role in scientific discovery. This is best exemplified by AlphaFold 2's remarkable progress in the grand challenge of protein structure prediction, which has revolutionized structural biology and opened new avenues for drug discovery and materials science~\citep{jumper2021highly}. Other notable examples include the development of novel antibiotics, protein binder design,  and material discovery with AI~\citep{wong2024discovery, zambaldi2024novo, merchant2023scaling}. 

Building on these successes with specialized, bespoke AI models, there has been recent work exploring the even more ambitious goal of fully integrating AI, especially modern LLM-based systems, into the complete research workflow, from initial hypothesis generation all the way to manuscript writing. This end-to-end integration represents a significant shift, presenting both unprecedented opportunities and significant challenges as the field moves beyond specialized AI tools toward realizing the potential of AI as an active collaborator, or even, as some envision, a nascent ``AI scientist''~\citep{lu2024ai, schmidgall2025agent}.

As an example of this shift, Liang et al.~\cite{liang2024can} directly assessed the utility of LLMs for providing feedback on research manuscripts. Through both a retrospective analysis of existing peer reviews and a prospective user study, they demonstrated the significant concordance between LLM-generated feedback and that of human reviewers. Their study, using GPT-4~\citep{openai2023gpt4}, found that a majority of researchers perceived LLM-generated feedback as helpful, and in some instances, even more beneficial than feedback from human colleagues. However, while valuable, their work focuses solely on the feedback stage of the scientific process, leaving open the question of how LLMs might be integrated into the full research cycle, from hypothesis formation to experimental validation and manuscript writing.

Another effort embodying this shift is PaperQA2~\citep{skarlinski2024language}, an AI agent for scientific literature search and summarization. The authors claimed to surpass PhD and postdoc researchers on multiple literature research tasks, as measured both by performance on objective benchmarks and human evaluations. While the system is a useful for synthesizing information, it does not engage in scientific reasoning for novel hypothesis generation.

HypoGeniC, a system proposed by Zhou et al.~\cite{zhou2022least}, tackles hypothesis generation by iteratively refining hypotheses using LLMs and a multi-armed bandit-inspired approach. The process begins with a small set of examples, from which initial hypotheses are generated. These hypotheses are then iteratively updated through exploration and exploitation, guided by a reward function based on training accuracy.  This refined set of hypotheses is subsequently used to construct an interpretable classifier. However, the method's reliance on retrospective data for evaluation means the degree to which the system can generate truly novel hypotheses remains an open question. Furthermore, the system lacks end-to-end validation beyond subjective human evaluations.

Ifargan et al.~\cite{ifargan2025autonomous} present ``data-to-paper'', a platform that systematically guides multiple LLM and rule-based agents to generate research papers, with automated feedback mechanisms and information tracing for verification. However, the evaluations are limited to recapitulating existing peer-reviewed publications and its unclear if the system can generate truly novel, yet grounded hypothesis and research proposals.

Virtual Lab~\citep{swanson2024virtual} is another closely related work. Here, the authors propose a team of LLM agents with a ``principal investigator'' LLM guiding a team of specialized LLM agents to solve a scientific problem. The LLM team receives high level human supervision. The authors demonstrate the utility of their work by leveraging Virtual Lab to design nanobody binders to recent variants of SARS-CoV-2 with experimental validation. While similar in spirit, there are significant design differences to our approach and the generality of the system remains unclear.

Boiko et al.~\citep{boiko2023autonomous} introduced ``Coscientist'', a multi-agent system powered by GPT-4, designed for autonomous execution of complex chemical experiments. This system integrates capabilities such as web and document searching, and code execution, to facilitate independent experimental design, planning, and execution. In addition to similar sounding names, both ``Coscientist'' and our system share the overarching goal of accelerating scientific discovery through AI. However, there are several important distinctions. Notably, ``Coscientist'' is quite narrowly focused on chemical research while ours is much broadly applicable across science. Secondly, our system has important technical innovations that lead to a self-improving system that can uncover new, original knowledge while their approach is a more vanilla-stitching of GPT-4 based agents. Finally, despite the name, ``Coscientist'' prioritizes a high degree of autonomy in experimental execution, directly interfacing with laboratory hardware. Our system, instead, is explicitly designed as a collaborative tool, emphasizing a ``scientist-in-the-loop'' approach and centers on the more cognitive aspects of the research process.

Finally, Lu et al.~\cite{lu2024ai} propose ``The AI Scientist'', a fully automated system designed to conduct research using multiple collaborating LLM agents. These agents handle all stages of the research process, from defining research problems and conducting literature reviews to designing and executing experiments, and even writing up the results. The design shares similarities with our work---the key differences being our focus on the scaling of the test-time compute paradigm to generate high quality hypotheses and research proposals. Secondly, their proposed system has limited automated evaluations; in contrast, our work has a combination of automated, human expert and end-to-end wet lab validations. Finally, our goal is to not to automate scientific discovery, rather to build a helpful AI collaborator for scientists.

\subsection{AI for biomedicine}
More broadly, large AI models are increasingly demonstrating their potential in biomedical science. Both general purpose (GPT-4, Gemini) and specialized LLMs (Med-PaLM, Med-Gemini, Galactica, Tx-LLM) have shown strong performance on biomedical reasoning and question-answering benchmarks~\citep{team2023gemini, achiam2023gpt, singhal2022large, saab2024capabilities, taylor2022galactica, chaves2024tx}. Beyond benchmarks, Med-PaLM 2, was successfully applied to identify causative murine genetic factors for traits such as diabetes, cataracts, and hearing loss~\cite{tu2023genetic}---an early example of hypothesis generation and LLM-assisted discovery. We have also seen the exciting development of specialized foundation and large language models trained on DNA, RNA and protein sequences with a variety of applications~\citep{nguyen2024sequence, lin2023evolutionary, ruffolo2024design, shaw2024protex}. Although AI in biology and medicine often necessitates specialization, the rapid progress of frontier AI models has blurred the distinction. As these models grow in scale, data diversity, and complexity, they continue to achieve breakthroughs in areas once thought to require domain-specific AI. Our co-scientist system, with its modular multi-agent architecture, is flexibly designed to build on top of these advancements in general-purpose frontier AI models and leverage specialized AI models as tools to enhance the capabilities.
 
Drug repurposing is an important area of validation experiments in this work. The traditional approach to this task requires both computational and experimental approaches and a comprehensive understanding of disease-drug interactions~\cite{pushpakom2019drug, krishnamurthy2022drug}. While methods like knowledge graphs with graph convolutional networks have shown promise~\cite{zitnik2018modeling, morselli2021network}, their applicability is limited by the initial knowledge graph's scope.  TxGNN~\cite{huang2024foundation}, an example of a specialized biomedical foundation model with a graph based approach, addresses ``zero-shot'' repurposing for novel diseases but remains dependent on the underlying knowledge graph's quality and lacks sufficient scalability and explainability. Furthermore, no end-to-end validations of the model predictions were reported in the study. In contrast, our work, leveraging state-of-the-art LLMs in the co-scientist setup, is more scalable. We report a combination of expert evaluations and wet-lab experiments to validate the system predictions.

\section{Method}\label{sec:method}
\begin{figure}
    \centering
    \includegraphics[width=0.85\textwidth]{imgs/heatmap_acc.pdf}
    \caption{\textbf{Visualization of the proposed periodic Bayesian flow with mean parameter $\mu$ and accumulated accuracy parameter $c$ which corresponds to the entropy/uncertainty}. For $x = 0.3, \beta(1) = 1000$ and $\alpha_i$ defined in \cref{appd:bfn_cir}, this figure plots three colored stochastic parameter trajectories for receiver mean parameter $m$ and accumulated accuracy parameter $c$, superimposed on a log-scale heatmap of the Bayesian flow distribution $p_F(m|x,\senderacc)$ and $p_F(c|x,\senderacc)$. Note the \emph{non-monotonicity} and \emph{non-additive} property of $c$ which could inform the network the entropy of the mean parameter $m$ as a condition and the \emph{periodicity} of $m$. %\jj{Shrink the figures to save space}\hanlin{Do we need to make this figure one-column?}
    }
    \label{fig:vmbf_vis}
    \vskip -0.1in
\end{figure}
% \begin{wrapfigure}{r}{0.5\textwidth}
%     \centering
%     \includegraphics[width=0.49\textwidth]{imgs/heatmap_acc.pdf}
%     \caption{\textbf{Visualization of hyper-torus Bayesian flow based on von Mises Distribution}. For $x = 0.3, \beta(1) = 1000$ and $\alpha_i$ defined in \cref{appd:bfn_cir}, this figure plots three colored stochastic parameter trajectories for receiver mean parameter $m$ and accumulated accuracy parameter $c$, superimposed on a log-scale heatmap of the Bayesian flow distribution $p_F(m|x,\senderacc)$ and $p_F(c|x,\senderacc)$. Note the \emph{non-monotonicity} and \emph{non-additive} property of $c$. \jj{Shrink the figures to save space}}
%     \label{fig:vmbf_vis}
%     \vspace{-30pt}
% \end{wrapfigure}


In this section, we explain the detailed design of CrysBFN tackling theoretical and practical challenges. First, we describe how to derive our new formulation of Bayesian Flow Networks over hyper-torus $\mathbb{T}^{D}$ from scratch. Next, we illustrate the two key differences between \modelname and the original form of BFN: $1)$ a meticulously designed novel base distribution with different Bayesian update rules; and $2)$ different properties over the accuracy scheduling resulted from the periodicity and the new Bayesian update rules. Then, we present in detail the overall framework of \modelname over each manifold of the crystal space (\textit{i.e.} fractional coordinates, lattice vectors, atom types) respecting \textit{periodic E(3) invariance}. 

% In this section, we first demonstrate how to build Bayesian flow on hyper-torus $\mathbb{T}^{D}$ by overcoming theoretical and practical problems to provide a low-noise parameter-space approach to fractional atom coordinate generation. Next, we present how \modelname models each manifold of crystal space respecting \textit{periodic E(3) invariance}. 

\subsection{Periodic Bayesian Flow on Hyper-torus \texorpdfstring{$\mathbb{T}^{D}$}{}} 
For generative modeling of fractional coordinates in crystal, we first construct a periodic Bayesian flow on \texorpdfstring{$\mathbb{T}^{D}$}{} by designing every component of the totally new Bayesian update process which we demonstrate to be distinct from the original Bayesian flow (please see \cref{fig:non_add}). 
 %:) 
 
 The fractional atom coordinate system \citep{jiao2023crystal} inherently distributes over a hyper-torus support $\mathbb{T}^{3\times N}$. Hence, the normal distribution support on $\R$ used in the original \citep{bfn} is not suitable for this scenario. 
% The key problem of generative modeling for crystal is the periodicity of Cartesian atom coordinates $\vX$ requiring:
% \begin{equation}\label{eq:periodcity}
% p(\vA,\vL,\vX)=p(\vA,\vL,\vX+\vec{LK}),\text{where}~\vec{K}=\vec{k}\vec{1}_{1\times N},\forall\vec{k}\in\mathbb{Z}^{3\times1}
% \end{equation}
% However, there does not exist such a distribution supporting on $\R$ to model such property because the integration of such distribution over $\R$ will not be finite and equal to 1. Therefore, the normal distribution used in \citet{bfn} can not meet this condition.

To tackle this problem, the circular distribution~\citep{mardia2009directional} over the finite interval $[-\pi,\pi)$ is a natural choice as the base distribution for deriving the BFN on $\mathbb{T}^D$. 
% one natural choice is to 
% we would like to consider the circular distribution over the finite interval as the base 
% we find that circular distributions \citep{mardia2009directional} defined on a finite interval with lengths of $2\pi$ can be used as the instantiation of input distribution for the BFN on $\mathbb{T}^D$.
Specifically, circular distributions enjoy desirable periodic properties: $1)$ the integration over any interval length of $2\pi$ equals 1; $2)$ the probability distribution function is periodic with period $2\pi$.  Sharing the same intrinsic with fractional coordinates, such periodic property of circular distribution makes it suitable for the instantiation of BFN's input distribution, in parameterizing the belief towards ground truth $\x$ on $\mathbb{T}^D$. 
% \yuxuan{this is very complicated from my perspective.} \hanlin{But this property is exactly beautiful and perfectly fit into the BFN.}

\textbf{von Mises Distribution and its Bayesian Update} We choose von Mises distribution \citep{mardia2009directional} from various circular distributions as the form of input distribution, based on the appealing conjugacy property required in the derivation of the BFN framework.
% to leverage the Bayesian conjugacy property of von Mises distribution which is required by the BFN framework. 
That is, the posterior of a von Mises distribution parameterized likelihood is still in the family of von Mises distributions. The probability density function of von Mises distribution with mean direction parameter $m$ and concentration parameter $c$ (describing the entropy/uncertainty of $m$) is defined as: 
\begin{equation}
f(x|m,c)=vM(x|m,c)=\frac{\exp(c\cos(x-m))}{2\pi I_0(c)}
\end{equation}
where $I_0(c)$ is zeroth order modified Bessel function of the first kind as the normalizing constant. Given the last univariate belief parameterized by von Mises distribution with parameter $\theta_{i-1}=\{m_{i-1},\ c_{i-1}\}$ and the sample $y$ from sender distribution with unknown data sample $x$ and known accuracy $\alpha$ describing the entropy/uncertainty of $y$,  Bayesian update for the receiver is deducted as:
\begin{equation}
 h(\{m_{i-1},c_{i-1}\},y,\alpha)=\{m_i,c_i \}, \text{where}
\end{equation}
\begin{equation}\label{eq:h_m}
m_i=\text{atan2}(\alpha\sin y+c_{i-1}\sin m_{i-1}, {\alpha\cos y+c_{i-1}\cos m_{i-1}})
\end{equation}
\begin{equation}\label{eq:h_c}
c_i =\sqrt{\alpha^2+c_{i-1}^2+2\alpha c_{i-1}\cos(y-m_{i-1})}
\end{equation}
The proof of the above equations can be found in \cref{apdx:bayesian_update_function}. The atan2 function refers to  2-argument arctangent. Independently conducting  Bayesian update for each dimension, we can obtain the Bayesian update distribution by marginalizing $\y$:
\begin{equation}
p_U(\vtheta'|\vtheta,\bold{x};\alpha)=\mathbb{E}_{p_S(\bold{y}|\bold{x};\alpha)}\delta(\vtheta'-h(\vtheta,\bold{y},\alpha))=\mathbb{E}_{vM(\bold{y}|\bold{x},\alpha)}\delta(\vtheta'-h(\vtheta,\bold{y},\alpha))
\end{equation} 
\begin{figure}
    \centering
    \vskip -0.15in
    \includegraphics[width=0.95\linewidth]{imgs/non_add.pdf}
    \caption{An intuitive illustration of non-additive accuracy Bayesian update on the torus. The lengths of arrows represent the uncertainty/entropy of the belief (\emph{e.g.}~$1/\sigma^2$ for Gaussian and $c$ for von Mises). The directions of the arrows represent the believed location (\emph{e.g.}~ $\mu$ for Gaussian and $m$ for von Mises).}
    \label{fig:non_add}
    \vskip -0.15in
\end{figure}
\textbf{Non-additive Accuracy} 
The additive accuracy is a nice property held with the Gaussian-formed sender distribution of the original BFN expressed as:
\begin{align}
\label{eq:standard_id}
    \update(\parsn{}'' \mid \parsn{}, \x; \alpha_a+\alpha_b) = \E_{\update(\parsn{}' \mid \parsn{}, \x; \alpha_a)} \update(\parsn{}'' \mid \parsn{}', \x; \alpha_b)
\end{align}
Such property is mainly derived based on the standard identity of Gaussian variable:
\begin{equation}
X \sim \mathcal{N}\left(\mu_X, \sigma_X^2\right), Y \sim \mathcal{N}\left(\mu_Y, \sigma_Y^2\right) \Longrightarrow X+Y \sim \mathcal{N}\left(\mu_X+\mu_Y, \sigma_X^2+\sigma_Y^2\right)
\end{equation}
The additive accuracy property makes it feasible to derive the Bayesian flow distribution $
p_F(\boldsymbol{\theta} \mid \mathbf{x} ; i)=p_U\left(\boldsymbol{\theta} \mid \boldsymbol{\theta}_0, \mathbf{x}, \sum_{k=1}^{i} \alpha_i \right)
$ for the simulation-free training of \cref{eq:loss_n}.
It should be noted that the standard identity in \cref{eq:standard_id} does not hold in the von Mises distribution. Hence there exists an important difference between the original Bayesian flow defined on Euclidean space and the Bayesian flow of circular data on $\mathbb{T}^D$ based on von Mises distribution. With prior $\btheta = \{\bold{0},\bold{0}\}$, we could formally represent the non-additive accuracy issue as:
% The additive accuracy property implies the fact that the "confidence" for the data sample after observing a series of the noisy samples with accuracy ${\alpha_1, \cdots, \alpha_i}$ could be  as the accuracy sum  which could be  
% Here we 
% Here we emphasize the specific property of BFN based on von Mises distribution.
% Note that 
% \begin{equation}
% \update(\parsn'' \mid \parsn, \x; \alpha_a+\alpha_b) \ne \E_{\update(\parsn' \mid \parsn, \x; \alpha_a)} \update(\parsn'' \mid \parsn', \x; \alpha_b)
% \end{equation}
% \oyyw{please check whether the below equation is better}
% \yuxuan{I fill somehow confusing on what is the update distribution with $\alpha$. }
% \begin{equation}
% \update(\parsn{}'' \mid \parsn{}, \x; \alpha_a+\alpha_b) \ne \E_{\update(\parsn{}' \mid \parsn{}, \x; \alpha_a)} \update(\parsn{}'' \mid \parsn{}', \x; \alpha_b)
% \end{equation}
% We give an intuitive visualization of such difference in \cref{fig:non_add}. The untenability of this property can materialize by considering the following case: with prior $\btheta = \{\bold{0},\bold{0}\}$, check the two-step Bayesian update distribution with $\alpha_a,\alpha_b$ and one-step Bayesian update with $\alpha=\alpha_a+\alpha_b$:
\begin{align}
\label{eq:nonadd}
     &\update(c'' \mid \parsn, \x; \alpha_a+\alpha_b)  = \delta(c-\alpha_a-\alpha_b)
     \ne  \mathbb{E}_{p_U(\parsn' \mid \parsn, \x; \alpha_a)}\update(c'' \mid \parsn', \x; \alpha_b) \nonumber \\&= \mathbb{E}_{vM(\bold{y}_b|\bold{x},\alpha_a)}\mathbb{E}_{vM(\bold{y}_a|\bold{x},\alpha_b)}\delta(c-||[\alpha_a \cos\y_a+\alpha_b\cos \y_b,\alpha_a \sin\y_a+\alpha_b\sin \y_b]^T||_2)
\end{align}
A more intuitive visualization could be found in \cref{fig:non_add}. This fundamental difference between periodic Bayesian flow and that of \citet{bfn} presents both theoretical and practical challenges, which we will explain and address in the following contents.

% This makes constructing Bayesian flow based on von Mises distribution intrinsically different from previous Bayesian flows (\citet{bfn}).

% Thus, we must reformulate the framework of Bayesian flow networks  accordingly. % and do necessary reformulations of BFN. 

% \yuxuan{overall I feel this part is complicated by using the language of update distribution. I would like to suggest simply use bayesian update, to provide intuitive explantion.}\hanlin{See the illustration in \cref{fig:non_add}}

% That introduces a cascade of problems, and we investigate the following issues: $(1)$ Accuracies between sender and receiver are not synchronized and need to be differentiated. $(2)$ There is no tractable Bayesian flow distribution for a one-step sample conditioned on a given time step $i$, and naively simulating the Bayesian flow results in computational overhead. $(3)$ It is difficult to control the entropy of the Bayesian flow. $(4)$ Accuracy is no longer a function of $t$ and becomes a distribution conditioned on $t$, which can be different across dimensions.
%\jj{Edited till here}

\textbf{Entropy Conditioning} As a common practice in generative models~\citep{ddpm,flowmatching,bfn}, timestep $t$ is widely used to distinguish among generation states by feeding the timestep information into the networks. However, this paper shows that for periodic Bayesian flow, the accumulated accuracy $\vc_i$ is more effective than time-based conditioning by informing the network about the entropy and certainty of the states $\parsnt{i}$. This stems from the intrinsic non-additive accuracy which makes the receiver's accumulated accuracy $c$ not bijective function of $t$, but a distribution conditioned on accumulated accuracies $\vc_i$ instead. Therefore, the entropy parameter $\vc$ is taken logarithm and fed into the network to describe the entropy of the input corrupted structure. We verify this consideration in \cref{sec:exp_ablation}. 
% \yuxuan{implement variant. traditionally, the timestep is widely used to distinguish the different states by putting the timestep embedding into the networks. citation of FM, diffusion, BFN. However, we find that conditioned on time in periodic flow could not provide extra benefits. To further boost the performance, we introduce a simple yet effective modification term entropy conditional. This is based on that the accumulated accuracy which represents the current uncertainty or entropy could be a better indicator to distinguish different states. + Describe how you do this. }



\textbf{Reformulations of BFN}. Recall the original update function with Gaussian sender distribution, after receiving noisy samples $\y_1,\y_2,\dots,\y_i$ with accuracies $\senderacc$, the accumulated accuracies of the receiver side could be analytically obtained by the additive property and it is consistent with the sender side.
% Since observing sample $\y$ with $\alpha_i$ can not result in exact accuracy increment $\alpha_i$ for receiver, the accuracies between sender and receiver are not synchronized which need to be differentiated. 
However, as previously mentioned, this does not apply to periodic Bayesian flow, and some of the notations in original BFN~\citep{bfn} need to be adjusted accordingly. We maintain the notations of sender side's one-step accuracy $\alpha$ and added accuracy $\beta$, and alter the notation of receiver's accuracy parameter as $c$, which is needed to be simulated by cascade of Bayesian updates. We emphasize that the receiver's accumulated accuracy $c$ is no longer a function of $t$ (differently from the Gaussian case), and it becomes a distribution conditioned on received accuracies $\senderacc$ from the sender. Therefore, we represent the Bayesian flow distribution of von Mises distribution as $p_F(\btheta|\x;\alpha_1,\alpha_2,\dots,\alpha_i)$. And the original simulation-free training with Bayesian flow distribution is no longer applicable in this scenario.
% Different from previous BFNs where the accumulated accuracy $\rho$ is not explicitly modeled, the accumulated accuracy parameter $c$ (visualized in \cref{fig:vmbf_vis}) needs to be explicitly modeled by feeding it to the network to avoid information loss.
% the randomaccuracy parameter $c$ (visualized in \cref{fig:vmbf_vis}) implies that there exists information in $c$ from the sender just like $m$, meaning that $c$ also should be fed into the network to avoid information loss. 
% We ablate this consideration in  \cref{sec:exp_ablation}. 

\textbf{Fast Sampling from Equivalent Bayesian Flow Distribution} Based on the above reformulations, the Bayesian flow distribution of von Mises distribution is reframed as: 
\begin{equation}\label{eq:flow_frac}
p_F(\btheta_i|\x;\alpha_1,\alpha_2,\dots,\alpha_i)=\E_{\update(\parsnt{1} \mid \parsnt{0}, \x ; \alphat{1})}\dots\E_{\update(\parsn_{i-1} \mid \parsnt{i-2}, \x; \alphat{i-1})} \update(\parsnt{i} | \parsnt{i-1},\x;\alphat{i} )
\end{equation}
Naively sampling from \cref{eq:flow_frac} requires slow auto-regressive iterated simulation, making training unaffordable. Noticing the mathematical properties of \cref{eq:h_m,eq:h_c}, we  transform \cref{eq:flow_frac} to the equivalent form:
\begin{equation}\label{eq:cirflow_equiv}
p_F(\vec{m}_i|\x;\alpha_1,\alpha_2,\dots,\alpha_i)=\E_{vM(\y_1|\x,\alpha_1)\dots vM(\y_i|\x,\alpha_i)} \delta(\vec{m}_i-\text{atan2}(\sum_{j=1}^i \alpha_j \cos \y_j,\sum_{j=1}^i \alpha_j \sin \y_j))
\end{equation}
\begin{equation}\label{eq:cirflow_equiv2}
p_F(\vec{c}_i|\x;\alpha_1,\alpha_2,\dots,\alpha_i)=\E_{vM(\y_1|\x,\alpha_1)\dots vM(\y_i|\x,\alpha_i)}  \delta(\vec{c}_i-||[\sum_{j=1}^i \alpha_j \cos \y_j,\sum_{j=1}^i \alpha_j \sin \y_j]^T||_2)
\end{equation}
which bypasses the computation of intermediate variables and allows pure tensor operations, with negligible computational overhead.
\begin{restatable}{proposition}{cirflowequiv}
The probability density function of Bayesian flow distribution defined by \cref{eq:cirflow_equiv,eq:cirflow_equiv2} is equivalent to the original definition in \cref{eq:flow_frac}. 
\end{restatable}
\textbf{Numerical Determination of Linear Entropy Sender Accuracy Schedule} ~Original BFN designs the accuracy schedule $\beta(t)$ to make the entropy of input distribution linearly decrease. As for crystal generation task, to ensure information coherence between modalities, we choose a sender accuracy schedule $\senderacc$ that makes the receiver's belief entropy $H(t_i)=H(p_I(\cdot|\vtheta_i))=H(p_I(\cdot|\vc_i))$ linearly decrease \emph{w.r.t.} time $t_i$, given the initial and final accuracy parameter $c(0)$ and $c(1)$. Due to the intractability of \cref{eq:vm_entropy}, we first use numerical binary search in $[0,c(1)]$ to determine the receiver's $c(t_i)$ for $i=1,\dots, n$ by solving the equation $H(c(t_i))=(1-t_i)H(c(0))+tH(c(1))$. Next, with $c(t_i)$, we conduct numerical binary search for each $\alpha_i$ in $[0,c(1)]$ by solving the equations $\E_{y\sim vM(x,\alpha_i)}[\sqrt{\alpha_i^2+c_{i-1}^2+2\alpha_i c_{i-1}\cos(y-m_{i-1})}]=c(t_i)$ from $i=1$ to $i=n$ for arbitrarily selected $x\in[-\pi,\pi)$.

After tackling all those issues, we have now arrived at a new BFN architecture for effectively modeling crystals. Such BFN can also be adapted to other type of data located in hyper-torus $\mathbb{T}^{D}$.

\subsection{Equivariant Bayesian Flow for Crystal}
With the above Bayesian flow designed for generative modeling of fractional coordinate $\vF$, we are able to build equivariant Bayesian flow for each modality of crystal. In this section, we first give an overview of the general training and sampling algorithm of \modelname (visualized in \cref{fig:framework}). Then, we describe the details of the Bayesian flow of every modality. The training and sampling algorithm can be found in \cref{alg:train} and \cref{alg:sampling}.

\textbf{Overview} Operating in the parameter space $\bthetaM=\{\bthetaA,\bthetaL,\bthetaF\}$, \modelname generates high-fidelity crystals through a joint BFN sampling process on the parameter of  atom type $\bthetaA$, lattice parameter $\vec{\theta}^L=\{\bmuL,\brhoL\}$, and the parameter of fractional coordinate matrix $\bthetaF=\{\bmF,\bcF\}$. We index the $n$-steps of the generation process in a discrete manner $i$, and denote the corresponding continuous notation $t_i=i/n$ from prior parameter $\thetaM_0$ to a considerably low variance parameter $\thetaM_n$ (\emph{i.e.} large $\vrho^L,\bmF$, and centered $\bthetaA$).

At training time, \modelname samples time $i\sim U\{1,n\}$ and $\bthetaM_{i-1}$ from the Bayesian flow distribution of each modality, serving as the input to the network. The network $\net$ outputs $\net(\parsnt{i-1}^\mathcal{M},t_{i-1})=\net(\parsnt{i-1}^A,\parsnt{i-1}^F,\parsnt{i-1}^L,t_{i-1})$ and conducts gradient descents on loss function \cref{eq:loss_n} for each modality. After proper training, the sender distribution $p_S$ can be approximated by the receiver distribution $p_R$. 

At inference time, from predefined $\thetaM_0$, we conduct transitions from $\thetaM_{i-1}$ to $\thetaM_{i}$ by: $(1)$ sampling $\y_i\sim p_R(\bold{y}|\thetaM_{i-1};t_i,\alpha_i)$ according to network prediction $\predM{i-1}$; and $(2)$ performing Bayesian update $h(\thetaM_{i-1},\y^\calM_{i-1},\alpha_i)$ for each dimension. 

% Alternatively, we complete this transition using the flow-back technique by sampling 
% $\thetaM_{i}$ from Bayesian flow distribution $\flow(\btheta^M_{i}|\predM{i-1};t_{i-1})$. 

% The training objective of $\net$ is to minimize the KL divergence between sender distribution and receiver distribution for every modality as defined in \cref{eq:loss_n} which is equivalent to optimizing the negative variational lower bound $\calL^{VLB}$ as discussed in \cref{sec:preliminaries}. 

%In the following part, we will present the Bayesian flow of each modality in detail.

\textbf{Bayesian Flow of Fractional Coordinate $\vF$}~The distribution of the prior parameter $\bthetaF_0$ is defined as:
\begin{equation}\label{eq:prior_frac}
    p(\bthetaF_0) \defeq \{vM(\vm_0^F|\vec{0}_{3\times N},\vec{0}_{3\times N}),\delta(\vc_0^F-\vec{0}_{3\times N})\} = \{U(\vec{0},\vec{1}),\delta(\vc_0^F-\vec{0}_{3\times N})\}
\end{equation}
Note that this prior distribution of $\vm_0^F$ is uniform over $[\vec{0},\vec{1})$, ensuring the periodic translation invariance property in \cref{De:pi}. The training objective is minimizing the KL divergence between sender and receiver distribution (deduction can be found in \cref{appd:cir_loss}): 
%\oyyw{replace $\vF$ with $\x$?} \hanlin{notations follow Preliminary?}
\begin{align}\label{loss_frac}
\calL_F = n \E_{i \sim \ui{n}, \flow(\parsn{}^F \mid \vF ; \senderacc)} \alpha_i\frac{I_1(\alpha_i)}{I_0(\alpha_i)}(1-\cos(\vF-\predF{i-1}))
\end{align}
where $I_0(x)$ and $I_1(x)$ are the zeroth and the first order of modified Bessel functions. The transition from $\bthetaF_{i-1}$ to $\bthetaF_{i}$ is the Bayesian update distribution based on network prediction:
\begin{equation}\label{eq:transi_frac}
    p(\btheta^F_{i}|\parsnt{i-1}^\calM)=\mathbb{E}_{vM(\bold{y}|\predF{i-1},\alpha_i)}\delta(\btheta^F_{i}-h(\btheta^F_{i-1},\bold{y},\alpha_i))
\end{equation}
\begin{restatable}{proposition}{fracinv}
With $\net_{F}$ as a periodic translation equivariant function namely $\net_F(\parsnt{}^A,w(\parsnt{}^F+\vt),\parsnt{}^L,t)=w(\net_F(\parsnt{}^A,\parsnt{}^F,\parsnt{}^L,t)+\vt), \forall\vt\in\R^3$, the marginal distribution of $p(\vF_n)$ defined by \cref{eq:prior_frac,eq:transi_frac} is periodic translation invariant. 
\end{restatable}
\textbf{Bayesian Flow of Lattice Parameter \texorpdfstring{$\boldsymbol{L}$}{}}   
Noting the lattice parameter $\bm{L}$ located in Euclidean space, we set prior as the parameter of a isotropic multivariate normal distribution $\btheta^L_0\defeq\{\vmu_0^L,\vrho_0^L\}=\{\bm{0}_{3\times3},\bm{1}_{3\times3}\}$
% \begin{equation}\label{eq:lattice_prior}
% \btheta^L_0\defeq\{\vmu_0^L,\vrho_0^L\}=\{\bm{0}_{3\times3},\bm{1}_{3\times3}\}
% \end{equation}
such that the prior distribution of the Markov process on $\vmu^L$ is the Dirac distribution $\delta(\vec{\mu_0}-\vec{0})$ and $\delta(\vec{\rho_0}-\vec{1})$, 
% \begin{equation}
%     p_I^L(\boldsymbol{L}|\btheta_0^L)=\mathcal{N}(\bm{L}|\bm{0},\bm{I})
% \end{equation}
which ensures O(3)-invariance of prior distribution of $\vL$. By Eq. 77 from \citet{bfn}, the Bayesian flow distribution of the lattice parameter $\bm{L}$ is: 
\begin{align}% =p_U(\bmuL|\btheta_0^L,\bm{L},\beta(t))
p_F^L(\bmuL|\bm{L};t) &=\mathcal{N}(\bmuL|\gamma(t)\bm{L},\gamma(t)(1-\gamma(t))\bm{I}) 
\end{align}
where $\gamma(t) = 1 - \sigma_1^{2t}$ and $\sigma_1$ is the predefined hyper-parameter controlling the variance of input distribution at $t=1$ under linear entropy accuracy schedule. The variance parameter $\vrho$ does not need to be modeled and fed to the network, since it is deterministic given the accuracy schedule. After sampling $\bmuL_i$ from $p_F^L$, the training objective is defined as minimizing KL divergence between sender and receiver distribution (based on Eq. 96 in \citet{bfn}):
\begin{align}
\mathcal{L}_{L} = \frac{n}{2}\left(1-\sigma_1^{2/n}\right)\E_{i \sim \ui{n}}\E_{\flow(\bmuL_{i-1} |\vL ; t_{i-1})}  \frac{\left\|\vL -\predL{i-1}\right\|^2}{\sigma_1^{2i/n}},\label{eq:lattice_loss}
\end{align}
where the prediction term $\predL{i-1}$ is the lattice parameter part of network output. After training, the generation process is defined as the Bayesian update distribution given network prediction:
\begin{equation}\label{eq:lattice_sampling}
    p(\bmuL_{i}|\parsnt{i-1}^\calM)=\update^L(\bmuL_{i}|\predL{i-1},\bmuL_{i-1};t_{i-1})
\end{equation}
    

% The final prediction of the lattice parameter is given by $\bmuL_n = \predL{n-1}$.
% \begin{equation}\label{eq:final_lattice}
%     \bmuL_n = \predL{n-1}
% \end{equation}

\begin{restatable}{proposition}{latticeinv}\label{prop:latticeinv}
With $\net_{L}$ as  O(3)-equivariant function namely $\net_L(\parsnt{}^A,\parsnt{}^F,\vQ\parsnt{}^L,t)=\vQ\net_L(\parsnt{}^A,\parsnt{}^F,\parsnt{}^L,t),\forall\vQ^T\vQ=\vI$, the marginal distribution of $p(\bmuL_n)$ defined by \cref{eq:lattice_sampling} is O(3)-invariant. 
\end{restatable}


\textbf{Bayesian Flow of Atom Types \texorpdfstring{$\boldsymbol{A}$}{}} 
Given that atom types are discrete random variables located in a simplex $\calS^K$, the prior parameter of $\boldsymbol{A}$ is the discrete uniform distribution over the vocabulary $\parsnt{0}^A \defeq \frac{1}{K}\vec{1}_{1\times N}$. 
% \begin{align}\label{eq:disc_input_prior}
% \parsnt{0}^A \defeq \frac{1}{K}\vec{1}_{1\times N}
% \end{align}
% \begin{align}
%     (\oh{j}{K})_k \defeq \delta_{j k}, \text{where }\oh{j}{K}\in \R^{K},\oh{\vA}{KD} \defeq \left(\oh{a_1}{K},\dots,\oh{a_N}{K}\right) \in \R^{K\times N}
% \end{align}
With the notation of the projection from the class index $j$ to the length $K$ one-hot vector $ (\oh{j}{K})_k \defeq \delta_{j k}, \text{where }\oh{j}{K}\in \R^{K},\oh{\vA}{KD} \defeq \left(\oh{a_1}{K},\dots,\oh{a_N}{K}\right) \in \R^{K\times N}$, the Bayesian flow distribution of atom types $\vA$ is derived in \citet{bfn}:
\begin{align}
\flow^{A}(\parsn^A \mid \vA; t) &= \E_{\N{\y \mid \beta^A(t)\left(K \oh{\vA}{K\times N} - \vec{1}_{K\times N}\right)}{\beta^A(t) K \vec{I}_{K\times N \times N}}} \delta\left(\parsn^A - \frac{e^{\y}\parsnt{0}^A}{\sum_{k=1}^K e^{\y_k}(\parsnt{0})_{k}^A}\right).
\end{align}
where $\beta^A(t)$ is the predefined accuracy schedule for atom types. Sampling $\btheta_i^A$ from $p_F^A$ as the training signal, the training objective is the $n$-step discrete-time loss for discrete variable \citep{bfn}: 
% \oyyw{can we simplify the next equation? Such as remove $K \times N, K \times N \times N$}
% \begin{align}
% &\calL_A = n\E_{i \sim U\{1,n\},\flow^A(\parsn^A \mid \vA ; t_{i-1}),\N{\y \mid \alphat{i}\left(K \oh{\vA}{KD} - \vec{1}_{K\times N}\right)}{\alphat{i} K \vec{I}_{K\times N \times N}}} \ln \N{\y \mid \alphat{i}\left(K \oh{\vA}{K\times N} - \vec{1}_{K\times N}\right)}{\alphat{i} K \vec{I}_{K\times N \times N}}\nonumber\\
% &\qquad\qquad\qquad-\sum_{d=1}^N \ln \left(\sum_{k=1}^K \out^{(d)}(k \mid \parsn^A; t_{i-1}) \N{\ydd{d} \mid \alphat{i}\left(K\oh{k}{K}- \vec{1}_{K\times N}\right)}{\alphat{i} K \vec{I}_{K\times N \times N}}\right)\label{discdisc_t_loss_exp}
% \end{align}
\begin{align}
&\calL_A = n\E_{i \sim U\{1,n\},\flow^A(\parsn^A \mid \vA ; t_{i-1}),\N{\y \mid \alphat{i}\left(K \oh{\vA}{KD} - \vec{1}\right)}{\alphat{i} K \vec{I}}} \ln \N{\y \mid \alphat{i}\left(K \oh{\vA}{K\times N} - \vec{1}\right)}{\alphat{i} K \vec{I}}\nonumber\\
&\qquad\qquad\qquad-\sum_{d=1}^N \ln \left(\sum_{k=1}^K \out^{(d)}(k \mid \parsn^A; t_{i-1}) \N{\ydd{d} \mid \alphat{i}\left(K\oh{k}{K}- \vec{1}\right)}{\alphat{i} K \vec{I}}\right)\label{discdisc_t_loss_exp}
\end{align}
where $\vec{I}\in \R^{K\times N \times N}$ and $\vec{1}\in\R^{K\times D}$. When sampling, the transition from $\bthetaA_{i-1}$ to $\bthetaA_{i}$ is derived as:
\begin{equation}
    p(\btheta^A_{i}|\parsnt{i-1}^\calM)=\update^A(\btheta^A_{i}|\btheta^A_{i-1},\predA{i-1};t_{i-1})
\end{equation}

The detailed training and sampling algorithm could be found in \cref{alg:train} and \cref{alg:sampling}.





\clearpage

\begin{table*}[t]
\centering
\fontsize{11pt}{11pt}\selectfont
\begin{tabular}{lllllllllllll}
\toprule
\multicolumn{1}{c}{\textbf{task}} & \multicolumn{2}{c}{\textbf{Mir}} & \multicolumn{2}{c}{\textbf{Lai}} & \multicolumn{2}{c}{\textbf{Ziegen.}} & \multicolumn{2}{c}{\textbf{Cao}} & \multicolumn{2}{c}{\textbf{Alva-Man.}} & \multicolumn{1}{c}{\textbf{avg.}} & \textbf{\begin{tabular}[c]{@{}l@{}}avg.\\ rank\end{tabular}} \\
\multicolumn{1}{c}{\textbf{metrics}} & \multicolumn{1}{c}{\textbf{cor.}} & \multicolumn{1}{c}{\textbf{p-v.}} & \multicolumn{1}{c}{\textbf{cor.}} & \multicolumn{1}{c}{\textbf{p-v.}} & \multicolumn{1}{c}{\textbf{cor.}} & \multicolumn{1}{c}{\textbf{p-v.}} & \multicolumn{1}{c}{\textbf{cor.}} & \multicolumn{1}{c}{\textbf{p-v.}} & \multicolumn{1}{c}{\textbf{cor.}} & \multicolumn{1}{c}{\textbf{p-v.}} &  &  \\ \midrule
\textbf{S-Bleu} & 0.50 & 0.0 & 0.47 & 0.0 & 0.59 & 0.0 & 0.58 & 0.0 & 0.68 & 0.0 & 0.57 & 5.8 \\
\textbf{R-Bleu} & -- & -- & 0.27 & 0.0 & 0.30 & 0.0 & -- & -- & -- & -- & - &  \\
\textbf{S-Meteor} & 0.49 & 0.0 & 0.48 & 0.0 & 0.61 & 0.0 & 0.57 & 0.0 & 0.64 & 0.0 & 0.56 & 6.1 \\
\textbf{R-Meteor} & -- & -- & 0.34 & 0.0 & 0.26 & 0.0 & -- & -- & -- & -- & - &  \\
\textbf{S-Bertscore} & \textbf{0.53} & 0.0 & {\ul 0.80} & 0.0 & \textbf{0.70} & 0.0 & {\ul 0.66} & 0.0 & {\ul0.78} & 0.0 & \textbf{0.69} & \textbf{1.7} \\
\textbf{R-Bertscore} & -- & -- & 0.51 & 0.0 & 0.38 & 0.0 & -- & -- & -- & -- & - &  \\
\textbf{S-Bleurt} & {\ul 0.52} & 0.0 & {\ul 0.80} & 0.0 & 0.60 & 0.0 & \textbf{0.70} & 0.0 & \textbf{0.80} & 0.0 & {\ul 0.68} & {\ul 2.3} \\
\textbf{R-Bleurt} & -- & -- & 0.59 & 0.0 & -0.05 & 0.13 & -- & -- & -- & -- & - &  \\
\textbf{S-Cosine} & 0.51 & 0.0 & 0.69 & 0.0 & {\ul 0.62} & 0.0 & 0.61 & 0.0 & 0.65 & 0.0 & 0.62 & 4.4 \\
\textbf{R-Cosine} & -- & -- & 0.40 & 0.0 & 0.29 & 0.0 & -- & -- & -- & -- & - & \\ \midrule
\textbf{QuestEval} & 0.23 & 0.0 & 0.25 & 0.0 & 0.49 & 0.0 & 0.47 & 0.0 & 0.62 & 0.0 & 0.41 & 9.0 \\
\textbf{LLaMa3} & 0.36 & 0.0 & \textbf{0.84} & 0.0 & {\ul{0.62}} & 0.0 & 0.61 & 0.0 &  0.76 & 0.0 & 0.64 & 3.6 \\
\textbf{our (3b)} & 0.49 & 0.0 & 0.73 & 0.0 & 0.54 & 0.0 & 0.53 & 0.0 & 0.7 & 0.0 & 0.60 & 5.8 \\
\textbf{our (8b)} & 0.48 & 0.0 & 0.73 & 0.0 & 0.52 & 0.0 & 0.53 & 0.0 & 0.7 & 0.0 & 0.59 & 6.3 \\  \bottomrule
\end{tabular}
\caption{Pearson correlation on human evaluation on system output. `R-': reference-based. `S-': source-based.}
\label{tab:sys}
\end{table*}



\begin{table}%[]
\centering
\fontsize{11pt}{11pt}\selectfont
\begin{tabular}{llllll}
\toprule
\multicolumn{1}{c}{\textbf{task}} & \multicolumn{1}{c}{\textbf{Lai}} & \multicolumn{1}{c}{\textbf{Zei.}} & \multicolumn{1}{c}{\textbf{Scia.}} & \textbf{} & \textbf{} \\ 
\multicolumn{1}{c}{\textbf{metrics}} & \multicolumn{1}{c}{\textbf{cor.}} & \multicolumn{1}{c}{\textbf{cor.}} & \multicolumn{1}{c}{\textbf{cor.}} & \textbf{avg.} & \textbf{\begin{tabular}[c]{@{}l@{}}avg.\\ rank\end{tabular}} \\ \midrule
\textbf{S-Bleu} & 0.40 & 0.40 & 0.19* & 0.33 & 7.67 \\
\textbf{S-Meteor} & 0.41 & 0.42 & 0.16* & 0.33 & 7.33 \\
\textbf{S-BertS.} & {\ul0.58} & 0.47 & 0.31 & 0.45 & 3.67 \\
\textbf{S-Bleurt} & 0.45 & {\ul 0.54} & {\ul 0.37} & 0.45 & {\ul 3.33} \\
\textbf{S-Cosine} & 0.56 & 0.52 & 0.3 & {\ul 0.46} & {\ul 3.33} \\ \midrule
\textbf{QuestE.} & 0.27 & 0.35 & 0.06* & 0.23 & 9.00 \\
\textbf{LlaMA3} & \textbf{0.6} & \textbf{0.67} & \textbf{0.51} & \textbf{0.59} & \textbf{1.0} \\
\textbf{Our (3b)} & 0.51 & 0.49 & 0.23* & 0.39 & 4.83 \\
\textbf{Our (8b)} & 0.52 & 0.49 & 0.22* & 0.43 & 4.83 \\ \bottomrule
\end{tabular}
\caption{Pearson correlation on human ratings on reference output. *not significant; we cannot reject the null hypothesis of zero correlation}
\label{tab:ref}
\end{table}


\begin{table*}%[]
\centering
\fontsize{11pt}{11pt}\selectfont
\begin{tabular}{lllllllll}
\toprule
\textbf{task} & \multicolumn{1}{c}{\textbf{ALL}} & \multicolumn{1}{c}{\textbf{sentiment}} & \multicolumn{1}{c}{\textbf{detoxify}} & \multicolumn{1}{c}{\textbf{catchy}} & \multicolumn{1}{c}{\textbf{polite}} & \multicolumn{1}{c}{\textbf{persuasive}} & \multicolumn{1}{c}{\textbf{formal}} & \textbf{\begin{tabular}[c]{@{}l@{}}avg. \\ rank\end{tabular}} \\
\textbf{metrics} & \multicolumn{1}{c}{\textbf{cor.}} & \multicolumn{1}{c}{\textbf{cor.}} & \multicolumn{1}{c}{\textbf{cor.}} & \multicolumn{1}{c}{\textbf{cor.}} & \multicolumn{1}{c}{\textbf{cor.}} & \multicolumn{1}{c}{\textbf{cor.}} & \multicolumn{1}{c}{\textbf{cor.}} &  \\ \midrule
\textbf{S-Bleu} & -0.17 & -0.82 & -0.45 & -0.12* & -0.1* & -0.05 & -0.21 & 8.42 \\
\textbf{R-Bleu} & - & -0.5 & -0.45 &  &  &  &  &  \\
\textbf{S-Meteor} & -0.07* & -0.55 & -0.4 & -0.01* & 0.1* & -0.16 & -0.04* & 7.67 \\
\textbf{R-Meteor} & - & -0.17* & -0.39 & - & - & - & - & - \\
\textbf{S-BertScore} & 0.11 & -0.38 & -0.07* & -0.17* & 0.28 & 0.12 & 0.25 & 6.0 \\
\textbf{R-BertScore} & - & -0.02* & -0.21* & - & - & - & - & - \\
\textbf{S-Bleurt} & 0.29 & 0.05* & 0.45 & 0.06* & 0.29 & 0.23 & 0.46 & 4.2 \\
\textbf{R-Bleurt} & - &  0.21 & 0.38 & - & - & - & - & - \\
\textbf{S-Cosine} & 0.01* & -0.5 & -0.13* & -0.19* & 0.05* & -0.05* & 0.15* & 7.42 \\
\textbf{R-Cosine} & - & -0.11* & -0.16* & - & - & - & - & - \\ \midrule
\textbf{QuestEval} & 0.21 & {\ul{0.29}} & 0.23 & 0.37 & 0.19* & 0.35 & 0.14* & 4.67 \\
\textbf{LlaMA3} & \textbf{0.82} & \textbf{0.80} & \textbf{0.72} & \textbf{0.84} & \textbf{0.84} & \textbf{0.90} & \textbf{0.88} & \textbf{1.00} \\
\textbf{Our (3b)} & 0.47 & -0.11* & 0.37 & 0.61 & 0.53 & 0.54 & 0.66 & 3.5 \\
\textbf{Our (8b)} & {\ul{0.57}} & 0.09* & {\ul 0.49} & {\ul 0.72} & {\ul 0.64} & {\ul 0.62} & {\ul 0.67} & {\ul 2.17} \\ \bottomrule
\end{tabular}
\caption{Pearson correlation on human ratings on our constructed test set. 'R-': reference-based. 'S-': source-based. *not significant; we cannot reject the null hypothesis of zero correlation}
\label{tab:con}
\end{table*}

\section{Results}
We benchmark the different metrics on the different datasets using correlation to human judgement. For content preservation, we show results split on data with system output, reference output and our constructed test set: we show that the data source for evaluation leads to different conclusions on the metrics. In addition, we examine whether the metrics can rank style transfer systems similar to humans. On style strength, we likewise show correlations between human judgment and zero-shot evaluation approaches. When applicable, we summarize results by reporting the average correlation. And the average ranking of the metric per dataset (by ranking which metric obtains the highest correlation to human judgement per dataset). 

\subsection{Content preservation}
\paragraph{How do data sources affect the conclusion on best metric?}
The conclusions about the metrics' performance change radically depending on whether we use system output data, reference output, or our constructed test set. Ideally, a good metric correlates highly with humans on any data source. Ideally, for meta-evaluation, a metric should correlate consistently across all data sources, but the following shows that the correlations indicate different things, and the conclusion on the best metric should be drawn carefully.

Looking at the metrics correlations with humans on the data source with system output (Table~\ref{tab:sys}), we see a relatively high correlation for many of the metrics on many tasks. The overall best metrics are S-BertScore and S-BLEURT (avg+avg rank). We see no notable difference in our method of using the 3B or 8B model as the backbone.

Examining the average correlations based on data with reference output (Table~\ref{tab:ref}), now the zero-shoot prompting with LlaMA3 70B is the best-performing approach ($0.59$ avg). Tied for second place are source-based cosine embedding ($0.46$ avg), BLEURT ($0.45$ avg) and BertScore ($0.45$ avg). Our method follows on a 5. place: here, the 8b version (($0.43$ avg)) shows a bit stronger results than 3b ($0.39$ avg). The fact that the conclusions change, whether looking at reference or system output, confirms the observations made by \citet{scialom-etal-2021-questeval} on simplicity transfer.   

Now consider the results on our test set (Table~\ref{tab:con}): Several metrics show low or no correlation; we even see a significantly negative correlation for some metrics on ALL (BLEU) and for specific subparts of our test set for BLEU, Meteor, BertScore, Cosine. On the other end, LlaMA3 70B is again performing best, showing strong results ($0.82$ in ALL). The runner-up is now our 8B method, with a gap to the 3B version ($0.57$ vs $0.47$ in ALL). Note our method still shows zero correlation for the sentiment task. After, ranks BLEURT ($0.29$), QuestEval ($0.21$), BertScore ($0.11$), Cosine ($0.01$).  

On our test set, we find that some metrics that correlate relatively well on the other datasets, now exhibit low correlation. Hence, with our test set, we can now support the logical reasoning with data evidence: Evaluation of content preservation for style transfer needs to take the style shift into account. This conclusion could not be drawn using the existing data sources: We hypothesise that for the data with system-based output, successful output happens to be very similar to the source sentence and vice versa, and reference-based output might not contain server mistakes as they are gold references. Thus, none of the existing data sources tests the limits of the metrics.  


\paragraph{How do reference-based metrics compare to source-based ones?} Reference-based metrics show a lower correlation than the source-based counterpart for all metrics on both datasets with ratings on references (Table~\ref{tab:sys}). As discussed previously, reference-based metrics for style transfer have the drawback that many different good solutions on a rewrite might exist and not only one similar to a reference.


\paragraph{How well can the metrics rank the performance of style transfer methods?}
We compare the metrics' ability to judge the best style transfer methods w.r.t. the human annotations: Several of the data sources contain samples from different style transfer systems. In order to use metrics to assess the quality of the style transfer system, metrics should correctly find the best-performing system. Hence, we evaluate whether the metrics for content preservation provide the same system ranking as human evaluators. We take the mean of the score for every output on each system and the mean of the human annotations; we compare the systems using the Kendall's Tau correlation. 

We find only the evaluation using the dataset Mir, Lai, and Ziegen to result in significant correlations, probably because of sparsity in a number of system tests (App.~\ref{app:dataset}). Our method (8b) is the only metric providing a perfect ranking of the style transfer system on the Lai data, and Llama3 70B the only one on the Ziegen data. Results in App.~\ref{app:results}. 


\subsection{Style strength results}
%Evaluating style strengths is a challenging task. 
Llama3 70B shows better overall results than our method. However, our method scores higher than Llama3 70B on 2 out of 6 datasets, but it also exhibits zero correlation on one task (Table~\ref{tab:styleresults}).%More work i s needed on evaluating style strengths. 
 
\begin{table}%[]
\fontsize{11pt}{11pt}\selectfont
\begin{tabular}{lccc}
\toprule
\multicolumn{1}{c}{\textbf{}} & \textbf{LlaMA3} & \textbf{Our (3b)} & \textbf{Our (8b)} \\ \midrule
\textbf{Mir} & 0.46 & 0.54 & \textbf{0.57} \\
\textbf{Lai} & \textbf{0.57} & 0.18 & 0.19 \\
\textbf{Ziegen.} & 0.25 & 0.27 & \textbf{0.32} \\
\textbf{Alva-M.} & \textbf{0.59} & 0.03* & 0.02* \\
\textbf{Scialom} & \textbf{0.62} & 0.45 & 0.44 \\
\textbf{\begin{tabular}[c]{@{}l@{}}Our Test\end{tabular}} & \textbf{0.63} & 0.46 & 0.48 \\ \bottomrule
\end{tabular}
\caption{Style strength: Pearson correlation to human ratings. *not significant; we cannot reject the null hypothesis of zero corelation}
\label{tab:styleresults}
\end{table}

\subsection{Ablation}
We conduct several runs of the methods using LLMs with variations in instructions/prompts (App.~\ref{app:method}). We observe that the lower the correlation on a task, the higher the variation between the different runs. For our method, we only observe low variance between the runs.
None of the variations leads to different conclusions of the meta-evaluation. Results in App.~\ref{app:results}.

\section{Limitations}
\label{sec:limitations}
We are encouraged by the early promise of the AI co-scientist evaluations, which highlight its potential to augment scientific research.
However, the system has several limitations. Responsible innovation necessitates thoughtful consideration of these alongside the potential impacts to researchers and scientific research. 

\paragraph{Limitations with literature search, reviews and reasoning.} The reviews undertaken by the AI co-scientist system may miss critical prior works due to reliance on open-access literature. In the presented work, the AI co-scientist does not access the entire published literature due to compliance with license or access restrictions where applicable. The system may also omit consideration of prior work on occasions where it has incorrectly reasoned that the work is not relevant.

\paragraph{Lack of access to negative results data.} The AI co-scientist system's use of only open published literature means it likely has limited access to negative experimental results or records of failed experiments. It is known that such data may be more rarely published than positive results, yet experienced scientists working in the field may nonetheless possess and utilize this knowledge to prioritize research~\citep{brazil2024illuminating}. Strategies to overcome this phenomenon might further improve the performance of the co-scientist as a tool for scientific discovery.

\paragraph{Improved multimodal reasoning and tool-use capabilities.} Some of the most interesting data in scientific publications is not written in text but may be encoded visually in figures and charts. However, even state-of-the-art frontier models may not comprehensively utilize such data with optimal  reasoning~\citep{roberts2024scifibench} and the AI co-scientist system is unlikely to be an exception. Stronger benchmarks and evaluations are necessary to improve these capabilities. We have also not evaluated the ability of our system to reason over and integrate information from domain-specific biomedical multimodal datasets (such as large multi-omics datasets) and knowledge graphs. More work is needed to integrate the AI co-scientist system with specialized scientific tools, AI models and databases, and evaluate the ability to utilize them effectively.

\paragraph{Inherited limitations of frontier LLMs.} LLM limitations include imperfect factuality and hallucinations, which may be propagated in the co-scientist system. The system's reliance on existing LLMs and web-search, while providing immediate access to broad knowledge, may propagate errors of factuality, biases or limitations present in those resources.

\paragraph{Need for better metrics and broader evaluations.} While the current AI co-scientist evaluation includes AI auto-ratings, expert reviews and targeted \textit{in vitro} validations, the evaluation of system performance remains preliminary. A comprehensive, systematic evaluation across diverse biomedical and scientific disciplines is necessary to determine the generalizability of co-scientist. Furthermore, the system requires continued improvement to produce outputs that meet the rigor and detail of high-quality publications. Furthermore, the Elo rating implemented to help the system self-improve for hypothesis generation is a limited auto-evaluation metric. Continued investigation into alternative, more objective, less intrinsically-favored, evaluation metrics that better represent perspectives and preferences from expert scientists could strengthen future work.

\paragraph{Limitations of existing validations.} At present, the AI co-scientist focuses on identifying potential therapeutic targets and mechanisms, but many not be addressing the complexities of drug delivery systems. Pharmaceutical factors such as tissue-specific targeting, formulation requirements, and delivery efficiency—while critical for clinical translation—remain beyond the scope of the present system.

The AI co-scientist is currently also not designed to generate comprehensive clinical trial designs or to fully account for factors such as drug bioavailability, pharmacokinetics, and any complex drug interactions when applied for drug repurposing or discovery. These aspects require much deeper understanding, extensive expertise, and appropriate data beyond the scope of the current system. A dedicated translational scientific team is needed for onward clinical translation of the predictions. These limitations also highlights the need for continued development and integration of the system with more tools, such as specialized AI models and real-time databases.


\section{Safety and Ethical Implications}
\label{sec:safetyethics}
While AI systems such as the co-scientist offers the potential to accelerate scientific discovery, it also poses significant safety and ethical challenges, distinct from its impact on the scientific method itself. Safety risks center on the dual-use and the possibility that scientific breakthroughs could be exploited for harmful purposes. Ethical risks, conversely, involve research that contradicts established ethical norms and conventions within specific scientific disciplines. We review these distinct risk categories, emphasizing that further research is crucial to fully understand and mitigate them.

\paragraph{Evolving ethics frameworks, policy and regulations for advanced AI use in scientific endeavors.}
Research ethics is a central aspect of scientific endeavor and a prominent research field in its own right~\citep{shrader1994ethics, resnik2005ethics, rollin2006science, fisher2008research, edel2018science, menapace2019scientific}.  A key focus is directing research towards positive societal impact, although questions remain about potentially dual-use knowledge~\citep{miller2007ethical, selgelid2009governance, pustovit2010philosophical, forge2010note, kuhlau2013ethics}.

Core ethics principles are being complemented by emerging regulation, and formal processes involving organizational ethics reviews that are meant to provide an assessment of adherence to the code of conduct, as well as an assessment of present and future risks involved with research proposals~\citep{shaw2006research, rothstein2006risks, ludlow2015regulating, verschraegen2018regulating}.

The acceleration of science through AI, especially with advanced agentic AI systems, requires advances in science and AI ethics policy and regulation~\citep{jobin2019global, wansley2016regulation}. This adaptation is crucial to address the changing research landscape and the unique risks associated with AI agents of varying capabilities and autonomy.

Advancements in AI systems, like the co-scientist, require moving beyond the limited ethical considerations designed for earlier, specialized AI models with restricted application and action spaces~\citep{gabriel2024ethics}. Some preliminary frameworks have developed to understand the impact of LLM agents in science, specifically mapping risks across user intent, domain, and broader impact~\citep{tang2024prioritizingsafeguardingautonomyrisks}.

\paragraph{Dual-use risks and technical safeguards.}
Beyond the scientific domain, broad frameworks are being developed for evaluating the emergence of potentially dangerous capabilities in AI agents~\citep{shevlane2023model, bova2024quantifying, phuong2024evaluating}. These frameworks assess capabilities related to persuasion, deception, cybersecurity, self-proliferation, and self-reasoning. As AI agents advance, safety evaluations in science must integrate these broader assessments. A long-term risk is that agentic systems could develop intrinsic goals influencing research directions. Human susceptibility to AI manipulation, already observed in other contexts~\citep{sabour2025humandecisionmakingsusceptibleaidriven}, underscores the need for robust frameworks ensuring instruction-following and value alignment.

More immediately on a shorter time-scale, technical safeguards are needed to address unethical research queries, malicious user intent, and the potential for extracting dangerous or dual-use knowledge from scientific AI systems. Because verification is computationally `easier' than generation, significant research focuses on using advanced LLMs as `critics' or `judges' to evaluate both user queries and AI outputs acting as a scalable oversight mechanism. These critics operate based on predefined criteria, provided through direct instructions, examples (few-shot or many-shot prompting), or fine-tuning~\citep{ke2023critiquellm, vu2024foundational, wei2024systematic, lan2024criticbench, zheng2024judging, gu2024survey}. They can also leverage external tools for grounding~\citep{gou2023critic} and have shown promise in multimodal scenarios~\citep{chen2024mllm}. However, limitations remain; human expert involvement is crucial, as LLMs may not align with human judgment in specialized domains~\citep{szymanski2024limitationsllmasajudgeapproachevaluating}.

\paragraph{Adversarial robustness of scientific AI systems.}
Recognizing and mitigating adversarial attacks is a crucial, ongoing research area in the development of foundation models and advanced AI assistants~\citep{shayegani2023survey, he2023large, zhu2023promptbench, fu2023misusing, zhang2023defending, chao2024jailbreakbench, zhao2024evaluating, ma2025safetyscalecomprehensivesurvey}. While manual red teaming has identified vulnerabilities, automated approaches now allow for optimizing prompt suffixes to bypass safety measures, using techniques like greedy, gradient-based, or evolutionary methods~\citep{zou2023universal, lapid2023open}. Attacks can also exploit few-shot demonstrations, in-context learning~\citep{wang2023adversarial, qiang2023hijacking}, and multimodal inputs~\citep{qi2023visual}. Furthermore, LLMs can be used to generate and refine attacks against other LLMs~\citep{chao2023jailbreaking}, and attacks can be iterative, spanning multiple steps~\citep{wang2024footdoorunderstandinglarge}. Defenses are being developed to counter both human and automated attacks, which is increasingly important in an agentic AI future~\citep{zhang2024adversarial}.

Advances in post-training of base models will likely improve overall adversarial robustness. However, domain-specific recognition of malicious use may still require dedicated development and integration into scientific AI assistants. In AI systems employing iterative reasoning (e.g., request interpretation, hypothesis generation, internal thoughts, evaluation, user queries), each component must be tested independently.  This comprehensive testing should account for all potential failure modes, including the handling of unsafe queries, the safety of hypotheses (intermediate and final), and the accuracy of internal checks and filters.

\paragraph{Need for a comprehensive safety approach.}
Scientific AI assistants, like the co-scientist, require integrated, configurable guidelines within their safeguards. Developers should anticipate the complexity of this challenge and prioritize flexible safeguarding to rapidly incorporate community feedback. These semantic safeguards may need to be augmented by traditional software safety measures, including trusted testers, gradual feature rollouts, access controls, request logging, and flagging uncertain outputs for manual review.

Ensuring the safety of these systems, in line with existing AI safety guidelines~\citep{shneiderman2020bridging, anthropicscalingpolicy}, necessitates a multi-pronged approach. This includes:
\begin{itemize}
    \item Comprehensive threat modeling to identify potential vulnerabilities.
    \item Defense mechanisms for each identified threat.
    \item Extensive red-teaming and security testing.
    \item Rapid response procedures for quick resolution of issues including vulnerability patches.
    \item Continuous monitoring and performance tracking.
\end{itemize}

These considerations highlight the need for responsible development, governance and careful deployment of technologies designed for advancing science, appropriate safeguards and ethical guidelines and close compliance with applicable regulations. They also further underscore the need for broad community engagement and an inclusive development of best practices and recommendations around safe and ethical use for AI in science.

\paragraph{Current safeguards in the AI co-scientist.}
To mitigate these risks, the AI co-scientist currently employs the following safety mechanisms:
\begin{itemize}[leftmargin=1.5em,rightmargin=0em]
    \item\textbf{Reliance on public frontier LLMs.} The system utilizes established public Gemini 2.0 models, which already incorporate extensive safety evaluation and safeguards.
    \item\textbf{Initial research goal safety review.} Upon input, each research goal undergoes automated safety evaluation. Goals deemed potentially unsafe are rejected.
    \item\textbf{Research hypothesis safety review.} Generated hypotheses are reviewed for safety, even when the overarching research goal is deemed safe. Potentially unsafe hypotheses are excluded from the tournament, not developed any further, and are not presented to the user.
    \item\textbf{Continuous monitoring of research directions.} A meta-review agent provides an overview of research directions, enabling the AI co-scientist to continuously monitor for potential safety concerns and alert users if a research direction is detected as being potentially unsafe.
    \item\textbf{Explainability and transparency.} All system components, including the safety review, provide not only the final recommendation but also a detailed reasoning trace that can be used to justify and audit system decisions.
    \item\textbf{Comprehensive logging.} All system activities are logged and stored for future analysis and auditing.
    \item\textbf{Safety evaluations and red teaming.} A preliminary red teaming effort has been undertaken to ensure that the current implementation of unsafe research goal detection is robust and accurate. This evaluation includes an assessment of the system behavior when presented with 1,200 adversarial research goals across 40 distinct topic areas as discussed in \cref{sec:result_safety}.
    \item\textbf{Trusted tester program.} We are enthused by the early promise of the AI co-scientist system and believe it is important to more rigorously understand its strengths and limitations in many more areas of science and biomedicine; alongside making the system available to many more researchers who it is intended to support and assist. To facilitate this responsibly and with rigour, we will be enabling access to the system for scientists through a Trusted Tester Program to gather real-world feedback on the utility and robustness of the system.
\end{itemize}
Crucially, the AI co-scientist is designed to operate with continuous human expert oversight, ensuring that final decisions are always made by scientists exercising their expert judgment.


\vspace{-0.1cm}
\section{Future Work}
\vspace{-0.1cm}
\label{sec:future work}

\paragraph{Immediate improvements.}
The AI co-scientist is in its early development, with many opportunities for improvement. Immediate improvement opportunities include enhanced literature reviews, cross-checks with external tools, improved factuality checking, and increased citation recall to minimize missed relevant research. Coherence checks would also improve the system by reducing the burden of reviewing flawed hypotheses.

\paragraph{Expanded evaluations.}
Developing more objective evaluation metrics, potentially incorporating automated literature-based validation and simulated experiments, is a key area. Methods to mitigate biases or error patterns inherited from the base LLMs are also important, alongside analysis of the complementarity and optimal combination of different agent components.

A critical need is a larger-scale evaluation involving more subject matter experts with diverse, high-resolution research goals. Stress-testing the system at every level of resolution (from disease mechanisms to protein design, and expanding to other scientific disciplines) will reveal further areas for improvement. Finally, since laboratory resources are limited, improved evaluation frameworks could assist with hypothesis selection.

\paragraph{Capabilities advancements.}
Several opportunities remain to expand co-scientist's capabilities. Reinforcement learning could enhance hypothesis ranking, proposal generation, and evolutionary refinement.

Currently, the system assesses text from open-access publications but not images, data sets, or major public databases. Integrating these publicly available resources would significantly enhance the co-scientist's ability to generate and justify proposed hypotheses.

Future work will focus on handling more complex experimental designs, such as multi-step experiments and those involving conditional logic. Integrating co-scientist with laboratory automation systems could potentially create a closed-loop for validation and a grounded basis for iterative improvement. Exploring more structured user interfaces for providing feedback and insights from targeted user research studies, beyond free text, could improve the efficiency of human-AI collaboration in this paradigm.


\vspace{-0.1cm}
\section{Discussion}
\vspace{-0.1cm}
\label{sec:discussion}
Our study represents an initial foray into accelerating novel scientific discovery with agentic AI systems and here, we discuss some of the broader implications. The co-scientist iteratively refines its generated hypotheses through a generate, debate, evolve'' approach with specialized agents under the hood. This design creates a self-improving cycle for research hypothesis generation, as measured by automated evaluation metrics, and showcases the potential benefits of test-time compute scaling for scientific reasoning.

Instead of brute-force generation of a vast number of hypotheses and relying on volume to chance into a few potentially useful ones, the system is designed to mimic key aspects of the scientific reasoning method in an intelligent manner. As detailed in~\cref{sec:methods}, the co-scientist employs principled internal mechanisms, including scientific debates, tournaments, iterative refinement, and human feedback loops to progressively improve the quality of its proposals, and converge on high quality and well-reasoned hypotheses.

\paragraph{AI co-scientist novelty is grounded in prior evidence.} The AI co-scientist facilitates the generation of novel scientific hypotheses and uncovering new insights by synthesizing extensive literature and identifying latent relationships. While its primary utility in its current form may lie in enabling more incremental advancements — such as the computational repurposing of existing therapeutics — it may also be able to support exploratory, breakthrough research. When researchers define such open-ended research goals requiring complex and out-of-the box thinking, the system may produce outputs of varying confidence. Therefore, rigorous validation and critical appraisal by domain experts remain paramount. This system is intended to augment, not supplant, human scientific reasoning, empowering researchers to accelerate discovery while maintaining intellectual control over the generated insights. We further expand on the novelty aspects in the specific context of the applications considered in this work in~\cref{sec:glossary}.

\paragraph{Multiple experimental validations of novel co-scientist hypotheses.} Notably, this work demonstrates the validation of co-scientist hypotheses via experimental findings in multiple laboratories. In drug repurposing, co-scientist identifies novel candidates for AML that demonstrated \textit{in vitro} efficacy at clinically relevant concentrations, including the identification of new repurposing opportunities beyond current preclinical knowledge. For liver fibrosis, the co-scientist proposes new epigenetic treatment targets, with subsequent \textit{in vitro} experiments validating the anti-fibrotic activity of several suggested compounds, including an FDA-approved drug. In the realm of antimicrobial resistance, the co-scientist independently recapitulates a novel, unpublished finding regarding the mechanism of cf-PICI transfer between bacterial species. Early results over several queries of varying scientific complexity suggests the co-scientist has a potential to contribute to discovery within various biomedical domains.

\paragraph{Test-time compute scaling with scientific reasoning priors and inductive biases.} In the experiments reported here, the co-scientist did not require specialized pre-training, post-training, or a reinforcement learning framework. It leverages the capabilities of existing base LLMs, potentially benefiting from updates to those models without requiring retraining of the co-scientist system itself, which presents advantages of compute efficiency and generalizability. The system's architecture incorporates self-play, internal consistency checks, and tournament-based ranking, which support iterative hypothesis generation, evaluation, and refinement. This is reflected in the observed improvement in hypothesis quality over time. This self-evolution can be improved further by expanded tool use integration, including database queries, allowing the co-scientist to ground its proposals in existing knowledge and identify novel connections. In the future, we may leverage data and tournament ranking generated by the co-scientist itself as feedback to improve the whole system using reinforcement learning.

\paragraph{Frontier LLM advancements and the AI co-scientist.}
The frontier LLMs used within the co-scientist system have demonstrated a continuing trend of rapidly improved capabilities, including reasoning, logic, and also some aspects of scientific literature comprehension. As our system is designed to be model-agnostic, we hypothesize that further improvements in frontier LLMs will also result in improved co-scientist performance, and enable new avenues of research including optimal agentic use of tools.

\paragraph{Implications for drug repurposing and discovery.}
These advancements have significant implications for various biomedical and scientific domains. For example, the integration of the co-scientist into the drug candidate selection process represents a significant advancement in evidence-based drug repurposing. Beyond simple literature mining, the co-scientist maybe capable of synthesizing novel mechanistic insights by connecting molecular pathways, existing preclinical evidence, and potential therapeutic applications into structured, testable specific-aims. This capability is particularly valuable as it provides researchers with literature-supported rationales and suggests specific experimental approaches for validation. Notably, the co-scientist's structured output can be leveraged to develop comprehensive single-patient IND (Investigational New Drug) applications for compassionate use cases. By systematically presenting mechanistic evidence, relevant preclinical data, and proposed monitoring parameters, the co-scientist facilitates the development of well-reasoned treatment protocols for patients with refractory (treatment-resistant) disease who have exhausted standard therapeutic options and are ineligible for clinical trials. This application is particularly valuable in rare or aggressive diseases where traditional drug development timelines may not align with urgent patient needs. The platform's ability to rapidly generate evidence-based therapeutic hypotheses, complete with safety considerations and monitoring parameters, can help clinicians and regulatory bodies make informed decisions about compassionate use applications while maintaining scientific rigor.

The application of the co-scientist in drug repurposing presents a very compelling opportunity for orphan drugs, where extensive safety and clinical data already exist from their original rare disease indications. Given that Phase III clinical trials can cost hundreds of millions of dollars, repurposing these well-characterized therapeutics offers an efficient path to expanding treatment options across multiple diseases. This is especially relevant as orphan drugs often target fundamental biological pathways that may be relevant in other conditions, but these connections might not be immediately apparent through traditional research approaches. By systematically evaluating existing clinical data, safety outcomes, and mechanistic insights, the co-scientist can help identify promising new therapeutic applications while taking advantage of the investment already made in drug development and safety validation. This approach not only maximizes the utility of existing therapeutics but also provides a more rapid path to addressing unmet medical needs across a broader patient population.

More broadly, the co-scientist may also be potentially impactful throughout the entire drug discovery spectrum as evidenced by the early work on co-scientist assisted target discovery for liver fibrosis. 

\paragraph{Automation bias and impact on human scientific creativity.} 
Realizing the full potential of AI in biomedicine and science requires proactively addressing potential pitfalls. Over-reliance on AI-generated suggestions in collaborative AI systems could diminish critical thinking and increase homogeneity in research. Studies on AI's impact on creativity and ideation show mixed results; some suggest a risk of homogenization of ideas across populations~\citep{homogeneity_ideas}, while others are less conclusive~\citep{ashkinaze2024aiideasaffectcreativity}. The correlated success / failure modes of LLMs~\citep{wenger2025weredifferentweresame}, due to similar training data, could also artificially narrow scientific inquiry. Furthermore, AI system blind spots and performance variations across research domains must be considered. Therefore, scalable factuality and verification methods, alongside peer review and careful consideration of potential biases, are essential. Careful design and use of systems like the co-scientist are crucial to mitigate these risks.

\paragraph{AI as a catalyst for both scientific discovery and equity.}
Despite these risks, AI holds immense potential to democratize access to scientific information and accelerate discovery, particularly benefiting historically neglected and resource-constrained areas~\citep{sun2018addressing, george2023addressing}. In essence, AI can ``raise the tide'' of scientific progress, lifting all boats, especially those that have historically been left behind. Realizing this potential requires strategic investments and careful calibration of AI systems to foster ideation and innovation while minimizing false positives. This includes focusing on historically neglected research topics and addressing variations in performance across different scientific domains with varying amounts of pre-existing data. While current AI systems may tend to produce incremental ideas and research hypotheses, ongoing development aims to create systems capable of generating truly original, unorthodox and transformative scientific theories. Proactive mitigation of these challenges will ensure that AI serves as a powerful tool for all scientists, promoting a more equitable and innovative future for scientific explorations.

\clearpage

\section{Conclusion}
The AI co-scientist represents a promising step towards AI-assisted augmentation of scientists and acceleration of scientific discovery. Its ability to generate novel testable hypotheses across diverse scientific and biomedical domains, some supported by experimental findings, along with the capacity for recursive self-improvement with increasing compute, demonstrates the promise of meaningfully accelerating scientists’ endeavours to resolve grand challenges in human health, medicine and science. This innovation opens numerous questions and opportunities. Applying the empiric and responsible approach of science to the AI co-scientist system itself can thereby enable safe exploration of its undoubted potential, including how collaborative and human-centred AI systems might be able to augment human ingenuity and accelerate scientific discovery.

\vspace{12pt}
\subsubsection*{Acknowledgments}
We thank our teammates Subhashini Venugopalan, John Platt, Erica Brand, and Yun Liu for their detailed technical feedback on the manuscript. We thank Jakob T Rostoel, Cora Chmielowska and Jonasz B Patkowski from Imperial College London and Jakkapong Inchai, Weida Liu, and Wenlong Ren from Stanford University for providing expert feedback on the AI system introduced in this work, and the lab of Ravi Majeti from Stanford University for generously providing the AML cell lines used in this work. We thank Ritu Raman, Ryan Flynn, Charlie Hempstead, Lord Ara Darzi, Omar Abudayyeh, Jonathan Gootenberg, Nic Fishman, Jason Lequyer, Dan Leesman, Ravi Solanki, Dennis Gong and Ananthan Sadagopan for feedback on different aspects of the AI system and the work. We also thank Maen Abdelrahim, Ethan Burns, Preethi Prasad and Hanh Mai for their clinical expertise and expert evaluation.

We thank our teammates Thomas Wagner, Alessio Orlandi, Natasha Latysheva, Nir Kerem, Yaniv Carmel, Hussein Hassan Harrirou, Laurynas Tamulevičius and Grzegorz Glowaty for their technical support. We thank Taylor Goddu, Resham Parikh, Siyi Kou, Rachelle Sico, Amanda Ferber, Cat Kozlowski, Alison Lentz, KK Walker, Roma Ruparel, Jenn Sturgeon, Lauren Winer, Juanita Bawagan, Ed-Allt Graham, Tori Milner, MK Blake, Jack Mason, Erika Radhansson, Indranil Ghosh, Jay Nayar, Brian Cappy, Celeste Grade, Abi Jones, Laura Vardoulakis, Lizzie Dorfman, Ashmi Chakraborthy, Delia Williams-Falokun, Maggie Shiels, Kalyan Pamarthy, Sarah Brown,  Christian Wright, and S. Sara Mahdavi for their support and guidance during the course of this project. Finally, we thank Michael Brenner, Zoubin Ghahramani, Dale Webster, Joelle Barral, Michael Howell, Susan Thomas, Karen DeSalvo, Jason Freidenfelds, Ronit Levavi Morad, Vladimir Vuskovic, Ali Eslami, Anna Koivuniemi, Greg Corrado, Royal Hansen, Andy Berndt, Noam Shazeer, Oriol Vinyals, Koray Kavukcuoglu, James Manyika, Jeff Dean and Demis Hassabis for their support of this work.


\vspace{12pt}

\newpage
\setlength\bibitemsep{3pt}
\printbibliography
\balance
\clearpage
