\section{Introduction}
\label{sec:introduction}

\begin{figure}[htbp!]
\centering
\includegraphics[width=\textwidth,keepaspectratio]{figures/fig1.pdf}
\vspace{0.1cm}
\caption{\textbf{The AI co-scientist system design and experimental validation summary.} (a) Here, we illustrate the different components of the the AI co-scientist multi-agent system, and its interaction paradigm with scientists. Given a research goal in natural language, the co-scientist generates novel research hypotheses and proposals. The system employs specialized agents — Generation, Reflection, Ranking, Evolution, Proximity (which evaluates relatedness), Meta-review (which provides high level analysis) — to continuously generate, debate, and evolve research hypotheses within a tournament framework. Feedback from the tournament enables iterative improvement, creating a self-improving loop towards novel and high-quality outputs. The co-scientist leverages tools, including web search and specialized AI models to improve the grounding and quality of generated research hypotheses. Scientists can converse with the co-scientist in natural language to specify research goals, incorporate constraints, provide feedback and suggest new directions for explorations via the designated user interface. (b) We perform end-to-end validation of the co-scientist generated hypotheses in three important topics of biomedicine with varied complexity--- suggesting novel drug repurposing candidates for acute myeloid leukemia (AML) (upper panel), discovering novel epigenetic targets for liver fibrosis treatment (middle panel), and recapitulating the discovery of novel mechanism of gene transfer evolution in bacteria key to anti-microbial resistance (lower panel). The co-scientist's hypotheses for these three settings are externally, independently validated by \textit{in vitro} laboratory experiments and detailed in separate preprints co-timed with this work. In the figure, blue denotes expert scientist inputs while red denotes the co-scientist agents or outputs.
}
\label{fig:system-overview}
\end{figure}

Human ingenuity and creativity propel the advancement of fundamental research in science and medicine. However, researchers, particularly in biomedicine, are faced with a breadth and depth conundrum. The complexity of biomedical topics require increasingly deep and specific subject matter expertise, while leaps in insight may still arise from broad knowledge bridging across disciplines. With the rapid rise in scientific publications and the availability of numerous technologies for specialized high-throughput assays, mastery of both discipline-specific depth and trans-disciplinary insight can be challenging.

Despite these challenges, many modern breakthroughs have emerged from trans-disciplinary endeavours. Emmanuelle Charpentier and Jennifer Doudna won the 2020 Nobel Prize in Chemistry for their work on CRISPR~\citep{jinek2012programmable}, which combined techniques and strategies ranging from microbiology to genetics to molecular biology. These benefits of synergy have also been seen beyond experimental biomedicine in numerous other areas of science. Notably, Geoffrey Hinton and John Hopfield combined ideas from physics and neuroscience~\citep{hopfield1982neural, hinton1986learning} to develop artificial intelligence (AI) systems, which were awarded the 2024 Nobel Prize in Physics.

There has been rapid technological progress in AI towards generally intelligent and collaborative systems, which might empower scientists in creatively traversing and expertly reasoning across disciplinary domains. Such systems are capable of advanced reasoning~\citep{guo2025deepseek, jaech2024openai, team2024gemini}, multimodal understanding~\citep{team2024gemini}, and agentic behaviors~\citep{wiesinger2024agents} such as the ability to use tools to solve complex tasks over long time horizons. Further, the trends with distillation~\cite{hinton2015distilling} and inference time compute costs~\citep{team2024gemini, team2024gemma} indicate that such intelligent and general AI systems are rapidly becoming more affordable and available. Motivated by the aforementioned unmet needs in the modern discovery process in science and medicine and building on the advancements in frontier AI~\citep{leslie2024frontier}, we develop and introduce an AI co-scientist system.

The co-scientist is designed to act as a helpful assistant and collaborator to scientists and to help accelerate the scientific discovery process. The system is a compound, multi-agent AI system~\cite{chen2024more} building on Gemini 2.0 and designed to mirror the reasoning process underpinning the scientific method~\citep{gower2012scientific}. Given a research goal specified in natural language, the system can search and reason over relevant literature to summarize and synthesize prior work and build on it to propose novel, original research hypotheses and experimental protocols for downstream validations~(\cref{fig:system-overview}a). The co-scientist provides grounding for its recommendations by citing relevant literature and explaining the reasoning behind its proposals. 

This work does not aim to completely automate the scientific process with AI. Instead, the co-scientist is purpose-built for a ``scientist-in-the-loop'' collaborative paradigm, to help domain experts augment their hypothesis generation process and guide the exploration that follows. Scientists can specify their research goals in simple natural language, including informing the system of desirable attributes for the hypotheses or research proposals it should create and the constraints that the synthesized outputs should satisfy. They can also collaborate and provide feedback in a variety of ways, including directly supplying their own ideas and hypotheses, refining those generated by the system, or using natural language chat to guide the system and ensure alignment with their expertise.

The co-scientist works through a significant scaling of the test-time compute paradigm~\citep{snell2024scaling, brown2019superhuman, silver2016mastering} to iteratively reason, evolve, and improve the outputs as it gathers more knowledge and understanding. Underpinning the system are thinking and reasoning steps---notably a self-play based scientific debate step for generating novel research hypotheses; tournaments that compare and rank hypotheses via the process of finding win and loss patterns, and a hypothesis evolution process to improve their quality. Finally, the agentic nature of the system enables it to recursively self-critique its output and use tools such as web-search to provide itself with feedback to iteratively refine its hypotheses and research proposals.

While the co-scientist system is general-purpose and applicable across multiple scientific disciplines, in this study we focus our development and validation of the system to biomedicine. We validate the co-scientist's capability in three impactful areas of biomedicine with varied complexity: (1) drug repurposing, (2) novel treatment targets discovery, and (3) new mechanistic explanations for antimicrobial resistance (\cref{fig:system-overview}b).

Drug development is an increasingly time-consuming and expensive process~\citep{ringel2020breaking} in which new therapeutics require restarting many aspects of the discovery and development process for each indication or disease (roughly 70\% of drug approvals are for new drugs). In contrast, drug repurposing—--identifying novel therapeutic indications for drugs beyond their original intended use--—has emerged as a compelling strategy to address these challenges~\citep{pushpakom2019drug}. Successful examples of repurposing include Humira (adalimumab) and Keytruda (pembrolizumab), both of which have become among the most successful drugs in history.~\citep{pushpakom2019drug}. The process typically involves analyzing molecular signatures, signaling pathways, drug interactions, clinical trial results, adverse event reports, and other literature-based information~\citep{xia2024drug}, along with off-label use data and, in some cases, patient experiences. However, drug repurposing is limited by several factors: (1) the need for extensive expertise across biomedical, molecular biology, and biochemical systems, (2) the inherent complexity of mammalian biological systems, and (3) the time-intensive nature of traditional computational biology analyses required. We leverage the co-scientist to generate predictions for large-scale drug repurposing, validating the generated predictions using a combination of computational biology, expert clinician feedback, and \textit{in vitro} wet-lab validation approaches. Notably, our system has proposed novel repurposing candidates for acute myeloid leukamia (AML) that inhibit tumor viability at clinically relevant concentrations \textit{in vitro} across multiple AML cell lines.

Unlike drug repurposing, which is a combinatorial search problem through a large but constrained set of drugs and diseases, identifying novel treatment targets for diseases presents a more significant challenge, traditionally requiring extensive literature review, deep biological understanding, sophisticated hypothesis generation and complex experimental validation strategies. The uncertainty of identifying novel treatment targets is significantly greater than in drug repurposing, as it involves not only repurposing existing compounds but also uncovering entirely new components and mechanisms within biological systems. This target discovery process can be inefficient, potentially leading to suboptimal hypothesis selection and prioritization for \textit{in vitro} and \textit{in vivo} experimentation. Given the high costs and time associated with experimental validation, a more effective approach is needed. We probe the capabilities of the co-scientist to propose, rank, and provide experimental protocols for novel research hypotheses pertaining to target discovery. To demonstrate this capability, we focus on liver fibrosis, a prevalent and serious disease, showcasing the co-scientist's potential to discover novel treatment targets amenable to experimental validation. In particular, the co-scientist has suggested novel epigenetic targets demonstrating significant anti-fibrotic activity in human hepatic organoids.

As a third validation of the capabilities of our system, we focus on generation of hypotheses to explain mechanisms related to gene transfer evolution in bacteria pertaining to antimicrobial resistance (AMR) - mechanisms developed by microbes to circumvent drug applications used to fight infections. This is arguably an even more complex challenge than drug repurposing and target discovery and involves understanding of not only the molecular mechanisms of gene transfer (conjugation, transduction, and transformation) but also the ecological and evolutionary pressures that drive the spread of AMR genes: a system-level problem with many interacting variables. This is also an important healthcare challenge with increasing rates of infections and deaths worldwide~\citep{keown2014antimicrobial}. In this validation, researchers instructed the AI co-scientist to explore a topic that had already been subject to novel discovery by their independent research group. Notably, at the time of instructing the AI co-scientist system, the researchers' novel experimental insights had not yet been published or revealed in the public domain. The system was instructed to hypothesize how capsid-forming phage-inducible chromosomal islands (cf-PICIs) exist across multiple bacterial species. The system independently proposed that cf-PICIs interact with diverse phage tails to expand their host range. This \textit{in silico} discovery mirrored the novel and experimentally validated results that expert researchers had already performed, as detailed in the co-timed report~\citep{he2025chimeric, penades2025ai}.

Overall, our key contributions are summarized as follows:
\begin{itemize}[leftmargin=1.5em,rightmargin=0em]
    \item\textbf{Introducing an AI co-scientist.} We develop and introduce an AI co-scientist that goes beyond literature summarization and ``deep research'' tools to assist scientists in uncovering new knowledge, novel hypothesis generation and experimental planning.
    \item\textbf{Significant scaling of test-time compute paradigm for scientific reasoning.} The co-scientist is built on a Gemini 2.0 multi-agent architecture, utilizing an asynchronous task execution framework. This framework allows the system to flexibly allocate computational resources to scientific reasoning, mirroring key aspects of the scientific method. Specifically, the system uses self-play strategies, including a scientific debate and a tournament-based evolution process, to iteratively refine hypotheses and research proposals creating a self-improving loop. Using automated evaluations across 15 complex expert curated open scientific goals, we demonstrate the benefits of scaling the test-time compute paradigm with the AI co-scientist outperforming other state-of-the-art (SOTA) agentic and reasoning models in generating high quality hypotheses for complex problems.
    \item\textbf{Expert-in-the-loop scientific workflow.} Our system is designed for collaboration with scientists. The system can flexibly incorporate conversational feedback in natural language from scientists and co-develop, evolve and refine outputs.
    \item \textbf{End-to-end validation of the co-scientist in important topics in biomedicine.} We present end-to-end validation of novel AI-generated hypotheses through new empirical findings in three distinct and increasingly complex areas of biomedicine: drug repurposing, novel target discovery, and antimicrobial resistance. The AI co-scientist predicts novel repurposing drugs for AML, identifies novel epigenetic treatment targets grounded in preclinical evidence for liver fibrosis, and proposes novel mechanisms for gene transfer in bacterial evolution and antimicrobial resistance. These discoveries from the AI co-scientist have been validated in wet-lab settings and are detailed in separate, co-timed technical reports.
\end{itemize}


\section{Related Works}
\subsection{Reasoning models and test-time compute scaling}
The modern revolution in foundation AI models~\citep{bommasani2021opportunities} and large language models (LLMs) has been largely driven by advances in pre-training techniques~\citep{erhan2010does, radford2018improving}, leading to breakthroughs in models like the GPT and Gemini family~\citep{team2023gemini, achiam2023gpt}. These models, trained on increasingly massive internet-scale and multimodal datasets, have demonstrated impressive abilities in language understanding and generation leading to breakthrough performance in a variety of benchmarks~\citep{chowdhery2022palm, google2023palm2}.  However, a key area of ongoing development is enhancing their \textit{reasoning} capabilities. This has led to the emergence of ``reasoning models'' which go beyond simply predicting the next word and instead attempt to mimic human thought processes~\citep{wei2022chain}. One promising direction in this pursuit is the test-time compute paradigm.  This approach moves beyond solely relying on the knowledge acquired during pre-training and allocates additional computational resources during inference to enable System-2 style thinking---slower deliberate reasoning to reduce uncertainty and progress optimally towards the goal~\citep{kahneman2011thinking}. This concept emerged with early successes such as AlphaGo~\citep{silver2016mastering}, which used Monte Carlo Tree Search (MCTS) to explore game states and strategically select moves, and Libratus~\citep{brown2019superhuman}, which employed similar techniques to achieve superhuman performance in poker. This paradigm has now found applications in LLMs, where increased compute at test-time allows for more thorough exploration of possible responses, leading to improved reasoning and accuracy~\citep{wei2022chain, yao2024tree, zelikman2022star, chen2024more, snell2024scaling,4928, muennighoff2025s1, tu2024towards}. Recent advancements, like the Deepseek-R1 model~\citep{guo2025deepseek}, further demonstrate the potential of test-time compute by leveraging reinforcement learning to refine the model's ``chain-of-thought'' and enhance complex reasoning abilities over longer horizons. In this work, we propose a significant scaling of the test-time compute paradigm using inductive biases derived from the scientific method to design a multi-agent framework for scientific reasoning and hypothesis generation without any additional learning techniques.

\subsection{AI-driven scientific discovery}
AI-driven scientific discovery represents a paradigm shift in how research is conducted across various scientific domains. Recent advancements, particularly the development of large deep learning and generative models, have cemented AI's role in scientific discovery. This is best exemplified by AlphaFold 2's remarkable progress in the grand challenge of protein structure prediction, which has revolutionized structural biology and opened new avenues for drug discovery and materials science~\citep{jumper2021highly}. Other notable examples include the development of novel antibiotics, protein binder design,  and material discovery with AI~\citep{wong2024discovery, zambaldi2024novo, merchant2023scaling}. 

Building on these successes with specialized, bespoke AI models, there has been recent work exploring the even more ambitious goal of fully integrating AI, especially modern LLM-based systems, into the complete research workflow, from initial hypothesis generation all the way to manuscript writing. This end-to-end integration represents a significant shift, presenting both unprecedented opportunities and significant challenges as the field moves beyond specialized AI tools toward realizing the potential of AI as an active collaborator, or even, as some envision, a nascent ``AI scientist''~\citep{lu2024ai, schmidgall2025agent}.

As an example of this shift, Liang et al.~\cite{liang2024can} directly assessed the utility of LLMs for providing feedback on research manuscripts. Through both a retrospective analysis of existing peer reviews and a prospective user study, they demonstrated the significant concordance between LLM-generated feedback and that of human reviewers. Their study, using GPT-4~\citep{openai2023gpt4}, found that a majority of researchers perceived LLM-generated feedback as helpful, and in some instances, even more beneficial than feedback from human colleagues. However, while valuable, their work focuses solely on the feedback stage of the scientific process, leaving open the question of how LLMs might be integrated into the full research cycle, from hypothesis formation to experimental validation and manuscript writing.

Another effort embodying this shift is PaperQA2~\citep{skarlinski2024language}, an AI agent for scientific literature search and summarization. The authors claimed to surpass PhD and postdoc researchers on multiple literature research tasks, as measured both by performance on objective benchmarks and human evaluations. While the system is a useful for synthesizing information, it does not engage in scientific reasoning for novel hypothesis generation.

HypoGeniC, a system proposed by Zhou et al.~\cite{zhou2022least}, tackles hypothesis generation by iteratively refining hypotheses using LLMs and a multi-armed bandit-inspired approach. The process begins with a small set of examples, from which initial hypotheses are generated. These hypotheses are then iteratively updated through exploration and exploitation, guided by a reward function based on training accuracy.  This refined set of hypotheses is subsequently used to construct an interpretable classifier. However, the method's reliance on retrospective data for evaluation means the degree to which the system can generate truly novel hypotheses remains an open question. Furthermore, the system lacks end-to-end validation beyond subjective human evaluations.

Ifargan et al.~\cite{ifargan2025autonomous} present ``data-to-paper'', a platform that systematically guides multiple LLM and rule-based agents to generate research papers, with automated feedback mechanisms and information tracing for verification. However, the evaluations are limited to recapitulating existing peer-reviewed publications and its unclear if the system can generate truly novel, yet grounded hypothesis and research proposals.

Virtual Lab~\citep{swanson2024virtual} is another closely related work. Here, the authors propose a team of LLM agents with a ``principal investigator'' LLM guiding a team of specialized LLM agents to solve a scientific problem. The LLM team receives high level human supervision. The authors demonstrate the utility of their work by leveraging Virtual Lab to design nanobody binders to recent variants of SARS-CoV-2 with experimental validation. While similar in spirit, there are significant design differences to our approach and the generality of the system remains unclear.

Boiko et al.~\citep{boiko2023autonomous} introduced ``Coscientist'', a multi-agent system powered by GPT-4, designed for autonomous execution of complex chemical experiments. This system integrates capabilities such as web and document searching, and code execution, to facilitate independent experimental design, planning, and execution. In addition to similar sounding names, both ``Coscientist'' and our system share the overarching goal of accelerating scientific discovery through AI. However, there are several important distinctions. Notably, ``Coscientist'' is quite narrowly focused on chemical research while ours is much broadly applicable across science. Secondly, our system has important technical innovations that lead to a self-improving system that can uncover new, original knowledge while their approach is a more vanilla-stitching of GPT-4 based agents. Finally, despite the name, ``Coscientist'' prioritizes a high degree of autonomy in experimental execution, directly interfacing with laboratory hardware. Our system, instead, is explicitly designed as a collaborative tool, emphasizing a ``scientist-in-the-loop'' approach and centers on the more cognitive aspects of the research process.

Finally, Lu et al.~\cite{lu2024ai} propose ``The AI Scientist'', a fully automated system designed to conduct research using multiple collaborating LLM agents. These agents handle all stages of the research process, from defining research problems and conducting literature reviews to designing and executing experiments, and even writing up the results. The design shares similarities with our work---the key differences being our focus on the scaling of the test-time compute paradigm to generate high quality hypotheses and research proposals. Secondly, their proposed system has limited automated evaluations; in contrast, our work has a combination of automated, human expert and end-to-end wet lab validations. Finally, our goal is to not to automate scientific discovery, rather to build a helpful AI collaborator for scientists.

\subsection{AI for biomedicine}
More broadly, large AI models are increasingly demonstrating their potential in biomedical science. Both general purpose (GPT-4, Gemini) and specialized LLMs (Med-PaLM, Med-Gemini, Galactica, Tx-LLM) have shown strong performance on biomedical reasoning and question-answering benchmarks~\citep{team2023gemini, achiam2023gpt, singhal2022large, saab2024capabilities, taylor2022galactica, chaves2024tx}. Beyond benchmarks, Med-PaLM 2, was successfully applied to identify causative murine genetic factors for traits such as diabetes, cataracts, and hearing loss~\cite{tu2023genetic}---an early example of hypothesis generation and LLM-assisted discovery. We have also seen the exciting development of specialized foundation and large language models trained on DNA, RNA and protein sequences with a variety of applications~\citep{nguyen2024sequence, lin2023evolutionary, ruffolo2024design, shaw2024protex}. Although AI in biology and medicine often necessitates specialization, the rapid progress of frontier AI models has blurred the distinction. As these models grow in scale, data diversity, and complexity, they continue to achieve breakthroughs in areas once thought to require domain-specific AI. Our co-scientist system, with its modular multi-agent architecture, is flexibly designed to build on top of these advancements in general-purpose frontier AI models and leverage specialized AI models as tools to enhance the capabilities.
 
Drug repurposing is an important area of validation experiments in this work. The traditional approach to this task requires both computational and experimental approaches and a comprehensive understanding of disease-drug interactions~\cite{pushpakom2019drug, krishnamurthy2022drug}. While methods like knowledge graphs with graph convolutional networks have shown promise~\cite{zitnik2018modeling, morselli2021network}, their applicability is limited by the initial knowledge graph's scope.  TxGNN~\cite{huang2024foundation}, an example of a specialized biomedical foundation model with a graph based approach, addresses ``zero-shot'' repurposing for novel diseases but remains dependent on the underlying knowledge graph's quality and lacks sufficient scalability and explainability. Furthermore, no end-to-end validations of the model predictions were reported in the study. In contrast, our work, leveraging state-of-the-art LLMs in the co-scientist setup, is more scalable. We report a combination of expert evaluations and wet-lab experiments to validate the system predictions.

% \begin{figure}
%     \centering
%     \includegraphics[width=0.5\linewidth]{Move_teaser.pdf}
%     \caption{Comparison of different dynamic compute approaches. length of arrow indicates residual transformation per token while width indicates velocity of transformation.}
%     \label{fig:enter-label}
% \end{figure}

\section{Method}
\label{sec:method}
Residual connections play a crucial role in shaping token representations, yet their dynamics remain underexplored in the context of efficient decoding. In this work, we delve deeper into transformer residual dynamics and investigate how modulating residual transformation velocity can improve inference efficiency in token-level processing, optimizing both dense and sparse MoE transformers.


\subsection{Residual Dynamics and Motivation for Multi-rate Residuals} \label{sec:motivation}

To analyze how hidden representations evolve across different layers of a transformer architecture, it's crucial to consider the effect of residual connections. Each transformer decoder layer typically has residual connections across attention and MLP submodules. As the residual stream $h_i$ traverses from interval $E_j$ to $E_{j+1}$, it undergoes a residual transformation given by:  
% \begin{equation}
% \label{eq:slow_residual_transformation}
% H_{E_{j+1}} = H_{E_j} \prod_{i=E_j}^{E_{j+1}} \left( I + \mathcal{A}_i \right) \left( I + \mathcal{M}_i \right) \quad \text{where} \quad \mathcal{A}_i = f(c_i, h_{i}), \mathcal{M}_i = g(h_i)
% \end{equation}

\begin{equation} \label{eq:slow_residual_transformation}
h_{E_{j+1}} = h_{E_j} + \sum_{i=E_j}^{E_{j+1}-1} \left( \mathcal{A}_i(h_i) + \mathcal{M}_i(h_i + \mathcal{A}_i(h_i)) \right) \quad \text{where} \quad \mathcal{A}_i = f(c_i, h_{i}), \mathcal{M}_i = g(h_i). 
\end{equation}

Here, \( \mathcal{A}_i \) denotes the non-linear transformation introduced by the multi-head attention mechanism at layer \( i \), while \( \mathcal{M}_i \) corresponds to the non-linear transformation of the MLP block at the same layer. These transformations depend on the input residual stream \( h_i \) and, in the case of \( \mathcal{A}_i \), the previous contextual representation \( c_i \).\footnote{Normalization layers are typically applied in practice but are omitted here for simplicity of the argument.}


% For easy tokens, the magnitude and direction of this delta transformation become progressively smaller with each successive layer as shown in \cref{fig:delta_transformation}. Consequently, it is feasible to predict these tokens after only a few residual connections, whereas harder tokens necessitate more extensive processing through additional layers.

\begin{figure}[ht]
    \centering
    \begin{subfigure}{0.48\textwidth}
        \centering
        \includegraphics[width=\textwidth]{sections/figures/residual_change.pdf}
        \caption{}
        \label{fig:residual_change}
    \end{subfigure}%
    \hfill
    \begin{subfigure}{0.48\textwidth}
        \centering
        \includegraphics[width=\textwidth]{sections/figures/alignment_wrt_dedicated_model.pdf}
        \caption{}
    \label{fig:alignment_wrt_dedicated_model}
    \end{subfigure}
    \caption{(a) As residual streams propagate through the model, the directional shifts in the residuals become progressively smaller. (b) A dedicated model with $k$ layers achieves a faster rate of change in residual streams and higher alignment than base model leveraging early exit mechanisms at layer $k$.}
    \label{fig}
\end{figure}


To examine whether residual transformations can be accelerated across layers, we conducted experiments using a diverse set of prompts on a pre-trained Phi3 model~\cite{phi3_report}. As illustrated in \cref{fig:residual_change}, we measured the directional shift in residual states as \( 1 - \mathcal{C}(h_{i-1}, h_i) \), where \(\mathcal{C}\) denotes normalized cosine similarity. This shift is notably higher in the initial layers, gradually decreasing in subsequent layers. This behavior allows traditional early exit approaches to effectively accelerate decoding by enabling earlier exits for simpler tokens. However, these approaches typically rely on a distance-based approximation, where the full residual transformation of the model is approximated by the residual transformations of the initial layers. To gain deeper insights into the distance versus velocity aspects of residual transformation, we conducted a comparative study. Specifically, we trained an early exit head at layer $k$ of the Phi3 model, which consists of 32 layers, restricting the distance traveled by each token. To accelerate the residual transformation relative to number of layers, we trained a smaller model consisting of only $k$ layers, while keeping all other hyperparameters consistent. We then compared the next-token prediction accuracy of the early exit head of the base model with that of the smaller model. To ensure an equal number of trainable parameters, we inserted low-rank adapters into the smaller model and trained only these adapters, whereas, in the distance-based approach, we trained solely the early exit head. In addition, to accelerate the residual transformation in smaller model, we distilled the residual streams from the larger model by incorporating a distillation loss ~\cite{sanh2019distilbert} between the residual state at layer \(i\) of the smaller model and the residual state at layer \(4 \times i\) of the larger model. As shown in ~\cref{fig:alignment_wrt_dedicated_model} the smaller model demonstrates a significantly faster rate of change in residual streams, leading to higher next token prediction accuracy after $k$ layers compared to the base model that employs traditional early exit mechanisms after $k$ layers \cite{schuster2022confident, chen2023eellm, varshney-etal-2024-investigating}. This experimental setup, which modifies only the rate of change in residual streams while keeping other factors constant, suggests that dense transformers, trained with a fixed number of layers, may inherently possess a slow residual transformation bias.

This observation raises an intriguing question: if the rate of change in residual streams could be accelerated relative to the number of layers, is it possible to facilitate earlier alignment for a greater proportion of tokens? Earlier alignment would be beneficial to not only facilitate dynamic computation but also for generating speculative tokens efficiently with high acceptance rates in speculative decoding setups ~\cite{leviathan2023fast, chen2023accelerating}. 

%thereby enhancing the efficiency of early exiting? 
 % This bias likely constrains the effectiveness of early exiting, particularly for easier tokens. By addressing this limitation through accelerated residual transformations, we hypothesize that it is possible to substantially improve the efficiency and accuracy of early exit strategies in transformer models.

\subsection{Multi-Rate Residual Transformation} \label{m2r2_method}

To address the slow residual transformation bias described in ~\cref{sec:motivation}, we introduce \textit{accelerated residual streams} that operate at rate $R$ relative to original slow residual stream. We pair slow residual stream, $h$ with an accelerated residual stream, $p$, which has an intrinsic bias towards earlier alignment. Relative to ~\cref{eq:slow_residual_transformation}, accelerated residual transformation from interval $E_j$ to $E_{j+1}$ can be represented as: 

% \begin{equation}
% \label{eq:fast_residual_transformation}
% P_{E_{j+1}} = P_{E_j} \prod_{i=E_j}^{E_{j+1}} \left( I + \hat{\mathcal{A}_i} \right) \left( I + \hat{\mathcal{M}_i} \right) \quad \text{where} \quad \hat{\mathcal{A}_i} = \hat{f}(c_i, P_{i}), \hat{\mathcal{M}_i} = \hat{g}(P_{i})
% \end{equation}


\begin{equation} \label{eq:fast_residual_transformation}
p_{E_{j+1}} = p_{E_j} + \sum_{i=E_j}^{E_{j+1}-1} \left( \hat{\mathcal{A}_i}(p_i) + \hat{\mathcal{M}_i}(p_i + \hat{\mathcal{A}_i}(p_i)) \right) \quad \text{where} \quad \hat{\mathcal{A}_i} = \hat{f}(c_i, p_{i}), \hat{\mathcal{M}_i} = \hat{g}(h_i), 
\end{equation}



where $\hat{\mathcal{A}_i}$ and $\hat{\mathcal{M}_i}$ denote non-linear transformation added by layer $i$ to previous accelerated residual $p_{i}$. Similar to $\mathcal{A}_i$, non-linear transformation $\hat{\mathcal{A}_i}$ attends to same context $c_i$ but uses a different transformation $\hat{f}$ for accelerating $p_{E_j}$ relative to $h_{E_j}$. 

We integrate accelerated residual transformation directly into the base network using parallel accelerator adapters such that rank of accelerator adapters $R_p << d$ where $d$ denotes base model hidden dimension. This setup allows the slow residual stream $h_{E_j}$ to pass through the base model layers while the accelerated residual stream $p_{E_j}$ utilizes these parallel adapters as shown in ~\cref{fig:m2r2_main}. Both slow and accelerated residuals are processed in same forward pass via attention masking and incur negligible additional inference latency in memory bound decoding setups, while in compute bound decoding setups where FLOPs optimization is essential, accelerated residual stream utilizes a fraction of attention heads that of slow residual (see ~\cref{sec:flops_optimization}). Additionally, to maximize the utility of accelerated residual transformations without introducing dedicated KV caches, we propose a shared caching mechanism between the slow and accelerated streams which minimally impact alignment benefits of our approach while offering substantial memory savings (see ~\cref{fig:koala_alignment}). Specifically, the attention operation on the slow residuals \( \text{MHA}(h_t, h_{\leq t}, h_{\leq t}) \) is redefined for accelerated residuals as 
\[
\hat{\mathcal{A}} = MHA(p_t, h_{<t} \oplus p_t, h_{<t} \oplus p_t),
\]
where the accelerated residual at time-step $t$, \( p_t \) attends to the slow residual’s KV cache, facilitating the reuse of contextual information across both residual streams without incurring additional caching costs. Here, \(MHA(q, k, v) \) represents multi-head attention between query \( q \), key \( k \), and value \( v \).

\begin{figure}
    \centering
    \includegraphics[width=0.8\linewidth]{sections//figures/m2r2_main2.pdf}
    \caption{Multi-rate Residuals Framework: Slow residual stream of base model is accompanied by a faster stream that operates at a $2-(J+1)\times$ rate relative to the slow stream, undergoing transformations via accelerator adapters as detailed in \cref{m2r2_method}, where J denotes number of early exit intervals. Colors within the slow and fast residual streams indicate similarity, with matching colors representing the most closely aligned residual states. At the beginning of the forward pass and at each exit point, the accelerated residual state is initialized from the corresponding slow residual state to avoid gradient conflict during training (see ~\cref{sec:grad_conflict}). Early exiting decisions are informed by the Accelerated Residual Latent Attention (ARLA) mechanism, described in \cref{method_arla}, which evaluates residual dynamics across consecutive exit gates.}
    \label{fig:m2r2_main}
\end{figure}

% Furthermore. to maximize the benefits of fast residual transformations without using dedicated KV caches, we propose sharing the fast network’s cache with the slow network. Formally speaking, We modify attention operation on slow residuals $MHA(H_t, H_{<=t}, H_{<=t})$ as $MHA(P_{t}, H_{<t} \oplus P_t, H_{<t}  \oplus P_t)$ such that accelerated residuals attend to previous slow context KV cache, where $MHA(q,k,v)$ denotes multi head attention between query, $q$, key $k$ and value $v$.


\subsection{Enhanced Early Residual Alignment}
Early residual alignment is instrumental in optimizing early exiting, speculative decoding, and Mixture-of-Experts (MoE) inference mechanisms. In this section, we provide a detailed analysis of how accelerated residuals enhance these inference setups.

% By aligning the residual states of intermediate layers with the final output representations, the model can maintain high prediction accuracy even when computations are truncated at earlier layers. This enables more reliable early exiting, reducing the overall computational cost while preserving performance. Additionally, in speculative decoding, early residual alignment allows the model to make confident predictions using faster, partial computations, thereby accelerating inference without sacrificing output quality.


\subsubsection{Early Exiting} \label{method_early_exiting}

A prevalent strategy for enabling early exiting at an intermediate layer $E_{j}$ involves approximating the residual transformation between $E_{j}$ and the final layer $N-1$ using a linear, context independent mapping, $\mathcal{T}$, such that $H_{N-1} \approx \mathcal{T}(H_{E_{j}})$. This approximation has been extensively employed in conventional approaches ~\cite{schuster2022confident, chen2023eellm, varshney-etal-2024-investigating}, providing a computationally efficient means to project the output of deeper layers from intermediate states. Specifically, residual state of layer $N-1$ with this approximation can be expressed as:


% \begin{equation}
% \label{eq: vanila_ea_assumption}
% \Phi(H_{E_{j}}) \sim H_{E_{j}} \prod_{i=E_{j}}^{N}\left( I + \mathcal{A}_i \right) \left( I + \mathcal{M}_i \right) \quad \text{where} \quad \Phi \perp C
% \end{equation}

\begin{equation} \label{eq:early_exiting}
h_{E_j} + \sum_{i=E_j}^{N-1} \left( \mathcal{A}_i(h_i) + \mathcal{M}_i(h_i + \mathcal{A}_i(h_i)) \right) \sim \mathcal{T}(h_{E_{j}})  \quad \text{where} \quad \mathcal{T} \perp c. 
\end{equation}


Here, $\mathcal{A}_i$ and $\mathcal{M}_i$ represent the residual contributions of the multi-head attention and MLP layers, respectively, while $\mathcal{T}$ remains independent of $c$, the preceding context.

This approach is inherently limited by two major factors: first, the assumption of linearity between $h_{E_{j}}$ and $h_{N-1}$ may not hold uniformly for all tokens, particularly when $E_j \ll N$. Second, the linear transformation $\mathcal{T}$ disregards the influence of the context $c$ and fails to account for the latent representations of previous contextual states. In contrast, M2R2 accelerated residual states mitigate both of these challenges by approximating the slow residual transformation of all layers via a faster residual transformation of fewer layers as:
% \begin{equation}
% H_{E_j} \prod_{i=E_j}^{N}\left( I + \mathcal{A}_i \right) \left( I + \mathcal{M}_i \right) \sim P_{E_j} \prod_{i=E_j}^{E_j+1}\left( I + \hat{\mathcal{A}_i} \right) \left( I + \hat{\mathcal{M}_i} \right)
% \end{equation}


\begin{equation} \label{eq:m2r2_approximating_ea}
h_{E_j} + \sum_{i=E_j}^{N-1} \left( \mathcal{A}_i(h_i) + \mathcal{M}_i(h_i + \mathcal{A}_i(h_i)) \right) \sim p_{E_j} + \sum_{i=E_j}^{E_{j+1}-1} \left( \hat{\mathcal{A}_i}(p_i) + \hat{\mathcal{M}_i}(p_i + \hat{\mathcal{A}_i}(p_i)) \right), 
\end{equation}

% \begin{equation} \label{eq:fast_residual_transformation}
% p_{E_{j+1}} = p_{E_j} + \sum_{i=E_j}^{E_{j+1}-1} \left( \hat{\mathcal{A}_i}(p_i) + \hat{\mathcal{M}_i}(p_i + \hat{\mathcal{A}_i}(p_i)) \right) \quad \text{where} \quad \hat{\mathcal{A}_i} = \hat{f}(c_i, p_{i}), \hat{\mathcal{M}_i} = \hat{g}(h_i) 
% \end{equation}






where $p_{E_j}$ is initialized from the slow residual state $h_{E_j}$ at each early exit interval $E_j$ using an identity transformation (see ~\cref{fig:m2r2_main}). As shown in ~\cref{fig:m2r2_residual_sim}, accelerated residuals offer a smoother, more consistent shift in residual direction across layers, in contrast to the abrupt changes typically seen at early exit points in standard early exit methods. Moreover, the normalized cosine similarity between accelerated states at early exit intervals and final residual states is substantially higher compared to traditional early exit techniques, highlighting improved alignment with final layer representations. Traditional adaptive compute methods are constrained by two principal factors: the number of tokens eligible for early exit at intermediate layers and the precision of early exit decision. If residual streams fail to saturate early, the majority of tokens remain ineligible for exit, thereby diminishing potential speedups. Additionally, imprecise delineations between tokens suitable for early exit can lead to underthinking (premature exits that adversely affect accuracy) or overthinking (unnecessary processing that compromises efficiency) ~\cite{zhou2020self, dai2020dynamic}. Enhanced early alignment using ~\cref{eq:m2r2_approximating_ea} helps to address  first issue. To address the second issue we introduce Accelerated Residual Latent Attention, which dynamically assesses the saturation of the residual stream, allowing for a more precise differentiation between tokens that can exit early and those requiring further processing.

% This results in uniform change in residual direction    
% % We keep $\mathcal{A} = \hat{\mathcal{A}}$, while $\hat{\mathcal{M}}$ is accelerated by a factor of $2 - (N_{E}+1)X$ relative to the slower residual transformation $\mathcal{M}$, where $N_E$ represents number of early exiting intervals.
% Figure~\cref{fig:rate_change_comparison} illustrates the comparative rate of change between these transformation streams.



% fig:rate_change_comparison
% - grid plot x axis -> layer id (0, 8) , y axis -> layer id -> dark color cell for max similarity , lighter for lower 
% 
-------------------------------------------------------
Let's consider residual stream $h_i$ traverses through interval $E_j$ to $E_{j+1}$ and undergoes residual transformation given by 
\begin{equation}
h_{E_{j+1}} = h_{E_j} \prod_{i=E_j}^{E_{j+1}} \left( 1 + \delta_i \right)    
\end{equation}

where $\delta_i$ denotes non-linear transformation added by layer $i$. Each non-linear transformation of layer $i$ is a function of previous contextual representation, $c_i$ and input residual stream $h_i-1$ as
$\delta_i = f(c_i, h_{i-1})$ 

One way to exit early at exit $E_j+1$ is to assume that residual transformation from $E_j+1$ to final layer $N-1$ can be approximated by a linear function $\phi$ as $h_{N-1} \sim \Phi(h_{E_j+1})$ and most conventional approaches such as \todo{cite EA papers} use this approach. In other words, 

\begin{equation}
\Phi(h_{E_j+1} \sim h_{E_j+1} \prod_{i=E_j+1}^{N} \left( 1 + \delta_i \right)   
\end{equation}

This approach suffers from two primary issues, linearity assumption from $h_E_j+1$ to $H_N-1$ if often incorrect, particularly when $E_j << N$. More importantly, linear transformation $\Phi$ doesn't consider effect of context $C_i$. M2R2  effectively addresses these issues as accelerated residual stream at interval $E_j+1$ can be represented as 

\begin{equation}
r_{E_{j+1}} = r_{E_j} \prod_{i=E_j}^{E_{j+1}} \left( 1 + \gamma_i \right)    
\end{equation}

where $\gamma_i$ denotes non-linear transformation added by layer $i$ to previous accelerated residual $r_i-1$. Similar to $\delta_i$, non-linear transformation $\gamma_i$ considers context $C_i$ as 
$\gamma_i = g(c_i, r_{i-1})$. So in summary, slow residual transformation is approximated by accelerated residual as: 

\begin{equation}
h_{E_j} \prod_{i=E_j}^{N} \left( 1 + \delta_i \right) \sim h_{E_j} \prod_{i=E_j}^{E_j+1} \left( 1 + \gamma_i \right)
\end{equation}

It's worth noting that accelerated residual $r_i$ and slow residual $h_i$ are processed concurrently at layer $i$ by constructing proper attention mask such as attention of slow residual is represented as 

$MHA(H_it, H_{i<=t}, H_{i<=t}$ while attention of fast residual is computed as 

$MHA(r_it, H_{i<=t}, H_{i<=t}$ where $MHA(q,k,v$ denotes multi head attention between query, $q$, key $k$ and value $v$.


------------------------------------------------------------------

Vertical latent attention on accelerated residual is computed as 
$MHA(S_mt, S(Ej<=i<=m)t, S(Ej<=i<=m)t)$ where $Smt$ denotes query/key/value projection in latent domain at layer $m$ at time $t$. 
------------------------------------------------------------------

Gradient conflict Avoidance: 

Let's consider $w_j$ is a trainable parameter that belongs to a layer between $E_j$ and $E_j+1$. Consider early exit loss at gate $E_j+1$, $L_j+1$, gradient propagation of $w_j$ at another trainable parameter $w_j-n$ can be gives as 

$\sum_{k=E_j-n}^{E_j} \beta_k \frac{\partial L_{E_k}}{\partial w_k}$

where $\beta_j$ denotes backward transformation coefficient for weight $w_j$ to reach gate $E_j$. 
 
On the other hand, gradient propagation in proposed approach can be represented as 

\[
\frac{\partial L_{E_j}}{\partial w_j} = 
\begin{cases} 
\beta_j \frac{\partial L_{E_j}}{\partial w_j} & \text{if } E_j \leq w_j \leq E_{j+1} \\
0 & \text{otherwise}
\end{cases}
\]







% \begin{figure}[ht]
%     \centering
%     \includegraphics[width=0.8\textwidth, height=5cm]{rate_change_comparison.png}
%     \caption{Rate of change comparison between fast and slow residual streams.}
%     \label{fig:rate_change_comparison}
% \end{figure}

%vary k and and plot EA accuracy for larger and smaller models. 

% \begin{figure}[ht]
%     \centering
%     \includegraphics[width=0.5\textwidth,height=5cm]{sections/figures/alignment_comparison_dialogsum.pdf}
%     \caption{Alignment of exited tokens for different early exit layers using traditional early exiting heads, dedicated faster networks, and faster residuals.}
%     \label{fig:small_model_early_exiting}
% \end{figure}


\textbf{Accelerated Residual Latent Attention} \label{method_arla}

In the context of residual streams, we observe that the decision to exit at a given layer can be more effectively informed by analyzing the dynamics of residual stream transformations, instead of solely relying on a classification head applied at the early exit interval $E_j$. To capture the subtle dynamics of residual acceleration, we propose a \textit{Accelerated Residual Latent Attention} (ARLA) mechanism. This approach involves making the exit decision at gate $E_j$ by attending to the residuals spanning from gate $E_{j-1}$ to $E_j$, rather than considering only the residual at gate $E_j$. To minimize the computational overhead associated with exit decision-making, the attention mechanism operates within the latent domain as depicted in ~\cref{fig:arla_arch}. Formally, for each interval $[E_j, E_{j+1}]$, the accelerated residuals are projected into Query ($Q^s_{E_j}, \ldots, Q^s_{E_{j+1}}$), Key ($K^s_{E_j}, \ldots, K^s_{E_{j+1}}$), and Value ($V^s_{E_j}, \ldots, V^s_{E_{j+1}}$) vectors, with latent dimension $d^s$ for $Q^s$, $K^s$, and $V^s$ being significantly smaller than hidden dimension of $p$.\footnote{We use $d^s = 64$ for experiments described in ~\cref{sec:experiments}.} Notably, when the router is allowed to make exit decisions at gate $E_j$ based on residual change dynamics, we observe that the attention is not confined to the residual state at $E_j$ but is distributed across residual states from $E_{j-1}$ to $E_j$, %as illustrated in Figure~\ref{fig:vertical_latent_attention_dynamics}. 
This broader focus on residual dynamics significantly reduces decision ambiguity in early exits, as demonstrated in Figure~\ref{fig:roc_arla}, which contrasts routers based on the last hidden state, and the proposed ARLA router.

%show R -> S transformation. 
%show parameter and flop overhead as compared to adapter on last hidden state.

% \begin{figure}[ht]
%     \centering
%     \includegraphics[width=0.5\textwidth,height=5cm]{sections/figures/roc_arla.pdf}
%     \caption{ROC curves of early exit decision strategies: confidence-based methods (CALM/LITE), routers based on the accelerated hidden state, and latent attention routers.}
%     \label{fig:decision_making_comparison}
% \end{figure}

% \begin{figure}[ht]
%     \centering
%     \includegraphics[width=0.5\textwidth,height=5cm]{vertical_latent_attention.png}
%     \caption{Vertical latent attention mechanism for optimizing early exit decisions by considering residuals from gate \(M\) through \(M-1\).}
%     \label{fig:vertical_latent_attention}
% \end{figure}

\begin{figure}[ht]
    \centering
    \begin{subfigure}{0.52\textwidth}
        \centering
        \includegraphics[width=\textwidth, height = 4cm]{sections/figures/arla_arch.pdf}
        \caption{Accelerated Residual Latent Attention (ARLA): Accelerated residuals between early exit gates are projected into latent domain and attention over residual states within the interval is computed to capture residual dynamics and exit decision is made based on residual saturation.}
        \label{fig:arla_arch}
    \end{subfigure}%
    \hfill
    \begin{subfigure}{0.45\textwidth}
        \centering
        \includegraphics[width=\textwidth, height = 4.5cm]{sections/figures/vla_roc.pdf}
        \caption{ROC classification curves of early exit decision strategies using a linear router used on last residual state ~\cite{schuster2022confident, varshney-etal-2024-investigating, chen2023eellm}  and using ARLA approach that considers residual dynamics. }
        \label{fig:roc_arla}
    \end{subfigure}
    \caption{Effectiveness of ARLA in capturing residual dynamics for early exiting decisions.}


\end{figure}



% \begin{figure}[ht]
%     \centering
%     \includegraphics[width=1\textwidth,height=5cm]{sections/figures/arla.pdf}
%     \caption{fig that plots 32 rows 2 cols heatmap showing attention at each gate}
%     \label{fig:vertical_latent_attention_dynamics}
% \end{figure}

\subsubsection{Self Speculative Decoding} \label{method_self_speculative_decoding}

An alternative means to exploit the early alignment properties of our approach is through the use of accelerated residual states for speculative token sampling to accelerate autoregressive decoding. Speculative decoding aims to speed up memory-bound transformer inference by employing a lightweight draft model to predict candidate tokens, while verifying speculated tokens in parallel and advancing token generation by more than one token per full model invocation \cite{leviathan2023fast, chen2023accelerating, xia2023speculative, miao2023specinfer}. Despite its effectiveness in accelerating large language models (LLMs), speculative decoding introduces substantial complexity in both deployment and training. A separate draft model must be specifically trained and aligned with the target model for each application, which increases the training load and operational complexity ~\cite{chen2023accelerating}. Additionally, this approach is resource-inefficient, as it requires both the draft and target models to be simultaneously maintained in memory during inference \cite{leviathan2023fast, chen2023accelerating}. 

One strategy to address this inefficiency is to leverage the initial layers of the target model itself to generate speculative candidates, as depicted in ~\cite{Tang2024}. While this method reduces the autoregressive overhead associated with speculation, it suffers from suboptimal acceptance rates. This occurs because the linear transformation employed for translating hidden states from layer $k$ to the final layer $N$ is typically a poor approximation, as discussed in ~\cref{sec:motivation} and ~\cref{method_early_exiting}. Our approach resolves this limitation by utilizing accelerated residuals, which demonstrate higher fidelity to their slower counterparts. By utilizing accelerated residuals operating at a rate of $N/k$, where $k$ denotes the number of layers used for candidate speculation, we are able to efficiently generate speculative tokens for decoding.\footnote{We typically set $k = 4$ to balance the trade-off between autoregressive drafting overhead and acceptance rate, as discussed in~\cref{sec:experiments}.}
 This technique not only obviates the need for multiple models during inference but also improves the overall efficiency and effectiveness of speculative decoding.

\begin{figure}
    \centering    \includegraphics[width=1\linewidth]{sections/figures/m2r2_aot_loading.pdf}
    \caption{Ahead-of-Time Expert Loading: M2R2 accelerated residual stream predicts experts required for future layers, reducing reliance on on-demand lazy loading. Speculative pre-loading is efficiently overlapped with computation of multi-head attention (MHA) and MLP transformations. Only incorrectly speculated experts are loaded lazily, resulting in faster inference steps and improved computational efficiency. Here, H indicates LBM Host while D indicates HBM Device.}
    \label{fig:moe_expert_aot_loading}
\end{figure}


\subsubsection{Ahead of Time Expert Loading:} \label{method_aot_expert_loading}

Recent advancements in sparse Mixture-of-Experts (MoE) architectures ~\cite{shazeer2017outrageously, fedus2022switch, artetxe2019massively, lepikhin2020gshard, zoph2022designing} have introduced a paradigm shift in token generation by dynamically activating only a subset of experts per input, achieving superior efficiency in comparison to dense models, particularly under memory-bound constraints of autoregressive decoding \cite{fedus2022switch, zoph2022designing}. This sparse activation approach enables MoE-based language models to generate tokens more swiftly, leveraging the efficiency of selective expert usage and avoiding the overhead of full dense layer invocation. In dense transformer models, pre-loading layers is a common strategy to enhance throughput, as computations of current layer can be overlapped with pre-loading of next layer parameters ~\cite{narayanan2021efficient, shoeybi2020megatron}. However, MoE models face a unique challenge: expert selection occurs dynamically based on previous layer’s output, making it infeasible to preload next layer’s experts in parallel. This limitation results in inherent latency, as expert loading becomes a sequential, on-demand process ~\cite{lepikhin2020gshard, fedus2022switch}.

To address this inefficiency, our method introduces a mechanism with \textit{accelerated residuals}, which not only captures key characteristics of base slower residual states but also exhibit high cosine similarity with their final counterparts (as illustrated in \cref{fig:m2r2_residual_sim}). By employing accelerated residual streams, we can effectively predict the necessary experts for future layers well in advance of their actual invocation. Specifically, using a $2\times$ accelerated residual, the experts needed for layers $2i+2$ and $2i+3$ can be identified while still computing in layer $i$, thus overcoming the bottleneck of sequential, on-demand expert selection and mitigating latency in the decoding pipeline, as shown in \cref{fig:moe_expert_aot_loading}. Note that, we use fixed set of accelerator adapters for transforming accelerated residuals (as discussed in ~\cref{m2r2_method}) while slow residual is transformed via expert routing mechanism. 

Furthermore, our approach integrates a Least Recently Used (LRU) caching strategy, which enhances memory efficiency by replacing the least recently used experts with speculated experts that are anticipated to be needed in upcoming layers. This hybrid approach of preemptive expert loading with LRU caching yields substantial improvements over traditional on-demand loading or standalone caching strategies. By minimizing cache misses and efficiently managing memory, this approach addresses both compute and memory bottlenecks, leading to faster, more resource-efficient token generation in MoE architectures. A comprehensive evaluation of this strategy, in relation to state-of-the-art methods, is provided in \cref{experiments_aot}, and the compute and memory traces on an A100 GPU are detailed in \cref{fig:moe_aot_cuda_trace}.



% Recent advancements in sparse Mixture-of-Experts (MoE) architectures have introduced the concept of utilizing distinct computational paths for different tokens \cite{shazeer2017outrageously}. This approach, wherein only a subset of experts are activated per input, enables MoE-based language models to generate tokens more swiftly compared to their dense counterparts due to memory-bound nature of auto-regressive decoding. In dense models, pre-loading layers in advance is a common strategy to enhance computational efficiency. However, this technique is not applicable to MoE models, where expert selection occurs dynamically based on the outputs of previous layers, preventing parallel pre-fetching of experts.

% Our proposed method addresses this inefficiency. Accelerated residuals, which are highly similar to their slower counterparts (see \cref{fig:similarity}), can reliably predict the necessary experts ahead of time. For instance, by utilizing $2X$ accelerated residual stream, we can predict the experts needed for the layer $2i+1$ and $2i+3$ while carrying out computation in layer $i$. This enables us to commence expert loading significantly earlier, as illustrated in \cref{expert_loading}, effectively mitigating the delays observed with the naive on-demand expert loading. Additionally, our method benefits from incorporating a Least Recently Used (LRU) strategy, where speculated experts replace those that are least recently utilized, resulting in improved performance compared to using either strategy alone. For a comprehensive evaluation, refer to \cref{moe_trace}, which provides a CUDA compute and memory trace of our approach executed on <>.



% A naive solution involves using the residual state of the previous layer along with the gating function of the next layer to predict which experts need to be loaded, and initiating the expert loading process in parallel with the attention computation of the next layer. Yet, as shown in \cref{fig:MOE_attn_vs_loading_time}, the attention computation for medium to long contexts is considerably faster than the expert loading time, making this approach inefficient.




\subsection{Training} \label{method_training}
% This approach is feasible due to the absence of gradient conflicts, as discussed in \cref{sec:grad_conflict}.

To accelerate residual streams, we employ parallel accelerator adapters as described in \cref{m2r2_method}.  For the early exiting use-case outlined in \cref{method_early_exiting}, we define the training objective for these adapters using the following loss function, which combines cross-entropy loss at each exit $E_j$ with distillation loss at each layer $i$. Loss weights coefficients $\alpha_0$ and $\alpha_1$ are employed to balance contribution of corresponding losses.

\begin{align} \label{eq:mr_loss}
L_{\text{m2r2}} = \underbrace{-\alpha_0 \sum_{j=1}^{J} \sum_{t=1}^{T} \log p_{\theta} \left( \hat{y}_t^{E_j} \mid y_{<t}, x \right)}_{\text{cross-entropy loss}} 
+ \underbrace{\alpha_1\sum_{i=1}^{E_{J-1}} \sum_{t=1}^{T} \| \mathbf{p}_{t}^{i} - \mathbf{h}_{t}^{((i - E_{j(i)}) \cdot R_i) + E_{j(i)})} \|^2}_{\text{distillation loss}}.
\end{align}

where $\hat{y}_t^{E_j}$ denotes the predictions from the accelerated residual stream at layer $E_j$ and time step $t$, $y_t$ represents the corresponding ground truth tokens, and $x$ indicates previous context tokens. The distillation loss at each layer $i$ is computed by comparing accelerated residuals at layer $i$ with slow residuals at layer $(i - E_{j(i)}) \cdot R_i + E_{j(i)}$, where $R_i$ denotes the rate of accelerated residuals at layer $i$ while $E_{j(i)}$ represents the most recent gate layer index such that $E_{j(i)} <= i$. \( J \) represents the total number of early exit gates, N denotes number of hidden layers and $E_j$ denotes layer index corresponding to gate index $j$ and \( T \) denotes the sequence length. 

In dynamic compute settings, after training of accelerator adapters, we optimize the query, key, and value parameters governing the ARLA routers (see ~\cref{method_arla}) across all exits in parallel on binary cross entropy loss between predicted decision and ground truth exiting decision. The ground truth labels for the router are determined based on whether the application of the final logit head on $\hat{y}_t^{E_j}$ yields the correct next-token prediction. 


% The objective for this optimization is defined by the following loss function:


%TODO are equations required ? 
% \begin{equation} \label{eq:arla_loss_combined}\small
%     L_{\text{arla}} = -\frac{1}{N} \sum_{t=1}^{T} \left( \sum_{j=1}^{E_n} \left[ O_t^{E_j} \log(\hat{O}_t^{E_j}) + (1 - O_t^{E_j}) \log(1 - \hat{O}_t^{E_j}) \right] \right), \quad \text{where} \quad 
%     O_t^{E_j} = \begin{cases} 
%     1, & \text{if } L(\hat{y}_t^{E_j}) = y_t^{E_j} \\
%     0, & \text{otherwise}
%     \end{cases}
% \end{equation}

% where $\hat{O}_t^{E_j}$ represents the binary predicted logits produced by the vertical latent attention router, as described in \cref{sec:arla}, at gate $E_j$ and time step $t$, and $O_t^{E_j}$ denotes the corresponding ground truth labels. The ground truth labels for the router are determined based on whether the application of the logit head on $\hat{y}_t^{E_j}$ yields the correct next-token prediction. The parameters controlling vertical latent attention are trained concurrently to ensure consistency and efficient use of computational resources.

For self-speculative decoding, as described in \cref{method_self_speculative_decoding}, the training objective remains the same as \cref{eq:mr_loss}, but with the number of intervals set to $J = 1$ and the rate of residual transformation set to $R_n = N/k$, where the first $k$ layers generate speculative candidate tokens. In the context of Ahead-of-Time Expert Loading for Mixture-of-Experts (MoE) models (see \cref{method_aot_expert_loading}), setting the rate of residual transformation to $R_n = 2$ typically offers a good trade-off between the accuracy of expert speculation and AoT pre-loading of experts. 

% Thus, we set $J = 1$ and $E_1 = 16$.


~\subsection{FLOPs Optimization} \label{sec:flops_optimization}

Naively implemented, M2R2 incurs higher FLOP overhead compared to traditional speculative decoding and early exiting approaches such as ~\cite{medusa, schuster2022confident, Tang2024}. However, modern accelerators demonstrate compute bandwidth that exceeds memory access bandwidth by an order of magnitude or more~\cite{databricksLLMInference2023, jouppi2021ten}, meaning increased FLOPs do not necessarily translate to increased decoding latency. Nevertheless, to ensure fair comparison and efficiency in compute bound scenarios, we introduce targeted optimizations.

~\textbf{Attention FLOPs Optimization} For medium-to-long context lengths, attention computation dominates FLOPs in the self-attention layer, surpassing the contribution from MLP layers. Specifically, matrix multiplications involving queries, cached keys, and cached values scale with $l_{kv} * l_{q}$ where $l_{kv}$ denotes previous context length and $l_q$ denotes current query length. Since M2R2 pairs accelerated residuals with slow residuals, a naive implementation results in twice the FLOPs consumption compared to a standard attention layer. To address this, we limit the attention of accelerated residual stream to selectively attend to the top-k most relevant tokens, identified by the slow residual stream based on top attention coefficients\footnote{We set to k = 64 and attend to top 64 tokens as identified by the slow residual stream.}. This is possible since slow and accelerated residual streams are processed in same forward pass and accelerated streams have access to attention coefficients of slow stream. Note that, the faster residual stream still retains the flexibility to assign distinct attention coefficients to these tokens. Furthermore, we design the faster residual stream to employ only 8 attention heads, compared to the 32 heads used in the slow residual stream of the Phi-3 model, reducing query, key, value, and output projection FLOPs by a factor of 1/4. ~\cref{fig:m2r2_num_heads_ablation} indicates effect of using a slicker stream on alignment. As depicted, using $\hat{n}_h = 8$ offers a good trade-off between alignment and FLOPs overhead. 

~\textbf{MLP FLOPs Optimization} The accelerator adapters operating on the accelerated residual stream are intentionally designed with lower rank than their counterparts in the base model. This reduces FLOP overhead by a factor proportional to $hiddenSize / rank$. Additionally, since the faster residual stream uses only 8 attention heads (compared to 32 in the slow residual stream of Phi-3), the subsequent MLP layers process a smaller set of activations, further reducing FLOPs by another factor of 1/4.

These optimizations significantly reduce the FLOP overhead per speculative draft generation, as illustrated in ~\cref{fig:flops_optmization}. Notably, while traditional early-exiting speculative approaches such as DEED require propagating the full slow residual state through the initial layers, incurring substantial computational costs, M2R2 achieves efficient token generation via slimmer, low-rank faster residual streams. In contrast, Medusa introduces considerable FLOP overhead due to per-head computations scaling with $d^2+dv$\footnote{Here $d$ denotes hidden state dimension while $v$ denotes vocab size.}, whereas M2R2 employs low-rank layers for both MLP and language modeling heads, maintaining computational efficiency. All experiments involving the M2R2 approach, as detailed in ~\cref{sec:experiments}, are conducted using these FLOPs optimizations.









% \[
% O_t^{E_j} = 
% \begin{cases} 
% 1, & \text{if } L(\hat{y}_t^{E_j}) = y_t^{E_j} \\
% 0, & \text{otherwise}
% \end{cases}
% \]




%add distillation
% We train accelerator adapters described in \cref{m2r2_method} to accelerate residual streams on next token prediction all in parallel since there are no gradient conflict issues as described in \cref{sec:grad_conflict}.

% \begin{align} \label{eq:mr_loss}
% L_{mr} =  & -\sum_{j = 1}^{E_n} (\sum_{t=1}^{T}\log p_{\theta} (\hat{y}_t^{E_j} | \hat{y}_{<t}, x)) \nonumber
% \end{align}

% where $\hat{y_t^{E_j}}$ denotes predicted logits obtained from accelerated residual stream at gate $E_j$ and time-step $t$ while $y_t^{E_j}$ denotes corresponding truth tokens. 

% Upon training of adapters responsible for accelerating residual streams, we train query, key, value parameters responsible for vertical latent attention of all gates in parallel as

% \begin{equation} \label{eq:arla_loss}
%     L_{arla} = -\frac{1}{N} (\sum_{t=1}^{T}(1\sum_{j=1}^{E_n} \left[ O_t^{E_j} \log(\hat{O}_t^{E_j}) + (1 - o_t^{E_j}) \log(1 - \hat{o_t}_{E_j}) \right]))
% \end{equation}

% where $\hat{O_t^{E_j}}$ denotes binary predicted logits obtained from vertical latent attention router described in \cref{sec:arla} at gate $E_j$ and timestep $t$ while $O_t^{E_j}$ denotes corresponding truth label. Truth labels for router are obtained by computing whether logit head application on $\hat{y}_t^j$ results in true next token prediction. Formally speaking, 

% $O_t^{E_j} = 1 if L(\hat{y_t^{E_j}}) == y_t^{E_j} , 0 otherwise$. 

% Parameters responsible for vertical latent attention are also trained in parallel as well. 

%todo: training slow and fast residuals together and distillation can be two training mdoes. 
%Distillation can be an ablation. 




% Although transformer decoding is memory bound on most mainstream accelerators, there could be scenarios where flop savings are crucial. For instance, on on-device settings power consumption is directly correlated with flops per decoding step and reducing flops does help with overall energy consumption. Vanilla early exiting methods help with flop reduction but suffer from mismatch between training and inference due to early exited tokens. If token at decoding step $t$, $T_t$ exited at layer $E_i$, while token $T_{t+k}$ exits at layer $E_j$ such that $E_i < E_j$, hidden state $H_{t+k}l$ does not have corresponding hidden state $H_tl$ to attend to where $E_i < l <= E_j$. One solution that's often used in literature is to rely on last hidden state available, $H_t{E_j}$, however it tends to be sub-optimal and does affect generation quality \cite{ref}.  To alleviate this mismatch while reducing flops, we train router such that attention mask between token $T_{t+k}$ and token $T_{<t+k}$ is given by: 

% \begin{equation}
%     a_{T_{{t+k}{T_{<t+k}}} = 1 if  E_{T_{<t+k}} >= E{T_{t+k}}
%     else 0
% \end{equation}

% This attention mask enables router to account for exited tokens and get trained accordingly. Since attention mechanism during decoding remains exactly same as that during training, impact on generation quality tends to be minimal as noted in \cref{fig:gen_auality_with_and_without_recompute_attention_show_flops}.  Although MoD does not suffer from training and inference mismatch, we observe that it suffers from discountinuity between pre-training and super-vised fine-tuning resulting in sub-optimal perplexity. On the other hand, our method doesn't not require pre-training , doesn't suffer from discountinuity, and achieves much better perplexity in super-vised fine-tuning and instruction tuning setups as shown in \cref{fig:Mod_vs_m2r2_loss_curves}.






% Our techniques are directly applicable in such scenarios.    




%expert loading with cuda streams in experiments

\clearpage
\begin{table}[ht!]
\centering
\caption{\textbf{Super Resolution Performance Results.} Our proposed WGAN EEG Spatial Upsampling method significantly outperforms a baseline of Bicubic Interpolation commonly used in EEG upsampling pipelines.}
\label{tab:results}
\resizebox{0.8\linewidth}{!}{%
\begin{tabular}{@{}cccccc@{}}
\toprule
\multirow{2}{*}{\textbf{Dataset}} & \multirow{2}{*}{\textbf{Scale}} & \multicolumn{2}{c}{\textbf{Bicubic}} & \multicolumn{2}{c}{\textbf{WGAN}} \\ \cmidrule(l){3-6} 
                      &   & \textbf{MSE} & \textbf{MAE} & \textbf{MSE}    & \textbf{MAE}   \\
\toprule
\multirow{2}{*}{Val}  & 2 & 3.71E7       & 3.89E3       & \textbf{2.01E3} & \textbf{24.38} \\
                      & 4 & 7.23E7       & 6.42E3       & \textbf{8.53E3} & \textbf{63.83} \\
\midrule
\multirow{2}{*}{Test} & 2 & 3.75E7       & 3.91E3       & \textbf{2.06E3} & \textbf{24.66} \\
                      & 4 & 7.30E7       & 6.45E3       & \textbf{8.68E3} & \textbf{64.39} \\
\bottomrule
\end{tabular}%
}
\end{table}

\section{Limitations}
\label{sec:limitations}
We are encouraged by the early promise of the AI co-scientist evaluations, which highlight its potential to augment scientific research.
However, the system has several limitations. Responsible innovation necessitates thoughtful consideration of these alongside the potential impacts to researchers and scientific research. 

\paragraph{Limitations with literature search, reviews and reasoning.} The reviews undertaken by the AI co-scientist system may miss critical prior works due to reliance on open-access literature. In the presented work, the AI co-scientist does not access the entire published literature due to compliance with license or access restrictions where applicable. The system may also omit consideration of prior work on occasions where it has incorrectly reasoned that the work is not relevant.

\paragraph{Lack of access to negative results data.} The AI co-scientist system's use of only open published literature means it likely has limited access to negative experimental results or records of failed experiments. It is known that such data may be more rarely published than positive results, yet experienced scientists working in the field may nonetheless possess and utilize this knowledge to prioritize research~\citep{brazil2024illuminating}. Strategies to overcome this phenomenon might further improve the performance of the co-scientist as a tool for scientific discovery.

\paragraph{Improved multimodal reasoning and tool-use capabilities.} Some of the most interesting data in scientific publications is not written in text but may be encoded visually in figures and charts. However, even state-of-the-art frontier models may not comprehensively utilize such data with optimal  reasoning~\citep{roberts2024scifibench} and the AI co-scientist system is unlikely to be an exception. Stronger benchmarks and evaluations are necessary to improve these capabilities. We have also not evaluated the ability of our system to reason over and integrate information from domain-specific biomedical multimodal datasets (such as large multi-omics datasets) and knowledge graphs. More work is needed to integrate the AI co-scientist system with specialized scientific tools, AI models and databases, and evaluate the ability to utilize them effectively.

\paragraph{Inherited limitations of frontier LLMs.} LLM limitations include imperfect factuality and hallucinations, which may be propagated in the co-scientist system. The system's reliance on existing LLMs and web-search, while providing immediate access to broad knowledge, may propagate errors of factuality, biases or limitations present in those resources.

\paragraph{Need for better metrics and broader evaluations.} While the current AI co-scientist evaluation includes AI auto-ratings, expert reviews and targeted \textit{in vitro} validations, the evaluation of system performance remains preliminary. A comprehensive, systematic evaluation across diverse biomedical and scientific disciplines is necessary to determine the generalizability of co-scientist. Furthermore, the system requires continued improvement to produce outputs that meet the rigor and detail of high-quality publications. Furthermore, the Elo rating implemented to help the system self-improve for hypothesis generation is a limited auto-evaluation metric. Continued investigation into alternative, more objective, less intrinsically-favored, evaluation metrics that better represent perspectives and preferences from expert scientists could strengthen future work.

\paragraph{Limitations of existing validations.} At present, the AI co-scientist focuses on identifying potential therapeutic targets and mechanisms, but many not be addressing the complexities of drug delivery systems. Pharmaceutical factors such as tissue-specific targeting, formulation requirements, and delivery efficiency—while critical for clinical translation—remain beyond the scope of the present system.

The AI co-scientist is currently also not designed to generate comprehensive clinical trial designs or to fully account for factors such as drug bioavailability, pharmacokinetics, and any complex drug interactions when applied for drug repurposing or discovery. These aspects require much deeper understanding, extensive expertise, and appropriate data beyond the scope of the current system. A dedicated translational scientific team is needed for onward clinical translation of the predictions. These limitations also highlights the need for continued development and integration of the system with more tools, such as specialized AI models and real-time databases.


\section{Safety and Ethical Implications}
\label{sec:safetyethics}
While AI systems such as the co-scientist offers the potential to accelerate scientific discovery, it also poses significant safety and ethical challenges, distinct from its impact on the scientific method itself. Safety risks center on the dual-use and the possibility that scientific breakthroughs could be exploited for harmful purposes. Ethical risks, conversely, involve research that contradicts established ethical norms and conventions within specific scientific disciplines. We review these distinct risk categories, emphasizing that further research is crucial to fully understand and mitigate them.

\paragraph{Evolving ethics frameworks, policy and regulations for advanced AI use in scientific endeavors.}
Research ethics is a central aspect of scientific endeavor and a prominent research field in its own right~\citep{shrader1994ethics, resnik2005ethics, rollin2006science, fisher2008research, edel2018science, menapace2019scientific}.  A key focus is directing research towards positive societal impact, although questions remain about potentially dual-use knowledge~\citep{miller2007ethical, selgelid2009governance, pustovit2010philosophical, forge2010note, kuhlau2013ethics}.

Core ethics principles are being complemented by emerging regulation, and formal processes involving organizational ethics reviews that are meant to provide an assessment of adherence to the code of conduct, as well as an assessment of present and future risks involved with research proposals~\citep{shaw2006research, rothstein2006risks, ludlow2015regulating, verschraegen2018regulating}.

The acceleration of science through AI, especially with advanced agentic AI systems, requires advances in science and AI ethics policy and regulation~\citep{jobin2019global, wansley2016regulation}. This adaptation is crucial to address the changing research landscape and the unique risks associated with AI agents of varying capabilities and autonomy.

Advancements in AI systems, like the co-scientist, require moving beyond the limited ethical considerations designed for earlier, specialized AI models with restricted application and action spaces~\citep{gabriel2024ethics}. Some preliminary frameworks have developed to understand the impact of LLM agents in science, specifically mapping risks across user intent, domain, and broader impact~\citep{tang2024prioritizingsafeguardingautonomyrisks}.

\paragraph{Dual-use risks and technical safeguards.}
Beyond the scientific domain, broad frameworks are being developed for evaluating the emergence of potentially dangerous capabilities in AI agents~\citep{shevlane2023model, bova2024quantifying, phuong2024evaluating}. These frameworks assess capabilities related to persuasion, deception, cybersecurity, self-proliferation, and self-reasoning. As AI agents advance, safety evaluations in science must integrate these broader assessments. A long-term risk is that agentic systems could develop intrinsic goals influencing research directions. Human susceptibility to AI manipulation, already observed in other contexts~\citep{sabour2025humandecisionmakingsusceptibleaidriven}, underscores the need for robust frameworks ensuring instruction-following and value alignment.

More immediately on a shorter time-scale, technical safeguards are needed to address unethical research queries, malicious user intent, and the potential for extracting dangerous or dual-use knowledge from scientific AI systems. Because verification is computationally `easier' than generation, significant research focuses on using advanced LLMs as `critics' or `judges' to evaluate both user queries and AI outputs acting as a scalable oversight mechanism. These critics operate based on predefined criteria, provided through direct instructions, examples (few-shot or many-shot prompting), or fine-tuning~\citep{ke2023critiquellm, vu2024foundational, wei2024systematic, lan2024criticbench, zheng2024judging, gu2024survey}. They can also leverage external tools for grounding~\citep{gou2023critic} and have shown promise in multimodal scenarios~\citep{chen2024mllm}. However, limitations remain; human expert involvement is crucial, as LLMs may not align with human judgment in specialized domains~\citep{szymanski2024limitationsllmasajudgeapproachevaluating}.

\paragraph{Adversarial robustness of scientific AI systems.}
Recognizing and mitigating adversarial attacks is a crucial, ongoing research area in the development of foundation models and advanced AI assistants~\citep{shayegani2023survey, he2023large, zhu2023promptbench, fu2023misusing, zhang2023defending, chao2024jailbreakbench, zhao2024evaluating, ma2025safetyscalecomprehensivesurvey}. While manual red teaming has identified vulnerabilities, automated approaches now allow for optimizing prompt suffixes to bypass safety measures, using techniques like greedy, gradient-based, or evolutionary methods~\citep{zou2023universal, lapid2023open}. Attacks can also exploit few-shot demonstrations, in-context learning~\citep{wang2023adversarial, qiang2023hijacking}, and multimodal inputs~\citep{qi2023visual}. Furthermore, LLMs can be used to generate and refine attacks against other LLMs~\citep{chao2023jailbreaking}, and attacks can be iterative, spanning multiple steps~\citep{wang2024footdoorunderstandinglarge}. Defenses are being developed to counter both human and automated attacks, which is increasingly important in an agentic AI future~\citep{zhang2024adversarial}.

Advances in post-training of base models will likely improve overall adversarial robustness. However, domain-specific recognition of malicious use may still require dedicated development and integration into scientific AI assistants. In AI systems employing iterative reasoning (e.g., request interpretation, hypothesis generation, internal thoughts, evaluation, user queries), each component must be tested independently.  This comprehensive testing should account for all potential failure modes, including the handling of unsafe queries, the safety of hypotheses (intermediate and final), and the accuracy of internal checks and filters.

\paragraph{Need for a comprehensive safety approach.}
Scientific AI assistants, like the co-scientist, require integrated, configurable guidelines within their safeguards. Developers should anticipate the complexity of this challenge and prioritize flexible safeguarding to rapidly incorporate community feedback. These semantic safeguards may need to be augmented by traditional software safety measures, including trusted testers, gradual feature rollouts, access controls, request logging, and flagging uncertain outputs for manual review.

Ensuring the safety of these systems, in line with existing AI safety guidelines~\citep{shneiderman2020bridging, anthropicscalingpolicy}, necessitates a multi-pronged approach. This includes:
\begin{itemize}
    \item Comprehensive threat modeling to identify potential vulnerabilities.
    \item Defense mechanisms for each identified threat.
    \item Extensive red-teaming and security testing.
    \item Rapid response procedures for quick resolution of issues including vulnerability patches.
    \item Continuous monitoring and performance tracking.
\end{itemize}

These considerations highlight the need for responsible development, governance and careful deployment of technologies designed for advancing science, appropriate safeguards and ethical guidelines and close compliance with applicable regulations. They also further underscore the need for broad community engagement and an inclusive development of best practices and recommendations around safe and ethical use for AI in science.

\paragraph{Current safeguards in the AI co-scientist.}
To mitigate these risks, the AI co-scientist currently employs the following safety mechanisms:
\begin{itemize}[leftmargin=1.5em,rightmargin=0em]
    \item\textbf{Reliance on public frontier LLMs.} The system utilizes established public Gemini 2.0 models, which already incorporate extensive safety evaluation and safeguards.
    \item\textbf{Initial research goal safety review.} Upon input, each research goal undergoes automated safety evaluation. Goals deemed potentially unsafe are rejected.
    \item\textbf{Research hypothesis safety review.} Generated hypotheses are reviewed for safety, even when the overarching research goal is deemed safe. Potentially unsafe hypotheses are excluded from the tournament, not developed any further, and are not presented to the user.
    \item\textbf{Continuous monitoring of research directions.} A meta-review agent provides an overview of research directions, enabling the AI co-scientist to continuously monitor for potential safety concerns and alert users if a research direction is detected as being potentially unsafe.
    \item\textbf{Explainability and transparency.} All system components, including the safety review, provide not only the final recommendation but also a detailed reasoning trace that can be used to justify and audit system decisions.
    \item\textbf{Comprehensive logging.} All system activities are logged and stored for future analysis and auditing.
    \item\textbf{Safety evaluations and red teaming.} A preliminary red teaming effort has been undertaken to ensure that the current implementation of unsafe research goal detection is robust and accurate. This evaluation includes an assessment of the system behavior when presented with 1,200 adversarial research goals across 40 distinct topic areas as discussed in \cref{sec:result_safety}.
    \item\textbf{Trusted tester program.} We are enthused by the early promise of the AI co-scientist system and believe it is important to more rigorously understand its strengths and limitations in many more areas of science and biomedicine; alongside making the system available to many more researchers who it is intended to support and assist. To facilitate this responsibly and with rigour, we will be enabling access to the system for scientists through a Trusted Tester Program to gather real-world feedback on the utility and robustness of the system.
\end{itemize}
Crucially, the AI co-scientist is designed to operate with continuous human expert oversight, ensuring that final decisions are always made by scientists exercising their expert judgment.


\vspace{-0.1cm}
\section{Future Work}
\vspace{-0.1cm}
\label{sec:future work}

\paragraph{Immediate improvements.}
The AI co-scientist is in its early development, with many opportunities for improvement. Immediate improvement opportunities include enhanced literature reviews, cross-checks with external tools, improved factuality checking, and increased citation recall to minimize missed relevant research. Coherence checks would also improve the system by reducing the burden of reviewing flawed hypotheses.

\paragraph{Expanded evaluations.}
Developing more objective evaluation metrics, potentially incorporating automated literature-based validation and simulated experiments, is a key area. Methods to mitigate biases or error patterns inherited from the base LLMs are also important, alongside analysis of the complementarity and optimal combination of different agent components.

A critical need is a larger-scale evaluation involving more subject matter experts with diverse, high-resolution research goals. Stress-testing the system at every level of resolution (from disease mechanisms to protein design, and expanding to other scientific disciplines) will reveal further areas for improvement. Finally, since laboratory resources are limited, improved evaluation frameworks could assist with hypothesis selection.

\paragraph{Capabilities advancements.}
Several opportunities remain to expand co-scientist's capabilities. Reinforcement learning could enhance hypothesis ranking, proposal generation, and evolutionary refinement.

Currently, the system assesses text from open-access publications but not images, data sets, or major public databases. Integrating these publicly available resources would significantly enhance the co-scientist's ability to generate and justify proposed hypotheses.

Future work will focus on handling more complex experimental designs, such as multi-step experiments and those involving conditional logic. Integrating co-scientist with laboratory automation systems could potentially create a closed-loop for validation and a grounded basis for iterative improvement. Exploring more structured user interfaces for providing feedback and insights from targeted user research studies, beyond free text, could improve the efficiency of human-AI collaboration in this paradigm.


\vspace{-0.1cm}
\section{Discussion}
\vspace{-0.1cm}
\label{sec:discussion}
Our study represents an initial foray into accelerating novel scientific discovery with agentic AI systems and here, we discuss some of the broader implications. The co-scientist iteratively refines its generated hypotheses through a generate, debate, evolve'' approach with specialized agents under the hood. This design creates a self-improving cycle for research hypothesis generation, as measured by automated evaluation metrics, and showcases the potential benefits of test-time compute scaling for scientific reasoning.

Instead of brute-force generation of a vast number of hypotheses and relying on volume to chance into a few potentially useful ones, the system is designed to mimic key aspects of the scientific reasoning method in an intelligent manner. As detailed in~\cref{sec:methods}, the co-scientist employs principled internal mechanisms, including scientific debates, tournaments, iterative refinement, and human feedback loops to progressively improve the quality of its proposals, and converge on high quality and well-reasoned hypotheses.

\paragraph{AI co-scientist novelty is grounded in prior evidence.} The AI co-scientist facilitates the generation of novel scientific hypotheses and uncovering new insights by synthesizing extensive literature and identifying latent relationships. While its primary utility in its current form may lie in enabling more incremental advancements — such as the computational repurposing of existing therapeutics — it may also be able to support exploratory, breakthrough research. When researchers define such open-ended research goals requiring complex and out-of-the box thinking, the system may produce outputs of varying confidence. Therefore, rigorous validation and critical appraisal by domain experts remain paramount. This system is intended to augment, not supplant, human scientific reasoning, empowering researchers to accelerate discovery while maintaining intellectual control over the generated insights. We further expand on the novelty aspects in the specific context of the applications considered in this work in~\cref{sec:glossary}.

\paragraph{Multiple experimental validations of novel co-scientist hypotheses.} Notably, this work demonstrates the validation of co-scientist hypotheses via experimental findings in multiple laboratories. In drug repurposing, co-scientist identifies novel candidates for AML that demonstrated \textit{in vitro} efficacy at clinically relevant concentrations, including the identification of new repurposing opportunities beyond current preclinical knowledge. For liver fibrosis, the co-scientist proposes new epigenetic treatment targets, with subsequent \textit{in vitro} experiments validating the anti-fibrotic activity of several suggested compounds, including an FDA-approved drug. In the realm of antimicrobial resistance, the co-scientist independently recapitulates a novel, unpublished finding regarding the mechanism of cf-PICI transfer between bacterial species. Early results over several queries of varying scientific complexity suggests the co-scientist has a potential to contribute to discovery within various biomedical domains.

\paragraph{Test-time compute scaling with scientific reasoning priors and inductive biases.} In the experiments reported here, the co-scientist did not require specialized pre-training, post-training, or a reinforcement learning framework. It leverages the capabilities of existing base LLMs, potentially benefiting from updates to those models without requiring retraining of the co-scientist system itself, which presents advantages of compute efficiency and generalizability. The system's architecture incorporates self-play, internal consistency checks, and tournament-based ranking, which support iterative hypothesis generation, evaluation, and refinement. This is reflected in the observed improvement in hypothesis quality over time. This self-evolution can be improved further by expanded tool use integration, including database queries, allowing the co-scientist to ground its proposals in existing knowledge and identify novel connections. In the future, we may leverage data and tournament ranking generated by the co-scientist itself as feedback to improve the whole system using reinforcement learning.

\paragraph{Frontier LLM advancements and the AI co-scientist.}
The frontier LLMs used within the co-scientist system have demonstrated a continuing trend of rapidly improved capabilities, including reasoning, logic, and also some aspects of scientific literature comprehension. As our system is designed to be model-agnostic, we hypothesize that further improvements in frontier LLMs will also result in improved co-scientist performance, and enable new avenues of research including optimal agentic use of tools.

\paragraph{Implications for drug repurposing and discovery.}
These advancements have significant implications for various biomedical and scientific domains. For example, the integration of the co-scientist into the drug candidate selection process represents a significant advancement in evidence-based drug repurposing. Beyond simple literature mining, the co-scientist maybe capable of synthesizing novel mechanistic insights by connecting molecular pathways, existing preclinical evidence, and potential therapeutic applications into structured, testable specific-aims. This capability is particularly valuable as it provides researchers with literature-supported rationales and suggests specific experimental approaches for validation. Notably, the co-scientist's structured output can be leveraged to develop comprehensive single-patient IND (Investigational New Drug) applications for compassionate use cases. By systematically presenting mechanistic evidence, relevant preclinical data, and proposed monitoring parameters, the co-scientist facilitates the development of well-reasoned treatment protocols for patients with refractory (treatment-resistant) disease who have exhausted standard therapeutic options and are ineligible for clinical trials. This application is particularly valuable in rare or aggressive diseases where traditional drug development timelines may not align with urgent patient needs. The platform's ability to rapidly generate evidence-based therapeutic hypotheses, complete with safety considerations and monitoring parameters, can help clinicians and regulatory bodies make informed decisions about compassionate use applications while maintaining scientific rigor.

The application of the co-scientist in drug repurposing presents a very compelling opportunity for orphan drugs, where extensive safety and clinical data already exist from their original rare disease indications. Given that Phase III clinical trials can cost hundreds of millions of dollars, repurposing these well-characterized therapeutics offers an efficient path to expanding treatment options across multiple diseases. This is especially relevant as orphan drugs often target fundamental biological pathways that may be relevant in other conditions, but these connections might not be immediately apparent through traditional research approaches. By systematically evaluating existing clinical data, safety outcomes, and mechanistic insights, the co-scientist can help identify promising new therapeutic applications while taking advantage of the investment already made in drug development and safety validation. This approach not only maximizes the utility of existing therapeutics but also provides a more rapid path to addressing unmet medical needs across a broader patient population.

More broadly, the co-scientist may also be potentially impactful throughout the entire drug discovery spectrum as evidenced by the early work on co-scientist assisted target discovery for liver fibrosis. 

\paragraph{Automation bias and impact on human scientific creativity.} 
Realizing the full potential of AI in biomedicine and science requires proactively addressing potential pitfalls. Over-reliance on AI-generated suggestions in collaborative AI systems could diminish critical thinking and increase homogeneity in research. Studies on AI's impact on creativity and ideation show mixed results; some suggest a risk of homogenization of ideas across populations~\citep{homogeneity_ideas}, while others are less conclusive~\citep{ashkinaze2024aiideasaffectcreativity}. The correlated success / failure modes of LLMs~\citep{wenger2025weredifferentweresame}, due to similar training data, could also artificially narrow scientific inquiry. Furthermore, AI system blind spots and performance variations across research domains must be considered. Therefore, scalable factuality and verification methods, alongside peer review and careful consideration of potential biases, are essential. Careful design and use of systems like the co-scientist are crucial to mitigate these risks.

\paragraph{AI as a catalyst for both scientific discovery and equity.}
Despite these risks, AI holds immense potential to democratize access to scientific information and accelerate discovery, particularly benefiting historically neglected and resource-constrained areas~\citep{sun2018addressing, george2023addressing}. In essence, AI can ``raise the tide'' of scientific progress, lifting all boats, especially those that have historically been left behind. Realizing this potential requires strategic investments and careful calibration of AI systems to foster ideation and innovation while minimizing false positives. This includes focusing on historically neglected research topics and addressing variations in performance across different scientific domains with varying amounts of pre-existing data. While current AI systems may tend to produce incremental ideas and research hypotheses, ongoing development aims to create systems capable of generating truly original, unorthodox and transformative scientific theories. Proactive mitigation of these challenges will ensure that AI serves as a powerful tool for all scientists, promoting a more equitable and innovative future for scientific explorations.

\clearpage

\section{Conclusion}
The AI co-scientist represents a promising step towards AI-assisted augmentation of scientists and acceleration of scientific discovery. Its ability to generate novel testable hypotheses across diverse scientific and biomedical domains, some supported by experimental findings, along with the capacity for recursive self-improvement with increasing compute, demonstrates the promise of meaningfully accelerating scientists’ endeavours to resolve grand challenges in human health, medicine and science. This innovation opens numerous questions and opportunities. Applying the empiric and responsible approach of science to the AI co-scientist system itself can thereby enable safe exploration of its undoubted potential, including how collaborative and human-centred AI systems might be able to augment human ingenuity and accelerate scientific discovery.

\vspace{12pt}
\subsubsection*{Acknowledgments}
We thank our teammates Subhashini Venugopalan, John Platt, Erica Brand, and Yun Liu for their detailed technical feedback on the manuscript. We thank Jakob T Rostoel, Cora Chmielowska and Jonasz B Patkowski from Imperial College London and Jakkapong Inchai, Weida Liu, and Wenlong Ren from Stanford University for providing expert feedback on the AI system introduced in this work, and the lab of Ravi Majeti from Stanford University for generously providing the AML cell lines used in this work. We thank Ritu Raman, Ryan Flynn, Charlie Hempstead, Lord Ara Darzi, Omar Abudayyeh, Jonathan Gootenberg, Nic Fishman, Jason Lequyer, Dan Leesman, Ravi Solanki, Dennis Gong and Ananthan Sadagopan for feedback on different aspects of the AI system and the work. We also thank Maen Abdelrahim, Ethan Burns, Preethi Prasad and Hanh Mai for their clinical expertise and expert evaluation.

We thank our teammates Thomas Wagner, Alessio Orlandi, Natasha Latysheva, Nir Kerem, Yaniv Carmel, Hussein Hassan Harrirou, Laurynas Tamulevičius and Grzegorz Glowaty for their technical support. We thank Taylor Goddu, Resham Parikh, Siyi Kou, Rachelle Sico, Amanda Ferber, Cat Kozlowski, Alison Lentz, KK Walker, Roma Ruparel, Jenn Sturgeon, Lauren Winer, Juanita Bawagan, Ed-Allt Graham, Tori Milner, MK Blake, Jack Mason, Erika Radhansson, Indranil Ghosh, Jay Nayar, Brian Cappy, Celeste Grade, Abi Jones, Laura Vardoulakis, Lizzie Dorfman, Ashmi Chakraborthy, Delia Williams-Falokun, Maggie Shiels, Kalyan Pamarthy, Sarah Brown,  Christian Wright, and S. Sara Mahdavi for their support and guidance during the course of this project. Finally, we thank Michael Brenner, Zoubin Ghahramani, Dale Webster, Joelle Barral, Michael Howell, Susan Thomas, Karen DeSalvo, Jason Freidenfelds, Ronit Levavi Morad, Vladimir Vuskovic, Ali Eslami, Anna Koivuniemi, Greg Corrado, Royal Hansen, Andy Berndt, Noam Shazeer, Oriol Vinyals, Koray Kavukcuoglu, James Manyika, Jeff Dean and Demis Hassabis for their support of this work.


\vspace{12pt}

\newpage
\setlength\bibitemsep{3pt}
\printbibliography
\balance
\clearpage
