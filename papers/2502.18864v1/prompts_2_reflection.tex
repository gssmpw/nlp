\subsection{Prompt for the Reflection agent}
\label{sec:prompts_2_reflection}


\begin{figure}[htbp!]
\begin{tcolorbox}[
    colback=black!5!white,
    colframe=black!60!white,
    title=\textbf{Prompt for generating observations which can be explained by the hypothesis },
    fonttitle=\bfseries,
    arc=3mm,
    boxrule=1pt,
    bottomrule=2pt,
]
\footnotesize
\begin{verbatim}
You are an expert in scientific hypothesis evaluation. Your task is to analyze the 
relationship between a provided hypothesis and observations from a scientific article.
Specifically, determine if the hypothesis provides a novel causal explanation 
for the observations, or if they contradict it.

Instructions:

1.  Observation extraction: list relevant observations from the article.
2.  Causal analysis (individual): for each observation:
    a.  State if its cause is already established.
    b.  Assess if the hypothesis could be a causal factor (hypothesis => observation).
    c.  Start with: "would we see this observation if the hypothesis was true:".
    d.  Explain if it's a novel explanation. If not, or if a better explanation exists, 
        state: "not a missing piece."
3.  Causal analysis (summary): determine if the hypothesis offers a novel explanation 
    for a subset of observations. Include reasoning. Start with: "would we see some of
    the observations if the hypothesis was true:".
4.  Disproof analysis: determine if any observations contradict the hypothesis. 
    Start with: "does some observations disprove the hypothesis:".
5.  Conclusion: state: "hypothesis: <already explained, other explanations more likely, 
    missing piece, neutral, or disproved>".

Scoring:
    *   Already explained: hypothesis consistent, but causes are known. No novel explanation.
    *   Other explanations more likely: hypothesis *could* explain, but better explanations exist.
    *   Missing piece: hypothesis offers a novel, plausible explanation.
    *   Neutral: hypothesis neither explains nor is contradicted.
    *   Disproved: observations contradict the hypothesis.

Important: if observations are expected regardless of the hypothesis, and don't disprove it, 
it's neutral.

Article:
{article}

Hypothesis:
{hypothesis}

Response {provide reasoning. end with: "hypothesis: <already explained, other explanations 
more likely, missing piece, neutral, or disproved>".)
\end{verbatim}
\end{tcolorbox}
\vspace{0.1cm}
\caption{\textbf{Example Reflection agent prompt for generating observations from prior experimental results which can be explained by the hypothesis under consideration.}}
\label{fig:GENERATE_OBSERVATIONS_PROMPT}
\end{figure}
