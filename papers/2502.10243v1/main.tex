\documentclass{ieeeaccess} 

% --------------------------------------------------------------------
% Packages
% --------------------------------------------------------------------


\usepackage{amssymb,amsmath,amsfonts}
\usepackage{algorithmic}
\usepackage{graphicx}
\usepackage{hyperref} 

\usepackage{textcomp}

\usepackage{verbatim}
\usepackage[style=ieee,
bibstyle=ieee,
%citestyle=numeric-comp,
backend=biber,
minnames=1,
maxcitenames=2,
maxbibnames=6,
doi=true,
isbn=false,
url=false,
%natbib=true,
%clearlang=true,
date=year,
]{biblatex} 

\AtEveryBibitem{\clearlist{language}}
\addbibresource{bibliography.bib}

\input{etc/stolte_abbrev}
\usepackage{xpatch}
\usepackage{xstring}

% For correct arXiv bibliography entries, just add a ``tex.referencetype=unpublished'' in Zotero's ``extra'' field, maybe remove unncessary information in Zotero

% Automatically indicates if publication is in German, Chinese, or Japanes, requires language field in Zotero set to ``de'', zh, ja, or ko
\DeclareSourcemap{
	\maps[datatype=bibtex]{
		\map{
			\step[fieldsource=langid, match=\regexp{\A(n)?((swiss)?german|austrian)\Z}, final]
			\step[fieldset=language, fieldvalue={(in German)}]
		}
		\map{
			\step[fieldsource=langid, match=pinyin, final]
			\step[fieldset=language, fieldvalue={(in Chinese)}]
		}
		\map{
			\step[fieldsource=langid, match=japanese, final]
			\step[fieldset=language, fieldvalue={(in Japanese)}]
		}
		\map{
			\step[fieldsource=langid, match=ko, final]
			\step[fieldset=language, fieldvalue={(in Korean)}]
		}
	}
}

% removes year in @inproceedings, if already contained in proceedings name
%\xpatchbibmacro{publisher+location+date}
%	{\usebibmacro{date}}
%		{	\ifentrytype{inproceedings}{
%				\IfSubStr{\strfield{booktitle}}{\strfield{year}}{}{\usebibmacro{date}} % do not print year if already contained in the standards number
%			}
%		}
%	{}
%	{\typeout{There was an error patching biblatex-ieee (specifically, ieee.bbx's @inproceedings driver)}
%	}


% removes year in @inproceedings, if already contained in proceedings name
\xpatchbibdriver{inproceedings}					 	% driver to be patched
	{\usebibmacro{publisher+location+date}}			% part to be patched
	{\IfSubStr{\strfield{booktitle}}{\strfield{year}}{\usebibmacro{publisher+location}}{\usebibmacro{publisher+location+date}}} % how to patch
	{} % If success
	{\typeout{There was an error patching biblatex-ieee (specifically, ieee.bbx's @inproceedings driver)}} % If no success

\newbibmacro*{publisher+location}{%
	\printlist{location}%
	\iflistundef{publisher}
	{\setunit*{\addcomma\space}}
	{\setunit*{\addcolon\space}}%
	\printlist{publisher}%
	\newunit}

% Enables fixing empty year field in URL citation: https://tex.stackexchange.com/questions/151217/remove-parentheses-for-empty-year-field-biblatex-ieee-style
\xpatchbibdriver{online}
{\printtext[parens]{\usebibmacro{date}}}
	{\iffieldundef{year}
		{}
		{\printtext[parens]{\usebibmacro{date}}}}
	{}
	{\typeout{There was an error patching biblatex-ieee (specifically, ieee.bbx's @online driver)}}

% Does also something
\DeclareSourcemap{
	\maps[datatype=biber]{
		\map{
			\step[fieldsource=note, final]
			\step[fieldset=addendum, origfieldval, final]
			\step[fieldset=note, null]
		}
	}
}


% Print @standard entries according to IEEE guidelines, requires zotero entrytype ``report'' as well as the ``extra'' field filled with ``tex.referencetype=standard''
\DeclareBibliographyDriver{standard}{%
	\usebibmacro{bibindex}%
	\usebibmacro{begentry}%
	%	\usebibmacro{author}%
	%	\setunit{\addcomma\addspace}%
	\usebibmacro{maintitle+title}%
	%	\printfield{version}%
	\setunit{\addcomma\addspace}%
	\usebibmacro{publisher+type+number}%
	\setunit{\labelnamepunct}\newblock
	\IfSubStr{\strfield{number}}{\strfield{year}}{}{\usebibmacro{date}} % do not print year if already contained in the standards number
%	\usebibmacro{date}%
	%	\newunit\newblock
	%	\usebibmacro{url+urldate}%
	\newunit\newblock
	\usebibmacro{finentry}%
}

\newbibmacro*{publisher+type+number}{%
	\printtext{%
		\printlist{publisher}
		\iflistundef{publisher}
		{\space}{}%
		\iffieldundef{type}
		{Standard}
		{\printfield{type}}
		\printfield{number}
	}%	
}

% Print date in report only if year is not included in number
\xpatchbibdriver{report}					 	% driver to be patched
	{\usebibmacro{date}}			% part to be patched
	{\IfSubStr{\strfield{number}}{\strfield{year}}{}{\usebibmacro{date}}} % how to patch
	{} % If success
	{\typeout{There was an error patching biblatex-ieee (specifically, ieee.bbx's @report driver)}} % If no success



% Print @software entries according to IEEE guidelines, requires zotero entrytype ``report'' as well as the ``extra'' field filled with ``tex.referencetype=standard''
\DeclareBibliographyDriver{software}{%
	\usebibmacro{bibindex}%
	\usebibmacro{begentry}%
%	\usebibmacro{author}%
%	\setunit{\addcomma\addspace}%
	\usebibmacro{maintitle+title}%
%	\printfield{version}%
	\setunit{\addcomma\addspace}%
	\usebibmacro{version+year}%
	\setunit{\labelnamepunct}\newblock
	\usebibmacro{author}%
	\setunit{\addcomma\addspace}%
%	\IfSubStr{\strfield{number}}{\strfield{year}}{}{\usebibmacro{date}} % do not print year if already contained in the standards number
%	\usebibmacro{date}%
	\newunit\newblock
	\usebibmacro{url+urldate}%
	\newunit\newblock
	\usebibmacro{finentry}%
}

\newbibmacro*{version+year}{%
	\printtext{%
		\iffieldundef{version}{\iffieldundef{year}{}{\printfield{year}}}{\printfield{version}}
		\printfield{number}
		\printlist{publisher}
		\iflistundef{publisher}{\space}{}%
	}%	
}


% Für Verlinkungen zwischen den Kapiteln
\usepackage{cleveref}
\crefname{figure}{Fig.}{Fig.}
\crefname{equation}{}{}
\Crefname{equation}{Equation}{Equations}

\usepackage{siunitx}
\sisetup{per-mode=symbol,mode=match,group-digits=integer}
\DeclareSIUnit\foot{ft}


\def\BibTeX{{\rm B\kern-.05em{\sc i\kern-.025em b}\kern-.08em
    T\kern-.1667em\lower.7ex\hbox{E}\kern-.125emX}}

%%Fußnoten repariert (Counter-Reset in Vorlage nach \maketitle?)
\let\oldmaketitle\maketitle
\renewcommand{\maketitle}{\oldmaketitle\setcounter{footnote}{1}}

% ORCID
\newcommand{\orcid}[1]{\href{https://orcid.org/#1}{\includegraphics[width=8pt]{ORCIDiD_icon128x128.png}}}

\usepackage{hologo}
\usepackage{listings}
\lstset{breaklines,basicstyle=\small,columns=fullflexible,basicstyle=\ttfamily,language={[plain]TeX}}



\begin{document}

% --------------------------------------------------------------------
% arXiv Header
% --------------------------------------------------------------------


\thispagestyle{empty}
%\pagestyle{empty}
\twocolumn[
\begin{@twocolumnfalse}
	%
	%
	\large {This work has been submitted to the IEEE for possible publication. Copyright may be transferred without notice, after which this version may no longer be accessible.} \\ \\
	
\end{@twocolumnfalse}
]
\setcounter{page}{0}

% --------------------------------------------------------------------
% 0 Header
% --------------------------------------------------------------------

\history{Date of publication xxxx 00, 0000, date of current version January 23, 2025.}
\doi{XX.XXXX/ACCESS.20XX.XXXXXXX}

\title{
Safety Blind Spot in Remote Driving: Considerations for Risk Assessment of Connection Loss Fallback Strategies
}

\author{\uppercase{Leon Johann Brettin\,\orcid{0009-0007-2796-0860}\authorrefmark{1}, Niklas Braun\,\orcid{0000-0002-9312-8446}\authorrefmark{1}, Robert Graubohm\,\orcid{0000-0002-6682-4788}\authorrefmark{1} and Markus Maurer\,\orcid{0000-0002-5357-9701}\authorrefmark{1}}}

\address[1]{Institute of Control Engineering at Technische Universit\"at Braunschweig, 38106 Braunschweig, Germany}
\tfootnote{This work was partly supported by the German Federal Ministry for Economic Affairs and Climate Action within the project ‘‘Automatisierter Transport zwischen Logistikzentren auf Schnellstraßen im Level 4 (ATLAS-L4).’’}% <-this % stops a space

\markboth
{Brettin \headeretal: Safety Blind Spot in Remote Driving: Considerations for Risk Assessment of Connection Loss Fallback Strategies}
{Brettin \headeretal: Safety Blind Spot in Remote Driving: Considerations for Risk Assessment of Connection Loss Fallback Strategies}

\corresp{Corresponding author: Leon Johann Brettin (e-mail: l.brettin@tu-braunschweig.de).}



% --------------------------------------------------------------------
% 1 Abstract
% --------------------------------------------------------------------

\begin{abstract}

As part of the overall goal of driverless road vehicles, remote driving is a major emerging field of research of its own.
Current remote driving concepts for public road traffic often establish a fallback strategy of immediate braking to a standstill in the event of a connection loss.
This may seem like the most logical option when human control of the vehicle is lost.
However, our simulation results from hundreds of scenarios based on naturalistic traffic scenes indicate high collision rates for any immediate substantial deceleration to a standstill in urban settings.
We show that such a fallback strategy can result in a SOTIF relevant hazard, making it questionable whether such a design decision can be considered acceptable.
Therefore, from a safety perspective, we would call this problem a \emph{safety blind spot}, as safety analyses in this regard seem to be very rare. 

In this article, we first present a simulation on a naturalistic dataset that shows a high probability of collision in the described case.
Second, we discuss the severity of the resulting potential rear-end collisions and provide an even more severe example by including a large commercial vehicle in the potential collision.


\end{abstract}


\begin{keywords}
Safety Blind Spot,
Remote Driving,
Fallback Strategy, 
Safety, 
Risk
\end{keywords}

\titlepgskip=-21pt

\maketitle

% --------------------------------------------------------------------
% 2 Introduction
% --------------------------------------------------------------------
\section{Introduction}

The research on teleoperated vehicles and as a subclass \mbox{remote} driving has seen increased research interest in recent years:
There are legal developments (see, e.g. German Traffic Law~\cite[§~1d]{StVG_1d_BGB} or new UK BSI-Flex~\cite{BSI_1886}), start-up \mbox{companies} (see, e.g. Vay\footnote{\url{https://vay.io}}, Elmo\footnote{\url{https://www.elmoremote.com}}, Fernride\footnote{\url{https://www.fernride.com}}, Voysis\footnote{\url{https://www.voysys.se}}), market opportunities (see, e.g. \cite{Kelkar2025_McKinsey}) and new research questions (see, e.g. \cite{WG_Teleoperation2024_en}) in this area, motivating new system designs and publications.

We observe that current teleoperation initiatives and regulatory efforts \emph{assume} that emergency stopping in the event of major system failures is a design decision that, at least in urban operating environments, generally does not pose unreasonable risk. 
In this context, connection loss would be one of the failure events that would cause remote-driven vehicles to come to an immediate stop. 
In this article we take a closer look at the probability and potential consequences of collisions with closely following vehicles in urban traffic scenarios and find that they can be significant. 
Assuming that, depending on the specific operating environment (i.e., geographic area), unstable connection links may be common and likely failures of remote driving systems, and finding that published safety analyses in this regard are very rare, we consider this to be a \emph{safety blind spot}.

This article is intended for developers, safety engineers, system designers, regulators, and other stakeholders planning to introduce remotely operated vehicles into regular traffic:
Because when considering this safety blind spot from a \mbox{system} design perspective, identifying a lack of an \mbox{adequate} fallback strategy can potentially avoid unknown costs, \mbox{injuries}, or further damage.
Therefore, it should be beneficial to design a sound fallback strategy as early as the concept phase of development, rather than hoping that the connection will break very rarely and never under the wrong circumstances.
However, the purpose of this article is not to solve this blind spot, but to highlight it in the hope of stimulating discussion about a possible specification for a solution.

One of the main motivations for this work was the \mbox{collaborative} work done in the Working Group ``Research Needs in Teleoperation'' \cite{WG_Teleoperation2024_en}, in which some of the authors of this article participated. 
In this report, research questions on different research areas of teleoperation have been collected. 
One of the questions raised in the first cluster, ``Vehicle, area of operation and functional safety'', was what would happen in the event of a loss of connectivity in a teleoperated vehicle, which motivated the work in this article.

We structure the argumentation of this article around two main arguments based on the definition of risk in ISO~26262, which is a ``combination of the probability of occurrence of harm [...] and the severity [...] of that harm [...]''~\cite{ISO26262}.
First we look at the \emph{probability of occurrence} by conducting a \mbox{simulation} using naturalistic driving data:
We simulate \mbox{braking} as a fallback strategy for the rear vehicle with two driver models, one based on the \mbox{\emph{intelligent driver model}}~\cite{Treiber2000} and the other based on a simplified braking model that applies full braking after a set reaction time.
The simulation is configured purely longitudinal with start scenes and initial parameters based on the naturalistic real-world \emph{uniD} dataset~\cite{Krajewski2018, Bock2019, Krajewski2020, Moers2022}.
Secondly, we look at the potential \emph{severity} by assessing potential hazards associated with the analyzed fallback strategy.
For that, we will discuss the same scenario based on ISO~21448~\cite{ISO21448} considerations, as this type of collision could result in a potentially high severity.

The remainder of the article is therefore structured as follow: In Section II, we present terminology and definitions in the context of remote driving.
Related work is presented and discussed in Section III.
A discussion of potential fallback options is provided in Section IV.
The analysis and simulation is described in Section V, followed by the risk assessment in Section VI.
The results of the simulation and assessment are then discussed in Section VII.
The conclusion can be found in Section VIII.


% --------------------------------------------------------------------
% 3 Terminology
% --------------------------------------------------------------------

\section{Terminology and Definitions}

This section provides definitions of important terms used in this article.

\subsection*{Teleoperation}

Teleoperation systems according to \textcite{Winfield2009} require three main parts to function: the robot, the communication link, and the operator interface where an operator can \mbox{remotely} control the vehicle. The robot, in our case a vehicle, must at least have sensors that allow a human to perceive the environment, an interface to access the vehicle's actuators, and a communication device to establish and use a connection to the operator interface. \textcite{Winfield2009} also mentions an ``[o]nboard power and energy management sub-system'' which we are not discussing in this work. 
The communication link transports the signals between the operator interface and the vehicle.
Finally, \textcite{Winfield2009} explains that the operator interface must have some sort of display to receive environment information from the sensors and an input device to enter commands.


\subsection*{Remote Driving}

This article and the following discussion are focused on \emph{Remote Driving} as defined in SAE~J3016 as ``Real-time performance of part or all of the [Dynamic Driving Task (DDT)] and/or DDT fallback (including, real-time braking, steering, acceleration, and transmission shifting), by a remote driver.''~\cite[Clause~3.24]{J3016}. 
To our understanding, as also mentioned in \cite{WG_Teleoperation2024_en}, the definition of Remote Driving does not address the need for an automated driving capability to operate a remote-driven vehicle. 

\citeauthor{Majstorovic2022}~\cite{Majstorovic2022} go into further design features for differentiating between forms of Remote Driving, but we do not refer to these in more detail in this article.
We understand teleoperation as a generic term for the research area discussed in this article, with Remote Driving being a mode of teleoperation.
However, teleoperation can also take other modes, such as \emph{Remote Assistance}~\cite[Clause~3.23]{J3016} which is not part of this article.

\subsection*{Corner/Edge-Cases}
Since the scenario we are considering can be seen as an edge or corner case, both definitions based on \citeauthor{Koopman2019_credible_safety} are given here:
They define an edge-case as ``[...] a rare situation that will occur only occasionally, but still needs specific design attention to be dealt with in a reasonable and safe way. 
The quantification of `rare' is relative, and generally refers to situations or conditions that will occur often enough in a full-scale deployed fleet to be a problem but have not been captured in the design or requirements process.''~\cite[17]{Koopman2019_credible_safety}.
A corner-case, in contrast, is defined as ``combinations of normal operational parameters'' which lead to a rare situation \cite[17]{Koopman2019_credible_safety}.

In Section VII, we discuss how well corner cases or edge cases fit the definition of the scenarios discussed in this article.

% --------------------------------------------------------------------
% 4 Related Work
% --------------------------------------------------------------------

\section{Related Work}


As described in the introduction, one motivation for this work was the research questions raised in \cite{WG_Teleoperation2024_en}. 
In this work, some of the questions raised focus on the connection between the operator and the remote-driven vehicle and the problem that a loss of connection cannot be excluded. 

That the connection between the operator and the vehicle is an vital part of remote driven vehicles is stated in several sources:
\citeauthor{Zhang2020}~\cite{Zhang2020} for example discusses that there are still challenges regarding latency and poor network conditions.
It is shown that the worse the overall latency, the harder it is to control the vehicle (\cite{Zhang2020} following \cite{Sheridan1963, Luck2006, Chen2007, Davis2010, Gnatzig2013, Bodell2016, Liu2017}).
There is also research on how to deal with potentially unreliable connections; some examples are given below:

In an early publication, \textcite{Kay1995} propose a strategy in which the operator plans a route based on waypoints, which is then driven by a remote-driven vehicle equipped with an early automated driving function.
\textcite{Neumeier2019} propose a route planning strategy to mitigate routes with potential for poor network connectivity and vary the speed of the vehicle based on the current latency. 
\textcite{Georg2020} measure the latency of the system and optimize the hardware of the system based on latency and quality. 
The latency is differentiated between the feedback latency for an operator to see an event and the control latency until the commands from an operator are given to the vehicle.
Their work shows that it is important to consider not only the network latency, but also the latency added by the hardware used. 

Of even greater importance is the problem of connection loss.
For this purpose, \textcite{Zhang2020} shows the complexity of how to prepare for and properly respond to a connection loss:
According to \citeauthor{Zhang2020} there needs to be: 
\begin{enumerate}
    \item awareness of a potential connection loss,
    \item the ability to go into a ``safe mode'' as a precaution, and
    % --- Es wird "precaution" als Wort so in der Arbeit verwendet. 
    \item the ability for the vehicle to go into such a ``safe mode'' without the operator.
\end{enumerate}

In another publication by \citeauthor{Tang2014} it is also mentioned that ``one of the main challenges of teleoperated vehicles is the choice of an appropriate safety concept in case of a connection loss''~\cite[1399]{Tang2014}.
To address this challenge, a discussion is needed on how to reach an adequate minimal risk condition when the connection is lost and what characteristics fallback strategies must have to avoid unreasonable risk.


Efforts are already being made to address this problem. For example, \cite{Diermeyer2011, Tang2014, Hoffmann2022} are introducing the concept of a \emph{free corridor} where an area in front of the vehicle has to be kept free from obstacles by the operator in order to allow the vehicle to decelerate to a standstill within the set corridor.
In that case, maintaining the corridor as a free space is a key task for the operator.

In this article, however, we focus on other hazards due to sudden braking of a remote-driven vehicles after a connection loss.
Unlike previous studies, our simulation and investigations focus on collisions caused by vehicles \emph{behind} the remote-driven vehicle rather than the \emph{free corridor} in front of it.


Emergency braking maneuvers that cannot be anticipated by drivers of other vehicles can lead to collisions, even in lower speed areas such as an urban area (as seen e.g. in~\cite{NHTSA2007},~\cite{Distner2009} following \cite{Avery2008}, \cite{Beyerer2019}). However, it should be noted that higher speeds correlate with a higher probability of an incident or crash \cite{NHTSA2007}.
In \cite{Heck2015} and \cite{Beyerer2019} it is shown that triggering automatic emergency braking may even increase the severity of harm under certain conditions (for example, false-triggers of automatic emergency braking is simulated for closed freeways in \cite{Faerber2005}).

For the simulation, we present in Section V, we use the \emph{uniD} dataset ~\cite{Krajewski2018, Bock2019, Krajewski2020, Moers2022} which focuses on a straight, urban road, as this best suits our purpose. 
Besides this dataset, there are four other datasets that are recorded in the same way: \emph{highD}, a dataset focusing on a highway section \cite{Krajewski2018}, \emph{inD}, focusing on an intersection \cite{Bock2019}, \emph{rounD}, focusing on a roundabout \cite{Krajewski2020} and \emph{exiD}, focusing on a highway exit \cite{Moers2022}.
All these datasets are recorded top-down with a drone over the selected road section and are labeled similarly. 

Another topic addressed in this article is to discuss about human behavior in emergency situations. 
For example, \citeauthor{Certad2023}~\cite{Certad2023} also use the \emph{uniD} dataset to investigate the behavior of road users. 
As a methodology they use the \mbox{IEEE 2846-2022~\cite{IEEE2846-2022}} standard. 
They analyzed four scenarios to derive assumptions about kinematic values for ``reasonably foreseeable behavior''. 
The overall goal of their work was to derive potential starting conditions for testing vehicles equipped with an automated driving system \cite{Certad2023}.

As we can see, a lot of work is being done on teleoperation and the safety implications.
To the best of our knowledge, we are not aware of any other work that uses naturalistic data to qualitatively describe the risk of accidents for a specific operational environment and discusses the hazards when remote-driven vehicles exhibit a certain behavior.



% --------------------------------------------------------------------
% 5 Methods
% --------------------------------------------------------------------

\section{Discussion about potential deceleration methods on radio-loss for Remote Driving}

\Figure[ht](topskip=0pt, botskip=0pt, midskip=0pt)[width=0.99\columnwidth]{img/1.pdf}{\textbf{
    A human-driven vehicle is following a remote-driven vehicle. 
    A loss of connection occurs in the remote-driven vehicle, leading to a fallback strategy of braking which could result in a collision.
}\label{fig: overview}}


The chosen situation shown in figure \ref{fig: overview}, is in our opinion, a relatively common one.
Therefore it may be a common scenario in the future to see a human-driven vehicle closely following a remote-driven vehicle.
If the remote-driven vehicle brakes due to a radio-link loss, the vehicle behavior could potentially be hazardous:

In this situation, generally speaking, the remote-driven \mbox{vehicle} has two overall possibilities to react to such a \mbox{situation}:
First, the vehicle could continue to move \mbox{without} any navigation or guidance (one could say `blindly' or `stupidly'), which involves an unreasonably high risk that can be considered \emph{unsafe} since neither a driver nor a remote driver is performing the dynamic driving task.
As a second option, the vehicle can use a failure response strategy and apply braking.
Whether the steering wheel of the remote-driven vehicle is set straight in such a case or the steering angle is fixed, either way, lane departure is imminent.
Here the vehicle still has different approaches to braking: Either it can brake with maximum deceleration to come to a standstill with minimal stopping distance, or it can brake with a more moderate deceleration.
In the latter case, the stopping distance is longer, but a human-operated vehicle following the remote-driven vehicle has more available time and stopping distance for collision avoidance. 
As we discuss in the next subsection, having more time to react can be crucial for the human driver behind the decelerating, (previously) remote-driven vehicle.

\subsection{Emergency Stop}
The first possible reaction is to perform an emergency stop when the radio-link is lost. At first, this might seem the most natural because from the authors' personal perspective, it would probably be what a human would try to do.
If a human driver lost sight and control from one second to the next: Using a failure response strategy that applies braking deceleration at the physical limits reduces the probability of hitting something or someone in the direction of motion.
As will be discussed in the next section, a naturalistic emergency brake is usually not at the physical limits, but for the theoretical thinking of hitting the brake as hard as possible, it seems like a natural idea.


\subsubsection*{Humans cannot foresee such a stop}
A human operating the closely following vehicle behind the stopping remote-driven vehicle has no chance to anticipate the situation because a connection loss is something that cannot be seen from the outside, such as a red traffic light or a child or adult pedestrian crossing the street in front of the vehicle ahead. 
In such a scenario, the human driver has a fundamentally different understanding of the situation than the remote-driven vehicle:
The human driver will not anticipate the connection loss before it occurs and must react to the sudden deceleration, braking lights, and possibly hazard warning lights of the remote-driven vehicle ahead.
This unpredictability for the human driver increases the probability of a rear-end collision compared to sudden braking events in today's traffic.

\subsection{Moderate Deceleration}
Another possible reaction is to use moderate deceleration instead of full braking, resulting in a longer stopping distance. 
This increases the probability of colliding with an object in the vehicle's direction of motion compared to full braking. 

For the sake of simplicity, we make no assumptions about the vehicle's \mbox{capability} of lateral guidance. Hence a longer stopping distance may significantly increase the probability of exiting the lane boundaries as well. 

As mentioned in Section III, researchers from TU Munich~\cite{Diermeyer2011, Tang2014, Hoffmann2022} introduce the \emph{free corridor}, assuming some remaining lateral guidance capability on the vehicle side. 
Here, a path is projected to the operator that the vehicle would follow in the event of a connection loss. 
The path must be kept free by the operator, so that the vehicle has a collision-free path to follow and to brake within if a connection loss would occur, assuming that indeed no objects are in the path.

One open question which time horizon the \emph{free corridor} needs to cover:
If it is too long, the operator might not be able to keep it clear of objects or might be perceived to drive unreasonably slowly.
In complicated situations such as certain curvy roads, lane modifications due to construction zones, or parked vehicles occupying parts of the traffic lane, it might not always even be feasible for the operator to keep the corridor free.
If the corridor is too short, the vehicle has to apply almost full braking to stop within the short corridor and still have the problem of rear-end collisions. 

There is also the problem that future movements of other objects through the corridor are not initially mapped. For example, if people are obscured, the remote operator is unlikely to make a speed adjustment that sufficiently shortens the stopping distance if a person suddenly steps into the lane. Studies and discussions of the speed reductions required from this perspective have been conducted by \textcite{Graubohm2023_Occlusion}.
This is a more general problem, as it can be applied to automated driving as well.

The rear-end collision problem is demonstrated in the next section using a simulation study.


\subsection*{Why not use ADAS lateral and longitudinal guidance capabilities as a fallback?}

Most modern cars are equipped with ADAS realizing lateral and longitudinal guidance capabilities. Many of those vehicles are equipped with driving automation systems allowing up to SAE Level 2 functionality. Additional active safety systems, such as automatic emergency braking, may be present that do not impact the level of driving automation according to SAE~J3016~\cite{J3016}.
%
In \cite{Shi2020_Principle} and \cite{Shi2025_Classification}, emergency braking systems are classified in a separate category (\emph{Principle of Operation C $\beta_{\text{II}}$}) because such systems should temporarily intervene in ``accident-prone situations'' without the driver having enough time to take control.

What cannot be done when thinking of a failure response strategy is to simply continue driving and rely on such systems.
Automatic emergency braking only superficially seems to provide a solution when no automated driving capability is on board. 
The system is only designed to mitigate harm in imminent collisions. In those cases, such a system can reduce the consequences of accidents and potentially even prevent accidents (\cite{Reschka2017} following \cite{Daimler2009, vonHolt2004}), but it is not guaranteed that they will avoid a collision, as the purpose of these systems is different.
As noted in \cite[1198]{Rieken2016}, automatic emergency braking can also be viewed as hazardous if the system activates in the absence of an unavoidable collision criterion.
If there is a failure in the automatic emergency braking function itself, \cite{UN152_AEB} states that safe operation should not be endangered, indicating that the normal safe state is not to use the automatic emergency braking system in this type of situation.

On the other hand, Level 1 or Level 2 features cannot be used as a \emph{safe} fallback either.
By definition, the driver is responsible to monitor the environment and the driving assistance system during Level 1 or Level 2 driving automation~\cite{J3016}.
But there is no driver, when the operator is disconnected from the vehicle.
Such systems are not designed and approved to operate without a human supervisor with authority to overrule.
Therefore, ADAS providing Level 1 or Level 2 features are inappropriate as a fallback option for remote-driving.



% --------------------------------------------------------------------
% 6 Analysis UniD
% --------------------------------------------------------------------

\section{Analysis based on the uniD dataset}
\label{sec: analysis}

To illustrate the aforementioned problems, we performed an analysis on the ``naturalistic'' \emph{uniD} dataset~\cite{Krajewski2018, Bock2019, Krajewski2020, Moers2022}, which contains drone recordings of vehicles and pedestrians on a straight road section in an urban area.

The simulated scenario contains the mentioned scenario of a human-driven vehicle following a remote-driven vehicle on a straight road. 
The simulation starts with a connection loss of the remote-driven vehicle, which leads to a failure response strategy of braking.
By choosing different parameters for the reaction time for the human-driven vehicle and braking behavior of the human-driven and remote-driven vehicle, it is checked whether the human-driven vehicle collides with the remote-driven vehicle.

In the overall structure of this article, this can be seen as looking at the \emph{probability of occurrence} of such a rear-end collision.
The simulation can be seen here as a plausibility check to obtain a significant probability of a rear-end collision occurring with the failure response strategy of braking. 


This section is structured as follows: 
\Cref{sec: model prequisites} deals with the models used and their prerequisites.
The parameterization of the models based on the relevant literature is also discussed in this part.
In \Cref{sec: data preprocessing}, the preprocessing steps performed to obtain naturalistic scenes of vehicles following each other from the dataset are discussed.
\mbox{\Cref{sec: simulation}} describes the simulation and \Cref{sec: limitations} discusses its limitations.
The technical results of our analysis are presented in \Cref{sec: results}.

The overall results and the acceptability of the findings will be used in the discussion in Section VI and VII. 


\subsection{Model Prerequisites}
\label{sec: model prequisites}

\Figure[ht](topskip=0pt, botskip=0pt, midskip=0pt)[width=0.99\columnwidth, page=1]{img/2}{\textbf{One-dimensional simulation of a pair of vehicles on a straight road. The remote-driven vehicle is given index zero and the following human-operated vehicle is given index one. Each vehicle has a position and velocity state and length parameter.}\label{fig:simulation_sketch}}

We implemented two different longitudinal driver models for our simulation.
We call the first a \emph{sudden braking model} which applies an immediate maximum deceleration \mbox{$\dot{v}^\mathrm{min} \leq 0$} after a reaction time $\tau_r$. The second driver model is the \emph{intelligent driver model} introduced by \citeauthor{Treiber2000}~\cite{Treiber2000}. 
We implemented adjustments from \textcite{Albeaik2022} to deal with deceleration values going to $-\infty$ in the model and unwanted reversing of simulated vehicles.
The main part of the \emph{intelligent driver model} as we use it is shown in \cref{eq: IDM 1,eq: IDM 2,eq: IDM 3}:
\begin{align}
    \beta^\mathrm{free} &= \left(\frac{v}{v_0}\right)^\delta \label{eq: IDM 1}\\
    \beta^{\neg\mathrm{free}} &= \left( \frac{ s_0 + v T + \frac{v d}{2 \sqrt{a b }}}{s} \right) ^2 \label{eq: IDM 2}\\
    \dot{v} &= a \cdot ( 1 -\beta^\mathrm{free} - \beta^{\neg\mathrm{free}})
    \label{eq: IDM 3}
\end{align}
where 
\begin{itemize}
    \item $\beta^\mathrm{free}$ is the term governing the free road behavior;
    \item $\beta^{\neg\mathrm{free}}$ is the term governing the following of a vehicle in front (the lead vehicle);
    \item $\dot{v}$ is the resulting commanded acceleration;
    \item $v$ is the current speed of the considered vehicle;
    \item $v_0$ is the desired speed;
    \item $\delta$ is an arbitrary tuning parameter exponent used by the \emph{intelligent driver model} that provides a degree of freedom to adapt the model's characteristics to observed driver behavior;
    \item $s$ is the actual spacing between considered and lead vehicle;
    \item $s_0$ is the desired minimum spacing between considered and the lead vehicle;
    \item $T$ is the desired time headway to the lead vehicle;
    \item $d$ is the distance between the considered vehicle and lead vehicle;
    \item $a$ is the maximum acceleration commanded by the simulated driver;
    \item $b$ is a tuning parameter related to the desired deceleration.
\end{itemize}
For comparison: The algorithm of \textcite{Treiber2000} has an additional summand $s_1 \sqrt{v/v_0}$ inside the numerator in~\cref{eq: IDM 2}, where \citeauthor{Treiber2000} recommend, for simplicity, to set $s_1$ to zero which is what we did.


As mentioned before, we implemented the modifications from \textcite{Albeaik2022} limiting $v$ and $\dot{v}$ to 
\begin{align}
v_{\mathrm{lim}} &= \max{}(v, 0) \\
\dot{v}_{\mathrm{lim}} &= \max{}(\dot{v}, \dot{v}^\mathrm{min})
\end{align}
where $\dot{v}^\mathrm{min}$ is the maximum deceleration (or minimum \mbox{acceleration}).

\begin{table}[ht]
    \centering
    
    \caption{\textbf{Model parametrization}}
    \begin{tabular}{lcccc}
         \textbf{Description} & \textbf{Variable} & \textbf{Value} & \textbf{Unit} & \textbf{Justification}\\
         \hline
         Maximum acceleration& $a$ &\num{0.73} & \unit{\metre\per\second\squared} & \textbf{*1}\\
         Desired deceleration& $b$ &\num{-1.67}& \unit{\metre\per\second\squared} & \textbf{*1}\\
         Maximum deceleration& $\dot{v}^\mathrm{min}$ & \num{-3.41} & \unit{\metre\per\second\squared} & \textbf{*2}\\
         Desired speed& $v_0$  & \num{50} & \unit{\kilo\metre\per\hour}& \textbf{*3}\\
         Desired time headway& $T$ &  \num{1.6}& \unit{\second}& \textbf{*1}\\
         Minimum spacing& $s_0$ &  \num{2} & \unit{\metre}& \textbf{*1}\\
        Length of the vehicle& $d$ & \num{5} & \unit{\metre}& \textbf{*1}\\
        Acceleration exponent& $\delta$ & \num{4}& & \textbf{*1}\\
        &\\
    \end{tabular}
    \begin{tabular}{c|l}
        & The aforementioned values are required as model parameters \\ & for the selected model. The variable names were derived \\ & from \textcite{Treiber2000}.\\
        \textbf{*1} & \citeauthor{Albeaik2022}~\cite{Albeaik2022} state that these are ``typical and meaningful'' \\ &values according to \citeauthor{Treiber2000}~\cite{Treiber2000}. \\
        \textbf{*2} & \citeauthor{Wood2021}~\cite{Wood2021} state that current standards use a \\& deceleration value of $\qty{11.2}{\foot\per\second\squared}  \approxeq  \qty{3.41}{\metre\per\second\squared}$. \\&
        Since this is debatable, a discussion can be found in the text. \\
        \textbf{*3} & Maximum speed limit on the selected street.
    \end{tabular}
    \label{tab: selected values}
\end{table}

Since one of the most important variables for analysing braking behavior for a human driver is the reaction time, we added a ring buffer to the model, which sets a simplified fixed-time delay of the commanded acceleration signal.


\subsubsection*{Discussion about chosen deceleration}
The values listed in \Cref{tab: selected values} are used for calculation within the \emph{intelligent driver model}.
One could argue that the desired deceleration is still quite low compared to some of the other deceleration values used in the literature. Depending on which source is used, deceleration values below or above the selected value can be found. 
For example, \textcite{Graubohm2023_Occlusion} (derived from \cite{Vangi2007}) use an emergency deceleration of $\qty{6.5}{\meter\per\second\squared}$ and \textcite{Reschka2017} (derived from \cite{ISO22839}) suggests a maximum deceleration of $\qty{6.0}{\meter\per\second\squared}$.
On the other hand, when looking at common braking behavior, e.g., \textcite{Stolte2019} derive $\qty{3.0}{\meter\per\second\squared}$ for a naturalistic drive on an inner city ring road; \textcite{Certad2023} derived $\qty{2.0145}{\meter \per \second \squared}$ for a ``reasonable foreseeable'' deceleration value; and for a non emergency situation \textcite{Deligianni2017} come to average deceleration values of $\qty{2.42}{\meter\per\second\squared}$ (with outliers reaching up to $\qty{5.17}{\meter\per\second\squared}$).
The difference here is that we are looking for a value that real drivers would use in a normal braking situation.
As stated e.g. by \citeauthor{Rieken2016}~\mbox{\cite[p. 1178-1179]{Rieken2016}}, it cannot be assumed that every driver will brake with the maximum deceleration possible (or brake at all) when emergency braking would be necessary. 
Since this simulation aims to simulate human behavior, using full braking power does not seem to achieve this goal. 
While Forward Vehicle Collision Mitigation Systems (FVCMS) according to ISO 22839~\cite{ISO22839} can mitigate many of the simulated collisions with an average deceleration of up to $\qty{6}{\meter\per\second\squared}$ if there is sufficient clearance and a working detection, it cannot be generally assumed that the human driven vehicle in this scenario can use such functions, as the market share and activation status of such functions is unclear.

Further determination of realistic human deceleration is beyond the scope of this work.
In our opinion, $\qty{3.41}{\metre\per\second\squared}$ given by \textcite{Wood2021} seems to be acceptable as a compromise between emergency braking and a normal situation.
For comparison, we will run a simulation with the given emergency deceleration of $\qty{6.5}{\meter\per\second\squared}$, which we will address later in \Cref{sec: results}.

\hfill

We define the \emph{sudden braking model} to have only one parameter other than the reaction time, which is the \mbox{deceleration} after the reaction time. We set the deceleration to the same value as used for the \emph{intelligent driver model} by setting $\dot{v}^\mathrm{min}=\qty{-3.41}{\meter\per\second\squared}$. For a moderate braking configuration of the \emph{sudden braking model}, we set the deceleration to half of that value. 

We decided to use the \emph{intelligent driver model} to simulate human behavior regarding longitudinal vehicle \mbox{control}. While we acknowledge that the parameterization and validation of this and other driver models pose research questions of their own we consider this driver model and its parameters sufficiently valid to investigate naturalistic start scenes with respect to potential rear-end collisions.

An example simulation run can be seen in \cref{fig:simulation_run}. 
The figure shows two simulations with the same starting parameters, but with different driver models. 
The orange line represents the leading vehicle, which performs a naturalistic emergency stop at the beginning of the simulation, resulting in a complete stop of the vehicle.


We have taken the start scene from the later described preprocessing of the uniD dataset: A lead vehicle drives with a speed of $v_0$ and another vehicle follows the vehicle at a distance of $x_0 - x_1$ and a speed of $v_1$ on a straight inner-state route.
From there, an emergency stop of the front vehicle is simulated.

The blue dashed line and the green dotted line represent the following vehicle simulated once with the \emph{intelligent driver model} (IDM) and once with the \emph{sudden braking model} (SBM).
The orange line represents the front \mbox{vehicle}, which is simulated using a \emph{sudden braking model} that starts braking at the beginning of the simulation with $a_0 = \qty{-3.41}{\meter \per \second ^2}$.

As it can be seen in the figure, both rear vehicles start braking after a reaction time of $\qty{0.5}{\second}$, as this was set for this example simulation. 
The \emph{intelligent driver model} does not use maximum deceleration and decelerates over a longer time period behind the already stopped vehicle in front.
The IDM-controlled vehicle then stops at the set minimum spacing behind the front vehicle.

\Figure[ht](topskip=0pt, botskip=0pt, midskip=0pt)[width=0.99\columnwidth, page=2]{img/2}{\textbf{Example of a simulation run. The index zero represents the leading vehicle, while the index one denotes the following vehicle. IDM stands for the Intelligent Driver Model, while the SBM represents the Sudden Braking Model. The reaction time for this example is 0.5 s. No collision occurred in this run.}\label{fig:simulation_run}}


\subsection{Data Preprocessing}
\label{sec: data preprocessing}

As we use naturalistic driving scenes in order to initialize the simulation runs, we preprocess the trajectories from the \emph{uniD} dataset in a pipeline. The pipeline identifies vehicles that follow each other in the recorded stretch of road and yields their respective dynamic states. The filtering aims to prevent false positives and rather strictly discards pairs of vehicles that are not clearly following each other or that may be influenced by other traffic participants. In the following, we explain the filtering steps and parameters used for this study.
It should be noted that the detection of driving maneuvers is a research topic in its own right, and that this processing pipeline is only a simplified version for our purposes where the values and filtering steps had to be chosen based on subjective considerations.

For each frame in the recording the pipeline identifies all pairs of dynamic objects in the scene.
At the start, a frame consists of a list of vehicles with their current information regarding position, speed, type, and others.
We then filter these data objects as follows:

\begin{enumerate}
    \item Make pairs of all vehicle data objects that are in a frame.
    \item Only keep pairs of vehicles that point in roughly the same direction by filtering for a maximum absolute heading difference of $\ang{15}$.
    \item The lead vehicle must be in front of the following vehicle, which we partly assure by limiting the bearing to the lead vehicle from the following vehicle to $\ang{15}$.
    \item Discard pairs where the absolute lateral offset of the lead vehicle from the following vehicle's longitudinal axis is more than $\qty{1}{\metre}$. This accounts for possible maneuver ambiguity, e.g. in case of passing or parking vehicles.
    \item Filter out pairs where other road users are nearby such as \emph{pedestrians, bicycles} or \emph{motorcycles} directly alongside or between the pairing.
    \item Per vehicle pair keep only the pair with the closest lead vehicle.
    \item Discard pair where the following vehicle is following the lead vehicle for less than $\qty{1}{\second}$.
    \item Discard pairs where the following vehicle type is neither \emph{car} or \emph{van} or where the lead vehicle is neither of \emph{car, van, truck, bus} or \emph{trailer}. 
\end{enumerate}

\subsection{Simulation}
\label{sec: simulation}

We generate our scenarios as follows:
For each frame of the preprocessed pairs, we define a start scene for the simulation including the initial dynamic states of both the lead and the following vehicles. 
The connection loss of the remote-driven lead vehicle is triggered at the start of the simulation, i.e. in the start scene. The simulation is executed until any termination criterion is reached.
Termination conditions are:  
\begin{enumerate}
    \item reaching maximum simulation duration, 
    \item a collision occurs,
    \item all vehicles have come to a standstill,
\end{enumerate}

The initial scene serves as the starting point for subsequent scenes, which are developed with the help of the simulation. 
With that, the sequence of scenes can be seen as a scenario as defined in \textcite{Ulbrich2015}.

As shown in \Cref{tab: collision rate 1,tab: collision rate 2}, the reaction time $\tau_r$ and the desired deceleration $b$ are varied. Using steps of $\qty{0.5}{\second}$ for the reaction time up to a reaction time of $\qty{2.5}{\second}$. 
In comparison to the literature, \textcite{NHTSA2007} use a reaction time of $\qty{2}{\second}$ for the ``overwhelming majority'' of drivers. The \textcite{AASHTO2018} even uses $\qty{2.5}{\second}$ as the ``Perception-reaction time''. 

For the desired deceleration we use $\qty{-3.41}{\meter\per\second\squared}$ in \Cref{tab: collision rate 1} and $\qty{-1.71}{\meter\per\second\squared}$ in \Cref{tab: collision rate 2} for the lead vehicle.
The rear vehicle, which represents the human-driven vehicle, uses the selected naturalistic emergency deceleration of $\qty{-3.41}{\meter\per\second\squared}$.

We detect a collision between two vehicles by determining whether there is an overlap based on the positions and lengths of the vehicles.
The calculated collision rate in \Cref{tab: collision rate 1,tab: collision rate 2} is the sum of the scenarios with a collision divided by all scenarios we preprocessed, which is a total of $\sim 186000$.

\vspace{0.3cm}

\begin{table}[ht]
    \centering
    \caption{\textbf{Collision Rate - naturalistic emergency deceleration}}
    \begin{tabular}{ccc}
         \textbf{Reaction Time} & \multicolumn{2}{c}{\textbf{Collision Rate}}   \\
                                &  Sudden Braking Model & Intelligent Driver Model   \\
         \hline
         0.0&  0.03& 0.03\\
         0.5&  0.92& 0.92\\
         1.0&  8.06& 8.74\\
         1.5&  27.44& 46.65\\
         2.0& 49.04&73.09\\
         2.5& 65.13&86.09\\
         \hline
        in $\unit{\second}$ & in \unit{\percent} & in \unit{\percent} \\
    \end{tabular}
    
    \label{tab: collision rate 1}
\end{table}

\vspace{0.3cm}

\begin{table}[ht]
    \centering
    \caption{\textbf{Collision Rate - moderate deceleration}}
    \begin{tabular}{ccc}
         \textbf{Reaction Time} & \multicolumn{2}{c}{\textbf{Collision Rate}}   \\
                                &  Sudden Braking Model & Intelligent Driver Model   \\
            \hline
         0.0&  0.00& 0.00\\
         0.5&  0.01& 0.01\\
         1.0&  1.13& 1.13\\
         1.5&  6.24& 7.19\\
         2.0& 19.12&34.96\\
         2.5& 37.44&82.63\\
         \hline
        in $\unit{\second}$ & in \unit{\percent} & in \unit{\percent} \\
         
    \end{tabular}
    \label{tab: collision rate 2}
\end{table}
\subsection{Limitations}
\label{sec: limitations}
This simulation has the following limitations:
\begin{itemize}
    \item With the \emph{intelligent driver model} and our parameter selection we aim to simulate human behavior. As mentioned before, we acknowledge that this is a separate research topic and that with a more complex model there is a high probability of achieving more human-like behavior. 
    \item The \emph{intelligent driver model} itself is mainly used for traffic flow simulations and not for rear-end collisions. For this reason, we also added the \emph{sudden braking model} to show what would happen if a human driver would consistently brake with a constant maximum deceleration after a given reaction time. 
    However, this model is also limited by its simplicity and therefore cannot be considered a one-to-one simulation of human behavior.
    \item The data preprocessing pipeline to identify scenes where one vehicle is following another vehicle is based on a subjective considerations. We acknowledge that maneuver identification is also a research topic in its own right and we used a simplified approach to get the chosen initial parameters for our simulation. 
    \item Our simulation does not  take into account real-world aspects of deceleration conditions such as friction or aerodynamics or geographic variations such as road gradient and slope.
    \item Our initial parameters rely on the correctness of the \emph{uniD}~\cite{Krajewski2018, Bock2019, Krajewski2020, Moers2022} dataset and the correctness of the data recorded there. 
    Also, we only look at scenarios in the setting recorded in the \emph{uniD} dataset, which means that the vehicles following each other are limited to the road section recorded.
    \item As mentioned previously, we do not simulate lateral guidance and focus on longitudinal braking. There may be fallback strategies using lateral guidance that could perform better. We do not address those strategies in this work. 
\end{itemize}

\subsection{Results}
\label{sec: results}

This subsection considers the simulation results from a technical perspective.
A discussion about the general context of the results can be found in \Cref{sec: discussion}.

Our results, as seen in \Cref{tab: collision rate 1} and \Cref{tab: collision rate 2}, show that there is a high collision rate given our assumptions in the simulation, and that the collision rate strongly correlates with the reaction time.

Some of the selected findings are described below:
\begin{itemize}
    \item  The overall collision rate, given the scenarios selected through preprocessing, is high, ranging up to a collision rate of $\qty{86.09}{\percent}$ when a reaction time of $\qty{2.5}{\second}$ is used.
    \item  At moderate or half deceleration, the collision rates are lower, but still result in collision rates of $\qty{19.12}{\percent}$ and $\qty{37.44}{\percent}$ for reaction times of $\qty{2}{\second}$ and $\qty{2.5}{\second}$, respectively. 
    \item  For $\qty{0.0}{\second}$ reaction time and full emergency braking for both vehicles, there is still one vehicle combination with \num{63} starting scenes in the dataset that would lead to a collision. Our hypothesis for this case is that the following vehicle did not expect the front vehicle to brake in this scenario, allowing it to go faster than the front vehicle for a short time and keeping a too short distance to the front vehicle.
    \item For comparison with the discussion of the deceleration value, we also performed the simulation with an emergency deceleration of $\qty{6.5}{\meter\per\second\squared}$ for both vehicles with the \emph{intelligent driver model} and a reaction time of $\qty{1}{\second}$. With these parameters we would have a collision rate of $\qty{10.25}{\percent}$ in our considered scenarios.
\end{itemize}


We can also see that the lower the deceleration of the front vehicle, the lower the collision rate, but our simulation only considers rear-end collisions. 
It will considerably increase the potential danger to the front if the vehicle comes to a standstill very slowly.
This means that if the deceleration is set too low, the remote-driven vehicle may decelerate with a low collision rate according to the analysis here, but because of the long stopping distance, it may slowly drive into a vehicle in front.

Returning to the considered scenario of a human-driven vehicle following a remote-driven vehicle, the results show that, given our assumptions, the fallback strategy of immediate braking to a standstill in the event of a connection loss results in a high collision rate in our simulation. 
This means that our results show that even in urban areas, if the only option for a remote-driven vehicle in the event of a radio-link loss is to apply a naturalistic emergency brake, we will potentially see an increase of occurrence in rear-end collisions against current urban traffic.

With such a high probability of a rear-end collision, the next logical step is to see if this hazard can be considered acceptable, which will be discussed in the next section. 


% --------------------------------------------------------------------
% 7 Ris-Analysis
% --------------------------------------------------------------------

\section{Risk assessment with respect to simple fallback strategies}

In the last section we looked at the \emph{probability of occurrence} of such a rear-end collision when a sudden braking occurs due to a connection loss. 
As stated, our simulation can be seen here as a plausibility check to obtain a significant probability of a rear-end collision occurring with the failure response strategy of braking. 

In this part we will look at the \emph{severity} of such a collision.
The overall goal is to arrive at a conceptual risk assessment for the described fallback strategy.
It makes sense to look at this case with respect to ISO~21448~\cite{ISO21448}, as it deals with hazardous vehicle behavior due to functional insufficiencies and specific triggering conditions of road vehicle systems.
Specifying a fallback strategy unable to safely deal with foreseeable operational situations can be seen as an insufficiency of specification \cite[Clause 3.12]{ISO21448}.
Therefore, we will look at the guidance process defined in the standard to examine whether the scenario considered is of relevance and potentially requires functional modifications of the remote driving system to address SOTIF-related risk.

% I moved this up so that it will be on the same page as the Subsection "Hazardour Event Model", which is not the case when moving it to this subsection.
\Figure[ht!](topskip=0pt, botskip=0pt, midskip=0pt)[width=\linewidth, page=3]{img/2}{\textbf{Example of hazardous event model according to ISO~21448~\cite{ISO21448}. 
The figure shows the event chain for radio loss of a remote-driven vehicle with braking as a failure-response strategy.
In ISO~21448~\cite[28]{ISO21448} the description of unintended braking due to a radar reflection was used as a reference. The part considering harm will be discussed in the article.}\label{fig: hazards-case}}


To further illustrate the potential risk involved in the rear-end crash scenario, we will also consider what would happen if a large truck is included in the potential collision.

\subsubsection*{Why we chose this example?}
We have chosen a truck as an example here because, in the authors' view, this figurative example makes it even clearer that the potential harm is even higher when a truck is involved in this type of collision.
This again raises the question of whether the described fallback behavior can be considered acceptable when functionally specifying a remote driving system.

\subsection{Hazardous Event Model}


In \Cref{fig: hazards-case} we have applied the hazardous event model based on ISO~21448~\cite{ISO21448} to the case described. 
We will only show the relevant aspects of a scenario here and go into more detail for the whole scenario later.

\begin{itemize}
%
% Triggering Condition
\item As described in the sections before, the \emph{Triggering Condition} in this case would be the loss of the radio-link from the operator station to the remote-driven vehicle. 
%
% Hazardous Behavior
\item This results in a \emph{Hazardous Behavior} namely a sudden and extreme braking of the remote-driven vehicle.
%
% Hazard
\item One \emph{Hazard} in this case would be a rear-end collision with a following vehicle. 
%
% Scenario ascepct
\item The important \emph{Scenario Aspect} for the selected hazard is that another human-driven vehicle is following the remote-driven vehicle closely, as only then a rear-end collision can occur. 
%
% Hazardous Event
\item The \emph{Hazardous Event} is the presence of a following vehicle combined with a sudden and extreme braking of the remote-driven vehicle. For this case, as the scenario is within an urban area, the impact velocity for the rear-end collision can be up to  $\qty{50}{\kilo \meter \per \hour}$. The deceleration of the remote-driven vehicle depends on the specified fallback strategy, which  is set to an immediate sudden deceleration.
%
% Hazardous event not controlled
\item Since the driver of the following vehicle cannot \mbox{anticipate} the unpredictable sudden braking of the remote-driven vehicle, the collision \emph{is not controllable}. The driver's reaction time is too short and the stopping distance too long.
%
% Harm
\item Depending on parameters, e.g. type of colliding vehicle or speed difference at impact, a \emph{Harm} to the life and health of the occupants of both vehicles can occur. This is discussed later in this section.

\end{itemize}

\subsection{Description of the scenario}
As is common in hazard analysis, a description of the scenario is given here:

A human-driven vehicle is following a remote-driven \mbox{vehicle} on a straight road in an urban area.
The speed of the vehicles is in the range of the normal flow speed in urban road traffic in Germany.
At a given time, the remote-driven vehicle experiences a radio-link loss, to which it responds with a set fallback strategy.
According to the remote driving system specification, a braking maneuver is started immediately and is performed with a relatively strong deceleration to prevent the vehicle from leaving the lane or colliding with the vehicle in front.
The scenario includes an unavoidable collision between the leading, initially remote-driven,  vehicle and the human-driven vehicle due to a sudden braking maneuver after radio-link loss.
In this scenario, the weather and overall visibility conditions are good.


\subsection{Evaluation of Hazards}

The fact that sudden braking to a standstill causes unavoidable collision and thus represents \emph{Hazardous Behavior} shows that the failure-response strategy analyzed in the case study can also be interpreted as a functional insufficiency of the teleoperation system in relation to the \emph{Triggering Condition} "connection failure" within the framework of ISO~21448~\cite{ISO21448}.


With regard to ISO~21448~\cite[Clause 6.4]{ISO21448}, when looking at the \emph{severity} and \emph{controllability} of the \emph{Hazardous Event}, it should be clear that this event cannot be classified as neither ``controllable in general'' (C=0) nor ``no resulting harm'' (S=0), making it SOTIF-relevant. 

That the collision can be given at least an S1 severity rating can be seen by looking at J2980~\cite[18]{J2980}, where an S1 severity has to be assigned for frontal and rear impacts at speeds larger than $4 - \qty {10}{\kilo \meter \per \hour}$.
This is given in this scenario.

\subsection{Acceptance Criteria}

With respect to ISO~21448, the next aspect to consider is whether the specification of a sudden braking maneuver in the event of a radio-link loss can be considered an acceptable behavior or whether the behavior must be classified as unacceptable.

For example, ISO~21448 gives a reference to the positive risk balance, which is defined in ISO/TR~4804 as the ``benefit of sufficiently mitigating residual risk of traffic participation due to automated vehicles''~\cite{ISOTR4804:2020}.

The sudden failure of a connection, resulting in the \mbox{inability} to continue operation, is a completely new event, which can be expected to occur commonly, especially in individual operating environments due to specific circumstances

If this new type of event is fundamentally associated with vehicle behavior that leads to unavoidable accidents in normal traffic situations within the operating range of the system, possibly with serious injuries to those involved, social acceptance from a risk point of view or approval is very doubtful.

The risk balance in this kind of situation can even be seen as negative, as the introduction of remote-driven vehicles with these kind of fallback strategies would even increase the number of rear-end collisions.


\subsection{Concrete example with increased potential harm}

The described event is generally dangerous, but if we now consider the possibility of a truck following behind, the \mbox{severity} of the collision increases significantly.

For this argumentation, we use the definition of large trucks from \cite{NHTSA2023_facts}, where they are defined as ``Trucks over 10,000 pounds GVWR [gross vehicle weight rating], including single-unit trucks and truck tractors''~\cite{NHTSA2023_facts}.

\Figure[t!](topskip=0pt, botskip=0pt, midskip=0pt)[width=0.7\linewidth]{img/3.pdf}
{ \textbf{A large truck is following a remote-driven vehicle, which initiates emergency braking due to radio-link loss.}\label{fig: case-1}}


When considering a potential collision between a large truck at the rear and a remote-driven passenger vehicle at the front, a distinction can be made between the case where passengers are in the remote-driven vehicle and the case where no passengers are in the remote-driven vehicle:

\subsubsection*{Passengers inside the front vehicle}
The potential for increased harm is addressed in several accident research publications:

E.g. \citeauthor{Evans1993} state that there is a ``large influence of mass on relative driver fatality risk when two vehicles of different mass crash into each other''\cite[223]{Evans1993}.
This shows that even low-speed collisions between a truck and a passenger vehicle have a higher risk of fatality than collisions between two passenger vehicles.

That trucks involved in crashes lead to a high severity is stated in several sources \cite{Stigson2006, Chen2020, Liu2022}.

Research and publications in this area of such accidents is a vast field: reaching from looking at neck injuries for the passengers (eg. in \cite{Erbulut2014}) to collision calculations (eg. in \cite{Yang2005, Kostek2017}) to accident data (eg. in \cite{NHTSA2023_facts}).

We can conclude from this literature analysis that truck involvement increases the likelihood of significant harm to life and health in rear-end collisions.

\subsubsection*{No Passengers inside the front vehicle}
The potential for harm in the lead vehicle does not exist if there are no passengers in that vehicle.
Nevertheless, this does not change the other sources of harm of the potential collision, as the same collision could still occur here.

Looking again at ISO~21448~\cite{ISO21448}, there could still be different hazards resulting from this potential collision.
A few examples are described below:
\begin{itemize}
    \item Colliding into the rear of the remote-driven vehicle harms the truck driver.
    \item The collision pushes the vehicle onto the sidewalk, \mbox{endangering} pedestrians.
    \item The truck driver is able to react in time, but the vehicle behind the truck driver is not, resulting in a collision caused by, but not involving, the remote-driven vehicle.
\end{itemize}


% --------------------------------------------------------------------
% 8 Discussion
% --------------------------------------------------------------------

\section{Discussion}
\label{sec: discussion}
In summary, we have two main findings from our analysis:
\begin{itemize}
    \item The simulation results show that there is a high probability of a rear-end collision when the remote-driven vehicle brakes as a fallback strategy.
    \item The identified SOTIF implications according to ISO~21448~\cite{ISO21448} question whether potential vehicle behavior with respect to the analyzed remote-driving fallbacks can be considered acceptable in a system design context.
\end{itemize}
We describe the potential for significant harm, even in low-speed urban environments, by including large trucks in the accident scenarios considered.

As mentioned, risk is defined as the ``combination of the probability of occurrence of harm [...] and the severity [...] of that harm [...]''~\cite{ISO26262}. 
We are able to show that, based on our assumptions, the described scenario has both a high probability and a relevant potential for harm.

% ---
% Corner Case Diskussion
To further elaborate on the probability of occurrence, we believe it is even questionable to call this scenario a corner or edge case: 
A vehicle following another vehicle is a common scenario.
And for the emergency braking to initiate in this type of fallback scenario, it does not require multiple operational parameters, just a connection loss of the remote-driven vehicle.
Whether a connection loss can be considered a rare operating parameter is beyond the scope of this work.
However, when considering at the concept phase of designing a remote-driven vehicle, it cannot be considered rare from a safety perspective unless there is evidence of a specifically reliable radio-link in the intended operating environment.

% ---
% Social Backlash
Another problem that could arise is that even if no fatal harm results from a collision, there is still a high potential for social backlash:
\citeauthor{Cummings2024}~\cite{Cummings2024} points out that for self-driving vehicles, public sentiment is shifting against these operations due to minor accidents, lane obstructions, and vehicle not moving when ``they cannot negotiate a particular situation''.
She also points to a number of surveys that undermine this statement \cite{Smith2014, Smith2017, Rainie2022}.
Also, as can be seen, for example, in driverless field tests of SAE Level 4 automated driving functions, a collision can quickly lead to distrust and potential suspension of approval, as seen in the Cruise accident, where a collision with a pedestrian was the initial trigger for a suspension of the driverless testing permit \cite{Cruise_with_Link, DMV_Cruise2023, Mathews2024}.


To close the loop, considering this problem from a system design perspective:
When designing a remote-driven vehicle for an urban area, taking into account a fallback strategy in the event of a connection loss could lead to design decisions other than the probably simplest one of just braking.
Unless it is revisited and replaced with an appropriate fallback strategy, braking in the event of a connection loss may be considered an inadequate specification.
However, omitting this detail and relying on the hope that there will never be a connection loss, or that accident scenarios will not occur or not be harmful, can lead to much higher costs when found while testing or operating the designed system.

For the design phase of a remote driving function in an urban area, we acknowledge that this finding probably makes the development of such a function a much more complex task:
Designing an appropriate fallback strategy is, by definition, not possible with a Level 2 function.
Specifically, for continued lateral and longitudinal guidance, the system would need to be capable of performing the dynamic driving task without human supervision.
But in the end, such system designs have to be discussed and regulated for the overall safety of all road users.
As stated in the introduction, the purpose of this article is not to solve this problem, but to point it out, in the hope of stimulating discussion about a possible specification for a solution.


% --------------------------------------------------------------------
% 9 Conclusion
% --------------------------------------------------------------------

\section{Conclusion and Future Research}

Despite the current emergence of remote driving in public road traffic, the question of how to deal with a radio-link loss, which is a major concern in the field of teleoperation, has not yet been fully addressed and can be seen as a \emph{safety blind spot}.
Depending on the intended use of the remote driving function, it is not trivial to design a fallback strategy that leads to an adequate minimal risk condition. 
This article shows that there are non-trivial problems with rear-end collisions for remote-driven vehicles when the connection is lost.

Our work also shows two things regarding the \emph{probability} and the \emph{severity} that could result from an inadequate fallback strategy:
There is (a) a \emph{high probability} of collisions if the only option available to the remote-driven vehicle is sudden braking maneuver in case of a radio-link loss.
We have demonstrated this using simple, but reasoned simulations based on a naturalistic dataset.

Also (b), in terms of \emph{severity}, we discussed the hazardous behavior showing that it is potentially unsafe and rather questionable whether such a fallback strategy can be considered acceptable. 

Furthermore, we substantiated that the potential harm of a collision caused by unpredictable deceleration is significantly higher when a large truck is involved.
In conclusion, we find that a fallback that applies naturalistic emergency deceleration on radio-link loss is not a proper fallback strategy in the scenarios analyzed.
We hope that this article will help developers, safety engineers, system designers, regulators, and other involved stakeholders to consider this issue in the design phase, and that it will stimulate discussion on how to deal with fallback strategies for remote-driven vehicles in urban areas. 

In future work, we will look at the problem for \mbox{vehicles} with an Automated Driving System on board, where teleoperation is used to ``extend'' the operational domain of this system. 
Another potential research question is to examine how the chosen models behave on datasets other than the \emph{uniD} dataset. 
Possible interactions in front of the remote-driven vehicle could also be considered depending on real datasets. 
In addition to considering the rear of the remote-driven vehicle, a follow-up question would be to assess the collision potential of vehicles in front of the remotely operated vehicle.

\renewcommand*{\bibfont}{\footnotesize} 

\printbibliography 


% --------------------------------------------------------------------
% Bios
% --------------------------------------------------------------------

\begin{IEEEbiography}[{\includegraphics[width=1in,clip,keepaspectratio]{img/bios/leonb.png}}]{Leon Johann Brettin} received the B.Sc. degree in computer science from Ostfalia University of Applied Sciences, Wolfenbüttel, Germany (2017) and the M.Sc. degree in computer science from Technische Universität Braunschweig, Germany (2021).
He is currently a Research Associate with the Institute of Control Engineering, Technische Universität Braunschweig. His main research interests include the areas of teleoperated vehicles and human-machine-interfaces.
\end{IEEEbiography}

\begin{IEEEbiography}[{\includegraphics[width=1in,clip,keepaspectratio]{img/bios/nikla.jpg}}]{Niklas Braun} received the B.Eng. degree in the field of automotive engineering from Ostfalia University of Applied Sciences, Wolfsburg, Germany (2019) and the M. Sc. degree in the field of electrical engineering from Technische Universität Braunschweig, Germany (2022). He is currently a Research Associate with the Institute of Control Engineering, Technische Universität Braunschweig. His main research interests includes validity considerations of simulation in safety assurance of Automated Driving Systems.
\end{IEEEbiography}

\begin{IEEEbiography}[{\includegraphics[width=1in,clip,keepaspectratio]{img/bios/graub.jpg}}]{Robert Graubohm} received the B.Sc. (2013) and M.Sc. (2016) degree in industrial engineering in the field mechanical engineering from Technische Universität Braunschweig, Germany, and the M.B.A. (2015) degree from the University of Rhode Island, Kingston, RI, USA. 
He is currently a Research Associate with the Institute of Control Engineering, Technische Universität Braunschweig. His main research interests include development processes of automated driving functions and the safety conception in an early design stage.
\end{IEEEbiography}

\begin{IEEEbiography}[
{\includegraphics[width=1in,clip,keepaspectratio]{img/bios/maure.jpg}}]{Markus Maurer} received the Diploma degree in electrical engineering from the Technische Universität München, in 1993, and the PhD degree in automated driving from the Group of Prof. E. D. Dickmanns, Universität der Bundeswehr München, in 2000. 
From 1999 to 2007, he was a Project Manager and the Head of the Development Department of Driver Assistance Systems, Audi. 
Since 2007, he has been a Full Professor of automotive electronics systems with the Institute of Control Engineering, Technische Universität Braunschweig.
His research interests include both functional and systemic aspects of automated road vehicles.
\end{IEEEbiography}


\EOD
\end{document}
