\section{Related Work}
As described in the introduction, one motivation for this work was the research questions raised in \cite{WG_Teleoperation2024_en}. 
In this work, some of the questions raised focus on the connection between the operator and the remote-driven vehicle and the problem that a loss of connection cannot be excluded. 

That the connection between the operator and the vehicle is an vital part of remote driven vehicles is stated in several sources:
\citeauthor{Zhang2020}~\cite{Zhang2020} for example discusses that there are still challenges regarding latency and poor network conditions.
It is shown that the worse the overall latency, the harder it is to control the vehicle (\cite{Zhang2020} following \cite{Sheridan1963, Luck2006, Chen2007, Davis2010, Gnatzig2013, Bodell2016, Liu2017}).
There is also research on how to deal with potentially unreliable connections; some examples are given below:

In an early publication, \textcite{Kay1995} propose a strategy in which the operator plans a route based on waypoints, which is then driven by a remote-driven vehicle equipped with an early automated driving function.
\textcite{Neumeier2019} propose a route planning strategy to mitigate routes with potential for poor network connectivity and vary the speed of the vehicle based on the current latency. 
\textcite{Georg2020} measure the latency of the system and optimize the hardware of the system based on latency and quality. 
The latency is differentiated between the feedback latency for an operator to see an event and the control latency until the commands from an operator are given to the vehicle.
Their work shows that it is important to consider not only the network latency, but also the latency added by the hardware used. 

Of even greater importance is the problem of connection loss.
For this purpose, \textcite{Zhang2020} shows the complexity of how to prepare for and properly respond to a connection loss:
According to \citeauthor{Zhang2020} there needs to be: 
\begin{enumerate}
    \item awareness of a potential connection loss,
    \item the ability to go into a ``safe mode'' as a precaution, and
    % --- Es wird "precaution" als Wort so in der Arbeit verwendet. 
    \item the ability for the vehicle to go into such a ``safe mode'' without the operator.
\end{enumerate}

In another publication by \citeauthor{Tang2014} it is also mentioned that ``one of the main challenges of teleoperated vehicles is the choice of an appropriate safety concept in case of a connection loss''~\cite[1399]{Tang2014}.
To address this challenge, a discussion is needed on how to reach an adequate minimal risk condition when the connection is lost and what characteristics fallback strategies must have to avoid unreasonable risk.


Efforts are already being made to address this problem. For example, \cite{Diermeyer2011, Tang2014, Hoffmann2022} are introducing the concept of a \emph{free corridor} where an area in front of the vehicle has to be kept free from obstacles by the operator in order to allow the vehicle to decelerate to a standstill within the set corridor.
In that case, maintaining the corridor as a free space is a key task for the operator.

In this article, however, we focus on other hazards due to sudden braking of a remote-driven vehicles after a connection loss.
Unlike previous studies, our simulation and investigations focus on collisions caused by vehicles \emph{behind} the remote-driven vehicle rather than the \emph{free corridor} in front of it.


Emergency braking maneuvers that cannot be anticipated by drivers of other vehicles can lead to collisions, even in lower speed areas such as an urban area (as seen e.g. in~\cite{NHTSA2007},~\cite{Distner2009} following \cite{Avery2008}, \cite{Beyerer2019}). However, it should be noted that higher speeds correlate with a higher probability of an incident or crash \cite{NHTSA2007}.
In \cite{Heck2015} and \cite{Beyerer2019} it is shown that triggering automatic emergency braking may even increase the severity of harm under certain conditions (for example, false-triggers of automatic emergency braking is simulated for closed freeways in \cite{Faerber2005}).

For the simulation, we present in Section V, we use the \emph{uniD} dataset ~\cite{Krajewski2018, Bock2019, Krajewski2020, Moers2022} which focuses on a straight, urban road, as this best suits our purpose. 
Besides this dataset, there are four other datasets that are recorded in the same way: \emph{highD}, a dataset focusing on a highway section \cite{Krajewski2018}, \emph{inD}, focusing on an intersection \cite{Bock2019}, \emph{rounD}, focusing on a roundabout \cite{Krajewski2020} and \emph{exiD}, focusing on a highway exit \cite{Moers2022}.
All these datasets are recorded top-down with a drone over the selected road section and are labeled similarly. 

Another topic addressed in this article is to discuss about human behavior in emergency situations. 
For example, \citeauthor{Certad2023}~\cite{Certad2023} also use the \emph{uniD} dataset to investigate the behavior of road users. 
As a methodology they use the \mbox{IEEE 2846-2022~\cite{IEEE2846-2022}} standard. 
They analyzed four scenarios to derive assumptions about kinematic values for ``reasonably foreseeable behavior''. 
The overall goal of their work was to derive potential starting conditions for testing vehicles equipped with an automated driving system \cite{Certad2023}.

As we can see, a lot of work is being done on teleoperation and the safety implications.
To the best of our knowledge, we are not aware of any other work that uses naturalistic data to qualitatively describe the risk of accidents for a specific operational environment and discusses the hazards when remote-driven vehicles exhibit a certain behavior.



% --------------------------------------------------------------------
% 5 Methods
% --------------------------------------------------------------------