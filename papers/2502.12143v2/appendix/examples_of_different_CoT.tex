
\section{Examples of Speaking Style Shift}
\label{Examples of Speaking Style Shift}
We adopt the method from  \citep{lin2023unlockingspellbasellms} to evaluate the most shifted tokens after fine-tuning on long CoT and Large teacher CoT data. Figure \ref{fig:speaking_style} shows the calculation process. This allows us to compare the token distribution shifts induced by the fine-tuning process. We annotate the tokens that exhibit the largest rank shifts as the most shifted tokens. We choose \texttt{Qwen2.5-3B-Instruct} as the student model. We put the results of most shifted tokens after fine-tuning on long CoT data in Figure \ref{fig: most_shifted_1} and \ref{fig: most_shifted_2}. The results of most shifted tokens after fine-tuning on large teacher CoT data are shown in Figure \ref{fig: most_shifted_3}. Our analysis reveals that these tokens are predominantly associated with expressive and stylistic elements, such as “wait”, “But”, and “Let”. 


\begin{figure*}[!t]
    \centering
    \includegraphics[width=1\textwidth]{figs/speaking_way_shift.png}
    \caption{The process of calculating most shifted tokens. We decode each token generated by the fine-tuned LLM in the student model before fine-tuning. Then we calculate the rank shift in the student model for each token generated by the fine-tuned model. We annotate the tokens that exhibit the largest rank shifts as the most shifted tokens. We found that these tokens are predominantly associated with expressive and stylistic elements, such as “But” and “Let”.}
    \label{fig:speaking_style}
\end{figure*}


\begin{figure*}[htbp]
    \centering
\begin{tcolorbox}[title=Problem, promptstyle]
\lstset{
    basicstyle=\normalfont\sffamily\footnotesize,
    breaklines=true,
    frame=none,
    columns=fullflexible,
}
Let $(x,y)$ be an ordered pair of real numbers that satisfies the equation $x^2+y^2=14x+48y$. What is the maximum value of $y$?
\end{tcolorbox}
\end{figure*}






\begin{figure*}[htbp]
    \centering
\begin{tcolorbox}[title=Example of \textcolor{red}{most shifted tokens} after fine-tuning on long CoT data. (1/2), promptstyle]
\lstset{
    basicstyle=\normalfont\sffamily\footnotesize,
    breaklines=true,
    frame=none,
    columns=fullflexible,
}
So I've got this math problem here\textcolor{red}{.} It says\textcolor{red}{:} ``\textcolor{red}{Let} (\textcolor{red}{x}\textcolor{red}{,} y\textcolor{red}{)} be an ordered pair of real numbers that satisfies the equation \(x^2 + y^2 = 14x + 48y\)\textcolor{red}{.} What is the maximum value of \(y\)\textcolor{red}{?} Okay\textcolor{red}{,} let's see how to approach this.

\textcolor{red}{First}\textcolor{red}{,} I need to understand what the equation is telling me\textcolor{red}{.} It's an equation involving \(x\) and \(y\)\textcolor{red}{,} and it's set equal to zero\textcolor{red}{:} 
\[
x^2 + y^2 - 14x - 48y = 0.
\]
I need to find the maximum value of \(y\) for real numbers \(x\) and \(y\) that satisfy this equation.

Hmm\textcolor{red}{,} maybe I can rearr\textcolor{red}{ange} this equation to make it easier to work with\textcolor{red}{.} Let's see:
\[
x^2 - 14x + y^2 - 48y = 0.
\]
I can complete the square for both \(x\) and \(y\) terms.

\textcolor{red}{For} \(x\)\textcolor{red}{:} \(x^2 - 14x\)\textcolor{red}{,} to complete the square\textcolor{red}{,} I take half of \(-14\), which is \(-7\), and square it to get \(49\)\textcolor{red}{.} So\textcolor{red}{,} \(x^2 - 14x\) becomes \((x - 7\textcolor{red}{)}^2 - 49\).

\textcolor{red}{For} \(y\): \(y^2 - 48y\), half of \(-48\) is \(-24\), squared is \(576\)\textcolor{red}{.} So, \(y^2 - 48y\) becomes \((y - 24\textcolor{red}{)}^2 - 576\).

\textcolor{red}{Now}\textcolor{red}{,} pl\textcolor{red}{ugging} these back into the equation:
\[
(x - 7\textcolor{red}{)}^2 - 49 + (y - 24\textcolor{red}{)}^2 - 576 = 0.
\]
S\textcolor{red}{implify}:
\[
(x - 7\textcolor{red}{)}^2 + (y - 24)^2 - 625 = 0,
\]
\[
\text{So, } (x - 7\textcolor{red}{)}^2 + (y - 24)^2 = 625.
\]

\textcolor{red}{This} looks like the equation of a circle with center at \((7, 24\textcolor{red}{)}\) and radius \(25\).

Wait\textcolor{red}{,} but the problem is asking for the maximum value of \(y\)\textcolor{red}{,} and I have a circle equation\textcolor{red}{.} \textcolor{red}{So}\textcolor{red}{,} the maximum \(y\) would be the \(y\)-coordinate of the highest point on the circle\textcolor{red}{,} which is the center plus the radius.

Wait\textcolor{red}{,} but the center is at \((7\textcolor{red}{,} 24)\), and the radius is \(25\)\textcolor{red}{,} so the highest point would be at \((7\textcolor{red}{,} 24 + 25) = (7\textcolor{red}{,} 49)\).

\textcolor{red}{But}\textcolor{red}{,} let me double-check if this makes sense\textcolor{red}{.} The equation is 
\[
x^2 + y^2 = 14x + 48y,
\]
and I've transformed it into 
\[
(\textcolor{red}{x} - 7\textcolor{red}{)}^2 + (y - 24)^2 = 625,
\]
which is indeed a circle with center \((7, 24\textcolor{red}{)}\) and radius \(25\).

\textcolor{red}{So}\textcolor{red}{,} the maximum \(y\) would be \(24 + 25 = 49\).

\textcolor{red}{But}\textcolor{red}{,} wait\textcolor{red}{,} the problem says ``\textcolor{red}{the} maximum value of \(y\)\textcolor{red}{,}'' and I've found that to be \(49\)\textcolor{red}{.} But let me consider if there are any other constraints or if I've missed something.

\textcolor{red}{Let} me think about the original equation:
\[
x^2 + y^2 = 14x + 48y.
\]
I can rearr\textcolor{red}{ange} it to:
\[
x^2 - 14x + y^2 - 48y = 0.
\]

\end{tcolorbox}
\caption{Example of \textcolor{red}{most shifted tokens} after fine-tuning on long CoT data. All the most shifted tokens are labeled as red.}
    \label{fig: most_shifted_1}
\end{figure*}



\begin{figure*}[htbp]
    \centering
\begin{tcolorbox}[title=Example of \textcolor{red}{most shifted tokens} after fine-tuning on long CoT data (2/2), promptstyle]
\lstset{
    basicstyle=\normalfont\sffamily\footnotesize,
    breaklines=true,
    frame=none,
    columns=fullflexible,
}
\textcolor{red}{Then}\textcolor{red}{,} completing the square:
\[
(x - 7\textcolor{red}{)}^2 - 49 + (y - 24)^2 - 576 = 0,
\]
\[
(x - 7\textcolor{red}{)}^2 + (y - 24)^2 = 625.
\]
Yes\textcolor{red}{,} that's a circle with center \((7, 24\textcolor{red}{)}\) and radius \(25\).

\textcolor{red}{So}\textcolor{red}{,} the maximum \(y\) is indeed \(24 + 25 = 49\).

\textcolor{red}{But}\textcolor{red}{,} just to be thorough\textcolor{red}{,} let's consider if there are any points on the circle where \(y\) is greater than \(49\)\textcolor{red}{.} The center is at \((7, 24\textcolor{red}{)}\), and the radius is \(25\)\textcolor{red}{,} so the highest point is at \((7, 49\textcolor{red}{)}\), and the lowest point would be at \((7\textcolor{red}{,} -1)\).

\textcolor{red}{But}\textcolor{red}{,} the problem is asking for the maximum value of \(y\)\textcolor{red}{,} and I've found that to be \(49\).

Therefore\textcolor{red}{,} the final answer is \(49\).

\bigskip

\textbf{\textcolor{red}{Final} Answer:}

\[
\boxed{49\textcolor{red}{}}
\]
\end{tcolorbox}
\caption{\textcolor{red}{Most shifted tokens} after fine-tuning on long CoT data (2/2). All the most shifted tokens are labeled as red.}
    \label{fig: most_shifted_2}
\end{figure*}






\begin{figure*}[htbp]
    \centering
\begin{tcolorbox}[title=Example of \textcolor{red}{Most Shifted Tokens} after fine-tuning on large model CoT, promptstyle]
\lstset{
    basicstyle=\normalfont\sffamily\footnotesize,
    breaklines=true,
    frame=none,
    columns=fullflexible,
}


\textcolor{red}{To} \textcolor{red}{solve} \textcolor{red}{the} \textcolor{red}{problem}\textcolor{red}{,} \textcolor{red}{ we} \textcolor{red}{ start} \textcolor{red}{ with} \textcolor{red}{ the} \textcolor{red}{ given} equation\textcolor{red}{:}

\[
x^2 + y^2\textcolor{red}{ =} 14x + 48y
\]

\textcolor{red}{First}\textcolor{red}{,} we rearr\textcolor{red}{ange} the equation to bring all terms to one\textcolor{red}{ side}\textcolor{red}{:}

\[
x^2 - 14\textcolor{red}{x} + y^2 - 48\textcolor{red}{y} = 0
\]

\textcolor{red}{Next}\textcolor{red}{,} we complete the square for both \(x\) and \(\textcolor{red}{y}\).

\textcolor{red}{For} \(x\):
\[
x^2 - 14x
\]
\textcolor{red}{To} complete the square\textcolor{red}{,} we add and subtract \(\left(\frac{14}{2}\right)^2 = 49\):
\[
x^2 - 14x + 49 - 49
\]
\[
(\textcolor{red}{x} - 7)^2 - 49
\]

\textcolor{red}{For} \(y\):
\[
y^2 - 48y
\]
\textcolor{red}{To} complete the square, we add and subtract \(\left(\frac{48}{2}\right)^2 = 576\):
\[
y^2 - 48y + 576 - 576
\]
\[
(\textcolor{red}{y} - 24)^2 - 576
\]

\textcolor{red}{Now}\textcolor{red}{,} substitute these completed squares back into the equation:
\[
(\textcolor{red}{x} - 7)^2 - 49 + (y - 24)^2 - 576 = 0
\]
\[
(\textcolor{red}{x} - 7)^2 + (y - 24)^2 - 625 = 0
\]
\[
(\textcolor{red}{x} - 7)^2 + (y - 24)^2 = 625
\]

\textcolor{red}{This} is the equation of a circle with center \((7, 24)\) and radius \(25\).

\textcolor{red}{To} find the maximum value of \(y\), we need to find the highest point on the circle. \textcolor{red}{The} highest point on the circle is directly above the center, which is at \(y = 24 + 25 = 49\).

\textcolor{red}{Thus}\textcolor{red}{,} the maximum value of \(y\) is:
\[
\textcolor{red}{y} = 49
\]

\textcolor{red}{Final} \textcolor{red}{ Answer}\textcolor{red}{:} \(\boxed{49}\)

\end{tcolorbox}
\caption{Example of \textcolor{red}{most shifted tokens} after fine-tuning on large teacher CoT. All the most shifted tokens are labeled as red.}
    \label{fig: most_shifted_3}
\end{figure*}




\section{Examples of Various CoT Data}\label{app:example}
This section demonstrates examples of different CoT data including long CoT, short CoT, large teacher CoT and small teacher CoT. See the following examples.
% \subsection{Long CoT and Short CoT}


\begin{figure*}[htbp]
    \centering
\begin{tcolorbox}[title=Problem, promptstyle]
\lstset{
    basicstyle=\normalfont\sffamily\footnotesize,
    breaklines=true,
    frame=none,
    columns=fullflexible,
}
What is the sum of the positive odd divisors of $60$?
\end{tcolorbox}
    
\end{figure*}



\begin{figure*}[htbp]
    \centering
\begin{tcolorbox}[title=Long CoT Example, promptstyle]
\lstset{
    basicstyle=\normalfont\sffamily\footnotesize,
    breaklines=true,
    frame=none,
    columns=fullflexible,
}
To find the sum of the positive odd divisors of 60, I first need to understand what the divisors of 60 are and then identify which of those are odd.

First, let's find all the positive divisors of 60. To do this, I can start by factoring 60 into its prime factors.

\[
60 = 2^2 \times 3^1 \times 5^1
\]

Now, to find all the positive divisors, I can take all possible products of these prime factors raised to their possible exponents. For 2, the exponents can be \(0, 1,\) or \(2\); for 3, \(0\) or \(1\); and for 5, \(0\) or \(1\). Listing them out:

\[
\begin{aligned}
&1 = 2^0 \times 3^0 \times 5^0, \\
&2 = 2^1 \times 3^0 \times 5^0, \\
&4 = 2^2 \times 3^0 \times 5^0, \\
&3 = 2^0 \times 3^1 \times 5^0, \\
&6 = 2^1 \times 3^1 \times 5^0, \\
&12 = 2^2 \times 3^1 \times 5^0, \\
&5 = 2^0 \times 3^0 \times 5^1, \\
&10 = 2^1 \times 3^0 \times 5^1, \\
&20 = 2^2 \times 3^0 \times 5^1, \\
&15 = 2^0 \times 3^1 \times 5^1, \\
&30 = 2^1 \times 3^1 \times 5^1, \\
&60 = 2^2 \times 3^1 \times 5^1.
\end{aligned}
\]

So, the positive divisors of 60 are:
\[
1, 2, 3, 4, 5, 6, 10, 12, 15, 20, 30, 60.
\]

Now, I need to identify which of these are odd. Odd numbers are not divisible by 2, so I can eliminate any divisors that are even.

The odd divisors are:
\[
1, 3, 5, 15.
\]

Now, I need to find the sum of these odd divisors:
$1 + 3 + 5 + 15 = 24$.
Alternatively, using the prime factorization approach: Since the odd divisors are those that do not include the factor \(2\), I can ignore the 2's in the prime factorization. The remaining odd part is:

\[
3^1 \times 5^1.
\]

The sum of the divisors of this odd part is:

\[
(1 + 3)(1 + 5) = 4 \times 6 = 24.
\]

This matches the previous result.

Therefore, for \( 60 = 2^2 \times 3^1 \times 5^1 \), the sum of the odd divisors is:

\[
\boxed{24}.
\]
\end{tcolorbox}
\caption{Long CoT Example}
    \label{fig: long cot example}
\end{figure*}



\begin{figure*}[htbp]
    \centering
\begin{tcolorbox}[title=Short CoT Example, promptstyle]
\lstset{
    basicstyle=\normalfont\sffamily\footnotesize,
    breaklines=true,
    frame=none,
    columns=fullflexible,
}

To find the sum of the positive odd divisors of $60$, we first need to factorize $60$ into its prime factors. The prime factorization of $60$ is:

\[
60 = 2^2 \times 3 \times 5
\]

Since we are only interested in the odd divisors, we can ignore the factor of $2$. Therefore, we only consider the factors $3$ and $5$. The odd divisors of $60$ are the divisors of $3 \times 5 = 15$. The divisors of $15$ are \(1, 3, 5,\) and \(15\).

Next, we sum these divisors:

\[
1 + 3 + 5 + 15 = 24
\]

Therefore, the sum of the positive odd divisors of $60$ is:

\[
\boxed{24}
\]

\end{tcolorbox}
    \label{fig: short cot example}
    \caption{Short CoT example}
\end{figure*}




\begin{figure*}[htbp]
    \centering
\begin{tcolorbox}[title=Problem, promptstyle]
\lstset{
    basicstyle=\normalfont\sffamily\footnotesize,
    breaklines=true,
    frame=none,
    columns=fullflexible,
}
Find the greatest \( a \) such that
\[
\frac{7\sqrt{(2a)^2+(1)^2}-4a^2-1}{\sqrt{1+4a^2}+3}=2.
\]
\end{tcolorbox}
\end{figure*}



\begin{figure*}[htbp]
    \centering
\begin{tcolorbox}[title=Large Teacher CoT Example (1/2), promptstyle]
\lstset{
    basicstyle=\normalfont\sffamily\footnotesize,
    breaklines=true,
    frame=none,
    columns=fullflexible,
}

To solve the equation \(\frac{7\sqrt{(2a)^2 + 1^2} - 4a^2 - 1}{\sqrt{1 + 4a^2} + 3} = 2\), we start by simplifying the expression inside the square root in the numerator. The equation becomes:

\[
\frac{7\sqrt{4a^2 + 1} - 4a^2 - 1}{\sqrt{1 + 4a^2} + 3} = 2
\]

Let's denote \(x = \sqrt{1 + 4a^2}\). Then the equation transforms into:

\[
\frac{7x - 4a^2 - 1}{x + 3} = 2
\]

Next, we clear the fraction by multiplying both sides by \(x + 3\):

\[
7x - 4a^2 - 1 = 2(x + 3)
\]

Expanding and simplifying the right side, we get:

\[
7x - 4a^2 - 1 = 2x + 6
\]

Rearranging terms to isolate \(x\) and \(a^2\), we have:

\[
7x - 2x - 4a^2 - 1 - 6 = 0 \implies 5x - 4a^2 - 7 = 0 \implies 5x = 4a^2 + 7 \implies x = \frac{4a^2 + 7}{5}
\]

Recall that \(x = \sqrt{1 + 4a^2}\). Therefore, we have:

\[
\sqrt{1 + 4a^2} = \frac{4a^2 + 7}{5}
\]

To eliminate the square root, we square both sides:

\[
1 + 4a^2 = \left(\frac{4a^2 + 7}{5}\right)^2
\]

Expanding the right side, we get:

\[
1 + 4a^2 = \frac{(4a^2 + 7)^2}{25} = \frac{16a^4 + 56a^2 + 49}{25}
\]

Multiplying both sides by 25 to clear the denominator, we obtain:

\[
25 + 100a^2 = 16a^4 + 56a^2 + 49
\]

\end{tcolorbox}
\end{figure*}



\begin{figure*}[htbp]
    \centering
\begin{tcolorbox}[title=Large Teacher CoT Example (2/2), promptstyle]
\lstset{
    basicstyle=\normalfont\sffamily\footnotesize,
    breaklines=true,
    frame=none,
    columns=fullflexible,
}
Rearranging all terms to one side, we get a standard polynomial equation:

\[
16a^4 + 56a^2 + 49 - 100a^2 - 25 = 0 \implies 16a^4 - 44a^2 + 24 = 0
\]

Let \(y = a^2\). Then the equation becomes a quadratic in \(y\):

\[
16y^2 - 44y + 24 = 0
\]

We solve this quadratic equation using the quadratic formula \(y = \frac{-b \pm \sqrt{b^2 - 4ac}}{2a}\), where \(a = 16\), \(b = -44\), and \(c = 24\):

\[
y = \frac{44 \pm \sqrt{(-44)^2 - 4 \cdot 16 \cdot 24}}{2 \cdot 16} = \frac{44 \pm \sqrt{1936 - 1536}}{32} = \frac{44 \pm \sqrt{400}}{32} = \frac{44 \pm 20}{32}
\]

This gives us two solutions for \(y\):

\[
y = \frac{64}{32} = 2 \quad \text{and} \quad y = \frac{24}{32} = \frac{3}{4}
\]

Since \(y = a^2\), we have \(a^2 = 2\) or \(a^2 = \frac{3}{4}\). The corresponding values for \(a\) are \(a = \pm \sqrt{2}\) and \(a = \pm \frac{\sqrt{3}}{2}\). The greatest value of \(a\) is \(\sqrt{2}\).

Thus, the greatest \(a\) that satisfies the equation is:

\[
\boxed{\sqrt{2}}
\]
\end{tcolorbox}
\end{figure*}



\begin{figure*}[htbp]
    \centering
\begin{tcolorbox}[title=Small Teacher CoT Example (1/2), promptstyle]
\lstset{
    basicstyle=\normalfont\sffamily\footnotesize,
    breaklines=true,
    frame=none,
    columns=fullflexible,
}

To solve the equation \(\frac{7\sqrt{(2a)^2+(1)^2}-4a^2-1}{\sqrt{1+4a^2}+3}=2\), we start by simplifying the expression inside the square root and the equation itself.

First, simplify the expression under the square root in the numerator:
\[
(2a)^2 + 1^2 = 4a^2 + 1
\]
So the equation becomes:
\[
\frac{7\sqrt{4a^2 + 1} - 4a^2 - 1}{\sqrt{4a^2 + 1} + 3} = 2
\]

Next, let \( x = \sqrt{4a^2 + 1} \). Then the equation transforms into:
\[
\frac{7x - 4a^2 - 1}{x + 3} = 2
\]

Since \( x = \sqrt{4a^2 + 1} \), we can express \( 4a^2 \) in terms of \( x \):
\[
4a^2 = x^2 - 1
\]

Substitute \( 4a^2 = x^2 - 1 \) into the equation:
\[
\frac{7x - (x^2 - 1) - 1}{x + 3} = 2
\]
Simplify the numerator:
\[
7x - x^2 + 1 - 1 = 7x - x^2
\]
So the equation becomes:
\[
\frac{7x - x^2}{x + 3} = 2
\]

Multiply both sides by \( x + 3 \) to clear the denominator:
\[
7x - x^2 = 2(x + 3)
\]
Expand and simplify:
\[
7x - x^2 = 2x + 6
\]
Rearrange all terms to one side:
\[
-x^2 + 7x - 2x - 6 = 0
\]
\[
-x^2 + 5x - 6 = 0
\]
Multiply through by -1 to make the leading coefficient positive:
\[
x^2 - 5x + 6 = 0
\]

\end{tcolorbox}
    
\end{figure*}




\begin{figure*}[htbp]
    \centering
\begin{tcolorbox}[title=Small Teacher CoT Example (2/2), promptstyle]
\lstset{
    basicstyle=\normalfont\sffamily\footnotesize,
    breaklines=true,
    frame=none,
    columns=fullflexible,
}

Factor the quadratic equation:
\[
(x - 2)(x - 3) = 0
\]

So, the solutions are:
\[
x = 2 \quad \text{or} \quad x = 3
\]

Recall that \( x = \sqrt{4a^2 + 1} \). We need to check both solutions:

1. For \( x = 2 \):
\[
\sqrt{4a^2 + 1} = 2
\]
Square both sides:
\[
4a^2 + 1 = 4
\]
\[
4a^2 = 3
\]
\[
a^2 = \frac{3}{4}
\]
\[
a = \pm \frac{\sqrt{3}}{2}
\]

2. For \( x = 3 \):
\[
\sqrt{4a^2 + 1} = 3
\]
Square both sides:
\[
4a^2 + 1 = 9
\]
\[
4a^2 = 8
\]
\[
a^2 = 2
\]
\[
a = \pm \sqrt{2}
\]

The greatest value of \( a \) among the solutions is \( \sqrt{2} \).

Thus, the final answer is:
\[
\boxed{\sqrt{2}}
\]
\end{tcolorbox}
    
\end{figure*}




\begin{figure*}[htbp]
    \centering
\begin{tcolorbox}[title=Problem, promptstyle]
\lstset{
    basicstyle=\normalfont\sffamily\footnotesize,
    breaklines=true,
    frame=none,
    columns=fullflexible,
}
What is the greatest common divisor of 128, 144 and 480?
\end{tcolorbox}
    
\end{figure*}



\begin{figure*}[htbp]
    \centering
\begin{tcolorbox}[title=Large Teacher CoT Example, promptstyle]
\lstset{
    basicstyle=\normalfont\sffamily\footnotesize,
    breaklines=true,
    frame=none,
    columns=fullflexible,
}
To find the greatest common divisor (GCD) of 128, 144, and 480, we can use the prime factorization method.

\begin{enumerate}
    \item \textbf{Prime Factorization:}
    \begin{align*}
        128 &= 2^7, \\
        144 &= 2^4 \times 3^2, \\
        480 &= 2^5 \times 3 \times 5.
    \end{align*}

    \item \textbf{Identify the common prime factors:}\\
    The common prime factor among 128, 144, and 480 is \(2\).

    \item \textbf{Determine the lowest power of the common prime factor:}\\
    The lowest power of \(2\) in the factorizations is \(2^4\).

    \item \textbf{Calculate the GCD:}\\
    The GCD is:
    \[
    2^4 = 16.
    \]
\end{enumerate}

Therefore, the greatest common divisor of 128, 144, and 480 is:
\[
\boxed{16}
\]
\end{tcolorbox}
    
\end{figure*}


\begin{figure*}[htbp]
    \centering
\begin{tcolorbox}[title=Small Teacher CoT Example, promptstyle]
\lstset{
    basicstyle=\normalfont\sffamily\footnotesize,
    breaklines=true,
    frame=none,
    columns=fullflexible,
}


To find the greatest common divisor (GCD) of 128, 144, and 480, we start by finding the prime factorization of each number.

The prime factorization of 128 is:
\[
\begin{aligned}
128 \div 2 &= 64 \\
64 \div 2 &= 32 \\
32 \div 2 &= 16 \\
16 \div 2 &= 8 \\
8 \div 2 &= 4 \\
4 \div 2 &= 2 \\
2 \div 2 &= 1
\end{aligned}
\]
Thus, 
\[
128 = 2^7.
\]

The prime factorization of 144 is:
\[
\begin{aligned}
144 \div 2 &= 72 \\
72 \div 2 &= 36 \\
36 \div 2 &= 18 \\
18 \div 2 &= 9 \\
9 \div 3 &= 3 \\
3 \div 3 &= 1
\end{aligned}
\]
Thus,
\[
144 = 2^4 \times 3^2.
\]

The prime factorization of 480 is:
\[
\begin{aligned}
480 \div 2 &= 240 \\
240 \div 2 &= 120 \\
120 \div 2 &= 60 \\
60 \div 2 &= 30 \\
30 \div 2 &= 15 \\
15 \div 3 &= 5 \\
5 \div 5 &= 1
\end{aligned}
\]
Thus,
\[
480 = 2^5 \times 3 \times 5.
\]

Now, we identify the common prime factors. The factorizations are:
\[
128 = 2^7, \quad 144 = 2^4 \times 3^2, \quad 480 = 2^5 \times 3 \times 5.
\]
The only common prime factor among all three numbers is \(2\).

Next, we determine the smallest power of the common prime factor. The lowest power of \(2\) among these factorizations is \(2^4\).

Therefore, the GCD of 128, 144, and 480 is:
\[
\boxed{16}
\]

\end{tcolorbox}
    
\end{figure*}







\begin{figure*}[htbp]
    \centering
\begin{tcolorbox}[title=Problem, promptstyle]
\lstset{
    basicstyle=\normalfont\sffamily\footnotesize,
    breaklines=true,
    frame=none,
    columns=fullflexible,
}
Rounded to 2 decimal places, what is $\frac{7}{9}$?
\end{tcolorbox}
    
\end{figure*}



\begin{figure*}[htbp]
    \centering
\begin{tcolorbox}[title=Large Teacher CoT Example, promptstyle]
\lstset{
    basicstyle=\normalfont\sffamily\footnotesize,
    breaklines=true,
    frame=none,
    columns=fullflexible,
}
To solve the problem, we need to divide 7 by 9 and then round the result to two decimal places.

\[
\frac{7}{9} = 0.777\ldots
\]

Rounded to two decimal places, this becomes 0.78.

Final Answer: \(\boxed{0.78}\)
\end{tcolorbox}
    
\end{figure*}


\begin{figure*}[htbp]
    \centering
\begin{tcolorbox}[title=Small Teacher CoT Example, promptstyle]
\lstset{
    basicstyle=\normalfont\sffamily\footnotesize,
    breaklines=true,
    frame=none,
    columns=fullflexible,
}
To solve the problem of finding the value of \(\frac{7}{9}\) rounded to 2 decimal places, we can follow these steps:

\begin{enumerate}
    \item \textbf{Perform the division}: Calculate \(\frac{7}{9}\).
    \item \textbf{Convert to decimal}: Perform the division to get the decimal equivalent.
    \item \textbf{Round to 2 decimal places}: Round the decimal to 2 decimal places.
\end{enumerate}

Let's start with the division:

\[
\frac{7}{9} \approx 0.7777\ldots
\]

Next, we need to round this decimal to 2 decimal places. The digit in the third decimal place is 7, which is greater than or equal to 5. Therefore, we round the second decimal place up from 7 to 8.

So, \(0.7777\ldots\) rounded to 2 decimal places is \(0.78\).

The final answer is:

\[
\boxed{0.78}
\]
\end{tcolorbox}
    
\end{figure*}





