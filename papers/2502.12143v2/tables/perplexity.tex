\begin{table}[ht]
\centering
\resizebox{0.5\textwidth}{!}{
    \begin{tabular}{l|ccc|ccc}
    \toprule
    \makecell{Model \\ Size} & \makecell{Long \\ CoT} & \makecell{Short \\ CoT} & Diff. & \makecell{Large \\ Teacher \\ CoT} & \makecell{Small \\ Teacher \\ CoT} & Diff. \\
    \midrule
    0.5B & \textbf{2.238} & 1.278 & 0.959 & \textbf{1.247} & 1.218 & 0.029 \\
    1.5B & \textbf{1.969} & 1.226 & 0.743 & \textbf{1.205} & 1.178 & 0.026 \\
    3B   & \textbf{1.964} & 1.247 & 0.717 & \textbf{1.226} & 1.156 & 0.070 \\
    7B   & \textbf{1.923} & 1.223 & 0.700 & \textbf{1.198} & 1.181 & 0.017 \\
    14B  & \textbf{1.902} & 1.218 & 0.684 & \textbf{ 1.199 } & 1.189 & 0.009 \\
    32B  & \textbf{1.266} & 1.051 & 0.215 & \textbf{1.054} & 1.051 & 0.002 \\
    \bottomrule
    \end{tabular}
}
\caption{ Perplexity (\texttt{PPL}) of various CoT training data across student models in the Qwen 2.5 family with different sizes.
Long CoT data and large teacher CoT data generally have higher \texttt{PPL} than short CoT and small teacher CoT, respectively. These gaps gradually narrows down as model size increases, indicating larger student models better match the distribution of complex reasoning data such as long CoT and large teacher CoT.}
\label{ppl}
\end{table}