
% \pdfoutput=1

\documentclass[11pt]{article}


\usepackage[preprint]{acl}


\usepackage{times}
\usepackage{latexsym}
\usepackage[T1]{fontenc}
\usepackage[utf8]{inputenc}
\usepackage{microtype}
\usepackage{inconsolata}
\usepackage{graphicx}
\usepackage{microtype}
\usepackage{tabularx}
% \usepackage[table]{xcolor}
% \usepackage{xcolor}
\usepackage{amsmath}
\usepackage{amssymb}
\usepackage{mathtools}
\usepackage{amsthm}
\usepackage{multirow}
\usepackage{makecell}
\usepackage{booktabs}
\usepackage{array}
\usepackage{longtable}
\usepackage{subcaption}
\usepackage{fontawesome}
\usepackage{centernot}
\usepackage{pgf} 
\usepackage{tikz} 
\usepackage{stfloats}

\theoremstyle{plain}
\newtheorem{theorem}{Theorem}[section]
\newtheorem{proposition}[theorem]{Proposition}
\newtheorem{lemma}[theorem]{Lemma}
\newtheorem{corollary}[theorem]{Corollary}
\theoremstyle{definition}
\newtheorem{definition}[theorem]{Definition}
\newtheorem{assumption}[theorem]{Assumption}
\theoremstyle{remark}
\newtheorem{remark}[theorem]{Remark}


\usepackage[most,skins,theorems]{tcolorbox}
\tcbset{
  aibox/.style={
    width=\linewidth,
    top=8pt,
    bottom=4pt,
    colback=blue!6!white,
    colframe=black,
    colbacktitle=black,
    enhanced,
    center,
    attach boxed title to top left={yshift=-0.1in,xshift=0.15in},
    boxed title style={boxrule=0pt,colframe=white,},
  }
}
\newtcolorbox{AIbox}[2][]{aibox,title=#2,#1}







\newenvironment{findingBox}[2]{%
	\begin{tcolorbox}[
colframe=black!80,
colback=gray!10,
 boxrule=.5pt,
 left=1pt,
 right = 1pt,
 top=0pt,
 bottom=0pt,
 size=small,
 fonttitle=\bfseries,
coltitle=black,
boxrule=0.4mm,
arc=2mm
 ]{\textbf{Observation #1:} #2} 
}{%
	\end{tcolorbox}
}


\tcbset{
    promptstyle/.style={
        enhanced,
        colback=white,
        colframe=black,
        colbacktitle=gray!20,
        width=0.98\linewidth, 
        coltitle=black,
        rounded corners,
        sharp corners=north,
        boxrule=0.5pt,
        drop shadow=black!50!white,
        attach boxed title to top left={
            xshift=-2mm,
            yshift=-2mm
        },
        title code={%
            \begin{minipage}{0.8\linewidth}
                \centering\thetitle
            \end{minipage}
        },
        boxed title style={
            rounded corners,
            size=small,
            colback=gray!20,
        },
        fonttitle=\normalsize\bfseries,
        before upper={\parindent15pt}
    }
}


\usepackage{ifthen} 
\newcommand{\coloredAvg}[1]{%
  \pgfmathsetmacro{\val}{#1}%
  \pgfmathsetmacro{\absval}{abs(\val)}%
  \pgfmathsetmacro{\perc}{min(100,100*\absval/3)}%
  \ifthenelse{\lengthtest{\val pt > 0pt}}%
    {\cellcolor{red!\perc}#1}%
    {\ifthenelse{\lengthtest{\val pt < 0pt}}%
      {\cellcolor{green!\perc}#1}%
      {\cellcolor{white}#1}}%
}



\tcbset{
    userstyle/.style={
        enhanced,
        colback=white,
        colframe=black,
        colbacktitle=gray!20,
        coltitle=black,
        rounded corners,
        sharp corners=north,
        boxrule=0.5pt,
        drop shadow=black!50!white,
        attach boxed title to top left={
            xshift=-2mm,
            yshift=-2mm
        },
        boxed title style={
            rounded corners,
            size=small,
            colback=gray!20
        }
    },
    replystyleg/.style={
        enhanced,
        colback=green!15,
        colframe=black,
        colbacktitle=green!30,
        coltitle=black,
        boxrule=0.5pt,
        drop shadow=black!50!white,
        rounded corners,
        sharp corners=north,
        attach boxed title to top right={
            xshift=-2mm,
            yshift=-2mm
        },
        boxed title style={
            rounded corners,
            size=small,
            colback=green!40
        }
    },
    replystyler/.style={
        enhanced,
        colback=red!15,
        colframe=black,
        colbacktitle=red!40,
        coltitle=black,
        boxrule=0.5pt,
        drop shadow=black!50!white,
        rounded corners,
        sharp corners=north,
        attach boxed title to top right={
            xshift=-2mm,
            yshift=-2mm
        },
        boxed title style={
            rounded corners,
            size=small,
            colback=red!40
        }
    }
}



\newtcolorbox{prompt}[2][]{
    colback=white,
    colframe=gray!45,
    fonttitle=\bfseries,
    coltitle=black,
    sharp corners,
    title=#2,
    #1
}

\tcbset{
    promptstyle/.style={
        enhanced,
        colback=white,
        colframe=black,
        colbacktitle=gray!20,
        coltitle=black,
        rounded corners,
        sharp corners=north,
        boxrule=0.5pt,
        drop shadow=black!50!white,
        attach boxed title to top left={
            xshift=-2mm,
            yshift=-2mm
        },
        boxed title style={
            rounded corners,
            size=small,
            colback=gray!20
        }
    },
    replystyleg/.style={
        enhanced,
        colback=green!15,
        colframe=black,
        colbacktitle=green!30,
        coltitle=black,
        boxrule=0.5pt,
        drop shadow=black!50!white,
        rounded corners,
        sharp corners=north,
        attach boxed title to top right={
            xshift=-2mm,
            yshift=-2mm
        },
        boxed title style={
            rounded corners,
            size=small,
            colback=green!40
        }
    },
    replystyler/.style={
        enhanced,
        colback=red!15,
        colframe=black,
        colbacktitle=red!40,
        coltitle=black,
        boxrule=0.5pt,
        drop shadow=black!50!white,
        rounded corners,
        sharp corners=north,
        attach boxed title to top right={
            xshift=-2mm,
            yshift=-2mm
        },
        boxed title style={
            rounded corners,
            size=small,
            colback=red!40
        }
    }
}

\newtcolorbox{promptbox}[1][]{
    promptstyle,
    title=Prompt,
    #1
}






\title{Small Models Struggle to Learn from Strong Reasoners}

\author{
\textbf{Yuetai Li}\textsuperscript{$\clubsuit$} \;\;\;  
\textbf{Xiang Yue}\textsuperscript{$\diamondsuit$} \;\;\; 
\textbf{Zhangchen Xu}\textsuperscript{$\clubsuit$} \;\;\;  
\textbf{Fengqing Jiang}\textsuperscript{$\clubsuit$} \;\;\;
\textbf{Luyao Niu}\textsuperscript{$\clubsuit$} \;\;\; \\ 
\textbf{Bill Yuchen Lin}\textsuperscript{$\clubsuit$} \;\;\;
\textbf{Bhaskar Ramasubramanian}\textsuperscript{$\spadesuit$} \; \;\;
\textbf{Radha Poovendran}\textsuperscript{$\clubsuit$}\\
  \textsuperscript{$\clubsuit$}University of Washington \; 
  \textsuperscript{$\diamondsuit$}Carnegie Mellon University \;
  \textsuperscript{$\spadesuit$}Western Washington University \\
  \texttt{\{yuetaili,zxu9,fqjiang,luyaoniu,byuchenl,rp3\}@uw.edu},\\
  \texttt{xyue2@andrew.cmu.edu}, \texttt{ramasub@wwu.edu} \vspace{1em} \\
   \textbf{Huggingface}: \url{https://huggingface.co/UWNSL} \\
   \textbf{Project Page}: \url{https://small-model-gap.github.io/}   
}


\begin{document}
\maketitle
\begin{abstract}

Large language models (LLMs) excel in complex reasoning tasks, and distilling their reasoning capabilities into smaller models has shown promise. However, we uncover an interesting phenomenon, which we term the \textit{Small Model Learnability Gap}: small models ($\leq$3B parameters) do not consistently benefit from long chain-of-thought (CoT) reasoning or distillation from larger models. Instead, they perform better when fine-tuned on shorter, simpler reasoning chains that better align with their intrinsic learning capacity. To address this, we propose Mix Distillation, a simple yet effective strategy that balances reasoning complexity by combining long and short CoT examples or reasoning from both larger and smaller models. Our experiments demonstrate that Mix Distillation significantly improves small model reasoning performance compared to training on either data alone. These findings highlight the limitations of direct strong model distillation and underscore the importance of adapting reasoning complexity for effective reasoning capability transfer.

 
\end{abstract}


% 头图修改
% 有些query是不必要检索的/有些knowledge是需要检索的
% 基于上下文的query而不是knowledge的
% 人类思考过程dynamic knowledge decision 标题,已解决

\section{Introduction}

% 大模型已经展现出remarkable能力。
% 然而,由于模型容量和能力的限制,模型经常生成错误信息
% 一个promising的方法来解决幻觉就是RAG
Large Language Models (LLMs) have gained increasing prominence in the field of artificial intelligence.
However, limited by the capacity and capabilities of LLM, it still suffers from severe factual hallucination problems due to the lack of intrinsic knowledge~\cite{zhang2023hallucination}.
Retrieval-augmented generation (RAG) has been proposed as a promising paradigm to address this issue by integrating relevant factual information from external knowledge bases or search engines, thereby improving the factual accuracy and reliability of LLM outputs~\cite{zhao2024retrieval}.

% Unfortunately,检索并不是always有必要的,
% 一方面,额外的检索时浪费推理时间
% 另一方面,引入noise甚至错误信息导致错误答案
However, retrieval is not always helpful, and employing it for every instruction proves sub-optimal~\cite{chen2023understanding,tan2024blinded}.
When confronted with tasks that do not necessitate external knowledge, the introduction of RAG can lead to increased computational overhead during inference and may introduce potentially irrelevant text, compromising the factuality of LLMs' outputs.
% 
Specifically, previous research has demonstrated that by eliminating unnecessary retrieval operations, the efficiency of Retrieval-Augmented Generation can be significantly enhanced.
% 
Furthermore, the injection of superfluous information may adversely affect the model's performance, causing the model to struggle to maintain focus on the core task~\cite{liu-etal-2024-lost}.
% 
More critically, studies have revealed that LLMs face challenges in handling noisy or counterfactual text~\cite{chen2024benchmarking}. In scenarios where retrieved information is highly relevant to the query but contains inaccuracies or false statements, LLMs often struggle to discriminate it. This limitation can result in the model being misled, potentially leading to incorrect responses and undermining the purpose of knowledge augmentation~\cite{bian2024influence}.


\begin{figure}[t]
    \centering
    \includegraphics[width=0.98\linewidth]{figure/fig1.pdf}
    \caption{Comparison of traditional RAG and our DeepRAG.}
    \label{fig:comparison}
\end{figure}


% from human perspecitve, 人类并不会对会的问题搜索,而只会针对不会的问题搜索。
% However,仅仅依靠模型自身的输出来判断是否需要检索是不可靠的。
% Therefore, we aim to guide LLM's awareness of retrieval need based on a self-calibration process.
% In this paper, we propose Adaptive Inference-time Compute Generation(DeepRAG),a simple yet effective method desinged to 让模型认识到自我知识边界。
% DeepRAG的主要思想是通过对模型 关于回答query所需要的知识是否需要检索 这一行为进行校准。
% 通过让模型意识到自我知识边界,能够有效的提高检索效率、规避不必要的文档噪声

From a human perspective, individuals typically search for information only when encountering unfamiliar questions, relying on their existing knowledge for known topics. 
However, LLMs often lack this nuanced self-awareness of their knowledge boundaries, making it unreliable to depend solely on the model's own output to determine the necessity of retrieval~\cite{yin2023large}.
% 
In this paper, we propose \textbf{DeepRAG}, a simple yet effective method designed to guide LLM's awareness of retrieval needs based on a self-calibration process. 
% 
The core idea of DeepRAG is to calibrate the model's behavior regarding whether knowledge retrieval is necessary to answer a given query.
% 
By enabling the model to recognize its own knowledge boundaries, DeepRAG effectively improves retrieval efficiency and mitigates unnecessary document noise. 



% 然而,实现上述过程具有挑战性,xxx 已经表明仅仅根据query去检索知识通常无法检索到所有回答所需的信息,following xxx,我们将query分解成多个subquery,对于each subquery,我们让模型adaptively选择是否进行检索。
% 为了校准模型对知识边界的认知,我们设计了二叉树搜索方法,对于每个subquery,我们分别探索其在是否使用检索对于推理结果的影响。基于这个方法,我们首先生成数据让llm通过模仿学习到“子问题生成-是否检索-中间答案”的模式。接着,我们使用chain of calibration的方法校准模型对内部知识的认知,从而更加准确的决定是否需要检索。
% 这样,我们的方法不仅可以通过问题分解的过程简化每个子问题从而简化检索的难度,adaptively 和 accurately 根据每个子问题结合内部知识决定是否要检索。
% 此外,由于过程中的所有阶段都仅使用数据的输入和answer,不需要任何额外的监督,因此可以基于任何有监督数据集来校准llm在rag场景下的知识边界。


Unfortunately, achieving the above process is challenging. 
% 
Previous research has shown that retrieving knowledge based solely on the initial query often fails to gather all the necessary information for a comprehensive answer. 
Following \citet{yue2024inference}, we decompose the query into multiple subqueries, allowing the model to adaptively decide whether retrieval is necessary for each subquery iteratively.
To calibrate the model's perception of its knowledge boundaries, we designed a binary tree search method. For each subquery, we explore the impact of retrieval on the reasoning process. 
Based on it, we first synthesize data to make LLM focusing on the pattern of ``subquery generation - retrieval decision - intermediate answer''. 
Subsequently, we employ a chain of calibration approach to enhance model's awareness of its internal knowledge, enabling more accurate decisions on retrieval necessity.
% 
In this way, our method can not only simplify the retrieval process but also allow for adaptive and accurate retrieval decisions based on internal knowledge for each subquery. 
% 
Furthermore, because all phases in the procedure rely solely on input data and answers without additional supervision, our method can calibrate LLM's knowledge boundaries in RAG scenarios using any supervised dataset. 


We conduct experiments on five open-domain QA to validate the effectiveness of DeepRAG, including HotpotQA, 2WikiMultihopQA, and PopQA for multi-hop factual QA, CAG for time-sensitive QA, and WebQuestions for heterogeneous knowledge base QA.
% 
Experiments show that DeepRAG can significantly improve both the accuracy and efficiency of retrieval-augmented generation.

In summary, the contributions are as follows:


% 1. 提出一种new方法that enhance RAG via 让模型自适应的决定检索的时机 in inference time.
% 2. 提出了一种自动化框架可以基于任何有监督数据集校准RAG场景下的self-awareness.
% 3. 实验效果导致accurate efficient by exploring konwledge boundary

\begin{itemize}
    \item We propose a new framework that enhances Retrieval-Augmented Generation by enabling LLMs to adaptively determine the necessity of retrieval during inference time, thereby improving both accuracy and efficiency.
    \item We introduce an automated framework that calibrates the self-awareness of LLMs in RAG scenarios that can be applied to any supervised dataset without additional supervision.
    \item Experimental on multi-hop QA, time-sensitive QA, and heterogeneous knowledge base QA confirm that DeepRAG can significantly enhance both the accuracy and efficiency of retrieval-augmented generation with great generalization ability by exploring their knowledge boundary.
\end{itemize}


\begin{figure*}[htbp]
    \centering
    \includegraphics[width=0.98\linewidth]{figure/srag.pdf}
    \caption{An overview of DeepRAG, our framework comprises three key components: (1) Binary Tree Search, (2) Imitation Learning, and (3) Chain of Calibration. Given a set of supervised datasets, we first employ binary tree search to synthesize data for imitation learning, enabling the model to learn retrieval patterns. Subsequently, we use binary tree search to construct preference data for further calibrating the LLM's awareness of its knowledge boundaries.}
    \label{fig:main}
\end{figure*}



\section{Preliminaries}\label{sec:preliminaries}



%We denote by $(\Ac(x_\Ac),\Bc(x_\Bc))(z)$ a random execution of $\pi$ with private inputs $(x_\Ac,y_\Ac)$, and common input $z$.

%\Jnote{Move to DP}
% At the end of such an execution, the protocol outputs a public transcript denoted by the random variable $\trans_\pi(x_\Ac,x_\Ac,z)$ we denotes the common as $\out(\trans_\pi(x_\Ac,x_\Ac,z)$, and each party $\Pc \in \set{\Ac,\Bc}$ obtains his view denoted $\view^\Pc_\pi(x_\Ac,x_\Bc,z)$, which may also contain a ``local output'' \Jnote{Local} $\out^\Pc(x_\Ac,x_\Bc,z)$ (if the protocol specifies such an output). \Jnote{Common output, and parties output}


\subsection{Distributions and Random Variables}\label{sec:prelim:dist}
The support of a distribution $P$ over a finite set $\cS$ is defined by $\Supp(P) \eqdef \set{x\in \cS: P(x)>0}$. For a distribution or a random variable $D$, let $d\from D$ denote that $d$ was sampled according to $D$. Similarly,  for a set $\cS$, let $x \from \cS$ denote that $x$ is drawn uniformly from $\cS$, and denote by $\cU_{\cS}$ the uniform distribution over $\cS$. For a finite set $\cX$ and a distribution $C_X$ over $\cX$, we use the capital letter $X$ to denote the random variable that takes values in $\cX$ and is sampled according to $C_X$. The {\sf statistical distance} (\aka {\sf~variation distance}) of two distributions $P$ and $Q$ over a discrete domain $\cX$ is defined by $\sdist{P}{Q} \eqdef \max_{\cS\subseteq \cX} \size{P(\cS)-Q(\cS)} = \frac{1}{2} \sum_{x \in \cS}\size{P(x)-Q(x)}$. 
For a vector $x = (x_1,\ldots,x_n)$ and index $i\in [n]$, we let $x_{-i} = (x_1,\ldots,x_{i-1},x_{i+1},\ldots,x_n)$ and $x^{(i)} = (x_1,\ldots,x_{i-1}, -x_i, x_{i+1},\ldots,x_n)$, for a set $\cS \subseteq [n]$ we let $x_{\cS} = (x_i)_{i \in \cS}$ and $x_{-\cS} = (x_i)_{i \in [n]\setminus \cS}$, and for a vector $r \in \zo^n$ we let $x_r = (x_i)_{\set{i \colon r_i = 1}}$ and $x_{-r} = (x_i)_{\set{i \colon r_i = 0}}$.

%For $n \in \N$ we let $U_n$ be the uniform distribution over $\oo^n$, and let $S_n$ be the distribution induces by the sum of $n$ i.i.d.\ random variables, each is distributed according to $U_1$. Let $\cN(0,1)$ be the standard normal distribution.
%For a distribution $\cD$ and a function $f$, we define by $f(\cD)$ the distribution that is induced by the output of $f(x)$ for $x \from \cD$. 





% \begin{theorem}[\cite{McGregorMPRTV10}]\label{thm:sv-extracotr}
% 	\Enote{Remove if not needed}
% 	There is a constant $c$ to make the following holds. Let $X$ be an $\alpha$-SV source on $\{0,1\}^n$, let $Y$ be a source on $\{0,1\}^n$ with min-entropy at least $\beta n$ (independent from $X$), and let $Z=\ip{X,Y}\mbox{mod m}$ for some $m\in\mathbb{N}$. Then for every $\delta\in[0,1]$, the random variable $(Y,Z)$ is $\delta$-close to $(Y,U)$ where $U$ is uniform on $\mathbb{Z}_m$ and independent of $Y$, provided that
% 	$$
% 	n\geq c\cdot\frac{m^2}{\alpha\beta}\cdot\log(\frac{m}{\beta})\cdot\log(\frac{m}{\delta}).
% 	$$
% \end{theorem}



\Enote{I removed the definition of DP since it already appears in the intro}
\remove{
\subsection{Differential Privacy}\label{sec:prelim:DP}
We use the following standard definition of (information theoretic) differential privacy, due to \citet{DMNS06}. For notational convenience, we focus on databases over $\oo$.
\begin{definition}[Differentially private mechanisms]\label{def:mech}
	A randomized function $f\colon\oo^n\mapsto \zs$ is an {\sf $n$-size, $(\eps,\delta)$-differentially private mechanism} (denoted $(\eps,\delta)$-\DP) if for every neighboring $w,w'\in \oo^n$ and every function $g\colon \zs\mapsto \zo$, it holds that 
	$$
	\pr{g(f(w))=1}\leq e^{\eps}\cdot \pr{g(f(w'))=1} +\delta.
	$$ 	
	If $\delta=0$, we omit it from the notation.
\end{definition}
}


\subsubsection{Computational Differential Privacy}
There are several ways for defining computational differential privacy (see \cref{sec:related-works}). We use the most relaxed version due to \cite{BNO08}. For notational convenience, we focus on databases over $\oo$.
\begin{definition}[Computational differentially private mechanisms]\label{def:ComMech}
	A randomized function ensemble $f=\set{f_\pk\colon\oo^{n(\pk)}\mapsto \zs}$ is an {\sf $n$-size, $(\eps,\delta)$-computationally differentially private} (denoted $(\eps,\delta)$-$\CDP$) if for every poly-size circuit family $\set{\Ac_\pk}_{\pk\in \N}$, the following holds for every large enough $\pk$ and every neighboring $w,w'\in\oo^{n(\pk)}$:
	$$
	\pr{\Ac_\pk(f_\pk(w))=1}\leq e^{\eps(\pk)}\cdot \pr{\Ac_\pk(f_\pk(w'))=1} +\delta(\pk).
	$$ 
	If $\delta(\pk) = \negl(\pk)$, we omit it from the notation. 
\end{definition}



\subsubsection{Two-Party Differential Privacy}\label{sec:DP}
In this section we formally define distributed differential privacy mechanism (\ie protocols). %For the ease of notation, we consider protocol with no common input.

\begin{definition}\label{def:DP}%\Nnote{fix security parameter}
	A two-party protocol $\Pi=(\Ac,\Bc)$ is {\sf $(\eps,\delta)$-differentially private}, denoted $(\eps,\delta)$-$\DP$, if the following holds for every algorithm $\Dc$: let $\V^\Pc(x,y)(\pk)$ be the view of party $\Pc$ in a random execution of $\Pi(x,y)(1^\pk)$. Then for every $\pk,n \in \N$, $x\in \oo^n$ and neighboring $y,y'\in\oo^n$:
	\begin{align*}
	\pr{\Dc(V^\Ac(x,y)(\pk))=1}\le e^{\eps(\pk)}\cdot \pr{\Dc(V^\Ac (x,y')(\pk))=1}+\delta(\pk),
	\end{align*} 
	and for every $y\in \oo^n$ and neighboring $x,x'\in\oo^{n}$:
	\begin{align*}
	\pr{\Dc(V^\Bc(x,y)(\pk))=1}\le e^{\eps(\pk)}\cdot \pr{\Dc(V^\Bc (x',y)(\pk))=1}+\delta(\pk).
	\end{align*} 	
	Protocol $\Pi$ is {\sf $(\eps,\delta)$-computational differentially private}, denoted $(\eps,\delta)$-$\CDP$, if the above inequalities only hold for a non-uniform \ppt $\Dc$ and large enough $\pk$. We omit $\delta = \negl(\pk)$ from the notation. \footnote{Note that define we give for two-party differentially private protocols is a semi-honest definition, in which we ask for the security to hold when the parties interact in an honest execution of the protocol. Since we are proving a lower bound, starting from this weaker guarantee (as opposed to security against malicious players), yields a stronger result.}
\end{definition}
%We omit $\delta$ from the notation if $\delta$ is a negligible function of $n$.

%\Enote{simulation-based}
\begin{remark}[The definition for computational differential privacy we use]\label{rem:comDPChannel} 
	An alternative, stronger definition of computational differential privacy, known as simulation-based computational differential privacy, requires that the distribution of each party’s view be computationally indistinguishable from a distribution that ensures privacy in an information-theoretic sense. \cref{def:DP} is a weaker notion in comparison. Consequently, establishing a lower bound for a protocol that satisfies this weaker guarantee (as we do in this work) yields a stronger result.%Actually, our lower bound only requires the privacy to hold against \emph{uniform} external observer.
	%\Nnote{Maybe add: When only interesting in \Dp against external observer, the two definitions can be achieve using key-agreement and (single-party) \Dp mechanism. }
\end{remark}




\subsection{Useful Claims}
\remove{
In this section, we state generic lemmas and propositions that we will use later in our proofs.

The following lemma which we prove in \cref{sec:missing-proofs:distance-I}, measures the distance between two uniform stings conditioned one a random index $i$ either being fixed to $0$ or to $1$.

\def\distanceILemma{
    Let $R \la \zo^n$. For any (randomized) function $f:\{0,1\}^n\rightarrow \{0,1\}$ and $\alpha > 0$, it holds that
    \begin{align}\label{eq:f-alpha}
        \ppr{i \la [n]}{\size{\:\ex{f(R) \mid R_i = 0}-\ex{f(R) \mid R_i = 1}\:}\geq \alpha} \leq \frac{2}{n \alpha^2},
    \end{align}
    where the expectations are taken over $R$ and the randomness of $f$.
}

\begin{lemma}\label{lem:distance-I}
    \distanceILemma
\end{lemma}
}

The following two propositions state that given the output of a differentially private function, it is not possible to predict well even a random index (even if all other indexes are leaked). The first proposition handles the information-theoretic case and the second handles the computation case. Both propositions are proven in \cref{sec:missing-proofs:hard-to-guess}. 

\def\propHardToGuessInf{
    Let $f\colon \oo^n \rightarrow \cY$ be an $(\eps,\delta)$-\DP function, let $g \colon [n] \times \oo^{n-1} \times \cY \rightarrow \set{-1,1,\bot}$ be a (randomized) function, and let $X = (X_1,\ldots,X_n) \la \oo^n$. Then the following holds for every $i \in [n]$ where $X_i^* = g(i,X_{-i},f(X_1,\ldots,X_n))$:
    \begin{align*}
        \pr{X_i^* = X_i} \leq e^{\eps}\cdot \pr{X_i^* = -X_i} + \delta.
    \end{align*}
}

\begin{proposition}\label{prop:hard-to-guess-inf}
    \propHardToGuessInf
\end{proposition}


\def\propHardToGuessComp{
    Let $f = \set{f_{\pk} \colon \oo^{n(\pk)} \rightarrow \zo^{m(\pk)}}_{\pk \in \bbN}$ be an $(\eps,\delta)$-\CDP function ensemble, and let $\set{g_{\pk}}_{\pk \in \bbN}$ be a poly-size circuit family. Then, for large enough $\pk$ and $X = (X_1,\ldots,X_{n(\pk)}) \la \oo^{n(\pk)}$, the following holds for every $i \in [n(\pk)]$ where $X_i^* = g_{\pk}(i,X_{-i},f_{\pk}(X_1,\ldots,X_n))$:
    \begin{align*}
        \pr{X_i^* = X_i} \leq e^{\eps(\pk)}\cdot \pr{X_i^* = -X_i} + \delta(\pk).
    \end{align*}
}

\begin{proposition}\label{prop:hard-to-guess-comp}
    \propHardToGuessComp
\end{proposition}





\remove{
\Enote{Chao's old statement:}
\begin{lemma}\label{lem:distance-I-old}
        Let $R \la \zo^n$. 
	For any function $f:\{0,1\}^n\rightarrow \{0,1\}$ and $\alpha<0.01$, it holds that
	$$
	\Pr_{i\la[n]}\left[\: \size{\:\mathbb{E}[f(R) \mid R_i = 0]-\mathbb{E}[f(R) \mid R_i = 1]\:}\geq \alpha\right]\leq \frac{2+2\log(\frac{1}{\alpha})}{n\alpha^2}.
	$$
\end{lemma}
\begin{proof}
	Define $S_1=\{r \in \zo^n \colon f(r)=1\}$. Then for any $i\in[n]$, we have
	$$
	\begin{array}{rl}
		\size{\mathbb{E}[f(R) \mid R_i = 0]-\mathbb{E}[f(R) \mid R_i = 1]}
		&=\size{\Pr[R\in S_1|R_i=0]-\Pr[R\in S_1|R_i=1]}\\
		&=\size{\frac{\Pr[R_i=0|R\in S_1]\cdot\Pr[R\in S_1]}{\Pr[R_i=0]}-\frac{\Pr[R_i=1|R\in S_1]\cdot\Pr[R\in S_1]}{\Pr[R_i=1]}}\\
		&=\frac{2\size{S_1}}{2^n}\size{\Pr[R_i=0|R\in S_1]-\Pr[R_i=1|R\in S_1]}
	\end{array}
	$$
	When $|S_1|\leq \alpha\cdot 2^{n-1}$, we have $\size{\mathbb{E}[f(R) \mid R_i = 0]-\mathbb{E}[f(R) \mid R_i = 1]}\leq\frac{2\size{S_1}}{2^n}\leq \alpha$ for any $i\in[n]$. Hence, in the following, we assume $|S_1|> \alpha\cdot 2^{n-1}$.

	%Define $I_{bad}=\{i|\size{\Pr[R_i=0|R\in S_1]-\Pr[R_i=1|R\in S_1]}>2\alpha\}$ and $k=\size{I_{bad}}$, then for any $i\notin I_{bad}$, we have 
    %$$
    %\begin{array}{rl}
    %    2\alpha&\geq \size{\Pr[R_i=0|R\in S_1]-\Pr[R_i=1|R\in S_1]}\\
    %    &=\size{\frac{\Pr[R\in S_1|R_i=0]\cdot\Pr[R_i=0]}{\Pr[R\in S_1]}-\frac{\Pr[R\in S_1|R_i=1]\cdot\Pr[R_i=1]}{\Pr[R\in S_1]}}\\
    %    &=\size{\Pr[R\in S_1|R_i=0]-\Pr[R\in S_1|R_i=1]}\cdot\frac{1}{2\Pr[R\in S_1]}\\
    %    &\geq \size{\mathbb{E}[f(R) \mid R_i = 0]-\mathbb{E}[f(R) \mid R_i = 1]}\cdot \frac{1}{2},
    %\end{array}
    %$$ 
    %where the last inequality is because $\Pr[R\in S_1]\leq 1$. So that $\size{\mathbb{E}}[f(R) \mid R_i = 0]-\mathbb{E}[f(R) \mid R_i = 1]\leq %4\alpha$.
    Define $I_{bad}=\{i \colon \size{\Pr[R_i=0|R\in S_1]-\Pr[R_i=1|R\in S_1]} \geq 2\alpha\}$ and $k=\size{I_{bad}}$, and denote $I_{bad}=\{i_1,\dots,i_k\}$. Define $(X_{i_1}, \ldots X_{i_k}) = (R_{i_1},\dots,R_{i_k})\mid_{R \in S_1}$. 
    Consider the min-entropy
	$$
	\begin{array}{rl}
		H_{min}(X_{i_1},\dots,X_{i_k})&\leq H(X_{i_1},\dots,X_{i_k})\\
		&\leq \sum_{j=1}^k H(X_{i_j})\\
		&\leq k\cdot \left(-(\frac{1}{2}+2\alpha)\cdot\log(\frac{1}{2}+2\alpha)-(\frac{1}{2}-2\alpha)\cdot\log(\frac{1}{2}-2\alpha)\right)\\
            &=k\cdot \left(-(\frac{1}{2}+2\alpha)\cdot(\log(1+4\alpha)-1)-(\frac{1}{2}-2\alpha)\cdot(\log(1-4\alpha)-1)\right)\\
            &=k\cdot \left(1-(\frac{1}{2}+2\alpha)\cdot\log(1+4\alpha)-(\frac{1}{2}-2\alpha)\cdot\log(1-4\alpha)\right),
		
	\end{array}
	$$
	where $H_{min}(Y)$ is the minimum entropy of $Y$ and $H(Y)$ is the Shannon entropy of $Y$.\Enote{add to preliminaries.}
        The third inequality holds since by the definition of $I_{bad}$, for every $j \in [k]$ it holds that $\size{\pr{X_{i_j} = 1}-\pr{X_{i_j} = 0}} > 2\alpha$, and therefore $H(X_{i_j}) \leq H(1/2 + 2\alpha)$\Enote{define}.
	
	Therefore, there exists $b_1,\dots,b_k\in\{0,1\}$, such that 
	
	\begin{align}\label{eq:min-entropy-result}
		\Pr\left[(R_{i_1},\ldots,R_{i_k}) = (b_1,\ldots,b_k) \mid R\in S_1\right]
		&= \pr{(X_{i_1},\ldots,X_{i_k}) = (b_1,\ldots,b_k)}\\
		&= 2^{-H_{min}(X_{i_1},\dots,X_{i_k})}\nonumber\\
		&\geq 2^{k\cdot \left(-1+(\frac{1}{2}+2\alpha)\cdot\log(1+4\alpha)+(\frac{1}{2}-2\alpha)\cdot\log(1-4\alpha)\right)}.\nonumber
	\end{align}
	
	Let $S_{bad}=\{r \in \zo^n  \colon \set{(r_{i_1},\ldots,r_{i_k}) = (b_1,\ldots,b_k)} \land \set{r\in S_1}\}$.
	It holds that
	\begin{align*}
		|S_{bad}|
		&= \size{S_1} \cdot \Pr\left[(R_{i_1},\ldots,R_{i_k}) = (b_1,\ldots,b_k) \mid R\in S_1\right]\\
		&\geq \alpha\cdot 2^{n-1}\cdot2^{k\cdot \left(-1+(\frac{1}{2}+2\alpha)\cdot\log(1+4\alpha)+(\frac{1}{2}-2\alpha)\cdot\log(1-4\alpha)\right)},
	\end{align*} 
	where the inequality holds by \cref{eq:min-entropy-result} and since $\size{S_1} \geq \alpha\cdot 2^{n-1}$.
	Notice that any string in $S_{bad}$ depends on at most $n-k$ bits. It implies that $|S_{bad}|\leq 2^{n-k}$. Therefore, we have
	$$
	\begin{array}{rl}
		&2^{n-k}\geq \alpha\cdot 2^{n-1}\cdot2^{k\cdot \left(-1+(\frac{1}{2}+2\alpha)\cdot\log(1+4\alpha)+(\frac{1}{2}-2\alpha)\cdot\log(1-4\alpha)\right)} \\
		\Rightarrow& n-k \geq \log \alpha+n-1+k\cdot \left(-1+(\frac{1}{2}+2\alpha)\cdot\log(1+4\alpha)+(\frac{1}{2}-2\alpha)\cdot\log(1-4\alpha)\right)\\
		\Rightarrow& 1-\log \alpha \geq k\cdot((\frac{1}{2}+2\alpha)\cdot\log(1+4\alpha)+(\frac{1}{2}-2\alpha)\cdot\log(1-4\alpha))\\
		\Rightarrow& 1-\log \alpha \geq k\cdot(4\alpha\cdot\log(1+4\alpha)+(\frac{1}{2}-2\alpha)\cdot\log(1-16\alpha^2))\\
        \Rightarrow& 1-\log\alpha \geq k\cdot(15.9\alpha^2-8\alpha^2+32\alpha^3)=k\cdot(7.9\alpha^2+32\alpha^3)>0.5k\alpha^2\\
		\Rightarrow& k\leq \frac{2-2\log \alpha}{\alpha^2} = \frac{2+2\log (1/\alpha)}{\alpha^2},
	\end{array}
	$$
	Where the third transition holds since 
	\begin{align*}
		\lefteqn{(\frac{1}{2}+2\alpha)\cdot\log(1+4\alpha)+(\frac{1}{2}-2\alpha)\cdot\log(1-4\alpha)}\\
		&= 4\alpha\cdot\log(1+4\alpha) + (\frac{1}{2}-2\alpha)\paren{\log(1+4\alpha)+\log(1-4\alpha)}\\
		&= 4\alpha\cdot\log(1+4\alpha)+(\frac{1}{2}-2\alpha)\cdot\log(1-16\alpha^2),
	\end{align*}
	and the forth transition holds since $4\alpha\cdot\log(1+4\alpha)+(\frac{1}{2}-2\alpha)\cdot\log(1-16\alpha^2) > 15.9\alpha^2-8\alpha^2+32\alpha^3$ for $\alpha < 0.01$.
	Thus, we conclude that 
	$$
	\Pr_{i\la[n]}\left[\size{\mathbb{E}[f(R) \mid R_i=0]-\mathbb{E}[f(R) \mid R_i = 1]}\geq \alpha\right]\leq \frac{k}{n}\leq \frac{2+2\log (1/\alpha)}{n\alpha^2}.
	$$
\end{proof}
}


\subsection{Channels and Two-Party Protocols}\label{sec:protocol}

\paragraph{Channels.}A channel is simply a distribution of a pair of tuples defined as follows. 
\begin{definition}[Channels]\label{def:channel} A {\sf channel} $C_{(X,U)(Y,V)}$ of size $\isize$ over alphabet $\Sigma$ is a probability distribution over $(\Sigma^\isize \times\zo^\ast) \times(\Sigma^\isize \times\zo^\ast)$. The ensemble $C_{(X,U)(Y,V)}= \set{C_{(X_\pk,U_\pk)(Y_\pk,V_\pk)}}_{\pk\in \N}$ is an $\isize$-size channel ensemble, if for every $\pk\in \N$, $C_{(X_\pk,U_\pk)(Y_\pk,V_\pk)}$ is an $\isize(\pk)$-size channel. %We denote a channel of size one by a \emph{single-bit} channel. 
We refer to $X$ and $Y$ as the {\sf local outputs}, and to $U$ and $V$ as the {\sf views}.	
\end{definition}

We view a  channel as the experiment in which there are two parties $\Ac$ and $\Bc$.  Party $\Ac$ receives ``output'' $X$ and ``view'' $U$, and party $\Bc$ receives ``output'' $Y$ and ``view'' $V$. Unless stated otherwise, the channels we consider are over the alphabet $\Sigma = \oo$. We naturally identify channels with the distribution that characterizes their output.








\subsubsection{Two-Party Protocols}

A two-party protocol $\Pi=(\Ac,\Bc)$ is \ppt if the running time of both parties is polynomial in their input length. We let $\Pi(x,y)(z)$ or $(\Ac(x),\Bc(y))(z)$ denote a random execution of $\Pi$ on a common input $z$, and private inputs $x,y$.%We assume \wlg that a protocol has a common output (part of its transcript).\Jnote{This is not really the case we consider in this paper..}

\begin{definition}[Oracle-aided protocols]\label{def:ChannelAidedProtocol}
	In a two-party protocol $\Pi$ with oracle access to a {\sf protocol} $\Psi$, denoted $\Pi^\Psi$, the parties make use of the \textit{next-message function} of $\Psi$.\footnote{The function that on a partial view of one of the parties, returns its next message.} In a two-party protocol $\Pi$ with oracle access to a {\sf channel} $C_{Z W}$, denoted $\Pi^C$, the parties can jointly invoke $C$ for several times. In each call, an independent pair $(z,w)$ is sampled according to $C_{Z W}$, one party gets $z$, the other gets $w$.
\end{definition}


\begin{definition}[The channel of a protocol]\label{def:ChannlOfProtocol}
	For a no-input two-party protocol $\Pi= (\Ac,\Bc)$, we associate the channel $C_\Pi$, defined by $\C_\Pi= C_{(X, U),(Y, V)}$, where $X$ and $Y$ are the local outputs of $\Ac$ and $\Bc$ (respectively) and
	$U$ and $V$ are the local views of $\Ac$ and $\Bc$ (respectively).
    
	For a two-party protocol $\Pi$ that gets a security parameter $1^\pk$ as its (only, common) input, we associate the channel ensemble $ \set{C_{\Pi(1^\pk)}}_{\pk\in \N}$. 
\end{definition}

\begin{definition}[$(\alpha,\gamma)$-Accurate channel]\label{def:accurate-func}
	A channel $C = C_{(X, U),(Y, V)}$ is {\sf $(\alpha,\gamma)$-accurate for the function $f$}, if $\ppr{C}{\size{\out(V)-f(X,Y)}\leq \alpha}\ge \gamma$, where $\out(V)$ is the designated output.
    A channel ensemble $C_{(X, U),(Y, V)}= \set{C_{(X_\pk, U_\pk),(Y_\pk, V_\pk)}}_{\pk\in \N}$ is  $(\alpha,\gamma)$-accurate for  $f$ if $C_{(X_\pk, U_\pk),(Y_\pk, V_\pk)}$ is $(\alpha(\pk),\gamma(\pk))$-accurate for $f$, for every $\pk \in \N$.
\end{definition}

\subsubsection{Differentially Private Channels}\label{sec:DPChannel}
Differentially private channels are naturally defined as follows:
\begin{definition}[Differentially private channels]\label{def:DPChannel}
	An $n$-size channel $C = C_{(X, U),(Y, V)}$ with $X, Y$ over $\oo^n$ 
	is {\sf$(\eps,\delta)$-differentially private} (denoted $(\eps,\delta)$-$\DP$) if for every $x \in \Supp(X)$ there exists an $n$-size $(\eps,\delta)$-$\DP$ mechanisms $\Mc_x$ such that $(X,Y,U) \equiv (X,Y,\Mc_X(Y))$, and for every $y \in \Supp(Y)$ there exists an $n$-size $(\eps,\delta)$-$\DP$ mechanisms $\Mc_y'$ such that $(X,Y,V) \equiv (X,Y,\Mc_Y'(X))$. In addition, we say that the channel is \emph{uniform} if $X$ and $Y$ are independent random variables uniformly distributed in $\oo^n$. 
\end{definition}

\begin{definition}[Computational differentially private channels]\label{def:CDPChannel}
	An $n$-size channel ensemble $C = \set{C_{(X_\pk, U_\pk),(Y_\pk, V_\pk)}}_{\pk\in\N}$ with $X_\pk, Y_\pk$ over $\oo^n$ 
	is {\sf$(\eps,\delta)$-computationally differentially private} (denoted $(\eps,\delta)$-$\CDP$) if for every ensemble $\set{x_\pk \in \Supp(X_\pk)}_{\pk\in\N}$ there exists an $n$-size $(\eps,\delta)$-\CDP mechanisms ensemble $\set{\Mc_{x_\pk}}_{\pk\in\N}$ such that $(X_\pk,Y_\pk,U_\pk) \equiv (X_\pk,Y_\pk,\Mc_{X_\pk}(Y_\pk))$, for every $\pk\in\N$, and for every ensemble $\set{y_\pk \in \Supp(Y_\pk)}_{\pk\in\N}$ there exists an $n$-size $(\eps,\delta)$-$\CDP$ mechanisms ensemble $\set{\Mc'_{y_\pk}}_{\pk\in\N}$ such that $(X_\pk,Y_\pk,V_\pk) \equiv (X_\pk,Y_\pk,\Mc_{Y_\pk}'(X_\pk))$ for every $\pk\in \N$. In addition, we say that the channel is \emph{uniform} if $X_\pk$ and $Y_\pk$ are independent random variables uniformly distributed in $\{\pm 1\}^n$ for all $\pk\in\N$.
\end{definition}




% \begin{lemma}~\label{lem:dp-sv-source}
% 	Let $P$ be an $\varepsilon$-DP randomized protocol. Let $X$ and $Y$ be independent random variables uniformly distributed in $\{\pm 1\}^n$ and let random variable $\Pi(X,Y)$ denote the transcript of running $P(X,y)$. Then for every $\pi\in Supp(\Pi)$, the random variables corresponding to the inputs conditioned on transcript $\pi$, $X_\pi$ and $Y_\pi$, are independent $e^{-\varepsilon}$-strong SV source.
% \end{lemma}





\subsubsection{Weak Erasure Channel (\WEC)}

\begin{definition}[\WEC]\label{def:WEC}
	A channel $((O_A,V_A), (O_B,V_B))$ with $O_A \in \set{0,1}$ and $O_B \in \set{0,1,\bot}$ is a {\sf weak erasure channel}, denoted $(\alpha,p,q)$-$\WEC$, if:
	\begin{itemize}
		%\item $O_A\in \set{-1,1}$ and $O_B\in \set{-1,1,\bot}$.
		\item Random erasure: $\pr{O_B = \perp} = 1/2$.
		
		\item Agreement: $\pr{O_A\ne O_B\mid O_B\ne \bot}\le \alpha$.
		
		\item Secrecy:
		
		\begin{enumerate}
			\item For every algorithm $\Dc$ it holds that\label{WEC:item:A}
			\begin{align*}
				%\size{\pr{\Ac(O_A,V_A) = 1 \mid O_B \neq \perp} - \pr{\Ac(O_A,V_A) = 1 \mid O_B = \perp}} \le p
				\size{\pr{\Dc(V_A) = 1 \mid O_B \neq \perp} - \pr{\Dc(V_A) = 1 \mid O_B = \perp}} \le p
			\end{align*}
			(Alice doesn't know if $O_B = \perp$.)
			
			\item For every algorithm $\Dc$ it holds that\label{WEC:item:B}
			\begin{align*}
				\pr{\Dc(V_B) = O_A \mid O_B=\bot} \leq \frac{1+q}{2}.
			\end{align*}
			(i.e., if $O_B=\bot$, Bob don't know what is the value of $O_A$).
			
			%\item $SD((O_A U|O_B=\bot),(O_A U|O_B\ne \bot))\le p$ (The sender don't know if $O_B=\bot$).
			
			%\item $SD(V O_A|O_B=\bot,V(-O_A)|O_B=\bot)\le q$ (If $O_B=\bot$, Bob don't know what the value of $O_A$).
		\end{enumerate}
	\end{itemize}
   We say that a channel ensemble $C=\set{C_\pk}_{\pk\in N}$ is a {\sf computational weak erasure channel}, denoted $(\alpha,p,q)$-\CompWEC, if for every \ppt algorithm $\Dc$ and every sufficiently large $\pk\in\N$, $C_\pk$ satisfies the properties stated in the items above, where the secrecy property holds with respect to a \ppt algorithm $\Dc$. A protocol $\Lambda$ is said to be $(\alpha,p,q)$-$\CompWEC$, if the ensemble induces by the protocol (that is, $C=\set{C_{\Lambda(\pk)}}_{\pk\in\N}$) is $(\alpha,p,q)$-$\CompWEC$.  
\end{definition}



\subsubsection{Approximate Weak Erasure Channel (\AWEC)}\label{sec:AWEC}

\begin{definition}[\AWEC]\label{def:AWEC}
	A channel $C = ((O_A,V_A), (O_B,V_B))$ over $([-n,n] \times \zo^*) \times (([-n,n] \cup \bot)  \times \zo^*)$ is an {\sf approximate weak erasure channel}, denoted $(\ell,\alpha,p,q)$-\AWEC if:
	\begin{itemize}
		
		\item Random erasure: $\pr{O_B = \perp} = 1/2$.
		
		\item Accuracy: $\pr{\size{O_A - O_B} > \ell \mid O_B \ne \bot}\le \alpha$.
		
		\item Secrecy:
		
		\begin{enumerate}
			\item For every algorithm $\Dc$ it holds that\label{AWEC:item:A}
			\begin{align*}
				%\size{\pr{\Ac(O_A,V_A) = 1 \mid O_B \neq \perp} - \pr{\Ac(O_A,V_A) = 1 \mid O_B = \perp}} \le p
				\size{\pr{\Dc(V_A) = 1 \mid O_B \neq \perp} - \pr{\Dc(V_A) = 1 \mid O_B = \perp}} \le p
			\end{align*}
			(Alice doesn't know if $O_B=\bot$).
			
			\item For every algorithm $\Dc$ it holds that\label{AWEC:item:B}
			\begin{align*}
				\pr{\size{\Dc(V_B) - O_A} \leq 1000 \ell \mid O_B=\bot} \leq q.
			\end{align*}
			(i.e., if $O_B=\bot$, Bob can't estimate the value of $O_A$ with error $\leq 1000 \ell$).
		\end{enumerate}
	\end{itemize}
     We say that a channel ensemble $C=\set{C_\pk}_{\pk\in N}$ is a {\sf computational approximate weak erasure channel}, denoted $(\ell,\alpha,p,q)$-\CompAWEC, if for every \ppt algorithm $\Dc$ and every sufficiently large $\pk\in\N$, $C_\pk$ satisfies the properties stated in the items above. A protocol $\Gamma$ is said to be $(\ell,\alpha,p,q)$-$\CompAWEC$, if the ensemble induced by the protocol (that is, $C=\set{C_{\Gamma(\pk)}}_{\pk\in\N}$) is $(\ell,\alpha,p,q)$-$\CompAWEC$.  
\end{definition}

We will make use of the following lemma, which shows that for some choices of the parameters, \AWEC implies \WEC. The lemma is proven in \cref{sec:AWEC-to-WEC}.

\begin{lemma}\label{lemma:AWEC-to-WEC}
	For every $\ell> 0$, there exists a \ppt protocol $\Lambda = (\Pc_1,\Pc_2)$ such that given an oracle access to an $(\ell,\alpha,p,q)$-\AWEC $C$, the channel $\tilde{C}$ induced by $\Lambda^C$ is $(\alpha'=\alpha+0.001,\: p' = p ,\:  q' = 1/2 + 2(q+0.01))$-\WEC.
	Furthermore, the proof is constructive in a black-box manner:
	\begin{enumerate}
		\item There exists an oracle-aided \ppt algorithm $\Ec_1$ such that for every channel $C = ((\OA,\VA), (\OB,\VB))$ and algorithm $\Dc$ violating the \WEC secrecy property~\ref{WEC:item:A} of $\tilde{C}$, algorithm $\Ec_1^{\Dc}$ violates the \AWEC secrecy property~\ref{AWEC:item:A} of $C$.
		
		\item There exists an oracle-aided \ppt algorithm $\Ec_2$ such that for every channel $C = ((\OA,\VA), (\OB,\VB))$ and algorithm $\Dc$ violating the \WEC secrecy property~\ref{WEC:item:B} of $\tilde{C}$, algorithm $\Ec_2^{\Dc}$ violates the \AWEC secrecy property~\ref{AWEC:item:B} of $C$.
	\end{enumerate}
\end{lemma}

Since \cref{lemma:AWEC-to-WEC} is constructive, the following is an immediate corollary.
\begin{corollary}\label{cor:CompAWEC to CompWEC}
There exists an oracle aided \ppt protocol $\Lambda$, such that given a protocol $\Gamma$ that induces $(\ell,\alpha,p,q)$-\CompAWEC, it holds that $\Lambda^\Gamma$ is $(\alpha'=\alpha+0.001,\: p' = p ,\:  q' = 1/2 + 2(q+0.01))$-\CompWEC.  
\end{corollary}
\begin{proof}[Proof of \ref{cor:CompAWEC to CompWEC}]
Let $\Lambda$ be the \ppt algorithm guaranteed  by Lemma \ref{lemma:AWEC-to-WEC}. Given an $(\ell,\alpha,p,q)$-\CompAWEC protocol $\Gamma$, we define $\Lambda(\pk)={\Lambda^{\Gamma(\pk)}(\pk)}$. Assume towards a contradiction that $\Lambda$ is not a $(\alpha',p',q')$-\CompWEC. It follows that there exists a \ppt $\Dc$ that for infinity many $\pk\in\N$ contradicts one of the \WEC secrecy properties of channel ensemble $\set{C_{\Lambda(\pk)}}_{\pk\in\N}$. Fix $\pk\in\N$ for which this holds. By Lemma \ref{lemma:AWEC-to-WEC}, there exists a \ppt $\Ec^\Dc$ that for every such $\pk$  contradicts one of the secrecy properties of the channel $C_{\Gamma(\pk)}$. This implies that for infinity many $\pk\in\N$, $\Ec^\Dc$  contradict the secrecy of the channel ensemble $\set{C_{\Gamma(\pk)}}_{\pk\in\N}$, which is a contradiction since this would means that $\Gamma$ is not a $(\ell,\alpha,p,q)$-\CompAWEC.       
\end{proof}



\subsection{Oblivious Transfer (\OT)}

\paragraph{Secure Computation.}
We use the standard notion of securely computing a functionality, \cf  \cite{Goldreich04}.
\begin{definition}[Secure computation]\label{def:SFE}
	A two-party protocol {\sf securely computes a functionality $f$}, if it does so according to the real/ideal paradigm.   We add the term perfectly/statistically/computationally/non-uniform computationally, if the simulator's output is  perfect/statistical/computationally indistinguishable/  non-uniformly indistinguishable from  the real distribution.  The protocol have the above notions of security {\sf against semi-honest  adversaries}, if its security only  guaranteed to holds against an adversary that follows the prescribed protocol.   Finally, for the case of perfectly secure computation, we naturally apply the above notion also to the non-asymptotic case: the protocol with no security parameter perfectly  compute a functionality $f$.
	
	A two-party protocol {\sf securely computes a functionality ensemble $f$ with oracle to a channel $C$}, if it does so according to the above definition when the parties have access to a trusted party computing $C$. All the above adjectives naturally extend to this setting.
\end{definition}

\paragraph{Oblivious Transfer.}
The (one-out-of-two) oblivious transfer functionality is defined as follows.
\begin{definition}[oblivious transfer functionality $f_{\OT}$]\label{def:OTfunc}
	The oblivious transfer functionality over $\zo \times (\zs)^2$ is defined by  $f_{\OT} (i,(\sigma_0,\sigma_1)) = (\perp,\sigma_i)$.
\end{definition}
A protocol is $\ast$ secure OT,   for \\$\ast\in \set{\text{semi-honest statistically/computationally/computationally non-uniform}}$, if it  compute the $f_{\OT}$  functionality with $\ast$ security.





% \begin{definition}[Computational oblivious transfer, semi-honest model]
% A protocol $\Pi=(\Ac,\Bc)$ is a semi-honest 1-out-of-2 computational oblivious transfer (comp-OT) protocol if the following holds. Given a common input $1^{\pk}$, the parties $\Ac$ and $\Bc$ run the protocol $\Pi(1^\pk)$ (in an honest manner) and    
% $\Ac$ outputs $X=(m_1,m_2)\in \zo\times\zo$ and has a view $U$ and $\Bc$ outputs $Y=(i,\hat{m})\in\zo\times\zo$ and has a view $V$, and the following properties are satisfied:
% \begin{enumerate}
%     \item \textbf{Correctness:} 
%     $\pr{\hat{m}\neq m_i}<\negl(\pk).$ 
    
%     \item \textbf{A's Privacy:} For every \ppt $\Dc$ and every sufficiently large $\pk$:
%     $\pr{\Dc(V)=m_{i-1}}<(1+\negl(\pk))/2$
    
%     \item \textbf{B's Privacy:} For every \ppt $\Dc$ and every sufficiently large $\pk$:
%     $\pr{\Dc(U)=i}<(1+\negl(\pk))/2$  
% \end{enumerate}
% \end{definition}

We make use of the following useful results by Wullschleger on oblivious transfer amplification from weak channels.
\begin{theorem}[\cite{Wullschleger09}, from \WEC to statistically secure \OT]\label{thm:WEC TO OT IT}
    There exists an oracle aided protocol $\Pi$ such that the following holds: Given a $(\alpha,p,q)$-\WEC $C$, if $44(\alpha+p)\le 1-q$ then $\Pi^{C}(1^\pk)$ is a semi-honest statistically secure \OT.
\end{theorem}

The following computational version of \cref{thm:WEC TO OT IT} is implicit in \cite{Wullschleger09} and is based on the computational proof explicitly stated in \cite{Wul07} (see Section 6 in \cite{Wullschleger09} for discussion).   

\begin{theorem}[\cite{Wullschleger09,   Wul07}, from \CompWEC to computinally secure \OT]\label{thm:WEC TO OT Comp}
    There exists an oracle aided protocol $\Pi$ such that the following holds: Given a $(\alpha,p,q)$-\CompWEC protocol $\Lambda$, if $44(\alpha+p)\le 1-q$ then $\Pi^{\Lambda}$ is a semi-honest computational secure \OT.
\end{theorem}



% \begin{definition}[Computational 1-out-of-2 Oblivious Transfer, semi-honest model]
% A protocol $\Pi=(\Ac,\Bc)$ is a semi-honest 1-out-of-2 $(\eps,\alpha,\beta)$-oblivious transfer (OT) protocol if the following holds. 

% The parties $\Ac$ and $\Bc$ run the protocol (in an honest manner) and    
% $\Ac$ outputs $X=(m_1,m_2)\in \zo\times\zo$ and has a view $U$ and $\Bc$ outputs $Y=(i,\hat{m})\in\zo\times\zo$ and has a view $V$, and following properties are satisfied:
% \begin{enumerate}
%     \item \textbf{Correctness:} 
%     $\pr{\hat{m}\neq m_i}<\eps.$ 
    
%     \item \textbf{A's Privacy:} For every adversary $\Dc$:
%     $\pr{\Dc(V)=m_{i-1}}<(1+\alpha)/2$
    
%     \item \textbf{B's Privacy:} For every adversary $\Dc$: $\pr{\Dc(U)=i}<(1+\beta)/2$  
% \end{enumerate}
% \end{definition}
\section{Small Model Learnability Gap}
\label{sec: empirical}
In this section, we fine-tune student models using different CoT data. 
We then reveal the small model learnability gap given the performance of fine-tuned models.

\subsection{Experiment Setup}

\paragraph{Datasets.}
We use the 7,500 prompt set of MATH \citep{hendrycks2021measuringmathematicalproblemsolving}. This dataset encompasses seven math topics such as advanced calculus, geometry, and linear algebra. 





\paragraph{Student models.}
Our study considers ten student models from the Qwen \citep{qwen2.5} and Llama \citep{llama32,llama31} model families of varying sizes. These models include the Instruct version of \texttt{Qwen2.5-0.5B}, \texttt{Qwen2.5-1.5B}, \texttt{Qwen2.5-3B}, \texttt{Qwen2.5-7B}, \texttt{Qwen2.5-14B}, and \texttt{Qwen2.5-32B}, and the Instruct version of \texttt{Llama3.2-1B}, \texttt{Llama3.2-3B}, \texttt{Llama3.1-8B}, and \texttt{Llama3.3-70B}. A comprehensive overview of the student models is presented in Table \ref{tab:models_overview} of Appendix \ref{appendix:More on Experimental Setups}. 


\paragraph{Teacher models.} 


To compare long CoT with short CoT, we use \texttt{QwQ-32B-Preview} \citep{QwenTeam2024b} to generate long CoT sequences and \texttt{Qwen2.5-32B-Instruct} as the response generator for short CoT. 
Within each model family, we designate the larger scale model as the large teacher and the smaller scale model as the small teacher. 
This includes \texttt{Qwen2.5-72B-Instruct} vs \texttt{Qwen2.5-\allowdisplaybreaks3B-Instruct},~\texttt{Llama3.1-70B-Instruct} vs~\texttt{Llama3.1-8B-Instruct},~and \texttt{Gemma2-27B-it} vs \texttt{Gemma2-9B-it}.



\begin{figure*}[!t]
    \centering
    \includegraphics[width=\textwidth]{figs/long_cot_combined_model_performance.pdf}
    \caption{Long CoT Gap ($\Delta_{Long}=P_{Long} - P_{Short}$) of student models with different models sizes for (a) Qwen family (b) Llama family. For teacher models, \texttt{QwQ-32B-Preview} is chosen to generate long CoT responses, while \texttt{Qwen2.5-32B-Instruct} is chosen to generate short CoT responses. Negative (Positive) $\Delta_{Long}$ indicates that long CoT is worse (better) than short CoT. Our results demonstrate that short CoT is better for smaller student models (indicated by $\Delta_{Long}$ < 0), while long CoT is better for larger student models (indicated by $\Delta_{Long}$ > 0).}
    % \vspace{-1em}
    \label{fig:combined_model_performance}
\end{figure*}
\begin{figure*}[!t]
    \centering
    \includegraphics[width=\textwidth]{figs/lmp_combined_model_performance.pdf}
    \caption{Large model CoT Gap ($\Delta_{Large}=P_{Large} - P_{Small}$) of student models with different models sizes for (a) Qwen family (b) Llama family. For teacher models, \texttt{Qwen2.5-72B-Instruct} is chosen as the large teacher to generate responses, while \texttt{Qwen2.5-3B-Instruct} is chosen as the small teacher to generate responses. Negative (positive) $\Delta_{Large}$ indicates that large teacher CoT is worse (better) than small teacher CoT. Our results demonstrate that small teacher CoT is better for smaller student models (indicated by $\Delta_{Large}$ < 0), while large model CoT is better for larger student models (indicated by $\Delta_{Large}$ > 0).}
    \label{fig:lmp_combined_model_performance}
\end{figure*}

\paragraph{Evaluation Benchmarks.} 


We evaluate the reasoning capability of fine-tuned student models on a set of commonly used benchmarks, including MATH \citep{hendrycks2021measuringmathematicalproblemsolving}, GSM8K \citep{cobbe2021trainingverifierssolvemath}, AMC 2023, AIME 2024, and the English math subset of OlympiadBench \citep{he2024olympiadbenchchallengingbenchmarkpromoting}. These benchmarks span a wide range of challenge levels, from elementary mathematics to advanced competition problems. 
We define the student model performance as the average score on five benchmarks. Unless otherwise specified, all fine-tuned models are evaluated in a zero-shot setting using greedy decoding. We set the maximum generation tokens as 16k. Please see Appendix \ref{appendix:More on Experimental Setups} for detailed experimental setup.

We define the following performance scores:
\begin{itemize}
    \item \(P_{Long}\): Performance score of a student model fine-tuned on long CoT data.
    \item \(P_{Short}\): Performance score of a student model fine-tuned on short CoT data.
    \item \(P_{Large}\): Performance score of a student model fine-tuned on CoT from a larger teacher.
    \item \(P_{Small}\): Performance score of a student model fine-tuned on CoT from a smaller teacher.
\end{itemize}




\paragraph{Training Setup.} 
Teacher models generate responses by rejection sampling 
\citep{dong2023raftrewardrankedfinetuning, gulcehre2023reinforcedselftrainingrestlanguage, tong2024dartmathdifficultyawarerejectiontuning, yuan2023scalingrelationshiplearningmathematical, yue2023mammothbuildingmathgeneralist, zelikman2022starbootstrappingreasoningreasoning}
By default, teacher models employ greedy decoding.  
By combining the math problem instructions with corresponding solutions generated by teacher models, we construct problem-solution pairs to fine-tune student models. 
We train the models using the LLaMA-Factory framework \citep{zheng2024llamafactory}. 
For student models of scale less than 14B, we use full-parameter SFT and implement a cosine learning rate schedule with a maximum learning rate of $10^{-5}$ to fine-tune student models for two epochs \citep{touvron2023llama}. 
For student models larger than 14B, we adopt LoRA fine-tuning with a learning rate of $10^{-4}$ for two epochs. Detailed hyperparameters and information about the experimental platform are provided in Appendix \ref{appendix:More on Experimental Setups}.





\subsection{Long CoT Gap}
This section evaluates the reasoning capabilities of student models fine-tuned over long CoT data and short CoT data. We quantify the performance difference between long and short CoT data using \emph{long CoT gap} \(\Delta_{Long}\), defined as:
\begin{equation*}
\Delta_{Long} = P_{Long} - P_{Short}.
\end{equation*}

Figure \ref{fig:combined_model_performance} provides a comprehensive overview of the long CoT gap $\Delta_{Long}$ across different student models. 
The detailed benchmark scores on MATH, GSM8K, AIME, AMC, and OlympiadBench are deferred to Table \ref{tab:full_performance_lg} in Appendix \ref{appendix: More Experiments}.
We report the following key takeaways. 

\begin{AIbox}{Takeaway 1: Long CoT Gap}
Small student models tend to benefit more from short CoT, while large student models gain greater advantages from long CoT.

\end{AIbox}

We observe that long CoT is more effective for larger models, consistently leading to improved performance across most math benchmarks. 
For example, the student model \texttt{Qwen2.5-32B-Instruct} improves about 15 points across all math metrics on average. 

However, long CoT data is not effective for smaller models, yielding significantly less improvement compared to short CoT. On the MATH and AMC benchmarks, student model \texttt{Qwen2.5-1.5B-Instruct}  performs over 10 points lower when fine-tuned with long CoT data. This shows that smaller models may not be able to effectively learn and utilize the long CoT paradigm. 
Please see more attribution analysis in Section \ref{More Analysis Results}.





\begin{table}[ht]
\centering
\small
\label{tab:comparison_table}
\resizebox{\columnwidth}{!}{
\begin{tabular}{lcccc}
\toprule
\textbf{Student Model} & \textbf{$P_{Long}$} & \textbf{$P_{Short}$} & \textbf{$\Delta_{Long}$} & \textbf{\makecell{ Better?}} \\
\midrule
Qwen2.5-0.5B   & 14.8 & 19.5 & \cellcolor{red!60}{-4.7}  & Short \\
Qwen2.5-1.5B   & 27.0 & 34.2 & \cellcolor{red!80}{-7.1}  & Short \\
Qwen2.5-3B     & 40.3 & 43.4 & \cellcolor{red!30}{-3.1}  & Short \\
Qwen2.5-7B     & 48.9 & 47.2 & \cellcolor{green!10}{1.7}     & Long   \\
Qwen2.5-14B    & 59.2 & 54.3 & \cellcolor{green!30}{4.9}     & Long   \\
Qwen2.5-32B    & 73.0 & 59.3 & \cellcolor{green!80}{13.7}    & Long   \\
\midrule
Llama-3.2-1B   & 15.8 & 19.5 & \cellcolor{red!40}{-3.7}  & Short \\
Llama-3.2-3B   & 32.5 & 33.1 & \cellcolor{red!10}{-0.6}  & Short \\
Llama-3.1-8B   & 35.2 & 31.5 & \cellcolor{green!25}{3.7}     & Long   \\
Llama-3.3-70B  & 58.2 & 54.3 & \cellcolor{green!25}{3.8}     & Long   \\
\bottomrule
\end{tabular}
}
\caption{Comparison of the average performance between fine-tuning with long CoT ($P_{Long}$) and short CoT ($P_{Short}$). We find that small student models may struggle to learn from long CoT data. }
\end{table}

\begin{table}[ht]
\centering
\small
\label{tab:strong_vs_weak}
\resizebox{\columnwidth}{!}{
\begin{tabular}{lcccc}
\toprule
\textbf{Student Model} & \textbf{$P_{Large}$} & \textbf{$P_{Small}$} & \textbf{$\Delta_{Large}$} & \textbf{\makecell{ Better?}} \\
\midrule
Qwen2.5-0.5B   & 16.9 & 20.4 & \cellcolor{red!50}{-3.5} & Weak   \\
Qwen2.5-1.5B   & 32.2 & 33.0 & \cellcolor{red!15}{-0.8} & Weak   \\
Qwen2.5-3B     & 39.7 & 39.4 & \cellcolor{green!10}{0.3}    & Strong \\
Qwen2.5-7B     & 48.9 & 42.3 & \cellcolor{green!70}{6.6}    & Strong \\
Qwen2.5-14B    & 52.9 & 49.9 & \cellcolor{green!30}{3.0}    & Strong \\
Qwen2.5-32B    & 59.5 & 53.0 & \cellcolor{green!70}{6.5}    & Strong \\
\midrule
Llama-3.2-1B   & 16.5 & 18.5 & \cellcolor{red!30}{-1.9} & Weak   \\
Llama-3.2-3B   & 32.8 & 31.2 & \cellcolor{green!20}{1.6}    & Strong \\
Llama-3.1-8B   & 25.6 & 25.1 & \cellcolor{green!10}{0.5}    & Strong \\
Llama-3.3-70B  & 57.6 & 53.3 & \cellcolor{green!50}{4.3}    & Strong \\
\bottomrule
\end{tabular}
}
\caption{Comparison of average performance between fine-tuning with large teacher CoT ($P_{Long}$) and small teacher CoT ($P_{Small}$). We find that small student models may struggle to learn from large teacher CoT data.}
\end{table}





\subsection{Large Teacher CoT Gap}
We investigate how effective small models may learn from large teacher and small teachers.
We define a \emph{large teacher CoT gap} as:
\[
\Delta_{Large} = P_{Large} - P_{Small}.
\]

Figure \ref{fig:lmp_combined_model_performance} provides a comprehensive comparison of the $\Delta_{Large}$ incurred by all student models. 
The detailed benchmark scores of MATH, GSM8K, AIME, AMC and OlympiadBench are deferred to Table \ref{tab:lmp-full_comparison} in Appendix \ref{appendix: More Experiments}. More experimental results of different teacher models, including \texttt{Llama3.1-70B} vs \texttt{Llama3.1-8B} and \texttt{Gemma2-27B} vs \texttt{Gemma2-9B} are in Table \ref{tab:lmp_comparison2} of Appendix \ref{appendix: More Experiments}.

We observe that larger student models learn effectively from large teacher CoT. 
For example, \texttt{Qwen2.5-7B-Instruct} and \texttt{Qwen2.5-32B-Instruct} student models improve over 5 points on average, with \texttt{Qwen2.5-32B-Instruct} achieving more than a 15 point increase on the AIMC benchmark. 
However, smaller models do not learn effectively from large teacher models such as \texttt{Qwen2.5-72B-Instruct}. 
Instead, small teacher models such as \texttt{Qwen2.5-3B-Instruct} may serve as better teacher models for small student models.
For instance, the performance of \texttt{Qwen2.5-0.5B-Instruct} degrades by more than 10 points on the AMC benchmark.


Note that prior studies \citep{kim2024evaluatinglanguagemodelssynthetic} also demonstrated that stronger models are not necessarily stronger teachers, 
emphasizing response generator and teacher-side factors. Our work differs in that we attribute this phenomenon primarily to the size of the student model.


\begin{AIbox}{\makecell{Takeaway 2: Large Teacher CoT Gap}}
Small student models tend to learn better from small teachers, while large student models benefit more from large teachers.
\end{AIbox}





\begin{table*}[ht]
    \small
    \centering
    \label{tab:combined-performance}
    % \resizebox{0.9\textwidth}{!}{%
    \begin{tabular}{l l c c c c c c}
        \toprule
        Student Model & Distillation Method & MATH & AMC & GSM8k & \makecell{Olympiad \\ Bench} & AIME & Average \\
        \midrule
        \multirow{7}{*}{Qwen2.5-3B} 
            & Long CoT                  & 56.2  & 37.5  & 80.0  & 24.4  & \underline{3.3}  & 40.3  \\
            & Short CoT             & 61.0  & 37.5  & \textbf{82.0}  & 26.4  & \textbf{10.0}       & 43.4  \\
            & Large Model CoT  & 57.5  & 35.0  & 80.0  & 25.9  & 0.0              & 39.7  \\
            & Small Model CoT      & 60.3  & 27.5  & 79.5  & 26.4  & \underline{3.3}  & 39.4  \\
            & Deepseek-R1-32B (Long CoT)           & 50.7  & 20.0  & 81.2  & 15.7  & 0.0              & 33.5  \\
        \cmidrule(lr){2-8}
            & \multicolumn{1}{l}{\textit{Ours}} & & & & & & \\
            & \textbf{Mix-Long}                           & \underline{64.7}  & \textbf{45.0}  & \underline{81.4}  & \underline{28.6}  & \textbf{10.0}  & \textbf{45.9}  \\
            & \textbf{Mix-Large}                         & \textbf{65.8}  & \underline{42.5}  & 81.7  & \textbf{29.0}  & \textbf{10.0}  & \underline{45.8}  \\
        \midrule
        \multirow{7}{*}{Llama3.2-3B} 
            & Long CoT                  & 48.7  & 17.5  & 75.1  & \underline{17.6}  & \underline{3.3}  & 32.5  \\
            & Short CoT             & 50.9  & 15.0  & 77.5  & \textbf{18.7}      & \underline{3.3}  & 33.1  \\
            & Large Model CoT  & 47.4  & \textbf{25.0}  & 71.2  & 16.9  & \underline{3.3}  & 32.8  \\
            & Small Model CoT      & 47.9  & 17.5  & 74.1  & 16.4  & \underline{3.3}  & 31.2  \\
            & Deepseek-R1-32B (Long CoT)           & 48.5  & 17.5  & \underline{77.7}  & 16.1  & \textbf{6.7}   & 33.3  \\
        \cmidrule(lr){2-8}
            & \multicolumn{1}{l}{\textit{Ours}} & & & & & & \\
            & \textbf{Mix-Long}                           & \textbf{53.0}  & \underline{22.5}  & \textbf{79.4}  & 17.2  & \underline{3.3}  & \textbf{35.1}  \\
            & \textbf{Mix-Large}                         & \underline{51.8}  & \textbf{25.0}  & 76.3  & 17.2  & \underline{3.3}  & \underline{34.7}  \\
        \bottomrule
    \end{tabular}%
    % }
    \caption{\textbf{Mix Distillation} outperforms the baseline models across most metrics. We use \texttt{Llama3.2-3B-Instruct} and \texttt{Qwen2.5-3B-Instruct} as the student model and 7.5k samples in MATH dataset as the training set. We distill different teacher models to generate responses as the baseline. Our proposed Mix-Long combines long CoT data and normal CoT data in a 1:4 ratio, while Mix-Large combines strong model response and weak model response with the same proportion. Experimental results demonstrate that both Mix-Long and Mix-Large surpass baselines in most evaluation metrics. The highest score is bolded, and the second highest score is \underline{underlined}.}
\end{table*}


\subsection{Analysis of Small Model Learnability Gap}
\label{More Analysis Results}


\paragraph{Domain knowledge affects learnability gap.}

We observe that math expert models, in spite of small model size,  exhibit a smaller learnability gap for both long CoT and large teacher CoT data compared to general models in Figure \ref{fig:math_expert_vs_general}. 
Specifically, we compare the learnability gaps between the student models \texttt{Qwen2.5-Math-1.5B-Instruct} and \texttt{Qwen2.5-1.5B-Instruct}. Our findings show that the long CoT gap of the small math expert model is significantly smaller than that of general small models. 
Furthermore, the performance improvement of \texttt{Qwen2.5-Math-1.5B} when fined-tuned with large teacher CoT exceeds that of \texttt{Qwen2.5-1.5B}, suggesting that math expert models benefit more substantially from large teacher CoT. We conjecture that a key factor leading to the small model learnability gap is the \textit{limited in-domain knowledge of small student models}.
We summarize this observation in the following takeaway.

\begin{AIbox}{\makecell{Takeaway 3: Effect of Domain Knowledge}} Limited domain knowledge of small models may hinder their learning from strong reasoning teachers.  \end{AIbox}

\begin{figure*}[!t]
    \centering
    \includegraphics[width=\textwidth]{figs/math_expert_vs_general.pdf}
    \caption{Math expert models usually have a less significant Learnability Gap than the general models. 
    A positive Gap means long CoT or large teacher CoT is better while negative means worse. This indicates that the math expert model could more easily learn from long CoT data or large teacher CoT. }
    \label{fig:math_expert_vs_general}
\end{figure*}



\paragraph{Base models exhibit a more significant learnability gap.}
We observe that base models generally exhibit a more significant learnability gap than Instruct models in Figure \ref{fig:Base_vs_Instruct_Gap}. 
This suggests that it is more challenging for small base models to effectively learn from long CoT data or large teacher CoT.


\begin{AIbox}{Takeaway 4: Base vs Instruct}
Small base models experience more significant learnability gap than Instruct models.
\end{AIbox}
\begin{figure*}[!t]
    \centering
    \includegraphics[width=\textwidth]{figs/Base_vs_Instruct_Gap.pdf}
    \caption{Base models generally exhibit a more significant learnability gap than Instruct models. A positive gap indicates that long CoT data or large teacher CoT enhance performance, whereas a negative gap suggests they have the opposite effect. This implies that it is more challenging for small base models to effectively learn from long CoT data or large teacher CoT.}
    % \vspace{-1em}
    \label{fig:Base_vs_Instruct_Gap}
\end{figure*}





\paragraph{Speaking styles shift.}
We adopt the method from  \citep{lin2023unlockingspellbasellms} to evaluate the rank shift of each token before and after fine-tuning on long CoT and Large teacher CoT data. This allows us to compare the token distribution shifts induced by the fine-tuning process. We then annotate the tokens that exhibit the largest rank shifts as the most shifted tokens. Our analysis reveals that these tokens are predominantly associated with expressive and stylistic elements, such as “wait”, “But”, and “Let”. Please see Appendix \ref{Examples of Speaking Style Shift} for more details.


\begin{AIbox}{\makecell{Takeaway 5: Speaking Styles Shift}} 
Long CoT and large teacher CoT primarily shift the student model's  distribution of tokens associated with speaking styles. 
\end{AIbox}




\section{Mix Distillation: Bridge Small Model Learnability Gap}
This section presents our Mix Distillation approach to bridge the small model learnability gap.
\subsection{Mix Distillation}
We propose \textit{Mix Distillation} to address the learnability gap observed in small models. This approach blends easier-to-learn data with more challenging data for small models, thereby leveraging the strengths of both. 

Our insight is that small models tend to perform better on data that closely matches their inherent distribution (such as short CoT or small teacher CoT), while they struggle with data that exhibits greater distribution shifts. The token distribution of the mixed long CoT and large teacher CoT data may become closer to that of small models' inherent distribution, thereby enabling them to learn more effectively from challenging datasets. 

We propose Mix-Long, which combines long CoT and short CoT data with a weight of long CoT $\alpha$ and short CoT $1-\alpha$. Similarly, we proposed Mix-Large, which combines large teacher CoT with a weight of $\alpha$ and small teacher CoT with a weight of $1-\alpha$.


\subsection{Experiment Results}

We use Qwen2.5-3B-Instruct as the student model and MATH (7.5k) as the training set. We distill different teacher models to generate responses as the baseline. They include \texttt{QwQ-32B} (long CoT), \texttt{Qwen2.5-32B} (short CoT), \texttt{Qwen2.5-72B} (large teacher CoT), \texttt{Qwen2.5-3B} (small teacher CoT). We add \texttt{Deepseek-R1-32B} \citep{DeepSeekAI2025DeepseekR1} as the teacher model to generate another set of long CoT data as baseline. We set $\alpha=0.2$ in both configurations of Mix-Long and Mix-Large.


Experimental results demonstrate that both Mix-Long and Mix-Large surpass baselines in most evaluation metrics. We show that the small student model could achieve improved performance by Mix Distillation compared to training on a single dataset. For instance, \texttt{Qwen2.5-3B-Instruct} improves by more than 8 points on MATH and AMC using Mix-Long, compared to direct training on long CoT data. It also shows a more than 7-point gain on MATH, AIME and AMC for \texttt{Qwen2.5-3B-Instruct} by Mix-Large compared with training on large teacher CoT data. This implies that it is easier for small student models to learn from datasets generated by Mix Distillation.

\begin{AIbox}{\makecell{Takeaway 6: Mix Distillation Bridges Gap}}
By mixing long CoT data (resp. large teacher CoTs) and short CoT data (resp. small teacher CoT), the small student model could achieve better performance compared to training on either data alone.
\end{AIbox}


\begin{figure}[!t]
    \centering
    \includegraphics[width=0.35\textwidth]{figs/lg_lmp_mix_weight_ablation.pdf}
    \caption{The average performance varies with the mix weight of long CoT or large teacher CoT data. \texttt{Qwen2.5-3B-Instruct} is chosen as the student model. At a weight of 0.2, mix distillation achieves the highest average performance.}
    % \vspace{-1em}
    \label{fig:lg_lmp_mix_weight_ablation}
\end{figure}







\begin{figure}[!t]
    \centering
    \includegraphics[width=0.5\textwidth]{figs/mix_long.png}
    \caption{Case Study of Mix-Long. Models fine-tuned on long CoT tended to overthink, while those trained on short CoT produced incorrect answers. In contrast, Mix-Long, incorporating branching elements (e.g., “Alternatively”), achieved a balanced reasoning process and arrived at the correct answer.}
    \label{fig:balanced CoT}
\end{figure}

Figure \ref{fig:lg_lmp_mix_weight_ablation} shows the average performance when taking different mix weight $\alpha$ of long CoT data or large teacher CoT. We choose \texttt{Qwen2.5-3B-Instruct} as the student model and find that a weight $\alpha$ of 0.2 achieves the highest average performance across five benchmarks for both Mix-Long and Mix-Large. 

Interestingly, we find that after mixing long CoT and short CoT data, the small student model’s output incorporates characteristics of long CoT, such as a branching process, while maintaining a reduced token length and avoiding overly elaborate thinking. This is illustrated in Figure \ref{fig:balanced CoT}. We observed that the small student model fine-tuned on long CoT data becomes overwhelmed by repeated thoughts and fails to stop, whereas the model fine-tuned on short CoT data produces incorrect answers. In contrast, our proposed Mix-Long, which incorporates branching elements (e.g., the use of “Alternatively”), delivers the correct answer. Additionally, the average token lengths of responses generated by long CoT, short CoT, and Mix-Long are 3384.7, 575.7, and 1248.9, respectively. We suggest that mixing long CoT and short CoT data is a practical approach to achieving a balanced CoT length, thereby enhancing the reasoning capabilities of small student models.


\section{Related Work} \label{sec:related}

% \textbf{Adversarial Attack}
\textbf{Attacks on SLAM.} 
%With the rise of machine learning, 
The robustness of computer vision systems is being actively investigated. With the emergence of adversarial images in the digital domain by adding optimized noise directly to images~\cite{szegedy2013intriguing,carlini2017towards}, researchers find that such attacks also exist physically in the real world \cite{eykholt2018robust,song2018physical,zhao2019seeing}. To fill the gap between attacks in the digital and physical worlds, recent studies have demonstrated that attacks on real-world computer vision systems are practical \cite{eykholt2018robust,li2019adversarial,man2020ghostimage,sharif2016accessorize,zhao2019seeing,zhou2018invisible}. However, attacks on traditional computer vision methods such as SLAM are relatively less explored. \cite{yoshida2022adversarial} proposes an attack against the scan matching algorithm in LiDAR-based SLAM, while most SLAMs in AR/VR devices rely on different sensors like RGB/depth cameras and IMUs. \cite{ikram2022perceptual} and \cite{chen2024adversary} mislead visual SLAM by poisoning the images with special patterns, and \cite{wang2021can} causes the camera to fail using infrared light. In our work, we demonstrate attacks on Visual-Inertial SLAM (VI-SLAM) by perturbing the IMU readings, rather than cameras, and showing its impact on XR user experience. 

\textbf{Acoustic Injection Attacks.} Among various physical attacks, acoustic injection attacks are attractive due to their low cost. Son~\etal~\cite{son2015rocking} were the first to introduce acoustic attacks on MEMS gyroscopes, demonstrating how these attacks could lead to sensor denial-of-service and result in drone crashes. WALNUT~\cite{trippel2017walnut} expanded on this by developing output biasing and control attacks that enable precise manipulation of MEMS accelerometer outputs using modulated sound waves. Wang et al.~\cite{wang2017sonic} demonstrated a sonic gun, showcasing the vulnerability of various smart devices (\eg drones and self-balancing vehicles) to acoustic attacks. Tu et al. \cite{tu2018injected} designed side-swing and switching attacks to alter the outputs of MEMS gyroscopes and accelerometers. Furthermore, Ji et al. \cite{ji2021poltergeist} fool the object detectors by applying acoustic attack to the image stabilizers commonly used in modern cameras. However, none of the existing works study the relationship between the acoustic injections and SLAM outputs on recent XR devices. 

% \zijian{Do we need one session about security in AR/VR?}
% \yicheng{TODO}
%\jiasi{cite the AIVR paper (UMass Amherst?) paper is we have not already. They add IMU perturbation but w/o SLAM, iirc} \yicheng{Cited}

\textbf{XR Security and Privacy.} 
%Security and privacy concerns in XR systems have gained significant attention. 
For single-user XR systems, researchers have demonstrated various side-channel attacks to extract sensitive information (\eg keystrokes) through video feeds~\cite{ling2019know}, head movements~\cite{nair2023unique, slocum2023going}, architectural hints~\cite{zhang2023its,shang2020arspy}, power usage~\cite{li2024dangers}, and EM side-channel leakages~\cite{al2021vr}. In multi-user XR systems, Su et al.~\cite{su2024remote} use avatar motion data to infer keystrokes in shared VR environments. Slocum et al.~\cite{slocum2024doesn} reveal vulnerabilities in the shared state frameworks of multi-user AR. Similarly, Lebeck et al.~\cite{lebeck2017securing} highlight risks like deceptive virtual objects and emphasize access control for managing shared physical and virtual spaces. Ruth et al.~\cite{ruth2019secure} further propose a secure multi-user AR framework focusing on content sharing and permissions.
Chandio et al.~\cite{chandio2024stealthy} %introduced a multi-modal spatiotemporal attack that 
simultaneously manipulated visual and inertial sensors to disrupt XR pose estimation. However, their study evaluated the attack using offline datasets and assumed the attacker's capability to manipulate IMU data streams through acoustic means, without real experiments. Ours is the first to demonstrate acoustic injection attacks on recent XR devices, like the Hololens 2, in the real world.
 



\section*{Conclusion}
This paper aims to enhance our understanding of the computational complexity of computing various Shapley value variants. We found that for various ML models --- including decision trees, regression tree ensembles, weighted automata, and linear regression --- both local and global interventional and baseline SHAP can be computed in polynomial time under HMM modeled distributions. This extends popular algorithms, such as TreeSHAP, beyond their empirical distributional scope. We also establish strict complexity gaps between the various SHAP variants (baseline, interventional, and conditional) and prove the intractability of computing SHAP for tree ensembles and neural networks in simplified scenarios. Overall, we present SHAP as a versatile framework whose complexity depends on four key factors: \begin{inparaenum}[(i)] \item model type, \item SHAP variant, \item distribution modeling approach, \item and local vs. global explanations\end{inparaenum}. We believe this perspective provides deeper insight into the computational complexity of SHAP, paving the way for future work.




%We believe that our framework provides a more intricate understanding of SHAP computation complexity across different models, distributions, and variants, paving the way for further research.

Our work opens promising directions for future research. First, expanding our computational analysis to other SHAP-related metrics, such as asymmetric SHAP~\citep{frye20} and SAGE~\citep{covert2020understanding}, would be valuable. Additionally, we aim to explore more expressive distribution classes and relaxed assumptions beyond those in Section \ref{sec:tractable} while maintaining tractable SHAP computation. Finally, when exact computation is intractable (Section \ref{sec:intractable}), investigating the approximability of SHAP metrics through approximation and parameterized complexity theory~\citep{downey2012parameterized} is an important direction.

%Our work opens several promising avenues for future research on the computational properties of explainable AI methods, with a particular focus on SHAP. First, it would be interesting to broaden the computational analysis conducted in this work to include other popular SHAP-related metrics in the literature, such as asymmetric SHAP \cite{frye20} and SAGE \cite{covert2020understanding}. Also, in the future, we aim to explore more expressive distribution classes and relaxed distributional assumptions—extending beyond those examined in Section \ref{sec:tractable} —that still yield tractable SHAP computation. Finally, when exact computation proves intractable (Section \ref{sec:intractable}), it is worthwhile to theoretically investigate the question of the approximability of computing the SHAP metrics across various configurations, through the lens of approximation and parametrized complexity theory \cite{arora2009computational}.

%This paper aims to deepen our understanding of the computational complexity involved in obtaining different Shapley value variants. We found that for a variety of ML models, including decision trees, tree ensembles for regression, weighted automata, and linear regression models — computing both local and global interventional and baseline SHAP can be done in polynomial time when distributions are modeled by HMMs. This extends the distributional scope of popular algorithms like TreeSHAP, which is limited to empirical distributions. Additionally, we demonstrate a strict complexity gap between SHAP variants, showing that interventional and baseline SHAP can be strictly easier to compute than conditional SHAP. Despite these positive results, we uncovered intractability for various SHAP variants in neural networks and tree ensembles. Finally, we provided generalized complexity relations across SHAP variants. We believe that our framework offers a deeper understanding of the complexity involved in computing SHAP across various variants, models, distributions, as well as in both local and global computations, laying the groundwork for future research.
One limitation of this study is that it only evaluated LLaVA as the target Vision Language Model (VLM), which may limit the generalizability of the findings to other models. Additionally, the alignment of visual attention heatmaps for non-existing objects was not assessed, indicating that further analysis is needed in this area. 

Moreover, the experiments were conducted solely using the MSCOCO dataset, and future work should expand the evaluation to include additional datasets to ensure the robustness and broader applicability of the results. Furthermore, since datasets that contain both questions and corresponding answers alongside matching segmentation data, which can be used to evaluate object hallucination, are scarce, it may be necessary to develop such datasets.

This research focuses on analyzing the internal representations of numerical values in Large Language Models and does not involve human subjects, sensitive data, or direct societal impact.


\bibliography{custom}


\clearpage
\appendix
\section{Detailed Experimental Setups}
\label{appendix:More on Experimental Setups}


\begin{table}[h]
    \centering
    \small
    \begin{tabular}{ll}
        \toprule
        \textbf{Category} & \textbf{Models} \\
        \midrule
        \multicolumn{2}{c}{\textbf{Teacher Models}} \\
        \midrule
        \textbf{Long CoT vs} & QwQ-32B-Preview vs\\\textbf{ShortCoT}& Qwen2.5-32B-Instruct \\
        \midrule
        \textbf{Large Teacher vs}  & \\
        \textbf{Small Teacher}& \\
        \textit{Qwen Family} & Qwen2.5-72B-Instruct vs\\& Qwen2.5-3B-Instruct \\
        \textit{Llama Family} & Llama3.1-70B-Instruct vs\\& Llama3.1-8B-Instruct \\
        \textit{Gemma Family} & Gemma2-27B-it vs\\& Gemma2-9B-it \\
        \midrule
        \multicolumn{2}{c}{\textbf{Student Models}} \\
        \midrule
        \textit{Qwen Family} & Qwen2.5-0.5B-Instruct,\\& Qwen2.5-1.5B-Instruct, \\
        & Qwen2.5-3B-Instruct, \\&Qwen2.5-7B-Instruct, \\
        & Qwen2.5-14B-Instruct,\\& Qwen2.5-32B-Instruct \\
        \textit{Llama Family} & Llama3.2-1B-Instruct,\\& Llama3.2-3B-Instruct, \\
        & Llama3.1-8B-Instruct,\\& Llama3.3-70B-Instruct \\
        \bottomrule
    \end{tabular}
    \caption{Overview of Teacher and Student Models}
    \label{tab:models_overview}
\end{table}




\subsection{Models}
Table \ref{tab:models_overview} presents a comprehensive overview of student and teacher models used in our paper. 




\subsection{Training Setup}
\label{appx:training-setup}
Our model training is conducted using LLaMA-Factory \citep{zheng2024llamafactory}, on a server with four NVIDIA A100-SXM4-80GB GPUs, an AMD EPYC 7763 64-Core Processor, and 512 GB of RAM. We use full parameter fine-tuning on student models less than 14B parameters. 
When the student model is larger than 14B, we use LoRA fine-tuning \cite{hu2021loralowrankadaptationlarge}. 
Table \ref{tab: training-hyperparameters} and Table \ref{tab: training-lora-hyperparameters} list hyper-parameters for full parameter fine-tuning and LoRA fine-tuning respectively. 


\begin{table}[!h]
\small
\centering
\resizebox{0.8\columnwidth}{!}{
\begin{tabular}{ll}
\toprule
\textbf{Hyper-parameter} & \textbf{Value} \\ \midrule
Learning Rate & $1 \times 10^{-5}$ \\
Number of Epochs & $2$ \\
Number of Devices & $4$ \\
Per-device Batch Size & $2$ \\
Optimizer & \texttt{Adamw} \\
Learning Rate Scheduler & \texttt{cosine} \\
Max Sequence Length  & $16384$ \\ \bottomrule
\end{tabular}
}
\caption{This table shows the hyper-parameters for full parameter fine-tuning.}
\label{tab: training-hyperparameters}
\end{table}


\begin{table}[!h]
\small
\centering
\resizebox{0.8\columnwidth}{!}{
\begin{tabular}{ll}
\toprule
\textbf{Hyper-parameter} & \textbf{Value} \\ \midrule
Learning Rate & $1 \times 10^{-4}$ \\
Number of Epochs & $2$ \\
Number of Devices & $4$ \\
Per-device Batch Size & $1$ \\
Lora Target & \texttt{full} \\
Learning Rate Scheduler & \texttt{cosine} \\
Warmup Ratio & $0.03$ \\
Max Sequence Length  & $16384$ \\ \bottomrule
\end{tabular}
}
\caption{This table shows the hyper-parameters for LoRA fine-tuning.}
\label{tab: training-lora-hyperparameters}
\end{table}



Teacher models generate responses by rejection sampling \citep{zelikman2022starbootstrappingreasoningreasoning,tong2024dartmathdifficultyawarerejectiontuning,yue2023mammothbuildingmathgeneralist,singh2024humandatascalingselftraining,gulcehre2023reinforcedselftrainingrestlanguage,yuan2023scalingrelationshiplearningmathematical,dong2023raftrewardrankedfinetuning}. The prompt used to make teacher models generate responses to the math questions is shown as follows.
By default, teacher models employ greedy decoding. 
By combining the math problem instructions with corresponding solutions generated by teacher models, we construct problem-solution pairs to fine-tune student models. 
We perform pairwise comparisons of solutions generated by different teacher models and filter out problem-solution pairs that are correct for both models to fine-tune student models.

\begin{figure}[htbp]
    \centering
\begin{tcolorbox}[title=Prompt, promptstyle]
\lstset{
    basicstyle=\normalfont\sffamily\footnotesize,
    breaklines=true,
    frame=none,
    columns=fullflexible,
}
Solve the following math problem. Present the final answer in the format: Final Answer: $\boxed{\{\texttt{your answer}\}}$

Problem: \{problem\}

Answer:
\end{tcolorbox}
    \label{fig: train_sampling_prompt}
\end{figure}





\subsection{Evaluation Setup} 


We evaluate the reasoning capability of fine-tuned student models on a set of commonly used benchmarks, including MATH \citep{hendrycks2021measuringmathematicalproblemsolving}, GSM8K \citep{cobbe2021trainingverifierssolvemath}, AMC 2023, AIME 2024, and the English math subset of OlympiadBench \citep{he2024olympiadbenchchallengingbenchmarkpromoting}. 

Unless otherwise specified, all fine-tuned models are evaluated in a zero-shot setting using greedy decoding. We set the maximum generation tokens as 16k. The evaluation prompt is shown below. 


\begin{figure}[htbp]
    \centering
\begin{tcolorbox}[title=Prompt, promptstyle]
\lstset{
    basicstyle=\normalfont\sffamily\footnotesize,
    breaklines=true,
    frame=none,
    columns=fullflexible,
}
Solve the following math problem and present the final answer in the format: Final Answer: $\boxed{\{\texttt{your answer}\}}$

Problem: \{problem\}

Answer:
\end{tcolorbox}
    \label{fig: evaluation_prompt}
\end{figure}


After extracting the final answer of the evaluated model, we first employ exact matching to determine the correctness of the answer. If the answer is incorrect, we use Qwen-32B-Instruct as a judge to compare the extracted final answers against that of the ground truth. The prompt is shown below.


\begin{figure}[htbp]
    \centering
\begin{tcolorbox}[title=Prompt, promptstyle]
\lstset{
    basicstyle=\normalfont\sffamily\footnotesize,
    breaklines=true,
    frame=none,
    columns=fullflexible,
}
Given a math problem, its correct final answer, and the model's generated final answer, determine if the model's answer is correct. Respond with 'True' if the it is correct and 'False' if it is incorrect. 

Problem: \texttt{\{problem\}}

Correct Final Answer: \texttt{\{ground truth\}}

Model's Generated Final Answer: \texttt{\{resp answer\}}

Your Judgement:
\end{tcolorbox}
    \label{fig: score_prompt}
\end{figure}





            




\section{More Experiments Results}
\label{appendix: More Experiments}
In this section we present additional experiment results of long CoT gap and large teacher CoT gap.
\subsection{Long CoT Gap: Additional Results}
Table \ref{tab:full_performance_lg} shows the detailed performance scores and gap of each benchmark for different student models fine-tuned on long CoT and short CoT.
\texttt{QwQ-32B-Preview} is chosen to generate long CoT and awhile \texttt{Qwen-2.5-32B-Instruct} is chosen to generate short CoT. We observe that small student models tend to benefit more from short CoT, while large student models gain greater advantages from long CoT. 

\begin{table*}[htbp]
  \centering
  \resizebox{1\textwidth}{!}{%
  \begin{tabular}{l*{5}{ccc}c}
    \toprule
    & \multicolumn{3}{c}{MATH} 
    & \multicolumn{3}{c}{GSM8K} 
    & \multicolumn{3}{c}{AIME} 
    & \multicolumn{3}{c}{AMC} 
    & \multicolumn{3}{c}{Olympiad} 
    & \multicolumn{1}{c}{Average $\Delta_{\rm Long}$} \\
    \cmidrule(lr){2-4} \cmidrule(lr){5-7} \cmidrule(lr){8-10} \cmidrule(lr){11-13} \cmidrule(lr){14-16}
    Model 
    & $P_{\rm Long}$ & $P_{\rm Short}$ & $\Delta_{\rm Long}$ 
    & $P_{\rm Long}$ & $P_{\rm Short}$ & $\Delta_{\rm Long}$ 
    & $P_{\rm Long}$ & $P_{\rm Short}$ & $\Delta_{\rm Long}$ 
    & $P_{\rm Long}$ & $P_{\rm Short}$ & $\Delta_{\rm Long}$ 
    & $P_{\rm Long}$ & $P_{\rm Short}$ & $\Delta_{\rm Long}$ 
    & \\ 
    \midrule

    Llama-3.2-1B  & 28.6  & 33.4  & \cellcolor{red!32} -4.78  & 42.3  & 49.2  & \cellcolor{red!46} -6.90  & 0.00  & 0.00  & 0.00  & 2.50  & 7.50  & \cellcolor{red!33} -5.00  & 5.48  & 7.40  & \cellcolor{red!13} -1.92  & \cellcolor{red!25} -3.72 \\
    Llama-3.2-3B  & 48.7  & 50.9  & \cellcolor{red!14} -2.14  & 75.1  & 77.5  & \cellcolor{red!16} -2.42  & 3.33  & 3.33  & \cellcolor{white} 0.00  & 17.5  & 15.0  & \cellcolor{green!17} 2.50  & 17.6  & 18.7  & \cellcolor{red!7} -1.04  & \cellcolor{red!4} -0.619 \\
    Llama-3.1-8B  & 50.0  & 44.6  & \cellcolor{green!36} 5.36   & 81.4  & 75.5  & \cellcolor{green!39} 5.84   & 0.00  & 0.00  & \cellcolor{white} 0.00   & 27.5  & 22.5  & \cellcolor{green!33} 5.00   & 17.3  & 14.8  & \cellcolor{green!17} 2.52   & \cellcolor{green!25} 3.74 \\
    Llama-3.3-70B & 75.3  & 74.9  & \cellcolor{green!2} 0.340   & 92.7  & 91.2  & \cellcolor{green!10} 1.44   & 26.7  & 13.3  & \cellcolor{green!89} 13.3   & 55.0  & 52.5  & \cellcolor{green!17} 2.50   & 41.3  & 39.7  & \cellcolor{green!11} 1.63   & \cellcolor{green!26} 3.85 \\
    \midrule
    Qwen2.5-0.5B  & 23.0  & 31.5  & \cellcolor{red!56} -8.44  & 39.5  & 45.3  & \cellcolor{red!39} -5.84  & 0.00  & 0.00  & \cellcolor{white} 0.00  & 7.50  & 15.0  & \cellcolor{red!50} -7.50  & 4.00  & 5.93  & \cellcolor{red!13} -1.93  & \cellcolor{red!32} -4.74 \\
    Qwen2.5-1.5B  & 41.6  & 52.3  & \cellcolor{red!71} -10.7  & 63.8  & 71.7  & \cellcolor{red!53} -7.89  & 0.00  & 0.00  & \cellcolor{white} 0.00  & 17.5  & 27.5  & \cellcolor{red!67} -10.0  & 12.3  & 19.4  & \cellcolor{red!47} -7.11  & \cellcolor{red!48} -7.13 \\
    Qwen2.5-3B   & 56.2  & 61.0  & \cellcolor{red!32} -4.84  & 80.0  & 82.0  & \cellcolor{red!13} -1.98  & 3.33  & 10.0  & \cellcolor{red!44} -6.67  & 37.5  & 37.5  & \cellcolor{white} 0.00  & 24.4  & 26.4  & \cellcolor{red!13} -1.93  & \cellcolor{red!21} -3.08 \\
    Qwen2.5-7B   & 68.2  & 67.8  & \cellcolor{green!3} 0.460    & 86.2  & 85.7  & \cellcolor{green!4} 0.560    & 13.3  & 6.67  & \cellcolor{green!44} 6.67    & 40.0  & 40.0  & \cellcolor{white} 0.00   & 36.6  & 35.7  & \cellcolor{green!6} 0.889    & \cellcolor{green!11} 1.72 \\
    Qwen2.5-14B  & 78.3  & 76.2  & \cellcolor{green!14} 2.04   & 93.3  & 92.5  & \cellcolor{green!5} 0.760    & 20.0  & 6.67  & \cellcolor{green!89} 13.3   & 60.0  & 55.0  & \cellcolor{green!33} 5.00   & 44.4  & 40.9  & \cellcolor{green!24} 3.56   & \cellcolor{green!33} 4.94 \\
    Qwen2.5-32B  & 84.8  & 82.3  & \cellcolor{green!16} 2.44   & 94.9  & 94.3  & \cellcolor{green!4} 0.610    & 40.0  & 10.0  & \cellcolor{green!100} 30.0  & 85.0  & 62.5  & \cellcolor{green!100} 22.5  & 60.4  & 47.3  & \cellcolor{green!88} 13.2  & \cellcolor{green!91} 13.7 \\

    \bottomrule
  \end{tabular}
  }


\caption{This table summarizes the performance of models in Llama and Qwen families fine-tuned with long CoT and short CoT data. They are evaluated on MATH, GSM8K, AIME, AMC, and OlympiadBench. \texttt{QwQ-32B-Preview} is chosen to generate long CoT and awhile \texttt{Qwen-2.5-32B-Instruct} is chosen to generate short CoT. We observe that small student models tend to benefit more from short CoT, while large student models gain greater advantages from long CoT.}
\label{tab:full_performance_lg}
\end{table*}

\subsection{Large Teacher CoT Gap: Additional Results}
Table \ref{tab:lmp-full_comparison} shows the detailed performance scores and gap of each benchmark for different student models distilled from large teacher and small teacher.
We summarize the performance of 10 student models from the Llama and Qwen families across various model sizes. \texttt{Qwen-2.5-72B-Instruct} is chosen as the large teacher while \texttt{Qwen-2.5-3B-Instruct} is chosen as the small teacher. The results are shown in Table \ref{tab:lmp-full_comparison}. Our findings indicate that small student models may experience degraded performance when distilled from a large teacher compared to a small teacher, whereas larger student models benefit more from distilling a large teacher.

Table \ref{tab:lmp_comparison2} shows more experiment results for teacher models in different model families, including \texttt{Gemma-27B-it} vs \texttt{Gemma-9B-it} and \texttt{Llama3.1-72B-Instruct} vs \texttt{Llama3.1-8B-Instruct}. 


\begin{table*}[htbp]
  \centering
  \resizebox{1\textwidth}{!}{%
  \begin{tabular}{l*{5}{ccc}c}
    \toprule
    & \multicolumn{3}{c}{MATH} 
    & \multicolumn{3}{c}{GSM8k} 
    & \multicolumn{3}{c}{AIME} 
    & \multicolumn{3}{c}{AMC} 
    & \multicolumn{3}{c}{Olympiad} 
    & \multicolumn{1}{c}{\makecell{Average \\ $\Delta_{\rm Strong}$}} \\
    \cmidrule(lr){2-4} \cmidrule(lr){5-7} \cmidrule(lr){8-10} \cmidrule(lr){11-13} \cmidrule(lr){14-16}
    Model 
    & $P_{\rm Strong}$ & $P_{\rm Weak}$ & $\Delta_{\rm Strong}$ 
    & $P_{\rm Strong}$ & $P_{\rm Weak}$ & $\Delta_{\rm Strong}$ 
    & $P_{\rm Strong}$ & $P_{\rm Weak}$ & $\Delta_{\rm Strong}$ 
    & $P_{\rm Strong}$ & $P_{\rm Weak}$ & $\Delta_{\rm Strong}$ 
    & $P_{\rm Strong}$ & $P_{\rm Weak}$ & $\Delta_{\rm Strong}$ 
    & \\ 
    \midrule
    Llama-3.2-1B      & 29.8   & 29.6   & \cellcolor{green!2}{0.160} 
                      & 44.4   & 47.5   & \cellcolor{red!32}{-3.18} 
                      & 0.00   & 0.00   & \cellcolor{white}{0.00} 
                      & 2.50   & 7.50   & \cellcolor{red!50}{-5.00} 
                      & 6.07   & 7.70   & \cellcolor{red!16}{-1.63} 
                      & \cellcolor{red!19}{-1.93} \\
    Llama-3.2-3B      & 47.4   & 47.9   & \cellcolor{red!5}{-0.500} 
                      & 71.2   & 74.1   & \cellcolor{red!29}{-2.88} 
                      & 3.33   & 0.00   & \cellcolor{green!33}{3.33} 
                      & 25.0   & 17.5   & \cellcolor{green!75}{7.50} 
                      & 16.9   & 16.4   & \cellcolor{green!4}{0.445} 
                      & \cellcolor{green!16}{1.58} \\
    Llama-3.2-8B      & 37.6   & 37.6   & \cellcolor{red!1}{-0.040} 
                      & 67.0   & 69.2   & \cellcolor{red!22}{-2.20} 
                      & 6.67   & 0.00   & \cellcolor{green!67}{6.67} 
                      & 7.50   & 7.50   & \cellcolor{white}{0.00} 
                      & 9.19   & 11.0   & \cellcolor{red!18}{-1.78} 
                      & \cellcolor{green!5}{0.530} \\
    Llama-3.2-70B     & 74.5   & 72.2   & \cellcolor{green!23}{2.28} 
                      & 92.0   & 92.2   & \cellcolor{red!2}{-0.152} 
                      & 16.7   & 16.7   & \cellcolor{white}{0.00} 
                      & 67.5   & 50.0   & \cellcolor{green!100}{17.5} 
                      & 37.3   & 35.7   & \cellcolor{green!16}{1.63} 
                      & \cellcolor{green!43}{4.25} \\
    \midrule
    Qwen2.5-0.5B      & 30.0   & 31.0   & \cellcolor{red!9}{-0.920} 
                      & 43.1   & 45.4   & \cellcolor{red!24}{-2.35} 
                      & 0.00   & 0.00   & \cellcolor{white}{0.00} 
                      & 5.00   & 17.5   & \cellcolor{red!100}{-12.5} 
                      & 6.52   & 8.30   & \cellcolor{red!18}{-1.78} 
                      & \cellcolor{red!35}{-3.51} \\
    Qwen2.5-1.5B      & 50.3   & 50.7   & \cellcolor{red!4}{-0.440} 
                      & 70.6   & 71.0   & \cellcolor{red!5}{-0.455} 
                      & 0.00   & 3.33   & \cellcolor{red!33}{-3.33} 
                      & 22.5   & 20.0   & \cellcolor{green!25}{2.50} 
                      & 17.8   & 20.0   & \cellcolor{red!22}{-2.22} 
                      & \cellcolor{red!8}{-0.790} \\
    Qwen2.5-3B        & 57.5   & 60.3   & \cellcolor{red!28}{-2.82} 
                      & 79.9   & 79.5   & \cellcolor{green!4}{0.379} 
                      & 0.00   & 3.33   & \cellcolor{red!33}{-3.33} 
                      & 35.0   & 27.5   & \cellcolor{green!75}{7.50} 
                      & 25.9   & 26.4   & \cellcolor{red!4}{-0.444} 
                      & \cellcolor{green!3}{0.256} \\
    Qwen2.5-7B        & 71.3   & 63.6   & \cellcolor{green!77}{7.66} 
                      & 87.8   & 84.1   & \cellcolor{green!37}{3.72} 
                      & 6.67   & 0.00   & \cellcolor{green!67}{6.67} 
                      & 40.0   & 35.0   & \cellcolor{green!50}{5.00} 
                      & 38.8   & 29.0   & \cellcolor{green!98}{9.78} 
                      & \cellcolor{green!66}{6.56} \\
    Qwen2.5-14B       & 76.4   & 72.8   & \cellcolor{green!37}{3.66} 
                      & 93.1   & 89.6   & \cellcolor{green!35}{3.49} 
                      & 6.67   & 3.33   & \cellcolor{green!33}{3.33} 
                      & 47.5   & 45.0   & \cellcolor{green!25}{2.50} 
                      & 41.0   & 39.0   & \cellcolor{green!21}{2.07} 
                      & \cellcolor{green!30}{3.01} \\
    Qwen2.5-32B       & 80.5   & 76.8   & \cellcolor{green!37}{3.72} 
                      & 92.2   & 92.7   & \cellcolor{red!5}{-0.531} 
                      & 20.0   & 3.33   & \cellcolor{green!100}{16.7} 
                      & 57.5   & 50.0   & \cellcolor{green!75}{7.50} 
                      & 47.4   & 42.4   & \cellcolor{green!50}{5.04} 
                      & \cellcolor{green!65}{6.48} \\
    \bottomrule
  \end{tabular}
  }


\caption{This table summarizes the performance of models in Llama and Qwen families fine-tuned with large teacher CoT and small teacher CoT when evaluated on MATH, GSM8K, AIME, AMC, and OlympiadBench. \texttt{Qwen-2.5-72B-Instruct} is chosen as the large teacher while \texttt{Qwen-2.5-3B-Instruct} is chosen as the small teacher. We observe that small student models may experience degraded performance when distilled from a large teacher compared to a small teacher, whereas larger student models benefit more from the distilling a large teacher.}
\label{tab:lmp-full_comparison}
\end{table*}




\begin{table*}[htbp]
  \centering
  \resizebox{1\textwidth}{!}{%
  \begin{tabular}{l *{6}{c} *{6}{c}}
    \toprule
    & \multicolumn{6}{c}{Gemma2-9B vs Gemma2-27B} & \multicolumn{6}{c}{Llama3.1-8B vs Llama3.1-70B} \\
    \cmidrule(lr){2-7} \cmidrule(lr){8-13}
    Model & MATH & AMC & Olympiad & AIME & GSM8k & Average & MATH & AMC & Olympiad & AIME & GSM8k & Average \\
    \midrule
    Llama3.2-1B   & \cellcolor{white}{-1.42}  & \cellcolor{white}{-7.50}  & \cellcolor{white}{0.00}   & \cellcolor{white}{0.00}   & \cellcolor{white}{-0.227}  & \cellcolor{red!61}{-1.83}   & \cellcolor{white}{-1.42}  & \cellcolor{white}{-5.00}  & \cellcolor{white}{-0.296} & \cellcolor{white}{3.33}  & \cellcolor{white}{0.152}  & \cellcolor{red!22}{-0.646} \\
    Llama3.2-3B   & \cellcolor{white}{2.08}   & \cellcolor{white}{-7.50}  & \cellcolor{white}{-0.888} & \cellcolor{white}{0.00}   & \cellcolor{white}{1.67}   & \cellcolor{red!31}{-0.928}  & \cellcolor{white}{-0.14}  & \cellcolor{white}{10.0}   & \cellcolor{white}{-0.593} & \cellcolor{white}{3.33}  & \cellcolor{white}{1.06}   & \cellcolor{green!91}{2.73} \\
    Llama3.1-8B   & \cellcolor{white}{0.56}   & \cellcolor{white}{0.00}   & \cellcolor{white}{0.078} & \cellcolor{white}{0.00}   & \cellcolor{white}{-0.516}  & \cellcolor{green!5}{0.0243}  & \cellcolor{white}{-2.18}  & \cellcolor{white}{7.50}   & \cellcolor{white}{2.67}  & \cellcolor{white}{0.00}  & \cellcolor{white}{-1.29}  & \cellcolor{green!45}{1.34} \\
    Llama3.1-70B  & \cellcolor{white}{0.02}   & \cellcolor{white}{7.50}   & \cellcolor{white}{-0.741} & \cellcolor{white}{10.0}   & \cellcolor{white}{0.152}   & \cellcolor{green!100}{3.39}  & \cellcolor{white}{2.72}   & \cellcolor{white}{17.5}   & \cellcolor{white}{5.48}  & \cellcolor{white}{6.67}  & \cellcolor{white}{0.986}   & \cellcolor{green!100}{6.67} \\
\midrule
    Qwen2.5-0.5B  & \cellcolor{white}{-4.56}  & \cellcolor{white}{0.00}   & \cellcolor{white}{0.741} & \cellcolor{white}{0.00}   & \cellcolor{white}{0.592}   & \cellcolor{red!22}{-0.645}  & \cellcolor{white}{-1.88}  & \cellcolor{white}{0.00}   & \cellcolor{white}{0.185} & \cellcolor{white}{0.00}  & \cellcolor{white}{-1.74}  & \cellcolor{red!23}{-0.688} \\
    Qwen2.5-1.5B  & \cellcolor{white}{-1.20}  & \cellcolor{white}{2.50}   & \cellcolor{white}{-1.19} & \cellcolor{white}{0.00}   & \cellcolor{white}{-0.986}  & \cellcolor{red!6}{-0.174}  & \cellcolor{white}{-1.48}  & \cellcolor{white}{5.00}   & \cellcolor{white}{-0.148} & \cellcolor{white}{3.33}  & \cellcolor{white}{-1.14}  & \cellcolor{green!37}{1.11} \\
    Qwen2.5-3B    & \cellcolor{white}{0.44}   & \cellcolor{white}{5.00}   & \cellcolor{white}{1.78}  & \cellcolor{white}{0.00}   & \cellcolor{white}{-0.758}  & \cellcolor{green!43}{1.29}   & \cellcolor{white}{-1.26}  & \cellcolor{white}{5.00}   & \cellcolor{white}{-0.741} & \cellcolor{white}{-3.33} & \cellcolor{white}{-1.29}  & \cellcolor{red!11}{-0.325} \\
    Qwen2.5-7B    & \cellcolor{white}{0.22}   & \cellcolor{white}{5.00}   & \cellcolor{white}{1.04}  & \cellcolor{white}{-3.33}  & \cellcolor{white}{3.94}   & \cellcolor{green!46}{1.37}   & \cellcolor{white}{3.68}   & \cellcolor{white}{20.0}   & \cellcolor{white}{4.15}  & \cellcolor{white}{3.33}  & \cellcolor{white}{2.81}   & \cellcolor{green!100}{6.79} \\
    Qwen2.5-14B   & \cellcolor{white}{1.32}   & \cellcolor{white}{2.50}   & \cellcolor{white}{-0.148} & \cellcolor{white}{0.00}   & \cellcolor{white}{-0.986}  & \cellcolor{green!18}{0.537}  & \cellcolor{white}{2.18}   & \cellcolor{white}{0.00}   & \cellcolor{white}{0.445}  & \cellcolor{white}{3.33}  & \cellcolor{white}{-0.303}  & \cellcolor{green!38}{1.13} \\
    Qwen2.5-32B   & \cellcolor{white}{0.10}   & \cellcolor{white}{2.50}   & \cellcolor{white}{1.48}  & \cellcolor{white}{3.44}   & \cellcolor{white}{1.36}   & \cellcolor{green!59}{1.78}  & \cellcolor{white}{2.72}   & \cellcolor{white}{-2.50}  & \cellcolor{white}{5.63}  & \cellcolor{white}{3.33}  & \cellcolor{white}{0.834}   & \cellcolor{green!67}{2.00} \\
    \bottomrule
  \end{tabular}
  }
  \caption{This table presents the performance of student models distilled from different teacher models, including \texttt{Gemma-27B-it} vs \texttt{Gemma-9B-it} and \texttt{Llama3.1-72B-Instruct} vs \texttt{Llama3.1-8B-Instruct}. We observe that small student models may experience degraded performance when distilled from a large teacher compared to a small teacher, whereas larger student models benefit more from the distilling a large teacher.}

  \label{tab:lmp_comparison2}
\end{table*}

\section{Examples of Speaking Style Shift}
\label{Examples of Speaking Style Shift}
We adopt the method from  \citep{lin2023unlockingspellbasellms} to evaluate the most shifted tokens after fine-tuning on long CoT and Large teacher CoT data. Figure \ref{fig:speaking_style} shows the calculation process. This allows us to compare the token distribution shifts induced by the fine-tuning process. We annotate the tokens that exhibit the largest rank shifts as the most shifted tokens. We choose \texttt{Qwen2.5-3B-Instruct} as the student model. We put the results of most shifted tokens after fine-tuning on long CoT data in Figure \ref{fig: most_shifted_1} and \ref{fig: most_shifted_2}. The results of most shifted tokens after fine-tuning on large teacher CoT data are shown in Figure \ref{fig: most_shifted_3}. Our analysis reveals that these tokens are predominantly associated with expressive and stylistic elements, such as “wait”, “But”, and “Let”. 


\begin{figure*}[!t]
    \centering
    \includegraphics[width=1\textwidth]{figs/speaking_way_shift.png}
    \caption{The process of calculating most shifted tokens. We decode each token generated by the fine-tuned LLM in the student model before fine-tuning. Then we calculate the rank shift in the student model for each token generated by the fine-tuned model. We annotate the tokens that exhibit the largest rank shifts as the most shifted tokens. We found that these tokens are predominantly associated with expressive and stylistic elements, such as “But” and “Let”.}
    \label{fig:speaking_style}
\end{figure*}


\begin{figure*}[htbp]
    \centering
\begin{tcolorbox}[title=Problem, promptstyle]
\lstset{
    basicstyle=\normalfont\sffamily\footnotesize,
    breaklines=true,
    frame=none,
    columns=fullflexible,
}
Let $(x,y)$ be an ordered pair of real numbers that satisfies the equation $x^2+y^2=14x+48y$. What is the maximum value of $y$?
\end{tcolorbox}
\end{figure*}






\begin{figure*}[htbp]
    \centering
\begin{tcolorbox}[title=Example of \textcolor{red}{most shifted tokens} after fine-tuning on long CoT data. (1/2), promptstyle]
\lstset{
    basicstyle=\normalfont\sffamily\footnotesize,
    breaklines=true,
    frame=none,
    columns=fullflexible,
}
So I've got this math problem here\textcolor{red}{.} It says\textcolor{red}{:} ``\textcolor{red}{Let} (\textcolor{red}{x}\textcolor{red}{,} y\textcolor{red}{)} be an ordered pair of real numbers that satisfies the equation \(x^2 + y^2 = 14x + 48y\)\textcolor{red}{.} What is the maximum value of \(y\)\textcolor{red}{?} Okay\textcolor{red}{,} let's see how to approach this.

\textcolor{red}{First}\textcolor{red}{,} I need to understand what the equation is telling me\textcolor{red}{.} It's an equation involving \(x\) and \(y\)\textcolor{red}{,} and it's set equal to zero\textcolor{red}{:} 
\[
x^2 + y^2 - 14x - 48y = 0.
\]
I need to find the maximum value of \(y\) for real numbers \(x\) and \(y\) that satisfy this equation.

Hmm\textcolor{red}{,} maybe I can rearr\textcolor{red}{ange} this equation to make it easier to work with\textcolor{red}{.} Let's see:
\[
x^2 - 14x + y^2 - 48y = 0.
\]
I can complete the square for both \(x\) and \(y\) terms.

\textcolor{red}{For} \(x\)\textcolor{red}{:} \(x^2 - 14x\)\textcolor{red}{,} to complete the square\textcolor{red}{,} I take half of \(-14\), which is \(-7\), and square it to get \(49\)\textcolor{red}{.} So\textcolor{red}{,} \(x^2 - 14x\) becomes \((x - 7\textcolor{red}{)}^2 - 49\).

\textcolor{red}{For} \(y\): \(y^2 - 48y\), half of \(-48\) is \(-24\), squared is \(576\)\textcolor{red}{.} So, \(y^2 - 48y\) becomes \((y - 24\textcolor{red}{)}^2 - 576\).

\textcolor{red}{Now}\textcolor{red}{,} pl\textcolor{red}{ugging} these back into the equation:
\[
(x - 7\textcolor{red}{)}^2 - 49 + (y - 24\textcolor{red}{)}^2 - 576 = 0.
\]
S\textcolor{red}{implify}:
\[
(x - 7\textcolor{red}{)}^2 + (y - 24)^2 - 625 = 0,
\]
\[
\text{So, } (x - 7\textcolor{red}{)}^2 + (y - 24)^2 = 625.
\]

\textcolor{red}{This} looks like the equation of a circle with center at \((7, 24\textcolor{red}{)}\) and radius \(25\).

Wait\textcolor{red}{,} but the problem is asking for the maximum value of \(y\)\textcolor{red}{,} and I have a circle equation\textcolor{red}{.} \textcolor{red}{So}\textcolor{red}{,} the maximum \(y\) would be the \(y\)-coordinate of the highest point on the circle\textcolor{red}{,} which is the center plus the radius.

Wait\textcolor{red}{,} but the center is at \((7\textcolor{red}{,} 24)\), and the radius is \(25\)\textcolor{red}{,} so the highest point would be at \((7\textcolor{red}{,} 24 + 25) = (7\textcolor{red}{,} 49)\).

\textcolor{red}{But}\textcolor{red}{,} let me double-check if this makes sense\textcolor{red}{.} The equation is 
\[
x^2 + y^2 = 14x + 48y,
\]
and I've transformed it into 
\[
(\textcolor{red}{x} - 7\textcolor{red}{)}^2 + (y - 24)^2 = 625,
\]
which is indeed a circle with center \((7, 24\textcolor{red}{)}\) and radius \(25\).

\textcolor{red}{So}\textcolor{red}{,} the maximum \(y\) would be \(24 + 25 = 49\).

\textcolor{red}{But}\textcolor{red}{,} wait\textcolor{red}{,} the problem says ``\textcolor{red}{the} maximum value of \(y\)\textcolor{red}{,}'' and I've found that to be \(49\)\textcolor{red}{.} But let me consider if there are any other constraints or if I've missed something.

\textcolor{red}{Let} me think about the original equation:
\[
x^2 + y^2 = 14x + 48y.
\]
I can rearr\textcolor{red}{ange} it to:
\[
x^2 - 14x + y^2 - 48y = 0.
\]

\end{tcolorbox}
\caption{Example of \textcolor{red}{most shifted tokens} after fine-tuning on long CoT data. All the most shifted tokens are labeled as red.}
    \label{fig: most_shifted_1}
\end{figure*}



\begin{figure*}[htbp]
    \centering
\begin{tcolorbox}[title=Example of \textcolor{red}{most shifted tokens} after fine-tuning on long CoT data (2/2), promptstyle]
\lstset{
    basicstyle=\normalfont\sffamily\footnotesize,
    breaklines=true,
    frame=none,
    columns=fullflexible,
}
\textcolor{red}{Then}\textcolor{red}{,} completing the square:
\[
(x - 7\textcolor{red}{)}^2 - 49 + (y - 24)^2 - 576 = 0,
\]
\[
(x - 7\textcolor{red}{)}^2 + (y - 24)^2 = 625.
\]
Yes\textcolor{red}{,} that's a circle with center \((7, 24\textcolor{red}{)}\) and radius \(25\).

\textcolor{red}{So}\textcolor{red}{,} the maximum \(y\) is indeed \(24 + 25 = 49\).

\textcolor{red}{But}\textcolor{red}{,} just to be thorough\textcolor{red}{,} let's consider if there are any points on the circle where \(y\) is greater than \(49\)\textcolor{red}{.} The center is at \((7, 24\textcolor{red}{)}\), and the radius is \(25\)\textcolor{red}{,} so the highest point is at \((7, 49\textcolor{red}{)}\), and the lowest point would be at \((7\textcolor{red}{,} -1)\).

\textcolor{red}{But}\textcolor{red}{,} the problem is asking for the maximum value of \(y\)\textcolor{red}{,} and I've found that to be \(49\).

Therefore\textcolor{red}{,} the final answer is \(49\).

\bigskip

\textbf{\textcolor{red}{Final} Answer:}

\[
\boxed{49\textcolor{red}{}}
\]
\end{tcolorbox}
\caption{\textcolor{red}{Most shifted tokens} after fine-tuning on long CoT data (2/2). All the most shifted tokens are labeled as red.}
    \label{fig: most_shifted_2}
\end{figure*}






\begin{figure*}[htbp]
    \centering
\begin{tcolorbox}[title=Example of \textcolor{red}{Most Shifted Tokens} after fine-tuning on large model CoT, promptstyle]
\lstset{
    basicstyle=\normalfont\sffamily\footnotesize,
    breaklines=true,
    frame=none,
    columns=fullflexible,
}


\textcolor{red}{To} \textcolor{red}{solve} \textcolor{red}{the} \textcolor{red}{problem}\textcolor{red}{,} \textcolor{red}{ we} \textcolor{red}{ start} \textcolor{red}{ with} \textcolor{red}{ the} \textcolor{red}{ given} equation\textcolor{red}{:}

\[
x^2 + y^2\textcolor{red}{ =} 14x + 48y
\]

\textcolor{red}{First}\textcolor{red}{,} we rearr\textcolor{red}{ange} the equation to bring all terms to one\textcolor{red}{ side}\textcolor{red}{:}

\[
x^2 - 14\textcolor{red}{x} + y^2 - 48\textcolor{red}{y} = 0
\]

\textcolor{red}{Next}\textcolor{red}{,} we complete the square for both \(x\) and \(\textcolor{red}{y}\).

\textcolor{red}{For} \(x\):
\[
x^2 - 14x
\]
\textcolor{red}{To} complete the square\textcolor{red}{,} we add and subtract \(\left(\frac{14}{2}\right)^2 = 49\):
\[
x^2 - 14x + 49 - 49
\]
\[
(\textcolor{red}{x} - 7)^2 - 49
\]

\textcolor{red}{For} \(y\):
\[
y^2 - 48y
\]
\textcolor{red}{To} complete the square, we add and subtract \(\left(\frac{48}{2}\right)^2 = 576\):
\[
y^2 - 48y + 576 - 576
\]
\[
(\textcolor{red}{y} - 24)^2 - 576
\]

\textcolor{red}{Now}\textcolor{red}{,} substitute these completed squares back into the equation:
\[
(\textcolor{red}{x} - 7)^2 - 49 + (y - 24)^2 - 576 = 0
\]
\[
(\textcolor{red}{x} - 7)^2 + (y - 24)^2 - 625 = 0
\]
\[
(\textcolor{red}{x} - 7)^2 + (y - 24)^2 = 625
\]

\textcolor{red}{This} is the equation of a circle with center \((7, 24)\) and radius \(25\).

\textcolor{red}{To} find the maximum value of \(y\), we need to find the highest point on the circle. \textcolor{red}{The} highest point on the circle is directly above the center, which is at \(y = 24 + 25 = 49\).

\textcolor{red}{Thus}\textcolor{red}{,} the maximum value of \(y\) is:
\[
\textcolor{red}{y} = 49
\]

\textcolor{red}{Final} \textcolor{red}{ Answer}\textcolor{red}{:} \(\boxed{49}\)

\end{tcolorbox}
\caption{Example of \textcolor{red}{most shifted tokens} after fine-tuning on large teacher CoT. All the most shifted tokens are labeled as red.}
    \label{fig: most_shifted_3}
\end{figure*}




\section{Examples of Various CoT Data}\label{app:example}
This section demonstrates examples of different CoT data including long CoT, short CoT, large teacher CoT and small teacher CoT. See the following examples.
% \subsection{Long CoT and Short CoT}


\begin{figure*}[htbp]
    \centering
\begin{tcolorbox}[title=Problem, promptstyle]
\lstset{
    basicstyle=\normalfont\sffamily\footnotesize,
    breaklines=true,
    frame=none,
    columns=fullflexible,
}
What is the sum of the positive odd divisors of $60$?
\end{tcolorbox}
    
\end{figure*}



\begin{figure*}[htbp]
    \centering
\begin{tcolorbox}[title=Long CoT Example, promptstyle]
\lstset{
    basicstyle=\normalfont\sffamily\footnotesize,
    breaklines=true,
    frame=none,
    columns=fullflexible,
}
To find the sum of the positive odd divisors of 60, I first need to understand what the divisors of 60 are and then identify which of those are odd.

First, let's find all the positive divisors of 60. To do this, I can start by factoring 60 into its prime factors.

\[
60 = 2^2 \times 3^1 \times 5^1
\]

Now, to find all the positive divisors, I can take all possible products of these prime factors raised to their possible exponents. For 2, the exponents can be \(0, 1,\) or \(2\); for 3, \(0\) or \(1\); and for 5, \(0\) or \(1\). Listing them out:

\[
\begin{aligned}
&1 = 2^0 \times 3^0 \times 5^0, \\
&2 = 2^1 \times 3^0 \times 5^0, \\
&4 = 2^2 \times 3^0 \times 5^0, \\
&3 = 2^0 \times 3^1 \times 5^0, \\
&6 = 2^1 \times 3^1 \times 5^0, \\
&12 = 2^2 \times 3^1 \times 5^0, \\
&5 = 2^0 \times 3^0 \times 5^1, \\
&10 = 2^1 \times 3^0 \times 5^1, \\
&20 = 2^2 \times 3^0 \times 5^1, \\
&15 = 2^0 \times 3^1 \times 5^1, \\
&30 = 2^1 \times 3^1 \times 5^1, \\
&60 = 2^2 \times 3^1 \times 5^1.
\end{aligned}
\]

So, the positive divisors of 60 are:
\[
1, 2, 3, 4, 5, 6, 10, 12, 15, 20, 30, 60.
\]

Now, I need to identify which of these are odd. Odd numbers are not divisible by 2, so I can eliminate any divisors that are even.

The odd divisors are:
\[
1, 3, 5, 15.
\]

Now, I need to find the sum of these odd divisors:
$1 + 3 + 5 + 15 = 24$.
Alternatively, using the prime factorization approach: Since the odd divisors are those that do not include the factor \(2\), I can ignore the 2's in the prime factorization. The remaining odd part is:

\[
3^1 \times 5^1.
\]

The sum of the divisors of this odd part is:

\[
(1 + 3)(1 + 5) = 4 \times 6 = 24.
\]

This matches the previous result.

Therefore, for \( 60 = 2^2 \times 3^1 \times 5^1 \), the sum of the odd divisors is:

\[
\boxed{24}.
\]
\end{tcolorbox}
\caption{Long CoT Example}
    \label{fig: long cot example}
\end{figure*}



\begin{figure*}[htbp]
    \centering
\begin{tcolorbox}[title=Short CoT Example, promptstyle]
\lstset{
    basicstyle=\normalfont\sffamily\footnotesize,
    breaklines=true,
    frame=none,
    columns=fullflexible,
}

To find the sum of the positive odd divisors of $60$, we first need to factorize $60$ into its prime factors. The prime factorization of $60$ is:

\[
60 = 2^2 \times 3 \times 5
\]

Since we are only interested in the odd divisors, we can ignore the factor of $2$. Therefore, we only consider the factors $3$ and $5$. The odd divisors of $60$ are the divisors of $3 \times 5 = 15$. The divisors of $15$ are \(1, 3, 5,\) and \(15\).

Next, we sum these divisors:

\[
1 + 3 + 5 + 15 = 24
\]

Therefore, the sum of the positive odd divisors of $60$ is:

\[
\boxed{24}
\]

\end{tcolorbox}
    \label{fig: short cot example}
    \caption{Short CoT example}
\end{figure*}




\begin{figure*}[htbp]
    \centering
\begin{tcolorbox}[title=Problem, promptstyle]
\lstset{
    basicstyle=\normalfont\sffamily\footnotesize,
    breaklines=true,
    frame=none,
    columns=fullflexible,
}
Find the greatest \( a \) such that
\[
\frac{7\sqrt{(2a)^2+(1)^2}-4a^2-1}{\sqrt{1+4a^2}+3}=2.
\]
\end{tcolorbox}
\end{figure*}



\begin{figure*}[htbp]
    \centering
\begin{tcolorbox}[title=Large Teacher CoT Example (1/2), promptstyle]
\lstset{
    basicstyle=\normalfont\sffamily\footnotesize,
    breaklines=true,
    frame=none,
    columns=fullflexible,
}

To solve the equation \(\frac{7\sqrt{(2a)^2 + 1^2} - 4a^2 - 1}{\sqrt{1 + 4a^2} + 3} = 2\), we start by simplifying the expression inside the square root in the numerator. The equation becomes:

\[
\frac{7\sqrt{4a^2 + 1} - 4a^2 - 1}{\sqrt{1 + 4a^2} + 3} = 2
\]

Let's denote \(x = \sqrt{1 + 4a^2}\). Then the equation transforms into:

\[
\frac{7x - 4a^2 - 1}{x + 3} = 2
\]

Next, we clear the fraction by multiplying both sides by \(x + 3\):

\[
7x - 4a^2 - 1 = 2(x + 3)
\]

Expanding and simplifying the right side, we get:

\[
7x - 4a^2 - 1 = 2x + 6
\]

Rearranging terms to isolate \(x\) and \(a^2\), we have:

\[
7x - 2x - 4a^2 - 1 - 6 = 0 \implies 5x - 4a^2 - 7 = 0 \implies 5x = 4a^2 + 7 \implies x = \frac{4a^2 + 7}{5}
\]

Recall that \(x = \sqrt{1 + 4a^2}\). Therefore, we have:

\[
\sqrt{1 + 4a^2} = \frac{4a^2 + 7}{5}
\]

To eliminate the square root, we square both sides:

\[
1 + 4a^2 = \left(\frac{4a^2 + 7}{5}\right)^2
\]

Expanding the right side, we get:

\[
1 + 4a^2 = \frac{(4a^2 + 7)^2}{25} = \frac{16a^4 + 56a^2 + 49}{25}
\]

Multiplying both sides by 25 to clear the denominator, we obtain:

\[
25 + 100a^2 = 16a^4 + 56a^2 + 49
\]

\end{tcolorbox}
\end{figure*}



\begin{figure*}[htbp]
    \centering
\begin{tcolorbox}[title=Large Teacher CoT Example (2/2), promptstyle]
\lstset{
    basicstyle=\normalfont\sffamily\footnotesize,
    breaklines=true,
    frame=none,
    columns=fullflexible,
}
Rearranging all terms to one side, we get a standard polynomial equation:

\[
16a^4 + 56a^2 + 49 - 100a^2 - 25 = 0 \implies 16a^4 - 44a^2 + 24 = 0
\]

Let \(y = a^2\). Then the equation becomes a quadratic in \(y\):

\[
16y^2 - 44y + 24 = 0
\]

We solve this quadratic equation using the quadratic formula \(y = \frac{-b \pm \sqrt{b^2 - 4ac}}{2a}\), where \(a = 16\), \(b = -44\), and \(c = 24\):

\[
y = \frac{44 \pm \sqrt{(-44)^2 - 4 \cdot 16 \cdot 24}}{2 \cdot 16} = \frac{44 \pm \sqrt{1936 - 1536}}{32} = \frac{44 \pm \sqrt{400}}{32} = \frac{44 \pm 20}{32}
\]

This gives us two solutions for \(y\):

\[
y = \frac{64}{32} = 2 \quad \text{and} \quad y = \frac{24}{32} = \frac{3}{4}
\]

Since \(y = a^2\), we have \(a^2 = 2\) or \(a^2 = \frac{3}{4}\). The corresponding values for \(a\) are \(a = \pm \sqrt{2}\) and \(a = \pm \frac{\sqrt{3}}{2}\). The greatest value of \(a\) is \(\sqrt{2}\).

Thus, the greatest \(a\) that satisfies the equation is:

\[
\boxed{\sqrt{2}}
\]
\end{tcolorbox}
\end{figure*}



\begin{figure*}[htbp]
    \centering
\begin{tcolorbox}[title=Small Teacher CoT Example (1/2), promptstyle]
\lstset{
    basicstyle=\normalfont\sffamily\footnotesize,
    breaklines=true,
    frame=none,
    columns=fullflexible,
}

To solve the equation \(\frac{7\sqrt{(2a)^2+(1)^2}-4a^2-1}{\sqrt{1+4a^2}+3}=2\), we start by simplifying the expression inside the square root and the equation itself.

First, simplify the expression under the square root in the numerator:
\[
(2a)^2 + 1^2 = 4a^2 + 1
\]
So the equation becomes:
\[
\frac{7\sqrt{4a^2 + 1} - 4a^2 - 1}{\sqrt{4a^2 + 1} + 3} = 2
\]

Next, let \( x = \sqrt{4a^2 + 1} \). Then the equation transforms into:
\[
\frac{7x - 4a^2 - 1}{x + 3} = 2
\]

Since \( x = \sqrt{4a^2 + 1} \), we can express \( 4a^2 \) in terms of \( x \):
\[
4a^2 = x^2 - 1
\]

Substitute \( 4a^2 = x^2 - 1 \) into the equation:
\[
\frac{7x - (x^2 - 1) - 1}{x + 3} = 2
\]
Simplify the numerator:
\[
7x - x^2 + 1 - 1 = 7x - x^2
\]
So the equation becomes:
\[
\frac{7x - x^2}{x + 3} = 2
\]

Multiply both sides by \( x + 3 \) to clear the denominator:
\[
7x - x^2 = 2(x + 3)
\]
Expand and simplify:
\[
7x - x^2 = 2x + 6
\]
Rearrange all terms to one side:
\[
-x^2 + 7x - 2x - 6 = 0
\]
\[
-x^2 + 5x - 6 = 0
\]
Multiply through by -1 to make the leading coefficient positive:
\[
x^2 - 5x + 6 = 0
\]

\end{tcolorbox}
    
\end{figure*}




\begin{figure*}[htbp]
    \centering
\begin{tcolorbox}[title=Small Teacher CoT Example (2/2), promptstyle]
\lstset{
    basicstyle=\normalfont\sffamily\footnotesize,
    breaklines=true,
    frame=none,
    columns=fullflexible,
}

Factor the quadratic equation:
\[
(x - 2)(x - 3) = 0
\]

So, the solutions are:
\[
x = 2 \quad \text{or} \quad x = 3
\]

Recall that \( x = \sqrt{4a^2 + 1} \). We need to check both solutions:

1. For \( x = 2 \):
\[
\sqrt{4a^2 + 1} = 2
\]
Square both sides:
\[
4a^2 + 1 = 4
\]
\[
4a^2 = 3
\]
\[
a^2 = \frac{3}{4}
\]
\[
a = \pm \frac{\sqrt{3}}{2}
\]

2. For \( x = 3 \):
\[
\sqrt{4a^2 + 1} = 3
\]
Square both sides:
\[
4a^2 + 1 = 9
\]
\[
4a^2 = 8
\]
\[
a^2 = 2
\]
\[
a = \pm \sqrt{2}
\]

The greatest value of \( a \) among the solutions is \( \sqrt{2} \).

Thus, the final answer is:
\[
\boxed{\sqrt{2}}
\]
\end{tcolorbox}
    
\end{figure*}




\begin{figure*}[htbp]
    \centering
\begin{tcolorbox}[title=Problem, promptstyle]
\lstset{
    basicstyle=\normalfont\sffamily\footnotesize,
    breaklines=true,
    frame=none,
    columns=fullflexible,
}
What is the greatest common divisor of 128, 144 and 480?
\end{tcolorbox}
    
\end{figure*}



\begin{figure*}[htbp]
    \centering
\begin{tcolorbox}[title=Large Teacher CoT Example, promptstyle]
\lstset{
    basicstyle=\normalfont\sffamily\footnotesize,
    breaklines=true,
    frame=none,
    columns=fullflexible,
}
To find the greatest common divisor (GCD) of 128, 144, and 480, we can use the prime factorization method.

\begin{enumerate}
    \item \textbf{Prime Factorization:}
    \begin{align*}
        128 &= 2^7, \\
        144 &= 2^4 \times 3^2, \\
        480 &= 2^5 \times 3 \times 5.
    \end{align*}

    \item \textbf{Identify the common prime factors:}\\
    The common prime factor among 128, 144, and 480 is \(2\).

    \item \textbf{Determine the lowest power of the common prime factor:}\\
    The lowest power of \(2\) in the factorizations is \(2^4\).

    \item \textbf{Calculate the GCD:}\\
    The GCD is:
    \[
    2^4 = 16.
    \]
\end{enumerate}

Therefore, the greatest common divisor of 128, 144, and 480 is:
\[
\boxed{16}
\]
\end{tcolorbox}
    
\end{figure*}


\begin{figure*}[htbp]
    \centering
\begin{tcolorbox}[title=Small Teacher CoT Example, promptstyle]
\lstset{
    basicstyle=\normalfont\sffamily\footnotesize,
    breaklines=true,
    frame=none,
    columns=fullflexible,
}


To find the greatest common divisor (GCD) of 128, 144, and 480, we start by finding the prime factorization of each number.

The prime factorization of 128 is:
\[
\begin{aligned}
128 \div 2 &= 64 \\
64 \div 2 &= 32 \\
32 \div 2 &= 16 \\
16 \div 2 &= 8 \\
8 \div 2 &= 4 \\
4 \div 2 &= 2 \\
2 \div 2 &= 1
\end{aligned}
\]
Thus, 
\[
128 = 2^7.
\]

The prime factorization of 144 is:
\[
\begin{aligned}
144 \div 2 &= 72 \\
72 \div 2 &= 36 \\
36 \div 2 &= 18 \\
18 \div 2 &= 9 \\
9 \div 3 &= 3 \\
3 \div 3 &= 1
\end{aligned}
\]
Thus,
\[
144 = 2^4 \times 3^2.
\]

The prime factorization of 480 is:
\[
\begin{aligned}
480 \div 2 &= 240 \\
240 \div 2 &= 120 \\
120 \div 2 &= 60 \\
60 \div 2 &= 30 \\
30 \div 2 &= 15 \\
15 \div 3 &= 5 \\
5 \div 5 &= 1
\end{aligned}
\]
Thus,
\[
480 = 2^5 \times 3 \times 5.
\]

Now, we identify the common prime factors. The factorizations are:
\[
128 = 2^7, \quad 144 = 2^4 \times 3^2, \quad 480 = 2^5 \times 3 \times 5.
\]
The only common prime factor among all three numbers is \(2\).

Next, we determine the smallest power of the common prime factor. The lowest power of \(2\) among these factorizations is \(2^4\).

Therefore, the GCD of 128, 144, and 480 is:
\[
\boxed{16}
\]

\end{tcolorbox}
    
\end{figure*}







\begin{figure*}[htbp]
    \centering
\begin{tcolorbox}[title=Problem, promptstyle]
\lstset{
    basicstyle=\normalfont\sffamily\footnotesize,
    breaklines=true,
    frame=none,
    columns=fullflexible,
}
Rounded to 2 decimal places, what is $\frac{7}{9}$?
\end{tcolorbox}
    
\end{figure*}



\begin{figure*}[htbp]
    \centering
\begin{tcolorbox}[title=Large Teacher CoT Example, promptstyle]
\lstset{
    basicstyle=\normalfont\sffamily\footnotesize,
    breaklines=true,
    frame=none,
    columns=fullflexible,
}
To solve the problem, we need to divide 7 by 9 and then round the result to two decimal places.

\[
\frac{7}{9} = 0.777\ldots
\]

Rounded to two decimal places, this becomes 0.78.

Final Answer: \(\boxed{0.78}\)
\end{tcolorbox}
    
\end{figure*}


\begin{figure*}[htbp]
    \centering
\begin{tcolorbox}[title=Small Teacher CoT Example, promptstyle]
\lstset{
    basicstyle=\normalfont\sffamily\footnotesize,
    breaklines=true,
    frame=none,
    columns=fullflexible,
}
To solve the problem of finding the value of \(\frac{7}{9}\) rounded to 2 decimal places, we can follow these steps:

\begin{enumerate}
    \item \textbf{Perform the division}: Calculate \(\frac{7}{9}\).
    \item \textbf{Convert to decimal}: Perform the division to get the decimal equivalent.
    \item \textbf{Round to 2 decimal places}: Round the decimal to 2 decimal places.
\end{enumerate}

Let's start with the division:

\[
\frac{7}{9} \approx 0.7777\ldots
\]

Next, we need to round this decimal to 2 decimal places. The digit in the third decimal place is 7, which is greater than or equal to 5. Therefore, we round the second decimal place up from 7 to 8.

So, \(0.7777\ldots\) rounded to 2 decimal places is \(0.78\).

The final answer is:

\[
\boxed{0.78}
\]
\end{tcolorbox}
    
\end{figure*}






\end{document}
