
% \pdfoutput=1

\documentclass[11pt]{article}


\usepackage[preprint]{acl}


\usepackage{times}
\usepackage{latexsym}
\usepackage[T1]{fontenc}
\usepackage[utf8]{inputenc}
\usepackage{microtype}
\usepackage{inconsolata}
\usepackage{graphicx}
\usepackage{microtype}
\usepackage{tabularx}
% \usepackage[table]{xcolor}
% \usepackage{xcolor}
\usepackage{amsmath}
\usepackage{amssymb}
\usepackage{mathtools}
\usepackage{amsthm}
\usepackage{multirow}
\usepackage{makecell}
\usepackage{booktabs}
\usepackage{array}
\usepackage{longtable}
\usepackage{subcaption}
\usepackage{fontawesome}
\usepackage{centernot}
\usepackage{pgf} 
\usepackage{tikz} 
\usepackage{stfloats}

\theoremstyle{plain}
\newtheorem{theorem}{Theorem}[section]
\newtheorem{proposition}[theorem]{Proposition}
\newtheorem{lemma}[theorem]{Lemma}
\newtheorem{corollary}[theorem]{Corollary}
\theoremstyle{definition}
\newtheorem{definition}[theorem]{Definition}
\newtheorem{assumption}[theorem]{Assumption}
\theoremstyle{remark}
\newtheorem{remark}[theorem]{Remark}


\usepackage[most,skins,theorems]{tcolorbox}
\tcbset{
  aibox/.style={
    width=\linewidth,
    top=8pt,
    bottom=4pt,
    colback=blue!6!white,
    colframe=black,
    colbacktitle=black,
    enhanced,
    center,
    attach boxed title to top left={yshift=-0.1in,xshift=0.15in},
    boxed title style={boxrule=0pt,colframe=white,},
  }
}
\newtcolorbox{AIbox}[2][]{aibox,title=#2,#1}







\newenvironment{findingBox}[2]{%
	\begin{tcolorbox}[
colframe=black!80,
colback=gray!10,
 boxrule=.5pt,
 left=1pt,
 right = 1pt,
 top=0pt,
 bottom=0pt,
 size=small,
 fonttitle=\bfseries,
coltitle=black,
boxrule=0.4mm,
arc=2mm
 ]{\textbf{Observation #1:} #2} 
}{%
	\end{tcolorbox}
}


\tcbset{
    promptstyle/.style={
        enhanced,
        colback=white,
        colframe=black,
        colbacktitle=gray!20,
        width=0.98\linewidth, 
        coltitle=black,
        rounded corners,
        sharp corners=north,
        boxrule=0.5pt,
        drop shadow=black!50!white,
        attach boxed title to top left={
            xshift=-2mm,
            yshift=-2mm
        },
        title code={%
            \begin{minipage}{0.8\linewidth}
                \centering\thetitle
            \end{minipage}
        },
        boxed title style={
            rounded corners,
            size=small,
            colback=gray!20,
        },
        fonttitle=\normalsize\bfseries,
        before upper={\parindent15pt}
    }
}


\usepackage{ifthen} 
\newcommand{\coloredAvg}[1]{%
  \pgfmathsetmacro{\val}{#1}%
  \pgfmathsetmacro{\absval}{abs(\val)}%
  \pgfmathsetmacro{\perc}{min(100,100*\absval/3)}%
  \ifthenelse{\lengthtest{\val pt > 0pt}}%
    {\cellcolor{red!\perc}#1}%
    {\ifthenelse{\lengthtest{\val pt < 0pt}}%
      {\cellcolor{green!\perc}#1}%
      {\cellcolor{white}#1}}%
}



\tcbset{
    userstyle/.style={
        enhanced,
        colback=white,
        colframe=black,
        colbacktitle=gray!20,
        coltitle=black,
        rounded corners,
        sharp corners=north,
        boxrule=0.5pt,
        drop shadow=black!50!white,
        attach boxed title to top left={
            xshift=-2mm,
            yshift=-2mm
        },
        boxed title style={
            rounded corners,
            size=small,
            colback=gray!20
        }
    },
    replystyleg/.style={
        enhanced,
        colback=green!15,
        colframe=black,
        colbacktitle=green!30,
        coltitle=black,
        boxrule=0.5pt,
        drop shadow=black!50!white,
        rounded corners,
        sharp corners=north,
        attach boxed title to top right={
            xshift=-2mm,
            yshift=-2mm
        },
        boxed title style={
            rounded corners,
            size=small,
            colback=green!40
        }
    },
    replystyler/.style={
        enhanced,
        colback=red!15,
        colframe=black,
        colbacktitle=red!40,
        coltitle=black,
        boxrule=0.5pt,
        drop shadow=black!50!white,
        rounded corners,
        sharp corners=north,
        attach boxed title to top right={
            xshift=-2mm,
            yshift=-2mm
        },
        boxed title style={
            rounded corners,
            size=small,
            colback=red!40
        }
    }
}



\newtcolorbox{prompt}[2][]{
    colback=white,
    colframe=gray!45,
    fonttitle=\bfseries,
    coltitle=black,
    sharp corners,
    title=#2,
    #1
}

\tcbset{
    promptstyle/.style={
        enhanced,
        colback=white,
        colframe=black,
        colbacktitle=gray!20,
        coltitle=black,
        rounded corners,
        sharp corners=north,
        boxrule=0.5pt,
        drop shadow=black!50!white,
        attach boxed title to top left={
            xshift=-2mm,
            yshift=-2mm
        },
        boxed title style={
            rounded corners,
            size=small,
            colback=gray!20
        }
    },
    replystyleg/.style={
        enhanced,
        colback=green!15,
        colframe=black,
        colbacktitle=green!30,
        coltitle=black,
        boxrule=0.5pt,
        drop shadow=black!50!white,
        rounded corners,
        sharp corners=north,
        attach boxed title to top right={
            xshift=-2mm,
            yshift=-2mm
        },
        boxed title style={
            rounded corners,
            size=small,
            colback=green!40
        }
    },
    replystyler/.style={
        enhanced,
        colback=red!15,
        colframe=black,
        colbacktitle=red!40,
        coltitle=black,
        boxrule=0.5pt,
        drop shadow=black!50!white,
        rounded corners,
        sharp corners=north,
        attach boxed title to top right={
            xshift=-2mm,
            yshift=-2mm
        },
        boxed title style={
            rounded corners,
            size=small,
            colback=red!40
        }
    }
}

\newtcolorbox{promptbox}[1][]{
    promptstyle,
    title=Prompt,
    #1
}






\title{Small Models Struggle to Learn from Strong Reasoners}

\author{
\textbf{Yuetai Li}\textsuperscript{$\clubsuit$} \;\;\;  
\textbf{Xiang Yue}\textsuperscript{$\diamondsuit$} \;\;\; 
\textbf{Zhangchen Xu}\textsuperscript{$\clubsuit$} \;\;\;  
\textbf{Fengqing Jiang}\textsuperscript{$\clubsuit$} \;\;\;
\textbf{Luyao Niu}\textsuperscript{$\clubsuit$} \;\;\; \\ 
\textbf{Bill Yuchen Lin}\textsuperscript{$\clubsuit$} \;\;\;
\textbf{Bhaskar Ramasubramanian}\textsuperscript{$\spadesuit$} \; \;\;
\textbf{Radha Poovendran}\textsuperscript{$\clubsuit$}\\
  \textsuperscript{$\clubsuit$}University of Washington \; 
  \textsuperscript{$\diamondsuit$}Carnegie Mellon University \;
  \textsuperscript{$\spadesuit$}Western Washington University \\
  \texttt{\{yuetaili,zxu9,fqjiang,luyaoniu,byuchenl,rp3\}@uw.edu},\\
  \texttt{xyue2@andrew.cmu.edu}, \texttt{ramasub@wwu.edu} \vspace{1em} \\
   \textbf{Huggingface}: \url{https://huggingface.co/UWNSL} \\
   \textbf{Project Page}: \url{https://small-model-gap.github.io/}   
}


\begin{document}
\maketitle
\begin{abstract}

Large language models (LLMs) excel in complex reasoning tasks, and distilling their reasoning capabilities into smaller models has shown promise. However, we uncover an interesting phenomenon, which we term the \textit{Small Model Learnability Gap}: small models ($\leq$3B parameters) do not consistently benefit from long chain-of-thought (CoT) reasoning or distillation from larger models. Instead, they perform better when fine-tuned on shorter, simpler reasoning chains that better align with their intrinsic learning capacity. To address this, we propose Mix Distillation, a simple yet effective strategy that balances reasoning complexity by combining long and short CoT examples or reasoning from both larger and smaller models. Our experiments demonstrate that Mix Distillation significantly improves small model reasoning performance compared to training on either data alone. These findings highlight the limitations of direct strong model distillation and underscore the importance of adapting reasoning complexity for effective reasoning capability transfer.

 
\end{abstract}


\section{Introduction}

The convergence of vision and language in artificial intelligence has led to the development of Vision-Language Models (VLMs) that can interpret and generate multimodal content. Among these, OpenAI's Contrastive Language-Image Pre-training (CLIP) model~\cite{radford2021learningtransferablevisualmodels} has been particularly influential, demonstrating remarkable capabilities in zero-shot image classification and setting new standards for multimodal understanding~\cite{Cherti_2023, gadre2023datacompsearchgenerationmultimodal, schuhmann2021laion400mopendatasetclipfiltered, thrush2022winoground}. The success of CLIP has catalyzed a wide array of applications---from image retrieval and visual question answering to text-to-image generation---signifying a paradigm shift in how models perceive and relate visual and linguistic information.




Visual Language Models like CLIP face significant challenges in understanding and reasoning about complex scenes with multiple objects and intricate relationships. CLIP struggles to identify distinct objects and model their relationships accurately, especially when captions contain the same objects but differ in their relationships. This results in difficulty distinguishing between similar captions with different object relationships. Several benchmark datasets have been introduced to elucidate the limitations of existing models in capturing subtle relational nuances. Notably, Winoground \cite{thrush2022winoground}, VL-CheckList \cite{zhao2022vl}, ARO \cite{yuksekgonul2023and}, and CREPE \cite{ma2023crepe} have been instrumental in evaluating models' capacities to accurately match images with semantically appropriate captions. 



\begin{figure*}[t]
    \centering
    \includegraphics[width=\textwidth]{figs/main_fig_v5.pdf}
    \caption{Overview of our key contributions. Step 1: We create ComCO dataset for controlled multi-object experiments. Step 2: We identify biases in CLIP's image encoder (favoring larger objects) and text encoder (prioritizing first-mentioned objects). Step 3: We investigate the origin of these biases, finding a connection to training data characteristics. Step 4: We demonstrate the practical impacts of these biases on image-text matching task, showing how they affect model performance in multi-object scenarios.}
    \label{fig:mainfig}
    \vspace{-0.5cm} 
\end{figure*}

Numerous studies have addressed compositionality challenges in multi-object scenarios, often through end-to-end methods like fine-tuning with hard-negative samples \cite{yuksekgonul2023and} to improve model performance. However, these approaches have faced criticism and subsequent refinement, as seen in methods like SUGARCREPE \cite{hsieh2024sugarcrepe} and \cite{sahin2024enhancing}, which generate negative captions with minor structural changes or LLMs to highlight semantic distinctions. While most focus on CLIP’s ability to distinguish structurally similar yet conceptually different captions, few studies, such as Dumpala et al. \cite{dumpala2024sugarcrepe++}, explore CLIP’s performance on semantically equivalent but structurally distinct captions, revealing a gap in understanding CLIP's inconsistency with such prompts.




% While previous studies have made significant strides in understanding CLIP's limitations, our work distinguishes itself in several key aspects. Firstly, we shift the focus from evaluating CLIP's ability to differentiate between conceptually distinct captions to examining its performance with semantically equivalent but structurally varied captions. This approach allows us to probe deeper into the model's understanding of language and visual content beyond surface-level differences. Here, model systematic mistakes give an indication the potential baises. Secondly, unlike many previous works that primarily introduced benchmarks or proposed end-to-end solutions, we conduct a thorough investigation into the underlying causes of CLIP's behavior. Our study delves into the internal mechanisms of both the image and text encoders, providing insights into why the model is biased and lacks invariance to certain types of linguistic and visual variations. 

% To facilitate this in-depth analysis, we introduce the \textbf{ComCO} dataset, specifically designed to isolate and examine different aspects of CLIP's performance in {\it controlled} multi-object scenarios. Furthermore, our research spans multiple versions of CLIP trained on various datasets and architectures, ensuring the broad applicability and generalizability of our findings. By focusing on these underexplored areas and employing a more comprehensive analytical approach, our work aims to provide a deeper understanding of CLIP's limitations and pave the way for more robust and versatile vision-language models. It is important to note that such an analysis not only benefits the improvement of CLIP but also has significant implications for related models, such as text-to-image (T2I) generative models and multimodal large language models (MLLMs). Understanding the intricacies of CLIP's encoding process can inform and enhance the development of these technologies, potentially leading to advancements across various domains of artificial intelligence. As shown in Figure \ref{fig:mainfig}, our key contributions are as follows:


While previous studies have advanced our understanding of CLIP's limitations, our work uniquely focuses on CLIP's performance with semantically equivalent but structurally varied captions rather than simply distinguishing conceptually different captions. This shift enables a deeper examination of the model’s grasp of language and visual content, where systematic errors reveal potential biases. Unlike prior works that primarily propose benchmarks or end-to-end solutions, we investigate the root causes of CLIP's behavior, delving into the mechanisms of both image and text encoders to uncover why the model displays biases and lacks robustness to certain linguistic and visual variations. To support this analysis, we introduce the \textbf{ComCO} dataset, purpose-built for examining CLIP's performance under {\it controlled} multi-object scenarios. Our study spans multiple versions of CLIP trained on diverse datasets and architectures, ensuring the broad applicability of our findings. This comprehensive approach aims to deepen our understanding of CLIP’s limitations and pave the way for more adaptable vision-language models. Beyond CLIP, our insights have significant implications for text-to-image (T2I) generative models and multimodal large language models (MLLMs), where decoding CLIP’s encoding intricacies can inform advancements in artificial intelligence across domains. As shown in Figure \ref{fig:mainfig}, our key contributions are as follows:



\begin{itemize} \item \textbf{Development of Novel Dataset}: We introduce \textit{ComCO}, a specialized dataset for creating {\it controlled} multi-object scenarios. Unlike previous benchmarks, ComCO allows control over object size and caption order, enabling precise analysis of model performance across compositional challenges and enhancing understanding of VLMs' strengths and weaknesses.

\item \textbf{Encoder Analysis}: We conduct an in-depth examination of CLIP’s image and text encoders in multi-object scenes, revealing weaknesses in preserving information for object distinction and identifying where compositional information is lost.

\item \textbf{Bias Identification}: Our study reveals that CLIP’s image encoder prefers larger objects, while the text encoder favors first-mentioned and visually larger objects, highlighting biases in CLIP's handling of visual and linguistic information.

\item \textbf{Investigation of Bias Origins}: We explore the origins of these biases, showing that larger objects are often mentioned earlier in CLIP’s training captions, and are favored in embeddings due to the abundance of their visual tokens. We substantiate this with analyses of the LAION dataset and CLIP’s training progression.

\item \textbf{Practical Impact}: We show how these biases affect performance in multi-object tasks, with significant drops in image-text matching accuracy in ComCO and COCO ~\cite{lin2015microsoftcococommonobjects}. These biases also extend to text-to-image models, influencing object prominence based on prompt order.

\end{itemize}



% \begin{itemize} \item \textbf{Development of Novel Dataset}: We introduce \textit{ComCO}, a specialized dataset specifically designed to create {\it controlled} multi-object scenarios. Here, unlike previous benchmarks, we can control the object size in the image, and their ordering in the caption. Hence, this dataset enables precise, fine-grained analysis of model performance across a spectrum of compositional challenges, facilitating a deeper understanding of VLMs' strengths and weaknesses.

%     \item \textbf{Comprehensive Encoder Analysis}: We perform an in-depth examination of both the image and text encoders in CLIP when processing multi-object scenes and descriptions. This includes text-based, and object-based image retrievals, that reveal each text and image encoder weaknesses in preserving the information necessary to discern various objects. By analyzing the embedding space, we identify the stages at which compositional information is lost or distorted, providing insights into the internal mechanisms of the model.
    
%     \item \textbf{Identification of Specific Biases}: Our research uncovers significant biases in CLIP models. The image encoder prefers larger objects in multi-object images, while the text encoder favors first-mentioned objects and also objects that are usually visually larger in real-world. These biases reveal the complex interplay between visual and linguistic information processing in CLIP, influencing its interpretation of multi-object scenarios.
    
%     \item \textbf{Investigation of the Bias Origin }: We explore the origins of observed biases in CLIP's performance, particularly in various multi-object scenarios. Our investigation delves into both the image and text encoders. We hypothesize that the visually larger objects are mostly mentioned earlier in the caption in CLIP training datasets. But it is evident that the image encoding naturally favors such objects in the embedding due to the abundance of their visual tokens. Therefore, the text encoder may get biased towards such objects, and consequently earlier mentioned text tokens. We provide evidence for these biases through analyses of the LAION dataset and CLIP's training progression, revealing a consistent trend where larger objects tend to be mentioned earlier in image captions.
    
%     \item \textbf{Practical impacts of encoder biases}: We demonstrate how the identified biases in CLIP's image and text encoders significantly impact performance in multi-object analysis/synthesis scenarios. Using our ComCO dataset, we show substantial drops in image-text matching accuracy when manipulating object sizes and caption order. We further reveal how these biases propagate to text-to-image generation models like Stable Diffusion, influencing the prominence and likelihood of object appearance in generated images based on prompt order.
    
%     \end{itemize}

These findings reveal how biases in CLIP’s text and image encoders significantly reduce its performance in multi-object scenarios, emphasizing the need to address these biases to enhance vision-language models' robustness. Our work offers key insights into CLIP's behavior and lays groundwork for improving model performance in real-world applications.






% !TEX root =  ../main.tex
\section{Background on causality and abstraction}\label{sec:preliminaries}

This section provides the notation and key concepts related to causal modeling and abstraction theory.

\spara{Notation.} The set of integers from $1$ to $n$ is $[n]$.
The vectors of zeros and ones of size $n$ are $\zeros_n$ and $\ones_n$.
The identity matrix of size $n \times n$ is $\identity_n$. The Frobenius norm is $\frob{\mathbf{A}}$.
The set of positive definite matrices over $\reall^{n\times n}$ is $\pd^n$. The Hadamard product is $\odot$.
Function composition is $\circ$.
The domain of a function is $\dom{\cdot}$ and its kernel $\ker$.
Let $\mathcal{M}(\mathcal{X}^n)$ be the set of Borel measures over $\mathcal{X}^n \subseteq \reall^n$. Given a measure $\mu^n \in \mathcal{M}(\mathcal{X}^n)$ and a measurable map $\varphi^{\V}$, $\mathcal{X}^n \ni \mathbf{x} \overset{\varphi^{\V}}{\longmapsto} \V^\top \mathbf{x} \in \mathcal{X}^m$, we denote by $\varphi^{\V}_{\#}(\mu^n) \coloneqq \mu^n(\varphi^{\V^{-1}}(\mathbf{x}))$ the pushforward measure $\mu^m \in \mathcal{M}(\mathcal{X}^m)$. 


We now present the standard definition of SCM.

\begin{definition}[SCM, \citealp{pearl2009causality}]\label{def:SCM}
A (Markovian) structural causal model (SCM) $\scm^n$ is a tuple $\langle \myendogenous, \myexogenous, \myfunctional, \zeta^\myexogenous \rangle$, where \emph{(i)} $\myendogenous = \{X_1, \ldots, X_n\}$ is a set of $n$ endogenous random variables; \emph{(ii)} $\myexogenous =\{Z_1,\ldots,Z_n\}$ is a set of $n$ exogenous variables; \emph{(iii)} $\myfunctional$ is a set of $n$ functional assignments such that $X_i=f_i(\parents_i, Z_i)$, $\forall \; i \in [n]$, with $ \parents_i \subseteq \myendogenous \setminus \{ X_i\}$; \emph{(iv)} $\zeta^\myexogenous$ is a product probability measure over independent exogenous variables $\zeta^\myexogenous=\prod_{i \in [n]} \zeta^i$, where $\zeta^i=P(Z_i)$. 
\end{definition}
A Markovian SCM induces a directed acyclic graph (DAG) $\mathcal{G}_{\scm^n}$ where the nodes represent the variables $\myendogenous$ and the edges are determined by the structural functions $\myfunctional$; $ \parents_i$ constitutes then the parent set for $X_i$. Furthermore, we can recursively rewrite the set of structural function $\myfunctional$ as a set of mixing functions $\mymixing$ dependent only on the exogenous variables (cf. \cref{app:CA}). A key feature for studying causality is the possibility of defining interventions on the model:
\begin{definition}[Hard intervention, \citealp{pearl2009causality}]\label{def:intervention}
Given SCM $\scm^n = \langle \myendogenous, \myexogenous, \myfunctional, \zeta^\myexogenous \rangle$, a (hard) intervention $\iota = \operatorname{do}(\myendogenous^{\iota} = \mathbf{x}^{\iota})$, $\myendogenous^{\iota}\subseteq \myendogenous$,
is an operator that generates a new post-intervention SCM $\scm^n_\iota = \langle \myendogenous, \myexogenous, \myfunctional_\iota, \zeta^\myexogenous \rangle$ by replacing each function $f_i$ for $X_i\in\myendogenous^{\iota}$ with the constant $x_i^\iota\in \mathbf{x}^\iota$. 
Graphically, an intervention mutilates $\mathcal{G}_{\mathsf{M}^n}$ by removing all the incoming edges of the variables in $\myendogenous^{\iota}$.
\end{definition}

Given multiple SCMs describing the same system at different levels of granularity, CA provides the definition of an $\alpha$-abstraction map to relate these SCMs:
\begin{definition}[$\abst$-abstraction, \citealp{rischel2020category}]\label{def:abstraction}
Given low-level $\mathsf{M}^\ell$ and high-level $\mathsf{M}^h$ SCMs, an $\abst$-abstraction is a triple $\abst = \langle \Rset, \amap, \alphamap{} \rangle$, where \emph{(i)} $\Rset \subseteq \datalow$ is a subset of relevant variables in $\mathsf{M}^\ell$; \emph{(ii)} $\amap: \Rset \rightarrow \datahigh$ is a surjective function between the relevant variables of $\mathsf{M}^\ell$ and the endogenous variables of $\mathsf{M}^h$; \emph{(iii)} $\alphamap{}: \dom{\Rset} \rightarrow \dom{\datahigh}$ is a modular function $\alphamap{} = \bigotimes_{i\in[n]} \alphamap{X^h_i}$ made up by surjective functions $\alphamap{X^h_i}: \dom{\amap^{-1}(X^h_i)} \rightarrow \dom{X^h_i}$ from the outcome of low-level variables $\amap^{-1}(X^h_i) \in \datalow$ onto outcomes of the high-level variables $X^h_i \in \datahigh$.
\end{definition}
Notice that an $\abst$-abstraction simultaneously maps variables via the function $\amap$ and values through the function $\alphamap{}$. The definition itself does not place any constraint on these functions, although a common requirement in the literature is for the abstraction to satisfy \emph{interventional consistency} \cite{rubenstein2017causal,rischel2020category,beckers2019abstracting}. An important class of such well-behaved abstractions is \emph{constructive linear abstraction}, for which the following properties hold. By constructivity, \emph{(i)} $\abst$ is interventionally consistent; \emph{(ii)} all low-level variables are relevant $\Rset=\datalow$; \emph{(iii)} in addition to the map $\alphamap{}$ between endogenous variables, there exists a map ${\alphamap{}}_U$ between exogenous variables satisfying interventional consistency \cite{beckers2019abstracting,schooltink2024aligning}. By linearity, $\alphamap{} = \V^\top \in \reall^{h \times \ell}$ \cite{massidda2024learningcausalabstractionslinear}. \cref{app:CA} provides formal definitions for interventional consistency, linear and constructive abstraction.
\section{Small Model Learnability Gap}
\label{sec: empirical}
In this section, we fine-tune student models using different CoT data. 
We then reveal the small model learnability gap given the performance of fine-tuned models.

\subsection{Experiment Setup}

\paragraph{Datasets.}
We use the 7,500 prompt set of MATH \citep{hendrycks2021measuringmathematicalproblemsolving}. This dataset encompasses seven math topics such as advanced calculus, geometry, and linear algebra. 





\paragraph{Student models.}
Our study considers ten student models from the Qwen \citep{qwen2.5} and Llama \citep{llama32,llama31} model families of varying sizes. These models include the Instruct version of \texttt{Qwen2.5-0.5B}, \texttt{Qwen2.5-1.5B}, \texttt{Qwen2.5-3B}, \texttt{Qwen2.5-7B}, \texttt{Qwen2.5-14B}, and \texttt{Qwen2.5-32B}, and the Instruct version of \texttt{Llama3.2-1B}, \texttt{Llama3.2-3B}, \texttt{Llama3.1-8B}, and \texttt{Llama3.3-70B}. A comprehensive overview of the student models is presented in Table \ref{tab:models_overview} of Appendix \ref{appendix:More on Experimental Setups}. 


\paragraph{Teacher models.} 


To compare long CoT with short CoT, we use \texttt{QwQ-32B-Preview} \citep{QwenTeam2024b} to generate long CoT sequences and \texttt{Qwen2.5-32B-Instruct} as the response generator for short CoT. 
Within each model family, we designate the larger scale model as the large teacher and the smaller scale model as the small teacher. 
This includes \texttt{Qwen2.5-72B-Instruct} vs \texttt{Qwen2.5-\allowdisplaybreaks3B-Instruct},~\texttt{Llama3.1-70B-Instruct} vs~\texttt{Llama3.1-8B-Instruct},~and \texttt{Gemma2-27B-it} vs \texttt{Gemma2-9B-it}.



\begin{figure*}[!t]
    \centering
    \includegraphics[width=\textwidth]{figs/long_cot_combined_model_performance.pdf}
    \caption{Long CoT Gap ($\Delta_{Long}=P_{Long} - P_{Short}$) of student models with different models sizes for (a) Qwen family (b) Llama family. For teacher models, \texttt{QwQ-32B-Preview} is chosen to generate long CoT responses, while \texttt{Qwen2.5-32B-Instruct} is chosen to generate short CoT responses. Negative (Positive) $\Delta_{Long}$ indicates that long CoT is worse (better) than short CoT. Our results demonstrate that short CoT is better for smaller student models (indicated by $\Delta_{Long}$ < 0), while long CoT is better for larger student models (indicated by $\Delta_{Long}$ > 0).}
    % \vspace{-1em}
    \label{fig:combined_model_performance}
\end{figure*}
\begin{figure*}[!t]
    \centering
    \includegraphics[width=\textwidth]{figs/lmp_combined_model_performance.pdf}
    \caption{Large model CoT Gap ($\Delta_{Large}=P_{Large} - P_{Small}$) of student models with different models sizes for (a) Qwen family (b) Llama family. For teacher models, \texttt{Qwen2.5-72B-Instruct} is chosen as the large teacher to generate responses, while \texttt{Qwen2.5-3B-Instruct} is chosen as the small teacher to generate responses. Negative (positive) $\Delta_{Large}$ indicates that large teacher CoT is worse (better) than small teacher CoT. Our results demonstrate that small teacher CoT is better for smaller student models (indicated by $\Delta_{Large}$ < 0), while large model CoT is better for larger student models (indicated by $\Delta_{Large}$ > 0).}
    \label{fig:lmp_combined_model_performance}
\end{figure*}

\paragraph{Evaluation Benchmarks.} 


We evaluate the reasoning capability of fine-tuned student models on a set of commonly used benchmarks, including MATH \citep{hendrycks2021measuringmathematicalproblemsolving}, GSM8K \citep{cobbe2021trainingverifierssolvemath}, AMC 2023, AIME 2024, and the English math subset of OlympiadBench \citep{he2024olympiadbenchchallengingbenchmarkpromoting}. These benchmarks span a wide range of challenge levels, from elementary mathematics to advanced competition problems. 
We define the student model performance as the average score on five benchmarks. Unless otherwise specified, all fine-tuned models are evaluated in a zero-shot setting using greedy decoding. We set the maximum generation tokens as 16k. Please see Appendix \ref{appendix:More on Experimental Setups} for detailed experimental setup.

We define the following performance scores:
\begin{itemize}
    \item \(P_{Long}\): Performance score of a student model fine-tuned on long CoT data.
    \item \(P_{Short}\): Performance score of a student model fine-tuned on short CoT data.
    \item \(P_{Large}\): Performance score of a student model fine-tuned on CoT from a larger teacher.
    \item \(P_{Small}\): Performance score of a student model fine-tuned on CoT from a smaller teacher.
\end{itemize}




\paragraph{Training Setup.} 
Teacher models generate responses by rejection sampling 
\citep{dong2023raftrewardrankedfinetuning, gulcehre2023reinforcedselftrainingrestlanguage, tong2024dartmathdifficultyawarerejectiontuning, yuan2023scalingrelationshiplearningmathematical, yue2023mammothbuildingmathgeneralist, zelikman2022starbootstrappingreasoningreasoning}
By default, teacher models employ greedy decoding.  
By combining the math problem instructions with corresponding solutions generated by teacher models, we construct problem-solution pairs to fine-tune student models. 
We train the models using the LLaMA-Factory framework \citep{zheng2024llamafactory}. 
For student models of scale less than 14B, we use full-parameter SFT and implement a cosine learning rate schedule with a maximum learning rate of $10^{-5}$ to fine-tune student models for two epochs \citep{touvron2023llama}. 
For student models larger than 14B, we adopt LoRA fine-tuning with a learning rate of $10^{-4}$ for two epochs. Detailed hyperparameters and information about the experimental platform are provided in Appendix \ref{appendix:More on Experimental Setups}.





\subsection{Long CoT Gap}
This section evaluates the reasoning capabilities of student models fine-tuned over long CoT data and short CoT data. We quantify the performance difference between long and short CoT data using \emph{long CoT gap} \(\Delta_{Long}\), defined as:
\begin{equation*}
\Delta_{Long} = P_{Long} - P_{Short}.
\end{equation*}

Figure \ref{fig:combined_model_performance} provides a comprehensive overview of the long CoT gap $\Delta_{Long}$ across different student models. 
The detailed benchmark scores on MATH, GSM8K, AIME, AMC, and OlympiadBench are deferred to Table \ref{tab:full_performance_lg} in Appendix \ref{appendix: More Experiments}.
We report the following key takeaways. 

\begin{AIbox}{Takeaway 1: Long CoT Gap}
Small student models tend to benefit more from short CoT, while large student models gain greater advantages from long CoT.

\end{AIbox}

We observe that long CoT is more effective for larger models, consistently leading to improved performance across most math benchmarks. 
For example, the student model \texttt{Qwen2.5-32B-Instruct} improves about 15 points across all math metrics on average. 

However, long CoT data is not effective for smaller models, yielding significantly less improvement compared to short CoT. On the MATH and AMC benchmarks, student model \texttt{Qwen2.5-1.5B-Instruct}  performs over 10 points lower when fine-tuned with long CoT data. This shows that smaller models may not be able to effectively learn and utilize the long CoT paradigm. 
Please see more attribution analysis in Section \ref{More Analysis Results}.





\begin{table}[ht]
\centering
\small
\label{tab:comparison_table}
\resizebox{\columnwidth}{!}{
\begin{tabular}{lcccc}
\toprule
\textbf{Student Model} & \textbf{$P_{Long}$} & \textbf{$P_{Short}$} & \textbf{$\Delta_{Long}$} & \textbf{\makecell{ Better?}} \\
\midrule
Qwen2.5-0.5B   & 14.8 & 19.5 & \cellcolor{red!60}{-4.7}  & Short \\
Qwen2.5-1.5B   & 27.0 & 34.2 & \cellcolor{red!80}{-7.1}  & Short \\
Qwen2.5-3B     & 40.3 & 43.4 & \cellcolor{red!30}{-3.1}  & Short \\
Qwen2.5-7B     & 48.9 & 47.2 & \cellcolor{green!10}{1.7}     & Long   \\
Qwen2.5-14B    & 59.2 & 54.3 & \cellcolor{green!30}{4.9}     & Long   \\
Qwen2.5-32B    & 73.0 & 59.3 & \cellcolor{green!80}{13.7}    & Long   \\
\midrule
Llama-3.2-1B   & 15.8 & 19.5 & \cellcolor{red!40}{-3.7}  & Short \\
Llama-3.2-3B   & 32.5 & 33.1 & \cellcolor{red!10}{-0.6}  & Short \\
Llama-3.1-8B   & 35.2 & 31.5 & \cellcolor{green!25}{3.7}     & Long   \\
Llama-3.3-70B  & 58.2 & 54.3 & \cellcolor{green!25}{3.8}     & Long   \\
\bottomrule
\end{tabular}
}
\caption{Comparison of the average performance between fine-tuning with long CoT ($P_{Long}$) and short CoT ($P_{Short}$). We find that small student models may struggle to learn from long CoT data. }
\end{table}

\begin{table}[ht]
\centering
\small
\label{tab:strong_vs_weak}
\resizebox{\columnwidth}{!}{
\begin{tabular}{lcccc}
\toprule
\textbf{Student Model} & \textbf{$P_{Large}$} & \textbf{$P_{Small}$} & \textbf{$\Delta_{Large}$} & \textbf{\makecell{ Better?}} \\
\midrule
Qwen2.5-0.5B   & 16.9 & 20.4 & \cellcolor{red!50}{-3.5} & Weak   \\
Qwen2.5-1.5B   & 32.2 & 33.0 & \cellcolor{red!15}{-0.8} & Weak   \\
Qwen2.5-3B     & 39.7 & 39.4 & \cellcolor{green!10}{0.3}    & Strong \\
Qwen2.5-7B     & 48.9 & 42.3 & \cellcolor{green!70}{6.6}    & Strong \\
Qwen2.5-14B    & 52.9 & 49.9 & \cellcolor{green!30}{3.0}    & Strong \\
Qwen2.5-32B    & 59.5 & 53.0 & \cellcolor{green!70}{6.5}    & Strong \\
\midrule
Llama-3.2-1B   & 16.5 & 18.5 & \cellcolor{red!30}{-1.9} & Weak   \\
Llama-3.2-3B   & 32.8 & 31.2 & \cellcolor{green!20}{1.6}    & Strong \\
Llama-3.1-8B   & 25.6 & 25.1 & \cellcolor{green!10}{0.5}    & Strong \\
Llama-3.3-70B  & 57.6 & 53.3 & \cellcolor{green!50}{4.3}    & Strong \\
\bottomrule
\end{tabular}
}
\caption{Comparison of average performance between fine-tuning with large teacher CoT ($P_{Long}$) and small teacher CoT ($P_{Small}$). We find that small student models may struggle to learn from large teacher CoT data.}
\end{table}





\subsection{Large Teacher CoT Gap}
We investigate how effective small models may learn from large teacher and small teachers.
We define a \emph{large teacher CoT gap} as:
\[
\Delta_{Large} = P_{Large} - P_{Small}.
\]

Figure \ref{fig:lmp_combined_model_performance} provides a comprehensive comparison of the $\Delta_{Large}$ incurred by all student models. 
The detailed benchmark scores of MATH, GSM8K, AIME, AMC and OlympiadBench are deferred to Table \ref{tab:lmp-full_comparison} in Appendix \ref{appendix: More Experiments}. More experimental results of different teacher models, including \texttt{Llama3.1-70B} vs \texttt{Llama3.1-8B} and \texttt{Gemma2-27B} vs \texttt{Gemma2-9B} are in Table \ref{tab:lmp_comparison2} of Appendix \ref{appendix: More Experiments}.

We observe that larger student models learn effectively from large teacher CoT. 
For example, \texttt{Qwen2.5-7B-Instruct} and \texttt{Qwen2.5-32B-Instruct} student models improve over 5 points on average, with \texttt{Qwen2.5-32B-Instruct} achieving more than a 15 point increase on the AIMC benchmark. 
However, smaller models do not learn effectively from large teacher models such as \texttt{Qwen2.5-72B-Instruct}. 
Instead, small teacher models such as \texttt{Qwen2.5-3B-Instruct} may serve as better teacher models for small student models.
For instance, the performance of \texttt{Qwen2.5-0.5B-Instruct} degrades by more than 10 points on the AMC benchmark.


Note that prior studies \citep{kim2024evaluatinglanguagemodelssynthetic} also demonstrated that stronger models are not necessarily stronger teachers, 
emphasizing response generator and teacher-side factors. Our work differs in that we attribute this phenomenon primarily to the size of the student model.


\begin{AIbox}{\makecell{Takeaway 2: Large Teacher CoT Gap}}
Small student models tend to learn better from small teachers, while large student models benefit more from large teachers.
\end{AIbox}





\begin{table*}[ht]
    \small
    \centering
    \label{tab:combined-performance}
    % \resizebox{0.9\textwidth}{!}{%
    \begin{tabular}{l l c c c c c c}
        \toprule
        Student Model & Distillation Method & MATH & AMC & GSM8k & \makecell{Olympiad \\ Bench} & AIME & Average \\
        \midrule
        \multirow{7}{*}{Qwen2.5-3B} 
            & Long CoT                  & 56.2  & 37.5  & 80.0  & 24.4  & \underline{3.3}  & 40.3  \\
            & Short CoT             & 61.0  & 37.5  & \textbf{82.0}  & 26.4  & \textbf{10.0}       & 43.4  \\
            & Large Model CoT  & 57.5  & 35.0  & 80.0  & 25.9  & 0.0              & 39.7  \\
            & Small Model CoT      & 60.3  & 27.5  & 79.5  & 26.4  & \underline{3.3}  & 39.4  \\
            & Deepseek-R1-32B (Long CoT)           & 50.7  & 20.0  & 81.2  & 15.7  & 0.0              & 33.5  \\
        \cmidrule(lr){2-8}
            & \multicolumn{1}{l}{\textit{Ours}} & & & & & & \\
            & \textbf{Mix-Long}                           & \underline{64.7}  & \textbf{45.0}  & \underline{81.4}  & \underline{28.6}  & \textbf{10.0}  & \textbf{45.9}  \\
            & \textbf{Mix-Large}                         & \textbf{65.8}  & \underline{42.5}  & 81.7  & \textbf{29.0}  & \textbf{10.0}  & \underline{45.8}  \\
        \midrule
        \multirow{7}{*}{Llama3.2-3B} 
            & Long CoT                  & 48.7  & 17.5  & 75.1  & \underline{17.6}  & \underline{3.3}  & 32.5  \\
            & Short CoT             & 50.9  & 15.0  & 77.5  & \textbf{18.7}      & \underline{3.3}  & 33.1  \\
            & Large Model CoT  & 47.4  & \textbf{25.0}  & 71.2  & 16.9  & \underline{3.3}  & 32.8  \\
            & Small Model CoT      & 47.9  & 17.5  & 74.1  & 16.4  & \underline{3.3}  & 31.2  \\
            & Deepseek-R1-32B (Long CoT)           & 48.5  & 17.5  & \underline{77.7}  & 16.1  & \textbf{6.7}   & 33.3  \\
        \cmidrule(lr){2-8}
            & \multicolumn{1}{l}{\textit{Ours}} & & & & & & \\
            & \textbf{Mix-Long}                           & \textbf{53.0}  & \underline{22.5}  & \textbf{79.4}  & 17.2  & \underline{3.3}  & \textbf{35.1}  \\
            & \textbf{Mix-Large}                         & \underline{51.8}  & \textbf{25.0}  & 76.3  & 17.2  & \underline{3.3}  & \underline{34.7}  \\
        \bottomrule
    \end{tabular}%
    % }
    \caption{\textbf{Mix Distillation} outperforms the baseline models across most metrics. We use \texttt{Llama3.2-3B-Instruct} and \texttt{Qwen2.5-3B-Instruct} as the student model and 7.5k samples in MATH dataset as the training set. We distill different teacher models to generate responses as the baseline. Our proposed Mix-Long combines long CoT data and normal CoT data in a 1:4 ratio, while Mix-Large combines strong model response and weak model response with the same proportion. Experimental results demonstrate that both Mix-Long and Mix-Large surpass baselines in most evaluation metrics. The highest score is bolded, and the second highest score is \underline{underlined}.}
\end{table*}


\subsection{Analysis of Small Model Learnability Gap}
\label{More Analysis Results}


\paragraph{Domain knowledge affects learnability gap.}

We observe that math expert models, in spite of small model size,  exhibit a smaller learnability gap for both long CoT and large teacher CoT data compared to general models in Figure \ref{fig:math_expert_vs_general}. 
Specifically, we compare the learnability gaps between the student models \texttt{Qwen2.5-Math-1.5B-Instruct} and \texttt{Qwen2.5-1.5B-Instruct}. Our findings show that the long CoT gap of the small math expert model is significantly smaller than that of general small models. 
Furthermore, the performance improvement of \texttt{Qwen2.5-Math-1.5B} when fined-tuned with large teacher CoT exceeds that of \texttt{Qwen2.5-1.5B}, suggesting that math expert models benefit more substantially from large teacher CoT. We conjecture that a key factor leading to the small model learnability gap is the \textit{limited in-domain knowledge of small student models}.
We summarize this observation in the following takeaway.

\begin{AIbox}{\makecell{Takeaway 3: Effect of Domain Knowledge}} Limited domain knowledge of small models may hinder their learning from strong reasoning teachers.  \end{AIbox}

\begin{figure*}[!t]
    \centering
    \includegraphics[width=\textwidth]{figs/math_expert_vs_general.pdf}
    \caption{Math expert models usually have a less significant Learnability Gap than the general models. 
    A positive Gap means long CoT or large teacher CoT is better while negative means worse. This indicates that the math expert model could more easily learn from long CoT data or large teacher CoT. }
    \label{fig:math_expert_vs_general}
\end{figure*}



\paragraph{Base models exhibit a more significant learnability gap.}
We observe that base models generally exhibit a more significant learnability gap than Instruct models in Figure \ref{fig:Base_vs_Instruct_Gap}. 
This suggests that it is more challenging for small base models to effectively learn from long CoT data or large teacher CoT.


\begin{AIbox}{Takeaway 4: Base vs Instruct}
Small base models experience more significant learnability gap than Instruct models.
\end{AIbox}
\begin{figure*}[!t]
    \centering
    \includegraphics[width=\textwidth]{figs/Base_vs_Instruct_Gap.pdf}
    \caption{Base models generally exhibit a more significant learnability gap than Instruct models. A positive gap indicates that long CoT data or large teacher CoT enhance performance, whereas a negative gap suggests they have the opposite effect. This implies that it is more challenging for small base models to effectively learn from long CoT data or large teacher CoT.}
    % \vspace{-1em}
    \label{fig:Base_vs_Instruct_Gap}
\end{figure*}





\paragraph{Speaking styles shift.}
We adopt the method from  \citep{lin2023unlockingspellbasellms} to evaluate the rank shift of each token before and after fine-tuning on long CoT and Large teacher CoT data. This allows us to compare the token distribution shifts induced by the fine-tuning process. We then annotate the tokens that exhibit the largest rank shifts as the most shifted tokens. Our analysis reveals that these tokens are predominantly associated with expressive and stylistic elements, such as “wait”, “But”, and “Let”. Please see Appendix \ref{Examples of Speaking Style Shift} for more details.


\begin{AIbox}{\makecell{Takeaway 5: Speaking Styles Shift}} 
Long CoT and large teacher CoT primarily shift the student model's  distribution of tokens associated with speaking styles. 
\end{AIbox}




\section{Mix Distillation: Bridge Small Model Learnability Gap}
This section presents our Mix Distillation approach to bridge the small model learnability gap.
\subsection{Mix Distillation}
We propose \textit{Mix Distillation} to address the learnability gap observed in small models. This approach blends easier-to-learn data with more challenging data for small models, thereby leveraging the strengths of both. 

Our insight is that small models tend to perform better on data that closely matches their inherent distribution (such as short CoT or small teacher CoT), while they struggle with data that exhibits greater distribution shifts. The token distribution of the mixed long CoT and large teacher CoT data may become closer to that of small models' inherent distribution, thereby enabling them to learn more effectively from challenging datasets. 

We propose Mix-Long, which combines long CoT and short CoT data with a weight of long CoT $\alpha$ and short CoT $1-\alpha$. Similarly, we proposed Mix-Large, which combines large teacher CoT with a weight of $\alpha$ and small teacher CoT with a weight of $1-\alpha$.


\subsection{Experiment Results}

We use Qwen2.5-3B-Instruct as the student model and MATH (7.5k) as the training set. We distill different teacher models to generate responses as the baseline. They include \texttt{QwQ-32B} (long CoT), \texttt{Qwen2.5-32B} (short CoT), \texttt{Qwen2.5-72B} (large teacher CoT), \texttt{Qwen2.5-3B} (small teacher CoT). We add \texttt{Deepseek-R1-32B} \citep{DeepSeekAI2025DeepseekR1} as the teacher model to generate another set of long CoT data as baseline. We set $\alpha=0.2$ in both configurations of Mix-Long and Mix-Large.


Experimental results demonstrate that both Mix-Long and Mix-Large surpass baselines in most evaluation metrics. We show that the small student model could achieve improved performance by Mix Distillation compared to training on a single dataset. For instance, \texttt{Qwen2.5-3B-Instruct} improves by more than 8 points on MATH and AMC using Mix-Long, compared to direct training on long CoT data. It also shows a more than 7-point gain on MATH, AIME and AMC for \texttt{Qwen2.5-3B-Instruct} by Mix-Large compared with training on large teacher CoT data. This implies that it is easier for small student models to learn from datasets generated by Mix Distillation.

\begin{AIbox}{\makecell{Takeaway 6: Mix Distillation Bridges Gap}}
By mixing long CoT data (resp. large teacher CoTs) and short CoT data (resp. small teacher CoT), the small student model could achieve better performance compared to training on either data alone.
\end{AIbox}


\begin{figure}[!t]
    \centering
    \includegraphics[width=0.35\textwidth]{figs/lg_lmp_mix_weight_ablation.pdf}
    \caption{The average performance varies with the mix weight of long CoT or large teacher CoT data. \texttt{Qwen2.5-3B-Instruct} is chosen as the student model. At a weight of 0.2, mix distillation achieves the highest average performance.}
    % \vspace{-1em}
    \label{fig:lg_lmp_mix_weight_ablation}
\end{figure}







\begin{figure}[!t]
    \centering
    \includegraphics[width=0.5\textwidth]{figs/mix_long.png}
    \caption{Case Study of Mix-Long. Models fine-tuned on long CoT tended to overthink, while those trained on short CoT produced incorrect answers. In contrast, Mix-Long, incorporating branching elements (e.g., “Alternatively”), achieved a balanced reasoning process and arrived at the correct answer.}
    \label{fig:balanced CoT}
\end{figure}

Figure \ref{fig:lg_lmp_mix_weight_ablation} shows the average performance when taking different mix weight $\alpha$ of long CoT data or large teacher CoT. We choose \texttt{Qwen2.5-3B-Instruct} as the student model and find that a weight $\alpha$ of 0.2 achieves the highest average performance across five benchmarks for both Mix-Long and Mix-Large. 

Interestingly, we find that after mixing long CoT and short CoT data, the small student model’s output incorporates characteristics of long CoT, such as a branching process, while maintaining a reduced token length and avoiding overly elaborate thinking. This is illustrated in Figure \ref{fig:balanced CoT}. We observed that the small student model fine-tuned on long CoT data becomes overwhelmed by repeated thoughts and fails to stop, whereas the model fine-tuned on short CoT data produces incorrect answers. In contrast, our proposed Mix-Long, which incorporates branching elements (e.g., the use of “Alternatively”), delivers the correct answer. Additionally, the average token lengths of responses generated by long CoT, short CoT, and Mix-Long are 3384.7, 575.7, and 1248.9, respectively. We suggest that mixing long CoT and short CoT data is a practical approach to achieving a balanced CoT length, thereby enhancing the reasoning capabilities of small student models.

\putsec{related}{Related Work}

\noindent \textbf{Efficient Radiance Field Rendering.}
%
The introduction of Neural Radiance Fields (NeRF)~\cite{mil:sri20} has
generated significant interest in efficient 3D scene representation and
rendering for radiance fields.
%
Over the past years, there has been a large amount of research aimed at
accelerating NeRFs through algorithmic or software
optimizations~\cite{mul:eva22,fri:yu22,che:fun23,sun:sun22}, and the
development of hardware
accelerators~\cite{lee:cho23,li:li23,son:wen23,mub:kan23,fen:liu24}.
%
The state-of-the-art method, 3D Gaussian splatting~\cite{ker:kop23}, has
further fueled interest in accelerating radiance field
rendering~\cite{rad:ste24,lee:lee24,nie:stu24,lee:rho24,ham:mel24} as it
employs rasterization primitives that can be rendered much faster than NeRFs.
%
However, previous research focused on software graphics rendering on
programmable cores or building dedicated hardware accelerators. In contrast,
\name{} investigates the potential of efficient radiance field rendering while
utilizing fixed-function units in graphics hardware.
%
To our knowledge, this is the first work that assesses the performance
implications of rendering Gaussian-based radiance fields on the hardware
graphics pipeline with software and hardware optimizations.

%%%%%%%%%%%%%%%%%%%%%%%%%%%%%%%%%%%%%%%%%%%%%%%%%%%%%%%%%%%%%%%%%%%%%%%%%%
\myparagraph{Enhancing Graphics Rendering Hardware.}
%
The performance advantage of executing graphics rendering on either
programmable shader cores or fixed-function units varies depending on the
rendering methods and hardware designs.
%
Previous studies have explored the performance implication of graphics hardware
design by developing simulation infrastructures for graphics
workloads~\cite{bar:gon06,gub:aam19,tin:sax23,arn:par13}.
%
Additionally, several studies have aimed to improve the performance of
special-purpose hardware such as ray tracing units in graphics
hardware~\cite{cho:now23,liu:cha21} and proposed hardware accelerators for
graphics applications~\cite{lu:hua17,ram:gri09}.
%
In contrast to these works, which primarily evaluate traditional graphics
workloads, our work focuses on improving the performance of volume rendering
workloads, such as Gaussian splatting, which require blending a huge number of
fragments per pixel.

%%%%%%%%%%%%%%%%%%%%%%%%%%%%%%%%%%%%%%%%%%%%%%%%%%%%%%%%%%%%%%%%%%%%%%%%%%
%
In the context of multi-sample anti-aliasing, prior work proposed reducing the
amount of redundant shading by merging fragments from adjacent triangles in a
mesh at the quad granularity~\cite{fat:bou10}.
%
While both our work and quad-fragment merging (QFM)~\cite{fat:bou10} aim to
reduce operations by merging quads, our proposed technique differs from QFM in
many aspects.
%
Our method aims to blend \emph{overlapping primitives} along the depth
direction and applies to quads from any primitive. In contrast, QFM merges quad
fragments from small (e.g., pixel-sized) triangles that \emph{share} an edge
(i.e., \emph{connected}, \emph{non-overlapping} triangles).
%
As such, QFM is not applicable to the scenes consisting of a number of
unconnected transparent triangles, such as those in 3D Gaussian splatting.
%
In addition, our method computes the \emph{exact} color for each pixel by
offloading blending operations from ROPs to shader units, whereas QFM
\emph{approximates} pixel colors by using the color from one triangle when
multiple triangles are merged into a single quad.



\section{Conclusion}
In this work, we propose a simple yet effective approach, called SMILE, for graph few-shot learning with fewer tasks. Specifically, we introduce a novel dual-level mixup strategy, including within-task and across-task mixup, for enriching the diversity of nodes within each task and the diversity of tasks. Also, we incorporate the degree-based prior information to learn expressive node embeddings. Theoretically, we prove that SMILE effectively enhances the model's generalization performance. Empirically, we conduct extensive experiments on multiple benchmarks and the results suggest that SMILE significantly outperforms other baselines, including both in-domain and cross-domain few-shot settings.
\section*{Limitations and Ethical Considerations}

\noindent\textbf{Limitations.} The primary limitation of our work is that it extends only the dataset provided by MUSE and employs DeepSeek-v3 for question generation. 
To mitigate this generalization risk, we have released our code and the generated audit suite, allowing researchers to utilize our framework to create additional audit datasets and evaluate their quality. Meanwhile, this is also our future work to extend our framework to other benchmarks.

\noindent\textbf{Ethical Considerations.} Machine unlearning can be employed to mitigate risks associated with LLMs in terms of privacy, security, bias, and copyright. Our work is dedicated to providing a comprehensive evaluation framework to help researchers better understand the unlearning effectiveness of LLMs, which we believe will have a positive impact on society.
This research focuses on analyzing the internal representations of numerical values in Large Language Models and does not involve human subjects, sensitive data, or direct societal impact.


\bibliography{custom}


\clearpage
\appendix
\section{Detailed Experimental Setups}
\label{appendix:More on Experimental Setups}


\begin{table}[h]
    \centering
    \small
    \begin{tabular}{ll}
        \toprule
        \textbf{Category} & \textbf{Models} \\
        \midrule
        \multicolumn{2}{c}{\textbf{Teacher Models}} \\
        \midrule
        \textbf{Long CoT vs} & QwQ-32B-Preview vs\\\textbf{ShortCoT}& Qwen2.5-32B-Instruct \\
        \midrule
        \textbf{Large Teacher vs}  & \\
        \textbf{Small Teacher}& \\
        \textit{Qwen Family} & Qwen2.5-72B-Instruct vs\\& Qwen2.5-3B-Instruct \\
        \textit{Llama Family} & Llama3.1-70B-Instruct vs\\& Llama3.1-8B-Instruct \\
        \textit{Gemma Family} & Gemma2-27B-it vs\\& Gemma2-9B-it \\
        \midrule
        \multicolumn{2}{c}{\textbf{Student Models}} \\
        \midrule
        \textit{Qwen Family} & Qwen2.5-0.5B-Instruct,\\& Qwen2.5-1.5B-Instruct, \\
        & Qwen2.5-3B-Instruct, \\&Qwen2.5-7B-Instruct, \\
        & Qwen2.5-14B-Instruct,\\& Qwen2.5-32B-Instruct \\
        \textit{Llama Family} & Llama3.2-1B-Instruct,\\& Llama3.2-3B-Instruct, \\
        & Llama3.1-8B-Instruct,\\& Llama3.3-70B-Instruct \\
        \bottomrule
    \end{tabular}
    \caption{Overview of Teacher and Student Models}
    \label{tab:models_overview}
\end{table}




\subsection{Models}
Table \ref{tab:models_overview} presents a comprehensive overview of student and teacher models used in our paper. 




\subsection{Training Setup}
\label{appx:training-setup}
Our model training is conducted using LLaMA-Factory \citep{zheng2024llamafactory}, on a server with four NVIDIA A100-SXM4-80GB GPUs, an AMD EPYC 7763 64-Core Processor, and 512 GB of RAM. We use full parameter fine-tuning on student models less than 14B parameters. 
When the student model is larger than 14B, we use LoRA fine-tuning \cite{hu2021loralowrankadaptationlarge}. 
Table \ref{tab: training-hyperparameters} and Table \ref{tab: training-lora-hyperparameters} list hyper-parameters for full parameter fine-tuning and LoRA fine-tuning respectively. 


\begin{table}[!h]
\small
\centering
\resizebox{0.8\columnwidth}{!}{
\begin{tabular}{ll}
\toprule
\textbf{Hyper-parameter} & \textbf{Value} \\ \midrule
Learning Rate & $1 \times 10^{-5}$ \\
Number of Epochs & $2$ \\
Number of Devices & $4$ \\
Per-device Batch Size & $2$ \\
Optimizer & \texttt{Adamw} \\
Learning Rate Scheduler & \texttt{cosine} \\
Max Sequence Length  & $16384$ \\ \bottomrule
\end{tabular}
}
\caption{This table shows the hyper-parameters for full parameter fine-tuning.}
\label{tab: training-hyperparameters}
\end{table}


\begin{table}[!h]
\small
\centering
\resizebox{0.8\columnwidth}{!}{
\begin{tabular}{ll}
\toprule
\textbf{Hyper-parameter} & \textbf{Value} \\ \midrule
Learning Rate & $1 \times 10^{-4}$ \\
Number of Epochs & $2$ \\
Number of Devices & $4$ \\
Per-device Batch Size & $1$ \\
Lora Target & \texttt{full} \\
Learning Rate Scheduler & \texttt{cosine} \\
Warmup Ratio & $0.03$ \\
Max Sequence Length  & $16384$ \\ \bottomrule
\end{tabular}
}
\caption{This table shows the hyper-parameters for LoRA fine-tuning.}
\label{tab: training-lora-hyperparameters}
\end{table}



Teacher models generate responses by rejection sampling \citep{zelikman2022starbootstrappingreasoningreasoning,tong2024dartmathdifficultyawarerejectiontuning,yue2023mammothbuildingmathgeneralist,singh2024humandatascalingselftraining,gulcehre2023reinforcedselftrainingrestlanguage,yuan2023scalingrelationshiplearningmathematical,dong2023raftrewardrankedfinetuning}. The prompt used to make teacher models generate responses to the math questions is shown as follows.
By default, teacher models employ greedy decoding. 
By combining the math problem instructions with corresponding solutions generated by teacher models, we construct problem-solution pairs to fine-tune student models. 
We perform pairwise comparisons of solutions generated by different teacher models and filter out problem-solution pairs that are correct for both models to fine-tune student models.

\begin{figure}[htbp]
    \centering
\begin{tcolorbox}[title=Prompt, promptstyle]
\lstset{
    basicstyle=\normalfont\sffamily\footnotesize,
    breaklines=true,
    frame=none,
    columns=fullflexible,
}
Solve the following math problem. Present the final answer in the format: Final Answer: $\boxed{\{\texttt{your answer}\}}$

Problem: \{problem\}

Answer:
\end{tcolorbox}
    \label{fig: train_sampling_prompt}
\end{figure}





\subsection{Evaluation Setup} 


We evaluate the reasoning capability of fine-tuned student models on a set of commonly used benchmarks, including MATH \citep{hendrycks2021measuringmathematicalproblemsolving}, GSM8K \citep{cobbe2021trainingverifierssolvemath}, AMC 2023, AIME 2024, and the English math subset of OlympiadBench \citep{he2024olympiadbenchchallengingbenchmarkpromoting}. 

Unless otherwise specified, all fine-tuned models are evaluated in a zero-shot setting using greedy decoding. We set the maximum generation tokens as 16k. The evaluation prompt is shown below. 


\begin{figure}[htbp]
    \centering
\begin{tcolorbox}[title=Prompt, promptstyle]
\lstset{
    basicstyle=\normalfont\sffamily\footnotesize,
    breaklines=true,
    frame=none,
    columns=fullflexible,
}
Solve the following math problem and present the final answer in the format: Final Answer: $\boxed{\{\texttt{your answer}\}}$

Problem: \{problem\}

Answer:
\end{tcolorbox}
    \label{fig: evaluation_prompt}
\end{figure}


After extracting the final answer of the evaluated model, we first employ exact matching to determine the correctness of the answer. If the answer is incorrect, we use Qwen-32B-Instruct as a judge to compare the extracted final answers against that of the ground truth. The prompt is shown below.


\begin{figure}[htbp]
    \centering
\begin{tcolorbox}[title=Prompt, promptstyle]
\lstset{
    basicstyle=\normalfont\sffamily\footnotesize,
    breaklines=true,
    frame=none,
    columns=fullflexible,
}
Given a math problem, its correct final answer, and the model's generated final answer, determine if the model's answer is correct. Respond with 'True' if the it is correct and 'False' if it is incorrect. 

Problem: \texttt{\{problem\}}

Correct Final Answer: \texttt{\{ground truth\}}

Model's Generated Final Answer: \texttt{\{resp answer\}}

Your Judgement:
\end{tcolorbox}
    \label{fig: score_prompt}
\end{figure}





            




\section{More Experiments Results}
\label{appendix: More Experiments}
In this section we present additional experiment results of long CoT gap and large teacher CoT gap.
\subsection{Long CoT Gap: Additional Results}
Table \ref{tab:full_performance_lg} shows the detailed performance scores and gap of each benchmark for different student models fine-tuned on long CoT and short CoT.
\texttt{QwQ-32B-Preview} is chosen to generate long CoT and awhile \texttt{Qwen-2.5-32B-Instruct} is chosen to generate short CoT. We observe that small student models tend to benefit more from short CoT, while large student models gain greater advantages from long CoT. 

\begin{table*}[htbp]
  \centering
  \resizebox{1\textwidth}{!}{%
  \begin{tabular}{l*{5}{ccc}c}
    \toprule
    & \multicolumn{3}{c}{MATH} 
    & \multicolumn{3}{c}{GSM8K} 
    & \multicolumn{3}{c}{AIME} 
    & \multicolumn{3}{c}{AMC} 
    & \multicolumn{3}{c}{Olympiad} 
    & \multicolumn{1}{c}{Average $\Delta_{\rm Long}$} \\
    \cmidrule(lr){2-4} \cmidrule(lr){5-7} \cmidrule(lr){8-10} \cmidrule(lr){11-13} \cmidrule(lr){14-16}
    Model 
    & $P_{\rm Long}$ & $P_{\rm Short}$ & $\Delta_{\rm Long}$ 
    & $P_{\rm Long}$ & $P_{\rm Short}$ & $\Delta_{\rm Long}$ 
    & $P_{\rm Long}$ & $P_{\rm Short}$ & $\Delta_{\rm Long}$ 
    & $P_{\rm Long}$ & $P_{\rm Short}$ & $\Delta_{\rm Long}$ 
    & $P_{\rm Long}$ & $P_{\rm Short}$ & $\Delta_{\rm Long}$ 
    & \\ 
    \midrule

    Llama-3.2-1B  & 28.6  & 33.4  & \cellcolor{red!32} -4.78  & 42.3  & 49.2  & \cellcolor{red!46} -6.90  & 0.00  & 0.00  & 0.00  & 2.50  & 7.50  & \cellcolor{red!33} -5.00  & 5.48  & 7.40  & \cellcolor{red!13} -1.92  & \cellcolor{red!25} -3.72 \\
    Llama-3.2-3B  & 48.7  & 50.9  & \cellcolor{red!14} -2.14  & 75.1  & 77.5  & \cellcolor{red!16} -2.42  & 3.33  & 3.33  & \cellcolor{white} 0.00  & 17.5  & 15.0  & \cellcolor{green!17} 2.50  & 17.6  & 18.7  & \cellcolor{red!7} -1.04  & \cellcolor{red!4} -0.619 \\
    Llama-3.1-8B  & 50.0  & 44.6  & \cellcolor{green!36} 5.36   & 81.4  & 75.5  & \cellcolor{green!39} 5.84   & 0.00  & 0.00  & \cellcolor{white} 0.00   & 27.5  & 22.5  & \cellcolor{green!33} 5.00   & 17.3  & 14.8  & \cellcolor{green!17} 2.52   & \cellcolor{green!25} 3.74 \\
    Llama-3.3-70B & 75.3  & 74.9  & \cellcolor{green!2} 0.340   & 92.7  & 91.2  & \cellcolor{green!10} 1.44   & 26.7  & 13.3  & \cellcolor{green!89} 13.3   & 55.0  & 52.5  & \cellcolor{green!17} 2.50   & 41.3  & 39.7  & \cellcolor{green!11} 1.63   & \cellcolor{green!26} 3.85 \\
    \midrule
    Qwen2.5-0.5B  & 23.0  & 31.5  & \cellcolor{red!56} -8.44  & 39.5  & 45.3  & \cellcolor{red!39} -5.84  & 0.00  & 0.00  & \cellcolor{white} 0.00  & 7.50  & 15.0  & \cellcolor{red!50} -7.50  & 4.00  & 5.93  & \cellcolor{red!13} -1.93  & \cellcolor{red!32} -4.74 \\
    Qwen2.5-1.5B  & 41.6  & 52.3  & \cellcolor{red!71} -10.7  & 63.8  & 71.7  & \cellcolor{red!53} -7.89  & 0.00  & 0.00  & \cellcolor{white} 0.00  & 17.5  & 27.5  & \cellcolor{red!67} -10.0  & 12.3  & 19.4  & \cellcolor{red!47} -7.11  & \cellcolor{red!48} -7.13 \\
    Qwen2.5-3B   & 56.2  & 61.0  & \cellcolor{red!32} -4.84  & 80.0  & 82.0  & \cellcolor{red!13} -1.98  & 3.33  & 10.0  & \cellcolor{red!44} -6.67  & 37.5  & 37.5  & \cellcolor{white} 0.00  & 24.4  & 26.4  & \cellcolor{red!13} -1.93  & \cellcolor{red!21} -3.08 \\
    Qwen2.5-7B   & 68.2  & 67.8  & \cellcolor{green!3} 0.460    & 86.2  & 85.7  & \cellcolor{green!4} 0.560    & 13.3  & 6.67  & \cellcolor{green!44} 6.67    & 40.0  & 40.0  & \cellcolor{white} 0.00   & 36.6  & 35.7  & \cellcolor{green!6} 0.889    & \cellcolor{green!11} 1.72 \\
    Qwen2.5-14B  & 78.3  & 76.2  & \cellcolor{green!14} 2.04   & 93.3  & 92.5  & \cellcolor{green!5} 0.760    & 20.0  & 6.67  & \cellcolor{green!89} 13.3   & 60.0  & 55.0  & \cellcolor{green!33} 5.00   & 44.4  & 40.9  & \cellcolor{green!24} 3.56   & \cellcolor{green!33} 4.94 \\
    Qwen2.5-32B  & 84.8  & 82.3  & \cellcolor{green!16} 2.44   & 94.9  & 94.3  & \cellcolor{green!4} 0.610    & 40.0  & 10.0  & \cellcolor{green!100} 30.0  & 85.0  & 62.5  & \cellcolor{green!100} 22.5  & 60.4  & 47.3  & \cellcolor{green!88} 13.2  & \cellcolor{green!91} 13.7 \\

    \bottomrule
  \end{tabular}
  }


\caption{This table summarizes the performance of models in Llama and Qwen families fine-tuned with long CoT and short CoT data. They are evaluated on MATH, GSM8K, AIME, AMC, and OlympiadBench. \texttt{QwQ-32B-Preview} is chosen to generate long CoT and awhile \texttt{Qwen-2.5-32B-Instruct} is chosen to generate short CoT. We observe that small student models tend to benefit more from short CoT, while large student models gain greater advantages from long CoT.}
\label{tab:full_performance_lg}
\end{table*}

\subsection{Large Teacher CoT Gap: Additional Results}
Table \ref{tab:lmp-full_comparison} shows the detailed performance scores and gap of each benchmark for different student models distilled from large teacher and small teacher.
We summarize the performance of 10 student models from the Llama and Qwen families across various model sizes. \texttt{Qwen-2.5-72B-Instruct} is chosen as the large teacher while \texttt{Qwen-2.5-3B-Instruct} is chosen as the small teacher. The results are shown in Table \ref{tab:lmp-full_comparison}. Our findings indicate that small student models may experience degraded performance when distilled from a large teacher compared to a small teacher, whereas larger student models benefit more from distilling a large teacher.

Table \ref{tab:lmp_comparison2} shows more experiment results for teacher models in different model families, including \texttt{Gemma-27B-it} vs \texttt{Gemma-9B-it} and \texttt{Llama3.1-72B-Instruct} vs \texttt{Llama3.1-8B-Instruct}. 


\begin{table*}[htbp]
  \centering
  \resizebox{1\textwidth}{!}{%
  \begin{tabular}{l*{5}{ccc}c}
    \toprule
    & \multicolumn{3}{c}{MATH} 
    & \multicolumn{3}{c}{GSM8k} 
    & \multicolumn{3}{c}{AIME} 
    & \multicolumn{3}{c}{AMC} 
    & \multicolumn{3}{c}{Olympiad} 
    & \multicolumn{1}{c}{\makecell{Average \\ $\Delta_{\rm Strong}$}} \\
    \cmidrule(lr){2-4} \cmidrule(lr){5-7} \cmidrule(lr){8-10} \cmidrule(lr){11-13} \cmidrule(lr){14-16}
    Model 
    & $P_{\rm Strong}$ & $P_{\rm Weak}$ & $\Delta_{\rm Strong}$ 
    & $P_{\rm Strong}$ & $P_{\rm Weak}$ & $\Delta_{\rm Strong}$ 
    & $P_{\rm Strong}$ & $P_{\rm Weak}$ & $\Delta_{\rm Strong}$ 
    & $P_{\rm Strong}$ & $P_{\rm Weak}$ & $\Delta_{\rm Strong}$ 
    & $P_{\rm Strong}$ & $P_{\rm Weak}$ & $\Delta_{\rm Strong}$ 
    & \\ 
    \midrule
    Llama-3.2-1B      & 29.8   & 29.6   & \cellcolor{green!2}{0.160} 
                      & 44.4   & 47.5   & \cellcolor{red!32}{-3.18} 
                      & 0.00   & 0.00   & \cellcolor{white}{0.00} 
                      & 2.50   & 7.50   & \cellcolor{red!50}{-5.00} 
                      & 6.07   & 7.70   & \cellcolor{red!16}{-1.63} 
                      & \cellcolor{red!19}{-1.93} \\
    Llama-3.2-3B      & 47.4   & 47.9   & \cellcolor{red!5}{-0.500} 
                      & 71.2   & 74.1   & \cellcolor{red!29}{-2.88} 
                      & 3.33   & 0.00   & \cellcolor{green!33}{3.33} 
                      & 25.0   & 17.5   & \cellcolor{green!75}{7.50} 
                      & 16.9   & 16.4   & \cellcolor{green!4}{0.445} 
                      & \cellcolor{green!16}{1.58} \\
    Llama-3.2-8B      & 37.6   & 37.6   & \cellcolor{red!1}{-0.040} 
                      & 67.0   & 69.2   & \cellcolor{red!22}{-2.20} 
                      & 6.67   & 0.00   & \cellcolor{green!67}{6.67} 
                      & 7.50   & 7.50   & \cellcolor{white}{0.00} 
                      & 9.19   & 11.0   & \cellcolor{red!18}{-1.78} 
                      & \cellcolor{green!5}{0.530} \\
    Llama-3.2-70B     & 74.5   & 72.2   & \cellcolor{green!23}{2.28} 
                      & 92.0   & 92.2   & \cellcolor{red!2}{-0.152} 
                      & 16.7   & 16.7   & \cellcolor{white}{0.00} 
                      & 67.5   & 50.0   & \cellcolor{green!100}{17.5} 
                      & 37.3   & 35.7   & \cellcolor{green!16}{1.63} 
                      & \cellcolor{green!43}{4.25} \\
    \midrule
    Qwen2.5-0.5B      & 30.0   & 31.0   & \cellcolor{red!9}{-0.920} 
                      & 43.1   & 45.4   & \cellcolor{red!24}{-2.35} 
                      & 0.00   & 0.00   & \cellcolor{white}{0.00} 
                      & 5.00   & 17.5   & \cellcolor{red!100}{-12.5} 
                      & 6.52   & 8.30   & \cellcolor{red!18}{-1.78} 
                      & \cellcolor{red!35}{-3.51} \\
    Qwen2.5-1.5B      & 50.3   & 50.7   & \cellcolor{red!4}{-0.440} 
                      & 70.6   & 71.0   & \cellcolor{red!5}{-0.455} 
                      & 0.00   & 3.33   & \cellcolor{red!33}{-3.33} 
                      & 22.5   & 20.0   & \cellcolor{green!25}{2.50} 
                      & 17.8   & 20.0   & \cellcolor{red!22}{-2.22} 
                      & \cellcolor{red!8}{-0.790} \\
    Qwen2.5-3B        & 57.5   & 60.3   & \cellcolor{red!28}{-2.82} 
                      & 79.9   & 79.5   & \cellcolor{green!4}{0.379} 
                      & 0.00   & 3.33   & \cellcolor{red!33}{-3.33} 
                      & 35.0   & 27.5   & \cellcolor{green!75}{7.50} 
                      & 25.9   & 26.4   & \cellcolor{red!4}{-0.444} 
                      & \cellcolor{green!3}{0.256} \\
    Qwen2.5-7B        & 71.3   & 63.6   & \cellcolor{green!77}{7.66} 
                      & 87.8   & 84.1   & \cellcolor{green!37}{3.72} 
                      & 6.67   & 0.00   & \cellcolor{green!67}{6.67} 
                      & 40.0   & 35.0   & \cellcolor{green!50}{5.00} 
                      & 38.8   & 29.0   & \cellcolor{green!98}{9.78} 
                      & \cellcolor{green!66}{6.56} \\
    Qwen2.5-14B       & 76.4   & 72.8   & \cellcolor{green!37}{3.66} 
                      & 93.1   & 89.6   & \cellcolor{green!35}{3.49} 
                      & 6.67   & 3.33   & \cellcolor{green!33}{3.33} 
                      & 47.5   & 45.0   & \cellcolor{green!25}{2.50} 
                      & 41.0   & 39.0   & \cellcolor{green!21}{2.07} 
                      & \cellcolor{green!30}{3.01} \\
    Qwen2.5-32B       & 80.5   & 76.8   & \cellcolor{green!37}{3.72} 
                      & 92.2   & 92.7   & \cellcolor{red!5}{-0.531} 
                      & 20.0   & 3.33   & \cellcolor{green!100}{16.7} 
                      & 57.5   & 50.0   & \cellcolor{green!75}{7.50} 
                      & 47.4   & 42.4   & \cellcolor{green!50}{5.04} 
                      & \cellcolor{green!65}{6.48} \\
    \bottomrule
  \end{tabular}
  }


\caption{This table summarizes the performance of models in Llama and Qwen families fine-tuned with large teacher CoT and small teacher CoT when evaluated on MATH, GSM8K, AIME, AMC, and OlympiadBench. \texttt{Qwen-2.5-72B-Instruct} is chosen as the large teacher while \texttt{Qwen-2.5-3B-Instruct} is chosen as the small teacher. We observe that small student models may experience degraded performance when distilled from a large teacher compared to a small teacher, whereas larger student models benefit more from the distilling a large teacher.}
\label{tab:lmp-full_comparison}
\end{table*}




\begin{table*}[htbp]
  \centering
  \resizebox{1\textwidth}{!}{%
  \begin{tabular}{l *{6}{c} *{6}{c}}
    \toprule
    & \multicolumn{6}{c}{Gemma2-9B vs Gemma2-27B} & \multicolumn{6}{c}{Llama3.1-8B vs Llama3.1-70B} \\
    \cmidrule(lr){2-7} \cmidrule(lr){8-13}
    Model & MATH & AMC & Olympiad & AIME & GSM8k & Average & MATH & AMC & Olympiad & AIME & GSM8k & Average \\
    \midrule
    Llama3.2-1B   & \cellcolor{white}{-1.42}  & \cellcolor{white}{-7.50}  & \cellcolor{white}{0.00}   & \cellcolor{white}{0.00}   & \cellcolor{white}{-0.227}  & \cellcolor{red!61}{-1.83}   & \cellcolor{white}{-1.42}  & \cellcolor{white}{-5.00}  & \cellcolor{white}{-0.296} & \cellcolor{white}{3.33}  & \cellcolor{white}{0.152}  & \cellcolor{red!22}{-0.646} \\
    Llama3.2-3B   & \cellcolor{white}{2.08}   & \cellcolor{white}{-7.50}  & \cellcolor{white}{-0.888} & \cellcolor{white}{0.00}   & \cellcolor{white}{1.67}   & \cellcolor{red!31}{-0.928}  & \cellcolor{white}{-0.14}  & \cellcolor{white}{10.0}   & \cellcolor{white}{-0.593} & \cellcolor{white}{3.33}  & \cellcolor{white}{1.06}   & \cellcolor{green!91}{2.73} \\
    Llama3.1-8B   & \cellcolor{white}{0.56}   & \cellcolor{white}{0.00}   & \cellcolor{white}{0.078} & \cellcolor{white}{0.00}   & \cellcolor{white}{-0.516}  & \cellcolor{green!5}{0.0243}  & \cellcolor{white}{-2.18}  & \cellcolor{white}{7.50}   & \cellcolor{white}{2.67}  & \cellcolor{white}{0.00}  & \cellcolor{white}{-1.29}  & \cellcolor{green!45}{1.34} \\
    Llama3.1-70B  & \cellcolor{white}{0.02}   & \cellcolor{white}{7.50}   & \cellcolor{white}{-0.741} & \cellcolor{white}{10.0}   & \cellcolor{white}{0.152}   & \cellcolor{green!100}{3.39}  & \cellcolor{white}{2.72}   & \cellcolor{white}{17.5}   & \cellcolor{white}{5.48}  & \cellcolor{white}{6.67}  & \cellcolor{white}{0.986}   & \cellcolor{green!100}{6.67} \\
\midrule
    Qwen2.5-0.5B  & \cellcolor{white}{-4.56}  & \cellcolor{white}{0.00}   & \cellcolor{white}{0.741} & \cellcolor{white}{0.00}   & \cellcolor{white}{0.592}   & \cellcolor{red!22}{-0.645}  & \cellcolor{white}{-1.88}  & \cellcolor{white}{0.00}   & \cellcolor{white}{0.185} & \cellcolor{white}{0.00}  & \cellcolor{white}{-1.74}  & \cellcolor{red!23}{-0.688} \\
    Qwen2.5-1.5B  & \cellcolor{white}{-1.20}  & \cellcolor{white}{2.50}   & \cellcolor{white}{-1.19} & \cellcolor{white}{0.00}   & \cellcolor{white}{-0.986}  & \cellcolor{red!6}{-0.174}  & \cellcolor{white}{-1.48}  & \cellcolor{white}{5.00}   & \cellcolor{white}{-0.148} & \cellcolor{white}{3.33}  & \cellcolor{white}{-1.14}  & \cellcolor{green!37}{1.11} \\
    Qwen2.5-3B    & \cellcolor{white}{0.44}   & \cellcolor{white}{5.00}   & \cellcolor{white}{1.78}  & \cellcolor{white}{0.00}   & \cellcolor{white}{-0.758}  & \cellcolor{green!43}{1.29}   & \cellcolor{white}{-1.26}  & \cellcolor{white}{5.00}   & \cellcolor{white}{-0.741} & \cellcolor{white}{-3.33} & \cellcolor{white}{-1.29}  & \cellcolor{red!11}{-0.325} \\
    Qwen2.5-7B    & \cellcolor{white}{0.22}   & \cellcolor{white}{5.00}   & \cellcolor{white}{1.04}  & \cellcolor{white}{-3.33}  & \cellcolor{white}{3.94}   & \cellcolor{green!46}{1.37}   & \cellcolor{white}{3.68}   & \cellcolor{white}{20.0}   & \cellcolor{white}{4.15}  & \cellcolor{white}{3.33}  & \cellcolor{white}{2.81}   & \cellcolor{green!100}{6.79} \\
    Qwen2.5-14B   & \cellcolor{white}{1.32}   & \cellcolor{white}{2.50}   & \cellcolor{white}{-0.148} & \cellcolor{white}{0.00}   & \cellcolor{white}{-0.986}  & \cellcolor{green!18}{0.537}  & \cellcolor{white}{2.18}   & \cellcolor{white}{0.00}   & \cellcolor{white}{0.445}  & \cellcolor{white}{3.33}  & \cellcolor{white}{-0.303}  & \cellcolor{green!38}{1.13} \\
    Qwen2.5-32B   & \cellcolor{white}{0.10}   & \cellcolor{white}{2.50}   & \cellcolor{white}{1.48}  & \cellcolor{white}{3.44}   & \cellcolor{white}{1.36}   & \cellcolor{green!59}{1.78}  & \cellcolor{white}{2.72}   & \cellcolor{white}{-2.50}  & \cellcolor{white}{5.63}  & \cellcolor{white}{3.33}  & \cellcolor{white}{0.834}   & \cellcolor{green!67}{2.00} \\
    \bottomrule
  \end{tabular}
  }
  \caption{This table presents the performance of student models distilled from different teacher models, including \texttt{Gemma-27B-it} vs \texttt{Gemma-9B-it} and \texttt{Llama3.1-72B-Instruct} vs \texttt{Llama3.1-8B-Instruct}. We observe that small student models may experience degraded performance when distilled from a large teacher compared to a small teacher, whereas larger student models benefit more from the distilling a large teacher.}

  \label{tab:lmp_comparison2}
\end{table*}

\section{Examples of Speaking Style Shift}
\label{Examples of Speaking Style Shift}
We adopt the method from  \citep{lin2023unlockingspellbasellms} to evaluate the most shifted tokens after fine-tuning on long CoT and Large teacher CoT data. Figure \ref{fig:speaking_style} shows the calculation process. This allows us to compare the token distribution shifts induced by the fine-tuning process. We annotate the tokens that exhibit the largest rank shifts as the most shifted tokens. We choose \texttt{Qwen2.5-3B-Instruct} as the student model. We put the results of most shifted tokens after fine-tuning on long CoT data in Figure \ref{fig: most_shifted_1} and \ref{fig: most_shifted_2}. The results of most shifted tokens after fine-tuning on large teacher CoT data are shown in Figure \ref{fig: most_shifted_3}. Our analysis reveals that these tokens are predominantly associated with expressive and stylistic elements, such as “wait”, “But”, and “Let”. 


\begin{figure*}[!t]
    \centering
    \includegraphics[width=1\textwidth]{figs/speaking_way_shift.png}
    \caption{The process of calculating most shifted tokens. We decode each token generated by the fine-tuned LLM in the student model before fine-tuning. Then we calculate the rank shift in the student model for each token generated by the fine-tuned model. We annotate the tokens that exhibit the largest rank shifts as the most shifted tokens. We found that these tokens are predominantly associated with expressive and stylistic elements, such as “But” and “Let”.}
    \label{fig:speaking_style}
\end{figure*}


\begin{figure*}[htbp]
    \centering
\begin{tcolorbox}[title=Problem, promptstyle]
\lstset{
    basicstyle=\normalfont\sffamily\footnotesize,
    breaklines=true,
    frame=none,
    columns=fullflexible,
}
Let $(x,y)$ be an ordered pair of real numbers that satisfies the equation $x^2+y^2=14x+48y$. What is the maximum value of $y$?
\end{tcolorbox}
\end{figure*}






\begin{figure*}[htbp]
    \centering
\begin{tcolorbox}[title=Example of \textcolor{red}{most shifted tokens} after fine-tuning on long CoT data. (1/2), promptstyle]
\lstset{
    basicstyle=\normalfont\sffamily\footnotesize,
    breaklines=true,
    frame=none,
    columns=fullflexible,
}
So I've got this math problem here\textcolor{red}{.} It says\textcolor{red}{:} ``\textcolor{red}{Let} (\textcolor{red}{x}\textcolor{red}{,} y\textcolor{red}{)} be an ordered pair of real numbers that satisfies the equation \(x^2 + y^2 = 14x + 48y\)\textcolor{red}{.} What is the maximum value of \(y\)\textcolor{red}{?} Okay\textcolor{red}{,} let's see how to approach this.

\textcolor{red}{First}\textcolor{red}{,} I need to understand what the equation is telling me\textcolor{red}{.} It's an equation involving \(x\) and \(y\)\textcolor{red}{,} and it's set equal to zero\textcolor{red}{:} 
\[
x^2 + y^2 - 14x - 48y = 0.
\]
I need to find the maximum value of \(y\) for real numbers \(x\) and \(y\) that satisfy this equation.

Hmm\textcolor{red}{,} maybe I can rearr\textcolor{red}{ange} this equation to make it easier to work with\textcolor{red}{.} Let's see:
\[
x^2 - 14x + y^2 - 48y = 0.
\]
I can complete the square for both \(x\) and \(y\) terms.

\textcolor{red}{For} \(x\)\textcolor{red}{:} \(x^2 - 14x\)\textcolor{red}{,} to complete the square\textcolor{red}{,} I take half of \(-14\), which is \(-7\), and square it to get \(49\)\textcolor{red}{.} So\textcolor{red}{,} \(x^2 - 14x\) becomes \((x - 7\textcolor{red}{)}^2 - 49\).

\textcolor{red}{For} \(y\): \(y^2 - 48y\), half of \(-48\) is \(-24\), squared is \(576\)\textcolor{red}{.} So, \(y^2 - 48y\) becomes \((y - 24\textcolor{red}{)}^2 - 576\).

\textcolor{red}{Now}\textcolor{red}{,} pl\textcolor{red}{ugging} these back into the equation:
\[
(x - 7\textcolor{red}{)}^2 - 49 + (y - 24\textcolor{red}{)}^2 - 576 = 0.
\]
S\textcolor{red}{implify}:
\[
(x - 7\textcolor{red}{)}^2 + (y - 24)^2 - 625 = 0,
\]
\[
\text{So, } (x - 7\textcolor{red}{)}^2 + (y - 24)^2 = 625.
\]

\textcolor{red}{This} looks like the equation of a circle with center at \((7, 24\textcolor{red}{)}\) and radius \(25\).

Wait\textcolor{red}{,} but the problem is asking for the maximum value of \(y\)\textcolor{red}{,} and I have a circle equation\textcolor{red}{.} \textcolor{red}{So}\textcolor{red}{,} the maximum \(y\) would be the \(y\)-coordinate of the highest point on the circle\textcolor{red}{,} which is the center plus the radius.

Wait\textcolor{red}{,} but the center is at \((7\textcolor{red}{,} 24)\), and the radius is \(25\)\textcolor{red}{,} so the highest point would be at \((7\textcolor{red}{,} 24 + 25) = (7\textcolor{red}{,} 49)\).

\textcolor{red}{But}\textcolor{red}{,} let me double-check if this makes sense\textcolor{red}{.} The equation is 
\[
x^2 + y^2 = 14x + 48y,
\]
and I've transformed it into 
\[
(\textcolor{red}{x} - 7\textcolor{red}{)}^2 + (y - 24)^2 = 625,
\]
which is indeed a circle with center \((7, 24\textcolor{red}{)}\) and radius \(25\).

\textcolor{red}{So}\textcolor{red}{,} the maximum \(y\) would be \(24 + 25 = 49\).

\textcolor{red}{But}\textcolor{red}{,} wait\textcolor{red}{,} the problem says ``\textcolor{red}{the} maximum value of \(y\)\textcolor{red}{,}'' and I've found that to be \(49\)\textcolor{red}{.} But let me consider if there are any other constraints or if I've missed something.

\textcolor{red}{Let} me think about the original equation:
\[
x^2 + y^2 = 14x + 48y.
\]
I can rearr\textcolor{red}{ange} it to:
\[
x^2 - 14x + y^2 - 48y = 0.
\]

\end{tcolorbox}
\caption{Example of \textcolor{red}{most shifted tokens} after fine-tuning on long CoT data. All the most shifted tokens are labeled as red.}
    \label{fig: most_shifted_1}
\end{figure*}



\begin{figure*}[htbp]
    \centering
\begin{tcolorbox}[title=Example of \textcolor{red}{most shifted tokens} after fine-tuning on long CoT data (2/2), promptstyle]
\lstset{
    basicstyle=\normalfont\sffamily\footnotesize,
    breaklines=true,
    frame=none,
    columns=fullflexible,
}
\textcolor{red}{Then}\textcolor{red}{,} completing the square:
\[
(x - 7\textcolor{red}{)}^2 - 49 + (y - 24)^2 - 576 = 0,
\]
\[
(x - 7\textcolor{red}{)}^2 + (y - 24)^2 = 625.
\]
Yes\textcolor{red}{,} that's a circle with center \((7, 24\textcolor{red}{)}\) and radius \(25\).

\textcolor{red}{So}\textcolor{red}{,} the maximum \(y\) is indeed \(24 + 25 = 49\).

\textcolor{red}{But}\textcolor{red}{,} just to be thorough\textcolor{red}{,} let's consider if there are any points on the circle where \(y\) is greater than \(49\)\textcolor{red}{.} The center is at \((7, 24\textcolor{red}{)}\), and the radius is \(25\)\textcolor{red}{,} so the highest point is at \((7, 49\textcolor{red}{)}\), and the lowest point would be at \((7\textcolor{red}{,} -1)\).

\textcolor{red}{But}\textcolor{red}{,} the problem is asking for the maximum value of \(y\)\textcolor{red}{,} and I've found that to be \(49\).

Therefore\textcolor{red}{,} the final answer is \(49\).

\bigskip

\textbf{\textcolor{red}{Final} Answer:}

\[
\boxed{49\textcolor{red}{}}
\]
\end{tcolorbox}
\caption{\textcolor{red}{Most shifted tokens} after fine-tuning on long CoT data (2/2). All the most shifted tokens are labeled as red.}
    \label{fig: most_shifted_2}
\end{figure*}






\begin{figure*}[htbp]
    \centering
\begin{tcolorbox}[title=Example of \textcolor{red}{Most Shifted Tokens} after fine-tuning on large model CoT, promptstyle]
\lstset{
    basicstyle=\normalfont\sffamily\footnotesize,
    breaklines=true,
    frame=none,
    columns=fullflexible,
}


\textcolor{red}{To} \textcolor{red}{solve} \textcolor{red}{the} \textcolor{red}{problem}\textcolor{red}{,} \textcolor{red}{ we} \textcolor{red}{ start} \textcolor{red}{ with} \textcolor{red}{ the} \textcolor{red}{ given} equation\textcolor{red}{:}

\[
x^2 + y^2\textcolor{red}{ =} 14x + 48y
\]

\textcolor{red}{First}\textcolor{red}{,} we rearr\textcolor{red}{ange} the equation to bring all terms to one\textcolor{red}{ side}\textcolor{red}{:}

\[
x^2 - 14\textcolor{red}{x} + y^2 - 48\textcolor{red}{y} = 0
\]

\textcolor{red}{Next}\textcolor{red}{,} we complete the square for both \(x\) and \(\textcolor{red}{y}\).

\textcolor{red}{For} \(x\):
\[
x^2 - 14x
\]
\textcolor{red}{To} complete the square\textcolor{red}{,} we add and subtract \(\left(\frac{14}{2}\right)^2 = 49\):
\[
x^2 - 14x + 49 - 49
\]
\[
(\textcolor{red}{x} - 7)^2 - 49
\]

\textcolor{red}{For} \(y\):
\[
y^2 - 48y
\]
\textcolor{red}{To} complete the square, we add and subtract \(\left(\frac{48}{2}\right)^2 = 576\):
\[
y^2 - 48y + 576 - 576
\]
\[
(\textcolor{red}{y} - 24)^2 - 576
\]

\textcolor{red}{Now}\textcolor{red}{,} substitute these completed squares back into the equation:
\[
(\textcolor{red}{x} - 7)^2 - 49 + (y - 24)^2 - 576 = 0
\]
\[
(\textcolor{red}{x} - 7)^2 + (y - 24)^2 - 625 = 0
\]
\[
(\textcolor{red}{x} - 7)^2 + (y - 24)^2 = 625
\]

\textcolor{red}{This} is the equation of a circle with center \((7, 24)\) and radius \(25\).

\textcolor{red}{To} find the maximum value of \(y\), we need to find the highest point on the circle. \textcolor{red}{The} highest point on the circle is directly above the center, which is at \(y = 24 + 25 = 49\).

\textcolor{red}{Thus}\textcolor{red}{,} the maximum value of \(y\) is:
\[
\textcolor{red}{y} = 49
\]

\textcolor{red}{Final} \textcolor{red}{ Answer}\textcolor{red}{:} \(\boxed{49}\)

\end{tcolorbox}
\caption{Example of \textcolor{red}{most shifted tokens} after fine-tuning on large teacher CoT. All the most shifted tokens are labeled as red.}
    \label{fig: most_shifted_3}
\end{figure*}




\section{Examples of Various CoT Data}\label{app:example}
This section demonstrates examples of different CoT data including long CoT, short CoT, large teacher CoT and small teacher CoT. See the following examples.
% \subsection{Long CoT and Short CoT}


\begin{figure*}[htbp]
    \centering
\begin{tcolorbox}[title=Problem, promptstyle]
\lstset{
    basicstyle=\normalfont\sffamily\footnotesize,
    breaklines=true,
    frame=none,
    columns=fullflexible,
}
What is the sum of the positive odd divisors of $60$?
\end{tcolorbox}
    
\end{figure*}



\begin{figure*}[htbp]
    \centering
\begin{tcolorbox}[title=Long CoT Example, promptstyle]
\lstset{
    basicstyle=\normalfont\sffamily\footnotesize,
    breaklines=true,
    frame=none,
    columns=fullflexible,
}
To find the sum of the positive odd divisors of 60, I first need to understand what the divisors of 60 are and then identify which of those are odd.

First, let's find all the positive divisors of 60. To do this, I can start by factoring 60 into its prime factors.

\[
60 = 2^2 \times 3^1 \times 5^1
\]

Now, to find all the positive divisors, I can take all possible products of these prime factors raised to their possible exponents. For 2, the exponents can be \(0, 1,\) or \(2\); for 3, \(0\) or \(1\); and for 5, \(0\) or \(1\). Listing them out:

\[
\begin{aligned}
&1 = 2^0 \times 3^0 \times 5^0, \\
&2 = 2^1 \times 3^0 \times 5^0, \\
&4 = 2^2 \times 3^0 \times 5^0, \\
&3 = 2^0 \times 3^1 \times 5^0, \\
&6 = 2^1 \times 3^1 \times 5^0, \\
&12 = 2^2 \times 3^1 \times 5^0, \\
&5 = 2^0 \times 3^0 \times 5^1, \\
&10 = 2^1 \times 3^0 \times 5^1, \\
&20 = 2^2 \times 3^0 \times 5^1, \\
&15 = 2^0 \times 3^1 \times 5^1, \\
&30 = 2^1 \times 3^1 \times 5^1, \\
&60 = 2^2 \times 3^1 \times 5^1.
\end{aligned}
\]

So, the positive divisors of 60 are:
\[
1, 2, 3, 4, 5, 6, 10, 12, 15, 20, 30, 60.
\]

Now, I need to identify which of these are odd. Odd numbers are not divisible by 2, so I can eliminate any divisors that are even.

The odd divisors are:
\[
1, 3, 5, 15.
\]

Now, I need to find the sum of these odd divisors:
$1 + 3 + 5 + 15 = 24$.
Alternatively, using the prime factorization approach: Since the odd divisors are those that do not include the factor \(2\), I can ignore the 2's in the prime factorization. The remaining odd part is:

\[
3^1 \times 5^1.
\]

The sum of the divisors of this odd part is:

\[
(1 + 3)(1 + 5) = 4 \times 6 = 24.
\]

This matches the previous result.

Therefore, for \( 60 = 2^2 \times 3^1 \times 5^1 \), the sum of the odd divisors is:

\[
\boxed{24}.
\]
\end{tcolorbox}
\caption{Long CoT Example}
    \label{fig: long cot example}
\end{figure*}



\begin{figure*}[htbp]
    \centering
\begin{tcolorbox}[title=Short CoT Example, promptstyle]
\lstset{
    basicstyle=\normalfont\sffamily\footnotesize,
    breaklines=true,
    frame=none,
    columns=fullflexible,
}

To find the sum of the positive odd divisors of $60$, we first need to factorize $60$ into its prime factors. The prime factorization of $60$ is:

\[
60 = 2^2 \times 3 \times 5
\]

Since we are only interested in the odd divisors, we can ignore the factor of $2$. Therefore, we only consider the factors $3$ and $5$. The odd divisors of $60$ are the divisors of $3 \times 5 = 15$. The divisors of $15$ are \(1, 3, 5,\) and \(15\).

Next, we sum these divisors:

\[
1 + 3 + 5 + 15 = 24
\]

Therefore, the sum of the positive odd divisors of $60$ is:

\[
\boxed{24}
\]

\end{tcolorbox}
    \label{fig: short cot example}
    \caption{Short CoT example}
\end{figure*}




\begin{figure*}[htbp]
    \centering
\begin{tcolorbox}[title=Problem, promptstyle]
\lstset{
    basicstyle=\normalfont\sffamily\footnotesize,
    breaklines=true,
    frame=none,
    columns=fullflexible,
}
Find the greatest \( a \) such that
\[
\frac{7\sqrt{(2a)^2+(1)^2}-4a^2-1}{\sqrt{1+4a^2}+3}=2.
\]
\end{tcolorbox}
\end{figure*}



\begin{figure*}[htbp]
    \centering
\begin{tcolorbox}[title=Large Teacher CoT Example (1/2), promptstyle]
\lstset{
    basicstyle=\normalfont\sffamily\footnotesize,
    breaklines=true,
    frame=none,
    columns=fullflexible,
}

To solve the equation \(\frac{7\sqrt{(2a)^2 + 1^2} - 4a^2 - 1}{\sqrt{1 + 4a^2} + 3} = 2\), we start by simplifying the expression inside the square root in the numerator. The equation becomes:

\[
\frac{7\sqrt{4a^2 + 1} - 4a^2 - 1}{\sqrt{1 + 4a^2} + 3} = 2
\]

Let's denote \(x = \sqrt{1 + 4a^2}\). Then the equation transforms into:

\[
\frac{7x - 4a^2 - 1}{x + 3} = 2
\]

Next, we clear the fraction by multiplying both sides by \(x + 3\):

\[
7x - 4a^2 - 1 = 2(x + 3)
\]

Expanding and simplifying the right side, we get:

\[
7x - 4a^2 - 1 = 2x + 6
\]

Rearranging terms to isolate \(x\) and \(a^2\), we have:

\[
7x - 2x - 4a^2 - 1 - 6 = 0 \implies 5x - 4a^2 - 7 = 0 \implies 5x = 4a^2 + 7 \implies x = \frac{4a^2 + 7}{5}
\]

Recall that \(x = \sqrt{1 + 4a^2}\). Therefore, we have:

\[
\sqrt{1 + 4a^2} = \frac{4a^2 + 7}{5}
\]

To eliminate the square root, we square both sides:

\[
1 + 4a^2 = \left(\frac{4a^2 + 7}{5}\right)^2
\]

Expanding the right side, we get:

\[
1 + 4a^2 = \frac{(4a^2 + 7)^2}{25} = \frac{16a^4 + 56a^2 + 49}{25}
\]

Multiplying both sides by 25 to clear the denominator, we obtain:

\[
25 + 100a^2 = 16a^4 + 56a^2 + 49
\]

\end{tcolorbox}
\end{figure*}



\begin{figure*}[htbp]
    \centering
\begin{tcolorbox}[title=Large Teacher CoT Example (2/2), promptstyle]
\lstset{
    basicstyle=\normalfont\sffamily\footnotesize,
    breaklines=true,
    frame=none,
    columns=fullflexible,
}
Rearranging all terms to one side, we get a standard polynomial equation:

\[
16a^4 + 56a^2 + 49 - 100a^2 - 25 = 0 \implies 16a^4 - 44a^2 + 24 = 0
\]

Let \(y = a^2\). Then the equation becomes a quadratic in \(y\):

\[
16y^2 - 44y + 24 = 0
\]

We solve this quadratic equation using the quadratic formula \(y = \frac{-b \pm \sqrt{b^2 - 4ac}}{2a}\), where \(a = 16\), \(b = -44\), and \(c = 24\):

\[
y = \frac{44 \pm \sqrt{(-44)^2 - 4 \cdot 16 \cdot 24}}{2 \cdot 16} = \frac{44 \pm \sqrt{1936 - 1536}}{32} = \frac{44 \pm \sqrt{400}}{32} = \frac{44 \pm 20}{32}
\]

This gives us two solutions for \(y\):

\[
y = \frac{64}{32} = 2 \quad \text{and} \quad y = \frac{24}{32} = \frac{3}{4}
\]

Since \(y = a^2\), we have \(a^2 = 2\) or \(a^2 = \frac{3}{4}\). The corresponding values for \(a\) are \(a = \pm \sqrt{2}\) and \(a = \pm \frac{\sqrt{3}}{2}\). The greatest value of \(a\) is \(\sqrt{2}\).

Thus, the greatest \(a\) that satisfies the equation is:

\[
\boxed{\sqrt{2}}
\]
\end{tcolorbox}
\end{figure*}



\begin{figure*}[htbp]
    \centering
\begin{tcolorbox}[title=Small Teacher CoT Example (1/2), promptstyle]
\lstset{
    basicstyle=\normalfont\sffamily\footnotesize,
    breaklines=true,
    frame=none,
    columns=fullflexible,
}

To solve the equation \(\frac{7\sqrt{(2a)^2+(1)^2}-4a^2-1}{\sqrt{1+4a^2}+3}=2\), we start by simplifying the expression inside the square root and the equation itself.

First, simplify the expression under the square root in the numerator:
\[
(2a)^2 + 1^2 = 4a^2 + 1
\]
So the equation becomes:
\[
\frac{7\sqrt{4a^2 + 1} - 4a^2 - 1}{\sqrt{4a^2 + 1} + 3} = 2
\]

Next, let \( x = \sqrt{4a^2 + 1} \). Then the equation transforms into:
\[
\frac{7x - 4a^2 - 1}{x + 3} = 2
\]

Since \( x = \sqrt{4a^2 + 1} \), we can express \( 4a^2 \) in terms of \( x \):
\[
4a^2 = x^2 - 1
\]

Substitute \( 4a^2 = x^2 - 1 \) into the equation:
\[
\frac{7x - (x^2 - 1) - 1}{x + 3} = 2
\]
Simplify the numerator:
\[
7x - x^2 + 1 - 1 = 7x - x^2
\]
So the equation becomes:
\[
\frac{7x - x^2}{x + 3} = 2
\]

Multiply both sides by \( x + 3 \) to clear the denominator:
\[
7x - x^2 = 2(x + 3)
\]
Expand and simplify:
\[
7x - x^2 = 2x + 6
\]
Rearrange all terms to one side:
\[
-x^2 + 7x - 2x - 6 = 0
\]
\[
-x^2 + 5x - 6 = 0
\]
Multiply through by -1 to make the leading coefficient positive:
\[
x^2 - 5x + 6 = 0
\]

\end{tcolorbox}
    
\end{figure*}




\begin{figure*}[htbp]
    \centering
\begin{tcolorbox}[title=Small Teacher CoT Example (2/2), promptstyle]
\lstset{
    basicstyle=\normalfont\sffamily\footnotesize,
    breaklines=true,
    frame=none,
    columns=fullflexible,
}

Factor the quadratic equation:
\[
(x - 2)(x - 3) = 0
\]

So, the solutions are:
\[
x = 2 \quad \text{or} \quad x = 3
\]

Recall that \( x = \sqrt{4a^2 + 1} \). We need to check both solutions:

1. For \( x = 2 \):
\[
\sqrt{4a^2 + 1} = 2
\]
Square both sides:
\[
4a^2 + 1 = 4
\]
\[
4a^2 = 3
\]
\[
a^2 = \frac{3}{4}
\]
\[
a = \pm \frac{\sqrt{3}}{2}
\]

2. For \( x = 3 \):
\[
\sqrt{4a^2 + 1} = 3
\]
Square both sides:
\[
4a^2 + 1 = 9
\]
\[
4a^2 = 8
\]
\[
a^2 = 2
\]
\[
a = \pm \sqrt{2}
\]

The greatest value of \( a \) among the solutions is \( \sqrt{2} \).

Thus, the final answer is:
\[
\boxed{\sqrt{2}}
\]
\end{tcolorbox}
    
\end{figure*}




\begin{figure*}[htbp]
    \centering
\begin{tcolorbox}[title=Problem, promptstyle]
\lstset{
    basicstyle=\normalfont\sffamily\footnotesize,
    breaklines=true,
    frame=none,
    columns=fullflexible,
}
What is the greatest common divisor of 128, 144 and 480?
\end{tcolorbox}
    
\end{figure*}



\begin{figure*}[htbp]
    \centering
\begin{tcolorbox}[title=Large Teacher CoT Example, promptstyle]
\lstset{
    basicstyle=\normalfont\sffamily\footnotesize,
    breaklines=true,
    frame=none,
    columns=fullflexible,
}
To find the greatest common divisor (GCD) of 128, 144, and 480, we can use the prime factorization method.

\begin{enumerate}
    \item \textbf{Prime Factorization:}
    \begin{align*}
        128 &= 2^7, \\
        144 &= 2^4 \times 3^2, \\
        480 &= 2^5 \times 3 \times 5.
    \end{align*}

    \item \textbf{Identify the common prime factors:}\\
    The common prime factor among 128, 144, and 480 is \(2\).

    \item \textbf{Determine the lowest power of the common prime factor:}\\
    The lowest power of \(2\) in the factorizations is \(2^4\).

    \item \textbf{Calculate the GCD:}\\
    The GCD is:
    \[
    2^4 = 16.
    \]
\end{enumerate}

Therefore, the greatest common divisor of 128, 144, and 480 is:
\[
\boxed{16}
\]
\end{tcolorbox}
    
\end{figure*}


\begin{figure*}[htbp]
    \centering
\begin{tcolorbox}[title=Small Teacher CoT Example, promptstyle]
\lstset{
    basicstyle=\normalfont\sffamily\footnotesize,
    breaklines=true,
    frame=none,
    columns=fullflexible,
}


To find the greatest common divisor (GCD) of 128, 144, and 480, we start by finding the prime factorization of each number.

The prime factorization of 128 is:
\[
\begin{aligned}
128 \div 2 &= 64 \\
64 \div 2 &= 32 \\
32 \div 2 &= 16 \\
16 \div 2 &= 8 \\
8 \div 2 &= 4 \\
4 \div 2 &= 2 \\
2 \div 2 &= 1
\end{aligned}
\]
Thus, 
\[
128 = 2^7.
\]

The prime factorization of 144 is:
\[
\begin{aligned}
144 \div 2 &= 72 \\
72 \div 2 &= 36 \\
36 \div 2 &= 18 \\
18 \div 2 &= 9 \\
9 \div 3 &= 3 \\
3 \div 3 &= 1
\end{aligned}
\]
Thus,
\[
144 = 2^4 \times 3^2.
\]

The prime factorization of 480 is:
\[
\begin{aligned}
480 \div 2 &= 240 \\
240 \div 2 &= 120 \\
120 \div 2 &= 60 \\
60 \div 2 &= 30 \\
30 \div 2 &= 15 \\
15 \div 3 &= 5 \\
5 \div 5 &= 1
\end{aligned}
\]
Thus,
\[
480 = 2^5 \times 3 \times 5.
\]

Now, we identify the common prime factors. The factorizations are:
\[
128 = 2^7, \quad 144 = 2^4 \times 3^2, \quad 480 = 2^5 \times 3 \times 5.
\]
The only common prime factor among all three numbers is \(2\).

Next, we determine the smallest power of the common prime factor. The lowest power of \(2\) among these factorizations is \(2^4\).

Therefore, the GCD of 128, 144, and 480 is:
\[
\boxed{16}
\]

\end{tcolorbox}
    
\end{figure*}







\begin{figure*}[htbp]
    \centering
\begin{tcolorbox}[title=Problem, promptstyle]
\lstset{
    basicstyle=\normalfont\sffamily\footnotesize,
    breaklines=true,
    frame=none,
    columns=fullflexible,
}
Rounded to 2 decimal places, what is $\frac{7}{9}$?
\end{tcolorbox}
    
\end{figure*}



\begin{figure*}[htbp]
    \centering
\begin{tcolorbox}[title=Large Teacher CoT Example, promptstyle]
\lstset{
    basicstyle=\normalfont\sffamily\footnotesize,
    breaklines=true,
    frame=none,
    columns=fullflexible,
}
To solve the problem, we need to divide 7 by 9 and then round the result to two decimal places.

\[
\frac{7}{9} = 0.777\ldots
\]

Rounded to two decimal places, this becomes 0.78.

Final Answer: \(\boxed{0.78}\)
\end{tcolorbox}
    
\end{figure*}


\begin{figure*}[htbp]
    \centering
\begin{tcolorbox}[title=Small Teacher CoT Example, promptstyle]
\lstset{
    basicstyle=\normalfont\sffamily\footnotesize,
    breaklines=true,
    frame=none,
    columns=fullflexible,
}
To solve the problem of finding the value of \(\frac{7}{9}\) rounded to 2 decimal places, we can follow these steps:

\begin{enumerate}
    \item \textbf{Perform the division}: Calculate \(\frac{7}{9}\).
    \item \textbf{Convert to decimal}: Perform the division to get the decimal equivalent.
    \item \textbf{Round to 2 decimal places}: Round the decimal to 2 decimal places.
\end{enumerate}

Let's start with the division:

\[
\frac{7}{9} \approx 0.7777\ldots
\]

Next, we need to round this decimal to 2 decimal places. The digit in the third decimal place is 7, which is greater than or equal to 5. Therefore, we round the second decimal place up from 7 to 8.

So, \(0.7777\ldots\) rounded to 2 decimal places is \(0.78\).

The final answer is:

\[
\boxed{0.78}
\]
\end{tcolorbox}
    
\end{figure*}






\end{document}
