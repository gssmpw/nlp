\begin{abstract}
\label{sec:abstract}
Recent advancements in protein design have leveraged diffusion models to generate structural scaffolds, followed by a process known as protein inverse folding, which involves sequence inference on these scaffolds. However, these methodologies face significant challenges when applied to hyper-variable structures such as antibody Complementarity-Determining Regions (CDRs), where sequence inference frequently results in non-functional sequences due to hallucinations. Distinguished from prevailing protein inverse folding approaches, this paper introduces {\igseek}, a novel structure-retrieval framework that infers CDR sequences by retrieving similar structures from a natural antibody database. Specifically, {\igseek} employs a simple yet effective multi-channel equivariant graph neural network to generate high-quality geometric representations of CDR backbone structures. Subsequently, it aligns sequences of structurally similar CDRs and utilizes structurally conserved sequence motifs to enhance inference accuracy. Our experiments demonstrate that {\igseek} not only proves to be highly efficient in structural retrieval but also outperforms state-of-the-art approaches in sequence recovery for both antibodies and T-Cell Receptors, offering a new retrieval-based perspective for therapeutic protein design.
\end{abstract}