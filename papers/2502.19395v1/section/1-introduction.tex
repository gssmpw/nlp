\section{Main}
\label{sec:intro}

Antibodies, known for their high specificity and affinity, have emerged as pivotal therapeutic agents in the treatment of complex diseases, including cancer \cite{adams2005cancer}, autoimmune disorders \cite{feldmann2003tnf}, and infectious diseases \cite{abraham2020passive}. In 2023, the global best-selling drug was Keytruda, a cancer treatment antibody, with sales reaching $\$25$ billion, surpassing Humira, another antibody used for treating rheumatoid arthritis, which had dominated the market for the past decade \citep{dunleavy2024keytruda}. Traditionally, the discovery of antibodies has predominantly relied on immunizing animals with antigens \cite{van1980okt3} or employing various display techniques such as phage \cite{maccallum1996antibody} and yeast displays \cite{chao2006yeast}. However, these approaches face significant challenges when dealing with structurally intricate proteins, which are difficult to express in a soluble and functional form. Additionally, even when numerous candidate antibodies are generated through these techniques, they may not necessarily bind to the desired domain or exhibit therapeutic efficacy.



To overcome these limitations, deep learning models have been introduced to design synthetic antibodies by learning from natural antibody-antigen complexes \cite{luo2022diffab,jin2022refinegnn,kong2023dymean,kong2023mean,bennett2024atomically}. Despite significant strides in protein design \cite{dauparas2022mpnn,hsu2022esmif1,notin2024protein}, antibodies present a distinct challenge for deep learning due to the high flexibility of their binding regions, known as complementarity-determining regions (CDRs). Inspired by RFdiffusion's \cite{watson2023rfdiffusion} remarkable achievements in monomeric protein and binder design, Bennett et al. \cite{bennett2024atomically} advanced the field by fine-tuning the RFdiffusion model with antibody-antigen complex structural data to facilitate epitope-targeted antibody design. 
Their approach aligns well with established pharmaceutical practices by generating different CDRs on the same framework for different antigen targets, thereby enhancing developability and reducing downstream optimization requirements. 
While structural and functional analyses validated the its capability to generate antibodies that bind to predetermined epitopes, the approach was constrained by notably low success rates.



One reason for the low success rate of this AI-based antibody design pipeline is the occurrence of hallucinations, 
especially during the process of protein inverse folding, which predicts the CDR sequence based on the backbone structure \cite{dauparas2022mpnn,hsu2022esmif1,gruver2023lambo,gao2023pifold}. To be specific, given an antigen epitope and an antibody backbone,
the amino acid sequences inferred through methods like ProteinMPNN \cite{dauparas2022mpnn} and ESM-IF1 \cite{hsu2022esmif1} may not fold into the desired structures in real biological systems. More critically, there are currently no effective computational methods to reduce these hallucinations, aside from conducting time-consuming, labor-intensive, and expensive wet-lab experiments for validation. Typically, using independent structure prediction models to fold and verify the inferred sequences cannot effectively eliminate non-functional sequences caused by hallucinations. That is because even state-of-the-art models exhibit structural deviations of 1 to 3 \AA\ and have low confidence in predicting the structures of antibody CDRs. 

\begin{wrapfigure}{r}{0.5\textwidth}
  \centering
  \hspace{-2mm}
  \includegraphics[width=\linewidth]{section/fig/framework-igseek.pdf}
  \vspace{-3mm}
  \caption{The Framework IgSeek: (a) Pre-train an MEGNN encoder by a self-supervised learning task. (b) Construct a CDR vector database. (c) Sequence generation by K-NN search.}
  \label{fig:framework}
  \vspace{-4mm}
\end{wrapfigure}
To deal with the challenge of hallucinations arising from previous models, we propose an antibody CDR sequence design framework from a novel perspective of similar structure retrieval, named as  {\igseek} (Ig for Immunoglobulin, a.k.a. antibody). Our framework is enlightened by a noteworthy empirical discovery made 25 years ago, which revealed that antibodies exhibit a limited set of canonical structures within 5 out of 6 CDRs despite the vast diversity in sequences, and that certain CDR conformations are scaffolded by a few highly conserved residues \cite{chothia1989conformations}. Further inspired by retrieval-augmented prediction for hallucination reduction in protein structure prediction~\cite{jumper2021alphafold}, and natural language generation~\cite{DBLP:journals/corr/abs-2312-10997}, 
{\igseek} leverages neural retrieval in a database of natural antibodies to retrieve structurally similar sequence templates of CDR, and ensembles the queried templates for sequence prediction. 
Extensive experimental validation demonstrates that our structure-guided retrieval approach effectively improves the accuracy of CDR sequence prediction, notably outperforming state-of-the-art sequence design methods.



