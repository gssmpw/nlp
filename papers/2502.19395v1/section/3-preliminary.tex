\section{Preliminaries and Problem Formulation}
\label{sec:preliminary}

\begin{figure}[h]
  \centering
  \includegraphics[width=0.85\textwidth]{section/fig/antibody-igseek.pdf}
  \caption{Antibody Structure}
  \label{fig:antibody}
\end{figure}

Antibodies are sophisticated Y-shaped protein characterized by their distinctive structural architecture, comprising two identical sets of polypeptide chains. Each set consists of a heavy chain and a light chain, with both chains exhibiting a modular organization of constant and variable regions. While the constant regions maintain high sequence conservation across different antibody molecules, the variable regions display significant diversity, enabling specific antigen recognition and binding capabilities. Within the variable domains, the structural organization follows a precise pattern defined by the IMGT numbering scheme, alternating between four framework regions (FRs) and three complementarity determining regions (CDRs). The FRs provide the structural scaffold, while the CDRs form the antigen-binding interface and are primarily responsible for the specificity and affinity of antigen recognition. These hypervariable CDR loops represent the most critical determinants of antibody-antigen interactions and are consequently the primary focus of rational antibody engineering and design efforts.


We represent the 3D structure of a CDR as a geometric graph $G=(V,E)$ with node set $V$ and edge set $E$ \citep{jing2020gvp,jin2022refinegnn,zhang2023gearnet}. Each node $v_i \in V$ denotes an amino acid residue, associated with a multi-channel 3D coordinate matrix $\vect{X}_i \in \mathbb{R}^{c \times 3}$, where $c$ is the channel size, i.e., the number of atoms in the residue $v_i$. In this paper, we consider the four backbone atoms \{N, C$_\alpha$, C, O\} that are independent to residue type, i.e., $c = 4$. Each edge $e_{ij} \in E$ denotes an interaction between $v_i$ and $v_j$, if the Euclidean distance between their C$_\alpha$ atoms is within a threshold $\theta$. The neighborhood of a node $v_i$, denoted as $\mathcal{N}_{i}$, consists of the adjacency nodes of $v_i$, that is, $\{v_j | (v_i, v_j)\in E\}$. 


{\bf CDR Sequence Design.} Given the structure $G=(E, V)$ of a CDR and the multi-channel 3D coordinate of each residue, in this paper, we aim to reconstruct the corresponding sequence of the CDR, denoted as $\vect{s} = \{s(i) | i \in [1, \cdots, |V| ]\}$, where $s(i)$ is the amino acid type of residue $v_i$.

E(3) Equivalence is an important property in modeling the 3D structures \citep{fuchs2020se3transformer,batzner2022nequip,liao2023equiformer}. Formally, let $\mathcal{X}$ and $\mathcal{Y}$ be two vector spaces, with $T_{\mathcal{X}}(g): \mathcal{X} \rightarrow \mathcal{X}$ and $T_{\mathcal{Y}}(g): \mathcal{Y} \rightarrow \mathcal{Y}$ representing two sets of transformations for the abstract group $g \in E(3)$. A function $\phi: \mathcal{X} \rightarrow \mathcal{Y}$ is E(3) Equivariant to $g$ if it satisfies the following condition: 
\begin{align}
    \phi(\{T_{\mathcal{X}}(g)\vect{x}_i, \vect{h}_i \}_{i=1}^n) = T_{\mathcal{Y}}(g) \phi(\{ \vect{x}_i, \vect{h}_i \}_{i=1}^n),
\end{align}
where $\vect{x}_{i} \in \mathbb{R}^3$ denotes the input 3D coordinates and $\vect{h}_i \in \mathbb{R}^{d}$ is the $d$-dimensional features of a node, respectively. This inductive bias guarantees that $\phi$ preserves equivariant transformation regarding transformation of the coordinate system in E(3) group \citep{satorras2021egnn,huang2022gmn,liao2023equiformer}. A typical example for this transformation operation in the space $\mathcal{X}$ is given by $T_{\mathcal{X}}(g)\vect{x}_i^{(0)} = \vect{R}\vect{x}_i^{(0)} + \vect{b}$, where $\vect{R}\in \mathbb{R}^{3 \times 3}$ is an orthogonal matrix and $\vect{b}$ is the bias term.


To achieve equivalence, equivariant graph neural networks are proposed~\citep{satorras2021egnn, huang2022gmn, kong2023dymean, kong2023mean}, which follows a general message-passing framework as shown in Eq.~\ref{eq:egnn:message}-\ref{eq:egnn:coordinate}.
Here, $\vect{m}_{j \rightarrow i}^{(l)}$ denotes the messages propagated from node $v_j$ to $v_i$, and $d_{ij}^{(l-1)} = dist(v_i, v_j)$ denotes the Euclidean distance between $v_i$ and $v_j$, and $\vect{x}_{ij}^{(l-1)}$ denotes coordinate differences between $v_i$ and $v_j$ at the $(l-1)$-th layer.
\begin{align}
    \vect{m}_{j \rightarrow i}^{(l)} &= \psi_1 \left( \vect{h}_i^{(l-1)}, \vect{h}_j^{(l-1)}, \vect{x}_{ij}^{(l-1)}, d_{ij}^{(l-1)} \right), \label{eq:egnn:message} \\
    \vect{h}_i^{(l)} &= \psi_2 \left(\vect{h}_i^{(l-1)}, \sum_{v_j \in \mathcal{N}_i} \vect{m}_{j \rightarrow i}^{(l)} \right), \label{eq:egnn:feature} \\
    \vect{x}_i^{(l)} &= \psi_3 \left( \vect{x}_i^{(l-1)}, \vect{x}_{ij}^{(l-1)} \sum_{j} \psi_4 \left( \sum_{v_j \in \mathcal{N}_i} \vect{m}_{j \rightarrow i}^{(l)} \right) \right). \label{eq:egnn:coordinate}
\end{align}
The functions $\{\psi_1, \psi_2,\psi_3, \psi_4\}$ are equivariant transformations, typically implemented as Multi-Layer Perceptrons (MLPs) to leverage the universal approximation \citep{funahashi1989mlp,cybenko1989mlp,hornik1991mlp}. In this process, the feature $\vect{h}_i^{(l)}$ remains E(3) invariant, while the coordinate $\vect{x}_i^{(l)}$ is E(3) equivariant.