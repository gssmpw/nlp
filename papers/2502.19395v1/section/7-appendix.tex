\appendix
\section*{Appendix}
\label{sec:appendix}




\section{Related Work}
Our work draws on and contributes to research in mobility aids and the built environment, online image-based survey for urban assessment, personalized routing applications and accessibility maps.

\subsection{Mobility Aids and the Built Environment}
People who use mobility aids (\textit{e.g.,} canes, walkers, mobility scooters, manual wheelchairs and motorized wheelchairs) face significant challenges navigating their communities.
Studies have repeatedly found that sidewalk conditions can significantly impede mobility among these users~\cite{bigonnesse_role_2018,fomiatti_experience_2014,f_bromley_city_2007,rosenberg_outdoor_2013, harris_physical_2015,korotchenko_power_2014}. 
In a review of the physical environment's role in mobility, \citet{bigonnesse_role_2018} summarized factors affecting mobility aid users, including uneven or narrow sidewalks (\textit{e.g.,}~\cite{fomiatti_experience_2014,f_bromley_city_2007}), rough pavements (\textit{e.g.,}~\cite{fomiatti_experience_2014,f_bromley_city_2007}), absent or poorly designed curb ramps (\textit{e.g.,}~\cite{rosenberg_outdoor_2013, f_bromley_city_2007, korotchenko_power_2014}), lack of crosswalks (\textit{e.g.,}~\cite{harris_physical_2015}), and various temporary obstacles (\textit{e.g.,}~\cite{harris_physical_2015}).

Though most research on mobility disability and the built environment has focused on wheelchair users~\cite{bigonnesse_role_2018}, mobility challenges are not experienced uniformly across different user populations~\cite{prescott_factors_2020, bigonnesse_role_2018}. 
For example, crutch users could overcome a specific physical barrier (such as two stairs down to a street), whereas motorized wheelchair users could not (without a ramp)~\cite{bigonnesse_role_2018}. 
Such variability demonstrates how person-environment interaction can differ based on mobility aids and environmental factors~\cite{sakakibara_rasch_2018,smith_review_2016}.
Further, mobility aids such as canes, crutches, or walkers are more commonly used than wheelchairs in the U.S.~\cite{taylor_americans_2014, firestine_travel_2024}: in 2022, approximately 4.7 million adults used a cane, crutches, or a walker, compared to 1.7 million who used a wheelchair~\cite{firestine_travel_2024}.
This underscores the importance of considering a diverse range of mobility aid users in urban accessibility research.
For example, \citet{prescott_factors_2020} explored the daily path areas of users of manual wheelchairs, motorized wheelchairs, scooters, walkers, canes, and crutches and found that the type of mobility device had a strong association with users' daily path area size.
Our study aims to further advance knowledge of how different mobility aid users perceive sidewalk barriers, with a more inclusive understanding of urban accessibility.

\begin{figure*}
    \centering
    \includegraphics[width=1\linewidth]{figures/figure-tutorial.png}
    \caption{Survey Part 2.1 showed all 52 images and asked participants to rate their passability based on their lived experience and use of their mobility aid. Above is the interactive tutorial we showed at the beginning of this part.}
    \Description{This figure shows a screenshot from the online survey. In survey part 2.1, participants were presented with 52 images and were asked to rate their passibility based on their lived experience and use of their mobility aid. The screenshot shows the interactive tutorial shown before this section.}
    \label{fig:survey-part2-instructions}
\end{figure*}

\subsection{Online Image-Based Survey for Urban Assessment}
Sidewalk barriers hinder individuals with mobility impairments not just by preventing particular travel paths but also by reducing confidence in self-navigating and decreasing one's willingness to travel to areas that might be physically challenging or unsafe~\cite{vasudevan_exploration_2016,clarke_mobility_2008}.
Prior work in this area traditionally uses three main study methods: in-person interviews (\textit{e.g}.~\cite{rosenberg_outdoor_2013,castrodale_mobilizing_2018}), GPS-based activity studies (\textit{e.g.,}~\cite{prescott_exploration_2021, prescott_factors_2020,rosenberg_outdoor_2013}), and online-questionnaires (\textit{e.g.,}~\cite{carlson_wheelchair_2002}). 
In-person interviews, while providing detailed and nuanced information, are limited by small sample sizes~\cite{rosenberg_outdoor_2013}. GPS-based activity studies involve tracking mobility aids user activity over a period of time, offering insights into movement patterns and activity space; however, these studies are constrained by geographical location~\cite{prescott_exploration_2021}. In contrast, online questionnaires can reach much larger populations and cover broader geographical regions, but they often yield high-level information that lacks the depth and nuance of the other approaches~\cite{carlson_wheelchair_2002}.
Our study aims to strike a balance between these approaches, capturing nuanced perspectives of mobility aid users about the built environment while maintaining a sufficiently large enough sample size for robust statistical analysis. 
Building on~\citet{bigonnesse_role_2018}'s work, we explore not only the types of factors considered to be barriers, but the \textit{intensity} of these barriers and their differential impacts.

Visual assessment of environmental features has long been employed by researchers across diverse fields, including human well-being~\cite{humpel_environmental_2002}, ecosystem sustainability~\cite{gobster_shared_2007}, and public policy~\cite{dobbie_public_2013}. 
These studies examine the relationship between images and the reactions they provoke in respondents or compare differences in reactions between groups.
Over the past decade, online visual preference surveys have gained popularity (\textit{e.g.,}~\cite{evans-cowley_streetseen_2014, salesses_collaborative_2013, goodspeed_research_2017}), where respondents are asked to make pairwise comparisons between randomly selected images.
Using this approach has two advantages: it adheres to the law of comparative judgment~\cite{thurstone_law_2017} by allowing respondents to make direct comparisons, and it prevents inter-rater inconsistency possible with scale ratings~\cite{goodspeed_research_2017}.
Additionally, online surveys generally offer advantages of increased sample sizes, reduced costs, and greater flexibility~\cite{wherrett_issues_1999}.
For people with disabilities, online surveys can be particularly beneficial. They help reach hidden or difficult-to-access populations~\cite{cook_challenges_2007,wright_researching_2005} and are believed to encourage more honest answers to sensitive questions~\cite{eckhardt_research_2007} by providing a higher level of anonymity and confidentiality~\cite{cook_challenges_2007, wright_researching_2005}.

\begin{figure*}
    \centering
    \includegraphics[width=1\linewidth]{figures/figure-comaprison-screenshot.png}
    \caption{In survey Part 2.2, participants were asked to perform a series of pairwise comparisons based on their 2.1 responses.}
    \Description{This figure shows a screenshot from the online survey. In Survey Part 2.2, participants were asked to perform a series of pairwise comparisons based on their 2.1 responses.}
    \label{fig:survey-part2b-pairwise}
\end{figure*}

\subsection{Personalized Routing Applications and Accessibility Maps}
Navigation challenges faced by mobility aid users can be mitigated through the provision of routes and directions that guide them to destinations safely, accurately, and efficiently~\cite{kasemsuppakorn_understanding_2015}. However, current commercial routing applications (\textit{e.g.}, \textit{Google Maps}) do not provide sufficient guidance for mobility aid users.
To address this gap, significant research has focused on routing systems for this population over the past two decades~\cite{barczyszyn_collaborative_2018, karimanzira_application_2006, matthews_modelling_2003, kasemsuppakorn_understanding_2015, volkel_routecheckr_2008, holone_people_2008, wheeler_personalized_2020, gharebaghi_user-specific_2021, ding_design_2007}.
One early, well-known prototype system is \textit{MAGUS}~\cite{matthews_modelling_2003}, which computes optimal routes for wheelchair users based on shortest distance, minimum barriers, fewest slopes, and limits on road crossings and challenging surfaces.
\textit{U-Access}~\cite{sobek_u-access_2006} provides the shortest route for people with three accessibility levels: unaided mobility, aided mobility (using crutch, cane, or walker), and wheelchair users.
However, U-Access only considers distance and ignores other
important factors for mobility aid users~\cite{barczyszyn_collaborative_2018}.
A series of projects by Kasemsuppakorn \textit{et al}.~\cite{kasemsuppakorn_personalised_2009, kasemsuppakorn_understanding_2015} attempted to create personalized routes for wheelchair users using fuzzy logic and \textit{Analytic Hierarchy Process} (AHP).

While influential, many personalized routing prototypes face limited adoption due to a scarcity of accessibility data for the built environment. 
Geo-crowdsourcing~\cite{karimi_personalized_2014}, a.k.a. volunteered geographic information (VGI)~\cite{goodchild_citizens_2007}, has emerged as an effective solution~\cite{karimi_personalized_2014, wheeler_personalized_2020}.
In this approach, users annotate maps with specific criteria or share personal experiences of locations, typically using web applications based on Google Maps or \textit{OpenStreetMap} (OSM)~\cite{karimi_personalized_2014}.
Examples include \textit{Wheelmap}~\cite{mobasheri_wheelmap_2017}, \textit{CAP4Access}~\cite{cap4access_cap4access_2014}, \textit{AXS Map}~\cite{axs_map_axs_2012}, and \textit{Project Sidewalk}~\cite{saha_project_2019}.
Recent research demonstrated the potential of using crowdsourced geodata for personalized routing~\cite{goldberg_interactive_2016, bolten_accessmap_2019,menkens_easywheel_2011, neis_measuring_2015}.
For example, \textit{EasyWheel}~\cite{menkens_easywheel_2011}, a mobile social navigation system based on OSM, provides wheelchair users with optimized routing, accessibility information for points of interest, and a social community for reporting barriers. 
\textit{AccessMap}~\cite{bolten_accessmap_2019} offers routing information tailored to users of canes, manual wheelchairs, or powered wheelchairs, calculating routes based on OSM data that includes slope, curbs, stairs and landmarks. 
Our work builds on the above by gathering perceptions of sidewalk obstacles from different mobility aid users to create generalizable profiles based on mobility aid type. We envision that these profiles can provide starting points in tools like Google Maps for personalized routing but can be further customized by the end user to specify additional needs (\textit{e.g.}, ability to navigate hills, \textit{etc.})

Beyond routing applications, our study data can contribute to modeling and visualizing higher-level abstractions of accessibility. 
Similar to \textit{AccessScore}~\cite{li_interactively_2018}, data from our survey can provide personalizable and interactive visual analytics of city-wide accessibility. By identifying both differences between mobility groups and common barriers within groups, we can develop analytical tools to prioritize barriers and assess the impact of their mitigation or removal, potentially benefiting the broadest range of mobility group users. Incorporating perceptions of passibility into urban planning processes provides a new dimension for urban planners' toolkits, which are often narrowly focused on compliance with ADA standards.






\section{Preliminary Study}
\label{sec:preliminary-study}


\noindent
\shepherd{ \textbf{Study 1: WPT Depths and Spectrogram Resolution.}
As discussed in Section~\ref{sec:sound-recognition}, the Wavelet Packet Transform (WPT) decomposes signals into finer sub-frequency bands at each level, with spectrogram resolution depending on WPT depth. Greater depth improves classification but increases computational cost. We conduct a preliminary study on WPT depth in environmental sound classification using ESC10~\cite{piczak2015esc} and US8K~\cite{salamon2017us8k}. On an MSP430 microcontroller~\cite{texas2021msp430}, we implemented a simple CNN classifier using WPT spectrograms at varying resolutions, measuring accuracy and energy consumption. Figure~\ref{fig:resolution-accuracy-energy} shows that higher resolution improves accuracy but greatly increases energy consumption, highlighting the need for cost-efficient approaches to balance performance and efficiency. This experiment also implies that to achieve good classification in on-cloud inference, high-resolution spectrogram will need to be transmitted. This results in even larger energy and communication overhead for edge devices, hence motivating keeping the inference pipeline local.}

\noindent
\shepherd{\textbf{Study 2: Effects of Frequency Bands.} WPT also allows us to selectively upsample frequency-domain resolutions on certain frequency bands. We argue that the discriminative information for different sound classes is distributed differently across different frequency bands.} To verify that, in the second preliminary experiment, we classify spectrograms of the same resolution but with either high-frequency bands only or low-frequency bands only. The results, shown in Figure~\ref{fig:high-low-frequency}, indicate that, for sounds of helicopters, waves, and drilling, high-frequency bands are more important for making the correct classification, whereas low-frequency bands are more important for some other classes.

\shepherd{These observations motivate the use of frequency-domain attention to guide the wavelet transform in generating multi-resolution spectrograms, achieving high accuracy while minimizing WPT and classification costs. This insight informs the design of our novel neural architecture, detailed in the following section.}



%\Eric{Exp2: Discrimnative information to distinguish different sound classes is distributed non-uniformly across spectral bands. Expected result: For some class high-freq bands are more important for classification, others low-freq bands. ExpPlan: Find appropriate classes mask out certain bands and do classification.}

%%%%%%%%%%%%%%%%%%%%%%%%%%%%%%%%%%%%%%%%%%%%%
\begin{figure}[tp]
    \centering
    \includegraphics[width=\linewidth]{figures/resolution-accuracy-energy.png}
    \vspace{-0.8cm}
    \caption{\shepherd{Accuracy (left) and energy consumption (right) at various spectrogram resolutions.}}
    \label{fig:resolution-accuracy-energy}
    \vspace{-0.3cm}
\end{figure}
%%%%%%%%%%%%%%%%%%%%%%%%%%%%%%%%%%%%%%%%%%%%%
\begin{figure}[tp]
    \centering
    \includegraphics[width=\linewidth]{figures/high-low-frequency.png}
    \vspace{-0.8cm}
    \caption{\shepherd{Accuracy of using high- and low-frequency band for ESC10 (left) and US8k (right).}}
    \vspace{-0.5cm}
    \label{fig:high-low-frequency}
\end{figure}
%%%%%%%%%%%%%%%%%%%%%%%%%%%%%%%%%%%%%%%%%%%%%
\section{Method}

In Fig. \ref{fig:overview}, we illustrate two major stages of MedForge for collaborative model development, including feature branch development (Sec~\ref{branch}) and model merging (Sec~\ref{forging}). In the feature branch development, individual contributors (i.e., medical centers) could make individual knowledge contributions asynchronously. Our MedForge allows each contributor to develop their own plugin module and distilled data locally without the need to share any private data. In the model merging stage, MedForge enables multi-task knowledge integration by merging the well-prepared plugin module asynchronously. This key integration process is guided by the distilled dataset produced by individual branch contributors, resulting in a generalizable model that performs strongly among multiple tasks.


\subsection{Preliminary}
\label{pre}
In MedForge, the development of a multi-capability model relies on the multi-center and multi-task knowledge introduced by branch plugin modules and the distilled datasets.
The relationship between the main base model and branch plugin modules in our proposed MedForge is conceptually similar to the relationship between the main repository and its branches in collaborative software version control platforms (e.g., GitHub~\cite{github}). 
To facilitate plugin module training on branches and model merging, we use the parameter-efficient finetuning (PEFT) technique~\cite{hu2021lora} for integrating knowledge from individual contributors into the branch plugin modules. 

\subsubsection{Parameter-efficient Finetuning}
Compared to resource-intensive full-parameter finetuning, parameter-efficient finetuning (PEFT) only updates a small fraction of the pretrained model parameters to reduce computational costs and accelerate training on specific tasks. These benefits are particularly crucial in medical scenarios where computational resources are often limited.
As the representative PEFT technique, LoRA (Low-Rank Adaptation)~\cite{hu2021lora} is widely utilized in resource-constrained downstream finetuning scenarios. In our MedForge, each contributor trains a lightweight LoRA on a specific task as the branch plugin module. LoRA decomposes the weight matrices of the target layer into two low-rank matrices to represent the update made to the main model when adapting to downstream tasks. If the target weight matrix is $W_0 \in R^{d \times k}$, during the adaptation, the updated weight matrix can be represented as $W_0+\Delta W=W_0+B A$, where $B \in \mathbb{R}^{d \times r}, A \in \mathbb{R}^{r \times k}$ are the low-rank matrices with rank $r \ll  \min (d, k)$ and $AB$ constitute the LoRA module. 



\subsubsection{Dataset Distillation}
Dataset distillation~\cite{wang2018dataset, yu2023dataset, lei2023comprehensive} is particularly valuable for medicine scenarios that have limited storage capabilities, restricted transmitting bandwidth, and high concerns for data privacy~\cite{li2024dataset}. 
We leverage the power of dataset distillation to synthesize a small-scale distilled dataset from the original data.

The distilled datasets serve as the training set in the subsequent merging stage to allow multi-center knowledge integration. Models trained on this distilled dataset maintain comparable performance to those trained on the original dataset (\ref{tab:main_res}). Moreover, the distinctive visual characteristics among images of the raw dataset are blurred (see \ref{fig:overview}(a)), which alleviates the potential patient information leakage. 

To perform dataset distillation, we define the original dataset as $\mathcal{T}=\{x_i,y_i\}^N_{i=1}$ and the model parameters as $\theta$. The dataset distillation aims to synthesize a distilled dataset ${\mathcal{S}=\{{s_i},\tilde{y_i}\}^M_{i=1}}$ with a much smaller scale (${M \ll N}$), while models trained on $\mathcal{S}$ can show similar performance as models trained on $\mathcal{T}$. 
This process is achieved by narrowing the performance gap between the real dataset $\mathcal{T}$ and the synthesized dataset $\mathcal{S}$. In MedForge, we utilize the distribution matching (DM)~\cite{zhao2023dataset}, which increases data distribution similarity between the synthesized distilled data and the real dataset
The distribution similarity between the real and synthesized dataset is evaluated through the empirical estimate of the Maximum Mean Discrepancy (MMD)~\cite{gretton2012kernel}:
\begin{equation}
\mathbb{E}_{\boldsymbol{\vartheta} \sim P_{\vartheta}}\left\|\frac{1}{|\mathcal{T}|} \sum_{i=1}^{|\mathcal{T}|} \psi_{\boldsymbol{\vartheta}}\left(\boldsymbol{x}_i\right)-\frac{1}{|\mathcal{S}|} \sum_{j=1}^{|\mathcal{S}|} \psi_{\boldsymbol{\vartheta}}\left(\boldsymbol{s}_j\right)\right\|^2
\end{equation}

where $P_\vartheta$ is the distribution of network parameters, $\psi_{\boldsymbol{\vartheta}}$ is a feature extractor. Then the distillation loss $\mathcal{L}_{DM}$ is:
\begin{equation}\scalebox{0.9}{$
\mathcal{L}_{\mathrm{DM}}(\mathcal{T},\mathcal{S},\psi_{\boldsymbol{\vartheta}})=\sum_{c=0}^{C-1}\left\|\frac{1}{\left|\mathcal{T}_c\right|} \sum_{\mathbf{x} \in \mathcal{T}_c} \psi(\mathbf{x})-\frac{1}{\left|\mathcal{S}_c\right|} \sum_{\mathbf{s} \in \mathcal{S}_c} \psi(\mathbf{s})\right\|^2$}
\end{equation}

We also applied the Differentiable Siamese Augmentation (DSA) strategy~\cite{zhao2021dataset} in the training process of distilled data to enhance the quality of the distilled data. DSA could ensure the distilled dataset is representative of the original data by exploiting information in real data with various transformations. The distilled images extract invariant and critical features from these augmented real images to ensure the distilled dataset remains representative.
\begin{figure}[t]
    \centering
    \includegraphics[width=\linewidth]{assets/img/model_arch.png}
    \caption{\textbf{Main model architecture.} We adopt CLIP as the base module and attach LoRA modules to the visual encoder and visual projection as the plugin module. During all the procedures, only the plugin modules are tuned while the rest are frozen. We get the classification result by comparing the cosine similarity of the visual and text embeddings.}
    \label{fig:model_arch}
\end{figure}

\subsection{Feature Branch Development}
\label{branch}
In the feature branch development stage, the branch contributors are responsible for providing the locally trained branch plugin modules and the distilled data to the MedForge platform, as shown in Fig~\ref{fig:overview} (a).
In collaborative software development, contributors work on individual feature branches, push their changes to the main platform, and later merge the changes into the main branch to update the repository with new features. Inspired by such collaborative workflow, branch contributors in MedForge follow similar preparations before the merging stage, enabling the integration of diverse branch knowledge into the main branch while effectively utilizing local resources.

MedForge consistently keeps a base module and a forge item as the main branch. The base module preserves generative knowledge of the foundation model pretrained on natural image datasets (i.e., ImageNet~\cite{deng2009imagenet}), while the forge item contains model merging information that guides the integration of feature branch knowledge (i.e., a merged plugin module or the merging coefficients assigned to plugin modules). 
Similar to individual software developers working in their own branches, each branch contributor (e.g., individual medical centers) trains a task-specific plugin module using their private data to introduce feature branch knowledge into the main branch. These branch plugin modules are then committed and pushed to update the forge items of the main branch in the merging stage, thus enhancing the model's multi-task capabilities.


\begin{figure*}
    \centering
    \includegraphics[width=\textwidth]{assets/img/fusion.png}
    \caption{\textbf{The detailed methodology of the proposed Fusion.} Branch contributors can asynchronously commit and push their branch plugin modules and the distilled datasets. the plugin modules will then be weighted fused to the current main plugin module.}

    \label{fig:merge}
\end{figure*}


Regarding model architecture, MedForge contains a base module and a plugin module (Fig ~\ref{fig:model_arch}). The base module is pretrained on general datasets (e.g., ImageNet) and remains the model parameters frozen in all processes and branches (main and feature branches) to avoid catastrophic forgetting of foundational knowledge acquired from pretraining. Meanwhile, the plugin module is adaptable for knowledge integration and can be flexibly added or removed from the base module, allowing updates without affecting the base model. In our study, we use the pretrained CLIP~\cite{radford2021learning} model as the base module. For the language encoder and projection layer of the CLIP model, all the parameters are frozen, which enables us to directly leverage the language capability of the original CLIP model. For the visual encoder, we apply LoRA on weight matrices of query ($W_q$) and value ($W_v$), following the previous study~\cite{hu2021lora}. To better adapt the model to downstream visual tasks, we apply the LoRA technique to both the visual encoder and the visual projection, and these LoRA modules perform as the plugin module. During the training, only the plugin module (LoRA modules) participates in parameter updates, while the base module (the original CLIP model) remains unchanged. 

In addition to the plugin modules, the feature branch contributors also develop a distilled dataset based on their private local data, which encapsulates essential patterns and features, serving as the foundation for training the merging coefficients in the subsequent merging stage~\ref{forging}. Compared to previous model merging approaches that rely on whole datasets or few-shot sampling, distilled data is lightweight and representative, mitigating the privacy risks associated with sharing raw data. 
We illustrate our distillation procedure in Algorithm~\ref{algorithm:alg1}. In each distillation step, the synthesized data $\mathcal{S}$ will be updated by minimizing $\mathcal{L}_{DM}$.
\begin{algorithmic}[1]
    \STATE \textbf{Input:} A list of clauses $C$
    \STATE \textbf{Output:} List of primary outputs $PO$, primary inputs $PI$, intermediate variables $IV$, and Boolearn expressions $BE$
    \STATE $SC$ = [] \COMMENT{List of sub-clauses}
    \FOR{$i = 1$ to length($C$)}
        % \IF{$C[i] \cap SC = \emptyset$}
        %     \STATE Append \text{Simplify}(\text{FindBooleanExpression}([], $SC$)) to $BE$
        %     %\COMMENT{Simplify Boolean expression}
        %     \FOR{each item $w$ in $SC$}
        %         \IF{$w \notin IV$ and $w \neq v$}
        %             \STATE Append $w$ to $PI$
        %         \ENDIF
        %     \ENDFOR
        %     \STATE $SC$ = []
        % \ELSE
            \STATE Append $C[i]$ to $SC$
            \FOR{each item $v$ in $SC$}
                \IF{$v \notin PI$ and $v \notin IV$}
                    \STATE $f \gets \text{FindBooleanExpression}(v, SC)$ %\COMMENT{Find Boolean expression for $v$}
                    \STATE $g \gets \text{FindBooleanExpression}(\neg v, SC)$ %\COMMENT{Find Boolean expression for $\neg v$}
                    \IF{$f = \neg g$}
                        \STATE Append \text{Simplify}($f$) to $BE$ %\COMMENT{Simplify Boolean expression}
                        \IF{$f = True$ or $f = False$}
                            \STATE Append $v$ to $PO$
                        \ELSE
                            \STATE Append $v$ to $IV$
                        \ENDIF
                        \FOR{each item $w$ in $SC$}
                            \IF{$w \notin IV$ and $w \neq v$}
                                \STATE Append $w$ to $PI$
                            \ENDIF
                        \ENDFOR
                        \STATE SC = []
                        \STATE \textbf{break}
                    \ENDIF
                \ENDIF    
            \ENDFOR
        % \ENDIF
    \ENDFOR
    \STATE \textbf{Return} $PO, PI, IV, BE$
    \vspace{-0.65cm}
\end{algorithmic}



\subsection{MedForge Merging Stage}
\label{forging}
Following the feature branch development stage illustrated in Fig~\ref{fig:overview} (a), branch contributors push and merge their branch plugin modules along with the corresponding distilled dataset into the main branch, as shown in Fig~\ref{fig:overview} (b). Our MedForge allows an incremental capability accumulation from branches to construct a comprehensive medical model that can handle multiple tasks.

In the merging stage, the $i^{th}$ branch contributor is assigned a coefficient $w'_i$ for the contribution of merging, while the coefficient for the current main branch is $w_i$. By adaptively adjusting the value of coefficients, the main branch can balance and coordinate updates from different contributors, ultimately enhancing the overall performance of the model across multiple tasks.
The optimization of the coefficients is done by minimizing the cross-entropy loss for classification based on the distilled datasets. We also add $L1$ regularization to the loss to regulate the weights to avoid outlier coefficient values (e.g., extremely large or small coefficient values)~\cite{huang2023lorahub}. During optimization, following~\cite{huang2023lorahub}, we utilize Shiwa algorithm~\cite{liu2020versatile} to enable model merging under gradient-free conditions, with lower computational and time costs. The optimizer selector~\cite{liu2020versatile} automatically chooses the most suitable optimization method for coefficient optimization. 

In the following sections, we introduce the two merging methods used in our MedForge: Fusion and Mixture. In MedForge-Fusion, the parameters of the branch plugin modules are fused into the main branch after each round of the merging stage. For MedForge-Mixture, the outputs of the branch modules are weighted and summed based on their respective coefficients rather than directly applying the weighted sum to the model parameters. This largely preserves the internal parameter structure of each branch module.

\paragraph{MedForge-Fusion}
In MedForge, forge items are utilized to facilitate the integration of branch knowledge into the main branch.
For MedForge-Fusion, the forge item refers to adaptable main plugin modules. When the $i^{th}$ branch contributor pushes its branch plugin module $\theta'_i=A'_iB'_i$ to the main branch, the current main plugin module $\theta_{i-1}=A_{i-1}B_{i-1}$ will be updated to $\theta_{i}=A_{i}B_{i}$. The parameters of the branch and the current main plugin modules are weighted with coefficients and added to fuse a new version. The $A_i$, $B_i$ are the low-rank matrices composing the LoRA module $\theta_i$. The detailed fusion process can be represented as:
\begin{equation}
\theta_{i}=(w_i A_{i-1}+w'_i A'_i)(w_i B_{i-1}+ w'_i B'_i)
\end{equation}
Where $w_i$ is the coefficient assigned to the current main branch, while $w'_i$ is the coefficient assigned to the branch contributor. After this round of merging, the resulting plugin module $\theta_{i}$ is the updated version of main forging item, thus the main model is able to obtain new capacity introduced by the current branch contributor. When new contributors push their plugin modules and distilled datasets, the main branch can be incrementally updated through merging stages, and the optimization of the coefficients is guided by distilled data.
As shown in Fig.~\ref{fig:merge}, though multiple contributors commit their branch plugin modules and distilled datasets at different times, they can flexibly merge their plugin modules with the current main branch. After each merging round, the plugin module of the main branch will be updated, and thus the version iteration has been achieved.
\begin{figure*}[t]
    \centering
    \includegraphics[width=\textwidth]{assets/img/mixture.png}
    \caption{\textbf{The detailed methodology of the proposed Mixture.} Branch contributors can asynchronously commit and push their branch plugin modules and the distilled datasets. the outputs of different plugin modules will be weighted aggregated. The weights of each merging step will be saved.}

    \label{fig:mixmerge}
\end{figure*}


\paragraph{MedForge-Mixture}
To further improve the model merging performance, inspired by~\cite{zhao2024loraretriever}, we also propose medForge-mixture. For MedForge-Mixture, the forge items refer to the optimized coefficients.
As shown in Fig.~\ref{fig:mixmerge}, for MedForge-Mixture, the coefficient of each branch contributor is acquired and optimized based on distilled datasets. Then the outputs of plugin modules will be weighted combined with these coefficients to get the merged output. 

For each merging round, with branch contributor $i$, the branch coefficient is $w'_i$, the main coefficient is $w_i$, the branch plugin module is $\theta'_i=A'_iB'_i$, and the current main plugin module is $\theta_i=A_iB_i$. With the input $x$, the resulted MedForge-Mixture output can be represented as:
\begin{equation}
y_{i}=w_i A_{i-1} B_{i-1} x+w'_i A'_i B'_i x
\end{equation}

In this way, MedForge encourages additional contributors as the workflow supports continuous incremental knowledge updates.

Overall, both MedForge merging strategies greatly improve the communication efficiency among contributors. We use this design to build a multi-task medical foundation model that enhances the full utilization of resources in the medical community. For the MedForge-Fusion strategy, the main plugin module is updated after each merging round, thus avoiding storing the previous plugin modules and saving space. Meanwhile, the MedForge-Mixture strategy avoids directly updating the parameters of each plugin module, thus preserving their original structure and preventing the introduction of additional noise, which enhances the robustness and stability of the models.



\begin{table}[t]
\small
\caption{Hyperparameters of {\igseek}.}
\vspace{-2mm}
\centering
\begin{tabular}{ccl}
\toprule 
Hyperparameter  & Value & Description \\ 
\midrule
\multicolumn{3}{c}{Input}\\
\midrule
noise\_ratio   & $0.15$             & Ratio of the input coordinates with added Gaussian noise. \\
noise\_scale   & $1$                & The standard deviation $\sigma$ in the Gaussian noise. \\
$\theta$       & $10$ \AA\             & The Euclidean distance threshold when constructing the graph $G$. \\
\midrule
\multicolumn{3}{c}{MEGNN}\\
\midrule
learning\_rate & $5 \times 10^{-3}$ & Learning rate of MEGNN. \\
weight\_decay  & $1 \times 10^{-4}$ & Weight decay factor of the optimizer. \\
hidden\_dim    & $256$              & Size of hidden feature dimension in MEGNN. \\
emb\_dim       & $128$              & Size of output embedding dimension in MEGNN. \\
n\_layer       & $4$                & Number of layers in MEGNN. \\
epoch          & $50$               & Number of the iterations during training\\
batch\_size    & $8$                & Number of batch size in MEGNN. \\
drop\_out      & $0.1$              & Number of dropout rate in MEGNN.\\
\midrule
\multicolumn{3}{c}{Retrieval}\\
\midrule
$k$            & $10$               & Number of nearest neighbor retrieved in the CDR vector database. \\
n\_sample      & $2$                & Number of generated samples for each query.\\
\bottomrule
\end{tabular}
\label{tab:appendix-param}
\vspace{-3mm}
\end{table}

\section{Datasets and Labels}
\label{sec:appendix-data}
{\bf Datasets.} 
We selected all experimentally solved antibody structures released in the SAbDab antibody database \citep{dunbar2013sabdab,schneider2021sabdab} before January 1, 2024, to sample our training set. Notice that we remove CDR sequences that are identical to those in the dataset to eliminate redundancy in the dataset.
Following FoldSeek \citep{van2024foldseek}, for each CDR in the SAbDab-before-2024 dataset, we randomly sample equal-length CDRs with TM-score large than $0.6$ to generate training pairs. The final training set consisted of $45,043$ antibody CDR pairs. 
After finishing model training, all $24,479$ unique CDR structures in the SAbDab-before-2024 dataset are utilized to construct the CDR vector database. 
The test set of SAbDab-2024 include experimentally solved antibody released in SAbDab antibody database between January 3, 2024 and May 29, 2024. 
This process resulted in $4,449$ test CDR samples that are completely unseen during the model training process.

The sequence similarity distribution between the training set and test set is illustrated in Figure \ref{fig:seq_sim}. As we can observe, the average sequence similarity for each CDR region in the training and test set is around 0.3 to 0.5, which shows that there is no potential data leakage issue in this data split strategy.
In addition, we utilize a T-cell receptor dataset released in the structural T-cell receptor database \citep{leem2018stcrdab} to construct a test set with $5,111$ receptors, referred to as STCRDab.
To evaluate the model efficiency, we utilize $5,000$ predicted CDR-H3 loops from the Observed Antibody Space (OAS) \citep{olsen2022oas}, denoted as OAS-H3. Redundant CDR loops are removed from the test set. Statistics of these datasets are listed in Table \ref{tab:datasets}.

{\bf Labels.}
PyIgClassify cluster labels \citep{north2011fccc,adolf2015fccc} are employed as ground-truth labels to assess the retrieval performance of antibody CDR regions. For each PDB structure containing an identified antibody heavy or light chain, PyIgClassify categorizes the conformations of CDRs using a three-tier strategy: chain and position, length, and the similarity of dihedral angles. For instance, the cluster ID L1-10-1 denotes a CDR-L1 with a length of 10 amino acids, where the subcluster 1 is determined based on the similarity of dihedral angles using the affinity propagation clustering method \citep{frey2007clustering}.


\begin{table}[t]
\small
\caption{Profile of Datasets}
\vspace{-2mm}
\centering
\label{tab:datasets}
\begin{tabular}{crrrrrr}
\toprule 
{SAbDab}  & \#CDR-H1& \#CDR-H2& \#CDR-H3& \#CDR-L1& \#CDR-L2& \#CDR-L3\\ 
\midrule
 \# Data (before-2024) & 4,464     & 4,466    & 4,463    & 3,693        & 3,696  & 3,897  \\ 
\# Query (2024) & 809     & 823    & 513    & 580        & 607  & 578  \\ 
\midrule
\midrule
{STCRDab}  & \#CDR-A1& \#CDR-A2& \#CDR-A3& \#CDR-B1& \#CDR-B2& \#CDR-B3\\ 
\midrule
 \# Data  & 680  & 680    & 680 & 741        & 741       & 741         \\
\# Query    & 138    & 140   & 120 & 158        & 154       & 138      \\ 
\bottomrule
\end{tabular}
\vspace{-4mm}
\end{table}


\section{Implementation Details}
\label{sec:appendix-param}
In this section, we introduce the implementation details of our {\igseek}.
The MEGNN model introduced in Section \ref{sec:method} consists of three key learnable functions:
\begin{itemize}[topsep=0.5mm, partopsep=0pt, itemsep=0pt, leftmargin=10pt]
    \item The edge module $\phi_e$ (refer to Eq. \ref{eq:megnn:edge}) consists of a two-layer MLP with two Leaky Rectified Linear Unit (LeakyReLU) activation functions \citep{xu2015leakyrelu}. Besides, a dropout function \citep{srivastava2014dropout} with $0.1$ dropout rate is employed on the output of $\phi_e$: 
    $$
    \begin{aligned}
        &\text{CONCAT(Features)} \rightarrow \text{Input} \rightarrow \{ \text{LinearLayer()} \rightarrow \text{LeakyReLU()} \rightarrow \text{LinearLayer()} \\
        &\rightarrow \text{LeakyReLU()} \} \rightarrow 
        \text{Dropout} \rightarrow \text{Output}.
    \end{aligned}
    $$
    \item The coordinate module $\phi_X$ (refer to Eq. \ref{eq:megnn:coordinate}) contains a two-layer MLP that shares weights with the MLP in the edge module $\phi_e$. 
    \item The node module $\phi_h$ (refer to Eq. \ref{eq:megnn:feature}) is a two-layer MLP with one LeakyReLU activation function:
    $$    
        \text{CONCAT(Features)} \rightarrow \text{Input} \rightarrow \{ \text{LinearLayer()} \rightarrow \text{LeakyReLU()} \rightarrow \text{LinearLayer()} \} \rightarrow \text{Output}.
    $$
\end{itemize}

In our experiments, we train the MEGNN model in {\igseek} using PyTorch \citep{paszke2019pytorch} with an Adam optimizer \citep{kingma2015adam} on 4 NVIDIA Tesla A100 GPUs. Table \ref{tab:appendix-param} lists the hyperparameters of {\igseek}.



\begin{figure}[h]
  \centering
  \includegraphics[width=0.55\textwidth]{section/fig/seq_iden.pdf}
  \vspace{-4mm}
  \caption{Sequence similarity between SAbDab train/test set.}
  \label{fig:seq_sim}
  \vspace{-4mm}
\end{figure}

\section{Baselines}
\label{sec:appendix-baseline}

The first category is structure retrieval model:
\begin{itemize}[topsep=0.5mm, partopsep=0pt, itemsep=0pt, leftmargin=10pt]
    \item {\bf FoldSeek} \citep{van2024foldseek} represents tertiary amino acid interactions using 3D interaction (3Di) structural alphabet, achieving 4 to 5 orders of magnitude speed-up compared to traditional iterative or stochastic structure retrieval methods like CE \citep{shindyalov1998ce}, Dali \citep{holm2020dali}, and TM-align \citep{zhang2005tmalign}. Official code is available at: \href{https://github.com/steineggerlab/foldseek}{https://github.com/steineggerlab/foldseek}.
\end{itemize}

\begin{figure}[t]
  \centering
  \includegraphics[width=0.95\textwidth]{section/fig/bar_aar_antibody_3aa.pdf}
  \vspace{-2mm}
  \caption{The Comparison of Average AAR on the SAbDab-2024 Dataset using CDRs with extensions of 1 to 3 amino acids on each side in the flanking regions.}
  \label{fig:exp-extension}
  %\vspace{-5mm}
\end{figure}


The second category is protein and antibody design models:
\begin{itemize}[topsep=0.5mm, partopsep=0pt, itemsep=0pt, leftmargin=10pt]
    \item {\bf ProteinMPNN}  \citep{dauparas2022mpnn} 
   is a deep learning–based method for protein sequence design that excels in both in silico and experimental evaluations, achieving a sequence recovery of 52.4\% on native protein backbones, compared to 32.9\% for Rosetta \citep{adolf2018rosetta,baek2021rosetta}. By leveraging a message-passing neural network with enhanced input features and edge updates, ProteinMPNN is capable of designing monomers, cyclic oligomers, protein nanoparticles, and protein-protein interfaces, rescuing previously failed designs generated by Rosetta \citep{baek2021rosetta} or AlphaFold \citep{jumper2021alphafold}. 
    Official code is available at: \href{https://github.com/dauparas/ProteinMPNN}{https://github.com/dauparas/ProteinMPNN}.
    \item {\bf ESM-IF1} \citep{hsu2022esmif1} 
    employs a sequence-to-sequence Transformer to predict protein sequences from backbone atom coordinates, which is pre-trained on structures of 12M protein sequences. It achieves 51\% native sequence recovery and 72\% for buried residues.
    Official code is available at: \href{https://github.com/facebookresearch/esm/tree/main/examples/inverse\_folding}{https://github.com/facebookresearch/esm/tree/main/examples/inverse\_folding}.
    \item {\bf AbMPNN} \citep{dreyer2023abmpnn} fine-tunes ProteinMPNN on the SAbDab \citep{dunbar2013sabdab,schneider2021sabdab} dataset for antibody design, outperforming generic protein models in sequence recovery and structure robustness, especially for the hypervariable CDR-H3 loop. The profile of model weights is available at: \href{https://zenodo.org/records/8164693}{https://zenodo.org/records/8164693}.
    \item {\bf AntiFold} \citep{hoie2024antifold} is an antibody-specific inverse folding model fine-tuned from ESM-IF1 \citep{hsu2022esmif1} on solved antibody structures from the SAbDab dataset \citep{dunbar2013sabdab,schneider2021sabdab} and predicted antibody structures from the OAS dataset \citep{kovaltsuk2018oas, olsen2022oas}.
    AntiFold excels in sequence recovery and structural similarity while also demonstrates stronger correlations in predicting antibody-antigen binding affinity in a zero-shot manner.
    Official code is available at: \href{https://github.com/oxpig/AntiFold}{https://github.com/oxpig/AntiFold}.
\end{itemize}


\section{Additional Experiments}
\label{sec:appeidix-exp}
{\bf CDR with extensions.}
In this set of experiments, we compare {\igseek} with protein and antibody design baselines using the SAbDab-2024 dataset. We focus on CDRs with backbone extensions of $n$ amino acids on each side in the flanking regions. Fig. \ref{fig:exp-extension} illustrates the results for varying values of $n = 0, 1, 2, 3$. As we can observe, the performance of {\igseek} improves with the inclusion of additional amino acids in the given structure, , which aligns with the fact that more input structural information can be encoded into the CDR representation. In contrast, other baseline models are adversely affected by hallucinations stemming from conserved backbone structures. Notably, when $n=3$, {\igseek} consistently outperforms its competitors by at least $5\%$ and $18\%$ for heavy chain and light chain CDR loops, respectively. This further demonstrates that the retrieval-based strategy employed by {\igseek} effectively mitigates hallucinations during CDR sequence generation.

\begin{table}[t]
\small
\caption{The Comparison of Average AAR with varying $K$ in SAbDab-2024.}
\vspace{-2mm}
\centering
\label{tab:param-k}
\begin{tabular}{cccccc}
\toprule 
$K$  & 5 & 10 &	20 & 50 & 100 \\
\midrule
CDR-L1 & 0.660 & 0.658 & 0.645 & 0.620 & 0.593 \\
\midrule
CDR-L2 & 0.580 & 0.580 & 0.573 & 0.573 & 0.550 \\
\midrule
CDR-L3 & 0.586 & 0.586 & 0.576 & 0.574 & 0.564 \\
\midrule
CDR-H1 & 0.560 & 0.561 & 0.560 & 0.553 & 0.537 \\
\midrule
CDR-H2 & 0.440 & 0.435 & 0.432 & 0.429 & 0.430 \\
\midrule
CDR-H3 & 0.473 & 0.464 & 0.455 & 0.447 & 0.441 \\
\bottomrule
\end{tabular}
\vspace{-4mm}
\end{table}


{\bf Influence of value $K$.}
In this set of experiments, we conduct experiments on the SAbDab-2024 dataset to evaluate the impact of varying parameter $K$ in {\igseek}. Table \ref{tab:param-k} reports the average AAR of {\igseek} across different values of $K$ on the SAbDab-2024 dataset. As we can observe, the performance of IgSeek exhibits a decline as $K$ increases. In our implementation, we set $K=10$ rather than $5$ as {\igseek} achieves comparable results while preserving enhanced sequence diversity.



