%!TEX root = paper.tex
\section{Conclusion}

We created \mt{} after we got tired of answering ``no'' to application developers who asked ``Does userspace stack X work on the cloud?''
Initially, we believed this would be a simple matter of porting an existing stack to handle all the quirks of cloud vNICs.
However, it required a fundamentally new way of thinking about vNIC features as absent-by-default instead of present-by-default as in stacks designed for bare-metal NICs.
This paper attempts to crystallize our learnings---formed after numerous mistakes where we depended on a NIC feature that was unavailable in some vNIC on some cloud (e.g., RSS key inspection, application memory DMA registration)---into the Least Common Denominator NIC model.
Another crucial lesson was that the libOS model for userspace stacks is incompatible with how developers of non-infrastructure applications wish to run applications, i.e., with multiple processes and many threads, in various programming languages, and without touching DPDK.
We hope the research community will build upon \mt{}, with future projects like large-scale evaluations and new applications, to finally bring userspace networking benefits to the cloud.