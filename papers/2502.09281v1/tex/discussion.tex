\section{Discussion}
\label{sec:discussion}

\noindent \textbf{What happens if the LCD model changes?}
We derive the LCD model after analyzing the capabilities of the network virtualization layer exposed in the three major cloud providers.
%many NICs deployed in cloud data center spanning more than 10 years.
It should be noted that large cloud providers typically spend millions of dollars to equip their fleet with a particular version of NICs~\cite{nic-purchase}. 

Table~\ref{table:unsupported_features} shows that the LCD model has seen little progress in 15 years, with features such as flow steering, available since 2009, still not accessible to cloud customers. Although the LCD model may improve as cloud providers upgrade NICs, we prioritized fast networking today, making RSS-- essential.

% Even if newer NIC generations would potentially support more features, it is unlikely that all NICs in data centers located in multiple regions will be replaced at once.

Nonetheless, \mt{} should adapt to the vNIC features available in its environment. \mt{} partial supports selective feature selection, using hardware checksum offloads (AWS and Azure) and software checksums in GCP where offloads are unavailable. Adding support for more features requires significant engineering, which we are addressing with the open-source community.

%It is not feasible to replace all of them at a time.%with new NICs that support LCD features. 

% Indeed, as also shown in Table~\ref{table:unsupported_features}, an old feature like flow steering that, to the best of our knowledge, appeared in 2009, after almost 15 years, is not yet exposed.

\vspace{0.5em}
\label{disc:fops}
\noindent \textbf{\mt{} and existing kernel bypass operating systems.}
A large body of research had focused on building full operating systems to support microsecond-scale applications in data centers such as Junction~\cite{junction:nsdi24}, ZygOS~\cite{zygos}, IX~\cite{belay2014ix}, Caladan~\cite{caladan}, Shenango~\cite{shenango}, and Shinjuku~\cite{shinjuku}. These systems are full-stack solutions, while \mt{}, being a networking stack, is an orthogonal effort. Indeed, \mt{} can be incorporated into these systems to provide more efficient transport and higher compatibility in the public cloud.

\vspace{0.5em}
\label{disc:custom_protocol}
\noindent \textbf{Does custom protocol hamper wider adoption?}
\mt{} targets cloud-based, latency-critical RPC workloads, using RDMA-like message-oriented transport suited for request-response tasks, avoiding TCP-like abstractions and overhead. Future plans include adding TCP for external service connectivity. Experiments used custom transport, with RSS-- orthogonal to transport protocols and implementable on stock TCP.