%!TEX root = paper.tex
\begin{abstract}
After a decade of research in userspace network stacks, why do new solutions remain inaccessible to most developers?
We argue that this is because they ignored (1) the hardware constraints of public cloud NICs (vNICs) and (2) the flexibility required by applications.
Concerning the former, state-of-the-art proposals rely on specific NIC features (e.g., flow steering, deep buffers) that are not broadly available in vNICs.
As for the latter, most of these stacks enforce a restrictive execution model that does not align well with cloud application requirements.

We propose a new userspace network stack, \mt{}, built for public cloud VMs.
Central to \mt{} is a new ``Least Common Denominator'' model, a conceptual NIC with a minimal feature set supported by all kernel-bypass vNICs.
The challenge is to build a new solution with performance comparable to existing stacks while relying only on basic features (e.g., no flow steering, no RSS reconfiguration).
\mt{} uses a microkernel design to provide higher flexibility in application execution compared to a library OS design; we show that microkernels' inter-process communication overhead is negligible on large cloud networks.
%Central to \mt{} is a new ``Least Common Denominator'' NIC model, a conceptual NIC with the minimal feature set supported by all kernel-bypass Ethernet NICs.
%We create a new technique called \rssminus{}~(minus minus) that provides flow-steering-like functionality in LCD NICs, which support only a restricted version of receive-side scaling.
%\mt{} uses a microkernel design since it provides higher flexibility in application execution compared to a library OS design; we show that on large cloud networks, the inter-process communication overhead of microkernels is negligible.
Our experiments show that \mt{} works on today's three largest public clouds.
We also demonstrate the latency and throughput benefits of \mt{} for two real-world applications: a key-value store and state-machine replication.
For the key-value store application, \mt{} achieves 80\% lower latency and 75\% lower CPU utilization compared to the best-existing cloud solution.

% Linux TCP/IP.
% \pg{The last line might not come across as very impactful. Can we instead say compared to the best existing solution in cloud?}

\end{abstract}
