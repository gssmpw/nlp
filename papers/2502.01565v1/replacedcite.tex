\section{Related Work}
\noindent
\textbf{Oriented Object Detectors:} OOD architectures are analogous to traditional HBB object detectors, and the main difference is the regression of an additional parameter related to the orientation. They can rely on anchors and predict offsets and shape adjustment factors on top of predefined bounding boxes____, or be anchor-free and directly regress the BB parameters____. They can also be categorized into single-stage methods, where the BBs are produced directly from the network____ sometimes with an additional refinement step____, or two-stage, in which proposals are initially created and then refined in a second stage____. Within these basic categories of OOD approaches, several improvements have been proposed to specific modules for enhancing the performance____ by incorporating dedicated modules that are specific for OOD, such as rotation equivariant (RE) backbones. Still, they rely on detectors based on OBB regression heads, which leads to parametrization ambiguities that might generate boundary discontinuity or encoder ambiguity problems, as noted in____. 

\noindent
\textbf{Oriented Object Representation and Regression Loss functions:}
A critical component of an OOD architecture is an appropriate OBB representation and its regression (or localization) loss. The earliest and easiest approach is to represent an OBB by parameters $(x,y,w,h, \theta)$ and regress them using a per-parameter $\ell_1$ loss____, where $(x,y)$ represent the center, $(w,h)$ the dimensions, and $\theta$ the orientation. However, the angular component can produce large $\ell_1$ loss values for similar OBBs due to the discontinuity in the OBB parametrization (recall Figure~\ref{fig:disc:parametrizations}). As mitigation strategies, joint optimization using  IoU-based loss functions have been proposed, such as rotated-IoU (rIoU)____, Pixels IoU (PIoU)____ or convex-IoU____. However, they might face differentiability or implementation issues____. Another set of methods converts OBBs to 2D Gaussian distributions, and explores distribution-based regression loss functions, such as Gauss Wasserstein Distance (GWD)____, Kullback-Leibler Divergence (KLD)____, Bhattacharyya Distance (BD)____, or Probabilistic Intersection-over-Union (ProbIoU) loss____. Gaussian-based methods involve simple-to-compute and differentiable regression loss functions, but suffer from the \textit{decoding ambiguity} for square-like objects, for which the angular information cannot be retrieved. Furthermore, they can still suffer from angular discontinuity, as recently mentioned in works that explicitly handle the boundary discontinuity problem, such as____. On the other hand, they provide a natural solution for the \textit{encoding ambiguity} problem for circular objects, unlike loss functions based directly on OBBs. Recent solutions that focus on the boundary discontinuity problem____ have shown promising results, but are still affected by the encoding ambiguity problem for circular objects since the explore OBBs and enforce angular consistency.

This paper presents a novel regression head for oriented object detection that directly produces the parameters of a 2D Gaussian distribution (i.e., the mean vector and covariance matrix) called GauCho, which can be coupled to any Gaussian-based loss function. To avoid a constrained optimization imposed by the structure of covariance matrices (they need to be positive-definite), we rely on the Cholesky decomposition. As we show in this paper, there is a continuous one-to-one mapping between the GauCho head and the parameters of a Gaussian, which naturally mitigates the boundary discontinuity problem. We also propose to use Oriented Ellipses (OEs) instead of OBBs as the final output of a GauCho-based detector. As shown in Section~\ref{sec:discussion}, OEs are suitable representations for oriented objects typically present in aerial imagery applications and are a natural choice when Gaussian-based loss functions are used. Although GauCho still suffers from the \textit{decoding ambiguity} for square-like objects, it fully solves the \textit{encoding ambiguity} for circular objects.