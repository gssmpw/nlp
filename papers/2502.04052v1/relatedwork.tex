\section{Related Work}
Classical DT learning algorithms, such as C4.5 \citep{quinlan2014c4} or CART \citep{breiman2017classification}, are based on growing a DT by greedily splitting the input space
in a componentwise fashion to optimize the reduction in the chosen error metric at each step of building the tree.
No method has been yet proposed to incorporate updates of an internal memory state based on these algorithms.

Nevertheless, the idea of using explicit time-dependency in the DT framework is not new.
\citep{chen2016learning} propose a model which they call recurrent DT, for camera planning. In contrast to ReMeDe, no internal memory state is used but previous outputs are fed back into the model as inputs, which renders this approach a special case of NARX models in the terminology used here. 
The same holds for \citep{chegini2010prediction}, who extend the LoLiMoT algorithm \citep{nelles1996basis} to include output feedback for financial time series prediction.
\citep{alaniz2021learningdecisiontreesrecurrently} propose an intricate scheme to learn a recurrent model, involving a DT, but also a combination between an LSTM and an Attribute-Learning System, where a DT uses the hidden state of an LSTM. 

Others have taken the converse route and combine classical DT with recurrent models in leaf nodes, such as \citep{ren2021tree}. Therein, first the input data is split using classical DT algorithms and then separate RNNs are trained for each leaf node, inheriting the potential suboptimality of the former.
Also worth mentioning is a family of approaches that uses hierarchical, tree structured switching linear systems for dynamics modeling, such as \citep{nassar2018tree}, which share some structural similarities with ReMeDe Trees, although the resulting models are quite different. In particular, the hidden state used there is discrete and some of the involved operations are soft, i.e. stochastic.
In contrast, a ReMeDe Tree consists only of a single hard, axis-aligned DT which performs read and write operations on its own hidden memory state, enabled by training the complete model via gradient descent. To the best of our knowledge, no other recurrent DT using continuous hidden state feedback has been proposed yet.