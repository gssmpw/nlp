\section{Background}
\subsection{Context-Free Grammars}

Context-free grammar (CFG) is a type of formal grammar where the productions rules govern how to generate text from non-terminals and terminals.
%
A context-free grammar is defined by $G=(V,\Sigma, R, S)$ where $V$ and $\Sigma$ denotes nonterminal and terminal respectively. $R$ is a finite relation in $V \times (V \cup \Sigma)^{*}$ which specifies the production rules of the grammar. $S \in V$ is the start symbol. 
A production rule in $R$ has the form 
\begin{equation}
    \alpha \to \beta
\end{equation}
where $\alpha \in V$, $\beta \in (V \cup \Sigma)^{*}$.
It is conventional to list all rules with the same left-hand side on the same line and separate the right-hand side with ``$|$'' like $\alpha \to \beta_1 \,|\, \beta_2$.