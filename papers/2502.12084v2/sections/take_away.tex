\section{Takeaways}

Based on our findings, we highlight three key areas for future improvements:

\begin{itemize} \item \textbf{Strengthening Fundamental Visual Capabilities.} Improving core visual abilities not only enhances overall performance but also increases adaptability. A stronger visual foundation maximizes the effectiveness of visual prompting and reduces reliance on prior knowledge, enabling models to operate more independently in vision-centric tasks.

\item \textbf{Balancing Language-Based Reasoning in Vision-Centric Tasks.} Integrating language into vision-centric tasks requires careful calibration. Future research should establish clearer principles on when language-based reasoning aids visual understanding and when it introduces unnecessary biases, ensuring models leverage language appropriately.

\item \textbf{Evolving Vision-Text Training Paradigms.} Current training paradigms focus heavily on emphasizing vision-language associations. However, as models expand their visual context window, their ability to reason purely within the visual domain becomes increasingly crucial. We should prioritize developing models that can structure, organize, and infer relationships among visual cues.
\end{itemize}