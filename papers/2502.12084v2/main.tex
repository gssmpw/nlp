% This must be in the first 5 lines to tell arXiv to use pdfLaTeX, which is strongly recommended.
\pdfoutput=1
% In particular, the hyperref package requires pdfLaTeX in order to break URLs across lines.

\documentclass[11pt]{article}

% Change "review" to "final" to generate the final (sometimes called camera-ready) version.
% Change to "preprint" to generate a non-anonymous version with page numbers.
%\usepackage[review]{acl}
\usepackage[preprint]{acl}
% \usepackage[final]{acl}

\usepackage[skins,most]{tcolorbox} 
\tcbuselibrary{breakable} 
\newtcolorbox[auto counter, number within=section, list type=subsubsection, list inside=toc]{sectionbox}[2][]{
colback=white!98!gray, colframe=black, 
colbacktitle=white!90!gray, coltitle=black, 
fonttitle=\bfseries,
title={#2}, 
list entry={Comment \thetcbcounter\quad}
}

% Standard package includes
\usepackage{booktabs}
\usepackage{subcaption}
\usepackage{times}
\usepackage{latexsym}
\usepackage{amsmath}
\usepackage{algorithm}
\usepackage{algpseudocode}
\usepackage{xcolor}
\usepackage{color}
\usepackage{listings}
\definecolor{customTeal}{RGB}{0, 128, 128}
\definecolor{emphasisColor}{RGB}{255, 0, 0} % Red color for emphasis
\usepackage{tabularx}
\usepackage{ragged2e}
\newcolumntype{Y}{>{\RaggedRight\arraybackslash}X} % 自适应的左对齐列
% \usepackage[most]{tcolorbox}
\usepackage[skins,most]{tcolorbox}
\usepackage{colortbl}
\usepackage{multirow} 
\usepackage{booktabs}
\usepackage{wrapfig}



\usepackage{pifont}  % 引入 pifont 以支持 \ding 命令
% 自定义颜色
\definecolor{customgreen}{HTML}{16C47F}  % 使用指定的 16C47F 绿色
\definecolor{customred}{HTML}{C62300}   % 酒红色(不变)

% 定义带颜色的✔和✘
\newcommand{\cmark}{\textcolor{customgreen}{\ding{51}}}  % ✔
\newcommand{\xmark}{\textcolor{customred}{\ding{55}}}   % ✘

\usepackage{float}


\usepackage{caption}

\lstset{
    language=Python,         
    basicstyle=\fontsize{7.0pt}{7.5pt}\ttfamily\selectfont,
    keywordstyle=\color{customTeal},    
    stringstyle=\color{customTeal},    
    commentstyle=\color{customTeal},     
    morecomment=[l][\color{green}]{\#},
    breaklines=true,                
    showstringspaces=false,
    escapeinside={(*@}{@*)}, % 
    numbers=left,          
    stepnumber=1,           
    numberstyle=\tiny\color{gray}, 
    numbersep=5pt,         
    xleftmargin=1.5em,      
    frame=none,              
}

% For proper rendering and hyphenation of words containing Latin characters (including in bib files)
\usepackage[T1]{fontenc}
% For Vietnamese characters
% \usepackage[T5]{fontenc}
% See https://www.latex-project.org/help/documentation/encguide.pdf for other character sets

% This assumes your files are encoded as UTF8
\usepackage[utf8]{inputenc}

% This is not strictly necessary, and may be commented out,
% but it will improve the layout of the manuscript,
% and will typically save some space.
\usepackage{microtype}

% This is also not strictly necessary, and may be commented out.
% However, it will improve the aesthetics of text in
% the typewriter font.
\usepackage{inconsolata}

\usepackage{enumitem} 
\usepackage{lipsum}
%Including images in your LaTeX document requires adding
%additional package(s)
\usepackage{graphicx}

\NewDocumentCommand{\yi}
{ mO{} }{\textcolor{blue}{\textsuperscript{\textit{May}}\textsf{\textbf{\small[#1]}}}}

\newcommand{\name}{}
\newcommand{\briefname}{\texttt{PC-Bench} }

% If the title and author information does not fit in the area allocated, uncomment the following
%
%\setlength\titlebox{<dim>}
%
% and set <dim> to something 5cm or larger.


% \title{No Need to Know Where Cars Were Made to Tell Them Apart: \\How Well Do VLMs Visually Link Matching Cues? \yi{Alternative Candidates: teaser phrase - ("Caught the Connection?" or "Is that the Same Car or a Different One?"); main part - ("A Closer Look at How Well VLMs Implicitly Link Explicitly Matching Visual Cues")} \yi{Current teaser is sort of too low and not on-point/comprehensive/precise enough}} 


% \title{Caught the Connection? A Closer Look at How Well VLMs Implicitly Link Explicit Matching Visual Cues}
\title{\raisebox{-0.15cm}{\includegraphics[width=0.8cm]{img/vlm2-bench-icon_final.png}}VLM$^2$-Bench: A Closer Look at How Well VLMs Implicitly Link \\ Explicit Matching Visual Cues}


% Author information can be set in various styles:
% For several authors from the same institution:

\DeclareSymbolFont{extraup}{U}{zavm}{m}{n}
\DeclareMathSymbol{\vardiamond}{\mathalpha}{extraup}{87}
\author{\bf Jianshu Zhang\textsuperscript{$^{\heartsuit}$\thanks{These authors contribute to this work equally.}}, ~ Dongyu Yao$^{\spadesuit*}$,  ~ Renjie Pi$^{\heartsuit}$, ~ Paul Pu Liang$^{\vardiamond}$, ~ Yi R. (May) Fung$^{\heartsuit}$\\
$^{\heartsuit}$HKUST ~~~~~~$^{\spadesuit}$CMU
~~~~~~$^{\vardiamond}$MIT\\
\texttt{jianshu.zhang777@gmail.com}
~~~\texttt{raindy@cmu.edu} ~~~\texttt{rpi@ust.hk} \\
\texttt{ppliang@mit.edu} ~~~\texttt{yrfung@ust.hk}
}


\begin{document}
\maketitle



\begin{abstract}

% Recent works to jointly reconstruct 3D human and object from a single RGB image, are mostly model-based, that fail to capture the fine details of the clothed human body and object surface. In this paper, we introduce ReCHOR, a novel, model-free, first-method to produce realistic clothed human-object reconstructions from a monocular view. This is extremely challenging due to human-object occlusions, diverse interactions and depth ambiguity, as it needs to infer both 3D spatial awareness and high resolution details. Our core idea is based on estimating neural implicit representations for human and object respectively by an attention-based neural implicit model that attends to pixel-aligned features from both the global human-object image for spatial awareness and  the local separate view of human and object images for high quality details. Additionally, the network is conditioned on semantic features from an initial estimated human-object pose prior and a generative diffusion model that inpaints occluded regions, thus enabling the retrieval of details from them.
% We also propose a synthetic dataset with rendered scenes of diverse, inter-occluded 3D human and object scans, to train our network. We evaluate our method on the synthetic and real world BEHAVE dataset. Our experiments show that our method outperforms the SOTA in achieving realistic clothed human-object reconstructions.
Recent approaches to jointly reconstruct 3D humans and objects from a single RGB image represent 3D shapes with template-based or coarse models, which fail to capture details of loose clothing on human bodies. In this paper, we introduce a novel implicit approach for jointly reconstructing realistic 3D clothed humans and objects from a monocular view. For the first time, we model both the human and the object with an implicit representation, allowing to capture more realistic details such as clothing. This task is extremely challenging due to human-object occlusions and the lack of 3D information in 2D images, often leading to poor detail reconstruction and depth ambiguity. To address these problems, we propose a novel attention-based neural implicit model that leverages image pixel alignment from both the input human-object image for a global understanding of the human-object scene and from local separate views of the human and object images to improve realism with, for example, clothing details. Additionally, the network is conditioned on semantic features derived from an estimated human-object pose prior, which provides 3D spatial information about the shared space of humans and objects. To handle human occlusion caused by objects, we use a generative diffusion model that inpaints the occluded regions, recovering otherwise lost details. For training and evaluation, we introduce a synthetic dataset featuring rendered scenes of inter-occluded 3D human scans and diverse objects. Extensive evaluation on both synthetic and real-world datasets demonstrates the superior quality of the proposed human-object reconstructions over competitive methods.
\end{abstract}
\section{Introduction}
\label{sec:intro}
% Image editing methods in diffusion models depend on user-defined control directions - users can unlock their creativity using these methods by specifying the desired manipulation through prompts~\cite{gandikota2023concept}, reference images~\cite{ruiz2022dreambooth, kumari2022customdiffusion, gal2022image, chen2024trainingfreeregionalpromptingdiffusion}, or attribute vectors~\cite{parmar2023zero,hertz2022prompt}. In this work, we ask a fundamentally different question: \emph{Can we automatically discover the underlying visual structure of a concept within diffusion model's knowledge?} %Rather than requiring user-specified controls, we aim to decompose the model's internal knowledge into meaningful directions.

% This question touches on a fundamental limitation in how we interact with diffusion models. Current control methods ~\cite{zhang2023addingconditionalcontroltexttoimage, gandikota2023concept, ye2023ipadaptertextcompatibleimage,ye2023ipadaptertextcompatibleimage, hertz2024stylealignedimagegeneration, li2023photomaker, shi2024instantbooth, chen2024trainingfreeregionalpromptingdiffusion} require users to specify their desired manipulations in advance, limiting interactive creativity. This contrasts with natural human artistic workflows, where creators dynamically explore creative ideas while jointly refining them toward meaningful artistic outcomes~\cite{hoffmann2016modeling}. This synergy between specification and exploration is not new to generative models. Early GAN architectures naturally developed disentangled latent spaces that enabled continuous\cite{harkonen2020ganspace,radford2015unsupervised, wu2021stylespace, shen2020interfacegan}, compositional control over generated images. Users could explore these spaces to discover interesting variations that would be difficult to describe in words~\cite{wu2021stylespace}, then combine them to achieve their creative goals~\cite{grabe2022towards}. 


% While diffusion models have largely superseded GANs in conditional image synthesis~\cite{dhariwal2021diffusion},  their underlying structure remains less understood. Diffusion models achieve remarkable diversity through high-dimensional latents, unlike GANs' compact latent spaces.  With a single prompt, diffusion models can generate radically different variations through different random initializations of input noise. We ask - Is it possible to discover interpretable structure within this vast space of variations?

Text-to-image diffusion models are capable of generating remarkable visual variations from a single prompt through different random initializations. However, this vast creative potential remains largely opaque to users---while we can generate diverse images, we lack understanding of the underlying structure of these variations. This presents a fundamental challenge: how can we discover and expose the latent visual capabilities encoded within these models?

\let\thefootnote\relax \footnote{$^{*}$Correspondence to \texttt{gandikota.ro@northeastern.edu}}

The challenge touches on a key limitation in how we interact with diffusion models today. Current control methods require users to explicitly specify their desired edits in advance through prompts~\cite{gandikota2023concept}, reference images~\cite{zhang2023addingconditionalcontroltexttoimage, chen2024trainingfreeregionalpromptingdiffusion, ruiz2022dreambooth,kumari2022customdiffusion, Ryu_lora, hu2021lora}, or attribute vectors~\cite{ye2023ipadaptertextcompatibleimage, hertz2024stylealignedimagegeneration, li2023photomaker, shi2024instantbooth,parmar2023zero,hertz2022prompt}. That contrasts sharply with natural human creative workflows, where artists dynamically explore creative ideas and jointly refine them toward meaningful artistic outcomes~\cite{hoffmann2016modeling}. The need for pre-specified controls creates a barrier between users and the full creative potential of these models.

Interestingly, earlier generative models like GANs~\cite{gans,karras2019style,brock2018large} naturally developed more interpretable internal structures. Their compact latent spaces often exhibited emergent disentanglement~\cite{harkonen2020ganspace,radford2015unsupervised, wu2021stylespace, shen2020interfacegan}, enabling continuous and compositional control over generated images. Users could explore these spaces to discover interesting variations that would be difficult to describe in words~\cite{wu2021stylespace}, then combine them to achieve their creative goals~\cite{grabe2022towards}.

Diffusion models have largely superseded GANs in conditional image synthesis~\cite{dhariwal2021diffusion}, achieving greater diversity through much higher-dimensional latents. And yet an understanding of the underlying structure of these larger latent spaces has remained elusive. In this work, we ask a fundamental question: \emph{Can we automatically discover the visual structure within a diffusion model's knowledge of a concept?} Rather than requiring user-specified controls, we aim to decompose the model's internal representations into expressive directions that users can explore and combine.

To address these needs, we present \textbf{SliderSpace}, a framework that brings systematic explorability to diffusion models. Given just a text prompt, SliderSpace discovers a canonical set of meaningful, diverse, and controllable directions within the model's knowledge of that concept. Each direction is implemented as a low-rank adapter~\cite{hu2021lora} that can be scaled and composed with others, allowing users to explore and smoothly combine different aspects of variation, as shown in Figure~\ref{fig:intro}.

We ground SliderSpace discovery in three key requirements for meaningful decomposition of a diffusion model's visual manifold: 
\begin{enumerate}
    \item \textbf{Unsupervised Discovery:} The decomposition process should emerge from the intrinsic structure of the model's learned representation, rather than being guided by predefined attributes. This ensures we capture the true topology of the model's knowledge space rather than projecting our assumptions onto it.
    
    \item \textbf{Semantic Orthogonality:} Each discovered control must represent a distinct semantic direction. This is enforced in a semantic feature space, like CLIP, where every slider has an orthogonal effect in embeddings. This prevents discovering multiple controls that create similar semantic effects, making the system more efficient and easier.
    
    \item \textbf{Distribution Consistency:} Directions must induce consistent transformations across both random seeds and prompt variations. 
\end{enumerate}

These requirements naturally lead to our proposed framework, which we formalize in Section~\ref{sec:method}. As we show in our experiments, SliderSpace is architecture-agnostic, working with both conventional U-Net based models like Stable Diffusion~\cite{rombach2022high, rombach2022sd20, podell2023sdxl, turbo, dmd} and recent transformer-based architectures like Flux~\cite{flux}.

We demonstrate the expressiveness of SliderSpace through three applications: First, we show how SliderSpace can decompose high-level concepts into diverse and expressive components, revealing the natural axes of variation in the model's understanding. Second, we explore artistic style variation, where SliderSpace discovers directions that match or exceed the diversity of manually curated artist lists while being judged more useful by human evaluators. Finally, we show how SliderSpace can help reverse the mode collapse commonly observed in distilled diffusion models, restoring diversity while maintaining generation speed.

Beyond providing practical creative control, SliderSpace opens new avenues for understanding and utilizing the latent capabilities of diffusion models. By mapping these models' visual potential into intuitive, composable directions, we take a step toward making their creative possibilities more accessible and interpretable to users.

% Image editing methods in diffusion models unlock the creativity of users. In this work we ask an alternate question: \emph{Can we organize and expose what of the diffusion model is already capable of?}.
% Existing methods for controlling image generation typically require users to manually specify edit directions for desired changes. This process is time-consuming, requires technical expertise, and limits the spontaneity of the creative process. For instance, if a user wants to adjust the smile of a generated person, they must explicitly request this edit, often through imprecise prompt engineering or model fine-tuning. This approach of predefined controls or manual specifications restricts users from fully exploring the latent capabilities of the model. There may be interesting stylistic variations or attributes that the model can generate, but users have no easy way to discover or utilize these.

% Natural visual disentanglement was an emergent property in the latent space of Generative Adversarial Models (GANs) \cite{harkonen2020ganspace,radford2015unsupervised, wu2021stylespace, shen2020interfacegan}. In particular, it has been observed that StyleGAN~\cite{karras2019style} stylespace neurons offer detailed control over many meaningful aspects of images that would be difficult to describe in words~\cite{wu2021stylespace}. However, diffusion models do not share such a compact latent space~\cite{park2023unsupervised}; and efforts to uncover such a space in the semantic embeddings of the text conditioning have met with limited success \nik{Nick - is there a specific citation you were thinking about?}.

% In this work we introduce \textbf{SliderSpace}, which takes a step towards uncovering an analogous low dimensional representation of diffusion models' visual breadth; in essence treating the diffusion model as many generators sharing parameters, where a particular generator is defined by a specific prompt. For a given prompt we sample many random seeds (and optionally prompt expansions using an LLM), generate the corresponding images, and apply an off the shelf feature extractor (in this work CLIP, but our method can be applied to any differentiable feature extractor). We use PCA to analyze these features, and for each of the leading $k$ principal components we train a LoRA \cite{} which causes the diffusion model to produces images which increase the feature magnitude along that component when passed back through the same feature extractor. This leads to a 'Slider' for each principal component, because each LoRA can be scaled and applied to the original diffusion model, continuously varying those visual features in the generated results (as measured, in our case, by CLIP).

% There are many other works that enhance the controllability of diffusion models. One common approach is enabling users to add spatial constraints to a generation either manually, or via a reference image \cite{zhang2023addingconditionalcontroltexttoimage, chen2024trainingfreeregionalpromptingdiffusion}, a second is leveraging more abstract embeddings (e.g. identity, style) extracted from a reference image \cite{ye2023ipadaptertextcompatibleimage, hertz2024stylealignedimagegeneration, li2023photomaker, shi2024instantbooth}, a third is finetuning a foundation model to better generate a concept important to the user \cite{ruiz2022dreambooth, kumari2022customdiffusion, Ryu_lora, hu2021lora}, and a fourth (most relevant to this work) is finding low-rank adaptors of the model based on a prompt or small training set which can be scaled to provide continous control over one aspect of generated image (e.g. night vs day, basic vs luxury, etc.) \cite{gandikota2023concept}. SliderSpace is complementary to all of these methods and offers something distinct. All of the other methods we are aware require the user (and / or model designer) to know in advance what type of control they want. In contrast SliderSpace assists users in discovering and controlling hidden capabilities present in the diffusion model's distribution of possible generations.

%We propose that truly intuitive creative control in a text-to-image model should meet three key criteria: \emph{discoverability}, \emph{intuitiveness}, and \emph{specificity}. The model should reveal controllable attributes that may not be immediately obvious, offer controls that are easy to understand and manipulate, and ensure each control affects a distinct attribute of the generated image.

% We demonstrate the utility and power of SliderSpace using three applications built on top of SDXL-DMD \cite{dmd}, because its fast generation speed lends itself well to the continuous control offered by SliderSpace.

% First, we study concept decomposition (Section \ref{sec:concept_exp}), where we learn sliders for a specific concept (e.g. 'monster', 'waterfall', 'car'). Through quantitative metrics of diversity and text alignment we demonstrate that the learned sliders dramatically boost the diversity of generations when randomly applied without harming text alignment; we also ask humans to qualitatively judge these results in a user study where they find the SliderSpace results to be more 'Diverse', 'Useful', and 'Creative' than our baselines.

% Second, we attempt to compare the automatic discoveries of SliderSpace to a large scale manual study of artistic styles (Section \ref{sec:art_exp}), open-sourced by ParrotZone \cite{parrotzone}. In this study SDXL was prompted with over 4300 artist names,  and based on visual inspection the cases of successful stylistic mimicry recorded. Quantitatively SliderSpace more closely matches the distribution of artistic variation discovered by ParrotZone than other baselines, and in our user studies was judged to be significantly more 'Diverse' and 'Useful' than the baselines. To our surprise humans even judged SliderSpace results to be slightly more 'Diverse' than the results generated by the manually discovered artist names of \cite{parrotzone}.

% Third, we attempt to use SliderSpace to reverse the mode collapse commonly observed in distilled few-step diffusion models relative to the original teacher model (Section \ref{sec:diverse_exp}). We quantitatively demonstrate that applying SliderSpace to SDXL-DMD leads to more closely matching the distribution of images by the original teacher, SDXL.

%Through extensive experiments on various state-of-the-art text-to-image models, we demonstrate that SliderSpace significantly enhances user control and creative expression in AI-assisted image generation tasks. Our method enables a range of applications, including concept decomposition and control, diversity improvement in generated images, customization dissection and edits, and the exploration of artistic styles inherent in the model.

% SliderSpace goes beyond providing a practical tool for enhanced creative control. By mapping the visual potential of diffusion models it can open new avenues for generative creativity and deepens our understanding of each model's hidden potential.
\section{VLM$^2$-Bench}
VLM$^2$-Bench is a benchmark designed to assess models' ability to visually link matching cues when processing multiple images or videos. This section introduces the three main categories of VLM$^2$-Bench—\textit{general cue} (\S\ref{gc}), \textit{object-centric cue} (\S\ref{oc}), and \textit{person-centric cue} (\S\ref{pc})—detailing their associated subtasks, data collection process, and QA pair construction.

\begin{figure*}[t]
  \centering
  \includegraphics[width=0.99\textwidth]{img/general-pipe.png} 
  % \vspace{-0.8cm}
  \caption{Construction of \textbf{GC}: (i) We start by manually verifying the edited image data based on three key criteria. (ii) A VLM is then prompted to generate captions for each image, followed by salient score-based filtering to retain the challenging cases. (iii) Finally, visual cues are extracted from two sources and incorporated into a QA prompt, guiding an LLM to generate both positive and negative answer pairs. 
  % \yi{"a event hall" is hard to read in the top right -- rule of thumb is need to make paper read-friendly for color blind ppl and when ppl print in black and white (otherwise reviewer might crib); turn that shade into darker shade of color please}
  }
  \label{fig:gc construction}
\end{figure*}

\subsection{General Cue (GC)}
\label{gc}
GC is designed to assess a model's ability to link matching cues across diverse contexts, encompassing a broad range of \textit{general cues}. Given two images containing both matched and mismatched cues, an ideal model should accurately identify mismatched ones and associate matched ones.

\paragraph{Subtasks.}
Here we introduce two subtasks: (i) \textbf{\textit{Matching (Mat)}} evaluates a model’s ability to link corresponding visual cues across two images to determine whether they match. Instead of merely identifying differences, the model must associate identical visual elements in both images to recognize what has remained the same and what has changed. 
(ii) \textbf{\textit{Tracking (Trk)}} focuses on a model’s ability to track a specific visual cue that appears in only one of the two images and determine how it has changed. Rather than simply detecting a difference, the model must link the cue across contexts to understand the transformation process. 



\paragraph{Data Collection.} 
We repurpose data from two image editing datasets~\citep{wei2024omniedit,ku2023imagenhub}, where each data sample includes an original image $I_{ori}$, an edited image with subtle modifications $I_{edit}$, and a corresponding edit instruction $\mathcal{P}$ describing the changes. Our data collection is carried out across two dimensions. First, to ensure diversity in the mismatched cues, GC encompasses various types of changes, such as instance-level modifications (e.g., add/remove, swap, attribute change), which focus on specific items, as well as environment-level changes.

\paragraph{QA Construction.}
We predefine a T/F question template for \textit{Mat} and \textit{Trk} with a placeholder for the candidate answer (refer to Appendix~\ref{appendix: more details on benchmark construction}). Figure~\ref{fig:gc construction} illustrates the construction process, which follows a three-stage approach. 

\textit{Manual Screening \& Refinement:} We ensure that $\mathcal{P}$ accurately reflects the changes (correctness), corresponds uniquely to the modified cues (uniqueness), and is unambiguous (clarity).


\textit{Salient Sampling:} Here, we automate the removal of overly simple cases (e.g., mismatched cues are too salient). To achieve this, a VLM first generates separate descriptions for \( I_{ori} \) and \( I_{edit} \), denoted as \( Cap_{ori} \) and \( Cap_{edit} \). These descriptions are then combined with \( \mathcal{P} \) into a single passage using a predefined template \(\mathcal{T}\) (see Table \ref{template salient score} for details). The probability assigned by a language model (e.g., Llama3-8B~\citep{dubey2024llama}) to \( \mathcal{P} \) given this text-based information is used to compute the salient score, formulated as:

\vspace{-13pt}
\begin{equation}
S_{\text{salient}} = \frac{1}{|\mathcal{P}|} \sum_{i=1}^{|\mathcal{P}|} \log P_{\theta}(p_i \mid C \cup p_{<i}),
\end{equation}

\noindent where \( \mathcal{P} = \{p_1, p_2, ..., p_{|\mathcal{P}|}\} \) represents the tokenized \(\mathcal{P}\), and \( C = \mathcal{T} (Cap_{ori}, Cap_{edit}) \) denotes the context filled with template \(\mathcal{T}\). Samples with scores below \( \theta \) (-2.0 here) are retained, ensuring that the benchmark includes more challenging examples requiring nuanced visual cue association. 
% \yi{what's your actual threshold used? which LLM exactly did you use to compute salient score and construct your dataset here? I know it's tempting to leave many details into appendix but if you don't provide sufficient information for readers to follow and reproduce your methods at bare minimum from the 8 main pgs then you may put the paper at greater risk for reviewer attack (things to consider) -- good to cross reference other benchmark construction papers and see how they write these details}


\textit{Pair-wise Answer Generation:} Finally, we extract visual cues using a dual-level approach. First, cues parsed from VLM-generated descriptions compensate for the limitations of open-set detectors when handling out-of-distribution scenes. Meanwhile, the open-set detector~\citep{wu2022grit} extracts fine-grained cues that VLMs might overlook. With these extracted cues, we prompt an LLM to generate a pair of answers for \textit{Mat} and \textit{Trk}, each consisting of one positive and one negative answer.




\subsection{Object-centric Cue (OC)}
\label{oc}
OC aims to assess a model's ability to link matching cues associated with everyday objects using \textit{object-centric cues}. Even when encountering an object for the first time, a well-aligned model should be able to leverage its unique visual cues to establish associations, enabling it to recognize and track the object across different scenes. This capability is essential for coherent perception and interaction in real-world deployments.

\paragraph{Subtasks.} 
\label{subtasks}
Based on the complexity of linking cues to solve the problem, we define three subtasks in OC. (i) \textbf{\textit{Comparison (Cpr)}} requires the model to determine whether the objects appearing in different images are the same. This task primarily assesses the model’s ability to perceive visual consistency or change. Notably, we observe that models exhibit significant model-specific bias when making a binary decision~\citep{pair, ye2024justice, song2024large,li2024naturalbench}, leading to discrepancies between results and their actual capabilities. To mitigate this, we introduce consistency-pair validation, where for each statement (e.g., ``X is Y”, with the answer being T), we generate a corresponding negation (e.g., ``X is not Y”, with the answer being F). The model is only considered correct if it correctly answers both statements, ensuring consistency in its decision-making.
(ii) \textbf{\textit{Counting (Cnt)}} involves identifying the number of unique objects, requiring the model not only to recognize variations or consistencies but also to track distinct cues to avoid double-counting the same object. (iii) \textbf{\textit{Grouping (Grp)}}, the most challenging one, requires the model to identify all instances of the same object, building on precise cue matching across multiple images.

\paragraph{Data Collection.}  
We manually collect various categories of everyday objects (e.g., pets, cups). For each category, we define multiple subcategories and collect a set of images \( \mathcal{I}_{O_i}\)—four images that depict the same object in different scenarios. Additionally, we also collect a set \( \mathcal{I}_{\neg O_i} \), consisting of four images of different objects, each containing some matching visual cues with \( \mathcal{I}_{O_i} \), which are used as distractors.
% We manually collect 8 major categories of everyday objects (e.g., pets, cups) from various online sources. Within each category, we defined multiple subcategories \( O_i \) and collected \( \mathcal{I}_{O_i} = \{ I_1, I_2, I_3, I_4 \} \)—4 images that depict the same object in different scenarios. Additionally, we also collect \( \mathcal{I}_{\neg O_i} \), a set of 4 images featuring different objects, each containing some matching visual cues with \( \mathcal{I}_{O_i} \). \( \mathcal{I}_{\neg O_i} \) are used as distractors to test the model's ability to differentiate between objects that share many cues but are distinct.


\paragraph{QA Construction.}
For each subtask, we define a question template that includes a placeholder for \( \mathcal{I}_{O_i} \), which allows us to tailor the question based on different objects (see Appendix~\ref{appendix: more details on benchmark construction}). For answer generation, we first curate the multi-image sequences according to predefined rules. For each specific sequence, we generate the ground truth answers for the questions related to \textit{Cpr}, \textit{Cnt}, and \textit{Grp}.




\subsection{Person-centric Cue (PC)}
\label{pc}
PC aims to evaluate a model's ability to link \textit{person-centric cues}. While a model cannot memorize every individual, it should possess the capability to associate the same person across different images or frames by leveraging distinctive visual cues such as facial features, clothing, or body posture. This ability is essential for ensuring coherent perception of human actions and is a fundamental requirement for real-world VLM applications.

\paragraph{Subtasks.} 
Similar to OC's subtasks (refer to \S\ref{subtasks}), PC includes (i) \textbf{\textit{Comparison (Cpr)}}, (ii) \textbf{\textit{Counting (Cnt)}}, and (iii) \textbf{\textit{Grouping (Grp)}}. However, unlike objects, individuals can be observed through their actions in videos. Therefore, we introduce (iv) \textbf{\textit{Video Identity Describing (VID)}}. This subtask assesses whether a model can correctly link the same person by analyzing its description of a video containing that person.


\paragraph{Data Collection.}  
We manually select several individuals, each denoted as \( \mathcal{P}_i \). For each individual, we collect \( \mathcal{I}_{\mathcal{P}_i}\)—4 images depicting the same individual. For each image \( I_i  \in  \mathcal{I}_{\mathcal{P}_i} \), we select the distractor images \( I_{\neg i} \notin \mathcal{I}_{\mathcal{P}_i} \) that has the highest CLIP similarity~\citep{hessel2021clipscore}. This allows us to obtain images of different individuals where most cues are matched.
% We manually select 30 different individuals, each denoted as \( \mathcal{P}_i \). For each individual, we collect \( \mathcal{I}_{\mathcal{P}_i} = \{ I_1, I_2, I_3, I_4 \} \)—4 images depicting the same individual in different scenarios. For each image \( I_i  \in  \mathcal{I}_{\mathcal{P}_i} \), we select the distractor images \( I_{\neg i} \notin \mathcal{I}_{\mathcal{P}_i} \) that has the highest CLIP similarity. This allows us to obtain images of different individuals where most cues are matched.
For the subtask of \textit{VID}, we collect videos of different individuals, denoted as \( V_{\mathcal{P}_i} \), and pair each with another video \( V_{\neg \mathcal{P}_i} \) featuring a different individual with highly similar cues (e.g., actions, scene, clothing). We then construct two video sequences: 
(i) \( \mathcal{P}_i \xrightarrow{} \neg \mathcal{P}_i \), assessing the model's ability to distinguish individuals. 
(ii) \( \mathcal{P}_i \xrightarrow{} \neg \mathcal{P}_i \xrightarrow{} \mathcal{P}_i \), evaluating whether the model detects changes and links the final occurrence of \( \mathcal{P}_i \) to its first appearance.

% Additionally, for the subtask of \textit{VID}, we collect videos of different individuals, denoted as \( V_{\mathcal{P}_i} \). For each \( V_{\mathcal{P}_i} \), we find another video of a different individual, denoted as \( V_{\neg \mathcal{P}_i} \), with highly similar cues (such as actions, scene, clothing, etc.). We then construct two types of videos based on different person sequences. (i) \( \mathcal{P}_i \xrightarrow{} \neg \mathcal{P}_i \), which tests the model's ability to distinguish between the two individuals. (ii) \( \mathcal{P}_i \xrightarrow{} \neg \mathcal{P}_i \xrightarrow{} \mathcal{P}_i \), which examines whether the model can detect changes between \( \mathcal{P}_i \) and \( \neg \mathcal{P}_i \), and link the final occurrence of \( \mathcal{P}_i \) to its first appearance.


\paragraph{QA Construction.} The construction for the overall QA in PC’s \textit{Cpr}, \textit{Cnt}, and \textit{Grp} subtasks follows a similar approach to OC. For the \textit{VID} task, we emphasize the model's ability to describe individuals when designing open-ended questions, aiming to better test the model's capacity to link individuals appearing in different scenes.

\begin{figure}[hb]
  \centering
  \includegraphics[width=0.49\textwidth]{img/bench-stastics.png} 
  % \vspace{-0.8cm}
  \caption{Statistical overview of \textbf{VLM$^2$-Bench}. The pie chart shows the distribution of 9 subtasks across the 3 main categories of visual cues. The bar plot illustrates the percentage breakdown by question format. 
  }
  \label{fig:bench statistics main}
\end{figure}

\begin{figure*}[t]
    \captionsetup{type=table}
    % \vspace{-0.4cm}
    \centering
    \begin{minipage}{0.99\textwidth}
        \centering
        \resizebox{1\textwidth}{!}{%
        \begin{tabular}{l||cc|ccc|cccc|cc}
            \toprule
            \textbf{Baselines or Models} & \multicolumn{2}{c|}{\textbf{GC}} & \multicolumn{3}{c|}{\textbf{OC}} & \multicolumn{4}{c|}{\textbf{PC}} & \multicolumn{2}{c}{\textbf{Overall*}}\\
            % \midrule
            & \textit{Mat} & \textit{Trk} & \textit{Cpr} & \textit{Cnt} & \textit{Grp} & \textit{Cpr} & \textit{Cnt} & \textit{Grp} & \textit{VID} & Avg & \textbf{$\Delta_{human}$} \\
            \midrule
            \textcolor{black!50}{Chance-Level} & 
            \textcolor{black!50}{25.00} & 
            \textcolor{black!50}{25.00} & 
            \textcolor{black!50}{50.00} & 
            \textcolor{black!50}{34.88} & 
            \textcolor{black!50}{25.00} & 
            \textcolor{black!50}{50.00} & 
            \textcolor{black!50}{34.87} & 
            \textcolor{black!50}{25.00} & 
            \textcolor{black!50}{-} & 
            \textcolor{black!50}{33.72} & 
            \textcolor{black!50}{-61.44} \\
            Human-Level  & 95.06 & 98.11 & 96.02 & 94.23 & 91.92 & 97.08 & 92.87 & 91.17 & 100.00 & 95.16 & 0.00 \\
            \midrule
            LLaVA-OneVision-7B    & 16.60 & 13.70 & 47.22 & 56.17 & 27.50 & \cellcolor{yellow!45}62.00 & 46.67 & 37.00 & \cellcolor{yellow!15}47.25 & 39.35 & -55.81 \\
            LLaVA-Video-7B        & 18.53 & 12.79 & 54.72 & \cellcolor{yellow!15}62.47 & 28.50 & \cellcolor{yellow!45}62.00 & \cellcolor{yellow!45}66.91 & 25.00 & \cellcolor{yellow!45}59.00 & 43.32 &  -51.84 \\
            LongVA-7B             & 14.29 & 19.18 & 26.67 & 42.53 & 18.50 & 21.50 & 38.90 & 18.00 & 3.75  & 22.59 & -72.57 \\
            mPLUG-Owl3-7B         & 17.37 & 18.26 & 49.17 & \cellcolor{yellow!45}62.97 & 31.00 & \cellcolor{yellow!15}63.50 & 58.86 & 26.00 & 13.50 & 37.85 & -57.31 \\
            Qwen2-VL-7B  & 27.80 & 19.18 & \cellcolor{yellow!15}68.06 & 45.99 & 35.00 & 61.50 & 58.59 & 49.00 & 16.25 & 42.37 & -52.79 \\
            Qwen2.5-VL-7B & \cellcolor{yellow!45}35.91 & \cellcolor{yellow!75}43.38 & \cellcolor{yellow!45}71.39 & 41.72 & \cellcolor{yellow!15}47.50 & \cellcolor{yellow!75}80.00 & 57.98 & \cellcolor{yellow!75}69.00 & 46.50 & \cellcolor{yellow!45}54.82 & -40.34 \\
            InternVL2.5-8B        & 21.24 & 26.03 & 53.33 & 55.23 & 46.50 & 51.50 & \cellcolor{yellow!15}60.00 & \cellcolor{yellow!15}52.00 & 5.25  & 41.23 & -53.93 \\
            InternVL2.5-26B        & \cellcolor{yellow!15}30.50 & \cellcolor{yellow!15}30.59 & 43.33 & 51.48 & \cellcolor{yellow!45}52.50 & 59.50 & 59.70 & \cellcolor{yellow!45}61.00 & 21.75 & \cellcolor{yellow!15}45.59 & -49.57 \\
            \midrule
            GPT-4o                & \cellcolor{yellow!75}37.45 & \cellcolor{yellow!45}39.27 & \cellcolor{yellow!75}74.17 & \cellcolor{yellow!75}80.62 & \cellcolor{yellow!75}57.50 & 50.00 & \cellcolor{yellow!75}90.50 & 47.00 & \cellcolor{yellow!75}66.75 & \cellcolor{yellow!75}60.36 & -34.80 \\
            \bottomrule
        \end{tabular}
        }
    \end{minipage}
    % \hfill
    % \begin{minipage}[c]{0.22\textwidth}
    %     \centering
    %     \vspace{-0.5em}
    %     \includegraphics[width=\textwidth]{img/radar_chart.png}
    % \end{minipage}
    % \vspace{-0.2cm}
    \caption{Evaluation results on \textbf{VLM$^2$-Bench}, covering \textit{Mat} (Matching), \textit{Trk} (Tracking), \textit{Cpr} (Comparison), \textit{Cnt} (Counting), \textit{Grp} (Grouping), and \textit{VID} (Video Identity Describing). The \colorbox{yellow!75}{highest}, \colorbox{yellow!45}{second}, and \colorbox{yellow!15}{third} highest scores are highlighted. *: Overall excludes the \textit{VID} due to the lack of a chance-level baseline for open-ended tasks.}
    \label{exp:main_exp}
    % \vspace{-0.4cm}
\end{figure*}

\subsection{Benchmark Statistics}
\label{bench_statistics_main}
Our benchmark is organized into three main categories, comprising a total of 9 subtasks. After careful verification, it contains 3,060 question-answer pairs, with varying formats including T/F, multi-choice (MC), numerical (Nu), and open-ended (Oe). To ensure the quality of the annotations, we perform an inter-annotator agreement (IAA) evaluation~\citep{thorne2018fever} involving three annotators, resulting in a high Fleiss' Kappa score~\citep{fleiss1971measuring} of 0.983. Figure~\ref{fig:bench statistics main} presents the distribution of these subtasks across the three categories, along with the breakdown of different question formats. For additional details, refer to Appendix~\ref{appendix: statistics}.

% \yi{TODO: add overall descriptive stats for dataset, and interannotator agreement}


\section{Experimental Results}
We demonstrate the effectiveness of STAIR through extensive experiments on multiple benchmarks that reflect both the safety guardrails and general capabilities of LLMs. 

\subsection{Experimental Settings}

We hereby introduce the key experimental settings, with more details explained in~\cref{sec:appendix_data} and~\ref{sec:appendix_exp}.


\textbf{Models and Datasets.} We take two base LLMs for safety alignment, LLaMA-3.1-8B-Instruct~\cite{dubey2024llama} and Qwen-2-7B-Instruct~\cite{qwen2}. For test-time scaling and ablation studies, only LLaMA is utilized. All experiments use a seed dataset $\mathcal{D}$ comprising 50k samples from three sources. For safety-focused data, we use a modified version of 22k preference samples from PKU-SafeRLHF~\cite{ji2024pku} along with 3k jailbreak data from JailbreakV-28k~\cite{luo2024jailbreakv}. Additionally, 25k pairwise data are drawn from UltraFeedback~\cite{cui2024ultrafeedback} to maintain helpfulness, as done in prior works~\cite{qi2024safety,wu2024thinking}. Note that responses in $\mathcal{D}$ are in normal conversational style rather than reasoning-oriented. While we use the whole dataset with labels for training baselines, we only take 10k samples each from PKU-SafeRLHF and UltraFeedback to construct structured CoT data $\mathcal{D}_{\text{CoT}}$. During each self-improvement iteration, 5k safety and 5k helpfulness samples are utilized. Jailbreak prompts are used only in the final two iterations, with 1k and 2k samples, respectively.

\textbf{Baselines.} We first evaluate the performance of CoT prompting~\cite{wei2022chain} to assess the contribution of available reasoning capability to safety consolidation. We then include SFT and DPO~\cite{rafailov2024direct} on standard datasets as representative alignment techniques, both of which are employed in our framework. Besides, SACPO~\cite{wachi2024stepwise}, designed to mitigate the safety-performance trade-off with two-step DPO, and Self-Rewarding~\cite{yuanself}, which leverages self-generated and self-rewarded data in iterative DPO, are also used as baselines for comparison.


\textbf{Evaluation.} We use 10 popular benchmarks to evaluate harmlessness and general performance of the trained models. For harmlessness, models are required to provide clear refusals to harmful queries, following~\cite{guan2024deliberative}. We test the models on StrongReject~\cite{souly2024strongreject}, XsTest~\cite{rottger2023xstest}, highly toxic prompts from WildChat~\cite{zhaowildchat}, and the stereotype-related split from Do-Not-Answer~\cite{wang2023not}. We report the average goodness score on the top-2 jailbreak methods of PAIR~\cite{chaojailbreaking} and PAP~\cite{zeng2024johnny} for StrongReject, and refusal rates for the rest. For general performance, we use benchmarks reflecting diverse aspects of trustworthiness in addition to the popular ones for helpfulness like GSM8k~\cite{hendrycks2measuring}, AlpacaEval2.0~\cite{dubois2024length} and BIG-bench HHH~\cite{zhou2024beyond}. We take SimpleQA~\cite{wei2024measuring} for truthfulness, InfoFlow~\cite{mireshghallahcan} for privacy awareness, and AdvGLUE~\cite{wang2adversarial} for adversarial robustness. Official metrics are reported for all.

% We leave other details including hyperparameters and evaluation strategies in~\cref{sec:appendix_exp}.


\begin{table*}[ht]
    \centering
    \caption{Performance on diverse benchmarks reflecting both harmlessness and general performance. CoT Style represents whether the method adopt Chain-of-Thought reasoning, while Self Gen. denotes whether the method use self-generated data for training. For all reported metrics, the best results are marked in \textbf{bold} and the second best results are marked by \underline{underline}.}
    \renewcommand{\arraystretch}{1.1} % Increase row height
    
\resizebox{\textwidth}{!}{%
    \begin{tabular}{l@{\;\,}|@{\;\,}c@{\;\,}|@{\;\,}c@{\;\,}|c@{\;\,}c@{\;\,}c@{\;\,}c|c@{\;\,}c@{\;\,}c@{\;\,}c@{\;\,}c@{\;\,}c}
        \toprule[1.5pt]
       & \multirow{2}{*}{\makecell{CoT\\Style}} & \multirow{2}{*}{\makecell{Self\\Gen.}}  &  \multicolumn{4}{c|}{\textbf{Harmlessness}} & \multicolumn{6}{c}{\textbf{General}}  \\ \cmidrule(lr){4-7}\cmidrule(lr){8-13}
       & & & StrongReject  & XsTest  & WildChat  & Stereotype  &  SimpleQA 	&  InfoFlow  &  AdvGLUE  & GSM8k  & AlpacaEval  & HHH  \\\midrule
        \multicolumn{13}{c}{\sc Llama-3.1-8B-Instruct} \\ \midrule
        Base &  - & - & 0.4054 & 88.00\% & 47.94\% & 87.37\% & 2.52\% & 0.4229 & 58.33\% &85.60\% &  25.55\% & 82.50\%\\ 
        CoT & \cmark & - & 0.3790 & 87.00\% & 50.23\% & 65.26\% & 4.09\% &  0.7041 & 58.40\% & 87.11\% &22.04\% & 81.63\% \\
        SFT & \xmark & \xmark & 0.4698 & 94.50\% & 50.68\% & 94.74\% & 4.72\% &  0.7134 & 57.53\% &72.02\% & 9.21\% & 82.63\% \\
        DPO & \xmark & \xmark & 0.5054 & 86.00\% & 54.79\% & \bf 97.89\% & 4.46\% & 0.7081 & 66.27\% &84.15\% &  15.26\% & 83.84\% \\ 
        SACPO & \xmark & \xmark  & 0.7264 & 88.50\% & 58.45\% & 96.84\% & 0.74\% &  0.0503 & 65.60\% &86.50\% & 20.44\% & 85.21\%\\ 
        Self-Rewarding & \xmark & \cmark & 0.4633 & \bf 99.00\% & 49.77\% & 94.74\% & 2.70\%  & 0.6618 & 59.10\% & \bf 88.10\%& 26.41\% & 82.09\%\\\midrule
        STAIR-SFT & \cmark & \xmark & 0.6536 & 85.50\% & 50.68\% & 94.74\% & \underline{6.31\%} & \underline{0.7876} & \bf 70.57\% & 86.05\%  &  31.21\% & 83.13\%\\
        +DPO-1 & \cmark & \cmark & 0.6955 & 94.00\% & 57.99\% & \bf 97.89\% & 6.08\% & \bf 0.7998 & 65.93\% & 86.81\% & 34.48\% & 84.53\% \\
        +DPO-2 & \cmark & \cmark & \underline{0.7973} & 96.50\% & \underline{68.95\%} & 96.84\% & 6.00\% &  0.7700 & \underline{69.43\%} & 87.26\% &\underline{36.24\%} & \bf 87.09\% \\
        +DPO-3 & \cmark & \cmark & \bf  0.8798 &  \bf 99.00\% & \bf 69.86\% & 96.84\% & \bf 6.38\% &  0.7395 & 69.20\% &\underline{87.64\%} &\bf  38.66\% & \underline{85.66\%} \\ \midrule
        \multicolumn{13}{c}{\sc Qwen-2-7B-Instruct} \\ \midrule
        Base &  - & - & 0.3808 & 72.50\% & 47.49\% & 90.53\% & 3.79\% & 0.7221 & 66.50\%& \underline{87.49\%}  & 20.06\% & 87.87\%\\ 
        CoT & \cmark & -  & 0.3792 & 70.00\% & 42.92\% & 88.42\% & 3.03\%& 0.7628 & 65.60\% & \bf 88.10\%  & \underline{25.97\%} & 88.30\%\\
        SFT & \xmark & \xmark & 0.4952 & 84.00\% & 58.45\% & 91.58\% & 3.47\% & 0.6267 & 66.90\% &82.34\% &  8.94\% & 89.74\% \\
        DPO & \xmark & \xmark & 0.5026 & 69.00\% & 66.21\% & 88.42\% & 2.59\% &  0.6793 & 70.97\% & 81.43\% & 11.48\% & 88.08\% \\
        SACPO & \xmark & \xmark & 0.5577 & 75.00\% & 60.27\% & 95.79\% & 0.62\%  & 0.6213 & 64.10\% & 85.22\% & 17.04\% & 89.60\% \\ 
        Self-Rewarding & \xmark & \cmark & 0.5062 & 96.00\% & 52.51\% &  94.74\% & 3.37\% & 0.7140 & 66.13\% & 87.34\% & 14.69\% & 88.31\% \\\midrule
        STAIR-SFT & \cmark & \xmark & 0.7356 & 83.50\% & 62.56\% & 95.79\% & 3.81\% &  0.8215 & 70.57\% &84.61\% & 20.31\% & \underline{90.38\%} \\
        +DPO-1 & \cmark & \cmark & 0.7606 & 96.50\% & 65.19\% & 95.79\% & \underline{3.88\%} & \underline{0.8235} & \underline{73.10\%} & 84.76\% & 23.29\% & 90.21\% \\
        +DPO-2 & \cmark & \cmark & \underline{0.8137} & \underline{98.50\%} & \underline{67.90\%} & \underline{97.89\%} & 3.79\% & \bf 0.8646 & 72.83\% & 86.05\% & 24.86\% & 90.11\% \\
        +DPO-3 & \cmark & \cmark & \bf 0.8486 & \bf 99.00\% & \bf 80.56\% & \bf 98.95\% & \bf 4.07\% & 0.7644 & \bf 74.13\% & 85.75\% & \bf 26.31\% & \bf 90.71\% \\ \bottomrule[1.5pt]
    \end{tabular}}
    \label{tab:benchmarks}
    \vspace{-2ex}
\end{table*}



\subsection{Main Results}

We present the results on diverse benchmarks evaluating both the harmlessness and the general performance in~\cref{tab:benchmarks}, which shows the superiority of STAIR, attributed to the incorporation of introspective reasoning to safety alignment and the self-improvement on stepwise data generated with SI-MCTS. 
We use STAIR-SFT to represent the model trained on $\mathcal{D}_\text{CoT}$ with SFT and DPO-k to denote the model after the k-th iteration of self-improvement. Some qualitative examples are displayed in~\cref{sec:appendix_examples}.

First of all, though initially aligned with instruction tuning, the base LLMs remain vulnerable to harmful queries, especially jailbreak attacks. This is evidenced by the goodness scores below 0.40 on StrongReject. We then explore CoT prompting to stimulate the existing reasoning capability in LLMs. While it leads to improvements in reasoning-dependent tasks like GSM8k and InfoFlow, it shows no enhancement in safety. When applying SFT or DPO to the whole dataset $\mathcal{D}$, we observe significant safety-performance trade-offs due to the conflicting objectives. For instance, for both LLaMA-3.1 and Qwen-2 trained with SFT and DPO, their winning rates against GPT-4 on AlpacaEval decline sharply compared to base models. By employing safety-constrained optimization, the trade-off issue is mitigated to a large extent by SACPO, with better safety enhancements compared to previous methods. However, the performance on SimpleQA and InfoFlow degrades, reflecting losses in factual knowledge and over-refusals to benign privacy-related queries. For Self-Rewarding, their improvements on XsTest, which contains queries apparently harmful, are considerable due to the original behaviors of direct refusals in base LLMs. Nevertheless, the behaviors of refusals fail to generalize to jailbreak attacks, as they lack sufficient capabilities to analyze the underlying risks. 

In comparison, STAIR demonstrates more balanced and continuous improvements on diverse benchmarks. With CoT format alignment, the models acquire the basic ability of safety-aware reasoning, enhancing their resilience against harmful inputs. Further training with stepwise preference data generated by SI-MCTS leads to consistent safety enhancements while maintaining or even improving general performance. For example, LLaMA-3.1 achieves an increase of over 20\% in refusal rate on WildChat after three iterations of self-improvement, while its winning rate against GPT-4 on AlpacaEval reaches 38.66\%, a significant improvement compared to 25.55\% for the base model. Similar trends are observed on other benchmarks like SimpleQA and GSM8k. Besides, the accuracy on AdvGLUE is substantially higher than other baselines, highlighting the benefit to robustness from step-by-step reasoning. On StrongReject, both LLMs eventually reach goodness scores of 0.8798 and 0.8486 respectively, which firmly confirm the positive impact of integrating reasoning with safety alignment.

\subsection{Test-time Scaling}

Using the trained process reward model, we investigate the impact of test-time scaling. Since both stepwise and full-trajectory data are used for training, we employ Best-of-N (BoN) and Beam Search, with results presented in~\cref{fig:tts-safe} and~\ref{fig:tts-helpful} for StrongReject and AlpacaEval respectively. Extra computational costs are estimated based on the number of generated steps relative to one-time greedy decoding, expressed in logarithmic form. For example, Bo8 and beam search generating 4 successors with a beam width of 2 correspond to $\log_2(N)=3$. The results indicate that test-time scaling consistently improves both safety and helpfulness. Both searching methods bring improvements of 0.06 for goodness on StrongReject and more than 3.0\% for winning rates on Alpaca.
This supports that the effect of test-time scaling can generalize from math and coding~\cite{snell2024scaling,xie2024self} to more general scenarios like safety.


\begin{figure*}[t]
     \centering
     \begin{minipage}{0.3\textwidth}
         \centering
         \includegraphics[width=\textwidth, trim={1cm 1cm 1cm 1cm}]{images/draft/strongreject.png}
         \vspace{-4ex}
         \caption{Changes in goodness scores on StrongReject with test-time scaling.}
         \label{fig:tts-safe}
     \end{minipage}
     \hfill
     \begin{minipage}{0.3\textwidth}
         \centering
         \includegraphics[width=\textwidth, trim={1cm 1cm 1cm 1cm}]{images/draft/alpaca.png}
         \vspace{-4ex}
         \caption{Changes in winning rates on AlpacaEval when with test-time scaling.}
         \label{fig:tts-helpful}
     \end{minipage}
     \hfill
     \begin{minipage}{0.3\textwidth}
         \centering
         \includegraphics[width=\textwidth, trim={1cm 1cm 1cm 1cm}]{images/draft/balance.png}
         \vspace{-4ex}
         \caption{Results on StrongReject and AlpacaEval as the ratio of safety data varies.}
         \label{fig:data}
     \end{minipage}
        % \caption{Three simple graphs}
        % \label{fig:three graphs}
    \vspace{-1ex}
\end{figure*}


\subsection{Detailed Analysis}

We then conduct some ablation studies to confirm the effectiveness of our framework.

\textbf{Balance between Safety and Helpfulness Data.} To evaluate the impact of the ratio between safety and helpfulness data in the training dataset, we conduct a study during the CoT format alignment stage as a representative. We plot the performance in terms of safety and helpfulness to the varying ratios in~\cref{fig:data}. While a trade-off between safety and helpfulness is observed, consistent with prior findings~\cite{bai2022training}, the performance in both dimensions consistently exceeds that of the base model. This highlights the effectiveness of training with structured CoT data.

\textbf{Step-level Optimization.} To verify the effectiveness of stepwise preference data in the stage of self-improvement, we compare the performance of DPO-1, which is trained on stepwise data based on STAIR-SFT using DPO, with models trained on full trajectory data using either SFT or DPO. The full trajectory data is selected from the same search trees of SI-MCTS, with the total number of training samples kept equal to that of DPO-1. Results in~\cref{tab:iterative} support our strategy of step-level optimization, which brings more fine-grained supervision to safety-aware reasoning.

\textbf{Iterative Training.} We adopt iterative optimization for continuous improvement, motivated by the belief that data generated in later iterations is of higher quality. To validate this, we compare the results of DPO-3 with the model trained using data crafted from all prompts in a single iteration and the model trained on data from the first iteration for three times as many epochs. Results in~\cref{tab:iterative} demonstrate superior improvements on different benchmarks, confirming the improving data quality throughout iterations.




\begin{table}[ht]
\vspace{-1ex}
    \centering
    \caption{Ablation studies on iterative training on stepwise data}
    % \renewcommand{\arraystretch}{1.2} % Increase row height
\resizebox{\linewidth}{!}{%
    \begin{tabular}{l@{\;\,}|@{\;\,}c@{\;\,}c@{\;\,}c@{\;\,}c}
    \toprule[1.5pt]
         & StrongReject & XsTest & GSM8k & AlpacaEval  \\ \midrule
      \multicolumn{5}{c}{Stepwise Data}\\\midrule
      STAIR-SFT + Full (SFT) &  0.6222 & 87.00\% & 85.29\% & 28.10\% \\
      STAIR-SFT + Full (DPO) &  0.6663 & 92.50\% & 86.50\% & 32.87\%\\\midrule
      STAIR-SFT + Step (DPO) & \bf 0.6955 & \bf 94.00\% & \bf 86.81\% & \bf 34.48\% \\\midrule
      \multicolumn{5}{c}{Iterative Training}\\\midrule
      1st Split, 3$\times$ Epochs & 0.6745 & 97.50\%  & 85.75\% & 37.28\% \\
      Full Dataset, 1 Iteration   & 0.7342 & 90.00\%  & 86.58\% & 36.96\%\\\midrule
      STAIR-DPO-3 & \bf 0.8798 & \bf 99.00\% &  \bf 87.64\% & \bf 38.66\% \\\bottomrule[1.5pt]
    \end{tabular}}
    \label{tab:iterative}
    \vspace{-2ex}
\end{table}


\section{How Prompting Methods affect VLMs}
% \yi{TODO: Name this section better}

In this section\footnote{Due to space limits, we reference most case studies, figures, and details in the Appendix within this section.}, we investigate various prompting methods (language-side and vision-side) to evaluate their impact on performance in VLM$^2$-Bench. We select the top 3 performing open-source models (Qwen2.5-VL-8B, InternVL2.5-8B, InternVL2.5-26B), along with GPT-4o, and explore different approaches of CoT~\citep{cot1,cot2} and visual prompting (VP)~\citep{lei2024scaffoldingcoordinatespromotevisionlanguage, vp-zoom-in} (refer to Appendix~\ref{appendix: prompting approaches} for details). The goal is to investigate whether these techniques can improve performance across the benchmark and to identify the underlying factors that contribute to their success or failure. 
% \yi{Some NLP readers may not immediately understand what visual prompting looks like -- some figure in main body of paper could help (you can pick or trim down for the most importantly findings to allot space)}



\subsection{Probing for General Cue (GC)}
\paragraph{Methods.} (i) \textbf{CoT-normal} (Table~\ref{cot prompt}) encourages the model to solve the task step by step, allowing it to reason through the problem. (ii) \textbf{CoT-special} (Table~\ref{cot-special}) guides the model to solve the task using a thought process closer to how humans typically approach it. (iii) \textbf{VP-grid} (Figure~\ref{fig:vp-grid}) is adapted from previous work ~\citep{lei2024scaffoldingcoordinatespromotevisionlanguage} for our tasks, overlaying a dot matrix on the image as visual anchors to provide positional references and enhance the model's performance.

\begin{figure}[t]
  \centering
  \begin{subfigure}[b]{0.48\textwidth}
    \centering
    \includegraphics[width=\textwidth]{img/gc-analysis.png}
    \caption{Results of CoT-normal, CoT-special, and VP-grid on GC.}
    \label{fig:gc-analysis}
  \end{subfigure}
  \\ % This ensures vertical spacing
  \begin{subfigure}[b]{0.48\textwidth}
    \centering
    \includegraphics[width=\textwidth]{img/oc-analysis.png}
    \caption{Results of CoT and VP-zoom-o on OC.}
    \label{fig:oc-analysis}
  \end{subfigure}
  \\ 
  \begin{subfigure}[b]{0.48\textwidth}
    \centering
    \includegraphics[width=\textwidth]{img/pc-analysis.png}
    \caption{Results of CoT and VP-zoom-p on PC.}
    \label{fig:pc-analysis}
  \end{subfigure}
  \caption{Performance \textcolor{teal}{gains} or \textcolor{red}{losses} (\%) when applying different prompting methods on VLM$^2$-Bench.}
  \label{fig:analysis}
\end{figure}
\paragraph{Finding IV: Reasoning in language aids models in logically linking visual cues.} From Figure~\ref{fig:gc-analysis}, it is evident that both CoT-normal and CoT-special, which reasoning in language, positively impact model performance in most cases. As demonstrated in Figure~\ref{fig:cot-special increase}, CoT-special improves performance by first having the model explicitly write out the cues present in each image, followed by using language to make inferences. This process helps reduce the model's error rate by structuring the task and providing clearer logical guidance. This suggests that when models are linking general visual cues, using language to help structure the logical flow of the process can be beneficial.

\paragraph{Finding V: Effectiveness of visual prompting depends on models' ability to interpret both prompting cues and the visual content.} 
As shown in Figure~\ref{fig:gc-analysis}, VP-grid negatively impacts GC performance for QwenVL2.5, causing a significant drop compared to the vanilla approach. Figure~\ref{fig:vp-grid decrease} reveals that this decline stems from the model's difficulty in interpreting the visual coordinates within the prompt, leading to misinterpretation of the cues and causing it to fail cases it originally answered correctly under the vanilla setting. However, as shown in Figure~\ref{fig:vp-grid increase}, GPT-4o successfully resolves a previously incorrect case by effectively leveraging the cues introduced through visual prompting while utilizing its strong visual perception abilities.
% As seen for QwenVL2.5-7B in Figure~\ref{fig:gc-analysis}, regardless of the subtask in GC, the performance drops significantly after using VP-grid compared to the vanilla approach. From Figure~\ref{fig:vp-grid decrease} we can observe that this happens because the model struggles to comprehend the visual coordinates we provided. In such cases, asking the model to infer the meaning of these coordinates leads to confusion and misguides the model. However, as shown in Figure~\ref{fig:vp-grid increase}, GPT-4o successfully answer a previously incorrect case by properly understanding the VP-grid.


\subsection{Probing for Object-centric Cue (OC)}
\paragraph{Methods.} (i) \textbf{CoT} (Table~\ref{cot prompt}). (ii) \textbf{VP-zoom-o} (Figure~\ref{fig:VP-zoom-o}) uses an open-set detector~\citep{grounded-sam} to obtain bounding boxes, which are then cropped to focus the model’s attention on object-centric cues. By eliminating irrelevant non-object cues and emphasizing the object-centric cues, it enhances the model's ability to better focus on the most relevant visual information.



\paragraph{Finding VI: The open-ended nature of language may hinder object grouping.}
Unlike GC that link instance-level cues, OC requires grouping similar objects based on fine-grained visual details. As shown in Figure~\ref{fig:oc-analysis}, InternVL2.5 using CoT struggles with this task because the open-ended nature of language leads to both limited coverage of subtle visual cues (see Figure~\ref{fig:oc_cot_normal_decrease}) and inconsistent representations of the same cues, introducing ambiguity, making it harder for models to reliably align and group matching objects.

\paragraph{Finding VII: Amplifying object cues benefits stronger models while having minimal impact on others.} From Figure~\ref{fig:oc-analysis}, we observe that for models with strong vision capabilities like GPT-4o, our VP-zoom-o method further enhances performance. For other models, this method at least ensures that the performance remains on par with the vanilla approach, without causing any degradation.


\subsection{Probing for Person-centric Cue (PC)} 
\paragraph{Methods.} (i) \textbf{CoT} (Table~\ref{cot prompt}). (ii) \textbf{VP-zoom-p} (Figure~\ref{fig:vp-zoom-p}) utilizes a face detector~\citep{facedetector} to obtain bounding boxes of faces-the most distinguishing feature of different individuals. It then crops the image to focus only on the face, thereby minimizing the interference from distractor cues such as clothing and other background elements.

\paragraph{Finding VIII: CoT and visual prompting fail to improve linking on highly abstract person-centric cues, leading to a performance drop.} From Figure~\ref{fig:pc-analysis}, we observe that for almost all models, neither CoT (language-based) nor VP-zoom-p (vision-based) lead to improved performance. This is because facial features are highly abstract, and CoT methods struggle to effectively describe them in words. Additionally, VP-zoom-p fails because current models' visual capabilities are insufficient to accurately perceive facial features.

% \paragraph{Finding IX: Models in comparison tasks prioritize their knowledge over direct matching.} In Figure~\ref{fig:pc-analysis}, we observe that CoT leads to a significant drop in GPT-4o's performance in \textit{Cpr}. We find that when the model compares whether two images depict the same person, its reasoning chain tends to focus on identifying who each person is first, leading to poor performance for individuals it has not encountered before. However, for \textit{Grp}, the model tends to describe the appearance features of each person. For models with strong visual capabilities like GPT-4o, this approach naturally leads to relatively higher performance improvements.

\section{Related Work}

% Reaction Diffusion
\paragraph{Wave-based Computing}
While prior work on wave-based computing in trainable task-oriented neural networks remains scarce, there is a rich history of using wave-like or other spatiotemporal field dynamics generally for computation.  
Early work studied the ability for waves to perform simple logical operations and thereby compute in a distributed manner \citep{pwc, wave_compute}, while other work has studied the ability for physical water waves to act as literal instantiations of classic `reservoir computers' \citep{maksymov2023analoguephysicalreservoircomputing}. Classically, the domain of `Neural Field Theory' has studied the role of spatiotemporal field dynamics in neural computation from a rigorous mathematical standpoint, although to-date these models have not been adapted to deep-neural network task-oriented performance. We refer readers to \cite{nft} for a thorough review of such models. 

More recently, \cite{hughes2019wave} have noted the analogy between the wave equation and recurrent neural networks, as we have done here, and used this to suggest that wave-based RNNs with learnable wave speeds may perform a type of analog computation. The authors use this to perform acoustic signal classification in a simplified setting, similar to our study in spirit, but differing in how waves are used and their computational purpose. Most related to the present study, \cite{BALKENHOL20244288} use an architecture similar to ours, with a Laplacian recurrent operator, damping, and gating, to show that when provided with an audio signal at a specific spatial location of the network, neurons at more distant locations can perfectly reconstruct the signal. The authors also show that this network is able to reproduce electrical recordings from macaque monkeys in response to simple grating stimuli, hypothesizing that their detection of high frequency waves is highly related to the transfer of information over large cortical distances.  

\looseness=-1
In terms of task-oriented wave-based models, recent work by \cite{felix} extensively studies the computational abilities of oscillatory neural networks, and specifically notes the emergence of traveling waves in these models in response to visual stimuli. Similarly, work by \cite{nwm, wrnn} studies wave-based RNNs for sequence processing and prediction. Our work fundamentally differs from these in the precise study of how these waves may be utilized for the spatial integration of visual information, as is hypothesized to happen in the visual cortex. Furthermore, our work uniquely demonstrates that a timeseries based readout is crucial for performing this type of integration, inspired by Kac's question, opening the door for future novel applications of these models. 

\vspace{-4mm}
\paragraph{Recurrence vs. Depth}
Another relevant line of research concerns the ability to trade off depth for recurrence in CNNs. 
Early work in this area was performed by \citet{liao2020bridginggapsresiduallearning}, with a more extensive recent study performed by \citet{schwarzschild2022the}. The authors demonstrate how iterating a single convolutional layer in a deep CNN yields similar performance to equivalently deep fully untied CNNs. Our work differs from these in that we demonstrate the advantage of a timeseries readout mechanism, inspired by Kac's question, whereas prior work can be seen as using the 'last' hidden state mechanism, that we see underperforms in this work. Interestingly, our findings thus suggest a potential novel method to improve the performance of these recurrent alternatives to deep networks through the use of our readout, a direction we intend to study in future work. Other more machine learning focused work has studied the impact of various weight-sharing schemes in deep convolutional networks \citep{eigen2014understandingdeeparchitecturesusing, jastrzębski2018residualconnectionsencourageiterative, boulch2017sharesnetreducingresidualnetwork}, however these share the same distinction with the present study in terms of their readout mechanism, while our proposed timeseries readouts appear to be uniquely linked to the wave dynamics that emerge in our models. 


\subsubsection{Binding By Synchrony}
Finally, we believe our work shares an interesting connection with the ``binding by synchrony'' concept \citep{Singer:2007} from early neuroscience research. Specifically, while our model's `binding' of parts into wholes does not rely on precise zero-lag synchrony—where oscillators within an object are perfectly in phase, as in the original framework; our method does rely on traveling waves of activity within objects that can be interpreted as a type of phase-lag synchrony. The ``binding operation'' then involves a transformation of the time signal using a suitable linear projection (our proposed timeseries readout). We believe this connection is valuable precisely since it enables a connection with the extensive historical literature on this concept, while simultaneously forming novel predictions on how such phenomena might manifest in natural neural systems. 
On the machine learning side of this concept, our work shares a strong connection with a class of object-centric learning methods which leverage a notion of synchrony of neural activations to define `bound' visual units for computational purposes. This includes models such as complex autoencoders \citep{lowe_complex-valued_2022, lowe_rotating_2024, stanic_contrastive_2024, gopalakrishnan_recurrent_2024} and recent Artificial Kuramoto Oscillatory Neurons (AKOrN) \cite{miyato_artificial_2024}. 
Unlike our method, the waves in the AKOrN model are not used directly as a representation themselves, but instead are neglected through the use of the `last hidden state' readout method. Perhaps most related to our work, \cite{liboni_image_2023} use a complex-valued recurrent neural network designed to generate traveling waves for image segmentation, with binding information encoded in the temporal phase sequence of these waves. This method can indeed be seen as using traveling waves to integrate information spatially, but contains no trainable components, offering a more theoretical exposition to the problem, as opposed to the task-oriented empirical study presented here. 
\section{Takeaways}

Based on our findings, we highlight three key areas for future improvements:

\begin{itemize} \item \textbf{Strengthening Fundamental Visual Capabilities.} Improving core visual abilities not only enhances overall performance but also increases adaptability. A stronger visual foundation maximizes the effectiveness of visual prompting and reduces reliance on prior knowledge, enabling models to operate more independently in vision-centric tasks.

\item \textbf{Balancing Language-Based Reasoning in Vision-Centric Tasks.} Integrating language into vision-centric tasks requires careful calibration. Future research should establish clearer principles on when language-based reasoning aids visual understanding and when it introduces unnecessary biases, ensuring models leverage language appropriately.

\item \textbf{Evolving Vision-Text Training Paradigms.} Current training paradigms focus heavily on emphasizing vision-language associations. However, as models expand their visual context window, their ability to reason purely within the visual domain becomes increasingly crucial. We should prioritize developing models that can structure, organize, and infer relationships among visual cues.
\end{itemize}
\section{Concluding Remarks}
In this paper, we proposed a novel approach utilizing multimodal LLMs to generate gesture-aware speech recognition transcripts for patients with language disorders. Our framework integrates verbal speech and iconic gestures, enabling the generation of enriched transcripts that capture the latent meaning conveyed through both modalities. Through extensive experimentation, we demonstrated that the proposed method effectively contextualizes incomplete or disfluent speech by incorporating gesture information, leading to more accurate and meaningful representations of the speaker's intent. These findings highlight the potential of our approach to significantly contribute to the field of speech and language therapy, offering innovative tools that can enhance the quality of life for individuals with language disorders by facilitating better communication and assessment methods.

\subsection{Ethical Statement} 
Our dataset was obtained from AphasiaBank with the approval of the Institutional Review Board (IRB) and adheres to the data sharing guidelines set by TalkBank\footnote{https://talkbank.org/share/ethics.html}. This includes complying with the Ground Rules for all TalkBank databases, which are based on the American Psychological Association Code of Ethics~\cite{american2002ethical}.

\subsection{Limitation \& Future Work} 
%This study represents a preliminary investigation into using multimodal LLMs to generate gesture-aware speech recognition transcripts. 
While the results are promising, we recognize several limitations and outline our plans to extend this work further.

One primary limitation is the absence of a definitive ground truth for quantitative evaluation. Since our model generates transcripts by synthesizing speech and gesture data from scratch, traditional benchmarks, such as comparisons with standard speech recognition outputs, are insufficient. Moreover, existing original transcripts lack gesture annotations, making direct comparisons challenging. In future work, we aim to address this gap by collaborating with certified pathologists to conduct qualitative assessments, such as A-B preference tests, to evaluate the effectiveness of gesture-enriched transcripts in accurately conveying the speaker's intentions.

To support quantitative evaluations, we plan to develop novel metrics that assess transcript quality, including grammar accuracy, semantic consistency, and the integration of multimodal information. Such metrics will provide a more objective basis for assessing our model's performance and facilitate comparisons with other multimodal and unimodal approaches.

Another limitation of this study is its focus on structured gestures from a specific task, the Peanut Butter Sandwich Task. While this task offers a controlled context for testing our approach, it does not encompass the diversity of gestures and communication patterns seen in everyday scenarios. As part of our future work, we plan to expand the scope of our model to include tasks such as the Cinderella Story Recall Task~\cite{bird1996cinderella}, which involves unstructured and complex narrative gestures. This expansion will allow us to evaluate the adaptability and robustness of our model in handling varied linguistic and gestural contexts.

In summary, while this study establishes a strong foundation for gesture-aware speech recognition, we aim to refine and extend our methods through collaborative qualitative evaluations, the development of robust quantitative metrics, and broader task applications. These efforts will ensure that our approach continues to evolve, ultimately contributing to more effective communication tools and interventions for individuals with language disorders.




\section{Limitations}\label{sec:limitations}
In this work, we introduced a framework for approximating ASR metrics, evaluated across various ASR models and datasets. Despite the promising results, there are several limitations to consider.

\noindent\textbf{Evaluation.} While our evaluation setup is comprehensive, consisting of over 40 models and 14 datasets representing various acoustic and linguistic conditions such as natural noise, dialects, and accents—far surpassing previous works—we have not explored more nuanced conditions such as gender, non-native speech, and approximation across various age groups. Additionally, while the framework has demonstrated strong performance in approximating ASR metrics across multiple datasets, its generalization to highly diverse or extreme real-world conditions might still require further investigation.

\noindent\textbf{Language.} Furthermore, the evaluation is currently limited to a single language; expanding this framework to multiple languages or achieving language-agnostic ASR metric approximation remains an important direction for future work.

\noindent\textbf{Compute.} While, unlike previous works, our final approximator is a simple machine learning model that does not require GPUs to run, we do utilize a single GPU for multimodal embedding extraction, which could be performed on any consumer-grade GPU.



%\section*{Acknowledgments}



% Bibliography entries for the entire Anthology, followed by custom entries
%\bibliography{anthology,custom}
% Custom bibliography entries only
\bibliography{custom}

\appendix
% \section{Possible Data Curation Design - Demo and scripts}
% Question format: Currently, all questions are about "counting problems" (direct) and will ask a model to "describe" the scene as well as hints on object number (indirect).


% Counting object categories:
% \begin{itemize}
%     \item Human action: The question will only focus on human attributes
%     \item Object Movement: Questions on objects
%     % \item Human-object Interaction: Questions regarding object and human
% \end{itemize}





% \subsection{Setting 1: Human Action}
% \textbf{Scripts}: 

% \begin{enumerate}[itemsep=0.1em]
%     \item (Fixed camera) A human face appears in the camera, disappears for a few seconds, comes back in \textbf{different clothing}, and \textbf{still faces the camera}.
%     \item (Fixed camera) A human face appears in the camera, disappears for a few seconds, and comes back in different clothing, but \textbf{back towards} the camera.
%     \item (Fixed camera) A human face appears in the camera, disappears for a few seconds and comes back in different clothing with \textbf{extra facial decoration} (probably a half mask), still facing the camera.
% \end{enumerate}

% \textbf{Visual Cues}: human facial features; hairstyle; makeup; clothing; decorations



% \subsection{Setting 2: Object Movement}

% \textbf{Scripts}: 
% \begin{enumerate}[itemsep=0.1em]
%     \item Several cups are placed on the table. The camera moves away and then returns; the number of cups remains \textbf{unchanged}.
%     \item Several cups are placed on the table. \textbf{A new cup is added off-camera}, and when the camera pans back, the number of cups in view changes.
%     \item Several cups are placed on the table. Off-camera, one cup is \textbf{removed and relocated} to the bottom of the previous frame. When the camera returns, the number of cups in view remains \textbf{unchanged}.
%     \item Several cups are placed on the table. Off-camera, one cup is \textbf{removed and replaced with a different one}. When the camera pans back, the number of cups in view remains \textbf{unchanged}.
%     \item Several cups are placed on the table. While off-camera, the \textbf{liquid content or color} inside the cups changes. When the camera returns, the number of cups in view remains \textbf{unchanged}.
% \end{enumerate}

% \textbf{Visual Cues for cup objects}: shape; color; material texture; liquid color and amount;




% \subsection{Setting 3: Human-Object Interaction}

%\clearpage
%\newpage
\onecolumn
\section{Appendix Outline}
In the appendix, we provide:
\begin{itemize} \item \textbf{Appendix \ref{appendix: licence}} provides details on the licensing terms and usage rights for our benchmark.

\item \textbf{Appendix \ref{appendix: statistics}} presents the statistical analysis of the VLM$^2$-Bench.

\item \textbf{Appendix \ref{appendix: baselines}} details on how we obtain the chance-level and human-level baselines.

\item \textbf{Appendix \ref{appendix: more details on benchmark construction}}  elaborates more details on the construction of the VLM$^2$-Bench.

\item \textbf{Appendix \ref{appendix: prompting approaches}} provides a deeper dive into the various prompting techniques we use.

\item \textbf{Appendix \ref{Appendix: case study}} a detailed breakdown and analysis of failure and success examples regarding different prompting methods.
\end{itemize}



\section{Licencing and Intended Use}
\label{appendix: licence}
Our VLM$^2$-Bench is available under the CC-BY 4.0 license for academic use with proper attribution. The images, videos, and annotations in this benchmark are intended solely for research purposes. These data were sourced from publicly available online platforms, and while efforts were made to use them responsibly, explicit permissions may not have been obtained for all content. Users are responsible for ensuring that their use of the data complies with applicable intellectual property laws and ethical guidelines. We encourage users to verify the sources and ensure compliance with any terms of service or licensing agreements.


\section{VLM$^2$-Bench Statistics}
\label{appendix: statistics}
% 1 paragraph (maybe some figures)
% overall statistics for all tasks.

% OC: how many images in total; sequences, questions...

% PC: how many images in total; sequence, questions...

Here we provide additional details regarding the construction and statistics of our \textbf{VLM$^2$-Bench} benchmark. As described in the main paper (\S~\ref{bench_statistics_main}), our benchmark comprises three main categories---\textit{General Cue (GC)}, \textit{Object-centric Cue (OC)}, and \textit{Person-centric Cue (PC)}---with a total of 3,060 visual-text query pairs. Below, we elaborate on the specific data composition, including the distribution of question types (T/F, multiple-choice (MC), numerical (Nu), and open-ended (Oe)) and the rationale behind each subtask.

\subsection{Overall Composition}
\begin{wrapfigure}{r}{0.5\textwidth} % 表格靠右,宽度为文本宽度的50%
\centering
\footnotesize % 缩小表格字体
\setlength{\tabcolsep}{4pt} % 调整列间距
\begin{tabular}{@{}lccccc@{}}
\toprule
\textbf{Category} & \textbf{T/F} & \textbf{MC} & \textbf{Nu} & \textbf{Oe} & \textbf{Total} \\
\midrule
GC & 960 & -- & -- & -- & 960 \\
OC & 720 & 200 & 360 & -- & 1,280 \\
PC & 400 & 100 & 120 & 200 & 820 \\
\midrule
Total & 2,080 & 300 & 480 & 200 & 3,060 \\
\bottomrule
\end{tabular}
\caption{Overview of query distribution across the three categories of VLM$^2$-Bench. T/F = True/False, MC = multiple-choice, Nu = numerical, Oe = open-ended.}
\label{tab:overall_stats}
\end{wrapfigure}
Table~\ref{tab:overall_stats} provides a detailed summary of the total query counts across different categories and subtasks in our benchmark. The dataset is structured into three primary categories: General Cue (GC), Object-centric Cue (OC), and Person-centric Cue (PC), comprising a total of 3,060 visual-text query pairs.

The General Cue (GC) category consists of 960 queries, which include 260 Matching (Mat) true/false pairs, resulting in 520 queries, and 220 Tracking (Trk) true/false pairs, leading to 440 queries. 

The Object-centric Cue (OC) category contains 1,280 queries, covering three subtasks: Comparison (Cpr) with 360 true/false pairs (720 queries), Counting (Cnt) with 360 numerical queries, and Grouping (Grp) with 200 multiple-choice questions. 

Lastly, the Person-centric Cue (PC) category includes 820 queries, comprising 200 Comparison (Cpr) true/false pairs (400 queries), 120 Counting (Cnt) numerical queries, 100 Grouping (Grp) multiple-choice questions, and 200 Free-form (VID) open-ended queries.

Overall, these components collectively sum up to 3,060 visual-text query pairs, offering a comprehensive benchmark for evaluating vision-language models across various types of contextual cues.
% Table~\ref{tab:overall_stats} summarizes the total query counts within each category and subtask. The benchmark is divided as follows:
% \begin{itemize}
%     \item \textbf{General Cue (GC):} 960 queries  
    
%           - \emph{Matching (Mat)}: 260 T/F pairs $\rightarrow$ 520 queries  
          
%           - \emph{Tracking (Trk)}: 220 T/F pairs $\rightarrow$ 440 queries  
%     \item \textbf{Object-centric Cue (OC):} 1,280 queries  
    
%           - \emph{Comparison (Cpr)}: 360 T/F pairs $\rightarrow$ 720 queries  
          
%           - \emph{Counting (Cnt)}: 360 numerical (counting) queries 
          
%           - \emph{Grouping (Grp)}: 200 multiple-choice questions  
%     \item \textbf{Person-centric Cue (PC):} 820 queries  
    
%           - \emph{Comparison (Cpr)}: 200 T/F pairs $\rightarrow$ 400 queries  
          
%           - \emph{Counting (Cnt)}: 120 numerical (counting) queries 
          
%           - \emph{Grouping (Grp)}: 100 multiple-choice questions  
          
%           - \emph{Free-form (VID)}: 200 open-ended queries  
% \end{itemize}

% Summing these yields a total of 3,060 visual-text query pairs.


% \begin{table}[t]
%     \centering
%     \small
%     \setlength{\tabcolsep}{4pt} % Reduce column spacing slightly
%     \renewcommand{\arraystretch}{1.1} % Adjust row spacing slightly
%     \resizebox{\columnwidth}{!}{ % Scale the table to fit within text width
%     \begin{tabular}{lcccccc}
%     \toprule
%     \textbf{Category} & \textbf{Subtask} & \textbf{T/F} & \textbf{MC} & \textbf{Nu} & \textbf{Oe} & \textbf{Total} \\
%     \midrule
%     \multirow{2}{*}{GC} & Mat & 520 & -- & -- & -- & 520 \\
%                         & Trk & 440 & -- & -- & -- & 440 \\
%     \cmidrule(lr){2-7}
%     & \textbf{Subtotal} & 960 & -- & -- & -- & 960 \\
%     \midrule
%     \multirow{3}{*}{OC} & Cpr & 720 & -- & -- & -- & 720 \\
%                         & Cnt & -- & -- & 360 & -- & 360 \\
%                         & Grp & -- & 200 & -- & -- & 200 \\
%     \cmidrule(lr){2-7}
%     & \textbf{Subtotal} & 720 & 200 & 360 & -- & 1{,}280 \\
%     \midrule
%     \multirow{4}{*}{PC} & Cpr & 400 & -- & -- & -- & 400 \\
%                         & Cnt & -- & -- & 120 & -- & 120 \\
%                         & Grp & -- & 100 & -- & -- & 100 \\
%                         & VID & -- & -- & -- & 200 & 200 \\
%     \cmidrule(lr){2-7}
%     & \textbf{Subtotal} & 400 & 100 & 120 & 200 & 820 \\
%     \midrule
%     \textbf{Total} & & 2{,}480 & 300 & 480 & 200 & 3{,}060 \\
%     \bottomrule
%     \end{tabular}
%     } % End of resizebox
%     \caption{Overview of query distribution across the three categories of VLM$^2$-Bench. T/F = True/False, MC = multiple-choice, Nu = numerical, Oe = open-ended.}
%     \label{tab:overall_stats}
% \end{table}


\subsection{Details per Subtask and Question Type}

\paragraph{General Cue (GC).}
\mbox{}\par
\noindent
\texttt{Matching (Mat).} We collect 260 True/False (T/F) pairs focused on verifying the alignment between a visual instance and a textual description (e.g., object presence, basic attributes). Each T/F pair forms two distinct queries (one True, one False), yielding 520 queries in total. 

\noindent
\texttt{Tracking (Trk).} We design 220 T/F pairs that test an understanding of object or entity continuity across frames. For example, a question might ask whether the same object reappears in subsequent frames. Each T/F pair similarly results in two queries, totaling 440.

\paragraph{Object-centric Cue (OC).} All the visual query cases are built upon the 360 image sequences we construct. Details about image sequences can be found in Section~\ref{appendix_oc_construct}.

\noindent
\texttt{Comparison (Cpr).} This subtask examines the model's ability to compare object properties (e.g., size, color, quantity) across different frames. We produce 360 T/F pairs, each yielding two queries (720 total). Among these 360 pairs, we maintain a 1:2 ratio of True to False for ground-truth answers (i.e., 120 True vs.\ 240 False).

\noindent
\texttt{Counting (Cnt).} We provide 360 numerical questions, each asking for a count of objects in a given scene or sequence. Possible numeric answers are typically small integers (e.g., 1, 2, 3), reflecting the number of relevant objects.

\noindent
\texttt{Grouping (Grp).} We generate 200 multiple-choice (MC) questions that ask about grouping objects according to certain criteria (e.g., AAB, ABC, AAAB, AABC, ABCD). Each question presents multiple group-configuration options plus a \textit{``None''} option, which can serve as either a correct or distractor choice. For image sequences of length 4, the options include various plausible groupings (two-of-a-kind, three-of-a-kind, etc.) along with at least one additional distractor grouping that also involves three-of-a-kind to ensure sufficient challenge.

\paragraph{Person-centric Cue (PC).} Similar to OC, the construction of 260 image sequences as well as 200 video clips for PC is detailed in Section~\ref{appendix_pc_construct}.

\par

\noindent
\texttt{Comparison (Cpr).} We create 200 T/F pairs (400 queries total) focusing on comparing attributes or actions related to one or more human individuals across multiple images in a sequence. The ground truth is balanced at 100 True vs.\ 100 False.

\noindent
\texttt{Counting (Cnt).} This subtask involves 120 numerical questions asking for the number of people present or the frequency of certain actions in a sequence. Typical numeric answers range from 1 to 4, given the scope of each visual sequence.

\noindent
\texttt{Grouping (Grp).} We provide 100 MC questions based on sequences containing at least three images, with at least two images featuring the same main ``meta-human.'' The goal is to identify correct groupings of persons based on appearance, role, or action. As with \textit{OC-Grp}, each question includes a ``None'' option as either the correct or a distractor choice.

\noindent
\texttt{Open-ended (VID).} We introduce 200 open-ended queries that focus on various person-centric aspects, such as identifying roles or describing activities. These questions allow more flexibility in model responses and assess the ability to generate context-relevant answers.

\subsection{Annotation Quality and Agreement}
As noted in the main text, three annotators reviewed all 3,060 question-answer pairs. An inter-annotator agreement study showed a high consensus rate of 98.74\%, ensuring that the data is both accurate and consistent.

\subsection{Summary}
Our construction methodology ensures a balanced coverage of both object-centric and person-centric reasoning, as well as basic general cues such as element matching and tracking. The inclusion of multiple question types (T/F, MC, numerical, and open-ended) further promotes comprehensive evaluation of vision-language models. Figure~\ref{fig:bench statistics main} in the main paper illustrates the distribution of these subtasks and their question-format breakdown. We believe that the richness and diversity of VLM$^2$-Bench make it a robust platform for advancing multimodal research.



%\newpage
\section{Baselines}
\label{appendix: baselines}
\subsection{Chance-level}
In this part, we explain the calculation of chance-level accuracy for all subtasks in Table~\ref{exp:main_exp}.

\paragraph{GC-Mat, GC-Trk.} % 1/2*1/2
The Matching (Mat) and Tracking (Trk) tasks in General Cue (GC) follow a \textbf{True-False (TF) paired-question format}, where each pair consists of a \textbf{positive question} and a \textbf{negative question}:

\begin{itemize}
    \item \textbf{Positive Question}: Derive from the correct \textit{element} or \textit{change}.
    The ground truth (GT) answer is True (T).
    \item \textbf{Negative Question}: Derive from the distractor \textit{element} or \textit{change}.
    The ground truth (GT) answer is False (F).
\end{itemize}

A question pair example is shown in Table~\ref{tf_pair_example}.

\begin{table}[h]
    \begin{tcolorbox}[colframe=black, colback=gray!10!white, coltitle=black, boxrule=0.5mm]
    \textbf{Positive Question:}  
    
    \textit{"Is the answer \textcolor{customgreen}{`the salad'} correct for the given question: 'What object that was present in the first image is no longer visible in the second?'"}  
    \\ GT Answer: \textbf{\textcolor{customgreen}{T}}

    \vspace{3mm}
    
    \textbf{Negative Question:}  
    
    \textit{"Is the answer \textcolor{customred}{`the ciabatta roll'} correct for the given question: 'What object that was present in the first image is no longer visible in the second?'"}  
    \\ GT Answer: \textbf{\textcolor{customred}{F}}
    \end{tcolorbox}
    \caption{Example of True-False paired questions in GC-Mat, with a positive and negative question.}
    \label{tf_pair_example}
\end{table}


During the construction of these questions, we ensure that the queried content originates from either the correct answer or a distractor answer. These elements are designed to be \textbf{independent and identically distributed}. Since each question in the pair has an independent 50\% chance of being answered correctly, the expected accuracy under random guessing would be $P(\text{correct answer}) = \frac{1}{2} \times \frac{1}{2} = \frac{1}{4} = 25\%$.





\paragraph{OC-Cpr, PC-Cpr.} % 1/2

The OC-Cpr and PC-Cpr tasks utilize a \textbf{True-False (TF) paired-question format} where both questions in a pair originate from the same correct answer but are framed in two different ways:

\begin{itemize}
    \item \textbf{Positive Question}: A direct affirmative statement that correctly represents the ground truth.
    \item \textbf{Negative Question}: A negated version of the positive question, often by inserting "not" after the verb.
\end{itemize}

An example is shown in Table~\ref{tf_cpr_example}.

\begin{table}[h]
    \begin{tcolorbox}[colframe=black, colback=gray!10!white, coltitle=black, boxrule=0.5mm]
    \textbf{Positive Question:}  
    
    \textit{"Given the images, the claim `The pets in these images \textcolor{customgreen}{are} the same pet.' is right."}  
    \\ GT Answer: \textbf{\textcolor{customgreen}{T}}

    \vspace{3mm}

    \textbf{Negative Question:}  
    
    \textit{"Given the images, the claim `The pets in these images \textcolor{customred}{are not} the same pet.' is right."}  
    \\ GT Answer: \textbf{\textcolor{customred}{F}}
    \end{tcolorbox}
    \caption{Example of True-False paired questions in OC-Cpr, with a positive and negative question.}
    \label{tf_cpr_example}
\end{table}



This construction aims to eliminate \textbf{language bias} by ensuring that the model does not favor one phrasing over another. For a language model that is free from bias, these two questions are \textbf{logically equivalent}—answering one correctly implies answering the other correctly as well. Consequently, under random guessing, the expectation is $P(\text{correct answer}) = \frac{1}{2} = 50\%$.


\paragraph{OC-Cnt, PC-Cnt.}
The calculation formulas for the accuracy of the chance-level accuracy are the same as in Section \ref{metrics}.





Under a pure random guessing strategy, the predicted answer \( \hat{N}_i \) is uniformly sampled from the set \(\{1,2,\ldots,L\}\), where \(L\) is the number of images (i.e., the sequence length for that instance). For a fixed sequence length \(L\), we can compute the expected normalized accuracy \(E(L)\) by averaging over all possible ground-truth and guess pairs:
\[
E(L) = 1 - \frac{1}{L^2} \sum_{N=1}^{L} \sum_{\hat{N}=1}^{L} w(L) \cdot \epsilon(N,\hat{N})^{-\alpha},
\]
where
\[
\epsilon(N,\hat{N}) = \frac{|\hat{N} - N|}{\max(N-1,\, L-N)}
\]
and the weight is defined as
\[
w(L) = \frac{L_{\max}}{L},
\]
with \( L_{\max} = 4 \) being the maximum sequence length in our dataset.

\textbf{OC-Cnt Task:} The OC-Cnt task exhibits the following distribution:
\begin{itemize}
    \item Length 2: 80 sequences (22.2\%)
    \item Length 3: 120 sequences (33.3\%)
    \item Length 4: 160 sequences (44.4\%)
\end{itemize}
Thus, the overall chance level accuracy is obtained as the weighted average:
$
Acc_{\text{OC-Cnt}} = \frac{80\,E(2) + 120\,E(3) + 160\,E(4)}{360} \approx 34.88\%.
$

\textbf{PC-Cnt Task:} For the PC-Cnt task, the sequence distribution is:
\begin{itemize}
    \item Length 2: 30 sequences (25.0\%)
    \item Length 3: 25 sequences (20.8\%)
    \item Length 4: 65 sequences (54.2\%)
\end{itemize}
Accordingly, the overall chance level accuracy is given by:
$
Acc_{\text{PC-Cnt}} = \frac{30\,E(2) + 25\,E(3) + 65\,E(4)}{120} \approx 34.87\%.
$


% In summary, under a random guessing strategy and using the above normalized error formulation, the expected (chance-level) accuracies are approximately 34.88\% for the OC-Cnt task and 34.87\% for the PC-Cnt task.





\subsection{Human-level}
To facilitate human participants in providing responses to our questions, we integrated all model-prompted questions and answer choices into a graphical user interface (GUI), as illustrated in Figure~\ref{fig: gui_human}. This interface enabled participants to select their answers conveniently, ensuring consistency in data collection. We then gathered all responses and conducted statistical analysis on the collected human evaluations.



\begin{figure*}[htbp]
  \centering
  \includegraphics[width=0.85\textwidth]{img/GUI_human.png} 
  % \vspace{-0.8cm}
  \caption{The GUI used for human-level testing.}
  \label{fig: gui_human}
\end{figure*}





\clearpage
\newpage
\section{More details on Benchmark Construction}
\label{appendix: more details on benchmark construction}

\subsection{GC (General Cue)}
\paragraph{Manual Screening and Refine.}  

Figure~\ref{fig: gui} demonstrates the Graphic User Interface (GUI) we build for manually screening image editing data.






\begin{figure*}[htbp]
  \centering
  \includegraphics[width=0.85\textwidth]{img/gui.jpg} 
  % \vspace{-0.8cm}
  \caption{The GUI used for manually screening image editing data and refining edited prompts in General Cue (GC).}
  \label{fig: gui}
\end{figure*}

\paragraph{Salient Sampling.}

The pseudocode in Figure~\ref{fig:salient score algorithm} and Table~\ref{template salient score} displays the calculation process for the salient sampling score mentioned in Section~\ref{gc}.

\begin{table*}[htbp]
    \begin{tcolorbox}[colframe=black, colback=gray!10!white, coltitle=black, boxrule=0.5mm]
    Supposed you are looking at two images:

    Image 1: \textcolor{teal!70}{\textbf{<Cap\_{src}>}}

    Image 2: \textcolor{orange!70}{\textbf{<Cap\_{edit}>}}

    From Image 1 to Image 2, the change can be summarized as: \textcolor{blue!70}{\textbf{<P>}}
    \end{tcolorbox}
    \caption{Template for salient-score calculation, which contain three placeholders for each sample.}
    \label{template salient score}
\end{table*}

\begin{figure}[htbp]
\centering
\begin{minipage}{0.90\textwidth} 
    \begin{algorithm}[H]
    \caption{\footnotesize Salient Score Computation}
    \begin{lstlisting}[language=Python]
# cap_src: caption for the source image
# cap_edit: caption for the edited image
# T: template for constructing a paragraph
# P: editing prompt
input_text = concat(cap_src, cap_edit, T)
in_tokens = tokenizer.encode(input_text)
out_tokens = tokenizer.encode(P)
log_sum = 0
tokens = in_tokens

# Model Forward Pass
for i in range(1, len(out_tokens)):
    outputs = model(tokens)
    logits = outputs.logits

    # Extract log probability of next token
    probs = log_softmax(logits[0, -1, :])
    prob = probs[out_tokens[i]]
    log_sum += prob
    
    # Update Input Sequence
    tokens = concat(tokens, out_tokens[i])

# Normalize the total log probability as the salient_score
salient_score = log_sum / len(out_tokens)

# Return: salient_score
    \end{lstlisting}
\end{algorithm}
\end{minipage}
\caption{Pseudocode for salient score computation in the phrase of Salient Sampling in the construction of GC.}
\label{fig:salient score algorithm}
\end{figure}


\paragraph{Prompts for Pair-wise Answer Generation.}

Table~\ref{tab:mat_pair_generation_prompt} and \ref{tab:trk_pair_generation_prompt}
provides the complete prompts used to generate pair-wise answers for our evaluation tasks. The prompts were designed to instruct the language model to produce two distinct answers—a positive (T) answer and a negative (F) answer—for each task. The dual-answer format is intended to capture both the expected response and its direct opposite, thereby offering a more balanced insight into the model's understanding.




\begin{table*}[htbp]
    \begin{tcolorbox}[colframe=black, colback=gray!10!white, coltitle=black, boxrule=0.5mm]
\textbf{\#Task Description}

Given the change between the first image and the second image, you need to generate four choices to the question ``What new element can be observed in the second image that was not present in the first?" \textcolor{blue}{(this question varies based on the mis-matched cue types, here shows the question for `add" from the ``Add/Remove" category)}. Remember, the choices' lengths should be similar. Additionally, your response should start with "\textit{Choices:}".

\textbf{\#Pair Design}

In these two choices, you need to contain *only* the names of objects, but be specific:

1. Correct Answer (You need to infer the *only* from the \textit{Editing Information})

2. Distractor (You need to pick a random object *only* in the \textit{Description}, but differ from the correct answer object)

\textbf{\#In-context example}

\textit{Editing Information}: 

Add a katana held in the figure's left hand, angled downwards.

\textit{Description}: 

The image depicts a person dressed in traditional Japanese armor, standing in a misty, snowy landscape. The armor is detailed and appears to be made of metal, with various straps and buckles. The person is wearing a black mask that covers their entire face, adding to the mysterious and stealthy appearance. The background features stone lanterns and other traditional Japanese structures, which are partially obscured by the mist. The overall atmosphere is serene yet somewhat eerie, with the mist adding a sense of mystery and isolation. The scene suggests a historical or fantasy setting, possibly a samurai or ninja in a snowy, misty environment.

\textit{Choices}:

Correct Answer: katana held

Distractor: black mask

\textbf{\#Task}

\textit{Editing Information}: 

\textcolor{teal}{\textbf{<Edit Prompt>}}

\textit{Description}: 

\textcolor{orange}{\textbf{<Description>}}

    \end{tcolorbox}
    \caption{Prompt for generating paired answers in the Matching (Mat) subtask of General Cue (GC).}
    \label{tab:mat_pair_generation_prompt}
\end{table*}



\begin{table*}[htbp]
    \begin{tcolorbox}[colframe=black, colback=gray!10!white, coltitle=black, boxrule=0.5mm]
\textbf{Task Description}

Given the change between the first image and the second image, you need to generate four choices to the question "What key visual difference can be observed from the first image to the second image?". Remember, the choices' lengths should be similar. Additionally, your response should start with ``\textit{Choices}:" and must contain Correct Answer and Direct Reverse Answer.

\textbf{Pair Design}

In the two choices, you need to contain:

1. Correct Answer (You need to infer from the \textit{Editing Information})

2. Direct Reverse Answer (You need to infer from the \textit{Editing Information} and change it to the opposite)

\textbf{In-context example} 

\textit{Editing Information}: 

Swap the black ninja gloves with clean white gloves appropriate for serving.

\textit{Description}: 

The image depicts a person dressed in formal attire, standing in a doorway. The individual is wearing a black tuxedo with a white dress shirt and a black bow tie. They are holding a tray with several items on it. The tray contains a small glass container, a bottle, and a small white object, possibly a salt shaker or a similar item. The person is also wearing black gloves, which are typical for serving or formal dining scenarios. The background shows a wooden door with a brass hinge and a light-colored wall. The setting appears to be indoors, possibly in a house or a formal establishment.

\textit{Choices}:

Correct Answer: The black ninja gloves were replaced with clean white gloves.

Direct Reverse Answer: The clean white gloves were replaced with black ninja gloves.

\textbf{\#Task}

\textit{Editing Information}: 

\textcolor{teal}{\textbf{<Edit Prompt>}}

\textit{Description}: 

\textcolor{orange}{\textbf{<Description>}}

    \end{tcolorbox}
    \caption{Prompt for generating paired answers in the Tracking (Trk) subtask of General Cue (GC).}
    \label{tab:trk_pair_generation_prompt}
\end{table*}





% \paragraph{Question Templates.}

% Table \ref{gc_mat_tf_example} and \ref{gc_trk_tf_example} list detailed standard question templates for General Cue - Matching and Tracking tasks, including the format instruction prompt.

% \begin{table}[h]
%     \begin{tcolorbox}[colframe=black, colback=gray!10!white, coltitle=black, boxrule=0.5mm]
%     \textbf{GC-Mat Positive Question:}  
    
%     \textit{"Is the answer \textcolor{customgreen}{'correct element'} correct for the given question: 'What new element can be observed in the second image that was not present in the first?'"}  
%     \\ GT Answer: \textbf{True (T)}

%     \vspace{3mm}
    
%     \textbf{GC-Mat Negative Question:}  
    
%     \textit{"Is the answer \textcolor{customred}{`distractor element'} correct for the given question: 'What new element can be observed in the second image that was not present in the first?'"}  
%     \\ GT Answer: \textbf{False (F)}
%     \end{tcolorbox}
%     \caption{GC-Mat True-False paired-question}
%     \label{gc_mat_tf_example}
% \end{table}

% \begin{table}[h]
%     \begin{tcolorbox}[colframe=black, colback=gray!10!white, coltitle=black, boxrule=0.5mm]
%     \textbf{GC-Trk Positive Question:}  

%     \textit{"Is the answer \textcolor{customgreen}{'correct change'} correct for the given question: 'What key visual change can be observed from the first image to the second image?'"}  
%     \\ GT Answer: \textbf{True (T)}

%     \vspace{3mm}
    
%     \textbf{GC-Trk Negative Question:}  
    
%     \textit{"Is the answer \textcolor{customred}{'distractor change (reversed process)'} correct for the given question: 'What key visual change can be observed from the first image to the second image?'"}  
%     \\ GT Answer: \textbf{False (F)}
%     \end{tcolorbox}
%     \caption{GC-Trk True-False paired-question }
%     \label{gc_trk_tf_example}
% \end{table}






\subsection{OC (Object-centric Cue)}
\label{appendix_oc_construct}
% \paragraph{Data Collection.} how to collect: xxx Figure~\ref{fig:oc overview}

\paragraph{Data Collection.} 
To construct the dataset, we follow a structured approach to collect object-centric images, as illustrated in Figure~\ref{fig:oc overview}. In total, we manually collected 320 images for objects.



\paragraph{Main Meta-Object Selection.} 
We predefine 8 types of common objects, with each type containing 5 meta-objects. For each meta-object, we collect four images that represent the same object from different angles and scene conditions.

\paragraph{Distractor Meta-Object Selection.} 
To build meaningful object image sequences, we introduce visually distractive elements for each main meta-object, referred to as ``distractor meta-objects''. Specifically, for each main meta-object, we collect four additional images that belong to different but visually similar meta-objects within the same object category. These images are selected following predefined visual cue confusion principles, ensuring that they provide meaningful challenges for vision language models. We ensure that each distractor image belongs to a different distractor meta-object, fundamentally guaranteeing that the count of different meta-objects in the final constructed sequence strictly follows our design. The principle of selecting distractor meta-objects is illustrated in the outer ring of Figure~\ref{fig:oc overview}. 





\paragraph{Image Sources.} 
The images are gathered from various sources based on the nature of the objects:
\begin{itemize}
    \item \textbf{Plush Objects:} Images of plush toys are entirely sourced from the \href{https://us.jellycat.com/}{Jellycat website} and its review sections, where diverse user-uploaded images provide a wide variety of object angles and scenes.
    \item \textbf{Pet Objects:} For the pet category of meta-objects, we source images from a combination of social media accounts of popular pet influencers' pet photography. We also include images of a ragdoll cat owned by one of the authors. As a result, this approach guarantees that each pet meta-object within the dataset belongs to the same individual cat or dog, minimizing variability unrelated to visual cue confusion.
    \item \textbf{Other Objects:} Most images are collected from \href{https://www.amazon.com/}{Amazon} product listings and review sections containing user-uploaded photos. A smaller portion of the dataset is curated using Google Lens image search, where specific visual distractive cues are used to retrieve and manually select images. The detailed visual cue principles guiding this selection process can be found in Figure~\ref{fig:oc overview}.
\end{itemize}


\begin{figure*}[htbp]
  \centering
  \includegraphics[width=0.99\textwidth]{img/pc-o-all-cases.pdf} 
  % \vspace{-0.8cm}
  \caption{The overview of the structured design of the Object-centric Cue (OC) images.
 \textbf{Central Layer (Main Meta-Objects)}: The innermost circle represents the predefined \textbf{8 object categories}, which serve as the foundation for our dataset. These categories include \textit{Pet, Plush, Bag, Book, Cup, Shirt, Shoes, and Toy}. Each category consists of 4 main meta-objects.
    \textbf{Middle Layer (Example Meta-Objects within Each Category)}: Each segment surrounding the center showcases a representative \textbf{main meta-object} within its category. These meta-objects serve as core instances for data collection. For example, the \textit{Pet} category includes \textit{Cat} and \textit{Dog}, while the \textit{Bag} category includes \textit{Backpack}, \textit{Schoolbag} and \textit{Fashion Bag}.
    \textbf{Outer Layer (Distractor Meta-Objects \& Visual Cue Distraction Principles)}: The outermost ring presents 1 out of 4 \textbf{distractor meta-objects} specifically selected to create challenging image sequences. Each distractor meta-object shares one or more \textbf{distractive visual cues} with its corresponding main meta-object.
  }
  \label{fig:oc overview}
  % \vspace{-4em}
\end{figure*}







\paragraph{Images Sequence Construction.} 
The construction of image sequences in OC (a total of 360 sequences) follows the structure in Table~\ref{tab:sequence construction oc}. More specific details are listed below:

\textbf{Two-Image Sequences (\texttt{image\_seq\_len} = 2)}
\begin{enumerate}
    \item \textbf{Main Meta-Object Only (AA)}:
        Two images are randomly sampled from the same main meta-object.  
        40 sequences are constructed (one for each main meta-object).
    \item \textbf{Main Meta-Object + Distractor Meta-Object (AB)}:
        One image is randomly selected from the main meta-object, and one from the corresponding distractor meta-object.  
        40 sequences are constructed.
\end{enumerate}

\textbf{Three-Image Sequences (\texttt{image\_seq\_len} = 3)}
\begin{enumerate}
    \item \textbf{Main Meta-Object Only (AAA)}:
        Three images are randomly sampled from the same main meta-object.  
        40 sequences are constructed.
    \item \textbf{Main Meta-Object + Distractor Meta-Object (AAB)}:
        Two images are selected from the main meta-object, and one from the distractor meta-object.  
        The order of images is shuffled.  
        40 sequences are constructed.
    \item \textbf{Main Meta-Object + Distractor Meta-Objects (ABC)}:
        One image is selected from the main meta-object, while two are selected from different distractor meta-objects.  
        40 sequences are constructed.
\end{enumerate}

\textbf{Four-Image Sequences (\texttt{image\_seq\_len} = 4)}
\begin{enumerate}
    \item \textbf{Main Meta-Object Only (AAAA)}:
        All four images are sampled from the same main meta-object and shuffled.  
        40 sequences are constructed.
    \item \textbf{Main Meta-Object + Distractor Meta-Object (AAAB)}:
        Three images are sampled from the same main meta-object, while one is selected from a distractor meta-object.  
        40 sequences are constructed.
    \item \textbf{Main Meta-Object + Distractor Meta-Objects (AABC)}:
        Two images are selected from the main meta-object, while two are selected from different distractor meta-objects.  
        40 sequences are constructed.
    \item \textbf{Main Meta-Object + Distractor Meta-Objects (ABCD)}:
        One image is selected from the main meta-object, while three are selected from different distractor meta-objects.  
        40 sequences are constructed.
\end{enumerate}















\begin{table*}[h]
\centering
\begin{tabularx}{\textwidth}{ccYccc}
\toprule
\textbf{Num} & \textbf{Src} & \textbf{Process of Image Sequences Construction} &  \textbf{\textit{Cpr}} & \textbf{\textit{cnt}} & \textbf{\textit{Grp}} \\ \hline
\multirow{2}{*}{2} & \multirow{2}{*}{\textbf{AA}} & 
2 images from the same object \( O_i \), randomly sampled as \( \mathcal{I}_{O_i} = \{ I_i, I_j \} \), and shuffled. 
& \multirow{2}{*}{T} & \multirow{2}{*}{2} & \multirow{2}{*}{-}  \\ \hline
\multirow{2}{*}{2} & \multirow{2}{*}{\textbf{AB}} & 
1 image \( I_{i} \) from \( \mathcal{I}_{O_i} \) and 1 image \( I_{\neg i} \) from distractor set \( \mathcal{I}_{\neg O_i}\), randomly shuffled. 
& \multirow{2}{*}{F} & \multirow{2}{*}{1} & \multirow{2}{*}{-}  \\ \hline
\multirow{2}{*}{3} & \multirow{2}{*}{\textbf{AAA}} & 
3 images from the same object \( O_i \), randomly sampled as \( \mathcal{I}_{O_i} = \{ I_i, I_j, I_k \} \), and shuffled. 
& \multirow{2}{*}{T} & \multirow{2}{*}{3} & \multirow{2}{*}{-}  \\ \hline
\multirow{3}{*}{3} & \multirow{3}{*}{\textbf{AAB}} & 
2 images from the same object \( O_i \), randomly sampled as \( \mathcal{I}_{O_i} = \{ I_i, I_j\} \) and 1 \( I_{\neg i} \) from distractor set \( \mathcal{I}_{\neg O_i}\), randomly shuffled. 
& \multirow{3}{*}{F} & \multirow{3}{*}{2} & \multirow{3}{*}{[$I_i$, $I_j$]}  \\ \hline
\multirow{3}{*}{3} & \multirow{3}{*}{\textbf{ABC}} & 
1 images from the same object \( O_i \), randomly sampled as \( \mathcal{I}_{O_i} = \{ I_i\} \) and 2 images \( \{I_{\neg i}, I_{\neg j}\} \) from distractor set \( \mathcal{I}_{\neg O_i}\), randomly shuffled. 
& \multirow{3}{*}{F} & \multirow{3}{*}{3} & \multirow{3}{*}{[]}  \\ \hline
\multirow{2}{*}{4} & \multirow{2}{*}{\textbf{AAAA}} & 
4 images from the same object \( O_i \), randomly sampled as \( \mathcal{I}_{O_i} = \{ I_i, I_j, I_k, I_p  \} \), and shuffled. 
& \multirow{2}{*}{T} & \multirow{2}{*}{4} & \multirow{2}{*}{-}  \\ \hline
\multirow{3}{*}{4} & \multirow{3}{*}{\textbf{AAAB}} & 
3 images from the same object \( O_i \), randomly sampled as \( \mathcal{I}_{O_i} = \{ I_i, I_j, I_k \} \) and 1 image \( I_{\neg i} \) from distractor set \( \mathcal{I}_{\neg O_i}\), randomly shuffled. 
& \multirow{3}{*}{F} & \multirow{3}{*}{2} & \multirow{3}{*}{[\(I_i, I_j, I_k\)]}  \\ \hline
\multirow{3}{*}{4} & \multirow{3}{*}{\textbf{AABC}} & 
2 images from the same object \( O_i \), randomly sampled as \( \mathcal{I}_{O_i} = \{ I_i, I_j\} \) and 2 images \( \{I_{\neg i}, I_{\neg j}\} \) from distractor set \( \mathcal{I}_{\neg O_i}\), randomly shuffled. 
& \multirow{3}{*}{F} & \multirow{3}{*}{3} & \multirow{3}{*}{[\(I_i, I_j\)]}  \\ \hline
\multirow{3}{*}{4} & \multirow{3}{*}{\textbf{ABCD}} & 
1 images from the same object \( O_i \), randomly sampled as \( I_i\) and 3 images \( \{I_{\neg i}, I_{\neg j}, I_{\neg k} \} \) from distractor set \( \mathcal{I}_{\neg O_i}\), randomly shuffled. 
& \multirow{3}{*}{F} & \multirow{3}{*}{3} & \multirow{3}{*}{[]}  \\ \bottomrule
\end{tabularx}
\caption{Summary of multi-images sequence construction for Object-centric Cue (OC) tasks.}
\label{tab:sequence construction oc}
\end{table*}



\paragraph{Question Templates.}

Table \ref{oc_cpr_tf_example}, \ref{oc_cnt_example} and \ref{oc_grp_example} list detailed standard question templates (with format instructions) for the Object-centric Cue task, including 3 subtasks: Comparison (cpr), Counting (Cnt), and Grouping (Grp). 

\begin{table}[htbp]
    \begin{tcolorbox}[colframe=black, colback=gray!10!white, coltitle=black, boxrule=0.5mm]
    \textbf{OC-Cpr Positive Question:}  
    
    \textit{Judge the following statement based on the images: `The \{obj\}s in these images are the same \{obj\}.' Provide only one correct answer: `T' (True) or `F' (False). Respond with either `T' or `F'.}  
    \\ GT Answer: \textbf{\textcolor{customgreen}{T}}

    \vspace{3mm}
    
    \textbf{OC-Cpr Negative Question:}  
    
    \textit{Judge the following statement based on the images: `The \{obj\}s in these images are \textcolor{customred}{not} the same \{obj\}.' Provide only one correct answer: `T' (True) or `F' (False). Respond with either `T' or `F'.}   
    \\ GT Answer: \textbf{\textcolor{customred}{F}}
    \end{tcolorbox}
    \caption{Question templates used for consistency-pair evaluation in the Comparison (Cpr) subtask of Object-centric Cue (OC).}
    \label{oc_cpr_tf_example}
\end{table}

\begin{table}[htbp]
    \begin{tcolorbox}[colframe=black, colback=gray!10!white, coltitle=black, boxrule=0.5mm]
    \textbf{OC-Cnt Question:}  

    \textit{Answer the following question according to this rule: You only need to provide *ONE* correct numerical answer. For example, if you think the answer is `1', your response should only be `1'. The Question is: How many different \{obj\}s are there in the input images?}  
    \\ GT Answer: \textbf{3} \textcolor{blue}{(Example Answer)}
    \end{tcolorbox}
    \caption{The question template used for the counting (Cnt) subtask of Object-centric Cue (OC).}
    \label{oc_cnt_example}
\end{table}

\begin{table}[htbp]
    \begin{tcolorbox}[colframe=black, colback=gray!10!white, coltitle=black, boxrule=0.5mm]
    \textbf{OC-Grp Question:}  

    \textit{Answer the following question based on this rule: You only need to provide *ONE* correct answer, selecting from the options listed below. For example, if you think the correct answer is `B) 1 and 2', your response should be `B) 1 and 2'. \\
    The Question is: Which images show the same \{obj\} in the input images? Choices: A) 1 and 3; B) None; C) 2 and 3; D) 1 and 2.}  
    \\ GT Answer: \textbf{A) 1 and 3} \textcolor{blue}{(Example Answer)}
    \end{tcolorbox}
    \caption{The question template used for the grouping (Grp) subtask of Object-centric Cue (OC).}
    \label{oc_grp_example}
\end{table}






\subsection{PC (Person-centric Cue)}

\label{appendix_pc_construct}

\paragraph{Data Collection.}
We collect images of \emph{meta-humans} mainly from \url{https://www.imdb.com/} and some are from the actor or actress's social media.

\paragraph{Main Meta-human Selection.} 

Our dataset is evenly distributed across different racial groups (Asian, Black, and White) and genders (Male and Female). 
For every race-gender combination, we select five main meta-humans, each contributing four images, yielding a total of 120 images. 

To ensure consistency, all selected individuals are within a similar age range, preventing significant age-related facial changes that could interfere with identity recognition. Additionally, each actor's appearance remains relatively consistent in terms of makeup and overall styling, ensuring that different images of the same meta-human retain distinct yet comparable visual cues (e.g. face shape, eye spacing, nose structure, and lip contours). By preserving these features, we avoid manipulating a single individual's visual cues that could potentially mislead VLMs. Rather, we ensure that the evaluation genuinely tests whether the model can visually link matching cues to recognize the same or different individuals without prior identity knowledge.


\paragraph{Distractor Meta-human Selection.}
To introduce challenging distractors in our sequences, we compute the CLIP embedding for every image and store these embeddings in a reference base. 
When a distractor image is needed, we perform an image-to-image similarity search within this base to identify the most visually similar image that originates from a different meta-human. 
This fine-grained matching ensures that the distractor image closely resembles the main meta-human’s image, leading to more challenging image sequences.


\paragraph{Discussion on Why Objects Require Dedicated Distractors, While Humans Do Not.}  
In object-centric tasks, objects are categorized into eight distinct types, with substantial differences among different types (e.g. pets and bags). Therefore, each main meta-object requires dedicated distractors from the same object type to ensure meaningful comparisons.  
In contrast, humans belong to a single category, meaning that any meta-human can serve as a distractor for another. Given that we compute CLIP embeddings to select visually similar distractors, the constructed image sequences already present a significant challenge without the need for type-specific distractors.  
We also ensure diversity by selecting five main meta-humans for each race-gender pair, providing a sufficiently large pool from which to choose suitable distractors. Corresponding to our hypothesis, in the final curated sequences, most distractor meta-humans chosen were of the same race or gender as the main meta-human. Additionally, as shown in Table \ref{exp:main_exp}, these curated image sequences along with our designed questions effectively challenge tested models, revealing their limited performances in visually linking matching cues on person-centric data.




\paragraph{Images Sequence Construction.}
The construction of image sequences in PC (a total of 260 sequences) follows the structure in Table~\ref{tab:sequence construction pc}. 
More specific details are listed below:

\textbf{Two-Image Sequences (\texttt{image\_seq\_len} = 2)}
\begin{enumerate} \item \textbf{Main Meta-Human Only (PP):}
Two images are randomly selected from the same main meta-human, resulting in 50 sequences.
\item \textbf{Main Meta-Human + Distractor Meta-Human (PQ):}  
One image is randomly selected from the main meta-human, and the other from a distractor meta-human. The order of the images is shuffled. This results in 50 sequences.

\end{enumerate}

\textbf{Three-Image Sequences (\texttt{image\_seq\_len} = 3)}
\begin{enumerate}
    \item \textbf{Main Meta-Human Only (PPP)}:  
    Three images are randomly sampled from the same main meta-human.  
    20 sequences are constructed.
    \item \textbf{Main Meta-Human + Distractor Meta-Human (PPQ)}:  
    Two images are selected from the main meta-human, and one from a single distractor meta-human.  
    The order of images is shuffled.  
    30 sequences are constructed.
    \item \textbf{Main Meta-Human + Distractor Meta-Humans (PQR)}:  
    One image is selected from the main meta-human, while the other two come from distinct distractor meta-humans.  
    The order is shuffled.  
    10 sequences are constructed.
\end{enumerate}

\textbf{Four-Image Sequences (\texttt{image\_seq\_len} = 4)}
\begin{enumerate}
    \item \textbf{Main Meta-Human Only (PPPP)}:  
    All four images are sampled from the same main meta-human.  
    The order is shuffled.  
    30 sequences are constructed.
    \item \textbf{Main Meta-Human + Distractor Meta-Human (PPPQ)}:  
    Three images are sampled from the main meta-human, while one is selected from a single distractor meta-human.  
    The order is shuffled.  
    20 sequences are constructed.
    \item \textbf{Main Meta-Human + Distractor Meta-Humans (PPQR)}:  
    Two images are selected from the main meta-human, while two are selected from distinct distractor meta-humans.  
    The order is shuffled.  
    20 sequences are constructed.
    \item \textbf{Main Meta-Human + Distractor Meta-Humans (PQRS)}:  
    One image is selected from the main meta-human, while three are selected from distinct distractor meta-humans.  
    The order is shuffled.  
    30 sequences are constructed.
\end{enumerate}










% \paragraph{Images Sequence Construction.} how to construct images sequence: Table~\ref{tab:sequence construction pc}



\begin{table*}[htbp]
\centering
\begin{tabularx}{\textwidth}{ccYccc}
\toprule
\textbf{Num} & \textbf{Src} & \textbf{Process of Image Sequences Construction} & \textbf{\textit{Cpr}} & \textbf{\textit{cnt}} & \textbf{\textit{Grp}} \\ \hline
\multirow{2}{*}{2} & \multirow{2}{*}{\textbf{PP}} & 2 images from the same person \( P_i \), randomly sampled as \( \mathcal{I}_{P_i} = \{ I_i, I_j \} \), and shuffled. & \multirow{2}{*}{T} & \multirow{2}{*}{2} & \multirow{2}{*}{-} \\ \hline
\multirow{2}{*}{2} & \multirow{2}{*}{\textbf{PQ}} & 1 image \( I_{i} \) from \( \mathcal{I}_{P_i} \) and 1 image \( I_{\neg i} \) from distractor set \( \mathcal{I}_{\neg P_i} \), randomly shuffled. & \multirow{2}{*}{F} & \multirow{2}{*}{1} & \multirow{2}{*}{-} \\ \hline
\multirow{2}{*}{3} & \multirow{2}{*}{\textbf{PPP}} & 3 images from the same person \( P_i \), randomly sampled as \( \mathcal{I}_{P_i} = \{ I_i, I_j, I_k \} \), and shuffled. & \multirow{2}{*}{T} & \multirow{2}{*}{3} & \multirow{2}{*}{-} \\ \hline
\multirow{3}{*}{3} & \multirow{3}{*}{\textbf{PPQ}} & 2 images from the same person \( P_i \), randomly sampled as \( \mathcal{I}_{P_i} = \{ I_i, I_j \} \) and 1 \( I_{\neg i} \) from distractor set \( \mathcal{I}_{\neg P_i} \), randomly shuffled. & \multirow{3}{*}{F} & \multirow{3}{*}{2} & \multirow{3}{*}{[$I_i$, $I_j$]} \\ \hline
\multirow{3}{*}{3} & \multirow{3}{*}{\textbf{PQR}} & 1 image from the same person \( P_i \), randomly sampled as \( \mathcal{I}_{P_i} = \{ I_i \} \) and 2 images \( \{I_{\neg i}, I_{\neg j} \} \) from distractor set \( \mathcal{I}_{\neg P_i} \), randomly shuffled. & \multirow{3}{*}{F} & \multirow{3}{*}{3} & \multirow{3}{*}{[]} \\ \hline
\multirow{2}{*}{4} & \multirow{2}{*}{\textbf{PPPP}} & 4 images from the same person \( P_i \), randomly sampled as \( \mathcal{I}_{P_i} = \{ I_i, I_j, I_k, I_p \} \), and shuffled. & \multirow{2}{*}{T} & \multirow{2}{*}{4} & \multirow{2}{*}{-} \\ \hline
\multirow{3}{*}{4} & \multirow{3}{*}{\textbf{PPPQ}} & 3 images from the same person \( P_i \), randomly sampled as \( \mathcal{I}_{P_i} = \{ I_i, I_j, I_k \} \) and 1 image \( I_{\neg i} \) from distractor set \( \mathcal{I}_{\neg P_i} \), randomly shuffled. & \multirow{3}{*}{F} & \multirow{3}{*}{2} & \multirow{3}{*}{[$I_i$, $I_j$, $I_k$]} \\ \hline
\multirow{3}{*}{4} & \multirow{3}{*}{\textbf{PQQR}} & 2 images from the same person \( P_i \), randomly sampled as \( \mathcal{I}_{P_i} = \{ I_i, I_j \} \) and 2 images \( \{I_{\neg i}, I_{\neg j} \} \) from distractor set \( \mathcal{I}_{\neg P_i} \), randomly shuffled. & \multirow{3}{*}{F} & \multirow{3}{*}{3} & \multirow{3}{*}{[$I_i$, $I_j$]} \\ \hline
\multirow{3}{*}{4} & \multirow{3}{*}{\textbf{PQRV}} & 1 image from the same person \( P_i \), randomly sampled as \( I_i \) and 3 images \( \{I_{\neg i}, I_{\neg j}, I_{\neg k} \} \) from distractor set \( \mathcal{I}_{\neg P_i} \), randomly shuffled. & \multirow{3}{*}{F} & \multirow{3}{*}{3} & \multirow{3}{*}{[]} \\ \bottomrule
\end{tabularx}
\caption{Summary of multi-images sequence construction for Person-centric Cue (PC) tasks.}
\label{tab:sequence construction pc}
\end{table*}



\paragraph{Video Construction.}
\begin{wrapfigure}{r}{0.35\textwidth} % 图片靠右,宽度为文本宽度的50%
  \centering
  \vspace{-10pt} % 调整图片与上方文本的间距
  \includegraphics[width=0.99\linewidth]{img/duration_distribution.png} % 图片宽度为 wrapfigure 宽度的95%
  \caption{Distribution of video duration in the subtask of Video Identity Description (VID) in Person-centric Cue (PC).}
  \label{fig:video distribution}
  \vspace{-15pt} % 调整图片与下方文本的间距
\end{wrapfigure}
% how to let different sampling methods  
% 1, table (summary of different sampling methods)
% 2, ways to concat video to ensure sampling

The video data for this benchmark is manually collected from Shutterstock\footnote{\url{https://www.shutterstock.com}}. We selected ten common activity categories that an individual can perform: \textbf{clean, cook, drink, exercise, listen, play, read, ride, walk, and work}. For each category, we curated \textbf{10 sets of candidate video pairs}, and each set consists of two videos.

To ensure motion consistency and length diversity, we carefully structured the final videos by concatenating clips while keeping the total duration within the \textbf{0-100}s time range. Figure~\ref{fig:video distribution} displays the sketch of concatenated video length distribution. The final compositions followed two formats:
\paragraph{$P$->$\neg P$ format}: A direct concatenation of two distinct clips (same length for each clip).


\paragraph{$P$->$\neg P$->$P$ format}: A sequence where the first clip and the third clip are sampled from the same candidate video, while the second clip is sampled from the second candidate video (same length for the three clips).




% \begin{figure}[h]
%   \centering
%   \includegraphics[width=0.70\textwidth]{img/duration_distribution.png} 
%   % \vspace{-0.8cm}
%   \caption{Distribution of video duration in the subtask of Video Identity Description (VID) in Person-centric Cue (PC).}
%   \label{fig:video distribution}
%   % \vspace{-4em}
% \end{figure}





Regardless of the different default sampling methods for our baseline models in Table \ref{different_sampling}, both $P$->$\neg P$ and $P$->$\neg P$->$P$ formats ensure that every video clip has frames included in the sampling process:

\begin{itemize}
    \item \textbf{Uniform Sampling (8/16 frame)}: Each clip contributes a proportionate number of frames based on the total video length. Since in one concatenated video, all the sampled clips are the same length, this method guarantees at least 2 frames for each clip can be sampled as model input frames.
    \item \textbf{FPS Sampling (1fps)}: Since frames are sampled at a fixed rate, the structure of $P$->$\neg P$ and $P$->$\neg P$->$P$ ensures that each clip is present long enough for multiple frames to be captured, regardless of its placement in the sequence.
\end{itemize}



\begin{table}[H]
\centering
{
\begin{tabular}{lcc}
    \toprule
    \textbf{Model Name} & \textbf{Uniform (8/16)} & \textbf{FPS (1fps)} \\
    \midrule
    LLaVA-OneVision-7B  & \cmark & \xmark \\
    LLaVA-Video-7B      & \cmark & \xmark \\
    LongVA-7B           & \cmark & \xmark \\
    mPLUG-Owl3-7B       & \cmark & \xmark \\
    Qwen2-VL-7B         & \xmark & \cmark \\
    Qwen2.5-VL-7B       & \xmark & \cmark \\
    InternVL2.5-8B      & \cmark & \xmark \\
    InternVL2.5-26B     & \cmark & \xmark \\
    \midrule
    GPT-4o              & \cmark & \xmark \\
    \bottomrule
\end{tabular}
}
    \caption{Comparison of different video sampling methods of VLMs.}
    \label{different_sampling}
\end{table}



Thus, by maintaining the integrity of each clip's temporal structure, both $P$->$\neg P$ and $P$->$\neg P$->$P$ formats effectively ensure that every clip contributes frames to the final sampled frame input for all models.



\paragraph{Question Templates.}

Table~\ref{pc_cpr_tf_example}, Table~\ref{pc_cnt_example}, Table~\ref{pc_grp_example}, and Table~\ref{pc_vid_example} present the detailed standard question templates for the Person-centric Cue task, covering the four subtasks: Comparison (PC-Cpr), Counting (PC-Cnt), Grouping (PC-Grp), and Video Identity Description (PC-VID).



\begin{table}[H]
    \begin{tcolorbox}[colframe=black, colback=gray!10!white, coltitle=black, boxrule=0.5mm]
    \textbf{PC-Cpr Positive Question:}  
    
    \textit{
    Judge the following statement based on the images: `The individuals in these images are the same person.' Provide only one correct answer: `T' (True) or `F' (False). Respond with either `T' or `F'.}  
    \\ GT Answer: \textbf{\textcolor{customgreen}{T}}

    \vspace{3mm}
    
    \textbf{PC-Cpr Negative Question:}  
    
    \textit{Judge the following statement based on the images: `The individuals in these images are \textcolor{customred}{not} the same person.' Provide only one correct answer: `T' (True) or `F' (False). Respond with either `T' or `F'.}  
    \\ GT Answer: \textbf{\textcolor{customred}{F}}
    \end{tcolorbox}
    \caption{Question templates used for consistency-pair evaluation in the Comparison (Cpr) subtask of Person-centric Cue (PC).}
    \label{pc_cpr_tf_example}
\end{table}

\begin{table}[H]
    \begin{tcolorbox}[colframe=black, colback=gray!10!white, coltitle=black, boxrule=0.5mm]
    \textbf{PC-Cnt Question:}  
    
    \textit{"Answer the following question according to this rule: You only need to provide *ONE* correct numerical answer. For example, if you think the answer is '1', your response should only be '1'. The Question is: How many distinct individuals are in the input images?"}  
    \\ GT Answer: \textbf{2} \textcolor{blue}{(Example Answer)}
    \end{tcolorbox}
    \caption{The question template used for the counting (Cnt) subtask of Person-centric Cue (PC).}
    \label{pc_cnt_example}
\end{table}

\begin{table}[H]
    \begin{tcolorbox}[colframe=black, colback=gray!10!white, coltitle=black, boxrule=0.5mm]
    \textbf{PC-Grp Question:}  
    
    \textit{Answer the following question according to this rule: You only need to provide *ONE* correct answer, selecting from the options listed below. For example, if you think the correct answer is `B) 2 and 3', your response should only be `B) 2 and 3'. The Question is: Which images correspond to the same person in the input images? Choices: A) None; B) 2 and 3; C) 1 and 3; D) 1 and 2."}  
    \\ GT Answer: \textbf{D) 1 and 2} \textcolor{blue}{(Example Answer)}
    \end{tcolorbox}
    \caption{The question template used for the grouping (Grp) subtask of Person-centric Cue (PC).}
    \label{pc_grp_example}
\end{table}

\begin{table}[H]
    \begin{tcolorbox}[colframe=black, colback=gray!10!white, coltitle=black, boxrule=0.5mm]
    \textbf{PC-VID Question:}  
    
    \textit{"Give a comprehensive description of the whole video, prioritizing details about the individuals in the video."}  
    \end{tcolorbox}
    \caption{The question template used for the Video Identity Description (VID) subtask of Person-centric Cue (PC).}
    \label{pc_vid_example}
\end{table}





\clearpage
\newpage
\section{More details on Prompting Approaches}
\label{appendix: prompting approaches}
\subsection{Prompts for LLM-as-Evaluator}

When models answer our free-form PC-VID questions, their responses are evaluated by GPT-4o using the scoring prompts detailed in Tables~\ref{score prompt ab} and \ref{score prompt aba}. Specifically, for videos following a \( \mathcal{P} \rightarrow \neg \mathcal{P} \) sequence, GPT-4o assesses whether the model explicitly distinguishes that the first individual (\( \mathcal{P} \)) and the second individual (\( \neg \mathcal{P} \)) are different. In this case, if the model successfully makes this distinction, it receives a score of 1; otherwise, it is given a score of 0.

For videos that exhibit a \( \mathcal{P} \rightarrow \neg \mathcal{P} \rightarrow \mathcal{P} \) (PQP) pattern, the evaluation is more nuanced. The evaluator model (GPT-4o) checks two aspects: (1) whether the model correctly identifies that there are two distinct individuals (i.e., \( \mathcal{P} \) and \( \neg \mathcal{P} \)), and (2) whether the model explicitly recognizes that the final appearance belongs to the same individual as the first (\( \mathcal{P} \)). A perfect identification of both aspects yields a score of 2, while correctly distinguishing the individuals without explicitly linking the final appearance to the first results in a score of 1. If the model fails to distinguish between the individuals, a score of 0 is assigned.

% \paragraph{Task Context and Accuracy Calculation}
% In the subtask of \textit{VID}, we collect videos featuring different individuals. For each video \( V_{\mathcal{P}_i} \) depicting a person \( \mathcal{P}_i \), we identify another video \( V_{\neg \mathcal{P}_i} \) showing a different individual who shares similar visual cues (e.g., actions, scene, clothing). Two types of video sequences are constructed: (i) \( \mathcal{P}_i \xrightarrow{} \neg \mathcal{P}_i \), which tests the model's ability to discern distinct individuals, and (ii) \( \mathcal{P}_i \xrightarrow{} \neg \mathcal{P}_i \xrightarrow{} \mathcal{P}_i \), which examines whether the model can both detect the switch between individuals and correctly link the final occurrence of \( \mathcal{P}_i \) back to its first appearance.

% We prompt the model with the following question: ``Give a comprehensive description of the whole video, prioritizing details about the individuals in the video.'' This free-form question is designed to probe the model’s capacity to observe scene transitions and changes in individual identity solely through visual link matching cues, without any prior knowledge of the individuals' identities.

% The final accuracy \(Acc_{oe}\) is computed by averaging the scores across all open-ended responses and rescaling them to the range \([0,1]\). In addition, we perform manual verification of GPT-4o’s scoring by randomly sampling 20 scored responses per model. This review revealed only 2 discrepancies out of 180 responses, resulting in an accuracy rate of 98.89\%.






\begin{table*}[htbp]
    \begin{tcolorbox}[colframe=black, colback=gray!10!white, coltitle=black, boxrule=0.5mm]
    \textbf{\#Task}
    
    You are evaluating a model's ability to accurately distinguish between two different individuals, P and Q, who appear sequentially in a video (first P, then Q). Given a description, your task is to determine if the model explicitly identifies that the first person (P) and the second person (Q) are different individuals.
    
    \textbf{\#Return Format}
    
    You only need return a number after "Score:". If you think the model correctly identifies that the two appearances belong to different individuals, return "Score: 1". If you think the model fails to explicitly state that there are two different individuals, return "Score: 0".
    
    \textbf{\#Description}
    
    \textcolor{teal}{\textbf{<Model's Description>}}
    \end{tcolorbox}
    \caption{Scoring prompt for \textit{VID} (when video belongs to category of $P$->$\neg P$).}
    \label{score prompt ab}
\end{table*}

\begin{table*}[htbp]
    \begin{tcolorbox}[colframe=black, colback=gray!10!white, coltitle=black, boxrule=0.5mm]
    \textbf{\#Task}
    
    You are evaluating a model's ability to accurately distinguish between two different individuals, P and Q, who appear sequentially in a video following an PQP pattern (first P, then Q, then P again). Given a description, your task is to determine whether the model explicitly identifies that: (1) P and Q are different individuals, and (2) The person in the final scene is the same as the first (P).
    
    \textbf{\#Return Format}
    
    You only need return a number after "Score:". 
    
    (1) If the model correctly describes that the video follows an PQP sequence, explicitly recognizing that the first and last appearances belong to the same person (P), while the middle appearance is a different person (Q), return "Score: 2".
    
    (2) If the model correctly identifies that there are two different people in the video (P and Q) but does not explicitly mention that the last scene returns to P, return "Score: 1".

    (3) If the model fails to recognize that two different individuals appear (e.g., treats all appearances as the same person or does not distinguish between P and Q), return "Score: 0".
    
    \textbf{\#Description}
    
    \textcolor{teal}{\textbf{<Model's Description>}}
    \end{tcolorbox}
    \caption{Scoring prompt for \textit{VID} (when video belongs to category of $P$->$\neg P$->$P$).}
    \label{score prompt aba}
\end{table*}

\subsection{Prompting Approaches for Probing on VLM$^2$-Bench}
\paragraph{CoT (CoT-normal).}
The normal version of the Chain-of-Thought prompt is shown in Table~\ref{cot prompt}. We simply require the model to think 'step-by-step' to ensure self-reflection and self-correction, as well as the transparent thinking process.





\paragraph{CoT-special for GC.}
Table~\ref{cot-special} shows a special version of the Chain-of-Thought prompt. According to the task features, we carefully analyze how a human being approaches and visually links matching cues for questions in GC, then curate this prompt as an imitation of the human visual linking process.

\paragraph{VP-grid for GC.}
Figure~\ref{fig:vp-grid} displays a complete version of Visual Prompting with Grid assistance (VP-grid). Here we follow ~\citep{lei2024scaffoldingcoordinatespromotevisionlanguage} to print a set of dot matrix onto the input image, accompanied by the image order dimension concatenated with Cartesian coordinates as (\textit{image order index}, \textit{colum index)}, \textit{row index}). In the detailed textual prompt design, we also integrated references and explanations for the grids, allowing VLMs to leverage this visual assistance as spatial and visual matching references.


\paragraph{VP-zoom-o for OC.}
In Figure~\ref{fig:VP-zoom-o}, we demonstrate the visual prompting process for OC. We leverage the Grounded-SAM \citep{grounded-sam} model to detect bounding boxes for objects based on their types then crop the ``zoomed-in'' objects as the image input for further VQA pairs.

\paragraph{VP-zoom-p for PC.}
The visual prompting process in similar to that of OC (Figure~\ref{fig:vp-zoom-p}). We use a face detection model~\citep{facedetector} to ``zoom in'' on the individual's face and occlude other irrelevant information.

\begin{table*}[htbp]
    \begin{tcolorbox}[colframe=black, colback=gray!10!white, coltitle=black, boxrule=0.5mm]
    \textcolor{blue!70}{\textbf{<Question>}}
    % \vspace{1em}
    
    Let's think `step by step' to answer this question, you need to output the thinking process of how you get the answer.
    \end{tcolorbox}
    \caption{CoT prompt for GC (here we denote as CoT-normal to distinguish it from the CoT-special in Table~\ref{cot-special} that specifically designed for GC), OC, and PC.}
    \label{cot prompt}
\end{table*}

\begin{table*}[htbp]
    \begin{tcolorbox}[colframe=black, colback=gray!10!white, coltitle=black, boxrule=0.5mm]
    \textcolor{blue!70}{\textbf{<Question>}}
    % \vspace{1em}
    
    Use the following 4 steps to answer the question:
    \vspace{0.5em}
    
    \textbf{Step 1. Understand the Question} \\
    - Identify the question's purpose.\\
    - Check for any format requirements.\\
    % \vspace{0.5em}
    
    \textbf{Step 2. Perceive (List Elements)} \\
    - List every details in each image respectively. \\
    - Note positions and attributes of elements.\\
    % \vspace{0.5em}
    
    \textbf{Step 3. Connect (Compare \& Reason)} \\
    - Compare corresponding elements in each image.\\
    - List all the unchanged elements and the changed element.\\
    % \vspace{0.5em}
    
    \textbf{Step 4. Conclude (Answer the Question)} \\
    \end{tcolorbox}
    \caption{CoT-special specifically designed for GC.}
    \label{cot-special}
\end{table*}

\begin{figure*}[htbp]
\centering
\includegraphics[width=0.95\textwidth]{img/grid_prompt.png} 
% \vspace{-0.8cm}
\caption{An illustration of how VP-grid works for GC.}
\label{fig:vp-grid}
% \vspace{-4em}
\end{figure*}

\begin{figure*}[h]
\centering
\includegraphics[width=0.95\textwidth]{img/pc-o_prompt.png} 
% \vspace{-0.8cm}
\caption{An illustration of how VP-zoom-o works for OC.}
\label{fig:VP-zoom-o}
% \vspace{-4em}
\end{figure*}

\begin{figure*}[h]
\centering
\includegraphics[width=0.95\textwidth]{img/pc-p_prompt.png} 
% \vspace{-0.8cm}
\caption{An illustration of how VP-zoom-p works for PC.}
\label{fig:vp-zoom-p}
% \vspace{-4em}
\end{figure*}















\newpage
\section{Case Study}
\label{Appendix: case study}

This section focuses on how various prompting techniques influence model performance, highlighting their successes and limitations across different models.



\subsection{Case for CoT-special prompting in General Cue (GC) Task}

We observe that the CoT-special prompt boosts InternVL2.5-8B's performance by over 25\% than the standard query in both Matching and Tracking tasks for General Cue. While for the traditional CoT-normal prompting technique, this boost is only 13\%. The CoT-special prompt (Table~\ref{cot-special}) directs the model through four explicit steps: understanding the question, perceiving (listing elements), connecting (comparing and reasoning), and concluding. This structured approach mirrors the human process of visual matching and is effective even for a rather smaller model like InternVL2.5-8B, which might otherwise struggle with the ambiguity of a complex generic step-by-step instruction (which we will discuss later in the next Subsection~\ref{vp-decrease case study}).

For example, in the provided InternVL2.5-8B response Figure~\ref{fig:cot-special increase}, the model correctly executes the following: In Step 2, it identifies critical details such as "Vase with flowers on the table" and "Chandelier above" in Image 1, while noting the absence of the vase in Image 2. In Step 3, it systematically compares the two images, highlighting that while many elements remain unchanged (e.g., the chandelier, kitchen area, bowl of fruit, window), the removal of the vase is the key difference. Finally, in Step 4, the model concludes that the statement "The vase on top of the table was removed" accurately describes the visual change, thereby arriving at the correct answer.

This detailed, multi-step breakdown not only ensures that all pertinent visual cues are captured and processed but also reduces errors by structuring the logical flow of reasoning. The CoT-special prompt's explicit instructions help InternVL2.5-8B align visual information with textual descriptions more effectively, thus enhancing overall performance. Compared to the less specific CoT-normal prompt—which may leave the model with gaps in reasoning—the CoT-special prompt provides clear, task-specific guidance that is essential for complex visual reasoning tasks, as evidenced by the substantial performance improvement.




\begin{figure*}[htbp]
  \centering
  \includegraphics[width=0.75\textwidth]{img/gc_intern_pcp_boost.pdf} 
  % \vspace{-0.8cm}
  \caption{Case study illustrating how CoT-special improves performance of the subtask of Tracking (Trk) in General Cue (GC). The model, InternVL2.5-8B, demonstrates a step-by-step reasoning process: In Step 2, it identifies key details such as ``Vase with flowers on the table" and "Chandelier above" in Image 1, while noting the absence of the vase in Image 2. In Step 3, it compares the images, recognizing that while many elements remain unchanged (e.g., chandelier, kitchen area, fruit bowl, window), the vase's removal is the primary difference. In Step 4, the model concludes that the statement "The vase on top of the table was removed" accurately reflects the visual change, leading to the correct answer.}

  \label{fig:cot-special increase}
\end{figure*}





\subsection{Case for VP-grid in General Cue Task}
\label{vp-decrease case study}
The VP-grid (Visual Prompting with Grid assistance) method enhances visual matching in General Cue tasks by overlaying a dot matrix grid onto the input image. Each dot is annotated with a three-dimensional coordinate tuple, \((\textit{image order index}, \textit{column index}, \textit{row index})\), where the first dimension distinguishes the sequence of images (e.g., the first image is indexed as 1 and the second as 2). This grid is further supported by detailed textual descriptions that clarify the coordinate system, enabling Vision-Language Models (VLMs) to use these cues for spatial and visual matching.

\paragraph{A example failure case in VP-grid.}
However, this approach does not yield consistent improvements across all models. For instance, the Qwen2.5-VL-7B model demonstrates a significant performance drop—nearly 20\%—when using VP-grid. An example failure case is in Figure ~\ref{fig:vp-grid decrease}. Our analysis reveals that although the model correctly identifies visual elements (e.g., a pedestrian with a high-visibility vest at coordinates \((2,5,3)\)), it fails to properly interpret the image sequence. Specifically, the model incorrectly associates the coordinates \((2,5,3)\) with the first image, rather than the second, despite the explicit definition provided in the textual prompt. This misinterpretation leads to erroneous linking of visual matching cues and subsequent faulty reasoning. We suspect that the underlying issue is the limited semantic comprehension capability of the relatively smaller 7B model, which struggles with complex, predefined spatial instructions and visual assistance.


\paragraph{A example of success case in VP-grid.} In contrast to models that often misinterpret or neglect spatial cues provided by VP-grid—leading to errors such as mismatching image indices—GPT-4o successfully leverages these visual prompts to achieve correct visual-textual alignment. In the example at Figure~\ref{fig:vp-grid increase}, the model identifies the cat's nose at coordinates \((1,2,4)\) in the first image and at \((2,2,4)\) in the second image, enabling it to accurately capture the change in the visual attribute (from a lighter pink to a darker black).

This success stems from several key aspects of GPT-4o's processing capabilities:
\begin{enumerate}
    \item \textbf{Precise Disambiguation of Image Order:} The VP-grid explicitly encodes image order, which GPT-4o uses to differentiate between multiple images. This prevents the common error of conflating spatial information from distinct images—a problem seen in smaller models.
    \item \textbf{Robust Visual Matching in space:} With clear coordinate annotations, the model effectively locates and compares the same physical regions across images. In this case, the exact correspondence between the cat's nose in different images is recognized, which is crucial for detecting subtle visual changes.
    \item \textbf{Structured Reasoning Process:} GPT-4o adheres to a well-defined reasoning sequence in our textual guidance(perception, connection, and conclusion). By systematically linking the provided grid coordinates with the textual descriptions, it is able to deduce the key visual change accurately.
\end{enumerate}

\paragraph{Implications on Model Scale.} Our analysis suggests that the enhanced performance of GPT-4o with VP-grid can be attributed to its larger model capacity. Although the detailed architecture of GPT-4o is proprietary, its ability to process complex multi-modal prompts implies that:
\begin{itemize}
    \item \textbf{Enhanced Semantic Understanding:} Larger models are inherently better at comprehending intricate, structured prompts that combine visual and textual information. This results in a more nuanced interpretation of spatial cues.
    \item \textbf{Superior Visual-Textual Alignment:} With greater capacity, GPT-4o can integrate and correlate the detailed spatial data (visual assistance) from the VP-grid with the corresponding textual descriptions, minimizing the risk of mis-association or errors.
    \item \textbf{Effective Handling of Complexity:} The advanced reasoning capabilities of larger models enable them to navigate the additional complexity introduced by VP-grid without suffering from the side effects seen in smaller models. This ensures that the additional spatial guidance improves performance rather than causing confusion.
\end{itemize}

The success of GPT-4o in utilizing the VP-grid approach demonstrates that model scale plays a critical role in effectively integrating complex visual and textual cues. By accurately disambiguating image order and performing precise spatial matching, GPT-4o not only avoids the pitfalls encountered by smaller models but also benefits significantly from the additional visual assistance, leading to an overall performance improvement of approximately 10\%.





\begin{figure*}[h]
  \centering
  \includegraphics[width=0.75\textwidth]{img/gc_qwen_vlp_drop.pdf} 
  % \vspace{-0.8cm}
  \caption{Case study illustrating why VP-grid leads to performance degradation in Qwen2.5-VL-7B. The model correctly identifies visual elements (e.g., a pedestrian with a high-visibility vest at coordinates \((2,5,3)\)) but fails to interpret the image sequence correctly. It mistakenly associates the coordinates with the first image instead of the second, despite the explicit definition in the textual prompt. This misinterpretation results in incorrect visual cue linking and faulty reasoning, highlighting the model's difficulty in handling structured spatial instructions and visual prompts.}
  \label{fig:vp-grid decrease}
\end{figure*}






\begin{figure*}[h]
  \centering
  \includegraphics[width=0.75\textwidth]{img/gc_gpt4o_vlp_boost.pdf} 
  % \vspace{-0.8cm}
  \caption{Case study demonstrating why VP-grid leads to performance improvement for GPT-4o. Unlike models that often misinterpret or overlook spatial cues, GPT-4o effectively uses VP-grid to align visual and textual information. In the example shown in Figure~\ref{fig:vp-grid increase}, the model correctly identifies the cat's nose at coordinates \((1,2,4)\) in the first image and \((2,2,4)\) in the second, accurately capturing the visual change in the attribute (from a lighter pink to a darker black). This success highlights GPT-4o’s ability to handle structured spatial prompts and improve performance through visual prompting.}
  \label{fig:vp-grid increase}
\end{figure*}



\subsection{Case for CoT prompting in Object-centric Cue Task}

The task design for Object-centric cue (OC) and person-centric cue (PC) requires multiple images (more than 2) as sequence input. We observe that, unlike General Cue (GC) tasks where models are required to link instance-level cues, OC tasks demand that models group similar objects based on fine-grained visual features. As illustrated in Figure~\ref{fig:oc-analysis}, models using the CoT approach sometimes struggle to provide a comprehensive overview of vision-based cues across a sequence of images. 

A detailed case in Figure~\ref{fig:oc_cot_normal_decrease} is provided by InternVL2.5-26B's response. The ground truth and Vanilla responses correctly identify that there is no grouping for the same meta-object in the sequence, with the answer `D) None'. In the CoT response, the model states: "The second
and third images \textcolor{customgreen}{both have dinosaurs wearing sunglasses}". Although the description here is true, its ambiguity and lack of detailed coverage lead the model to incorrectly select option C) 2 and 3, rather than the correct option D) None. Because if we take a closer look at the design on the backpack in image 3, the dinosaur with sunglasses is actually holding a keyboard instead of a skateboard in image 2. This is a distractive visual matching cue we intend to capture during the distractor meta-object selection. This major difference should have prevented models from grouping image 2 and image 3 together.


According to our findings, this misgrouping occurs for two main reasons:
\begin{enumerate}
    \item \textbf{Insufficient Overview of Visual Cues:} The CoT prompt does not force the model to systematically verify all critical details across multiple images. As a result, the model overlooks nuanced differences, such as the design discrepancy on the backpack in image 3, where the dinosaur holds a keyboard rather than a skateboard.
    \item \textbf{Variability in Descriptive Language:} The open-ended language generated by the CoT approach can lead to inconsistent descriptions. In this case, the model generalized the visual cue of a "dinosaur design" without capturing the specific attribute (i.e., the object the dinosaur is holding), which is crucial for correct grouping.
\end{enumerate}

Thus, the lack of structured guidance in the CoT prompt leads to the dropping or misinterpretation of critical cues, resulting in incorrect grouping decisions for multi-image sequences in OC tasks. This analysis underscores the importance of more detailed structured intermediate reasoning strategies, such as those provided by a tailored CoT-special prompt, to ensure that all relevant visual details are captured and compared accurately.










\begin{figure*}[h]
  \centering
  \includegraphics[width=0.75\textwidth]{img/oc_intern26_cot_drop2.pdf} 
  % \vspace{-0.8cm}
  \caption{Case study illustrating why CoT leads to performance degradation. In the example shown in Figure~\ref{fig:oc_cot_normal_decrease}, InternVL2.5-26B's response correctly identifies that no grouping occurs for the same meta-object in the sequence, with the correct answer being `D) None'. However, in the CoT response, the model incorrectly selects option C) 2 and 3. While it correctly states that ``the second and third images both have dinosaurs wearing sunglasses," the lack of detailed analysis leads to an inaccurate conclusion. A closer examination reveals a key difference between the images—the dinosaur in image 3 is holding a keyboard instead of a skateboard, which should have prevented the grouping of the two images. This highlights the importance of providing more detailed and unambiguous cues in CoT reasoning.}

  \label{fig:oc_cot_normal_decrease}
\end{figure*}




\end{document}
