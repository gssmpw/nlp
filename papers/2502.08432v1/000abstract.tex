Hypergraphs provide a superior modeling framework for representing complex multidimensional relationships in the context of real-world interactions that often occur in groups, overcoming the limitations of traditional homogeneous graphs. However, there have been few studies on hypergraph-based contrastive learning, and existing graph-based contrastive learning methods have not been able to fully exploit the high-order correlation information in hypergraphs. Here, we propose a Hypergraph Fine-grained contrastive learning (HyFi) method designed to exploit the complex high-dimensional information inherent in hypergraphs. While avoiding traditional graph augmentation methods that corrupt the hypergraph topology, the proposed method provides a simple and efficient learning augmentation function by adding noise to node features. Furthermore, we expands beyond the traditional dichotomous relationship between positive and negative samples in contrastive learning by introducing a new relationship of weak positives. It demonstrates the importance of fine-graining positive samples in contrastive learning. Therefore, HyFi is able to produce high-quality embeddings, and outperforms both supervised and unsupervised baselines in average rank on node classification across 10 datasets. Our approach effectively exploits high-dimensional hypergraph information, shows significant improvement over existing graph-based contrastive learning methods, and is efficient in terms of training speed and GPU memory cost. The source code is available at \url{https://github.com/Noverse0/HyFi.git}.


% 하이퍼그래프는 집단에서 자주 발생하는 실제 상호작용의 맥락에서 복잡한 다차원 관계를 표현하는 데 탁월한 모델링 프레임워크를 제공하여 기존의 동질적인 그래프의 한계를 극복할 수 있습니다. 하지만 하이퍼그래프 기반 대조 학습과 관련된 연구는 많지 않으며, 기존의 그래프 기반 대조 학습 방법은 하이퍼그래프의 고차 상관관계 정보를 충분히 활용하지 못했습니다. 여기서는 하이퍼그래프에 내재된 복잡한 고차원 정보를 활용하기 위해 고안된 세분화된 하이퍼그래프 대비 학습(FG-HGCL) 방법을 소개합니다. 제안된 방법은 하이퍼그래프 토폴로지를 손상시키는 기존의 그래프 증강 방법을 피하면서 노드 특징에 노이즈를 추가하여 간단하고 효율적인 학습 증강 기능을 제공합니다. 또한 공유 하이퍼에지와 공유 노드를 동질성의 지표로 사용하는 독특한 대비 학습 방식을 사용합니다. 이 방법은 쌍을 이루는 노드 관계를 4개의 세분으로 효율적으로 분류하고 고품질 임베딩을 생성하며 10개의 데이터 세트에서 노드 분류 및 클러스터링 작업에서 감독 및 비감독 기준선보다 뛰어난 성능을 보입니다. 이 접근 방식은 고차원 하이퍼그래프 정보를 효과적으로 활용하여 기존의 그래프 기반 대비 학습 방법에 비해 상당한 개선을 보여주며, 훈련 속도와 GPU 메모리 비용 측면에서 효율적입니다. 소스 코드는 https://github.com/Noverse0/FG-HGCL.git 에서 확인할 수 있습니다.
