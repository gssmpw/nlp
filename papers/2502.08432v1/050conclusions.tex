This study has successfully demonstrated the efficacy of the HyFi model in hypergraph-based contrastive learning, demonstrating its ability to capture the nuanced relationships inherent in hypergraph results in higher quality embeddings, and superior performance in node classification task across a variety of datasets. Our novel augmentation technique developed for hypergraphs preserves the original topology while adding noise to node features, ensuring the integrity of the hypergraph structure. Furthermore, the contrastive learning method introduced in this study effectively utilizes the homophily characteristic of hypergraphs, providing a more effective approach than existing HGCL methods. Also, the proposed HyFi model stands out for its efficiency in processing speed and memory usage, making it an ideal solution for various applications. 
% This research paves the way for future exploration in hypergraph-based learning, representing an important advancement and also underscoring the potential for further advancements in this field.

