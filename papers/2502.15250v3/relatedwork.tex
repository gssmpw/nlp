\section{Related work}
In this section, a combination of gradient-based method, Bayesian decison and mathematical morphology is introduced, which can detect frontal zone and fronts. Further improvements with innovative algorithms are in the next section.

\subsection{Data}
The sea surface temperature(SST) is obtained from the NOAA Coral Reef Watch daily global $5$km($0.05$ degree exactly) satellite coral bleaching heat stress monitoring product(v3.1). It addresses errors identified in previous versions of the SST data, particularly for coral reef regions during the period from 2013-2016. What is more, this fine-grained resolution is essential for understanding ocean fronts and it can be downloaded from \url{https://coralreefwatch.noaa.gov/product/5km/}. 

Results of Lyapunov method for comparison are composed of multimission altimetry-derived gridded backward-in-time FSLE and Orientations of associated eigenvectors\cite{Hernandez-2011}. Spatial resolution is $4$km($0.04$ degree exactly) and can be downloaded from \url{https://www.aviso.altimetry.fr/en/data/products/value-added-products/fsle-finite-size-lyapunov-exponents.html}.

\subsection{Gradient calculation}
\begin{equation}\label{eq:kernel}
	\begin{gathered}
		F_{x}=\begin{pmatrix}-1&0&+1\\-2&0&+2\\-1&0&+1\end{pmatrix}\\
		F_{y}=\begin{pmatrix}+1&+2&+1\\0&0&0\\-1&-2&-1\end{pmatrix}
	\end{gathered}
\end{equation}

\begin{equation}
	\begin{gathered}
		T_x=F_x*SST\\
		T_y=F_y*SST
	\end{gathered}
\end{equation}

The $F_x$ and $F_y$ in \eqref{eq:kernel} are two convolutional kernels which perform weighted operations on the surrounding points of each point, respectively, to calculate the longitudinal gradient $T_y$ and latitudinal gradient $T_x$.

\begin{equation}\label{eq:mag_dir}
	\begin{gathered}
		T_{mag}=\sqrt{T_x^2+T_y^2}\\
		T_{dir}=arctan(\frac{T_y}{T_x})
	\end{gathered}
\end{equation}

Using $T_x$ and $T_y$, the magnitude and direction of gradient can be calculated by \eqref{eq:mag_dir}. The range of gradient direction is $[-\pi, \pi)$.

\subsection{Double thresholding}
Sort according to $T_{mag}$ in descending order to generate a culmulative distribution function(CDF). Then, take the top 10\% $T_{mag}$ position in the gradient sequence as the first percentile, i.e., the upper threshold(noted as $u_u$) and 20\% the lower threshold(noted as $u_l$). The points with $T_{mag}$ that exceed the upper threshold are classified as frontal zone points, while below the lower threshold non frontal zone points and between the two thresholds undetermined frontal zone points. The CDF of one day is shown in figure \ref{fig:cdf}.

\begin{figure}[!t]
	\centering
	\includegraphics[width=0.35\textwidth]{./figures/cdf.png}
	\caption{CDF of SST gradient magnitude in 2025.01.01} 
	\label{fig:cdf} 
\end{figure}

\subsection{Bayesian decision}
\begin{figure}
	\centering
	\includegraphics[width=0.1\textwidth]{./figures/surround.png}
	\caption{Undetermined frontal zone point} 
	\label{fig:surround} 
\end{figure}

For the undetermined frontal zone points obtained above, a Bayesian decision procedure is imposed to judge if the points are frontal zone points. As shown in figure \ref{fig:surround}, suppose the point to be determined is $E$, which is surrounded by points $A, B, C, D, F, G, H$ and $I$.

\begin{equation}\label{eq:prior}
	\begin{gathered}
		P(E=i)=\frac{T_E-u_l}{u_u-u_l}\\
		P(E=j)=\frac{u_u-T_E}{u_u-u_l}
	\end{gathered}
\end{equation}

Firstly, the prior probability of $E$ being frontal zone point or non frontal zone point is calculated with \eqref{eq:prior} where $i$ is frontal zone point, $j$ is non frontal zone point and $T_E$ is gradient magnitude of $E$.

It is clear that the prior probability is a linear transform of gradient magnitude and they are positively relevant. In another word, prior probability is the information of first derivative.

Secondly, likelihood is obtained by using more information of the whole field, i.e., local degree of edge(LDE) and block deviation(BD)\cite{Ping-2014}. LDE and BD are calculated for the $4$ pairs of points($A-I$, $B-H$, $C-G$ and $D-F$) around $E$. 

\begin{equation}\label{eq:lde_bd_ai}
	\begin{gathered}
		LDE(A,I)=\frac{4}{7}\frac{V_{max}-\overline{V}-|A-I|}{V_{max}-V_{min}}+\frac{1}{2}\\
		BD(A,I)=\frac{|A-I|}{V_{max}-V_{min}}
	\end{gathered}
\end{equation}

In \eqref{eq:lde_bd_ai}, $V=[A, B, C, D, F, G, H, I]$ is the vector of $8$ SST values around $E$. $V_{max}$ is the maximum of these components in $V$, $V_{min}$ is the minimum and $\overline{V}$ the average. It can be verified that LDE and BD are two mathematical operators which fall into $[0, 1]$ for normalization\cite{Mansoori-2006, Bauer-1996}. 

It can also be concluded that $LDE(A,I)$ is the similarity between $|A-I|$ and local maximum amplitude($V_{max}-\overline{V}$) and $BD(A,I)$ is normalized gradient magnitude in $A-I$ direction.

\begin{equation}\label{eq:lde_bd_e}
	\begin{gathered}
		LDE(E)=\frac{1}{4}(LDE(A,I)+LDE(B,H)+LDE(C,G)+\\
		LDE(D,F))\\
		BD(E)=\frac{1}{4}(BD(A,I)+BD(B,H)+BD(C,G)+\\
		BD(D,F))\\
	\end{gathered}
\end{equation}

The process is the same for $B-H$, $C-G$ and $D-F$. LDE and BD average of these $4$ pairs generate the LDE and BD of $E$, as shown in \eqref{eq:lde_bd_e}.

\begin{equation}\label{eq:likelihood}
	\begin{gathered}
		P(fact|E=i)=\frac{F_{LDE,E}}{F_{E}}\frac{F_{BD,E}}{F_{E}}\\
		P(fact|E=j)=\frac{NF_{LDE,E}}{NF_{E}}\frac{NF_{BD,E}}{NF_{E}}
	\end{gathered}
\end{equation}

Then comes the calculation of likelihood. In \eqref{eq:likelihood}, $F_E$ represents the number of points whose gradient magnitude is greater than $T_E$, $F_{LDE,E}$ is the number with LDE difference smaller than $0.1$ in $F_E$ compared to $LDE(E)$ and $F_{BD,E}$ is the same logic for BD difference. The derivation of $NF_E$, $NF_{LDE,E}$ and $NF_{BD,E}$ follows the same steps as described above.

The term $fact$ in \eqref{eq:likelihood} is based on the assumption that if $E$ is a frontal zone point($E=i$), then LDE and BD of greater gradient magnitude points should be similar and vice versa. The assumption is natural for it introduces $2$ mathematical operators, i.e. patterns to generate likelihood.

\begin{equation}\label{eq:bayesian}
	\begin{gathered}
		P(E=i|fact)=\frac{P(fact|E=i)P(E=i)}{P(fact)}\\
		P(E=j|fact)=\frac{P(fact|E=j)P(E=j)}{P(fact)}
	\end{gathered}
\end{equation}

\begin{equation}\label{eq:decision}
	{E} = \begin{cases}
		i,&{\text{if}}P(fact|E=i)P(E=i) \geq P(fact|E=j)\\
		&P(E=j)\\
		{j,}&{\text{otherwise.}} 
	\end{cases}
\end{equation}

\begin{algorithm}
	\caption{Bayesian decision.}
	\label{alg:bd}
	\begin{algorithmic}[1]
		\Require
		SST and SST gradient magnitude;
		\Ensure
		Frontal zone;
		\State Use SST gradient magnitude to get prior probability for every point by \eqref{eq:prior};
		\label{code:fram:prior}
		\State Use SST to calculate LDE and BD by \eqref{eq:lde_bd_ai} and \eqref{eq:lde_bd_e};
		\label{code:fram:lde_bd}
		\State Obtain likelihood by \eqref{eq:likelihood}; 
		\label{code:fram:likelihood}
		\State Get posterior and make Bayesian decision by \eqref{eq:bayesian};
		\label{code:fram:posterior}
		\Return Frontal zone;
	\end{algorithmic}
\end{algorithm}

Finally, with Bayes theorem and decision for this paper--\eqref{eq:bayesian}\eqref{eq:decision}, the undetermined frontal zone point $E$ can be judged by posterior probability. If $P(fact|E=i)P(E=i)>=P(fact|E=j)P(E=j)$, then $E$ is seen as frontal zone point and non frontal zone point otherwise. The whole process is shown in algorithm \ref{alg:bd}.

\subsection{Skeletonization}
With maximum disk method(MDM) in mathematical morphology, these one-pixel-width fronts can be extracted from the frontal zone obtained above. MDM is a skeletonization algorithm that determines whether each point in the frontal zone is a front point, also known as a skeleton point. The specific principle is to expand the disk with each point as the center. When there are two or more tangent points with the boundary during the expansion process, it is determined as a skeleton point. 

\begin{figure}
	\centering
	\includegraphics[width=0.15\textwidth]{./figures/mdm.png}
	\caption{Examples of MDM} 
	\label{fig:mdm} 
\end{figure}

\begin{equation}\label{eq:mdm}
	\begin{gathered}
		S(Z)=\bigcup_{k=0}^K S_k (Z)\\
		S_k (Z)=(Z \ominus kM)-(Z \ominus k)\degree M\\
		K= \max \{k|(Z \ominus kM) \neq \varnothing \}
	\end{gathered}
\end{equation}

The principle can be seen in figure \ref{fig:mdm}, and the mathematical principle can be seen in \eqref{eq:mdm}. In it, $Z$ is the determined frontal zone, $(Z \ominus kM)$ means performing $k$ rounds of corrosion on the identified frontal zone, $M$ is the convolutional kernel, $\degree$ is the open operation and $K$ is the maximum number of corrosion cycles before being corroded into an empty set.

\subsection{Skeleton trimming}
Although these fronts have been found, they have many branches, and in order to extract the main features and adapt to the research scale, redundant branches must be removed. Here, a greedy algorithm called the discrete skeleton evolution(DSE) algorithm is used to assign weights to each branch, iteratively removing the branch with the minimum weight to obtain the main fronts\cite{Bai-2007, Vincent-1993}.

The points in $S(Z)$ need to be classified into $3$ categories: endpoint, connection point and intersection point. The endpoint has $1$ point in $S(Z)$ connected to itself, the connection point has $2$ and the intersection point has $3$. It is clear to see that endpoint and intersection point determine a branch in $S(Z)$. Suppose $l_i(i=1,2,...,N)$ are the endpoints in $S(Z)$, $f(l_i)$ are the closest intersection points to $l_i$, $Q(l_i,f(l_i))$ are the branches.

The frontal zone after skeleton reconstruction was determined according to $R(S)=\cup_{s\in S(Z)}U(s,r(s))$; among them, $R(S)$ is the frontal zone after skeleton reconstruction, $r(s)$ is the radius of the largest disk $U(s,r(s))$ with the center $s$ and in the frontal zone.

\begin{equation}\label{eq:weight}
	w_i=V(R(S))-V(R(S-Q(l_i,f(l_i))))
\end{equation}

Using \eqref{eq:weight} to assign weights to every branch $Q(l_i,f(l_i))$ and $V()$ is the area function. $w_i$ is the number of pixels loss when comparing the reconstructed frontal zone and the original. It can be seen from this equation that if $w_i$ approches $0$, it will mean that the reconstructed frontal zone after removing $Q(l_i,f(l_i))$ is similar to the original, so it can be removed and a greedy algorithm can be obtained as in algorithm \ref{alg:dse}.

\begin{algorithm}
	\caption{DSE.}
	\label{alg:dse}
	\begin{algorithmic}[1]
		\Require
		The original skeleton of frontal zone $S^0(Z)$(S(Z));
		\Ensure
		Trimmed skeleton $S^{K}(Z)$(assume $K$ iterations);
		\State Set an upper threhold $t$ for minumum weight;
		\label{code:fram:threshold}
		\State In the $kth$ iteration, update weights $w^k_i$ in $S^k(Z)$ according to  \eqref{eq:weight};
		\label{code:fram:kth_weight}
		\State Choose the mimimum weight $w^k_{min}$; 
		if $w^k_{min} \leq t$, skip to \ref{code:fram:dse_trim} otherwise \Return $S^{k}(Z)$;
		\label{code:fram:dse_judge}
		\State Remove the branch and update skeleton: $S^{k+1}(Z)=S^k(Z)-Q(l_{min}^k,f(l_{min}^k))$;
		skip to \ref{code:fram:kth_weight};
		\label{code:fram:dse_trim}
	\end{algorithmic}
\end{algorithm}

\begin{figure}
	\centering
	\includegraphics[width=0.45\textwidth]{./figures/prune.png}
	\caption{SST fronts in 2025.01.01 with different $t$} 
	\label{fig:dse} 
\end{figure}

Figure \ref{fig:dse} is a demo of DSE and it shows the influence of threshold $t$ on the performance for SST fronts in Jan 1st , 2025 in South China Sea.