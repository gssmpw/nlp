\section{Related Work}
\label{sec:related_work}
% \paragraph{Deformable 3D Gaussians}
\paragraph{Dynamic Gaussian Modeling}
\label{sec:related_work:dynamic_gs}
Recent advancements have shown the potential of Gaussian Splatting~\citep{kerbl20233d} for 4D reconstruction~\citep{jung2023deformable, katsumata2023efficient,wu20244d,luiten2024dynamic,li2024spacetime,lu20243d, lei2024gart,guo2024motion,qian20243dgs,bae2024per,wan2024template}. A central focus of these efforts is the deformation modeling of 3D Gaussians. While effective for dynamics capturing, most approaches learn transformations implicitly, limiting their capability for controllable dynamics modeling. To address this issue, recent studies use superpoints~\citep{huang2024sc,wan2024superpoint} for improved dynamics modeling and control. However, as superpoint learning is based primarily on rendering without considering object physics, these methods fail to reliably capture accurate physical parameters (\eg, joints and axes). Another line of works~\citep{xie2024physgaussian,jiang2024vr} introduce controllable Gaussians by integrating physics-based modeling for graphics simulations. These models require intricate priors of objects (\eg, material properties), making them impractical for reconstructing everyday articulated objects. To overcome these challenges, our work combines the explicit 3D Gaussian modeling with articulation modeling, enabling efficient and high-quality reconstruction with precise articulation parameter estimation for more practical digital-twin construction of articulated objects.


% Our work leverages the efficiency and flexibility of Gaussian representations while tailoring the approach to the specific demands of articulated object reconstruction, achieving high-quality mesh reconstruction, part segmentation and joint articulation estimation.


% . Articulated objects, with requirements for controllable dynamics modeling yet with simple part-axis representations, demands a simple yet effective method for modeling the controllable 

% in 3D scene representation have seen the emergence of Gaussian Splatting \citep{kerbl20233d} as a powerful technique for efficient and high-quality 3D reconstruction. Building upon this, many works have extended 3D Gaussians to dynamic scenes, enabling the representation of non-rigid transformations. \cite{luiten2024dynamic, katsumata2023efficient} use frame-by-frame training to learn the dynamic Gaussians and \cite{li2024spacetime} propose to model the motion trajectory of Gaussians. \cite{yang2024deformable} proposes deformable 3D Gaussian, deforming a set of canonical Gaussians to model the dynamic of 3D scenes, which is adopted and improved by many works \citep{wu20244d, wan2024superpoint, lu20243d, huang2024sc, lei2024gart, guo2024motion, wu20244d, jung2023deformable, qian20243dgs, bae2024per, xie2024physgaussian, xie2024surgicalgaussian}. 
% While effective in capturing complex transformations, these methods are focused on general scene rather than specifically addressing the challenges of articulated objects.
% Articulated object reconstruction is challenging due to the complex interplay between rigid parts and constrained joint movements, which requires simultaneous part segmentation and precise joint parameter estimation—tasks that are not typically encountered in general scene reconstruction or even human body modeling. 
% Our work leverages the efficiency and flexibility of Gaussian representations while tailoring the approach to the specific demands of articulated object reconstruction, achieving high-quality mesh reconstruction, part segmentation and joint articulation estimation.

\paragraph{Articulation Parameter Estimation}
\label{sec:related_work:artmodel}
Estimating joint articulation parameters for articulated objects has been extensively studied, with approaches broadly categorized into two main categories. First, prediction-based methods estimate joint parameters from sensory inputs of different object configurations \citep{huang2014occlusion,katz2013interactive} or use end-to-end models \citep{hu2017learning,yi2018deep,li2020category,wang2019shape2motion,sun2023opdmulti,liu2022toward,weng2021captra,sturm2011probabilistic,chu2023command,martin2016integrated,liu2023self,gadre2021act,mo2021where2act,jain2021screwnet,yan2020rpm,lei2023nap} to predict part segmentation, kinematic structure, as well as joint parameters. Second, reconstruction-based methods optimize articulation parameters by reconstructing multi-view images or videos~\citep{wei2022self,tseng2022cla,mu2021sdf,lewis2022narf22,jiayi2023paris,lei2024gart,deng2024articulate,swaminathan2024leia,noguchi2022watch,zhang2021strobenet,pillai2015learning,liu2023building}.
Most of these methods treat articulation parameter estimation as a separate task, without generating high-quality, interactable part-mesh reconstructions. \model aims to address this gap by integrating part-mesh reconstruction and articulation parameter estimation, enabling the creation of high-quality, interactable replicas.
% Based on reconstruction of multi-view images, our \model addresses both articulation parameter estimation and part mesh reconstruction, building high-quality interactable replicas.

\paragraph{Articulated Object Reconstruction}
\label{sec:related_work:reconstruction}
Articulated object reconstruction, differing from human and animal motion modeling~\citep{joo2018total,loper2023smpl,mihajlovic2021leap,noguchi2021neural,yang2021viser,yang2021lasr,romero2022embodied,zuffi20173d,yang2024attrihuman,xu2020ghum,tan2023distilling, yang2022banmo,yang2023ppr,song2023moda,yang2023reconstructing,song2023total}, focus on the piece-wise rigidity of each part, requiring both part-level geometry reconstruction and joint articulation parameter estimation. While end-to-end models predict joint parameters and segment object parts from single-stage~\citep{heppert2023carto, wei2022self,kawana2021unsupervised} or interaction observations\citep{jiang2022ditto, ma2023sim2real, nie2022structure, hsu2023ditto}, they struggle to generalize to unseen objects. Per-object optimization approaches~ \citep{jiayi2023paris,liu2023building,weng2024neural,deng2024articulate,swaminathan2024leia}, using multi-state observations for articulation modeling, offer better adaptability to unknown objects but face scaling issues of multiple joints. Methods like DTA~\citep{weng2024neural} attempt to handle multi-part objects but still struggle with those having more than three movable parts. We address the reliability, flexibility, and scalability issues of previous works with our canonical Gaussian design and skinning-inspired part dynamics modeling, achieving higher accuracy, robustness, and efficiency for articulated object reconstruction.


% These limitations call for more flexible, scalable methods in articulated object reconstruction.
% We address these limitations through our novel part assignment module and initialization strategy, making accurate and robust performance with much higher efficiency.