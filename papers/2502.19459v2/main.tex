
\documentclass{article} % For LaTeX2e
\usepackage{iclr2025_conference}

% Optional math commands from https://github.com/goodfeli/dlbook_notation.
%%%%% NEW MATH DEFINITIONS %%%%%

% \usepackage{amsmath,amsfonts,bm}
\usepackage{amsmath,amsfonts}

\usepackage{pifont}


\newcommand{\R}{\mathbb{R}}


\def\va{{\mathbf{a}}}
\def\vg{{\mathbf{g}}}

% Sets
\def\sR{\mathbb{R}}
\def\sC{\mathbb{C}}
\def\sZ{\mathbb{Z}}
\def\sN{\mathbb{N}}
\def\sQ{\mathbb{Q}}

\def\sS{\mathcal{S}}



% Vectors
\def\vzero{{\mathbf{0}}}
\def\vone{{\mathbf{1}}}
\def\vmu{{\mathbf{\mu}}}
\def\vtheta{{\mathbf{\theta}}}
\def\va{{\mathbf{a}}}
\def\vb{{\mathbf{b}}}
\def\vc{{\mathbf{c}}}
\def\vd{{\mathbf{d}}}
\def\ve{{\mathbf{e}}}
\def\vf{{\mathbf{f}}}
\def\vg{{\mathbf{g}}}
\def\vh{{\mathbf{h}}}
\def\vi{{\mathbf{i}}}
\def\vj{{\mathbf{j}}}
\def\vk{{\mathbf{k}}}
\def\vl{{\mathbf{l}}}
\def\vm{{\mathbf{m}}}
\def\vn{{\mathbf{n}}}
\def\vo{{\mathbf{o}}}
\def\vp{{\mathbf{p}}}
\def\vq{{\mathbf{q}}}
\def\vr{{\mathbf{r}}}
\def\vs{{\mathbf{s}}}
\def\vt{{\mathbf{t}}}
\def\vu{{\mathbf{u}}}
\def\vv{{\mathbf{v}}}
\def\vw{{\mathbf{w}}}
\def\vx{{\mathbf{x}}}
\def\vy{{\mathbf{y}}}
\def\vz{{\mathbf{z}}}
\def\vzeta{{\mathbf{\zeta}}}

% Matrix
\def\mA{{\mathbf{A}}}
\def\mB{{\mathbf{B}}}
\def\mC{{\mathbf{C}}}
\def\mD{{\mathbf{D}}}
\def\mE{{\mathbf{E}}}
\def\mF{{\mathbf{F}}}
\def\mG{{\mathbf{G}}}
\def\mH{{\mathbf{H}}}
\def\mI{{\mathbf{I}}}
\def\mJ{{\mathbf{J}}}
\def\mK{{\mathbf{K}}}
\def\mL{{\mathbf{L}}}
\def\mM{{\mathbf{M}}}
\def\mN{{\mathbf{N}}}
\def\mO{{\mathbf{O}}}
\def\mP{{\mathbf{P}}}
\def\mQ{{\mathbf{Q}}}
\def\mR{{\mathbf{R}}}
\def\mS{{\mathbf{S}}}
\def\mT{{\mathbf{T}}}
\def\mU{{\mathbf{U}}}
\def\mV{{\mathbf{V}}}
\def\mW{{\mathbf{W}}}
\def\mX{{\mathbf{X}}}
\def\mY{{\mathbf{Y}}}
\def\mZ{{\mathbf{Z}}}
\def\mBeta{{\mathbf{\beta}}}
\def\mPhi{{\mathbf{\Phi}}}
\def\mLambda{{\mathbf{\Lambda}}}
\def\mSigma{{\mathbf{\Sigma}}}


% Expectation
% \def\eE{\mathop{\mathbb{E}}\limits}
\def\eE{\mathbb{E}}

% Probability
\def\pP{\mathbb{P}}

% Tilde
\def\tf{\tilde{f}}
\def\tS{\tilde{S}}
\def\wtF{\widetilde{\mathcal{F}}}
\def\whR{\widehat{R}}
\def\tvx{\tilde{\mathbf{x}}}
\def\ty{\tilde{y}}


\def\defeq{\overset{\textup{def}}{=}}
% \def\defeq{\overset{.}{=}}
\def\defone{\overset{\text{\ding{172}}}{=}}
\def\deftwo{\overset{\text{\ding{173}}}{=}}
\def\leqone{\overset{\text{\ding{172}}}{\leq}}
\def\leqtwo{\overset{\text{\ding{173}}}{\leq}}
\def\leqthree{\overset{\text{\ding{174}}}{\leq}}
\def\leqfour{\overset{\text{\ding{175}}}{\leq}}
\def\eqone{\overset{\text{\ding{172}}}{=}}
\def\eqtwo{\overset{\text{\ding{173}}}{=}}
\def\eqthree{\overset{\text{\ding{174}}}{=}}
\def\eqfour{\overset{\text{\ding{175}}}{=}}
\def\geqfive{\overset{\text{\ding{176}}}{\geq}}
\usepackage[utf8]{inputenc} % allow utf-8 input
\usepackage[T1]{fontenc}    % use 8-bit T1 fonts
\usepackage{microtype,inconsolata}
\usepackage{times,latexsym}
\usepackage{graphicx} \graphicspath{{figures/}}
\usepackage{amsmath,amssymb,mathabx,mathtools,amsthm,nicefrac}
\usepackage[linesnumbered,ruled,vlined]{algorithm2e}
\usepackage{acronym}
\usepackage{enumitem}
\usepackage[pagebackref,breaklinks,colorlinks]{hyperref}
\usepackage{balance}
\usepackage{xspace}
\usepackage{setspace}
\usepackage[skip=3pt,font=small]{subcaption}
\usepackage[skip=3pt,font=small]{caption}
\usepackage[capitalise,noabbrev,nameinlink]{cleveref}
\usepackage{booktabs,tabularx,colortbl,multirow,multicol,array,makecell,tabularray}
\usepackage{overpic,wrapfig}
\usepackage{dblfloatfix}
\usepackage[misc]{ifsym}
\usepackage{pifont}
\usepackage{fancyvrb}

% Add a period to the end of an abbreviation unless there's one
% already, then \xspace.
\makeatletter
\DeclareRobustCommand\onedot{\futurelet\@let@token\@onedot}
\def\@onedot{\ifx\@let@token.\else.\null\fi\xspace}

\def\eg{\emph{e.g}\onedot} \def\Eg{\emph{E.g}\onedot}
\def\ie{\emph{i.e}\onedot} \def\Ie{\emph{I.e}\onedot}
\def\cf{\emph{c.f}\onedot} \def\Cf{\emph{C.f}\onedot}
\def\etc{\emph{etc}\onedot} \def\vs{\emph{vs}\onedot}
\def\wrt{w.r.t\onedot} \def\dof{d.o.f\onedot}
\def\etal{\emph{et al}\onedot}

\makeatother

\acrodef{sota}[SOTA]{State-of-the-Art}
\acrodef{method}[\textsc{PRA}]{Preference-based Robot Assistant}
\acrodef{pbp}[\textsc{PbP}]{Preference-based Planning}
\acrodef{vln}[VLN]{Vision-and-Language Navigation}
\acrodef{llm}[LLM]{Large Language Model}
\acrodef{EILEV}[EILEV]{Efficient In-context Learning on Egocentric Videos}
\acrodef{vlm}[VLM]{Vision-Language Model}
\acrodef{vivit}[ViViT]{Video Vision Transformer}
\acrodef{llava}[LLaVA]{Large Language and Vision Assistant}
\acrodef{ai}[AI]{Artificial Intelligence}
\acrodef{ik}[IK]{Inverse Kinematics}
\acrodef{ompl}[OMPL]{Open Motion Planning Library}
\acrodef{sem}[SEM]{Structural Equation Model}

% Spacing
% \medmuskip=2mu   % reduce spacing around binary operators
% \thickmuskip=3mu % reduce spacing around relational operators
\setlength{\abovedisplayskip}{3pt}
\setlength{\belowdisplayskip}{3pt}
\setlength{\abovecaptionskip}{3pt}
\setlength{\belowcaptionskip}{3pt}
% \setlength\floatsep{1\baselineskip plus 3pt minus 2pt}
% \setlength\textfloatsep{1\baselineskip plus 3pt minus 2pt}
% \setlength\dbltextfloatsep{1\baselineskip plus 3pt minus 2pt}
% \setlength\intextsep{1\baselineskip plus 3pt minus 2pt}

\newcolumntype{x}{>{\columncolor{LightCyan1}}c}
\newcolumntype{y}{>{\columncolor{MistyRose}}c}

% \title{Efficient Reconstruction of Articulated \\ Objects via ArtGS}
\title{Building Interactable Replicas of
Complex \\Articulated Objects via Gaussian Splatting}
% Authors must not appear in the submitted version. They should be hidden
% as long as the \iclrfinalcopy macro remains commented out below.
% Non-anonymous submissions will be rejected without review.

\iclrfinalcopy

\author{
% Yu Liu$^{1,2,*,\ddagger}$\quad Baoxiong Jia$^{2,*}$\quad Ruijie Lu$^{2,3}$ \quad Junfeng Ni$^{1,2}$\\
% \textbf{Song-Chun Zhu}$^{1,2,3}$ \quad \textbf{Siyuan Huang}$^{2}$\\
\hspace{-9pt} Yu Liu$^{1,2,*,\ddagger}$, Baoxiong Jia$^{2,*}$, Ruijie Lu$^{2,3}$, Junfeng Ni$^{1,2}$, \textbf{Song-Chun Zhu}$^{1,2,3}$, \textbf{Siyuan Huang}$^{2}$
\\
\hspace{-8pt} \small $^1$Tsinghua University~$^2$State Key Laboratory of General Artificial Intelligence, BIGAI~$^3$\small Peking University
}

% The \author macro works with any number of authors. There are two commands
% used to separate the names and addresses of multiple authors: \And and \AND.
%
% Using \And between authors leaves it to \LaTeX{} to determine where to break
% the lines. Using \AND forces a linebreak at that point. So, if \LaTeX{}
% puts 3 of 4 authors names on the first line, and the last on the second
% line, try using \AND instead of \And before the third author name.

\newcommand{\fix}{\marginpar{FIX}}
\newcommand{\new}{\marginpar{NEW}}

\iclrfinalcopy % Uncomment for camera-ready version, but NOT for submission.
\begin{document}


\maketitle

\begin{abstract}
Building interactable replicas of articulated objects is a key challenge in computer vision. Existing methods often fail to effectively integrate information across different object states, limiting the accuracy of part-mesh reconstruction and part dynamics modeling, particularly for complex multi-part articulated objects. We introduce \model, a novel approach that leverages 3D Gaussians as a flexible and efficient representation to address these issues. Our method incorporates canonical Gaussians with coarse-to-fine initialization and updates for aligning articulated part information across different object states, and employs a skinning-inspired part dynamics modeling module to improve both part-mesh reconstruction and articulation learning. Extensive experiments on both synthetic and real-world datasets, including a new benchmark for complex multi-part objects, demonstrate that \model achieves state-of-the-art performance in joint parameter estimation and part mesh reconstruction. Our approach significantly improves reconstruction quality and efficiency, especially for multi-part articulated objects. Additionally, we provide comprehensive analyses of our design choices, validating the effectiveness of each component to highlight potential areas for future improvement. Our work is made publicly available at: \url{https://articulate-gs.github.io}.

\let\thefootnote\relax\footnote{$^*$Equal contribution. $^{\ddagger}$Work done as an intern at BIGAI.}

\end{abstract}

\section{Introduction}
\label{sec:intro}
Articulated objects, central to everyday human-environment interactions, have become a key focus in computer vision research~\citep{yang2023reconstructing,weng2024neural,luo2024physpart,liu2024cage,deng2024articulate}. Accurately reconstructing real-world scenes~\citep{chen2024single,ni2024phyrecon,lu2024movis} and creating interactable digital replicas of these objects are essential for various applications, including scene understanding~\citep{jia2024sceneverse, huang2024embodied, zhu2024unifying,linghu2024multi} and robotics learning~\citep{liu2022akb,geng2023gapartnet,geng2023partmanip,gong2023arnold,yang2024physcene,zhao2024tac,lu2024manigaussian}. By building high-fidelity digital twins of articulated objects, we bridge the gap between synthetic and real-world scenarios, thus facilitating the sim-to-real transfer of robotic systems~\citep{torne2024reconciling,kerr2024rsrd}. As we advance towards more sophisticated robotic systems and immersive virtual environments, there is a growing need for improved and efficient modeling techniques for the reconstruction of articulated objects.

% Articulated objects are ubiquitous in our daily lives and industrial settings, ranging from household items like laptops and drawers to complex robotic arms. The ability to accurately reconstruct these objects and create interactable digital replicas is crucial for various applications, including robotic manipulation, augmented reality, and scene understanding. The importance of articulated object reconstruction lies in its potential to training embodied robot in the simulator, enhancing manipulation capabilities of robots \citep{geng2023gapartnet, geng2023partmanip, liu2022akb}. 

% By building high-fidelity digital twins of articulated objects, we enable the training of embodied agents in simulators that closely mimic real-world conditions. These accurate replicas bridge the gap between synthetic data and real scenarios, facilitating the development of robotic systems that can generalize more effectively to real-world environments. This approach not only enhances the quality of robot training but also accelerates the deployment of robust, adaptable robotic solutions across diverse domains~\citep{torne2024reconciling}. As we move towards more sophisticated robotic systems and immersive virtual environments, the need for accurate and efficient reconstruction of articulated objects becomes increasingly critical. 

The problem of reconstructing articulated objects has been extensively studied~\citep{jiayi2023paris,liu2023building,weng2024neural,deng2024articulate,yang2023reconstructing}, with a key challenge being the learning of object geometry when only partial views of the object are available at any given state. To accurately reconstruct object parts (\eg, a closed drawer), it is essential to integrate observations from multiple object states during interactions (\eg, the opening process of the drawer). This necessitates the simultaneous learning and alignment of fine-grained object parts across different states, which must be achieved jointly during the reconstruction of object geometries. Such a requirement presents significant challenges in object modeling, especially for complex everyday articulated objects that often consist of multiple interactable parts. Additionally, uncertainties in object geometry reconstruction introduce further challenges in modeling articulation, as errors in geometry modeling can result in inaccurate learning of articulation parameters. These challenges highlight the need for improved models that handle the complexities of multi-part articulated objects.

Recent approaches attempt to address these challenges using part priors from pre-trained models. These models provide either part segmentation masks via models like SAM~\citep{kirillov2023segment}, or 2D pixel correspondences for aligning pixels across states~\citep{sun2021loftr}. However, these methods rely heavily on priors from pre-trained models, often using single-state inputs and neglecting critical motion information~\citep{mandi2024real2code} and struggling with the complexity of multi-part objects when accurately matching pixels across states becomes difficult~\citep{weng2024neural}. These limitations result in unstable and inconsistent learning of object parts, posing significant challenges to the joint learning of part motion and geometry.

% This inability to effectively leverage motion information across multiple states, coupled with the limitations of existing segmentation and prior models, underscores the need for connecting multiple states with articulation transformation and achieving more accurate part segmentation, particularly for objects with complex articulation structures.
% Real2Code \citep{mandi2024real2code} employs the powerful generalization capabilities of DUSt3R \citep{wang2024dust3r}, SAM \citep{kirillov2023segment}, and Large Language Models (LLMs) to reconstruct complex articulated objects through code generation. While innovative, this approach has two main drawbacks: a) It relies on a single state of objects, lacking .
% b) Its performance is largely dependent on the segmentation results produced by SAM, which may not always accurately capture the articulation structure of complex objects.
% , the learning of object geometry given limited visibility at any single object state is difficult. One needs to combine the observation of multiple object states (\eg, the openning process of a drawer) to properly infer the geometry of object parts (\ie, the drawer in this case). Finding this alignment is difficult as it requires fine-grained object part modeling which needs to be provided as prior or learned jointly during object part movements. This mesh reconstruction issue further causes uncertainty on assuming the geometry of objects during learning, and therefore adds difficulty in articulation modeling as errors in object geometry modeling leads to incorrect articulation parameter estimates. These challenges are further aggravated under the case of complex multi-part objects, as the uncertainty of object geometry and part identification significantly increases as the number of parts increases.

% object parts at any single object states. For reconstructing a drawer, one needs to align the observation of multiple object states describing the openning process of objects for 

% visibility issue
% geometry / articulation interplay

% One key challenge in this task is the intricate interplay between object geometry reconstruction and articulation modeling. As the state of articulated objects naturally contain occlusion of different parts,

% where uncertanties in geometric reconstruction leads to errors in estimating articulation parameters, and changes in artic


% This task presents significant challenges due to the intricate interplay between object geometry and articulation. That is, changes in articulation alters the visible geometry, while uncertainties in geometric reconstruction leads to errors in estimating articulation parameters, particularly for complex multi-part objects. Complex multi-part objects often comprise numerous interconnected parts with diverse joint types, intricate geometries, and varying degrees of visibility across different configurations. These characteristics make it difficult to accurately capture the object's structure and motion from limited observations. 

% At the same time, complex multi-part objects dominate our everyday environment, from intricate machinery to sophisticated household appliances. The accurate reconstruction of these complex articulated structures represents a critical challenge in our quest to create faithful digital replicas of the physical world. Overcoming this hurdle is essential for bridging the gap between virtual and real environments, enabling more realistic simulations, and advancing our ability to interact with and manipulate digital representations of real-world objects. As we strive to build an ideal digital twin of our world, mastering the reconstruction of complex multi-part objects stands as a fundamental stepping stone towards achieving truly immersive and functional virtual realities.

% However, existing methods~\citep{jiang2022ditto, jiayi2023paris, weng2024neural} often struggle with complex multi-part objects, especially in real-world scenarios where observations may be limited or noisy. Even for simple 2-part objects, most current methods face stability problems due to randomness. Many approaches rely heavily on motion information for part segmentation, neglecting crucial spatial relationships. This becomes particularly problematic for multi-part objects, where motion cues alone are insufficient to distinguish between similarly moving parts. 
% Recent works have attempted to address the challenges of reconstructing complex articulated objects, but significant limitations remain. DTA \citep{weng2024neural} makes strides by incorporating the LOFTER prior model \citep{sun2021loftr}, which provides additional guidance for articulation estimation. However, this approach still struggles to extend beyond objects with three or more parts, falling short when faced with the intricate structures common in many real-world articulated objects. Real2Code \citep{mandi2024real2code} takes a different approach, leveraging a combination of advanced tools including DUSt3R \citep{wang2024dust3r}, SAM \citep{kirillov2023segment}, a pre-trained shape-completion model, and a fine-tuned Large Language Model (LLM) to reconstruct complex articulated objects through code generation. While innovative, this method relies heavily on a single object state and lacks motion information, making its performance largely dependent on the segmentation results produced by SAM. This reliance on static segmentation without motion cues limits the method's ability to accurately capture the articulation properties of complex objects. These approaches, while advancing the field, highlight the persistent challenges in reconstructing complex multi-part articulated objects. The inability to effectively leverage motion information across multiple states, coupled with the limitations of existing segmentation and prior models, underscores the need for more robust and versatile methods. 

To address these challenges, we propose \model, which introduces several key innovations for handling complex multi-part articulated objects. Specifically, we adopt the commonly used two-state setting for learning articulated objects, as established in prior works~\citep{jiayi2023paris, weng2024neural}. Central to our approach is the use of 3D Gaussians~\citep{kerbl20233d} as the foundational representation, chosen for their ability to explicitly maintain spatial information while offering efficiency and high reconstruction quality. To effectively model object dynamics and integrate information across multiple object states, we employ canonical Gaussians with a carefully designed coarse-to-fine initialization and update scheme. These Gaussians act as a bridge between different input object states, enabling accurate deformation modeling that improves both mesh reconstruction and articulation learning. Building on the canonical Gaussians, we draw inspiration from Gaussian skinning~\citep{song2024reacto} and introduce a center-based clustering module for part and dynamics learning. This approach leverages motion priors of Gaussians, which are summarized during the learning process, serving as a guide to better align object parts between states and improve articulation learning.
These designs allow our method to achieve state-of-the-art performance in joint parameter estimation and part mesh reconstruction, excelling on both existing benchmarks and our newly curated complex multi-part articulated object reconstruction benchmark. Our approach outperforms existing methods in both synthetic and real-world scenarios, with significant improvements in axis modeling and overall efficiency. Through extensive experiments, we demonstrate the effectiveness of our model in efficiently delivering high-quality reconstruction of complex multi-part articulated objects. We also provide comprehensive analyses of our design choices, highlighting the critical role of these modules and identifying areas for future improvement. 

\paragraph{Contributions} Our main contributions of this work can be summarized as follows:

\begin{itemize}[leftmargin=*,nolistsep,noitemsep]
 \item We propose \model, a novel and efficient method for articulated object reconstruction that achieves state-of-the-art performance, particularly for complex multi-part objects.
 \item We introduce coarse-to-fine canonical Gaussian initialization and skinning-inspired part dynamics modeling with self-guided motion priors to improve object part and articulation learning, effectively addressing the limitations of existing methods in using object motion information.
 \item We conduct extensive experiments on both synthetic and real-world articulated objects, demonstrating the effectiveness, efficiency, scalability, and robustness of our approach. We also provide comprehensive ablation studies to validate our designs and highlight areas for future improvement.
\end{itemize}

% handling objects with significantly more parts and achieving more accurate and stable reconstructions. 
% We leverage the efficiency and flexibility of Gaussian representations, using canonical Gaussians to connect different states and mitigate visibility issues. Moreover, we propose to obtain coarse canonical Gaussians by matching two states of Gaussians, leading to more robust and high-quality reconstructions. To incorporate spatial information modeling into our part segmentation module, we use center-based segmentation, enabling more accurate separation of parts even in complex scenarios. Additionally, we propose a novel initialization strategy that leverages motion priors and clustering techniques to provide an improved starting point for optimization.

% These innovations enable our method to achieve state-of-the-art performance in joint parameter estimation and part mesh reconstruction for both two-part and multi-part objects, from synthetic to real-world scenarios. Moreover, our approach demonstrates remarkable efficiency, reconstructing complex articulated objects in a fraction of the time required by current state-of-the-art methods.

% Key advantages of our approach include:
% - More accurate estimation of joint parameters, as evidenced by smaller Axis Angle Error, Axis Position Error, and Part Motion Error.
% - Better reconstruction of part meshes, with lower Chamfer Distance (CD) metrics for both static and movable parts.
% - Significantly improved performance on real-world objects and multi-part articulated structures compared to existing methods.
% - High computational efficiency, requiring less than half or even a third of the training time of current state-of-the-art methods.

% Our contributions can be summarized as follows:
% \begin{itemize}[leftmargin=*,nolistsep,noitemsep]
% \item We propose \model, a novel articulated object reconstruction method that achieves state-of-the-art performance in joint parameter estimation and mesh reconstruction, particularly for real objects and complex multi-part objects.
% \item We propose an efficient and effective initialization strategy for the canonical Gaussians and part segmentation module, leading to more stable and accurate reconstructions.
% \item Extensive experiments on both synthetic and real-world datasets, demonstrating the effectiveness, efficiency and scalability of our approach for objects with varying complexity. We conduct comprehensive experiments and ablation studies to validate the effectiveness of our approach and its individual components.
% \end{itemize}

\section{Related Work}
\label{sec:related_work}
% \paragraph{Deformable 3D Gaussians}
\paragraph{Dynamic Gaussian Modeling}
\label{sec:related_work:dynamic_gs}
Recent advancements have shown the potential of Gaussian Splatting~\citep{kerbl20233d} for 4D reconstruction~\citep{jung2023deformable, katsumata2023efficient,wu20244d,luiten2024dynamic,li2024spacetime,lu20243d, lei2024gart,guo2024motion,qian20243dgs,bae2024per,wan2024template}. A central focus of these efforts is the deformation modeling of 3D Gaussians. While effective for dynamics capturing, most approaches learn transformations implicitly, limiting their capability for controllable dynamics modeling. To address this issue, recent studies use superpoints~\citep{huang2024sc,wan2024superpoint} for improved dynamics modeling and control. However, as superpoint learning is based primarily on rendering without considering object physics, these methods fail to reliably capture accurate physical parameters (\eg, joints and axes). Another line of works~\citep{xie2024physgaussian,jiang2024vr} introduce controllable Gaussians by integrating physics-based modeling for graphics simulations. These models require intricate priors of objects (\eg, material properties), making them impractical for reconstructing everyday articulated objects. To overcome these challenges, our work combines the explicit 3D Gaussian modeling with articulation modeling, enabling efficient and high-quality reconstruction with precise articulation parameter estimation for more practical digital-twin construction of articulated objects.


% Our work leverages the efficiency and flexibility of Gaussian representations while tailoring the approach to the specific demands of articulated object reconstruction, achieving high-quality mesh reconstruction, part segmentation and joint articulation estimation.


% . Articulated objects, with requirements for controllable dynamics modeling yet with simple part-axis representations, demands a simple yet effective method for modeling the controllable 

% in 3D scene representation have seen the emergence of Gaussian Splatting \citep{kerbl20233d} as a powerful technique for efficient and high-quality 3D reconstruction. Building upon this, many works have extended 3D Gaussians to dynamic scenes, enabling the representation of non-rigid transformations. \cite{luiten2024dynamic, katsumata2023efficient} use frame-by-frame training to learn the dynamic Gaussians and \cite{li2024spacetime} propose to model the motion trajectory of Gaussians. \cite{yang2024deformable} proposes deformable 3D Gaussian, deforming a set of canonical Gaussians to model the dynamic of 3D scenes, which is adopted and improved by many works \citep{wu20244d, wan2024superpoint, lu20243d, huang2024sc, lei2024gart, guo2024motion, wu20244d, jung2023deformable, qian20243dgs, bae2024per, xie2024physgaussian, xie2024surgicalgaussian}. 
% While effective in capturing complex transformations, these methods are focused on general scene rather than specifically addressing the challenges of articulated objects.
% Articulated object reconstruction is challenging due to the complex interplay between rigid parts and constrained joint movements, which requires simultaneous part segmentation and precise joint parameter estimation—tasks that are not typically encountered in general scene reconstruction or even human body modeling. 
% Our work leverages the efficiency and flexibility of Gaussian representations while tailoring the approach to the specific demands of articulated object reconstruction, achieving high-quality mesh reconstruction, part segmentation and joint articulation estimation.

\paragraph{Articulation Parameter Estimation}
\label{sec:related_work:artmodel}
Estimating joint articulation parameters for articulated objects has been extensively studied, with approaches broadly categorized into two main categories. First, prediction-based methods estimate joint parameters from sensory inputs of different object configurations \citep{huang2014occlusion,katz2013interactive} or use end-to-end models \citep{hu2017learning,yi2018deep,li2020category,wang2019shape2motion,sun2023opdmulti,liu2022toward,weng2021captra,sturm2011probabilistic,chu2023command,martin2016integrated,liu2023self,gadre2021act,mo2021where2act,jain2021screwnet,yan2020rpm,lei2023nap} to predict part segmentation, kinematic structure, as well as joint parameters. Second, reconstruction-based methods optimize articulation parameters by reconstructing multi-view images or videos~\citep{wei2022self,tseng2022cla,mu2021sdf,lewis2022narf22,jiayi2023paris,lei2024gart,deng2024articulate,swaminathan2024leia,noguchi2022watch,zhang2021strobenet,pillai2015learning,liu2023building}.
Most of these methods treat articulation parameter estimation as a separate task, without generating high-quality, interactable part-mesh reconstructions. \model aims to address this gap by integrating part-mesh reconstruction and articulation parameter estimation, enabling the creation of high-quality, interactable replicas.
% Based on reconstruction of multi-view images, our \model addresses both articulation parameter estimation and part mesh reconstruction, building high-quality interactable replicas.

\paragraph{Articulated Object Reconstruction}
\label{sec:related_work:reconstruction}
Articulated object reconstruction, differing from human and animal motion modeling~\citep{joo2018total,loper2023smpl,mihajlovic2021leap,noguchi2021neural,yang2021viser,yang2021lasr,romero2022embodied,zuffi20173d,yang2024attrihuman,xu2020ghum,tan2023distilling, yang2022banmo,yang2023ppr,song2023moda,yang2023reconstructing,song2023total}, focus on the piece-wise rigidity of each part, requiring both part-level geometry reconstruction and joint articulation parameter estimation. While end-to-end models predict joint parameters and segment object parts from single-stage~\citep{heppert2023carto, wei2022self,kawana2021unsupervised} or interaction observations\citep{jiang2022ditto, ma2023sim2real, nie2022structure, hsu2023ditto}, they struggle to generalize to unseen objects. Per-object optimization approaches~ \citep{jiayi2023paris,liu2023building,weng2024neural,deng2024articulate,swaminathan2024leia}, using multi-state observations for articulation modeling, offer better adaptability to unknown objects but face scaling issues of multiple joints. Methods like DTA~\citep{weng2024neural} attempt to handle multi-part objects but still struggle with those having more than three movable parts. We address the reliability, flexibility, and scalability issues of previous works with our canonical Gaussian design and skinning-inspired part dynamics modeling, achieving higher accuracy, robustness, and efficiency for articulated object reconstruction.


% These limitations call for more flexible, scalable methods in articulated object reconstruction.
% We address these limitations through our novel part assignment module and initialization strategy, making accurate and robust performance with much higher efficiency.

\section{Preliminaries}
\label{sec:prel}
\paragraph{3D Gaussian Splatting} \ac{3dgs} represents a static 3D scene using 3D Gaussians~\citep{kerbl20233d}. Each Gaussian $G_i$ is associated with a center $\vmu_i$, covariance matrix $\mSigma_i$, opacity $\sigma_i$ and spherical harmonics coefficients $\vh_i$. The final opacity of a 3D Gaussian at a spatial point $\vx$ can be calculated as: 
\begin{equation}
 \begin{aligned}
 \alpha_i(\vx)= \sigma_i \exp\left(-\frac{1}{2}(\vx-\vmu_i)^T\mSigma_i^{-1}(\vx - \vmu_i)\right), && \text{where} && \mSigma_i = \mR_i\mS_i\mS_i^T\mR_i^T.
 \end{aligned}
 \label{eq:gs_opacity}
\end{equation}
As the physical meaning of a covariance matrix is only valid if it is positive semi-definite, we decompose the covariance matrix $\mSigma_i$ following~\cref{eq:gs_opacity} into a scaling diagonal matrix $\mS_i$ and a rotation matrix $\mR_i$ parameterized by a quaternion $\vr_i$. A scene is then described with a collection of such Gaussians $\mathcal{G} = \{G_i:\vmu_i, \vr_i, {\bm{s}}_i, \sigma_i, \vh_i\}_{i=1}^{N}$. We render an image $\mI$ and optionally its depth image $\mD$ from the 3D scene $\mathcal{G}$ by projecting each Gaussian onto the 2D image plane and aggregating them using $\alpha$-blending:
\begin{equation}
 \begin{aligned}
 \mI = \sum_{i=1}^{N}T_i\alpha_i^{\text{2D}}\mathcal{SH}(\vh_i, \vv_i), && \mD=\sum_{i=1}^N T_i\alpha_i^{\text{2D}}d_i, && \text{where} && T_i = \prod_{j=1}^{i-1}(1 - \alpha_j^{\text{2D}}).
 \end{aligned}
 \label{eq:gs_rendering}
\end{equation}
$\alpha_i^{\text{2D}}$ is a 2D version of~\cref{eq:gs_opacity}, with $\vmu_i$, $\mSigma_i$, $\vx$ replaced by the projected $\vmu_i^{\text{2D}}$, $\mSigma_i^{\text{2D}}$, and the pixel coordinate $\vu$. $\mathcal{SH}(\cdot)$ is the spherical harmonic function, $\vv_i$ is the view direction from the camera to $\vmu_i$, $d_i$ is the depth of the $i$-th Gaussian. Given $N_v$ input view images $\{\bar{\mI}_i, \bar{\mD}_i\}_{i=1}^{N_v}$, \ac{3dgs} learns Gaussians $\mathcal{G}$ with:
\begin{equation}
\begin{aligned}
\mathcal{L}_{\text{render}} = (1-\lambda_{\text{SSIM}})\mathcal{L}_I + \lambda_{\text{SSIM}}\mathcal{L}_{\text{D-SSIM}} + \mathcal{L}_D,
\end{aligned}
\label{eq:rendering_loss}
\end{equation}
where $\mathcal{L}_I = ||\mI - \bar{\mI}||_1$ is the L1-loss, $\mathcal{L}_{\text{D-SSIM}}$ is the D-SSIM loss~\citep{kerbl20233d}, $\lambda_{\text{SSIM}}$ is the weight of D-SSIM loss, and $\mathcal{L}_D = \log\left(1 + ||\mD - \bar{\mD}||_1\right)$ is the optional depth supervision. 

\paragraph{Mesh Extraction from Gaussians} To extract meshes from Gaussian splats $\mathcal{G}$, we can render depth maps and utilize \ac{tsdf} to fuse the reconstructed depth maps, and extract the object mesh $\mathcal{M}$ with marching cubes~\citep{huang20242d}. This process can be done with Open3D~\citep{zhou2018open3d} with proper choice of voxel size and truncated threshold.

% Gaussian Splatting uses a set of 3D Gaussians to represent 3D scenes~\citep{kerbl20233d}. Each Gaussian $g$ is defined by its color (represented by spherical harmonic coefficients $sh$), opacity $\sigma$, center position $x \in \mathbb{R}^3$ and a 3D covariance matrix $\Sigma \in \mathbb{R}^{3 \times 3}$, which governs the spatial extent and orientation of the Gaussian as $\alpha_i(p)=\sigma_i e^{-\frac{1}{2}(p-x_i)^T\Sigma^{-1}(p-x_i)}$. The covariance matrix $\Sigma$ can be decomposed into a rotation matrix $R \in SO(3)$ parameterized by a quaternion $r$ and a scaling diagonal matrix $S \in \mathbb{R}^{3\times3}$ parameterized by a scaling vector $s$, such that $\Sigma = RSS^TR^T$. A scene is then described by a collection of Gaussians $G = \{g_i: x_i, r_i, s_i, \sigma_i, sh_i\}$.

% To render an image from the 3D scene, each Gaussian is projected onto the 2D image plane. The final color $C(u)$ of a pixel $u$ is computed by aggregating the contributions of all Gaussians that project onto the pixel. This is done through $\alpha$-blending, where each Gaussian's color is weighted based on its opacity and spherical harmonic coefficients. By optimizing the Gaussian parameters $G = \{g_i : x_i, r_i, s_i, \sigma_i, sh_i\}$ and adaptively adjusting the density of the Gaussians, we can achieve real-time rendering of high-quality images. The color is expressed as $C(u) = \sum_{i=1}^{N} T_i \alpha_i SH(sh_i, v_i)$, where $v_i$ is the view direction, $SH(sh_i, v_i)$ is the evaluation of the spherical harmonic function for Gaussian $G_i$, and the term $T_i$ accumulates transparency and is defined as $T_i = \prod_{j=1}^{i-1} (1 - \alpha_j)$.



\section{Method}
\label{sec:method}
Given $N_v$ RGB-D images of an unknown articulated object $\{\bar{\mI}^t_i, \bar{\mD}^t_i\}_{i=1}^{N_v}$ at two joint states $t\in\{0,1\}$, we aim to reconstruct its part-level meshes $\mathcal{M}$ and joint articulation parameters $\Psi$. We define a set of learnable canonical Gaussians $\mathcal{G}^c$ which can be transformed into joint state Gaussians $\mathcal{G}^t$ via a per-Gaussian SE(3) transformation $T^{c\to t}$, parameterized by $\Psi$. Formally,
\begin{equation}
 \begin{aligned}
 \mathcal{G}^t = T^{c\to t}\cdot\mathcal{G}_c \quad \text{and} \quad \mathcal{G}^{c} = (T^{c\to t})^{-1}\cdot\mathcal{G}^t \quad \text{for} \quad t\in\{0,1\}. \\
 \end{aligned}
 \label{eq:motion_model}
\end{equation}
We impose the continuity of motion between the joint states by setting the canonical Gaussians $\mathcal{G}^{c}$ at the mid-state ($c$ : $t$ = 0.5), enforcing that $T^{c\to0}=(T^{c\to1})^{-1}$. This simplifies the articulation learning and connects the two input joint states through the canonical Gaussians $\mathcal{G}^c$, solving potential issues of occlusion and misinformation when reconstructing object meshes separately on the two joint states.



Using this motion model, we leverage multi-view RGB-D images from the two input states to learn both the canonical Gaussian $\mathcal{G}^c$, the transformation $T^{c\to1}$ or equivalently the joint parameters $\Psi$, and extract object meshes $\mathcal{M}^t$ for different joint states following~\cref{sec:prel}. An overview of \model is presented in~\cref{fig:overview}, with details on key designs provided in the following sections.

% Given multi-view RGB-D images $\{\mI_m^t, D_m^t\}_{m=1}^M$ at two states $t\in\{0,1\}$, we aim to build an interactable replica of an unknown articulated object by reconstructing its part-level meshes and estimating its joint articulation parameters $\phi$. We adopt Gaussian Splatting to reconstruct the geometry of the object, extracting its meshes by TSDF. To estimate its joint parameters $\phi$, we model its articulation by deforming canonical Gaussians $\mathcal{G}^c$ to two observation spaces as $\mathcal{G}^0$ and $\mathcal{G}^1$. For each canonical Gaussian $G^c_i$, its deformation $T_i^{c\rightarrow t}$ defined as $G^t_i=T_i^{c\rightarrow t}\cdot G^c_i$, is determined by its part segmentation mask and per-joint transformations parameterized by $\phi$. As illustrated in \cref{fig:overview}, our method is divided into 3 stages. Stage one trains Gaussians at two states individually to obtain the coarse canonical Gaussians by Hungarian Algorithm. Stage two learns type-free joint transformations to predict joint types (revolute or prismatic). Stage three further optimizes the joint parameters with type-constrained articulation. 

\begin{figure}[t!]
 \centering
 \resizebox{\linewidth}{!}{\includegraphics[width=\linewidth]{figure/overview.pdf}}
 \caption{\textbf{The overview of \model.} Our method is divided into two stages: (i) obtaining coarse canonical Gaussians $\mathcal{G}^c_{\text{init}}$ by matching the Gaussians $\mathcal{G}^0_{\text{single}}$ and $\mathcal{G}^1_{\text{single}}$ trained with each single-state individually and initializing the part assignment module with clustered centers, (ii) jointly optimizing canonical Gaussians $\mathcal{G}^c$ and articulation model (including the articulation parameters $\Psi$ and the part assignment module in \cref{sec:method:skinning}).}
 \label{fig:overview}
\end{figure}

\subsection{Coarse-to-Fine Canonical Gaussian Initialization with Motion Analysis}
\label{sec:method:canonical}
The initialization of the canonical Gaussians $\mathcal{G}^c$ is crucial for articulation learning. A good initialization leverages the consistency between input joint states, improving mesh reconstruction and articulation modeling. In contrast, a random initialization leads to undesirable local minima, adversely affecting the learning process (see in~\cref{fig:ablation}). To tackle this issue, we propose a coarse-to-fine strategy for the canonical Gaussian initialization, incorporating preliminary motion information from the two input joint states to enhance subsequent articulation modeling.
% Initializing the canonical Gaussians $\mathcal{G}^c$ is a challenging task. A well-chosen initialization can exploit the consistency between the provided input joint state, leading to improved mesh reconstruction and articulation modeling. In contrast, randomly initialized canonical Gaussians are susceptible to undesired local minima, which negatively impacts the articulation learning process (see an example in~\cref{fig:ablation}). To address this issue, our key observation is to utilize the continuity in object part movement between the input joint states as a constraint for canonical Gaussian initialization. We meticulously set the canonical Gaussians to be at the middle state of the two input joint states (\ie, $t=0.5$) and use the forward-backward consistency of the transformation as a constraint:
% \begin{equation}
% \mathcal{G}^1 = T^{c\rightarrow1} \cdot \mathcal{G}^c,\quad \mathcal{G}^0 = T^{c\rightarrow0} \cdot \mathcal{G}^c, \quad T^{c\rightarrow0}=(T^{c\rightarrow1})^{-1},
% \end{equation}
% where $T^{c\to 0} = (T^{c\to1})^{-1}$ comes from the assumption of continuous motion and that canonical state is the middle state. Under this assumption, we chain the two input joint states via the canonical Gaussian, solving potential issues of occlusion and view-point differences when modeling object meshes on separate states.


% % The major motivation for assuming the canonical Gaussians $\mathcal{G}^c$ is to chain the input two joint states via the canonical Gaussian, thereby addressing potential issues of previous works which only considers static states separately, thus suffering from occlusion and view-points issues. Nonetheless, learning the transformation and the canonical Gaussian jointly via only the multi-view input of two joint states is still challenging as it is essentially an joint optimization problem where object geometry and transformation compromise each other during learning, leading to degenerate solutions. To solve this issue, our key observation is to utilize the consistency between the two-joint states, setting the canonical state at an intermediate state ($t=0.5$), thereby adding constraint on the transformation, \ie:
% % % Deforming shared canonical Gaussians connects the two states as cues for each other, helping to address the problems of occlusion and viewpoint issues, which is a challenge for DTA that extracts meshes from static NeRFs of the two input states individually. 
% % % To further utilize the dynamic consistency between the two states, we set the canonical state at an intermediate state ($t=0.5$) with: 
% % \begin{equation}
% % \mathcal{G}^1 = T^{c\rightarrow1} \cdot \mathcal{G}^c,\quad \mathcal{G}^0 = T^{c\rightarrow0} \cdot \mathcal{G}^c, \quad T^{c\rightarrow0}=(T^{c\rightarrow1})^{-1},
% % \end{equation}
% % where we introduce forward-backward consistency from the canonical state (middle state) to the two input state.

% Obtaining canonical Gaussians $\mathcal{G}^c$ at a hypothetical state is challenging, due to the absence of direct supervision on the canonical space. Starting from randomly initialized canonical Gaussians makes it easy to fall into an unwanted local optimal solution, as shown in \cref{fig:ablation}. In the case of an error in the canonical Gaussians, it's hard to learn the correct articulated parts and their articulation parameters. 

% Given input images of the two object states, we first learn the Gaussian representation $\mathcal{G}_{\text{coarse}}^s$ for the two object states $s\in\{0,1\}$ following~\cref{eq:rendering_loss}. To model the deformation between these states, we define a canonical Gaussian representation $\mathcal{G}^c$ and a transformation function $T^{c\rightarrow s}$ parameterized by articulation parameters $\Psi$ that transforms $\mathcal{G}^c$ to a refined object state Gaussian $\mathcal{G}^s$.

% We learn the articulation parameters $\Psi$ using the deformation from the learned Gaussian representation $\mathcal{G}^0$ and $\mathcal{G}^1$ of the two states as a supervisory signal. Specifically, we define a canonical Gaussian $\mathcal{G}^c$ based on $\mathcal{G}^0$ and $\mathcal{G}^1$, and use $\Psi$ to parameterize the transformation $T^{c\rightarrow t}$ from the canonical Gaussian to the Gaussian $\mathcal{G}^t$ at state $t$ for modeling the deformation process. 

% \subsubsection{Canonical Gaussians Initialization}
% \label{sec:method:articulation:cano}

\paragraph{Coarse Initialization by Matching Single-state Gaussians} 
In this phase, we first separately train two sets of single-state Gaussians $\mathcal{G}^t_{\text{single}}$ with input multi-view images following~\cref{eq:rendering_loss}. We then apply Hungarian Matching to obtain matched Gaussian pairs between $\mathcal{G}_{\text{single}}^0$ and $\mathcal{G}_{\text{single}}^1$, based on the distance between Gaussian centers. We take the mean of each pair of matched Gaussians as the coarse canonical Gaussian initialization $\mathcal{G}_{\text{coarse}}^c$. To reduce the significant computation time associated with matching a large number of Gaussians, we use \ac{fps} to downsample the learned single-state Gaussians to a set of 5K Gaussians prior to matching. 

% \paragraph{Obtaining Canonical Gaussians by Matching Coarse Gaussians.} To solve this problem, we propose to obtain coarse canonical Gaussians by matching coarse Gaussians of 2 states. 
% Specifically, we individually train 2 sets of coarse Gaussians $\hat{\mathcal{G}}^0$, $\hat{\mathcal{G}}^1$ with multi-view RGB images. Given the positions of these Gaussians, we match two sets of Gaussians by Hungarian Algorithm to obtain paired Gaussians. We take the mean of all properties of the paired Gaussians to initialize the canonical Gaussians. The canonical Gaussians obtained by Hungarian Matching are not accurate, especially for prismatic joints, but they can provide a good initialization for subsequent training and prevent the model from learning various distorted canonical spaces. 

\paragraph{Initialization Refinement with Motion Analysis} 
To support geometry reconstruction and articulation modeling, relying solely on 5K matched coarse Gaussians alone is insufficient. Therefore, we refine the coarse initialization $\mathcal{G}_{\text{coarse}}^c$ guided by the motion information of object parts.
Intuitively, single-state Gaussians, $\mathcal{G}_{\text{single}}^{0}$ and $\mathcal{G}_{\text{single}}^{1}$, should exhibit consistency for static object parts discrepancies for movable parts, \ie, the static parts of these Gaussians are well-learned. Based on this insight, we refine the set of coarse canonical Gaussians $\mathcal{G}_{\text{coarse}}^c$ by including Gaussians corresponding to static parts, allowing more focused learning of movable parts during articulation modeling. In practice, we classify each Gaussian $G_i$ in a joint state $t$ as static or dynamic by calculating its minimum Chamfer Distance to all Gaussians in the opposite state $\bar{t}$:

\begin{equation}
\text{CD}_i^{t\to \bar{t}} = \min_{j}||\vmu^t_i - \vmu^{\bar{t}}_j||_2,\quad G_i \in \mathcal{G}_{\text{single}}^t, G_j\in\mathcal{G}_{\text{single}}^{\bar{t}}\quad\text{and}\quad \text{CD}^{t\to\bar{t}}=\underset{i}{\mathrm{Mean}}\left(\text{CD}_i^{t\to\bar{t}}\right).
\label{eq:mobility_identification}
\end{equation}

If the distance $\text{CD}_i^{t\to\bar{t}}$ exceeds a threshold $\epsilon_{\text{static}}$, $G_i$ is classified as dynamic; otherwise it is static. To determine which state, $t$ or $\bar{t}$, contains more motion information, we compare the mean distance $\text{CD}^{t\to\bar{t}}$ of all Gaussians in state $t$ following~\cref{eq:mobility_identification} and classify the higher state as the more motion informative state. For instance, a cabinet with open drawers provides clearer identification of movable parts than one with closed drawers. With this information, we add the static Gaussians from the more motion informative state to refine $\mathcal{G}_{\text{coarse}}^c$ into the final initialization of the canonical Gaussian $\mathcal{G}^c_{\text{init}}$.

% At the same time, the number of matched Gaussians (5K) is not enough, so we hope to add static Gaussians shared by the two states. To identify which points are static or mobile, we compute the single-directional Chamfer Distance between two states of Gaussians, denoted as $CD_{0\rightarrow1} \in \mathbb{R}^{N_0}$ and $CD_{1\rightarrow0} \in \mathbb{R}^{N_1}$. If the Chamfer Distance of a Gaussian exceeds a threshold $\eps_{cd}$, the Gaussian is considered mobile; otherwise, it is considered static. If the average Chamfer Distance of all Gaussians in a state is higher, we consider that state to have greater motion and contain more information (imagine that a cabinet with open drawers reveals more information than one with closed drawers). We select all static Gaussians from this state to initialize the canonical space. 
% Additionally, these static and movable Gaussians could be used to better initialize our part segmentation module, which will be described in \cref{sec:method:skinning}. 

% Since the computation time required for the matching process increases significantly with the number of Gaussians, we utilize farthest point sampling to downsample 5K Gaussians at each stage. At the same time, the number of matched Gaussians (5K) is not enough, so we hope to add static Gaussians shared by the two states. To identify which points are static or mobile, we compute the single-directional Chamfer Distance between two states of Gaussians, denoted as $CD_{0\rightarrow1} \in \mathbb{R}^{N_0}$ and $CD_{1\rightarrow0} \in \mathbb{R}^{N_1}$. If the Chamfer Distance of a Gaussian exceeds a threshold $\eps_{cd}$, the Gaussian is considered mobile; otherwise, it is considered static. If the average Chamfer Distance of all Gaussians in a state is higher, we consider that state to have greater motion and contain more information (imagine that a cabinet with open drawers reveals more information than one with closed drawers). We select all static Gaussians from this state to initialize the canonical space. Additionally, these static and movable Gaussians could be used to better initialize our part segmentation module, which will be described in \cref{sec:method:skinning}. 

\subsection{Part Discovery for Articulation Modeling}
\label{sec:method:skinning}
Following~\cref{eq:motion_model}, we use a part-based formulation for articulation modeling. Specifically, given the number of parts $K$, we aim to decompose the Gaussians into $K$ parts and learn the articulation paramerters $\Psi = \{T_{k}^{c\to 1}\}_{k=1}^{K}$. In contrast to existing works that leverage prior information for part discovery~\citep{mandi2024real2code,weng2024neural}, we discover parts in an unsupervised manner during learning.

% To transform Gaussians from the canonical space to the observation space, we start by inputting their positions $\rmX \in \mathbb{R}^{N\times 3}$ into a segmentation module to assign these Gaussians to $K$ parts, obtaining a part-level mask $\rmM \in \mathbb{R}^{N\times K}$.

% \subsubsection{Part Segmentation with Gaussian Skinning}
\paragraph{Center-based Part Modeling and Assignment}
Given input canonical Gaussians $\mathcal{G}^c = \{G_i\}_{i=1}^{N}$, our objective is to compute part-level masks $\rmM\in\mathbb{R}^{N\times K}$ that assign each Gaussian $\mathcal{G}_i$ to a specific part. A common approach to generating these assignment masks is through unsupervised segmentation modules using MLPs or slot-attention~\citep{locatello2020sa,jia2023improving,liu2025slotlifter}. However, these models implicitly segment parts and fail to leverage the explicit spatial and dynamic information present in 3D Gaussians. We observe that such methods struggle with parts that exhibit similar motion patterns, leading to incorrect assignments. To address this issue, we adopt a center-based part modeling approach that explicitly utilizes spatial information, inspired by sparse control points from SC-GS~\citep{huang2024sc} and quasi-rigid blend skinning in REACTO~\citep{song2024reacto}. Specifically, we define $K$ learnable centers $C_k = (\vp_k, \mV_k, {\bm \lambda}_k)$ with center location $\vp_k\in\mathbb{R}^3$, rotation matrix $\mV_k\in\mathbb{R}^{3\times 3}$, and scale vector ${\bm \lambda}_k\in\mathbb{R}^3$. 
For a given Gaussian $G_i \in \mathcal{G}^c$, we compute the Mahalanobis distance $\mD_{ik}$ between $G_i$ and center $C_k$ as:
\begin{equation}
 \rmX^k_i = \frac{[\mV_k(\vmu^c_i-\vp_k)]}{{\bm\lambda}_k} \quad
 \rmD_{i}^k = (\rmX^k_i)^T\cdot\rmX^k_i\quad\text{and}\quad\bm{M} = \mathrm{GumbelSoftmax}\left(\frac{-\mD + \mW_{\Delta}}{\tau}\right)
 \label{eq:part_assignment}
\end{equation}
where $\mD_{i}^k$ is the distance matrix for part assignment. One challenge of using the distance matrix for part assignment is identifying sharp boundaries when two parts overlap spatially (\eg, in the case of a closed drawer). To improve boundary identification, we introduce a residual term $\mW_\Delta = \mathrm{MLP}(\vmu, \mX, \mD)$, predicted by a shallow MLP that concatenates the absolute position of each Gaussian and the distance matrix $\mD$ as input. This residual is added to the original distance matrix $\mD$ to refine the part assignment mask following \cref{eq:part_assignment}. Notably, we use Gumbel Softmax to ensure that each Gaussian is assigned to only one part, which simplifies the optimization of joint parameters. Detailed implementation can be found in~\cref{app:imp:assignment}.

% The part segmentation masks could be calculated as $M = \mathrm{Gumbel Softmax}(\frac{-D + \delta}{\tau})$, where $\tau$ is a temperature factor that anneals with the training process. Notably, we use Gumbel Softmax to ensure that each Gaussian is assigned to only one part, thereby facilitating the optimization of the joint parameters for each part. 

% the motion of two parts is similar, these implicit models cannot distinguish the parts based on motion information alone (see \cref{fig:ablation_seg}).
% In order to make better use of spatial information to cluster Gaussian, we adopt a center-based segmentation strategy similar to the control points in SC-GS~\citep{huang2024sc} and Gaussian-based skinning in REACTO~\citep{song2024reacto}. 
% Specifically, given the positions $\rmX^c_i$ of canonical Gaussians, we compute the Mahalanobis distance $\rmD_{ik}$ between $\rmX^c_i$ and $K$ learnable centers $c_k=(\mu_k, R_k, \lambda_k)$ as:
% \begin{equation}
% \rmX^k_i = [R_k(\rmX^c_i-\mu_k)] / \lambda_k \quad
% \rmD_{ik} = (\rmX^k_i)^T\cdot\rmX^k_i ,
% \end{equation}
% where $\mu_k\in \mathbb{R}^3, R_k\in \mathbb{R}^{3\times3}, \lambda_k\in \mathbb{R}^3$ are center vector, rotation matrix and scale vector respectively. 
% Although this modeling approach effectively utilizes spatial information, it cannot model sharp boundaries, especially when two parts have some overlap in space (e.g., a closed drawer), making it difficult to segment the parts accurately. Therefore, we use a hash grid to encode the absolute position of each Gaussian, predicting a residual $\delta$ with a shallow MLP and adding it to the original distance $\rmD_{ik}$. The part segmentation masks could be calculated as $M = \mathrm{Gumbel Softmax}(\frac{-D + \delta}{\tau})$, where $\tau$ is a temperature factor that anneals with the training process. Notably, we use Gumbel Softmax to ensure that each Gaussian is assigned to only one part, thereby facilitating the optimization of the joint parameters for each part. 

\paragraph{Center Initialization by Clustering Coarse Gaussians}
We empirically find that the initialization of centers $\vp_k$ and scale ${\bm \lambda}_k$ have great impacts on the correctness of part discovery in later learning process (see~\cref{fig:ablation}). Therefore, similar to the canonical Gaussian initialization described in~\cref{sec:method:canonical}, we utilize the motion type of each joint as additional information for providing good initializations of part centers. Specifically, we select the input joint state with more motion information to identify static and dynamic parts. For static parts, we take the mean of the Gaussians as the part center. For movable parts, we do spectral clustering on the positions of movable Gaussians ($K-1$ clusters) and take the mean of each cluster for part center initialization. We use the distance from the farthest point to the center of each cluster as the initial scale.

% Although this center-based segmentation method certainly makes use of spatial information, we found that the random initialization of the centers $\mu_k$ and scales $\lambda_k$ have a great impact on the final result. Therefore, we consider getting a better initialization by clustering existing coarse Gaussians. 
% As shown in the \cref{fig:ablation}, simply clustering all these Gaussians (\ie, w/o motion prior) does not result in reasonable clustering centers, as clustering by spatial location alone is a big challenge. We propose to solve this problem with motion information. 
% Firstly, as described in \cref{sec:method:cano}, we choose Gaussians of states with more motion. The mean of the positions of the static Gaussians is used as the center of the static part. We do spectral clustering on the positions of movable Gaussians and take the mean of each cluster to get its center. Half of the distance from the farthest point to the center of each cluster is used as the initial value of scale.

\subsection{Self-guided Articulation Type and Parameter Learning}\label{sec:method:dual_quaternion}
After obtaining object part representations, we define the per-part articulation parameters via dual-quaternions. Formally, the joints articulation parameters $\Psi = \{T_k^{c\to 1}\}_{k=1}^{K} = \{\vq_k^{c\to1}:({\vq}_{k,r}, \vq_{k, d})\}_{k=1}^K$, where $\vq_{k,r}$ and ${\vq}_{k,d}$ are the real and dual part of the dual-quaternion that determine the rotation and translation of the joint transformation respectively. For notational simplicity, we use $\vq_k^t$ for $\vq_k^{c\rightarrow t}$ in the following texts. With the mid-state assumption in~\cref{sec:method}, we have $\vq_{k}^0 = (\vq_k^{1})^{-1}$ is the inverse of dual-quaternion $\vq_{k}^1$. Given object masks $\mM$ obtained in~\cref{sec:method:skinning}, the per-gaussian dual-quaternion $\vq_i$ for Gaussian $G_i \in \mathcal{G}^c$ is given by:
\begin{equation}
\begin{aligned}
\vq_i^t =(\sum_{k=1}^K \mM_{ik}\cdot {\vq^t_{k,r}},\sum_{k=1}^K \mM_{ik}\cdot {\vq^t_{k,d}}). 
\end{aligned}
\end{equation}
where $(\vq_k^1)^{-1}$ is the inverse of dual-quaternion $\vq_k^{1}$ and $\mM_{ik}$ denotes the probability of Gaussian $i$ belongs to part $k$.
With the per-gaussian transformation given $\vq_{i}^{t}$, we transform the canonical Gaussian $\mathcal{G}^c$ to get the two joint state Gaussians $\mathcal{G}^t$ with:
\begin{equation}
\begin{aligned}
\vmu_i^t = \mR_i^{c\rightarrow t}\cdot \vmu_i^c + \vt_i^{c\rightarrow t}, \quad \vr_i^t = {\vq}_{i,r}^{t} \otimes \vr_i^c,\\
\end{aligned}
\label{eq:articulation_modeling}
\end{equation}
where $\mR_i^{c\rightarrow t}$ and $\vt_i^{c\rightarrow t}$ is the per-gaussian rotation matrix and translation vector derived from $\vq_i^{t}$, and $\otimes$ denotes quaternion multiplication operation. We assume that the scale ${\bm s}_i$ and opacity $\sigma_i$ of the Gaussian $G_i$ remains consistent under transformation.

To enhance the learning of articulation parameters, we adopt a warm-up strategy for predicting the joint type of each part. During the warm-up stage, we optimize the articulation parameters $\Psi=\{\vq_k^{1}\}_{k=1}^{K}$ without any constraints. Next, we develop a heuristic for joint type prediction based on the learned rotation ${\vq_{k,r}}$. Specifically, we classify the joint as revolute if the rotation degree of ${\vq_{k,r}}$ exceeds a threshold $\epsilon_{\text{revol}}$, and otherwise prismatic. With predicted joint types, we constrain the joint transformation for each part. Specifically, we manually set the rotation quaternion $\vq_{k,r}$ of prismatic joints as identity quaternion. This operation allows the model to focus on optimizing the translation term $\vq_{k,d}$ of the prismatic joint, thereby obtaining a more accurate estimate of the joint parameters. 
% Similarly, we can also restrict the translation term of revolute joint as \bx{$\vq_{k,d}=\frac{1}{2}(\vq_{d0}\otimes \vq_r-\vq_r\otimes \vq_{d0})$, where $\vq_{d0}=(0, \vt_0)$ and $\vt_0\in \mathbb{R}^3$ is the axis position of the revolute joint. d0????} However, we found in our experiments that restricting revolute joints does not always bring benefits. Some objects are beneficial while others are not, because this reduction of the optimization space may make it more difficult to find an optimal solution. Thus, we only restrict the prismatic joints.
\subsection{Optimization}
We train our model using the rendering loss with depth supervision $\mathcal{L}_{\text{render}}$ described in~\cref{sec:prel} on the reconstructed $\mathcal{G}^t$ for the two joint states as discussed in~\cref{sec:method:dual_quaternion}. To reduce the chances of learning artifacts during update, we use the single-state reconstructed Gaussians $\mathcal{G}_\text{single}^t$ as an additional supervision:
\begin{equation}
\label{eq:cd_loss}
 \mathcal{L}_{\text{CD}}=\frac{1}{N}\sum_{i=1}^{N}\min_{j}||\vmu^t_i - \vmu^{t}_j||_2\quad,\quad G_i \in \mathcal{G}^t,\quad\text{and}\quad G_j\in\mathcal{G}_{\text{single}}^{t},
\end{equation}
where we calculate the single-direction Chamfer Distance between the deformed Gaussians $\mathcal{G}^t$ and single-state reconstructed Gaussians $\mathcal{G}^t_\text{single}$ as the loss signal. As these single-state Gaussians are only a rough estimate, we only introduce this loss in the first 1K to 5K steps.
% Additionally, to compare with DTA fairly, we add a depth loss $\mathcal{L}_{depth}=\log (1 + |\mathrm{depth_{gt}} - \mathrm{depth_{pred}}|)$. 
Additionally, to regularize the learning of part centers $\vp_k$, we add another regularization loss as:
\begin{equation}
\mathcal{L}_{\text{reg}}=\frac{1}{K}\sum_{k=1}^K||\vp_k-\hat{\vp}_k||_2, \quad \mathrm{where}\quad \hat{\vp}_k=\sum_{i=1}^{N}\frac{\mM_{ik}}{\sum_{i=1}^{N}\mM_{ik}}\vmu_i,
\end{equation}
which enforces that the centers $\vp_k$ should be close to the average spatial position of Gaussians in canonical Gaussians $\mathcal{G}^c$ that belong to part $k$. Above all, our supervision could be summarized as:
\begin{equation}
\mathcal{L} = \mathcal{L}_{\text{render}} + \lambda_{\text{CD}}\mathcal{L}_{\text{CD}} + \lambda_{\text{reg}}\mathcal{L}_{\text{reg}}.
\end{equation}
We provide more implementation and model training details in~\cref{app:imp}.

\section{Experiments}
\label{sec:exp}

% \bx{In this section, we conducted extensive experiments to investigate the following questions:
% \begin{itemize}[leftmargin=*,noitemsep,nolistsep]
% \item How good is our proposed \model on both synthetic and real-world objects?
% \item Is \model able to generalize to complex multi-part articulated objects?
% \item How significant are our designs (obtaining canonical Gaussians, center-based segmentation and center initialization) in \model?
% \end{itemize}
% }

\paragraph{Datasets}
We evaluate our method on three datasets:
(1) PARIS, a two-part dataset proposed by~\cite{jiayi2023paris}, which features articulated objects consisting of one static and one movable part. It includes 10 synthetic objects from the PartNet-Mobility dataset~\citep{xiang2020sapien} and 2 real-world objects captured using the MultiScan~\citep{mao2022multiscan} toolset. 
(2) DTA-Multi, a dataset proposed by~\cite{weng2024neural}, containing 2 synthetic multi-part articulated objects from PartNet-Mobility, each with one static part and two movable parts. 
(3) \model-Multi, our newly curated dataset, featuring 5 complex articulated objects from PartNet-Mobility with 3 to 6 movable parts. 

\paragraph{Metrics} Following the evaluation protocols of PARIS~\citep{jiayi2023paris} and DTA~\citep{weng2024neural}, we assess the performance of all methods using both mesh reconstruction and articulation estimation metrics. 
For mesh reconstruction, we compute the bi-directional Chamfer Distance between the reconstructed mesh and the ground truth mesh with 10K uniformly sampled points from each mesh. We report the Chamfer Distance for the whole object (CD-w), the static parts (CD-s), and the movable parts (CD-m).
For articulation estimation, we evaluate the predicted articulation using the angular error (Axis Ang.) and the distance (Axis Pos., revolute joint only) between the predicted and ground-truth joint axes. We also report the part motion error (Part Motion) which measures the rotation geodesic distance error (in degrees) for revolute joints and Euclidean distance error (in meters) for prismatic joints.

% Axis Angle Error ($^\degree$): the angular error of the predicted joint axis, (2) Axis Position Error (0.1m): Distance between predicted and ground-truth joint axes for revolute joints, which is not used for prismatic joints. (3) Part Motion Error: Rotation geodesic distance error ($^\degree$) for revolute joints, or Euclidean distance error (m) for prismatic joints.

\subsection{Results on Simple Articulated Objects}
\label{sec:exp:two-part}
\paragraph{Experimental Setup} We use the PARIS dataset as the benchmark and select Ditto~\citep{hsu2023ditto}, PARIS~\citep{jiayi2023paris}, CSG-reg~\citep{weng2024neural}, 3Dseg-reg~\citep{weng2024neural}, and DTA~\citep{weng2024neural} as baselines for quantitative evaluation. Following the evaluation setting from DTA~\citep{weng2024neural}, we report all metrics with mean $\pm$ std over 10 trials calculated at the high-visibility joint state. We re-train DTA on the same device (NVIDIA RTX 3090) for training time comparison. Additional results on all joint states are provided in~\cref{tab:app:exp_2part}.

\paragraph{Results}
As shown in \cref{tab:exp_2part}, our method significantly outperforms existing approaches across all metrics, especially for joint articulation parameter estimation, where \model achieves substantially lower errors. This improvement stems from our motion model with Gaussian Splatting, which explicitly deforms Gaussians for more precise part transformation modeling, leading to more precise joint parameter estimation. For mesh reconstruction, \model excels in reconstructing movable parts, yielding lower CD-m values, especially for real-world objects. While DTA performs well on CD-w and CD-s due to its state-by-state reconstruction, we show in~\cref{fig:2part} that it struggles with the low-visibility state. In contrast, \model achieves significantly better results on the low-visibility state while maintaining competitive results on the high-visibility state. This is attributed to the canonical Gaussians modeling that connects the two input joint states for mutually improved mesh reconstruction. Additionally, \model shows consistently better results on real-world objects with significantly faster training time, positioning it as an efficient solution for building digital twins of real-world articulated objects.
\begin{table*}[t!]
\caption{\textbf{Quantitative evaluation on PARIS.} Metrics are reported as mean $\pm$ std over 10 trials at the joint state with higher visibility, following \citep{weng2024neural}. PARIS$^*$~\citep{jiayi2023paris} is augmented with depth for fair comparison. DTA is re-trained for time efficiency comparison. Lower ($\downarrow$) is better on all metrics and we highlight \colorbox[HTML]{ffc5c5}{best} and \colorbox[HTML]{ffebd8}{second best} results. Objects with $\dagger$ are seen categories trained in Ditto. F indicates wrong motion type predictions. Axis Pos. is omitted for prismatic joints (Blade, Storage, and Real Storage).}
\label{tab:exp_2part}
\renewcommand{\arraystretch}{1.2}
\resizebox{\linewidth}{!}{
\begin{tabular}{cc|ccccccccccc|ccc}
\hline
\multirow{2}{*}{Metric} &\multirow{2}{*}{Method} &\multicolumn{11}{c}{Synthetic Objects} &\multicolumn{3}{|c}{Real Objects} \\
% \cmidrule(lr){}
& &FoldChair &Fridge &Laptop$^\dagger$ &Oven$^\dagger$ &Scissor &Stapler &USB 
&Washer&Blade &Storage$^\dagger$ &All & Fridge &Storage &All \\
\hline
\multirow{6}{*}{\shortstack{Axis\\Ang}} 
&Ditto
&89.35 &89.30 &3.12 &0.96 &4.50 &89.86 &89.77 &89.51 &79.54 &6.32 &54.22 &\best{1.71} &\secbest{5.88} &\secbest{3.80} \\
% &PARIS
% &8.08\tiny{$\pm$13.2} &9.15\tiny{$\pm$28.3} &0.02\tiny{$\pm$0.0} &0.04\tiny{$\pm$0.0} &3.82\tiny{$\pm$3.4} &39.73\tiny{$\pm$35.1} &0.13\tiny{$\pm$0.2} &25.36\tiny{$\pm$30.3} &15.38\tiny{$\pm$14.9} &0.03\tiny{$\pm$0.0} &10.17\tiny{$\pm$12.5} &1.64\tiny{$\pm$0.3} &43.13\tiny{$\pm$23.4} &22.39\tiny{$\pm$11.9} \\
&PARIS*
&15.79\tiny{$\pm$29.3} &2.93\tiny{$\pm$5.3} &\secbest{0.03\tiny{$\pm$0.0}} &7.43\tiny{$\pm$23.4} &16.62\tiny{$\pm$32.1} &8.17\tiny{$\pm$15.3} &0.71\tiny{$\pm$0.8} &18.40\tiny{$\pm$23.3} &41.28\tiny{$\pm$31.4} &\secbest{0.03\tiny{$\pm$0.0}} &11.14\tiny{$\pm$16.1} &\secbest{1.90\tiny{$\pm$0.0}} &30.10\tiny{$\pm$10.4} &16.00\tiny{$\pm$5.2} \\
&CSG-reg
&0.10\tiny{$\pm$0.0} &0.27\tiny{$\pm$0.0} &0.47\tiny{$\pm$0.0} &0.35\tiny{$\pm$0.1} &0.28\tiny{$\pm$0.0} &0.30\tiny{$\pm$0.0} &11.78\tiny{$\pm$10.5} &71.93\tiny{$\pm$6.3} &7.64\tiny{$\pm$5.0} &2.82\tiny{$\pm$2.5} &9.60\tiny{$\pm$2.4} &8.92\tiny{$\pm$0.9} &69.71\tiny{$\pm$9.6} &39.31\tiny{$\pm$5.2} \\
&3Dseg-reg
&- &- &2.34\tiny{$\pm$0.11} &- &- &- &- &- &9.40\tiny{$\pm$7.5} &- &- &- &- &- \\
&DTA
&\secbest{0.03\tiny{$\pm$0.0}} &\secbest{0.09\tiny{$\pm$0.0}} &{0.07\tiny{$\pm$0.0}} &\secbest{0.22\tiny{$\pm$0.1}} &\secbest{0.10\tiny{$\pm$0.0}} &\secbest{0.07\tiny{$\pm$0.0}} &\secbest{0.11\tiny{$\pm$0.0}} &\secbest{0.36\tiny{$\pm$0.1}} &\secbest{0.20\tiny{$\pm$0.1}} &{0.09\tiny{$\pm$0.0}} &\secbest{0.13\tiny{$\pm$0.0}} &{2.08\tiny{$\pm$0.0}} 
&13.64\tiny{$\pm$3.6} &7.86\tiny{$\pm$1.8} \\
&Ours
&\best{0.01\tiny{$\pm$0.0}} &\best{0.03\tiny{$\pm$0.0}} &\best{0.01\tiny{$\pm$0.0}} &\best{0.01\tiny{$\pm$0.0}} &\best{0.05\tiny{$\pm$0.0}} &\best{0.01\tiny{$\pm$0.0}} &\best{0.04\tiny{$\pm$0.0}} &\best{0.02\tiny{$\pm$0.0}} &\best{0.03\tiny{$\pm$0.0}} &\best{0.01\tiny{$\pm$0.0}} &\best{0.02\tiny{$\pm$0.0}} & 2.09\tiny{$\pm$0.0} &\best{3.47\tiny{$\pm$0.3}} &\best{2.78\tiny{$\pm$0.2}} \\
\hline
\multirow{6}{*}{\shortstack{Axis\\Pos}} 
&Ditto
&3.77 &1.02 &\secbest{0.01} &0.13 &5.70 &0.20 &5.41 &0.66 &- &- &2.11 &1.84 &- &1.84 \\
% &PARIS 
% &0.45\tiny{$\pm$0.9} &0.38\tiny{$\pm$1.0} &\best{0.00\tiny{$\pm$0.0}} &0.00\tiny{$\pm$0.0} &2.10\tiny{$\pm$1.4} &2.27\tiny{$\pm$3.4} &2.36\tiny{$\pm$3.4} &1.50\tiny{$\pm$1.3} &- &- &1.13\tiny{$\pm$1.1} &\secbest{0.34\tiny{$\pm$0.2}} &- &\secbest{0.34\tiny{$\pm$0.2}} \\
&PARIS*
&0.25\tiny{$\pm$0.5} &1.13\tiny{$\pm$2.6} &\best{0.00\tiny{$\pm$0.0}} &0.05\tiny{$\pm$0.2} &1.59\tiny{$\pm$1.7} &4.67\tiny{$\pm$3.9} &3.35\tiny{$\pm$3.1} &3.28\tiny{$\pm$3.1} &- &- &1.79\tiny{$\pm$1.5} &\secbest{0.50\tiny{$\pm$0.0}} &{-} &\secbest{0.50\tiny{$\pm$0.0}} \\
&CSG-reg 
&0.02\tiny{$\pm$0.0} &\best{0.00\tiny{$\pm$0.0}} &0.20\tiny{$\pm$0.2} &0.18\tiny{$\pm$0.0} &\secbest{0.01\tiny{$\pm$0.0}} &\secbest{0.02\tiny{$\pm$0.0}} &\secbest{0.01\tiny{$\pm$0.0}} &2.13\tiny{$\pm$1.5} &- &- &0.32\tiny{$\pm$0.2} &1.46\tiny{$\pm$1.1} &- &1.46\tiny{$\pm$1.1} \\
&3Dseg-reg 
&- &- &0.10\tiny{$\pm$0.0} &- &- &- &- &- &- &- &- &- &- &- \\
&DTA
&\secbest{0.01\tiny{$\pm$0.0}} &\secbest{0.01\tiny{$\pm$0.0}} &\secbest{0.01\tiny{$\pm$0.0}}
&\secbest{0.01\tiny{$\pm$0.0}} &{0.02\tiny{$\pm$0.0}} &\secbest{0.02\tiny{$\pm$0.0}} &\best{0.00\tiny{$\pm$0.0}} &\secbest{0.05\tiny{$\pm$0.0}} 
&{-} &{-} &\secbest{0.02\tiny{$\pm$}0.0} &0.59\tiny{$\pm$0.0}
&- &0.59\tiny{$\pm$0.0} \\
&Ours
&\best{0.00\tiny{$\pm$0.0}} &\best{0.00\tiny{$\pm$0.0}} &\secbest{0.01\tiny{$\pm$0.0}} &\best{0.00\tiny{$\pm$0.0}} &\best{0.00\tiny{$\pm$0.0}} &\best{0.01\tiny{$\pm$0.0}} &\best{0.00\tiny{$\pm$0.0}} &\best{0.00\tiny{$\pm$0.0}} & {-} & {-} &\best{0.00\tiny{$\pm$0.0}} &\best{0.47\tiny{$\pm$0.0}} & {-} &\best{0.47\tiny{$\pm$0.0}} \\
\hline
\multirow{6}{*}{\shortstack{Part\\Motion}}
&Ditto
&99.36 &F &5.18 &2.09 &19.28 &56.61 &80.60 &55.72 &F &0.09 &39.87 &8.43 &0.38 &4.41 \\
% &PARIS
% &131.66\tiny{$\pm$78.9} &24.58\tiny{$\pm$57.7} &0.03\tiny{$\pm$0.0} &0.03\tiny{$\pm$0.0} &120.70\tiny{$\pm$50.1} &110.80\tiny{$\pm$47.1} &64.85\tiny{$\pm$84.3} &60.35\tiny{$\pm$23.3} &0.34\tiny{$\pm$0.2} &0.30\tiny{$\pm$0.0} &51.36\tiny{$\pm$34.2} &\best{2.16\tiny{$\pm$1.1}} &0.56\tiny{$\pm$0.4} &1.36\tiny{$\pm$0.7} \\
&PARIS*
&127.34\tiny{$\pm$75.0} &45.26\tiny{$\pm$58.5} &\secbest{0.03\tiny{$\pm$0.0}} &9.13\tiny{$\pm$28.8} &68.36\tiny{$\pm$64.8} &107.76\tiny{$\pm$68.1} &96.93\tiny{$\pm$67.8} &49.77\tiny{$\pm$26.5} &0.36\tiny{$\pm$0.2} &0.30\tiny{$\pm$0.0} &50.52\tiny{$\pm$39.0} &\best{1.58\tiny{$\pm$0.0}} &0.57\tiny{$\pm$0.1} &1.07\tiny{$\pm$0.1} \\
&CSG-reg 
&0.13\tiny{$\pm$0.0} &0.29\tiny{$\pm$0.0} &0.35\tiny{$\pm$0.0} &0.58\tiny{$\pm$0.0} &\secbest{0.20\tiny{$\pm$0.0}} &0.44\tiny{$\pm$0.0} &10.48\tiny{$\pm$9.3} &158.99\tiny{$\pm$8.8} &\secbest{0.05\tiny{$\pm$0.0}} &\secbest{0.04\tiny{$\pm$0.0}} &17.16\tiny{$\pm$1.8} &14.82\tiny{$\pm$0.1}
&0.64\tiny{$\pm$0.1} &7.73\tiny{$\pm$0.1} \\
&3Dseg-reg
&- &- &1.61\tiny{$\pm$0.1} &- &- &- &- &- &0.15\tiny{$\pm$0.0} &- &- &- &- &- \\
&DTA
&\secbest{0.10\tiny{$\pm$0.0}} &\secbest{0.12\tiny{$\pm$0.0}} &0.11\tiny{$\pm$0.0} &\secbest{0.12\tiny{$\pm$0.0}} &{0.37\tiny{$\pm$0.6}} &\secbest{0.08\tiny{$\pm$0.0}} &\secbest{0.15\tiny{$\pm$0.0}} 
&\secbest{0.28\tiny{$\pm$0.1}} &\best{0.00\tiny{$\pm$0.0}} &\best{0.00\tiny{$\pm$0.0}} &\secbest{0.13\tiny{$\pm$0.1}} &\secbest{1.85\tiny{$\pm$0.0}} &\secbest{0.14\tiny{$\pm$0.0}} &\secbest{1.00\tiny{$\pm$0.0}} \\
&Ours
&\best{0.03\tiny{$\pm$0.0}} &\best{0.04\tiny{$\pm$0.0}}
&\best{0.02\tiny{$\pm$0.0}} &\best{0.02\tiny{$\pm$0.0}}
&\best{0.04\tiny{$\pm$0.0}} &\best{0.01\tiny{$\pm$0.0}}
&\best{0.03\tiny{$\pm$0.0}} &\best{0.03\tiny{$\pm$0.0}}
&\best{0.00\tiny{$\pm$0.0}} &\best{0.00\tiny{$\pm$0.0}}
&\best{0.02\tiny{$\pm$0.0}} & 1.94\tiny{$\pm$0.0}
&\best{0.04\tiny{$\pm$0.0}} &\best{0.99\tiny{$\pm$0.0}} \\
\hline
\multirow{6}{*}{\shortstack{CD-s}} 
&Ditto
&33.79 &3.05 &\secbest{0.25} &\best{2.52} &39.07 &41.64 &2.64 &10.32 &46.90 &9.18 &18.94 &47.01 &16.09 &31.55 \\
% &PARIS 
% &9.16\tiny{$\pm$5.0} &3.65\tiny{$\pm$2.7} &\best{0.16\tiny{$\pm$0.0}} &\secbest{12.95\tiny{$\pm$1.0}} &1.94\tiny{$\pm$3.8} &\best{1.88\tiny{$\pm$0.2}} &\best{2.69\tiny{$\pm$0.3}} &25.39\tiny{$\pm$2.2} &1.19\tiny{$\pm$0.6} &12.76\tiny{$\pm$2.5} &7.18\tiny{$\pm$1.8} &42.57\tiny{$\pm$34.1} &54.54\tiny{$\pm$30.1} &48.56\tiny{$\pm$32.1} \\
&PARIS*
&10.20\tiny{$\pm$5.8} &8.82\tiny{$\pm$12.0} &\best{0.16\tiny{$\pm$0.0}} &\secbest{3.18\tiny{$\pm$0.3}} &15.58\tiny{$\pm$13.3} &\best{2.48\tiny{$\pm$1.9}} &\best{1.95\tiny{$\pm$0.5}} &12.19\tiny{$\pm$3.7} &1.40\tiny{$\pm$0.7} &8.67\tiny{$\pm$0.8} &6.46\tiny{$\pm$3.9} &11.64\tiny{$\pm$1.5} &20.25\tiny{$\pm$2.8} &15.94\tiny{$\pm$2.1} \\
&CSG-reg
&1.69 &1.45 &0.32 &3.93 &\secbest{3.26} &\secbest{2.22} &\best{1.95} &\best{4.53} &0.59 &\secbest{7.06} &2.70 &6.33 &12.55 &9.44 \\
&3Dseg-reg
&- &- &0.76 &- &- &- &- &- &66.31 &- &- &- &- &- \\
&DTA
&\best{0.18\tiny{$\pm$0.0}} &\secbest{0.62\tiny{$\pm$0.0}} &0.30\tiny{$\pm$0.0} &4.60\tiny{$\pm$0.1} &{3.55\tiny{$\pm$6.1}} &{2.91\tiny{$\pm$0.1}} &2.32\tiny{$\pm$0.1} 
&\secbest{4.56\tiny{$\pm$0.1}} &\secbest{0.55\tiny{$\pm$0.0}} &\best{4.90\tiny{$\pm$0.5}} &\best{2.45\tiny{$\pm$0.7}} &\secbest{2.36\tiny{$\pm$0.1}} &\secbest{10.98\tiny{$\pm$0.1}} &\secbest{6.67\tiny{$\pm$0.1}} \\
&Ours
&\secbest{0.26\tiny{$\pm$0.3}} &\best{0.52\tiny{$\pm$0.0}} 
&0.63\tiny{$\pm$0.0} &3.88\tiny{$\pm$0.0} 
&\best{0.61\tiny{$\pm$0.3}} &3.83\tiny{$\pm$0.1} 
&\secbest{2.25\tiny{$\pm$0.2}} &{6.43\tiny{$\pm$0.1}} &\best{0.54\tiny{$\pm$0.0}} &{7.31\tiny{$\pm$0.2}} 
&\secbest{2.63\tiny{$\pm$0.1}} &\best{1.64\tiny{$\pm$0.2}} 
&\best{2.93\tiny{$\pm$0.3}} &\best{2.29\tiny{$\pm$0.3}} \\
\hline
\multirow{6}{*}{\shortstack{CD-m}}
&Ditto
&141.11 &0.99 &\secbest{0.19} &0.94 &20.68 &31.21 &15.88 &12.89 &195.93 &2.20 &42.20 &50.60 &\secbest{20.35} &35.48 \\
% &PARIS 
% &8.99\tiny{$\pm$7.6} &7.76\tiny{$\pm$11.2} &\best{0.21\tiny{$\pm$0.2}} &\secbest{28.70\tiny{$\pm$15.2}} &46.64\tiny{$\pm$40.7} &9.27\tiny{$\pm$30.7}
% &5.32\tiny{$\pm$5.9} &178.43\tiny{$\pm$131.7} &25.21\tiny{$\pm$9.5} &76.69\tiny{$\pm$6.1} &39.72\tiny{$\pm$25.9} &45.66\tiny{$\pm$31.7} &864.82\tiny{$\pm$382.9} &455.24\tiny{$\pm$207.3} \\
&PARIS*
&17.97\tiny{$\pm$24.9} &7.23\tiny{$\pm$11.5} &0.15\tiny{$\pm$0.0} &6.54\tiny{$\pm$10.6} &16.65\tiny{$\pm$16.6} &30.46\tiny{$\pm$37.0} &10.17\tiny{$\pm$6.9} &265.27\tiny{$\pm$248.7} &117.99\tiny{$\pm$213.0} &52.34\tiny{$\pm$11.0} &52.48\tiny{$\pm$58.0} &77.85\tiny{$\pm$26.8} &474.57\tiny{$\pm$227.2} &276.21\tiny{$\pm$127.0} \\
&CSG-reg
&1.91 &21.71 &0.42 &256.99 &\secbest{1.95} &6.36 &29.78 &436.42 &26.62 &1.39 &78.36 &442.17 &521.49 &481.83 \\
&3Dseg-reg
&- &- &1.01 &- &- &- &- &- &6.23 &- &- &- &- &- \\
&DTA
&\best{0.15\tiny{$\pm$0.0}} &\secbest{0.27\tiny{$\pm$0.0}} &\best{0.13\tiny{$\pm$0.0}} &\best{0.44\tiny{$\pm$0.0}} &{10.11\tiny{$\pm$19.4}} &\secbest{1.13\tiny{$\pm$0.5}} &\secbest{1.47\tiny{$\pm$0.0}} 
&\best{0.45\tiny{$\pm$0.0}} &\secbest{2.05\tiny{$\pm$0.3}} &\best{0.36\tiny{$\pm$0.0}} &\secbest{1.66\tiny{$\pm$2.0}} &\secbest{1.12\tiny{$\pm$0.0}} &30.78\tiny{$\pm$2.6} &\secbest{15.95\tiny{$\pm$1.3}} \\
&Ours
&\secbest{0.54\tiny{$\pm$0.1}} &\best{0.21\tiny{$\pm$0.0}} &\best{0.13\tiny{$\pm$0.0}} & \secbest{0.89\tiny{$\pm$0.2}} &\best{0.64\tiny{$\pm$0.4}} &\best{ 0.52\tiny{$\pm$0.1}} &\best{1.22\tiny{$\pm$0.1}} &\best{0.45\tiny{$\pm$0.2}} &\best{1.12\tiny{$\pm$0.2}} & \secbest{1.02\tiny{$\pm$0.4}} &\best{0.67\tiny{$\pm$0.2}} &\best{0.66\tiny{$\pm$0.2}} &\best{6.28\tiny{$\pm$3.6}} &\best{3.47\tiny{$\pm$1.9}} \\

\hline
\multirow{6}{*}{\shortstack{CD-w}}
&Ditto
&6.80 &2.16 &\secbest{0.31} &\best{2.51} &1.70 &2.38 &2.09 &7.29 &42.04 &\best{3.91} &7.12 &6.50 &14.08 &10.29 \\
&PARIS* 
&4.37\tiny{$\pm$6.4} &5.53\tiny{$\pm$4.7} &\best{0.26\tiny{$\pm$0.0}} &{3.18\tiny{$\pm$0.3}} &3.90\tiny{$\pm$3.6} &5.27\tiny{$\pm$5.9} &1.78\tiny{$\pm$0.2} &10.11\tiny{$\pm$2.8} &0.58\tiny{$\pm$0.1} &7.80\tiny{$\pm$0.4} &4.28\tiny{$\pm$2.4} &8.99\tiny{$\pm$1.4} &32.10\tiny{$\pm$8.2} &20.55\tiny{$\pm$4.8} \\
&CSG-reg
&0.48 &0.98 &0.40 &\secbest{3.00} &1.70 &\secbest{1.99} &\secbest{1.20} &\best{4.48} &\secbest{0.56} &4.00 &\secbest{1.88} &5.71 &14.29 &10.00 \\
&3Dseg-reg
&- &- &0.81 &- &- &- &- &- &0.78 &- &- &- &- &- \\
&DTA
&\best{0.27\tiny{$\pm$0.0}} &\secbest{0.70\tiny{$\pm$0.0}} &0.32\tiny{$\pm$0.0} &4.24\tiny{$\pm$0.1} &\best{0.41\tiny{$\pm$0.0}} &\best{1.92\tiny{$\pm$0.0}} &\best{1.17\tiny{$\pm$0.0} }
&\best{4.48\tiny{$\pm$0.2}} &\best{0.36\tiny{$\pm$0.0}} &\secbest{3.99\tiny{$\pm$0.4}} &\best{1.79\tiny{$\pm$0.1}} &\secbest{2.08\tiny{$\pm$0.1}} &\secbest{8.98\tiny{$\pm$0.1}} &\secbest{5.53\tiny{$\pm$0.1}} \\
&Ours
&\secbest{0.43\tiny{$\pm$0.2}} &\best{0.58\tiny{$\pm$0.0}} & 0.50\tiny{$\pm$0.0} &3.58\tiny{$\pm$0.0} & \secbest{0.67\tiny{$\pm$0.3}} & {2.63\tiny{$\pm$0.0}} & {1.28\tiny{$\pm$0.0}} & \secbest{5.99\tiny{$\pm$0.1}} & 0.61\tiny{$\pm$0.0} & 5.21\tiny{$\pm$0.1} & {2.15\tiny{$\pm$0.1}} &\best{1.29\tiny{$\pm$0.1}} &\best{3.23\tiny{$\pm$0.1}} &\best{2.26\tiny{$\pm$0.1}} \\
% \hline
% \multirow{2}{*}{\shortstack{CD-w\\mean}}
% &DTA
% &0.26 &0.71 &0.34 &4.27 &0.41 &2.01 &1.36 
% &4.55 &0.36 &4.09 &1.84 &2.10 &9.03 &5.57 \\
% &Ours
% &0.30 &0.59 &0.48 &3.62 &0.53 &{2.87} &1.57 
% &6.04 &0.62 &5.12 &2.17 &1.45 &2.72 &2.09 \\
% \hline
% \multirow{2}{*}{\shortstack{CD-w\\mean}}
% &DTA
% &0.26 &0.71 &0.34 &4.27 &0.41 &2.01 &1.36 
% &4.55 &0.36 &4.09 &1.84 &2.10 &9.03 &5.57 \\
% &Ours
% &0.30 &0.59 &0.48 &3.62 &0.53 &{2.87} &1.57 
% &6.04 &0.62 &5.12 &2.17 &1.45 &2.72 &2.09 \\
% \hline
% \multirow{2}{*}{\shortstack{CD-w\\mean}}
% &DTA
% &0.26 &0.71 &0.34 &4.27 &0.41 &2.01 &1.36 
% &4.55 &0.36 &4.09 &1.84 &2.10 &9.03 &5.57 \\
% &Ours
% &0.30 &0.59 &0.48 &3.62 &0.53 &{2.87} &1.57 
% &6.04 &0.62 &5.12 &2.17&1.45 &2.72 &2.09 \\
\hline
\multirow{2}{*}{\shortstack{Time\\(min)}}
&DTA
&29 &30 &31 &29 &28 &29 &31 &28 &27 &28 &29 &29 &29 &29 \\
&Ours 
&9 &8 &7 &7 &7 &7 &7 &8 &7 &8 &8 &9 &9 &9 \\
\hline
\end{tabular}
}
\vspace{-10pt}
\end{table*}
\begin{figure}[t!]
 \centering
 \resizebox{\linewidth}{!}{\includegraphics[width=\linewidth]{figure/vis_exp_2part_1.pdf}}
 \vspace{-10pt}
 \caption{\textbf{Qualitative visualizations of PARIS objects.} We present reconstruction comparisons between DTA and our model on Real Storage (Top) and Synthetic Blade (Bottom). DTA struggles with mesh reconstruction at the low-visibility state, as it processes each state separately. In contrast, our method leverages the connection between states to improve the reconstruction for both low- and high-visibility states.}
 \label{fig:2part}
 \vspace{-10pt}
\end{figure}

% For CD-w and CD-s that reflect the reconstruction quality of whole and the static part, DTA has some advantages because it's reconstructed for each state individually, but \model still achieves the best or comparable results. Additionally, as shown in \cref{fig:2part}, the DTA faces a challenge in reconstructing objects when one of the states has low visibility because the states are reconstructed separately. Instead, our method takes advantage of the connection between the two states, ensuring that the mesh of the two states remains highly consistent and enabling them to assist each other in the reconstruction process.

% Notably, our method shows markable improvement for real-world objects (Fridge and Storage), demonstrating its ability to generalize to the real-world. Furthermore, our approach is highly efficient, requiring an average of 8-9 minutes of training time compared to 28-31 minutes for DTA, demonstrating a significant reduction in computational cost.



\subsection{Results on Complex Articulated Objects with Multiple Movable Parts}
\label{sec:exp:multi-part}

\paragraph{Experimental Setup} We use DTA-Multi and \model-Multi as benchmarks for evaluating complex articulated object reconstruction. On DTA-Multi, we compare our model against PARIS and DTA, while on \model-Multi we use DTA as the main baseline given its strong performance. Similar to~\cref{sec:exp:two-part}, we report all metrics with a mean over 10 trials for DTA-Multi and 3 trials for \model-multi because of the training time required for baselines. For \model-multi, we report the average of all movable parts for articulation estimation and mesh reconstruction due to the large number of parts. Considering the potential error prediction with no mesh for one of the parts, we manually set the Chamfer Distance of the empty prediction to 1000. 


\begin{table*}[t]
\caption{\textbf{Quantitative evaluation on DTA-Multi.} We report averaged metrics over 10 trials with different random seeds. Lower ($\downarrow$) is better on all metrics. Joint 1 of ``Storage-m'' is prismatic with no Axis Pos.}
\label{tab:exp_mpart_dta}
\renewcommand{\arraystretch}{1.2}
\resizebox{\linewidth}{!}{
\begin{tabular}{ccccccccccccc}
\toprule
Object &Method
&Axis Ang 0 &Axis Ang 1 &Axis Pos 0 &Axis Pos 1 
&Part Motion 0 &Part Motion 1 &CD-s &CD-m 0 &CD-m 1 &CD-w &Time (min)\\
\midrule
\multirow{3}{*}{Fridge-m} 
&PARIS
&34.52 &15.91 &3.60 &1.63 &86.21 &105.86 &8.52 &526.19 &160.86 &15.00 &- \\
&DTA
&0.25 &0.06 &0.01 &0.01 &0.23 &0.08 &0.63 &0.44 &0.53 &0.88 &32 \\
&Ours 
&\textbf{0.02} &\textbf{0.00} &\textbf{0.00} &\textbf{0.00} &\textbf{0.02} &\textbf{0.03} & \textbf{0.62} &\textbf{0.07} &\textbf{0.18} &\textbf{0.75} &\textbf{8} \\
\midrule
\multirow{3}{*}{Storage-m} 
&PARIS
&43.26 &26.18 &10.42 &- &79.84 &0.64 &8.56 &128.62 &266.71 &8.66 &- \\
&DTA
&0.17 &0.40 &0.04 &- &0.13 &0.00 &0.86 &0.20 & \textbf{0.25} &0.97 &32 \\
&Ours
&\textbf{0.01} & \textbf{0.02} & \textbf{0.01} & - & \textbf{0.01} & \textbf{0.00} & \textbf{0.78} & \textbf{0.19} & 0.27 & \textbf{0.93} &\textbf{8} \\
 \bottomrule
\end{tabular}
}
\vspace{-5pt}
\end{table*}

\begin{table*}[t]
\caption{\textbf{Quantitative evaluation on \model-Multi}. Metrics are averaged over 3 trials. Due to the large number of parts, we report the average metric for all movable parts. Lower ($\downarrow$) is better on all metrics. ``Table-31249'' has 3 prismatic joints with no Axis Pos. }
\label{tab:exp_mpart_our}
\renewcommand{\arraystretch}{1.2}
\resizebox{\linewidth}{!}{
\begin{tabular}{ccccccccccc}
\toprule
Object &Method
&Axis Ang &Axis Pos &Part Motion &CD-s &CD-m &CD-w &Time (min)\\
\midrule
\multirow{2}{*}{\shortstack{Table \\ \scriptsize{25493 (4 parts)}}}
&DTA 
&24.35 &- &0.12 &\textbf{0.59} &104.38 &\textbf{0.55} &34 \\
&Ours 
&\textbf{1.16} &- &\textbf{0.00} &0.74 &\textbf{3.53} &0.74 &\textbf{8} \\
\midrule
\multirow{2}{*}{\shortstack{Table \\ \scriptsize{31249 (5 parts)}}}
&DTA 
&20.62 &4.2 &30.8 &1.39 &230.38 &\textbf{1.00} &37 \\
&Ours 
&\textbf{0.04} &\textbf{0.00} &\textbf{0.01} &\textbf{1.22} &\textbf{3.09} &1.16 &\textbf{8} \\
\midrule
\multirow{2}{*}{\shortstack{Storage \\ \scriptsize{45503 (4 parts)}}}
&DTA 
&51.18 &2.44 &43.77 &5.74 &246.63 &0.88 &35 \\
&Ours 
&\textbf{0.02} &\textbf{0.00} &\textbf{0.03} &\textbf{0.75} &\textbf{0.13} &\textbf{0.88} &\textbf{8} \\
\midrule
\multirow{2}{*}{\shortstack{Storage \\ \scriptsize{47468 (7 parts)}}}
&DTA
&19.07 &0.31 &10.67 &0.82 &476.91 &0.71 &45 \\
&Ours 
&\textbf{0.14} &\textbf{0.02} &\textbf{0.62} &\textbf{0.67} &\textbf{3.70} &\textbf{0.70} &\textbf{8} \\
\midrule
\multirow{2}{*}{\shortstack{Oven \\ \scriptsize{101908 (4 parts)}}}
&DTA 
&17.83 &6.51 &31.80 &1.17 &359.16 &\textbf{1.01} &35 \\
&Ours 
&\textbf{0.04} &\textbf{0.01} &\textbf{0.23} &\textbf{1.08} &\textbf{0.25} &1.03 &\textbf{8} \\
\bottomrule 
\end{tabular}
}
\vspace{-5pt}
\end{table*}
\begin{figure}[t!]
 \centering
 \resizebox{\linewidth}{!}{\includegraphics[width=\linewidth]{figure/vis_exp_multi.pdf}}
 \caption{\textbf{Qualitative results on multi-part objects.} We present reconstruction comparisons between DTA and our model on Storage-47648 (Left) and Table-31249 (Bottom). On \model-Multi, DTA struggles with movable part identification and axis prediction as the number of parts increases, whereas our model maintains high performance regardless of part count, achieving high-quality reconstruction of part mesh and joint articulation.}
 \label{fig:mpart}
 \vspace{-10pt}
\end{figure}

\paragraph{Results} As demonstrated in~\cref{tab:exp_mpart_dta} and~\cref{tab:exp_mpart_our}, our method consistently outperforms existing methods by a large margin in both mesh reconstruction and articulation estimation. Notably, on \model-Multi, the baseline model DTA struggles with movable part identification and axis prediction as the number of parts increases, whereas our model maintains high performance regardless of part count. We also provide a qualitative comparison in~\cref{fig:mpart} for better visualization. Moreover, our method maintains the same time efficiency while the training time of existing methods scales with the number of parts. These results underscore the robustness and effectiveness of our method in modeling complex, multi-part articulated objects.

% \paragraph{Simple Multi-part Objects} As demonstrated in \cref{tab:exp_mpart_dta}, our method exhibits substantial advantages in modeling multi-part objects. Our method outperforms existing methods on almost all metrics. For joint parameter estimation, \model's estimation of joint parameters maintains high accuracy when it's extended to multi-part objects, bringing a large improvement compared with the existing methods. 
% In terms of mesh reconstruction, our method yields lower CD-s, CD-m, and CD-w values for all objects.

% \paragraph{Complex Multi-part Objects} Importantly, as the number of parts continues to increase, DTA struggles with objects having more than three movable parts (as shown in \cref{fig:mpart} and \cref{tab:exp_mpart_our}). \model maintains high quality on both joint parameter estimation and part mesh reconstruction as the number of parts increases. This demonstrates the generalization ability of \model to more complex articulated objects.

% In addition, the efficiency of our method is particularly evident in multi-part scenarios, requiring only about one-forth of the training time compared to DTA (8 minutes vs. 32-45 minutes).

% These results underscore the effectiveness of our method in handling complex, multi-part articulated objects, demonstrating significant improvements in both accuracy and computational efficiency over existing approaches.


\subsection{Ablative Studies}
\label{sec:exp:ablation}
\paragraph{Experimental Setup} To verify the effectiveness of our model design, we meticulously design four ablations of \model to identify the impact of key components in our method: (i) Randomly initializing canonical Gaussians (\textit{w/o} Cano. Init.), (ii) predicting part assignments with MLP (\textit{w/} MLP Seg) or Slot-Attention (\textit{w/} SA Seg), (iii) randomly initializing part centers $C_k$ (\textit{w/o} Center Init.), (iv) clustering all Gaussians instead of clustering movable Gaussians for part center initialization (\textit{w/o} Motion Prior), and (v) learning articulation parameters without the joint prediction warmup stage (\textit{w/o} Joint Pred.). We select two representative objects: ``Storage-47648'' with 4 revolute and 2 prismatic joints and ``Oven-101908'' with 3 revolute joints for ablative analysis. Similar to~\cref{sec:exp:multi-part}, we report the average of all parts over 10 trials for all metrics.

\begin{table}[t]
\centering
\caption{\textbf{Ablative experiments}. Lower ($\downarrow$) is better on all metrics.}
\label{tab:ablation}
\renewcommand{\arraystretch}{1.2}
\resizebox{\linewidth}{!}{
\begin{tabular}{lcccccccccccc}
\toprule
\multirow{2}[2]{*}{Method} & \multicolumn{6}{c}{Storage 47648 (7 parts)} & \multicolumn{6}{c}{Oven 101908 (4 parts)} \\
\cmidrule(lr){2-7}\cmidrule{8-13}
&Axis Ang &Axis Pos &Part Motion &CD-s &CD-m & CD-w & Axis Ang &Axis Pos &Part Motion &CD-s &CD-m &CD-w \\
\midrule
Full
&\textbf{0.14} &\textbf{0.02} &\textbf{0.62} &\textbf{0.67} &\textbf{3.70} &\textbf{0.70} 
&\textbf{0.04} &\textbf{0.01} &\textbf{0.23} &\textbf{1.08} &\textbf{0.25} &\textbf{1.03} \\
\textit{w/o} Cano. init.
&24.15 & 0.73 & 20.61 & 0.83 & 495.07 & 1.25 
& 57.87 & 2.95 & 54.45 & 1.73 & 1030.19 & 2.36 \\
\textit{w/o} Center Init.
& 52.78 & 0.83 & 33.04 & 1.09 & 344.19 & 1.69 
& 28.94 & 2.36 & 22.46 & 1.41 & 8.86 & 2.13 \\ 
\textit{w/o} Motion Prior
& 26.74 & 0.22 & 21.16 & 258.23 & 599.46 & 1.15 
& 40.08 & 0.98 & 41.06 & 1.75 & 503.44 & 2.35 \\
\textit{w/o} Joint Pred. 
&{0.16} &{0.02} &{0.72} &{0.67} &{3.90} &{0.71}
&{0.04} &{0.01} &{0.23} &{1.08} &{0.25} &{1.03} \\
\textit{w/} MLP Seg 
& 21.84 & 3.46 & 31.43 & 1.82 & 664.25 & 1.28 
& 12.08 & 3.33 & 27.28 & 7.78 & 126.95 & 2.19 \\
\textit{w/} SA Seg 
& 25.43 & 0.7 & 23.22 & 1.52 & 459.89 & 1.16 
& 58.04 & 4.53 & 51.28 & 1.26 & 496.64 & 2.35 \\
\bottomrule
\end{tabular}
}
\vspace{-10pt}
\end{table}

\begin{figure}[t!]
 \centering
 \resizebox{\linewidth}{!}{\includegraphics[width=\linewidth]{figure/ablation.pdf}}
 \caption{\textbf{Abaltion Studies}. We visualize the initialized and optimized canonical Gaussians with their part assignment and centers for the full model, w/o Motion Prior and w/o Cano. Init. We highlight center error, part assignment error, and canonical Gaussian error with \textcolor{red}{red}, \textcolor{green}{green}, and \textcolor{blue}{blue} bounding boxes separately.}
 \label{fig:ablation}
 \vspace{-10pt}
\end{figure}

\paragraph{Results and Discussions} As shown in~\cref{tab:ablation} and \cref{fig:ablation}, we make the following observations:
\begin{itemize}[leftmargin=*,nolistsep,noitemsep]
\item\textit{Canonical Gaussians Initialization.} Omitting this initialization strategy significantly degrades the model performance across all metrics, particularly for movable parts. As illustrated in \cref{fig:ablation} (5), the absence of our initialization strategy leads to malformed canonical Gaussians, making the model converge to suboptimal local minima during optimization.

\item\textit{Center-based Part Modeling and Assignment.} Replacing our center-based part assignment module with MLP or Slot-Attention ("w/ MLP Seg" and "w/ SA Seg") leads to substantial performance drops, especially in joint parameter estimation and movable part reconstruction. This demonstrates the superiority of our center-based approach in accurately segmenting articulated parts.

\item\textit{Center Initialization.} Random center initialization performs well for static parts but poorly for movable parts. Clustering all Gaussians fails to reconstruct both static and movable parts due to incorrect center initialization. As illustrated in \cref{fig:ablation} (1), clustering on movable Gaussians still produces an incorrect center but provides a good starting point for optimization. Our \model will refine the centers in the optimization process as shown in \cref{fig:ablation} (3). In contrast, clustering on all Gaussians results in entirely wrong center initialization (\cref{fig:ablation} (2)), which is difficult to correct (\cref{fig:ablation} (4)), leading to even worse performance than random initialization. This highlights the importance of our center initialization strategy in achieving accurate part articulation modeling.

\item\textit{Joint Prediction Warmup.} This technique primarily affects prismatic joints, as we do not constrain the transformation of revolute joints. As shown in \cref{tab:ablation}, predicting the joint type and then refining joint parameters with type constraints slightly improves the articulation reconstruction.
\end{itemize}

In summary, these ablation studies confirm that each component contributes significantly to its overall performance, playing crucial roles in achieving accurate joint parameter estimation and high-quality part mesh reconstruction. We provide further discussions in~\cref{app:discussion} and ~\cref{app:limitation}.

\section{Conclusion}
\label{sec:conclusion}
In conclusion, we propose \model, a novel approach for reconstructing articulated objects from two states of multi-view images. By leveraging 3D Gaussians and introducing novel techniques for state alignment and part dynamics modeling, our approach overcomes key limitations of existing methods. The performance improvements in joint parameter estimation and part mesh reconstruction, particularly for complex multi-part objects, demonstrate the effectiveness of our innovations. Our comprehensive experiments across synthetic and real-world datasets validate the robustness and efficiency of \model, while also revealing promising directions for future research. As the demand for accurate digital replicas of articulated objects continues to grow in fields such as robotics and augmented reality, \model provides a solid foundation for bridging the gap between physical and virtual environments. Moving forward, we anticipate that the principles introduced in this work will inspire further advancements in the field, ultimately enabling more sophisticated and realistic simulations for a wide range of applications.

% \bibliography{ref}
\bibliographystyle{iclr2025_conference}
\documentclass{MITstyle}

%\usepackage[table]{xcolor}
\usepackage{chngcntr}
\usepackage{hyperref}
\usepackage{microtype}

\title{A Lightweight and Extensible Cell Segmentation and Classification Model for Whole Slide Images}

\author{Nikita Shvetsov~$^{1, }$\footnote{Correspondence e-mail: nikita.shvetsov@uit.no}, Thomas K. Kilvaer~$^{2, 3}$, Masoud Tafavvoghi~$^{4}$, Anders Sildnes~$^{1}$, \\ Kajsa Møllersen~$^{4}$, Lill-Tove Rasmussen Busund~$^{5, 6}$, Lars Ailo Bongo~$^{1}$ \\
%
\vspace{1em} % Space between authors and afilliations
%
\normalfont{\small $^{1}$Department of Computer Science, UiT The Arctic University of Norway}\\
\normalfont{\small $^{2}$Department of Oncology, University Hospital of North Norway}\\
\normalfont{\small $^{3}$Department of Clinical Medicine, UiT The Arctic University of Norway}\\
\normalfont{\small $^{4}$Department of Community Medicine, UiT The Arctic University of Norway}\\
\normalfont{\small $^{5}$Department of Medical Biology, UiT The Arctic University of Norway} \\
\normalfont{\small $^{6}$Department of Clinical Pathology, University Hospital of North Norway} %\vspace{2em}
}

\begin{document}
\maketitle

\section*{Abstract}

% \begin{abstract}
% Developing clinically useful cell-level analysis tools in digital pathology remains challenging due to limitations in dataset granularity, inconsistent annotations, computational demands of advanced models, and difficulties in integrating new technologies into clinical workflows. To address these challenges, we propose a multi-faceted solution that enhances data quality, model performance, and usability to create a lightweight and extensible cell segmentation and classification model.

% First, we update data labels by employing a cross-relabeling process that refines the labels of two existing datasets, PanNuke and MoNuSAC, to create a new unified dataset with enhanced granularity, encompassing seven distinct cell types. Second, we leverage the H-Optimus foundation model as a fixed encoder to improve feature representation for simultaneous cell segmentation and classification tasks. Third, to address the computational demands of foundation models, we employ knowledge distillation to reduce model size and complexity while maintaining comparable performance. Finally, to facilitate integration into clinical workflows, we integrate the distilled model into the QuPath software, a widely used open-source platform in digital pathology.

% Our results demonstrate improvements in cell segmentation and classification performance using the H‑Optimus-based model compared to a CNN-based model. Specifically, the average $R^2$ improved from 0.575 to 0.871, and the average $PQ$ score improved from 0.450 to 0.492, indicating better alignment with actual cell counts and enhanced segmentation and classification quality. Furthermore, the distilled student model maintains performance comparable to the larger foundation model while reducing the parameter count by a factor of 48.
% Overall, by reducing computational complexity and integrating it into existing workflows, the proposed approach may significantly impact diagnostic processes, reduce the workload of pathologists, and contribute to improved patient outcomes. Though our approach shows potential enhancements in efficiency and usability of cell segmentation and classification models in digital pathology, extensive validation is needed to deploy these models in clinical practice.
% \end{abstract}

%%% shortened abstract
\begin{abstract}
Developing clinically useful cell-level analysis tools in digital pathology remains challenging due to limitations in dataset granularity, inconsistent annotations, high computational demands, and difficulties integrating new technologies into workflows. To address these issues, we propose a solution that enhances data quality, model performance, and usability by creating a lightweight, extensible cell segmentation and classification model. 

First, we update data labels through cross-relabeling to refine annotations of PanNuke and MoNuSAC, producing a unified dataset with seven distinct cell types. Second, we leverage the H-Optimus foundation model as a fixed encoder to improve feature representation for simultaneous segmentation and classification tasks. Third, to address foundation models' computational demands, we distill knowledge to reduce model size and complexity while maintaining comparable performance. Finally, we integrate the distilled model into QuPath, a widely used open-source digital pathology platform. 

Results demonstrate improved segmentation and classification performance using the H-Optimus-based model compared to a CNN-based model. Specifically, average $R^2$ improved from 0.575 to 0.871, and average $PQ$ score improved from 0.450 to 0.492, indicating better alignment with actual cell counts and enhanced segmentation quality. The distilled model maintains comparable performance while reducing parameter count by a factor of 48. By reducing computational complexity and integrating into workflows, this approach may significantly impact diagnostics, reduce pathologist workload, and improve outcomes. Although the method shows promise, extensive validation is necessary prior to clinical deployment.
\end{abstract}
\clearpage

\section{Introduction}
In digital pathology, accurate segmentation and classification of cells are crucial for many diagnostic, prognostic, and predictive analyses \cite{Jaber_Beziaeva_etal._2019,Lin_Pan_etal._2022,Park_Ock_etal._2022,Shen_Choi_etal._2024}. Nowadays, developments in computational pathology offer multiple solutions \cite{H._Qu_P._Wu_etal._2020,Javed_Mahmood_etal._2020} to utilize cell-level datasets to train machine learning models that solve these problems. The quality and specificity of training datasets are critical for robust and accurate models. Adhering to the principle of "garbage in, garbage out", it is essential to ensure that these datasets are extensively and accurately labeled with distinct classes that reflect the diverse biological characteristics of different cell types. Unfortunately, the number of open-source datasets comprising such high-quality annotations is limited. Existing cell segmentation datasets \cite{Gamper_Koohbanani_etal._2019,Graham_Vu_etal._2019,Verma_Kumar_etal._2021} may offer extensive annotations for certain cell types while providing more general labels for others. For example, in PanNuke, which is one of the largest open-source datasets comprising labeled cells, various types of morphologically and functionally different inflammatory cells like macrophages and lymphocytes are clustered in a broad "inflammatory" class. Consequently, these classes are frequently omitted from analyses or aggregated into broader meta-classes \cite{Gamper_Koohbanani_etal._2020} and likely interfere with other cell classes included in the dataset. This and similar inconsistencies in annotation granularity limit the ability of machine learning models to learn the comprehensive and nuanced features necessary for accurate cell segmentation and classification. To address these challenges, methods for refining and standardizing dataset annotations are essential to enhance the quality of training data.

A complementary approach to mitigate the absence of high-quality training data is the use of foundation models. Foundation models as encoders are defined as large-scale, versatile networks pre-trained on vast, diverse datasets using self-supervised learning, contrasting with convolutional neural network (CNN) pre-trained encoders that rely on supervised learning with labeled data. In practice, foundation models leverage enormous amounts of weakly or unlabeled data from millions of whole slide images (WSIs) and employ self-attention mechanisms to capture long-range dependencies and global context \cite{Chen_Ding_etal._2024,Saillard_Jenatton_etal._2024,Vorontsov_Bozkurt_etal._2024,Xu_Usuyama_etal._2024}. As a consequence, foundation models are able to produce transferable feature representations across different cell types and tissue environments. The feature representations can be leveraged by decoder networks to produce segmentation masks and pixel-level classifications. Because foundation models have comprehensive feature representations, they can be effectively fine-tuned using much smaller amounts of cell-level data compared to the large datasets needed to train models from scratch. Furthermore, foundation models incorporate adversarial training elements or contrastive learning \cite{Chen_Ding_etal._2024,Xu_Usuyama_etal._2024}, enhancing their resilience and adaptability by exposing them to challenging and varied scenarios during training. This may result in more generalizable models, often making them well-suited for diverse and complex tasks in digital pathology.

Despite the inherent advantages of foundation models, their deployment for practical use faces its own obstacles. In particular, they require substantial computational power, financial investments and rigorous testing to ensure reliability and efficacy for a given task \cite{Akkus_Dangott_etal._2022,Dragomir_Cocuz_etal._2022,Go_2022,Jafri_Farooqui_etal._2024}. Moreover, while foundation models enhance feature representation and performance, they depend on the quality of available annotations for decoder fine-tuning and, like any other model, cannot resolve existing inconsistencies or ambiguities in data labels. Therefore, there remains a critical need for solutions that address both data quality and practical deployment considerations.
Further, integrating new technologies into existing clinical workflows often encounters resistance, as it necessitates adjustments to established diagnostic processes. So, there is a need to develop solutions that could be integrated into current practices, minimizing the burden on medical professionals to adopt new tools \cite{King_Williams_etal._2023}.

Existing solutions \cite{Goldsborough_Philps_etal._2024,Hörst_Rempe_etal._2024}, while addressing some aspects of these challenges, fall short in providing a comprehensive approach. To address the data quality and clinical deployment issues, we propose a multi-faceted solution that encompasses data refinement, model optimization, and integration with existing pathology tools (\hyperref[fig:fig1]{Figure 1}). The outcome is a lightweight cell segmentation and classification model that can be integrated into digital pathology workflows for practical clinical use.

\begin{figure}[h!]
    \centering
    \includegraphics[width=\textwidth, height=0.82\textheight, keepaspectratio]{images/Figure_1.pdf}
    \caption{Overview of the proposed solution, including 1) Data refinement using cross-relabeling, 2) Teacher model development and fine tuning, 3) Student model optimization with knowledge distillation and 4) Student model and QuPath integration}
    \label{fig:fig1}
\end{figure}
\clearpage

Our approach begins with preparing the data for the fine-tuning and training of the machine learning models. We create a refined dataset, acquired via cross-relabeling two cell-level datasets, enhancing annotation specificity and consistency of the labeled data. Subsequently, we create a cell segmentation and classification model based on the foundation model. We leverage the foundation model as a fixed encoder and fine-tune a decoder using the refined dataset to improve generalization across diverse tissue- and cell types.
To ensure that the model remains lightweight and deployable in a possibly resource-constrained environment, we employ knowledge distillation to approximate the functionality of the foundation model. Finally, to facilitate the practical application of our model in digital pathology workflows, we integrate it with the QuPath \cite{Bankhead_Loughrey_etal._2017} application. Each methodological component contributes to the overarching goal of enhancing model performance, generalizability, and usability in clinical settings.

The primary contributions of this paper are:
\begin{enumerate}
    \item \textit{Data labels refinement through cross-relabeling:}
    
    We propose a new method for refining labels of cell-level datasets through cross-relabeling. This method employs classification models to re-label broad and ambiguous instances, resulting in a more diverse dataset. Our evaluation demonstrates that these classification models achieve high accuracy on test subsets, indicating the reliability of the method for label refinement.

    \item \textit{Enhanced model performance via foundation models:}
    
    We employ a foundation model as a feature extractor for the cell segmentation and classification task. In comparison with training a CNN model from scratch, the foundation model backbone only needs fine-tuning, which significantly reduces training time, computational resources and data requirements. We show that using a foundation model encoder leads to better performance in cell segmentation and classification networks than using a CNN-based encoder. This improvement may enable the model to generalize more effectively across various tissue types and imaging methods.
    
    \item \textit{Model optimization through knowledge distillation:}
    
    We show that a smaller student model trained using knowledge distillation on the refined dataset obtained via our cross-relabeling approach from a foundation model achieves comparable performance in cell segmentation and quantification tasks. As a result, this model is more suitable for deployment in environments without high-performance computing resources.
    
    \item \textit{Integration with QuPath:}
    
    We integrate the distilled cell segmentation and classification model into QuPath, a widely used open-source digital pathology platform, to accelerate clinical adaptation by enabling pathologists to more easily incorporate advanced computational tools into their existing workflows.
\end{enumerate}

Through these methodological steps, we aim to bridge the gap between advanced machine learning techniques and practical clinical applications, making accurate and efficient digital pathology accessible in a broader range of healthcare settings.

\section{Refining Existing Datasets Using Cross-Relabeling}
To address the limitations of sparse and ambiguous labeling of cell-level datasets, we propose a generalizable cross-relabeling strategy that can be applied to any dataset containing broadly categorized or imprecisely labeled cell types. This approach involves training and subsequently leveraging classification models to refine broad categories into more specific or biologically relevant classes.
When applied to cell-level data, the methodology includes extracting individual cell images from the dataset patches, preprocessing these images to standardize the size and accommodate partial cells, and then training deep learning classifiers capable of distinguishing between the finer cell subtypes within the coarser categories. 
To illustrate our approach, we focus on the PanNuke \cite{Gamper_Koohbanani_etal._2020, Gamper_Koohbanani_etal._2019} and MoNuSAC \cite{Verma_Kumar_etal._2021} datasets that we have used to train models for cell quantification in our previous works \cite{Shvetsov_Grønnesby_etal._2022,Shvetsov_Sildnes_etal._2024}. We find that for better cell differentiation we have to introduce more granular labels. PanNuke includes a broad classification of "inflammatory" cells, encompassing lymphocytes, macrophages, and neutrophils. Each cell type differs significantly in structure, function, and clinical relevance. Conversely, MoNuSAC uses the label "epithelial" for a class that comprises both benign epithelial cells and malignant neoplastic cells. This practice makes it challenging to differentiate between benign and malignant epithelial cells in the dataset, which is a critical distinction when identifying tumor areas within tissue samples. To address these issues, we implement a cross-relabeling strategy as shown in \hyperref[fig:fig2]{Figure 2}. The key components are two classification models: one is trained on singular cell images from PanNuke data to classify the epithelial meta-class into epithelial and neoplastic classes. The other is trained on MoNuSAC to refine the inflammatory class into lymphocytes, neutrophils, and macrophages.

\begin{figure}[h!]
    \centering
    \includegraphics[width=\textwidth]{images/Figure_2.pdf}
    \caption{Refined dataset generation via cross relabeling}
    \label{fig:fig2}
\end{figure}

The refining approach consists of three consecutive steps. The first is the preprocessing step, in which we extract individual cells from both datasets (\hyperref[fig:fig3]{Figure 3}). The specifics of PanNuke and MoNuSAC patch preparation before cell preprocessing are provided in \hyperref[chap:S1]{Appendix S1}.

\begin{figure}[h!]
    \centering
    \includegraphics[width=\textwidth]{images/Figure_3.pdf}
    \caption{Cell instances preprocessing including (1) cell map extraction, (2) bounding box delineation, (3) adjusting cell boxes and (4) cropping and resizing of cell images}
    \label{fig:fig3}
\end{figure}

During preprocessing, we extract cell type maps from the ground truth label mask and calculate bounding boxes around each cell instance. To accommodate partial cells at patch borders, a common issue in cropped patch images, we employ mirror padding and extend the field of view of the cell label by 15 pixels to capture adjacent cells. We then crop and resize the identified regions to $64 \times 64$ pixels using bicubic interpolation.

The preprocessed PanNuke dataset comprises 68,031 neoplastic and 23,207 epithelial cell images, while MoNuSAC comprises  33,104 lymphocytes, 1,252 neutrophils, and 1,695 macrophages, which we subsequently use in training cell classification models and classifying the cell image data \hyperref[fig:S2]{Appendix Figure S2 (1)}. 

The next step is to train two distinct ResNet50-based classifiers tailored to address the specific labeling challenges inherent in each dataset. We use ResNet50 for classification models due to its proven effectiveness for image classification tasks in histopathology \cite{pan2022reviewmachinelearningapproaches}, and its compatibility with small images. For the PanNuke dataset, we design the classifier, trained on MoNuSAC data, to disaggregate the heterogeneous "inflammatory" cell category into distinct subtypes: lymphocytes, macrophages, and neutrophils. Similarly, for the MoNuSAC dataset, the classifier is trained on PanNuke data and distinguishes between benign and malignant epithelial cells within the overarching "epithelial" label. By applying these targeted classifiers to their respective datasets, we assign more specific labels to individual cell instances, thus enabling us to create a unified dataset.
To ensure a balanced representation of classes, we train both models on datasets that had been equalized to match the size of the least represented class. Thus, we obtain datasets comprising 23,207 samples per class for PanNuke and 1,252 samples per class for MoNuSAC data. Next, we partition both of them into training (70\%), validation (20\%), and testing (10\%) subsets. To mitigate the risk of overfitting, we use a single dropout layer with a rate of p=0.5 in both models and data augmentation using randomized color perturbations, rotation, and horizontal and vertical flipping. We employ AdamW optimizer and the cross-entropy loss function for the training criterion.

To evaluate the two trained models, we measure the classification accuracy on the respective test subsets. The accuracies on the test subset for both classifiers are presented in \hyperref[tab:1]{Table 1}. The PanNuke model achieves an average accuracy of 93.57\%, with higher accuracy for neoplastic cells (96.06\%) compared to epithelial cells (86.26\%). The confusion matrix in Figure A3.1 shows that the model predominantly distinguishes accurately between epithelial and neoplastic tissues, with a substantial number of correct classifications and relatively few misclassifications. The MoNuSAC model demonstrates an average accuracy of 98.92\%, excelling in classifying lymphocytes (99.67\%) and macrophages (94.12\%), with lower performance for neutrophils (85.71\%). The confusion matrix in Figure A3.2 shows that the model identifies lymphocytes and performs reasonably well with macrophages and neutrophils.

\begin{table}[h!]
\renewcommand{\arraystretch}{1.5}
  \centering
  \caption{Cell classification results for PanNuke and MoNuSAC trained models (CI 95\%).}
  \label{tab:1}
  \begin{tabular}{|l|c|c|}
   \hline
   %\rowcolor{gray!30}
    Accuracy               & PanNuke model              & MoNuSAC model              \\
    \hline
    Average      & 0.936 (0.931--0.941)         & 0.989 (0.986--0.993)        \\
    \hline
    Neoplastic   & 0.961 (0.956--0.965)         & -                          \\
    \hline
    Epithelial   & 0.863 (0.849--0.877)         & -                          \\
    \hline
    Lymphocytes  & -                          & 0.997 (0.995--0.999)        \\
    \hline
    Neutrophils  & -                          & 0.857 (0.796--0.918)        \\
    \hline
    Macrophages  & -                          & 0.941 (0.906--0.976)        \\
    \hline
  \end{tabular}
\end{table}

Finally, during the last step, we use the model trained on PanNuke data for epithelial cells in MoNuSAC and the model trained on MoNuSAC for the inflammatory cells class in PanNuke. Specifically, we use classifier models to relabel epithelial cells in MoNuSAC and inflammatory cells in PanNuke data. Then we combine cells with refined labels and the rest of the cells in both datasets to create a refined dataset (\hyperref[fig:S2]{Appendix Figure S2 (2)}). The process of relabeling cells and visualizing them on a patch is shown in \hyperref[fig:fig4]{Figure 4}. The cell counts in the refined dataset are provided in \hyperref[tab:S4]{Appendix Table S4}.

\begin{figure}[h!]
    \centering
    \includegraphics[width=\textwidth, height=0.42\textheight, keepaspectratio]{images/Figure_4.pdf}
    \caption{Cell relabeling procedure for epithelial and inflammatory cell classes}
    \label{fig:fig4}
\end{figure}

%\hfill

Relabeling and combining datasets have been explored in a prior study \cite{Parulekar_Kanwat_etal._2023}, where consecutive fine-tuning on multiple datasets was employed to account for hierarchical class label structures. While the method presented in \cite{Parulekar_Kanwat_etal._2023} is intuitive, it often lacks consistency and requires multiple fine-tuning runs, which can be cumbersome and time-consuming. 
In contrast, cross-relabeling simplifies this process by using specialized classification models tailored to each dataset's specific labeling challenges. This approach provides better transparency and produces a unified dataset encompassing seven distinct cell types across multiple tissue samples, enhancing data diversity for further model training or fine-tuning.

Despite these improvements, cross-relabeling does not entirely resolve issues related to poor labeling quality or the amount of labeled data. Specifically, our results show lower accuracies persist for underrepresented classes, such as macrophages, which may stem from a limited sample availability and intrinsic challenges in distinguishing these cells based solely on H\&E staining. Furthermore, while our method enhances label specificity, it relies on the initial quality of the broad labels; thus, any fundamental inaccuracies in the original annotations can propagate through the relabeling process. Addressing the overall problem of limited data labels may require integrating additional data sources or utilizing complementary immunohistochemical staining methods.
Although the reported performance metrics are obtained from evaluations on the native test sets of each dataset, it is important to note that the primary application of these classifiers is to perform cross-relabeling, where a model trained on one dataset (e.g., PanNuke) is applied to another (e.g., MoNuSAC) and vice versa. We acknowledge that a more systematic evaluation of cross-dataset generalization is needed and could be performed in future work.

Overall, the refined dataset produced by our approach can enhance the supervised training or fine-tuning of cell segmentation and classification models, especially those that utilize pre-trained foundation models to improve feature extraction robustness. In addition, these models can detect nuanced classes that enable researchers to conduct more detailed analyses of biological processes in computational pathology.

\section{Foundation models for robust cell segmentation and classification}

Accurate cell segmentation and classification in digital pathology are hindered by limited labeled data and the fact that conventional CNNs are unable to capture global contextual information due to their local receptive field constraints \cite{Gheflati_Rivaz_2022,Yang_Marcus_etal.}. Traditional approaches in cell quantification have predominantly relied on CNN encoders, such as ResNet50, given their proven effectiveness in semantic segmentation tasks \cite{Deshmane_2023,Graham_Vu_etal._2019,Mukasheva_Koishiyeva_etal._2024,Stringer_Wang_etal._2021}. However, approaches that include fine-tuning of pretrained CNNs, data augmentation, and stain normalization to partially increase data variability and address staining differences often fail to achieve the necessary generalization and robustness across diverse tissue types and staining conditions \cite{G._Wang_W._Li_etal._2018,Gao_Bagci_etal._2018,Karim_El_Khoury_Martin_Fockedey_etal._2021}.

To overcome these challenges, we leverage an encoder-decoder network that uses a foundation model as the encoder and a CNN upsampling decoder (\hyperref[fig:fig5]{Figure 5}) for simultaneous cell segmentation and classification in 2D patches extracted from WSIs. Foundation models with transformer-based architectures are viable alternatives to CNN-based encoders \cite{Shamshad_Khan_etal._2023,Sourget_2023}. They enable the creation of more advanced architectures that can decode or transform learned features more effectively \cite{Chen_Duan_etal._2023,Cheng_Misra_etal._2022,Xie_Wang_etal._2021}.

\begin{figure}[h!]
    \centering
    \includegraphics[width=\textwidth]{images/Figure_5.pdf}
    \caption{UNETR-like model with foundational model as backbone}
    \label{fig:fig5}
\end{figure}

By utilizing a transformer-based encoder, we incorporate global contextual information into the feature extraction process, which is a key advantage of such architectures \cite{Chen_Lu_etal._2021}. This foundation model integration facilitates accurate pixel-wise segmentation and classification without the need for extensive encoder training, thereby potentially improving generalization across varied cellular structures and tissue types.
In our implementation, we employ a modified UNETR \cite{Hatamizadeh_Tang_etal._2021} architecture that combines a vision transformer (ViT) \cite{Dosovitskiy_Beyer_etal._2021} encoder with a CNN-based decoder. The encoder utilizes the pretrained H-Optimus foundation model, which contains 1.1 billion parameters and is trained on over 500,000 H\&E stained WSIs \cite{Saillard_Jenatton_etal._2024}. We extract outputs from four evenly spaced transformer blocks $Z_i$, where $i \in [1, 14, 26, 38]$, to serve as residual connections for the CNN decoder. We select these blocks based on our observation that features from non-adjacent levels of the encoder lead to better overall performance on the test subset.

The CNN decoder upsamples the feature representations, acquired from the transformer blocks, to generate an intermediate vector that is handled by two task-specific layers that generate cell segmentation and classification masks. The first task-specific layer is the ‘Cellpose head’,  which is used to delineate cell instances. The layer generates horizontal and vertical gradient maps to form vector fields that are refined through gradient tracking in a post-processing step using the Cellpose algorithm \cite{Stringer_Wang_etal._2021}, known for its efficacy in cell segmentation tasks and generalizability across multiple domains \cite{Pachitariu_Stringer_2022,Stringer_Pachitariu_2024}. The second task-specific layer is the "Cell type head", which assigns labels to individual pixels. In the post-processing step, we determine the output classification label of each segmented cell instance by majority voting over the labeled pixels that comprise the cell in the segmentation map.

To evaluate model performance and measure the impact of adding a foundation model as backbone, we compare it to a ResNet50-based model. ResNet50 is a widely used solution for encoders in segmentation architectures in the medical domain \cite{Deshmane_2023,Graham_Vu_etal._2019,Mukasheva_Koishiyeva_etal._2024,Stringer_Wang_etal._2021}. For the H-Optimus-based model, we utilize frozen weights for the encoder and only fine-tune the decoder to take advantage of the extensive pre-training of the foundation model. For the ResNet50-based model we start with ImageNet \cite{Deng_Dong_etal.} weights and train both encoder and decoder parts. Hyperparameters for the training step are set to be identical, where possible, for comparable evaluation. 
For this evaluation, we deliberately use the PanNuke dataset to provide a standardized and controlled comparison between the H‑Optimus and ResNet50-based models (\hyperref[fig:S2]{Appendix Figure S2 (3)}). Specifically, we use two of the default PanNuke dataset splits (66\%) for training and validation, and reserve the third split (33\%) for testing.

To address the challenge of cell class imbalance in the PanNuke dataset, which is a common characteristic in most cell-level H\&E patch datasets, both models’ training processes employ a weighted loss function comprising cross-entropy and focal loss \cite{Lin_Goyal_etal._2018}. The focal loss component is adjusted with coefficients derived from each cell class' instance frequency, emphasizing learning from underrepresented classes and enhancing the model's sensitivity to rare but significant cellular patterns. The cross-entropy loss is augmented with spectral decoupling regularization \cite{Pezeshki_Kaba_etal._2021,Pohjonen_Stürenberg_etal._2022} and spatially varying label smoothing \cite{Islam_Glocker_2021}, which potentially stabilizes training and improves generalization in case of complex tissue morphologies. For optimization, we employ the \textit{AdamW} \cite{Loshchilov_Hutter_2019} to counter unbalanced class scenarios, with cosine annealing learning rate scheduler.

We utilize the scikit-learn library \cite{Van_der_Walt_Schönberger_etal._2014} and HoVer-Net \cite{Graham_Vu_etal._2019} implementations of $R^2$ (the coefficient of determination) and $PQ$ (panoptic quality) to evaluate our experiments. Complete mathematical formulations and detailed explanations of these metrics are provided in \hyperref[chap:S5]{Appendix S5}. To compute confidence intervals, we use nonparametric bootstrapping, where after calculating the metric on the full sample, we generated 1000 bootstrap replicates by resampling with replacement and then determined the 95\% confidence intervals as the 2.5th and 97.5th percentiles of the resulting empirical distribution.

%\hfill

The model comparisons are summarized in \hyperref[tab:2]{Table 2}. The H‑Optimus-based model achieves higher $R^2$ across all cell classes compared to the ResNet50-based model, which means that its predictions are more closely aligned with the PanNuke cell counts, indicating a stronger correlation with the observed data. Notably, the improvement of $R^2_{dead}$ may be an indicator of better global contextual representations provided by the foundation model backbone. In terms of segmentation and classification quality combined, measured by the PQ score, the H‑Optimus-based model demonstrates notable improvements across most cell classes. Overall, the average $R^2$ improved from 0.575 to 0.871, while the average $PQ$ score improved from 0.450 to 0.492, demonstrating better performance of the H-Optimus-based model.

\begin{table}[h!]
\renewcommand{\arraystretch}{1.5}
  \centering
  \caption{Cell quantification metrics for baseline and proposed models (CI 95\%).}
  \label{tab:2}
  \begin{tabular}{|l|c|c|}
    \hline
    %\rowcolor{gray!30}
    Metric             & Resnet50-based            & H-optimus-based              \\
    \hline
    $R^2_{neoplastic}$    & 0.681 (0.576--0.769)       & \textbf{0.941 (0.917--0.960)} \\
    \hline
    $R^2_{inflammatory}$  & 0.863 (0.778--0.903)       & \textbf{0.949 (0.918--0.966)} \\
    \hline
    $R^2_{connective}$    & 0.600 (0.488--0.698)       & 0.609 (0.436--0.772)          \\
    \hline
    $R^2_{dead}$          & 0.097 (-11.389--0.669)     & 0.925 (0.404--0.982)          \\
    \hline
    $R^2_{epithelial}$    & 0.635 (0.490--0.747)       & \textbf{0.930 (0.886--0.964)} \\
    \hline
    $PQ_{neoplastic}$       & 0.517 (0.499--0.535)       & \textbf{0.589 (0.575--0.604)} \\
    \hline
    $PQ_{inflammatory}$     & 0.455 (0.429--0.482)       & \textbf{0.528 (0.507--0.549)} \\
    \hline
    $PQ_{connective}$       & 0.416 (0.400--0.431)       & \textbf{0.451 (0.436--0.465)} \\
    \hline
    $PQ_{dead}$             & 0.374 (0.342--0.408)       & 0.292 (0.209--0.365)          \\
    \hline
    $PQ_{epithelial}$       & 0.488 (0.460--0.519)       & \textbf{0.599 (0.579--0.618)} \\
    \hline
  \end{tabular}
\end{table}

Our results  show that integrating the H‑Optimus foundation model within the UNETR architecture enhances the model's ability to segment and classify cells across diverse tissues from PanNuke data. The pretrained transformer encoder provides robust feature representations, resulting in higher average $R^2$ and $PQ$ scores compared to the CNN-based model. This leads to more reliable cell quantification and more accurate downstream analysis. Additionally, the streamlined fine-tuning process reduces computational overhead and training time, making the model more adaptable for new data.

Despite these advancements, the foundation model-based approach does not fully resolve all challenges related to cell segmentation and classification. We observe lower metric scores for underrepresented classes in the training data. Furthermore, foundation models typically encompass billions of parameters, resulting in substantial computational and memory requirements. It therefore poses challenges for deployment in resource-constrained environments, limiting their practical applicability in certain clinical settings.

\section{Model optimization via Knowledge Distillation}

To address the limitations posed by the extensive size of foundation models, we implement knowledge distillation — a model compression technique that leverages the teacher-student paradigm \cite{Hinton_Vinyals_etal._2015}. By training a smaller, more efficient student model to replicate the output of a larger, pre-trained teacher model, we retain performance while significantly reducing the model's complexity and resource requirements (\hyperref[fig:fig6]{Figure 6}).

\begin{figure}[h!]
    \centering
    \includegraphics[width=\textwidth, height=0.45\textheight, keepaspectratio]{images/Figure_6.pdf}
    \caption{Knowledge distillation framework for training a student model using a pre-trained teacher}
    \label{fig:fig6}
\end{figure}

We employ knowledge distillation to compress the H‑Optimus-based teacher model into a more efficient student model. The teacher model is the modified UNETR architecture with the H‑Optimus foundation model described in the previous chapter. The student model is based on a UNet architecture augmented with residual connections and incorporates a smaller ViT encoder with 9 million parameters \cite{Steiner_Kolesnikov_etal._2022,Wightman_2019}. 

First, we fine-tune the teacher model using the refined dataset from the cross-relabeling procedure (Section 2). Initially we train the decoder of the teacher model while keeping the encoder weights frozen. We split the refined dataset into train (70\%), validation (20\%) and test (10\%) subsets (\hyperref[fig:S2]{Appendix Figure S2 (4)}). During fine-tuning, we use the train and validation subsets, while leaving the test subset for model evaluation. We set the training procedure and model hyperparameters to be identical to those that were used to demonstrate the utility of foundation models for the simultaneous cell segmentation and classification task.

Next, we perform knowledge distillation from teacher to student using the refined dataset used to fine-tune the teacher model. The student model is trained to replicate the teacher model's outputs. We utilize a specialized loss function that aligns the student's predicted probability distribution with the teacher's, incorporating the teacher's class probability distribution derived from the output. Following the methodology of Hinton et al. \cite{Hinton_Vinyals_etal._2015}, we experiment with various hyperparameter settings for the temperature ($T$) and the balancing coefficients ($\alpha$ and $\beta$) in the loss function. We vary $T$ from 1 to 20 and adjust $\alpha$ and $\beta$ to balance the distillation and student losses. Through iterative tuning and evaluation, we identify that setting $T=14$, $\alpha=0.3$, and $\beta=0.7$ yields a configuration that converges and closely approximates the teacher model's performance during training.

Finally, we assess the performance of both models using the $R^2$ and $PQ$ (defined in \hyperref[chap:S5]{Appendix S5}) on the test set of the refined dataset (\hyperref[tab:3]{Table 3}). We observe that the 95\% confidence intervals overlap for most cell types, so we cannot claim statistically significant performance differences between the teacher and student models. One exception appears in the neoplastic class. The teacher model produces an $R^2$ of 0.919, while the student model shows an $R^2$ of 0.852. In addition, the student model achieves higher $PQ$ values for the neoplastic and connective classes, though the confidence intervals show overlap.

\begin{table}[h!]
\renewcommand{\arraystretch}{1.5}
  \centering
  \caption{Cell quantification metrics for teacher and distilled student models (CI 95\%).}
  \label{tab:3}
  \begin{tabular}{|l|c|c|}
    \hline
    %\rowcolor{gray!30}
    Metric & Teacher & Student \\
    \hline
    $R^2_{neoplastic}$    & \textbf{0.919} (0.898--0.939) & 0.852 (0.800--0.891) \\
    \hline
    $R^2_{lymphocyte}$    & 0.969 (0.956--0.977)         & 0.969 (0.956--0.978) \\
    \hline
    $R^2_{connective}$    & 0.694 (0.548--0.809)         & 0.618 (0.469--0.741) \\
    \hline
    $R^2_{dead}$          & 0.755 (0.400--0.908)         & 0.424 (0.100--0.731) \\
    \hline
    $R^2_{epithelial}$    & 0.922 (0.870--0.958)         & 0.843 (0.738--0.917) \\
    \hline
    $R^2_{macrophage}$    & 0.384 (-0.369--0.724)        & 0.704 (0.352--0.859) \\
    \hline
    $R^2_{neutrofil}$     & 0.854 (0.578--0.929)         & 0.833 (0.502--0.925) \\
    \hline
    $PQ_{neoplastic}$       & 0.581 (0.569--0.593)         & 0.601 (0.588--0.613) \\
    \hline
    $PQ_{lymphocyte}$       & 0.536 (0.520--0.553)         & 0.563 (0.544--0.579) \\
    \hline
    $PQ_{connective}$       & 0.436 (0.421--0.451)         & 0.457 (0.441--0.474) \\
    \hline
    $PQ_{dead}$             & 0.272 (0.235--0.315)         & 0.279 (0.201--0.369) \\
    \hline
    $PQ_{epithelial}$       & 0.522 (0.500--0.545)         & 0.530 (0.506--0.555) \\
    \hline
    $PQ_{macrophage}$       & 0.524 (0.459--0.588)         & 0.474 (0.405--0.543) \\
    \hline
    $PQ_{neutrofil}$        & 0.541 (0.490--0.592)         & 0.565 (0.522--0.607) \\
    \hline
  \end{tabular}
\end{table}


We further decompose the $PQ$ metric into its $SQ$ and $DQ$ components (\hyperref[tab:S6]{Appendix Table S6}). Both models produce nearly identical $SQ$ values, which indicates that they predict instance boundaries with similar precision. Although the student model shows some improvement in $DQ$ scores for certain classes, the confidence intervals overlap and do not confirm a statistically significant difference.

We observe that the student and teacher models yield comparable detection performance despite the student model using a much smaller and simpler architecture. A model with fewer parameters reduces the risk of overfitting when training data are scarce relative to the model’s complexity \cite{Farias_Ludermir_etal._2022}. The knowledge distillation process also encourages the student model to focus on the most generalizable detection features learned from the teacher. These factors enable the student model to achieve similar detection performance across different cell types.

Additionally, considering the model sizes reported in \hyperref[tab:4]{Table 4}, the distilled model achieves a significant reduction compared to the teacher model, with a 48-fold decrease in parameter count and a 5.5-fold reduction in on-disk size. In inference mode, the teacher model requires 16 GB of VRAM for a batch size of 32, while the distilled model only needs 3 GB of VRAM for the same batch size. These reductions make the distilled model significantly more practical for fine-tuning and deployment in resource-constrained environments.

\begin{table}[h!]
\renewcommand{\arraystretch}{1.5}
  \centering
  \caption{Parameter counts and size of teacher and distilled model}
  \label{tab:4}
  \adjustbox{max width=\textwidth}{%
  \begin{tabular}{|l|c|c|c|}
    \hline
    %\rowcolor{gray!30}
    Metric & H-optimus-based (Teacher) & mobileViT-based (Student) & Magnitude of difference \\
    \hline
    Parameters count       & 1,158,917,906   & \textbf{24,093,393}   & \textbf{48x}  \\
    \hline
    Estimated Total Size (MB) & 87,912       & \textbf{15,935}    & \textbf{5.5x} \\
    \hline
  \end{tabular}%
}
\end{table}

%\hfill

With recent advancements in complex network architectures and the use of pretrained encoders to achieve state-of-the-art performance \cite{Baumann_Dislich_etal._2024,Hörst_Rempe_etal._2024} in cell segmentation and classification tasks, model size, computational complexity, and processing times have increased. This limits the scalability and accessibility of these models. As we demonstrate, this may be mitigated using knowledge distillation. Studies in the field of natural language processing have demonstrated the efficacy of knowledge distillation in retaining the capabilities of the teacher model while achieving significant reductions in size and complexity \cite{Huangpu_Gao_2024,Sun_Yu_etal.}. 

We demonstrate the feasibility of knowledge distillation in digital pathology, specifically for cell segmentation and classification tasks. Moreover, we achieve this performance while also significantly reducing the parameter count. In addressing the challenge of knowledge transfer, we found that distillation from a transformer-based model to a smaller transformer is more straightforward than attempting to map transformer features to CNN blocks. In our experiments, using a CNN-based network as a student results in worse cell quantification performance due to the structural constraints of CNN feature space dimensions. 

Although our primary approach relies on a transformer-based student model that performs well, it can be further optimized to incorporate advantages from CNN architectures. For example, employing alternative techniques such as using ViT adapters \cite{Chen_Duan_etal._2023} or $1 \times 1$ convolutions to adjust feature map sizes may be beneficial for harnessing CNN advantages like enhanced local feature extraction. Moreover, if additional performance improvements are desired, the process can be further enhanced by applying supplementary knowledge distillation techniques, such as self-distillation \cite{Zhang_Song_etal._2019} or online distillation \cite{Houyon_Cioppa_etal._2023}.

Despite these promising results, further validation on independent datasets is necessary to fully understand the model's limitations. Underrepresented classes may pose challenges when addressing complex cases. Pathologists need to validate these models to adopt them in clinical settings. While the distilled models are smaller and more deployable, a technological gap persists because pathologists traditionally rely on established methods for inspecting WSIs and diagnosing diseases. Addressing the complexities involved in deploying models for inference and supporting pathologists in adopting new tools is essential for integrating these models into clinical workflows.

\section{Model integration with QuPath}
Digital pathology tools with graphical user interfaces are essential for visualizing and analyzing WSIs. To make our student model useful in clinical pathology workflows, it needs to be integrated into a tool that enables inspecting regions, creating annotations, and providing quantitative analyses of biomarkers. Therefore, we integrate the trained student model from the previous chapter into the QuPath open‑source platform \cite{Bankhead_Loughrey_etal._2017}. QuPath provides the required annotation, visualization, and analysis tools to interpret complex histological data, including workflows for cell segmentation, classification, and quantification (\hyperref[fig:fig7]{Figure 7}). 

\begin{figure}[h!]
    \centering
    \includegraphics[width=\textwidth]{images/Figure_7.pdf}
    \caption{Visualization of model-generated cell quantification annotations (left) and the corresponding unannotated slide (right) in QuPath}
    \label{fig:fig7}
\end{figure}

To identify the regions in a WSI critical for prognosticating tumor development, such as specific tumor areas or border regions without overlapping healthy tissue, the pathologist uses QuPath to outline these regions. Then, the pathologist initiates a cell segmentation and classification script through the QuPath interface for the selected regions. The resulting annotations and quantified cell information are then directly overlaid onto the WSI in the QuPath interface. Additional design and implementation details are in \hyperref[chap:S7]{Appendix S7}. 

Two common approaches for integrating deep learning models into QuPath are Java‑based native QuPath extensions \cite{Goldsborough_Philps_etal._2024} and the execution of RESTful API requests to a model server coupled with handling the response via an extension, as demonstrated in the application of cell segmentation models applied to immunofluorescence images \cite{Sugawara_2023}. While the community is actively working on these integration strategies, there is currently no universal solution that fully addresses all integration and performance requirements.

Extensions may offer better integration with QuPath, allowing slightly improved performance and more widespread usage of the built-in QuPath models, but they lack the flexibility to customize models and modify their behavior. For example, the newest version of QuPath includes models such as StarDist \cite{Weigert_Schmidt} and InstanSeg \cite{Goldsborough_Philps_etal._2024} that can perform cell segmentation. Both models pose limitations when applied to simultaneous cell segmentation and classification. StarDist performs well only on convex, round shapes by design, whereas some neoplastic, inflammatory, and connective cells exhibit complex and non-convex shapes. InstanSeg provides only semantic segmentation without assigning classes to the segmented cells.

%\hfill

In contrast, our approach offers an alternative integration strategy. It utilizes the paquo library to directly interact with QuPath’s internal application programming interface from within Python. This enables data exchange and processing without the need for intermediate conversion steps and provides greater control over model customization, retraining, and the incorporation of custom processing steps.

The integration of our custom model with QuPath underscores its potential to significantly enhance the diagnostic process by reducing the time burden on pathologists and enabling them to focus on more complex interpretative tasks using familiar software. Leveraging a tool that is already well-established among pathologists increases the likelihood of its adoption into daily clinical workflows. The quantitative data generated through the automated workflow is critical for both clinical decision-making and research, facilitating more accurate biomarker analysis, enabling robust statistical evaluations, and supporting hypothesis generation and testing. Additionally, by streamlining cell segmentation and classification, the tool enhances the scalability and reproducibility of pathological assessments, ultimately contributing to improved diagnostic accuracy and patient outcomes.

\section{Conclusion and future work}

In this study, we address critical challenges in digital pathology and tackle the usability and deployment issues of the developed models in standard computing environments without the need for high-performance computing systems. Our multi-faceted approach encompasses data refinement through cross-relabeling, leveraging foundation models for robust cell segmentation and classification, optimizing model performance via knowledge distillation, and integrating the optimized model into the QuPath software for practical application. This approach is used to construct a capable, versatile, and adjustable model for cell segmentation and classification, with enhanced performance and usability.

\begin{sloppypar}
While our approach shows potential in the field of computational pathology, certain limitations persist. 
For example, our implementation currently exhibits lower performance in detecting macrophages. 
This serves as an instance of the broader challenge of accurately identifying complex cell types. In order to address this issue, extending our approach to incorporate additional data sources, exploring alternative modeling approaches, and integrating other imaging modalities such as immunohistochemical staining may help improve detection accuracy. Moreover, although the distilled model reduces computational demands, integrating advanced deep learning models into clinical practice requires addressing technological gaps and potential resistance to adopting new tools within established diagnostic processes.
\end{sloppypar}

Future work could focus on several key areas to refine the proposed approach and facilitate its adoption in clinical environments. Enhancing the cell-relabeling process with additional datasets \cite{Graham_Jahanifar_etal._2021} could improve the representation of underrepresented cell types and enhance overall model performance. Also, incorporating additional data sources, such as multi-modal imaging or complementary staining methods, may address limitations related to cell type differentiation and class imbalance. Exploring other foundation models \cite{Vorontsov_Bozkurt_etal._2024,Zimmermann_Vorontsov_etal._2024} or introducing additional modalities \cite{Ding_Wagner_etal._2024,Vaidya_Zhang_etal._2025} may provide alternative architectures better suited to specific tasks or offer improved efficiency. Implementing more complex knowledge distillation techniques \cite{Houyon_Cioppa_etal._2023,Zhang_Song_etal._2019} could further optimize the model's performance and adaptability. Additionally, deeper integration with QuPath or other digital pathology software could provide pathologists more control over cell quantification analysis directly within the QuPath interface, thereby increasing accessibility and usability. Such enhancements would not only refine model performance but also ensure greater adaptability and scalability within various clinical environments. Finally, extensive validation of the model by pathologists and benchmarking against independent datasets are essential steps toward establishing the model's reliability and fostering confidence in its clinical utility.

\section*{Acknowledgments} 
This work was funded in part by the Research Council of Norway grant no. 309439 SFI Visual Intelligence, and the North Norwegian Health Authority grant no. HNF1521-20.

\bibliographystyle{IEEEtran}
\begin{sloppypar}
\begin{thebibliography}{99}

\bibitem{chaplot2020neural} Chaplot, Devendra Singh, et al. "Neural topological slam for visual navigation." Proceedings of the IEEE/CVF conference on computer vision and pattern recognition. 2020.

\bibitem{maksymets2021thda} Maksymets, Oleksandr, et al. "Thda: Treasure hunt data augmentation for semantic navigation." Proceedings of the IEEE/CVF International Conference on Computer Vision. 2021.

\bibitem{mezghan2022memory} Mezghan, Lina, et al. "Memory-augmented reinforcement learning for image-goal navigation." 2022 IEEE/RSJ International Conference on Intelligent Robots and Systems (IROS). IEEE, 2022.

\bibitem{al2022zero} Al-Halah, Ziad, Santhosh Kumar Ramakrishnan, and Kristen Grauman. "Zero experience required: Plug \& play modular transfer learning for semantic visual navigation." Proceedings of the IEEE/CVF Conference on Computer Vision and Pattern Recognition. 2022.

\bibitem{ye2021auxiliary} Ye, Joel, et al. "Auxiliary tasks and exploration enable objectgoal navigation." Proceedings of the IEEE/CVF international conference on computer vision. 2021.

\bibitem{chaplot2020object} Chaplot, Devendra Singh, et al. "Object goal navigation using goal-oriented semantic exploration." Advances in Neural Information Processing Systems 33 (2020)

\bibitem{ramakrishnan2022poni} Ramakrishnan, Santhosh Kumar, et al. "Poni: Potential functions for objectgoal navigation with interaction-free learning." Proceedings of the IEEE/CVF Conference on Computer Vision and Pattern Recognition. 2022.

\bibitem{ramrakhya2022habitat} Ramrakhya, Ram, et al. "Habitat-web: Learning embodied object-search strategies from human demonstrations at scale." Proceedings of the IEEE/CVF Conference on Computer Vision and Pattern Recognition. 2022.

\bibitem{mousavian2019visual} Mousavian, Arsalan, et al. "Visual representations for semantic target driven navigation." 2019 International Conference on Robotics and Automation (ICRA). IEEE, 2019.

\bibitem{dhariwal2021diffusion} Dhariwal, Prafulla, and Alexander Nichol. "Diffusion models beat gans on image synthesis." Advances in neural information processing systems 34 (2021)

\bibitem{ho2022classifier} Ho, Jonathan, and Tim Salimans. "Classifier-free diffusion guidance." arXiv preprint arXiv:2207.12598 (2022).

\bibitem{nichol2021glide} Nichol, Alex, et al. "Glide: Towards photorealistic image generation and editing with text-guided diffusion models." arXiv preprint arXiv:2112.10741 (2021)

\bibitem{brooks2023instructpix2pix} Brooks, Tim, Aleksander Holynski, and Alexei A. Efros. "Instructpix2pix: Learning to follow image editing instructions." Proceedings of the IEEE/CVF Conference on Computer Vision and Pattern Recognition. 2023.

\bibitem{fu2023guiding} Fu, Tsu-Jui, et al. "Guiding instruction-based image editing via multimodal large language models." arXiv preprint arXiv:2309.17102 (2023).

\bibitem{geng2024instructdiffusion} Geng, Zigang, et al. "Instructdiffusion: A generalist modeling interface for vision tasks." Proceedings of the IEEE/CVF Conference on Computer Vision and Pattern Recognition. 2024.

\bibitem{zhou2024minedreamer} Zhou, Enshen, et al. "Minedreamer: Learning to follow instructions via chain-of-imagination for simulated-world control." arXiv preprint arXiv:2403.12037 (2024).

\bibitem{zhou2023esc} Zhou, Kaiwen, et al. "Esc: Exploration with soft commonsense constraints for zero-shot object navigation." International Conference on Machine Learning. PMLR, 2023.

\bibitem{yu2023l3mvn} Yu, Bangguo, Hamidreza Kasaei, and Ming Cao. "L3mvn: Leveraging large language models for visual target navigation." 2023 IEEE/RSJ International Conference on Intelligent Robots and Systems (IROS). IEEE, 2023.

\bibitem{gadre2023cows} Gadre, Samir Yitzhak, et al. "Cows on pasture: Baselines and benchmarks for language-driven zero-shot object navigation." Proceedings of the IEEE/CVF Conference on Computer Vision and Pattern Recognition. 2023.

\bibitem{shah2023navigation} Shah, Dhruv, et al. "Navigation with large language models: Semantic guesswork as a heuristic for planning." Conference on Robot Learning. PMLR, 2023.

\bibitem{cai2024bridging} Cai, Wenzhe, et al. "Bridging zero-shot object navigation and foundation models through pixel-guided navigation skill." 2024 IEEE International Conference on Robotics and Automation (ICRA). IEEE, 2024.

\bibitem{yu2023co} Yu, Bangguo, Hamidreza Kasaei, and Ming Cao. "Co-NavGPT: Multi-robot cooperative visual semantic navigation using large language models." arXiv preprint arXiv:2310.07937 (2023).

\bibitem{wu2024voronav} Wu, Pengying, et al. "Voronav: Voronoi-based zero-shot object navigation with large language model." arXiv preprint arXiv:2401.02695 (2024).

\bibitem{qin2023mp5} Qin, Yiran, et al. "Mp5: A multi-modal open-ended embodied system in minecraft via active perception." arXiv preprint arXiv:2312.07472 (2023).

\bibitem{du2024learning} Du, Yilun, et al. "Learning universal policies via text-guided video generation." Advances in Neural Information Processing Systems 36 (2024).

\bibitem{ajay2024compositional} Ajay, Anurag, et al. "Compositional foundation models for hierarchical planning." Advances in Neural Information Processing Systems 36 (2024).

\bibitem{liang2024skilldiffuser} Liang, Zhixuan, et al. "Skilldiffuser: Interpretable hierarchical planning via skill abstractions in diffusion-based task execution." Proceedings of the IEEE/CVF Conference on Computer Vision and Pattern Recognition. 2024.

\bibitem{heusel2017gans} Heusel, Martin, et al. "Gans trained by a two time-scale update rule converge to a local nash equilibrium." Advances in neural information processing systems 30 (2017).

\bibitem{zhang2018unreasonable} Zhang, Richard, et al. "The unreasonable effectiveness of deep features as a perceptual metric." Proceedings of the IEEE conference on computer vision and pattern recognition. 2018.

\bibitem{brown2020language} Brown, Tom B. "Language models are few-shot learners." arXiv preprint arXiv:2005.14165 (2020).

\bibitem{podell2023sdxl} Podell, Dustin, et al. "Sdxl: Improving latent diffusion models for high-resolution image synthesis." arXiv preprint arXiv:2307.01952 (2023).

\bibitem{brohan2022rt} Brohan, Anthony, et al. "Rt-1: Robotics transformer for real-world control at scale." arXiv preprint arXiv:2212.06817 (2022).

\bibitem{brohan2023rt} Brohan, Anthony, et al. "Rt-2: Vision-language-action models transfer web knowledge to robotic control." arXiv preprint arXiv:2307.15818 (2023).

\bibitem{li2024manipllm} Li, Xiaoqi, et al. "Manipllm: Embodied multimodal large language model for object-centric robotic manipulation." Proceedings of the IEEE/CVF Conference on Computer Vision and Pattern Recognition. 2024.

\bibitem{shah2023vint} Shah, Dhruv, et al. "ViNT: A foundation model for visual navigation." arXiv preprint arXiv:2306.14846 (2023).

\bibitem{liu2024visual} Liu, Haotian, et al. "Visual instruction tuning." Advances in neural information processing systems 36 (2024).

\bibitem{hu2021lora} Hu, Edward J., et al. "Lora: Low-rank adaptation of large language models." arXiv preprint arXiv:2106.09685 (2021).

\bibitem{qin2023supfusion} Qin, Yiran, et al. "SupFusion: Supervised LiDAR-camera fusion for 3D object detection." Proceedings of the IEEE/CVF International Conference on Computer Vision. 2023.

\bibitem{qin2024worldsimbench} Qin, Yiran, et al. "Worldsimbench: Towards video generation models as world simulators." arXiv preprint arXiv:2410.18072 (2024).

\bibitem{yu2025gamefactory} Yu, Jiwen, et al. "GameFactory: Creating New Games with Generative Interactive Videos." arXiv preprint arXiv:2501.08325 (2025).

\bibitem{zhou2024code} Zhou, Enshen, et al. "Code-as-Monitor: Constraint-aware Visual Programming for Reactive and Proactive Robotic Failure Detection." arXiv preprint arXiv:2412.04455 (2024).

\bibitem{zhang2024ad} Zhang, Zaibin, et al. "AD-H: Autonomous Driving with Hierarchical Agents." arXiv preprint arXiv:2406.03474 (2024).

\bibitem{wang2024toward} Wang, Chaoqun, et al. "Toward Accurate Camera-based 3D Object Detection via Cascade Depth Estimation and Calibration." arXiv preprint arXiv:2402.04883 (2024).

\bibitem{huang2024story3d} Huang, Yuzhou, et al. "Story3d-agent: Exploring 3d storytelling visualization with large language models." arXiv preprint arXiv:2408.11801 (2024).

\bibitem{savinov2018semi} Savinov, Nikolay, Alexey Dosovitskiy, and Vladlen Koltun. "Semi-parametric topological memory for navigation." arXiv preprint arXiv:1803.00653 (2018).

\bibitem{majumdar2022zson} Majumdar, Arjun, et al. "Zson: Zero-shot object-goal navigation using multimodal goal embeddings." Advances in Neural Information Processing Systems 35 (2022): 32340-32352.

\bibitem{yadav2023offline} Yadav, Karmesh, et al. "Offline visual representation learning for embodied navigation." Workshop on Reincarnating Reinforcement Learning at ICLR 2023. 2023.

\bibitem{yadav2023ovrl} Yadav, Karmesh, et al. "Ovrl-v2: A simple state-of-art baseline for imagenav and objectnav." arXiv preprint arXiv:2303.07798 (2023).

\bibitem{sun2024fgprompt} Sun, Xinyu, et al. "FGPrompt: fine-grained goal prompting for image-goal navigation." Advances in Neural Information Processing Systems 36 (2024).

\bibitem{zhu2017target} Zhu, Yuke, et al. "Target-driven visual navigation in indoor scenes using deep reinforcement learning." 2017 IEEE international conference on robotics and automation (ICRA). IEEE, 2017.

\bibitem{koh2024generating} Koh, Jing Yu, Daniel Fried, and Russ R. Salakhutdinov. "Generating images with multimodal language models." Advances in Neural Information Processing Systems 36 (2024).

\bibitem{krantz2022instance} Krantz, Jacob, et al. "Instance-specific image goal navigation: Training embodied agents to find object instances." arXiv preprint arXiv:2211.15876 (2022).

\bibitem{schulman2017proximal} Schulman, John, et al. "Proximal policy optimization algorithms." arXiv preprint arXiv:1707.06347 (2017).

\bibitem{anderson2018evaluation} Anderson, Peter, et al. "On evaluation of embodied navigation agents." arXiv preprint arXiv:1807.06757 (2018).

\bibitem{lin2024navcot} Lin, Bingqian, et al. "NavCoT: Boosting LLM-Based Vision-and-Language Navigation via Learning Disentangled Reasoning." arXiv preprint arXiv:2403.07376 (2024).

\bibitem{NavGPT} Zhou, Gengze, Yicong Hong, and Qi Wu. "Navgpt: Explicit reasoning in vision-and-language navigation with large language models." Proceedings of the AAAI Conference on Artificial Intelligence.

\bibitem{hahn2021no} Hahn, Meera, et al. "No rl, no simulation: Learning to navigate without navigating." Advances in Neural Information Processing Systems 34 (2021): 26661-26673.

\bibitem{li2025t2isafety} Li, Lijun, et al. "T2ISafety: Benchmark for Assessing Fairness, Toxicity, and Privacy in Image Generation." arXiv preprint arXiv:2501.12612 (2025).

\bibitem{an2024agfsync} An, Jingkun, et al. "AGFSync: Leveraging AI-Generated Feedback for Preference Optimization in Text-to-Image Generation." arXiv preprint arXiv:2403.13352 (2024).


\end{thebibliography}
\end{sloppypar}

\clearpage
\beginsupplement
\section*{Appendix}
\renewcommand{\thesubsection}{S\arabic{subsection}}

\subsection{\label{chap:S1}PanNuke and MoNuSAC preprocessing}
The PanNuke dataset comprises a set of 7,901 RGB patches, each with dimensions of $256 \times 256$ pixels, which we set as the standard patch size for our analysis. In contrast, the MoNuSAC dataset encompasses 294 images of heterogeneous dimensions. To standardize the MoNuSAC images with our experiments, we implement a standardization protocol. Specifically, for images exceeding the dimensions of $256 \times 256$ pixels, we segment them into equal-sized patches and apply mirror padding to the remaining portions to avoid information loss at the peripherals. Patches with dimensions less than $128 \times 128$ pixels are excluded from the dataset due to the insufficient resolution to capture relevant cellular details. For patches where either dimension falls between 128 and 256 pixels, we employ upsampling to achieve the standard patch size. As a result, we obtain a total of 2,823 RGB patches derived from the MoNuSAC dataset for subsequent analysis. For additional details on the MoNuSAC data preparation process, refer to the source code \cite{Shvetsov_2025a}.
\clearpage

\subsection{\label{chap:S2}Data usage for the methodology}

\counterwithin{figure}{subsection}
\renewcommand{\thefigure}{S\arabic{subsection}}

\begin{figure}[h!]
    \centering
    \includegraphics[width=\textwidth, height=0.85\textheight, keepaspectratio]{images/A2.pdf}
    \caption{Overview of the methodology for cross-labeling, dataset refinement, and model comparison. (1) Cross-relabeling - training and testing cell classification models, (2) Cross-relabeling - using cell classification models to create refined dataset, (3) Fine-tuning and training models for comparison, (4) Student knowledge distillation with refined dataset}
    \label{fig:S2}
\end{figure}
\clearpage

\subsection{\label{chap:S3}Confusion matrices for classification models}
\counterwithin{figure}{subsection}
\renewcommand{\thefigure}{S\arabic{subsection}.\arabic{figure}}

\begin{figure}[h!]
    \centering
    \includegraphics[width=\textwidth, height=0.4\textheight, keepaspectratio]{images/A3_1.pdf}
    \caption{Confusion matrix for PanNuke trained model}
    \label{fig:S3.1}
\end{figure}

\begin{figure}[h!]
    \centering
    \includegraphics[width=\textwidth, height=0.4\textheight, keepaspectratio]{images/A3_2.pdf}
    \caption{Confusion matrix for MoNuSAC trained model}
    \label{fig:S3.2}
\end{figure}

\clearpage

\subsection{\label{chap:S4}Datasets cell counts}

\counterwithin{table}{subsection}
\renewcommand{\thetable}{S\arabic{subsection}}

\begin{table}[h!]
\renewcommand{\arraystretch}{2.0}
\centering
\caption{\label{tab:S4}Cell counts for PanNuke, MoNuSAC and refined datasets. Numbers in parentheses indicate preprocessed cell counts for cell classifier models training and testing.}
%\adjustbox{max width=\textwidth}{%
\begin{tabular}{|l|c|c|c|}
\hline
%\rowcolor{gray!30}
Cell type & PanNuke & MoNuSAC & Refined \\
\hline
Neoplastic & 77,403 (68,031) & - & 105,451 \\
\hline
Epithelial & 26,572 (23,207) & - & 29,926 \\
\hline
Epithelial (benign and malignant) & - & 31,402 & - \\
\hline
Inflammatory & 32,276 & - & - \\
\hline
Lymphocytes & - & 37,045 (33,104) & 65,275 \\
\hline
Neutrophils & - & 1,355 (1,252) & 3,833 \\
\hline
Macrophage & - & 1,842 (1,695) & 3,410 \\
\hline
Dead & 2,908 & - & 2,908 \\
\hline
Connective & 50,585 & - & 50,585 \\
\hline
\end{tabular}
%
%}
\end{table}



\clearpage

\subsection{\label{chap:S5}Definition of validation metrics}
\counterwithin{equation}{subsection}
\renewcommand{\theequation}{\arabic{equation}}

\subsubsection{\label{chap:S5.1}R\textsuperscript{2}}
The coefficient of determination, denoted as $R^2$, is a statistical measure that represents the proportion of variance in the dependent variable that is predictable from the independent variables. In the context of cell quantification in pathology, $R^2$ is used to assess how well the predicted quantities of different cell types in a patch align with the actual quantities observed in the ground truth data, with higher values representing more accurate quantification. $R^2$ is defined as
\begin{equation*}
R^2 = 1 - \frac{\sum_{i=1}^n (y_i - \hat{y}_i)^2}{\sum_{i=1}^n (y_i - \bar{y})^2},
\end{equation*}
where $y_i$ represents the actual number of cells of a specific type in the $i$-th image, $\hat{y}_i$ represents the predicted number of cells of that type in the $i$-th image, $\bar{y}$ is the mean of the actual numbers across all images, and $n$ is the total number of images in the dataset.

The $R^2$ metric has a range of $(-\infty, 1]$. An $R^2$ of 1 indicates perfect prediction, where all predicted values exactly match the actual values. An $R^2$ of 0 suggests that the model explains none of the variability of the response data around its mean. If $R^2$ is negative, it indicates that the model performs worse than a model that simply predicts the mean of the actual values for all observations.

\subsubsection{\label{chap:S5.2}PQ}
Panoptic Quality ($PQ$) is a comprehensive metric used to evaluate the performance of segmentation models in tasks that require both instance segmentation and classification. $PQ$ provides a single score that encapsulates both the detection accuracy (i.e., how many objects were correctly identified) and the segmentation quality (i.e., how accurately the objects' boundaries were delineated). This metric is particularly useful in multiclass scenarios where each pixel is classified into distinct categories, such as different cell types in pathology images.

$PQ$ is calculated as the product of two terms: Detection Quality ($DQ$) and Segmentation Quality ($SQ$). It can be expressed as
\begin{equation*}
PQ = DQ \cdot SQ,
\end{equation*}
where
\begin{equation*}
DQ = \frac{TP}{TP + 0.5\, FP + 0.5\, FN},
\end{equation*}
\begin{equation*}
SQ = \frac{\sum_{(p, g) \in \mathcal{M}} IoU(p, g)}{TP}.
\end{equation*}
In these formulas, $TP$ denotes the number of correctly matched instances between ground truth and prediction, $FP$ denotes the predicted instances that have no corresponding ground truth, $FN$ denotes the ground truth instances that were not detected, $IoU(p, g)$ is the Intersection over Union for a pair of matched instances $p$ (prediction) and $g$ (ground truth), and $\mathcal{M}$ is the set of matched pairs.

The $PQ$ metric is calculated for each class and is averaged across classes to provide a global performance measure.

The $PQ$ score has a range of $[0, 1.0]$, where a higher score indicates better performance in both detecting and segmenting the instances correctly. A $PQ$ of 1 signifies perfect identification and segmentation of all instances, whereas a $PQ$ of 0 indicates that no instances were correctly identified and segmented.

\clearpage

\subsection{\label{chap:S6}Segmentation and Detection quality metrics for teacher and student models}

\begin{table}[h!]
\renewcommand{\arraystretch}{2.0}
\centering
\caption{Segmentation and detection quality for student and teacher models (CI 95\%)}
\label{tab:S6}
%\adjustbox{max width=\textwidth}{%
\begin{tabular}{|l|c|c|}
\hline
%\rowcolor{gray!30}
Metric & Teacher & Student \\
\hline
$SQ_{neoplastic}$ & 0.819 (0.815--0.823) & 0.824 (0.819--0.828) \\
\hline
$SQ_{lymphocyte}$ & 0.795 (0.788--0.802) & 0.790 (0.783--0.796) \\
\hline
$SQ_{connective}$ & 0.770 (0.762--0.776) & 0.780 (0.772--0.786) \\
\hline
$SQ_{dead}$ & 0.659 (0.623--0.688) & 0.657 (0.624--0.695) \\
\hline
$SQ_{epithelial}$ & 0.780 (0.770--0.790) & 0.788 (0.779--0.797) \\
\hline
$SQ_{macrophage}$ & 0.788 (0.760--0.810) & 0.757 (0.730--0.783) \\
\hline
$SQ_{neutrofil}$ & 0.782 (0.761--0.801) & 0.775 (0.759--0.792) \\
\hline
$DQ_{neoplastic}$ & 0.706 (0.692--0.719) & 0.727 (0.712--0.741) \\
\hline
$DQ_{lymphocyte}$ & 0.675 (0.656--0.698) & 0.713 (0.691--0.734) \\
\hline
$DQ_{connective}$ & 0.566 (0.546--0.584) & 0.583 (0.565--0.602) \\
\hline
$DQ_{dead}$ & 0.410 (0.361--0.465) & 0.435 (0.306--0.561) \\
\hline
$DQ_{epithelial}$ & 0.668 (0.639--0.694) & 0.673 (0.644--0.702) \\
\hline
$DQ_{macrophage}$ & 0.657 (0.583--0.727) & 0.615 (0.531--0.703) \\
\hline
$DQ_{neutrofil}$ & 0.691 (0.625--0.753) & 0.729 (0.679--0.778) \\
\hline
\end{tabular}
%
%}
\end{table}

\clearpage

\subsection{\label{chap:S7}QuPath integration method}
We adopt an integration strategy leveraging the paquo \cite{Bayer_AG} library, a Python package that enables direct interaction with QuPath’s internal API, thereby facilitating seamless data exchange without intermediate conversion steps. The data processing pipeline (\hyperref[fig:S7]{Appendix Figure S7}) begins with the acquisition of WSIs and their associated annotations from QuPath, which are represented as Shapely \cite{Gillies_Wel_etal._2024} polygons. Utilizing paquo, we directly read, create, and modify these annotations and detections within a QuPath project in the Python environment. Images are then cropped using these polygons and processed by cell segmentation and classification models employing standard vision processing toolkits such as OpenCV, pyvips, and PyTorch. Additionally, QuPath employs Groovy scripts to initiate a Python process that starts the entire pipeline from QuPath graphical interface: fetching polygons, extracting images from them, and running deep learning model inference on the cropped images. 
The results are returned to QuPath, leveraging paquo's Python bindings to manipulate QuPath data while minimizing the computational overhead typically associated with cross-environment communication.

\counterwithin{figure}{subsection}
\renewcommand{\thefigure}{S\arabic{subsection}}

\begin{figure}[h!]
    \centering
    \includegraphics[width=\textwidth]{images/A7.pdf}
    \caption{QuPath integration workflow using Python environment}
    \label{fig:S7}
\end{figure}

Compared to traditional workflows that involve exporting annotations as GeoJSON, classifying them in Python, and reimporting them into QuPath, our approach offers several advantages. We eliminate the need to switch between programming languages, providing a cohesive and streamlined development process entirely within QuPath software and removing the necessity to use other tools. Meanwhile, we avoid storing annotations as intermediate JSON files unless required for external use or archiving. By conducting the entire inference and post-processing workflow within the Python environment, we leverage the power and flexibility of Python libraries for image processing and machine learning. This approach also enables adjustments to any set of labels and models, thereby improving its applicability.

%\hfill

The distilled model and QuPath integration code are packaged into a Docker container, enabling streamlined execution with the Docker engine. Detailed integration code and deployment instructions can be found in the GitHub repository \cite{Shvetsov_2025b}.

Despite these benefits, we acknowledge that the paquo library is a proof‑of‑concept project in its early development stage and has not been tested across all versions of QuPath.

\clearpage

\subsection{\label{chap:S8}Data and code availability statement}
All datasets, models, and code used in this study are publicly available and can be obtained from the repositories listed below. 
The PanNuke \cite{Gamper_Koohbanani_etal._2019} and MoNuSAC \cite{Verma_Kumar_etal._2021} datasets are publicly accessible, and download information along with detailed descriptions can be found in their respective articles. Preprocessing scripts for PanNuke and MoNuSAC data, as well as individual cell extraction scripts, are available on GitHub \cite{Shvetsov_2025a}. The H-Optimus foundation model used in our experiments can be downloaded from the HuggingFace repository \cite{hoptimus2024}, and model information is available on GitHub \cite{Saillard_Jenatton_etal._2024}. In addition, the integration code for QuPath and the distilled model packaged in a Docker container are provided in the repository \cite{Shvetsov_2025b}, and paquo Python library is available from the authors GitHub repository \cite{Bayer_AG}.
\clearpage

\end{document}


\clearpage

\appendix

\renewcommand{\thefigure}{A.\arabic{figure}}
\renewcommand{\thetable}{A.\arabic{table}}
\renewcommand{\theequation}{A.\arabic{equation}}
% \renewcommand{\thesection}{S.\arabic{section}}
\setcounter{section}{0}
\setcounter{figure}{0}
\setcounter{table}{0}
\setcounter{equation}{0}


\section{Implementation and Training Details}
\label{app:imp}
\paragraph{Canonical Gaussian Initialzation}
We train single-state Gaussians $\mathcal{G}^0$ and $\mathcal{G}^1$ for 10K steps with loss $\mathcal{L}=(1-\lambda_\text{{SSIM}})\mathcal{L}_I+\lambda_\text{{SSIM}}\mathcal{L}_{\text{D-SSIM}}+\lambda_o\mathcal{L}_o$, where $\lambda_\text{{SSIM}}=0.2,\lambda_o=0.01$ is used in experiments and $\mathcal{L}_o$ is an opacity entropy loss calculated as:
$$
\hat{\sigma}_i=\mathbb{1}\{\sigma_i>0.5\},\quad
\mathcal{L}_o=-\frac{1}{N}\sum_{i=1}^{N}[\hat{\sigma}_i\sigma_i+(1-\hat{\sigma}_i)\log(1-\sigma_i)],
$$ 
which encourages Gaussian opacities $\sigma_i$ to approach either 0 or 1, controlling Gaussian count and accelerating training. We then obtain coarse canonical Gaussians by matching $\mathcal{G}^0$ and $\mathcal{G}^1$ as described in \cref{sec:method:canonical}. This stage takes about 2 minutes per object.
\paragraph{Part Discovery for Articulation Modeling} 
\label{app:imp:assignment}
As described in \cref{sec:method:skinning}, given canonical Gaussians $\mathcal{G}^c = \{G_i\}_{i=1}^{N}$ and $K$ learnable part centers $C_k = (\vp_k, \mR_k, {\bm \lambda}_k)$, we calculate part-level masks $\mM$ using \cref{eq:part_assignment}. We use a learnable hash grid $H$ to encode Gaussian positions and predict the residual term in \cref{eq:part_assignment} as:
\begin{equation*}
\begin{aligned}
\rmX^k_i = \frac{[\mR_k(\vmu^c_i-\vp_k)]}{{\bm\lambda}_k}, &\quad
 \rmD_{ik} = (\rmX^k_i)^T\cdot\rmX^k_i \\
{\mW_\Delta}_{ik}=\mathrm{MLP}(\vmu^c_i,H(\vmu^c_i),\{X^k_i\}_{k=1}^K,\{D_{ik}\}_{k=1}^K), &\quad\bm{M} = \mathrm{GumbelSoftmax}\left(\frac{-\mD + \mW_{\Delta}}{\tau}\right) \\
\end{aligned}
\end{equation*}
Since the part assignment and articulation parameters are far from optimal at the beginning of training, using hard assignment for Gumbel-Softmax hinders the joint optimization of the part assignment and articulation parameters. To address this problem, we anneal the temperature $\tau$ from 1 to 0.1 over 10K steps, using soft assignment that is similar to Softmax when $\tau > 0.1$ and hard assignment otherwise for training stability. This approach allows for more flexible assignments during the early stages of training, facilitating better joint optimization, and gradually transitioning to decisive part assignment as the model converges.

\paragraph{Optimization}
To enhance the learning of articulation parameters, we adopt a warm-up strategy to predict the joint type of each part. This process requires 3K-5K steps that take 30 to 50 seconds. Then we train \model with joint type constraint for 20K steps, taking 5-7 minutes per object. 
For hyper-parameters, we set the threshold $\epsilon_{\text{static}}$ to identify static/movable Gaussians as $\epsilon_{\text{static}}=0.02 \cdot \max_{i} \text{CD}_{i}^{t\rightarrow \bar{t}}$ for two-part objects and $\epsilon_{\text{static}}=0.05 \cdot \max_{i} \text{CD}_{i}^{t\rightarrow \bar{t}}$ for multi-part objects. We use $\epsilon_{\text{revol}}=10^\degree$ for predicting joint types following PARIS~\citep{jiayi2023paris}.
$\lambda_{cd}$ and $\lambda_{reg}$ are set as 100 and 0.1 separately.
In addition, the CD loss in \cref{eq:cd_loss} aims to decrease the distance between a deformed Gaussian $G_i^t$ and its nearest Gaussians $\hat{G}_i^t$ in $\mathcal{G}^t_{\text{single}}$. Since the deformed Gaussians and canonical Gaussians for a prismatic joint have a large overlap, the nearest Gaussian may be in the opposite direction of the ideal one, making it ineffective for prismatic joints. Thus the CD loss is only used for regularizing the objects that only have revolute joints.
Moreover, the densification strategy of Gaussians is cloning or splitting one Gaussian when the gradient of its center $\vmu$ is greater than a threshold $\epsilon_{\text{densify}}$. This is effective for static scenes but meets challenges for dynamic scenes. In the early stage of training, the large gradient is often due to deformation error. To prevent excessive increase of Gaussian quantity due to deformation error, we raised this threshold $\epsilon_{\text{densify}}$ from 0.0002 used in previous works \citep{kerbl20233d,huang2024sc} to 0.001.

\section{Additional Discussions}
\label{app:discussion}
\begin{table*}[t]
\caption{\textbf{Quantitative evaluation of each state on PARIS data.} We report the average of metrics over 10 trials of each state. "metric-0/1" represents the metric evaluated at state 0/1 and "metric-m" is the average of two states. We highlight \colorbox[HTML]{ffc5c5}{best} results on average of two states. Axis Pos. is omitted for prismatic joints (Blade, Storage, and Real Storage).}
\label{tab:app:exp_2part}
\renewcommand{\arraystretch}{1.2}
\resizebox{\linewidth}{!}{
\begin{tabular}{cc|ccccccccccc|ccc}
\hline
\multirow{2}{*}{Metric} &\multirow{2}{*}{Method} &\multicolumn{11}{c}{Synthetic Objects} &\multicolumn{3}{|c}{Real Objects} \\
% \cmidrule(lr){}
& &FoldChair &Fridge &Laptop &Oven &Scissor &Stapler &USB 
&Washer&Blade &Storage &All & Fridge &Storage &All \\
\hline
\multirow{6}{*}{\shortstack{Axis\\Ang}} 
&DTA-0
&0.03 &0.09 &0.07 &0.22 &0.10 &0.06 &0.11 &0.36 &0.20 &0.07 &0.13 &2.08 &13.64 &7.86 \\
&Ours-0
&0.01 &0.03 &0.01 &0.01 &0.05 &0.01 &0.04 &0.02 &0.03 &0.01 &0.02 & 2.09 &3.47 &2.78 \\
&DTA-1
&0.04 &0.10 &0.07 &0.23 &0.10 &0.07 &0.11 &0.36 &0.26 &0.09 &0.14 &2.07 &8.08 &5.08 \\
&Ours-1
&0.01 &0.03 &0.01 &0.01 &0.05 &0.01 &0.04 &0.02 &0.03 &0.01 &0.02 & 2.09 &3.47 &2.78 \\
&DTA-m
&0.04 &0.10 &0.07 &0.22 &0.10 &0.06 &0.11 &0.36 &0.23 &0.08 &0.14 &\best{2.08} &10.86 &6.47 \\
&Ours-m
&\best{0.01} &\best{0.03} &\best{0.01} &\best{0.01} &\best{0.05} &\best{0.01} &\best{0.04} &\best{0.02} &\best{0.03} &\best{0.01} &\best{0.02} & 2.09 &\best{3.47} &\best{2.78} \\
\hline
\multirow{6}{*}{\shortstack{Axis\\Pos}} 
&DTA-0
&0.01 &0.01 &0.01 &0.01 &0.03 &0.02 &0.00 &0.04 &- &- &0.02 &0.59 &- &0.59 \\
&Ours-0 
&0.00 &0.00 &0.01 &0.00 &0.00 &0.01 &0.00 &0.00 & - & - &0.00 &0.47 & - &0.47 \\
&DTA-1
&0.01 &0.01 &0.01 &0.01 &0.02 &0.02 &0.00 &0.05 &- &- &0.02 &0.59 &- &0.59 \\
&Ours-1
&0.00 &0.00 &0.01 &0.00 &0.00 &0.01 &0.00 &0.00 & - & - &0.00 &0.47 & - &0.47 \\
&DTA-m
&0.01 &0.01 &0.01 &0.01 &0.03 &0.02 &0.00 &0.04 &- &- &0.02 &0.59 &- &0.59 \\
&Ours-m
&\best{0.00} &\best{0.00} &\best{0.01} &\best{0.00} &\best{0.00} &\best{0.01} &\best{0.00} &\best{0.00} & \best{-} & \best{-} &\best{0.00} &\best{0.47} & \best{-} &\best{0.47} \\
\hline
\multirow{6}{*}{\shortstack{Part\\Motion}}
&DTA-0
&0.10 &0.12 &0.11 &0.12 &0.38 &0.08 &0.15 &0.28 &0.00 &0.00 &0.13 &1.85 &0.14 &1.00 \\
&Ours-0
&0.03 &0.04
&0.02 &0.02
&0.04 &0.01
&0.03 &0.03
&0.00 &0.00
&0.02 & 1.94
&0.04 &0.99 \\
&DTA-1
&0.09 &0.13 &0.11 &0.13 &0.37 &0.08 &0.14 &0.28 &0.00 &0.00 &0.13 &1.85 &0.09 &0.97 \\
&Ours-1
&0.03 &0.04
&0.02 &0.02
&0.04 &0.01
&0.03 &0.03
&0.00 &0.00
&0.02 & 1.94
&0.04 &0.99 \\
&DTA-m
&0.09 &0.12 &0.11 &0.12 &0.38 &0.08 &0.15 &0.28 &0.00 &0.00 &0.13 &\best{1.85} &0.12 &0.99 \\
&Ours-m
&\best{0.03} &\best{0.04}
&\best{0.02} &\best{0.02}
&\best{0.04} &\best{0.01}
&\best{0.03} &\best{0.03}
&\best{0.00} &\best{0.00}
&\best{0.02} & 1.94
&\best{0.04} &\best{0.99} \\
\hline
\multirow{6}{*}{\shortstack{CD-s}} 
&DTA-0
&0.18 &0.62 &0.32 &4.60 &3.30 &2.68 &2.32 &4.77 &0.55 &4.71 &2.41 &2.36 &10.98 &6.67 \\
&Ours-0
&0.26 & 0.52 & 0.59 & 3.88 & 0.62 & 3.85 & 2.25 & 6.41 & 0.54 & 7.47 & 2.64 & 1.64 & 2.93 & 2.29 \\
&DTA-1
&0.19 &0.63 &0.30 &4.58 &3.55 &2.91 &2.90 &4.56 &0.45 &4.90 &2.50 &2.59 &9.60 &6.10 \\
&Ours-1
&0.26 & 0.48 & 0.63 & 4.00 & 0.61 & 3.83 & 2.56 & 6.43 & 0.54 & 7.31 & 2.67 & 2.01 & 4.02 & 3.02 \\
&DTA-m
&\best{0.19} &0.62 &\best{0.31} &4.59 &3.43 &\best{2.79} &2.61 &\best{4.66} &\best{0.50} &\best{4.80} &\best{2.46} &2.48 &10.29 &6.39 \\
&Ours-m
&0.26 &\best{0.50} 
&0.61 &\best{3.94} 
&\best{0.61} &3.84 
&\best{2.41} &6.42 &0.54 &7.39
&2.65 &\best{1.82 } 
&\best{3.48} &\best{2.65} \\
\hline
\multirow{6}{*}{\shortstack{CD-m}}
&DTA-0
&0.15 &0.27 &0.16 &0.44 &17.38 &2.34 &1.47 &0.37 &2.05 &0.36 &2.50 &1.12 &30.78 &15.95 \\
&Ours-0
&0.54 & 0.21 & 0.14 & 0.89 & 0.65 & 0.88 & 1.22 & 1.54 & 1.12 & 1.03 & 0.82 & 0.66 & 6.28 & 3.47 \\
&DTA-1
&0.13 &0.30 &0.13 &0.45 &10.11 &1.13 &1.51 &0.45 &61.38 &0.36 &7.60 &1.85 &365.74 &183.80 \\
&Ours-1
&0.12 & 0.21 & 0.13 & 0.76 & 0.64 & 0.52 & 1.43 & 0.45 & 1.01 & 1.02 & 0.63 & 1.31 & 87.81 & 44.56 \\
&DTA-m
&\best{0.14} &0.28 &0.15 &\best{0.44} &13.75 &1.73 &1.49 &\best{0.41} &31.72 &\best{0.36} &5.05 &1.48 &198.26 &99.88 \\
&Ours-m
&0.33 &\best{0.21} &\best{0.14} &0.82 &\best{0.65} &\best{0.70} &\best{1.33} &1.00
&\best{1.06} &1.02 &\best{0.73} &\best{0.99} &\best{47.05} &\best{24.02} \\

\hline
\multirow{6}{*}{\shortstack{CD-w}}
&DTA-0
&0.27 &0.70 &0.35 &4.24 &0.42 &2.13 &1.17 &4.59 &0.36 &4.09 &1.83 &2.08 &8.98 &5.53 \\
&Ours-0
&0.43 & 0.58 & 0.47 & 3.58 & 0.69 & 3.13 & 1.28 & 6.12 & 0.61 & 5.13 & 2.20 & 1.29 & 3.23 & 2.26 \\
&DTA-1
&0.26 &0.70 &0.32 &4.27 &0.41 &1.92 &1.52 &4.48 &0.38 &3.99 &1.83 &2.19 &9.03 &5.61 \\
&Ours-1
&0.30 & 0.59 & 0.50 & 3.71 & 0.67 & 2.63 & 1.87 & 5.99 & 0.65 & 5.21 & 2.21 & 1.45 & 2.45 & 1.95 \\
&DTA-m
&\best{0.26} &0.70 &\best{0.34} &4.25 &\best{0.41} &\best{2.02} &\best{1.34} &\best{4.53} &\best{0.37} &\best{4.04} &\best{1.83} &2.13 &9.01 &5.57 \\
&Ours-m
&0.36 &\best{0.59} &0.48 &\best{3.64} &0.68 &2.88 & 1.58 &6.05 & 0.63 & 5.17 &{2.21} &\best{1.37} &\best{2.84} &\best{2.11} \\

\hline
\end{tabular}
}
\end{table*}

We present a comprehensive analysis of \model and DTA through additional quantitative and qualitative results. 

\paragraph{Visibility Problem} Our results uncover an intriguing inconsistency in DTA's performance across different states of the same object. As illustrated in \cref{tab:app:exp_2part}, DTA demonstrates good reconstruction quality in the high-visibility state but shows markedly poor performance in the low-visibility state. This limitation is particularly pronounced for objects with prismatic joints, such as real storage and blade. In these cases, DTA struggles to accurately capture the geometry and articulation of partially occluded parts. 
The observed inconsistency and state-dependent performance fluctuations underscore the necessity for a more robust approach that effectively connects and leverages information from multiple states. This is precisely where \model's strengths become evident. By establishing connections between different articulation states, \model achieves more consistent and high-quality reconstructions across varying object configurations.
Jointly optimizing over multiple states allows \model to:
1) Leverage complementary information from different articulation states,
2) Maintain consistency in part assignment and geometry across states,
3) Better handle occlusions and low-visibility scenarios by inferring occluded geometries from other states.
These capabilities enable \model to produce more accurate and reliable reconstructions, particularly in challenging scenarios. The superior performance of \model demonstrates its potential for robust articulated object reconstruction in real-world applications.

\paragraph{Significance of Part Assignment}
Through analysis of both qualitative (\cref{fig:app:mpart}) and quantitative (\cref{tab:exp_mpart_our}) results, we have identified that the model's ultimate performance is primarily determined by the accuracy of part assignment. When the model fails to correctly divide an object into parts, it becomes impossible to obtain reasonable joint parameter estimation. Conversely, even when joint parameter estimation is inaccurate, the model may still correctly separate the object's parts. This insight reveals that accurate part assignment is a crucial prerequisite for high-quality articulated object reconstruction.
Our findings emphasize that to enhance the reconstruction of articulated objects, the ability to reasonably separate parts is of paramount importance. \model addresses this challenge through the center-based segmentation and improved initialization by clustering. These techniques work in synergy to significantly improve the part segmentation capabilities of \model. By enhancing the model's ability to correctly identify and separate object parts, we lay a solid foundation for subsequent stages of the reconstruction process, including joint parameter estimation and final mesh reconstruction.

\section{Limitations}
\label{app:limitation}
\paragraph{Stability of Randomness.} 
\model exhibits enhanced robustness and stability across different random seeds, primarily due to our innovative initialization strategy for canonical Gaussians and our part assignment module. We observe that stability issues often stem from the initialization of three key components: canonical Gaussians $\mathcal{G}^c$, part centers $C$ in the part assignment module, and joint articulation parameters $\Psi$.
As demonstrated in \cref{sec:exp:ablation}, faulty initialization of $\mathcal{G}^c$ and $C$ can lead to significant performance degradation, particularly for complex objects with multiple movable parts. While our current initialization strategy has greatly improved stability, severe initialization errors in center $C$ may still result in part mis-segmentation. We can integrate prior models such as SAM~\citep{kirillov2023segment} to enhance the ability to correct center initialization errors.
Although \model works with randomly initialized $\Psi$, we have observed that improved initialization of $\Psi$ brings enhanced performance. Future work could explore the integration of heuristic algorithms or feed-forward articulation estimation models to provide better initial estimation for $\Psi$.

\paragraph{Limited States} 
Our current approach is limited to modeling articulated objects using only two states, which may not fully capture the complexity of real-world multi-part objects. 
Moreover, as the number of parts increases, distinguishing parts with similar joint axes and motion patterns (such as parallel drawers) becomes increasingly challenging, complicating the segmentation process.
To address this, two main avenues could be explored for future research:
1) Multi-state Extension: Develop a methodology to extend \model to handle multiple states that interact with different parts, potentially by identifying movable parts with a sequential state update mechanism. This would involve iteratively updating the model as new state information becomes available, allowing for a more comprehensive representation of the object's articulation space.
2) Continuous Temporal Reconstruction: Adapt \model to reconstruct articulated objects from monocular video sequences. This approach would leverage temporal information to infer a continuous range of articulation states, providing a more nuanced understanding of the object's movement capabilities.

\paragraph{Mesh Reconstruction Fidelity} 
Our current implementation utilizes the original Gaussian Splatting technique, which, while effective, has limitations in terms of mesh reconstruction quality compared with NeRF-based methods\citep{wang2021neus,yariv2021volume,wen2023bundlesdf}. Integrating recent advancements in reconstruction with Gaussian Splatting~\citep{huang20242d,chen2024pgsr} may help to improve the reconstruction fidelity of \model.

{\section{Additional Experiments}}
% \label{app:addi_exp}
% \subsection{{Additional Ablation Studies}}
% \label{app:addi_ab}
{\subsection{Additional Quantitative Comparisons}}
\label{app:addi_ac}
\begin{table*}[t]
\caption{{\textbf{Quantitative evaluation of Axis Pos metric on PARIS.} Metrics are reported as mean $\pm$ std over 10 trials on average of 2 states. We report the value timed by 1000 and highlight the \colorbox[HTML]{ffc5c5}{best} results.}}
\label{tab:app:exp_ap}
\renewcommand{\arraystretch}{1.2}
\resizebox{\linewidth}{!}{
{\begin{tabular}{cc|ccccccccc}
\hline
{Metric} &{Method} &FoldChair &Fridge &Laptop &Oven &Scissor &Stapler &USB 
&Washer &All \\
\hline
\multirow{2}{*}{\shortstack{Axis\\Pos}} 
&DTA
&{0.53\tiny{$\pm$0.3}} &{0.62\tiny{$\pm$0.3}} &{1.10\tiny{$\pm$0.7}}
&{1.49\tiny{$\pm$1.0}} &{2.48\tiny{$\pm$2.8}} &{2.21\tiny{$\pm$1.8}} &{0.35\tiny{$\pm$0.2}} &{4.53\tiny{$\pm$2.8}}  &{1.66\tiny{$\pm$}1.2}  \\
&Ours
&\best{0.48\tiny{$\pm$0.2}} &\best{0.44\tiny{$\pm$0.2}} &\best{0.39\tiny{$\pm$0.3}} &\best{0.55\tiny{$\pm$0.4}} &\best{0.16\tiny{$\pm$0.1}} &\best{0.93\tiny{$\pm$0.4}} &\best{0.08\tiny{$\pm$0.1}} &\best{0.33\tiny{$\pm$0.3}} &\best{0.42\tiny{$\pm$0.3}}  \\
\hline
\end{tabular}
}
}
\end{table*}

\begin{table*}[t]
\caption{{\textbf{Quantitative evaluation for perception-based metrics on PARIS data.} We report the results on average of two states. We highlight \colorbox[HTML]{ffc5c5}{best} results.}}
\label{tab:app:exp_psnr}
\renewcommand{\arraystretch}{1.2}
\resizebox{\linewidth}{!}{
{
\begin{tabular}{cc|ccccccccccc|ccc}
\hline
\multirow{2}{*}{Metric} &\multirow{2}{*}{Method} &\multicolumn{11}{c}{Synthetic Objects} &\multicolumn{3}{|c}{Real Objects} \\
% \cmidrule(lr){}
& &FoldChair &Fridge &Laptop &Oven &Scissor &Stapler &USB 
&Washer&Blade &Storage &All & Fridge &Storage &All \\
\hline
\multirow{2}{*}{PSNR} 
&PARIS & 31.50 & \best{37.67} & \best{37.26} & 35.30 & \best{38.37} & 38.49 & 39.07 & \best{40.08} & 38.29 & 36.18 & 37.22 & 25.29 & \best{27.13} & 26.21 \\
&Ours & \best{34.46} & 37.11 & 34.09 & \best{37.06} & 38.29 & \best{39.13} & \best{39.64} & 38.50 & \best{41.16} & \best{37.24} & \best{37.67} & \best{27.05} & 25.38 & \best{26.22} \\
\hline
\multirow{2}{*}{SSIM} 
&PARIS & 0.985 & \best{0.994} & \best{0.991} & 0.980 & 0.996 & 0.995 & 0.992 & 0.991 & 0.996 & \best{0.993} & 0.991 & 0.898 & \best{0.953} & 0.926 \\
&Ours & \best{0.997} & 0.993 & 0.988 & \best{0.995} & \best{0.998} & \best{0.999} & \best{0.998} & \best{0.995} & \best{0.999} & 0.992 & \best{0.995} & \best{0.939} & 0.930 & \best{0.935} \\
\hline
\multirow{2}{*}{$\text{LPIPS}_{vgg}$} 
&PARIS & 0.045 & \best{0.032} & \best{0.020} & \best{0.045} & 0.015 & 0.019 & 0.029 & \best{0.029} & 0.017 & \best{0.095} & \best{0.035} & 0.188 & \best{0.139} & 0.164 \\
&Ours & \best{0.036} & 0.041 & 0.045 & 0.054 & \best{0.014} & \best{0.011} & \best{0.016} & 0.052 & \best{0.004} & 0.097 & 0.037 & \best{0.114} & 0.188 & \best{0.151} \\
\hline
\end{tabular}
}
}
\end{table*}

\begin{table*}[t]
\caption{{\textbf{Quantitative comparison for whole mesh reconstruction on PARIS data.} We report the average of CD-w over 10 trials. We bold \textbf{best} results on average of two states.}} 
\label{tab:app:exp_tsdf}
\renewcommand{\arraystretch}{1.2}
\resizebox{\linewidth}{!}{
{
\begin{tabular}{cc|ccccccccccc|ccc}
\hline
\multirow{2}{*}{Metric} &\multirow{2}{*}{Method} &\multicolumn{11}{c}{Synthetic Objects} &\multicolumn{3}{|c}{Real Objects} \\
% \cmidrule(lr){}
& &FoldChair &Fridge &Laptop &Oven &Scissor &Stapler &USB 
&Washer&Blade &Storage &All & Fridge &Storage &All \\
\hline
\multirow{3}{*}{\shortstack{CD-w}}
&DTA
&\textbf{0.26} &0.70 &\textbf{0.34} &4.25 &\textbf{0.41} &\textbf{2.02} &\textbf{1.34} &\textbf{4.53} &\textbf{0.37} &\textbf{4.04} &\textbf{1.83} &2.13 &9.01 &5.57 \\
&TSDF with gt depth
& 0.30 & \textbf{0.56} & 0.47 & \textbf{3.60} & 0.49 & 2.78 & 1.60 & 5.73 & 0.54 & 5.13 & 2.12 & 3.15 & 131.86 & 67.51\\
&Ours
&0.36 &{0.59} &0.48 &{3.64} &0.68 &2.88 & 1.58 &6.05 & 0.63 & 5.17 &{2.21} &\textbf{1.37} &\textbf{2.84} &\textbf{2.11} 
\\
\hline
\end{tabular}
}
}
\end{table*}

{We provide additional comparisons with previous methods in this section. 
\paragraph{Scaled Axis Pos Metric.} Following DTA and PARIS, we multiply the 'Axis Pos' metric by 10 in \cref{tab:exp_2part} and \cref{tab:app:exp_2part}. While this metric shows minimal variation among current methods for synthetic objects, we also report the Axis Pos metric multiplied by 1000. As shown in \cref{tab:app:exp_ap}, \model demonstrates superior performance compared to DTA.
\paragraph{Perception-based Metrics.} To evaluate rendering quality, we assess perception-based metrics including LPIPS~\cite{zhang2018unreasonable}, SSIM~\cite{wang2004image}, and PSNR, with results shown in \cref{tab:app:exp_psnr}. While our primary focus aligns with previous methods on mesh reconstruction and articulation estimation, \model achieves comparable or superior performance relative to PARIS. 
\paragraph{Limited Improvement for CD-w on Simple Synthetic Objects.} Our method's performance on simple synthetic objects, particularly in terms of CD-w metric, is constrained by our use of TSDF for mesh extraction from Gaussian Splatting-rendered depths. To analyze this limitation, we compare against meshes reconstructed using ground-truth depth with TSDF. As shown in \cref{tab:app:exp_tsdf}, even with ground-truth depth input, TSDF-based reconstruction cannot surpass algorithms using marching cubes with NeRF, primarily due to the fundamental differences between TSDF and marching cubes algorithms on simple geometries. However, for complex or real-world objects where articulation reconstruction becomes more critical, the advantages of our model become evident. Additionally, TSDF with ground truth depth on real-world objects may produce poor-quality meshes (e.g., real\_storage) due to depth sensor noise, while our \model achieves high-quality reconstruction. Importantly, our primary objective is to create digital twins of real-world articulated objects, where \model demonstrates significant performance improvements, particularly for complex and real-world scenarios. 
}

{\subsection{Failure Cases}}
{\paragraph{Incorrect Initialization of Part Centers.} For real-world objects with multiple parts, clustering-derived part centers may be inaccurate (\cref{fig:failure} (a)) due to sensor noise, occlusion, and varying illumination conditions. These incorrectly initialized centers often persist through optimization, degrading performance for parts with misaligned centers (\cref{fig:failure} (c)). Manual correction of erroneous part centers prior to training (\cref{fig:failure} (b)) yields improved results (\cref{fig:failure} (d)). As discussed in \cref{app:limitation}, incorporating prior models like SAM for automatic, accurate part center initialization remains a promising direction for future work.}

{\paragraph{Similar Motions.} Our method exhibits limitations when handling parts with identical motion across states, as demonstrated in case 2 of \cref{fig:failure} where two drawers are pulled with the same distance. In such scenarios, the model tends to learn a single joint to fit both parts, failing to distinguish between the independently movable parts. As discussed in \cref{app:limitation}, expanding \model to incorporate additional states would provide richer motion information, potentially enabling better part separation.}

\begin{figure}[t!]
 \centering
 \resizebox{\linewidth}{!}{\includegraphics[width=\linewidth]{figure/failure.pdf}}
  \caption{{\textbf{Failure cases}. We illustrate failure cases of our \model. 'Init./Opt. Cano.' represents initialized and optimized Canonical Gaussians, while the prefix 'M' indicates manual correction of erroneous part centers.}}
 \label{fig:failure}
 \vspace{-10pt}
\end{figure}

\begin{figure}[t!]
 \centering
 \resizebox{\linewidth}{!}{\includegraphics[width=\linewidth]{figure/refinement.pdf}}
\caption{{\textbf{Evolution of canonical Gaussians}. We visualize the evolution of canonical Gaussians, showing both their part assignments and centers. Our initialization strategy begins with dense static Gaussians and sparse dynamic Gaussians. As training progresses, the Gaussians undergo densification while simultaneously refining their part centers and assignments. These visualization results demonstrate the effectiveness of \model.}}
 \label{fig:evolution}
 \vspace{-10pt}
\end{figure}


{\subsection{Evolution of Canonical Gaussians}}
{We visualize the evolution of canonical Gaussians in \cref{fig:evolution}, showing both their part assignments and centers. Our initialization strategy begins with dense static Gaussians and sparse dynamic Gaussians. As training progresses, the Gaussians undergo densification while simultaneously refining their part centers and assignments. These visualization results demonstrate the effectiveness of \model.}
{\subsection{Additional Qualitative Comparisons}}
{We provide additional qualitative comparisons on different datasets in the following pages.}
\begin{figure}[t!]
 \centering
 \resizebox{\linewidth}{!}{\includegraphics[width=\linewidth]{figure/vis_exp_multi_1.pdf}}
 \caption{\textbf{Additional qualitative results on \model-Multi.}}
 \label{fig:app:mpart}
 \vspace{-10pt}
\end{figure}
\begin{figure}[t!]
 \centering
 \resizebox{\linewidth}{!}{\includegraphics[width=\linewidth]{figure/interp_supp.pdf}}
 \caption{{\textbf{Interpolation results on PARIS data.}}}
 \label{fig:interp_supp}
 \vspace{-10pt}
\end{figure}

\end{document}



