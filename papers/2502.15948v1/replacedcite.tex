\section{Related Work}
\label{sec:rel-work}
Robots are frequently used for health monitoring tasks, compared to screen-based interactions, as their physical presence as interaction partners can lead to more favorable responses and a more positive perception____.
In____, a robot and a \ac{VA} were compared for small talk conversations among adults aged 45-65. 
The robot was considered friendlier and more trustworthy, while the \ac{VA} was favored for its learnability and interaction speed. 
Social robots are also a suitable choice for the health assessment of older adults in particular, as they can help combat social isolation in hospital settings____.
According to a scoping review____, social robots can also help combat loneliness, but more research needs to be conducted on \acp{VA} in this domain. 
In an evaluation of both a \ac{VA} and a social robot to support isolated older adults____, participants preferred the robot in the home environment. 
However, the authors argue that \acp{VA} may help support older adults on the go, e.g., when getting exercise or going shopping, and that social robots and \acp{VA} may thus complement each other. 

In a comparison of the virtual and physical representation of the iCat robot with a text interface in the context of health self-management of older adults____, the virtual and physical character were seen as more empathetic and trustworthy than the text-based interface, and the older adults were more conversational with the characters. 
Research on user preference for either a \ac{VA} or a robot has explored several influencing factors: 
In____, the advantages of the robot's physical presence for older users are highlighted, suggesting increased engagement in a coaching context for mental support. 
However, the used \ac{VA} was a replica of the robot, limiting facial expressions and thus compromising its human-like appeal. 
In____, older adults preferred interacting with \acp{VA} over robots, citing the \ac{VA}'s less robotic appearance and higher scores in reliability, practicality, and engagement.

While acceptance and willingness to use social robots and \acp{VA} for daily assistance are important factors, long-term motivation to actually use the agent is not guaranteed. 
When deploying \acp{SIA} in long-term studies in real-world environments, the novelty effect____ needs to be considered, as the motivation and excitement can decline as soon as the first week of usage____. 
True long-term interactions may also be referred to as interactions that have gone beyond the point in time when the novelty effect wears off____. 
Here, differences between individual users are also crucial to consider, as preferences, expectations, and comfort with technology can vary widely among older adults____. 

% Research gap 
The current state of research suggests that user preferences for either \acp{VA} or robots are influenced by various subtle factors, often dependent on the specific application domain. 
However, previous studies have primarily focused on either highly realistic \acp{VA}____ or virtual representations of the robot____. 
These approaches may not fully exploit the potential of \acp{VA}: 
Overly realistic \acp{VA} may create high user expectations that are difficult to meet____, while designing the \ac{VA} as a replica of the robot means inheriting the robot's limitations, such as the inability to display facial expressions.
As such, while each interface has its advantages and disadvantages, direct comparisons of both are still lacking in literature, particularly in real-world settings and contexts, and over longer periods of time, when the novelty of the interaction with the \ac{SIA} has worn off. 
To address this gap, we present an eight-week, in-home comparison of a social robot and a \ac{VA}, in which older adults chose their preferred interface for health monitoring.
%