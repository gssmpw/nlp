\begin{figure*}
\centering
\newcommand*{\myfontsize}{\small}
\myfontsize
\begin{tikzpicture}
\newcommand*{\edgeArrowHeight}{2em}
\newcommand*{\minBlockWidth}{5em}
\newcommand*{\maxBlockWidth}{8em}
\newcommand*{\varwidthBlock}[2][\maxBlockWidth]{%
    \begin{varwidth}{#1}
        \centering
        #2
    \end{varwidth}
}
\small
% Create a matrix of nodes that spans the width of the text
\matrix[matrix of nodes,
        nodes={rectangle, draw, text centered, minimum height=2em, minimum width=3em, font=\myfontsize},
        row sep=2em,
        column sep=2em,
        %nodes in empty cells
        ] (m) 
{
    \varwidthBlock[6em]{Model derivation (Sect.~\ref{sec:manipulator-dynamic-model})} &
    \varwidthBlock{Linear dynamic parameters identification (Sect.~\ref{sec:linear-dynamic-parameters})} &
    \varwidthBlock{Nonlinear friction parameters computation (Sect.~\ref{sec:nonlinear-friction-parameters})} &
    \varwidthBlock[\minBlockWidth]{Motor drive gains estimation (Sect.~\ref{sec:motor-drive-gains})} &
    \node[draw, rounded corners=0.75em] (m-1-5) {\varwidthBlock[\minBlockWidth]{Estimated model \eqref{eq:estimated-torques-and-gains}}}; \\
    \varwidthBlock{Excitation trajectory design (Sect.~\ref{sec:excitation-trajectories})} &
    \node[draw, rounded corners=1em] (m-2-2){Robot}; & &
    \varwidthBlock{Payload description (Sect.~\ref{sec:modified-dynamic-coefficients-due-to-a-payload})} &
    \varwidthBlock[\minBlockWidth]{Payload integration (Sect.~\ref{sec:dynamic-model-identification-payload-integration})}\\
};

% Define a style for the circles
\tikzset{my-circle/.style={circle, fill=black, minimum size=1.5mm, inner sep=0pt}}

% Draw arrow entering block (1,1)
\draw[->, thick] ([xshift=-2em]m-1-1.west) -- node[pos=0,left,text width=5em,align=center] {DH parameters} (m-1-1.west);

% Draw arrow from block (1,1) to block (1,2)
\draw[->, thick] (m-1-1.east) -- (m-1-2.west) node[midway, below] {$\gls{c-minimal-regressor}$};

% Draw arrow from block (1,1) to block (1,3)
\path (m-1-1) -- (m-1-2) coordinate[pos=0.5] (midpoint-1-2);
\coordinate (m-1-3-above) at ([yshift=+2em]m-1-3.north);
\node[my-circle] at (midpoint-1-2) {};
\draw[->, thick] (midpoint-1-2) |- (m-1-3-above) -- (m-1-3.north) {};

% Draw arrow from block (1,2) to block (1,3)
\draw[->, thick] (m-1-2.east) -- (m-1-3.west) node[midway, above] {$\hat{\gls{c-d-coefficients}}$};

% Draw arrow from block (1,3) to block (1,4)
\draw[->, thick] (m-1-3.east) -- (m-1-4.west) node[midway, below] {$\hat{\gls{f-set}}$};

% Draw arrow from block (1,3) to block (1,5)
\path (m-1-3) -- (m-1-4) coordinate[pos=0.5] (midpoint-3-4);
\coordinate (m-1-4-above-1) at ([yshift=+1em]m-1-4.north);
\node[my-circle] at (midpoint-3-4) {};
\draw[-, thick] (midpoint-3-4) |- (m-1-4-above-1);
\draw[->, thick] (m-1-4-above-1) -| ([xshift=-0.83em]m-1-5.north);

% Draw arrow from block (1,4) to block (1,5)
\draw[->, thick] (m-1-4.east) -- (m-1-5.west) node[pos=0.6, above] {$\hat{\gls{mdg}}$};

% Draw arrow exiting block (1,5)
\draw[->, thick] (m-1-5.east) -- node[midway,above] {$\hat{\bm\tau}$} ++(2em, 0);

% Draw arrow from block (1,1) to block (1,4) (regressor line)
\coordinate (m-1-4-above-3) at ([yshift=+3em]m-1-4.north);
\draw[->, thick]
(m-1-1.north) |-
([xshift=+0.83em]m-1-4-above-3) node[near end,above]{$\gls{regressor}$} --
([xshift=+0.83em]m-1-4.north);

% Draw arrow from block (1,1) to block (1,4) (current-level minimal regressor line)
\node[my-circle] at (m-1-3-above) {};
\draw[->, thick] (m-1-3-above) -| ([xshift=-0.83em]m-1-4.north);

% Draw arrow from block (1,1) to block (1,5)
\coordinate (m-1-4-above-2) at ([yshift=2em]m-1-4.north);
\node[my-circle] at ([xshift=-0.83em]m-1-4-above-2) {};
\draw[->, thick] ([xshift=-0.83em]m-1-4-above-2) -| ([xshift=+0.83em]m-1-5.north);

% Draw arrow from block (1,1) to block (2,5)
\path (m-1-5) -- (m-2-5) coordinate[pos=0.5] (aux);
\node[my-circle] at ([xshift=+0.83em]m-1-4-above-3) {};
\draw[-, thick] ([xshift=+0.83em]m-1-4-above-3) -| ([xshift=1em]m-1-4.east);
\draw[-, thick] ([xshift=1em]m-1-4.east) |- (aux);
\draw[->, thick] (aux) -- (m-2-5.north);

% Draw arrow entering block (2,4)
\draw[->, thick] ([xshift=-2em]m-2-4.west) -- node[pos=0,left] {$\gls{mass}_L, \gls{com}_l, \gls{inertia-tensor}_l$} (m-2-4.west);

% Draw arrows connecting blocks in bottom row
\draw[->, thick] (m-2-1.east) -- node[midway,above] {$\bm q(t)$} (m-2-2.west);
\draw[->, thick] (m-2-2.north) -- node[midway,right] {$\underline{\bm q}, \underline{\dot{\bm q}}, \underline{\ddot{\bm q}}, \underline{\bm q}$} (m-1-2.south);
\draw[->, thick] (m-2-4.north) -- node[midway,left] {$\gls{d-parameters}_L$} (m-1-4.south);
\draw[->, thick] (m-2-4.east) -- node[midway,above]{$\gls{d-parameters}_L$} (m-2-5.west);
\draw[->, thick] (m-2-5.east) -- node[midway,above] {\gls{torques-p}} ++(2em, 0);

\end{tikzpicture}
\caption{Estimation procedure}
\label{fig:block-scheme}
\end{figure*}