\section{Problem Formulation}

\noindent \textbf{Problem.} Given an arbitrary textual description $t$ as the input condition, e.g. "a small bedroom for a young student", our goal is to generate plausible, realistic, and physically feasible scenes corresponding to the requirement. The generated scene includes the selected 3D models and the arranged placements of the objects therein, i.e. $S=\{(M_i, P_i)|i=1,...,n\}$, where $M_i$ and $P_i$ denote the selected 3D model and spatial placement respectively for the $i$th object. The placement of each object contains its center position coordinates, orientation angle, and size.

\noindent \textbf{Hierarchical Scene Representation.} 
We propose to use the hierarchical scene representation throughout our pipeline. It is a three-level hierarchical structure, as shown in Figure~\ref{fig:hierarchy}. The first level is the root node representing the entire scene, the second level is the internal nodes each representing a rectangular functional area, and the third level is the leaf nodes representing the objects belonging to the corresponding area. The nodes are connected with two types of edges,
i.e. the parent-child relation indicating the hierarchical structure and the pairwise relation between objects to represent
their spatial relationship. Specifically, to reduce the redundancy, we set one anchor object for each functional area and only allow the pairwise relations between the anchor object and other objects belonging to the same functional area.

Each node in the hierarchy contains some attributes. Assuming axis-aligned rectangular floorplans for the scenes, the root node $r$ has a size attribute $s_r$ and a text description $t_r$ of the scene, i.e. $r = \{t_r, s_r\}$. Each internal node $a$, which represents an axis-aligned functional area, carries the text description $t_a$, the size attributes $s_a$, as well as a center position $p_a$ and an orientation $\theta_a$, i.e. $a = \{t_a, s_a, p_a, \theta_a\}$. The position is a 2D coordinate while the orientation is a binary value representing either horizontal or vertical direction. Each object node $o$ contain the text description $t_o$, the category label $c_o$, the corresponding 3D model $M_o$, as well as the size $s_o$, center position $p_o$, orientation $\theta_o$ of the oriented bounding box of the object, i.e. $o = \{t_o, c_o, M_o, s_o, p_o, \theta_o\}$. Note that the size attributes are 2D vectors for room and functional areas, but 3D vectors for objects, since we also care about their heights. In addition, the pairwise spatial relationship $e$ stores the coarse text description $t_e$ such as "in front of" and the fine-grained relative placement coordinates including position $p_e$ and orientation $\theta_e$ of one object w.r.t. the anchor object, i.e. $e = \{t_e, p_e, \theta_e\}$.