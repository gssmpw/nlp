%File: formatting-instructions-latex-2025.tex
%release 2025.0
\documentclass[letterpaper]{article} % DO NOT CHANGE THIS
\usepackage{aaai25}  % DO NOT CHANGE THIS
\usepackage{times}  % DO NOT CHANGE THIS
\usepackage{helvet}  % DO NOT CHANGE THIS
\usepackage{courier}  % DO NOT CHANGE THIS
\usepackage[hyphens]{url}  % DO NOT CHANGE THIS
\usepackage{graphicx} % DO NOT CHANGE THIS
\urlstyle{rm} % DO NOT CHANGE THIS
\def\UrlFont{\rm}  % DO NOT CHANGE THIS
\usepackage{natbib}  % DO NOT CHANGE THIS AND DO NOT ADD ANY OPTIONS TO IT
\usepackage{caption} % DO NOT CHANGE THIS AND DO NOT ADD ANY OPTIONS TO IT
\frenchspacing  % DO NOT CHANGE THIS
\setlength{\pdfpagewidth}{8.5in}  % DO NOT CHANGE THIS
\setlength{\pdfpageheight}{11in}  % DO NOT CHANGE THIS

%
% These are recommended to typeset algorithms but not required. See the subsubsection on algorithms. Remove them if you don't have algorithms in your paper.
\usepackage{algorithm}
\usepackage{algorithmic}
\usepackage{amsmath}
\usepackage{amssymb}
\usepackage{color}
\usepackage{multirow} 
%
% These are are recommended to typeset listings but not required. See the subsubsection on listing. Remove this block if you don't have listings in your paper.
\usepackage{newfloat}
\usepackage{listings}
\DeclareCaptionStyle{ruled}{labelfont=normalfont,labelsep=colon,strut=off} % DO NOT CHANGE THIS
\lstset{%
	basicstyle={\footnotesize\ttfamily},% footnotesize acceptable for monospace
	numbers=left,numberstyle=\footnotesize,xleftmargin=2em,% show line numbers, remove this entire line if you don't want the numbers.
	aboveskip=0pt,belowskip=0pt,%
	showstringspaces=false,tabsize=2,breaklines=true}
\floatstyle{ruled}
\newfloat{listing}{tb}{lst}{}
\floatname{listing}{Listing}
%
% Keep the \pdfinfo as shown here. There's no need
% for you to add the /Title and /Author tags.
\pdfinfo{
/TemplateVersion (2025.1)
}

% DISALLOWED PACKAGES
% \usepackage{authblk} -- This package is specifically forbidden
% \usepackage{balance} -- This package is specifically forbidden
% \usepackage{color (if used in text)
% \usepackage{CJK} -- This package is specifically forbidden
% \usepackage{float} -- This package is specifically forbidden
% \usepackage{flushend} -- This package is specifically forbidden
% \usepackage{fontenc} -- This package is specifically forbidden
% \usepackage{fullpage} -- This package is specifically forbidden
% \usepackage{geometry} -- This package is specifically forbidden
% \usepackage{grffile} -- This package is specifically forbidden
% \usepackage{hyperref} -- This package is specifically forbidden
% \usepackage{navigator} -- This package is specifically forbidden
% (or any other package that embeds links such as navigator or hyperref)
% \indentfirst} -- This package is specifically forbidden
% \layout} -- This package is specifically forbidden
% \multicol} -- This package is specifically forbidden
% \nameref} -- This package is specifically forbidden
% \usepackage{savetrees} -- This package is specifically forbidden
% \usepackage{setspace} -- This package is specifically forbidden
% \usepackage{stfloats} -- This package is specifically forbidden
% \usepackage{tabu} -- This package is specifically forbidden
% \usepackage{titlesec} -- This package is specifically forbidden
% \usepackage{tocbibind} -- This package is specifically forbidden
% \usepackage{ulem} -- This package is specifically forbidden
% \usepackage{wrapfig} -- This package is specifically forbidden
% DISALLOWED COMMANDS
% \nocopyright -- Your paper will not be published if you use this command
% \addtolength -- This command may not be used
% \balance -- This command may not be used
% \baselinestretch -- Your paper will not be published if you use this command
% \clearpage -- No page breaks of any kind may be used for the final version of your paper
% \columnsep -- This command may not be used
% \newpage -- No page breaks of any kind may be used for the final version of your paper
% \pagebreak -- No page breaks of any kind may be used for the final version of your paperr
% \pagestyle -- This command may not be used
% \tiny -- This is not an acceptable font size.
% \vspace{- -- No negative value may be used in proximity of a caption, figure, table, section, subsection, subsubsection, or reference
% \vskip{- -- No negative value may be used to alter spacing above or below a caption, figure, table, section, subsection, subsubsection, or reference

\setcounter{secnumdepth}{2} %May be changed to 1 or 2 if section numbers are desired.

% The file aaai25.sty is the style file for AAAI Press
% proceedings, working notes, and technical reports.
%

% Title

% Your title must be in mixed case, not sentence case.
% That means all verbs (including short verbs like be, is, using,and go),
% nouns, adverbs, adjectives should be capitalized, including both words in hyphenated terms, while
% articles, conjunctions, and prepositions are lower case unless they
% directly follow a colon or long dash
\title{Hierarchically-Structured Open-Vocabulary Indoor Scene Synthesis  
\par with Pre-trained Large Language Model}
\author{
    % Authors
    Weilin Sun\textsuperscript{\rm 1},
    Xinran Li\textsuperscript{\rm 1},
    Manyi Li\textsuperscript{\rm 1}\thanks{Corresponding Author.},
    Kai Xu\textsuperscript{\rm 2}\footnotemark[1],
    Xiangxu Meng\textsuperscript{\rm 1},
    Lei Meng\textsuperscript{\rm 1, 3}\footnotemark[1]
}
\affiliations{
    % Affiliations
    \textsuperscript{\rm 1}School of Software, Shandong University, China\\
    \textsuperscript{\rm 2} School of Computer Science, National University of Defense Technology, China\\
    \textsuperscript{\rm 3} Shandong Research Institute of Industrial Technology, China\\
    \{sunweilin, xinranli\}@mail.sdu.edu.cn, manyili@sdu.edu.cn, kevin.kai.xu@gmail.com, \{mxx, lmeng\}@sdu.edu.cn, 
}

%Example, Single Author, ->> remove \iffalse,\fi and place them surrounding AAAI title to use it
\iffalse
\title{My Publication Title --- Single Author}
\author {
    Author Name
}
\affiliations{
    Affiliation\\
    Affiliation Line 2\\
    name@example.com
}
\fi

\iffalse
%Example, Multiple Authors, ->> remove \iffalse,\fi and place them surrounding AAAI title to use it
\title{My Publication Title --- Multiple Authors}
\author {
    % Authors
    First Author Name\textsuperscript{\rm 1,\rm 2},
    Second Author Name\textsuperscript{\rm 2},
    Third Author Name\textsuperscript{\rm 1}
}
\affiliations {
    % Affiliations
    \textsuperscript{\rm 1}Affiliation 1\\
    \textsuperscript{\rm 2}Affiliation 2\\
    firstAuthor@affiliation1.com, secondAuthor@affilation2.com, thirdAuthor@affiliation1.com
}
\fi


% REMOVE THIS: bibentry
% This is only needed to show inline citations in the guidelines document. You should not need it and can safely delete it.
\usepackage{bibentry}
% END REMOVE bibentry

\begin{document}

\maketitle

\begin{abstract}
Indoor scene synthesis aims to automatically produce plausible, realistic and diverse 3D indoor scenes, especially given arbitrary user requirements. Recently, the promising generalization ability of pre-trained large language models (LLM) assist in open-vocabulary indoor scene synthesis. However, the challenge lies in converting the LLM-generated outputs into reasonable and physically feasible scene layouts. In this paper, we propose to generate hierarchically structured scene descriptions with LLM and then compute the scene layouts. Specifically, we train a hierarchy-aware network to infer the fine-grained relative positions between objects and design a divide-and-conquer optimization to solve for scene layouts. The advantages of using hierarchically structured scene representation are two-fold. First, the hierarchical structure provides a rough grounding for object arrangement, which alleviates contradictory placements with dense relations and enhances the generalization ability of the network to infer fine-grained placements. Second, it naturally supports the divide-and-conquer optimization, by first arranging the sub-scenes and then the entire scene, to more effectively solve for a feasible layout. We conduct extensive comparison experiments and ablation studies with both qualitative and quantitative evaluations to validate the effectiveness of our key designs with the hierarchically structured scene representation. Our approach can generate more reasonable scene layouts while better aligned with the user requirements and LLM descriptions. We also present open-vocabulary scene synthesis and interactive scene design results to show the strength of our approach in the applications.
\end{abstract}

% Uncomment the following to link to your code, datasets, an extended version or similar.
%
% \begin{links}
%     \link{Code}{https://aaai.org/example/code}
%     \link{Datasets}{https://aaai.org/example/datasets}
%     \link{Extended version}{https://aaai.org/example/extended-version}
% \end{links}

%motivation
The prevalence of misinformation is a growing concern globally. Misinformation damages society in numerous ways. It threatens the maintenance of trust in vaccines and health policies\cite{do2022infodemics,macdonald2023meme}, can incite violence and harassment\cite{cdtFromFellows}, can undermine democratic processes (particularly elections)\cite{bovet2019influence,groshek2017helping}, and harms individual and societal well-being\cite{verma2022examining}. For example, during the 2018 Brazilian presidential election, manipulated photos, decontextualized videos, and hoax audio significantly influenced election results by aiding the victory of the main far-right candidate \cite{theguardianWhatsAppFake,santos2020social}. Another example is the spread of false information during the COVID-19 pandemic. Many people were persuaded that ineffective or counterproductive treatments such as alcohol-based cleaning products and an anti-parasitic drug could cure patients. The victims of this misinformation could suffer serious illness or even death \cite{bbcHundredsDead}.

Different countermeasures have been investigated to combat misinformation. These interventions can be categorized into two major groups: pre-emptive intervention ("prebunking") and reactive intervention ("debunking")\cite{ecker2022psychological}. Debunking involves fact-checking and correcting misinformation after it has been encountered. Fact-checkers use investigative practices to determine the veracity of content and dispute factual inaccuracies\cite{juneja2022human}.  %The most common debunking approach uses fact-checked information to dispute misinformation and provide accurate information\cite{hameleers2020misinformation,facebookIntoFacebook,aghajari2023reviewing}. 

%difficulties of overcoming misinfo
%describe debunking and spreading info. "two aspects of misinfo: how it spreads and how to debunk it once it's here."
However, the lasting effects of misinformation make it challenging to mitigate its influence once people have been exposed \cite{roets2017fake,lewandowsky2012misinformation}. Furthermore, fact-checking efforts are limited in terms of scale and reach, which restricts their effectiveness\cite{roozenbeek2020prebunking}. Given these challenges, it is not sufficient to rely purely on debunking efforts. Prebunking, on the other hand, works to build attitudinal inoculation, enabling people to identify and resist manipulative messages. This approach equips individuals to better manage misinformation they encounter in the real world\cite{prebin2024}. Prebunking is based on a range of educational measures \cite{dame2022combating,cook2017neutralizing}, which can include games \cite{roozenbeek2019fake,cook2023cranky}.

%helping by teaching people  

%preemptive corrections before misinformation is encountered, effectively reducing reliance on misinformation. 

%This includes digital media literacy education activities\cite{dame2022combating,cook2017neutralizing} and designing serious games to improve media literacy\cite{roozenbeek2019fake,cook2023cranky}. 
%Prebunking becoming an important 



%valuable complement by raising people's awareness and understanding of misinformation, helping them more effectively navigate credible, biased, and false information. One promising application of prebunking is through game-based learning. 

%games + research gap
Recent research indicates that the game-based learning approaches are potentially useful prebunking interventions. These games educate and build resistance in people exposed to misinformation\cite{traberg2022psychological,kiili2024tackling}. In games such as "Bad News,"\cite{roozenbeek2019fake} "Harmony Square,"\cite{harmonysquare} "Go Viral!"\cite{camCambridgeGame} and "Trustme!"\cite{yang2021can} players adopt the role of a misinformation producer whose task is to create and spread misinformation as efficiently as possible. Another approach, applied in the games "MAthE"\cite{katsaounidou2019mathe} and "Escape Fake"\cite{escapefake}, involves players acting as fact-checkers and challenging them to assess the validity of information. In some scenarios players can use tools such as search engines to gather clues. 

%the problem with standard PVP games (deterministic games), lack of natural interaction, use common language to explain better. doesn't feel real world. 
While gamified misinformation interventions have shown promise in teaching players about the nature of misinformation, %either from the perspective of the creator or the audience, 
the choice-based formats of these games can limit replayability. %and fun that make games engaging.
%Another Moreover, these games often present only one-shot interactions with misinformation, which may not capture the complexity and ever-changing nature of misinformation in the real world\cite{shin2018diffusion}. 
Additionally, such choice-based game formats require little cognitive effort from the player, who is presented with a limited number of pre-generated options. This can diminish player involvement in the game and reduce enjoyment of gameplay. Another factor limiting engagement is that most of these games are designed for single-player mode. Multiplayer mechanics, by contrast, are an effective way to enhance motivation and replayability in games\cite{mustaro2012immersion}. Social interactions also significantly contribute to player engagement in educational and other serious games\cite{lepper2021intrinsic}. Research has demonstrated that both collaborative and competitive gameplay can enhance the effectiveness of serious games\cite{cagiltay2015effect,bellotti2010designing}.
%and increase player motivation \cite{cagiltay2015effect,bellotti2010designing}.

%GAMES: 2 problems:
%1. not natural, not realistic, deterministic -> LLM solution.
%2. engagement and attitude -> two-player PvP (one against another goal).

To address the challenges of non-natural interactions and a deterministic game paths (with a view to fostering more open-ended exploration and engagement through Player versus Player (PvP)), we need to incorporate more complex scenarios. These would give players the ability to not only choose from preselected options, but also to actively generate content and implement their own strategies for debunking or creating misinformation. Additionally, implementing a PvP approach can be a training tool through which players can learn how to counter real-life misinformation.

%critically discern misinformation.


%During the game, players learn how several misinformation manipulation techniques can be used to produce credible fake news\cite{roozenbeek2019fake}. Another type of game involves players acting as fact-checkers whose task is to identify misinformation or fake news\cite{katsaounidou2019mathe,escapefake}.

%However, these games may overlook key features of misinformation. One noticeable feature is that misinformation typically involves more than isolated instances. During significant events like elections, wars, or health crises, misinformation evolves across various formats (text, pictures, infographics, videos) and stages. For example, during the COVID-19 pandemic, misinformation ranged from false cures and conspiracy theories about the virus's origin to vaccine misinformation, each gaining prominence at different stages[REF]. Another feature is that misinformation is constantly changing. Research found that political false rumors tend to become more intense and extreme over time\cite{shin2018diffusion}, and health misinformation statuses change as new evidence emerges\cite{tang2024knows}.

%evolution idea - using LLM to model public opinion.
%Therefore, there is a need to design a game that not only enhances the ability to distinguish misinformation but also raises awareness of its evolving characteristics and various forms. 
%Advancements in AI and Natural Language Processing (NLP) open the opportunities for creating engaging games. Large language models (LLMs), such as ChatGPT, have been applied in video games for generative narratives\cite{park2023generative}, NPC dialogue\cite{ashby2023personalized,uludaugli2023non}, and role-playing\cite{xu2023exploring}. 

%Integrating LLMs into misinformation learning games could dynamically adapt to player interactions, providing a more engaging and personalized experience. This is particularly useful for capturing and reflecting the dynamic nature of misinformation events. However, further exploration is needed to detail this integration.


Progress in AI and Natural Language Processing (NLP) provide new opportunities for creating more engaging game experiences. Large language models (LLMs) can simulate complex human interactions and societal dynamics\cite{ziems2024can}and have been applied in video games for tasks like generating narratives\cite{park2023generative}, non-playable characters (NPCs),dialogue\cite{ashby2023personalized,uludaugli2023non}, and role-playing scenarios\cite{xu2023exploring}. 
Previous work also demonstrated the opportunities for prompting the LLM to impersonate a specific character and create interactive dialogues from this perspective\cite{zhou2024eternagram,shao2023character}. Integrating LLMs into the game can introduce greater variability to in-game interactions and enhances both engagement and replayability. As the effectiveness of many inoculation interventions tends to diminish over time, developing an enjoyable and replayable game that consistently reinforces players' cognitive "resilience" against misinformation is an essential advancement in this field\cite{wells2024doomscroll}. Applying LLM to the game also has the potential to increase the educational impact of the intervention. It allows users not just to select from a range of options but also to put their input into the model and receive individual feedback tailored to this input.

%To bridge the gap in misinformation education games and explore LLM-driven games, 
Inspired by previous misinformation game interventions, this research has involved the development of a PvP misinformation education game called \textit{Breaking the News}. In our game, two players are assigned either the role of a misinformation creator (referred to as an "influencer" in the game) or a counteractor of misinformation (referred to as a "journalist-debunker" in the game). The influencer creates misinformation posts in a system that mimics a social media environment, while the debunker seeks to counter these messages by presenting compelling arguments. LLMs are used to represent public opinion in the "country" where the game events take place. The goal for both players is to earn the trust of the citizens and convince them to believe the information they present. 

In this paper we aim to answer the following research questions: 

\textbf{RQ1:}  
How may we empower players to understand the processes of misinformation generation and misinformation debunking through a GenAI-based PvP game?

\textbf{RQ2:} 
What behaviors do players exhibit when they are asked to generate versus protect against misinformation?
 
In this paper, we present the design and evaluation of the PvP game. We conducted a mixed-method study with 47 participants, using a within-subjects design and pre- and post-surveys for repeated measures. Our findings suggest that through gameplay, participants improved their ability to reflect on instances of misinformation, raised their levels of media literacy, expanded their repertoire of strategies applied to countering misinformation, and improved their discernment abilities. This study contributes to the growing body of work analyzing misinformation education games. Specifically, we provide insights into integrating LLMs and interactive PvP mechanics in media literacy contexts. We also offer practical guidance for the design of serious games aimed at combating misinformation in a dynamic, real-world manner.

%based on events rather than isolated posts or headlines. %In our game, %two different parties (debunker and disinformation creator (influencer) are battling for influence on people's opinion about event.control, and the hearts and minds of their citizens and the global community.
%two players are assigned either the role of a misinformation maker (influencer) or a misinformation stopper (debunker). %One player will start with an unfolding event and will experience the creation and dissemination of information and misinformation in various formats and stages. The other player will have to identify misinformation and use different countermeasures to respond and combat misinformation in the game. After the second player's responses to misinformation, they will receive simulated reactions from a group of citizens, generated by LLM, and decide on the next steps. After a few rounds, mimicking misinformation diffusion patterns, the game will have results. 

%The influencer will create misinformation posts, while the debunker tries to resolve the issue by proposing compelling arguments. The LLM-4 is representing represent the public opinion of several citizens in the "Country" where the game events take place.

%The goal for both players is to earn the trust of the citizens and convince them to believe the information they present.

%to align more closely with their direction. Once the game is finished, the players will have a chance to review their strategies at different stages. 

%The game aimed to enhance players' ability to reflect on instances of misinformation, raise their media literacy, expand their repertoire of strategies applied to counter misinformation, and improve their discernment abilities.

%for real-world encounters.
%Define player versus player.


%needs to be more conservative here:
%RQ2: What behaviors do players exhibit when they are asked to generate and protect against misinformation?

%what study did you do?
%what are the 3 major findings?
%our contribution (short version).

\section{Related Work}

\noindent \textbf{Indoor Scene Synthesis.} The common practice is to produce a set of objects and their placements~\cite{Patil2023AdvancesID}, i.e. scene layouts. Early works rely on the pre-defined rules~\cite{Yu2011MakeIH, Merrell2011InteractiveFL, Ma2016Actiondriven3I, Fu2017AdaptiveSO, Fu2020HumancentricMF} to generate interpretable and feasible scene layouts. To further capture diverse spatial arrangement, the data-driven approaches~\cite{fisher2012example, Qi2018HumanCentricIS, xu2014organizing, ma2018language, Sun2022SequentialFO, Sun2024SequentialSA} learn the object relationship from datasets~\cite{fu20213d, song2017semantic}. The researchers have developed all kinds of networks to learn the scenes represented as different data structures, including sequences~\cite{Wang2018DeepCP}, graphs~\cite{Zhou2019SceneGraphNetNM, Wang2019PlanITPA}, hierarchies~\cite{li2019grains, Gao2023SceneHGNHG}, sets~\cite{Paschalidou2021ATISSAT, Wei2023LEGONetLR, Tang2023DiffuSceneSG, zhai2024echoscene}, etc. However, due to the inherent complexity, it is difficult to capture the essential relationship from the observed layouts and generalize to other categories.

\begin{tabular}{cc}
\toprule
& SP Policy\\
\midrule
Hierarchical Policy  & $0.86\pm0.06$ \\
\bottomrule 
\end{tabular}


Incorporating additional knowledge enhances scene prior learning.  Graph-to-3D~\cite{Dhamo2021Graphto3DEG} and CommonScenes~\cite{zhai2024commonscenes} synthesize the layouts and object shapes for coherent scenes. Some methods~\cite{ye2022scene, yi2023mime} take human motion trajectory as conditions to populate the objects. Haisor~\cite{sun2024haisor} uses reinforcement learning with human interaction and space area consideration for scene synthesis. External expert knowledge~\cite{leimer2022layoutenhancer, yang2024learning} can also be incorporated during training to enhance network performance. These works refer to the same observation that indoor scene synthesis involves a comprehensive consideration of space partition, functional arrangement, and aesthetic creativity, thus requiring a generative model with extensive knowledge.


\noindent \textbf{Text-to-Scene Synthesis.} The challenges include semantic understanding of user requirements and scene synthesis. Given a natural language description, early works~\cite{ChangSM14, ChangMSPM15, ChangESM17, SavvaCA17, MaPFLPHYTGZ18, Yang2021} parse the input as scene templates, where the nodes represent objects and edges for spatial relations, and then sample the corresponding object models and placements. With the development of deep learning, some works~\cite{Paschalidou2021ATISSAT, Tang2023DiffuSceneSG, Dinh2024} take latent vectors such as textual embeddings as conditions, and train conditional scene synthesis model to output scenes. However, these works rely on detailed descriptions as input to specify the setting of target scenes, e.g. "there is a desk and there is a notepad on the desk", rather than reasoning the scene configuration from abstract instructions. Moreover, generalizing to diverse scene categories and open-vocabulary settings is a long-standing problem. 

\begin{figure*}[t]
    \centering
    \includegraphics[width=0.90\linewidth]{figures/pipeline.pdf}
    \vspace{-2mm}
    \caption{Overall pipeline of proposed {\ours}. An encoded low-resolution video latent is concatenated to the current noisy latent $x_t$, and it undergoes alternating denoising processes using both {\ssi} (\ssiabb) and {\tfi} (\tfiabb). At this stage, the noisy latent is split and merged before and after the denoising process, respectively. 
    After each denoising step, {\drg} ({\drgabb}) is applied to enhance the quality of the image further.}
    \vspace{-2mm}
    \Description{pipeline}
    \label{fig:main}
\end{figure*}

\noindent \textbf{LLM-Assisted Scene Synthesis.} Recent works have investigated utilizing the pre-trained LLMs to handle the multi-objects in the scenes, most of which focus on the 2D layout for controllable scene image synthesis~\cite{lian2023llm, gani2023llm, nie2024compositional}. LayoutGPT~\cite{Feng2023LayoutGPTCV} retrieves scene layouts in CSS format, with LLMs outputting numerical bounding boxes for each object. However, due to the lack of spatial reasoning ability, LLM cannot handle the complex relationships of 3D scenes, causing heavy object overlap and out-of-boundary problems.

Some others use LLM to generate a textual scene description and convert it to 3D scenes. Aladdin~\cite{Huang2023AladdinZH} introduced a pipeline to sample and generate 3D textured assets from an abstract description and manually organize them to construct a scene. Recent works~\cite{Wen2023AnyHomeOG, Yang2023HolodeckLG} require LLM to describe the object relations using pre-defined atom relations, which are interpreted as fixed relative positions between objects and refined using a rule-based optimization algorithm. However, our experiments show that defining a compact yet informative atom relations is difficult. Dense and detailed object relations are able to provide precise spatial arrangement but often cause self-contradiction in the LLM outputs, while a sparse set of coarse relations leads to coherent arrangements but fails to capture the diverse spatial placements among objects. 

% !TEX root = template.tex

\section{System Setup} \label{sec:problem}
Let a group of heterogeneous agents $\mathcal{N} = \{a_i \mid i=1,2,...,N\}$ operate within a partially known workspace. Each agent can execute primitive actions that may require assistance from others. The agents are connected via a shared network. Within this framework, agents can directly exchange messages with any other agent in the workspace. %Next, we define the agents' models and their assigned tasks.
\subsection{System description}
\subsubsection{Motion Transition System} \label{subsec:motion-ts}
Agent $a_i$'s motion within the workspace is modeled as an FTS. Our approach focuses on a set of ROIs while a low-level controller handles obstacle avoidance and inter-region movement. This approach significantly reduces computational complexity compared to a fully partitioned workspace but sacrifices the ability to track, at any time, the agent's exact location within the FTS. Each agent $a_i$ is aware only of the set of $M^{a_i}$ ROIs, denoted by $\Pi^{a_i}_{\mathcal{M}}=\{\pi^{a_i}_1,\pi^{a_i}_2,...,\pi^{a_i}_{M^{a_i}}\}$. The FTS assigned agent $a_i$ is:
\begin{equation} \label{eq:motion-fts} \mathcal{T}^{a_i}_{\mathcal{M}}\triangleq\left(\Pi^{a_i}_{\mathcal{M}}, \Pi^{a_i}_{\mathcal{M},0}, \Psi^{a_i}_{\mathcal{M}}, \Sigma^{a_i}_{\mathcal{M}}, \longrightarrow^{a_i}_{\mathcal{M}}, \mathrm{L}^{a_i}_{\mathcal{M}}, \mathrm{T}^{a_i}_{\mathcal{M}}\right),  
\end{equation}
where $\Pi^{a_i}_{\mathcal{M},0}\in\Pi^{a_i}_{\mathcal{M}}$ is the initial ROI, $\Psi^{a_i}_{\mathcal{M}}$ is the set of atomic propositions describing the properties of the workspace, $\Sigma^{a_i}_{\mathcal{M}}$ is the set of movement actions, $\longrightarrow^{a_i}_{\mathcal{M}}\subseteq \Pi^{a_i}_{\mathcal{M}}\times\Sigma^{a_i}_{\mathcal{M}}\times\Pi^{a_i}_{\mathcal{M}}$ is the transition relation, $\mathrm{L}^{a_i}_{\mathcal{M}}:\Pi^{a_i}\rightarrow2^{\Psi^{a_i}_{\mathcal{M}}}$ is the labeling function, indicating the properties held by each ROI, and $\mathrm{T} ^{a_i}_{\mathcal{M}}:\longrightarrow^{a_i}_{\mathcal{M}}\rightarrow\mathbb{R}^+$ is the transition time function, representing the estimated time necessary for each transition.

\subsubsection{Action Model}\label{subsec:action-model}
In addition to its movement actions, agent $a_i$ can perform actions  $\Sigma^{a_i}_{\mathscr{A}} \triangleq \Sigma^{a_i}_l \cup \Sigma^{a_i}_c \cup \Sigma^{a_i}_h$, where $\Sigma^{a_i}_l$ are \textit{local} actions performed independently, $\Sigma^{a_i}_c$ are \textit{collaborative} actions requiring assistance from other agents, and $\Sigma^{a_i}_h$ are \textit{assisting} actions carried out to help others. Lastly, $\sigma_0 = \mathit{None}\in \Sigma^{a_i}_l$ indicates that $a_i$ remains idle. The action model for agent $a_i$ is defined as the tuple:
\begin{equation}\label{eq:action-model}
\mathscr{A}^{a_i} \triangleq \left(\Sigma^{a_i}_{\mathscr{A}}, \Psi^{a_i}_{\mathscr{A}}, \mathrm{L}^{a_i}_{\mathscr{A}}, \mathrm{Cond}^{a_i}, \mathrm{Dura}^{a_i}, \mathrm{Depd}^{a_i}\right),
\end{equation}
where $\Psi^{a_i}_{\mathscr{A}}$ is the set of atomic propositions, $\mathrm{L}^{a_i}_{\mathscr{A}}$ is the labeling function as in \cite{meng_paper}, $\mathrm{Cond}^{a_i}$ is the region properties required to execute an action, $\mathrm{Dura}^{a_i}$ is the action duration, with $\mathrm{Dura}^{a_i}(\sigma_s) = T_s > 0$, and 
$\mathrm{Depd}^{a_i}: \Sigma^{a_i}_{\mathscr{A}} \rightarrow 2^{\Sigma^{\sim a_i}_h} \times 2^{\Pi^{\mathcal{N}}}$ denotes the dependence function, where  $\Sigma^{\sim a_i}_h$ is the set of \textit{external} assisting actions that agent $a_i$ depends on, and $\Pi^{\mathcal{N}}=\cup_{a_i\in\mathcal{N}}\Pi^{a_i}_{\mathcal{M}}$. Given $\sigma_c\in \Sigma^{a_i}_c $, we define the set of actions involved in a collaboration as:
\begin{equation}\label{eq:collaboration}
    \mathcal{C}(\sigma_c)=\{\sigma_c\}\cup\mathrm{Depd}^{a_i}(\sigma_c).
\end{equation}
\begin{definition}\label{def:succesful-collab}
    A collaboration is considered successful if all actions involved are synchronized; i.e. to complete $\sigma_c \in \Sigma_c^{a_i}$, it is necessary that all actions in $\mathcal{C}(\sigma_c)$ start simultaneously.
\end{definition}
\subsubsection{Agent Transition System} \label{subsec:agent-ts}

The planner in Sec. \ref{subsec:planning} requires to define agent $a_i$'s FTS by combining \eqref{eq:motion-fts} and \eqref{eq:action-model}.
\begin{definition}
 Given $\mathcal{T}^{a_i}_{\mathcal{M}}$ and $\mathscr{A}^{a_i}$, a valid FTS for agent $a_i$, according to \cite{model-checking}, can be constructed as follows:
    \begin{equation}\label{eq:agent-model}    \mathcal{T}^{a_i}_{\mathcal{G}}\triangleq\left(\Pi^{a_i}_{\mathcal{G}}, \Pi^{a_i}_{\mathcal{G},0}, \Psi^{a_i}_{\mathcal{G}}, \Sigma^{a_i}_{\mathcal{G}}, \longrightarrow^{a_i}_{\mathcal{G}}, \mathrm{L}^{a_i}_{\mathcal{G}}, \mathrm{T}^{a_i}_{\mathcal{G}}\right), 
    \end{equation}
where $\Pi^{a_i}_{\mathcal{G}} = \Pi^{a_i}_{\mathcal{M}}\times \Sigma^{a_i}_{\mathscr{A}}$ is the set states,
$\Pi^{a_i}_{\mathcal{G},0}=\langle\Pi^{a_i}_{\mathcal{M},0} , \mathit{None}\rangle$ is the initial state,
$\Psi^{a_i}_{\mathcal{G}}$ is the set of atomic propositions,
$\Sigma^{a_i}_{\mathcal{G}}=\Sigma^{a_i}_{\mathcal{M}}\bigcup\Sigma^{a_i}_{\mathscr{A}}$, with $\Sigma^{a_i}_{\mathcal{G}, l}=\Sigma^{a_i}_{\mathcal{M}}\bigcup\Sigma^{a_i}_l$, 
$\longrightarrow^{a_i}_{\mathcal{G}}$ is the transition relation,
$\mathrm{L}^{a_i}_{\mathcal{G}}$ is the labeling function, and 
$\mathrm{T}^{a_i}_{\mathcal{G}}$ is the transition estimated duration \cite{meng_paper}.
\end{definition}
As in \cite{meng_paper},  the path is denoted by  $\tau^{a_{i}}=\pi^{a_i}_{\mathcal{G}, 0} \pi^{a_i}_{\mathcal{G}, 1} \ldots$ its trace by $\mathit{trace}(\tau^{a_{i}})=L_{\mathcal{G}}^{a_{i}}(\pi^{a_i}_{\mathcal{G}, 0}) L_{\mathcal{G}}^{a_{i}}(\pi^{a_i}_{\mathcal{G}, 1}) \ldots$ and, the associated sequence of actions by $\rho^{a_i}=\sigma^{a_i}_0, \sigma^{a_i}_1,\ldots$, i.e., the actions that allow transition between the states of $\tau^{a_{i}}$. 

\subsubsection{Task Specification}\label{subsec:task}
%In \cite{meng_paper} sc-LTL was considered but o
Our focus is on implementing recurring tasks i.e., tasks that repeat infinitely often. We will consider the following syntax $\varphi' ::=\top \mid a \mid \neg a\mid \varphi'_1\wedge\varphi'_2 \mid \lozenge\varphi'$, and for agent $a_i$ we define the recurring task as
\begin{equation}\label{eq:recurringLTL}
\varphi^{a_i}_r=\varphi'_1\wedge\square\lozenge\varphi'_2.
\end{equation}
Note that $\varphi'_2$ cannot start with $\lozenge$ to guarantee the validity of the LTL formula.
Given any satisfying word of $\varphi^{a_i}_r$, inserting a detour i.e., a finite sequence of states, between two consecutive states results in a satisfying word.

%\subsection{Problem Statement}
%We can summarize the problem as follows:
\begin{problem}\label{problem:task}
Given $\mathcal{T}_{\mathcal{G}}^{a_i}$ and the locally assigned task $\varphi^{a_i}_r$, design a distributed coordination and synchronization scheme such that $\varphi^{a_i}_r$ is satisfied for all $a_i \in \mathcal{N}$.%, and 2)   %The algorithm must also compensate for delays induced by the experimental scenario and 
 %all joint actions involved in a collaboration in  \eqref{eq:collaboration} start simultaneously.
\end{problem}
%\begin{remark}
   %Synchronizing actions that require precise timing, such as loading boxes, is critical for successful collaboration. Otherwise, timing discrepancies could lead to failure. 
%\end{remark}

\section{Method}

We conduct a systematic literature review to address our research question. Following prior method~\cite{nightingale2009guide}, we aim to identify relevant research papers on RPAs and provide a comprehensive summary of the literature. We selected four widely used academic databases: Google Scholar, ACM Digital Library, IEEE Xplore, and ACL Anthology. These databases encompass a broad spectrum of research across AI, human-computer interaction, and computational linguistics. Given the rapid advancements in LLM research, we included both peer-reviewed and preprint studies (e.g., from arXiv) to capture the latest developments. Below, we detail our paper selection and annotation process.

\begin{table*}[t]
\small

\caption{Definition and examples of six agent attributes.}
% \vspace{-0.5em}
\resizebox{\textwidth}{!}{%
\begin{tabular}{@{}p{0.21\textwidth}p{0.45\textwidth}p{0.33\textwidth}@{}}
\toprule
\textbf{Agent attributes}     & \textbf{Definition}     & \textbf{Examples} \\ 
\midrule
Activity History        & A record of past actions, behaviors, and engagements, including schedules, browsing history, and lifestyle choices. & Backstory, plot, weekly schedule, browsing history, social media posts, lifestyle       \\ 
Belief and Value        & The principles, attitudes, and ideological stances that shape an individual's perspectives and decisions.           & Stances, beliefs, attitudes, values, political leaning, religion                            \\ 
Demographic Information & Personal identifying details such as name, age, education, career, and location.                                    & Name, appearance, gender, age, date of birth, education, location, career, household income \\ 
Psychological Traits    & Characteristics related to personality, emotions, interests, and cognitive tendencies.                              & Personality, hobby and interest, emotional                                                  \\ 
Skill and Expertise     & The knowledge level, proficiency, and capability in specific domains or technologies.                             & Knowledge level, technology proficiency, skills                                            \\ 
Social Relationships & The nature and dynamics of interactions with others, including roles, connections, and communication styles.        & Parenting styles, interactions with players                                                \\ 
\bottomrule
\end{tabular}
}
\vspace{-1em}
\label{attr_def}
\end{table*}

\subsection{Literature Search and Screening Method}

\begin{figure}
    \includegraphics[width=\linewidth]{Figures/simple-PRISMA-1.png}
    % \vspace{-1em}
    \caption{Screening process of literature review. We initially retrieved $1,676$ papers published between 2021 and 2024, and narrowed down to $122$ final selections.}
    \vspace{-1em}
    \label{fig:prisma}
\end{figure}

Our literature review focuses on LLM agents that role-play human behaviors, such as decision-making, reasoning, and deliberate actions. We specifically focus on studies where LLM agents demonstrate the ability to simulate human-like cognitive processes in their objectives, methodologies, or evaluation techniques. To ensure methodological rigor, we define explicit inclusion and exclusion criteria (Tab.~\ref{tab:criteria} in Appendix~\ref{tab: inclusion and exclusion criteria}). 

The inclusion criteria require that an LLM agent in the study exhibits human-like behavior, engages in cognitive activities such as decision-making or reasoning, and operates in an open-ended task environment. We excluded studies where LLM agents primarily serve as chatbots, task-specific assistants, evaluators, or agents operating within predefined and finite action spaces. Additionally, studies focusing solely on perception-based tasks (e.g., computer vision or sensor-based autonomous driving) without cognitive simulation were also excluded.

Using this scope, we searched four databases using the query string provided in Appendix~\ref{query string}, retrieving $1,676$ papers published between January 2021 to December 2024. After removing duplicates, $1,573$ unique papers remained. Two authors independently screened the paper titles and abstracts based on the inclusion criteria. If at least one author deemed a paper relevant, it proceeded to full-text screening, where two authors reviewed the paper in detail and resolved any disagreements through discussion (Fig.~\ref{fig:prisma}). The final set of selected studies comprised $122$ publications.


\subsection{Paper Annotation Method}
Our team followed established open coding procedures \cite{brod2009qualitative} to conduct an inductive coding process to identify key themes. Three co-authors with extensive experience in LLM agents (``annotators,'' hereinafter) collaboratively annotated the papers on three dimensions: \textbf{agent attributes}, \textbf{task attributes}, and \textbf{evaluation metrics}. 

To ensure consistency, two annotators independently annotated the same 20\% of articles and then held a meeting to discuss and refine an initial set of categories for the three dimensions. After reaching a consensus, each annotator annotated half of the remaining papers and cross-validated the other half annotated by the other annotator. Once the annotations were completed, a third annotator reviewed the coded data and identified potential discrepancies. 
Any discrepancies were discussed among the annotators to ensure consistency until disagreements were resolved, ensuring reliability and validity through an iterative refinement process.
% \vspace{-8pt}
\section{Experiment and Results}
\begin{table*}[t]
\centering
\resizebox{\linewidth}{!}{
\begin{tabular}{cccccc}
    \toprule
    \textbf{Task} & \makecell[c]{\textbf{Dataset}} & \textbf{Metrics} & \textbf{Model}  & \textbf{Integration Method} & \textbf{Performance} \\
    \midrule
    \multirow{22}{*}{ASR} & \multirow{12}{*}{\makecell[c]{\textbf{Librispeech}\\ \small \textit{dev-clean} | \textit{dev-other} | \\ \small \textit{test-clean} | \textit{test-other}}} & \multirow{12}{*}{WER $\downarrow$} &  Whisper-large-v2 \cite{radford-etal-2023-whisper} & Non-LLM  & --- | --- | 2.7 | 5.2 \\
    &&& HyPoradise \cite{chen-2023-hyporadise} & Text-based $\rightarrow$ LLM GER & --- | --- | 1.8 | 3.7 \\
    &&&  BLSP \cite{wang2023blsp} & Latent-representation-based $\rightarrow$ Convolutional Downsampling  & --- | --- | 10.4 | --- \\
    &&&  SpeechVerse \cite{das2024speechverse} & Latent-representation-based $\rightarrow$ Convolutional Downsampling  & --- | --- | 2.1 | 4.4 \\
    &&&  Seed-ASR \cite{bai2024seed} & Latent-representation-based $\rightarrow$ Convolutional Downsampling  & --- | --- | \textbf{1.5} | \textbf{2.8} \\
    &&&  SALMONN \cite{tang-etal-2024-salmonn} & Latent-representation-based $\rightarrow$ Q-Former  & --- | --- | 2.1 | 4.9\\
    &&&  SLM \cite{wang-2023-slm} & Latent-representation-based $\rightarrow$ Other Adaptation Methods  &   --- | --- | 2.6 | 5.0  \\
    &&& Qwen-Audio \cite{chu-2023-qwen-audio} & Latent-representation-based $\rightarrow$ Other Adaptation Methods   & 1.8 | 4.0 | 2.0 | 4.2 \\
    &&&  Qwen2-Audio \cite{chu-2024-qwen2-audio} & Latent-representation-based $\rightarrow$ Other Adaptation Methods  & \textbf{1.3} | \textbf{3.4} | 1.6 | 3.6\\
    &&& LauraGPT \cite{du-etal-2024-lauragpt} & Latent-representation-based $\rightarrow$ Other Adaptation Methods & --- | --- | 4.4 | 7.7 \\
    &&&  SpeechGPT-Gen \cite{zhang-etal-2024-speechgptgen} & Audio-token-based $\rightarrow$ Semantic Token & --- | --- | 2.4 | --- \\
    &&& SLAM-ASR \cite{ma2024embarrassingly} & Latent-representation-based $\rightarrow$ Convolutional Downsampling   & --- | --- | 1.9 | 3.8 \\
    &&& BESTOW \cite{chen2024bestow} & Latent-representation-based $\rightarrow$ Other Adaptation Methods   & --- | --- | --- | 3.2 \\
    &&& \citet{hao-2024-boosting} & Audio-token-based $\rightarrow$ Acoustic Tokens & 3.7 | 6.6 | 3.4 | 7.1 \\
    \cline{2-6}
    & \multirow{4}{*}{\makecell[c]{\textbf{Fleurs}\\ \small \textit{zh} | \textit{en}}} & \multirow{4}{*}{WER $\downarrow$} &  Whisper-large-v2 \cite{radford-etal-2023-whisper} & Non-LLM & 4.2 | 14.7 \\
    &&& \citet{huang-2024-fusion-comprehensive} w/ PaLM2 & Text-based $\rightarrow$ LLM Rescoring &  13.1 | --- \\
    &&&  Seed-ASR \cite{bai2024seed} & Latent-representation-based $\rightarrow$ Convolutional Downsampling &  \textbf{3.43} | ---  \\
    &&&  Qwen2-Audio \cite{chu-2024-qwen2-audio} & Latent-representation-based $\rightarrow$ Other Adaptation Methods  &   --- | \textbf{7.5}  \\
    &&&  DiscreteSLU \cite{shon2024discreteslu} & Audio-token-based $\rightarrow$ Semantic Token  &   12.6 | ---  \\
    \cline{2-6}
    & \multirow{4}{*}{\makecell[c]{\textbf{AISHELL-2} \\ \small \textit{Mic | iOS | Android}}} & \multirow{4}{*}{WER $\downarrow$} &  Seed-ASR \cite{bai2024seed} & Latent-representation-based $\rightarrow$ Convolutional Downsampling & \textbf{2.2} | \textbf{2.2} | \textbf{2.2} \\ 
    &&& Qwen-Audio \cite{chu-2023-qwen-audio} & Latent-representation-based $\rightarrow$ Other Adaptation Methods & 3.3 | 3.1 | 3.3 \\
    &&&  Qwen2-Audio \cite{chu-2024-qwen2-audio} & Latent-representation-based $\rightarrow$ Other Adaptation Methods & 3.0 | 3.0 | 2.9 \\
    &&& LauraGPT \cite{du-etal-2024-lauragpt} & Latent-representation-based $\rightarrow$ Other Adaptation Methods & --- | 3.2 | --- \\
    \cline{2-6}
    & \multirow{2}{*}{\makecell[c]{\textbf{VoxPopoli}\\ \small \textit{All}}} & \multirow{2}{*}{WER $\downarrow$} &  SpeechVerse \cite{das2024speechverse} & Latent-representation-based $\rightarrow$ Convolutional Downsampling  & \textbf{6.5} \\ 
    &&& AudioPaLM \cite{rubenstein-etal-2023-audiopalm} & Audio-token-based $\rightarrow$ Semantic and Acoustic Tokens & 9.8 \\
    \hline
    
    \multirow{11}{*}{S2TT} & \multirow{11}{*}{\makecell[c]{\textbf{CoVoST2}\\ \small \textit{en-de} | \textit{de-en} |\\ \small \textit{en-zh} | \textit{zh-en} |\\ \small \textit{es-en} | \textit{fr-en} |\\ \small \textit{it-en} | \textit{ja-en}}} & \multirow{11}{*}{BLEU $\uparrow$}
    &  Whisper-large-v2 \cite{radford-etal-2023-whisper} & Non-LLM & --- | 36.3 | --- | 18.0 | 40.1 | 36.4 | 30.9 | 26.1 \\
    &&&  GenTranslate \cite{hu-etal-2024-gentranslate} & Text-based $\rightarrow$ LLM GER  & --- | 39.2 | --- | 21.6 | 42.0 | 41.7 | --- | 22.9 \\
    &&&  GenTranslate-V2 \cite{hu-etal-2024-gentranslate} & Text-based $\rightarrow$ LLM GER & --- | 40.6 | --- | 23.3 | 43.6 | 42.7 | --- | 25.4 \\
    &&& BLSP \cite{wang2023blsp} & Latent-representation-based $\rightarrow$ Convolutional Downsampling & 24.4 | --- | 41.3 | --- | --- | --- | --- | --- \\
    &&&  Speech-Llama \cite{wu-2023-decoder-only} & Latent-representation-based $\rightarrow$ CTC Compression & --- | 27.1 | --- | 12.3 | 27.9 | 25.2 | 25.9 | 19.9 \\
    &&& SALMONN \cite{tang-etal-2024-salmonn} & Latent-representation-based $\rightarrow$ Q-Former & 18.6 | --- | 33.1 | --- | --- | --- | --- | --- \\
    &&&  Qwen-Audio \cite{chu-2023-qwen-audio} & Latent-representation-based $\rightarrow$ Other Adaptation Methods & 25.1 | 33.9 | 41.5 | 15.7 | 39.7 | 38.5 | 36.0 | --- \\
    &&& Qwen2-Audio \cite{chu-2024-qwen2-audio} & Latent-representation-based $\rightarrow$ Other Adaptation Methods & \textbf{29.9} | 35.2 | \textbf{45.2} | 24.4 | 40.0 | 38.5 | 36.3 | --- \\
    &&&  LauraGPT \cite{du-etal-2024-lauragpt} & Latent-representation-based $\rightarrow$ Other Adaptation Methods & --- | --- | 38.5 | --- | --- | --- | --- | --- \\
    &&& LLaST \cite{chen2024llast} & Latent-representation-based $\rightarrow$ Other Adaptation Methods & --- | 41.2 | --- | 24.8 | \textbf{46.1} | \textbf{45.1} | \textbf{43.0} | \textbf{28.8} \\
    &&&  AudioPaLM \cite{rubenstein-etal-2023-audiopalm} & Audio-token-based $\rightarrow$ Semantic and Acoustic Tokens & --- | \textbf{43.4} | --- | \textbf{25.5} | 44.2 | 44.8 | --- | 25.9 \\
    &&& Ideal-LLM \cite{xue2024ideal} & Latent-representation-based $\rightarrow$ Other Adaptation Methods & 25.9 | 38.5 | --- | --- | 41.5 | 40.0 | 38.0 | --- \\
    \hline
    
    \multirow{4}{*}{S2ST} & \multirow{4}{*}{\makecell[c]{\textbf{CVSS S2ST}\\ \small \textit{de-en} | \textit{zh-en} |\\ \small \textit{es-en} | \textit{fr-en} |\\ \small \textit{it-en} | \textit{ja-en}}} & \multirow{4}{*}{ASR-BLEU $\uparrow$} &  Translatotron 2 + pretraining &  &  \\
    &&&   + TTS aug \cite{jia-etal-2022-s2st} & \multirow{-2}{*}{Non-LLM} &  \multirow{-2}{*}{33.6 | 13.1 | 38.5 | 36.5 | 35.7 | 8.5} \\
    &&& & & \\
    &&& \multirow{-2}{*}{AudioPaLM \cite{rubenstein-etal-2023-audiopalm}} & \multirow{-2}{*}{Audio-token-based $\rightarrow$ Semantic and Acoustic Tokens} & \multirow{-2}{*}{\textbf{37.2} | \textbf{20.0} | \textbf{40.4} | \textbf{38.3} | \textbf{39.4} | \textbf{20.9}} \\
    \hline

    \multirow{18}{*}{TTS} & \multirow{3}{*}{\makecell[c]{\textbf{AISHELL-1}}} & \multirow{3}{*}{\makecell[c]{CER $\downarrow$ | SECS $\uparrow$ |\\ MOSNet $\uparrow$}} &  VALL-E Phone \cite{wang-etal-2023-valle} & Non-LLM & \textbf{4.75} | \textbf{0.91} | \textbf{3.22} \\
    &&& VALL-E Token \cite{wang-etal-2023-valle} & Non-LLM & 6.52 | \textbf{0.91} | 3.19 \\
    &&&  LauraGPT \cite{du-etal-2024-lauragpt} & Audio-token-based $\rightarrow$ Acoustic Tokens &  6.91 | 0.90 | 3.14 \\
    \cline{2-6}
    & \multirow{5}{*}{\makecell[c]{\textbf{LibriTTS}}} & \multirow{5}{*}{\makecell[c]{WER $\downarrow$ | SECS $\uparrow$ |\\ MOSNet $\uparrow$}} & VALL-E Phone \cite{wang-etal-2023-valle} & Non-LLM & \textbf{4.30} | 0.92 | 3.28 \\
    &&&  VALL-E Token \cite{wang-etal-2023-valle} & Non-LLM & 6.57 | \textbf{0.93} | 3.28 \\
    &&& SpeechGPT-Gen (zero-shot) \cite{zhang-etal-2024-speechgptgen} & Audio-token-based $\rightarrow$ Semantic Token & 3.10 | 0.63 | 3.63 \\
    &&&  VoxtLM \cite{maiti-etal-2024-voxtlm} & Audio-token-based $\rightarrow$ Semantic Token & --- | --- | \textbf{4.36} \\
    &&& LauraGPT \cite{du-etal-2024-lauragpt} & Audio-token-based $\rightarrow$ Acoustic Tokens &   8.62 | 0.91 | 3.26 \\
    \cline{2-6}
    & \multirow{5}{*}{\makecell[c]{\textbf{Topic-StoryCloze}}} & \multirow{5}{*}{\makecell[c]{TSC $\uparrow$}} &  Spirit-LM \cite{nguyen-etal-2025-spiritlm} & Audio-token-based $\rightarrow$ Semantic Token  & 82.9 \\
    &&& TWIST-1.3B \cite{hassid-etal-2023-twist} & Audio-token-based $\rightarrow$ Semantic Token & 70.6 \\
    &&&  TWIST-7B \cite{hassid-etal-2023-twist} & Audio-token-based $\rightarrow$ Semantic Token & 74.1 \\
    &&& TWIST-13B \cite{hassid-etal-2023-twist} & Audio-token-based $\rightarrow$ Semantic Token & 76.4 \\
    &&&  Moshi \cite{defossez-etal-2024-moshi} & Audio-token-based $\rightarrow$ Semantic and Acoustic Tokens & \textbf{83.6} \\
    \cline{2-6}
    & \multirow{5}{*}{\makecell[c]{\textbf{StoryCloze}}} & \multirow{5}{*}{\makecell[c]{SSC $\uparrow$}} & Spirit-LM \cite{nguyen-etal-2025-spiritlm} & Audio-token-based $\rightarrow$ Semantic Token & 61.0 \\
    &&&  TWIST-1.3B \cite{hassid-etal-2023-twist} & Audio-token-based $\rightarrow$ Semantic Token & 52.4 \\
    &&& TWIST-7B \cite{hassid-etal-2023-twist} & Audio-token-based $\rightarrow$ Semantic Token & 55.3 \\
    &&&  TWIST-13B \cite{hassid-etal-2023-twist} & Audio-token-based $\rightarrow$ Semantic Token & 55.4 \\
    &&& Moshi \cite{defossez-etal-2024-moshi} & Audio-token-based $\rightarrow$ Semantic and Acoustic Tokens & \textbf{62.7} \\
    
    \bottomrule
\end{tabular}
}
\caption{Quantitative comparison based on applications. The \textbf{Bold} denotes the best result.}
\label{tab:benchmarks}
\end{table*}


\subsection{Experiment Settings}

\noindent\textbf{Dataset.} We conduct the comparison and ablation study on the 3D-Front dataset~\cite{fu20213d}. That is, we train the deep-learning-based approaches on this dataset and constrain the LLM-assisted methods to synthesize scenes with object categories within this dataset, for a fair comparison. Following LayoutGPT~\cite{Feng2023LayoutGPTCV}, we take the room category and floor, i.e. its width and height, as input conditions and filter out the scenes with irregular floors. The sizes of the training sets are 3397 and 690 for bedrooms and living rooms, while the corresponding test sets are 60 and 53.

\noindent\textbf{Metrics.} We evaluate the generated scenes from two perspectives. One is the physical feasibility of the scenes, estimated by the overlap between oriented bounding boxes and out-of-boundary metrics, i.e. overlap and OOB. The other is the reasonable organization of scenes, for which we select some common object pairs, i.e. bed-nightstand, table-chair, table-sofa, and measure the averaged KL-divergence between the relative placement distributions of the ground-truth scenes in the test sets and the generated scenes.

\noindent\textbf{Implementation.} We use GPT-4\cite{achiam2023gpt} for all the LLM-assisted approaches for the evaluation (the open-source LLaMA also works well with our approach). We train the hierarchy-aware neural network with 500 epochs using the Adam optimizer, where the batch size is 4 and the learning rate is 1e-4. The network is trained on the combination of the bedroom and living room training sets, which takes about 8 hours on a Nvidia 4090 GPU. Our divide-and-conquer optimization is implemented with the GUROBI solver~\cite{gurobi}. Our approach takes about 2 minutes to synthesize a scene with 8 objects.  

%我们在一个具有40GB内存的NVIDIA A100 GPU上进行CommonScenes的训练、评估和可视化。我们采用初始学习率为1e-4的AdamW优化器对网络进行端到端训练。我们在所有的实验中设置{λ1, λ2, λ3} ={1.0, 1.0, 1.0}。分布Z中的Nc设为128,TSDF大小D设为64。
%We conduct the training, evaluation, and visualization of CommonScenes on a single NVIDIA A100 GPU with 40GB memory. We adopt the AdamW optimizer with an initial learning rate of 1e-4 to train the network in an end-to-end manner. We set {λ1, λ2, λ3} = {1.0, 1.0, 1.0} in all our experiments. Nc in distribution Z is set to 128 and TSDF size D is set as 64. We provide more details in the Supplementary Material.

% Only the hierarchy aware graph neural network needs to be trained in our approach. We train the network with 500 iterations using the Adam optimizer, where the batch size is 4 and the learning rate is 1e-4. The network is trained on the combination of the bedroom and living room training sets, which takes about 8 hours on a Nvidia 4090 GPU. Our divide-and-conquer optimization is implemented with the GUROBI solver~\cite{gurobi}, which takes about 2 minutes to synthesize a scene with 8 objects.


\subsection{Comparisons}

We compare with two types of state-of-the-art approaches, including those training deep neural networks from scratch, i.e. ATISS~\cite{Paschalidou2021ATISSAT} and DiffuScene~\cite{Tang2023DiffuSceneSG}, and LLM-assisted indoor scene synthesis, i.e. LayoutGPT~\cite{Feng2023LayoutGPTCV} and HOLODECK~\cite{Yang2023HolodeckLG}. We re-train the deep networks using their released code on the same train/test split. For the LLM-assisted approaches, we use their implementations of the pipelines and invoke the same version of GPT for the inference.

\noindent\textbf{Qualitative Evaluation.} Figure~\ref{fig:comparison_topview} presents generated scenes of different methods. The deep learning methods ATISS and DiffuScene generate results with reasonable placements of objects. But the networks are not guaranteed to ensure the physical feasibility of the scene layouts and sometimes cause object overlap and out of the floor boundary. LayoutGPT, which uses in-context learning to infer numerical layouts based on the demonstrated examples, generates many incorrect orientations and positions. HOLODECK produces relatively better results in terms of physical feasibility, but some objects are not placed in the optimal position as specified by the LLM. By contrast, our approach is able to produce more reasonable and feasible scene layouts.

\noindent\textbf{Quanlitative Evaluation.} Table~\ref{tab:comparison} validates the observations from the visual results. Among all the methods, we achieve the best in terms of both the physical feasibility (overlap and OOB) and the reasonable relative placements (KL Div.). It is interesting to see that the data-driven approaches are good at objects' relative positions and LLM-assisted optimization, i.e. HOLODECK, obtains more feasible results, while ours takes the merit of both and won the best on all the metrics.

\begin{table}
\centering
\setlength{\tabcolsep}{8.5pt}
% \vspace{-0.2cm}
\begin{tabular}{ccccc}
\hline
\multirow{2}{*}{Models}    & \multicolumn{2}{c}{Bedrooms}  &      \multicolumn{2}{c}{Living Rooms}  \\
\cline{2-5}
         & $\#$Rel. & $\#$Obj. & $\#$Rel. & $\#$Obj.  \\
\hline

HOLODECK & 0.77 & 0.93 &  0.69 & 0.97 \\
\hline

Our & \textbf{0.88} & \textbf{0.99} & \textbf{0.86} &  \textbf{1.00} \\
\hline
\end{tabular}
\caption{Semantic alignment between LLM-generated descriptions and the resulting scenes of HOLODECK and ours.}
\label{tab:comparison_holo}
% \vspace{-13pt}
\end{table}

We further provide an additional quantitative comparison with HOLODECK, the closest work to ours, as both use the LLM to generate textual scene descriptions and then solve for the scene layouts. The difference is that HOLODECK requires dense and detailed spatial relations while ours uses hierarchical structures with sparse relations as well as a neural network to infer the fine-grained relative placements. Table~\ref{tab:comparison_holo} reports the semantic alignment between the LLM-generated descriptions and the generated scenes. Specifically, $\#$Rel. counts the percentage of relative placements matching with LLM-generated spatial relations and $\#$Obj. counts the existence of LLM-specified objects. Obviously, our results align better with LLM arrangements, implying the advantages of using hierarchical scene representation with our approach.

\begin{table}[]
\centering
\setlength{\tabcolsep}{7pt}
% \vspace{-0.2cm}
\begin{tabular}{ccccc}
\hline
Models     &  Scene Effect. & Physical. & Layout. \\
\hline
ATISS      & 3.55           & 2.86              & 3.10       \\
\hline
DiffuScene & 3.38           & 2.55              & 2.57      \\
\hline
LayoutGPT  & 3.04          & 2.45           & 2.23       \\
\hline
Holodeck   & 3.12          & 3.75             & 3.23       \\
\hline
Ours       & \textbf{4.73}          & \textbf{4.69}            & \textbf{4.55}  \\
\hline
\end{tabular}
\caption{Averaged score of perceptual study on the generated scenes. The participants give scores between 1-5 w.r.t. scene effectiveness, physical feasibility and layout rationality.}
\label{tab:perceptual_study}
% \vspace{-0.2cm}
\end{table}

\noindent\textbf{Perceptual Study.} The perceptual study evaluates the quality of scenes generated by different methods (those used in the comparison experiments). We present 25 generated scenes with the input user requirements to 30 participants, who rate them on a 5-point Likert scale from three aspects: scene effectiveness, physical feasibility, and layout rationality. Since both the rendered scenes and the topview visualizations are presented, the participants are sensitive to unreasonable placements which might be covered by the occlusion in the renderings. Table~\ref{tab:perceptual_study} shows that our results get the highest scores w.r.t. all three aspects, i.e. high scene effectiveness as we prompt the LLM with functional area considerations, high feasibility and layout rationality as our approach can effectively generate feasible scenes aligned with LLM arrangement. Otherwise, although HOLODECK prevents object overlap or out-of-boundary, since its results may break the LLM arrangements, it may affect possible human activity, such as shown in the top row case in Figure~\ref{fig:comparison_topview}.

\begin{figure}
\centering
\includegraphics[width=1\linewidth]{CameraReady/Figures/ablation_color.pdf}
\caption{Topview visualizations of the ablation study.}
\label{fig:ablation_topview}
\end{figure}
% Please add the following required packages to your document preamble:
% \usepackage{multirow}
% \begin{table*}[]
% \begin{tabular}{cccccccc}
% \hline
% \multirow{2}{*}{}
% & \multirow{2}{*}{NObj} 
% & \multirow{2}{*}{OBB over} 
% & \multirow{2}{*}{Out-of-Boundary} 
% & \multicolumn{2}{c}{Rel. Pos. Distribution}  \\
% & 
% &
% &                                 
% & desk-chair        
% & bed-nightstand        
% &                         
% \\ \hline
% Baseline
% &                       
% &                                   
% &                             
% &                                 
% &                    
% &                       
% &                         
% \\ \hline
% Hybrid Pos.                                                                         
% &                        
% &     
% &                              
% &                                  
% &                   
% &                       
% &                         
% \\ \hline
% Relative Pos. + Optim.                                                              &                        &                                    &                              &                                  &                    &                       &                         \\ \hline
% \begin{tabular}[c]{@{}c@{}}Relative Pos. + Retrieval + Optim.\\ (Full)\end{tabular} &                        &                                    &                              &                                  &                    &                       &                \\ \hline        
% \end{tabular}
% \caption{Ablation table.}
% \end{table*}



% Please add the following required packages to your document preamble:
% \usepackage{multirow}
\begin{table*}
\setlength{\tabcolsep}{10pt}
\centering

% \vspace{-0.2cm}
\begin{tabular}{cc|ccc|ccc}
\hline
\multicolumn{2}{c}{Baselines} & \multicolumn{3}{c}{Bedrooms} & \multicolumn{3}{c}{Living Rooms} \\

\hline
Hierarchy-Aware Net. & D$\&$C Optim. & {OOB↓} & {Overlap↓} & {KL Div.↓}  & {OOB↓} & {Overlap↓} & {KL Div.↓} \\
\hline

$\times$ & $\times$ &  0.80 & 0.09 & 0.23 & 0.98 & 0.10 & 0.24  \\
\hline

\checkmark & $\times$ &  0.73&  0.18&  0.09&  0.93&  0.21& 0.16  \\\hline

$\times$ & \checkmark & 0.00 & 0.00 & 0.23 & 0.00 & 0.00 & 0.23  \\\hline

\checkmark & \checkmark &  \textbf{0.00}& \textbf{0.00}& \textbf{0.09}&  \textbf{0.00}&  \textbf{0.00}& \textbf{0.13} \\ 
\hline

\end{tabular}
\caption{Quantitative evaluation of ablation study to validate our key designs: the hierarchy-aware network (Hierarchy-Aware Net.) to infer fine-grained relative placements and divide-and-conquer optimization (D$\&$C optim.) to solve the final layouts.}
\label{tab:ablation_study}
% \vspace{-0.2cm}
\end{table*}


\subsection{Ablation Study}

The ablation study validates the key designs of our approach, i.e. the hierarchy-aware network and the divide-and-conquer optimization, by removing the corresponding stage from our pipeline. When removing the hierarchy-aware network, we pre-define the relative placement coordinates for the textual spatial relations to replace the predictions of the network. When removing the divide-and-conquer optimization, we directly require the LLM to generate coordinates of anchor objects and transform the relative placements of the other objects into global coordinates without optimization refinement. The results are shown in Table~\ref{tab:ablation_study} and Figure~\ref{fig:ablation_topview}. It indicates that our hierarchy-aware network is of vital importance in capturing the reasonable relative placements between objects, and the divide-and-conquer optimization ensures the physical feasibility of the generated scene layouts.

\begin{figure}
\centering
\includegraphics[width=\linewidth]{CameraReady/Figures/scene_diversity_image.pdf}
\caption{Diverse results in the open-vocabulary task.}
\label{scene_diversity}
\end{figure}


\section{Applications}

\noindent\textbf{Open-vocabulary scene synthesis.} To demonstrate the generalization of our approach, we show the open-vocabulary scene synthesis results in Figure~\ref{scene_diversity}. By removing object category constraints in the LLM prompt, we generate scenes of a dining room, museum, and arcade room, not trained on by our model, to validate its practical usage. The results show that the LLM generates reasonable hierarchical structures for arbitrary requirements, while our generalizable hierarchy-aware network and divide-and-conquer optimization produce realistic scenes with various descriptions.

\begin{figure}[t]
\centering
\includegraphics[width=\linewidth]{CameraReady/Figures/scene_edit_image.pdf}
\caption{Language-guided interactive scene editing.}
\label{scene_editing}
\end{figure}

\noindent\textbf{Interactive scene editing.} Our approach also supports user-friendly language-guided interactive scene editing. Specifically, we describe the current state of the scene and an editing instruction as the input to LLM, and add an additional constraint to the divide-and-conquer optimization to maintain the placements of unchanged objects as much as possible. As shown in Figure~\ref{scene_editing}, the LLM can modify the scene by adding and removing objects. Moreover, with the hierarchical scene structure and our approach, the edited scenes exhibit minimal changes from the original scenes while satisfying the LLM arrangments and the expectations of users.
%%%%%%%%%%%%%%%%%%%%%%%%%%%%%%%%%%%%%%%%%%%%%%%%%%%%%%%%%%%%%%%%%%%%%%%%%%%%%%%%
\section{CONCLUSION AND FUTURE WORK}
SoRos have shown immense potential with inherent conformability and adaptability to a multitude of surfaces, yet they previously lacked adequate grip stability to overcome non-uniform surfaces present outside of lab environments. Compliant microspines are one missing piece towards shrinking this real-life realization gap. We propose an elegant, compliant microspine design with a standardized soft-compliant integration technique. The stacked array configuration enables the SoRo to maintain surface interaction when extreme surface discrepancies are present. We provide results from a set of field experiments reflecting the improved performance of two microspine array configurations over a baseline SoRo on four different, ruggedized surfaces. %\hl{Our results indicate that the 1ML design is superior to 0ML with significantly increased planar movement on all tested surfaces, and the 2ML design results in improved performance on three of the surfaces with an emphasis on increased repeatability across trials.} 
Our results indicate that microspines are a vital technology for increasing terrain traversability in mobile SoRos. Future work includes optimizing microspine array configurations for different surfaces, performing additional field experiments, and exploring the generalizability of the design to different prototypes.





% %1. Designed microspine array grippers for soft robot.
% % 2. The modular design allowed for exploration of three spine configurations - xx,xx,xx
% % 3. The three test surfaces were yy, yy, yy
% % 4. The experimental setup allowed for real-time tracking of the robot pose. 
% % 180 experiments were conducted to investigate repeatability of locomotion.
% % A translation gait was used for these experiments which is an optimal translation gait for SoRo locomotion on rubber mat with no spines.
% % All three surfaces, there is increased surface interaction for both the spine configurations when compared with no spine configuration. 
% % This is visible in the increased coupled rotation and translation of the robot during the gaits.
% Modular soft microspine endcap grippers are designed for a three-limb SoRo to increase surface interaction regardless of surface topography. The modular design allows for different configurations to be explored, namely no spines, inward facing spines, and directional spines. Experiments are performed on three variable, modular surfaces: porous wood, rubber mat, and smooth whiteboard. Real-time tracking of the robot pose is used to record relative and cumulative displacement and rotation over a $49.5\sec$ trial. An optimal translation gait identified for the three-limb SoRo on a rubber mat without spines is used for all experiments. 20 experiments are performed for each surface and spine configuration combination for a total of 180 experiments to investigate locomotion repeatability. Increased surface interaction on every surface for inward facing and directional spines is observed when compared to no spines. Decreased variance in translation is seen in the smooth and porous surfaces, showing an increase in repeatability across trials when microspines are present.
% Additionally, the rotation and translation seen during the gaits are tightly coupled as there is never a significant increase in translation without a similar increase in rotation. In short, we find that microspine grippers provide a significant increase in the engagement between the robot and environment. %This is seen even more in preliminary testing of the SoRo on inclines as the spherically reconfigurable design orients itself towards the path of least resistance, introducing greater rotation on uneven surfaces.
% %During preliminary testing of the three-limbed SoRo on inclines, increased rotation was observed. The SoRo is spherically reconfigurable and wants to orient itself towards the path of least resistance which is down a given incline. 

% % Despite not being an optimal gait for all the other surfaces and spine configurations, an increased repeatability (decreased variance) for translation is observed.


% %Microspine grippers are small spines commonly found on insect legs that reinforce surface interaction by engaging with asperities to increase shear force and traction. An array of such microspines, when integrated into a robot can provide them with the ability to maneuver uneven terrains, inclines and even climb walls. The surface conformability and adaptability of soft robots (SoRos) makes them an ideal candidates for traversing complex terrains. %
% % Taking inspiration from nature, an array of these microspines can be attached to end effectors, enabling robots to traverse uneven terrain on inclines and climb vertical walls.
% %One way of Despite recent advancements, there remains a real-life realization gap for soft locomotors pertaining to their transition from controlled lab environment to the field.
% % in the field of soft robotics, the jump from controlled lab environments to real-world, unstructured test sites remains a core challenge. 

% % Integration of microspines into SoRos has the potential to shrink this gap by improving grip stability for traversability. %
% % % Integrating microspines into SoRos increases stability and traversal capabilities, effectively shrinking the reality gap.
% % In this paper, we propose a passive modular microspine array endcap for motor-tendon actuated (MTA) soft robots. Here, the direction of the microspine array in the endcap can be varied. 
% % % We use a spherically reconfigurable MTA three-limb SoRo design from a previous work as our experimental prototype. We propose a modular endcap design for MTA SoRos outfitted with different passive microspine configurations. 
% % Consequently, modularity enables exploration of nonidentical array configuration schemes to account for different surfaces. %
% % These microspine arrays are combined with a three-limb MTA SoRo for realizing  planar locomotion with real-time tracking. Experiments are conducted on smooth, rough, and porous surfaces with different array configurations. The directional configuration, where microspines are oriented in the robot forward direction, differs from the inward configuration where they are aligned in the same direction w.r.t. the individual limb.
% % %Planar locomotion tests are performed on flat ground with real-time tracking. We demonstrate experiments on smooth, rough, and porous surfaces to evaluate the impact of microspine placement on a variety of surfaces. 
% % Experiments are conducted using a push-pull translation gait for 180 trials where 20 trials were conducted for a particular configuration and surface. Each trial comprised of 45 gaits to examine repeatability. Results indicate that spine array integration increases displacement on all three surfaces and, on average, the directional configuration robot moves twice as much when compared with no spines. Additionally, for the given gait, they improve locomotion repeatability and reliability. The experiments show the microspine gripper consistently increases engagement of the robot with the surfaces.

% Future work will include experiments on more complex surfaces including inclines and dirt/grass (real-world). Controlled locomotion gait sequences will be explored to account for the changing inclines while minimizing rotation, but some amount of rotation will likely always be present. Increased surface area interaction is vital for the microspines to be effective. This can be achieved by attaching a larger array of spines to account for missed interactions. Consequently, we plan to investigate utilizing active microspine arrays. This will have the added capability of reconfiguring the SoRo to deploy microspines dependent on the surface topography as well as including microspines in only portions of gait sequences.
% %Different microspine materials will be tested as deterioration can occur over time on surfaces such as carpet where a large interaction takes place. 
% %This is clear in the literature where robots can have up to 250 spines/limb forcing a meaningful interaction occurs every time. 
% %Since the current robot is spherically reconfigurable, which innately introduces rotation into any gait and the system as a whole. This is a much larger problem on inclined surfaces as the robot wants to orient itself towards the path of least resistance, ultimately moving down the incline. Additionally, it is clear from the testing that the robot is not a perfectly stable system. Despite aligning the robot in the same orientations for each test case, the variance in the results is very noticeable. Simply stating that a symmetric robot would fix this problem is not the case. The researchers believe that stability is the more important factor than symmetry, hence designing and manufacturing a robot that is agnostic towards inclined surfaces through stability is a must for future works.
% %All mentioned approaches above will increase the overall performance, but without a holistic approach, the robot will not be able to be taken into the real world with meaningful testing cases. 
% % The addition of passive microspine endcaps increased the distance and speed traveled as well as the consistency of locomotion. The future improvements mentioned above will increase the overall performance, but a holistic approach must be taken as all parameters are linked. As progress continues, more surfaces and real world tests will be able to be conducted and applications can be built from this soft robotic platform.

\section*{Acknowledgments}
% This work is supported in part by the National Natural Science Foundation of China (Grant no. 62325211, 62132021), the Excellent Young Scientists Fund Program (Overseas) of Shandong Province (Grant no.2023HWYQ034).

This work is supported in part by the Shandong Province Excellent Young Scientists Fund Program (Overseas) (Grant No. 2022HWYQ-048 and 2023HWYQ-034), the TaiShan Scholars Program (Grant No. tsqn202211289), the National Natural Science Foundation of China (Grant No. 62325211 and 62132021).

\bibliography{references}

\end{document}
