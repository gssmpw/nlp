\section{Related Work \label{related}
}

Phishing detection systems generally employ a number of common techniques in combination to enable reliable protection against the attacks; this includes URL scanning, email content scanning, network metadata analysis (DNS blacklisting), rule-based methods and more recently AI-based phishing detection tools which use large amounts of data to train machine learning (ML) models to extract features from email contents, web-site characteristics, user interactions and other associated metadata ____. Broadly categorizing these systems, there are primarily two types - traditional and non-traditional methods. The traditional methods primarily incorporate legal protection, user education and awareness, blacklist/whitelist, visual similarity, and search engines. The non-traditional methods fundamentally include content-based, heuristics-based, AI, fuzzy logic, rule-based and data mining methods. The proposed work falls under the latter category of AI-enabled phishing detection systems.

In the latest research involving the aforementioned category, there has been greater focus on supervised learning based machine learning techniques. A phishing URL classification using recurrent neural networks is proposed by ____, where an LSTM model is trained on 2 million URLs containing legitimate and phishing URLs and each character of a URL input is translated to 128-dimensional embedding for training the model. This model trained on URLs was bench-marked against a random forest model trained on lexical and statistical features associated with URLs and although the Random Forest model yielded an overall accuracy of 94\%, the LSTM outperformed it by using just one feature, the URL string, yielding 99\% overall accuracy. This highlights the importance of analysing URL text in phishing detection.

The Deep Neural Network-based approach taken by ____ is another method where a large dataset consisting of malicious and benign URLs was used; in this case, to train 3 different types of Convolutional Neural Networks (CNNs). The first CNN model utilises character-level URL embeddings, the second model uses word-level embedding of the same URL and the third (and most effective) model used a hybrid approach where the embeddings used for training the CNN model uses word and character-level embeddings together.

The supervised learning approach has proven quite effective for detecting traditional phishing URLs, however, supervised machine learning are not generalizable in the face of the ever-evolving phishing URLs especially with the need for large amounts of labelled data and feature engineering processes, which are prone to human error ____. Additionally, classical machine learning techniques still suffer from a lack of efficiency in detecting zero-day phishing attacks ____. Consequently, supervised machine learning methods fail to detect unknown or newly evolved phishing attacks. Deep learning algorithms such as CNNs can detect zero-day attacks more efficiently ____. However, although these approaches are still dependent on large amounts of historical labelled data for training, and are thus not as effective against the evolving techniques of attack and not resilient to generative AI-based techniques. Prior to ____, most of the phishing URL detection research was focused on supervised classification techniques, and there is very limited work done on adapting clustering for efficient phishing URL detection.

The work published in ____ emphasises on the importance of using unsupervised methods such as clustering, in contrast to previously cited works which employ supervised algorithms requiring computation resources and large amounts of data to train models. It explores clustering methods such as K-Means and DBSCAN for detecting phishing URLs. The workflow starts with the browser analytics segment which monitors the user’s activity and any domain searched is matched with white and blacklists. If there is a match with blacklist the domain is blocked and vice-versa. If there are no matches, then certain features from the website are derived, such as website protocol, IP address subdomains, the amount of web traffic and Google Page-rank index etc. Based on the extracted features clustering is carried out using K-Means clustering algorithm to output two clusters (benign and malicious). Unlike the above supervised methods, here the analysis for classifying a URL is done by analysing the external features in addition to the URL itself. A shortcoming of these particular unsupervised approaches is their high computation requirements as data volumes increase and thus lack of scalability in addition to lack of accuracy with high-dimensional data such as text ____.

There is a need for further research into evolving web-based phishing detection approaches; more specifically into the unsupervised clustering techniques that hold the potential for detection of evolved threats (without the need for historical training data). The proposed method in this paper aims to fill that gap and offers a solution using unsupervised techniques, which not only addresses the aforementioned shortcomings but also enables scalability and resilience, traits which are unique to this method. Applications of our approach include but are not limited to detection of entire phishing campaigns using a scalable and efficient pipeline, enabling detection of zero-day attacks with detection of newly registered phishing domains (which do not offer additional features such as traffic data) and resilience against AI-generated phishing campaigns; all-the-while ensuring privacy of users’ content.