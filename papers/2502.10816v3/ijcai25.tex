%%%% ijcai25.tex

\typeout{BalanceBenchmark: A Survey for Imbalanced Learning}

% These are the instructions for authors for IJCAI-25.

\documentclass{article}
\pdfpagewidth=8.5in
\pdfpageheight=11in

% The file ijcai25.sty is a copy from ijcai22.sty
% The file ijcai22.sty is NOT the same as previous years'
\usepackage{ijcai25}
\usepackage{amsthm,amsmath,amssymb}
\usepackage{mathrsfs}
% Use the postscript times font!
\usepackage{times}
\usepackage{soul}
\usepackage{url}
\usepackage[hidelinks]{hyperref}
\usepackage[utf8]{inputenc}
\usepackage[small]{caption}
\usepackage{graphicx}
\usepackage{amsmath}
\usepackage{amsthm}
\usepackage{booktabs}
\usepackage{algorithm}
\usepackage{algorithmic}
\usepackage{multirow}
\usepackage[switch]{lineno}
\usepackage{xcolor} % 用于颜色
\usepackage{amssymb}
\usepackage{subcaption}
\usepackage{enumitem}
\usepackage{booktabs}
\newcommand{\noteBy}[2]{\textcolor{blue}{[#1: #2]}}
% Comment out this line in the camera-ready submission
% \linenumbers

\urlstyle{same}
\hyphenation{different}
\newcommand\blfootnote[1]{%
\begingroup
\renewcommand\thefootnote{}\footnote{#1}%
\addtocounter{footnote}{-1}%
\endgroup
}
\newtheorem{example}{Example}
\newtheorem{theorem}{Theorem}
\pdfinfo{
/TemplateVersion (IJCAI.2025.0)
}

\title{BalanceBenchmark: A Survey for Multimodal Imbalance Learning}

\author{
Shaoxuan Xu$^{1,\dagger}$
\and
Menglu Cui$^{2,\dagger}$\and
Chengxiang Huang$^{3}$ \and
Hongfa Wang$^{4,5}$ \And
Di Hu$^{1,*}$
\affiliations
$^1$Gaoling School of Artificial Intelligence, Renmin University of China, Beijing, China\\
$^2$Shanghai University of Finance and Economics, Shanghai, China\\
$^3$Beijing University of Posts and Telecommunications, Beijing, China\\
$^4$Tencent Data Platform, Shenzhen, China\\
$^5$Tsinghua Shenzhen International Graduate School, Shenzhen, China\\
\emails
\{xushaoxuan20040225, dihu\}@ruc.edu.cn,
Louise158@stu.sufe.edu.cn,
huangchengxiang2021@bupt.edu.cn,
hongfawang@tencent.com
}

\begin{document}

\maketitle

\begin{abstract}
Multimodal learning has gained attention for its capacity to integrate information from different modalities. However, it is often hindered by the \textit{multimodal imbalance problem}, where certain modality dominates while others remain underutilized. Although recent studies have proposed various methods to alleviate this problem, they lack comprehensive and fair comparisons. In this paper, we systematically categorize various mainstream multimodal imbalance algorithms into four groups based on the strategies they employ to mitigate imbalance. To facilitate a comprehensive evaluation of these methods, we introduce \textbf{BalanceBenchmark}, a benchmark including multiple widely used multidimensional datasets and evaluation metrics from three perspectives: performance, imbalance degree, and complexity. To ensure fair comparisons, we have developed a modular and extensible toolkit that standardizes the experimental workflow across different methods. Based on the experiments using  BalanceBenchmark, we have identified several key insights into the characteristics and advantages of different method groups in terms of performance, balance degree and computational complexity. We expect such analysis could inspire more efficient approaches to address the imbalance problem in the future, as well as foundation models. The code of the toolkit is available at \url{https://github.com/GeWu-Lab/BalanceBenchmark}.
\blfootnote{\noindent
\textsuperscript{$\dagger$}Equal contribution. 
\textsuperscript{*}Corresponding author.
}
\end{abstract}
\documentclass[../main.tex]{subfiles}
\graphicspath{{../images/}}
\makeatletter
\def\input@path{{../images/}}
\makeatother
\begin{document}
\section{Introduction}
\begin{figure}
\centering
\begin{tikzpicture}
\node[inner sep=0pt] (ws) at (0, 0) {
\includegraphics[height=.4\textwidth, trim={10cm 0 10cm 0},clip]{world_space.png}};
\node[inner sep=0pt] (cs) at (6,0) {\includegraphics[height=.4\textwidth, trim={10cm 1cm 10cm 4cm},clip]{conf_space.png}};
\end{tikzpicture}
\vspace{-5pt}
\label{fig:pbrm_intro}
\caption{\textbf{Left}: Shows world space obstacles as grey spheres. Robots start and goal configuration is colored red and green, respectively. Configurations along the computed path are colored transparent blue. \textbf{Right:} Mapped world space scenario to configuration space. Obstacle region is the grey mesh. Red spheres are collision-free regions computed by the neural SCDF. The optimized shortest path in the convex corridor is the blue curve.}
\vspace{-25pt}
\end{figure}
Motion planning is the problem of finding a collision-free trajectory that connects a given start and goal configuration. The planning takes place in the configuration space of the robot. For single body robots, like mobile robots or drones, the configuration space and the world space are usually the same. This simplifies the planning, since explicit obstacle representations are available which enables geometrical tools like separating hyperplanes, smallest distance to obstacles etc., to be used when designing motion planning algorithms. For multi-body robots like manipulators, the situation is completely different. The world space obstacles are usually mapped to non-convex regions, and to make the problem even harder, the mapping is usually not known. Forming explicit representations of the obstacle region in the configuration space is usually too expensive or intractable. Despite all of this, sampling based planners are used with great success, which mainly is due to their use of implicit representations of the obstacle region. The basic idea is to construct a graph in the configuration space that covers and connects the collision-free region. From this graph, a path can be extracted that connects a given start and goal configuration. The approach is computationally expensive, since the graph is constructed with the smallest geometrical building block available, points, which represents a collision-check. Furthermore, the extracted paths from the graph are non-smooth and jagged due to the stochastic nature of the approach. This adds an additional post-processing step to the process, where the paths are shortcutted and smoothened, before the path can be used for tracking. Clearly a lot of time is invested to form this graph and produce smooth paths. Thus, if the obstacles start to move, then all of this work is done in no use, since all points that make up this graph need to be re-verified, which is simply too time consuming to be done in real time.
\\\\
In this work, we want to address the existing drawbacks of the sampling based planners. Our main contribution is an improved motion planner where each vertex in the graph covers a collision-free region in the form of a sphere instead of a point and where the edges are formed with neighboring intersecting spheres. This representation has the advantage of instead of returning piecewise linear paths, returning a sequence of overlapping spheres, i.e. a convex corridor, that connects a given start and goal configuration, illustrated in Figure \ref{fig:pbrm_intro}. This convex corridor allows us to use convex optimization to produce smooth trajectories, instead of computationally expensive post-processing methods. The representation further allows us to estimate the coverage of the collision-free space, which gives us awareness and feedback in the offline roadmap construction phase. Finally, our representation is simple to adapt to moving obstacles, simply requery for the new radii and recheck for intersections. 
\\\\
The spherical collision-free regions are formed using a signed distance function (SDF), which is a function that returns the smallest distance from an arbitrary point to the boundary of an obstacle. As the name implies, the distance is signed, thus if the point is inside the obstacle it is negative otherwise positive. If the distance is positive, a sphere with radius equal to the distance is guaranteed to cover a collision-free region. Using an SDF in motion planning is not new, but what is novel about our approach is that we express the distance in the configuration space instead of the world space and by doing so allows us to form these convex collision-free regions. We refer to the resulting SDF as a signed configuration distance function (SCDF). Computing an SCDF analytically is non-trivial, our approach is therefore to parameterize the SCDF with a deep neural network and learn the mapping by supervised learning. Our resulting neural SCDF can compute distances for different parameter values of obstacle shapes and we also show how multiple distances can be combined, thus making our approach flexible.
\section{Related work}
Motion planning algorithms can roughly be divided into three families, grid-based, sampling based and optimization based methods. Grid-based methods (GBM) discretize the planning space from which a graph is then compiled. A standard search method is A$^\star$ \citep{a_star}, which is classified as an \textit{informed} search method, since it employs a heuristic function to speed up the search. A$^\star$ guarantees to return an optimal path at the level of discretization used. GBMs usually discretize the planning space by a regular lattice and this limits the GBMs to problems with low dimensionality due to the curse of dimensionality. Thus, GBMs are usually limited to single-body robots where the degrees of freedom (DOF) are low. To overcome the inherent scaling problem with the GBMs, stochastic methods are usually used for multi-body robots. These methods are termed as sampling-based methods (SBM) and core members within this family are the rapidly-exploring random trees (RRT) \citep{rrt} and the probabilistic roadmap (PRM) \citep{prm}. RRT grows a tree from the start configuration and explores the collision-free region in a rapid way until it is able to connect to the goal region. RRT is usually improved by bi-directional planning \citep{rrt_connect}, i.e. an additional tree is grown from the goal configuration and the trees are tested for connection after any tree has been expanded. RRT is a single-query method, thus it searches for a path from scratch each time it is queried. Contrary to this, PRM is a multi-query method, which solves for multiple queries without starting from scratch. PRM does this by creating a roadmap (graph) that covers the collision-free space as an offline step. The graph is then used to solve for multiple queries. PRMs are used in cases where the environment does not change since the extra offline step is too computationally costly and needs to be re-done if the environment is changed. In our work, we address this inherent issue by using a different roadmap representation. Our vertices in the graph cover a collision-free region in the form of spheres and we form the edges by checking for intersecting spheres. If something in the environment changes, we recompute the spheres radii and recheck the intersections, without relying on collision detection. We use a trained neural network to compute the sphere radius, therefore querying for the radius can be done fast, hence our representation enables the PRM for dynamic environments.
\\\\
In the recent decades, optimization based methods (OBM) \citep{chomp, schulman, itomp, stomp} have been introduced as an alternative to SBM for multi-body robots. Like the SBM, the OBMs scale well to higher dimensional problems and produce smoother motion. It is common to use a SDF in the optimization since it is a smooth function, thus enabling gradient-based methods. However, the standard way of expressing the SDF is in world space. The distance therefore needs to be mapped to the configuration space by the forward kinematics. This mapping makes the optimization problem a non-linear program (NLP), which is computationally expensive to solve. Recently, a different approach has been proposed. In \cite{mp_gcs} motion planning is formulated as a convex optimization problem by using the graph of convex sets framework \citep{gcs}. The underlying idea is to decompose the collision-free space into intersecting convex sets from which a convex optimization problem is formulated. In cases where an explicit representation of the obstacles in the configuration space exists, like for single-body robots, creating collision-free convex regions can be done fast \citep{iris}. For multi-body robots, this is non-trivial. Existing work does this successfully \citep{iris_nlp, iris_c} by an optimization based approach, but the methods are still too time consuming to be used in the presence of moving obstacles. Our approach is instead to use deep learning to learn an SDF expressed in the configuration space. With this, we can query for shortest distances to the collision boundary, which allows us to expand spherical regions which are collision-free. Our approach is fast and therefore enables our suggested roadmap planner to be used in dynamic environments.
\\\\
Recent research has focused on learning collision detection \citep{fk_kernel_distance, diffco, graphdistnet} by predicting the signed distance between the robot links and the surrounding obstacles in the world space. The learned SDF is used in trajectory optimization but since the distance is expressed in the world space, the problem becomes an NLP and therefore takes a long time to solve. We take a novel approach and suggest to instead express the signed distance in the configuration space. This allows us to improve the PRM at the same time as it enables convex optimization for trajectory optimization, which runs faster and is more reliable than NLP solvers. In \cite{cspf} a learned signed distance function in the configuration space is proposed similar to our approach. However, their approach is restricted to point cloud representations, while we propose to represent the obstacles as parameterized geometric shapes, e.g. spheres. Furthermore, we also show how to use our learned SCDF to improve an existing roadmap planner.
\section{Problem formulation}
A robot is located in the world space, $\W \subset \R^3 $. The unique location of the robot is given by its configuration $\q \in \C$, where $\C$ is the configuration space. The set of points covered by the robots bodies at a certain configuration is expressed as $\B(\q) \subset \W$. The robot is surrounded by $\NrObst$ obstacles $\O = \bigcup_{i=1}^{\NrObst} \O_i$, where  $\O_i \subset \W$. The representation of the obstacle in the configuration space is the set $\C\O_i = \{\q \in \C \: |\: \B(\q) \cap \O_i \neq \emptyset \}$. The obstacle space is formed as $\Co = \bigcup_{i=1}^{\NrObst} \C \O_i$. The complement is referred to as the free space, $\Cf = \C \setminus \Co$. The path planning problem is a tuple, ($\Cf$, $\qStart$, $\qGoal$), where we want to connect a query pair, consisting of a start, $\qStart$, and goal configuration, $\qGoal$, with a geometric path, $\q(s): [0, 1] \mapsto \Cf$, such that $\q(0)=\qStart$ and $\q(1)=\qGoal$, or report correctly when such a path does not exist.
\end{document}

In this section,  we introduce related concepts including GNN for recommendation, negative sampling, and Mixup \cite{mixup}.

\subsection{Graph Neural Networks for Recommendation}
 Recommendation is the most important technology in many e-commerce platforms, which has evolved from collaborative filtering to graph-based models. Graph-based recommendation represents all users and items by embedding and recommending items with maximum similarity score (by a inner product operation) for a given user. Here, we briefly describe the pipeline of GNN-based representation learning, including aggregation and optimization with negative sampling.

GNNs learn distributed vectors of nodes by leveraging node features and the graph structure. 
The neighborhood aggregation follows the ``message passing'' mechanism, which iteratively updates a node's embedding $h$ by aggregating the embeddings of its neighbors. Formally, the embedding $h_i^l$ of node $i$ in the $l$-th layer of GNN is defined as:
% GNNs use graph structures and node features to learn distributed vectors to represent graph information. Learning follows the "message passing" mechanism of neighborhood aggregation by iteratively updating a node's embedding $h$ by aggregating the embeddings of its neighbors. Formally, the representation $h_k^i$ of node $i$ in the $k$th layer of GNN is defined as:
 \begin{equation} \label{agg}
     \small
     h_i^l= \sigma\left(\text{AGG}\left(h_{i}^{l-1}, h_{j}^{l-1} \mid j \in N_{(i)},W_l\right)\right),
 \end{equation}
where the \(\sigma\) is activation function, $W_l$ denotes the trainable weights at layer l, $N_{(i)}$ denotes all nodes adjacent to $i$, $\text{AGG}$ is an aggregation function implemented by specific GNN model (\eg GraphSAGE, GCN, GAT, \etc), and $h_i^0$ is typically initialized as the input node feature $v_i$.

 
\subsection{Negative Sampling}

Negative sampling \cite{negsamp} is firstly proposed to serve as a simplified version of Noise Contrastive Estimation\cite{NCE}, which is an efficient way to compute the partition function of an unnormalized distribution to accelerate the training of Word2Vec\cite{word2vec}. The GNN has different non-Euclidean encoder layers with the following negative sampling objective:

\begin{equation}\label{negsampling}
    \mathcal{L} = \log(\sigma (e_{v_i}^Te_{v_p}))+\sum^{c}_{j=1}\mathbb{E}_{v_j\sim P_n(v)}  \log(1-\sigma (e_{v_i}^Te_{v_j})),
\end{equation}
where $v_i$ is a node in the graph, $v_p$ is sampled from the positive distribution of node $v_i$, $v_j$ is sampled from the negative distribution of node $v_i$, $e$ represents the embedding of the node, $\sigma$ represents the sigmoid function, $c$ represents the number of negative samples for each positive sample pair. 
%So Negative Sampling are free to simplify NCE as long as the vector representations retain their quality, it is an effective method to calculate the partition function of unnormalized distribution.


\subsection{Mixup}


\textbf{Mixup\cite{mixup}} is an simple yet effective data augmentation method that is originally proposed for image classification tasks. 
Mathematically, let $(x, y)$ denotes a sample of training data, where $x$ is the raw input samples and $y$ represents the one-hot label of $x$, the Mixup generates synthetic training samples $(\tilde{x}, \tilde{y})$  as follows:
\begin{equation}
% \vspace{-0.2cm}
\begin{split}
% \setlength\abovedisplayskip{0cm}
& \tilde{x}=\lambda x_{i}+(1-\lambda) x_{j}, \\
% \setlength\belowdisplayskip{0cm}
% \setlength\abovedisplayskip{0cm}
& \tilde{y}=\lambda y_{i}+(1-\lambda) y_{j}. \\
% \setlength\belowdisplayskip{0cm}
\end{split}
% \vspace{-0.2cm}
\end{equation}
It generates new samples by using linear interpolations to mix different images and their labels.
\section{Taxonomy of Approaches for Positive and Negative Pair Curation}
\label{sec: Taxonomy}



In contrastive learning, a commonly used loss function is the InfoNCE loss defined below. It pulls similar (positive) pairs together while pushing dissimilar (negative) pairs apart in the embedding space and computes the similarity between an anchor and its positive counterpart, using a softmax over similarity scores. %The InfoNCE loss is defined as follows:


\begin{math}
\hspace{-1em}
\centering
 \mathcal{L} = 
- \frac{1}{N} \sum_{i=1}^{N} \log 
\frac{
\exp\left(\text{sim}\left(\mathbf{z}_i, \mathbf{z}_i^+\right)/\tau\right)
}{
\exp\left(\text{sim}\left(\mathbf{z}_i, \mathbf{z}_i^+\right)/\tau\right) 
+ \sum_{j=1}^{N} 
\exp\left(\text{sim}\left(\mathbf{z}_i, \mathbf{z}_j\right)/\tau\right)}
\end{math}


where \( \mathbf{z}_i \) is the representation of the anchor sample and \( \mathbf{z}_i^+ \) is the representation of the positive sample obtained through augmenting the same instance or using a criterion to select another instance. \( \mathbf{z}_j \) represents all samples in the batch (including negatives). \( \text{sim}(\cdot, \cdot) \) denotes the similarity function (commonly cosine similarity). \( \tau \) is the temperature scaling parameter and \( N \) is the number of samples in the batch.


\subsection{Positive Pair Creation Taxonomy}
The taxonomy of Positive Pair Creation can be categorized into two main groups: \textbf{single-instance positives} and \textbf{multi-instance positives}, as shown in Fig. 1.

%\subsubsection{Single-instance Positives}

% Single-instance positives can be grouped into a single family of technique, as shown in Figure 1. 
Single-instance positive pair creation generates pairs by applying augmentations (e.g., cropping, color changes, geometric transformations) to a single sample \cite{chen2020simple}.  However, this approach limits diversity, as random augmentations fail to capture viewpoint changes, object deformations, or semantically similar instances within the same class. As a result, the model's generalization depends heavily on the augmentation strategy, which may not fully capture the intrinsic variations needed for learning robust embeddings.


% \subsubsection{Multi-instance Positives}

To overcome the limitations of single-instance pairs, multi-instance positive pair curation creates pairs from different data samples rather than augmented views of the same sample, leading to greater diversity \cite{dwibedi2021little}. As shown in Fig. 1, multi-instance curation techniques include: (1) Embedding-based, which selects semantically similar instances in embedding space; (2) Synthetic, which generates positive pairs using generative models; (3) Supervised, which uses human or oracle-labeled data; (4) Attribute-based, which groups samples based on spatial, temporal, or other object-based attributes.  and (5) Cross-modal, which associates samples across different modalities.  By ensuring higher diversity and semantic alignment, multi-instance positive pairs improve representation learning and align embeddings more effectively with downstream tasks.

%comment from here
% \noindent \textbf{Embedding-based similarity} identifies semantically similar samples in the embedding space to form positive pairs. For example, Nearest-Neighbour Contrastive Learning of visual Representations (NNCLR) \cite{dwibedi2021little} retrieves the nearest neighbor of a sample as its positive pair. Similarly, MSF \cite{koohpayegani2021mean} proposes to use the first and $k$ nearest neighbors as the multiple instance positives. All4One \cite{estepa2023all4one} improves MSF by incorporating a centroid contrastive objective to learn contextual information from multiple neighbors using a transformer network. \textbf{Synthetic data generation} employs generative models such as generative adversarial networks (GAN) \cite{wu2023synthetic} or diffusion models \cite{zeng2024contrastive} to create synthetic data points that are semantically similar but distinct from the original, serving as positive pairs. \cite{wu2023synthetic} is jointly trained with the main model to dynamically customize hard samples based on the training state of the main model. \cite{zeng2024contrastive} replaces the features of the intermediate layers in the diffusion model with the semantic features extracted from an anchor image during a random reversed diffusion process. This results in the generation of images possessing similar semantic content to the anchor image but differing in background and context due to the randomness of features in other layers. \textbf{Supervised pairing} utilizes label information to create positive pairs from samples of the same class, as seen in Supervised Contrastive Learning (SupCon) \cite{khosla2020supervised}. Another method \cite{ghose2023tailoring} proposes to create pairs and train the model in an online manner by using human-guided feedback. \cite{wang2022oracle} incorporates human or oracle feedback for a subset of samples to extend the set of positive instance pairs. \textbf{Attribute-based pairing} leverages specific attributes such as spatial location or temporal proximity to form positive pairs. For instance, geographically aligned images captured at different times can be paired \cite{ayush2021geography}. The attributes used to generate optimal views for contrastive representation learning are task-dependent. \textbf{Cross-modal positives} align samples across different modalities (e.g., images and text, audio, speech) that correspond to the same semantic content \cite{radford2021learning}, \cite{wang2022image}, \cite{baevski2020wav2vec}, \cite{li2020unimo}, \cite{morgado2021audio}.
% %comment till here

\subsection{Negative Pair Creation Taxonomy}

In typical contrastive learning approaches, negative pairs are created from samples not used to create the positive pair without considering their semantic content. However, recent work \cite{huynh2022boosting} suggests that uncurated negatives may lead to false negatives, where semantically similar samples are incorrectly treated as negatives. An effective negative sample selection strategy should balance easy and hard negatives while maintaining representativeness. Based on these principles, negative pair curation can be categorized into three main approaches (Fig. 1): (1) Hard Negative Selection, which prioritizes difficult negatives close to the anchor in embedding space; (2) False Negative Elimination, which removes or reclassifies semantically similar false negatives; and (3) Synthetic Negatives, where generative models create diverse, controlled negative samples. There is a subtle trade-off between (1) and (2). Hard negatives improve discrimination but risk overfitting, while false negative elimination reduces noise but may mistakenly remove challenging yet valid negatives, weakening the representations.

% \noindent \textbf{Hard negative selection} involves identifying samples that are particularly challenging for the model to distinguish. \cite{Hardnegativemixing} extends the MoCo v2 framework \cite{chen2020improved} by adding two sets of hard and harder negatives into the queue. The first set is a convex linear combination of pairs of its hardest existing negatives, whereas the second set is created by mixing the negatives with the query. \cite{unremix} introduces UnReMix, a method designed to enhance contrastive learning by selecting hard negative samples based on three key factors: anchor similarity, model uncertainty, and representativeness, ensuring that negative samples are similar to the anchor point, making them challenging for the model to distinguish. \textbf{Removal of false negatives} addresses negative pairs from the same semantic category. \cite{huynh2022boosting} introduces methods to identify these false negatives and propose two strategies to mitigate their impact: elimination and attraction. False Negative Elimination identifies potential false negatives and excludes them from the negative sample set, preventing the model from learning misleading distinctions. In False Negative Attraction, instead of excluding false negatives, this strategy reclassifies them as positives, encouraging the model to learn representations that acknowledge their semantic similarity. \textbf{Synthetic hard negatives} can be created using various techniques, including generative models, feature space interpolation, or rule-based algorithms that modify existing data. \cite{dong2024synthetic} proposes an approach that involves mixing existing negative samples in the feature space to create more challenging negatives, encouraging the model to learn more discriminative representations. It proposes a novel feature-level sampling method to generate more and harder negative samples by mixing them through linear combination and ensuring their reliability by debiasing.


% Next, we dive into details of the most commonly used techniques for crafting effective positive and negative pairs.







We based our analyses on the labeled data created in previous work~\cite{sanei2023characterizing}. The dataset distinguished 305 usability issues from five popular OSS projects (Jupyter Lab,
Google Colab, CoCalc, VSCode, and Atom) and identified their posters. In this paper, we focus on individuals who have ever posted a usability issue in that dataset. 

\subsection{Discovering the Role of Issue Posters}\label{sec: Discovering_role}

To detect the background of the usability issue posters in the dataset, we checked each user's \textit{Profile page} on GitHub, examining their bios, shared personal websites, LinkedIn pages, and/or shared resumes. If they have not shared these information, we searched for their LinkedIn profiles using their full names to extract details on their backgrounds and expertise. We considered their job titles posted in the information acquired this way and categorized them into (1) UX professionals, (2) managers, (3) data scientists, and (4) developers. UX professionals were defined as those indicating positions such as \textit{UX designer} and \textit{user interface and user experience designer}.

Among the 224 usability issue posters in the dataset, we were able to identify the role of 180 users. Within those 180 users, 121 (67.2\%) were developers, 34 (18.9\%) identified as data scientists, 21 (11.7\%) held managerial positions, and only four (2.2\%) were UX professionals. The UX professionals included one male contributed to \textit{VSCode}, another male contributed to \textit{Atom}, and two involved in \textit{Jupyter Lab} project, one male and one female. Notably, there were no UX professionals involved in \textit{CoCalc} and \textit{Google Colab} projects in our data sample. For easier referencing, in the following we call the UX professionals of VSCode as \textit{VSCode\_pro}, Atom \textit{Atom\_pro}, male of Jupyter Lab as \textit{Jupyter\_pro\_M} and female as \textit{Jupyter\_pro\_F}.

\subsection{Characteristics of Issues Posted by UX Professionals (RQ1)}

Once we identified the roles of the usability issue posters, we extracted all the issues posted by the four UX professionals across the five OSS projects. Next, we analyzed the extracted issues by adopting the following steps. First, following the approach outlined in \cite{sanei2023characterizing}, we labeled each issue with either usability or non-usability; and for each usability issue, we identified the main \textit{usability dimension} touched by the issue using the ten Nielsen heuristics~\cite{nielsen2005ten}. Then, similar to \cite{sanei2021impacts}, we identified the specific \textit{sentiment} and \textit{tone} expressed by the UX professionals when posting the usability issues. In our study, the sentiment captures the valence of the emotion that includes three categories (positive, negative, and neutral), while the tone describes emotion with seven affective factors (excited, frustrated, impolite, polite, sad, satisfied, and sympathetic). Subsequently, we analyzed the \textit{argument structure} of the usability issues to better understand the discursive device that the issue posters adopted to convince other discussion participants. We particularly identified whether a \textit{claim} and a \textit{premise} appeared in a usability issue post, using criteria proposed in prior work~\cite{skitalinskaya_learning_2021, wachsmuth_argumentation_2017, dowden1993logical}. Statements were considered as claims if they explicitly indicate the position or stance of the issue posters to the discussed usability issues; and premise means that a statement contains reasoning, evidence, or examples that support a stance. We compared how the above characteristics (i.e., usability dimensions, sentiments, tones, and argument structures) differed in issues posted by UX professionals and those without UX expertise.

\subsection{UX Professionals' Purpose Following Up on Issues (RQ2)}

% After investigating how UX professionals posted the usability issues, we recognized the importance of understanding their participation afterwards, particularly in following up on the discussion threads of the issues they posted. 
Thus, we first isolated comments made by the UX professionals posted to the usability issues they created within the datasets. Then, we employed an inductive content analysis~\cite{wamboldt1992content, Hsieh2005} and categorized the various purposes behind their contributions in posting each comment. For our analysis, the \textit{purpose} specifies the distinct goal that a particular comment serves within the context of the discussion thread. The purpose of a comment may vary based on its content and the immediate objective of the issue posters to write in the discussion to address one specific comment posted by another contributor. We grouped the identified purposes into themes through an iterative approach conducted by the two authors.

\section{Datasets and benchmark} \label{sec:datasets and benchmarks}
\subsection{Datasets} \label{sec:datasets}
BalanceBenchmark includes 7 datasets to evaluate different multimodal imbalance algorithms. These datasets include different types and numbers of modalities, as well as varying degrees of imbalance. \textbf{KineticsSounds} \cite{kinetics-sounds}, \textbf{CREMA-D} \cite{cremad}, \textbf{BalancedAV} \cite{balance}, and \textbf{VGGSound} \cite{vggsound} are audio-video datasets across various application scenarios. \textbf{UCF-101} \cite{ucf101} is a dataset with two modalities, RGB and optical flow. \textbf{FOOD-101} \cite{food101} is an image-text dataset. And \textbf{CMU-MOSEI} \cite{mosei} is a trimodal dataset (audio, video, text).

\subsection{Benchmark} \label{sec:benchmarks}
BalanceBenchmark is the first comprehensive framework designed to evaluate multimodal imbalance algorithms. It addresses three critical limitations of existing measurement approaches. Firstly, to tackle the absence of standardized metrics for imbalance analysis,  we introduce a systematic evaluation protocol in Section \ref{sec:tool4}, which measures three key dimensions in multimodal learning: performance, imbalance, and complexity. Secondly, to ensure reproducibility and fair comparison of multiple methods, we maintain consistent experimental settings through a modular toolkit with unified data loaders and backbone support. Thirdly, to prevent overfitting to specific scenarios, we incorporate 7 diverse datasets spanning different modality combinations such as audio-video, image-text, RGB-optical flow and trimodal scenarios, with varying degrees of modality imbalance.

\textbf{Implementation details.} 
To ensure a reliable comparison across methods, consistent experimental settings are maintained for each dataset. Most datasets utilize the SGD optimizer with momentum set to 0.9 and weight decay of 1e-4, while VGGSound employs an AdamW optimizer with weight decay of 1e-3. All datasets use the StepLR scheduler with a decay rate of 0.1, where the step size is 30 for most datasets and 10 for VGGSound. The batch size is fixed at 64 for most datasets, except for VGGSound which uses 32. Models on VGGSound are trained for 30 epochs, while models on other datasets are trained for 70 epochs. Learning rates are tailored to each dataset to accommodate varying training dynamics: CREMA-D, FOOD-101, KineticsSounds and VGGSound use 1e-3, BalancedAV uses 5e-3, UCF-101 and CMU-MOSEI use 1e-2. Regarding network architectures, ResNet18 is employed as the backbone for audio-video datasets (i.e., CREMA-D, KineticsSounds, BalancedAV, and VGGSound). FOOD-101 combines a pre-trained Transformer with ResNet18. UCF-101 uses ResNet18, and CMU-MOSEI applies a Transformer architecture across all three modalities. The experiments are conducted on different GPU platforms, ensuring consistency within each dataset: CREMA-D, BalancedAV, CMU-MOSEI and VGGSound are evaluated on NVIDIA GeForce RTX 3090, where VGGSound specifically uses two GPUs. KineticsSounds, FOOD-101, and UCF-101 experiments are performed on an NVIDIA A40.
In this section, we empirically compare the proposed algorithm on both sequence windows and time windows with existing methods.
\paragraph{Datasets} For the sequence-based model, we used two synthetic datasets and two cross-language datasets. The statistics of the datasets are provided in Table \ref{table:statistics}:

\begin{table}[t]
    \centering
    \caption{The statistics of the datasets. The datasets satisfy $1 \leq \|\vx\|\|\vy\| \leq R $.}
    \label{table:statistics}
    \begin{tabular}{|c|c|c|c|c|c|}
    \hline
        Dataset & $n$ & $m_x$ & $m_y$ & $N$ & $R$ \\ \hline
        SYNTHETIC(1) & 100,000 & 1,000 & 2,000 & 50,000 & 65 \\ \hline
        SYNTHETIC(2) & 100,000 & 1,000 & 2,000 & 50,000 & 724 \\ \hline
        APR & 23,235 & 28,017 & 42,833 & 10,000 & 773 \\ \hline
        PAN11 & 88,977 & 5,121 & 9,959 & 10,000 & 5,548 \\ \hline
        EURO & 475,834 & 7,247 & 8,768 & 100,000 & 107,840 \\ \hline
    \end{tabular}
\end{table}

\begin{itemize}
    \item Synthetic: The elements of the two synthetic datasets are initially uniformly sampled from the range (0,1), then multiplied by a coefficient to adjust the maximum column squared norm $R$. The X matrix has 1,000 rows, and the Y matrix has 2,000 rows, each with 100,000 columns. The window size is set to 50,000.
    \item APR: The Amazon Product Reviews (APR) dataset is a publicly available collection containing product reviews and related information from the Amazon website. This dataset consists of millions of sentences in both English and French. We structured it into a review matrix where the X matrix has 28,017 rows, and the Y matrix has 42,833 rows, with both matrices sharing 23,235 columns. The window size is 10,000.
    \item PAN11: PANPC-11 (PAN11) is a dataset designed for text analysis, particularly for tasks such as plagiarism detection, author identification, and near-duplicate detection. The dataset includes texts in English and French. The X and Y matrices contain 5,121 and 9,959 rows, respectively, with both matrices having 88,977 columns. The window size is 10,000.
\end{itemize}
We evaluate the time-based model on another real-world dataset:
\begin{itemize}
    \item EURO: The Europarl (EURO) dataset is a widely used multilingual parallel corpus, comprising the proceedings of the European Parliament. We selected a subset of its English and French text portions. The X and Y matrices contain 7,247 and 8,768 rows, respectively, and both matrices share 475,834 columns. Timestamps are generated using the $Poisson$ $Arrival$ $Process$ with a rate parameter of $\lambda=2$. The window size is set to 100,000, with approximately 30,000 columns of data on average in each window.
\end{itemize}

\paragraph{Setup} For the sequence-based model, we compare the proposed hDS-COD and  aDS-COD with EH-COD~\cite{yao2024approximate} and DI-COD~\cite{yao2024approximate}. We do not consider the Sampling algorithm as a baseline, as its performance is inferior to that of EH-COD and DI-CID, as demonstrated in \cite{yao2024approximate}. %The hDS-COD is adjusted by the parameter $\ell$ and the maximum number of levels $L = \log{R}$, where $R$ is the prior estimate of the maximum squared column norm of the dataset. DI-COD similarly requires a prior estimate of $R$ to limit the maximum number of levels $L = \log{(R/\varepsilon})$. In contrast, aDS-COD and EH-COD do not require an estimate of $R$; their error-space balance is controlled by the parameter $\ell = \frac{1}{\varepsilon}$. 
For the time-based model, we compare the proposed hDS-COD and  aDS-COD with EH-COD and the Sampling algorithm since DI-COD cannot be applied to time-based sliding window model. To achieve the same error bound, the maximum number of levels for hDS-COD is set to $L = \log{(\varepsilon NR)}$, and the initial threshold for aDS-COD is set to $1$.

Our experiments aim to illustrate the trade-offs between space and approximation errors. The x-axis represents two metrics for space: final sketch size and total space cost. The final sketch size refers to the number of columns in the result sketches $\mA$ and $\mB$ generated by the algorithm, representing a compression ratio. The total space cost refers to the maximum space required during the algorithm's execution, measured by the number of columns.We evaluate the approximation performance of all algorithms based on correlation errors $\operatorname{corr-err}(\mathbf{X}_W \mathbf{Y}_W^\top, \mathbf{A} \mathbf{B}^\top)$, which is reflected on the y-axis. Every 1,000 iterations, all algorithms query the window and record the average and maximum errors across all sampled windows.

The experiments for all algorithms were conducted using MATLAB (R2023a), with all algorithms running on a Windows server equipped with 32GB of memory and a single processor of Intel i9-13900K.

\paragraph{Performance} Figure \ref{fig:error vs l} and Figure \ref{fig:error vs space} illustrate the space efficiency comparison of the algorithms on sequence-based datasets. Panels (a-d) show the average errors across all sampled windows, while panels (e-h) display the maximum errors.

Figure \ref{fig:error vs l} evaluates the compression effect of the final sketch. The hDS-COD, aDS-COD, and EH-COD show similar compression performances. But the DS series is more stable, particularly on the synthetic datasets, where they significantly outperform EH-COD and DI-COD. The performance of hDS-COD and aDS-COD is nearly the same, indicating that the adaptive threshold trick in aDS-COD does not have a noticeable negative impact on it, maintaining the same error as hDS-COD.

Figure \ref{fig:error vs space} measures the total space cost of the algorithms. hDS-COD and aDS-COD show a significant advantage over existing methods, as they can achieve the  $\varepsilon$-approximation error with much less space. For the same space cost, the correlation errors of hDS-COD and aDS-COD are much smaller than those of EH-COD and DI-COD. Also, aDS-COD has better space efficiency than hDS-COD because aDS only uses a single-level structure while hDS requires $\log R+1$ levels. We find that hDS-COD requires more space on  SYNTHETIC(2) dataset compared to SYNTHETIC(1) dataset. This phenomenon occurs because SYNTHETIC(2) dataset has a larger $R$, which confirms the dependence on $R$ as stated in Theorem~\ref{thm:hds}. 

Figure \ref{fig:time-based} compares the performance of algorithms on time-based windows. Panels (a) and (b) present the error against the final sketch size, which show that our aDS-COD and hDS-COD algorithms enjoy similar performance as EH-COD and significantly outperform the sampling algorithm. On the other hand, as shown in panels (c) and (d), our methods outperform baselines in terms of total space cost.


\section{Conclusion}
In conclusion, we introduce BalanceBenchmark, a unified benchmark for fair and comprehensive evaluation of multimodal imbalance algorithms. By incorporating a systematic taxonomy, diverse evaluation metrics, a comprehensive dataset collection, and the modular toolkit BalanceMM, our benchmark enables thorough assessment of existing methods and provides a convenient tool for future work. 
\clearpage
\bibliographystyle{named}
\bibliography{ijcai25}

\end{document}

