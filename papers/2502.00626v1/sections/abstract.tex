\begin{abstract}
Cutting  thin-walled deformable structures is common in daily life, but poses significant challenges for simulation due to the introduced spatial discontinuities. Traditional methods rely on mesh-based domain representations, which require frequent remeshing and refinement to accurately capture evolving discontinuities. These challenges are further compounded in reduced-space simulations, where the basis functions are inherently geometry- and mesh-dependent, making it difficult or even impossible for the basis to represent the diverse family of discontinuities introduced by cuts.

Recent advances in representing basis functions with neural fields offer a promising alternative, leveraging their discretization-agnostic nature to represent deformations across varying geometries. However, the inherent continuity of neural fields is an obstruction to generalization, particularly if discontinuities are encoded in neural network weights.

%: when trained on simulation sequences, the basis shows relatively limited generalization to discontinuities not included in the training set. Conversely, when trained in a data-free manner, the basis fails to capture any discontinuities within the shape, restricting its applicability to scenarios involving complex cuts.

We present \emph{Wind Lifter}, a novel neural representation designed to accurately model complex cuts in thin-walled deformable structures. Our approach constructs neural fields that reproduce discontinuities precisely at specified locations, without ``baking in'' the position of the cut line. To achieve this, we augment the input coordinates of the neural field with the generalized winding number of any given cut line, effectively lifting the input from two to three dimensions. Lifting allows the network to focus on the easier problem of learning a 3D everywhere-\emph{continuous} volumetric field, while a corresponding restriction operator enables the final output field to precisely resolve \emph{strict} discontinuities. Crucially, our approach does not embed the discontinuity in the neural network's weights, opening avenues to generalization of cut placement.

Our method achieves real-time simulation speeds and supports dynamic updates to cut line geometry during the simulation. Moreover, the explicit representation of discontinuities makes our neural field intuitive to control and edit, offering a significant advantage over traditional neural fields, where discontinuities are embedded within the network’s weights, and enabling new applications that rely on general cut placement.
%Additionally, our approach seamlessly integrates into both data-free and data-driven frameworks, improving the accuracy of discontinuity representation.
    
\end{abstract}