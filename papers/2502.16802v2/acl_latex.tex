\pdfoutput=1
\documentclass[11pt]{article}
\usepackage[]{acl}
\usepackage{times}
\usepackage{latexsym}
\usepackage{booktabs}
\usepackage[T1]{fontenc}
\usepackage[utf8]{inputenc}
\usepackage{microtype}
\usepackage{inconsolata}
\usepackage{graphicx}
\usepackage{subfig}
\usepackage{float}
\usepackage{listings}
\lstset{
  basicstyle=\ttfamily,
  breaklines=true  % 设置自动换行
}
\title{Unsupervised Topic Models are Data Mixers for Pre-training Language Models}

% Author information can be set in various styles:
% For several authors from the same institution:
% \author{Author 1 \and ... \and Author n \\
%         Address line \\ ... \\ Address line}
% if the names do not fit well on one line use
%         Author 1 \\ {\bf Author 2} \\ ... \\ {\bf Author n} \\
% For authors from different institutions:
% \author{Author 1 \\ Address line \\  ... \\ Address line
%         \And  ... \And
%         Author n \\ Address line \\ ... \\ Address line}
% To start a separate ``row'' of authors use \AND, as in
% \author{Author 1 \\ Address line \\  ... \\ Address line
%         \AND
%         Author 2 \\ Address line \\ ... \\ Address line \And
%         Author 3 \\ Address line \\ ... \\ Address line}


\author{
  Jiahui Peng$^{1*}$, 
  Xinlin Zhuang$^{1,2}$\thanks{\quad Equal contributions.}, Jiantao Qiu$^{1}$, Ren Ma$^{1}$, Jing Yu$^{1}$, Tianyi Bai$^{1,3}$, Conghui He$^{1}$\thanks{\quad Corresponding authors.}\\
  $^1$Shanghai AI Laboratry\\
  $^2$School of Computer Science and Technology, East China Normal University\\
  $^3$ Hong Kong University of Science and Technology
  \\
    \texttt{pengjiahui@pjlab.org.cn, xinlinzhuang@stu.ecnu.edu.cn}\\
    }
\begin{document}
\maketitle
\begin{abstract}

The performance of large language models (LLMs) is significantly affected by the quality and composition of their pre-training data, which is inherently diverse, spanning various domains, sources, and topics. Effectively integrating these heterogeneous data sources is crucial for optimizing LLM performance. Previous research has predominantly concentrated on domain-based data mixing, often neglecting the nuanced topic-level characteristics of the data. To address this gap, we propose a simple yet effective topic-based data mixing strategy that utilizes fine-grained topics generated through our topic modeling method, DataWeave. 
DataWeave employs a multi-stage clustering process to group semantically similar documents and utilizes LLMs to generate detailed topics, thereby facilitating a more nuanced understanding of dataset composition. 
Our strategy employs heuristic methods to upsample or downsample specific topics, which significantly enhances LLM performance on downstream tasks, achieving superior results compared to previous, more complex data mixing approaches. 
Furthermore, we confirm that the topics \textit{Science} and \textit{Relationships} are particularly effective, yielding the most substantial performance improvements. 
We will make our code and datasets publicly available.
\end{abstract}

\section{Introduction}

\begin{figure*}[tb]
    \centering
    \includegraphics[scale=0.5]{figs/workflow_all.pdf}
    \caption{workflow of DataWeave and topic mix.}
    \label{fig:overview}
\end{figure*}

The performance of large language models (LLMs) is profoundly influenced by the quality and composition of their pre-training data \cite{pretrainersguide, dataeverywhere, phi}. 
To ensure high-quality data, two primary strategies are commonly employed: data selection and data mixing. 
Data selection involves filtering datasets based on predefined rules \cite{gopher, refinedweb, dolma} or classifiers \cite{fineweb, qurating, dsir}, while data mixing adjusts the proportions of data from different domains in the pre-training corpus. 
Compared to data selection, data mixing is more intuitive and controllable, making it a preferred choice in LLM pre-training. 
However, the specific strategies for data mixing are rarely open-sourced by companies or research institutions, limiting reproducibility and transparency. 
Existing data mixing methods often rely on simple methods, such as temperature-based sampling \cite{dataeverywhere}, or computationally expensive procedures, such as RegMix \cite{regmix}, which requires training numerous smaller models to explore optimal data ratios. 
These methods are either suboptimal and lack performance guarantees or require substantial computational resources, presenting significant challenges for researchers with limited access to large-scale computational infrastructure.
This necessitates the development of more efficient and scalable data mixing strategies that can be broadly adopted.
\par
Topic modeling has long been a fundamental tool in natural language processing (NLP) for uncovering the latent thematic structure in large, unlabeled document collections \cite{blei2003latent, grootendorst2020bertopic}. 
Traditional approaches, such as Latent Dirichlet Allocation (LDA) \cite{blei2003latent}, rely on probabilistic graphical models, while more recent methods, such as BERTopic \cite{grootendorst2020bertopic}, leverage contextualized embeddings from pre-trained language models like BERT \cite{bert}. 
These methods often select topics based on metrics like Term Frequency-Inverse Document Frequency (TF-IDF), which can fail to capture the nuanced semantics of document clusters. 
While LLMs exhibit remarkable capabilities in zero-shot text summarization and topic interpretation, existing topic modeling methods face two significant limitations: \textbf{(i) Limited scalability to large datasets.} 
These methods typically require substantial computational resources, making them impractical for modeling topics in massive corpora. 
\textbf{(ii) Limited application in model training.} 
Generated topics are predominantly used for analyzing dataset distributions rather than improving the performance of models on downstream tasks. 
This highlights the need for topic modeling techniques that are both suitable for tackling large-scale datasets and directly applicable to enhancing model training.
\par
To address these challenges, we first propose \textbf{DataWeave}, a novel topic modeling method. 
DataWeave leverages a multi-stage clustering process to extract fine-grained topics from large-scale pre-training datasets. 
Specifically, we first group semantically similar documents into clusters, each representing a distinct topic. 
LLMs are then employed to generate meaningful topic labels, utilizing their inherent understanding of language to capture the nuances of each cluster. 
Based on topics from DataWeave, we propose to upsample or downsample weights of certain topics as data mixing strategy for pre-training LLMs.
Our experiments demonstrate that this fine-grained, topic-based data mixing approach outperforms previous complex state-of-the-art (SOTA) domain-level mixing methods. 
Furthermore, we train a topic classifier to analyze the topic distribution of downstream task evaluation datasets, providing valuable insights into the relationship between topics and task performance. 
All topic weights and the topic classifier will be open-sourced to facilitate further research and reproducibility.
In summary, the main contributions of this paper are as follows:
\begin{itemize}
    \item We propose \textbf{DataWeave}, a novel topic modeling method that combines clustering and LLMs to extract fine-grained topics from large-scale datasets. 
    \item We develop a simple yet effective topic-based data mixing strategy using the topics extracted by DataWeave. Our results show that fine-grained topics are more effective than traditional domain-level approaches and our strategy outperforms baseline data mixing methods.
    \item We analyze the relationship between topics and downstream task performance. Notably, our experiments reveal that the topics \textit{Science} and \textit{Relationships} contribute the most to task improvement. These findings provide actionable insights for optimizing pre-training data.
\end{itemize}


\section{Related Work}

\subsection{Data Mixing} 
The quality of pre-training data has been demonstrated to play a critical role in model performance, as highlighted in several studies \cite{pretrainersguide, dataeverywhere}. 
One natural and intuitive approach to improving data quality involves adjusting the weights assigned to different data domains. 
Data mixing methods aim to optimize the distribution of attribute weights within pre-training datasets. 
For example, methods such as DoReMi \cite{doremi} and DOGE \cite{doge} utilize small proxy models to generate domain weights, while DMLaw \cite{dmlaw} and RegMix \cite{regmix} determine domain weights by training a set of smaller models.
More recently, Llama-3.1 \cite{dubey2024llama} employs downsampling to reduce the proportion of data from the arts and entertainment domain, and \citet{chen2024towards} investigates effective training strategies by adjusting topic weights. 
However, these studies lack details of domain weights and primarily explore data mixing from a domain-level perspective.
In this paper, we propose incorporating topic modeling to control data weights at a finer granularity, enabling more precise adjustments to the pre-training data and enhancing LLM's capabilities.

\subsection{Topic Model} 
Topic modeling is an unsupervised method used to uncover abstract topics within documents in the field of Natural Language Processing \cite{topicsurvey}. 
Traditional approaches, such as Latent Dirichlet Allocation (LDA), typically rely on probabilistic techniques to generate topics \cite{blei2003latent}. 
BERTopic \cite{grootendorst2020bertopic} leverages transformer-based architectures to enhance traditional topic modeling processes.
More recent research has explored the use of LLMs for topic modeling, particularly by utilizing their text summarization capabilities to automatically assign descriptive labels to clusters of words. 
For instance, \cite{rijcken2023towards} demonstrated that approximately half of the topic labels generated by ChatGPT were considered useful by human evaluators. 
Additionally, other studies \cite{mu2024addressing, mu2024large, rijcken2023towards} have conducted extensive experiments to improve the performance of LLMs in topic modeling through prompt engineering.
However, these methods become computationally expensive and impractical when applied to large-scale datasets. 
In contrast to these approaches, we propose utilizing topic model for data mixing, enabling more fine-grained control over data weights. 
This approach not only improves the efficiency of pre-training LLMs but also provides valuable insights into the role of topic-level adjustments in optimizing data distributions for enhanced model performance.

\section{DataWeave}
\label{sec:method}

DataWeave is a novel framework designed to address large-scale datasets by integrating multi-stage clustering with topic extraction, thereby facilitating effective data mixing for pre-training LLMs. The framework consists of three main stages, as illustrated in Figure \ref{fig:overview}.
In the first stage, semantic embeddings for all documents are generated using the BGE model \footnote{\url{https://huggingface.co/BAAI/bge-base-en-v1.5}}. 
Next, the documents are partitioned into $K_1$ clusters through clustering, and representative summaries are generated for each cluster. 
In the subsequent stage, the documents are further grouped into $K_2$ clusters, and abstract topics are derived for each cluster. 
Finally, a subset of $T$ topics is merged from the $K_2$ topics to ensure the resulting topics are more coherent and interpretable.
The determination of specific hyperparameters is dependent on the characteristics of the specific datasets and the available computational resources. 
The configurations used in our experiments are detailed in Section \ref{subsec:setup}.

\subsection{Step 1: Multi-stage Clustering}
Clustering large-scale datasets poses significant challenges due to high memory requirements, communication overhead, and other computational limitations, which often exceed the capabilities of standard computational devices. 
Given these constraints, and considering the trade-offs among commonly available clustering algorithms \cite{clusteringsurvey} and computational resources, we adopt the K-Means algorithm due to its relatively low computational complexity of $O(kNI)$, where $k$ denotes the number of clusters, $N$ denotes the number of data points, and $I$ denotes the maximum iteration times.
In this stage, the data is first partitioned into $K_1$ clusters using K-Means. 
For each cluster, we randomly sample $m_1$ data points and employ \textit{gpt-4o-2024-11-20} to generate an abstract summary to represent the semantics of this cluster. 
The generated summary provides a concise and comprehensive description of the cluster, constrained to no more than 20 words. 
The prompt template used for summary generation is detailed in Appendix \ref{app:prompt}. 


\subsection{Step 2: Topic Extraction}
Following the similar operations in Step 1, we continue to employ K-Means to partition the $K_1$ clusters into $K_2$  more compact clusters. 
For each of these $K_2$ clusters, we randomly sample $m_2$ summaries generated in Step 1 and use \textit{gpt-4o-2024-11-20} to produce an abstract topic with no more than 3 words, resulting in a total of $K_2$ topics.
Despite the high readability of the $K_2$ topics, two significant issues arise. 
First, there is the problem of extensive topic overlap. 
Upon manual inspection of the $K_2$ topics, we observe considerable redundancy, with many topics sharing similar content. 
Second, there is the issue of non-parallel topic granularity. 
Specifically, LLMs tend to generate topics that vary in specificity, ranging from highly detailed to overly abstract, which undermines the consistency and interpretability of the results.
To address these issues, we further utilize \textit{gpt-4o-2024-11-20} to merge the $K_2$ topics into $T$ ultimate topics, ensuring a more coherent and hierarchical topic structure. 
The prompt template used for topic merging is provided in Appendix \ref{app:prompt}. 

\section{Experiment}

\subsection{Experimental Setup}
\label{subsec:setup}

\paragraph{Dataset}
We utilize the widely adopted SlimPajama corpus \cite{slimpajama} as the dataset for our experiments. This corpus comprises a total of 591,399,449 documents, encompassing approximately 627B tokens. Additionally, SlimPajama categorizes the data into seven distinct domains: \textit{arXiv}, \textit{Books}, \textit{C4}, \textit{CommonCrawl}, \textit{GitHub}, \textit{StackExchange}, and \textit{Wikipedia}.

\paragraph{DataWeave Configuration}
For the clustering process, we set the number of clusters $K_1$ and $K_2$ in the two stages to 10,000 and 300, respectively. 
A hyperparameter search was conducted to determine these values: $K_1$ was explored over the set \{10,000, 30,000, 60,000, 90,000\}, and $K_2$ was searched within the range \{100, 150, \dots, 600\}. The optimal values for \(K_1\) and \(K_2\) were selected based on the maximum Silhouette Coefficient criterion \cite{silhouette}, ensuring well-defined and meaningful clusters.
Moreover, we merge 300 topics into 12 final topics.
Regarding topic extraction, we set the sample sizes $m_1$ and $m_2$ to 10 and 50, respectively. 
These values were chosen to account for the maximum context window length of \textit{gpt-4o-2024-11-20} as well as the average length of the input texts and the generated summaries.

\subsection{Implementation Details}
\label{subsec:implementation}

\paragraph{Training}
In the continual pre-training setting, the model is initially pre-trained on 30B uniformly sampled tokens, followed by further pre-training on an additional 30B tokens using different data mixtures. 
In contrast, in the standard pre-training setting, the model is directly pre-trained on 30B tokens using different data mixtures.
The model employed is a decoder-only transformer architecture with 1.3B parameters, incorporating Rotary Position Embeddings (RoPE) \cite{su2024roformer} and supporting a maximum context window of 1,024 tokens \cite{llama}. 
Further details regarding the model architecture and training configurations can be found in Appendix \ref{app:training}.


\paragraph{Baselines}
We leverage the topics generated by DataWeave to guide data mixing strategies for pre-training LLMs. Specifically, inspired by Llama-3.1 \cite{dubey2024llama}, we upsample one or more selected topics to a weight of 30\% while downsampling the remaining topics to a weight of original 70\%. 
To demonstrate the effectiveness of DataWeave, we compare it against several SOTA data mixing methods:
\begin{itemize}
\item Uniform: Tokens are randomly sampled with uniform probability across all domains, without applying any specific control over the data distribution.
\item Temperature: Temperature-based sampling \cite{dataeverywhere,bert} proportionally adjusts data source weights according to a scaled factor of their token counts. For our experiments, we set $t=0.4$
to compute topic weights based on token ratios.
\item RegMix: RegMix \cite{regmix} involves training a set of small 1M-parameter models on diverse data mixtures and fitting regression models to predict model performance based on the respective mixtures. Using the fitted regression model, the top-ranked mixture is simulated to determine the optimal topic weights.
\end{itemize}
Details regarding the data weights used in different settings are provided in Appendix \ref{app:detailed_weights}. 
\begin{figure*}[tb]
    \centering
    \hspace{-2.2cm}
    \subfloat{   %第一张子图
        \label{fig:topic_distribution}
        \begin{minipage}{0.45\textwidth}
        \centering 
        \includegraphics[scale=0.27]{figs/topic_distribution.png}
        \end{minipage}
    }
    \hspace{0.185\textwidth} 
    \subfloat{ 
        \label{fig:topic_distribution_domain}
        % \hspace{1cm}
        \begin{minipage}{0.45\textwidth}
        \centering   
        \includegraphics[scale=0.3]{figs/heatmap_no_value_topic_domain.pdf}
        \end{minipage}
    }
\caption{The DataWeave topic distribution of SlimPajama.}
\label{fig:topic_distribution}  
\end{figure*}
\paragraph{Evaluation}
To evaluate the capabilities of pre-trained LLMs, we assess their performance through in-context learning using the \textit{lm-evaluation-harness} framework \cite{eval-harness} and accuracy scores are reported.
The evaluation dataset encompasses three categories of downstream tasks and further evaluation details are provided in Appendix \ref{app:evaluation}.
\begin{itemize}
\item \textbf{General Knowledge}: ARC-Challenge \cite{arc}, ARC-Easy, and SciQ \cite{sciq}.
\item \textbf{Commonsense Reasoning}: PIQA \cite{bisk2020piqa}, SIQA \cite{siqa}, WinoGrande \cite{winogrande}, and CommonsenseQA \cite{commonsenseqa}.
\item \textbf{Reading Comprehension}: RACE \cite{race} and OpenBookQA \cite{openbookqa}.
\end{itemize}






\subsection{DataWeave Results}

\paragraph{Topic Distribution}
DataWeave yields 12 final topics: \textit{Technology}, \textit{Science}, \textit{Politics}, \textit{Health}, \textit{Lifestyle}, \textit{Law}, \textit{Entertainment}, \textit{Education}, \textit{Relationships}, \textit{Finance}, \textit{Community}, and \textit{Others}. 
Based on the analysis of the topic distribution in Figure \ref{fig:topic_distribution}, we have the following key observations:
\begin{enumerate}
    \item \textbf{Alignment with Human-Defined Categories.}
    The majority of topics, such as \textit{Technology} and \textit{Entertainment}, closely align with traditional human-defined categories. This indicates that DataWeave is capable of identifying coherent and interpretable topics that reflect common thematic structures in the dataset.
    \item \textbf{Emergence of Divergent Topics.} 
    Certain topics, such as \textit{Health} and \textit{Relationships}, diverge from pre-existing human-defined labels. This suggests that clustering process in DataWeave can uncover nuanced or less conventional themes that may not be explicitly represented in predefined taxonomies.
    \item \textbf{Limitations of Human-Defined Labels.} 
    The analysis highlights the insufficiency of human-defined labels in fully capturing the diversity of online content. DataWeave demonstrates the ability to reveal latent themes that are not immediately apparent in traditional classification schemes.
    \item \textbf{Topic Distribution Across Domains.} 
    Figure \ref{fig:topic_distribution}(b) illustrates the distribution of topics across various domains. Each column delineates the distribution of topics pertinent to its respective domain. The topic \textit{Technology} demonstrates a strong correlation with \textit{StackExchange} and \textit{GitHub}, as both platforms emphasize technical discussions and coding practices. In contrast, data derived from \textit{CommonCrawl} and \textit{C4} reveals a high correlation with a majority of topics. This finding underscores the significant diversity present within these domains.
\end{enumerate}


\paragraph{Effectiveness of DataWeave}
In the absence of ground-truth labels for topic modeling, establishing a robust and comprehensive evaluation framework for topic models remains a debated challenge within the research community. 
Some approaches propose assessing models based on the top-ranked words associated with each topic  \cite{bianchi2020pre,bouma2009normalized}. However, this method often entails considerable computational overhead, particularly when applied to large-scale datasets.
Therefore, we introduce an alternative evaluation method by leveraging \textit{gpt-4o-2024-11-20} to identify the most relevant topics for assessing the effectiveness of DataWeave. 
Given that content typically spans multiple labels, we report three evaluation metrics for reference: Top-1 Accuracy, Top-3 Accuracy, and Top-5 Accuracy. 
These metrics measure the proportion of instances where the DataWeave label appears within the top-$k$ labels identified by \textit{gpt-4o-2024-11-20}. 
The evaluation results for our method are as follows: Top-1 Accuracy is 57.23\%,Top-3 Accuracy is 81.19\%, and Top-5 Accuracy is 90.19\%. 
These results prove the effectiveness of DataWeave.
Moreover, additional case-specific details are provided in Appendix \ref{app:dataweave_fullresult}.
% TODO
\begin{table*}[!tb]
\small
\resizebox{\textwidth}{!}{
\centering
\begin{tabular}{@{}lcccc@{}}
\toprule
\multicolumn{1}{c}{\textbf{Upsampled Topic}} &
  \begin{tabular}[c]{@{}c@{}}\textbf{General Knowledge}\\ (3 tasks)\end{tabular} &
  \begin{tabular}[c]{@{}c@{}}\textbf{Commonsense Reasoning}\\ (4 tasks)\end{tabular} &
  \begin{tabular}[c]{@{}c@{}}\textbf{Reading Comprehension}\\ (2 tasks)\end{tabular} &
  \begin{tabular}[c]{@{}c@{}}\textbf{Average}\\ (9 tasks)\end{tabular} \\ \midrule
Random        & 58.11          & 46.95 & 32.61          & 47.48                  \\ \midrule
Technology    & 58.64          & 47.02          & 32.27          & 47.62 (+0.14)          \\
Science       & \textbf{61.97} & 46.49          & \textbf{34.10} & \textbf{48.90 (+1.42)} \\
Politics      & 58.39          & 47.22          & 32.02          & 47.57 (+0.09)          \\
Health        & 58.84          & \underline{47.92}          & 33.12          & 48.27 (+0.79)          \\
Lifestyle     & 59.14          & 47.07          & 32.48          & 47.85 (+0.37)          \\
Law           & 58.12          & 46.69          & \underline{34.00}          & 47.68 (+0.20)          \\
Entertainment & 57.90          & 46.91          & 32.27          & 47.32 (-0.16)          \\
Education     & \underline{59.50}          & 46.72          & 33.46          & 48.03 (+0.55)          \\
Relationships & 58.87          & \textbf{48.10}          & 33.11          & \underline{48.36 (+0.88)}        \\
Finance       & 57.99          & 46.87          & 32.86          & 47.46 (-0.02)         \\
Community     & 57.88          & 47.06          & 32.82          & 47.50 (+0.02)          \\
Others        & 58.38          & 46.75          & 33.54          & 47.69 (+0.21)          \\ \bottomrule
\end{tabular}
}
\caption{Downstream tasks results of continual pre-training. \textit{Random} denotes no any control over topic distribution of the 30B additional tokens. Full results are provided in Appendix \ref{app:datamix}.}
\label{tab:downstream}
\end{table*}

\subsection{Continual Pre-training Results}


We conducted experiments in continual pre-training setting to explore the effects of 12 topics generated through DataWeave. 
Specifically, we pre-trained the LLM using another 30B tokens at different data mixture where we upsampled data from each topic, as detailed in Section \ref{subsec:implementation}.
As shown in Table \ref{tab:downstream}, most scenarios show superior performance over random selecting 60B tokens without considering topic weights, indicating that targeted upsampling can significantly enhance model capabilities in specific tasks.
Among the topics, \textit{Science} stands out as the most effective, achieving the highest overall performance and the best results in General Knowledge and Reading Comprehension, which aligns well with human intuition given the structured and information-dense nature of scientific texts. 
\textit{Health} and \textit{Relationships} also yield notable gains, with \textit{Health} improving the average score by 0.79 and \textit{Relationships} by 0.88. 
These results suggest that topics containing practical, real-world knowledge or those closely tied to human reasoning may have a stronger impact on enhancing LLM capabilities across diverse tasks.
\par
Interestingly, some topics such as \textit{Technology} and \textit{Education}, while intuitively important for general knowledge and reasoning tasks, show only moderate improvements in the overall average. 
This could indicate that their data distributions or linguistic patterns are already well-represented in the base pre-training corpus, leading to diminishing returns from additional upsampling. 
On the other hand, topics like \textit{Entertainment} and \textit{Community}, which might be expected to have a more limited impact due to their less formal or specialized nature, show comparable improvements to other domains. 
This suggests that even seemingly less critical topics can contribute positively to overall performance, likely by diversifying the model’s linguistic and contextual understanding.



\subsection{Pre-training Results}

\begin{table*}[!tb]
\centering
\small
\resizebox{\textwidth}{!}{
\begin{tabular}{@{}lcccc@{}}
\toprule
\textbf{Data Mixing Method} &
  \begin{tabular}[c]{@{}c@{}}\textbf{General Knowledge}\\ (3 tasks)\end{tabular} &
  \begin{tabular}[c]{@{}c@{}}\textbf{Commonsense Reasoning}\\ (4 tasks)\end{tabular} &
  \begin{tabular}[c]{@{}c@{}}\textbf{Reading Comprehension}\\ (2 tasks)\end{tabular} &
  \begin{tabular}[c]{@{}c@{}}\textbf{Average}\\ (9 tasks)\end{tabular} \\ \midrule
Uniform                   & 54.52 & 44.42 & 25.07 & 43.49\\ \midrule
RegMix-domain   & 53.77 & 45.74 & 25.38 & 43.89 (+0.40) \\ 
RegMix-topic    & 54.39 & \underline{45.96} & 26.16 & 44.39 (+0.90) \\
Temperature-domain & 53.64 & 45.47 & 25.76 & 43.81(+0.31) \\ 
Temperature-topic & 55.62 & 44.96 & \textbf{27.66} & 44.67(+1.18) \\  \midrule
\textbf{$\downarrow$ Entertainment}      & 55.38 & 45.68 & 26.16 & 44.58 (+1.09) \\
\textbf{$\uparrow$ Science}    & \textbf{58.05} & 44.42 & \underline{26.84} & \underline{45.06(+1.57)} \\
\textbf{$\uparrow$ \{Science,Relationships,Health\}}   & \underline{56.36} & \textbf{46.23} & 26.52 & \textbf{45.23(+1.74)} \\ \bottomrule
\end{tabular}
}
\caption{Downstream tasks results of data mixing methods in pre-training setting. The symbol $\uparrow$ represents upsampling, while $\downarrow$ denotes downsampling of one or more topics. Full results are provided in Appendix \ref{app:datamix}.}
\label{tab:topicmix}
\end{table*}

We conducted experiments in  pre-training setting using various data mixing methods at both the domain level and the topic level generated by DataWeave. 
The results of these experiments are presented in Table \ref{tab:topicmix}.

\paragraph{Topic-level outperforms domain-level for data mixing.}
Our experimental results demonstrate that adjusting data weights at the topic level consistently outperforms adjustments at the domain level. 
As shown in Table~\ref{tab:topicmix}, both  RegMix and Temperature methods yield better results when applied to topics rather than domains. 
This can be attributed to the finer granularity of topics, which allows for more precise control over the diversity and relevance of the data. 
For instance, as shown in Figure \ref{fig:topic_distribution}, within the domain \textit{C4},  there may coexist highly beneficial topics like \textit{Science} and less impactful ones like \textit{Entertainment}. 
Adjusting domain weights alone fails to adequately highlight useful data, as the domain aggregates both high- and low-utility topics. 
In contrast, topic-level adjustments enable targeted amplification of valuable data while suppressing less relevant portions, leading to more significant performance gains. 
This result underscores the importance of topic granularity for optimizing data utilization in pre-training pipelines and highlights the superior flexibility and effectiveness of topic modeling.


\paragraph{Heuristic-based topic mixing is simple yet effective.}
Interestingly, we find that our straightforward heuristic-based approach to topic mixing achieves the best overall performance, surpassing more complex data mixing methods. 
As shown in Table~\ref{tab:topicmix}, downsampling the over-represented topic \textit{Entertainment} improves the average performance by 1.09\%. 
This aligns with findings from Llama-3.1 \cite{dubey2024llama}, which demonstrate that reducing the prevalence of web-dominant categories like \textit{Entertainment} enhances language model capabilities. 
Furthermore, our experiments reveal that upsampling beneficial topics such as \textit{Science}, \textit{Relationships}, and \textit{Health}—either individually or collectively—leads to substantial performance improvements, with the highest gain of 1.74\% observed when all three topics are upsampled together. 
Notably, these heuristic-based adjustments can be implemented with minimal overhead while delivering significant performance gains. 
This makes them an attractive option for practitioners seeking to balance efficiency with effectiveness.



\section{Analysis}

\subsection{Relation to Downstream Tasks}

\begin{figure}
    \centering
    \includegraphics[width=1.0\linewidth]{figs/heatmap_no_value.pdf}
    \caption{Topic distribution in downstream tasks.}
    \label{fig:heatmap}
\end{figure}

To investigate the impact of topics derived from DataWeave on a range of downstream tasks, we trained a BERT topic classifier to categorize documents into the identified 12 topics. 
Additional details regarding the topic classifier can be found in Appendix \ref{app:topic_classifier}. 
We employed the topic classifier to annotate the evaluation datasets, and the resulting distributions are illustrated in Figure \ref{fig:heatmap}.
In General Knowledge tasks, the topic \textit{Science} consistently constitutes the largest proportion across the ARC-C, ARC-E, and SciQ datasets, which may account for the significant performance improvements observed when upsampling data from \textit{Science} for these three tasks (see Table \ref{tab:downstream}). 
Similarly, in the realm of Commonsense Reasoning tasks, the topic \textit{Lifestyle} emerges as the most prominent. 
For Reading Comprehension tasks, the topic distribution remains relatively balanced among \textit{Lifestyle}, \textit{Entertainment}, \textit{Education}, and \textit{Science}.
These findings provide valuable insights into the effectiveness of various data mixing strategies.

\begin{table*}[!tb]
\small
\centering
\begin{tabular}{@{}lll@{}}
\toprule
 \textbf{Summary} & \textbf{Topic} & \textbf{Final Topic} \\
 (10, 000 items) &  (300 items) &  (12 items) \\ \midrule
\begin{tabular}[l]{@{}l@{}}Exploration of pirate culture, entertainment, and media across various \\ forms and events.\end{tabular} & Gaming & Entertainment\\ \midrule
Variety of salsa recipes and their uses as appetizers & Cooking & Lifestyle \\ \midrule
Analysis of the ongoing tensions and nuclear threats posed by North Korea. & Politics & Politics\\ \midrule
Overview of various zombie-themed films and their cultural impact. & Entertainment & Entertainment \\ \midrule
\begin{tabular}[l]{@{}l@{}}Overview of various antique jewelry businesses and services, including custom \\ designs and repair options.\end{tabular}  & Jewelry & Lifestyle \\ \midrule
\begin{tabular}[l]{@{}l@{}}The importance of love, connection, and communication in relationships is \\ emphasized throughout various challenges and experiences.\end{tabular}  & Relationships & Relationships \\ \midrule
\begin{tabular}[l]{@{}l@{}}Recent astronomical discoveries reveal insights into the universe's formation, \\ including ancient stars, black holes, and galaxy dynamics.\end{tabular}  & Space & Science \\ \midrule
\begin{tabular}[l]{@{}l@{}}Setting up development environments and compiling applications on Windows  \\ and Linux using various tools and libraries.\end{tabular} & Technology & Technology \\ 
 \bottomrule
\end{tabular}
\caption{Examples across different DataWeave stages.}
\label{tab:dataweave_process}
\end{table*}
\subsection{Cost Analysis} 

Understanding domain interactions poses significant challenges for human analysts. RegMix \cite{regmix} offers valuable insights into how different data domains influence one another, uncovering complex relationships that are often difficult for human experts to fully comprehend. Consequently, prior research on data mixtures has primarily concentrated on developing automated methods to efficiently identify high-performing combinations, rather than relying exclusively on human intuition. In contrast, our approach presents an efficient way of determining data weights. As demonstrated in Table \ref{tab:topicmix}, heuristic-based methods outperform all other data mixing techniques in downstream tasks, without any supplementary models, thereby further validating the efficiency of topic mixing.


\subsection{Case Study}


Table \ref{tab:dataweave_process} presents several examples in the DataWeave process, illustrating the progression from 10,000 summaries to 300 identified topics, ultimately distilled into 12 final topics.

\paragraph{LLMs can extract high-quality topics from summaries.}
Unlike individual words, summaries encapsulate information from multiple documents, providing a rich semantic foundation for topic extraction. 
This complexity allows LLMs to identify and extract high-quality, human-readable topics from these summaries effectively. 
The ability of LLMs to synthesize and distill nuanced themes underscores their potential in various NLP tasks, particularly in generating coherent and relevant topics that reflect the underlying content.

\paragraph{Merging topics is vital.}
The analysis reveals a notable issue of non-parallel topic granularity among the initial 300 human-interpretable topics. 
For example, the topic \textit{Gaming} serves as a specific subset within the broader category of \textit{Entertainment}, while \textit{Jewelry} and \textit{Lifestyle} exhibit similar hierarchical relationships. 
This discrepancy highlights the need for a systematic merging process to ensure clarity and coherence in topic categorization. 
Fortunately, this granularity issue has been effectively resolved in the final set of 12 topics, demonstrating the importance of refining topic definitions and relationships to enhance interpretability and usability in downstream applications.


\section{Conclusion}
In this study, we introduce a novel topic modeling method that combines clustering techniques with Large Language Models (LLMs) to facilitate data mixing, ultimately enhancing LLM performance on downstream tasks. 
Our approach demonstrates significant improvements in LLM pre-training effectiveness by strategically adjusting the weights of specific topics, thereby achieving a more balanced capability across various domains.
To further enhance performance in domain-specific applications, it is essential to curate relevant knowledge data meticulously. 
This curation process ensures that the LLMs are exposed to high-quality, contextually appropriate information, which is critical for their effective operation in specialized fields.
Looking ahead, our future work will focus on incorporating a greater number of topics per document. 
This expansion will allow for a richer representation of content, enabling more nuanced understanding and generation capabilities.

% 'Limitations' starts from Page 9
% \clearpage 
\newpage
\section*{Limitations}
There are two limitations in this work. 
First, due to the scale and complexity of web-scale data, the topic generation process shows potential for further enhancements in both effectiveness and efficiency. 
Second, the number of final topics in our method is determined as a hyperparameter by human judgment rather than model performance, necessitating additional experimentation. 
Our future work will focus on improving these aspects.

% 未来可以扩充topic数量,研究diversity的data mix
% 可以发展成多类别的data mix





\bibliography{custom}

\appendix


% clustering score

% topic data

\section{Prompt Templates}
\label{app:prompt}
We present three prompts utilized in DataWeave, including generating a brief summary, deriving topics from summaries, and producing final topics. These prompts are illustrated in Figures \ref{fig:prompt_brief_prompt}, \ref{fig:prompt_summary2prompt}, and \ref{fig:prompt_topic2topic}. 
We employ \textit{gpt-4o-2024-11-20} \footnote{\url{https://openai.com/index/hello-gpt-4o}} to obtain the corresponding results.
% Furthermore, specific examples about topics are provided. %TODO


\begin{figure*}[tb]
    \centering
    \includegraphics[scale=0.6]{figs/prompt_brief_summary.pdf}
    \caption{The prompt of extracting brief summary for each partition.}
    \label{fig:prompt_brief_prompt}
\end{figure*}


\begin{figure*}[tb]
    \centering
    \includegraphics[scale=0.5]{figs/prompt_summary2topic.pdf}
    \caption{The prompt of extracting summary to topic.}
    \label{fig:prompt_summary2prompt}
\end{figure*}

\begin{figure*}[tb]
    \centering
    \includegraphics[scale=0.5]{figs/prompt_topic2topic.pdf}
    \caption{The prompt of merging topics to final topics.}
    \label{fig:prompt_topic2topic}
\end{figure*}


\section{Training Details}
\label{app:training}

The architecture of the pre-trained model is detailed in Table \ref{tab:architecture}. 
Each model was trained on 32x NVIDIA A800 GPUs, utilizing a global batch size of $4 \times 2^{20}$ tokens and completing 7,500 training steps within approximately 14 hours. 
The learning rate was set to $5 \times 10^{-5}$, and the Adam optimizer was used with the following hyperparameters: $\beta_1 = 0.9$, $\beta_2 = 0.95$, and $\epsilon = 10^{-8}$.


\begin{table}[!tb]
\small
\begin{tabular}{@{}ll@{}}
\toprule
Hyperparameter               & Value               \\ \midrule
Vocabulary Size                   & 32,000              \\
MLP Ratio                    & 8/3                 \\
Hidden Dimension Size        & 2048                \\
Number of Layers             & 24                  \\
Number of Attention Heads    & 16                  \\
Number of KV Attention Heads & 16                  \\
RoPE Base                    & 10,000              \\
Maximum Context Window Length                    & 1024              \\
Number of Parameters         & 1,345,423,360 (1.3B) \\ \bottomrule
\end{tabular}
\caption{The architecture of pre-trained decoder-only model.}
\label{tab:architecture}
\end{table}

\section{Data Weights}
\label{app:detailed_weights}
The detailed topic weights in different settings are provided in Table \ref{tab:weights}. 
In our method, we upsample or downsample target topic before normalizing the weights. Specifically, in downsampling \textit{Entertainment} experiment, we reduce its weight from 23.91\% to 10\% and then normalize the results. In the unsampling experiment, we increase \textit{Science} weight from 5.73\% to 35.73\%. Additionally, we raise the weights of \textit{Science} \textit{Relationships} \textit{Health} by 10\% each, followed by normalization.

\begin{table*}[]
\fontsize{9}{10}\selectfont
\centering
\begin{tabular}{@{}lcccccc@{}}
\toprule
{Topic}        & {SlimPajama} & {RegMix} & {Temperature} & {$\downarrow$ Ent.} & {$\uparrow$ Sci.} & {$\uparrow$ S.R.H} \\ \midrule
{Technology}    & {17.55} & {14.91} & {10.35} & {20.39} & {13.5} & {13.5} \\
{Science}       & {5.73} & {5.54} & {7.7} & {6.66} & {27.49} & {12.1} \\
{Politics}      & {8.23} & {4.06} & {8.2} & {9.56} & {6.33} & {6.33} \\
{Health}        & {7.04} & {5.31} & {7.96} & {8.17} & {5.41} & {13.1} \\
{Lifestyle}     & {5.49} & {12.01} & {7.66} & {6.37} & {4.22} & {4.22} \\
{Law}           & {6.08} & {4.12} & {7.77} & {7.07} & {4.68} & {4.68} \\
{Entertainment} & {23.91} & {29.14} & {12.13} & {11.62} & {18.39} & {18.39} \\
{Education}     & {13.4} & {9.14} & {9.33} & {15.56} & {10.3} & {10.31} \\
{Relationships} & {1.14} & {6.16} & {6.87} & {1.32} & {0.87} & {8.57} \\
{Finance}       & {4.01} & {2.63} & {7.38} & {4.66} & {3.09} & {3.09} \\
{Community}     & {2.29} & {1.89} & {7.07} & {2.66} & {1.76} & {1.76} \\
{Others}        & {5.13} & {5.1} & {7.59} & {5.96} & {3.95} & {3.95} \\  \bottomrule
\end{tabular}
\caption{Exact topic weights (\%) on SlimPajama obtained in data mixing methods.}
\label{tab:weights}
\end{table*}


\section{Evaluation Details}
\label{app:evaluation}

We evaluated LLM performance under few-shot ICL settings using the \textbf{lm-evaluation-harness} framework \footnote{\url{https://github.com/EleutherAI/lm-evaluation-harness}}.
The details for evaluation are shown in Table \ref{tab:icl}.

\begin{table}[]
\centering
\small
\begin{tabular}{@{}cc@{}}
\toprule
Dataset       & Number of Examples \\ \midrule
ARC-E         & 15                 \\
ARC-C         & 15                 \\
SciQ          & 2                  \\ \midrule
SIQA          & 10                 \\
PIQA          & 10                 \\
WinoGrande    & 15                 \\
CommonsenseQA & 10                 \\ \midrule
RACE          & 2                  \\
OpenbookQA    & 10                 \\ \bottomrule
\end{tabular}
\caption{ICL evaluation details in our experiment.}
\label{tab:icl}
\end{table}


\section{Full Experimental Results}
\label{app:results}

\subsection{Continual Pre-training Results}
The full results of continual pre-training experiment are shown in Tables \ref{tab:fullcontinual_1}, \ref{tab:fullcontinual_2}, and \ref{tab:fullcontinual_3}.

\begin{table*}[!tb]
\centering
\setlength{\tabcolsep}{1mm}{
\begin{tabular}{@{}lcccc@{}}
\toprule
\textbf{Upsampled Topic} & \textbf{ARC-E} & \textbf{ARC-C} & \textbf{SciQ} & \textbf{Average} \\ \midrule
Random        & 56.99 & 27.73 & 89.60 & 58.11 \\ \midrule
Technology    & 58.08 & 28.33 & 89.50 & 58.64 \\
Science       & 64.18 & 31.74 & 90.00 & 61.97 \\
Politics      & 57.87 & 27.30 & 90.00 & 58.39 \\
Health        & 58.71 & 28.41 & 89.40 & 58.84 \\
Lifestyle     & 58.54 & 29.09 & 89.80 & 59.14 \\
Law           & 57.24 & 27.13 & 90.00 & 58.12 \\
Entertainment & 57.03 & 27.56 & 89.10 & 57.90 \\
Education     & 59.51 & 29.18 & 89.80 & 59.50 \\
Relationships & 58.41 & 28.41 & 89.80 & 58.87 \\
Finance       & 57.11 & 27.47 & 89.40 & 57.99 \\
Community     & 57.26 & 27.47 & 88.90 & 57.88 \\
Others        & 58.33 & 26.62 & 90.20 & 58.38 \\ \bottomrule
\end{tabular}
}
\caption{Full downstream tasks results of continual pre-training in General Knowledge.}
\label{tab:fullcontinual_1}
\end{table*}

\begin{table*}[!tb]
\centering
\setlength{\tabcolsep}{1mm}{
\begin{tabular}{@{}lccccc@{}}
\toprule
\textbf{Upsampled Topic} & \textbf{SIQA} & \textbf{PIQA} & \textbf{WinoGrande} & \textbf{CommonsenseQA} & \textbf{Average} \\ \midrule
Random        & 40.63 & 70.29 & 55.17 & 21.70 & 46.95 \\ \midrule
Technology    & 42.32 & 70.78 & 55.09 & 19.90 & 47.02 \\
Science       & 41.45 & 69.85 & 55.72 & 18.92 & 46.49 \\
Politics      & 41.91 & 70.08 & 56.99 & 19.90 & 47.22 \\
Health        & 41.91 & 71.59 & 55.64 & 22.52 & 47.92 \\
Lifestyle     & 41.45 & 72.58 & 55.32 & 18.92 & 47.07 \\
Law           & 41.10 & 70.08 & 55.25 & 20.31 & 46.69 \\
Entertainment & 41.81 & 70.29 & 56.12 & 19.41 & 46.91 \\
Education     & 42.02 & 69.64 & 55.80 & 19.41 & 46.72 \\
Relationships & 43.55 & 70.56 & 57.06 & 21.21 & 48.10 \\
Finance       & 41.25 & 69.91 & 54.93 & 21.38 & 46.87 \\
Community     & 41.35 & 70.18 & 55.33 & 21.38 & 47.06 \\
Others        & 41.04 & 69.10 & 56.36 & 20.48 & 46.75 \\ \bottomrule
\end{tabular}
}
\caption{Full downstream tasks results of continual pre-training in Commonsense Reasoning.}
\label{tab:fullcontinual_2}
\end{table*}

\begin{table*}[!tb]
\centering
\setlength{\tabcolsep}{1mm}{
\begin{tabular}{@{}lccc@{}}
\toprule
\textbf{Upsampled Topic} & \textbf{RACE} & \textbf{OpenbookQA} & \textbf{Average} \\ \midrule
Random                   & 33.01         & 32.20               & 32.61            \\ \midrule
Technology               & 32.34         & 32.20               & 32.27            \\
Science                  & 33.40         & 34.80               & 34.10            \\
Politics                 & 32.24         & 31.80               & 32.02            \\
Health                   & 32.63         & 33.60               & 33.12            \\
Lifestyle                & 31.96         & 33.00               & 32.48            \\
Law                      & 33.40         & 34.60               & 34.00            \\
Entertainment            & 32.73         & 31.80               & 32.27            \\
Education                & 33.11         & 33.80               & 33.46            \\
Relationships            & 33.01         & 33.20               & 33.11            \\
Finance                  & 33.11         & 32.60               & 32.86            \\
Community                & 32.63         & 33.00               & 32.82            \\
Others                   & 33.88         & 33.20               & 33.54            \\ \bottomrule
\end{tabular}
}
\caption{Full downstream tasks results of data mixing in Reading Comprehension.}
\label{tab:fullcontinual_3}
\end{table*}


\subsection{Pre-training Results}
\label{app:datamix}
The full results of pre-training experiment are shown in Table \ref{tab:fullmixing_1}, Table \ref{tab:fullmixing_2}, and Table \ref{tab:fullmixing_3}.

\begin{table*}[!tb]
\small
\centering
\setlength{\tabcolsep}{1mm}{
\begin{tabular}{@{}lccccc@{}}
\toprule
  & \textbf{ARC-E} & \textbf{ARC-C} & \textbf{SciQ} & \textbf{Average} \\ \midrule
Uniform                 & 52.44 & 26.20 & 84.90 & 54.52 \\ \midrule
RegMix-domain          & 51.81 & 25.60 & 83.90 & 53.77\\ 
RegMix-topic           & 51.26 & 26.71 & 85.20 & 54.39 \\ 
Temperature-domain    & 51.18 & 25.94 & 83.80 & 53.64\\ 
Temperature-topic     & 53.87 & 27.30 & 85.70 & 55.62 \\ \midrule
\textbf{$\downarrow$ Entertainment}      & 53.66 & 26.79 & 85.70 & 55.38\\
\textbf{$\uparrow$ \{Science\}} & 59.05 & 29.1 & 86 & 58.05 \\
\textbf{$\uparrow$ \{Science,Relationships,Health\}}       & 55.72 & 27.47 & 85.9 & 56.36 \\ \bottomrule
\end{tabular}
}
\caption{Full downstream tasks results of data mixing in General Knowledge.}
\label{tab:fullmixing_1}
\end{table*}


\begin{table*}[!tb]
\small
\centering
\setlength{\tabcolsep}{1mm}{
\begin{tabular}{@{}lccccc@{}}
\toprule
  & \textbf{SIQA} & \textbf{PIQA} & \textbf{WinoGrande} & \textbf{CommonsenseQA} & \textbf{Average} \\ \midrule
Uniform                 & 39.36 & 67.46 & 51.70 & 19.16 & 44.42 \\ \midrule
RegMix-domain          & 40.12 & 70.08 & 51.62 & 21.13 & 45.74\\ 
RegMix-topic           & 40.74 & 69.53 & 52.17 & 21.38 & 45.96 \\ 
Temperature-domain    & 39.46 & 67.79 & 53.83 & 20.80 & 45.47\\ 
Temperature-topic     & 40.43 & 68.50 & 52.57 & 18.35 & 44.96 \\ \midrule
\textbf{$\downarrow$ Entertainment}     & 39.92 & 68.44 & 52.09 & 22.28 & 45.68 \\
\textbf{$\uparrow$ \{Science\}} & 38.69 & 66.81 & 51.78 & 20.39 & 44.42 \\
\textbf{$\uparrow$ \{Science,Relationships,Health\}}  & 40.07 & 69.53 & 52.96 & 22.36 & 46.23\\ \bottomrule 
\end{tabular}
}
\caption{Full downstream tasks results of data mixing in Commonsense Reasoning.}
\label{tab:fullmixing_2}
\end{table*}


\begin{table*}[!tb]
\small
\centering
\setlength{\tabcolsep}{1mm}{
\begin{tabular}{@{}lccccc@{}}
\toprule
  & \textbf{RACE} & \textbf{OpenbookQA}  & \textbf{Average} \\ \midrule
Uniform                 & 21.34 & 28.80 & 25.07 \\ \midrule
RegMix-domain          & 20.96 & 29.80 & 25.38\\ 
RegMix-topic           & 23.16 & 29.40 & 26.28 \\ 
Temperature-domain    & 22.11 & 29.40 & 25.76\\ 
Temperature-topic     & 24.11 & 31.20 & 27.66 \\ \midrule
\textbf{$\downarrow$ Entertainment}      & 21.72 & 30.60 & 26.16 \\ 
\textbf{$\uparrow$ \{Science\}} & 22.68 & 31 & 26.84 \\
\textbf{$\uparrow$ \{Science,Relationships,Health\}}    & 21.44 & 31.60 & 26.52\\ \bottomrule
\end{tabular}
}
\caption{Full downstream tasks results of data mixing in Reading Comprehension.}
\label{tab:fullmixing_3}
\end{table*}


\section{More Examples}
\label{app:dataweave_fullresult}
To better showcase the results of topics generated through clustering methods, we have selected some examples that hit GPT's Top-1, Top-3 preferences, and those that did not hit the Top-3, for demonstration.


\begin{figure}[h]
\textbf{Text}:If you always end up going to night training, you could use a front light. We give you a few tips to choose it and we present several models.\\
LED: They have been imposed as an option in front of the old incandescent light bulb and offer a very bright white light.They are light-emitting diodes, are more compact, offer less energy consumption, longer life time and good resistance to vibrations.\\
When you buy a frontal look at the estimated reach of the same in meters depending on the activity you are going to do with it.Running you need less advance information than on a bicycle, for the simple matter of the speed at which you move.\\
It evaluates weight and volume according to the hours that you are going to be using it.It is not the same to run with a front 50 minutes to do it for 5 hours, the weight and its capacity to adapt it to your head, the helmet of the bike, etc.It will be determinate to choose one or the other.\\
Look at the details: Is it ready to use with rain (the bulb and the battery compartment is wa

\textbf{GPT Preference}: \textbf{Technology}, Lifestyle, Health, Others, Education, Community, Science, Finance, Politics, Law, Relationships, Entertainment

\textbf{Clustering Topic}: Technology
\caption{First case with hitting Top-1 GPT Preference.}
\end{figure}


\begin{figure}[h]
\textbf{Text}:Q: Disabling and enabling button submit based on radio input conditions I would like to disable an input field from being click-able if user hasn't selected a radio button. \\
Here is the simple HTML form:\\
<form method="POST">\\
    <input type='radio' name='a' value='a' id='checkMe' /> a \\
    <input type='radio' name='a' value='b' id='checkMe' /> b  \\
    <input type='radio' name='a' value='c' id='checkMe'  /> c    \\
    <input type='submit' value='choose' id='choose' disabled="disabled"/>    \\
</form>\\
\\
Now, I made this js, to see if one of the inputs is selected, then the disabled="disabled" part should be revered, but that is now the case in this JavaScript code \\
if(document.getElementById('checkMe').checked) {\\
   document.getElementById('choose').disabled=false;\\
}  \\
\\
Here is the online demo. http://jsfiddle.net/2HC6s/\\
\\
A: Try this | demo\\
<form method="POST" id="question">\\
    <input type='radio' name='a' value='a' id='checkMe' onclick="check()"/> a \\
    <input type='radio' name='a' value='b' id='checkMe1' onclic
    
\textbf{GPT Preference}: \textbf{Technology}, Education, Others, Lifestyle, Science, Community, Health, Finance, Politics, Law, Relationships, Entertainment

\textbf{Clustering Topic}: Technology
\caption{Second case with hitting Top-1 GPT Preference.}
\end{figure}

\begin{figure}
\textbf{Text}: CDs by various local artists, Art Glass by Jeri Danzig, handmade Christmas Cards \& labels by Holly Wayman.\\
Vital Signs hand block printed tops for adults and baby tees, sweatshirts and onesies.
Box interiors include plush satin, gold leaf and shadow box trinkets.\\
A box disguised as a book, two Men's dresser boxes and the little one with the insides show above.\\
Leah Crosby – Upcycled Bicycle Tire Earrings!
Color Photographs of street scenes over a number of decades.\\
Mixed media collage and paint, on canvas and wood panel.\\
Handmade jewelry made from recycled bicycle tires.\\
Color Photographs of the Vineyard.\\
Lathed bowls, plates, and gift boxes made from a variety of Vineyard woods. \\
\textbf{GPT Preference}: Entertainment, \textbf{Lifestyle}, Community, Art, Others, Technology, Education, Health, Science, Politics, Law, Financ"   \\
\textbf{Clustering Topic}: Lifestyle
\caption{First case with hitting Top-3 GPT Preference.}
\end{figure}

\begin{figure}
\textbf{Text}: OPEC and Allies Are Said to Have Already Cleared Oil Surplus \\
May 28, 2018 EnergyNow Media \\
May 27 by Wael Mahdi and Grant Smith \\
OPEC and allied oil producers including Russia concluded that the crude market re-balanced in April, when their output cuts achieved a key goal of eliminating the global surplus. \\
The excess in oil inventories, which has weighed on prices for three years, plunged in April to less than the five-year average for stockpiles in developed nations, according to people with knowledge of the data assessed at the meeting of the Joint Technical Committee of OPEC and other producers last week in Jeddah, Saudi Arabia. \\
The re-balance is sure to be the focus of a tense meeting between OPEC and its partners in the production cuts when they meet in Vienna next month. Top producers Saudi Arabia and Russia announced last week that the suppliers may boost output in the second half of the year. The trouble is, officials from several countries in the agreement, both inside OPEC and outside, said they disap 

\textbf{GPT Preference}: Politics, \textbf{Finance}, Community, Technology, Science, Health, Others, Lifestyle, Education, Relationships, Law, Entertainment

\textbf{Clustering Topic}: Finance
\caption{Second case with hitting Top-3 GPT Preference.}
\end{figure}

\begin{figure}
\textbf{Text}:Home/May Court History\\
The May Court Club of Oakville is part of The Association of May Court Clubs of Canada, the first service club in Canada and founded in 1898 on the eve of May Day in Ottawa, Ontario. May Court's founder, Lady Isabel Aberdeen, the wife of Canada's then Governor General, was a truly extraordinary woman. Lady Aberdeen had strong ideas about the role of women in society. More importantly she put her ideas into action. She was the founder of the Council of Women and the Victorian Order of Nurses (VON). To support these endeavours, she then founded a women's service club, The May Court Club. Over a century later, May Court Clubs have grown to more than 1,500 volunteer women located in nine Ontario cities driven by the same spirit and passion for making a difference in the communities they serve.

\textbf{GPT Preference}: Community, History, Lifestyle, \textbf{Politics}, Education, Health, Entertainment, Science, Technology, Finance, Law, Others

\textbf{Clustering Topic}: Politics
\caption{First case without hitting Top-3 GPT Preference.}
\end{figure}

\begin{figure}
\textbf{Text}:With a national tour footprint consisting of 18 live events on the schedule for 2019, Goodguys Giant Car Shows are a great way to expose your company's products and services to the loyal and affluent Goodguys marketplace. Our all-inclusive "You Gotta Drive 'Em" event culture brings together classic hot rodders and late model automotive enthusiasts from all walks of life in a face-to-face, fun and entertaining festival environment that has been delivering ROI for our business partners since 1983!\\
Don't hesitate, reserve your spot in a show near you today.

\textbf{GPT Preference}: Entertainment, Community, Lifestyle, Technology, \textbf{Finance}, Others, Education, Health, Politics, Science, Relationships, Law

\textbf{Clustering Topic}: Finance
\caption{Second case without hitting Top-3 GPT Preference.}
\end{figure}


\section{Topic Classifier Details}
\label{app:topic_classifier}
The training dataset for topic classifier is derived from a subset of SlimPajama, comprising a total of 100,000 samples, which were divided into training, development, and test sets in a ratio of 8:1:1. 
The training process required approximately 8 GPU hours. 
Upon completion of the training, the topic classifier attained an accuracy score of 84\% on the test set.


\end{document}
