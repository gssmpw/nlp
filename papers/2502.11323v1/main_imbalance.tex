\special{dvipdfmx:config z 0}

\documentclass[11pt]{article}


%
\setlength\unitlength{1mm}
\newcommand{\twodots}{\mathinner {\ldotp \ldotp}}
% bb font symbols
\newcommand{\Rho}{\mathrm{P}}
\newcommand{\Tau}{\mathrm{T}}

\newfont{\bbb}{msbm10 scaled 700}
\newcommand{\CCC}{\mbox{\bbb C}}

\newfont{\bb}{msbm10 scaled 1100}
\newcommand{\CC}{\mbox{\bb C}}
\newcommand{\PP}{\mbox{\bb P}}
\newcommand{\RR}{\mbox{\bb R}}
\newcommand{\QQ}{\mbox{\bb Q}}
\newcommand{\ZZ}{\mbox{\bb Z}}
\newcommand{\FF}{\mbox{\bb F}}
\newcommand{\GG}{\mbox{\bb G}}
\newcommand{\EE}{\mbox{\bb E}}
\newcommand{\NN}{\mbox{\bb N}}
\newcommand{\KK}{\mbox{\bb K}}
\newcommand{\HH}{\mbox{\bb H}}
\newcommand{\SSS}{\mbox{\bb S}}
\newcommand{\UU}{\mbox{\bb U}}
\newcommand{\VV}{\mbox{\bb V}}


\newcommand{\yy}{\mathbbm{y}}
\newcommand{\xx}{\mathbbm{x}}
\newcommand{\zz}{\mathbbm{z}}
\newcommand{\sss}{\mathbbm{s}}
\newcommand{\rr}{\mathbbm{r}}
\newcommand{\pp}{\mathbbm{p}}
\newcommand{\qq}{\mathbbm{q}}
\newcommand{\ww}{\mathbbm{w}}
\newcommand{\hh}{\mathbbm{h}}
\newcommand{\vvv}{\mathbbm{v}}

% Vectors

\newcommand{\av}{{\bf a}}
\newcommand{\bv}{{\bf b}}
\newcommand{\cv}{{\bf c}}
\newcommand{\dv}{{\bf d}}
\newcommand{\ev}{{\bf e}}
\newcommand{\fv}{{\bf f}}
\newcommand{\gv}{{\bf g}}
\newcommand{\hv}{{\bf h}}
\newcommand{\iv}{{\bf i}}
\newcommand{\jv}{{\bf j}}
\newcommand{\kv}{{\bf k}}
\newcommand{\lv}{{\bf l}}
\newcommand{\mv}{{\bf m}}
\newcommand{\nv}{{\bf n}}
\newcommand{\ov}{{\bf o}}
\newcommand{\pv}{{\bf p}}
\newcommand{\qv}{{\bf q}}
\newcommand{\rv}{{\bf r}}
\newcommand{\sv}{{\bf s}}
\newcommand{\tv}{{\bf t}}
\newcommand{\uv}{{\bf u}}
\newcommand{\wv}{{\bf w}}
\newcommand{\vv}{{\bf v}}
\newcommand{\xv}{{\bf x}}
\newcommand{\yv}{{\bf y}}
\newcommand{\zv}{{\bf z}}
\newcommand{\zerov}{{\bf 0}}
\newcommand{\onev}{{\bf 1}}

% Matrices

\newcommand{\Am}{{\bf A}}
\newcommand{\Bm}{{\bf B}}
\newcommand{\Cm}{{\bf C}}
\newcommand{\Dm}{{\bf D}}
\newcommand{\Em}{{\bf E}}
\newcommand{\Fm}{{\bf F}}
\newcommand{\Gm}{{\bf G}}
\newcommand{\Hm}{{\bf H}}
\newcommand{\Id}{{\bf I}}
\newcommand{\Jm}{{\bf J}}
\newcommand{\Km}{{\bf K}}
\newcommand{\Lm}{{\bf L}}
\newcommand{\Mm}{{\bf M}}
\newcommand{\Nm}{{\bf N}}
\newcommand{\Om}{{\bf O}}
\newcommand{\Pm}{{\bf P}}
\newcommand{\Qm}{{\bf Q}}
\newcommand{\Rm}{{\bf R}}
\newcommand{\Sm}{{\bf S}}
\newcommand{\Tm}{{\bf T}}
\newcommand{\Um}{{\bf U}}
\newcommand{\Wm}{{\bf W}}
\newcommand{\Vm}{{\bf V}}
\newcommand{\Xm}{{\bf X}}
\newcommand{\Ym}{{\bf Y}}
\newcommand{\Zm}{{\bf Z}}

% Calligraphic

\newcommand{\Ac}{{\cal A}}
\newcommand{\Bc}{{\cal B}}
\newcommand{\Cc}{{\cal C}}
\newcommand{\Dc}{{\cal D}}
\newcommand{\Ec}{{\cal E}}
\newcommand{\Fc}{{\cal F}}
\newcommand{\Gc}{{\cal G}}
\newcommand{\Hc}{{\cal H}}
\newcommand{\Ic}{{\cal I}}
\newcommand{\Jc}{{\cal J}}
\newcommand{\Kc}{{\cal K}}
\newcommand{\Lc}{{\cal L}}
\newcommand{\Mc}{{\cal M}}
\newcommand{\Nc}{{\cal N}}
\newcommand{\nc}{{\cal n}}
\newcommand{\Oc}{{\cal O}}
\newcommand{\Pc}{{\cal P}}
\newcommand{\Qc}{{\cal Q}}
\newcommand{\Rc}{{\cal R}}
\newcommand{\Sc}{{\cal S}}
\newcommand{\Tc}{{\cal T}}
\newcommand{\Uc}{{\cal U}}
\newcommand{\Wc}{{\cal W}}
\newcommand{\Vc}{{\cal V}}
\newcommand{\Xc}{{\cal X}}
\newcommand{\Yc}{{\cal Y}}
\newcommand{\Zc}{{\cal Z}}

% Bold greek letters

\newcommand{\alphav}{\hbox{\boldmath$\alpha$}}
\newcommand{\betav}{\hbox{\boldmath$\beta$}}
\newcommand{\gammav}{\hbox{\boldmath$\gamma$}}
\newcommand{\deltav}{\hbox{\boldmath$\delta$}}
\newcommand{\etav}{\hbox{\boldmath$\eta$}}
\newcommand{\lambdav}{\hbox{\boldmath$\lambda$}}
\newcommand{\epsilonv}{\hbox{\boldmath$\epsilon$}}
\newcommand{\nuv}{\hbox{\boldmath$\nu$}}
\newcommand{\muv}{\hbox{\boldmath$\mu$}}
\newcommand{\zetav}{\hbox{\boldmath$\zeta$}}
\newcommand{\phiv}{\hbox{\boldmath$\phi$}}
\newcommand{\psiv}{\hbox{\boldmath$\psi$}}
\newcommand{\thetav}{\hbox{\boldmath$\theta$}}
\newcommand{\tauv}{\hbox{\boldmath$\tau$}}
\newcommand{\omegav}{\hbox{\boldmath$\omega$}}
\newcommand{\xiv}{\hbox{\boldmath$\xi$}}
\newcommand{\sigmav}{\hbox{\boldmath$\sigma$}}
\newcommand{\piv}{\hbox{\boldmath$\pi$}}
\newcommand{\rhov}{\hbox{\boldmath$\rho$}}
\newcommand{\upsilonv}{\hbox{\boldmath$\upsilon$}}

\newcommand{\Gammam}{\hbox{\boldmath$\Gamma$}}
\newcommand{\Lambdam}{\hbox{\boldmath$\Lambda$}}
\newcommand{\Deltam}{\hbox{\boldmath$\Delta$}}
\newcommand{\Sigmam}{\hbox{\boldmath$\Sigma$}}
\newcommand{\Phim}{\hbox{\boldmath$\Phi$}}
\newcommand{\Pim}{\hbox{\boldmath$\Pi$}}
\newcommand{\Psim}{\hbox{\boldmath$\Psi$}}
\newcommand{\Thetam}{\hbox{\boldmath$\Theta$}}
\newcommand{\Omegam}{\hbox{\boldmath$\Omega$}}
\newcommand{\Xim}{\hbox{\boldmath$\Xi$}}


% Sans Serif small case

\newcommand{\Gsf}{{\sf G}}

\newcommand{\asf}{{\sf a}}
\newcommand{\bsf}{{\sf b}}
\newcommand{\csf}{{\sf c}}
\newcommand{\dsf}{{\sf d}}
\newcommand{\esf}{{\sf e}}
\newcommand{\fsf}{{\sf f}}
\newcommand{\gsf}{{\sf g}}
\newcommand{\hsf}{{\sf h}}
\newcommand{\isf}{{\sf i}}
\newcommand{\jsf}{{\sf j}}
\newcommand{\ksf}{{\sf k}}
\newcommand{\lsf}{{\sf l}}
\newcommand{\msf}{{\sf m}}
\newcommand{\nsf}{{\sf n}}
\newcommand{\osf}{{\sf o}}
\newcommand{\psf}{{\sf p}}
\newcommand{\qsf}{{\sf q}}
\newcommand{\rsf}{{\sf r}}
\newcommand{\ssf}{{\sf s}}
\newcommand{\tsf}{{\sf t}}
\newcommand{\usf}{{\sf u}}
\newcommand{\wsf}{{\sf w}}
\newcommand{\vsf}{{\sf v}}
\newcommand{\xsf}{{\sf x}}
\newcommand{\ysf}{{\sf y}}
\newcommand{\zsf}{{\sf z}}


% mixed symbols

\newcommand{\sinc}{{\hbox{sinc}}}
\newcommand{\diag}{{\hbox{diag}}}
\renewcommand{\det}{{\hbox{det}}}
\newcommand{\trace}{{\hbox{tr}}}
\newcommand{\sign}{{\hbox{sign}}}
\renewcommand{\arg}{{\hbox{arg}}}
\newcommand{\var}{{\hbox{var}}}
\newcommand{\cov}{{\hbox{cov}}}
\newcommand{\Ei}{{\rm E}_{\rm i}}
\renewcommand{\Re}{{\rm Re}}
\renewcommand{\Im}{{\rm Im}}
\newcommand{\eqdef}{\stackrel{\Delta}{=}}
\newcommand{\defines}{{\,\,\stackrel{\scriptscriptstyle \bigtriangleup}{=}\,\,}}
\newcommand{\<}{\left\langle}
\renewcommand{\>}{\right\rangle}
\newcommand{\herm}{{\sf H}}
\newcommand{\trasp}{{\sf T}}
\newcommand{\transp}{{\sf T}}
\renewcommand{\vec}{{\rm vec}}
\newcommand{\Psf}{{\sf P}}
\newcommand{\SINR}{{\sf SINR}}
\newcommand{\SNR}{{\sf SNR}}
\newcommand{\MMSE}{{\sf MMSE}}
\newcommand{\REF}{{\RED [REF]}}

% Markov chain
\usepackage{stmaryrd} % for \mkv 
\newcommand{\mkv}{-\!\!\!\!\minuso\!\!\!\!-}

% Colors

\newcommand{\RED}{\color[rgb]{1.00,0.10,0.10}}
\newcommand{\BLUE}{\color[rgb]{0,0,0.90}}
\newcommand{\GREEN}{\color[rgb]{0,0.80,0.20}}

%%%%%%%%%%%%%%%%%%%%%%%%%%%%%%%%%%%%%%%%%%
\usepackage{hyperref}
\hypersetup{
    bookmarks=true,         % show bookmarks bar?
    unicode=false,          % non-Latin characters in AcrobatÕs bookmarks
    pdftoolbar=true,        % show AcrobatÕs toolbar?
    pdfmenubar=true,        % show AcrobatÕs menu?
    pdffitwindow=false,     % window fit to page when opened
    pdfstartview={FitH},    % fits the width of the page to the window
%    pdftitle={My title},    % title
%    pdfauthor={Author},     % author
%    pdfsubject={Subject},   % subject of the document
%    pdfcreator={Creator},   % creator of the document
%    pdfproducer={Producer}, % producer of the document
%    pdfkeywords={keyword1} {key2} {key3}, % list of keywords
    pdfnewwindow=true,      % links in new window
    colorlinks=true,       % false: boxed links; true: colored links
    linkcolor=red,          % color of internal links (change box color with linkbordercolor)
    citecolor=green,        % color of links to bibliography
    filecolor=blue,      % color of file links
    urlcolor=blue           % color of external links
}
%%%%%%%%%%%%%%%%%%%%%%%%%%%%%%%%%%%%%%%%%%%


\usepackage{mathrsfs}
\usepackage{setspace} \singlespacing
\allowdisplaybreaks
%\usepackage{todonotes}
\usepackage{comment}
%\usepackage[inline]{showlabels}
\usepackage[normalem]{ulem}
\usepackage{mlmodern}
% Fix a issue in lmodern
% https://tex.stackexchange.com/questions/137141/strange-behaviour-of-binomial-coefficients-delimiters
\DeclareSymbolFont{largesymbols}{OMX}{cmex}{m}{n}

\usepackage{etoc} % separate TOC


\newcommand\KZ[1]{\textcolor{teal}{[KZ: #1]}}

\newcommand{\Err}{\mathrm{Err}}
\providecommand{\keywords}[1]
{
  {
  \small	
  \textbf{\textit{Keywords---}} #1
  }
}

\begin{document}


\title{A statistical theory of overfitting for imbalanced classification}
 
% old title
% \title{A modern statistical theory for imbalanced binary classification} 

%%%%% Alternative titles

%\title{The impact of overfitting in logistic regression with data imbalance}

%\title{A modern statistical theory for imbalanced binary classification}

%\title{The impact of overfitting on classification with data imbalance}

%"A Statistical Framework for Understanding Overfitting in Imbalanced Classification"

%"Theoretical Insights into Overfitting in Imbalanced Classification"

%"A Statistical Theory of Overfitting in Imbalanced Classification Models"

%"Exploring Overfitting in Imbalanced Classification: A Statistical Perspective"

%"Towards a Statistical Theory of Overfitting in Imbalanced Classification"


\author{Jingyang Lyu, Kangjie Zhou, and Yiqiao Zhong}
\author{Jingyang Lyu\thanks{Department of Statistics, University of Wisconsin--Madison, Madison, WI 53706, USA. Emails:
\texttt{jlyu55@wisc.edu}, \texttt{yiqiao.zhong@wisc.edu}} \and Kangjie Zhou\thanks{Department of Statistics, Columbia University, New York, NY 10027, USA. Email: \texttt{kz2326@columbia.edu}}
\and Yiqiao Zhong\footnotemark[1]}
\date{\today}


\pagenumbering{arabic}
\maketitle

\begin{abstract}
Classification with imbalanced data is a common challenge in data analysis, where certain classes (minority classes) account for a small fraction of the training data compared with other classes (majority classes). Classical statistical theory based on large-sample asymptotics and finite-sample corrections is often ineffective for high-dimensional data, leaving many overfitting phenomena in empirical machine learning unexplained.
%, which are ubiquitous in modern data analysis. %Indeed, in single-cell omics and deep learning, the feature vectors are often of hundreds or thousands dimensions. 
%Despite recent advances in high-dimensional asymptotics, overfitting in imbalanced classification is not well understood. For example, it is unclear why the logit distribution appear distorted, and why the accuracy of the minority class is more severely affected by overfitting. 

In this paper, we develop a statistical theory for high-dimensional imbalanced classification by investigating support vector machines and logistic regression. We find that dimensionality induces truncation or skewing effects on the logit distribution, which we characterize via a variational problem under high-dimensional asymptotics. In particular, for linearly separable data generated from a two-component Gaussian mixture model, the logits from each class follow a normal distribution $\normal(0,1)$ on the testing set, but asymptotically follow a rectified normal distribution $\max\{\kappa, \normal(0,1)\}$ on the training set---which is a pervasive phenomenon we verified on tabular data, image data, and text data. This phenomenon explains why the minority class is more severely affected by overfitting. Further, we show that margin rebalancing, which incorporates class sizes into the loss function, is crucial for mitigating the accuracy drop for the minority class. Our theory also provides insights into the effects of overfitting on calibration and other uncertain quantification measures.

%Our findings provide novel theoretical explanations for many empirical observations. Based on our theory, we propose practical applications of our analysis in model calibration and feature visualization.

%We characterize the truncation or skewing effects induced by dimensionality on the logit distribution via a variational problem

%on the training data; in particular, for linearly separable data generated from two-component Gaussian mixture model, the logits from each class are distributed as a normal distribution $N(0,1)$ on the test set, but as truncated distribution $\max\{\kappa, N(0,1)\}$ on the training set. 

%leading to overfitting that is more severe for the minority class---which is a pervasive phenomenon we verified on tabular data, image data, and text data. In particular, we characterize overfitting as a variational problem under the high-dimensional asymptotics where the sample size and dimension grow at a proportional ratio. 

%Further, we show that margin rebalancing, which incorporates class size into the loss function, is crucial for remedying the accuracy drop for the minority class, thereby providing theoretical foundation for empirical heuristics. Based on the theory, we suggest the practical implication of our analysis in model calibration and feature visualization.

\end{abstract}

\keywords{
    Imbalanced classification, overfitting, margin, logistic regression, support vector machine, overparametrization, calibration
}

% Main TOC
\etocdepthtag.toc{mtchapter}
\etocsettagdepth{mtchapter}{subsection}
\etocsettagdepth{mtappendix}{section}
% \setcounter{tocdepth}{2}
\tableofcontents

\newpage
\section{Introduction}


\begin{figure}[t]
\centering
\includegraphics[width=0.6\columnwidth]{figures/evaluation_desiderata_V5.pdf}
\vspace{-0.5cm}
\caption{\systemName is a platform for conducting realistic evaluations of code LLMs, collecting human preferences of coding models with real users, real tasks, and in realistic environments, aimed at addressing the limitations of existing evaluations.
}
\label{fig:motivation}
\end{figure}

\begin{figure*}[t]
\centering
\includegraphics[width=\textwidth]{figures/system_design_v2.png}
\caption{We introduce \systemName, a VSCode extension to collect human preferences of code directly in a developer's IDE. \systemName enables developers to use code completions from various models. The system comprises a) the interface in the user's IDE which presents paired completions to users (left), b) a sampling strategy that picks model pairs to reduce latency (right, top), and c) a prompting scheme that allows diverse LLMs to perform code completions with high fidelity.
Users can select between the top completion (green box) using \texttt{tab} or the bottom completion (blue box) using \texttt{shift+tab}.}
\label{fig:overview}
\end{figure*}

As model capabilities improve, large language models (LLMs) are increasingly integrated into user environments and workflows.
For example, software developers code with AI in integrated developer environments (IDEs)~\citep{peng2023impact}, doctors rely on notes generated through ambient listening~\citep{oberst2024science}, and lawyers consider case evidence identified by electronic discovery systems~\citep{yang2024beyond}.
Increasing deployment of models in productivity tools demands evaluation that more closely reflects real-world circumstances~\citep{hutchinson2022evaluation, saxon2024benchmarks, kapoor2024ai}.
While newer benchmarks and live platforms incorporate human feedback to capture real-world usage, they almost exclusively focus on evaluating LLMs in chat conversations~\citep{zheng2023judging,dubois2023alpacafarm,chiang2024chatbot, kirk2024the}.
Model evaluation must move beyond chat-based interactions and into specialized user environments.



 

In this work, we focus on evaluating LLM-based coding assistants. 
Despite the popularity of these tools---millions of developers use Github Copilot~\citep{Copilot}---existing
evaluations of the coding capabilities of new models exhibit multiple limitations (Figure~\ref{fig:motivation}, bottom).
Traditional ML benchmarks evaluate LLM capabilities by measuring how well a model can complete static, interview-style coding tasks~\citep{chen2021evaluating,austin2021program,jain2024livecodebench, white2024livebench} and lack \emph{real users}. 
User studies recruit real users to evaluate the effectiveness of LLMs as coding assistants, but are often limited to simple programming tasks as opposed to \emph{real tasks}~\citep{vaithilingam2022expectation,ross2023programmer, mozannar2024realhumaneval}.
Recent efforts to collect human feedback such as Chatbot Arena~\citep{chiang2024chatbot} are still removed from a \emph{realistic environment}, resulting in users and data that deviate from typical software development processes.
We introduce \systemName to address these limitations (Figure~\ref{fig:motivation}, top), and we describe our three main contributions below.


\textbf{We deploy \systemName in-the-wild to collect human preferences on code.} 
\systemName is a Visual Studio Code extension, collecting preferences directly in a developer's IDE within their actual workflow (Figure~\ref{fig:overview}).
\systemName provides developers with code completions, akin to the type of support provided by Github Copilot~\citep{Copilot}. 
Over the past 3 months, \systemName has served over~\completions suggestions from 10 state-of-the-art LLMs, 
gathering \sampleCount~votes from \userCount~users.
To collect user preferences,
\systemName presents a novel interface that shows users paired code completions from two different LLMs, which are determined based on a sampling strategy that aims to 
mitigate latency while preserving coverage across model comparisons.
Additionally, we devise a prompting scheme that allows a diverse set of models to perform code completions with high fidelity.
See Section~\ref{sec:system} and Section~\ref{sec:deployment} for details about system design and deployment respectively.



\textbf{We construct a leaderboard of user preferences and find notable differences from existing static benchmarks and human preference leaderboards.}
In general, we observe that smaller models seem to overperform in static benchmarks compared to our leaderboard, while performance among larger models is mixed (Section~\ref{sec:leaderboard_calculation}).
We attribute these differences to the fact that \systemName is exposed to users and tasks that differ drastically from code evaluations in the past. 
Our data spans 103 programming languages and 24 natural languages as well as a variety of real-world applications and code structures, while static benchmarks tend to focus on a specific programming and natural language and task (e.g. coding competition problems).
Additionally, while all of \systemName interactions contain code contexts and the majority involve infilling tasks, a much smaller fraction of Chatbot Arena's coding tasks contain code context, with infilling tasks appearing even more rarely. 
We analyze our data in depth in Section~\ref{subsec:comparison}.



\textbf{We derive new insights into user preferences of code by analyzing \systemName's diverse and distinct data distribution.}
We compare user preferences across different stratifications of input data (e.g., common versus rare languages) and observe which affect observed preferences most (Section~\ref{sec:analysis}).
For example, while user preferences stay relatively consistent across various programming languages, they differ drastically between different task categories (e.g. frontend/backend versus algorithm design).
We also observe variations in user preference due to different features related to code structure 
(e.g., context length and completion patterns).
We open-source \systemName and release a curated subset of code contexts.
Altogether, our results highlight the necessity of model evaluation in realistic and domain-specific settings.





\putsec{related}{Related Work}

\noindent \textbf{Efficient Radiance Field Rendering.}
%
The introduction of Neural Radiance Fields (NeRF)~\cite{mil:sri20} has
generated significant interest in efficient 3D scene representation and
rendering for radiance fields.
%
Over the past years, there has been a large amount of research aimed at
accelerating NeRFs through algorithmic or software
optimizations~\cite{mul:eva22,fri:yu22,che:fun23,sun:sun22}, and the
development of hardware
accelerators~\cite{lee:cho23,li:li23,son:wen23,mub:kan23,fen:liu24}.
%
The state-of-the-art method, 3D Gaussian splatting~\cite{ker:kop23}, has
further fueled interest in accelerating radiance field
rendering~\cite{rad:ste24,lee:lee24,nie:stu24,lee:rho24,ham:mel24} as it
employs rasterization primitives that can be rendered much faster than NeRFs.
%
However, previous research focused on software graphics rendering on
programmable cores or building dedicated hardware accelerators. In contrast,
\name{} investigates the potential of efficient radiance field rendering while
utilizing fixed-function units in graphics hardware.
%
To our knowledge, this is the first work that assesses the performance
implications of rendering Gaussian-based radiance fields on the hardware
graphics pipeline with software and hardware optimizations.

%%%%%%%%%%%%%%%%%%%%%%%%%%%%%%%%%%%%%%%%%%%%%%%%%%%%%%%%%%%%%%%%%%%%%%%%%%
\myparagraph{Enhancing Graphics Rendering Hardware.}
%
The performance advantage of executing graphics rendering on either
programmable shader cores or fixed-function units varies depending on the
rendering methods and hardware designs.
%
Previous studies have explored the performance implication of graphics hardware
design by developing simulation infrastructures for graphics
workloads~\cite{bar:gon06,gub:aam19,tin:sax23,arn:par13}.
%
Additionally, several studies have aimed to improve the performance of
special-purpose hardware such as ray tracing units in graphics
hardware~\cite{cho:now23,liu:cha21} and proposed hardware accelerators for
graphics applications~\cite{lu:hua17,ram:gri09}.
%
In contrast to these works, which primarily evaluate traditional graphics
workloads, our work focuses on improving the performance of volume rendering
workloads, such as Gaussian splatting, which require blending a huge number of
fragments per pixel.

%%%%%%%%%%%%%%%%%%%%%%%%%%%%%%%%%%%%%%%%%%%%%%%%%%%%%%%%%%%%%%%%%%%%%%%%%%
%
In the context of multi-sample anti-aliasing, prior work proposed reducing the
amount of redundant shading by merging fragments from adjacent triangles in a
mesh at the quad granularity~\cite{fat:bou10}.
%
While both our work and quad-fragment merging (QFM)~\cite{fat:bou10} aim to
reduce operations by merging quads, our proposed technique differs from QFM in
many aspects.
%
Our method aims to blend \emph{overlapping primitives} along the depth
direction and applies to quads from any primitive. In contrast, QFM merges quad
fragments from small (e.g., pixel-sized) triangles that \emph{share} an edge
(i.e., \emph{connected}, \emph{non-overlapping} triangles).
%
As such, QFM is not applicable to the scenes consisting of a number of
unconnected transparent triangles, such as those in 3D Gaussian splatting.
%
In addition, our method computes the \emph{exact} color for each pixel by
offloading blending operations from ROPs to shader units, whereas QFM
\emph{approximates} pixel colors by using the color from one triangle when
multiple triangles are merged into a single quad.


\newcommand{\tabincell}[2]{\begin{tabular}{@{}#1@{}}#2\end{tabular}}
\newcommand{\rowstyle}[1]{\gdef\currentrowstyle{#1}%
	#1\ignorespaces
}

\newcommand{\className}[1]{\textbf{\sf #1}}
\newcommand{\functionName}[1]{\textbf{\sf #1}}
\newcommand{\objectName}[1]{\textbf{\sf #1}}
\newcommand{\annotation}[1]{\textbf{#1}}
\newcommand{\todo}[1]{\textcolor{blue}{\textbf{[[TODO: #1]]}}}
\newcommand{\change}[1]{\textcolor{blue}{#1}}
\newcommand{\fetch}[1]{\textbf{\em #1}}
\newcommand{\phead}[1]{\vspace{1mm} \noindent {\bf #1}}
\newcommand{\wei}[1]{\textcolor{blue}{{\it [Wei says: #1]}}}
\newcommand{\peter}[1]{\textcolor{red}{{\it [Peter says: #1]}}}
\newcommand{\zhenhao}[1]{\textcolor{dkblue}{{\it [Zhenhao says: #1]}}}
\newcommand{\feng}[1]{\textcolor{magenta}{{\it [Feng says: #1]}}}
\newcommand{\jinqiu}[1]{\textcolor{red}{{\it [Jinqiu says: #1]}}}
\newcommand{\shouvick}[1]{\textcolor{violet(ryb)}{{\it [Shouvick says: #1]}}}
\newcommand{\pattern}[1]{\emph{#1}}
%\newcommand{\tool}{{{DectGUILag}}\xspace}
\newcommand{\tool}{{{GUIWatcher}}\xspace}


\newcommand{\guo}[1]{\textcolor{yellow}{{\it [Linqiang says: #1]}}}

\newcommand{\rqbox}[1]{\begin{tcolorbox}[left=4pt,right=4pt,top=4pt,bottom=4pt,colback=gray!5,colframe=gray!40!black,before skip=2pt,after skip=2pt]#1\end{tcolorbox}}

\section{Precise asymptotics of empirical logit distribution} \label{sec:logit}


In this section, we present our main results on the asymptotics of empirical logit distribution introduced in \cref{subsec:ELD}. Recall that data $\{(\xx_i, y_i)\}_{i = 1}^n$ are i.i.d.~generated from a 2-GMM \cref{model}, i.e., $\xx_i \,|\, y_i \sim \normal(y_i \bmu, \bI_d)$, with label distribution $P_y: \P(y_i = +1) = 1 - \P(y_i = -1) = \pi \in (0, \frac12]$. We consider proportional asymptotics where $n,d \to \infty$ and $n/d \to \delta$ with $\delta \in (0,\infty)$. Based on relations between $\bmu, \pi, \delta$, we will consider linearly separable data (fitted by SVM) and non-separable data (fitted by logistic regression) separately.

We define the following functions $\delta^*: \R \to \R_{\ge 0}$ and $H_\kappa: [-1, 1] \times \R \to \R_{\ge 0}$ that are related to the critical threshold of data separability:
\begin{equation}\label{eq:sep_functions}
     \delta^*(\kappa) := \max_{ \rho \in [-1, 1] , \beta_0 \in \R }  H_\kappa(\rho, \beta_0),
     \qquad
     H_\kappa(\rho, \beta_0) := \frac{1 - \rho^2}{\E\left[ \bigl(  s(Y) \kappa - \rho \norm{\bmu}_2 + G - \beta_0 Y \bigr)_+^2 \right]},
\end{equation}
where $(Y, G) \sim P_y \times \normal(0,1)$ and 
% $s: \{ \pm 1 \} \to \{ \tau, 1 \}$ is the logit-adjusting function
\begin{equation}\label{eq:s_fun}
     s(y) := \begin{cases} \ \tau , & \ \text{if} \ y = + 1, \\
        \ 1, & \ \text{if} \ y = -1. \end{cases}
\end{equation}
We will show in \cref{thm:SVM_main} that the relationship between $\delta$ and $\delta^*(0)$ determines separability, where $\delta^*(0)$ does not depend on $\tau$ by definition.

We summarize the asymptotics of logit distribution for both separable and non-separable case in \cref{tab:ELD}, which is the main contribution of our theoretical results (\cref{thm:SVM_main} and \ref{thm:logistic_main}).
\begin{table}[h!]
\begin{equation*}
\renewcommand{\arraystretch}{1.2}
    \begin{array}{rll}
    \hline
           &  \textbf{limiting ELD} ~ (\hat\nu_*)  &  \textbf{cause for overfitting} ~ (\xi^*)   \\
    \hline
      \text{separable data}
          &  \Law\left( Y, \,  Y \max \{ \kappa^*, \mathtt{LOGITS} \} \right)
          &  R^* \sqrt{1 - \rho^{*2}} \xi^* = \left( \kappa^* - \mathtt{LOGITS} \right)_+  \\
      \text{non-separable data}
          &  \Law\left( Y, \,  Y \, \prox_{ \lambda^* \ell}( \mathtt{LOGITS} ) \right)
          &   R^* \sqrt{1 - \rho^{*2}} \xi^*  =  
          % -\lambda^* \ell' \bigl( \prox_{ \lambda^* \ell}( \mathtt{LOGITS} ) \bigr) \\
            - \lambda^* \nabla \envelope_{ \lambda^* \ell}( \mathtt{LOGITS} ) \\
    \hline
          \textbf{limiting TLD} ~ (\hat\nu_*^\mathrm{test})
          &  \Law\left( Y, \,  Y \cdot \mathtt{LOGITS} \right)     &   \\
    \hline
    \multicolumn{3}{c}{\mathtt{LOGITS} := \rho^*\norm{\vmu}_2 R^* + R^*G + \beta_0^* Y 
    \quad \text{($R^* := 1$ in separable case)}
    }
    \end{array}
\end{equation*}
\vspace{-5mm}
\caption{Comparison of logit distributions on separable and non-separable data ($\tau = 1$).}
\label{tab:ELD}
\end{table}

\subsection{Separable data} \label{sec:logit_SVM}

For linearly separable data, recall the margin-rebalanced SVM in \cref{eq:SVM-m-reb} and \eqref{eq:SVM}. The following theorem summarizes the precise asymptotics of SVM under arbitrary $\tau$, including the limits of parameters, margin, and logit distribution. The proofs are deferred to the appendices.

Recall that data $\{(\xx_i, y_i)\}_{i = 1}^n$ are generated from 2-GMM with fixed parameters $\bmu \in \R^d$, $\pi \in (0, \frac12)$. Let $(\hat \vbeta_n, \hat \beta_{0, n})$ be an optimal solution to the margin-rebalanced SVM \cref{eq:SVM}, and let $\hat\kappa_n$ be the maximum margin as per \cref{def:max-margin}. Recall the cosine angle $\hat \rho_n:= \hat \rho$ between $\vmu$ and $\hat\vbeta_n$ defined in \cref{eq:rho_hat}. Let $\delta^*(\kappa)$ be defined as per \cref{eq:sep_functions}, and $\rho^*, \beta_0^*, \kappa^*, \xi^*$ be a solution to the variational problem
\begin{equation}\label{eq:SVM_variation}
    \begin{aligned}
        \begin{array}{cl}
            \underset{ \rho \in [-1, 1], \beta_0 \in \R, \kappa \in \R, \xi \in \cL^2  }{ \mathrm{maximize} } & \kappa, \\
            \underset{ \phantom{\smash{\bm\beta \in \R^d, \beta_0 \in \R, \kappa \in \R} } }{\text{subject to}} &  
            \rho \norm{\bmu}_2 + G + Y \beta_0 + \sqrt{1 - \rho^2} \xi \ge s(Y) \kappa,  
            \qquad \E[\xi^2]  \le  1/\delta .
        \end{array}
    \end{aligned}
    \end{equation}
    where $\cL^2$ is the space of all square integrable random variables in $(\Omega, \mathcal{F}, \P)$, and $(Y, G) \sim P_y \times \normal(0,1)$. 
    % The asymptotic distribution of logit margins on training set and test set are respectively defined as 
    %The limiting ELD and TLD are respectively defined as 
    We define
    \begin{equation*}
    \begin{aligned}
        % \cL_* & := \Law\left( 
        %   \max\bigl\{ \kappa^*,  s(Y)^{-1} ( \rho^*\norm{\vmu}_2 + G + Y \beta_0^* )  \bigr\}
        %  \right),
        % \\
        % \cL_*^\mathrm{test} & := \Law\left( 
        %  s(Y)^{-1} ( \rho^*\norm{\vmu}_2 + G + Y \beta_0^* ) 
        %  \right).
        \nu_* & := \Law \,\bigl( Y,  Y \max\{ s(Y)\kappa^*, \rho^* \| \bmu \| + G + Y \beta_0^* \} \bigr),  \\
        \nu^\mathrm{test}_* & := \Law \,\bigl( Y, Y (\rho^* \| \bmu \| + G + Y \beta_0^*) \bigr).
    \end{aligned}
    \end{equation*}
which we will prove to be the limiting ELD and TLD respectively.
\begin{thm}[Separable data] \label{thm:SVM_main}
    %Consider data $\{(\xx_i, y_i)\}_{i = 1}^n$ generated from 2-GMM with fixed parameters $\bmu \in \R^d$, $\pi \in (0, 1/2]$. Moreover, assume $n, d \to \infty$ with $n/d \to \delta \in (0, \infty)$. Fix $\tau \in (0, \infty)$.
    Assume $n, d \to \infty$ with $n/d \to \delta \in (0, \infty)$. Fix $\tau \in (0, \infty)$. 
    %Let $(\hat \vbeta_n, \hat \beta_{0, n})$ be an optimal solution to the margin-rebalanced SVM \cref{eq:SVM}, and let $\hat\kappa_n$ be the well-defined maximum margin as per \cref{def:max-margin}. Define the cosine angle between $\vmu$ and $\hat\vbeta_n$ as
    %\begin{equation}
    %    \label{eq:rho_hat}
    %    \hat\rho_n := \biggl\<  \frac{\hat\vbeta_n}{\| \hat\vbeta_n \|_2} , \frac{\vmu}{\norm{\vmu}_2} \biggr\>.
    %\end{equation}
    \begin{enumerate}[label=(\alph*)]
        \item \label{thm:SVM_main_trans}
        \textbf{(Phase transition)} With probability tending to one, the data is linearly separable if $\delta < \delta^*(0)$ and is not linearly separable if $\delta > \delta^*(0)$.

        \item \label{thm:SVM_main_var} 
        \textbf{(Variational problem)} In the separable regime $\delta < \delta^*(0)$, $(\rho^*, \beta_0^*, \kappa^*, \xi^*)$ is the unique solution to \cref{eq:SVM_variation} with $\rho^* \in (0, 1)$ (not depend on $\tau$), $\kappa^* > 0$, and the random variable $\xi^*$ satisfies (a.s.)
        \begin{equation}\label{eq:SVM_main_xi_star}
            \sqrt{1 - \rho^{*2}} \xi^* = \bigl( s(Y) \kappa^* - \rho^*\norm{\vmu}_2 - G - Y \beta_0^*) \bigr)_+.
        \end{equation}
        Moreover, $(\rho^*, \beta_0^*, \kappa^*)$ is also the unique solution to
        \begin{equation}\label{eq:SVM_asymp_simple}
        \begin{array}{rl}
        \maximize\limits_{\rho \in [-1, 1], \beta_0 \in \R, \kappa \in \R} & \kappa, \\
        \text{\emph{subject to}} & H_\kappa(\rho, \beta_0) \ge \delta
        \end{array}
        \end{equation}
        and $\kappa^* = \sup\left\{ \kappa \in \R: \delta^*(\kappa) \ge \delta \right\}$.
        
        \item \label{thm:SVM_main_mar} 
        \textbf{(Margin convergence)} In the separable regime $\delta < \delta^*(0)$,
        \begin{equation*}
            \hat\kappa_n \conL{2} \kappa^*.
        \end{equation*}
        In the non-separable regime $\delta > \delta^*(0)$ we have negative margin, i.e., with probability tending to one, for some $\overline{\kappa} > 0$,
        \begin{equation*}
            \max_{ \substack{ \norm{\bbeta}_2 = 1 \\ \beta_0 \in \R } } \min_{i \in [n]} \, \wt y_i ( \< \xx_i, \bbeta \> + \beta_0 )  \le  -\overline{\kappa}.
        \end{equation*}

        \item \label{thm:SVM_main_param} 
        \textbf{(Parameter convergence)} In the separable regime $\delta < \delta^*(0)$,
        \begin{equation*}
            \hat\rho_n \conp \rho^*,
            \qquad
            \hat\beta_{0,n} \conp \beta_0^*.
        \end{equation*}

        \item \label{thm:SVM_main_err}
        \textbf{(Asymptotic errors)} Recall the minority and majority test prediction errors , $\Err_{+,n}$ and $\Err_{-,n}$ respectively, of the max-margin classifier defined in \cref{eq:Err_n} (writing subscript $n$ for clarity). %as
        %\begin{equation}\label{eq:Err_n}
        %    \begin{aligned}
        %        \Err_{+,n} & := \P \left( \< \xx^\mathrm{new} , \hat\vbeta_n \> + \hat\beta_{0,n} < 0  \,\big|\, y^\mathrm{new} = +1 \right), \\
         %       \Err_{-,n} & := \P \left( \< \xx^\mathrm{new} , \hat\vbeta_n \> + \hat\beta_{0,n} \ge 0  \,\big|\, y^\mathrm{new} = -1 \right), 
         %   \end{aligned}
        %\end{equation}
        %where $(\xx^\mathrm{new}, y^\mathrm{new})$ is an i.i.d. test point. 
        Then in the separable regime $\delta < \delta^*(0)$,
        \begin{equation*}
            \Err_{+,n}  \to  \Phi \left(- \rho^* \norm{\bmu}_2 - \beta_0^* \right),
            \qquad
            \Err_{-,n}  \to  \Phi \left(- \rho^* \norm{\bmu}_2  + \beta_0^* \right).
        \end{equation*}
        \item \label{thm:SVM_main_logit}
        \textbf{(ELD/TLD convergence)} 
        % The empirical distribution of logit margins on training set and the distribution for a test point are defined as the following random measures, respectively:
        % \begin{equation}
        %     \hat \cL_{n} := \frac1n \sum_{i=1}^n \delta_{\wt y_i ( \< \xx_i, \hat\vbeta_n \> + \hat\beta_{0,n} )
        %     },
        %     \qquad
        %     \hat \cL_{n}^\mathrm{test} := \Law\left( \wt y^\mathrm{new} ( \< \xx^\mathrm{new}, \hat\vbeta_n \> + \hat\beta_{0,n} \right),
        % \end{equation}
        % where $\wt y^\mathrm{new} = y^\mathrm{new}/s(y^\mathrm{new})$ is the transformed label as \cref{eq:trans-labels}. 
        Recall the ELD $\hat\nu_n$ and TLD $\hat\nu_n^\mathrm{test}$ defined as per \cref{def:ELD_TLD}, where $\hat f(\xx) = \< \xx, \hat\vbeta_n \> + \hat\beta_{0, n}$.
        Then in the separable regime $\delta < \delta^*(0)$ we have logit convergence for both training and test data, i.e.,
        \begin{equation*}
            % W_2\bigl( \hat \cL_{n}, \cL_* \bigr) \conp 0,
            % \qquad
            % \hat\cL_{n}^\mathrm{test} \conw \cL_*^\mathrm{test}.
            W_2\bigl(\hat{\nu}_n , \nu_* \bigr) \conp 0,
            \qquad
            \hat{\nu}_n^\mathrm{test} \conw \nu^\mathrm{test}_*.
        \end{equation*}
    \end{enumerate}
\end{thm}

\begin{rem}
    By taking $\tau = 1$, the ELD convergence $W_2(\hat{\nu}_n , \nu_* ) \conp 0$ in \cref{thm:SVM}\ref{thm:SVM_c} is a consequence of \cref{thm:SVM_main}\ref{thm:SVM_main_logit}, and the TLD convergence $\hat{\nu}_n^\mathrm{test} \conw \nu^\mathrm{test}_*$ is a corollary of \cref{thm:SVM_main}\ref{thm:SVM_main_param}.
\end{rem}

As discussed in \cref{subsec:ELD}, random variable $\xi^*$ and the nonlinear transformation $\mathtt{T}^*(x) = \max\{x, \kappa^*\}$ therein characterize the effect of overfitting on logits. The following result provides an optimal transport perspective of this overfitting effect. For ease of description, we reformulate $\nu_*$ and $\nu^\mathrm{test}_*$ in terms of the following one-dimensional measures
\begin{equation*}
    \cL_*  := \Law \,\bigl(  \max\{ \kappa^*, \rho^* \| \bmu \| + G + Y \beta_0^* \} \bigr),  
        \qquad 
    \cL^\mathrm{test}_* := \Law \,\bigl(  \rho^* \| \bmu \| + G + Y \beta_0^* \bigr).
\end{equation*}

\begin{prop}[Optimal transport map]\label{prop:opt_transport}
    $\mathtt{T}^*(x) = \max\{\kappa^*, x\}$ is the unique optimal transport map from $\cL_*^\mathrm{test}$ to $\cL_*$ under the cost function $c(x, y) = h(x - y)$ for any strictly convex $h: \R^2 \to \R_{\ge 0}$. That is, 
    \[ 
    \mathtt{T}^* = \argmin_{\mathtt{T}: \R \to \R} \left\{ \int_{\R} c \bigl( x, \mathtt{T}(x) \bigr)  \d \cL_*^\mathrm{test}(x)
    \,\middle|\,
    \mathtt{T}_\sharp \cL_*^\mathrm{test} = \cL_*
    \right\},
    \]
    where $\mathtt{T}_\sharp$ is the pushforward operator.
\end{prop}
% \begin{proof}[Proof idea:] 
%     Let $\mu = \normal(0, 1)$ and $\nu = \Law( \max\{ \kappa, G \} )$. The optimal transport map from $\mu$ to $\nu$ is given by $\mathtt{T}(x) = F_{\nu}^{-} \circ \Phi (x) = \max\{ \kappa, x \}$ via optimal transport theory, where $F_{\nu}^{-}$ is the quantile function of $\nu$.
% \end{proof}














\subsection{Non-separable data}
\label{sec:logit_logistic}

For non-separable data, SVM yields a trivial solution $\vbeta=\boldsymbol{0}, \beta_0 =0$. 
% according to our discussion in \cref{sec:background}. 
A typical approach to fitting a classifier is to solve regression problem \cref{eq:logistic}. Similar to the margin-rebalanced SVM \cref{eq:SVM}, we can also incorporate $\tau$ into the objective function by substituting $y_i$ for $\wt y_i = y_i/s(y_i)$, that is,
\begin{equation}\label{eq:logistic_reg}
    \min \limits_{\vbeta \in \R^d, \beta_0 \in \R} \qquad 
    \frac1n \sum_{i=1}^n \ell \bigl( \wt y_i(\langle \vx_i, \vbeta \rangle + \beta_0) \bigr),
\end{equation}
where $\ell: \R \to \R_{\ge 0}$ is the loss function. We consider a more general form than logistic regression. We say that $\ell$ is \emph{pseudo-Lipschitz} if there exists a constant $L > 0$ such that, for all $x, y \in \R$,
\begin{equation*}
    \abs{ \ell(x) - \ell(y) } \le L \left( 1 + \abs{ x } + \abs{ y } \right) \abs{ x - y }.
\end{equation*}
This condition is satisfied, for instance, by the widely used logistic loss $\ell(t) = \log(1 + e^{-t})$. As the counterpart of \cref{thm:SVM_main} in the non-separable regime, the following theorem summarizes the precise asymptotics of regression \cref{eq:logistic_reg}, including the limits of parameters and logit distribution.

We consider the same 2-GMM setting as \cref{sec:logit_SVM}. For any non-increasing, strictly convex, pseudo-Lipschitz, twice differentiable function $\ell: \R \to \R_{\ge 0}$, let $(\hat \vbeta_n, \hat \beta_{0, n})$ be the optimal solution to regression \cref{eq:logistic_reg}. Recall $\hat \rho_n:= \hat \rho$ defined in \cref{eq:rho_hat} and $\delta^*(\kappa)$ defined in \cref{eq:sep_functions}. Let $\rho^*, R^*, \beta_0^*, \xi^*$ be a solution to the variational problem
    \begin{equation}\label{eq:logistic_variation}
    \begin{aligned}
        \begin{array}{cl}
            \underset{ \rho \in [-1, 1], R \ge 0, \beta_0 \in \R,\xi \in \cL^2 }{ \mathrm{minimize} }
            &
        \E \left[ \ell \biggl( \dfrac{ \rho\norm{\vmu}_2 R + RG + \beta_0 Y + R\sqrt{1 - \rho^2} \xi }{s(Y)} \biggr) \right], \\
            \text{subject to} & \vphantom{\dfrac11} \E \left[ \xi^2 \right]  \le  1/\delta .
        \end{array}
    \end{aligned}
    \end{equation}
    % \begin{equation}\label{eq:logistic_variation}
    % \begin{aligned}
    %     \begin{array}{cl}
    %         \underset{ \rho \in [-1, 1], R \ge 0, \beta_0 \in \R,\xi \in \cL^2 }{ \mathrm{minimize} }
    %         &
    %     \E \left[ \ell \bigl( \rho\norm{\vmu}_2 R + RG + \beta_0 Y + R\sqrt{1 - \rho^2} \xi \bigr) \right], \\
    %         \underset{ \phantom{\smash{\bm\beta \in \R^d, \beta_0 \in \R, \kappa \in \R} } }{\text{subject to}} & \E \left[ \xi^2 \right]  \le  1/\delta .
    %     \end{array}
    % \end{aligned}
    % \end{equation}
    where $(Y, G) \sim P_y \times \normal(0,1)$. 
    % The asymptotic distribution of logit margins on training set and test set are respectively defined as 
    %The limiting ELD and TLD are respectively defined as
    We define
    \begin{equation*}
    \begin{aligned}
        \nu_* & := 
        \Law \!\left( 
        Y, Ys(Y) \, \prox_{\frac{\lambda^* \ell}{s(Y)}}\biggl( \frac{\rho^* \norm{\vmu}_2 R^* + R^* G + \beta_0^* Y}{s(Y)} \biggr)
        \right).
        \\
        \nu^\mathrm{test}_* & := \Law \,\bigl( Y, Y ( R^*\rho^* \| \bmu \| + R^*G + Y \beta_0^*) \bigr),
    \end{aligned}
    \end{equation*}
    % \begin{equation}
    % \begin{aligned}
    %     \nu_* & := \Law \,\bigl( Y,
    %       Y \prox_{\lambda^* \ell}( \rho^*\norm{\vmu}_2 R^* + R^* G + \beta_0^* Y )
    %      \bigr),
    %     \\
    %     \nu^\mathrm{test}_* & := \Law \,\bigl( Y,
    %      Y \rho^*\norm{\vmu}_2 R^* + R^* G + \beta_0^* Y  
    %      \bigr),
    % \end{aligned}
    % \end{equation}
    aiming to show they are the limiting ELD and TLD respectively.

\begin{thm}[Non-separable data] \label{thm:logistic_main}
    Consider the same 2-GMM and proportional settings $n/d \to \delta$ as in \cref{thm:SVM_main}. 
    \begin{enumerate}[label=(\alph*)]
        \item \label{thm:logistic_main(a)}
        \textbf{(Variational problem)} In the non-separable regime $\delta > \delta^*(0)$, $(\rho^*, R^*, \beta_0^*, \xi^*)$ is the unique solution to \cref{eq:logistic_variation} with $\rho^* \in (0, 1)$, $R^* \in (0, \infty)$, and the random variable $\xi^*$ satisfies (a.s.)
        \begin{equation}\label{eq:logistic_xi_star}
            R^* \sqrt{1 - \rho^{*2}} \xi^*  =  -\lambda^* 
            \ell' \biggl( \prox_{\frac{\lambda^* \ell}{s(Y)}}\Bigl( \frac{\rho^* \norm{\vmu}_2 R^* + R^* G + \beta_0^* Y}{s(Y)} \Bigr) \biggr) ,
        \end{equation}
        % \begin{equation}
        %     R^* \sqrt{1 - \rho^{*2}} \xi^*  =  -\lambda^* 
        %     \ell' \bigl( \prox_{ \lambda^* \ell}( \rho^*\norm{\vmu}_2 R^* + R^*G + \beta_0^* Y ) \bigr) ,
        % \end{equation}
        where $\lambda^* \in (0, \infty)$ is the unique constant such that $\E[ \xi^2] = 1/\delta$.
        Moreover, $(\rho^*, R^*, \beta_0^*, \lambda^*)$ is also the unique solution to the following system of equations
        \begin{align*}
            - \frac{\tau R \rho}{2\pi \lambda \delta \norm{\vmu}_2}
            & = 
            \E\left[ \ell'\biggl( \prox_{\frac{\lambda\ell}{\tau}}\Bigl( \frac{\rho\norm{\vmu}_2 R + RG + \beta_0}{\tau} \Bigr) \biggr) \right] ,
            \\
            - \frac{R \rho}{2(1 - \pi) \lambda \delta \norm{\vmu}_2}
            & = 
            \E \left[ \ell'\bigl( \prox_{ \lambda \ell}( \rho\norm{\vmu}_2 R + RG - \beta_0 ) \bigr) \right] ,
            \\
            \frac{1}{\lambda \delta }
            & = 
            \E \left[ \dfrac{1}{s(Y)} \cdot \frac{
            \ell'' \biggl( \prox_{\frac{\lambda \ell}{s(Y)}}\Bigl( \dfrac{\rho\norm{\vmu}_2 R + RG + \beta_0 Y}{s(Y)} \Bigr) \biggr)
            }{s(Y) + 
            \lambda \ell'' \biggl( \prox_{\frac{\lambda \ell}{s(Y)}}\Bigl( \dfrac{\rho\norm{\vmu}_2 R + RG + \beta_0 Y}{s(Y)} \Bigr) \biggr)
            } \right] ,
            \\
            \frac{R^2 (1 - \rho^2)}{\lambda^2 \delta}
            & = 
            \E \left[ \Biggl(  
            \frac{1}{s(Y)} \cdot \ell' \biggl( \prox_{\frac{\lambda \ell}{s(Y)}}\Bigl( \frac{\rho\norm{\vmu}_2 R + RG + \beta_0 Y}{s(Y)} \Bigr) \biggr)
            \Biggr)^2 \right].
        \end{align*}
        % \begin{align*}
        %     - \frac{\rho R}{2\pi \lambda \delta \norm{\vmu}_2}
        %     & = 
        %     \E \left[ \ell'\bigl( \prox_{ \lambda \ell}( \rho\norm{\vmu}_2 R + RG + \beta_0 ) \bigr) \right] ,
        %     \\
        %     - \frac{\rho R}{2(1 - \pi) \lambda \delta \norm{\vmu}_2}
        %     & = 
        %     \E \left[ \ell'\bigl( \prox_{ \lambda \ell}( \rho\norm{\vmu}_2 R + RG - \beta_0 ) \bigr) \right] ,
        %     \\
        %     \frac{1}{\lambda \delta }
        %     & = 
        %     \E \left[ \frac{\ell''\bigl( \prox_{ \lambda \ell}( \rho\norm{\vmu}_2 R + RG + \beta_0 Y ) \bigr)}{1 + \lambda \ell''\bigl( \prox_{ \lambda \ell}( \rho\norm{\vmu}_2 R + RG + \beta_0 Y ) \bigr)} \right] ,
        %     \\
        %     \frac{R^2 (1 - \rho^2)}{\lambda^2 \delta}
        %     & = 
        %     \E \left[ \left(\ell'\bigl( \prox_{ \lambda \ell}( \rho\norm{\vmu}_2 R + RG + \beta_0 Y ) \bigr) \right)^2 \right] .
        % \end{align*}

        \item \label{thm:logistic_main(b)}
        \textbf{(Parameter convergence)} In the non-separable regime $\delta > \delta^*(0)$, as $n \to \infty$,
        \begin{equation*}
            \| \hat\vbeta_n \|_2 \conp R^*,
            \qquad
            \hat\rho_n \conp \rho^*,
            \qquad
            \hat\beta_{0,n} \conp \beta_0^*.
        \end{equation*}

        \item \label{thm:logistic_main(c)}
        \textbf{(Asymptotic errors)} Recall the prediction errors defined as per \cref{eq:Err_n}. Then in the non-separable regime $\delta > \delta^*(0)$, as $n \to \infty$,
        \begin{equation*}
            \Err_{+,n}  \to  \Phi \left(- \rho^* \norm{\bmu}_2 - \frac{\beta_0^*}{R^*} \right),
            \qquad
            \Err_{-,n}  \to  \Phi \left(- \rho^* \norm{\bmu}_2  + \frac{\beta_0^*}{R^*} \right).
        \end{equation*}

        \item \label{thm:logistic_main(d)}
        \textbf{(ELD/TLD convergence)} 
        % The empirical distribution of logit margins on training set and the distribution for a test point are defined as the following random measures, respectively:
        % \begin{equation}
        %     \hat \cL_{n} := \frac1n \sum_{i=1}^n \delta_{y_i ( \< \xx_i, \hat\vbeta_n \> + \hat\beta_{0,n} )
        %     },
        %     \qquad
        %     \hat \cL_{n}^\mathrm{test} := \Law\left( y^\mathrm{new} ( \< \xx^\mathrm{new}, \hat\vbeta_n \> + \hat\beta_{0,n} \right),
        % \end{equation}
        Recall the ELD $\hat\nu_n$ and TLD $\hat\nu_n^\mathrm{test}$ defined as per \cref{def:ELD_TLD}, where $\hat f(\xx) = \< \xx, \hat\vbeta_n \> + \hat\beta_{0, n}$.
        Then in the non-separable regime $\delta > \delta^*(0)$ we have logit convergence for both training and test data, i.e., as $n \to \infty$,
        \begin{equation*}
            % W_2\bigl( \hat \cL_{n}, \cL_* \bigr) \conp 0,
            % \qquad
            % \hat\cL_{n}^\mathrm{test} \conw \cL_*^\mathrm{test}.
            W_2\bigl(\hat{\nu}_n , \nu_* \bigr) \conp 0,
            \qquad
            \hat{\nu}_n^\mathrm{test} \conw \nu^\mathrm{test}_*.
        \end{equation*}
    \end{enumerate}
\end{thm}

\begin{rem}
    Compared to the separable regime, the random variable $\xi$ in the non-separable regime \cref{eq:logistic_variation} can also be interpreted as the cause for overfitting, but its distortion effect on ELD is not truncation. When $\tau = 1$, by \cref{eq:logistic_variation}, \eqref{eq:logistic_xi_star}, the following holds for a ``typical'' training point:
\begin{equation*}
\begin{aligned}
    y_i (\langle \vx_i, \hat \vbeta_n \rangle + \hat \beta_{0,n}) 
    & \approx 
    \rho^*\norm{\vmu}_2 R^* + R^*G + \beta_0^* Y + R^*\sqrt{1 - \rho^{*2}} \xi^* \\
    & = \rho^*\norm{\vmu}_2 R^* + R^*G + \beta_0^* Y 
    - \lambda^* 
            \ell' \bigl( \prox_{ \lambda^* \ell}( \rho^*\norm{\vmu}_2 R^* + R^*G + \beta_0^* Y ) \bigr) \\
    & = \prox_{ \lambda^* \ell}( \rho^*\norm{\vmu}_2 R^* + R^*G + \beta_0^* Y ),
\end{aligned}
\end{equation*}
where the equalities come from \cref{lem:prox}. Hence, the ELD in the non-separable regime is the TLD under nonlinear shrinkage due to the proximal operator of loss function $\ell$.
\end{rem}

\section{Analysis of margin rebalancing for separable data} \label{sec:rebalacing}


In this section, we show how margin rebalancing improves the test accuracies on imbalanced dataset by choosing the hyperparameter $\tau$ in \cref{fig:SVM_cartoon} appropriately. 


\subsection{Proportional regime}
\label{subsec:rebal_prop}
Consider the same 2-GMM and proportional settings in \cref{sec:logit_SVM} on linearly separable dataset ($\delta < \delta^*(0)$). According to \cref{thm:SVM_main}\ref{thm:SVM_main_err}, the asymptotic minority and majority test errors are
\begin{equation}\label{eq:asymp_Err}
    \Err_{+}^*  :=  \Phi \left(- \rho^* \norm{\bmu}_2 - \beta_0^* \right),
            \qquad
    \Err_{-}^*  :=  \Phi \left(- \rho^* \norm{\bmu}_2  + \beta_0^* \right).
\end{equation}
For the purpose of imbalanced classification, we define the \textit{asymptotic balanced error} as
\begin{equation*}
\Err_\mathrm{b}^* := \frac{1}{2} \Err_{+}^* + \frac{1}{2} \Err_{-}^*.
\end{equation*}

\paragraph{Monotonicity analysis.} 
% Before delving into the optimal $\tau$, 
We first provide some monotone results for test errors, which support our empirical observations in \cref{subsec:rebal}. 

% The following two preliminary results give the monotonicity of asymptotic parameters $\rho^*$ (cosine angle) and $\beta_0^*$ (intercept).

% Recall that $\rho^*$ is the unique solution to \cref{eq:SVM_variation}. According to \cref{thm:SVM_main}\ref{thm:SVM_main_var}, $\rho^* \in (0, 1)$ is invariant with respect to $\tau$. Hence $\rho^*$ can be viewed as a function of model parameters $(\pi, \norm{\bmu}_2, \delta)$ given by \cref{eq:SVM_variation}.

% \begin{prop}\label{prop:rho_mono}
% 	Let $\rho^*$ be the asymptotic cosine angle between $\bmu$ and $\hat\vbeta$ as defined in \cref{thm:SVM_main}. Then $\rho^* = \rho^* (\pi, \norm{\bmu}_2, \delta)$ is an increasing function of $\pi \in (0, \frac12)$, $\norm{\bmu}_2$, and $\delta$.
% \end{prop}

% Similarly, $\beta_0^*$ can also be viewed as a function of $(\pi, \norm{\bmu}_2, \delta)$ when $\tau$ is fixed.

% \begin{prop}\label{prop:beta0_mono}
% 	Let $\beta_0^*$ be the asymptotic intercept as defined in \cref{thm:SVM_main} without margin rebalancing ($\tau = 1$). Then $\beta_0^* = \beta_0^* (\pi, \norm{\bmu}_2, \delta)$ is an increasing function of $\pi \in (0, \frac12)$, $\delta$, and $\norm{\bmu}_2$.
% \end{prop}

% Combing these two propositions with \cref{eq:asymp_Err}, we have the following immediate result:

\begin{prop}\label{prop:Err-_mono}
    % The minority test error 
    $\Err_{+}^*$ is a decreasing function of $\pi \in (0, \frac12)$, $\norm{\bmu}_2$, and $\delta$ when $\tau = 1$.
\end{prop}

However, the majority error $\Err_{-}^*$ and balanced error $\Err_\mathrm{b}^*$ are not necessarily monotone under arbitrary $\tau$. Thus, we will focus on the monotonicity of these test errors when $\tau$ is chosen to be optimal.

% \ljycom{Can we show that $\Err_{+}^*$ is monotone increasing in $\pi \in (0, \frac12)$, $\delta$, and $\norm{\bmu}_2$ when $\tau = 1$?}

According to \cref{fig:SVM_cartoon} and \ref{fig:Err_tau}, by taking $\tau > 1$, we can improve the minority accuracy at the cost of harming majority accuracy. The opposite effects of $\tau$ on $\Err^*_+$ and $\Err^*_-$ are summarized in the following result.

\begin{prop}\label{prop:tau_mono}
    $\Err^*_+$ is decreasing in $\tau \in (0, \infty)$, and $\Err^*_-$ is increasing in $\tau \in (0, \infty)$.
\end{prop}

\paragraph{Choosing the optimal $\tau$.} A natural idea for margin rebalancing is to choose $\tau$ such that the balanced error $\Err_\mathrm{b}^*$ is minimized.

\begin{prop}[Optimal $\tau$] \label{prop:tau_optimal}
    Let $\tau^\mathrm{opt}$ be the optimal margin ratio $\tau$ defined in \cref{prop:tau_opt}. Denote $g_1 (x) := \E \left[ (G + x)_+ \right]$ where $G \sim \normal(0, 1)$. Then $\tau^\mathrm{opt}$ has the 
 explicit expression 
    %which minimizes the balanced error
%\begin{equation}
    %\tau^\mathrm{opt} :=  \argmin_{\tau \in \R} \Err_\mathrm{b} = \argmin_{\tau \in \R} \big\{ \Phi(- \rho^* \norm{\bmu}_2  - \beta_0^* )
    %+ \Phi(- \rho^* \norm{\bmu}_2 + \beta_0^* ) \big\}.
%\end{equation}
%When $\tau = \tau^\mathrm{opt}$, we have $\beta_0^* = 0$ and $\Err_+^* = \Err_-^* = \Err_\mathrm{b}^*$. In particular, $\tau^\mathrm{opt}$ has explicit expression
\begin{equation}\label{eq:tau_opt}
        \tau^\mathrm{opt} =  \dfrac{g_1^{-1} \left( \dfrac{\rho^*}{2 \pi \norm{\bmu}_2 \delta} \right) + \rho^* \norm{\bmu}_2}{g_1^{-1} \left( \dfrac{\rho^*}{2 (1 - \pi) \norm{\bmu}_2 \delta} \right) + \rho^* \norm{\bmu}_2}.
\end{equation}
\end{prop}
\begin{rem}\label{rem:tau_pi}
    % In contrast with \cite{cao2019learning} where the suggested $\tau$ scales with $\pi^{-1/4}$, 
    The optimal choice of $\tau$ has a complicated dependence on $\pi$. However, we note that the numerator scales as $\tau^\mathrm{opt} \asymp \sqrt{1/\pi}$ for small $\pi$ and fixed $\norm{\bmu}_2$ and $\delta$, which is consistent with the choice of $\tau$ in importance tempering \cite{lu2022importance}. In \cite{cao2019learning}, $\tau$ is suggested to scale with $\pi^{-1/4}$, however it was proved in \cite{kini2021label} that their algorithm won't converge to the solution with the desired $\tau$.

    It is worth noticing that in the near-degenerate cases where $\pi$ is very small or $\norm{\bmu}_2$, $\delta$ are very large, then $\rho^*$ is close to $0$ and the denominator can be negative,
    % where $\rho^*$ is close to $0$ or $\delta$ is very large, then the denominator can be negative, 
    leading to $\tau^\mathrm{opt} < 0$. While our theory (\cref{prop:Err_monotone}, \cref{prop:conf}) is still valid when we allow potential negative $\tau$, it is rarely used in practice. See \cref{subsec:tau_optimal} for a further discussion. 
    % In such cases, by slightly modifying the margin-rebalanced SVM problem \cref{eq:SVM-m-reb}, we can make all our theoretical results (\cref{prop:SVM_tau_relation} and \cref{thm:SVM_main}) still applicable. \TODO{Wait?! Does it mean we need additional assumptions in the previously stated results? Perhaps we just say we assume $\tau>0$ in previous results, and for convenience allow negative $\tau$ for this section and next section when we state monotone result?} \ljycom{We don't need additional assumption. For $\tau < 0$, I think we only need to change the objective in \cref{eq:SVM-m-reb} to $\minimize \kappa$ (and then we will get $\kappa^* < 0$). Then we consider the corresponding asymptotic problem, and \cref{prop:SVM_tau_relation}, \cref{thm:SVM_main} still hold. I agree with the suggestions.}
    The near-degenerate cases (small $\pi$, large $\norm{\bmu}_2$ or $\delta$) are better addressed under the high imbalance regime, as we analyze in the next subsection. 
\end{rem}
The minority/majority/balanced errors all equal $\Phi(- \rho^* \norm{\bmu}_2 )$ when $\tau = \tau^\mathrm{opt}$. 
% Using \cref{prop:rho_mono}, 
We can also obtain the monotonicity of test errors after margin rebalancing.
\begin{prop}\label{prop:Err_monotone}
    When $\tau = \tau^{\rm opt} > 0$, all the test errors $\Err^*_+$, $\Err^*_-$, $\Err^*_\mathrm{b}$ are decreasing functions of $\pi \in (0, 1/2)$ (imbalance ratio), $\delta$ (aspect ratio), and $\norm{\bmu}_2$ (signal strength).
\end{prop}


































\subsection{High imbalance regime}\label{subsec:high-imb}

Different from the proportional regime considered in \cref{sec:logit} and \ref{subsec:rebal_prop}, here we focus on a high-imbalanced scenario where $\pi$ is small, $\norm{\bmu}_2$ is large, and $n$ grows much faster than $d$. In this regime, we can even extend the feature distribution beyond Gaussian, and generalize the 2-GMM settings.

\begin{defn}[High imbalance]
We say a dataset $\{(\xx_i, y_i)\}_{i = 1}^n$ is i.i.d.~generated from a \emph{two-component sub-gaussian mixture model (2-subGMM)} if for any $i \in [n]$,
\begin{enumerate}
    \item[i.] Label distribution: $\P(y_i = +1) = 1 - \P(y_i = -1) = \pi$, 
    \item[ii.] Feature distribution: $\xx_i = y_i \bmu + \zz_i$, where $\zz_i$ has independent coordinates with uniformly bounded sub-gaussian norms. Namely, each coordinate $z_{ij}$ of $\zz_i$ satisfies $\E[z_{ij}] = 0$, $\var(z_{ij}) = 1$, and $\norm{z_{ij}}_{\psi_2} := \inf \{ K > 0: \E[\exp(X^2/K^2)] \le 2 \} \le C$ for all $j \in [d]$, where $C$ is an absolute constant.
    % $\zz_i \sim \subGind(\bzero, \bI_d; K)$, and $K$ is an absolute constant.
\end{enumerate}
For any constants $a,b,c >0$, we say a 2-subGMM is \emph{$(a,b,c)$-imbalanced} if \cref{setup-high-imbalance} holds.
% \begin{equation*}
% \pi = d^{-a}, \qquad \norm{\vmu}_2^2 = d^b, \qquad n = d^{c+1}.
% \end{equation*}
\end{defn}
\begin{rem}
    % When $\pi \to 0$, we require the condition $\norm{\vmu} \to \infty$ to make the classification problem workable. 
    Parameters $a$, $\frac{b}{2}$, and $c$ each specifies the degenerate rate of imbalance ratio $\pi$, and the growth rate of signal strength $\| \bmu \|_2$, aspect ratio $n/d$. We usually require $a < c + 1$ to make sure the minority class sample size $n_+ := \pi n = d^{c - a + 1} \to \infty$ does not degenerate.
\end{rem}

Our goal is to study the performance of margin-rebalanced SVM \cref{eq:SVM-m-reb} in this high imbalance regime, asymptotically as $d \to \infty$. Therefore, we allow $\tau = \tau_d$ depends on dimension $d$ and care about what order of $\tau_d$ would make the test errors vanish. We summarize our findings in the following theorem, which is consistent with the empirical observations in \cref{fig:High_imb_heat} and extends \cref{thm:high-imbalance} to the case of imbalanced 2-subGMM.

\begin{thm}[Phase transition in high imbalance regime]\label{thm:main_high-imbal}
Consider the high imbalance regime where the training data is i.i.d.~generated from an $(a,b,c)$-imbalanced 2-subGMM. Suppose that $a - c < 1$. A margin-rebalanced SVM is trained, with test errors calculated according to \cref{eq:Err_n}. Then as $d \to \infty$, 
the conclusions of the three phases in \cref{thm:high-imbalance} still hold.
% we have
% \begin{enumerate}[label=(\alph*)]
% \item \textbf{High signal} (no need for margin rebalancing): $a - c< b$. If we choose $1 \le \tau_d \ll d^{b/2}$, then 
% \begin{equation*}
% \hat\Err_{+,d} = o_{\P}(1), \qquad \hat\Err_{-,d} = o_{\P}(1).
% \end{equation*}
% \item \textbf{Moderate signal} (margin rebalancing is crucial): $b < a - c < 2b$. If we choose $d^{a-b-c} \ll \tau_d \ll d^{(a-c)/2}$, then
% \begin{equation*}
% \hat\Err_{+,d} = o_{\P}(1), \qquad \hat\Err_{-,d} = o_{\P}(1).
% \end{equation*}
% However, if we naively choose $\tau_d \asymp 1$, then 
% \begin{equation*}
% \hat\Err_{+,d} = 1 - o_{\P}(1), \qquad \hat\Err_{-,d} = o_{\P}(1).
% \end{equation*}
% \item \textbf{Low signal} (no better than random guess): $a - c > 2b$. For any $\tau_d$, we have
% \begin{equation*}
% \hat\Err_{\mathrm{b},d} := \frac12 \bigl( \hat\Err_{+,d} + \hat\Err_{-,d} \bigr) \ge \frac12 -  o_{\P}(1).
% \end{equation*}
% \end{enumerate}
\end{thm}

\section{Consequences for confidence estimation and calibration} \label{sec:calibration}


% In the deep learning literature, the \emph{confidence} of a classifier often refers to the probability of the correctness of a prediction, i.e., the probability that predicted label matches the true label \cite{?} The most common way to transform a logit into the predicted probability of each class is softmax function, or equivalently, sigmoid function in binary classification.

Recall the definition of confidence of the max-margin classifier $\hat p(\xx) := \sigma ( \hat f(\xx) ) = 
        \sigma( \< \xx, \hat\vbeta \> + \hat\beta_0 )$ in \cref{subsec:conf_calib}.
% \begin{defn}[Confidence]
%     For any classifier of the form $\hat y(\xx) = 2 \mathbbm{1}\{ f(\vx) > 0\} - 1$,
%     % linear classifier $f(\xx) = \< \xx, \bbeta \> + \beta_0$, 
%     define $p: \R^d \to [0, 1]$ as the \emph{confidence} of $f$, which is the sigmoid transformation of $f(\xx)$: 
%     \begin{equation*}
%         p(\xx) :=  \sigma\bigl( f(\xx) \bigr),
%         \qquad \text{where} ~ \sigma(t) := \frac{1}{1 + e^{-t}}.
%     \end{equation*}
% \end{defn}
Note that $\hat p(\xx)$ and $1 - \hat p(\xx)$ are the predicted probabilities of $\xx$ for the minority class ($y = +1$) and the majority class ($y = -1$) respectively. 

It is worth noticing that the confidence is sensitive to scales, i.e., $\sigma(t) \not= \sigma(ct)$ if $c \not= 1$, despite the fact that rescaling yields the same label prediction and thus does not affect accuracy. While small models tend to be calibrated, especially when parameter estimation is consistent, larger models such as DNNs are known to suffer from poor calibration \cite{guo2017calibration}. A simple theoretical explanation is that in a DNN, the last layer (usually a logistic regression) $\xx \mapsto \sigma\bigl( \< \xx, \hat\bbeta \> + \hat\beta_0 \bigr)$ trained by gradient descent on separable features often results in a very large $\| \hat\vbeta\|_2$ (as mentioned in \cref{subsec:LR_vs_SVM}), thereby inflating the predicted probabilities. Here we focus on the common form of SVM \eqref{eq:SVM-0} where normalization $\| \hat\vbeta \|_2 = 1$ is applied.
%Sometimes $( p(\xx), 1 - p(\xx) )$ also refers to confidence in a multi-class setting. \ljy{However, the confidence is NOT scale invariant, i.e., $\sigma(t) \not= \sigma(ct)$ if $c \not= 1$. It means even if the two classifiers based on $f_1(\vx)$ and $f_2(\vx) := c f_1(\vx)$ give the same prediction, the confidence of $f_1$ and $f_2$ are different. This partially explains why many deep neural networks are over-confident, since the last layer (usually a logistic regression) $\xx \mapsto \sigma\bigl( \< \xx, \hat\bbeta \> + \hat\beta_0 \bigr)$ trained by gradient descent on separable embeddings often results in a very large $\| \hat\vbeta\|_2$ (as mentioned in \cref{subsec:LR_vs_SVM}). In this section, we will focus on the the confidence of margin-rebalanced SVM \cref{eq:SVM-m-reb}, which fixes $\| \hat\vbeta \|_2 = 1$ on separable data.} 

Some probabilities regarding the confidence are as follows.

\begin{enumerate}
    \item \textbf{Max-margin confidence.}
    The confidence of the max-margin classifier is
    \begin{equation*}
        \hat p(\xx) := \sigma\bigl( \hat f(\xx) \bigr) = 
        \sigma\bigl( \< \xx, \hat\vbeta \> + \hat\beta_0 \bigr).
    \end{equation*}
    
    \item \textbf{Bayes optimal probability.}
    The true conditional probability is
    \begin{equation*}
        \begin{aligned}
            p^*(\xx) := \P( y = 1 \,|\, \xx ).
        \end{aligned}
    \end{equation*}
    
    \item \textbf{True posterior probability.} The probability conditioning on max-margin confidence is
    \begin{equation*}
        \begin{aligned}
            \hat p_0(\xx) := \P \bigl( y = 1 \,|\, \hat p(\xx) \bigr).
        \end{aligned}
    \end{equation*}
\end{enumerate}
Note that $p^*(\xx)$ is the confidence of the Bayes classifier $y^*(\xx) := 2\ind\{  \< \xx, 2\bmu \> + \log\frac{\pi}{1 - \pi} > 0 \} - 1$.

% We would like the confidence $\hat p$ to be a good approximation of the true conditional probability (Bayes optimal probability), that is
% \begin{equation}\label{eq:cal_strong}
%     \hat p(\xx) \approx p^*(\xx) = \P( y = 1 \,|\, \xx ).
% \end{equation}
% However, the Bayes optimal probability is usually intractable in black-box deep learning settings, and estimating this probability conditioned on the high dimensional $\xx$ can be difficult. The notion of \emph{calibration} is therefore widely used in literature, which captures the intuition of \cref{eq:cal_strong} in a weaker sense. In particular, the confidence $\hat p$ is (approximately) \emph{calibrated} if
% \begin{equation}\label{eq:cal_weak}
%     \hat p(\xx) \approx \hat p_0(\xx) = \P \bigl( y = 1 \,|\,  \hat p(\xx) \bigr).
% \end{equation}
% Motivated by these intuitions, some well-known miscalibration metrics are listed below \cite{kumar2019verified, kuleshov2015calibrated}:
Recall the definition of calibration \cref{eq:calibrated} and some miscalibration metrics \cref{eq:CalErr}---\eqref{eq:ConfErr} introduced in \cref{subsec:conf_calib}. We offer some further explanations for them.
\begin{itemize}
    \item \textbf{Calibration error:} The $\cL^2$ distance between confidence and posteriori, which is the most commonly used metric.
    \begin{equation*}
    % \begin{aligned}
    %     \mathrm{CalErr}(\hat p) & :=  \E\left[ \bigl( \hat p(\xx) - \hat p_0(\xx) \bigr)^2 \right]  = \E\left[ \Bigl(  
    %     \hat p(\xx) - \P \bigl( y = 1 \,|\,  \hat p(\xx) \bigr)
    %     \Bigr)^2 \right]
    %     \\
    %     & \phantom{:}=  \E\left[ \Bigl(
    %     \P\bigl( y \hat f(\xx) > 0 \,\big|\, \hat p(\xx) \bigr) - \max \bigl\{ \hat p(\xx), 1 - \hat p(\xx) \bigr\}
    %     \Bigr)^2 \right].
    % \end{aligned}
    \mathrm{CalErr}(\hat p) :=  \E\left[ \bigl( \hat p(\xx) - \hat p_0(\xx) \bigr)^2 \right]
    \end{equation*}

    \item \textbf{Mean squared error (MSE):} Also known as the Brier score, subject to a calibration budget \cite{brier1950verification, gneiting2007probabilistic}.
    \begin{equation*}
        \mathrm{MSE}(\hat p)  := \E\left[ \bigl( \mathbbm{1}\{ y = 1 \} - \hat p(\xx) \bigr)^2 \right]
    \end{equation*}
    It can be shown that MSE has the following decomposition
    % \begin{equation*}
    %     \begin{aligned}
    %     \mathrm{MSE}(\hat p) 
    %     & = \var\, \bigl[\mathbbm{1}\{ y = 1 \} \bigr]
    %     + \E\left[ \bigl( \hat p(\xx) - \hat p_0(\xx) \bigr)^2 \right] - \var\left[ \P\bigl( y = 1 \,\big|\, \hat p(\xx) \bigr) \right]
    %     \\
    %     & = \underbrace{ \var\, \bigl[\mathbbm{1}\{ y = 1 \} \bigr] }_{\text{irreducible}}  
    %     + 
    %     \underbrace{ \vphantom{\bigl[} \mathrm{CalErr}(\hat p) }_{\text{lack of calibration}}
    %     - 
    %     \underbrace{\var\, \bigl[ \hat p_0(\xx) \bigr]}_{\text{sharpness/resolution}}
    %     .
    %     \end{aligned}
    % \end{equation*}
    \begin{equation*}
        \mathrm{MSE}(\hat p) 
        \ = \ \underbrace{ \var\, \bigl[\mathbbm{1}\{ y = 1 \} \bigr] }_{\text{irreducible}}  
        + 
        \underbrace{ \vphantom{\bigl[} \mathrm{CalErr}(\hat p) }_{\text{lack of calibration}}
        - 
        \underbrace{\var\, \bigl[ \hat p_0(\xx) \bigr]}_{\text{sharpness/resolution}}
        .
    \end{equation*}
    Calibration error itself does not guarantee a useful predictor. Sharpness, also known as resolution \cite{murphy1973new, kuleshov2015calibrated}, is another desired property which measures the variance in the response $y$ explained by the probabilistic prediction $\hat p(\xx)$. Hence, a small MSE suggests a classifier to be calibrated with high sharpness. 
    
    Note that $\var[\mathbbm{1}\{ y = 1 \}] = \pi(1 - \pi)$ is an intrinsic quantity unrelated to $\hat f$. When study the effect of $\pi$ on model calibration, we may discard the irreducible variance term and define a modified MSE as
    \begin{equation*}
        \mathrm{mMSE}(\hat p) := \mathrm{CalErr}(\hat p) - \var \bigl[ \hat p_0(\xx) \bigr].
    \end{equation*}
    \item \textbf{Confidence estimation error:} The $\cL^2$ distance between confidence and Bayes optimum.
    \begin{equation*}
        \mathrm{ConfErr}(\hat p) :=  \E\left[ \bigl( \hat p(\xx) - p^*(\xx) \bigr)^2 \right]
        % = \E\left[ \bigl( \hat p(\xx) - \P( y = 1 \,|\, \xx ) \bigr)^2 \right]
        .
    \end{equation*}
    It has the following relation with MSE:
    % \begin{equation*}
    %     \begin{aligned}
    %     \mathrm{MSE}(\hat p) 
    %     & =  
    %     \E\left[ \bigl( \mathbbm{1}\{ y = 1 \} - p^*(\xx) \bigr)^2 \right]
    %     + \E\left[ \bigl( \hat p(\xx) - p^*(\xx) \bigr)^2 \right]
    %     \\
    %     & =
    %     \E \, \Bigl[ \var\, \bigl[\mathbbm{1}\{ y = 1 \} \,|\, \xx \bigr] \Bigr]
    %     + \mathrm{ConfErr}(\hat p)
    %     \\
    %     & = 
    %     \E \, \Bigl[ p^*(\xx) \bigl( 1 - p^*(\xx) \bigr) \Bigr]
    %     + \mathrm{ConfErr}(\hat p),
    %     \end{aligned}
    % \end{equation*}
    \begin{equation}
    \label{eq:MSE_vs_ConfErr}
        \mathrm{MSE}(\hat p) 
        =
        \E \left[ p^*(\xx) \bigl( 1 - p^*(\xx) \bigr) \right]
        + \mathrm{ConfErr}(\hat p),
    \end{equation}
    where the first term is intrinsic, which only depends on $\pi$ and $\norm{\bmu}_2$. 
    
    % \ljycom{This metric is less used in deep learning literature since conditioning on high-dimensional $\xx$ is often intractable. However, the definition of confidence estimation error is closer to the original purpose of confidence quantification/calibration.}
\end{itemize}

The asymptotics of these metrics, and some monotone effect of model parameters $\pi \in (0, 1/2)$, $\norm{\bmu}_2$, $\delta$ on them, are summarized in the following proposition.

\begin{prop}[Confidence estimation and calibration]\label{prop:conf}
    Consider 2-GMM and the proportional settings in \cref{sec:logit_SVM} on linearly separable dataset ($\delta < \delta^*(0)$).
    \begin{enumerate}[label=(\alph*)]
        \item \label{prop:conf_asymp}
        Let $(\rho^*, \beta_0^*)$ be defined as per \cref{thm:SVM_main}, and $(Y, G) \sim P_y \times \normal(0,1)$. Denote
        \begin{equation*}
            \begin{aligned}
                \mathrm{MSE}^*  
                & :=  \E \left[ \sigma \bigl( -\rho^* \norm{\bmu}_2 - \beta_0^* Y + G \bigr)^2 \right],
                \qquad
                \mathrm{mMSE}^* = \mathrm{MSE}^* - \pi(1 - \pi),
                \\
                \mathrm{CalErr}^*
                & :=  \E\left[ \left( \sigma\Bigl( 2\rho^* \norm{\bmu}_2 (\rho^* \norm{\bmu}_2 Y + G) + \log\frac{\pi}{1-\pi} \Bigr) - \sigma\bigl( \rho^* \norm{\bmu}_2 Y + G + \beta_0^* \bigr) \right)^2 \right], 
                \\
                V_{y|\xx}^*
                & :=  \E\left[ \sigma\Bigl( -2\norm{\bmu}_2( \norm{\bmu}_2 + G ) - \log\frac{\pi}{1-\pi} Y \Bigr)^2 \right],
                \qquad
                \mathrm{ConfErr}^* = \mathrm{MSE}^* - V_{y|\xx}^*.
            \end{aligned}
        \end{equation*}
        then
        \begin{equation*}
        \begin{aligned}
            \lim_{n \to \infty} \mathrm{MSE}(\hat p) = \mathrm{MSE}^*,
            \qquad
            & \lim_{n \to \infty} \mathrm{CalErr}(\hat p) = \mathrm{CalErr}^*, 
            \\
            \lim_{n \to \infty} \mathrm{mMSE}(\hat p) = \mathrm{mMSE}^*,
            \qquad
            & \lim_{n \to \infty} \mathrm{ConfErr}(\hat p) = \mathrm{ConfErr}^*.
        \end{aligned}
        \end{equation*}
        
        \item \label{prop:conf_mono}
        When $\tau = \tau^\mathrm{opt} > 0$, 
        \begin{itemize}
            \item $\mathrm{MSE}^*$ and $\mathrm{mMSE}^*$ are decreasing functions of $\pi \in (0, \frac12)$, $\norm{\bmu}_2$, $\delta$.
            % \item $\mathrm{CalErr}^*$ is decreasing in $\norm{\bmu}_2$ and $\delta$, for any $\pi < \overline{\pi} \approx 0.2$ fixed.
            % \ljycom{Based on the ``fact'' that both
            % \begin{align*}
            %     h_1(t) & = \E\left[ \Bigl( \sigma\bigl( 2 t (G+t) + c_+ \bigr) - \sigma( G+t ) \Bigr)^2 \right]
            %     \\
            %     h_2(t) & = \E\left[ \Bigl( \sigma\bigl( 2 t (G-t) + c_- \bigr) - \sigma( G-t ) \Bigr)^2 \right]
            % \end{align*}
            % are monotone decreasing when $c_\pm < 0$ are small enough. It seems difficult to prove, but can be veried numerically. 
            % }
            \item $\mathrm{ConfErr}^*$ is decreasing in $\delta$.
            
            % \ljycom{
            % For monotonicity in $\pi$, it suffices to show that
            % \begin{equation*}
            %     V_{y|\xx}^* = \E\left[ \sigma\Bigl( -2\norm{\bmu}_2( \norm{\bmu}_2 + G ) - \log\frac{\pi}{1-\pi} Y \Bigr)^2 \right] = \E \, \Bigl[ p^*(\xx) \bigl( 1 - p^*(\xx) \bigr) \Bigr]
            % \end{equation*}
            % is monotone increasing in $\pi \in (0, \frac12)$ (for any $\norm{\bmu}_2$ fixed).
            % }
        \end{itemize}
    \end{enumerate}
\end{prop}
In addition, there are some monotone relationships that can be verified numerically. We summarized them in the following claim.
\begin{claim}\label{claim:conf}
    Consider the same settings as \cref{prop:conf}. When $\tau = \tau^\mathrm{opt} > 0$, we have
    \begin{itemize}
            \item $\mathrm{CalErr}^*$ is decreasing in $\norm{\bmu}_2$ and $\delta$, for any $\pi \le \overline{\pi}$ fixed, where $\overline{\pi} \approx 0.25$ is some constant.
            \item $\mathrm{ConfErr}^*$ is decreasing in $\pi \in (0, \frac12)$.
        \end{itemize}
\end{claim}
Our findings challenge the conjecture that code-comment coherence, as measured by SIDE \cite{mastropaolo2024evaluating}, is a critical quality attribute for filtering instances of code summarization datasets. By selecting $\langle code, summary \rangle$ pairs with high-coherence for training allow to achieve the same results that would be achieved by randomly selecting such a number of instances. At the same time, we observed that reducing the datasets size up to 50\% of the training instances does not significantly affect the effectiveness of the models, even when the instances are randomly selected. These results have several implications.

First, code-comment consistency might not be a problem in state-of-the-art datasets in the first place, as also suggested in the results of RQ$_0$. Also, the DL models we adopted (and, probably, bigger models as well) are not affected by inconsistent code-comment pairs, even when these inconsistencies are present in the training set.
Despite the theoretical benefits of filtering by SIDE \cite{mastropaolo2024evaluating}, that is the state-of-the-art metric for measuring code-comment alignment, our results indicate its limitations in improving the \textit{overall} quality of the training sets for code summarization task.
Nevertheless, other quality aspects of code and comments that have not been explored yet (such as readability) may be important for smartly selecting the training instances.
Future work should explore such quality aspects further.

Our results clearly indicate that state-of-the-art datasets contain instances that do not contribute to improving the models' effectiveness. This finding is related to a general phenomenon observed in Machine Learning and Deep Learning. Models reach convergence when they are trained for a certain amount of time (epochs). Additional training provides smaller improvements and increases the risk of overfitting. We show that the same is true for data. In terms of effectiveness, model convergence is achieved with fewer training instances than previously assumed. Limiting the number of epochs may make it possible to reach model convergence with a subset of training data, maintaining model effectiveness, reducing resource demands and minimizing the risk of overfitting.
Future work could explore different criteria for data selection that identify the most informative subsets for training.
Conversely, this insight suggests that currently available datasets suffer from poor diversity (thus causing the previously discussed phenomenon).
This latter insight constitutes a clear warning for researchers interested in building code summarization datasets, which should include instances that add relevant information instead of adding more data, which might turn out to be useless.

Finally, it is worth pointing out that another benefit of the reduction we performed is the environmental impact. Reducing the number of training instances implies a reduced training time, which, in turn, lowers the resources necessary to perform training and, thus, energy consumption and CO$_2$ emissions.
We performed a rough estimation of the training time across different selections of \textit{TL-CodeSum} and \textit{Funcom} datasets and estimated a proxy of the CO$_2$ emissions for each model training phase by relying on the ML CO$_2$ impact calculator\footnote{\url{https://mlco2.github.io/impact/\#compute}} \cite{lacoste2019quantifying}. Such a calculator considers factors such as the total training time, the infrastructure used, the carbon efficiency, and the amount of carbon offset purchased. The estimation of CO$_{2}$ emissions needed to train the model with the \textit{Full} selection of \textit{Funcom} ($\sim$ 200 hours) is equal to 26.05 Kg, while with the optimized training set, \ie $SIDE_{0.9}$ ($\sim$ 90 hours), the estimation is 11.69 Kg of $CO_2$ (-55\% emissions).
While we recognize that this method provides an estimation rather than a precise measurement, it offers a glimpse into the environmental impact of applying data reduction.


\section*{Acknowledgments}

Y.Z.~is supported by NSF-DMS grant 2412052 and by the Office of the Vice Chancellor for Research and Graduate Education at the UW Madison with funding from the Wisconsin Alumni Research Foundation. K.Z.~is supported by the Founder's Postdoctoral Fellowship in Statistics at Columbia University. J.L.~is grateful for the feedback from Zhexuan Liu, Zexuan Sun, Zhuoyan Xu, Congwei Yang, Zhihao Zhao, and audience from the Institute for Foundations of Data Science (IFDS) Ideas Forum.

\newpage

\appendix

% Appendix TOC
\etocdepthtag.toc{mtappendix}
\etocsettagdepth{mtchapter}{none}
\etocsettagdepth{mtappendix}{subsubsection}
\tableofcontents

% Hide subsection titles in table of contents  
% \addtocontents{toc}{\protect\setcounter{tocdepth}{1}}

\section{Experiment details}
\label{append_sec:exp}

\subsection{Experiment setup and details}

We present the details of our experiments, including the computational configurations, information about the datasets, and the pretrained neural networks used in our study.

\paragraph{Optimization.} We used the functions \texttt{linear\_model.LogisticRegression} and \texttt{svm.SVC} from Python module \texttt{sklearn} to solve logistic regression \cref{eq:logistic} and SVM \cref{eq:SVM-0} (more precisely, \cref{eq:SVM-1} parametrization with $\tau = 1$). For logistic regression, we used the limited-memory BFGS (L-BFGS) solver, with maximum number of iterations $10^6$.\footnote{If logistic regression is far from converging after the maximum number of iterations is reached, we would add a small explicit regularizer as in \cref{eq:logistic-l2}. In practice, the parameter $\mathtt{C} = \lambda^{-1}$ in \texttt{linear\_model.LogisticRegression} can be chosen as $10^6 \sim 10^8$.} 
For SVM, we set the default value of cost parameter $\mathtt{C} = 1$.\footnote{Note that there is no hard-margin solver available in \texttt{sklearn} and \texttt{svm.SVC} is a soft-margin version. One may set $\mathtt{C}$ large enough, but usually a larger $\mathtt{C}$ will lead to longer running time. To handle this issue, (for separable data) we run \texttt{svm.SVC} with $\mathtt{C}$ increases from 1, until the training error attains zero.} Tolerance for both are set to be $10^{-8}$.

As discussed in \cref{subsec:LR_vs_SVM}, logistic regression and SVM are ``equivalent'' on separable dataset. Indeed, theoretically and empirically, there are advantages and disadvantages to both algorithms, summarized in \cref{tab:LR_vs_SVM}. In particular, SVM is preferred for theoretical analysis and precise 2-GMM simulation, while logistic regression is preferred for large scale real data analysis.
\begin{table}[h!]
% \renewcommand{\arraystretch}{1.2}
\centering
\begin{tabular}{rll}
    \hline
                 & \textbf{Pros} & \textbf{Cons} \\
    \hline
    \multirow{2}{*}{Logistic regression \eqref{eq:logistic}} 
    & robust to near-separability   & infinite-norm solution \\
    & computationally efficient       & slow convergence \\
    \hline
    \multirow{2}{*}{SVM \eqref{eq:SVM-0}}    
    & well-defined solution     & sensitive to outliers \\
    & support vectors available & quadratic programming \\
    \hline
\end{tabular}
\caption{Comparison of empirical behaviors of logistic regression and SVM on separable data.}
\label{tab:LR_vs_SVM}
\end{table}

% Note that the theoretical connection between SVM and logistic regression in \cite{rosset2003margin, Soudry_implicit_bias} does not include a intercept $\beta_0$. However, our extensive simulations and experiments reveals that the two remain equivalent numerically in the presence of intercept.

\paragraph{Datasets.} We provide the details of real data used in our study, including the source, size, and the preprocessing applied.
\begin{itemize}
    \item \textbf{IFNB} \cite{ifnb}: 
    single-cell RNA-seq dataset of peripheral blood mononuclear cells treated with interferon-$\beta$, which has $n=7,451$ cells, $d=2,000$ genes, and $K=13$ categories for cells. The original dataset is available from R package \texttt{SeuratData} (\url{https://github.com/satijalab/seurat-data}, version 0.2.2.9001) under the name \texttt{ifnb}. The data were preprocessed, normalized, and scaled by following the standard procedures by R package \texttt{Seurat} using functions \texttt{CreateSeuratObject}, \texttt{NormalizeData} and \texttt{ScaleData}.
    
    \item \textbf{CIFAR-10} \cite{KrizhevskyCIFAR102009}: the original dataset consists of 60,000 color images of size $32 \times 32$ in $K=10$ classes, with 6,000 images per class. There are 50,000 training images and 10,000 test images. It is available at \url{https://www.cs.toronto.edu/~kriz/cifar.html}. We followed the simple data augmentation in \cite{resnet, cao2019learning} for the training images: 4 pixels are padded on each side, and a $32 \times 32$ crop is randomly sampled from the padded image or its horizontal flip. Normalization is applied for both training and test images.
    
    \item \textbf{IMDb} \cite{IMDB}: the dataset consists of 50,000 movie reviews for binary sentiment classification ($K=2$), with the positive and negative reviews evenly distributed. There are 25,000 training texts and 25,000 test texts. The data can be found at \url{https://huggingface.co/datasets/stanfordnlp/imdb}. The maximum length in number of tokens for inputs was set as 512.
\end{itemize}

\paragraph{Pretrained models.} We downloaded and used pretrained models from Huggingface.
\begin{itemize}
    \item \textbf{ResNet-18} \cite{resnet}: 18-layer, 512-dim, 11.2M parameters, convolutional neural network (CNN), pretrained on CIFAR-10 training set (50,000 images). The pretrained model is downloaded from \url{https://huggingface.co/edadaltocg/resnet18_cifar10}. Notice that for extracting features, we manually removed the last fully-connected layer.
    \item \textbf{BERT} \cite{BERT}: 12-layer, 12-head, 768-dim, 110M parameters, encoder-only transformer, masked prediction, with absolute positional encoding at the input layer, pretrained on BooksCorpus (800M words) and English Wikipedia (2,500M words). The pretrained model is downloaded from \url{https://huggingface.co/google-bert/bert-base-uncased}.

    We also used a fine-tuned version of BERT (same structure as above) on IMDb dataset, which can be found at \url{https://huggingface.co/fabriceyhc/bert-base-uncased-imdb}.
\end{itemize}


\paragraph{Data splitting} For GMM simulated data and IFNB single-cell data, we split the whole dataset into training and test sets in equal proportions. For CIFAR-10 image data and IMDb movie review data, notice that we used the ResNet-18 and BERT model which are pretrained/fine-tuned on the training set of CIFAR-10 and IMDb, respectively. To avoid reusing the data when training the last fully-connected layer (i.e., logistic regression), we split the test set of CIFAR-10 and IMDb into a ``training subset'' and a ``test subset'' in equal proportions. We used this ``training subset'' for logistic regression training and ``test subset'' for evaluation.


% {We need a subsection ``experiment setup'' to discuss how we solving optimization problems: what python packages we used, what is the running time, we hyperparameter we chose, etc. If there are real data, explain the source of dataset, dataset size, what preprocessing is done, basic feature statistics, etc.}

% {Also be clear that we leveraged the connection between max-margin solution and ridgeless logistic regression, be clear about hyperparameter choice.}

\subsection{GMM simulation}

Figures~\ref{fig:GMM_main} and \ref{fig:Err_pi}---\ref{fig:reliability_GMM} are all generated from 2-GMM simulations. By rotational invariance, we may take $\bmu = (\mu, 0, \ldots, 0)^\top \in \R^d$ for some $\mu > 0$. Both minority and majority test errors are calculated on an independent balanced test set, to ensure the accuracy of estimating $\Err_+$.

\subsubsection{Rebalancing margin}

For \textbf{(ii) high imbalance regime}, we provide a simulation study by generating data from a 2-GMM model. More precisely, given $a, b, c > 0$, let
\begin{equation}\label{eq:abc_mod}
    \pi = C_\pi d^{-a},  \quad  \|\bmu\|_2^2 = C_{\mu} d^b,  \quad  n = C_n d^{c+1},
\end{equation}
for some fixed constant $C_\pi = 1, C_\mu = 0.75, C_n = 1$, where $\bmu = (\mu, 0, \ldots, 0)^\top \in \R^d$ and $\mu = \sqrt{C_{\mu} d^b}$. In the experiment, we fix $b = 0.3$, $c = 0.1$, and $d =2 000$ large enough to ensure data separability, while we change the value of $a$. For each tuple $(a, b, c)$, we compute the parameters $\pi, \bmu, n$ as per \cref{eq:abc_mod}, and generate training sets and test sets according to 2-GMM \cref{model}. 
% To reveal the importance of margin rebalancing, we choose $\tau = \tau_d = d^r$ for different values of $r \ge 0$. Then for each $\tau$, we train a margin-rebalanced SVM \cref{eq:SVM-m-reb}, and evaluate test errors $\hat\Err_+$, $\hat\Err_-$, and $\hat\Err_\mathrm{b}$ on the test set.

\subsubsection{Calibration}


The confidence reliability diagram \cref{fig:reliability_GMM} is created by partitioning $(0, 1]$ into $M$ interval bins $I_m := (\frac{m-1}{M}, \frac{m}{M}]$, $m \in [M]$, and calculating the average accuracy of each bin. Let $\hat p(\xx_i)$ be the confidence of the $i$-th test point ($i \in [n]$), and denote $\mathcal{B}_m := \{ i \in [n]: \hat p(\xx_i) \in I_m \}$ be the set of indices whose confidence falls into each bin. Then by our definition of confidence and the symmetry of binary classification, the accuracy and confidence of $\mathcal{B}_m$ can be estimated by
\begin{equation*}
    \hat{\mathrm{acc}}(\mathcal{B}_m) = \frac{1}{\abs{\mathcal{B}_m}} \sum_{i \in \mathcal{B}_m} \ind\{ y_i = 1 \},
    \qquad
    \hat{\mathrm{conf}}(\mathcal{B}_m) = \frac{1}{\abs{\mathcal{B}_m}} \sum_{i \in \mathcal{B}_m} \hat p(\xx_i).
\end{equation*}
We can also obtain a binning-based estimator of calibration error \cref{eq:CalErr} by using above quantities:
\begin{equation*}
    \hat{\mathrm{CalErr}} := \sum_{m=1}^M \frac{\abs{\mathcal{B}_m}}{n} \left( \hat{\mathrm{acc}}(\mathcal{B}_m) - \hat{\mathrm{conf}}(\mathcal{B}_m) \right)^2.
\end{equation*}
This is a variant of the prominent estimator called expected calibration error (ECE) \cite{guo2017calibration}.

The confidence reliability diagrams for additional 2-GMM simulations and IMDb movie review dataset are shown in Figures~\ref{fig:reliab_diag_mu=2}---\ref{fig:reliab_imdb}. These plots confirm a similar trend: miscalibration is getting worse when data becomes increasingly imbalanced (i.e., as $\pi$ decreases). 


% \begin{figure}[h]
%     \centering
%     \includegraphics[width=1\textwidth]{Figs/Calibration_pi=0.05, mu=1, n=1000, d=500.pdf}
%     \caption{
%     Reliability diagram for 2-GMM simulation ($\norm{\bmu}_2=1$, $n=1000$, $d=500$)
%     }
% \end{figure}
\begin{figure}[t]
    \centering
    \includegraphics[width=1\textwidth]{Figs/Calibration_pi=0.05,mu=2,n=1000,d=100.pdf}
    \caption{
    Reliability diagram for 2-GMM simulation ($\norm{\bmu}_2=2$, $n=1,\! 000$, $d=100$)
    }\label{fig:reliab_diag_mu=2}
\end{figure}
\begin{figure}[t]
    \centering
    \includegraphics[width=1\textwidth]{Figs/Calibration_pi=0.05,mu=0.5,n=1000,d=500.pdf}
    \caption{
    Reliability diagram for 2-GMM simulation ($\norm{\bmu}_2=0.5$, $n=1,\! 000$, $d=500$)
    }\label{fig:reliab_diag_mu=0.5}
\end{figure}
\begin{figure}[t]
    \centering
\includegraphics[width=1\textwidth]{Figs/Calibration_IMDb_BERT110M.pdf}
    \caption{
    Reliability diagram for IMDb dataset preprocessed by BERT base model (110M)
    }\label{fig:reliab_imdb}
\end{figure}

% \ljycom{The plot for single-cell and CIFAR-10 is not good, with non-monotone or reversed trend in $\pi$.}

\subsection{Function plot for the proximal operator $\prox_{\lambda \ell}(x)$}
\label{append_subsec_prox}
Recall that 
\begin{equation*}
    \prox_{\lambda \ell}(x) = \argmin_{t \in \R} \left\{ \ell(t) +  \frac1{2\lambda} (t - x)^2 \right\}.
\end{equation*}
We provide the plot for function $x \mapsto \prox_{\lambda \ell}(x)$, which is the specific form of overfitting effect in logistic regression \cref{eq:logistic} on non-separable data (i.e., $\delta > \delta^*(0)$). The plot is shown in \cref{fig:prox}, where $\ell(t) = \log(1 + e^{-t})$ is the logistic loss, and we choose $\lambda = 1, 5, 100$, and 10,000 for visualization. When $\lambda$ is close to zero, the function $x \mapsto \prox_{\lambda \ell}(x)$ is close to the identity map, which is because $\lim_{\lambda \to 0^+}\prox_{\lambda\ell}(x) = x$ by \cref{lem:prox}\ref{lem:prox(b)}. When $\lambda$ is large, the proximal operator (up to scaling) looks like a smooth approximation of the truncation map $x \mapsto \max\{ \kappa, x\}$ for some $\kappa > 0$. Intuitively, $\prox_{\lambda\ell}(x)$ behaves like minimizing $\ell$ when $\lambda$ is large. Therefore, a large $x$ yields $\prox_{\lambda\ell}(x) \approx x$ since $\ell(x) \approx 0$, and a small $x$ would be ``pushed'' to some $\kappa > 0$, since the logistic loss $\ell(x)$ locally is a smoothing of the hinge-type loss $x \mapsto (a - b x)_+$ for some $a, b > 0$.


According to our proof in \cref{append_sec:nonsep}, the limiting value of $\lambda$ as $n, d \to \infty$, $n/d \to \delta$ (denoted by $\lambda^* (\delta)$) is a decreasing function of the asymptotic aspect ratio $\delta$. Then \cref{fig:prox} graphically illustrates the effect of high-dimensionality on overfitting. When $n/d \to \delta$ is large, then $\lambda^*(\delta)$ is small and ELD $\approx$ TLD, and overfitting is negligible. In particular, this is the case for the classical setting where $d$ is fixed and $\delta = \infty$. When $\delta$ is moderate, the ELD is somewhat shrunken compared to TLD. When $\delta \downarrow \delta^*(0)$, approaching the interpolation threshold, then $\lambda^*(\delta)$ is very large, and the ELD is almost a rectified Gaussian and far away from the TLD.

\begin{figure}[h]
    \centering
    \includegraphics[width=1\textwidth]{Figs/prox_lambda.pdf}
    \caption{
    \textbf{Function plot for $x \mapsto \prox_{\lambda \ell}(x)$ under different $\lambda$.} The solid curve represents the function $y = \prox_{\lambda \ell}(x)$ and the dashed line represents the identity map $y = x$.
    }
    \label{fig:prox}
\end{figure}


\section{Preliminaries: Proofs for \Cref{sec:preliminary}}
\label{append_sec:prelim}
% \ljycom{Put it here temporarily...}
% \paragraph{On terminology.}
% \begin{itemize}
%     \item \textbf{Margin distribution}. We need some connection to this widely used terminology in ML literature, for example, \href{https://arxiv.org/abs/math/0405343}{[a]}
%     \href{https://arxiv.org/abs/1009.3613}{[b]}
%     \href{https://arxiv.org/abs/1706.08498}{[c]}
%     \href{https://arxiv.org/abs/1810.00113}{[d]}, 
%     which is defined as the empirical distribution of $y_i \hat f(\xx_i)$ in binary classification. But we prefer to call it logit distribution, to distinguish between logit and margin.
    
%     \item Indeed, people mix up the two meanings of the word \textbf{margin} (margin of an example v.s. margin for class), e.g., \cite{cao2019learning}.
% \end{itemize}

% \begin{rem}
%     Sometimes people informally (but incorrectly) call it ``truncated Gaussian distribution'', which in fact refers to the conditional distribution of $Z \,|\, Z \ge \kappa$ or $Z \,|\, Z \le \kappa$. More precisely, the rectified Gaussian distribution is essentially a mixture of a discrete distribution $\delta_\kappa$ and a continuous distribution (truncated Gaussian distribution with interval $(\kappa, \infty)$ or $(-\infty, \kappa)$). 
% \end{rem}

We first introduce some technical adjustments and terms that are used in our proofs.
\paragraph{Well-definedness of margin.}
We define the \emph{margin} of linear classifier $x \mapsto 2\ind\{ f(\xx) > 0 \} -1$ with $f(\xx) = \< \xx, \vbeta \> + \beta_0$ as 
\begin{equation}\label{eq:margin}
	\kappa = \kappa(\vbeta, \beta_0) := \min_{i \in [n]} \tilde{y}_i ( \< \xx_i, \bbeta \> + \beta_0 ),
\end{equation}
which is the objective of margin-rebalanced SVM \cref{eq:SVM-m-reb} and \eqref{eq:SVM}. Note there is a minor caveat about the one-class degenerate case, which is ignored in the main text for simplicity. When $n_+ = 0$ or $n$ (this happens with nonzero probability for any fixed $n$), we have $\kappa(\hat\vbeta, \hat\beta_0) = \infty$. It motivates us to redefine the maximum margin properly.
\begin{defn}\label{def:max-margin}
	The \emph{well-defined maximum margin} is
	\begin{equation}\label{eq:kappa_hat_def}
		\hat\kappa  := 
		\begin{cases} 
			\ \kappa(\hat\vbeta, \hat\beta_0) = \displaystyle\min_{i \in [n]} \tilde{y}_i ( \< \xx_i, \hat\bbeta \> + \hat\beta_0 ) , 
			& \ \text{if} \ 1 \le n_+ \le n - 1, \\
			\ 0    , 
			& \ \text{if} \ n_+ = 0 \text{ or } n. 
		\end{cases}
	\end{equation}
\end{defn}
Therefore, $\hat\kappa$ above is a proper random variable and $\hat\kappa \ge 0$ always holds\footnote{For degenerate case ($n_+ = 0$ or $n$), the dataset is considered as linearly separable.}. Further, $\hat{\kappa} = \kappa(\hat\vbeta, \hat\beta_0)$ with high probability as $n \to \infty$. We will apply similar adjustments to the definition of ELD in \cref{append_sec:sep} and elsewhere, whenever required for the proof.


\paragraph{Support vectors.} Consider the non-degenerate case ($1 \le n_+ \le n - 1$). To study the properties of optimal solution $(\hat\vbeta, \hat\beta_0, \hat\kappa)$ from a non-asymptotic perspective, we inherit the concept of \emph{support vectors} from SVM. Define the \emph{support vector of a linear classifier} $2\ind\{ \< \xx, \vbeta \> + \beta_0 > 0 \} -1$ as the vector(s) $\xx_i$ which attain(s) the smallest (rebalanced) logit margin $\wt y_i(\< \xx_i, \vbeta \> + \beta_0)$ from each class. Namely,
\begin{equation}\label{eq:SV_def}
	\begin{aligned}
		\mathcal{SV}_+ = \mathcal{SV}_+(\vbeta) & :=  \argmin_{i: y_i = +1} \wt y_i(\< \xx_i, \vbeta \> + \beta_0)
	= \argmin_{i: y_i = +1} + \< \xx_i, \vbeta \>, \\
		\mathcal{SV}_- = \mathcal{SV}_-(\vbeta) & :=  \argmin_{i: y_i = -1} \wt y_i(\< \xx_i, \vbeta \> + \beta_0)
	= \argmin_{i: y_i = -1} - \< \xx_i, \vbeta \>, \\
	\end{aligned}
\end{equation}
where $\mathcal{SV}_+, \mathcal{SV}_-$ are sets of (the indices of) \emph{positive} and \emph{negative support vectors}. A key observation from \cref{eq:SV_def} is that support vectors only depend on the data and parameter $\vbeta$, not $\beta_0$ or $\tau$.\footnote{Hence, we can view $\mathcal{SV}_\pm(\vbeta)$ as a mapping from $\R^d$ to the power set of $\{i: y_i = \pm 1\}$.} Let $\mathsf{sv}_+(\vbeta)$ and $\mathsf{sv}_-(\vbeta)$ be any element in $\mathcal{SV}_+(\vbeta)$ and $\mathcal{SV}_-(\vbeta)$, i.e.,
\begin{equation*}
	\mathsf{sv}_+(\vbeta) \in \mathcal{SV}_+(\vbeta),
	\qquad
	\mathsf{sv}_-(\vbeta) \in \mathcal{SV}_-(\vbeta),
\end{equation*} 
which are (the indices of) arbitrary positive and negative support vectors (only depends on $(\XX, \yy)$ and $\vbeta$). In particular, $\mathsf{sv}_+(\hat\vbeta) \in \mathcal{SV}_+(\hat\vbeta)$, $\mathsf{sv}_-(\hat\vbeta) \in \mathcal{SV}_-(\hat\vbeta)$ are support vectors of the max-margin classifier $2 \ind\{\< \xx, \hat\vbeta \> + \hat\beta_0 > 0\} - 1$, which aligns with the definition of support vectors in SVM.

\subsection{Proof of \cref{prop:SVM_tau_relation}}

The lemma below summarizes some important properties of the max-margin solution \cref{eq:SVM-m-reb} characterized by support vectors, which is a stronger statement than \cref{prop:SVM_tau_relation}.

\begin{lem}\label{lem:indep_tau}
	For non-degenerate case, let $(\hat\vbeta, \hat\beta_0, \hat\kappa)$ be an optimal solution to \cref{eq:SVM-m-reb}. Then
	\begin{enumerate}[label=(\alph*)]
		\item \label{lem:indep_tau(a)}
            $\hat\vbeta$ does NOT depend on $\tau$, and
		\begin{equation}\label{eq:kappa_hat}
			(\tau + 1) \hat\kappa = 
			\max_{\vbeta \in \S^{d-1}} \< \xx_{\mathsf{sv}_+(\vbeta)} - \xx_{\mathsf{sv}_-(\vbeta)}, \vbeta \>
			= 
			\< \xx_{\mathsf{sv}_+(\hat\vbeta)} - \xx_{\mathsf{sv}_-(\hat\vbeta)}, \hat\vbeta \>.
		\end{equation}
		\item \label{lem:indep_tau(b)}
            $\hat\beta_0$ depends on $\tau$ by
		\begin{equation}\label{eq:beta0_hat}
			\hat\beta_0 = -\frac{\tau \< \xx_{\mathsf{sv}_-(\hat\vbeta)}, \hat\vbeta \> + \< \xx_{\mathsf{sv}_+(\hat\vbeta)}, \hat\vbeta \>}{\tau + 1}.
		\end{equation}
		\item \label{lem:indep_tau(c)}
            If the data are linearly separable, then $(\hat\vbeta, \hat\beta_0)$ must be unique.
	\end{enumerate}
\end{lem}

\begin{proof}
	For any feasible solution $(\vbeta, \beta_0)$ of \cref{eq:SVM}, we denote the \emph{positive} and \emph{negative margin} of the classifier $\xx \mapsto 2\ind\{\< \xx, \vbeta \> + \beta_0 > 0 \}  -1$ as
	\begin{equation}\label{eq:margin_pm}
		\begin{aligned}
			\kappa_{+}(\vbeta, \beta_0)
			& := \min_{i: y_i = +1}\wt y_i(\< \xx_i, \vbeta \> + \beta_0)
			= \tau^{-1}(\< \xx_{\mathsf{sv}_+(\vbeta)}, \vbeta \> + \beta_0),  
			\\
			\kappa_{-}(\vbeta, \beta_0)
			& := \min_{i: y_i = -1}\wt y_i(\< \xx_i, \vbeta \> + \beta_0)
			= \mathmakebox[\widthof{$\tau^{-1}$}][r]{-\,} (\< \xx_{\mathsf{sv}_-(\vbeta)}, \vbeta \> + \beta_0).
		\end{aligned}
	\end{equation}
	%We apply the method of profiling for maximizing $\kappa(\vbeta, \beta_0)$. 
    According to \cref{eq:SVM}, we have
	\begin{equation*}
		\hat\vbeta = \argmax_{\vbeta \in \S^{d-1}} \kappa(\vbeta, \check\beta_0(\vbeta)) = \argmax_{\vbeta \in \S^{d-1}} \min_{i \in [n]} \tilde{y}_i ( \langle \xx_i, \bbeta \rangle + \check\beta_0(\vbeta) ),
	\end{equation*}
	where
	\begin{equation}
		\label{eq:beta0_optim}
		\begin{aligned}
			\check\beta_0(\vbeta) 
		: \! & = \argmax_{\beta_0 \in \R} \kappa(\vbeta, \beta_0)
		= \argmax_{\beta_0 \in \R} \min\limits_{i \in [n]} \tilde{y}_i ( \langle \xx_i, \bbeta \rangle + \beta_0 ) \\
		& = \argmax_{\beta_0 \in \R} \Bigl\{ \min_{i: y_i = +1}\wt y_i(\< \xx_i, \vbeta \> + \beta_0), \min_{i: y_i = -1}\wt y_i(\< \xx_i, \vbeta \> + \beta_0) \Bigr\}
		\\
		& = \argmax_{\beta_0 \in \R} \, \min\left\{ 
			\kappa_{+}(\vbeta, \beta_0),
			\kappa_{-}(\vbeta, \beta_0)
		 \right\}.
		\end{aligned}
	\end{equation}
	Here, $\check\beta_0(\vbeta)$ can be viewed as the optimal intercept for a linear classifier with slope given by $\vbeta$. 
	
	\vspace{0.5\baselineskip}
	\noindent
	\textbf{\ref{lem:indep_tau(b)}:}
	As defined in \cref{eq:margin_pm}, note $\min\{ 
		\kappa_{+}(\vbeta, \beta_0),
		\kappa_{-}(\vbeta, \beta_0) \}$ 
	is a piecewise linear concave function of $\beta_0$. Therefore, $\check\beta_0(\vbeta)$ must satisfy the \emph{margin-balancing} condition\footnote{
		As we have seen, the margin-balancing condition holds regardless of the sign of margin. It holds even if the data is not linearly separable.
	}, i.e., 
	 \begin{equation}\label{eq:margin-bal}
		\kappa_{+}(\vbeta, \check\beta_0(\vbeta)) = \kappa_{-}(\vbeta, \check\beta_0(\vbeta))
		= \kappa(\vbeta, \check\beta_0(\vbeta)).
	 \end{equation}
	In particular, recall that $\check\beta_0(\hat\vbeta) = \hat\beta_0$, then $\kappa_{+}(\hat\vbeta, \hat\beta_0) = \kappa_{-}(\hat\vbeta, \hat\beta_0)$. Substitute this back to \cref{eq:margin_pm} deduce
	\begin{equation*}
		\tau^{-1}(\< \xx_{\mathsf{sv}_+(\hat\vbeta)}, \hat\vbeta \> + \hat\beta_0)  
			= -(\< \xx_{\mathsf{sv}_-(\hat\vbeta)}, \hat\vbeta \> + \hat\beta_0),
	\end{equation*}
	which uniquely solves the expression for $\hat\beta_0$ in \cref{eq:beta0_hat}. This concludes the proof of part \ref{lem:indep_tau(b)}.

	\vspace{0.5\baselineskip}
	\noindent
	\textbf{\ref{lem:indep_tau(a)}:}
	Next, we show that $\hat\vbeta$ does not depend on $\tau$. According to \cref{eq:margin-bal} and \cref{eq:margin_pm},
	% Denote the optimization problem \cref{eq:SVM-epi} under a given $\tau$ as $\mathscr{P}_\tau$. Now suppose $(\hat\vbeta_{\tau_1}, \hat\beta_{0, \tau_1}, \hat\kappa_1)$ and $(\hat\vbeta_{\tau_2}, \hat\beta_{0, \tau_2}, \hat\kappa_2)$ are unique solutions to $\mathscr{P}_{\tau_1}, \mathscr{P}_{\tau_2}$, respectively, where $\tau_1 \not= \tau_2$.
	% \begin{equation*}
		\begin{align*}
			\hat\vbeta & = \argmax_{\vbeta \in \S^{d-1}} \kappa(\vbeta, \check\beta_0(\vbeta))  \\
		& =
		\argmax_{\vbeta \in \S^{d-1}} \frac{\tau \kappa_{+}(\vbeta, \check\beta_0(\vbeta)) +
		\kappa_{-}(\vbeta, \check\beta_0(\vbeta))}{\tau + 1} \\
		& = 
		\argmax_{\vbeta \in \S^{d-1}} \frac{\< \xx_{\mathsf{sv}_+(\vbeta)}, 
		\vbeta \> - \< \xx_{\mathsf{sv}_-(\vbeta)}, \vbeta \>}{\tau+1}
		= \argmax_{\vbeta \in \S^{d-1}} \< \xx_{\mathsf{sv}_+(\vbeta)} - \xx_{\mathsf{sv}_+(\vbeta)}, \vbeta \>,
		\end{align*}
	% \end{equation*}
	where $\< \xx_{\mathsf{sv}_+(\vbeta)} - \xx_{\mathsf{sv}_+(\vbeta)}, \vbeta \>$ only depends on $\vbeta$ and $(\XX, \yy)$ by definition. Hence, it deduces
	\begin{equation*}
		\begin{aligned}
			\hat\kappa & = \kappa(\hat\vbeta, \check\beta_0(\hat\vbeta)) 
		=
		\frac{\tau \kappa_{+}(\hat\vbeta, \check\beta_0(\hat\vbeta)) +
		\kappa_{-}(\hat\vbeta, \check\beta_0(\hat\vbeta))}{\tau + 1}
		=
		\frac{\< \xx_{\mathsf{sv}_+(\hat\vbeta)}, 
		\hat\vbeta \> - \< \xx_{\mathsf{sv}_-(\hat\vbeta)}, \hat\vbeta \>}{\tau+1}.
		\end{aligned}
	\end{equation*}
        This concludes the proof of part \ref{lem:indep_tau(a)}.

	\vspace{0.5\baselineskip}
	\noindent
	\textbf{\ref{lem:indep_tau(c)}:}
	Since \cref{eq:SVM-1} is a convex optimization problem with objective function $\norm{\bw}_2^2$, which is strictly convex in $\bw$, by equivalence between \cref{eq:SVM-m-reb}, \eqref{eq:SVM} and \eqref{eq:SVM-1}, we know that $\hat\bw$ and $\hat\vbeta = \hat\bw/\|\hat\bw\|_2$ must be unique. And by \ref{lem:indep_tau(a)}, $\hat\beta_0$ is also unique. This concludes the proof of part \ref{lem:indep_tau(c)}.
\end{proof}

Notice that \cref{lem:indep_tau} will also be used in the proof of \cref{lem:theta_hat_z} for the high imbalance regime. Below, we show that \cref{prop:SVM_tau_relation} is a direct consequence of \cref{lem:indep_tau}.

\begin{proof}[\textbf{Proof of \cref{prop:SVM_tau_relation}}]
We only show the relation on $\hat\kappa(\tau)$ and $\hat\beta_0(\tau)$, while the other results are simply restatements of \cref{lem:indep_tau}. According to \cref{eq:kappa_hat}, for any $\tau$, $(\tau + 1) \hat\kappa(\tau)$ equals a quantity which does not depend on $\tau$. Plugging in $\tau = 1$, we get $(\tau + 1) \hat\kappa(\tau) = 2 \hat\kappa(1)$.

Combining \cref{eq:kappa_hat} and \eqref{eq:beta0_hat}, we can solve
\begin{equation*}
    \< \xx_{\mathsf{sv}_+(\hat\vbeta)}, \hat\vbeta \> = \tau \hat\kappa(\tau) - \hat\beta_0(\tau), 
    \qquad
    \< \xx_{\mathsf{sv}_-(\hat\vbeta)}, \hat\vbeta \> = - \hat\kappa(\tau) - \hat\beta_0(\tau).
\end{equation*}
Notice the above holds for any $\tau > 0$. Taking $\tau = 1$ and substituting it into \cref{eq:beta0_hat}, we get
\begin{equation*}
\hat\beta_0(\tau) = -\frac{\tau \bigl(- \hat\kappa(1) - \hat\beta_0(1)\bigr) + \bigl(\hat\kappa(1) - \hat\beta_0(1) \bigr)}{\tau + 1} = \hat\beta_{0}(1) + \frac{\tau - 1}{\tau + 1} \hat\kappa(1).    
\end{equation*}
This completes the proof.
\end{proof}




\subsection{Proof of \cref{prop:explicit_bias}}
\begin{proof}[\textbf{Proof of \cref{prop:explicit_bias}}]
    Our argument follows the proof of \cite[Theorem 2.1]{Soudry_implicit_bias}. Assume that $\vbeta^*$ is a limit point of $\hat \vbeta_\lambda/\| \hat \vbeta_\lambda \|_2$ as $\lambda \to 0^+$, with $\| \vbeta^* \|_2 = 1$. The existence of $\vbeta^*$ is guaranteed by boundedness. Let $\beta_0^* := \limsup_{\lambda \to 0^+} \hat\beta_{0, \lambda}/\| \hat \vbeta_\lambda \|_2$. Now, suppose the max-margin classifier given by  $(\hat\vbeta, \hat\beta_0)$ (with $\| \hat\vbeta \|_2 = 1$) has a larger margin than $(\vbeta^*, \beta_0^*)$, that is,
    \begin{equation*}
        \kappa(\vbeta^*, \beta_0^*) = \min_{i \in [n]} y_i ( \< \xx_i, \vbeta^* \> + \beta_0^* )
        <
        \kappa(\hat\vbeta, \hat\beta_0) = \min_{i \in [n]} y_i ( \< \xx_i, \hat\vbeta \> + \hat\beta_0 ).
    \end{equation*}
    By continuity of $\kappa(\vbeta, \beta_0)$, there exists some open neighborhood of $(\vbeta^*, \beta_0^*)$:
    \begin{equation*}
        \mathcal{N}_{\vbeta^*, \beta_0^*} := \left\{
        \vbeta \in \R^d, \beta_0 \in R: \|\vbeta\|_2 = 1, \| \vbeta - \vbeta^* \|_2^2 + |\beta_0 - \beta_0^*|^2 < \delta^2 
        \right\}
    \end{equation*}
    and an $\varepsilon > 0$, such that
    \begin{equation*}
        \kappa(\vbeta, \beta_0) = \min_{i \in [n]} y_i ( \< \xx_i, \vbeta \> + \beta_0 ) < \kappa(\hat\vbeta, \hat\beta_0) - \varepsilon, \qquad \forall\, (\vbeta, \beta_0) \in \mathcal{N}_{\vbeta^*, \beta_0^*}.
    \end{equation*}
    Since $\ell$ is rapidly varying, now by \cite[Lemma 2.3]{Soudry_implicit_bias} we know that there exists some constant $T > 0$ (depends on $\kappa(\hat\vbeta, \hat\beta_0)$ and $\varepsilon$), such that
    \begin{equation*}
        \sum_{i=1}^n \ell\bigl(y_i ( \< \xx_i, t \hat\vbeta \> + t \hat\beta_0 ) \bigr)
        <
        \sum_{i=1}^n \ell\bigl(y_i ( \< \xx_i, t \vbeta \> + t \beta_0 ) \bigr)
        , \qquad \forall\, t > T , \  (\vbeta, \beta_0) \in \mathcal{N}_{\vbeta^*, \beta_0^*},
    \end{equation*}
    which implies $(t \hat\vbeta, t \hat\beta_0)$ has a smaller loss \cref{eq:logistic-l2} than $(t \vbeta, t \beta_0)$. This indicates that $\vbeta^*$ cannot be a limit point of $\hat \vbeta_\lambda/\| \hat \vbeta_\lambda \|_2$, which is a contradiction. Hence we must have $\kappa(\vbeta^*, \beta_0^*) = \kappa(\hat\vbeta, \hat\beta_0)$. Replacing $\limsup$ by $\liminf$ in the definition of $\beta_0^*$ gives the same conclusion. Then we complete the proof by noticing the max-margin solution is unique on separable data by \cref{lem:indep_tau}\ref{lem:indep_tau(c)}.
\end{proof}

\section{Logit distribution for separable data: Proofs for \cref{sec:logit_SVM}}
\label{append_sec:sep}

\subsection{Proof of \cref{thm:SVM_main}}
\label{append_subsec:sep}

Recall that the margin-rebalanced SVM can be rewritten as
\begin{equation}
    \label{eq:over_max-margin}
    \begin{array}{rl}
    \maximize\limits_{\bbeta \in \R^d, \, \beta_0, \kappa \in \R} & \kappa, \\
    \text{subject to} & \tilde{y}_i ( \< \xx_i, \bbeta \> + \beta_0 ) \ge \kappa,
	\quad \forall\, i \in [n], \\
	&  \norm{\bbeta}_2 \le 1.
    \end{array}
\end{equation}
Let $(\hat \vbeta_n, \hat \beta_{0, n})$ be an optimal solution and $\hat\kappa_n = \ind_{1 \le n_+ \le n - 1} \kappa(\hat \vbeta_n, \hat \beta_{0, n})$ be the well-defined maximum margin as per \cref{def:max-margin}. Our goal is to derive exact asymptotics for $(\hat \vbeta_n, \hat \beta_{0, n}, \hat\kappa_n)$. Similar to the development in \cite{montanari2023generalizationerrormaxmarginlinear}, for any positive margin $\kappa > 0$, we define the event
\begin{align*}
        \mathcal{E}_{n, \kappa} & = \bigl\{  \kappa(\hat \vbeta_n, \hat \beta_{0, n})  \ge \kappa \bigr\}  \\
        & =  \left\{ \text{$\exists\, \vbeta \in \R^d$, $\norm{\bbeta}_2 \le 1$, $\beta_0 \in \R$, such that $\tilde{y}_i \big( \langle \xx_i, \bbeta \rangle + \beta_0 \big) \ge \kappa$ for all $i \in [n]$} \right\} \\
        & = \left\{ \text{$\exists\, \vbeta \in \R^d$, $\norm{\bbeta}_2 \le 1$, $\beta_0 \in \R$, such that $\norm{ \left( \kappa \bs_\yy - \yy \odot \XX \vbeta  - \beta_0 \yy \right)_+ }_2 = 0$} \right\},
\end{align*}
where $\bs_\yy = (s(y_1), \dots, s(y_n))^\top$ and $s$ is the function defined in \cref{eq:s_fun}. Therefore, the data $(\XX, \yy)$ is linearly separable if and only if $\mathcal{E}_{n, \kappa}$ holds for some $\kappa > 0$. We would like to determine for which sets of parameters $(\pi, \vmu, \delta, \tau)$ we have
$\P(\mathcal{E}_{n, \kappa}) \to 1$ and for which instead $\P(\mathcal{E}_{n, \kappa}) \to 0$ as $n,d \to \infty$. To this end, we also define
\begin{equation}
    \label{eq:xi_n_kappa}
    \begin{aligned}
        \xi_{n, \kappa} & := \min_{ \substack{ \norm{\vbeta}_2 \le 1 \\ \beta_0 \in \R} } \frac{1}{\sqrt{d}} \norm{ \left( \kappa \bs_\yy - \yy \odot \XX \vbeta  - \beta_0 \yy \right)_+ }_2 \\
        & \phantom{:}\overset{\mathmakebox[0pt][c]{\text{(i)}}}{=} \min_{ \substack{ \norm{\vbeta}_2 \le 1 \\ \beta_0 \in \R} } \max_{ \substack{ \norm{\blambda}_2 \le 1 \\ \blambda \odot \yy \ge \bzero} } \frac{1}{\sqrt{d}} \blambda^\top \left( \kappa \bs_\yy \odot \yy - \XX \vbeta  - \beta_0 \bone \right),
    \end{aligned}
\end{equation}
where (i) is a consequence of Lagrange duality (dual norm) $\norm{(\ba)_+}_2 = \max_{\norm{\blambda}_2 \le 1, \blambda \ge \bzero} \blambda^\top \ba$. Then we established the following equivalence
\begin{equation*}
    \left\{ \xi_{n, \kappa} = 0 \right\} \Longleftrightarrow \mathcal{E}_{n ,\kappa}
    \qquad
    \left\{ \xi_{n, \kappa} > 0 \right\} \Longleftrightarrow \mathcal{E}^c_{n ,\kappa}.
\end{equation*}
Keep in mind that we are only concerned with the sign (positivity) of $\xi_{n, \kappa}$, not its magnitude. As a consequence, we have
\begin{equation*}
    \hat\kappa_n = \ind_{1 \le n_+ \le n - 1}\cdot \sup\{ \kappa \in \R:  \xi_{n,\kappa} = 0 \}.
\end{equation*}
Let $\mathcal{D}_n := \{ n_+ = 0 \text{ or } n \}$ be the event of degeneration for any datasets of size $n$. Clearly $\P(\mathcal{D}_n) = \pi^n + (1 - \pi)^n \to 0$ as $n \to \infty$. Technically, the empirical logit distribution (ELD) in \cref{eq:ELD} is not well-defined on $\mathcal{D}_n$. Similar as \cref{def:max-margin}, we can also redefine it as follows:
\begin{equation}\label{eq:over_ELD_well}
    \hat \nu_{n} := \frac1n \sum_{i=1}^n \delta_{(y_i, \< \xx_i, \hat\vbeta \> + \hat\beta_{0} ) \cdot \ind\{1 \le n_+ \le n - 1\} }.
\end{equation}



We provide an outline for the main parts of the proofs of \cref{thm:SVM_main}\ref{thm:SVM_main_trans}---\ref{thm:SVM_main_mar}, which involves several steps of transforming and simplifying the random variable $\xi_{n, \kappa}$.
\begin{equation*}
\begin{aligned}
    \xi_{n, \kappa}
    \, \xRightarrow[\text{\cref{lem:over_beta0}}]{\textbf{Step 1}} \, 
    \xi'_{n, \kappa, B}
    \, \xRightarrow[\text{\cref{lem:over_CGMT}}]{\textbf{Step 2}} \, 
    \xi'^{(1)}_{n, \kappa, B}
    \, \xRightarrow[\text{\cref{lem:over_ULLN}}]{\textbf{Step 3}} \, 
    \bar\xi'^{(2)}_{\kappa, B}
    \, \Rightarrow \,
    \bar\xi_{\kappa}^{(2)}
    \\
    \xRightarrow[\text{\cref{lem:over_sign}}]{\textbf{Step 4}} \, 
    \bar\xi_{\kappa}^{(3)}
    \, \Rightarrow \,
    \wt\xi_{\kappa}^{(3)}, F_\kappa(\rho, \beta_0)
    \, \xRightarrow[\text{\cref{lem:over_phase_trans}}]{\textbf{Step 5}} \, 
    \delta^*(\kappa), H_\kappa(\rho, \beta_0).
\end{aligned}
\left.
\vphantom{\begin{matrix} \dfrac12 \\ \dfrac12 \end{matrix}}
\right\} \text{\scriptsize{\cref{lem:over_mar_conp}}}
\end{equation*}

\paragraph{Step 1: Boundedness of the intercept (from $\xi_{n, \kappa}$ to $\xi'_{n, \kappa, B}$)} 
According to the definition of $\xi_{n, \kappa}$, parameters $\vbeta$ and $\blambda$ are optimized in compact sets, but $\beta_0$ is not. Such non-compactness might cause technical difficulties in the following steps, for example, when applying Gordon's Gaussian comparison inequality and establishing uniform convergence. However, it turns out that $\beta_0$ is asymptotically bounded on the event $\mathcal{E}_{n ,\kappa}$. More precisely, we define
\begin{equation}
    \label{eq:xi'_n_kappa_B}
    \xi'_{n, \kappa, B} := \min_{ \substack{ \norm{\vbeta}_2 \le 1 \\ \abs{\beta_0} \le B } } \max_{ \substack{ \norm{\blambda}_2 \le 1 \\ \blambda \odot \yy \ge 0} } \frac{1}{\sqrt{d}} \blambda^\top \left( \kappa \bs_\yy \odot \yy - \XX \vbeta  - \beta_0 \bone \right),
\end{equation}
where $B = B(\tau, \kappa, \pi, \norm{\bmu}_2, \delta)$ is a sufficiently large constant. Then we can show that $\xi_{n, \kappa}$ and $\xi'_{n, \kappa, B}$ have the same sign with high probability, which enables us to work with $\xi'_{n, \kappa, B}$ instead of $\xi_{n, \kappa}$.

\begin{lem}[Boundedness of $\beta_0$] 
\label{lem:over_beta0}    
There exists some constant $B \in (0, \infty)$ (depends on $\tau, \kappa, \pi, \norm{\bmu}_2, \delta$) such that
    \begin{equation*}
        \lim_{n \to \infty} \abs{ \P\bigl( \xi_{n, \kappa} = 0 \bigr) - \P\bigl(\xi'_{n, \kappa, B} = 0 \bigr) } = 0.
    \end{equation*}
\end{lem}
\noindent
See \cref{subsubsec:over_beta0} for the proof.

\paragraph{Step 2: Reduction via Gaussian comparison (from $\xi'_{n, \kappa, B}$ to $\xi'^{(1)}_{n, \kappa, B}$)} 
According to the expression of $\xi'_{n, \kappa, B}$, it is not hard to see the objective function (of $(\vbeta, \blambda)$) is a bilinear form of the Gaussian random matrix $\XX$. To simplify the bilinear term and make the calculation easier, we will use the convex Gaussian minimax theorem (CGMT, see \Cref{lem:CGMT}), i.e., Gordon's comparison inequality \cite{gordon1985some, thrampoulidis2015regularized}. To do so, we introduce another quantity:
\begin{equation}\label{eq:xi1_n_kappa_B}
    \xi_{n, \kappa, B}'^{(1)} := \min_{ \substack{ \rho^2 + \norm{\vtheta}_2^2 \le 1 \\ \abs{\beta_0} \le B } } \max_{ \substack{ \norm{\blambda}_2 \le 1 \\ \blambda \odot \yy \ge 0} } \frac{1}{\sqrt{d}}  \left(
    \norm{\blambda}_2 \vg^\top \btheta + \norm{\btheta}_2 \vh^\top \blambda + \blambda^\top \bigl( 
        \kappa \bs_\yy \odot \yy - \rho\norm{\bmu}_2 \yy + \rho \vu - \beta_0 \bone
     \bigr)
     \right),
\end{equation}
where $\rho \in \R$, $\vtheta \in \R^{d-1}$ are parameters, $\vg \sim \normal(\bzero, \bI_{d-1})$, $\hh \sim \normal(\bzero, \bI_{n})$, $\uu \sim \normal(\bzero, \bI_{n})$ are independent Gaussian vectors. The following lemma connects $\xi'_{n, \kappa, B}$ with $\xi_{n, \kappa, B}'^{(1)}$.

\begin{lem}[Reduction via CGMT] 
    \label{lem:over_CGMT}    
For any $v \in \R$ and $t \ge 0$,
    \begin{equation*}
        \P\Big( \big|\xi'_{n, \kappa, B} - v  \big| \ge t \Big) 
        \le 
        2 \P\Big( \big| \xi'^{(1)}_{n, \kappa, B} - v \big| \ge t \Big).
    \end{equation*}
\end{lem}
\noindent
See \cref{subsubsec:over_CGMT} for the proof.


\paragraph{Step 3: Dimension reduction (from $\xi'^{(1)}_{n, \kappa, B}$ to $\bar\xi'^{(2)}_{\kappa, B}$)} It turns out that $\xi'^{(1)}_{n, \kappa, B}$ can be further simplified for analytical purposes. We define a new (deterministic) quantity
\begin{equation*}
    \bar\xi'^{(2)}_{\kappa, B} :=  \min_{ \substack{ \rho^2 + r^2 \le 1, r \ge 0 \\  \abs{\beta_0} \le B } }
    -r + \sqrt{\delta} \left( \E\left[ \bigl(  s(Y) \kappa - \rho \norm{\bmu}_2 + \rho G_1 + rG_2 - \beta_0 Y \bigr)_+^2 \right] \right)^{1/2},
\end{equation*}
which is a constrained minimization over only three variables $\rho$, $r$, and $\beta_0$, with random variables $(Y, G_1, G_2) \sim P_y \times \normal(0, 1) \times \normal(0, 1)$. The two quantities of interest can be related via the uniform law of large numbers (ULLN) as shown in the following lemma.

\begin{lem}[ULLN]
\label{lem:over_ULLN}    
As $n,d \to \infty$, we have
    \begin{equation*}
        \xi'^{(1)}_{n, \kappa, B}  \conp \left( \bar\xi'^{(2)}_{\kappa, B} \right)_+.
    \end{equation*}
\end{lem}
\noindent
See \cref{subsubsec:over_ULLN} for the proof.


\paragraph{Step 4: Investigation of the positivity (from $\bar\xi'^{(2)}_{\kappa, B}$ to $\bar\xi^{(3)}_{\kappa}$)}
To further simplify the problem, we define the following quantities that are closely related to $\bar\xi'^{(2)}_{\kappa, B}$:
\begin{align}
        \bar\xi_{\kappa}^{(2)} & :=  \min_{ \substack{ \rho^2 + r^2 \le 1, r \ge 0 \\  \beta_0 \in  \R } }
    -r + \sqrt{\delta} \left( \E\left[ \bigl(  s(Y) \kappa - \rho \norm{\bmu}_2 + \rho G_1 + rG_2 - \beta_0 Y \bigr)_+^2 \right] \right)^{1/2} 
    \notag
    \\
        \bar\xi_{\kappa}^{(3)} & :=  \min_{ \substack{ \rho \in [-1, 1] \\  \beta_0 \in  \R } }
        -\sqrt{1 - \rho^2} + \sqrt{\delta} \left( \E\left[ \bigl(  s(Y) \kappa - \rho \norm{\bmu}_2 + G - \beta_0 Y \bigr)_+^2 \right] \right)^{1/2}.
        \label{eq:xi3_kappa}
\end{align}
Firstly, we argue that $\bar\xi'^{(2)}_{\kappa, B} = \bar\xi_{\kappa}^{(2)}$ for constant $B$ large enough, by noticing the optimal (unique) $\beta_0$ in $\bar\xi_{\kappa}^{(2)}$ is always bounded by some constant (depends on $\tau, \kappa, \pi, \norm{\bmu}_2, \delta$). Secondly, notice $\bar\xi_{\kappa}^{(3)}$ can be viewed as fixing $r = \sqrt{1 - \rho^2}$ in the optimization of $\bar\xi_{\kappa}^{(2)}$, and $G := \rho G_1 + \sqrt{1 - \rho^2} G_2 \sim \normal(0, 1)$. The following lemma shows that the sign won't change from $\bar\xi_{\kappa}^{(2)}$ to $\bar\xi_{\kappa}^{(3)}$.

\begin{lem}[Sign invariance] 
\label{lem:over_sign}    
For any $\kappa > 0$, the following result holds:
    \begin{enumerate}[label=(\alph*)]
        \item $\sign(\bar\xi_{\kappa}^{(2)}) = \sign(\bar\xi_{\kappa}^{(3)})$.
        \item If $\bar\xi_{\kappa}^{(2)} \le 0$, then $\bar\xi_{\kappa}^{(2)} = \bar\xi_{\kappa}^{(3)}$.
    \end{enumerate}
\end{lem}
\noindent
See \cref{subsubsec:over_positive} for the proof.



\paragraph{Step 5: Phase transition and margin convergence}
Note the function $\delta^*: \R \to \R_{\ge 0}$ defined in \cref{eq:sep_functions} is closely related to $\bar\xi_{\kappa}^{(3)}$. Let $\kappa^* := \sup\left\{ \kappa \in \R: \delta^*(\kappa) \ge \delta \right\}$. By combining the results from previous steps, we have the following relation.
\begin{lem}[Phase transition] 
\label{lem:over_phase_trans}
For any $\kappa > 0$, we have
\begin{equation*}
    \begin{aligned}
        \lim_{n \rightarrow \infty} \P\left( \xi_{n, \kappa} = 0 \right) = 1, \qquad & \text{if $\delta \le \delta^*(\kappa)$ (i.e., $\kappa \le \kappa^*$)}, \\
        \lim_{n \rightarrow \infty} \P\left( \xi_{n, \kappa} > 0 \right) = 1, \qquad & \text{if $\delta > \delta^*(\kappa)$ (i.e., $\kappa > \kappa^*$)}.
    \end{aligned}
\end{equation*}
In particular,
\begin{equation*}
    \begin{aligned}
        \lim_{n \rightarrow \infty} \P\left\{ \text{$(\XX, \yy)$ is linearly separable} \right\} = 1, \qquad & \text{if $\delta < \delta^*(0)$}, \\
        \lim_{n \rightarrow \infty} \P\left\{ \text{$(\XX, \yy)$ is not linearly separable} \right\} = 0, \qquad & \text{if $\delta > \delta^*(0)$}.
    \end{aligned}
\end{equation*}
\end{lem}
As a consequence, we can also derive the convergence of margin in probability. Notice that the following result is weaker than $\cL^2$ convergence \cref{thm:SVM_main}\ref{thm:SVM_main_mar}. However, we need this preliminary result for the subsequent proof of ELD convergence in \cref{lem:over_logit_conv}.
\begin{lem}[Margin convergence, in probability]
\label{lem:over_mar_conp}
If $\delta < \delta^*(0)$, we have $\hat\kappa_n \conp \kappa^*$.
\end{lem}
\noindent
See \cref{subsubsec:over_phase} for the proof.








\subsubsection{Step 1 --- Boundedness of the intercept: Proof of \cref{lem:over_beta0}}
\label{subsubsec:over_beta0}
\begin{proof}[\textbf{Proof of \cref{lem:over_beta0}}]
Recall that
\begin{equation*}
    \xi_{n, \kappa} = \min_{ \substack{ \norm{\vbeta}_2 \le 1 \\ \beta_0 \in \R} } \frac{1}{\sqrt{d}} \norm{ \left( \kappa \bs_\yy - \yy \odot \XX \vbeta  - \beta_0 \yy \right)_+ }_2.
\end{equation*}
Let $(\wt\vbeta_n, \wt\beta_{0, n})$ be a minimizer of the function above\footnote{
    In general $(\wt\vbeta_n, \wt\beta_{0, n})$ may not be unique and may not be equal to $(\hat\vbeta_n, \hat\beta_{0, n})$.
}. On the event $\mathcal{D}_n^c \cap \mathcal{E}_{n ,\kappa}$ ($\xi_{n, \kappa} = 0$), we have 
\begin{equation*}
        \bigl\| \bigl( \kappa \bs_\yy - \yy \odot \XX \wt\vbeta_n  -  \wt\beta_{0, n} \yy \bigr)_+  \bigr\|_2 = 0, 
        \quad
        \Longrightarrow
        \quad
        \begin{cases} 
            \ \tau \kappa - \< \xx_i, \wt\vbeta_n \> - \wt\beta_{0, n} \le 0, & \ \text{if} \ y_i = + 1, \\
            \ \mathmakebox[\widthof{$\tau\kappa$}][r]{\kappa} + \< \xx_i, \wt\vbeta_n \> + \wt\beta_{0, n} \le 0,      & \ \text{if} \ y_i = -1.
        \end{cases}
\end{equation*}
Write $\xx_i = y_i \bmu + \zz_i$, where $\zz_i \iidsim \normal(\bzero, \bI_d)$ and $y_i \indep \zz_i$. Then we obtain
\begin{equation*}
    \begin{cases} 
        \ \wt\beta_{0, n} \ge \mathmakebox[\widthof{$-$}][r]{\tau}
        \kappa - \< \bmu, \wt\vbeta_n \> - \< \zz_i, \wt\vbeta_n \> , & \ \text{if} \ y_i = + 1, \\
        \ \wt\beta_{0, n} \le    - \kappa + \< \bmu, \wt\vbeta_n \> - \< \zz_i, \wt\vbeta_n \> ,      & \ \text{if} \ y_i = -1,
    \end{cases}
\end{equation*}
which implies for all $i, j$ such that $y_i = +1, y_j = -1$,
\begin{equation*}
    \begin{aligned}
        | \wt\beta_{0, n} | & \le 
        \bigl| \tau \kappa - \< \bmu, \wt\vbeta_n \> - \< \zz_i, \wt\vbeta_n \> \bigr|
        + 
        \bigl| \kappa - \< \bmu, \wt\vbeta_n \> + \< \zz_j, \wt\vbeta_n \>  \bigr| \\
        & \le (\tau + 1)\kappa + 2\bigl| \< \bmu, \wt\vbeta_n \> \bigr| + 
        \bigl| \< \zz_i, \wt\vbeta_n \> \bigr| + \bigl| \< \zz_j, \wt\vbeta_n \> \bigr|.
    \end{aligned}
\end{equation*}
Using the inequality $(a+b+c)^2 \le 3(a^2 + b^2 + c^2)$, we have
\begin{equation*}
    \begin{aligned}
        | \wt\beta_{0, n} |^2 
        & \le
        3 \left\{  \bigl( (\tau + 1)\kappa + 2\bigl| \< \bmu, \wt\vbeta_n \> \bigr| \bigr)^2
        + 
        \min_{i: y_i = +1} \bigl| \< \zz_i, \wt\vbeta_n \> \bigr|^2 + 
        \min_{j: y_j = -1} \bigl| \< \zz_j, \wt\vbeta_n \> \bigr|^2  \right\} \\
        & \le 
        3 \, \biggl\{  \bigl( (\tau + 1)\kappa + 2\bigl| \< \bmu, \wt\vbeta_n \> \bigr| \bigr)^2
        + 
        \frac{1}{n_+}\sum_{i: y_i = +1} \bigl| \< \zz_i, \wt\vbeta_n \> \bigr|^2 + 
        \frac{1}{n_-}\sum_{j: y_j = -1} \bigl| \< \zz_j, \wt\vbeta_n \> \bigr|^2  \biggr\} \\
        & \overset{\mathmakebox[0pt][c]{\text{(i)}}}{=} 
        3 \, \biggl\{  \bigl( (\tau + 1)\kappa + 2\bigl| \< \bmu, \wt\vbeta_n \> \bigr| \bigr)^2
        + 
        \frac{1}{n_+} \bigl\| \ZZ_+ \wt\vbeta_n \bigr\|_2^2 + 
        \frac{1}{n_-} \bigl\| \ZZ_- \wt\vbeta_n \bigr\|_2^2  \biggr\} \\
        & \overset{\mathmakebox[0pt][c]{\text{(ii)}}}{\le}
        3 \, \biggl\{  \bigl( (\tau + 1)\kappa + 2 \norm{\bmu}_2 \bigr)^2
        + 
        \frac{1}{n_+} \norm{ \ZZ_+ }_{\mathrm{op}}^2 + 
        \frac{1}{n_-} \norm{ \ZZ_- }_{\mathrm{op}}^2  \biggr\} =: \wt B_{0, n},
    \end{aligned}
\end{equation*}
where in (i) we denote $\ZZ_+ \in \R^{n_+ \times d}$ as a Gaussian random matrix with rows $\zz_i$ such that $y_i = +1$, $\ZZ_- \in \R^{n_- \times d}$ with rows $\zz_j$ such that $y_j = +1$, while in (ii) we use Cauchy--Schwarz inequality, the definition of operator norm, and $\| \wt\vbeta_n \|_2 \le 1$. 

~\\
\noindent
Next, we show that $\wt B_{0, n}$ is asymptotically bounded. Notice $\ZZ_+, \ZZ_-$ have i.i.d. standard Gaussian entries. According to the tail bound of Gaussian matrices \cite[Corollary 7.3.3]{vershynin2018high}, for any $t_n \ge 0$ such that $t_n = o(\sqrt{n})$ and some absolute constants $c, C \in (0, \infty)$, we have
\begin{equation*}
    \begin{aligned}
        \wt B_{0, n} 
        & \overset{\mathmakebox[0pt][c]{\text{(i)}}}{\le} 
        3 \, \biggl\{  \bigl( (\tau + 1)\kappa + 2 \norm{\bmu}_2 \bigr)^2
        + 
        \frac{1}{n_+} \bigl(\sqrt{\smash[b]{n_+}} + \sqrt{d} + t_n \bigr)^2 + 
        \frac{1}{n_-} \bigl(\sqrt{\smash[b]{n_-}} + \sqrt{d} + t_n \bigr)^2  \biggr\} \\
        & \overset{\mathmakebox[0pt][c]{\text{(ii)}}}{\le}
        3 \, \biggl\{  \bigl( (\tau + 1)\kappa + 2 \norm{\bmu}_2 \bigr)^2
        + 
        \biggl(C + \frac{1}{\sqrt{\pi \delta}}       \biggr)^2 + 
        \biggl(C + \frac{1}{\sqrt{\smash[b]{(1 - \pi) \delta}}} \biggr)^2  \biggr\}
        =: B_{0} ,
    \end{aligned}
\end{equation*}
where (i) holds with probability as least $1 - 4 \exp(-c t_n^2)$, and (ii) holds with probability one based on the fact that $n_+/n \to \pi$, $n_-/n \to 1 - \pi$ a.s. (by strong law of large numbers), and $n/d \to \delta$ as $n \to \infty$. Notice the upper bound $B_0$ is a constant which depends on $(\tau, \kappa, \pi, \norm{\bmu}_2, \delta)$.
Let $t_n \to \infty$, then we conclude $\wt B_{0, n} \le  B_{0}$ with high probability.

~\\
\noindent
Combining these results, for any $B > \sqrt{B_0}$,
\begin{equation*}
    \left( \{ \xi_{n,\kappa} = 0 \} \cap \mathcal{D}_n^c \cap \{ \wt B_{0, n} \le  B_{0} \} \right) 
    \subseteq
    \left( \{ \xi_{n,\kappa} = 0 \} \cap \mathcal{D}_n^c \cap \{ | \wt\beta_{0, n} | \le B \} \right) 
    \subseteq
    \{ \xi'_{n,\kappa,B} = 0 \}.
\end{equation*}
Therefore, by union bound we have
\begin{equation*}
    \begin{aligned}
        \P\bigl(\xi_{n,\kappa} = 0\bigr) 
        & = \P\Bigl( \{ \xi_{n,\kappa} = 0 \} \cap 
        \bigl( \mathcal{D}_n^c \cap \{ \wt B_{0, n} \le  B_{0} \} \bigr)
         \Bigr)
         +
         \P\Bigl( \{ \xi_{n,\kappa} = 0 \} \cap 
        \bigl( \mathcal{D}_n \cup \{ \wt B_{0, n} >  B_{0} \} \bigr)
         \Bigr)
         \\
        & \le \P\bigl(\xi'_{n,\kappa,B} = 0\bigr) + \P(\mathcal{D}_n) + \P\bigl(\wt B_{0, n} > B_0\bigr).
    \end{aligned}
\end{equation*}
Finally, by noticing $\xi_{n,\kappa} \le \xi'_{n,\kappa,B}$, we conclude
\begin{equation*}
    0 \le \P\bigl(\xi_{n,\kappa} = 0\bigr) - \P\bigl(\xi'_{n,\kappa,B} = 0\bigr) \le 
    \P(\mathcal{D}_n) + \P\bigl(\wt B_{0, n} > B_0\bigr) \to 0,
    \qquad \text{as $n \to \infty$}.
\end{equation*}
This completes the proof.
\end{proof}














\subsubsection{Step 2 --- Reduction via Gaussian comparison: Proof of \cref{lem:over_CGMT}}
\label{subsubsec:over_CGMT}
\begin{proof}[\textbf{Proof of \cref{lem:over_CGMT}}]
Rewrite $\xx_i = y_i \bmu + \zz_i$, where $\zz_i \iidsim \normal(\bzero, \bI_d)$. Note that $y_i \indep \zz_i$. Denote the projection matrices
\begin{equation*}
    \bP_{\vmu} := \frac{1}{\norm{\vmu}_2^2} \vmu \vmu^\top,
    \qquad 
    \bP_{\vmu}^\perp := \bI_d - \frac{1}{\norm{\vmu}_2^2} \vmu \vmu^\top,
\end{equation*}
where $\bP_{\vmu}$ is the orthogonal projection onto $\spann\{ \vmu \}$ and $\bP_{\vmu}^\perp$ is the orthogonal projection onto the orthogonal complement of $\spann\{ \vmu \}$. Then we have the following decomposition:
\begin{equation*}
    \begin{aligned}
        \< \xx_i, \bbeta \> 
    & = y_i \< \bmu, \bbeta \> + \< \zz_i, \bbeta \> 
    = y_i \< \bmu, \bbeta \> + \< \zz_i, \bP_{\vmu} \bbeta \> + 
    \< \zz_i, \bP_{\vmu}^\perp \bbeta \>  \\
    & = y_i \left\< \bbeta, \frac{\bmu}{\norm{\bmu}_2} \right\> \norm{\bmu}_2 
    +  \left\< \bbeta, \frac{\bmu}{\norm{\bmu}_2} \right\> \left\< \zz_i, \frac{\bmu}{\norm{\bmu}_2} \right\>
    + \< \zz_i , \bP_{\bmu}^{\perp} \bbeta \> \\
    & = y_i \rho \norm{\vmu}_2 + \rho u_i + \< \zz_i , \bP_{\bmu}^{\perp} \bbeta \>,
    \end{aligned}
\end{equation*}
where
\begin{equation*}
    \rho := \left\< \bbeta, \frac{\bmu}{\norm{\bmu}_2} \right\>,
    \qquad
    u_i := \left\< \zz_i, \frac{\bmu}{\norm{\bmu}_2} \right\> \sim \normal(0, 1).
\end{equation*}
Let $\bQ \in \R^{n \times (n - 1)}$ be an orthonormal basis for the subspace $\spann\{ \vmu \}^\perp$ ($\bQ^\top \bQ = \bI_{n-1}$). Note that
\begin{equation*}
    \< \zz_i , \bP_{\bmu}^{\perp} \bbeta \>
    = \< \zz_i, \bQ \bQ^\top \bbeta \> 
    = \< \bQ^\top \zz_i,  \bQ^\top \bbeta \> 
    = \< \vg_i , \vtheta \>,
\end{equation*}
where
\begin{equation*}
    \begin{gathered}
        \vg_i := \bQ^\top \zz_i \sim \normal(\bzero, \bI_{d-1}),
    \qquad
    \vg_i \indep u_i,
    \\
    \vtheta := \bQ^\top \bbeta \in \R^{n-1},
    \qquad
    \norm{\vtheta}_2 
    = \sqrt{\norm{\vbeta}^2_2 - \norm{\bP_{\vmu}\vbeta}^2_2}
    \le \sqrt{1 - \rho^2}.
    \end{gathered}
\end{equation*}
We obtain a one-to-one map $\vbeta \leftrightarrow (\rho, \vtheta)$ in the unit ball. Therefore, we can reparametrize
\begin{equation*}
    \< \xx_i, \bbeta \> + \beta_0 \overset{\mathrm{d}}{=} y_i \rho \norm{\vmu}_2 - \rho u_i - \< \vg_i , \vtheta \> + \beta_0,
\end{equation*}
where $\rho^2 + \norm{\vtheta}_2^2 \le 1$, and $\{(y_i, u_i, \vg_i)\}_{i = 1}^n$ are i.i.d., each has joint distribution:
\begin{equation*}
    y_i \indep u_i \indep \vg_i,
    \qquad
    \P(y_i=+1) = 1 - \P(y_i=-1) = \pi,
    \quad
    u_i \sim \normal(0, 1),
    \quad
    \vg_i \sim \normal(\bzero, \bI_{d-1}).
\end{equation*}
Now denote
\begin{equation*}
    \uu = (u_1, \dots, u_n)^\top \in \R^{n},
    \qquad
    \GG = (\vg_1, \dots, \vg_n)^\top \in \R^{n \times (d-1)}.
\end{equation*}
Therefore, $\xi'^{(0)}_{n,\kappa, B} := \xi'_{n,\kappa, B}$ defined in \cref{eq:xi'_n_kappa_B} can be written as
\begin{align*}
        \xi'^{(0)}_{n,\kappa,B}
        & = 
        \min_{ \substack{ \norm{\vbeta}_2 \le 1 \\ \abs{\beta_0} \le B} } \max_{ \substack{ \norm{\blambda}_2 \le 1 \\ \blambda \odot \yy \ge \bzero} } \frac{1}{\sqrt{d}} \blambda^\top \left( \kappa \bs_\yy \odot \yy - \XX \vbeta  - \beta_0 \bone \right) \\
        & \overset{\mathmakebox[0pt][c]{\mathrm{d}}}{=} \min_{ \substack{ \rho^2 + \norm{\vtheta}_2^2 \le 1 \\  \abs{\beta_0} \le B } } \max_{ \substack{ \norm{\blambda}_2 \le 1 \\ \blambda \odot \yy \ge \bzero} } \frac{1}{\sqrt{d}} \blambda^\top \left( \kappa \bs_\yy \odot \yy - \rho \norm{\bmu}_2 \yy + \rho \uu + \GG \btheta  - \beta_0 \bone \right) \\
        & = \min_{ \substack{ \rho^2 + \norm{\vtheta}_2^2 \le 1 \\  \abs{\beta_0} \le B } } \max_{ \substack{ \norm{\blambda}_2 \le 1 \\ \blambda \odot \yy \ge \bzero} } \frac{1}{\sqrt{d}} \left( \blambda^\top \GG \btheta +
        \blambda^\top ( \kappa \bs_\yy \odot \yy - \rho \norm{\bmu}_2 \yy + \rho \uu   - \beta_0 \bone )
         \right).
        % \\
        % & =: \min_{ \substack{ \rho^2 + \norm{\vtheta}_2^2 \le 1 \\ \beta_0 \in \R } } \max_{ \substack{ \norm{\blambda}_2 \le 1 \\ \blambda \odot \yy \ge \bzero} } Q_0(\GG ; \rho, \vtheta, \blambda).
\end{align*}
On the other hand, recall $\xi_{n,\kappa}^{(1)}$ defined in \cref{eq:xi1_n_kappa_B}:
\begin{equation*}
    \xi'^{(1)}_{n, \kappa, B}
        = \min_{ \substack{ \rho^2 + \norm{\vtheta}_2^2 \le 1 \\  \abs{\beta_0} \le B } } \max_{ \substack{ \norm{\blambda}_2 \le 1 \\ \blambda \odot \yy \ge \bzero} } \frac{1}{\sqrt{d}}  \left(
    \norm{\blambda}_2 \vg^\top \btheta + \norm{\btheta}_2 \vh^\top \blambda + \blambda^\top \bigl( 
        \kappa \bs_\yy \odot \yy - \rho\norm{\bmu}_2 \yy + \rho \vu - \beta_0 \bone
     \bigr)
     \right).
    %  \\
    %  & =:  \min_{ \substack{ \rho^2 + \norm{\vtheta}_2^2 \le 1 \\ \beta_0 \in \R } } \max_{ \substack{ \norm{\blambda}_2 \le 1 \\ \blambda \odot \yy \ge \bzero} } Q_1(\vg, \hh; \rho, \vtheta, \blambda).
\end{equation*}
%%%%%%%%%%%%%%%%%%%%%%%%%%%%%%%%%%%%%%%%%%
% Deprecated: CGMT for unbounded \beta_0 % 
%%%%%%%%%%%%%%%%%%%%%%%%%%%%%%%%%%%%%%%%%%
% We couldn't apply CGMT to $\xi^{(0)}_{n,\kappa}$ and $\xi_{n,\kappa}^{(1)}$ directly since $\beta_0$ is not minimized on a compact set. To overcome this, for each integer $k \ge 1$, we define
% \begin{equation*}
%     M^{(0)}_{n, \kappa, k} :=  \min_{ \substack{ \rho^2 + \norm{\vtheta}_2^2 \le 1 \\ \abs{\beta_0} \le k } } \max_{ \substack{ \norm{\blambda}_2 \le 1 \\ \blambda \odot \yy \ge \bzero} } Q_0(\GG ; \rho, \vtheta, \blambda),
%     \qquad
%     M^{(1)}_{n, \kappa, k} :=
%     \min_{ \substack{ \rho^2 + \norm{\vtheta}_2^2 \le 1 \\ \abs{\beta_0} \le k } } \max_{ \substack{ \norm{\blambda}_2 \le 1 \\ \blambda \odot \yy \ge \bzero} } Q_1(\vg, \hh; \rho, \vtheta, \blambda).
% \end{equation*}
Note that both minimization and maximization above are defined over compact and convex constraint sets, and the objective function in $\xi'^{(0)}_{n, \kappa, B}$ is a bilinear in $(\vtheta, \blambda)$ (not $\beta_0$). In addition, $(\yy, \uu)$ is independent of $\GG, (\vg, \hh)$, so we can apply a variant of CGMT (\cref{lem:CGMT}) by conditioning on $(\yy, \uu)$, which yields for any $v \in \R$ and $t \ge 0$:
\begin{align*}
    & \P\left( \xi'^{(0)}_{n, \kappa, B} \le v+t \,|\, \yy, \uu \right) \le 2\, \P\left( \xi'^{(1)}_{n, \kappa, B} \le v+t \,|\, \yy, \uu \right),
    \\
    & \P\left( \xi'^{(0)}_{n, \kappa, B} \ge v-t \,|\, \yy, \uu \right) \le 2\, \P\left( \xi'^{(1)}_{n, \kappa, B} \ge v-t \,|\, \yy, \uu \right).
\end{align*}
Taking expectation over $(\yy, \uu)$ on both sides of the equation gives for any $v \in \R$ and $t \ge 0$:
\begin{equation*}
    \P\left( \xi'^{(0)}_{n, \kappa, B} \le v+t \right) \le 2\, \P\left( \xi'^{(1)}_{n, \kappa, B} \le v+t \right),
    \quad
    \P\left( \xi'^{(0)}_{n, \kappa, B} \ge v-t \right) \le 2\, \P\left( \xi'^{(1)}_{n, \kappa, B} \ge v-t \right),
\end{equation*}
which proves \Cref{lem:over_CGMT}.
% \begin{equation*}
%     \P\bigl( M^{(0)}_{n, \kappa, k} \le t \,|\, \yy, \uu \bigr) \le 2\, \P\bigl( M^{(1)}_{n, \kappa, k} \le t \,|\, \yy, \uu \bigr),
%     \qquad
%     \P\bigl( M^{(0)}_{n, \kappa, k} \ge t \,|\, \yy, \uu \bigr) \le 2\, \P\bigl( M^{(1)}_{n, \kappa, k} \ge t \,|\, \yy, \uu \bigr).
% \end{equation*}
% Taking expectation over $(\yy, \uu)$ on both sides of the equation gives for any $t \in \R$:
% \begin{equation*}
%     \P\bigl( M^{(0)}_{n, \kappa, k} \le t \bigr) \le 2\, \P\bigl( M^{(1)}_{n, \kappa, k} \le t \bigr),
%     \qquad
%     \P\bigl( M^{(0)}_{n, \kappa, k} \ge t \bigr) \le 2\, \P\bigl( M^{(1)}_{n, \kappa, k} \ge t \bigr).
% \end{equation*}
% Note $M^{(0)}_{n, \kappa, k} \searrow \xi^{(0)}_{n, \kappa}$ (in distribution) and $M^{(1)}_{n, \kappa, k} \searrow \xi^{(1)}_{n, \kappa}$ as $k \to \infty$. Hence, taking $k \to \infty$ in both inequalities above, we obtain
% \begin{equation*}
%     \P\bigl( \xi^{(0)}_{n, \kappa} \le t \bigr) \le 2\, \P\bigl( \xi^{(1)}_{n, \kappa} \le t \bigr),
%     \qquad
%     \P\bigl( \xi^{(0)}_{n, \kappa} \ge t \bigr) \le 2\, \P\bigl( \xi^{(1)}_{n, \kappa} \ge t \bigr).
% \end{equation*}
\end{proof}










\subsubsection{Step 3 --- Dimension reduction: Proof of \cref{lem:over_ULLN}}
\label{subsubsec:over_ULLN}
\begin{proof}[\textbf{Proof of \cref{lem:over_ULLN}}]
The expression of $\xi'^{(1)}_{n, \kappa, B}$ can be further simplified to
\begin{align*}
        \xi'^{(1)}_{n, \kappa, B}
        & = \min_{ \substack{ \rho^2 + \norm{\vtheta}_2^2 \le 1 \\ \abs{\beta_0} \le B } } \max_{ \substack{ \norm{\blambda}_2 \le 1 \\ \blambda \odot \yy \ge \bzero} } \frac{1}{\sqrt{d}} \left( \norm{\blambda}_2 \vg^\top \btheta +
        \blambda^\top ( \kappa \bs_\yy \odot \yy - \rho \norm{\bmu}_2 \yy + \rho \uu 
        + \norm{\btheta}_2 \vh - \beta_0 \bone )  \right)  \\
        & \overset{\mathmakebox[0pt][c]{\text{(i)}}}{=} 
        \min_{ \substack{ \rho^2 + \norm{\vtheta}_2^2 \le 1 \\ \abs{\beta_0} \le B } } \frac{1}{\sqrt{d}} \left( \vg^\top \btheta + \bigl\| \left( 
            \kappa \bs_\yy - \rho \norm{\bmu}_2 + \rho \uu \odot \yy
        + \norm{\vtheta}_2 \vh \odot \yy - \beta_0 \yy
         \right)_+ \bigr\|_2  \right)_+ \\
        & \overset{\mathmakebox[0pt][c]{\text{(ii)}}}{=}
         \min_{ \substack{ \rho^2 + r^2 \le 1 \\ r \ge 0, \abs{\beta_0} \le B } } \frac{1}{\sqrt{d}} \left( 
            - r \norm{\vg}_2 + \bigl\| \left( 
            \kappa \bs_\yy - \rho \norm{\bmu}_2 + \rho \uu \odot \yy
        + r \vh \odot \yy - \beta_0 \yy
         \right)_+ \bigr\|_2  \right)_+,
\end{align*}
where in (i) we use the fact
\begin{equation*}
    \begin{aligned}
        \max_{\norm{\blambda}_2 \le 1, \blambda \ge \bzero} 
    \left( a \norm{\blambda}_2 + \blambda^\top \mathrm{\bf b} \right)
    & = \max_{r \in [0, 1]} \max_{\norm{\bv}_2 = 1, \bv \ge \bzero} 
    r \bigl( a + \bv^\top \mathrm{\bf b} \bigr)
    = \left( \max_{\norm{\bv}_2 = 1, \bv \ge \bzero} \bigl( a + \bv^\top \mathrm{\bf b} \bigr) \right)_+
    \\
    & = \Bigl( a + \norm{(\mathrm{\bf b})_+}_2 \Bigr)_+,
    \end{aligned}
\end{equation*}
in (ii) we use Cauchy--Schwarz inequality $\vg^\top \btheta \ge - \norm{\vtheta}_2 \norm{\vg}_2$ and denote $r = \norm{\vtheta}_2$. For convenience, we write the parameter space as $\bar\Theta_{B} := \{ (\rho, r, \beta_0): \rho^2 + r^2 \le 1, r \ge 0, \abs{\beta_0} \le B \}$. Now, define
\begin{align*}
        \bar \xi'^{(1)}_{n, \kappa, B} & := \min_{ (\rho, r, \beta_0) \in \bar\Theta_{B} } \frac{1}{\sqrt{d}} \left( 
            - r \norm{\vg}_2 + \bigl\| \left( 
            \kappa \bs_\yy - \rho \norm{\bmu}_2 + \rho \uu \odot \yy
        + r \vh \odot \yy - \beta_0 \yy
         \right)_+ \bigr\|_2  \right) \\
         & \phantom{:}= 
         \min_{ (\rho, r, \beta_0) \in \bar\Theta_{B} }
         \left\{ 
            -r\frac{\norm{\vg}_2}{\sqrt{d}}
            + \sqrt{\frac{n}{d}} \sqrt{\frac{1}{n} \sum_{i=1}^n \bigl( s(y_i) \kappa - \rho \norm{\bmu}_2 + \rho u_i y_i
            + r h_i y_i - \beta_0 y_i \bigr)_+^2 }
         \right\}
         \\
         & \phantom{:} =\mathmakebox[0pt][c]{:} \min_{ (\rho, r, \beta_0) \in \bar\Theta_{B} } f^{(1)}_{n,\kappa}(\rho,r,\beta_0)  ,
\end{align*}
then $\xi'^{(1)}_{n, \kappa, B} = \bigl( \bar \xi'^{(1)}_{n, \kappa, B} \bigr)_+$. Recall that
\begin{align*}
        \bar\xi'^{(2)}_{\kappa, B} & = \min_{ (\rho, r, \beta_0) \in \bar\Theta_{B} }
    -r + \sqrt{\delta} \left( \E\left[ \bigl(  s(Y) \kappa - \rho \norm{\bmu}_2 + \rho Y G_1 + r Y G_2 - \beta_0 Y \bigr)_+^2 \right] \right)^{1/2} \\
    & = \min_{ (\rho, r, \beta_0) \in \bar\Theta_{B} }
    -r + \sqrt{\delta} \left( \E\left[ \bigl(  s(Y) \kappa - \rho \norm{\bmu}_2 + \rho G_1 + r G_2 - \beta_0 Y \bigr)_+^2 \right] \right)^{1/2} \\
    & =\mathmakebox[0pt][c]{:} \min_{ (\rho, r, \beta_0) \in \bar\Theta_{B} } f^{(2)}_{\kappa}(\rho,r,\beta_0),
\end{align*}
where $Y \indep G_1 \indep G_2$, $\P(Y = +1) = 1 - \P(Y = -1) = \pi$, and $G_1, G_2 \sim \normal(0, 1)$.
We also define
\begin{equation*}
    \begin{aligned}
        \phi^{(1)}_{n, \kappa}(\rho, r, \beta_0) & := \frac1n \sum_{i=1}^n \bigl( s(y_i) \kappa - \rho \norm{\bmu}_2 + \rho u_i y_i
        + r h_i y_i - \beta_0 y_i \bigr)_+^2
        = : \E_n\left[ f(Y, G_1, G_2; \rho, r, \beta_0) \right] , \\
        \phi^{(2)}_{\kappa}(\rho, r, \beta_0) & := \E\left[  \bigl(  s(Y) \kappa - \rho \norm{\bmu}_2 + \rho G_1 Y
        + r G_2 Y - \beta_0 Y \bigr)_+^2 \right]
        = : \mathmakebox[\widthof{$\E_n$}][l]{\E}\left[ f(Y, G_1, G_2 ; \rho, r, \beta_0) \right] ,
    \end{aligned}
\end{equation*}
where $\E_n[\cdot]$ denotes the expectation over the empirical distribution of $\{ (y_i, u_i, h_i) \}_{i = 1}^n$. In order to apply the uniform law of large numbers (ULLN), note that 
\begin{itemize}
    \item $\bar\Theta_{B}$ is compact. $(\rho, r, \beta_0) \mapsto f$ is continuous in $\bar\Theta_{B}$ for each $(Y, G_1, G_2)$, and $(Y, G_1, G_2) \mapsto f$ is measurable for each $(\rho, r, \beta_0)$
    \item $\abs{f(Y, G_1, G_2 ; \rho, r, \beta_0)} \le 3 \left( (\kappa\tau + \norm{\vmu}_2 + B)^2 + G_1^2 + G_2^2 \right)$ for all $(\rho, r, \beta_0) \in \bar\Theta_{B}$ and $\E[G_1^2] = \E[G_2^2] = 1 < \infty$.
\end{itemize}
Therefore, by ULLN \cite[Lemma 2.4]{newey1994large}, we have
\begin{align*}
    & \sup_{ (\rho, r, \beta_0) \in \bar\Theta_{B} }
    \abs{ \bigl( \phi^{(1)}_{n, \kappa}(\rho, r, \beta_0) \bigr)^{1/2} - \bigl( \phi^{(2)}_{\kappa}(\rho, r, \beta_0) \bigr)^{1/2} } \\
    \le {} & \sup_{ (\rho, r, \beta_0) \in \bar\Theta_{B} }
    \abs{ \phi^{(1)}_{n, \kappa}(\rho, r, \beta_0) -  \phi^{(2)}_{\kappa}(\rho, r, \beta_0) }^{1/2} = o_{\P}(1),
\end{align*}
where the inequality comes from the fact that $x \mapsto \sqrt{x}$ is $1/2$-Hölder continuous on $[0, \infty)$. Then
\begin{align*}
        & \sup_{ (\rho, r, \beta_0) \in \bar\Theta_{B} } \abs{ f^{(1)}_{n,\kappa}(\rho,r,\beta_0) - f^{(2)}_{\kappa}(\rho,r,\beta_0)} \\
        \le {} &  \sup_{r \in [-1, 1]} \abs{r - r\frac{\norm{\vg}_2}{\sqrt{d}}} + 
        \sup_{ (\rho, r, \beta_0) \in \bar\Theta_{B} } \abs{ \sqrt{\frac{n}{d}} \bigl( \phi^{(1)}_{n, \kappa}(\rho, r, \beta_0) \bigr)^{1/2}
        -  \sqrt{\delta} \bigl( \phi^{(2)}_{\kappa}(\rho, r, \beta_0) \bigr)^{1/2}  }  \\
        \le {} & \abs{1 - \frac{\norm{\vg}_2}{\sqrt{d}}} +  
        \sqrt{\frac{n}{d}}
        \sup_{ (\rho, r, \beta_0) \in \bar\Theta_{B} }
        \abs{ \bigl( \phi^{(1)}_{n, \kappa}(\rho, r, \beta_0) \bigr)^{1/2} - \bigl( \phi^{(2)}_{\kappa}(\rho, r, \beta_0) \bigr)^{1/2} } \\
        {} &  
        + \abs{ \sqrt{\frac{n}{d}} - \sqrt{\delta} } \sup_{ (\rho, r, \beta_0) \in \bar\Theta_{B} }\bigl( \phi^{(2)}_{\kappa}(\rho, r, \beta_0) \bigr)^{1/2} \\
        = {} & o_{\P}(1),
\end{align*}
by using $n/d \to \delta$ and law of large numbers $\norm{\vg}_2^2/(d - 1) \conp 1$. Finally, since the function $x \mapsto (x)_+$ is $1$-Lipschitz, we conclude
\begin{equation*}
    \Bigl| \xi'^{(1)}_{n, \kappa, B} - \bigl( \bar\xi'^{(2)}_{\kappa, B} \bigr)_+  \Bigr|
    \le
    \abs{ \bar\xi'^{(1)}_{n, \kappa, B} - \bar\xi'^{(2)}_{\kappa, B} }
    \le \sup_{ (\rho, r, \beta_0) \in \bar\Theta_{B} } \abs{ f^{(1)}_{n,\kappa}(\rho,r,\beta_0) - f^{(2)}_{\kappa}(\rho,r,\beta_0)} = o_{\P}(1).
\end{equation*}
This completes the proof.
\end{proof}









\subsubsection{Step 4 --- Investigation of the positivity: Proof of \cref{lem:over_sign}}
\label{subsubsec:over_positive}
\begin{proof}[\textbf{Proof of \cref{lem:over_sign}}]
We claim $\bar\xi'^{(2)}_{\kappa, B} = \bar\xi_{\kappa}^{(2)}$ when $B$ is large enough. Recall that
\begin{equation*}
        \bar\xi_{\kappa}^{(2)} =  \min_{ \substack{ \rho^2 + r^2 \le 1, r \ge 0 \\  \beta_0 \in  \R } }
    -r + \sqrt{\delta} \left( \E\left[ \bigl(  s(Y) \kappa - \rho \norm{\bmu}_2 + \rho G_1 + rG_2 - \beta_0 Y \bigr)_+^2 \right] \right)^{1/2}.
\end{equation*}
Let $(\wt\rho, \wt r, \wt\beta_{0})$ be a minimizer above and notice $\wt\rho G_1 + \wt r G_2 \overset{\mathrm{d}}{=} \wt R G$, where $\wt R = \sqrt{\wt\rho^2 + \wt r^2}$ and $G \sim \normal(0, 1)$. Then
\begin{align*}
        \wt\beta_{0} & \in \phantom{:} \argmin_{\beta_0 \in \R} \E\left[ \bigl(  s(Y) \kappa - \wt\rho \norm{\bmu}_2 + \wt R G - \beta_0 Y \bigr)_+^2 \right] \\
        & = \phantom{:} \argmin_{\beta_0 \in \R} \left\{ 
            \pi \E\left[ \bigl( \tau \kappa - \wt\rho \norm{\bmu}_2 + \wt R G - \beta_0 \bigr)_+^2 \right]
            + (1 - \pi) \E\left[ \bigl( \kappa - \wt\rho \norm{\bmu}_2 + \wt R G + \beta_0 \bigr)_+^2 \right] \right\} \\
        & =: \argmin_{\beta_0 \in \R} g_{\wt\rho, \wt r}(\beta_0).
\end{align*}
Notice that $g_{\wt\rho, \wt r}(\beta_0)$ is convex and continuously differentiable, since
\begin{equation*}
    g'_{\wt\rho, \wt r}(\beta_0)
    = -2\pi \E\left[ \bigl( \tau \kappa - \wt\rho \norm{\bmu}_2 + \wt R G - \beta_0 \bigr)_+ \right]
    + 2(1 - \pi) \E\left[ \bigl( \kappa - \wt\rho \norm{\bmu}_2 + \wt R G + \beta_0 \bigr)_+ \right] 
\end{equation*}
is non-decreasing, which is based on the fact that $x \mapsto \E[(G + x)_+]$ is increasing. Then $\wt\beta_{0}$ must satisfy $g'_{\wt\rho, \wt r}(\wt\beta_{0}) = 0$. Since $g'_{\wt\rho, \wt r}(+\infty) = +\infty$, $g'_{\wt\rho, \wt r}(-\infty) = -\infty$, by our construction in the proof of \cref{lem:over_beta0}, we can choose $B$ large enough such that $\bar\xi'^{(2)}_{\kappa, B} = \bar\xi_{\kappa}^{(2)}$.


~\\
\noindent
We can rewrite $\bar\xi_{\kappa}^{(2)}$ as follows by introducing an auxiliary parameter $c$:
\begin{equation*}
    \bar\xi_{\kappa}^{(2)} =  \min_{ \substack{ \rho^2 + r^2 \le 1, r \ge 0, \beta_0 \in  \R, \\
    \rho^2 + r^2 + \beta_0^2 = c^2, c \ge 0
} }
-r + \sqrt{\delta} \left( \E\left[ \bigl(  s(Y) \kappa - \rho \norm{\bmu}_2 + \rho G_1 + rG_2 - \beta_0 Y \bigr)_+^2 \right] \right)^{1/2},
\end{equation*}
and we also define the following quantity
\begin{equation*}
    \begin{aligned}
        \wt\xi_{\kappa}^{(2)} & :=  \min_{ \substack{ \rho^2 + r^2 \le 1, r \ge 0, \beta_0 \in  \R, \\
        \rho^2 + r^2 + \beta_0^2 = c^2, c \ge 0
    } }
        \frac1c \left\{  -r + \sqrt{\delta} \left( \E\left[ \bigl(  s(Y) \kappa - \rho \norm{\bmu}_2 + \rho G_1 + rG_2 - \beta_0 Y \bigr)_+^2 \right] \right)^{1/2} 
        \right\} \\
        & \phantom{:}=  \min_{ \substack{ \rho^2 + r^2 \le 1, r \ge 0, \beta_0 \in  \R, \\
        \rho^2 + r^2 + \beta_0^2 = c^2, c \ge 0
    } }
        -\frac{r}{c} + \sqrt{\delta} \left( \E\left[ \Bigl( s(Y) \frac{\kappa}{c} - \frac{\rho}{c} \norm{\bmu}_2 + \frac{\rho}{c} G_1 + \frac{r}{c} G_2 - \frac{\beta_0}{c} Y \Bigr)_+^2 \right] \right)^{1/2}.
    \end{aligned}
\end{equation*}
Then for any $\kappa > 0$, we have the following observations:
\begin{itemize}
    \item $\sign(\bar\xi_{\kappa}^{(2)}) = \sign(\wt\xi_{\kappa}^{(2)})$. (Their objective functions differ only by a multiplier $c \ge 0$.\footnote{We allow $c = 0$. If $c = 0$, then $\rho = r = \beta_0 = 0$ and the objective value in $\bar\xi_{\kappa}^{(2)}$ is $\sqrt{ \delta(\pi\tau^2\kappa^2 + (1 - \pi)\kappa^2) } > 0$, and the objective value in $\wt\xi_{\kappa}^{(2)}$ is defined as $+\infty$. Both of them are positive.
    })
    \item The minimizer in $\wt\xi_{\kappa}^{(2)}$ must satisfy $\rho^2 + r^2 = 1$. 
    
    Suppose $(\wt\rho, \wt r, \wt\beta_0, \wt c)$ is a minimizer in $\wt\xi_{\kappa}^{(2)}$ such that $\wt\rho^2 + \wt r^2 < 1$. We can increase $(\wt\rho, \wt r, \wt\beta_0, \wt c)$ proportionally, which results in a better solution. That is, define
    \begin{equation*}
        \check\rho := \frac{1}{\sqrt{\wt\rho^2 + \wt r^2}} \wt\rho, \qquad
        \check r := \frac{1}{\sqrt{\wt\rho^2 + \wt r^2}} \wt r,  \qquad
        \check\beta_0 := \frac{1}{\sqrt{\wt\rho^2 + \wt r^2}} \wt\beta_0,  \qquad
        \check c := \frac{1}{\sqrt{\wt\rho^2 + \wt r^2}} \wt c,
    \end{equation*}
    then $(\wt\rho, \wt r, \wt\beta_0, \wt c)$ has a smaller objective value (because $r/c$, $\rho/c$, $\beta_0/c$ all remain unchanged, but $\kappa/c$ decreases since $\check{c} > \wt c$), which contradicts the optimiality of $(\wt\rho, \wt r, \wt\beta_0, \wt c)$.
\end{itemize}
As a consequence, we can simplify
\begin{align*}
        \wt\xi_{\kappa}^{(2)} & =  \min_{ \substack{ \rho \in [-1, 1], \beta_0 \in  \R, \\
        \beta_0^2 = c^2 - 1, c \ge 1
    } }
        \frac1c \left\{  -\sqrt{1 - \rho^2} + \sqrt{\delta} \left( \E\left[ \bigl(  s(Y) \kappa - \rho \norm{\bmu}_2 + \rho G_1 + \sqrt{1 - \rho^2} G_2 - \beta_0 Y \bigr)_+^2 \right] \right)^{1/2} 
        \right\} \\
        & = \min_{ \substack{ \rho \in [-1, 1], \beta_0 \in  \R, \\
        \beta_0^2 = c^2 - 1, c \ge 1
    } }
        \frac1c \left\{  -\sqrt{1 - \rho^2} + \sqrt{\delta} \left( \E\left[ \bigl(  s(Y) \kappa - \rho \norm{\bmu}_2 + G - \beta_0 Y \bigr)_+^2 \right] \right)^{1/2} 
        \right\},
\end{align*}
where $G := \rho G_1 + \sqrt{1 - \rho^2} G_2 \sim \normal(0, 1)$. By the same argument, $\sign(\bar\xi_{\kappa}^{(3)}) = \sign(\wt\xi_{\kappa}^{(2)})$, where
\begin{equation*}
        \bar\xi_{\kappa}^{(3)}  =  \min_{ \substack{ \rho \in [-1, 1] \\  \beta_0 \in  \R } }
        -\sqrt{1 - \rho^2} + \sqrt{\delta} \left( \E\left[ \bigl(  s(Y) \kappa - \rho \norm{\bmu}_2 + G - \beta_0 Y \bigr)_+^2 \right] \right)^{1/2}.
\end{equation*}
Therefore, $\sign(\bar\xi_{\kappa}^{(2)}) = \sign(\bar\xi_{\kappa}^{(3)})$.

~\\
\noindent
In order to show $\bar\xi_{\kappa}^{(2)} = \bar\xi_{\kappa}^{(3)}$ when $\bar\xi_{\kappa}^{(2)} < 0$, we define the objective function of $\bar\xi_{\kappa}^{(2)}$ as
\begin{equation*}
    T_{\kappa}(\rho, r, \beta_0) := - r + \sqrt{\delta} \left( \E\left[ \bigl(  s(Y) \kappa - \rho \norm{\bmu}_2 + \rho G_1 + rG_2 - \beta_0 Y \bigr)_+^2 \right] \right)^{1/2}.
\end{equation*}
Then it suffices to show the minimizer of $T_{\kappa}$ must satisfy $\rho^2 + r^2 = 1$. Again, suppose $(\wt\rho, \wt r, \wt\beta_0)$ is a minimizer of $T_{\kappa}$ such that $\wt\rho^2 + \wt r^2 < 1$. We can increase $(\wt\rho, \wt r, \wt\beta_0)$ proportionally by defining
\begin{equation*}
        \check\rho := \frac{1}{\sqrt{\wt\rho^2 + \wt r^2}} \wt\rho, \qquad
        \check r := \frac{1}{\sqrt{\wt\rho^2 + \wt r^2}} \wt r,  \qquad
        \check\beta_0 := \frac{1}{\sqrt{\wt\rho^2 + \wt r^2}} \wt\beta_0,  \qquad
        \kappa' := \frac{1}{\sqrt{\wt\rho^2 + \wt r^2}} \kappa,
\end{equation*}
then
\begin{equation*}
    0 > \bar\xi_{\kappa}^{(2)} = T_{\kappa}(\wt\rho, \wt r, \wt\beta_0) 
    > 
    \frac{T_{\kappa}(\wt\rho, \wt r, \wt\beta_0) }{\sqrt{\wt\rho^2 + \wt r^2}} 
    =
    T_{\kappa'}(\check\rho, \check r, \check\beta_0) 
    >
    T_{\kappa}(\check\rho, \check r, \check\beta_0),
\end{equation*}
where the last inequality is because $x \mapsto \E[(G + c_1 x + c_2)_+^2]$ strictly increasing for any $c_1 > 0$ and $c_2 \in \R$, and the fact that $\kappa' > \kappa$. Therefore, a contradiction occurs and we complete the proof.
\end{proof}
















\subsubsection{Step 5 --- Phase transition and margin convergence: Proofs of \cref{lem:over_phase_trans}, \ref{lem:over_mar_conp}}
\label{subsubsec:over_phase}
\begin{proof}[\textbf{Proof of \cref{lem:over_phase_trans}}]
We define the following two functions:
\begin{equation}
    \label{eq:T_F_}
    \begin{aligned}
        T_\kappa(\rho, \beta_0) & := - \sqrt{1 - \rho^2} + \sqrt{\delta} \left( \E\left[ \bigl(  s(Y) \kappa - \rho \norm{\bmu}_2 + G - \beta_0 Y \bigr)_+^2 \right] \right)^{1/2}, \\
        F_\kappa(\rho, \beta_0) & := -(1 - \rho^2) + \delta  \E\left[ \bigl(  s(Y) \kappa - \rho \norm{\bmu}_2 + G - \beta_0 Y \bigr)_+^2 \right]
        \\
        & \phantom{:} = \pi \delta \E \left[ \bigl( G - \rho \norm{\bmu}_2 - \beta_0 + \kappa \tau \bigr)_+^2 \right]  + (1-\pi) \delta \E \left[ \bigl( G - \rho \norm{\bmu}_2 + \beta_0 + \kappa \bigr)_+^2 \right] + \rho^2 - 1,
    \end{aligned}
\end{equation}
and then
\begin{equation*}
    \bar\xi_{\kappa}^{(3)}  =  \min_{\rho \in [-1, 1] , \beta_0 \in  \R } T_\kappa(\rho, \beta_0),
    \qquad
    \wt\xi_{\kappa}^{(3)}  :=  \min_{\rho \in [-1, 1] , \beta_0 \in  \R } F_\kappa(\rho, \beta_0).
\end{equation*}
Clearly, $\sign(T_\kappa(\rho, \beta_0)) = \sign(F_\kappa(\rho, \beta_0))$ for any $\rho, \beta_0$ and $\sign(\bar\xi_{\kappa}^{(3)}) = \sign(\wt\xi_{\kappa}^{(3)})$. Also recall that
\begin{equation*}
    \delta^*(\kappa) = \max_{ \substack{\rho \in [-1, 1] \\ \beta_0 \in \R } }  H_\kappa(\rho, \beta_0),
    \qquad 
    H_\kappa(\rho, \beta_0) = \frac{1 - \rho^2}{\E\left[ \bigl(  s(Y) \kappa - \rho \norm{\bmu}_2 + G - \beta_0 Y \bigr)_+^2 \right]}.
\end{equation*}
We can see that $H_\kappa(\rho, \beta_0)$ is well-defined since $\E[ (  s(Y) \kappa - \rho \norm{\bmu}_2 + G - \beta_0 Y )_+^2 ]$ is bounded away from zero for any $\rho \in [-1, 1]$ and $\beta_0 \in \R \cup\{ \pm \infty \}$.

\vspace{0.5\baselineskip}
\noindent
Since $x \mapsto \E[(G + c_1 x + c_2)_+^2]$ is continuous and strictly increasing for any $c_1 > 0, c_2 \in \R$, it can be shown that both $\kappa \mapsto \bar\xi_{\kappa}^{(3)}$, $\kappa \mapsto \wt\xi_{\kappa}^{(3)}$ are continuous strictly increasing, and $\delta^*(\kappa)$ is continuous strictly decreasing (by restricting $\beta_0: \abs{\beta_0} \le B$ for some constant $B$ large enough, similar as Step 4, and then use compactness). Therefore, we have the following equivalent definitions of $\kappa^*$:
\begin{equation}\label{eq:kappa_star}
    \begin{aligned}
        \kappa^*  & := \sup\left\{ \kappa \in \R: \delta^*(\kappa) \ge \delta \right\} \\
        & \phantom{:} = \left\{ \kappa \in \R: \delta^*(\kappa) = \delta \right\} 
        = \left\{ \kappa \in \R: \bar\xi_{\kappa}^{(3)} = 0 \right\}
        = \left\{ \kappa \in \R:  \wt\xi_{\kappa}^{(3)} = 0 \right\}.
    \end{aligned}
\end{equation}
Now we can consider the following two regimes, each with a chain of equivalence:
\begin{equation}
    \label{eq:phase_equiv}
    \begin{aligned}
        \delta \le \delta^*(\kappa) 
        \quad   \overset{\mathmakebox[0pt][c]{\text{(i)}}}{\Longleftrightarrow} \quad
        \kappa \le \kappa^*
        \quad & \overset{\mathmakebox[0pt][c]{\text{(i)}}}{\Longleftrightarrow} \quad
        \bar\xi_{\kappa}^{(3)}, \wt\xi_{\kappa}^{(3)} \le 0
        \quad   \overset{\mathmakebox[0pt][c]{\text{(ii)}}}{\Longleftrightarrow} \quad
        \bar\xi_{\kappa}^{(2)} \le 0 
        \\
        & \overset{\mathmakebox[0pt][c]{\text{(iii)}}}{\Longleftrightarrow} \quad
        \xi'_{n, \kappa, B}  \conp \bigl( \bar\xi^{(2)}_{\kappa} \bigr)_+ = 0
        \quad \overset{\mathmakebox[0pt][c]{\text{(iv)}}}{\Longleftrightarrow} \quad
        \P(\xi_{n, \kappa} = 0) \to 1,
        \\
        \delta > \delta^*(\kappa) 
        \quad   \overset{\mathmakebox[0pt][c]{\text{(i)}}}{\Longleftrightarrow} \quad
        \kappa > \kappa^*
        \quad & \overset{\mathmakebox[0pt][c]{\text{(i)}}}{\Longleftrightarrow} \quad
        \bar\xi_{\kappa}^{(3)}, \wt\xi_{\kappa}^{(3)} > 0
        \quad   \overset{\mathmakebox[0pt][c]{\text{(ii)}}}{\Longleftrightarrow} \quad
        \bar\xi_{\kappa}^{(2)} > 0 
        \\
        & \overset{\mathmakebox[0pt][c]{\text{(iii)}}}{\Longleftrightarrow} \quad
        \xi'_{n, \kappa, B}  \conp \bigl( \bar\xi^{(2)}_{\kappa} \bigr)_+ > 0
        \quad \overset{\mathmakebox[0pt][c]{\text{(iv)}}}{\Longleftrightarrow} \quad
        \P(\xi_{n, \kappa} > 0) \to 1,
    \end{aligned}
\end{equation}
where (i) is from \cref{eq:kappa_star}, (ii) is from \cref{lem:over_sign}, (iii) is from \cref{lem:over_CGMT}, \ref{lem:over_ULLN}, and (iv) is from \cref{lem:over_beta0}. Linear separability considers the special case $\kappa = 0$. From definition \cref{eq:xi_n_kappa}, for any $\kappa \le 0$ we have $\xi_{n, \kappa} = 0$ (by taking $\vbeta = \bzero$, $\beta_0 = 0$). Therefore, 
\begin{itemize}
    \item If $\delta < \delta^*(0)$, by \cref{eq:phase_equiv} $\kappa^* > 0$ and $\P(\mathcal{E}_{n, \kappa^*}) = \P(\xi_{n, \kappa^*} = 0)
    \to 1$, which deduces the data is linearly separable with high probability.
    \item If $\delta > \delta^*(0)$, by \cref{eq:phase_equiv} $\kappa^* < 0$ and $\P(\mathcal{E}_{n, \kappa}) = \P(\xi_{n, \kappa} = 0)
    \to 0$ for any $\kappa > 0$ (as $\kappa \mapsto \xi_{n, \kappa}$ is non-decreasing), which implies the data is not linearly separable with high probability.
\end{itemize}
%%%%%%%%%%%%%%%%%%%%%%%%%%%%%%
% Old ver.: \norm{\beta} = 1 %
%%%%%%%%%%%%%%%%%%%%%%%%%%%%%%
% There is a small caveat that the constraint of $\vbeta$ in the max-margin optimization problem \cref{eq:over_max-margin} is $\norm{\vbeta}_2 = 1$, while in $\xi_{n,\kappa}$ \cref{eq:xi_n_kappa} it is $\norm{\vbeta}_2 \le 1$. To handle this, we can define 
% \begin{equation}
%     \label{eq:over_wt_kappa}
%     \wt\kappa_n := \sup\{ \kappa \in \R:  \xi_{n,\kappa} = 0 \}.
% \end{equation}
% Notice that $\wt\kappa_n = \hat\kappa_n \mathbbm{1}\{ \hat\kappa_n > 0 \} \ge 0$ always holds. 
% If $\delta < \delta^*(0)$, then $\hat\kappa_n > 0$ with high probability and hence $\P(\wt\kappa_n = \hat\kappa_n) \to 1$ as $n \to \infty$. Therefore, we only have to show that $\wt\kappa_n \conp \kappa^*$. Notice that $\kappa^* > 0$ if $\delta < \delta^*(0)$, and $ \bar\xi_{\kappa^*}^{(3)} = 0$. 
\end{proof}

\begin{proof}[\textbf{Proof of \cref{lem:over_mar_conp}}]
If $\delta < \delta^*(0)$, then $\kappa^* > 0$ and $\bar\xi_{\kappa^*}^{(3)} = 0$. According to \cref{eq:phase_equiv}, for any $\varepsilon > 0$ small enough, we have
\begin{equation*}
    \begin{aligned}
        \bar\xi_{\kappa^* - \varepsilon}^{(3)} < 0
    \quad & \Longrightarrow \quad
    \P(\mathcal{E}_{n, \kappa^* - \varepsilon}) = \P(\xi_{n, \kappa^* - \varepsilon} = 0) \to 1, \\
        \bar\xi_{\kappa^* + \varepsilon}^{(3)} > 0
    \quad & \Longrightarrow \quad
    \P(\mathcal{E}_{n, \kappa^* + \varepsilon}) = \P(\xi_{n, \kappa^* + \varepsilon} = 0) \to 0.
    \end{aligned}
\end{equation*}
% Combine these arguments above with \cref{eq:over_wt_kappa}, we see that $\kappa^* - \varepsilon \le \wt\kappa_n \le \kappa^* + \varepsilon$ with probability approaching one, which proves that $\wt\kappa_n \conp \kappa^*$, and hence $\hat\kappa_n \conp \kappa^*$.
Recall that $\hat\kappa_n = \ind_{1 \le n_+ \le n - 1} \sup\{ \kappa \in \R:  \xi_{n,\kappa} = 0 \}$.
By combining these arguments, we can see that $\kappa^* - \varepsilon \le \hat\kappa_n \le \kappa^* + \varepsilon$ holds on the event $\mathcal{D}_n^c$, with high probability. This proves $\hat\kappa_n \conp \kappa^*$.
% \begin{itemize}
%     \item If $\delta \le \delta^*(\kappa)$, or equivalently $\kappa \le \kappa^*$, then we get $\bar\xi_{\kappa}^{(3)}, \wt\xi_{\kappa}^{(3)} \le 0$, and \cref{lem:over_sign} gives $\bar\xi_{\kappa}^{(2)} \le 0$. Combining \cref{lem:over_beta0}, \ref{lem:over_CGMT} and \ref{lem:over_ULLN}, we obtain $\xi'_{n, \kappa, B}  \conp \bigl( \bar\xi^{(2)}_{\kappa} \bigr)_+ = 0$ and $\P(\xi_{n, \kappa} = 0) \to 1$.
%     \item If $\delta > \delta^*(\kappa)$, or equivalently $\kappa > \kappa^*$, then we get $\bar\xi_{\kappa}^{(3)}, \wt\xi_{\kappa}^{(3)} > 0$, and \cref{lem:over_sign} gives $\bar\xi_{\kappa}^{(2)} > 0$. Combining \cref{lem:over_beta0}, \ref{lem:over_CGMT} and \ref{lem:over_ULLN}, we obtain $\xi'_{n, \kappa, B}  \conp \bigl( \bar\xi^{(2)}_{\kappa} \bigr)_+ > 0$ and $\P(\xi_{n, \kappa} > 0) \to 1$.
% \end{itemize}
\end{proof}

\subsubsection{Convergence of ELD and parameters for $\tau = 1$: 
Proofs of \cref{lem:over_logit_conv}, \ref{lem:H_kappa_1}}
\label{subsubsec:over_logit_conv}

In this section, we provide a proof of parameter convergence in \cref{thm:SVM_main}\ref{thm:SVM_main_param} and ELD convergence in \ref{thm:SVM_main_logit} for the special case $\tau = 1$. For convenience of notation, we drop the subscripts and simply write $\hat\rho := \hat\rho_n$, $\hat\beta_{0} := \hat\beta_{0,n}$. 
% Recall the empirical distribution of logit margins (well-defined version, \cref{eq:over_ELD_well}) and asymptotic distribution are respectively defined as
% \begin{equation*}
%     \hat \cL_{n} = \frac1n \sum_{i=1}^n \delta_{y_i ( \< \xx_i, \hat\vbeta \> + \hat\beta_{0} ) \ind\{1 \le n_+ \le n - 1\} },
%     \qquad 
%     \cL_* = \Law\left( 
%     \max\bigl\{ \kappa^*,  \rho^*\norm{\vmu}_2 + G + \beta_0^* Y    \bigr\}
%     \right).
% \end{equation*}
Recall the ELD (well-defined version, i.e., \cref{eq:over_ELD_well}) and its asymptotics are respectively defined as
\begin{equation*}
    \hat \nu_{n} = \frac1n \sum_{i=1}^n \delta_{(y_i, \< \xx_i, \hat\vbeta \> + \hat\beta_{0} ) \cdot \ind\{\mathcal{D}_n^c\} },
    % \hat \nu_{n} = \frac1n \sum_{i=1}^n \delta_{(y_i, \< \xx_i, \hat\vbeta \> + \hat\beta_{0})},
    \qquad 
    \nu_* = \Law\left( Y,
    Y \max\bigl\{ \kappa^*, \rho^*\norm{\vmu}_2 + G + \beta_0^* Y \bigr\}
    \right).
\end{equation*}
Here $(\rho^*, \beta_0^*, \kappa^*)$ is defined as the maximizer of \cref{eq:SVM_asymp_simple}, and obviously $\kappa^*$ also satisfies \cref{eq:kappa_star}. The uniqueness of $(\rho^*, \beta_0^*)$ will be given by \cref{lem:H_kappa_1}. Analogous to the proof of \cite[Theorem 4.6]{montanari2022overparametrizedlineardimensionalityreductions}, by using the theory of projection pursuit therein, we have the following results.
\begin{lem}[ELD and parameter convergence] 
\label{lem:over_logit_conv}
Consider $\tau = 1$. As $n, d \to \infty$, we have
\begin{equation*}
    W_2 \bigl( \hat \nu_{n}, \nu_* \bigr)
    \conp 0.
\end{equation*}
The convergence of $\hat\rho \conp \rho^*$ and $\hat\beta_{0} \conp \beta_0^*$ are followed by continuity and convexity of $H_\kappa$ in \cref{eq:sep_functions}.
\end{lem}
\begin{proof}
Our proof primarily follows the setup in \cite[Section 4.1]{montanari2022overparametrizedlineardimensionalityreductions} and techniques in \cite[Section 4.3]{montanari2022overparametrizedlineardimensionalityreductions}. Recall that we can rewrite $\xx_i = y_i \bmu + \zz_i$, where $\zz_i \iidsim \normal(\bzero, \bI_d)$ and $y_i \indep \zz_i$. Using notation from \cite{montanari2022overparametrizedlineardimensionalityreductions}, $\P(y_i = 1 \,|\, \zz_i) = \varphi(\vmu_0^\top \zz_i)$, where $\vmu_0 = \vmu/\norm{\vmu}_2$ and $\varphi(x) \equiv \pi$ is a constant function. Recall that we reparametrize $\hat\rho = \vmu_0^\top \hat\vbeta$. Now, define random variables with joint distribution
\begin{equation*}
    Y \indep G \indep Z, 
    \quad \P(Y = +1 \,|\, G) = 1 - \P(Y = -1 \,|\, G) = \varphi(G) \equiv \pi, 
    \quad G, Z \sim \normal(0, 1).
\end{equation*}
Let $(Y, G, Z) \indep \hat\vbeta$. According to the definition in \cite[Lemma 4.2]{montanari2022overparametrizedlineardimensionalityreductions}, we have
\begin{equation*}
    \Law\left( Y, \vmu_0^\top \hat\vbeta \cdot G + \sqrt{1 - (\vmu_0^\top \hat\vbeta)^2} \cdot Z \right)
    = \Law\left( Y, \hat\rho G + \sqrt{1 - \hat\rho^2} Z \right)
    = \Law(Y, Z).
\end{equation*}
Therefore, by using \cite[Theorem 4.3]{montanari2022overparametrizedlineardimensionalityreductions}, for any $\varepsilon, \eta > 0$, with high probability we have
\begin{equation*}
    W_2^{(\eta)} \biggl( 
        \frac1n \sum_{i=1}^n \delta_{(y_i, \< \zz_i, \hat\vbeta \>)} , 
        \Law(Y, Z)        
     \biggr) 
     \le \frac{\sqrt{1 - \hat\rho^2}}{\sqrt{\delta}} + \varepsilon,
\end{equation*}
where $W_2^{(\eta)}$ is the $\eta$-constrained $W_2$ distance \cite[Definition 4.1]{montanari2022overparametrizedlineardimensionalityreductions}. Formally, for any $\eta > 0$, the $\eta$-constrained $W_2$ distance between any two probability measures $P$ and $Q$ in $\R^d$ is defined by
\begin{equation*}
    W_2^{(\eta)}(P, Q) := \left( \inf_{\gamma \in \Gamma^{(\eta)}(P, Q) }
    \int_{\R^d \times \R^d} \norm{\xx - \yy}_2^2 \gamma(\d\xx \times \d\yy)
    \right)^{1/2},
\end{equation*}
where $\Gamma^{(\eta)}(P, Q)$ denotes the set of all couplings $\gamma$ of $P$ and $Q$ which satisfy
\begin{equation}
    \label{eq:W2eta}
    \left(
        \int_{\R^d \times \R^d} |\< \be_1, \xx - \yy \>|^2 \gamma(\d\xx \times \d\yy) 
    \right)^{1/2}    
    \le \eta,
\end{equation}
where $\be_1 = (1, 0, \dots, 0)^\top$. 


The following proof is analogous to the proof of \cite[Theorem 4.6]{montanari2022overparametrizedlineardimensionalityreductions}. We show the convergence of logit margins $W_2( \hat\cL_n, \cL_* ) \conp 0$ first, where
\begin{equation}
    \label{eq:margin_logit_dist}
    \hat \cL_{n} := \frac1n \sum_{i=1}^n \delta_{y_i ( \< \xx_i, \hat\vbeta \> + \hat\beta_{0} ) },
    \qquad 
    \cL_* := \Law\left( 
    \max\bigl\{ \kappa^*,  \rho^*\norm{\vmu}_2 + G + \beta_0^* Y    \bigr\}
    \right).
\end{equation}
Throughout this subsection, all the expectations (including the one in $H_\kappa$) are conditional on $\{ (y_i, \zz_i) \}_{i=1}^n$, which will be denoted as $\E_{\cdot | n}[\cdot]$. Now, let
\begin{equation*}
    \frac1n \sum_{i=1}^n \delta_{(y_i, \< \zz_i, \hat\vbeta \>)} =: \Law(Y', Z'),
\end{equation*}
then by definition in \cref{eq:W2eta} and the same arguments in the proof of \cite[Theorem 4.6]{montanari2022overparametrizedlineardimensionalityreductions}, there exists a coupling $(Y, Z, Y', Z')$ and a sufficiently small $\eta$ ($\eta < \varepsilon^2/4$), such that
\begin{equation}
    \label{eq:pursuit_pre}
    % W_2 \biggl( 
    %     \frac1n \sum_{i=1}^n \delta_{y_i \< \zz_i, \hat\vbeta \>} , 
    %     \Law(YZ)        
    %  \biggr) 
    %  \le
    \left( \E_{\cdot | n}\bigl[(Y - Y')^2\bigr] \right)^{1/2} \le \eta,
    \qquad
    \left( \E_{\cdot | n}\bigl[(YZ - Y'Z')^2\bigr] \right)^{1/2} \le \frac{\sqrt{1 - \hat\rho^2}}{\sqrt{\delta}} + 2\varepsilon
\end{equation}
holds with high probability. 
We can express the empirical distribution of logit margins \cref{eq:margin_logit_dist} as
% Now, temporarily write the original (ill-defined) empirical distribution of logit margins as
\begin{equation}
    \label{eq:emp_ell}
    \hat \cL_{n}
    = \frac1n \sum_{i=1}^n \delta_{y_i \< \zz_i, \hat\vbeta \> + \hat\rho \norm{\vmu}_2 + y_i \hat\beta_0 ) }
    = \Law \Bigl( \underbrace{Y'Z' + \hat\rho \norm{\vmu}_2 + \hat\beta_0 Y' }_{=: V} \Bigr).
\end{equation}
For convenience, denote $\hat V := YZ + \hat\rho \norm{\vmu}_2 + \hat\beta_0 Y$, then with high probability we have
\begin{align}
        \bigl( \E_{\cdot | n}\bigl[(V - \hat V)^2\bigr] \bigr)^{1/2}
        & \overset{\mathmakebox[0pt][c]{\text{(i)}}}{\le} \left( \E_{\cdot | n}\bigl[(YZ - Y'Z')^2\bigr] \right)^{1/2} 
        + \left( \E_{\cdot | n}\bigl[(Y - Y')^2\bigr] \right)^{1/2} |\hat\beta_0|  \notag \\
        & \overset{\mathmakebox[0pt][c]{\text{(ii)}}}{\le} \frac{\sqrt{1 - \hat\rho^2}}{\sqrt{\delta}} + 2\varepsilon + \eta B \notag \\
        & \overset{\mathmakebox[0pt][c]{\text{(iii)}}}{\le} \frac{\sqrt{1 - \hat\rho^2}}{\sqrt{\delta}} + 3\varepsilon,
        \label{eq:V_diff0}
\end{align}
where (i) follows from Minkowski inequality, (ii) uses \cref{eq:pursuit_pre} and $|\hat\beta_0| \le B$ from \cref{lem:over_beta0}, by recalling that $\delta < \delta^*(0)$ and the data is linearly separable with high probability, while in (iii) we choose $\eta < \min\{ \varepsilon^2/4, \varepsilon/B \}$. According to $\hat\kappa_n \conp \kappa^*$ from \cref{lem:over_mar_conp}, we know that
\begin{equation*}
    \lim_{n \to \infty} \P\left( y_i (\< \hat\vbeta , \xx_i \> + \hat\beta_0 ) \ge \kappa^* - \varepsilon,
    \forall\, i \in [n] \right) = 1.
\end{equation*}
Then by definition of $V$ in \cref{eq:emp_ell}, with high probability we have 
\begin{equation}
    \label{eq:V_kappa_as}
    V \ge \kappa^* - \varepsilon,
    \qquad \text{almost surely}.
\end{equation}
Now, recall $\delta = \delta^*(\kappa^*) = H_{\kappa^*}(\rho^*, \beta_0^*)$ by \cref{eq:kappa_star}, where $(\rho^*, \beta_0^*) = \argmin_{\rho \in [-1, 1], \beta_0 \in \R} H_{\kappa^*}(\rho, \beta_0)$. Therefore,
\begin{equation}
    \label{eq:V_diff1}
        \bigl( \E_{\cdot | n}\bigl[(V - \hat V)^2\bigr] \bigr)^{1/2}
        \le \frac{\sqrt{1 - \hat\rho^2}}{\sqrt{\delta}} + 3\varepsilon
        = \frac{\sqrt{1 - \hat\rho^2}}{\sqrt{H_{\kappa^*}(\rho^*, \beta_0^*)}} + 3\varepsilon
\end{equation}
holds with high probability. For $\rho \in [-1, 1], \beta_0 \in \R$, let us define
\begin{equation*}
    h_{\kappa^*}^*(\rho, \beta_0) := \frac{1}{\sqrt{H_{\kappa^*}(\rho, \beta_0)}}
    - \frac{1}{\sqrt{H_{\kappa^*}(\rho^*, \beta_0^*)}}.
\end{equation*}
Note that $h_{\kappa^*}^*(\rho, \beta_0) \ge 0$. Hence, \cref{eq:V_diff1} implies that (reminding $\tau = 1$) with high probability
\begin{align*}
        \bigl( \E_{\cdot | n}\bigl[(V - \hat V)^2\bigr] \bigr)^{1/2}
        & \le \sqrt{1 - \hat\rho^2} \biggl( \sqrt{ \frac{1}{H_{\kappa^*}(\hat\rho, \hat\beta_0)} } - h_{\kappa^*}^*(\hat\rho, \hat\beta_0) \biggr) + 3\varepsilon \\
        & = \! \left( \E_{\cdot | n}\! \left[ \bigl(  \kappa^* - \hat\rho \norm{\bmu}_2 + G - \hat\beta_0 Y \bigr)_+^2 \right] \right)^{1/2}
        - \sqrt{1 - \hat\rho^2} \cdot h_{\kappa^*}^*(\hat\rho, \hat\beta_0) 
        + 3\varepsilon \\
        & \overset{\mathmakebox[0pt][c]{\text{(i)}}}{=} \bigl( \E_{\cdot | n}\bigl[(\kappa^* - \hat V)_+^2 \bigr] \bigr)^{1/2}
        - \sqrt{1 - \hat\rho^2} \cdot h_{\kappa^*}^*(\hat\rho, \hat\beta_0) 
        + 3\varepsilon,
\end{align*}
which can be further written as (with high probability)
\begin{align}
        \bigl( \E_{\cdot | n}\bigl[(V - \hat V)^2\bigr] \bigr)^{1/2} + \sqrt{1 - \hat\rho^2} \cdot h_{\kappa^*}^*(\hat\rho, \hat\beta_0) 
        & \le
        \bigl( \E_{\cdot | n}\bigl[(\kappa^* - \hat V)_+^2\bigr] \bigr)^{1/2}
        + 3\varepsilon \notag  \\
        & \overset{\mathmakebox[0pt][c]{\text{(ii)}}}{\le} 
        \bigl( \E_{\cdot | n}\bigl[(\kappa^* - \varepsilon - \hat V)_+^2\bigr] \bigr)^{1/2}
        + 4\varepsilon.
        % & = 
        % \! \left( \E_{\cdot | n}\!\left[ \bigl( \max\{ \kappa^* - \varepsilon , \hat V \}  - \hat V \bigr)^2 \right] \right)^{1/2}
        % + 4\varepsilon.
        \label{eq:V_diff2}
\end{align}
In the derivation above, equation (i) follows from $\hat V = YZ + \hat\rho \norm{\vmu}_2 + \hat\beta_0 Y 
\overset{\mathmakebox[0pt][c]{\mathrm{d}}}{=} -G + \hat\rho \norm{\vmu}_2 + \hat\beta_0 Y$ when conditioning on $\{(y_i, \zz_i)\}_{i = 1}^n$, and (ii) follows from the fact that
\begin{equation*}
    \begin{aligned}
        \frac{\d}{\d \kappa}  \bigl( \E_{\cdot | n}\bigl[(\kappa - \hat V)_+^2\bigr] \bigr)^{1/2}
        & = \frac{
            \E_{\cdot | n}\bigl[(\kappa - \hat V)_+^2\bigr]
        }{
            \bigl( \E_{\cdot | n}\bigl[(\kappa - \hat V)_+^2\bigr] \bigr)^{1/2}
        }
        \le 1.
    \end{aligned}
\end{equation*}
Besides, by using \cref{eq:V_kappa_as} and exactly the same arguments in the proof of \cite[Theorem 4.6]{montanari2022overparametrizedlineardimensionalityreductions}, we can show that with high probability,
\begin{equation}
    \label{eq:V_max}
    \E_{\cdot | n} \!\left[ \bigl(V - \max\{ \kappa^* - \varepsilon, \hat V \} \bigr)^2 \right]
    \le \E_{\cdot | n}\bigl[(V - \hat V)^2\bigr] - 
    \E_{\cdot | n}\bigl[(\kappa^* - \varepsilon - \hat V)_+^2\bigr].
\end{equation}
Combining \cref{eq:V_max} with \eqref{eq:V_diff2} gives the following implications:
\begin{itemize}
    \item \cref{eq:V_max} implies 
    \begin{equation*}
        \E_{\cdot | n}\bigl[(\kappa^* - \varepsilon - \hat V)_+^2\bigr]
        \le
        \E_{\cdot | n} \bigl[(V - \hat V)^2\bigr].
    \end{equation*}
    Plugging this into \cref{eq:V_diff2} yields that with high probability,
    \begin{equation*}
        \sqrt{1 - \hat\rho^2} \cdot h_{\kappa^*}^*(\hat\rho, \hat\beta_0) \le 4 \varepsilon,
    \end{equation*}
    i.e., $\sqrt{1 - \hat\rho^2} \cdot h_{\kappa^*}^*(\hat\rho, \hat\beta_0) \conp 0$. Note that if $\abs{\rho} \to 1$ (i.e., $\sqrt{1 - \rho^2} = o_\varepsilon(1)$), the quantity
    \begin{equation*}
            \sqrt{1 - \rho^2} \cdot h_{\kappa^*}^*(\rho, \beta_0)
        = \left( \E \left[ \bigl(  \kappa^* - \rho \norm{\bmu}_2 + G - \beta_0 Y \bigr)_+^2 \right] \right)^{1/2}
        - \frac{\sqrt{1 - \rho^2}}{\sqrt{H_{\kappa^*}(\rho^*, \beta_0^*)}}
    \end{equation*}
    is bounded away from 0, for any $\beta_0 \in \R \cup\{ \pm \infty\}$. Therefore, we must have $h_{\kappa^*}^*(\hat\rho, \hat\beta_0) \conp 0$. By \cref{lem:H_kappa_1} (proof is deferred to the end of this subsection), we know $h_{\kappa^*}^*(\rho, \beta_0) \ge 0$ for all $\rho \in [-1, 1], \beta_0 \in \R$, and $(\rho, \beta_0) \to (\rho^*, \beta_0^*)$ if and only if $h_{\kappa^*}^*(\rho, \beta_0) \to 0$. Hence, we conclude
    \begin{equation*}
        (\hat\rho, \hat\beta_0) \conp (\rho^*, \beta_0^*),
    \end{equation*}
    which gives parameter convergence.

    \item Let
    \begin{equation*}
        I := \bigl( \E_{\cdot | n}\bigl[(V - \hat V)^2\bigr] \bigr)^{1/2} 
        , \qquad
        I\!I := \bigl( \E_{\cdot | n}\bigl[(\kappa^* - \varepsilon - \hat V)_+^2\bigr] \bigr)^{1/2}.
    \end{equation*}
    Then \cref{eq:V_diff2} implies $I - I\!I \le 4\varepsilon$, and we also have (for $\varepsilon > 0$ small enough)
    \begin{equation*}
        \begin{aligned}
            I\!I & \le \abs{\kappa^* - \varepsilon} + \bigl( \E_{\cdot | n}\bigl[\hat V^2\bigr] \bigr)^{1/2} 
            \le \kappa^* + \bigl( \E_{\cdot | n}\bigl[ (
                G + \hat\rho \norm{\vmu}_2 + \hat\beta_0 Y
            )^2\bigr] \bigr)^{1/2}  \\
            & \le \kappa^* + \left(\E[G^2]\right)^{1/2} + |\hat\rho| \norm{\vmu}_2 + |\hat\beta_0| \\
            & \le \kappa^* + 1 + \norm{\vmu}_2 + B,
        \end{aligned}
    \end{equation*}
    by using Minkowski inequality and $|\hat\beta_0| \le B$ (with high probability) from \cref{lem:over_beta0}. Based on these results and \cref{eq:V_max}, with high probability, we have
    \begin{equation*}
        \begin{aligned}
            \E_{\cdot | n} \!\left[ \bigl(V - \max\{ \kappa^* - \varepsilon, \hat V \} \bigr)^2 \right] 
            & \le I^2 - I\!I^2 = (I - I\!I)(I - I\!I + 2I\!I) \\
            & \le 4\varepsilon \bigl(4\varepsilon + 2 (\kappa^* + 1 + \norm{\vmu}_2 + B) \bigr) \\
            & \le C \varepsilon,
        \end{aligned}
    \end{equation*}
    where $C \in (0, \infty)$ is some constant depending on $(\pi, \norm{\bmu}_2, \delta)$ (through $\kappa^*, B$). Therefore, by recalling $\hat V \overset{\mathmakebox[0pt][c]{\mathrm{d}}}{=} G + \hat\rho \norm{\vmu}_2 + \hat\beta_0 Y$, we obtain that with high probability,
    \begin{equation}
        \label{eq:W2_conv}
        W_2\left( \hat \cL_{n}, \Law \bigl(\max\{ \kappa^* - \varepsilon, G + \hat\rho \norm{\vmu}_2 + \hat\beta_0 Y \} \bigr) \right) \le \sqrt{C \varepsilon}.
    \end{equation}
\end{itemize}    
As a consequence, 
% there exists some event $\mathcal{A}_n$ with $\P(\mathcal{A}_n) \to 1$ as $n \to \infty$, such that the following bound holds on $\mathcal{A}_n$ (for $n$ large enough):
the following holds with high probability:
    \begin{align*}
            W_2\bigl( \hat \cL_{n}, \cL_* \bigr)
            & = W_2 \left( 
                \hat \cL_{n},
                \Law\bigl( \max\{ \kappa^*, G + \rho^*\norm{\vmu}_2 + \beta_0^* Y \} \bigr)
             \right) \\
            & \le
            W_2\left( \hat \cL_{n}, \Law \bigl(\max\{ \kappa^* - \varepsilon, G + \hat\rho \norm{\vmu}_2 + \hat\beta_0 Y \} \bigr) \right) \\
            & \phantom{\le} \  + W_2\left( \Law \bigl(\max\{ \kappa^* - \varepsilon, G + \hat\rho \norm{\vmu}_2 + \hat\beta_0 Y \} \bigr),
            \Law \bigl(\max\{ \kappa^* , G + \hat\rho \norm{\vmu}_2 + \hat\beta_0 Y \} \bigr) \right)
            \\
            & \phantom{\le} \  + W_2\left( \Law \bigl(\max\{ \kappa^* , G + \hat\rho \norm{\vmu}_2 + \hat\beta_0 Y 
            \} \bigr) 
            ,
            \Law\bigl( \max\{ \kappa^*, G + \rho^*\norm{\vmu}_2 + \beta_0^* Y \} \bigr)
            \right) \\
            & \le \sqrt{C \varepsilon} + \varepsilon + o_\varepsilon(1)
            = o_\varepsilon(1),
    \end{align*}
    where in the last inequality, we use that (i) the result from \cref{eq:W2_conv}, (ii) the fact that the mapping $\kappa \mapsto \max\{ \kappa, G + \hat\rho \norm{\vmu}_2 + \hat\beta_0 Y \}$ is 1-Lipschitz, and (iii) the consequence of $(\hat\rho, \hat\beta_0) \conp (\rho^*, \beta_0^*)$ and $|\hat\rho| \le 1$, $|\hat\beta_0| \le B$ (with high probability). 
    
    \vspace{0.5\baselineskip}
	\noindent
    % Recall $\wt\cL_n = \hat\cL_n$ on the event $\cD_n^c = \{ 1 \le n_+ \le n - 1 \}$ and $\P(\cA_n \cap \cD_n^c) \to 1$. Therefore, with high probability $W_2\bigl( \hat \cL_{n}, \cL_* \bigr) = o_\varepsilon(1)$.  
    % Finally, by taking $\varepsilon \to 0$, we prove
    % \begin{equation*}
    %     W_2\bigl( \hat \cL_{n}, \cL_* \bigr) \conp 0.
    % \end{equation*}
    Now we prove the convergence of ELD. Denote $\hat \cL_{n} =: \Law(L')$, $\cL_* =: \Law(L)$, where $(L, L')$ is a coupling such that
    \begin{equation}\label{eq:L_coupling}
        \left( \E_{\cdot | n}\bigl[(L - L')^2\bigr] \right)^{1/2} = o_\varepsilon(1).
    \end{equation}
    Therefore, for some constants $C_1, C_2 > 0$, with high probability, we have
    \begin{align*}
        W_2\bigl( \hat \nu_{n}, \nu_* \bigr)
        & \le
        \left( \E_{\cdot | n}\bigl[(Y - Y')^2\bigr] \right)^{1/2} 
        +
        \left( \E_{\cdot | n}\bigl[(YL - Y'L')^2\bigr] \right)^{1/2} \\
        & \overset{\mathmakebox[0pt][c]{\text{(i)}}}{\le} 
        \eta + \left( \E_{\cdot | n}\bigl[(YL - Y'L)^2\bigr] \right)^{1/2}
        + \left( \E_{\cdot | n}\bigl[(Y'L - Y'L')^2\bigr] \right)^{1/2} \\
        & \overset{\mathmakebox[0pt][c]{\text{(ii)}}}{\le}
        \eta + C_1 \left( \E_{\cdot | n}\bigl[(Y - Y')^2\bigr] \right)^{1/4} \left(\E[L^4]\right)^{1/4}
        + \left( \E_{\cdot | n}\bigl[(L - L')^2\bigr] \right)^{1/2} \\
        & \overset{\mathmakebox[0pt][c]{\text{(iii)}}}{\le}
        \eta + C_2 \sqrt{\eta} + o_\varepsilon(1)
        \overset{\mathmakebox[0pt][c]{\text{(iv)}}}{\le}  o_\varepsilon(1),
    \end{align*}
    where in (i) we use \cref{eq:pursuit_pre} and Minkowski inequality, in (ii) use Cauchy--Schwarz inequality and $Y, Y' \in \{ \pm 1 \}$, in (iii) use \cref{eq:pursuit_pre} and \eqref{eq:L_coupling}, while in (iv) recall that $\eta < \min\{ \varepsilon^2/4, \varepsilon/B \} = o_{\varepsilon}(1)$. By taking $\varepsilon \to 0$, we can show that $W_2\bigl( \hat \nu_{n}, \nu_* \bigr) \conp 0$ for $\tau = 1$. This completes the proof.
    % \begin{equation*}
    %     \begin{aligned}
    %         W_2\bigl( \hat \cL_{n}, \cL_* \bigr)
    %     & = W_2\, \Bigl( \Law( V \ind_{\cD_n^c} ), \Law \bigl(\max\{ \kappa^*, G + \rho^* \norm{\vmu}_2 + \beta_0^*  Y \} \bigr) \Bigr)
    %         \\
    %     & \le W_2\, \Bigl( \Law( V \ind_{\cD_n^c} ), \Law \bigl(\max\{ \kappa^*, G + \rho^* \norm{\vmu}_2 + \beta_0^*  Y \} \ind_{\cD_n^c} \bigr) \Bigr) \\
    %     & \phantom{\le} \  +  
    %     W_2\, \Bigl( \Law \bigl(\max\{ \kappa^*, G + \rho^* \norm{\vmu}_2 + \beta_0^*  Y \} \ind_{\cD_n^c} \bigr), \Law \bigl(\max\{ \kappa^*, G + \rho^* \norm{\vmu}_2 + \beta_0^*  Y \}  \bigr) \Bigr)  \\
    %     & \le W_2\, \Bigl( \Law( V ), \Law \bigl(\max\{ \kappa^*, G + \rho^* \norm{\vmu}_2 + \beta_0^*  Y \} \bigr) \Bigr) \\
    %     & \phantom{\le} \  +  
    %     \left( \E_{\cdot | n}\bigl[  (\max\{ \kappa^*, G + \rho^* \norm{\vmu}_2 + \beta_0^*  Y \} \ind_{\cD_n} )^2  \bigr] \right)^{1/2}  \\
    %     & \le W_2\bigl( \hat \cL_{n}, \cL_* \bigr)
    %     + \left( \E\bigl[  (\max\{ \kappa^*, G + \rho^* \norm{\vmu}_2 + \beta_0^*  Y \} )^2  \bigr] \right)^{1/2} \ind_{\cD_n} \\
    %     & \le o_\varepsilon(1) + 
    %     \end{aligned}
    % \end{equation*}
\end{proof}



Finally, we prove the following technical lemma.
\begin{lem}
    \label{lem:H_kappa_1}
    For any fixed $\kappa \in \R$ and $\tau > 0$, the function $H_\kappa(\rho, \beta_0)$ in \cref{eq:sep_functions} admits a unique maximizer $(\rho^*(\kappa), \beta_0^*(\kappa)) \in [0, 1) \times \R$.
\end{lem}
\begin{proof}
    For simplicity, write $\rho^* := \rho^*(\kappa)$, $\beta_0^* := \beta_0^*(\kappa)$. First, note that
    \begin{equation*}
        H_\kappa(\rho, \beta_0) = \frac{1 - \rho^2}{\E\left[ \bigl(  s(Y) \kappa - \rho \norm{\bmu}_2 + G - \beta_0 Y \bigr)_+^2 \right]} \le \, \frac{1}{\E\left[ \bigl(  s(Y) \kappa - \norm{\bmu}_2 + G - \beta_0 Y \bigr)_+^2 \right]},
    \end{equation*}
    which converges to $0$ as $\beta_0 \to \pm\infty$. Moreover, $H_\kappa(-\rho, \beta_0) < H_\kappa(\rho, \beta_0)$ for any $\rho \in (0, 1]$. Therefore, $H_\kappa(\rho, \beta_0)$ must have a maximizer $(\rho^*, \beta_0^*) \in [0, 1] \times \R$. Further, $\rho^* \in [0, 1)$ since $H_\kappa(1, \beta_0) \equiv 0$. We prove the uniqueness of $(\rho^*, \beta_0^*)$ by contradiction. For future convenience, we denote $H_{\max} := H_{\kappa} (\rho^*, \beta_0^*)$. Assume that there exist $(\rho_1, \beta_{0, 1})$ and $(\rho_2, \beta_{0, 2})$ such that $(\rho_1, \beta_{0, 1}) \neq (\rho_2, \beta_{0, 2})$, and
    \begin{equation*}
        H_{\kappa} (\rho_1, \beta_{0, 1}) = H_{\kappa} (\rho_2, \beta_{0, 2}) = H_{\max},
    \end{equation*}
    which implies
    \begin{equation*}
        G_{\kappa} (\rho_1, \beta_{0, 1}) = \frac{\sqrt{1 - \rho_1^2}}{\sqrt{H_{\max}}}, 
        \qquad 
        G_{\kappa} (\rho_2, \beta_{0, 2}) = \frac{\sqrt{1 - \rho_2^2}}{\sqrt{H_{\max}}},
    \end{equation*}
    where we define
    \begin{equation*}
        G_\kappa(\rho, \beta_0) := \left( \E\left[ \bigl(  s(Y) \kappa - \rho \norm{\bmu}_2 + G - \beta_0 Y \bigr)_+^2 \right] \right)^{1/2}.
    \end{equation*}
    Similar to \cite[Lemma 6.3]{montanari2023generalizationerrormaxmarginlinear}, we can show that $G_{\kappa}$ is strictly convex. Hence,
    \begin{align*}
        G_{\kappa} \left( \frac{\rho_1 + \rho_2}{2}, \frac{\beta_{0, 1} + \beta_{0, 2}}{2} \right) 
        & <  \frac{1}{2} \bigl( G_{\kappa} (\rho_1, \beta_{0, 1}) + G_{\kappa} (\rho_2, \beta_{0, 2}) \bigr) \\
        & =  \frac{1}{\sqrt{H_{\max}}} \frac{1}{2} \Bigl( 
        \sqrt{1 - \smash[b]{\rho_1^2}} + \sqrt{1 - \smash[b]{\rho_2^2}} \Bigr) \\
        & \le \frac{1}{\sqrt{H_{\max}}} \sqrt{1 - \left( \frac{\rho_1 + \rho_2}{2} \right)^2},
    \end{align*}
    where in the last line we use the concavity of the mapping $x \mapsto \sqrt{1 - x^2}$. It finally follows that
    \begin{equation*}
        H_{\kappa} \left( \frac{\rho_1 + \rho_2}{2}, \frac{\beta_{0, 1} + \beta_{0, 2}}{2} \right) > H_{\max},
    \end{equation*}
    a contradiction. This concludes the proof.
\end{proof}
%%%%% Old proof %%%%%
% \begin{proof}
%     For simplicity, we fix $\kappa \in \R$ and write $\rho = \rho^\star(\kappa)$, $\beta_0 = \beta_0^\star(\kappa)$.
%     \begin{itemize}
%         \item (Uniqueness of $\rho^\star$) \ \  Notice that $H_\kappa(\rho, \beta_0)$ is strictly increasing in $[-1, 0]$ and $H_\kappa(1, \beta_0) = 0$, so we must have $\rho^\star(\kappa) \in [0, 1)$. In order to show $\rho^\star(\kappa)$ is unique, we assume that there exists $\rho_1, \rho_2 \in [0, 1)$ and $\beta_{0,1}, \beta_{0,2} \in \R$, such that $\rho_1 \not= \rho_2$ and
%         \begin{equation*}
%             H_\kappa(\rho_1, \beta_{0, 1}) = H_\kappa(\rho_2, \beta_{0, 2})
%             = \max_{\rho \in [0, 1], \beta_0 \in \R} H_\kappa(\rho, \beta_0).
%         \end{equation*}
%         Denote this maximum by $H_{\max}$. Let us define a new function $G_\kappa: \mathsf{B}_2(1)_+ \times \R \to \R$ by
%         \ljy{\begin{equation*}
%             G_\kappa(\rho, r, \beta_0) := \left( \E\left[ \bigl(  s(Y) \kappa - \rho \norm{\bmu}_2 + \rho G_1 + r G_2 - \beta_0 Y \bigr)_+^2 \right] \right)^{1/2}
%             % G_\kappa(\rho, r, \beta_0) := \frac{r^2}{\E\left[ \bigl(  s(Y) \kappa - \rho \norm{\bmu}_2 + G - \beta_0 Y \bigr)_+^2 \right]},
%             \quad
%             \mathsf{B}_2(1)_+ := \mathsf{B}_2(1) \cap [0, 1]^2, 
%         \end{equation*}
%         where $G_1, G_2 \iidsim \normal(0, 1)$ are independent of $Y$.} We claim $G_\kappa$ is convex, by using the following facts:
%         \begin{itemize}
%             \item The mapping $(\rho, r, \beta_0) \mapsto s(Y(\omega)) \kappa - \rho \norm{\bmu}_2 + \rho G_1(\omega) + r G_2(\omega) - \beta_0 Y(\omega)$ is affine, $\forall\, \omega \in \Omega$.
%             \item The mapping $x \mapsto (x)_+^2$ is convex.
%             \item Expectation (integral) preserves convexity. %So the denominator of $G_\kappa$ is convex in $\rho, \beta_0$.
%             % \item The mapping $(x, y) \mapsto f(x)/g(y)$ is convex if both $f, g$ are convex and $f(x) \ge 0$, $g(y) > 0$.
%         \end{itemize}
%         \ljy{$H_\kappa$ and $G_\kappa$ are related by
%         \[  G_\kappa(\rho, r, \beta_0) = \frac{r}{ \sqrt{ H_{\kappa}(\rho, \beta_0) } },
%         \qquad
%         \forall\, \rho \in [0, 1),  \ \  r = \sqrt{1 - \rho^2},  \ \  \beta_0 \in \R. \]
%         }
%         Now, we denote
%         \begin{equation*}
%             r_1 := \sqrt{1 - \rho_1^2},
%             \qquad 
%             r_2 := \sqrt{1 - \rho_2^2},            
%         \end{equation*}
%         and
%         \begin{equation*}
%             \bar\rho :=
%             \frac{\rho_1 + \rho_2}{2}, 
%             \qquad
%             \bar r := \frac{r_1 + r_2}{2},
%             \qquad
%             \bar\beta_0  :=  \frac{\beta_{0, 1} + \beta_{0, 2}}{2} .
%         \end{equation*}
%         Then by convexity of $G_\kappa$, we have
%         \ljy{\begin{equation}
%             \label{eq:H_max1}
%             \begin{aligned}
%             %     H_{\max} = \frac12 H_\kappa(\rho_1, \beta_{0, 1}) + \frac12 H_\kappa(\rho_2, \beta_{0, 2})
%             % & = \frac12 G_\kappa(\rho_1, r_1, \beta_{0, 1}) + \frac12 G_\kappa(\rho_2, r_2,\beta_{0, 2}) \\
%             % & \le G_\kappa(\bar\rho, \bar r, \bar\beta_0).
%             G_\kappa(\bar\rho, \bar r, \bar\beta_0)
%             \le \frac12 G_\kappa(\rho_1, r_1, \beta_{0, 1}) + \frac12 G_\kappa(\rho_2, r_2,\beta_{0, 2})
%             = \frac{r_1}{2\sqrt{H_{\max}}} + \frac{r_2}{2\sqrt{H_{\max}}}
%             = \frac{\bar r}{\sqrt{H_{\max}}}.
%             \end{aligned}
%         \end{equation}
%         }
%         Since $\rho_1 \not= \rho_2$ and $\mathsf{B}_2(1)$ is strictly convex, we know that $\sqrt{\bar\rho^2 + \bar r^2} < 1$. We further denote
%         \begin{equation*}
%             \rho' := \frac{\bar\rho}{\sqrt{\bar\rho^2 + \bar r^2}},
%             \qquad
%             r' := \frac{\bar r}{\sqrt{\bar\rho^2 + \bar r^2}},
%             \qquad
%             \beta_0' := \frac{\bar \beta_0}{\sqrt{\bar\rho^2 + \bar r^2}},
%             \qquad
%             \kappa' := \frac{\kappa}{\sqrt{\bar\rho^2 + \bar r^2}}.
%         \end{equation*}
%         Then $\rho'{}^2 + r'{}^2 = 1$. %Since $G_\kappa$ is strictly increasing in $\rho$ and $r$, we have
%         \ljy{According to \cref{eq:H_max1}, we have
%         \begin{equation*}
%             \frac{1}{\sqrt{H_{\max}}} \ge \frac{G_\kappa(\bar\rho, \bar r, \bar\beta_0)}{\bar r}
%             = \frac{G_{\kappa'}(\rho', r', \beta_0')}{r'}
%             \overset{\text{(i)}}{>} \frac{G_{\kappa}(\rho', r', \beta_0')}{r'}
%             = \frac{1}{\sqrt{ H_\kappa(\rho', \beta_0') }},
%         \end{equation*}
%         where (i) is due to the fact that $\kappa' > \kappa$ and $c \mapsto (\E[ ( c + G )_+^2 ])^{1/2}$ is strictly increasing.
%         % \begin{equation}
%         %     \label{eq:H_max2}
%         %         G_\kappa(\bar\rho, \bar r, \bar\beta_0) 
%         %         < G_{\kappa}(\rho', r', \bar\beta_0)
%         %         = H_{\kappa}(\rho', \bar\beta_0).
%         % \end{equation}
%         %Combining \cref{eq:H_max1} and \eqref{eq:H_max2} 
%         This gives a contradiction $H_{\kappa}(\rho', \beta_0') > H_{\max}$. Therefore, the maximizer $\rho^\star(\kappa)$ must be unique.
%         }


%         \item (Uniqueness of $\beta_0^\star$) \ \ Similar to the proof at the beginning of step 4, $\beta_0^\star(\kappa)$ is the minimizer of
%         \begin{equation*}
%             h(\beta_0) :=   \E\left[ \bigl(  s(Y) \kappa - \rho^\star \norm{\bmu}_2 + G - \beta_0 Y \bigr)_+^2 \right].
%         \end{equation*}
%         We claim $h$ is strictly convex and continuously differentiable, since
% \begin{equation*}
%     h'(\beta_0)
%     = -2\pi \E\left[ \bigl( \tau \kappa - \rho^\star \norm{\bmu}_2 + G - \beta_0 \bigr)_+ \right]
%     + 2(1 - \pi) \E\left[ \bigl( \kappa - \rho^\star \norm{\bmu}_2 + G + \beta_0 \bigr)_+ \right] 
%     ,
% \end{equation*}
% which is strictly increasing and $h'(+\infty) = +\infty$, $h'(-\infty) = -\infty$. Hence, $\beta_0^\star(\kappa) \in \R$ is unique.
%     \end{itemize}
% \end{proof}


\subsubsection{Completing the proof of \cref{thm:SVM_main}}
\begin{proof}[\textbf{Proof of \cref{thm:SVM_main}}]
\noindent
\textbf{\ref{thm:SVM_main_trans}} is established by \cref{lem:over_phase_trans}.

\begin{proof}[\textbf{\emph{\ref{thm:SVM_main_var}:}}]
Notice the definition of $(\rho^*, \beta_0^*, \kappa^*)$ we used in our proof (\cref{subsubsec:over_phase}, \ref{subsubsec:over_logit_conv}) is based on \cref{eq:SVM_asymp_simple}. It suffices to show the equivalence of two optimization problems \cref{eq:SVM_variation} and \eqref{eq:SVM_asymp_simple}. Now we fix $\rho$, $\beta_0$ in \cref{eq:SVM_variation} and $X := \rho \| \bmu\|_2 + G + Y \beta_0$. Then \cref{eq:SVM_variation} can be written as
\begin{equation}\label{eq:SVM_var2}
    \begin{aligned}
        \begin{array}{cl}
            \underset{ \kappa > 0 , \, \xi \in \cL^2  }{ \mathrm{maximize} } & \kappa, \\
            \underset{ \phantom{\smash{\bm\beta \in \R^d, \beta_0 \in \R, \kappa \in \R} } }{\text{subject to}} &  
            X + \sqrt{1 - \rho^2} \xi \ge s(Y) \kappa,  
            \qquad \E[\xi^2]  \le  1/\delta .
        \end{array}
    \end{aligned}
\end{equation}
Note that it can be written as a convex optimization problem, and it is infeasible if $\rho = \pm 1$ (since $X$ has support $\R$). Take $\rho \in (-1, 1)$. According to the Karush--Kuhn--Tucker (KKT) and Slater's conditions for variational problems \cite[Theorem 2.9.2]{zalinescu2002convex}, $(\kappa, \xi)$ is the solution to \cref{eq:SVM_var2} if and only if it satisfies the following for some $\Lambda \in \cL^1, \Lambda \ge 0$ (a.s.) and $\nu \ge 0$:
\begin{equation*}
    \begin{aligned}
        -1 + \E[s(Y)\Lambda] = 0, \qquad
        -\sqrt{1 - \rho^2} \Lambda + 2 \nu \xi = 0 \ \  \text{(a.s.)},  \\
        \nu\left( \E[\xi^2] - \delta^{-1} \right) = 0,
        \qquad
        \Lambda \bigl( s(Y)\kappa - X - \sqrt{1 - \rho^2} \xi \bigr) = 0 \ \  \text{(a.s.)}.
    \end{aligned}
\end{equation*}
Clearly $\nu > 0$ (otherwise, $\Lambda = 0$ a.s., a contradiction). Consider the following two cases:
\begin{itemize}
    \item On the event $\{ s(Y(\omega))\kappa - X(\omega) < 0\}$, we obtain $s(Y(\omega))\kappa - X(\omega) - \sqrt{1 - \rho^2} \xi(\omega) < 0$, which implies $\Lambda(\omega) = 0$. Therefore, $\xi(\omega) = 0$.
    \item On the event $\{ s(Y(\omega))\kappa - X(\omega) > 0\}$, we obtain $\sqrt{1 - \rho^2} \xi(\omega) \ge s(Y(\omega))\kappa - X(\omega) > 0$, which implies $\xi(\omega) > 0$. Therefore, $\Lambda(\omega) > 0$, and thus $s(Y(\omega))\kappa - X(\omega) - \sqrt{1 - \rho^2} \xi(\omega) = 0$.
\end{itemize}
(Note $\P(s(Y)\kappa - X = 0) = 0$.) By combining these, we get $\sqrt{1 - \rho^2}\xi = (s(Y)\kappa - X)_+$. This proves \cref{eq:SVM_main_xi_star}. Plug in it into \cref{eq:SVM_variation} gives \cref{eq:SVM_asymp_simple}. The proof of $\rho^* \in (0, 1)$ and its independence of $\tau$ is given by \cref{lem:gordon_eq} in \cref{subsec:over_asymp}. 

This concludes the proof of part \ref{thm:SVM_main_var}.
\end{proof}





\begin{proof}[\textbf{\emph{\ref{thm:SVM_main_mar}, $\delta < \delta^*(0)$:}}]
We show that $\hat\kappa_n \conp \kappa^*$ in \cref{lem:over_mar_conp} can be strengthened to $\hat\kappa_n \conL{2} \kappa^*$. To this end, we show that $\hat\kappa_n^2$ is uniformly integrable (u.i.). Recall that $\kappa(\hat\vbeta_n, \hat\beta_{0,n}) \ge 0$ and
\begin{align*}
        \kappa(\hat\vbeta_n, \hat\beta_{0,n}) 
        & = \min_{i \in [n]} \wt y_i \bigl( \< \xx_i, \hat\vbeta_n \> + \hat\beta_{0,n} \bigr)
        = \min_{i \in [n]} \wt y_i \bigl(  y_i \< \vmu, \hat\vbeta_n \> + \< \zz_i, \hat\vbeta_n \> + \hat\beta_{0,n} \bigr) \\
        & = \min \left\{
            \min_{i: y_i = +1}  \tau^{-1}\bigl( \< \vmu, \hat\vbeta_n \> + \< \zz_i, \hat\vbeta_n \> + \hat\beta_{0,n} \bigr) ,
            \min_{i: y_i = -1}  \bigl( \< \vmu, \hat\vbeta_n \> - \< \zz_i, \hat\vbeta_n \> - \hat\beta_{0,n}\bigr)
        \right\}.
\end{align*}
Hence, on the event $\mathcal{D}_n^c$ (non-degenerate case), we have $\hat\kappa_n = \kappa(\hat\vbeta_n, \hat\beta_{0,n})$ and it can be bounded by the average from each class:
\begin{equation*}
    \begin{aligned}
        \kappa(\hat\vbeta_n, \hat\beta_{0,n}) & \le \tau^{-1}\bigl( \< \vmu, \hat\vbeta_n \> + \< \bar\zz^+_n, \hat\vbeta_n \> + \hat\beta_{0,n} \bigr) := \bar \kappa_n^+ , \\
        \kappa(\hat\vbeta_n, \hat\beta_{0,n}) & \le \phantom{\tau^{-1}\bigl(} \< \vmu, \hat\vbeta_n \> - \< \bar\zz^-_n, \hat\vbeta_n \> - \hat\beta_{0,n} \phantom{\bigr)} := \bar \kappa_n^-  ,
    \end{aligned}
\end{equation*}
where
\begin{equation*}
    \bar\zz^+_n := \frac{1}{n_+} \sum_{i: y_i = +1} \zz_i,
    \qquad
    \bar\zz^-_n := \frac{1}{n_-} \sum_{i: y_i = -1} \zz_i.
\end{equation*}
Combine these two bounds and apply Cauchy--Schwarz inequality, we obtain
\begin{equation*}
        \kappa(\hat\vbeta_n, \hat\beta_{0,n})
        \le \frac{\tau \bar \kappa_n^+  +  \bar \kappa_n^-}{\tau + 1}
        = \frac{2}{\tau + 1} 
        \!
        \left(
        \< \vmu, \hat\vbeta_n \> 
        + \left\< \frac{\bar\zz^+_n - \bar\zz^-_n}{2}, \hat\vbeta_n \right\>
        \right)
        \le 
        \frac{2}{\tau + 1} \left( \norm{\vmu}_2 + \norm{ \wt \zz_n }_2 \right),
\end{equation*}
where
\begin{equation*}
    \wt \zz_n := \frac{\bar\zz^+_n - \bar\zz^-_n}{2},
    \quad
    \wt \zz_n \,|\, \yy \sim \normal\left( \bzero, \frac14\Bigl( \frac{1}{n_+} + \frac{1}{n_-} \Bigr) \bI_d \right).
\end{equation*}
Therefore,
\begin{equation*}
    0   \le   \hat\kappa_n  =  \kappa(\hat\vbeta_n, \hat\beta_{0,n}) \ind_{1 \le n_+ \le n-1} 
    \le \frac{2}{\tau + 1} \left( 
        \norm{\vmu}_2 + \norm{ \wt \zz_n }_2 \ind_{1 \le n_+ \le n-1}
     \right).
\end{equation*}
In order to prove $\hat\kappa_n^2$ is u.i., it suffices to show that $\norm{ \wt \zz_n }_2^2 \ind_{1 \le n_+ \le n-1}$ is u.i.. Next, we prove this by establishing that $\norm{ \wt \zz_n }_2^2 \ind_{1 \le n_+ \le n-1}$ converges in $\cL^1$. It requires two steps (by Scheffé's Lemma):
\begin{subequations}
\begin{align}
    \E \left[ \norm{ \wt \zz_n }_2^2 \ind_{1 \le n_+ \le n-1} \right] \to \frac{1}{4\delta} \left( \frac{1}{\pi} + \frac{1}{1-\pi} \right),
    \label{eq:svm_ui_a}
    \\
    \norm{ \wt \zz_n }_2^2 \ind_{1 \le n_+ \le n-1} \conp \frac{1}{4\delta} \left( \frac{1}{\pi} + \frac{1}{1-\pi} \right).
    \label{eq:svm_ui_b}
\end{align}
\end{subequations}
For \cref{eq:svm_ui_a}, Observe
\begin{equation*}
    \begin{aligned}
      & \E \left[ \norm{ \wt \zz_n }_2^2 \ind_{1 \le n_+ \le n-1} \right]
    = \E \left[ \E\bigl[ \norm{ \wt \zz_n }_2^2 \ind_{1 \le n_+ \le n-1} \,|\, \yy \bigr] \right] \\
    = {} & \E \left[ \frac{d}{4} \Bigl( \frac{1}{n_+} + \frac{1}{n_-} \Bigr) \ind_{1 \le n_+ \le n-1} \right]
    = \frac{d}{4n} \E \left[ \Bigl( \frac{n}{n_+} + \frac{n}{n_-} \Bigr) \ind_{1 \le n_+ \le n-1} \right].
    \end{aligned}
\end{equation*}
To evaluate the expected value, note that (by law of large numbers)
\begin{equation*}
    \frac{n}{n_+} \ind_{1 \le n_+ \le n-1} \le \frac{2n}{n_+ + 1},
    \qquad
    \frac{n}{n_+}\ind_{1 \le n_+ \le n-1} \conp \frac{1}{\pi},
    \qquad
    \frac{2n}{n_+ + 1} \conp \frac{2}{\pi}.
\end{equation*}
A classical result \cite{chao1972negative} gives
\begin{equation*}
    \lim_{n \to \infty} \E\left[ \frac{2n}{n_+ + 1} \right]
    = \lim_{n \to \infty} \frac{2n\left( 1 - (1-\pi)^{n+1} \right)}{(n+1)\pi} = \frac{2}{\pi}.
\end{equation*}
So $\frac{2n}{n_+ + 1} \conL{1} \frac{2}{\pi}$, which implies $\frac{2n}{n_+ + 1}$ is u.i., and so is $\frac{n}{n_+} \ind_{1 \le n_+ \le n-1}$. Therefore, by Vitali convergence theorem, we have $\frac{n}{n_+} \ind_{1 \le n_+ \le n-1} \conL{1} \frac{1}{\pi}$. Similar arguments give $\frac{n}{n_-} \ind_{1 \le n_+ \le n-1} \conL{1} \frac{1}{1-\pi}$. Hence
\begin{equation*}
    \lim_{n \to \infty} \E \left[ \norm{ \wt \zz_n }_2^2 \ind_{1 \le n_+ \le n-1} \right]
    = \lim_{n \to \infty} \frac{d}{4n} \cdot \lim_{n \to \infty} \E \left[ \Bigl( \frac{n}{n_+} + \frac{n}{n_-} \Bigr) \ind_{1 \le n_+ \le n-1} \right]
    = \frac{1}{4\delta} \left( \frac{1}{\pi} + \frac{1}{1-\pi} \right).
\end{equation*}
For \cref{eq:svm_ui_b}, notice that $\norm{ \wt \zz_n }_2^2 \,|\, \yy \sim a_n \chi_d^2$, where $a_n = \frac{1}{4} ( \frac{1}{n_+} + \frac{1}{n_-})$. By concentration inequality (e.g., \cref{lem:subG_concentrate}\ref{lem:subG-Hanson-Wright-I}), we have
\begin{equation*}
    \P\left( \abs{ \norm{ \wt \zz_n }_2^2 - d a_n } \ge \varepsilon \,\big|\, \yy \right) 
    \le 2\exp\left( -c \min\left\{ \frac{\varepsilon^2}{d a_n^2}, \frac{\varepsilon}{a_n} \right\} \right)
    = o_\P(1),
\end{equation*}
where $c > 0$ is a constant, $a_n = o_\P(1)$, $d a_n \conp \frac{1}{4\delta} ( \frac{1}{\pi} + \frac{1}{1-\pi} )$. By taking expectation on both sides and using bounded convergence theorem, we have $\norm{ \wt \zz_n }_2^2 - d a_n = o_\P(1)$. Then we get \cref{eq:svm_ui_b}.
% {Kangjie's Proof of convergence in probability:} We already know that $n_+ / n \to \pi$ in probability, namely $\P (\vert n_+/n - \pi \vert \ge \veps) \to 0$ for all $\veps > 0$. Now fix any such $n_+$, the conditional distribution of $\norm{ \wt \zz_n }_2^2 \ind_{1 \le n_+ \le n-1}$ is just
% \begin{equation}
%     \frac{1}{4} \Bigl( \frac{1}{n_+} + \frac{1}{n_-} \Bigr) \chi^2 (d),
% \end{equation}
% for which we can use concentration inequality for chi-squared random variables. This leads to
% \begin{equation}
%     \P \left( \left\vert \norm{ \wt \zz_n }_2^2 \ind_{1 \le n_+ \le n-1} - \frac{1}{4 \delta} \Bigl( \frac{n}{n_+} + \frac{n}{n_-} \Bigr) \right\vert \ge \veps \Big\vert n_+ \right) \to 0
% \end{equation}
% uniformly for $n_+$ satisfying $\vert n_+/n - \pi \vert \le \veps$, or use bounded convergence to conclude that the unconditional probability goes to $0$:
% \begin{equation}
%     \P \left( \left\vert \norm{ \wt \zz_n }_2^2 \ind_{1 \le n_+ \le n-1} - \frac{1}{4 \delta} \Bigl( \frac{1}{\pi} + \frac{1}{1 - \pi} \Bigr) \right\vert \ge 2 \veps \right) \to 0.
% \end{equation}

Finally, \cref{eq:svm_ui_a} and \eqref{eq:svm_ui_b} imply that $\norm{ \wt \zz_n }_2^2 \ind_{1 \le n_+ \le n-1}$ converges in $\cL^1$, and thus is u.i.. So $\hat\kappa_n^2$ is also u.i.. By Vitali convergence theorem, convergence in probability of $\hat\kappa_n$ can be strengthen to $\cL^2$ convergence. 

This concludes the proof of part \ref{thm:SVM_main_mar} for $\delta < \delta^*(0)$.
\end{proof}






\begin{proof}[\textbf{\emph{\ref{thm:SVM_main_mar}, $\delta > \delta^*(0)$:}}]
For non-separable regime, we cannot work with $\xi_{n, \kappa}$ in \cref{eq:xi_n_kappa} to show a negative margin, since $\hat\kappa_n \ge 0$ always holds (by taking $\vbeta = 0$, $\beta_0 = 0$). To this end, we define
\begin{equation*}
    \Xi_{n, \kappa} := \min_{ \substack{ \norm{\vbeta}_2 = 1 \\ \beta_0 \in \R} } \frac{1}{\sqrt{d}} \norm{ \left( \kappa \bs_\yy - \yy \odot \XX \vbeta  - \beta_0 \yy \right)_+ }_2,
\end{equation*}
which replace the constraint $\norm{\vbeta}_2 \le 1$ in $\xi_{n, \kappa}$ by $\norm{\vbeta}_2 = 1$. Here we define the margin as
\begin{equation}\label{eq:wt_kappa_def}
    \wt\kappa_n := \sup\{ \kappa \in \R:  \Xi_{n,\kappa} = 0 \}.
\end{equation}
Note that $\wt\kappa_n = \hat\kappa_n$ on separable data, but $\wt\kappa_n$ is allowed to be negative. Then our goal is to show
\begin{equation}\label{eq:neg_kappa}
    \wt\kappa_n \le -\overline{\kappa}
\end{equation}
holds for some $\overline{\kappa} > 0$ with high probability. Then followed by the proof outline at the beginning of \cref{append_subsec:sep}, we can also define a series of random variables in a similar way:
\begin{align*}
    \Xi'_{n, \kappa, B} & := \min_{ \substack{ \norm{\vbeta}_2 = 1 \\ \abs{\beta_0} \le B } } \max_{ \substack{ \norm{\blambda}_2 \le 1 \\ \blambda \odot \yy \ge 0} } \frac{1}{\sqrt{d}} \blambda^\top \left( \kappa \bs_\yy \odot \yy - \XX \vbeta  - \beta_0 \bone \right),
    \\
    \Xi_{n, \kappa, B}'^{(1)} & := \min_{ \substack{ \rho^2 + \norm{\vtheta}_2^2 = 1 \\ \abs{\beta_0} \le B } } \max_{ \substack{ \norm{\blambda}_2 \le 1 \\ \blambda \odot \yy \ge 0} } \frac{1}{\sqrt{d}}  \left(
    \norm{\blambda}_2 \vg^\top \btheta + \norm{\btheta}_2 \vh^\top \blambda + \blambda^\top \bigl( 
        \kappa \bs_\yy \odot \yy - \rho\norm{\bmu}_2 \yy + \rho \vu - \beta_0 \bone
     \bigr)
     \right),
    \\
    \bar\Xi'^{(2)}_{\kappa, B} & :=  \min_{ \substack{ \rho^2 + r^2 = 1, r \ge 0 \\  \abs{\beta_0} \le B } }
    -r + \sqrt{\delta} \left( \E\left[ \bigl(  s(Y) \kappa - \rho \norm{\bmu}_2 + \rho G_1 + rG_2 - \beta_0 Y \bigr)_+^2 \right] \right)^{1/2},
\end{align*}
where the constraints $\norm{\vbeta}_2 \le 1$, $\rho^2 + \norm{\vtheta}_2^2 \le 1$, and $\rho^2 + r^2 \le 1$ in $\xi'_{n, \kappa, B}$, $\xi_{n, \kappa, B}'^{(1)}$, and $\bar\xi'^{(2)}_{\kappa, B}$ all become equality constraints. Then we follow the same arguments in Step 1---5 (\cref{lem:over_beta0}---\ref{lem:over_mar_conp}).
\begin{itemize}
    \item Analogous to the proof of \cref{lem:over_beta0}, we have $| \P\bigl( \Xi_{n, \kappa} = 0 \bigr) - \P\bigl(\Xi'_{n, \kappa, B} = 0 \bigr) | \to 0$.
    \item Analogous to the proof of \cref{lem:over_CGMT}, we can apply CGMT \cref{lem:CGMT} to connect $\Xi'_{n, \kappa, B}$ with $\Xi_{n, \kappa, B}'^{(1)}$. 
\begin{equation*}
    \P\left( \Xi'_{n, \kappa, B} \le t \vphantom{\Xi'^{(1)}_{n, \kappa, B}} \right) 
    \le 2\, \P\left( \Xi'^{(1)}_{n, \kappa, B} \le t \right).
\end{equation*}
    Here we only get a one-sided inequality since $\{(\rho, \vtheta): \rho^2 + \norm{\vtheta}_2^2 = 1 \}$ is non-convex.
    \item Analogous to the proof of \cref{lem:over_ULLN}, we have $\Xi'^{(1)}_{n, \kappa, B}  \conp ( \bar\Xi'^{(2)}_{\kappa, B} )_+$.
    \item Notice that the optimal $r$ in $\bar\Xi'^{(2)}_{\kappa, B}$ must be nonnegative. Hence, by substituting $r = \sqrt{1 - \rho^2}$, we have $\bar\Xi'^{(2)}_{\kappa, B} = \bar\xi_{\kappa}^{(3)}$ (\cref{eq:xi3_kappa}) for some $B > 0$ large enough.
\end{itemize}
Recall that in the proof of \cref{lem:over_phase_trans} and \ref{lem:over_mar_conp}, if $\delta > \delta^*(0)$, then there exists a $\kappa_0 < 0$, such that $\bar\xi_{\kappa_0}^{(3)} = 0$. According to \cref{eq:phase_equiv}, for any $\varepsilon > 0$ small enough, by using above relations, we have 
\begin{align*}
\bar\xi_{\kappa_0 + \varepsilon}^{(3)} = \bar\Xi'^{(2)}_{\kappa_0 + \varepsilon, B} > 0
\quad & \Longrightarrow \quad
\Xi'^{(1)}_{n, \kappa_0 + \varepsilon, B}  \conp \bigl( \bar\xi^{(3)}_{\kappa_0 + \varepsilon} \bigr)_+ > 0
\\
\quad \Longrightarrow \quad
\Xi'_{n, \kappa_0 + \varepsilon, B} > 0 ~ \text{w.h.p.}
\quad & \Longrightarrow \quad
\Xi_{n, \kappa_0 + \varepsilon} > 0 ~ \text{w.h.p.}.
\end{align*}
By \cref{eq:wt_kappa_def}, $\wt\kappa_n < \kappa_0 + \varepsilon < 0$ holds with high probability (by taking $\varepsilon$ to be sufficiently small), which proves \cref{eq:neg_kappa}.

This concludes the proof of part \ref{thm:SVM_main_mar} for $\delta > \delta^*(0)$.
\end{proof}








\begin{proof}[\textbf{\emph{\ref{thm:SVM_main_param}, \ref{thm:SVM_main_logit}:}}]
We have shown parameter and ELD convergence for the case $\tau = 1$ in \cref{lem:over_logit_conv}. Now for any $\tau \ge 1$, denote $\hat\vbeta_n(\tau), \hat\beta_{0, n}(\tau), \hat\kappa_n(\tau)$ as the max-margin solution to \cref{eq:over_max-margin}, and define
\begin{equation*}
     \hat\rho_n(\tau) :=  \left\< \hat\vbeta_n(\tau), \frac{\vmu}{\norm{\vmu}_2 } \right\>.
\end{equation*}
Similarly, denote $\rho^*(\tau), \beta_0^*(\tau), \kappa^*(\tau)$ as the optimal solution to \cref{eq:SVM_asymp_simple}. By \cref{prop:SVM_tau_relation},
\begin{itemize}
    \item We have
    \begin{equation}\label{eq:param_hat_tau}
        \hat\rho_n(\tau) = \hat\rho_n(1),
        \qquad
        \hat\beta_{0, n}(\tau) = \hat\beta_{0, n}(1) + \frac{\tau - 1}{\tau + 1} \hat\kappa_n(1). 
    \end{equation}

    \item We can write
    \begin{equation}\label{eq:Ln_hat_tau}
        \begin{aligned}
        %     \hat \cL_{n} & = \frac1n \sum_{i=1}^n \delta_{\wt y_i ( \< \xx_i, \hat\vbeta_n \> + \hat\beta_{0,n}(\tau) )  \ind_{\cD_n^c}  }
        % =: \Law\left( s(Y')^{-1} Y' \bigl( \< \xx', \hat\vbeta_n \> + \hat\beta_{0,n}(\tau) \bigr) \ind_{\cD_n^c} \right) \\
        % & = \Law \left(  s(Y')^{-1} \Bigl\{ Y' \bigl( \< \xx', \hat\vbeta_n \> + \hat\beta_{0,n}(1) \bigr) \ind_{\cD_n^c} \Bigr\}
        % + s(Y')^{-1} Y' \cdot \frac{\tau - 1}{\tau + 1} \hat\kappa_n(1) 
        % \right).
            \hat \nu_{n} & = \frac1n \sum_{i=1}^n \delta_{ \left(y_i, \< \xx_i, \hat\vbeta_n \> + \hat\beta_{0,n}(\tau) \right)  \ind\{\cD_n^c\}  }
        =: \Law\left( Y' \ind_{\cD_n^c}, \bigl( \< \xx', \hat\vbeta_n \> + \hat\beta_{0,n}(\tau) \bigr) \ind_{\cD_n^c} \right) \\
        & = \Law \left( Y' \ind_{\cD_n^c},  \bigl( \< \xx', \hat\vbeta_n \> + \hat\beta_{0,n}(1) \bigr) \ind_{\cD_n^c}
        + \frac{\tau - 1}{\tau + 1} \hat\kappa_n(1)
        \right).
        \end{aligned}
    \end{equation}
\end{itemize}
Besides, according to \cref{cor:asymp_tau_relation},
\begin{itemize}
    \item We have
    \begin{equation}\label{eq:param_star_tau}
        \rho^*(\tau) = \rho^*(1),
        \qquad
        \beta_0^*(\tau) = \beta_0^*(1) + \frac{\tau - 1}{\tau + 1} \kappa^*(1).
    \end{equation}

    \item We can also write
    \begin{equation}\label{eq:Ln_star_tau}
    \begin{aligned}
        \nu_*
        & = \Law \, \Bigl( Y, Y \max \bigl\{ s(Y)\kappa^*(\tau) , G + \rho^*\norm{\vmu}_2 + \beta_0^*(\tau) Y \bigr\} \Bigr) \\
        & = \Law \left( Y, Y \max \bigl\{ \kappa^*(1) , G + \rho^*\norm{\vmu}_2 + \beta_0^*(1) Y \bigr\} 
        + \frac{\tau - 1}{\tau + 1} \kappa^*(1)
        \right).
    \end{aligned}
    \end{equation}
\end{itemize}
We have shown $\hat\kappa_n(1) \conL{2} \kappa^*(1)$ and $\hat\beta_{0,n}(1) \conp \beta_0^*(1)$ in \cref{lem:over_logit_conv}. Then by continuous mapping theorem, comparing \cref{eq:param_hat_tau} and \eqref{eq:param_star_tau}, it follows that $\hat\beta_{0, n}(\tau) \conp \beta_0^*(\tau)$ for any $\tau > 0$. 

In \cref{lem:over_logit_conv}, we have shown that $W_2\bigl( \hat \nu_{n}, \nu_* \bigr) = o_\varepsilon(1)$ for $\tau = 1$ with high probability, i.e.,
\begin{equation}
\label{eq:logit_conv_1}
    W_2 \Bigl(
        \Law \Bigl(
        Y'\ind_{\cD_n^c} ,
        \underbrace{    
        \bigl( \< \xx', \hat\vbeta_n \> + \hat\beta_{0,n}(1) \bigr) \ind_{\cD_n^c}
        }_{ =: U_n} \Bigr)
        ,
        \Law \Bigl( 
        Y,
            \underbrace{    
        Y\max \bigl\{ \kappa^*(1) , G + \rho^*\norm{\vmu}_2 + \beta_0^*(1) Y \bigr\} 
        }_{=: U^*} \Bigr)
        \Bigr)
    \! = \!
    o_\varepsilon(1).
\end{equation}
Then there exists a coupling $(Y', Y, U_n, U^*)$ such that, with high probability, 
\begin{align*}
    W_2( \hat \nu_{n},  \nu_* )
    & \le   
        \left( \E_{\cdot | n} \bigl[ (
           Y'\ind_{\cD_n^c} - Y 
        )^2 \bigr] \right)^{\frac12} 
        + \left( \E_{\cdot | n} \biggl[ \Bigl(
           U_n - U^*  + \frac{\tau - 1}{\tau + 1} \hat\kappa_n(1) - \frac{\tau - 1}{\tau + 1} \kappa^*(1)
        \Bigr)^2 \biggr] \right)^{\frac12} 
        \\
    & \overset{\mathmakebox[0pt][c]{\text{(i)}}}{\le}
        \left( \E_{\cdot | n} \bigl[ (
           Y'\ind_{\cD_n^c} - Y 
        )^2 \bigr] \right)^{\frac12} 
        +
        \left( \E_{\cdot | n} \bigl[ (
            U_n - U^* 
        )^2 \bigr] \right)^{\frac12} 
        +
        \frac{\tau - 1}{\tau + 1}
         \left( \E_{\cdot | n} \bigl[ (
            \hat\kappa_n(1) - \kappa^*(1) 
        )^2 \bigr] \right)^{\frac12}
        \\
    & \overset{\mathmakebox[0pt][c]{\text{(ii)}}}{=}  o_\varepsilon(1),
\end{align*}
where in (i) we use Minkowski inequality, while in (ii) we use $\hat\kappa_n(1) \conL{2} \kappa^*(1)$ in \ref{thm:SVM_main_mar} and \cref{eq:logit_conv_1}. By taking $\varepsilon \to 0$, we can show that $W_2\bigl( \hat \nu_{n}, \nu_* \bigr) \conp 0$ holds for any $\tau > 0$.

% Then, by comparing \cref{eq:Ln_hat_tau} and \eqref{eq:Ln_star_tau}, with high probability, we have
% \begin{equation*}
%     \begin{aligned}
%            & W_2( \hat \cL_{n},  \cL_* ) \\
%     \overset{\mathmakebox[0pt][c]{\text{(i)}}}{\le} {} &  
%         \left( \E_{\cdot | n} \left[ \left(
%             s(Y')^{-1} U_n - s(Y)^{-1} U^*
%         \right)^2\right] \right)^{\frac12} 
%         + 
%         \frac{\tau - 1}{\tau + 1} \left( \E_{\cdot | n} \left[ \left(
%             s(Y')^{-1} Y' \hat\kappa_n(1) - s(Y)^{-1} Y \kappa^*(1)
%         \right)^2\right] \right)^{\frac12} \\
%     \overset{\mathmakebox[0pt][c]{\text{(ii)}}}{\le} {} & \left( \E_{\cdot | n} \left[ s(Y')^{-2} \left(
%          U_n -U^*
%     \right)^2\right] \right)^{\frac12}
%     + \left( \E_{\cdot | n} \left[ \left(
%         s(Y')^{-1} - s(Y)^{-1}
%     \right)^2  ( U^* )^2  \right] \right)^{\frac12}  
%     \\
%     & \phantom{\le}     
%     + \frac{\tau - 1}{\tau + 1}  \left\{ 
%     \left( \E_{\cdot | n} \left[ s(Y')^{-2} \bigl(
%         \hat\kappa_n(1) - \kappa^*(1)
%     \bigr)^2 \right] \right)^{\frac12}
%     +  \kappa^*(1) \left( \E_{\cdot | n} \left[ \left(
%         s(Y')^{-1}Y' - s(Y)^{-1}Y
%     \right)^2  \right] \right)^{\frac12}
%     \right\}
%     \\
%     \overset{\mathmakebox[0pt][c]{\text{(iii)}}}{\le} {} & 
%     (1 \vee \tau^{-1}) \left( \E_{\cdot | n} \left[ \left(
%         U_n -U^*
%    \right)^2\right] \right)^{\frac12}
%    + \left( \E_{\cdot | n} \left[ \left(
%        s(Y')^{-1} - s(Y)^{-1}
%    \right)^4  \right] \right)^{\frac14}
%    \left( \E \left[ ( U^* )^4  \right] \right)^{\frac14} \\
%    & \phantom{\le}     
%    + \frac{\tau - 1}{\tau + 1}  \left\{ 
%    (1 \vee \tau^{-1}) \left( \E_{\cdot | n} \left[ \bigl(
%        \hat\kappa_n(1) - \kappa^*(1)
%    \bigr)^2 \right] \right)^{\frac12}
%    +  \kappa^*(1) \left( \E_{\cdot | n} \left[ \left(
%     s(Y')^{-1}Y' - s(Y)^{-1}Y
% \right)^2  \right] \right)^{\frac12}
%    \right\} 
%    \\
%    \overset{\mathmakebox[0pt][c]{\text{(iv)}}}{\le} {} &  o_{\varepsilon}(1)  +  C_1 \left( \E_{\cdot | n} \left[ 
%     (Y' - Y)^2  \right] \right)^{\frac14}
%     + \frac{\tau - 1}{\tau + 1} \left\{
%         o_{\varepsilon}(1) + \kappa^*(1) \cdot C_2 \left( \E_{\cdot | n} \left[ 
%             (Y' - Y)^2  \right] \right)^{\frac12}
%      \right\}  
%      \\
%     \overset{\text{(v)}}{=} {} &   o_{\varepsilon}(1),
%     \end{aligned}
% \end{equation*}
% where in (i) and (ii) we use Minkowski inequality, in (iii) we use Cauchy--Schwarz inequality, in (iv) we use \cref{eq:logit_conv_1}, $\hat\kappa_n(1) \conL{1} \kappa^*(1)$ in \ref{thm:SVM_main_mar}, the fact that $y \mapsto s(y)^{-1}$ and $y \mapsto s(y)^{-1} y$ are Lipschitz, and $C_1, C_2 > 0$ are constants, while in (v) we use \cref{eq:pursuit_pre} can recall that $\eta < \min\{ \varepsilon^2/4, \varepsilon/B \} = o_{\varepsilon}(1)$. By taking $\varepsilon \to 0$, we can show that $W_2\bigl( \hat \cL_{n}, \cL_* \bigr) \conp 0$ holds for any $\tau > 0$.

For TLD convergence, we give a proof of $\hat\nu_{n}^\mathrm{test} \conw \nu_*^\mathrm{test}$. Write $\xx_\mathrm{new} = y_\mathrm{new} \bmu + \zz_\mathrm{new}$, $\zz_\mathrm{new} \sim \normal(\bzero, \bI_d)$, and recall $\hat\nu_{n}^\mathrm{test} = \Law \bigl(y^\mathrm{new}, \< \xx^\mathrm{new}, \hat\vbeta_n \> + \hat\beta_{0,n} \bigr)$. Let $G \sim \normal(0, 1)$ and $G \indep y^\mathrm{new}$, then
\begin{align*}
   \< \xx^\mathrm{new}, \hat\vbeta_n \> + \hat\beta_{0,n}
    & = \< y^\mathrm{new}\bmu + \zz^\mathrm{new}, \hat\vbeta_n \> + \hat\beta_{0,n} \\
    & = y^\mathrm{new} \hat\rho_n \norm{\bmu}_2 + \< \zz^\mathrm{new}, \hat\vbeta_n \>
    + \hat\beta_{0,n} \\
    & \cond y^\mathrm{new} ( \rho^* \norm{\bmu}_2 + G + y^\mathrm{new} \beta_{0}^*),
\end{align*}
where in the last line we use Slutsky's theorem and $y^\mathrm{new} \indep (y^\mathrm{new}\zz^\mathrm{new}, \hat\vbeta_n, \hat\beta_{0,n})$.

This concludes the proof of part \ref{thm:SVM_main_param} and \ref{thm:SVM_main_logit}.
\end{proof}

\begin{proof}[\textbf{\emph{\ref{thm:SVM_main_err}:}}]
In \ref{thm:SVM_main_logit} above, we showed that
\begin{equation*}
        \hat f(\xx_\mathrm{new}) = \< \xx_\mathrm{new}, \hat\vbeta_n \> + \hat\beta_{0,n}
        \cond y_\mathrm{new} \rho^* \norm{\bmu}_2 + G + \beta_{0}^*.
\end{equation*}
Therefore, by bounded convergence theorem, the errors have their limits
\begin{align*}
        \lim_{n \to \infty} \Err_{+,n} & = \P\left( + \rho^* \norm{\bmu}_2 + G + \beta_{0}^* \le 0 \right)
        = \Phi \left(- \rho^* \norm{\bmu}_2  - \beta_0^* \right), \\
        \lim_{n \to \infty} \Err_{-,n} & = \P\left( - \rho^* \norm{\bmu}_2 + G + \beta_{0}^* >  0 \right)
        = \Phi \left(- \rho^* \norm{\bmu}_2  + \beta_0^* \right).
\end{align*}
This concludes the proof of part \ref{thm:SVM_main_err}.
\end{proof}
Finally, we complete the proof of \cref{thm:SVM_main}.
\end{proof}
% \begin{rem}
% Some previous works \cite{kini2021label, deng2022model} also derived similar results on linear separability and parameter convergence, but logit distribution was not well studied. Our variational characterization of the optimization problem \cite{montanari2023generalizationerrormaxmarginlinear, montanari2022overparametrizedlineardimensionalityreductions} enables us to derive the asymptotics of logit distribution, which technically also simplifies the proof of parameter convergence.
% \end{rem}






\subsection{Analysis of the asymptotic optimization problem: Proof of \cref{lem:gordon_eq}}
\label{subsec:over_asymp}

We provide an analysis of the low dimensional asymptotic optimization problem \cref{eq:SVM_asymp_simple} in this subsection. The conclusion below has been used in the proofs of \cref{thm:SVM_main}\ref{thm:SVM_main_var}, \ref{thm:SVM_main_param} and \ref{thm:SVM_main_logit}. It will be also used in \cref{append_sec:mar_reb} to obtain monotonicity results.

For $G \sim \normal(0, 1)$ and $t \in \R$, we define two auxiliary functions
\begin{equation}
\label{eq:fun_g1_g2}
	g_1 (t) := \E \left[ (G + t)_+ \right], \qquad g_2 (t) := \E \left[ (G + t)_+^2 \right].
\end{equation}
Clearly both $g_1$ and $g_2$ are strictly increasing mappings from $\R$ to $\R_{>0}$. Then $g := g_2 \circ g_1^{-1}$ is also strictly increasing. The following lemma shows that the limiting parameters $(\rho^*, \beta_0^*, \kappa^*)$ defined in \cref{thm:SVM_main} can be characterized by the following system of equations, involving $g$ and $g_1^{-1}$.



\begin{lem}[Analysis of the asymptotic problem]
    \label{lem:gordon_eq}
    In the separable regime $\delta < \delta^*(0)$, $(\rho^*, \beta_0^*, \kappa^*)$ is the unique solution to the system of equations
    \begin{subequations}
    \begin{align}
    \label{eq:SVM_sys_eq_rho}
        \pi \delta \cdot g \left( \frac{\rho}{2 \pi \norm{\bmu}_2 \delta} \right) & + (1 - \pi) \delta \cdot g \left( \frac{\rho}{2(1 - \pi) \norm{\bmu}_2 \delta} \right) = 1 - \rho^2, \\
    \label{eq:SVM_sys_eq_bk1}
    - \beta_0 + \kappa \tau & = \rho \norm{\bmu}_2 + g_1^{-1} \left( \frac{\rho}{2 \pi \norm{\bmu}_2 \delta} \right), \\
    \label{eq:SVM_sys_eq_bk2}
	\beta_0 + \kappa & = \rho \norm{\bmu}_2 + g_1^{-1} \left( \frac{\rho}{2 (1 - \pi) \norm{\bmu}_2 \delta} \right),
    \end{align}
    \end{subequations}
    where $\rho^* \in (0, 1)$ does not depend on $\tau$ and $\kappa^* > 0$.
\end{lem}
\begin{proof}
    Recall that in the proof of \cref{lem:over_phase_trans} and \ref{lem:over_logit_conv}, we established that $(\rho^*, \beta_0^*, \kappa^*)$ is the unqiue solution to
    \begin{equation*}
        \begin{array}{rl}
        \maximize\limits_{\rho \in [0, 1], \beta_0 \in \R, \kappa \in \R} & \kappa, \\
        \text{\emph{subject to}} & H_\kappa(\rho, \beta_0) \ge \delta.
        \end{array}
    \end{equation*}
    Let $F(\rho, \beta_0, \kappa) := F_\kappa(\rho, \beta_0)$, where $F_\kappa$ is defined in \cref{eq:T_F_}. Then the above optimization problem is equivalent to
    \begin{equation*}
        \begin{array}{rl}
        \maximize\limits_{\rho \in [0, 1], \beta_0 \in \R, \kappa \in \R} & \kappa, \\
        \text{\emph{subject to}} & F(\rho, \beta_0, \kappa) \le 0,
        \end{array}
    \end{equation*}
    Note $F$ is convex (since $x \mapsto (x)_+^2$ is a convex map, and expectation preserves convexity). Setting $\partial_{\rho} F = 0$ and $\partial_{\beta_0} F = 0$, we obtain the first-order conditions satisfied by $(\rho^*, \beta_0^*)$:
    \begin{equation}
        \label{eq:SVM_foc1}
    \begin{aligned}
    & \E \left[ \big( G - \rho \norm{\bmu}_2 - \beta_0 + \kappa \tau \big)_+ \right] = \frac{\rho}{2 \pi \norm{\bmu}_2 \delta}, \\
    & \E \left[ \big( G - \rho \norm{\bmu}_2 + \beta_0 + \kappa \big)_+ \right] = \frac{\rho}{2 (1 - \pi) \norm{\bmu}_2 \delta}.
    \end{aligned}
    \end{equation}
Moreover, we have $\delta^*(\kappa^*) = \delta$ and thus $F(\rho, \kappa, \beta_0) = 0$ at $(\rho^*, \kappa^*, \beta_0^*)$, which leads to
\begin{equation}\label{eq:SVM_foc2}
    \pi \delta \E \left[ \big( G - \rho \norm{\bmu}_2 - \beta_0 + \kappa \tau \big)_+^2 \right]  + (1-\pi) \delta \E \left[ \big( G - \rho \norm{\bmu}_2 + \beta_0 + \kappa \big)_+^2 \right] = 1 - \rho^2.
\end{equation}
Using $g_1$, $g_2$ defined in \cref{eq:fun_g1_g2}, the first-order conditions \cref{eq:SVM_foc1} can be rewritten as
\begin{equation}
\label{eq:SVM_foc1_ref}
\begin{aligned}
	& g_1 \left( - \rho \norm{\bmu}_2 - \beta_0 + \kappa \tau \right) = \frac{\rho}{2 \pi \norm{\bmu}_2 \delta}, \\
	& g_1 \left( - \rho \norm{\bmu}_2 + \beta_0 + \kappa \right) = \frac{\rho}{2 (1 - \pi) \norm{\bmu}_2 \delta}.
\end{aligned}
\end{equation}
Similarly, we recast \cref{eq:SVM_foc2} into
\begin{equation}
\label{eq:SVM_foc2_ref}
    \pi \delta g_2 \left( - \rho \norm{\bmu}_2 - \beta_0 + \kappa \tau \right) 
    + (1-\pi) \delta g_2 \left( - \rho \norm{\bmu}_2 + \beta_0 + \kappa \right) = 1 - \rho^2.
\end{equation}
By combining \cref{eq:SVM_foc1_ref} and \eqref{eq:SVM_foc2_ref}, we get \cref{eq:SVM_sys_eq_rho}. \cref{eq:SVM_sys_eq_bk1} and \eqref{eq:SVM_sys_eq_bk2} directly come from \cref{eq:SVM_foc1_ref}.

Note that function $g: \R_{> 0} \to \R_{ > 0}$ satisfies $g(0^+) = 0$. As $\rho$ varies from $0$ to $1$, the L.H.S. of \cref{eq:SVM_sys_eq_rho} increases from $0$ to a positive number while the R.H.S. decays to $0$, which guarantees the existence and uniqueness of $\rho^* > 0$. Since \cref{eq:SVM_sys_eq_rho} does not depend on $\tau$ and $\kappa^*$, we know that $\rho^*$ does not depend on $\tau$ and $\kappa^*$. This concludes the proof.
\end{proof}

In parallel to \cref{prop:SVM_tau_relation} for the original non-asymptotic problem, we provide the following similar result on the asymptotic problem \cref{eq:SVM_asymp_simple}.

\begin{cor}\label{cor:asymp_tau_relation}
        In the separable regime $\delta < \delta^*(0)$, let $(\rho^*(\tau), \beta_0^*(\tau), \kappa^*(\tau))$ be the optimal solution to \cref{eq:SVM_asymp_simple} under hyperparameter $\tau$. Then
        \begin{equation}\label{eq:asymp_tau_relation}
		    \rho^*(\tau) = \rho^*(1),
        \qquad
        \beta_0^*(\tau) = \beta_0^*(1) + \frac{\tau - 1}{\tau + 1} \kappa^*(1),
        \qquad
        \kappa^*(\tau) = \frac{2}{\tau + 1} \kappa^*(1).
	\end{equation}
\end{cor}
\begin{proof}
    Conclusion for $\rho^*$ is already shown in \cref{lem:gordon_eq}. For $\beta_0^*$ and $\kappa^*$, note that the R.H.S. of \cref{eq:SVM_sys_eq_bk1} and \eqref{eq:SVM_sys_eq_bk2} are constants under $\rho = \rho^*$ (depending on $\pi$, $\norm{\bmu}_2$ and $\delta$). Then we have
    \begin{align*}
        - \beta_0^*(\tau) + \kappa^*(\tau) \tau & = - \beta_0^*(1) + \kappa^*(1), \\
	\beta_0^*(\tau) + \kappa^*(\tau) & = \beta_0^*(1) + \kappa^*(1).
    \end{align*}
    Combining these two equations gives the expression of $\beta_0^*(\tau), \kappa^*(\tau)$ in terms of $\beta_0^*(1), \kappa^*(1)$ as in \cref{eq:asymp_tau_relation}, completing the proof.
\end{proof}








\subsection{Proof of \cref{prop:opt_transport}}

\begin{proof}[\textbf{Proof of \cref{prop:opt_transport}}]
We can prove a more general result by replacing $\cL_*^\mathrm{test}$ with $\mu$ and $\cL_*$ with $\nu := \Law(\max\{ \kappa^*, X \})$, where $X \sim \mu$ and $\mu$ is any probability measure with atomless (continuous) CDF $F_\mu$. As a special case, in \cref{prop:opt_transport} we consider $\mu$ as a mixture of two Gaussian distributions, and the cost function $c(x, y) = (x - y)^2$. 

We now prove the general statement. Note the CDF of $\nu$ has the form
\begin{equation*}
    F_\nu(t) :=  
    \begin{cases} 
    F_\mu(t) , & \ \text{if} \ t < \kappa^*, \\
    1, & \ \text{if} \ t \ge \kappa^*.
    \end{cases}
\end{equation*}
According to the optimal transport theory \cite[Theorem 2.5]{santambrogio2015optimal}, the unique (also monotone) optimal transport map from $\mu$ to $\nu$ is given by $\mathtt{T}^* := F_{\nu}^{-} \circ F_{\mu}$, where $F_{\nu}^{-}$ is the quantile function of $\nu$:
\begin{equation*}
    F_{\nu}^{-}(x) = \inf\left\{ t \in \R: F_{\nu}(t) \ge x \right\}
    = 
     \begin{cases} 
    F_\mu^{-1}(x) , & \ \text{if} \ x < F_\mu(\kappa^*), \\
    \kappa^*, & \ \text{if} \ x \ge F_\mu(\kappa^*).
    \end{cases}
\end{equation*}
Then we have $\mathtt{T}^*(x) := F_{\nu}^{-} \bigl( F_{\mu}(x) \bigr) = \max\{ \kappa^*, x \}$, which concludes the proof.
\end{proof}





\section{Logit distribution for non-separable data: Proofs for \cref{sec:logit_logistic}}
\label{append_sec:nonsep}

\subsection{Proof of \cref{thm:logistic_main}}

Throughout this section, we assume the loss function $\ell: \R \to \R_{\ge 0}$ is non-increasing, strictly convex, and twice differentiable. Based on these assumptions, we establish the following properties of $\ell$.
\begin{lem}\label{lem:ell}
    Let $\ell \in C^1(\R)$ be a nonnegative, non-increasing, and strictly convex function. Then
    \begin{enumerate}[label=(\alph*)]
        \item $\ell$ is strictly decreasing.
        \item $\ell(-\infty) = +\infty$ and $\ell(+\infty) = \underline{\ell}$ for some $\underline{\ell} \in [0, +\infty)$. 
    \end{enumerate}
\end{lem}
\begin{proof}
    Notice that $\ell'(u) \le 0$ (by non-increasing) and $\ell'(u)$ is strictly increasing (by strict convexity), which implies that $\ell'(u) < 0$ for all $u \in \R$ and hence deduces part (a). For part (b), the limits $\lim_{u \to \pm \infty} \ell(u)$ are well-defined, and $\ell(+\infty) = \underline{\ell}$ for some $\underline{\ell} \in [0, +\infty)$ since $\ell$ is monotone and bounded from below. It remains to show $\ell(-\infty) = +\infty$.

    Assume $\ell(-\infty) = \overline{\ell} < \infty$ by contradiction. By convexity, we have $\ell(u) \le \frac12( \ell(2u) + \ell(0) )$ for any $u \in \R$. Taking $u \to -\infty$ on both sides yields $\overline{\ell} \le \frac{1}{2}(\overline{\ell} + \ell(0))$, hence $\overline{\ell} \le \ell(0)$, which contradicts the fact that $\ell$ is strictly decreasing. Therefore, we must have $\ell(-\infty) = +\infty$.
\end{proof}
Without loss of generality, assume $\underline{\ell} := \ell(+\infty) = 0$. Otherwise, we can just consider $\ell - \underline{\ell}$ instead of $\ell$. In addition, we also assume $\ell$ is pseudo-Lipschitz, i.e., there exists a constant $L > 0$ such that, for all $x, y \in \R$,
\begin{equation*}
    \abs{ \ell(x) - \ell(y) } \le L \left( 1 + \abs{ x } + \abs{ y } \right) \abs{ x - y }.
\end{equation*}
% This is true for logistic loss, squared loss, but not for exponential loss...}
For ease of exposition, we assume $\tau = 1$, as it is not fundamentally different from the case of arbitrary $\tau > 0$. In \cref{subsubsec:under_final}, we will discuss how to extend our proof to general $\tau > 0$.

Recall the original unconstrained empirical risk minimization (ERM) problem \cref{eq:logistic}:
\begin{equation}
\label{eq:ERM-0}
    M_n := 
    \min_{\bbeta \in \R^d, \, \beta_0 \in \R} \hat R_n(\bbeta, \beta_0)
    :=
    \min_{\bbeta \in \R^d, \, \beta_0 \in \R}  \frac1n \sum_{i=1}^n \ell\bigl( 
        y_i(\< \xx_i, \bbeta \> +  \beta_0 )
     \bigr).
\end{equation}

\noindent
We first provide an outline for the proof of \cref{thm:logistic_main}, which involves several intermediate steps of simplifying the random optimization problem $M_n$.
\begin{equation*}
\begin{aligned}
    M_{n}
    \, \xRightarrow[\text{\cref{lem:ERM_bound_beta}}]{\textbf{Step 1}} \,
    M_n(\bTheta_{\vbeta}, \bXi_{\bu})
    \, \xRightarrow[\text{\cref{lem:ERM_CGMT}}]{\textbf{Step 2}} \, 
    M_n^{(1)}(\bTheta_{\vbeta}, \bXi_{\bu})
    \, \Rightarrow \,
    M_n^{(2)}(\bTheta_{c}, \bXi_{\bu}) 
    \\
    \, \xRightarrow[\text{\cref{lem:M2-3}}]{\textbf{Step 3}} \, 
    M_n^{(3)}(\bTheta_{c}, \bXi_{\bu})
    \, \Rightarrow \,
    M_n^{(3)}(\bTheta_{c}) 
    \, \xRightarrow[\text{\cref{lem:M3-star}}]{\textbf{Step 4}} \, 
    M^*(\bTheta_{c})
    \, \Rightarrow \,
    M^*.
\end{aligned}
\left.
\vphantom{\begin{matrix} \dfrac12 \\ \dfrac12 \end{matrix}}
\right\} \text{\scriptsize{\cref{thm:ERM_conv}}}
\end{equation*}

\paragraph{Step 1: Boundedness of $\vbeta$ and $\beta_0$ (from $M_{n}$ to $M_n(\bTheta_{\vbeta}, \bXi_{\bu})$)}
Notice that by introducing the auxiliary variable $\bu = (u_1, \ldots, u_n)^\top \in \R^n$ and Lagrangian multiplier $\bv = (v_1, \ldots, v_n)^\top \in \R^n$, we can rewrite \cref{eq:ERM-0} as a minimax problem
\begin{align*}
        M_n & = \min_{ \substack{ \bbeta \in \R^d, \, \beta_0 \in \R \\  \bu \in \R^n } }
        \max_{ \bv \in \R^n }
        \left\{
        \frac1n \sum_{i=1}^n \ell( u_i )
         + \frac{1}{n} \sum_{i=1}^n v_i \bigl(  y_i(\< \xx_i, \bbeta \> +  \beta_0 ) - u_i\bigr)
         \right\} 
         \\
         & = \min_{ \substack{ \bbeta \in \R^d, \, \beta_0 \in \R \\  \bu \in \R^n } }
         \max_{ \bv \in \R^n }
         \left\{
         \frac1n \sum_{i=1}^n \ell( u_i )
          + \frac{1}{n} \sum_{i=1}^n v_i (  \< \bmu, \vbeta \> +  \< \zz_i, \vbeta \> + y_i \beta_0 - u_i )
          \right\},
\end{align*}
where in the second line, we reformulate $\xx_i = y_i(\vmu + \zz_i)$, $\zz_i \sim \normal(\bzero, \bI_d)$, $y_i \indep \zz_i$. For any closed subsets $\bTheta_{\vbeta} \subset \R^{d} \times \R$, $\bXi_{\bu} \subset \R^{n}$, we also define the quantity $M_n(\bTheta_{\vbeta}, \bXi_{\bu})$, which can be viewed as the constrained version of ERM problem $M_n$.
\begin{equation}
    \label{eq:Mn}
    \begin{aligned}
        M_n(\bTheta_{\vbeta}, \bXi_{\bu})
        :\! & = \min_{ \substack{ (\bbeta , \beta_0) \in \bTheta_{\vbeta} \\  \bu \in \bXi_{\bu} } }
    \max_{ \bv \in \R^n }
    \left\{
    \frac1n \sum_{i=1}^n \ell( u_i )
     + \frac{1}{n} \sum_{i=1}^n v_i (  \< \bmu, \vbeta \> +  \< \zz_i, \vbeta \> + y_i \beta_0 - u_i )
     \right\} \\
     & = \min_{ \substack{ (\bbeta , \beta_0) \in \bTheta_{\vbeta} \\  \bu \in \bXi_{\bu} } }
     \max_{ \bv \in \R^n }
     \left\{
     \frac1n \sum_{i=1}^n \ell( u_i )
      + \frac1n \bv^\top \bone \< \bmu, \vbeta \>
      + \frac1n \bv^\top \ZZ \vbeta + \frac1n \beta_0 \bv^\top \yy - \frac1n \bv^\top \bu
      \right\},
    \end{aligned}
\end{equation}
where $\ZZ = (\zz_1, \ldots, \zz_n)^\top \in \R^{n \times d}$. Let $(\hat\vbeta_n, \hat\beta_{0, n})$ be the unique minimizer of \cref{eq:ERM-0}. The following lemma implies that $\hat\vbeta_n$ and $\hat\beta_{0, n}$ are bounded with high probability, which enables us to work with $M_n(\bTheta_{\vbeta}, \bXi_{\bu})$ instead of $M_n$ for some compact sets $\bTheta_{\vbeta}$ and $\bXi_{\bu}$.

\begin{lem}[Boundedness of $\bbeta$ and $\beta_0$] \label{lem:ERM_bound_beta}
    In the non-separable regime $\delta > \delta^*(0)$, there exists some constants $C_{\bbeta}, C_{\beta_0}, C_{\bu} \in (0, \infty)$, such that $M_n = M_n(\bTheta_{\vbeta}, \bXi_{\bu})$ with high probability, where
    \begin{equation*}
    \bTheta_{\vbeta} = \{ (\vbeta, \beta_0) \in \R^d \times \R:  \norm{\vbeta}_2 \le C_{\vbeta},  \abs{\beta_0} \le C_{\beta_0}  \},
    \qquad
    \bXi_{\bu} = \{ \bu \in \R^n : \norm{\bu}_2 \le C_{\bu} \sqrt{n} \}.
    \end{equation*}
\end{lem}
\noindent
See \cref{subsubsec:under_step1} for the proof.

\paragraph{Step 2: Reduction via Gaussian comparison (from $M_n(\bTheta_{\vbeta}, \bXi_{\bu})$ to $M_n^{(1)}(\bTheta_{\vbeta}, \bXi_{\bu})$)}
The objective function of $M_n(\bTheta_{\vbeta}, \bXi_{\bu})$ in \cref{eq:Mn} is a bilinear form of the Gaussian random matrix $\ZZ$. To simplify the bilinear term, we will use the convex Gaussian minimax theorem (CGMT), i.e., Gordon's comparison inequality \cite{gordon1985some, thrampoulidis2015regularized}. To do so, we introduce another quantity:
\begin{equation*}
    \begin{aligned}
        M_n^{(1)}(\bTheta_{\vbeta}, \bXi_{\bu})
        : = \smash {\min_{ \substack{ (\bbeta , \beta_0) \in \bTheta_{\vbeta} \\  \bu \in \bXi_{\bu} } } }
        \max_{ \bv \in \R^n }
        \, \Biggl\{
        \frac1n \sum_{i=1}^n \ell( u_i )
         + \frac1n \bv^\top \bone \< \bmu, \vbeta \>
         & + \frac1n \norm{\bv}_2 \hh^\top \vbeta + \frac1n \norm{\vbeta}_2 \vg^\top \bv 
         \\
         & + \frac1n \beta_0 \bv^\top \yy - \frac1n \bv^\top \bu
         \Biggr\},
    \end{aligned}
\end{equation*}
where $\hh \sim \normal(\bzero, \bI_{d})$, $\vg \sim \normal(\bzero, \bI_{n})$ are independent Gaussian vectors. However, the classical CGMT cannot be directly applied to $M_n^{(1)}(\bTheta_{\vbeta}, \bXi_{\bu})$ since $\bv$ is maximized over an unbounded set. To this end, we proved the following version of CGMT, which connects $M_n^{(1)}(\bTheta_{\vbeta}, \bXi_{\bu})$ with $M_n(\bTheta_{\vbeta}, \bXi_{\bu})$.

\begin{lem}[CGMT, unbounded for maximum] \label{lem:ERM_CGMT}
For any compact sets $\bTheta_{\vbeta}$ and $\bXi_{\bu}$ (not necessarily convex) and $t \in \R$, we have
\begin{equation}\label{eq:unbounded_CGMT_1}
    \P \, \Bigl( M_n(\bTheta_{\vbeta}, \bXi_{\bu}) \le t \Bigr) \le 2 \, \P \,\Bigl( M_n^{(1)}(\bTheta_{\vbeta}, \bXi_{\bu}) \le t \Bigr).
\end{equation}
Additionally, if $\bTheta_{\vbeta}$ and $\bXi_{\bu}$ are convex, then
\begin{equation}\label{eq:unbounded_CGMT_2}
    \P \, \Bigl( M_n(\bTheta_{\vbeta}, \bXi_{\bu}) \ge t \Bigr) \le 2 \, \P \,\Bigl( M_n^{(1)}(\bTheta_{\vbeta}, \bXi_{\bu}) \ge t \Bigr).
\end{equation}
\end{lem}
\noindent
See \cref{subsubsec:under_step2} for the proof.




\paragraph{Reparametrization in low dimensions (from $M_n^{(1)}(\bTheta_{\vbeta}, \bXi_{\bu})$ to $M_n^{(2)}(\bTheta_{c}, \bXi_{\bu})$)}
To simplify $M^{(1)}_n(\bTheta_{\vbeta}, \bXi_{\bu})$, we consider the following change of variables
\begin{equation}\label{eq:change_of_var}
    \rho := \cos(\vmu, \vbeta) := 
    \begin{cases} 
    \, \displaystyle
    \left\< \frac{\vmu}{\norm{\vmu}_2}, \frac{\vbeta}{\norm{\vbeta}_2} \right\>, 
                  & \ \text{if} \ \vbeta \not= \bzero, \\
    \  0,         & \ \text{if} \ \vbeta = \bzero, 
    \end{cases}
    \qquad
    R := \norm{\vbeta}_2.
    % \qquad
    % \gamma := \norm{\bv}_2.
\end{equation}
Now, for any closed subset $\bTheta_{c} \subset [-1, 1] \times \R_{\ge 0} \times \R$, we define the quantity $M_n^{(2)}(\bTheta_{c}, \bXi_{\bu})$ by 
\begin{equation*}
    \begin{aligned}
        M_n^{(2)}(\bTheta_{\vbeta}, \bXi_{\bu})
        : = \smash{ \min_{ \substack{ (\bbeta , \beta_0) \in \R^d \times \R: 
        \\ (\cos(\vmu, \vbeta), \norm{\vbeta}_2, \beta_0) \in \bTheta_{c} \\  \bu \in \bXi_{\bu} } }
        }
        \max_{ \bv \in \R^n }
        \, \Biggl\{
        \frac1n \sum_{i=1}^n \ell( u_i )
         & + \frac1n \bv^\top \bone \< \bmu, \vbeta \>
         + \frac1n \norm{\bv}_2 \hh^\top \vbeta  
         \\
         & + \frac1n \norm{\vbeta}_2 \vg^\top \bv + \frac1n \beta_0 \bv^\top \yy - \frac1n \bv^\top \bu
         \Biggr\}.
    \end{aligned}
\end{equation*}
Therefore, $M_n^{(2)}(\bTheta_{c}, \bXi_{\bu})$ can be viewed as reparametrization of $M_n^{(1)}(\bTheta_{\vbeta}, \bXi_{\bu})$ when $\bTheta_{\vbeta} \subset \R^{d} \times \R$ takes the form
\begin{equation*}
        \bTheta_{\vbeta} = \left\{ 
        (\vbeta, \beta_0) \in \R^{d} \times \R:
        \bigl( \cos(\vmu, \vbeta), \norm{\vbeta}_2, \beta_0 \bigr) \in \bTheta_{c}
        \right\}.
\end{equation*}
Then we can simplify $M_n^{(2)}(\bTheta_{\vbeta}, \bXi_{\bu})$ as follows:
\begin{align}
        & M_n^{(2)}(\bTheta_{c}, \bXi_{\bu}) 
        \notag
        \\
        \overset{\mathmakebox[0pt][c]{\text{(i)}}}{=} {} & 
        \min_{ \substack{ (\rho, R, \beta_0) \in \bTheta_{c} \\ \bu \in \bXi_{\bu} } } \min_{ \substack{ \norm{\vbeta}_2 = R \\ \cos(\vmu, \vbeta) = \rho } }
        \max_{\gamma \ge 0} \max_{\norm{\bv_0}_2 = 1}
        \left\{ \frac1n \sum_{i=1}^n \ell( u_i ) +
        \frac{\gamma}{n} \bv_0^\top (\rho\norm{\vmu}_2 R \bone + R\vg + \beta_0 \yy  - \bu )
        + \frac{\gamma}{n} \hh^\top\vbeta 
        \right\} 
        \notag
        \\
        \overset{\mathmakebox[0pt][c]{\text{(ii)}}}{=} {} &  
        \min_{ \substack{ (\rho, R, \beta_0) \in \bTheta_{c} \\ \bu \in \bXi_{\bu} } } \min_{ \substack{ \norm{\vbeta}_2 = R \\ \cos(\vmu, \vbeta) = \rho } }
        \max_{\gamma \ge 0} 
        \left\{ \frac1n \sum_{i=1}^n \ell( u_i ) +
        \frac{\gamma}{n} \bigl\| \rho\norm{\vmu}_2 R \bone + R\vg + \beta_0 \yy  - \bu \bigr\|_2
        + \frac{\gamma}{n} \hh^\top\vbeta 
        \right\} 
        \notag
        \\
        \overset{\mathmakebox[0pt][c]{\text{(iii)}}}{=} {} & 
        \min_{ \substack{ (\rho, R, \beta_0) \in \bTheta_{c} \\ \bu \in \bXi_{\bu} } } 
        \max_{\gamma \ge 0} 
        \, \Biggl\{ 
            \frac1n \sum_{i=1}^n \ell( u_i )
        + \frac{\gamma}{n} \bigl\| \rho\norm{\vmu}_2 R \bone + R\vg + \beta_0 \yy  - \bu \bigr\|_2
        + \frac{\gamma}{n} \min_{ \substack{ \norm{\vbeta}_2 = R \\ \cos(\vmu, \vbeta) = \rho } } \hh^\top\vbeta  
        \Biggr\} 
        \notag
        \\
        \overset{\mathmakebox[0pt][c]{\text{(iv)}}}{=} {} & 
        \min_{ \substack{ (\rho, R, \beta_0) \in \bTheta_{c} \\ \bu \in \bXi_{\bu} } } 
        \max_{\gamma \ge 0} 
        \, \Biggl\{ 
            \frac1n \sum_{i=1}^n \ell( u_i )
        + \frac{\gamma}{n} \bigl\| \rho\norm{\vmu}_2 R \bone + R\vg + \beta_0 \yy  - \bu \bigr\|_2
        \notag
        \\
        & 
        \phantom{
        \min_{ \substack{ (\rho, R, \beta_0) \in \bTheta_{c} \\ \bu \in \bXi_{\bu} } } 
        \max_{\gamma \ge 0} 
        \, \Biggl\{ 
            \frac1n \sum_{i=1}^n \ell( u_i )
        }
        \, + \frac{\gamma}{n} R\biggl( \rho\frac{\hh^\top \vmu}{\norm{\vmu}_2} - \sqrt{1 - \rho^2} \| \bP_{\bmu}^\perp \hh \|_2 \biggr)  
        \Biggr\},
        \label{eq:Mn(2)}
\end{align}
where in (i) we apply the change of variables \cref{eq:change_of_var} and optimize $\bv$ by its length $\gamma$ and direction $\bv_0$ separately, (ii) follows from Cauchy--Schwarz inequality, (iii) is from the linearity of objective function in $\gamma$, and (iv) is based on direct calculation by decomposing $\vbeta$:
\begin{equation*}
    \min_{ \substack{ \vbeta \in \R^d: \norm{\vbeta}_2 = 1 \\ \cos(\vmu, \vbeta) = \rho } } \hh^\top\vbeta
    = \min_{ \vtheta \in \R^d: \substack{ \norm{\vtheta}_2 = 1 \\ \< \bmu, \vtheta \> = 0 } } \hh^\top 
    \biggl( \rho \frac{\bmu}{\norm{\vmu}_2} + \sqrt{1 - \rho^2} \vtheta \biggr)
    =  
    \rho \frac{\hh^\top \bmu}{\norm{\vmu}_2} - \sqrt{1 - \rho^2} \| \bP_{\bmu}^\perp \hh \|_2,
\end{equation*}
where $\bP_{\vmu}^\perp := \bI_d - \vmu \vmu^\top / \norm{\vmu}_2^2$.




\paragraph{Step 3: Convergence in variational forms (from $M_n^{(2)}(\bTheta_{c}, \bXi_{\bu})$ to $M_n^{(3)}(\bTheta_{c}, \bXi_{\bu})$)}
To proceed from \cref{eq:Mn(2)}, we adopt the following trick from \cite{montanari2023generalizationerrormaxmarginlinear}, where $\bu$ could be viewed as a functional of the empirical measure given by $\vg = (g_1, \ldots, g_n)^\top$ and $\yy = (y_1, \ldots, y_n)^\top$. Formally, let $\Q_n$ be the empirical distribution of the coordinates of $(\vg, \yy)$, i.e., the probability measure on $\R^2$ defined by
\begin{equation*}
    \Q_n := \frac{1}{n}\sum_{i=1}^n \delta_{(g_i, y_i)}.
\end{equation*}
Let $\cL^2(\Q_n) := \cL^2(\Q_n, \R^2)$ be the space of functions $U: \R^2 \to \R$, $(g, y) \mapsto U(g, y)$ that are square integrable with respect to $\Q_n$. Notice that the $n$ points that form $\Q_n$ are almost surely distinct, and therefore we can identify this space with the space of vectors $\bu \in \R^n$. We also define the two random variables in the same space by $G(g, y) = g$, $Y(g, y) = y$. Denote $\E_{\Q_n}$, $\norm{\,\cdot\,}_{\Q_n}$ the integral and norm with respect to $\Q_n$ in $\cL^2(\Q_n)$, i.e.,
\begin{equation*}
    \E_{\Q_n}[U] := \int_{\R^2} U(g,y) \, \d \Q_n(g, y)
    = \frac{1}{n} \sum_{i=1}^n U(g_i, y_i)
    ,
    \qquad
    \norm{U}_{\Q_n} := (\E_{\Q_n}[U^2])^{1/2}.
\end{equation*}
Let $\Xi_{u} \subseteq \cL^2(\Q_n)$ be the corresponding subset identified by $\bXi_{\bu} \subseteq \R^n$, that is,
\begin{equation*}
    \Xi_{u} := \left\{ U \in \cL^2(\Q_n) : \bu := \left( U(g_1, y_1), \ldots, U(g_n, y_n) \right)^\top \in \bXi_{\bu}  \right\}. 
\end{equation*}
Then with these definitions, we can rewrite the expression of $M_n^{(2)}(\bTheta_{c}, \bXi_{\bu})$ as
\begin{align*}
        M_n^{(2)}(\bTheta_{c}, \bXi_{\bu}) 
        & = \min_{ \substack{ (\rho, R, \beta_0) \in \bTheta_{c} \\  U \in \Xi_{u} } } 
        \max_{\gamma \ge 0}
        \, \Biggl\{ 
                \E_{\Q_n}[\ell(U)]
        + \frac{\gamma}{\sqrt{n}} \bigl\| \rho\norm{\vmu}_2 R + R G + \beta_0 Y - U \bigr\|_{\Q_n}
            \\
        & \phantom{.} \phantom{ 
            = \min_{ \substack{ (\rho, R, \beta_0) \in \bTheta_{c} \\ U \in \Xi_{u} } } 
        \max_{\gamma \ge 0}
        \, \Biggl\{ 
                % \E_{\Q_n}[\ell(U)]
        }
        + \frac{\gamma}{n} R\biggl( \rho\frac{\hh^\top \vmu}{\norm{\vmu}_2} - \sqrt{1 - \rho^2} \| \bP_{\bmu}^\perp \hh \|_2 \biggr)  
            \Biggr\}
        \\
        &   
        = \min_{ (\rho, R, \beta_0) \in \bTheta_{c} }
        \min_{ U \in \Xi_{u}  \cap   \mathcal{N}_n }
        \E_{\Q_n}[\ell(U)],
\end{align*}
where we define the (stochastic) subset $\mathcal{N}_n = \mathcal{N}_n(\rho, R, \beta_0)$ by
\begin{equation}
\label{eq:set_N_n}
    \mathcal{N}_n := \left\{
        U \in \cL^2(\Q_n): 
        \bigl\| \rho\norm{\vmu}_2 R + R G + \beta_0 Y - U \bigr\|_{\Q_n}
        \le \frac{R}{\sqrt{n}} \biggl( \sqrt{1 - \rho^2}  \| \bP_{\bmu}^\perp \hh \|_2  -  \rho\frac{\hh^\top \vmu}{\norm{\vmu}_2}  \biggr)
     \right\}.
\end{equation}
It can be shown that as $n, d \to \infty$,
\begin{equation*}
    \frac{R}{\sqrt{n}} \biggl( \sqrt{1 - \rho^2}  \| \bP_{\bmu}^\perp \hh \|_2  -  \rho\frac{\hh^\top \vmu}{\norm{\vmu}_2}  \biggr)
    \conp \frac{R \sqrt{1 - \rho^2}}{\sqrt{\delta}}.
\end{equation*}
This convergence then motivates us to define another quantity
\begin{equation}
    \label{eq:Mn(3)}
        M_n^{(3)}(\bTheta_{c}, \bXi_{\bu}) 
        :=  
        \min_{ (\rho, R, \beta_0) \in \bTheta_{c} } 
        \min_{ U \in \Xi_{u}  \cap   \mathcal{N}^\delta_n }
        \E_{\Q_n}[\ell(U)],
\end{equation}
where the subset $\mathcal{N}^\delta_n = \mathcal{N}^\delta_n(\rho, R, \beta_0)$ is given by
\begin{equation}
\label{eq:set_N_n_delta}
    \mathcal{N}_n^\delta := \left\{
        U \in \cL^2(\Q_n): 
        \bigl\| \rho\norm{\vmu}_2 R + R G + \beta_0 Y - U \bigr\|_{\Q_n}
        \le \frac{R \sqrt{1 - \rho^2}}{\sqrt{\delta}}
     \right\}.
\end{equation}
The following lemma shows that $M_n^{(2)}$ and $M_n^{(3)}$ are close to each other:
\begin{lem}\label{lem:M2-3}
    For any compact sets $\bTheta_{c} \subset [-1, 1] \times \R_{\ge 0} \times \R$ and $\bXi_{\bu} \subset \R^{n}$ (not necessarily convex), as $n \to \infty$, we have
    \begin{equation*}
        \abs{ M_n^{(2)}(\bTheta_{c}, \bXi_{\bu}) - M_n^{(3)}(\bTheta_{c}, \bXi_{\bu}) } \conp 0.
    \end{equation*}
\end{lem}
\noindent
See \cref{subsubsec:under_step3} for the proof.




\paragraph{Step 4: Asymptotic characterization (from $M_n^{(3)}(\bTheta_{c}, \bXi_{\bu})$, $M_n^{(3)}(\bTheta_{c})$ to $M^*(\bTheta_{c})$, $M^*$)}
For any closed subsets $\bTheta_{c} \subset [-1, 1] \times \R_{\ge 0} \times \R$, we define the quantity $M_n^{(3)}(\bTheta_{c})$ by 
\begin{equation*}
    M_n^{(3)}(\bTheta_{c}) 
        :=  
        \min_{ (\rho, R, \beta_0) \in \bTheta_{c} } 
        \min_{ U \in \mathcal{N}^\delta_n }
        \E_{\Q_n}[\ell(U)].
\end{equation*}
Compared with \cref{eq:Mn(3)}, clearly $M_n^{(3)}(\bTheta_{c}, \bXi_{\bu}) = M_n^{(3)}(\bTheta_{c})$ when $\bXi_{\bu}$ is large enough. To analyze $M_n^{(3)}(\bTheta_{c})$, we consider the change of variable\footnote{
We will show in \cref{lem:M_star_var} later that the minimizer of $M_n^{(3)}(\bTheta_{c})$ must satisfy $R \sqrt{1 - \rho^2} > 0$, hence the change of variable $\xi$ can be well-defined.
}
\begin{equation*}
    \xi := - \frac{\rho\norm{\vmu}_2 R + RG + \beta_0 Y - U}{R\sqrt{1 - \rho^2}},
\end{equation*}
Then we have
\begin{equation*}
        M_n^{(3)}(\bTheta_{c})
    = \min_{ \substack{ (\rho, R, \beta_0) \in \bTheta_{c} \\ \xi \in \cL^2(\Q_n), \norm{\xi}_{\Q_n} \le 1/\sqrt{\delta} } } 
    \E_{\Q_n} \left[ \ell \bigl( \rho\norm{\vmu}_2 R + RG + \beta_0 Y + R\sqrt{1 - \rho^2} \xi \bigr) \right].
\end{equation*}
Denote $\Q_\infty := \P$ the population measure of $(G, Y)$ (so that $(G, Y) \sim \normal(0, 1) \times P_y$ under $\Q = \Q_\infty$, and we have $\E_{\Q_\infty} := \E$, $\norm{U}_{\Q_\infty} := (\E[U^2])^{1/2}$). Then we also define the asymptotic counterpart of $M_n^{(3)}(\bTheta_{c})$ by replacing $\Q_n$ with $\Q_\infty$:
\begin{equation*}
    M^{*}(\bTheta_{c})
    := \min_{ \substack{ (\rho, R, \beta_0) \in \bTheta_{c} \\ \xi \in \cL^2(\Q_\infty), \norm{\xi}_{\Q_\infty} \le 1/\sqrt{\delta} } } 
    \E \left[ \ell \bigl( \rho\norm{\vmu}_2 R + RG + \beta_0 Y + R\sqrt{1 - \rho^2} \xi \bigr) \right].
\end{equation*}
The following lemma shows that $M_n^{(3)}(\bTheta_{c})$ converges to the deterministic quantity $M^{*}(\bTheta_{c})$:

\begin{lem}\label{lem:M3-star}
    For any compact subset $\bTheta_{c} \subset [-1, 1] \times \R_{\ge 0} \times \R$, as $n \to \infty$, we have
    \begin{equation*}
        M_n^{(3)}(\bTheta_{c}) \conp M^{*}(\bTheta_{c}).
    \end{equation*}
\end{lem}
\noindent
See \cref{subsubsec:under_step4} for the proof.


\vspace{0.5\baselineskip}
Finally, combining \cref{lem:ERM_CGMT}---\ref{lem:M3-star}, we obtain the following theorem.
\begin{thm}
\label{thm:ERM_conv}
    Consider any compact sets $\bTheta_{\vbeta}$ and $\bXi_{\bu}$ such that $\bTheta_{\vbeta}$ has the form of
    \begin{equation}\label{eq:Theta_link}
        \bTheta_{\vbeta} = \left\{ 
        (\vbeta, \beta_0) \in \R^{d} \times \R:
        \bigl( \cos(\vmu, \vbeta), \norm{\vbeta}_2, \beta_0 \bigr) \in \bTheta_{c}
        \right\}
    \end{equation}
    for some compact domain $\bTheta_{c} \subset [-1, 1] \times \R_{\ge 0} \times \R$ of $(\rho, R, \beta_0)$. Assume $\bXi_{\bu}$ is large enough. Then, for any $\veps > 0$, as $n \to \infty$ , we have
    \begin{equation*}
        \P \, \bigl( M_n(\bTheta_{\vbeta}, \bXi_{\bu}) \le M^*(\bTheta_{c}) - \veps \bigr) \to 0.
    \end{equation*}
    Further, if both $\bTheta_{\vbeta}$ and $\bXi_{\bu}$ are convex, then
    \begin{equation*}
        M_n(\bTheta_{\vbeta}, \bXi_{\bu}) \conp M^*(\bTheta_{c}).
    \end{equation*}
\end{thm}
\begin{proof}
According to \cref{lem:M2-3} and \ref{lem:M3-star}, we have $M_n^{(2)}(\bTheta_{c}, \bXi_{\bu}) \conp M^*(\bTheta_{c})$ for any compact sets $\bTheta_{c} \subset [-1, 1] \times \R_{\ge 0} \times \R$ and $\bXi_{\bu} \subset \R^{n}$ large enough such that $\Xi_{u} \subset \mathcal{N}^\delta_n$. When $\bTheta_{\vbeta}$ takes the form \cref{eq:Theta_link}, by CGMT \cref{lem:ERM_CGMT}, for any $\veps > 0$ we have
\begin{align*}
    \P \, \Bigl( M_n(\bTheta_{\vbeta}, \bXi_{\bu}) \le M^*(\bTheta_{c}) - \veps \Bigr) 
    & \le 2 \, \P \,\Bigl( M_n^{(1)}(\bTheta_{\vbeta}, \bXi_{\bu}) \le M^*(\bTheta_{c}) - \veps \Bigr) \\
    & = 2 \, \P \,\Bigl( M_n^{(2)}( \mathmakebox[\widthof{$\bTheta_{\vbeta}$}][l]{ \bTheta_{c} }, \bXi_{\bu}) \le M^*(\bTheta_{c}) - \veps \Bigr) \xrightarrow{n \to \infty} 0.
\end{align*}
If both $\bTheta_{\vbeta}$ and $\bXi_{\bu}$ are also convex, then we can similarly show that
\begin{equation*}
    \P \, \Bigl( M_n(\bTheta_{\vbeta}, \bXi_{\bu}) \ge M^*(\bTheta_{c}) + \veps \Bigr) 
    \le 2 \, \P \,\Bigl( M_n^{(2)}( \mathmakebox[\widthof{$\bTheta_{\vbeta}$}][l]{ \bTheta_{c} }, \bXi_{\bu}) \ge M^*(\bTheta_{c}) + \veps \Bigr) \xrightarrow{n \to \infty} 0.
\end{equation*}
Combining these implies $M_n(\bTheta_{\vbeta}, \bXi_{\bu}) \conp M^*(\bTheta_{c})$, which concludes the proof.
\end{proof}

\paragraph{Parameter convergence}

Next, we define $M^* := M^*( [-1, 1] \times \R_{\ge 0} \times \R )$ to be the unconstrained optimization problem \cref{eq:logistic_variation}, i.e.,
\begin{equation*}
    M^{*}
    = \min_{ \substack{ \rho \in [-1, 1], R \ge 0, \beta_0 \in \R \\ \xi \in \cL^2(\Q_\infty), \norm{\xi}_{\Q_\infty} \le 1/\sqrt{\delta} } } 
    \E \left[ \ell \bigl( \rho\norm{\vmu}_2 R + RG + \beta_0 Y + R\sqrt{1 - \rho^2} \xi \bigr) \right].
\end{equation*}
An analysis of the Karush--Kuhn--Tucker (KKT) conditions shows that $M^*$ has the unique solution $(\rho^*, R^*, \beta_0^*, \xi^*)$, with $\rho^* \in (0, 1)$, $R^* \in (0, \infty)$, and $\beta_0^* \in (-\infty, \infty)$. Combined with \cref{thm:ERM_conv}, it implies $M_n \conp M^*$, which leads to the convergence of parameters:
\begin{lem}[Parameter convergence] 
\label{lem:ERM_param_conv}
As $n, d \to \infty$, we have $M_n \conp M^*$, which implies
    \begin{equation*}
            \| \hat\vbeta_n \|_2 \conp R^*,
            \qquad
            \hat\rho_n = \biggl\< \frac{\hat \vbeta_n}{\| \hat \vbeta_n \|_2}, \frac{\vmu}{\| \vmu \|_2} \biggr\> \conp \rho^*,
            \qquad
            \hat\beta_{0,n} \conp \beta_0^*.
    \end{equation*}
\end{lem}
\noindent
See \cref{append_subsubsec:ERM_param} for the proof.

\paragraph{ELD convergence} Finally, to establish the ELD convergence, we use a proof strategy similar to that in \cref{lem:over_logit_conv} by first defining the following measures
\begin{align*}
    \hat \cL_{n} := \frac1n \sum_{i=1}^n \delta_{y_i ( \< \xx_i, \hat\vbeta \> + \hat\beta_{0} ) }, 
    \quad
    \cL_* := \Law \, (U^*)
    = \Law \, \bigl( \rho^*\norm{\vmu}_2 R^* + R^* G + \beta_0^* Y + R^* \sqrt{1 - \rho^*{}^2} \xi^* \bigr).
\end{align*}
Let $\mathsf{B}_{W_2}(\varepsilon)$ ($\varepsilon > 0$) be the $\varepsilon$-$W_2$ ball at $\cL_*$, i.e.,
\begin{equation*}
    \mathsf{B}_{W_2}(\varepsilon) := \left\{ \bu \in \R^n:   W_2 \biggl( 
        \frac{1}{n}\sum_{i=1}^n \delta_{u_i}, \cL_*
     \biggr)  < \varepsilon \right\}.
\end{equation*}
Then by showing that
\begin{equation*}
    \lim_{n \to \infty} \P \left( M_n(\R^{d + 1}, \mathsf{B}_{W_2}^c(\varepsilon) ) > M_n \right) = 1,
\end{equation*}
we can prove the convergence of logit margins $W_2( \hat \cL_{n}, \cL_* ) \conp 0$, and hence the ELD convergence. The result in summarized in the following lemma.

\begin{lem}[ELD convergence]
\label{lem:ERM_logit_conv}
    As $n, d \to \infty$, we have $W_2( \hat \cL_{n}, \cL_* ) \conp 0$ and $W_2 ( \hat \nu_{n}, \nu_* ) \conp 0$.
\end{lem}
\noindent
See \cref{append_subsubsec:ERM_logit} for the proof.


\subsubsection{Step 1 --- Boundedness of $\vbeta$ and $\beta_0$: Proof of \cref{lem:ERM_bound_beta}}
\label{subsubsec:under_step1}

\begin{proof}[\textbf{Proof of \cref{lem:ERM_bound_beta}}]
    % If $\hat\vbeta = \bzero$, then we have $\hat R_n(\bzero, \beta_0) = (n_+/n) \ell(\beta_0) + (n_-/n) \ell(-\beta_0) \to \pi \ell(\beta_0) + (1- \pi) \ell(-\beta_0)$ as $n \to \infty$ by LLN. Note that $\ell(-\infty) = +\infty$ by \cref{lem:ell}, then $|\hat\beta_0|$ must be bounded with high probability.
    We first assume $\hat\vbeta \not= \bzero$. By \cref{thm:SVM_main}\ref{thm:SVM_main_mar}, if $\delta > \delta^*(0)$, there exists $k \in [n]$ and constant $\overline{\kappa} > 0$, such that
    \begin{equation}
        \label{eq:neg_k_logit}
        y_k \biggl(  \biggl\< \xx_k, \frac{\hat\bbeta}{\|\hat\bbeta\|_2} \biggr\> + \frac{\hat\beta_0}{\|\hat\bbeta\|_2} \biggr) \le - \overline{\kappa}
    \end{equation}
    holds with high probability. Therefore, we have
    \begin{equation*}
            \ell(0) 
               \overset{\mathmakebox[0pt][c]{\smash{\text{(i)}}}}{\ge}
                \frac1n \sum_{i=1}^n \ell\bigl( 
                y_i(\< \xx_i, \hat\bbeta \> +  \hat\beta_0 ) \bigr)
               \overset{\mathmakebox[0pt][c]{\smash{\text{(ii)}}}}{\ge}
                \frac1n \ell\bigl( 
                y_k(\< \xx_k, \hat\bbeta \> +  \hat\beta_0 ) \bigr)
               \overset{\mathmakebox[0pt][c]{\smash{\text{(iii)}}}}{\ge}
                \frac1n \ell( - \overline{\kappa} \|\hat\bbeta\|_2 ),
    \end{equation*}
    where in (i) we note that $\hat R_n(\bzero, 0) \ge \hat R_n(\hat\bbeta, \hat\beta_0) = M_n$, in (ii) we use $\ell \ge 0$, and in (iii) we use \eqref{eq:neg_k_logit}. Clearly the above inequalities also hold for $\hat\vbeta = \bzero$. Notice that $\frac1n \ell( - \overline{\kappa} \|\hat\bbeta\|_2 ) \to +\infty$ as $\|\hat\bbeta\|_2 \to \infty$, which contradicts $\ell(0) < +\infty$. Hence, it implies $\|\hat\bbeta\|_2$ is bounded with high probability.

    Meanwhile, let $j, k \in [n]$ be any two indices $y_j = +1$, $y_k = -1$. Then as $\hat\beta_0 \to \pm\infty$, we have
    \begin{equation*}
        \ell(0) \ge \frac1n \sum_{i=1}^n \ell\bigl( 
            y_i(\< \xx_i, \hat\bbeta \> +  \hat\beta_0 ) \bigr)
        \ge
        \frac1n \ell\bigl( \< \xx_j, \hat\bbeta \> +  \hat\beta_0 \bigr)
        +
        \frac1n \ell\bigl( - \< \xx_k, \hat\bbeta \> - \hat\beta_0 \bigr)
        \to +\infty,
    \end{equation*}
    which leads to a contradiction. So $|\hat\beta_0|$ is also bounded with high probability.

    Finally, in the minimax representation of $M_n$, the optimal $\bu$ must satisfy $u_i = y_i(\< \xx_i, \hat\bbeta \> +  \hat\beta_0 )$ for all $i \in [n]$. Therefore, according to the tail bound of Gaussian matrices \cite[Corollary 7.3.3]{vershynin2018high},
    \begin{align*}
            \norm{\bu}_2 & = \| \yy \odot (\XX \hat\bbeta + \hat\beta_0 \bone_n ) \|_2
            = \| \< \bmu, \hat\bbeta \> \bone_n + \ZZ \hat\bbeta + \hat\beta_0 \yy \|_2 \\
            & \le \sqrt{n} \| \bmu \|_2 \| \hat\bbeta \|_2 +
            \| \ZZ \|_{\mathrm{op}} \| \hat\bbeta \|_2 + \sqrt{n} | \hat\beta_0 | \\
            & \le \sqrt{n} \| \bmu \|_2 \| C_{\bbeta} + \bigl( \sqrt{n}(1 + o(1)) + \sqrt{d} \bigr) C_{\bbeta} 
            + \sqrt{n} C_{\beta_0} \\
            & \le \sqrt{n} C_{\bu}
    \end{align*}
    with high probability, where $C_{\bu} > 0$ is some constant. This completes the proof.
\end{proof}






\subsubsection{Step 2 --- Reduction via Gaussian comparison: Proof of \cref{lem:ERM_CGMT}}
\label{subsubsec:under_step2}


\begin{proof}[\textbf{Proof of \cref{lem:ERM_CGMT}}]
    For $m \in \mathbb{N}_+$, denote $K_m = \{ \bv \in \R^n: \norm{\bv}_2 \le m \}$, and define
    \begin{align*}
        M_n (\bTheta_{\vbeta}, \bXi_{\bu}; K_m)
        & :=  \!\min_{ \substack{ (\bbeta , \beta_0) \in \bTheta_{\vbeta} \\  \bu \in \bXi_{\bu} } } \max_{ \bv \in K_m } \left\{
        \frac1n \sum_{i=1}^n \ell( u_i )
      + \frac1n \bv^\top \bone \< \bmu, \vbeta \>
      + \frac1n \bv^\top \ZZ \vbeta + \frac1n \beta_0 \bv^\top \yy - \frac1n \bv^\top \bu \right\} \! , \!\! 
      \\
        M_n^{(1)} (\bTheta_{\vbeta}, \bXi_{\bu}; K_m) 
        & :=  \!\min_{ \substack{ (\bbeta , \beta_0) \in \bTheta_{\vbeta} \\  \bu \in \bXi_{\bu} } }
        \max_{ \bv \in K_m }
        \, \Biggl\{
        \frac1n \sum_{i=1}^n \ell( u_i )
        + \frac1n \bv^\top \bone \< \bmu, \vbeta \>
         + \frac1n \norm{\bv}_2 \hh^\top \vbeta + \frac1n \norm{\vbeta}_2 \vg^\top \bv 
         \\
         &
         \phantom{:=  \!\min_{ \substack{ (\bbeta , \beta_0) \in \bTheta_{\vbeta} \\  \bu \in \bXi_{\bu} } }
        \max_{ \bv \in K_m }
        \, \Biggl\{}
         + \frac1n \beta_0 \bv^\top \yy - \frac1n \bv^\top \bu
         \Biggr\}.
    \end{align*}
    We first show that
    \begin{equation*}
        \lim_{m \to \infty} M_n (\bTheta_{\vbeta}, \bXi_{\bu}; K_m) = M_n (\bTheta_{\vbeta}, \bXi_{\bu}).
    \end{equation*}
    To this end, note that for any fixed $(\bbeta , \beta_0, \bu)$, by Cauchy--Schwarz inequality we have
    \begin{align}
        & \max_{ \bv \in K_m } \left\{
        \frac1n \sum_{i=1}^n \ell( u_i )
        + \frac1n \bv^\top \bone \< \bmu, \vbeta \>
        + \frac1n \bv^\top \ZZ \vbeta + \frac1n \beta_0 \bv^\top \yy - \frac1n \bv^\top \bu \right\} 
        \notag \\
        = {} & \frac1n \sum_{i=1}^n \ell( u_i ) + \frac{m}{n} \bigl\|\bu - \< \bmu, \vbeta \> \bone - \ZZ \vbeta - \beta_0 \yy \bigr\|_2.
        \label{eq:ell_Mn_ineq}
    \end{align}
    Let $(\bbeta_*^{(m)}, \beta_{0, *}^{(m)}, \bu_*^{(m)})$ be the minimizer of $M_n (\bTheta_{\vbeta}, \bXi_{\bu}; K_m)$. Since $\ell \ge 0$, we know that
    \begin{align*}
        & \frac{m}{n} \norm{\bu_*^{(m)} - \< \bmu, \vbeta_*^{(m)} \> \bone - \ZZ \vbeta_*^{(m)} - \beta_{0, *}^{(m)} \yy}_2 \le \, M_n (\bTheta_{\vbeta}, \bXi_{\bu}; K_m) \le M_n (\bTheta_{\vbeta}, \bXi_{\bu}) \\
        \implies \  & \frac{ \mathmakebox[\widthof{$m$}][c]{1} }{n} \norm{\bu_*^{(m)} - \< \bmu, \vbeta_*^{(m)} \> \bone - \ZZ \vbeta_*^{(m)} - \beta_{0, *}^{(m)} \yy}_2 \le \, \frac{1}{m} M_n (\bTheta_{\vbeta}, \bXi_{\bu}).
    \end{align*}
    Let $\bu' := \< \bmu, \vbeta_*^{(m)} \> \bone + \ZZ \vbeta_*^{(m)} + \beta_{0, *}^{(m)} \yy$, then we have
    \begin{equation}\label{eq:u_diff_Mn}
        \frac{1}{n} \, \bigl\| \bu_*^{(m)} - \bu' \bigr\|_2 \le \, \frac{1}{m} M_n (\bTheta_{\vbeta}, \bXi_{\bu}),
    \end{equation}
    which implies that ($\bu_*^{(m)} = (u_{*, 1}^{(m)}, \ldots, u_{*, n}^{(m)})^\top$, $\bu' = (u'_1, \ldots, u'_n)^\top$)
    \begin{align*}
        M_n (\bTheta_{\vbeta}, \bXi_{\bu}) 
        & = \min_{ \substack{ (\bbeta , \beta_0) \in \bTheta_{\vbeta} \\  \bu \in \bXi_{\bu} } }
        \left\{
        \frac1n \sum_{i=1}^n \ell( u_i )  \,\bigg|\,
         \< \bmu, \vbeta \> +  \< \zz_i, \vbeta \> + y_i \beta_0 - u_i = 0, \forall\, i \in [n]
         \right\}
        \\
        & \le
        \frac1n \sum_{i=1}^n \ell( u'_i ) \le \frac1n \sum_{i=1}^n \ell( u_{*, i}^{(m)} ) + \frac{1}{n} \sum_{i=1}^{n} \left\vert \ell( u_{*, i}^{(m)} ) - \ell( u'_i ) \right\vert 
        \\
        & \overset{\mathmakebox[0pt][c]{\text{(i)}}}{\le} 
        \frac1n \sum_{i=1}^n \ell( u_{*, i}^{(m)} ) + \frac{C_L}{n} \bigl\| \bu_*^{(m)} - \bu' \bigr\|_1
        \\
        & \overset{\mathmakebox[0pt][c]{\text{(ii)}}}{\le} 
        \frac1n \sum_{i=1}^n \ell( u_{*, i}^{(m)} ) + O_m \left( \frac{1}{m} \right) 
        \overset{\mathmakebox[0pt][c]{\text{(iii)}}}{\le}
        M_n (\bTheta_{\vbeta}, \bXi_{\bu}; K_m) + O_m \left( \frac{1}{m} \right),
    \end{align*}
    where (i) follows from the pseudo-Lipschitzness of $\ell$, the compactness of $\bXi_{\bu}$, and $C_L > 0$ is some constant, (ii) follows from \cref{eq:u_diff_Mn}, while (iii) follows from \cref{eq:ell_Mn_ineq}.
    This proves that
    \begin{equation*}
        \lim_{m \to \infty} M_n (\bTheta_{\vbeta}, \bXi_{\bu}; K_m) = M_n (\bTheta_{\vbeta}, \bXi_{\bu}).
    \end{equation*}
    Similarly, one can show that
    \begin{equation*}
        \lim_{m \to \infty} M_n^{(1)} (\bTheta_{\vbeta}, \bXi_{\bu}; K_m) = M_n^{(1)} (\bTheta_{\vbeta}, \bXi_{\bu}).
    \end{equation*}
    Now for any fixed $m$, applying \cref{lem:CGMT}\ref{lem:CGMT(a)} yields that $\forall\, t \in \R$:
    \begin{equation*}
        \P \, \Bigl( M_n(\bTheta_{\vbeta}, \bXi_{\bu}; K_m) \le t \Bigr) 
        \le 2 \, \P \, \Bigl( M_n^{(1)}(\bTheta_{\vbeta}, \bXi_{\bu}; K_m) \le t \Bigr),
    \end{equation*}
    thus leading to \cref{eq:unbounded_CGMT_1} (by continuity and using the two limits above)
    \begin{align*}
        & \P \left( M_n(\bTheta_{\vbeta}, \bXi_{\bu}) \le t \right) =  
        \lim_{m \to \infty} \P \left( M_n(\bTheta_{\vbeta}, \bXi_{\bu}; K_m) \le t \right) \\
        \le {} & 2 \lim_{m \to \infty} \P \left( M_n^{(1)}(\bTheta_{\vbeta}, \bXi_{\bu}; K_m) \le t \right)
        =  2 \, \P \left( M_n^{(1)}(\bTheta_{\vbeta}, \bXi_{\bu}) \le t \right).
    \end{align*}
    Further, if $\bTheta_{\vbeta}$ and $\bXi_{\bu}$ are convex, \Cref{lem:CGMT}(b) implies that
    \begin{equation*}
        \P \, \Bigl( M_n(\bTheta_{\vbeta}, \bXi_{\bu}; K_m) \ge t \Bigr) 
        \le 2 \, \P \, \Bigl( M_n^{(1)}(\bTheta_{\vbeta}, \bXi_{\bu}; K_m) \ge t \Bigr).
    \end{equation*}
    Sending $m \to \infty$ similarly proves the other inequality \cref{eq:unbounded_CGMT_2}.
\end{proof}




\begin{comment}
{Kangjie: Under our additional assumption, this lemma is not needed.}
It turns out $M_n^{(2)}(\bTheta_{c}, \bXi_{\bu}) \approx M_n^{(3)}(\bTheta_{c}, \bXi_{\bu})$. To elaborate, we need the following results.
\begin{lem}\label{lem:min_continous}
    Let $w \in \cL^2(\Q_n)$ be a fixed function. For any $c > 0$, define the $\Q_n$-ball
    \begin{equation*}
        \mathcal{B}(c) := \left\{  u \in \cL^2(\Q_n):  \norm{u - w}_{\Q_n} \le c \right\}.
    \end{equation*}
    Then for any closed set $\Xi_{u} \subseteq \cL^2(\Q_n)$, the function
    \begin{equation*}
        \zeta(c) := \min_{ u \in \Xi_{u}  \cap   \mathcal{B}(c)  }
        \E_{\Q_n}[\ell(u)]
    \end{equation*}
    is continuous in $c$.
\end{lem}
\begin{proof}
    Denote $\cM(c) := \Xi_{u}  \cap  \mathcal{B}(c)$. By viewing $\cM: \R_{> 0} \rightrightarrows \cL^2(\Q_n)$ as a correspondence between topological spaces, we can use Berge's Maximum Theorem \cite{???} to prove continuity. Then it suffices to establish the following facts. Let $c > 0$ be a fixed constant.
    \begin{itemize}
        \item The function $u \mapsto \E_{\Q_n}[\ell(u)]$ is continuous, due to continuity of $\ell$ and definition of $\Q_n$.
        \item $\cM$ is upper hemicontinuous (u.h.c.).
        
        We first prove $\mathcal{B}$ is u.h.c.. By sequential characterization, for any sequence $\{ c_m \}_{m=1}^\infty$ in $\R_{> 0}$ and $\{ u_m \}_{m=1}^\infty$ in $\mathcal{B}(c)$, such that $u_m \in \mathcal{B}(c_m)$. If $c_m \to c$ and $u_m \to u$ as $m \to \infty$, we have
        \begin{equation*}
            \norm{u - w}_{\Q_n} = \lim_{m \to \infty} \norm{u_m - w}_{\Q_n} \le \lim_{m \to \infty} c_m = c,
        \end{equation*}
        where the first equality is from continuity of $\norm{\,\cdot\,}_{\Q_n}$, and the inequality is from $u_m \in \mathcal{B}(c_m)$. This shows $u \in \mathcal{B}(c)$ and hence $\mathcal{B}$ is u.h.c.. Since $\cM$ is a closed subcorrespondence of $\mathcal{B}$, by \cite{???} $\cM$ is also u.h.c..

        \item $\cM$ is lower hemicontinuous (l.h.c.).
        
        By definition, consider any open set $U$ such that $U \cap \cM(c) \neq \varnothing$. Since $\cM(c)$ is compact, let $u_0 \in U \cap \cM(c)$ be an interior point. Then for any $c' > 0$ such that $\norm{u_0 - w}_{\Q_n} =: c_0 < c' < c$, we have $u_0 \in \cM(c')$. This implies $U \cap \cM(c) \neq \varnothing$ for any $c' \in (c_0, \infty)$, since $\cM(c_1) \subseteq \cM(c_2)$ for any $c_1 \le c_2$. Therefore, $U$ intersects with some neighborhood of $\cM(c)$, i.e., $\cM$ is l.h.c..
    \end{itemize}
    Hence, $\cM$ is a continuous correspondence. By Maximum Theorem, $c \mapsto \zeta(c)$ is continuous.
\end{proof}
\end{comment}



\subsubsection{Step 3 --- Convergence in variational forms: Proof of \cref{lem:M2-3}}
\label{subsubsec:under_step3}

\begin{proof}[\textbf{Proof of \cref{lem:M2-3}}]
    First, by definition of $M_n^{(2)}$ and $M_n^{(3)}$:
    \begin{equation*}
        \abs{ M_n^{(2)}(\bTheta_{c}, \bXi_{\bu}) - M_n^{(3)}(\bTheta_{c}, \bXi_{\bu}) } \le \, \sup_{(\rho, R, \beta_0) \in \bTheta_{c}}
            \abs{ 
            \min_{  
                U \in \Xi_{u}  \cap   \mathcal{N}_n  }
                \E_{\Q_n}[\ell(U)]    
        -
        \min_{  
        U \in \Xi_{u}  \cap   \mathcal{N}^\delta_n  }
        \E_{\Q_n}[\ell(U)]
        }.
    \end{equation*}
    For any fixed $(\rho, R, \beta_0) \in \bTheta_{c}$, by definition of $\mathcal{N}_n$ in \cref{eq:set_N_n} and $\mathcal{N}^\delta_n$ in \cref{eq:set_N_n_delta}, we have
    \begin{equation*}
        \left\vert \min_{U \in \Xi_{u} \cap \mathcal{N}_n} \E_{\Q_n}[\ell(U)] - \min_{U \in \Xi_{u} \cap \mathcal{N}^\delta_n}
        \E_{\Q_n}[\ell(U)] \right\vert
        \le  
        \max_{\substack{ U, U' \in \Xi_u \cap \mathcal{N}_n \cap \mathcal{N}^\delta_n \\ 
        \| U - U' \|_{\Q_n} \le \veps_n (\rho, R, \beta_0)}} \left\vert \E_{\Q_n} [\ell(U)] - \E_{\Q_n} [\ell(U')] \right\vert,
    \end{equation*}
    where
    \begin{equation*}
        \veps_n (\rho, R, \beta_0) := \left\vert \frac{R}{\sqrt{n}} \biggl( \sqrt{1 - \rho^2}  \| \bP_{\bmu}^\perp \hh \|_2  -  \rho\frac{\hh^\top \vmu}{\norm{\vmu}_2}  \biggr) - R \frac{\sqrt{1 - \rho^2}}{\sqrt{\delta}} \right\vert.
    \end{equation*}
    By our assumption that $\ell$ is pseudo-Lipschitz, the following estimate holds:
    \begin{align*}
        \left\vert \E_{\Q_n} [\ell(U)] - \E_{\Q_n} [\ell(U')] \right\vert 
        & \le 
        \frac{1}{n} \sum_{i=1}^{n} \vert \ell(u_i) - \ell(u'_i) \vert 
        \le \frac{L}{n} \sum_{i=1}^{n} (1 + \vert u_i \vert + \vert u'_i \vert ) \vert u_i - u'_i \vert \\
        & \stackrel{\text{(i)}}{\le}  L \left( 1 + \|U\|_{\Q_n} + \|U'\|_{\Q_n} \right) \|U - U'\|_{\Q_n} 
        \stackrel{\text{(ii)}}{\le}
        C ( 1 + o_{\P}(1) ) \, \veps_n (\rho, R, \beta_0) ,
    \end{align*}
    where (i) follows from Cauchy--Schwarz inequality, (ii) follows from the compactness of $\mathcal{N}_n^\delta$ and $\bTheta_{c}$, and the upper bound below:
    \begin{align*}
        \norm{U}_{\Q_n} & \le  \sup_{(\rho, R, \beta_0) \in \bTheta_{c}} \bigl\| \rho\norm{\vmu}_2 R + R G + \beta_0 Y \bigr\|_{\Q_n}
        + \frac{R \sqrt{1 - \rho^2}}{\sqrt{\delta}} 
        \\
        & \le \rho\norm{\vmu}_2 R_{\max} + R_{\max} \norm{G}_{\Q_n} + B_{0, \max} \norm{Y}_{\Q_n} + \frac{R_{\max}}{\sqrt{\delta}} \\
        & \overset{\mathmakebox[0pt][c]{\smash{\text{($*$)}}}}{=} 
        \rho\norm{\vmu}_2 R_{\max} + R_{\max} \bigl( 1 + o_{\P}(1) \bigr) + B_{0, \max} + \frac{R_{\max}}{\sqrt{\delta}},
    \end{align*}
    by denoting $R_{\max} := \max_{(\rho, R, \beta_0) \in \bTheta_{c}} R$, $B_{0, \max} := \max_{(\rho, R, \beta_0) \in \bTheta_{c}} \abs{\beta_0}$, and $C > 0$ is some constant. Here, ($*$) is from the law of large numbers: $\norm{G}_{\Q_n} \conp \norm{G}_{\Q_\infty} = (\E[G^2])^{1/2} = 1$. Combining these estimates, we finally deduce that
    \begin{align*}
        & \abs{M_n^{(2)}(\bTheta_{c}, \bXi_{\bu}) - M_n^{(3)}(\bTheta_{c}, \bXi_{\bu})} 
        \le  C ( 1 + o_{\P}(1) ) \max_{(\rho, R, \beta_0) \in \bTheta_{c}} \veps_n (\rho, R, \beta_0) \\
        = {} & 
        C ( 1 + o_{\P}(1) ) \max_{(\rho, R, \beta_0) \in \bTheta_{c}} \left\vert \frac{R}{\sqrt{n}} \biggl( \sqrt{1 - \rho^2}  \| \bP_{\bmu}^\perp \hh \|_2  -  \rho\frac{\hh^\top \vmu}{\norm{\vmu}_2}  \biggr) - R \frac{\sqrt{1 - \rho^2}}{\sqrt{\delta}} \right\vert \\
        \le {} & 
        C ( 1 + o_{\P}(1) ) \cdot R_{\max} \left( \left\vert \frac{1}{\sqrt{n}} \| \bP_{\bmu}^\perp \hh \|_2 - \frac{1}{\sqrt{\delta}} \right\vert + \frac{1}{\sqrt{n}} \frac{\vert \hh^\top \vmu \vert}{\norm{\vmu}_2} \right) \conp 0.
    \end{align*}
    The convergence in the last line follows from
\begin{equation*}
    \frac{\| \bP_{\bmu}^\perp \hh \|_2}{\sqrt{n}} = \frac{\| \bP_{\bmu}^\perp \hh \|_2}{\| \bP_{\bmu}^\perp \|_\mathrm{F}}
    \cdot \frac{\sqrt{d-1}}{\sqrt{n}}
    \conp \frac{1}{\sqrt{\delta}},
    \qquad
    \frac{ \hh^\top \vmu }{ \sqrt{n} \norm{\vmu}_2 } \conp 0,
\end{equation*}
according to \cref{lem:subG_concentrate}\ref{lem:subG-Hanson-Wright-II}\ref{lem:subG-Hoeffding} and $\|\bP_{\bmu}^\perp \|_\mathrm{op} = 1$, $\| \bP_{\bmu}^\perp \|_\mathrm{F} = \sqrt{d - 1}$. This completes the proof.
\end{proof}


\subsubsection{Step 4 --- Asymptotic characterization: Proofs of \cref{lem:M3-star}, \ref{lem:var_fixed}}
\label{subsubsec:under_step4}

We need the following auxiliary result, which studies a general variational problem for both $\Q = \Q_n$ and $\Q = \Q_\infty$ with parameters $(\rho, R, \beta_0)$ fixed. In particular, we are able to express the random variable $\xi$ by $(\rho, R, \beta_0)$, $(G, Y)$, and an additional scalar (Lagrange multiplier). Then, we can rewrite $M_n^{(3)}(\bTheta_{c})$, $M^{*}(\bTheta_{c})$ as low-dimensional convex-concave minimax problems.

\begin{lem}\label{lem:var_fixed}
    For any fixed parameters $\rho \in (-1, 1)$, $R > 0$, $\beta_0 \in \R$, and the probability measure $\Q = \Q_n$ or $\Q = \Q_\infty$, consider the following variational problem
    \begin{equation}
    \label{eq:ERM_var_fix}
        \zeta_{\rho, R, \beta_0}(\Q) :=
        \min_{\xi \in \cL^2(\Q), \norm{\xi}_{\Q}^2 \le 1/\delta} \mathscr{R}_{\Q}(\xi),
        \quad
        \mathscr{R}_{\Q}(\xi) := \E_{\Q} \left[ \ell \bigl( \rho\norm{\vmu}_2 R + RG + \beta_0 Y + R\sqrt{1 - \rho^2} \xi \bigr) \right] \! .
    \end{equation}
    \begin{enumerate}[label=(\alph*)]
        \item \label{lem:var_fixed(a)}
        $\mathscr{R}_{\Q}(\xi)$ has a unique minimizer $\xi^* := \xi^*_{\Q}(\rho, R, \beta_0)$, which must satisfy
        \begin{equation}\label{eq:xi_star}
            \xi^*_{\Q}(\rho, R, \beta_0) = - \frac{\lambda^*}{R \sqrt{1 - \rho^2}}\ell'\bigl(\prox_{ \lambda^* \ell}( \rho\norm{\vmu}_2 R + RG + \beta_0 Y )\bigr),
        \end{equation}
        where $\lambda^*$ is the unique solution such that $\norm{\xi^*}_\Q^2 = 1/\delta$.
        % {\color{blue} (logit) As a consequence,
        % \[ 
        % \begin{aligned}
        %     U^*(\rho, R, \beta_0)
        %     & := \rho\norm{\vmu}_2 R + RG + \beta_0 Y + R\sqrt{1 - \rho^2} \xi^*_{\Q}(\rho, R, \beta_0) \\
        %     &  \phantom{:}= \prox_{ \lambda \ell}( \rho\norm{\vmu}_2 R + RG + \beta_0 Y ).
        % \end{aligned}
        % \]
        % }
        As a consequence, we have
        \begin{equation*}
            \zeta_{\rho, R, \beta_0}(\Q)
                = 
            \E_{\Q} \left[ \ell \bigl( \prox_{ \lambda^* \ell}( \rho \norm{\vmu}_2 R + R G + \beta_0 Y )
                \bigr) \right],
        \end{equation*}
        where $\prox_{\lambda^*\ell}$ and $\envelope_{\lambda^*\ell}$ are the proximal operator and Moreau envelope of $\ell$ defined in \cref{append_subsec_Moreau}. Moreover, $\lambda^*$ is a decreasing function of $\delta$.
        \item \label{lem:var_fixed(b)}
        With change of variables $A := R \rho$, $B := R \sqrt{1 - \rho^2}$, the variational problem \cref{eq:ERM_var_fix} can be recast as $\zeta_{\rho, R, \beta_0}(\Q) = \sup_{\nu > 0} \mathscr{R}_{\nu, \Q}(A, B, \beta_0)$, where
        \begin{equation*}
            \mathscr{R}_{\nu, \Q}(A, B, \beta_0) 
            :=   - \frac{B \nu}{2 \delta }
            +
            \E_{\Q} \left[ \envelope_{\ell} \Bigl( A \norm{\vmu}_2 + A G_1 + B G_2 + \beta_0 Y ; \frac{B}{\nu} \Bigr) \right], 
        \end{equation*}
        and $(Y, G_1, G_2) \sim P_y \times \normal(0, 1) \times \normal(0, 1)$ under $\Q = \Q_\infty$.\footnote{According to the change of variables, we have relation $A G_1 + B G_2 \overset{\mathrm{d}}{=} R G$ under $\Q = \Q_\infty$. We can also construct the realizations $\{G_1(g_i, y_i), G_2(g_i, y_i)\}_{i=1}^n$ such that $A G_1 + B G_2 = R G$, $\Q_n$-a.s., that is, $A G_1(g_i, y_i) + B G_2(g_i, y_i) = R G(g_i, y_i)$, for all $i \in [n]$.} Moreover, $\mathscr{R}_{\nu, \Q}(A, B, \beta_0)$ is convex in $(A, B, \beta_0)$ over $\R_{>0} \times \R_{>0} \times \R$ and concave in $\nu$.
        
        % \item The function $c \mapsto \zeta(c)$ is continuous.
    \end{enumerate}
\end{lem}
\begin{proof}
    For \textbf{\ref{lem:var_fixed(a)}}, we first show the existence of a minimizer. The proof is a standard application of direct method in calculus of variations. Since $\ell$ is lower bounded, we know that
        \begin{equation*}
            \inf_{\xi \in \cL^2(\Q), \norm{\xi}_{\Q}^2 \le 1/\delta} \mathscr{R}_{\Q}(\xi) > - \infty.
        \end{equation*}
        Let $\{ \xi_m \}_{m \in \mathbb{N}} \in \cL^2 (\Q)$ be a minimizing sequence such that $\norm{\xi_m}_{\Q}^2 \le 1/\delta$, and
        \begin{equation*}
            \lim_{m \to \infty} \mathscr{R}_{\Q}(\xi_m) = \inf_{\xi \in \cL^2(\Q), \norm{\xi}_{\Q}^2 \le 1/\delta} \mathscr{R}_{\Q}(\xi).
        \end{equation*}
        Since $\cL^2 (\Q)$ is a Hilbert space (and hence self-reflexive), Banach-Alaoglu theorem implies that $\{ \xi_m \}$ has a weak-* convergent (and hence weak convergent) subsequence, which we still denote as $\{ \xi_m \}$. Let $\xi^*$ denote the weak limit of $\{ \xi_m \}$. By using Mazur's lemma, we know that there exists another sequence $\{ \xi'_m \}_{m \in \mathbb{N}}$, such that each $\xi'_m$ is a finite convex combination of $\{ \xi_k \}_{m \le k \le m + N(m)}$ ($N(m) \ge 0$ depends on $m$), and that $\xi'_m$ strongly converges to $\xi^*$. Now since $\mathscr{R}_{\Q}$ is convex (this follows from convexity of $\ell$ and the fact that integration $\E_{\Q}$ preserves convexity), we have
        \begin{equation*}
            \liminf_{m \to \infty} \mathscr{R}_{\Q} (\xi'_m) \le \liminf_{m \to \infty} \mathscr{R}_{\Q} (\xi_m) = \inf_{\xi \in \cL^2(\Q), \norm{\xi}_{\Q}^2 \le 1/\delta} \mathscr{R}_{\Q}(\xi).
        \end{equation*}
        On the other hand, Fatou's lemma implies that
        \begin{equation*}
            \mathscr{R}_{\Q}(\xi^*) \le \liminf_{m \to \infty} \mathscr{R}_{\Q}(\xi'_m).
        \end{equation*}
        This immediately leads to
        \begin{equation*}
            \mathscr{R}_{\Q}(\xi^*) = \inf_{\xi \in \cL^2(\Q), \norm{\xi}_{\Q}^2 \le 1/\delta} \mathscr{R}_{\Q}(\xi),
        \end{equation*}
        i.e., $\xi^*$ is a minimizer of $\mathscr{R}_{\Q}$.
        In order to prove uniqueness of the minimizer, we will show that $\mathscr{R}_{\Q} : \cL^2(\Q) \to \R_{> 0}$ is strictly convex. For any $\alpha \in (0, 1)$ and $\xi_1, \xi_2 \in \cL^2(\Q)$, with a shorthand $V := \rho\norm{\vmu}_2 R + RG + \beta_0 Y$, we notice that
    \begin{align*}
            & \mathscr{R}_{\Q}( \alpha \xi_1 + (1 - \alpha) \xi_2 )
            \\
            = {} & \E_{\Q} \left[  \ell \Bigl( \alpha \bigl(  V + R\sqrt{1 - \rho^2} \xi_1 \bigr) +  
            (1 - \alpha)\bigl( V + R\sqrt{1 - \rho^2} \xi_2 \bigr)
            \Bigr) \right] \\
            \le {} & \E_{\Q} \left[ \alpha \ell \bigl(  V + R\sqrt{1 - \rho^2} \xi_1 \bigr) +  
            (1 - \alpha) \ell\bigl( V + R\sqrt{1 - \rho^2} \xi_2 \bigr)
             \right] 
            = 
            \alpha \mathscr{R}_{\Q}( \xi_1 ) + (1 - \alpha) \mathscr{R}_{\Q}( \xi_2 ),
    \end{align*}
    where the inequality follows from strong convexity of $\ell$, and it becomes equality if and only if $\Q(\xi_1 \not= \xi_2) = 0$. Hence we conclude $\mathscr{R}_{\Q}$ is strictly convex. Since $\{ \xi: \norm{\xi}_{\Q}^2 \le 1/\delta \}$ is a convex set, it implies the uniqueness ($\Q$-a.s.) of the minimizer $\xi^*$.

    As a consequence, the unique minimizer is determined by the Karush--Kuhn--Tucker (KKT) and Slater's conditions for variational problems \cite[Theorem 2.9.2]{zalinescu2002convex}. $\xi$ is the minimizer if and only if, for some scalar $\nu$ (dual variable), the followings hold:
    \begin{equation}\label{eq:KKT_xi}
        \begin{aligned}
            U = \rho\norm{\vmu}_2 R + RG + \beta_0 Y + R\sqrt{1 - \rho^2} \xi,
            \qquad
            \ell' ( U ) + \nu \xi = 0, \\
            \norm{\xi}_{\Q}^2 - \delta^{-1} \le 0,
            \qquad
            \nu \ge 0,
            \qquad
            \nu(\norm{\xi}_{\Q}^2 - \delta^{-1}) = 0.
        \end{aligned}
    \end{equation}
    We claim that the KKT conditions imply that any minimizer $\xi$ and its associated dual variable $\nu$ must satisfy
    \begin{equation*}
        0 < \nu < \infty,  \qquad  \xi > 0 \ \  (\text{$\Q$-a.s.}), \qquad \norm{\xi}_{\Q}^2 = \delta^{-1}.
    \end{equation*}
    To show this, we notice that $R\sqrt{1 - \rho^2} > 0$ and $\ell$ is decreasing. Therefore, for any $\xi \in \cL^2(\Q)$, $\mathscr{R}_{\Q}(\xi) \ge \mathscr{R}_{\Q}(\abs{\xi})$. It implies that $\xi \ge 0$ if $\xi$ is the minimizer. Hence, by stationarity in \cref{eq:KKT_xi}: $\nu\xi = - \ell' ( U ) > 0$, which implies the positivity of $\nu$, $\xi$. Then $\norm{\xi}_{\Q}^2 = \delta^{-1}$ comes from complementary slackness in \cref{eq:KKT_xi}. To show $\nu$ must be finite, notice that $\nu \to +\infty$ implies $\ell'(U) \to -\infty$. Then $U \to -\infty$ since $\ell'$ is strictly increasing, while it contradicts $\xi > 0$ and $\norm{\xi}_{\Q}^2 = \delta^{-1}$.

    By change of variable $\lambda := R \sqrt{1 - \rho^2}/\nu$, now we can rewrite KKT conditions \cref{eq:KKT_xi} as
    \begin{equation}\label{eq:KKT_xi2}
            U + \lambda \ell' ( U ) = \rho\norm{\vmu}_2 R + RG + \beta_0 Y,
            \qquad
            0 < \lambda < \infty,
            \qquad
            \norm{\xi}_{\Q}^2 = \delta^{-1},
    \end{equation}
    where $\xi$ and $U$ are related by
    \begin{equation}\label{eq:xi-U}
        \xi = -\frac{\lambda}{R\sqrt{1 - \rho^2}} \ell'(U).
    \end{equation}
    Notice that \cref{eq:KKT_xi2} has a unique solution for $U$, since $x \mapsto x + \lambda \ell'(x)$ is a strictly increasing continuous function from $\R$ to $\R$, for any $\lambda \in (0, \infty)$. Then, according to \cref{lem:prox}, $U$ can be expressed by the proximal operator of $\ell$,
    \begin{equation}
    \label{eq:U_prox}
        U = \prox_{ \lambda \ell}( \rho\norm{\vmu}_2 R + RG + \beta_0 Y ).
    \end{equation}
    Combine it with \cref{eq:xi-U} gives the expression of $\xi^*$ in \cref{eq:xi_star}. To establish the uniqueness of $\lambda$, we show that $\nu$ satisfying \cref{eq:KKT_xi} must be unique. Note that $\xi = \xi (\nu)$ is determined by
    \begin{equation*}
        \nu \xi(\nu) + \ell' \left( \rho\norm{\vmu}_2 R + RG + \beta_0 Y + R\sqrt{1 - \rho^2} \xi (\nu) \right) = \, 0.
    \end{equation*}
    Since $\nu, \xi(\nu) > 0$ and $\ell'$ is strictly increasing (by strong convexity), we know that $\xi(\nu)$ is strictly decreasing in $\nu$. The uniqueness of $\nu$ immediately follows from the condition $\norm{\xi(\nu)}_{\Q}^2 = \delta^{-1}$. This also implies that $\xi(\nu) > 0$ is decreasing in $\delta$. Then we conclude $\nu$ is increasing in $\delta$, or equivalently $\lambda$ is decreasing in $\delta$. This completes the proof of part \ref{lem:var_fixed(a)}.


    \vspace{0.5\baselineskip}
    \noindent
    For \textbf{\ref{lem:var_fixed(b)}}, as a consequence we have
    \begin{align*}
    \zeta_{\rho, R, \beta_0}(\Q) & = \min_{\xi \in \cL^2(\Q), \norm{\xi}^2_{\Q} \le 1/\delta} \mathscr{R}_{\Q}(\xi) 
    \\
    & = \min_{\xi \in \cL^2(\Q), \norm{\xi}^2_{\Q} \le 1/\delta} \E_{\Q} \left[ \ell \bigl( \rho\norm{\vmu}_2 R + RG + \beta_0 Y + R\sqrt{1 - \rho^2} \xi \bigr) \right] \\
    & = \min_{\xi \in \cL^2(\Q)} \sup_{\nu \ge 0} \E_{\Q} \left[ \ell \bigl( \rho\norm{\vmu}_2 R + RG + \beta_0 Y + 
    R\sqrt{1 - \rho^2}\xi \bigr) + R\sqrt{1 - \rho^2} \cdot \frac{\nu}{2} \left( \xi^2 - \frac{1}{\delta} \right) \right] \\
    & \stackrel{\mathmakebox[0pt][c]{\smash{\text{(i)}}}}{=} 
    \sup_{\nu \ge 0} \min_{\xi \in \cL^2(\Q)} \E_{\Q} \left[ \ell \bigl( \rho\norm{\vmu}_2 R + RG + \beta_0 Y + 
    R\sqrt{1 - \rho^2}\xi \bigr) + R\sqrt{1 - \rho^2} \cdot \frac{\nu}{2} \left( \xi^2 - \frac{1}{\delta} \right) \right] \\
    & \stackrel{\mathmakebox[0pt][c]{\smash{\text{(ii)}}}}{=} 
    \sup_{\lambda > 0} \min_{U \in \cL^2(\Q)} \E_{\Q} \left[ \ell ( U ) + \frac{1}{2 \lambda} \left( U - \rho\norm{\vmu}_2 R - RG - \beta_0 Y \right)^2 - \frac{R^2 (1 - \rho^2)}{2 \lambda \delta} \right] \\
    & \stackrel{\mathmakebox[0pt][c]{\smash{\text{(iii)}}}}{=}
    \sup_{\lambda > 0} \left\{ \E_{\Q} \left[ \envelope_{\ell} \left( \rho\norm{\vmu}_2 R + RG + \beta_0 Y; \lambda \right) \right] - \frac{R^2 (1 - \rho^2)}{2 \lambda \delta} \right\},
\end{align*}
where (i) comes from strong duality in part \ref{lem:var_fixed(a)}, (ii) is by change of variable $U := \rho\norm{\vmu}_2 R + RG + \beta_0 Y + R\sqrt{1 - \rho^2} \xi$ and $\lambda = R\sqrt{1 - \rho^2}/\nu$, (iii) is from the definition of Moreau envelope \cref{eq:envelope}. Now, consider change of variable
\begin{equation*}
    A = R \rho, \qquad
    B = R \sqrt{1 - \rho^2}, \qquad
    \nu = R \sqrt{1 - \rho^2}/\lambda.
\end{equation*}
Note that $0 < \nu < \infty$ by part \ref{lem:var_fixed(a)}, then $\zeta_{\rho, R, \beta_0}(\Q)$ can be expressed as
\begin{equation*}
    \zeta_{\rho, R, \beta_0}(\Q) = 
    \min_{\xi \in \cL^2(\Q), \norm{\xi}^2_{\Q} \le 1/\delta} \mathscr{R}_{\Q}(\xi)  = 
        \sup_{\nu > 0} \,
        \mathscr{R}_{\nu, \Q}(A, B, \beta_0), 
\end{equation*}
where
\begin{equation*} 
    \mathscr{R}_{\nu, \Q}(A, B, \beta_0) 
    =   - \frac{B \nu}{2 \delta }
        +
        \E_{\Q} \left[ \envelope_{\ell} \Bigl( A \norm{\vmu}_2 + A G_1 + B G_2 + \beta_0 Y ; \frac{B}{\nu} \Bigr) \right].
\end{equation*}
Finally, we complete the proof by the following arguments:
\begin{itemize}
    \item $\mathscr{R}_{\nu, \Q}(A, B, \beta_0)$ is convex in $(A, B, \beta_0)$. It comes from \cref{lem:prox}\ref{lem:prox(a)} that $(x, \lambda) \mapsto \envelope_{\ell}(x; \lambda)$ is convex, and the fact that integration $\E_{\Q}$ preserves convexity.
    \item $\mathscr{R}_{\nu, \Q}(A, B, \beta_0)$ is concave in $\nu$. This comes from \cref{eq:envelope} that
    \begin{equation*}
        \envelope_{\ell} \Bigl( A \norm{\vmu}_2 + A G_1 + B G_2 + \beta_0 Y ; \frac{B}{\nu} \Bigr)
        = \min_{t \in \R} \left\{ 
        \ell(t) + \frac{\nu}{2B} (A \norm{\vmu}_2 + A G_1 + B G_2 - t)^2
        \right\},
    \end{equation*}
    with the fact that pointwise minimum and integration $\E_{\Q}$ preserves concavity. 
\end{itemize}
This concludes the proof of part \ref{lem:var_fixed(b)}.
\end{proof}



Then we can use \cref{lem:var_fixed} to show convergence $M_n^{(3)}(\bTheta_{c}) \conp M^{*}(\bTheta_{c})$ in \cref{lem:M3-star}.
\begin{proof}[\textbf{Proof of \cref{lem:M3-star}}]
    Recall the change of variables $A = R \rho$ and $B = R \sqrt{1 - \rho^2}$ defined in \cref{lem:var_fixed}\ref{lem:var_fixed(b)}.
    Note that $f: (\rho, R, \beta_0) \mapsto (R \rho, R \sqrt{1 - \rho^2}, \beta_0)$ is a continuous map. Then $f(\bTheta_{c}) \subset \R_{\ge 0} \times \R_{\ge 0} \times \R$ is still compact. Hence, by \cref{lem:var_fixed} we have
    \begin{equation*}
        M_n^{(3)}(\bTheta_{c}) =
        \min_{ (A, B, \beta_0) \in f(\bTheta_{c}) } \sup_{\nu > 0} \,
        \mathscr{R}_{\nu, \Q_n}(A, B, \beta_0),
        \quad
        M^*(\bTheta_{c}) =
        \min_{ (A, B, \beta_0) \in f(\bTheta_{c}) } \sup_{\nu > 0} \,
        \mathscr{R}_{\nu, \Q_\infty}(A, B, \beta_0).
    \end{equation*}
    For any fixed $A, B \ge 0$, $\beta_0 \in \R$, $\nu > 0$, by law of large numbers,
    \begin{align*}
        \mathscr{R}_{\nu, \Q_n}(A, B, \beta_0) 
        & = - \frac{B \nu}{2 \delta }
        +
        \E_{\Q_n} \left[ \envelope_{\ell} \Bigl( A \norm{\vmu}_2 + A G_1 + B G_2 + \beta_0 Y ; \frac{B}{\nu} \Bigr) \right] \\
        \conp \ \mathscr{R}_{\nu, \Q_\infty}(A, B, \beta_0)
        & = - \frac{B \nu}{2 \delta }
        +
        \E_{\phantom{\Q_n}} \left[ \envelope_{\ell} \Bigl( A \norm{\vmu}_2 + A G_1 + B G_2 + \beta_0 Y ; \frac{B}{\nu} \Bigr) \right]. 
    \end{align*}
    Recall $\mathscr{R}_{\nu, \Q_n}(A, B, \beta_0)$ is concave in $\nu$. Also, note that $\mathscr{R}_{\nu, \Q_\infty}(A, B, \beta_0) \to -\infty$ as $\nu \to \infty$, since by \cref{lem:prox}\ref{lem:prox(a)}, we have
    \begin{equation*}
        \lim_{\nu \to \infty} \E \left[ \envelope_{\ell} \Bigl( A \norm{\vmu}_2 + A G_1 + B G_2 + \beta_0 Y ; \frac{B}{\nu} \Bigr) \right]
        = \E \left[ \ell ( A \norm{\vmu}_2 + A G_1 + B G_2 + \beta_0 Y ) \right] < \infty.
    \end{equation*}
    This implies there exits $\overline{\nu} \in \R_{> 0}$, such that $\sup_{\nu \ge \overline{\nu}} \mathscr{R}_{\nu, \Q_\infty}(A, B, \beta_0) < \sup_{\nu > 0} \mathscr{R}_{\nu, \Q_\infty}(A, B, \beta_0)$. So, we can apply \cite[Lemma 10]{thrampoulidis2018precise} and conclude the uniform convergence
    \begin{equation*}
        \sup_{\nu > 0} \,
        \mathscr{R}_{\nu, \Q_n}(A, B, \beta_0)
        \ \conp \
        \sup_{\nu > 0} \,
        \mathscr{R}_{\nu, \Q_\infty}(A, B, \beta_0).
    \end{equation*}
    Recall that both $\sup_{\nu > 0} \mathscr{R}_{\nu, \Q_n}(A, B, \beta_0)$ and $\sup_{\nu > 0} \mathscr{R}_{\nu, \Q_\infty}(A, B, \beta_0)$ are convex in $(A, B, \beta_0)$ (since pointwise supremum preserves convexity). Then we could obtain uniform convergence on compact set $f(\bTheta_{c})$ by convexity \cite[Lemma 7.75]{liese2008statistical}:
    \begin{equation*}
    \abs{M_n^{(3)}(\bTheta_{c}) - M^*(\bTheta_{c})}
    \le
        \sup_{ (A, B, \beta_0) \in f(\bTheta_{c}) } \abs{
        \, \sup_{\nu > 0} \, \mathscr{R}_{\nu, \Q_n}(A, B, \beta_0)
        - \sup_{\nu > 0} \, \mathscr{R}_{\nu, \Q_\infty}(A, B, \beta_0)
        } \conp 0.
    \end{equation*}
    This completes the proof.
\end{proof}






\subsubsection{Parameter convergence and optimality analysis: Proofs of \cref{lem:boundedness_parameter}---\ref{lem:ERM_param_conv}}
\label{append_subsubsec:ERM_param}

% We first conclude the asymptotics of $M_n(\bTheta_{\vbeta}, \bXi_{\bu})$ by combining previous lemmas.
Recall that
\begin{equation}
    \label{eq:M_star}
    M^{*}
    = \min_{ \substack{ \rho \in [-1, 1], R \ge 0, \beta_0 \in \R \\ \xi \in \cL^2(\Q_\infty), \norm{\xi}_{\Q_\infty} \le 1/\sqrt{\delta} } } 
    \E \left[ \ell \bigl( \rho\norm{\vmu}_2 R + RG + \beta_0 Y + R\sqrt{1 - \rho^2} \xi \bigr) \right],
\end{equation}
where $R$, $\beta_0$ are optimized over unbounded sets. The following lemma shows that any minimizer $R^*$, $\beta_0^*$ of \cref{eq:M_star} must be bounded.

\begin{lem}[Boundedness of $R^*$ and $\beta_0^*$]\label{lem:boundedness_parameter}
    Let $(\rho^*, R^*, \beta_0^*, \xi^*)$ be any minimizer of \cref{eq:M_star}. Then in the non-separable regime ($\delta > \delta^*(0)$), we have $R^* < \infty$ and $\abs{\beta_0^*} < \infty$.
\end{lem}
\begin{proof}
    We first prove the following claim: There exists an $\veps > 0$, such that for any $(a, b) \in \R_{> 0} \times \R$ satisfying $a^2 + b^2 = 1$, any $\rho \in [-1, 1]$, and any $\xi \in \cL^2 (\Q_\infty)$, $\norm{\xi}_{\Q_\infty} \le 1 / \sqrt{\delta}$:
    \begin{equation}
    \label{eq:V_claim}
        \P \left( a \rho \norm{\bmu}_2 + a G + b Y + a \sqrt{1 - \rho^2} \xi \le - \veps \right) \ge \veps.
    \end{equation}
    We prove this claim by contradiction. Assume it is not true, then for any $m \in \mathbb{N}$, there exists the corresponding $(a_m, b_m, \rho_m, \xi_m)$ such that $(a_m, b_m) \in \R_{> 0} \times \R$, with $a_m^2 + b_m^2 = 1$, $\rho_m \in [-1, 1]$, and $\xi_m \in \cL^2 (\Q_\infty)$, $\norm{\xi_m}_{\Q_\infty} \le 1 / \sqrt{\delta}$, which satisfy
    \begin{equation}
    \label{eq:Vm_bound}
        \P \left( a_m \rho_m \norm{\bmu}_2 + a_m G + b_m Y + a_m \sqrt{1 - \rho_m^2} \xi_m \le - \frac{1}{m} \right) < \frac{1}{m}.
    \end{equation}
    We can always assume that $(a_m, b_m, \rho_m) \to (a, b, \rho)$ and $\xi_m \to \xi$ weakly in $\cL^2 (\Q_\infty)$ when $m \to \infty$. Otherwise, such a convergent subsequence always exists according to Heine--Borel Theorem and Banach--Alaoglu Theorem. Therefore, $a_m \rho_m \norm{\bmu}_2 + a_m G + b_m Y + a_m \sqrt{1 - \rho_m^2} \xi_m$ weakly converges to $a \rho \norm{\bmu}_2 + a G + b Y + a \sqrt{1 - \rho^2} \xi$ in $\cL^2 (\Q_\infty)$. For any nonnegative $Z \in \cL^2 (\Q_\infty)$, one has
    \begin{align*}
        & \E \left[ \bigl( a \rho \norm{\bmu}_2 + a G + b Y + a \sqrt{1 - \rho^2} \xi \bigr) Z \right] \\
        = {} & \lim_{m \to \infty} \E \left[ \bigl( a_m \rho_m \norm{\bmu}_2 + a_m G + b_m Y + a_m \sqrt{1 - \rho_m^2} \xi_m \bigr) Z \right].
    \end{align*}
    Denote $U_m := a_m \rho_m \norm{\bmu}_2 + a_m G + b_m Y + a_m \sqrt{1 - \rho_m^2} \xi_m$, then we obtain the following estimate:
    \begin{align*}
        \E [U_m Z] & =  \E \left[ U_m \ind_{U_m > - 1/m} Z \right] + \E \left[ U_m \ind_{U_m \le - 1/m} Z \right] 
        \\ 
        & \ge - \frac{1}{m} \E [Z] - \left( \E[ U_m^2 ] \right)^{1/2}  \left(\E[ Z^2 \ind_{U_m \le - 1/m}] \right)^{1/2},
    \end{align*}
    where the last line follows from Cauchy--Schwarz inequality. By definition of $U_m$, we know that $\E [U_m^2]$ is uniformly bounded for any $m \in \mathbb{N}$. Further, since $Z \in \cL^2 (\Q_\infty)$ and $\P (U_m \le -1/m) \le 1/m \to 0$ as $m \to \infty$ by \cref{eq:Vm_bound}, we know that $\E [Z^2 \ind_{U_m \le - 1/m}] \to 0$. It finally follows that
    \begin{equation*}
        \E \left[ \bigl( a \rho \norm{\bmu}_2 + a G + b Y + a \sqrt{1 - \rho^2} \xi \bigr) Z \right] 
        = \lim_{m \to \infty} \E [U_m Z] \ge 0.
    \end{equation*}
    Since this is true for any nonnegative $Z \in \cL^2 (\Q_\infty)$, we know that
    \begin{equation*}
        a \rho \norm{\bmu}_2 + a G + b Y + a \sqrt{1 - \rho^2} \xi \ge 0, \quad \text{almost surely},
    \end{equation*}
    or equivalently, there exists $(\rho, R, \beta_0) \in [-1, 1] \times \R_{>0} \times \R$ and $\xi \in \cL^2 (\Q_\infty)$, $\E[\xi^2] \le 1/\delta$ satisfying
    \begin{equation*}
        R \rho \norm{\bmu}_2 + R G + \beta_0 Y + R \sqrt{1 - \rho^2} \xi \ge 0, \quad \text{almost surely}.
    \end{equation*}
    It implies the constraint of the variational problem for the separable regime (SVM) \cref{eq:SVM_variation}, i.e., $\rho \norm{\bmu}_2 + G + \beta_0' Y + \sqrt{1 - \rho^2} \xi \ge \kappa$ holds for some $\kappa \ge 0$ (with change of variable $\beta_0' := \beta_0 / R$). According to \cref{thm:SVM_main}\ref{thm:SVM_main_var}, we obtain $\kappa^* \ge 0$, or equivalently $\delta \le \delta^*(0)$, which contradicts the non-separable regime $\delta > \delta^* (0)$. Our claim \cref{eq:V_claim} is thus proved. 
    
    Now for any $(\rho, R, \beta_0, \xi)$ such that $R > 0$, denote
    \begin{equation*}
        V(\rho, R, \beta_0, \xi) := \frac{1}{\sqrt{R^2 + \beta_0^2}} \bigl( \rho \norm{\vmu}_2 R + R G + \beta_0 Y + R\sqrt{1 - \rho^2} \xi \bigr). 
    \end{equation*}
    We know that $\P (V(\rho, R, \beta_0, \xi) \le - \veps) \ge \veps$ by \cref{eq:V_claim}. Therefore,
    \begin{align*}
        & \E \left[ \ell \bigl( \rho \norm{\vmu}_2 R + R G + \beta_0 Y + R\sqrt{1 - \rho^2} \xi \bigr) \right]
        \\ 
          = {} & \E \left[ \ell \Bigl( \sqrt{ R^2 + \smash{\beta_0^2} } \, V(\rho, R, \beta_0, \xi) \Bigr) \right] \\
        \ge {} & \E \left[ \ell \Bigl( \sqrt{ R^2 + \smash{\beta_0^2} } \, V(\rho, R, \beta_0, \xi) \Bigr) \ind_{V(\rho, R, \beta_0, \xi) \le - \veps} \right] \\
        \ge {} & \veps \ell \Bigl( - \veps \sqrt{ R^2 + \smash{\beta_0^2} } \Bigr),
    \end{align*}
    which diverges to infinity as $R^2 + \beta_0^2 \to \infty$. This completes the proof.
\end{proof}

A direct consequence of \cref{lem:boundedness_parameter} is that $M^* = M^*(\bTheta_{c})$ for $\bTheta_{c}$ large enough. The following result shows that $M^*$ in \cref{eq:M_star} has a unique minimizer.

\begin{lem}
\label{lem:M_star_var}
    Consider the variational problem $M^*$ defined in \cref{eq:M_star}.
    \begin{enumerate}[label=(\alph*)]
        \item \label{lem:M_star_var(a)}
        $M^*$ has a unique minimizer $(\rho^*, R^*, \beta_0^*, \xi^*)$, which must satisfy
        \begin{equation*}
            \xi^* = - \frac{\lambda^*}{R^* \sqrt{1 - \rho^*{}^2}} \ell'\bigl(\prox_{ \lambda^* \ell}( \rho^*\norm{\vmu}_2 R^* + R^* G + \beta_0^* Y )\bigr),
        \end{equation*}
        where $\lambda^*$ is the unique solution such that $\E[\xi^{* 2}] = 1/\delta$. As a consequence, we have
        \begin{equation*}
           M^* = \E \left[ \ell \bigl( \prox_{ \lambda^* \ell}( \rho^*\norm{\vmu}_2 R^* + R^* G + \beta_0^* Y )
                \bigr) \right].
        \end{equation*}

        
        \item \label{lem:M_star_var(b)}
        $(\rho^*, R^*, \beta_0^*, \lambda^*)$ is also the unique solution to the system of equations
        \begin{equation}\label{eq:sys_eq_Q}
            \begin{aligned}
                - \frac{R \rho}{\lambda \delta \norm{\vmu}_2}
                & = 
                \E \left[ \ell'\bigl( \prox_{ \lambda \ell}( \rho\norm{\vmu}_2 R + RG + \beta_0 Y ) \bigr) \right],
                \\
                \frac{R}{\lambda \delta }
                & = 
                \E \left[ \ell'\bigl( \prox_{ \lambda \ell}( \rho\norm{\vmu}_2 R + RG + \beta_0 Y ) \bigr) G \right],
                \\
                0
                & = 
                \E \left[\ell'\bigl( \prox_{ \lambda \ell}( \rho\norm{\vmu}_2 R + RG + \beta_0 Y ) \bigr) Y \right],
                \\
                \frac{R^2 (1 - \rho^2)}{\lambda^2 \delta}
                & = 
                \E \left[ \left(\ell'\bigl( \prox_{ \lambda \ell}( \rho\norm{\vmu}_2 R + RG + \beta_0 Y ) \bigr) \right)^2 \right],
            \end{aligned}
        \end{equation}
        where $(\rho^*, R^*, \beta_0^*, \lambda^*) \in (0, 1) \times \R_{>0} \times \R \times \R_{> 0}$.

        

        \item \label{lem:M_star_var(c)}
        % Moreover, the optimization problem \cref{eq:var_optim_Q} can also be expressed as
        % \begin{equation}\label{eq:var_optim_Q1}
        % \ljy{
        %     \mathscr{R}^*_{\Q} = 
        % \min_{ \substack{ \rho \in [0, 1], R \ge 0 \\ \beta_0 \in \R} }
        % \sup_{ \lambda > 0}
        % \biggl\{ 
        % - \frac{R^2(1 - \rho^2)}{2 \delta \lambda}
        % +
        % \E_{\Q} \left[ \envelope_{\lambda\ell} ( \rho\norm{\vmu}_2 R + RG + \beta_0 Y ) \right]
        % \biggr\}
        % ,
        % }
        % \end{equation}
        % where $(\rho^*, R^*, \beta_0^*, \lambda^*)$ is also the unique minimizer of this problem.

        With change of variables $A := R \rho$, $B := R \sqrt{1 - \rho^2}$, the original variational problem \cref{eq:M_star} can be reduced to the following minimax problem
        \[ 
        M^* = 
        \min_{\substack{ A \ge 0, B \ge 0 \\ \beta_0 \in \R} }
        \sup_{\nu > 0} 
        \,
         \biggl\{ 
        - \frac{B \nu}{2 \delta }
        +
        \E \left[ \envelope_{\ell} \Bigl( A \norm{\vmu}_2 + A G_1 + B G_2 + \beta_0 Y ; \frac{B}{\nu} \Bigr) \right]
        \biggr\},
        \]
        where $(Y, G_1, G_2) \sim P_y \times \normal(0, 1) \times \normal(0, 1)$, and the objective function is convex-concave.
    \end{enumerate}
\end{lem}

\begin{proof}
    We first show the optimization problem \cref{eq:M_star} has a unique minimizer. Since its original formulation is non-convex, we make the following change of variables:
    \begin{equation}\label{eq:change_var_AB}
        A := R \rho, \qquad B := R \sqrt{1 - \rho^2}, \qquad \xi_B := B \xi.
    \end{equation}
    Then, the optimization problem is recast as
    \begin{equation}\label{eq:var_optim_Q_recast}
        \min_{
        \substack{ A , B \ge 0, \beta_0 \in \R \\ \xi_B \in \cL^2(\Q_\infty)}
        } \ \E \left[ \ell \bigl( A \norm{\vmu}_2 + A G_1 + B G_2 + \beta_0 Y + \xi_B \bigr) \right], \quad \text{subject to} \ \norm{\xi_B}_{\Q} \le \frac{B}{\sqrt{\delta}},
    \end{equation}
    which is convex, where $(Y, G_1, G_2) \sim P_y \times \normal(0, 1) \times \normal(0, 1)$ (recall that $A G_1 + B G_2 \overset{\smash{\mathrm{d}}}{=} R G$). Now we show that the above optimization problem has a unique minimizer. Note that \cref{lem:boundedness_parameter} also implies that any minimizer of this optimization problem is finite. Therefore, a similar argument as in the proof of \cref{lem:var_fixed}\ref{lem:var_fixed(a)} shows that \cref{eq:var_optim_Q_recast} has a unique minimizer. Since the mapping $(\rho, R, \xi) \mapsto (A, B, \xi_B)$ is one-to-one, this also proves the original optimization problem \cref{eq:M_star} has a unique minimizer.
    
    As a consequence, the unique minimizer is determined by the KKT and Slater's conditions for variational problems \cite[Theorem 2.9.2]{zalinescu2002convex}. $(A, B, \beta_0, \xi_B)$ is the minimizer of \cref{eq:var_optim_Q_recast} if and only if, for some scalar $\nu_B$ (Lagrange multiplier), the followings hold:
    \begin{equation}
    \label{eq:M_star_KKT}
    \begin{aligned}
        A \norm{\vmu}_2 + A G_1 + B G_2 + \beta_0 Y + \xi_B & = U, \\
        \E \left[ \ell'(U) ( \norm{\vmu}_2 + G_1 ) \right] & = 0, \\
        \E \left[ \ell'(U) G_2 \right] - \nu_B \frac{B}{\delta} & = 0, \\
        \E \left[ \ell'(U) Y   \right] & = 0, \\
        \ell'(U) + \nu_B \xi_B & = 0, \\
        \delta  \, \E[\xi_B^2] \le B^2, \quad
        \nu_B \ge 0, \quad
        \nu_B \bigl( \delta \, \E[\xi_B^2] - B^2 \bigr) & = 0.
    \end{aligned}
    \end{equation}
    Using a similar argument as in the proof of \cref{lem:var_fixed}\ref{lem:var_fixed(a)}, we can also show that
    \begin{equation*}
        0 < \nu_B < \infty,  \qquad  \xi_B > 0 \ \  (\text{a.s.}), \qquad \E[\xi_B^2] = B^2/\delta,
    \end{equation*}
    which implies $B > 0$. Plugging this into \cref{eq:M_star_KKT} solves two conditions
    \begin{equation}
    \label{eq:M_star_KKT-1}
        \E \, \bigl[ \bigl( \ell'(U) \bigr)^2 \bigr] = \nu_B^2 \frac{B^2}{\delta},
        \qquad
        \E\left[ \ell'(U) Y \right] = 0.
    \end{equation}
    By Stein's identity, we also have relation
    \begin{equation*}
        \E\left[ \ell'(U) G_1 \right] = A \, \E\left[ \ell''(U) \right],
        \qquad
        \E\left[ \ell'(U) G_2 \right] = B \, \E\left[ \ell''(U) \right].
    \end{equation*}
    Combine the above with \cref{eq:M_star_KKT}, we obtain
    \begin{align*}
        \E \left[ \ell'(U) \right] = -\nu_B \frac{A}{\delta\norm{\bmu}_2},
    \qquad
        \E \left[ \ell'(U) G_1 \right] = \nu_B \frac{A}{\delta},
    \qquad
        \E \left[ \ell'(U) G_2 \right] = \nu_B \frac{B}{\delta},
    \end{align*}
    which is equivalent to (recall that $A G_1 + B G_2 \overset{\smash{\mathrm{d}}}{=} R G$)
    \begin{equation}
    \label{eq:M_star_KKT-2}
        \E \left[ \ell'(U) \right] = -\nu_B \frac{A}{\delta\norm{\bmu}_2},
        \qquad
        \E \left[ \ell'(U) G \right] = \nu_B \frac{R}{\delta}.
    \end{equation}
    The above implies $A > 0$ since $\ell' < 0$ by \cref{lem:ell}. Since both $A, B > 0$, by \cref{eq:change_var_AB} we have $\rho \in (-1, 1) \setminus \{ 0 \}$ and $R > 0$. Moreover, notice that for any $\rho > 0$,
    \begin{equation*}
        \E \left[ \ell \bigl( -\rho\norm{\vmu}_2 R + RG + \beta_0 Y + R\sqrt{1 - \rho^2} \xi \bigr) \right]
        >
        \E \left[ \ell \bigl( \rho\norm{\vmu}_2 R + RG + \beta_0 Y + R\sqrt{1 - \rho^2} \xi \bigr) \right].
    \end{equation*}
    Therefore, we must have $\rho \in (0, 1)$. Then we prove $(\rho^*, R^*, \beta_0^*, \lambda^*) \in (0, 1) \times \R_{>0} \times \R \times \R_{> 0}$. Lastly, by combining \cref{eq:M_star_KKT-1} and \eqref{eq:M_star_KKT-2} with change of variable $\lambda := 1/\nu_B$, and recalling \cref{eq:U_prox} in the proof of \cref{lem:var_fixed}, we obtain the KKT conditions \cref{eq:sys_eq_Q} expressed in $(\rho, R, \beta_0, \lambda)$. Then we complete the proof of part \ref{lem:M_star_var(b)}. Finally, part \ref{lem:M_star_var(c)} directly follows from \cref{lem:var_fixed}.
    % KKT conditions:
    %     \[
    %     \begin{aligned}
    %         \rho\norm{\vmu}_2 R + RG + \beta_0 Y + R\sqrt{1 - \rho^2} \xi & = U  \\
    %         \E_{\Q} \left[ \ell' ( U )
    %         \biggl( \norm{\vmu}_2 R - \frac{\rho R}{\sqrt{1 - \rho^2}} \xi \biggr) \right]
    %         & = 0 \\
    %         \E_{\Q} \left[ \ell' ( U )
    %         (\rho\norm{\vmu}_2 + G + \sqrt{1 - \rho^2} \xi ) \right] 
    %         & = 0 \\
    %         \E_{\Q} \left[ \ell' ( U )
    %         Y \right] 
    %         & = 0 \\
    %         \color{red} \ell' ( U )
    %         R \sqrt{1 - \rho^2} 
    %         + \lambda \xi
    %         & = 0 \\
    %         R & \ge 0 \\
    %         \delta \norm{\xi}_{\Q}^2 - 1 & \le 0 \\
    %         \lambda & \ge 0 \\
    %         \lambda\bigl( \delta \norm{\xi}_{\Q}^2 - 1 \bigr) & = 0 
    %     \end{aligned}
    %     \]
    %     If $\ell$ is strictly decreasing, we must have $\lambda > 0$, and therefore $\delta \norm{\xi}_{\Q}^2 - 1 = 0$. It can be shown that $\lambda < \infty$ ($\Q$-a.s.). Then, we apply variable transformation $\lambda \gets R^2 (1 - \rho^2)/\lambda$. KKT condition becomes
    %     \begin{equation*}
    %         \color{red} 
    %         \lambda \ell' ( U )
    %         + R \sqrt{1 - \rho^2} \xi
    %         = 0.
    %     \end{equation*}
    %     Let $U = U(G, Y)$ be the random (function) satisifying
    %     \[  
    %         U + \lambda \ell'(U) = \rho\norm{\vmu}_2 R + RG + \beta_0 Y.
    %     \]
    %     The existence and uniqueness of $U$ is ensured by strictly convexity of $\ell$. Recall that
    %     \begin{equation*}
    %         \prox_{\lambda f}(x) = \argmin_{t \in \R} \left\{ \lambda f(t) +  \frac12 (t - x)^2 \right\}.
    %     \end{equation*}
    %     Then we can write
    %     \begin{equation*}
    %         U = \prox_{ \lambda \ell}( \rho\norm{\vmu}_2 R + RG + \beta_0 Y ).
    %     \end{equation*}
    %     Recall
    %     \[ \ell'(U) = -\frac{R \sqrt{1 - \rho^2}}{\lambda} \xi. \]
    %     Therefore, the KKT simplifies to
    %     \[
    %     \begin{aligned}
    %         \norm{\vmu}_2 \sqrt{1 - \rho^2} \E_{\Q} [ \xi ] -  \rho \E_{\Q} [ \xi^2 ] 
    %         & = 0 \\
    %         \rho\norm{\vmu}_2 \E_{\Q}[\xi] +  \E_{\Q}[\xi G] + \sqrt{1 - \rho^2} \E_{\Q} [ \xi^2 ] 
    %         & = 0 \\
    %         \E_{\Q}[\xi Y]
    %         & = 0
    %     \end{aligned}
    %     \]
    %     Then we obtain the system of equations
    %     \[
    %     \begin{aligned}
    %         \E_{\Q} [ \xi ] & = \frac{\rho}{\delta \norm{\vmu}_2 \sqrt{1 - \rho^2}},
    %         \\
    %         \E_{\Q}[\xi G] & = - \frac{1}{\delta \sqrt{1 - \rho^2}},
    %         \\
    %         \E_{\Q}[\xi Y] & = 0,
    %         \\
    %         \E_{\Q}[\xi^2] & = \frac{1}{\delta}.
    %     \end{aligned}
    %     \]
    %     Plug-in these results,
    %     \[
    %     \begin{aligned}
    %         \E_{\Q} \left[ \ell'\bigl( \prox_{ \lambda \ell}( \rho\norm{\vmu}_2 R + RG + \beta_0 Y ) \bigr) \right] & = - \frac{\rho R}{\lambda \delta \norm{\vmu}_2},
    %         \\
    %         \E_{\Q} \left[ \ell'\bigl( \prox_{ \lambda \ell}( \rho\norm{\vmu}_2 R + RG + \beta_0 Y ) \bigr) G \right] & =  \frac{R}{\lambda \delta },
    %         \\
    %         \E_{\Q} \left[\ell'\bigl( \prox_{ \lambda \ell}( \rho\norm{\vmu}_2 R + RG + \beta_0 Y ) \bigr) Y \right] & = 0,
    %         \\
    %         \E_{\Q} \left[ \left(\ell'\bigl( \prox_{ \lambda \ell}( \rho\norm{\vmu}_2 R + RG + \beta_0 Y ) \bigr) \right)^2 \right] & = \frac{R^2 (1 - \rho^2)}{\lambda^2 \delta}.
    %     \end{aligned}
    %     \]
\end{proof}

We are now in position to establish the convergence of parameters.
\begin{proof}[\textbf{Proof of \cref{lem:ERM_param_conv}}]
Consider any $\varepsilon \ge 0$ and $C_R, C_{\beta_0} \in (0, \infty)$, let
\begin{equation*}
    \bTheta_{c}^*(\varepsilon) := \left\{ (\rho, R, \beta_0) \in [-1, 1] \times [0, C_R] \times [-C_{\beta_0}, C_{\beta_0}] : 
    \norm{ (\rho, R, \beta_0) - (\rho^*, R^*, \beta_0^*) }_2 \ge \varepsilon
    \right\}
\end{equation*}
and let $\bTheta^*_{\vbeta}(\varepsilon)$ defined as \cref{eq:Theta_link}. By \cref{lem:ERM_bound_beta} and \ref{lem:M_star_var}, we can choose some $C_R, C_{\beta_0} > 0$ and compact convex set $\bXi_{\bu} \subset \R^{n}$ large enough, such that as $n, d \to \infty$,
\begin{equation*}
    M_n =
    M_n(\bTheta^*_{\vbeta}(0), \bXi_{\bu}) \ \ (\text{w.h.p.}) ,
    \qquad
    M^* =
    M^*(\bTheta^*_{c}(0)).
\end{equation*}
Then according to \cref{thm:ERM_conv}, we have global convergence
\begin{equation*}
    M_n \conp M^*.
\end{equation*}
However, for any $\varepsilon > 0$ and $\zeta > 0$, by \cref{thm:ERM_conv} we have
\begin{equation*}
    M_n(\bTheta^*_{\vbeta}(\varepsilon), \R^n) =
    M_n(\bTheta^*_{\vbeta}(\varepsilon), \bXi_{\bu}) \ \ (\text{w.h.p.}) ,
    \quad
    \P \left( M_n(\bTheta^*_{\vbeta}(\varepsilon), \bXi_{\bu}) \le M^*(\bTheta_{c}^*(\varepsilon)) - \zeta \right) \to 0.
\end{equation*}
This implies
\begin{equation*}
    \pliminf_{n \to \infty} M_n(\bTheta^*_{\vbeta}(\varepsilon), \R^n) \ge
    M^*(\bTheta_{c}^*(\varepsilon))
    >
    M^*,
\end{equation*}
where the strict inequality comes from the uniqueness of minimizer $(\rho^*, R^*, \beta_0^*, \xi^*)$, established in \cref{lem:M_star_var}\ref{lem:M_star_var(a)}. Since $\varepsilon > 0$ can be arbitrarily small, this proves $(\hat\rho_n, \| \hat\vbeta_n \|_2, \hat\beta_{0,n}) \conp (\rho^*, R^*, \beta_0^*)$. Moreover, we know that $R^* > 0$ by \cref{lem:M_star_var}\ref{lem:M_star_var(b)}. So $\hat\vbeta_n \not= \bzero$ and therefore $\hat\rho_n$ is well-defined with high probability. This concludes the proof of \cref{lem:ERM_param_conv}.
\end{proof}



\subsubsection{ELD convergence: Proof of \cref{lem:ERM_logit_conv}}
\label{append_subsubsec:ERM_logit}

\begin{proof}[\textbf{Proof of \cref{lem:ERM_logit_conv}}]
We first establish the convergence of logit margins. Recall that
\begin{align*}
    \hat \cL_{n} = \  & \frac1n \sum_{i=1}^n \delta_{y_i ( \< \xx_i, \hat\vbeta \> + \hat\beta_{0} ) },
    \\ 
    \cL_* = \  & \Law \, (U^*)
    := \Law \, \bigl( \rho^*\norm{\vmu}_2 R^* + R^* G + \beta_0^* Y + R^* \sqrt{1 - \rho^*{}^2} \xi^* \bigr) 
    \\
    = \ & \Law \, \bigl( \prox_{ \lambda^* \ell}( \rho^* \norm{\vmu}_2 R^* + R^* G + \beta_0^* Y ) \bigr).
\end{align*}
For any $\varepsilon > 0$ small enough, we have defined the $\varepsilon$-$W_2$ open ball by
\begin{equation*}
    \mathsf{B}_{W_2}(\varepsilon) = \left\{ \bu \in \R^n:   W_2 \biggl( 
        \frac{1}{n}\sum_{i=1}^n \delta_{u_i}, \cL_*
     \biggr)  < \varepsilon \right\}.
\end{equation*}
For $C_R, C_{\beta_0} \in (0, \infty)$, let $\bTheta_{c} = [-1, 1] \times [0, C_R] \times [-C_{\beta_0}, C_{\beta_0}]$ and let $\bTheta_{\vbeta}$ be defined as \cref{eq:Theta_link}. When $C_R, C_{\beta_0} > 0$ and compact set $\bXi_{\bu} \subset \R^{n}$ are large enough, by \cref{lem:ERM_bound_beta} we have
\begin{align*}
    \wt M_n^\varepsilon & := M_n(\R^{d + 1}, \mathsf{B}_{W_2}^c(\varepsilon) ) 
    = M_n(\bTheta_{\vbeta}, \bXi_{\bu} \setminus \mathsf{B}_{W_2}(\varepsilon) ) \ \ (\text{w.h.p.}) ,
    \\
    % \P \, \Bigl( M_n(\bTheta_{\vbeta}, \bXi_{\bu} \setminus \mathsf{B}_{W_2}(\varepsilon) ) \le t \Bigr) 
    % \le 2 \, \P \,\Bigl( M_n^{(1)}(\bTheta_{\vbeta}, \bXi_{\bu} \setminus \mathsf{B}_{W_2}(\varepsilon) ) \le t \Bigr),
    % \ \ \forall\, t \in \R,
    % \\
    \wt M_n^{\varepsilon(3)} 
    & :=
    M_n^{(3)}(\bTheta_{c}, \mathsf{B}_{W_2}^c(\varepsilon) )
    = M_n^{(3)}(\bTheta_{c}, \bXi_{\bu} \setminus \mathsf{B}_{W_2}(\varepsilon) ).
\end{align*}
Combining these with \cref{lem:ERM_CGMT} and \ref{lem:M2-3} obtains that for any $\zeta > 0$,
\begin{equation}
    \label{eq:Mn-3-eps}
    \lim_{n \to \infty} \P \, \Bigl( \wt M_n^\varepsilon \le \wt M_n^{\varepsilon(3)} - \zeta \Bigr) = 0.
\end{equation}
In order to show $W_2( \hat \cL_{n}, \cL_* ) \conp 0$, our goal is to show that
\begin{equation*}
    \lim_{n \to \infty} \P \, \Bigl( \wt M_n^{\varepsilon} > M_n \Bigr) = 1.
\end{equation*}
Then according to \cref{eq:Mn-3-eps} and \cref{lem:ERM_param_conv}, it suffices to show that
\begin{equation}
    \label{eq:ERM_logit_goal}
    \pliminf_{n \to \infty} \wt M_n^{\varepsilon(3)}  > \plim_{n \to \infty} M_n = M^*.
\end{equation}
By \cref{eq:Mn(3)} and \eqref{eq:set_N_n_delta}, recall that
\begin{equation*}
    \wt M_n^{\varepsilon(3)} = \min_{ (\rho, R, \beta_0) \in \bTheta_{c} }
    \min_{\bu \in 
    \mathsf{N}^\delta_n(\rho, R, \beta_0) \setminus \mathsf{B}_{W_2}(\varepsilon) } 
    \frac1n \sum_{i=1}^n \ell(u_i),
\end{equation*}
where we temporarily define
\begin{equation*}
    \mathsf{N}^\delta_n(\rho, R, \beta_0) := \left\{
        \bu \in \R^n: 
        \frac{1}{\sqrt{n}} \bigl\| \rho\norm{\vmu}_2 R \bone_n + R \vg + \beta_0 \yy - \bu \bigr\|_2
        \le \frac{R \sqrt{1 - \rho^2}}{\sqrt{\delta}}
     \right\}.
\end{equation*}
Now we split $\wt M_n^{\varepsilon(3)}$ into two parts by
\begin{equation*}
    \wt M_n^{\varepsilon(3)} 
    = \min\left\{ I, I\!I \right\}
    := \min\left\{ 
        \min_{ \substack{ (\rho, R, \beta_0) \in \bTheta_{c} \setminus \mathsf{B}_{2,c^*}(\eta)
     \\ \bu \in 
    \mathsf{N}^\delta_n(\rho, R, \beta_0) \setminus \mathsf{B}_{W_2}(\varepsilon)  } } 
    \frac1n \sum_{i=1}^n \ell(u_i)
    ,
        \min_{ \substack{ (\rho, R, \beta_0) \in \bTheta_{c} \cap \mathsf{B}_{2,c^*}(\eta)
     \\ \bu \in 
    \mathsf{N}^\delta_n(\rho, R, \beta_0) \setminus \mathsf{B}_{W_2}(\varepsilon)  } } 
    \frac1n \sum_{i=1}^n \ell(u_i)
     \right\},
\end{equation*}
where $\eta > 0$ and
\begin{equation*}
\mathsf{B}_{2,c^*}(\eta) = \left\{ (\rho, R, \beta_0) \in \R^3 : \norm{(\rho, R, \beta_0) - (\rho^*, R^*, \beta^*_0)}_2 < \eta \right\}
\end{equation*}
is a $\eta$-$\cL^2$ open ball around the global minimizer $(\rho^*, R^*, \beta^*_0)$.

For the first term, with $\bTheta_{c}$ large enough such that $(\rho^*, R^*, \beta^*_0) \in \bTheta_{c}$, by \cref{lem:M3-star} we have
\begin{align*}
        I 
        & \ge    
        \min_{(\rho, R, \beta_0) \in \bTheta_{c} \setminus \mathsf{B}_{2,c^*}(\eta) }
        \min_{\bu \in 
    \mathsf{N}^\delta_n(\rho, R, \beta_0)  }
    \frac1n \sum_{i=1}^n \ell(u_i) 
       \\
       & =  \min_{ \substack{ (\rho, R, \beta_0) \in \bTheta_{c} \setminus \mathsf{B}_{2,c^*}(\eta) 
       \\ \xi \in \cL^2(\Q_n), \norm{\xi}^2_{\Q_n} \le 1/\delta } } 
       \E_{\Q_n} \left[ \ell \bigl( \rho\norm{\vmu}_2 R + RG + \beta_0 Y + R\sqrt{1 - \rho^2} \xi \bigr) \right] \\
       & =  M_n^{(3)}\bigl( \bTheta_{c} \setminus \mathsf{B}_{2,c^*}(\eta) \bigr)
       \conp  
       M_n^*\bigl( \bTheta_{c} \setminus \mathsf{B}_{2,c^*}(\eta) \bigr) >  M^*( \bTheta_{c} ) = M^*,
\end{align*}
where the strict inequality follows from the uniqueness of $(\rho^*, R^*, \beta^*_0)$ according to \cref{lem:M_star_var}\ref{lem:M_star_var(a)}.



For the second term, we can take $\eta > 0$ small enough, such that $(\rho, R, \beta_0) \in \mathsf{B}_{2,c^*}(\eta)$ implies
\begin{align*}
    &
    W_2 \, \Bigl( 
        \Law \left( U^*_{\rho, R, \beta_0} \right), \cL_*
     \Bigr) \\
    = {} &  
    W_2 \, \Bigl(
        \Law \left( U^*_{\rho, R, \beta_0} \right), \Law \, \bigl( U^*_{\rho^*, R^*, \beta_0^*} \bigr)
     \Bigr)
     \le \frac{\varepsilon}{2},
     \qquad
     \forall\, (\rho, R, \beta_0) \in \bTheta_{c} \cap \mathsf{B}_{2,c^*}(\eta),
\end{align*}
where $U^*_{\rho, R, \beta_0} := \rho\norm{\vmu}_2 R + RG + \beta_0 Y + R\sqrt{1 - \rho^2} \xi^*_{\Q_\infty}(\rho, R, \beta_0)$, and $\xi^*_{\Q_\infty}(\rho, R, \beta_0)$ is the unique minimizer of $\mathscr{R}_{\Q}(\xi)$ defined in \cref{eq:ERM_var_fix}, with an expression given by \cref{eq:xi_star}. The existence of such $\eta > 0$ is guaranteed by continuity of $W_2$ distance and $(\rho, R, \beta_0) \mapsto U^*_{\rho, R, \beta_0}$ by \cref{lem:var_fixed}. Then $\bu \notin \mathsf{B}_{W_2}(\varepsilon)$ implies (by triangle inequality)
\begin{equation*}
    W_2 \biggl( 
        \frac{1}{n}\sum_{i=1}^n \delta_{u_i}, \Law \left( U^*_{\rho, R, \beta_0} \right)
     \biggr)
     \ge \frac{\varepsilon}{2} 
     ,
     \qquad
     \forall\, (\rho, R, \beta_0) \in \bTheta_{c} \cap \mathsf{B}_{2,c^*}(\eta). 
\end{equation*} 
Thus we have
\begin{equation}
\label{eq:ERM_II0}
    I\!I = 
     \min_{ \substack{ (\rho, R, \beta_0) \in \bTheta_{c} \cap \mathsf{B}_{2,c^*}(\eta)
     \\ \bu \in 
    \mathsf{N}^\delta_n(\rho, R, \beta_0) \setminus \mathsf{B}_{W_2}(\varepsilon)  } } 
    \frac1n \sum_{i=1}^n \ell(u_i)
    \ge 
     \min_{ \substack{ (\rho, R, \beta_0) \in \bTheta_{c}
     \\ U \in 
    \mathcal{N}^\delta_n(\rho, R, \beta_0) \cap \mathcal{C}_n^\varepsilon(\rho, R, \beta_0)  } } 
    \E_{\Q_n}[\ell(U)],
\end{equation}
where denote
\begin{equation}
\label{eq:ERM_logit1}
    \mathcal{C}_n^\varepsilon(\rho, R, \beta_0) := \left\{ U \in \cL^2(\Q_\infty): 
    \norm{ U - U^*_{\rho, R, \beta_0} }_{\Q_n} \ge \frac{\varepsilon}{2}
    \right\}
\end{equation}
and recall \cref{eq:set_N_n_delta} that 
\begin{equation}
\label{eq:ERM_logit2}
    \mathcal{N}_n^\delta(\rho, R, \beta_0) = \left\{ 
        U \in \cL^2(\Q_n):  \bigl\| \rho\norm{\vmu}_2 R + RG + \beta_0 Y - U \bigr\|_{\Q_n}
        \le  \frac{ R\sqrt{1 - \rho^2} }{ \sqrt{\delta} }
     \right\}.
\end{equation}
Now, denote $\hat U_{\rho, R, \beta_0} := \rho\norm{\vmu}_2 R + RG + \beta_0 Y + R\sqrt{1 - \rho^2} \xi^*_{\Q_n}(\rho, R, \beta_0)$. According to \cref{lem:var_fixed}, we know that $\| \xi^*_{\Q_n}(\rho, R, \beta_0) \|_{\Q_n}^2 = 1/\delta$, that is,
\begin{equation}
\label{eq:ERM_logit3}
    \bigl\| \rho\norm{\vmu}_2 R + RG + \beta_0 Y - \hat U_{\rho, R, \beta_0} \bigr\|_{\Q_n}
        = \frac{ R\sqrt{1 - \rho^2} }{ \sqrt{\delta} }.
\end{equation}
We claim $\bigl\| U^*_{\rho, R, \beta_0} - \hat U_{\rho, R, \beta_0} \bigr\|_{\Q_n} \conp 0$. Otherwise, there exits a convergent sequence $\{ \hat\lambda_m \}_{m \in \mathbb{N}}$ such that $\plim_{m \to \infty} \hat\lambda_m \neq \lambda^*$, where $\hat\lambda_m$ satisfies the conditions in \cref{lem:var_fixed}\ref{lem:var_fixed(a)} under $\Q = \Q_m$, and $\lambda^*$ satisfies the conditions in \cref{lem:var_fixed}\ref{lem:var_fixed(a)} under $\Q = \Q_\infty$. This contradicts the convergence $\argmax_{\nu > 0} \mathscr{R}_{\nu, \Q_m}(A, B, \beta_0) \conp \argmax_{\nu > 0} \mathscr{R}_{\nu, \Q_\infty}(A, B, \beta_0)$ by an argmax theorem for the concave process \cite[Theorem 7.77]{liese2008statistical} according to \cref{lem:var_fixed}\ref{lem:var_fixed(b)}, and change of variable $\nu = R\sqrt{1 - \rho^2}/\lambda$. Hence, for all $n$ large enough, we have
\begin{equation*}
    \bigl\| U^*_{\rho, R, \beta_0} - \hat U_{\rho, R, \beta_0} \bigr\|_{\Q_n} \le \frac{\varepsilon}{2}.
\end{equation*}
Combining this with \cref{eq:ERM_logit1}---\eqref{eq:ERM_logit3} together, by triangle inequality, we obtain
\begin{equation}
\label{eq:ERM_logit_sets}
    \mathcal{N}^\delta_n(\rho, R, \beta_0) \cap \mathcal{C}_n^\varepsilon(\rho, R, \beta_0)
    \subseteq \wt{\mathcal{N}}^{\delta,\varepsilon}_n(\rho, R, \beta_0)
\end{equation}
where
\begin{equation*}
    \wt{\mathcal{N}}^{\delta,\varepsilon}_n(\rho, R, \beta_0)
    := \left\{ 
        U \in \cL^2(\Q_n):  \bigl\| \rho\norm{\vmu}_2 R + RG + \beta_0 Y - U \bigr\|_{\Q_n}
        \le \frac{ R\sqrt{1 - \rho^2} }{ \sqrt{\delta} } - \varepsilon
     \right\}.
\end{equation*}
Recall that $C_R = \max_{(\rho, R, \beta_0) \in \bTheta_{c}} R$. Denote $\delta'_\varepsilon > \delta$ as a constant such that
\begin{equation}
\label{eq:delta_eps}
    \frac{1}{\sqrt{\delta'_\varepsilon}} := \frac{1}{\sqrt{\delta}} - \frac{\varepsilon}{C_R}.
\end{equation}
Then following \cref{eq:ERM_II0}, we have
\begin{align*}
    I\!I 
        \ge & \ \min_{ (\rho, R, \beta_0) \in \bTheta_{c} }
    \min_{ U \in 
    \mathcal{N}^\delta_n(\rho, R, \beta_0) \cap \mathcal{C}_n^\varepsilon(\rho, R, \beta_0)  }
    \E_{\Q_n}[\ell(U)]
    \\
    \stackrel{\mathmakebox[0pt][c]{\smash{\text{(i)}}}}{\ge} & \ \min_{ (\rho, R, \beta_0) \in \bTheta_{c} }
    \min_{ U \in 
    \wt{\mathcal{N}}^{\delta,\varepsilon}_n(\rho, R, \beta_0)  }
    \E_{\Q_n}[\ell(U)]
    \\
    = & \ \min_{ (\rho, R, \beta_0) \in \bTheta_{c} }
       \min_{\xi \in \cL^2(\Q_n), \norm{\xi}_{\Q_n} \le \frac{1}{\sqrt{\delta}} - \frac{\varepsilon}{R\sqrt{1 - \rho^2}}} \E_{\Q_n} \left[ \ell \bigl( \rho\norm{\vmu}_2 R + RG + \beta_0 Y + R\sqrt{1 - \rho^2} \xi \bigr) \right]
    \\
    \stackrel{\mathmakebox[0pt][c]{\smash{\text{(ii)}}}}{\ge}  & \
    \min_{ (\rho, R, \beta_0) \in \bTheta_{c} }
       \min_{\xi \in \cL^2(\Q_n), \norm{\xi}^2_{\Q_n} \le 1/\delta'_\varepsilon } \E_{\Q_n} \left[ \ell \bigl( \rho\norm{\vmu}_2 R + RG + \beta_0 Y + R\sqrt{1 - \rho^2} \xi \bigr) \right]
    \\
    \conp & \ 
    \min_{ (\rho, R, \beta_0) \in \bTheta_{c} }
       \min_{\xi \in \cL^2(\Q_\infty), \norm{\xi}^2_{\Q_\infty} \le 1/\delta'_\varepsilon } \E \left[ \ell \bigl( \rho\norm{\vmu}_2 R + RG + \beta_0 Y + R\sqrt{1 - \rho^2} \xi \bigr) \right]
    \\
    \stackrel{\mathmakebox[0pt][c]{\smash{\text{(iii)}}}}{>} & \ \min_{ (\rho, R, \beta_0) \in \bTheta_{c} }
       \min_{\xi \in \cL^2(\Q_\infty), \norm{\xi}^2_{\Q_\infty} \le 1/\delta } \E \left[ \ell \bigl( \rho\norm{\vmu}_2 R + RG + \beta_0 Y + R\sqrt{1 - \rho^2} \xi \bigr) \right]
    \\
    = & \ M^*( \bTheta_{c} ) = M^*,
\end{align*}
where (i) follows from \cref{eq:ERM_logit_sets}, (ii) follows from \cref{eq:delta_eps} and the fact that 
\begin{equation*}
    \frac{1}{\sqrt{\delta}} - \frac{\varepsilon}{R\sqrt{1 - \rho^2}} \le \frac{1}{\sqrt{\delta'_\varepsilon}},
    \qquad \forall\, (\rho, R, \beta_0) \in \bTheta_{c},
\end{equation*}
the convergence follows from \cref{lem:M3-star}, and (iii) follows from the uniqueness of $(\rho^*, R^*, \beta_0^*)$ and KKT conditions $\norm{\xi^*}^2_{\Q_\infty} = 1/\delta$ in \cref{lem:M_star_var}. 

Finally, combining everthing together, we have
\begin{equation*}
    \pliminf_{n \to \infty} \wt M_n^{\varepsilon(3)}
    \ge \min\left\{ \pliminf_{n \to \infty} I, \ \pliminf_{n \to \infty} I\!I \right\}
    > M^*.
\end{equation*}
This shows \cref{eq:ERM_logit_goal}, and hence completes the proof.
\end{proof}

Using an argument similar to the one at the end of the proof of \cref{lem:over_logit_conv}, we can show the convergence of empirical logit distribution $W_2 ( \hat \nu_{n}, \nu_* ) \conp 0$ from $W_2( \hat \cL_{n}, \cL_* ) \conp 0$ given by \cref{lem:ERM_logit_conv}.

\subsubsection{Completing the proof of \cref{thm:logistic_main}}
\label{subsubsec:under_final}
\begin{proof}[\textbf{Proof of \cref{thm:logistic_main}}]
    Consider the ERM problem \cref{eq:logistic_reg} with arbitrary $\tau > 0$. Recall that $\wt y_i = y_i/s(y_i)$ where $s: \{ \pm 1 \} \to \{ 1 \} \cup \{ \tau \}$ is defined as per \cref{eq:s_fun}. $M_n$ is redefined as \cref{eq:logistic_reg}
    \begin{equation*}
        M_n := \min_{\bbeta \in \R^d, \, \beta_0 \in \R}  \frac1n \sum_{i=1}^n \ell\bigl( 
        \wt y_i(\< \xx_i, \bbeta \> +  \beta_0 )
        \bigr).
    \end{equation*}
    Under this modification, $M_n(\bTheta_{\vbeta}, \bXi_{\bu})$ can be redefined and expressed as
    \begin{align*}
        M_n(\bTheta_{\vbeta}, \bXi_{\bu})
        :\! & = \min_{ \substack{ (\bbeta , \beta_0) \in \bTheta_{\vbeta} \\  \bu \in \bXi_{\bu} } }
        \max_{ \bv \in \R^n } \left\{
        \frac1n \sum_{i=1}^n \ell \biggl( \frac{u_i}{s(y_i)} \biggr)
         + \frac{1}{n} \sum_{i=1}^n v_i \left( y_i(\< \xx_i, \bbeta \> +  \beta_0 ) - u_i \right)
         \right\} \\
        % & = \min_{ \substack{ (\bbeta , \beta_0) \in \bTheta_{\vbeta} \\  \bu \in \bXi_{\bu} } }
        % \max_{ \bv \in \R^n } \left\{
        % \frac1n \sum_{i=1}^n \ell \biggl( \frac{u_i}{s(y_i)} \biggr)
        %  + \frac{1}{n} \sum_{i=1}^n v_i \bigl(  y_i(\< \xx_i, \bbeta \> +  \beta_0 ) - u_i \bigr)
        %  \right\} \\
        & = \min_{ \substack{ (\bbeta , \beta_0) \in \bTheta_{\vbeta} \\  \bu \in \bXi_{\bu} } }
     \max_{ \bv \in \R^n }
     \left\{
     \frac1n \sum_{i=1}^n \ell \biggl( \frac{u_i}{s(y_i)} \biggr)
      + \frac1n \bv^\top \bone \< \bmu, \vbeta \>
      + \frac1n \bv^\top \ZZ \vbeta + \frac1n \beta_0 \bv^\top \yy - \frac1n \bv^\top \bu
      \right\}.
    \end{align*}
Consequently, quantities $M_n^{(k)}$, $k= 1,2,3$ and $M^*$ used in the proof can be similarly redefined as
    \begin{align*}
        M_n^{(1)}(\bTheta_{\vbeta}, \bXi_{\bu})
        & : = \smash {\min_{ \substack{ (\bbeta , \beta_0) \in \bTheta_{\vbeta} \\  \bu \in \bXi_{\bu} } } }
        \max_{ \bv \in \R^n }
        \, \Biggl\{
        \frac1n \sum_{i=1}^n \ell \biggl( \frac{u_i}{s(y_i)} \biggr)
         + \frac1n \bv^\top \bone \< \bmu, \vbeta \>
         + \frac1n \norm{\bv}_2 \hh^\top \vbeta 
         \\
        & \phantom{.} \phantom{ 
            : = \smash {\min_{ \substack{ (\bbeta , \beta_0) \in \bTheta_{\vbeta} \\  \bu \in \bXi_{\bu} } } }
        \max_{ \bv \in \R^n }
        \, \Biggl\{
        }
         + \frac1n \norm{\vbeta}_2 \vg^\top \bv 
         + \frac1n \beta_0 \bv^\top \yy - \frac1n \bv^\top \bu
         \Biggr\},
        \\
        M_n^{(2)}(\bTheta_{c}, \bXi_{\bu})
        & := \min_{ (\rho, R, \beta_0) \in \bTheta_{c} }
        \min_{ U \in \Xi_{u}  \cap   \mathcal{N}_n }
        \E_{\Q_n}\left[\ell \bigl( U/s(Y) \bigr)\right],
        \\
        M_n^{(3)}(\bTheta_{c}, \bXi_{\bu}) 
        & :=  
        \min_{ (\rho, R, \beta_0) \in \bTheta_{c} } 
        \min_{ U \in \Xi_{u}  \cap   \mathcal{N}^\delta_n }
        \E_{\Q_n}\left[\ell \bigl( U/s(Y) \bigr)\right],
        \\
        M_n^{(3)}(\bTheta_{c}) 
        & :=  
        \min_{ (\rho, R, \beta_0) \in \bTheta_{c} } 
        \min_{ U \in \mathcal{N}^\delta_n }
        \E_{\Q_n}\left[\ell \bigl( U/s(Y) \bigr)\right],
        \\
        & \phantom{:}= \min_{ \substack{ (\rho, R, \beta_0) \in \bTheta_{c} \\ \xi \in \cL^2(\Q_n), \norm{\xi}_{\Q_n} \le 1/\sqrt{\delta} } } 
    \E_{\Q_n} \left[ \ell \biggl( \frac{ \rho\norm{\vmu}_2 R + RG + \beta_0 Y + R\sqrt{1 - \rho^2} \xi }{s(Y)} \biggr) \right],
        \\
         M^*(\bTheta_{c})
     & := \min_{ \substack{ (\rho, R, \beta_0) \in \bTheta_{c} \\ \xi \in \cL^2(\Q_\infty), \norm{\xi}_{\Q_\infty} \le 1/\sqrt{\delta} } } 
    \E \left[ \ell \biggl( \frac{ \rho\norm{\vmu}_2 R + RG + \beta_0 Y + R\sqrt{1 - \rho^2} \xi }{s(Y)} \biggr) \right],
        \\
        M^* & := M^*([-1, 1] \times \R_{\ge 0} \times \R),
    \end{align*}
    where $\mathcal{N}_n$, $\mathcal{N}^\delta_n$ are still defined as \cref{eq:set_N_n}, \eqref{eq:set_N_n_delta}, and we still apply the change of variable
    \begin{equation*}
    U = \rho\norm{\vmu}_2 R + RG + \beta_0 Y + R\sqrt{1 - \rho^2} \xi.
    \end{equation*}
    One can use exactly similar arguments to conclude \cref{lem:ERM_bound_beta}---\ref{lem:M3-star} and \cref{thm:ERM_conv} with definitions above. For \cref{lem:var_fixed}, we can also get similar results, but the KKT condition in \cref{eq:KKT_xi2} now becomes
    % \begin{equation*}
    %     U = \rho\norm{\vmu}_2 R + RG + \beta_0 Y + R\sqrt{1 - \rho^2} \xi,
    %         \qquad
    %     \ell' ( U/s(Y) ) + \frac{R\sqrt{1 - \rho^2}}{\lambda} \xi = 0
    % \end{equation*}
    \begin{equation*}
        U + \lambda \ell' ( U/s(Y) ) = \rho\norm{\vmu}_2 R + RG + \beta_0 Y,
    \end{equation*}
    which implies
    \begin{equation}
    \label{eq:ERM_U_new}
        \frac{U}{s(Y)} = \prox_{\ell}\left( \frac{\rho\norm{\vmu}_2 R + RG + \beta_0 Y}{s(Y)} ; \frac{\lambda}{s(Y)} \right),
    \end{equation}
    as a substitute of \cref{eq:U_prox},
    and
    \begin{equation*}
        \xi^*_{\Q}(\rho, R, \beta_0) = -\frac{\lambda}{R\sqrt{1 - \rho^2}} \ell'\left( \prox_{\ell}\left( \frac{\rho\norm{\vmu}_2 R + RG + \beta_0 Y}{s(Y)} ; \frac{\lambda}{s(Y)} \right) \right),
    \end{equation*}
    as a substitute of \cref{eq:xi_star}.

\vspace{0.5\baselineskip}
\noindent
\textbf{\ref{thm:logistic_main(a)}:} According to the definition above, the KKT conditions \cref{lem:M_star_var} will become
\begin{align}
                - \frac{R \rho}{\lambda \delta \norm{\vmu}_2}
                & = 
                \E \left[ \wt\ell'_Y(U) \right],
                \label{eq:KKT_new1}
                \\
                \frac{R}{\lambda \delta }
                & = 
                \E \left[ \wt\ell'_Y(U) G \right],
                \label{eq:KKT_new2}
                \\
                0
                & = 
                \E \left[ \wt\ell'_Y(U) Y \right],
                \label{eq:KKT_new3}
                \\
                \frac{R^2 (1 - \rho^2)}{\lambda^2 \delta}
                & = 
                \E \left[ \bigl( \wt\ell'_Y(U) \bigr)^2 \right],
                \notag
\end{align}
where
\begin{equation*}
    \wt\ell'_Y(U) := \frac{1}{s(Y)} \ell'\left( \frac{U}{s(Y)} \right)
    = \frac{1}{s(Y)} \ell'\left( \prox_{\ell}\left( \frac{\rho\norm{\vmu}_2 R + RG + \beta_0 Y}{s(Y)} ; \frac{\lambda}{s(Y)} \right) \right)
    ,
\end{equation*}
and $U$ follows the relation \cref{eq:ERM_U_new}. By Stein's identity, \cref{eq:KKT_new2} can be expressed as
\begin{align*}
    \frac{R}{\lambda \delta } = \E \left[ \wt\ell'_Y(U) G \right]
    & = \E \left[ \frac{1}{s(Y)} \ell''\left( \frac{U}{s(Y)} \right)
    \cdot \frac{\d (U/s(Y))}{\d G} \right] \\
    & = \E \left[ \frac{1}{s(Y)} \ell''\left( \frac{U}{s(Y)} \right)
    \cdot \frac{1}{1 + \dfrac{\lambda}{s(Y)} \ell''\left( \dfrac{U}{s(Y)} \right) } \cdot \dfrac{R}{s(Y)}  \right],
\end{align*}
which gives the third KKT condition in \ref{thm:logistic_main(a)}. Besides, \cref{eq:KKT_new1} and \ref{eq:KKT_new3} can be rewritten as
\begin{align*}
    - \frac{R \rho}{\lambda \delta \norm{\vmu}_2}
    & = \pi \E\left[ \frac{1}{\tau} \ell'\left( \prox_{\ell}\left( \frac{\rho\norm{\vmu}_2 R + RG + \beta_0}{\tau} ; \frac{\lambda}{\tau} \right) \right) \right] \\
    & \phantom{=.} 
    + (1-\pi) \E\left[ \ell'\, \bigl( \prox_{\ell}\left( \rho\norm{\vmu}_2 R + RG - \beta_0 ; \lambda \right) \bigr)  \right],
    \\
    0 & = \pi \E\left[ \frac{1}{\tau} \ell'\left( \prox_{\ell}\left( \frac{\rho\norm{\vmu}_2 R + RG + \beta_0}{\tau} ; \frac{\lambda}{\tau} \right) \right) \right] \\
    & \phantom{=.}
    - (1-\pi) \E\left[ \ell'\, \bigl( \prox_{\ell}\left( \rho\norm{\vmu}_2 R + RG - \beta_0 ; \lambda \right) \bigr)  \right],
\end{align*}
which solves the first two KKT conditions in \ref{thm:logistic_main(a)}. This concludes the proof of part \ref{thm:logistic_main(a)}.


\vspace{0.5\baselineskip}
\noindent
\textbf{\ref{thm:logistic_main(b)}:} \cref{lem:ERM_param_conv} still remains valid under arbitrary $\tau > 0$, which concludes the proof.

\vspace{0.5\baselineskip}
\noindent
\textbf{\ref{thm:logistic_main(c)}:} Similar to the proof of \cref{thm:SVM_main}\ref{thm:SVM_main_err}, we can show that for any test point $(\xx_\mathrm{new}, y_\mathrm{new})$,
\begin{equation*}
        \hat f(\xx_\mathrm{new}) = \< \xx_\mathrm{new}, \hat\vbeta_n \> + \hat\beta_{0,n}
        \cond y_\mathrm{new} R^* \rho^* \norm{\bmu}_2 + R^* G + \beta_{0}^*,
\end{equation*}
where $(y^\mathrm{new}, G) \sim P_y \times \normal(0, 1)$. Therefore, by bounded convergence theorem, the errors have limits
\begin{align*}
        \lim_{n \to \infty} \Err_{+,n} & = \P\left( + R^*\rho^* \norm{\bmu}_2 + R^*G + \beta_{0}^* \le 0 \right)
        = \Phi \left(- \rho^* \norm{\bmu}_2  - \frac{\beta_0^*}{R^*} \right), \\
        \lim_{n \to \infty} \Err_{-,n} & = \P\left( - R^*\rho^* \norm{\bmu}_2 + R^*G + \beta_{0}^* >  0 \right)
        = \Phi \left(- \rho^* \norm{\bmu}_2  + \frac{\beta_0^*}{R^*} \right).
\end{align*}
This concludes the proof of part \ref{thm:logistic_main(c)}.

\vspace{0.5\baselineskip}
\noindent
\textbf{\ref{thm:logistic_main(d)}:} Based on \cref{eq:ERM_U_new}, we redefine $\cL_*$ in \cref{append_subsubsec:ERM_logit} by
\begin{equation*}
    \cL_* := \Law \, (U^*)
    =
    \Law \left( 
    s(Y) \, \prox_{\ell}\left( \frac{\rho^*\norm{\vmu}_2 R^* + R^*G + \beta_0^* Y}{s(Y)}; \frac{\lambda^*}{s(Y)} \right)
    \right).
\end{equation*}
Then \cref{lem:ERM_logit_conv} and the corresponding convergence of ELD still hold. The convergence of TLD directly comes from the proof of part \ref{thm:logistic_main(c)}. This concludes the proof of part \ref{thm:logistic_main(d)}.

Finally, we complete the proof of \cref{thm:logistic_main}.
\end{proof}












\begin{comment}
\begin{proof}
    \begin{enumerate}
        \item[(a)] 
    

    \item[(b)] The expression for $\zeta (c)$ directly follows from our conclusion of part (a). We next show that $\zeta(c)$ is continuous. In fact, for any $c_1, c_2 > 0$, we have \TODO{one expression is longer than page width}
    \begin{align*}
        & \left\vert \zeta(c_1) - \zeta(c_2) \right\vert \le \, \max_{\xi \in \cL^2(\Q), \norm{\xi}_{\Q}^2 \le c_1} \left\vert \mathscr{R}_{\Q}(\xi) - \mathscr{R}_{\Q} \left( \sqrt{\frac{c_2}{c_1}} \xi \right) \right\vert \\
        \le \, & \max_{\xi \in \cL^2(\Q), \norm{\xi}_{\Q}^2 \le c_1} \E_{\Q} \left[ \left\vert \ell \bigl( \rho\norm{\vmu}_2 R + RG + \beta_0 Y + R\sqrt{1 - \rho^2} \xi \bigr) - \ell \left( \rho\norm{\vmu}_2 R + RG + \beta_0 Y + R\sqrt{1 - \rho^2} \sqrt{\frac{c_2}{c_1}} \xi \right) \right\vert \right] \\
        \stackrel{(i)}{\le} \, & \max_{\xi \in \cL^2(\Q), \norm{\xi}_{\Q}^2 \le c_1} L \E_{\Q} \left[ \left( 1 + 2 \left\vert \rho\norm{\vmu}_2 R + RG + \beta_0 Y \right\vert + R\sqrt{1 - \rho^2} \left( 1 + \sqrt{\frac{c_2}{c_1}} \right) \xi \right) \left\vert 1 - \sqrt{\frac{c_2}{c_1}} \right\vert \xi \right] \\
        \stackrel{(ii)}{\le} \, & \max_{\xi \in \cL^2(\Q), \norm{\xi}_{\Q}^2 \le c_1} L \left(\rho, \norm{\vmu}_2, R, \beta_0 \right) \left\vert 1 - \sqrt{\frac{c_2}{c_1}} \right\vert \E_{\Q} [\xi^2]^{1/2} = \, C \left(\rho, \norm{\vmu}_2, R, \beta_0 \right) \left\vert \sqrt{c_1} - \sqrt{c_2} \right\vert,
    \end{align*}
    where in $(i)$ we use our assumption that $\ell$ is pseudo-Lipschitz, and $(ii)$ follows from Cauchy--Schwarz inequality. This establishes the continuity of $\zeta(c)$.
    \end{enumerate}
\end{proof}
\end{comment}








\begin{comment}
\begin{proof}
    \begin{enumerate}
        \item[(a)] 
        When $\Q = \Q_\infty$ and $\ell \in C^2(\R)$, by Stein's Formula we can simplify
        \[
        \begin{aligned}
            \E \left[ \ell'\bigl( \prox_{ \lambda \ell}( \rho\norm{\vmu}_2 R + RG + \beta_0 ) \bigr) \right] & = - \frac{\rho R}{2\pi \lambda \delta \norm{\vmu}_2},
            \\
            \E \left[ \ell'\bigl( \prox_{ \lambda \ell}( \rho\norm{\vmu}_2 R + RG - \beta_0 ) \bigr) \right] & = - \frac{\rho R}{2(1 - \pi) \lambda \delta \norm{\vmu}_2},
            \\
            \E \left[ \frac{\ell''\bigl( \prox_{ \lambda \ell}( \rho\norm{\vmu}_2 R + RG + \beta_0 Y ) \bigr)}{1 + \lambda \ell''\bigl( \prox_{ \lambda \ell}( \rho\norm{\vmu}_2 R + RG + \beta_0 Y ) \bigr)} \right] & =  \frac{1}{\lambda \delta },
            \\
            \E \left[ \left(\ell'\bigl( \prox_{ \lambda \ell}( \rho\norm{\vmu}_2 R + RG + \beta_0 Y ) \bigr) \right)^2 \right] & = \frac{R^2 (1 - \rho^2)}{\lambda^2 \delta}.
        \end{aligned}
        \]
        Solve the system of equations for $(\rho, R, \beta_0, \lambda)$. 
    \end{enumerate}
\end{proof}
\end{comment}



% \noindent
% Now, recall that
% \begin{equation*}
%     \begin{aligned}
%         M_n^{(3)}(\bTheta_{c})
%     & = \min_{ \substack{ (\rho, R, \beta_0) \in \bTheta_{c} \\ \xi \in \cL^2(\Q_n), \norm{\xi}^2_{\Q_n} \le 1/\delta } } 
%     \E_{\Q_n} \left[ \ell \bigl( \rho\norm{\vmu}_2 R + RG + \beta_0 Y + R\sqrt{1 - \rho^2} \xi \bigr) \right], \\
%     M^{*}(\bTheta_{c})
%     & = \min_{ \substack{ (\rho, R, \beta_0) \in \bTheta_{c} \\ \xi \in \cL^2(\Q_\infty), \norm{\xi}^2_{\Q_\infty} \le 1/\delta } } 
%     \E \left[ \ell \bigl( \rho\norm{\vmu}_2 R + RG + \beta_0 Y + R\sqrt{1 - \rho^2} \bigr) \right].
%     \end{aligned}
% \end{equation*}
% Let $M^{*} := M^{*}( [-1, 1] \times \R_{\ge 0} \times \R ) = \mathscr{R}^*_{\Q_\infty}$ be the unconstrained asymptotic minimum value.

% \begin{lem}
%     For any compact subset $\bTheta_{c} \subset [-1, 1] \times \R_{\ge 0} \times \R$, as $n \to \infty$, we have
%     \begin{equation*}
%         M_n^{(3)}(\bTheta_{c}) \conp M^{*}(\bTheta_{c}).
%     \end{equation*}
%     In particular, if $\delta > \delta^*(0)$ and $\bTheta_{c} = [-1, 1] \times [0, C_R] \times [-C_{\beta_0}, C_{\beta_0}]$, then we have
%     \begin{equation*}
%         M_n^{(3)}(\bTheta_{c}) \conp M^{*},
%     \end{equation*}
%     for $C_R, C_{\beta_0} \in (0, \infty)$ large enough.
% \end{lem}
% \begin{proof}
%     For any $\lambda > 0$ and probability measure $\Q$, define
%     \begin{equation*}
%         \mathscr{R}_{\lambda, \Q}(\rho, R, \beta_0)
%         =
%         - \frac{R^2(1 - \rho^2)}{2 \delta \lambda}
%         +
%         \E_{\Q} \left[ \envelope_{\lambda\ell} ( \rho\norm{\vmu}_2 R + RG + \beta_0 Y ) \right]
%     \end{equation*}
%     Then according to {\color{red} Lemma B.8 (c)}, 
%     \begin{equation*}
%         M_n^{(3)}(\bTheta_{c}) =
%         \min_{ \substack{ (\rho, R, \beta_0) \in \bTheta_{c}\\ \lambda \ge 0 } } 
%         \mathscr{R}_{\lambda, \Q_n}(\rho, R, \beta_0),
%         \qquad
%         M^*(\bTheta_{c}) =
%         \min_{ \substack{ (\rho, R, \beta_0) \in \bTheta_{c}\\ \lambda \ge 0 } } 
%         \mathscr{R}_{\lambda, \Q_\infty}(\rho, R, \beta_0).
%     \end{equation*}
%     We have the following facts:
%     \begin{itemize}
%         \item $\bTheta_{c}$ is compact. $(\rho, R, \beta_0) \mapsto \envelope_{\lambda\ell} ( \rho\norm{\vmu}_2 R + RG + \beta_0 Y )$ is continuous in $\bTheta_{c}$ for each $(G, Y)$, and $(G, Y) \mapsto \envelope_{\lambda\ell} ( \rho\norm{\vmu}_2 R + RG + \beta_0 Y )$ is measurable for each $(\rho, R, \beta_0)$
%         \item {\color{blue}Uniform measurable}.
%     \end{itemize}
%     Therefore, by ULLN \cite[Lemma 2.4]{newey1994large}, for each $\lambda > 0$, we have
%     \begin{equation*}
%         \begin{aligned}
%             & \abs{\min_{ (\rho, R, \beta_0) \in \bTheta_{c} } 
%             \mathscr{R}_{\lambda, \Q_n}(\rho, R, \beta_0)
%             -
%             \min_{ (\rho, R, \beta_0) \in \bTheta_{c} } 
%             \mathscr{R}_{\lambda, \Q_\infty}(\rho, R, \beta_0)
%             }
%             \\
%             = {} & \sup_{(\rho, R, \beta_0) \in \bTheta_{c}} \, \bigl| 
%                 \E_{\Q_n} \left[ \envelope_{\lambda\ell} ( \rho\norm{\vmu}_2 R + RG + \beta_0 Y ) \right]
%                 -
%                 \E_{\Q_\infty} \left[ \envelope_{\lambda\ell} ( \rho\norm{\vmu}_2 R + RG + \beta_0 Y ) \right]
%             \bigr|
%             \conp 0.
%         \end{aligned}
%     \end{equation*}
%     Note that function $\lambda \mapsto \min_{ (\rho, R, \beta_0) \in \bTheta_{c} } 
%             \mathscr{R}_{\lambda, \Q_n}(\rho, R, \beta_0)$ is convex. Moreover, {\color{blue} (details)}
%     \begin{equation*}
%         \lim_{\lambda \to 0^+} \min_{ (\rho, R, \beta_0) \in \bTheta_{c} } 
%         \mathscr{R}_{\lambda, \Q_\infty}(\rho, R, \beta_0) \to -\infty.
%     \end{equation*}
%     Then we may use {\color{blue} [TAH18, Lem. 10]} to conclude that
%     \begin{equation*}
%         M_n^{(3)}(\bTheta_{c}) = \inf_{\lambda > 0} \min_{ (\rho, R, \beta_0) \in \bTheta_{c} } 
%         \mathscr{R}_{\lambda, \Q_n}(\rho, R, \beta_0)
%         \conp 
%         M^{*}(\bTheta_{c}) =  \inf_{\lambda > 0} \min_{ (\rho, R, \beta_0) \in \bTheta_{c} } 
%         \mathscr{R}_{\lambda, \Q_\infty}(\rho, R, \beta_0) .
%     \end{equation*}
%     {\color{blue} Lemma B.9} implies that $R^*, \abs{\beta_0^*} < \infty$, which completes the proof.
% \end{proof}









% \noindent
% We denote the system of equations \cref{eq:sys_eq_Q} as
% \[ \E_{\Q_n}\left[ V_i(\rho, R, \beta_0, \lambda) \right] = 0,
% \qquad i = 1, 2, 3, 4, \]
% for some functions $V_i: (0, 1) \times \R_{>0} \times \R \times \R_{> 0} \to \R$.

% \begin{lem}
%     For any probability measure $\Q$, recall the optimization problem $\mathscr{R}^*_{\Q}$ in \cref{eq:var_optim_Q}. Let $(\rho^*(\Q), R^*(\Q), \beta_0^*(\Q), \lambda^*(\Q))$ be the unique solution to \cref{eq:sys_eq_Q}. {\color{red} non-separable regime condition.}
%     \begin{enumerate}
%         \item[(a)] For $i = 1,2,3$, we have
%         \begin{equation*}
%             \plim_{n \to \infty} \sup_{(\rho, R, \beta_0, \lambda) \in \mathcal{S}} \abs{ \E_{\Q_n}\left[ V_i(\rho, R, \beta_0, \lambda) \right] - \E_{\Q_\infty}\left[ V_i(\rho, R, \beta_0, \lambda) \right] }
%              = 0.
%         \end{equation*}
%         \item[(b)] We have parameter convergence
%         \begin{equation*}
%             \plim_{n \to \infty} \, \bigl(\rho^*(\Q_n), R^*(\Q_n), \beta_0^*(\Q_n), \lambda^*(\Q_n)\bigr)
%              = \bigl( \rho^*(\Q_\infty), R^*(\Q_\infty), \beta_0^*(\Q_\infty), \lambda^*(\Q_\infty) \bigr).
%         \end{equation*}
%         As a consequence, $\mathscr{R}^*_{\Q_n} \conp \mathscr{R}^*_{\Q_\infty}$.
%     \end{enumerate}
% \end{lem}

% \begin{proof}
%     \begin{enumerate}
%         \item[(a)] $\mathcal{S} = [0, 1] \times [0, M] \times [-M, M] \times [0, M]$ is a compact set, (by previous arguments) we claim $R^*(\Q_n), \beta_0^*(\Q_n), \lambda^*(\Q_n)$ are asymptotically bounded {(\color{blue} existence of $M >0$)}.
        
%         If each $V_i$ is integrable uniformly over $(\rho, R, \beta_0, \lambda)$ (a strong sufficient condition: $\ell$ is Lipschitz, so $\abs{\ell'} \le c$), the result is followed by ULLN.
        
%         \item[(b)] Exactly the same approach in \cite{montanari2023generalizationerrormaxmarginlinear} page 45.
%     \end{enumerate}
% \end{proof}
\section{Margin rebalancing in proportional regime: Proofs for \cref{subsec:rebal_prop}}
\label{append_sec:mar_reb}

\subsection{Proofs of \cref{prop:Err-_mono} and \ref{prop:tau_mono}}

% We prove the monotonicity of $\rho^*$ and $\beta_0^*$ in this subsection by studying the system of equations in \cref{lem:gordon_eq}. We restate these equations here.

We show the monotonicity of $\Err_+^*$ for $\tau = 1$ in this subsection by first analyzing the monotonicity of asymptotic parameters $\rho^*, \beta_0^*$, which are the solution to the system of equations in \cref{lem:gordon_eq}. We restate these equations here.
\begin{subequations}
\begin{align}
    \label{eq:reb_sys_eq_rho}
        \pi \delta \cdot g \left( \frac{\rho}{2 \pi \norm{\bmu}_2 \delta} \right) & + (1 - \pi) \delta \cdot g \left( \frac{\rho}{2(1 - \pi) \norm{\bmu}_2 \delta} \right) = 1 - \rho^2, \\
    \label{eq:reb_sys_eq_bk1}
    - \beta_0 + \kappa \tau & = \rho \norm{\bmu}_2 + g_1^{-1} \left( \frac{\rho}{2 \pi \norm{\bmu}_2 \delta} \right), \\
    \label{eq:reb_sys_eq_bk2}
	\beta_0 + \kappa & = \rho \norm{\bmu}_2 + g_1^{-1} \left( \frac{\rho}{2 (1 - \pi) \norm{\bmu}_2 \delta} \right).
\end{align}
\end{subequations}
The properties of functions $g_1, g_2, g$ therein are summarized below.
\begin{lem}\label{lem:g1_g2_g} 
Recall $g_1 (x) = \E \left[ (G + x)_+ \right]$, $g_2 (x) = \E \left[ (G + x)_+^2 \right]$, and $g = g_2 \circ g_1^{-1}$.
\begin{enumerate}[label=(\alph*)]
    \item $g_1$, $g_2$ are increasing maps from $\R$ to $\R_{> 0}$, and $g: \R_{> 0} \to \R_{> 0}$ is increasing with $g(0^+) = 0$.  
    \item $g_1$, $g_2$ have explicit expressions
    \begin{equation*}
        g_1(x) = x \Phi(x) + \phi(x), \qquad g_2(x) = (x^2+1)\Phi(x) + x\phi(x). 
    \end{equation*}
    \item \label{lem:g1_g2_g_asymp} 
    $g_1(x) \sim x$, $g_2(x) \sim x^2$, and $g(x) \sim x^2$, as $x \to +\infty$.
\end{enumerate}
\end{lem}



% \noindent
% Next we show that $\rho^*$ is increasing in $\pi \in (0, \frac12)$, $\norm{\bmu}_2$, and $\delta$.
The following preliminary result gives the monotonicity of $\rho^*$. By \cref{thm:SVM_main}\ref{thm:SVM_main_var}, $\rho^* \in (0, 1)$ is invariant with respect to $\tau$. Hence $\rho^*$ can be viewed as a function of model parameters $(\pi, \norm{\bmu}_2, \delta)$ determined by \cref{eq:reb_sys_eq_rho}.
\begin{lem}[Monotonicity of $\rho^*$]\label{lem:rho_mono}
	% Let $\rho^*$ be the asymptotic cosine angle between $\bmu$ and $\hat\vbeta$ as defined in \cref{thm:SVM_main}. Then 
    $\rho^*$ is an increasing function of $\pi \in (0, \frac12)$, $\norm{\bmu}_2$, and $\delta$.
\end{lem}
\begin{proof}
% [\textbf{Proof of \cref{lem:rho_mono}}]
Recall that $\rho^* \in (0, 1)$ as stated in \cref{thm:SVM_main}\ref{thm:SVM_main_var}.

\vspace{0.5\baselineskip}
\noindent
\textbf{(a) $\boldsymbol{\uparrow}$ in $\norm{\vmu}_2$:}
This point is obvious from \cref{eq:reb_sys_eq_rho} and Lemma~\ref{lem:g1_g2_g}(a). 

\vspace{0.5\baselineskip}
\noindent
\textbf{(b) $\boldsymbol{\uparrow}$ in $\delta$:}
Notice that \cref{lem:g_monotone} implies $x \mapsto x \cdot g(1 / x)$ is decreasing in $x$. As a consequence, if we fix $\rho$ and increase $\delta$ on the L.H.S. of \cref{eq:reb_sys_eq_rho}, then the L.H.S. will decrease, and $\rho^*$ have to increase to match the R.H.S.. Therefore, $\rho^*$ is an increasing function of $\delta$. 

\vspace{0.5\baselineskip}
\noindent
\textbf{(c) $\boldsymbol{\uparrow}$ in $\pi \in (0, \frac12)$:}
We prove this using a similar strategy. Define
	\begin{equation*}
		x_1 = x_1 (\pi) := g_1^{-1} \left( \frac{\rho}{2 \pi \norm{\bmu}_2 \delta} \right), 
            \quad
            x_2 = x_2 (\pi) := g_1^{-1} \left( \frac{\rho}{2(1 - \pi) \norm{\bmu}_2 \delta} \right),
	\end{equation*}
	then we know that the L.H.S. of \cref{eq:reb_sys_eq_rho} (for fixed $\delta$ and $\norm{\bmu}_2$) is proportional to
	\begin{equation}\label{eq:rho_LHS_prop}
        \rho \cdot \left(
		\frac{g_2 (x_1(\pi))}{g_1 (x_1(\pi))} + \frac{g_2 (x_2(\pi))}{g_1 (x_2(\pi))}
        \right),
	\end{equation}
	with the only constraint on $x_1$ and $x_2$ being
	\begin{equation*}
		\frac{1}{g_1 (x_1(\pi))} + \frac{1}{g_1 (x_2(\pi))} = C := \frac{2 \norm{\bmu}_2 \delta}{\rho}.
        % C = C(\delta, \norm{\bmu}_2, \rho).
	\end{equation*}
	Taking derivative with respect to $\pi$, it follows that
	\begin{equation*}
		- \frac{g_1' (x_1)}{g_1^2 (x_1)} \cdot x_1'(\pi) - \frac{g_1' (x_2)}{g_1^2 (x_2)} \cdot x_2' (\pi) = 0, 
        \quad
        \Longrightarrow 
        \quad x_1' (\pi) = - \frac{g_1^2 (x_1)}{g_1' (x_1)} \cdot \frac{g_1' (x_2)}{g_1^2 (x_2)}  \cdot x_2'(\pi),
	\end{equation*}
	thus leading to
	\begin{align*}
		& \frac{\d}{\d \pi} \left( \frac{g_2 (x_1(\pi))}{g_1 (x_1(\pi))} + \frac{g_2 (x_2(\pi))}{g_1 (x_2(\pi))} \right) \\
		= {} & \frac{g_2'(x_1) g_1(x_1) - g_2(x_1) g_1'(x_1)}{g_1^2(x_1)} \cdot x_1'(\pi) + \frac{g_2'(x_2) g_1(x_2) - g_2(x_2) g_1'(x_2)}{g_1^2(x_2)} \cdot x_2'(\pi) \\
		= {} & - \frac{g_1' (x_2)}{g_1^2(x_2)} \cdot x_2'(\pi) \cdot \left( \frac{g_2'(x_1) g_1(x_1) - g_2(x_1) g_1'(x_1)}{g_1' (x_1)} - \frac{g_2'(x_2) g_1(x_2) - g_2(x_2) g_1'(x_2)}{g_1' (x_2)} \right) \\
            = {} & - \frac{g_1' (x_2)}{g_1^2(x_2)} \cdot x_2'(\pi) \cdot \left( h(x_1) - h(x_2) \right),
	\end{align*}
        where 
        \begin{equation*}
		h(x) := \frac{g_2'(x) g_1(x) - g_2(x) g_1'(x)}{g_1'(x)}, \quad \forall\, x \in \R
        \end{equation*}
        is a monotone increasing function according to the proof of \cref{lem:g_monotone}. Therefore, $h(x_1) > h(x_2)$ (since $\pi < 1/2 \implies x_1 > x_2$). By definitions of $x_2$ and $g_1$, we know that $x_2'(\pi) > 0$ and $g_1'(x_2) > 0$. As a consequence,
	\begin{equation*}
		\frac{\d}{\d \pi} \left( \frac{g_2 (x_1(\pi))}{g_1 (x_1(\pi))} + \frac{g_2 (x_2(\pi))}{g_1 (x_2(\pi))} \right) < 0.
	\end{equation*}
	Similar to points (a) and (b), by combining \cref{eq:reb_sys_eq_rho} and \eqref{eq:rho_LHS_prop}, we conclude that $\rho^*$ is an increasing function of $\pi \in (0, \frac12)$. This completes the proof.
\end{proof}




% \noindent
% Next we show that $\beta_0^*$ is increasing in $\pi \in (0, \frac12)$, $\norm{\bmu}_2$, and $\delta$ when $\tau = 1$.
As long as $\tau \not= 0$, the linear system \cref{eq:reb_sys_eq_bk1} and \eqref{eq:reb_sys_eq_bk2} for $(\beta_0, \tau)$ is non-singular, so one can solve for $\beta_0$ and $\kappa$:
\begin{subequations}
\begin{align}
	\label{eq:beta0_tau}
	\beta_0 = \, & \frac{1}{1 + \tau} \left( \tau g_1^{-1} \left( \frac{\rho}{2 (1 - \pi) \norm{\bmu}_2 \delta} \right) - g_1^{-1} \left( \frac{\rho}{2 \pi \norm{\bmu}_2 \delta} \right) + (\tau - 1) \rho \norm{\bmu}_2 \right), \\
	\label{eq:kappa_tau}
	\kappa = \, & \frac{1}{1 + \tau} \left( g_1^{-1} \left( \frac{\rho}{2 (1 - \pi) \norm{\bmu}_2 \delta} \right) + g_1^{-1} \left( \frac{\rho}{2 \pi \norm{\bmu}_2 \delta} \right) + 2 \rho \norm{\bmu}_2 \right).
\end{align}
\end{subequations}
The following lemma establishes the monotonicity of $\beta_0^*$ when $\tau = 1$. 
\begin{lem}[Monotonicity of $\beta_0^*$]\label{lem:beta0_mono}
	% Let $\beta_0^*$ be the asymptotic intercept as defined in \cref{thm:SVM_main} without rebalancing of margin ($\tau = 1$). Then 
    $\beta_0^*$ is an increasing function of $\pi \in (0, \frac12)$, $\norm{\bmu}_2$, and $\delta$, when $\tau = 1$ (without margin rebalancing). Moreover, $\beta_0^* < 0$.
\end{lem}
\begin{proof}
% [\textbf{Proof of \cref{lem:beta0_mono}}]
When $\tau = 1$, the above equations reduce to
\begin{align}
	\beta_0 = \, & \frac{1}{2} \left( g_1^{-1} \left( \frac{\rho}{2 (1 - \pi) \norm{\bmu}_2 \delta} \right) - g_1^{-1} \left( \frac{\rho}{2 \pi \norm{\bmu}_2 \delta} \right) \right), 
    \label{eq:beta0_tau=1}
    \\
	\kappa = \, & \frac{1}{2} \left( g_1^{-1} \left( \frac{\rho}{2 (1 - \pi) \norm{\bmu}_2 \delta} \right) + g_1^{-1} \left( \frac{\rho}{2 \pi \norm{\bmu}_2 \delta} \right) + 2 \rho \norm{\bmu}_2 \right).
    \notag
\end{align}
Clearly $\beta_0^* < 0$, since $g_1^{-1}$ is an increasing function and $\pi < \frac12$.

\vspace{0.5\baselineskip}
\noindent
\textbf{(a) $\boldsymbol{\uparrow}$ in $\norm{\mu}_2$:}
Fixing $\pi$ and $\delta$, taking derivative with respect to $\norm{\bmu}_2$ in \cref{eq:beta0_tau=1}, we have
	\begin{equation*}
		\frac{\d \beta_0}{\d \norm{\bmu}_2} = \frac{1}{2} \left( \frac{1}{2 (1 - \pi) \delta} \cdot (g_1^{-1})' \left( \frac{\rho}{2 (1 - \pi) \norm{\bmu}_2 \delta} \right) - \frac{1}{2 \pi \delta} \cdot (g_1^{-1})' \left( \frac{\rho}{2 \pi \norm{\bmu}_2 \delta} \right) \right) \cdot \frac{\d}{\d \norm{\bmu}_2} \left( \frac{\rho}{\norm{\bmu}_2} \right).
	\end{equation*}
	Since $\pi < \frac12$, from \cref{lem:g_prime_monotone} we know that
	\begin{equation*}
		\frac{1}{2 (1 - \pi) \delta} \cdot (g_1^{-1})' \left( \frac{\rho}{2 (1 - \pi) \norm{\bmu}_2 \delta} \right) - \frac{1}{2 \pi \delta} \cdot (g_1^{-1})' \left( \frac{\rho}{2 \pi \norm{\bmu}_2 \delta} \right) < 0.
	\end{equation*}
	According to \cref{lem:rho_mono}, if we increase $\norm{\bmu}_2$, then $\rho$ will increase, and \cref{eq:reb_sys_eq_rho} implies that $\rho / \norm{\bmu}_2$ will decrease. Hence,
	\begin{equation*}
		\frac{\d}{\d \norm{\bmu}_2} \left( \frac{\rho}{\norm{\bmu}_2} \right) < 0.
	\end{equation*}
	Combining the above inequalities, we know that $\d \beta_0 / \d \norm{\bmu}_2 > 0$.

\vspace{0.5\baselineskip}
\noindent
\textbf{(b) $\boldsymbol{\uparrow}$ in $\delta$:}
Similarly, according to \cref{eq:reb_sys_eq_rho} and \cref{lem:rho_mono}, for fixed $\pi$ and $\norm{\bmu}_2$, we can show that $\rho / \delta$ will decrease if $\delta$ increases. By same approach as (a), we can conclude $\d \beta_0 / \d \delta > 0$.

\vspace{0.5\baselineskip}
\noindent
\textbf{(c) $\boldsymbol{\uparrow}$ in $\pi \in (0, \frac12)$:}
Lastly, we note that if $\pi \in (0, \frac12)$ increases, then $1 - \pi$ will decrease and $\rho$ will increase. According to \cref{lem:g_monotone}, we know that
\begin{equation*}
    \frac{(1 - \pi) \delta}{\rho} \cdot g \left( \frac{\rho}{2(1 - \pi) \norm{\bmu}_2 \delta} \right)
\end{equation*}
will increase. Since $(1 - \rho^2)/\rho$ will decrease, combining with \cref{eq:reb_sys_eq_rho}, we can show that
\begin{equation*}
    \frac{\pi \delta}{\rho} \cdot g \left( \frac{\rho}{2 \pi \norm{\bmu}_2 \delta} \right)
\end{equation*}
will decrease. By \cref{lem:g_monotone} again, we conclude that $\rho / (1 - \pi)$ will increase and $\rho / \pi$ will decrease, which implies that $\beta_0$ \cref{eq:beta0_tau=1} will increase. This completes the proof.
\end{proof}


The monotonicity of minority error is a direct consequence of the two lemmas above.
\begin{proof}[\textbf{Proof of \cref{prop:Err-_mono}}]
When $\tau = 1$, according to \cref{lem:rho_mono} and \ref{lem:beta0_mono}, both $\rho^*$ and $\beta_0^*$ are increasing in $\pi \in (0, \frac12)$, $\norm{\bmu}_2$, and $\delta$. We complete the proof by $\Err_{+}^* = \Phi \left(- \rho^* \norm{\bmu}_2 - \beta_0^* \right)$.
\end{proof}



Now we fix model parameters $\pi \in (0, \frac12)$, $\delta$, $\norm{\bmu}_2$, and consider test errors as functions of $\tau$. In order to prove \cref{prop:tau_mono}, we need the following result on the monotonicity of $\rho^*, \beta_0^*$ on $\tau$.
\begin{lem}[Dependence of $\tau$]\label{lem:tau_mono}
    % Let $(\rho^*, \beta_0^*, \kappa^*)$ be defined as per \cref{thm:SVM_main}. 
    Fix $\pi \in (0, \frac12)$, $\norm{\bmu}_2$, and $\delta$, then we have
    \begin{enumerate}[label=(\alph*)]
        \item \label{lem:tau_mono_rho} $\rho_0^*$ does not depend on $\tau$.
        \item \label{lem:tau_mono_beta0} $\beta_0^*$ is an increasing function of $\tau \in (0, \infty)$.
	\item \label{lem:tau_mono_kappa} $\kappa^*$ is a decreasing function of $\tau \in (0, \infty)$.
    \end{enumerate}
    As a consequence, $\Err^*_+$ is decreasing in $\tau \in (1, \infty)$, and $\Err^*_-$ is increasing in $\tau \in (1, \infty)$.
\end{lem}
\begin{proof}
    \ref{lem:tau_mono_rho} is already proved in \cref{thm:SVM_main}\ref{thm:SVM_main_var}. For \ref{lem:tau_mono_kappa}, the conclusion is followed by \cref{eq:kappa_tau}, since $\kappa^* \propto (1 + \tau)^{-1}$. For \ref{lem:tau_mono_beta0}, note that $\beta_0^* + \kappa^*$ is a fixed value according to \cref{eq:reb_sys_eq_bk2}. Then by using \ref{lem:tau_mono_kappa}, we conclude $\beta_0^*$ is increasing in $\tau$. This concludes the proof.
\end{proof}

These are consistent with the non-asymptotic monotonicity between $(\hat\rho, \hat\beta_0, \hat\kappa)$ and $\tau$ in \cref{prop:SVM_tau_relation}. Then the monotonicity of test errors is a direct consequence of \cref{lem:tau_mono}.


\begin{proof}[\textbf{Proof of \cref{prop:tau_mono}}]
According to \cref{lem:tau_mono}\ref{lem:tau_mono_rho}\ref{lem:tau_mono_beta0}, we know that $-\rho^* \norm{\bmu}_2 + \beta_0^*$ is increasing in $\tau$ and $-\rho^* \norm{\bmu}_2 - \beta_0^*$ is decreasing in $\tau$. This completes the proof.
\end{proof}







\subsection{Proofs of \cref{prop:tau_optimal} and \ref{prop:Err_monotone}}
\label{subsec:tau_optimal}

\begin{proof}[\textbf{Proof of \cref{prop:tau_optimal}}]
    Recall that
    \begin{equation*}
        \Err_\mathrm{b}^* = \frac12 \Bigl( 
        \Phi\left(- \rho^* \norm{\bmu}_2 - \beta_0^* \right) + \Phi\left(- \rho^* \norm{\bmu}_2  + \beta_0^* \right)
        \Bigr).
    \end{equation*}
    Notice that $\rho^*$ does not depend on $\tau$, and $\rho^*\norm{\bmu}_2 > 0$. We first show that $\tau = \tau^\mathrm{opt}$ if and only if $\beta_0^* = 0$. Then is suffices to show that for any fixed $a > 0$, function
    \begin{equation*}
        f(x) := \Phi(-a + x) + \Phi(-a - x), \qquad x \in \R
    \end{equation*}
    has unique minimizer $x = 0$. This is true by observing $f'(x) = \phi(-a + x) - \phi(-a - x) < 0$ for all $x < 0$, and $f'(x) > 0$ for all $x > 0$. Hence we conclude $\beta_0^* = 0$ and $\Err_+^* = \Err_-^* = \Err_\mathrm{b}^*$.

    Setting $\beta_0 = 0$ in \cref{eq:beta0_tau} and solving for $\tau$, we get \cref{eq:tau_opt}. This completes the proof.
\end{proof}

As stated in \cref{rem:tau_pi}, when $\norm{\bmu}_2$, $\delta$ are fixed and $\pi$ is small, the numerator of $\tau^\mathrm{opt}$ scales as $\sqrt{1/\pi}$. We formally prove this in the following lemma.

\begin{lem}
    When $\pi = o(1)$, we have
    \begin{equation*}
        g_1^{-1} \left( \dfrac{\rho^*}{2 \pi \norm{\bmu}_2 \delta} \right) + \rho^* \norm{\bmu}_2  \sim  \frac{1}{\sqrt{\pi \delta}}.
    \end{equation*}
\end{lem}
\begin{proof}
    By \cref{lem:rho_mono}, $\rho^*$ is monotone increasing in $\pi \in (0, \frac12)$. It can be easily shown that $\rho^* \to 0$ as $\pi \to 0$. Otherwise, suppose $\rho^* \to \underline{\rho} > 0$ as $\pi \to 0$, then by \cref{lem:g1_g2_g}\ref{lem:g1_g2_g_asymp} 
\[ 
    \pi \delta \cdot g \left( \frac{\rho^*}{2 \pi \norm{\bmu}_2 \delta} \right)
    \sim  \pi \delta \cdot \left( \frac{\underline{\rho}}{2 \pi \norm{\bmu}_2 \delta} \right)^2 \propto \frac{1}{\pi} \to \infty,
\]
while the other terms in \cref{eq:reb_sys_eq_rho} are all finite, which is a contradiction. Substitute $\rho^* \to 0$ into \cref{eq:reb_sys_eq_rho},
\[ 
g \left( \frac{\rho^*}{2 \pi \norm{\bmu}_2 \delta} \right) \sim \frac{1}{\pi\delta} \to \infty
\qquad \Longrightarrow \qquad
\frac{\rho^*}{2 \pi \norm{\bmu}_2 \delta} \sim \frac{1}{\sqrt{\pi\delta}}.
\]
The proof is complete by using \cref{lem:g1_g2_g}\ref{lem:g1_g2_g_asymp} again.
\end{proof}
\begin{rem}
    We notice that when $\pi$ is very small or $\norm{\bmu}_2$, $\delta$ are very large, then $\rho^*$ is close to $0$ and the denominator of $\tau^\mathrm{opt}$ can be zero or negative, leading $\tau^\mathrm{opt}$ infinity of negative. According to \cref{fig:SVM_cartoon}, this happens when the optimal decision boundary (the red solid line) falls on or under the margin of majority class (the black dashed line below with negative support vectors). In such cases, we have $\tau < -1$ and the training error for majority class is nonzero.

    Actually, our theory remains valid when $\tau < -1$. When $\tau < -1$, one can modify the objective of \cref{eq:SVM-m-reb} to minimizing $\kappa$ (since $\kappa < 0$ and $\tau \kappa > 0$), then the relation \cref{eq:margin-balance} in \cref{prop:SVM_tau_relation} still holds. For the asymptotic problem, one can similarly modify the variational problem \cref{eq:SVM_variation}. Then one may extend \cref{thm:SVM_main} to negative $\tau$ by relating \cref{eq:margin-balance} to \eqref{eq:asymp_tau_relation}, where \cref{eq:asymp_tau_relation} is derived from \cref{eq:reb_sys_eq_rho}---\eqref{eq:reb_sys_eq_bk2}, which also admits a unique solution when $\tau < -1$.
\end{rem}

Finally, prove the monotonicity of test errors after margin rebalancing.

\begin{proof}[\textbf{Proof of \cref{prop:Err_monotone}}]
    According to \cref{prop:tau_optimal}, $\Err_+^* = \Err_-^* = \Err_\mathrm{b}^* = \Phi(- \rho^* \norm{\bmu}_2 )$. Since $\rho^*$ is increasing in $\pi \in (0, \frac12)$, $\norm{\bmu}_2$, and $\delta$ by \cref{lem:rho_mono}, the proof is complete.
\end{proof}


% \ljy{
% \begin{lem}
% Properties:
%     \begin{enumerate}
%         \item[(a)] $ g_1(x) = x \Phi(x) + \phi(x)$ and $g_2(x) = (x^2+1)\Phi(x) + x\phi(x)$. 
%         \item[(b)] Some asymptotic results. \ljycom{Please check.}
%         \begin{equation*}
%         \renewcommand{\arraystretch}{2}
%             \begin{array}{rcccrcccl}
%                 g_1(x) & \!\!\!\! \sim \!\!\!\! & x 
%                 &  &
%                 g_2(x) & \!\!\!\! \sim \!\!\!\! & x^2
%                 &  &
%                 \text{as $x \to \infty$} 
%                 \\
%                 g_1(x) & \!\!\!\! \sim \!\!\!\! & \dfrac{\phi(x)}{x^2} 
%                 &  &
%                 g_2(x) & \!\!\!\! \sim \!\!\!\! & -\dfrac{2\phi(x)}{x^3}
%                 &  &
%                 \text{as $x \to -\infty$}
%                 \\
%                 g_1^{-1}(x) & \!\!\!\! \sim \!\!\!\! & -\sqrt{-2 \log x \vphantom{x^2}}
%                 &  &
%                 g_2^{-1}(x) & \!\!\!\! \sim \!\!\!\! & -\sqrt{-2 \log x \vphantom{x^2}}
%                 &  &
%                 \text{as $x \downarrow 0$}
%                 \\
%                 g(x) & \sim & x^2 & & & & & &  \text{as $x \to \infty$}
%                 \\
%                 g(x) & \sim & \dfrac{x}{2\sqrt{\mathrm{\pi}} (-\log x)^{3/2}} & & & & & &  \text{as $x \downarrow 0$}
%             \end{array}
%         \end{equation*}

%     \end{enumerate}
% \end{lem}
% }

% \ljy{When $\pi \to 0$...

% Recall \cref{eq:reb_sys_eq_rho},
% \begin{equation*}
% 	\pi \delta \cdot g \left( \frac{\rho^*}{2 \pi \norm{\bmu}_2 \delta} \right) + (1 - \pi) \delta \cdot g \left( \frac{\rho^*}{2(1 - \pi) \norm{\bmu}_2 \delta} \right) = 1 - \rho^{*2}.
% \end{equation*}
% By \cref{lem:rho_mono}, $\rho^*$ is monotone increasing in $\pi \in (0, \frac12)$. It can be easily shown that $\rho^* \to 0$ as $\pi \to 0$. Otherwise, suppose $\rho^* \to \underline{\rho} > 0$ as $\pi \to 0$, then
% \[ 
%     \pi \delta \cdot g \left( \frac{\rho^*}{2 \pi \norm{\bmu}_2 \delta} \right)
%     \sim  \pi \delta \cdot \left( \frac{\underline{\rho}}{2 \pi \norm{\bmu}_2 \delta} \right)^2 \propto \frac{1}{\pi} \to \infty,
% \]
% while the other terms in \cref{eq:reb_sys_eq_rho} are finite, which is a contradiction. Substitute $\rho^* \to 0$ into \cref{eq:reb_sys_eq_rho},
% \[ 
% g \left( \frac{\rho^*}{2 \pi \norm{\bmu}_2 \delta} \right) \sim \frac{1}{\pi\delta} \to \infty
% \qquad \Rightarrow \qquad
% \frac{\rho^*}{2 \pi \norm{\bmu}_2 \delta} \sim \frac{1}{\sqrt{\pi\delta}}.
% \]
% However, note that $g_1^{-1}(0^+) = -\infty$ and
% \[ 
% g_1^{-1} \left( \dfrac{\rho^*}{2 (1 - \pi) \norm{\bmu}_2 \delta} \right) \to -\infty.
% \]
% This leads to the denominator of $\tau^\mathrm{opt}$ in \cref{eq:tau_opt} to be negative ($-\infty$), while the numerator to be positive ($+\infty$). Then $\tau^\mathrm{opt} < 0$ when $\pi$ is small enough.


% Try to fix it...

% We know that $\beta_0^*$ is increasing in $\tau$ by \cref{lem:tau_mono}. According to the formula of $\beta_0^*$ in \cref{eq:beta0_tau},
% \[
% \lim_{\tau \to \infty} \beta_0^* = g_1^{-1} \left( \dfrac{\rho^*}{2 (1 - \pi) \norm{\bmu}_2 \delta} \right) + \rho^* \norm{\bmu}_2,
% \]
% which is the denominator of $\tau^\mathrm{opt}$ in \cref{eq:tau_opt}. When $\pi$ is small enough such that the R.H.S. of above is negative, we cannot find any $\tau \ge 1$ such that $\beta_0^* = 0$.
% \paragraph{Claim} (Modified \cref{prop:tau_opt}) There exist a threshold $\pi_c = \pi_c(\norm{\bmu}_2, \delta)$, such that
%     \[
%     \tau^\mathrm{opt} =  \argmin_{\tau \ge 1} \Err_b
%     = \begin{dcases}
%         \dfrac{g_1^{-1} \left( \dfrac{\rho^*}{2 \pi \norm{\bmu}_2 \delta} \right) + \rho^* \norm{\bmu}_2}{g_1^{-1} \left( \dfrac{\rho^*}{2 (1 - \pi) \norm{\bmu}_2 \delta} \right) + \rho^* \norm{\bmu}_2}
%         & \text{if}~ \pi \in (\pi_c, \tfrac12) \\
%         \infty & \text{if}~ \pi \in (0, \pi_c]
%     \end{dcases}
%     \]
% \paragraph{Q:} Does it make sense if we choose a negative $\tau$? (TODO more simulation)
% }


\subsection{Technical lemmas}

Some technical results used in the proof are summarized below.

\begin{lem}\label{lem:g_monotone}
The function $g_2(x) / g_1(x)$ is increasing in $x$. This implies $g(x) / x$ is increasing in $x$, and $x \cdot g(1/x)$ is decreasing in $x$.
\end{lem}
\begin{proof}
	By direct calculation, we have
	\begin{align*}
		g_2'(x) g_1(x) - g_2(x) g_1'(x) = 2 \left( \E[(G + x)_+] \right)^2 - \Phi(x) \E[(G+x)_+^2].
	\end{align*}
	It suffices to show that
	\begin{equation*}
		h(x) := \frac{2 \left( \E[(G + x)_+] \right)^2}{\Phi(x)} - \E[(G+x)_+^2] > 0, \qquad  \forall\, x \in \R.
	\end{equation*}
	To this end, note that $\lim_{x \to - \infty} h(x) = 0$, and that
	\begin{equation*}
		h'(x) = 2 \E [(G + x)_+] \left( 1 - \frac{\E [(G + x)_+] \phi(x)}{\Phi(x)^2} \right).
	\end{equation*}
	Hence, one only need to show that $h'(x) > 0$, $\forall \, x \in \R$, namely
	\begin{equation*}
		r(x) := \frac{\Phi(x)^2}{\phi(x)} - \E [(G + x)_+] > 0.
	\end{equation*}
	Notice again that $\lim_{x \to -\infty} r(x) = 0$, and
	\begin{equation*}
		r'(x) = \Phi(x) \left( 1 + \frac{x \Phi(x)}{\phi(x)} \right) > 0
	\end{equation*}
	by Mill's ratio, thus we finally conclude that $r (x) > 0$ for any $x \in \R$. Consequently, $g_2 (x) / g_1 (x)$ is increasing in $x$.

    By change of variable $y = g_1(x)$, we show that $g_2 (x)/g_1 (x) = g(y)/y$ is increasing in $y$.
\end{proof}





\begin{lem}\label{lem:g_prime_monotone}
	The function $x \mapsto x \cdot (g_1^{-1})' (x)$ is monotone increasing.
\end{lem}
\begin{proof}
	Let $x = g_1 (y)$, then we know that
	\begin{equation*}
		x \cdot (g_1^{-1})' (x) = \frac{g_1 (y)}{g_1' (y)}.
	\end{equation*}
	Since $y$ is increasing in $x$, it suffices to show that $g_1 (y) / g_1' (y)$ is increasing in $y$. Note that
	\begin{equation*}
		\frac{\d}{\d y} \left( \frac{g_1 (y)}{g_1' (y)} \right) = \frac{g_1' (y)^2 - g_1 (y) g_1'' (y)}{g_1' (y)^2} = \frac{\phi(y) r(y)}{g_1' (y)^2},
	\end{equation*}
	where the function $r(y)$ is defined in the proof of \cref{lem:g_monotone}, and we know that $r (y) > 0$ for all $y \in \R$. Therefore, $g_1 (y) / g_1' (y)$ is increasing. This completes the proof. 
\end{proof}
\section{Margin rebalancing in high imbalance regime: Proof of \cref{thm:main_high-imbal}}
\label{append_sec:high_imb}

Without loss of generality, we may consider the following case as a substitute of \cref{setup-high-imbalance}:
\begin{equation*}
\pi = d^{-a}, \qquad \norm{\vmu}_2^2 = d^b, \qquad n = d^{c+1}.
\end{equation*}
% The notations commonly used are summarized in the table below. Be aware that some are only used in the proof of this section and may have different meanings in other sections.

% \begin{table}[h!]
%     \centering
%     \label{tab:high_imb_notation}
%     \renewcommand{\arraystretch}{1.25}
%     \begin{tabular}{>{\centering\arraybackslash}m{0.125\linewidth}|>{\raggedright\arraybackslash}m{0.65\linewidth}|>{\centering\arraybackslash}m{0.12\linewidth}}
%         \hline\hline
%         \textbf{Symbol} & \textbf{Description} & \textbf{Definition} \\ 
%         \hline\hline
%         $\vbeta, \beta_0$ & 
%         the slope and intercept parameters for arbitrary linear classifier $f(\xx) = \< \xx, \vbeta \> + \beta_0$ ($\norm{\vbeta}_2 = 1$) &  \\ 
%         \hline
%         $\rho, \vtheta$ & 
%         reparametrization of $\vbeta$ via orthogonal projections $\bP_{\vmu}$ and $\bP_{\vmu}^\perp$ & \eqref{eq:def-rho-theta} \\ 
%         \hline
%         $\kappa_i, \kappa_i(\vbeta, \beta_0), $ $\kappa_i(\rho, \vtheta, \beta_0)$ & 
%         the (scaled) logit margin of $f(\xx) = \< \xx, \vbeta \> + \beta_0$ ($\norm{\vbeta}_2 = 1$) for $(\xx_i, y_i)$, expressed in parameters $(\vbeta, \beta_0)$ or $(\rho, \vtheta, \beta_0)$ & \eqref{eq:logits} \\
%         \hline
%         $\kappa, \kappa(\vbeta, \beta_0), $ $\kappa(\rho, \vtheta, \beta_0)$ & 
%         the margin of $f(\xx) = \< \xx, \vbeta \> + \beta_0$ ($\norm{\vbeta}_2 = 1$) for data $(\XX, \yy)$, expressed in parameters $(\vbeta, \beta_0)$ or $(\rho, \vtheta, \beta_0)$  & \eqref{eq:margin}, \eqref{eq:margin_reparam} \\
%         \hline
%         $\hat\vbeta, \hat\beta_0$ & 
%         the slope and intercept for max-margin linear classifier, the optimal solution to \cref{eq:SVM-m-reb} and \cref{eq:SVM} &  \\ 
%         \hline
%         $\hat\rho, \hat\vtheta$ & 
%         reparametrization of $\hat\vbeta$ via orthogonal projections $\bP_{\vmu}$ and $\bP_{\vmu}^\perp$, the optimal solution to \cref{eq:SVM-rho_theta}  & \eqref{eq:def-rho-theta_hat} \\ 
%         \hline
%         $\hat\kappa, \kappa(\hat\vbeta, \hat\beta_0), $ $\kappa(\hat\rho, \hat\vtheta, \hat\beta_0)$ & 
%         the maximum margin for data $(\XX, \yy)$, 
%         the optimal objective value of \cref{eq:SVM}, 
%         the optimal solution to \cref{eq:SVM-m-reb} and \eqref{eq:SVM-rho_theta},
%         can be expressed in $(\hat\vbeta, \hat\beta_0)$ or $(\hat\rho, \hat\vtheta, \hat\beta_0)$ & \eqref{eq:max-margin}, \eqref{eq:max-margin1} \\
%         \hline
%         $\wt\rho, \wt\vtheta, \wt\beta_0$ & 
%         a constructed solution with explicit expression, a good ``proxy'' for the max-margin solution $\hat\rho, \hat\vtheta, \hat\beta_0$  & \eqref{eq:param_star} \\ 
%         \hline
%         $\bar\kappa$ & 
%         a data-dependent (stochastic) upper bound on the maximum margin $\hat\kappa$ for data $(\XX, \yy)$  & \eqref{eq:kappa_upper} \\ 
%         \hline
%         $\check\beta_0, \check\beta_0(\vbeta)$ & the optimal intercept for a linear classifier with slope $\vbeta$, which satisfies margin-balancing condition \cref{eq:margin-bal} or \eqref{eq:svm_sv_bal}
%          & \eqref{eq:beta0_optim} \\ 
%         \hline
%         $\mathsf{sv}_+(\vtheta)$, $\mathsf{sv}_-(\vtheta)$\phantom{,} & the indices of the smallest $\kappa_i$ in each class, which is known as support vectors, depends on $\vtheta$ and data (with $\rho = \hat\rho$ fixed)  
%         & \eqref{eq:SV_def_theta} \\ 
%         \hline
%         $\mathsf{v}_+(\vtheta)$, $\mathsf{v}_-(\vtheta)$\phantom{,} & the indices of the smallest $y_i\< \zz_i, \vtheta \>$ in each class, a ``proxy'' of support vectors, depends on $\vtheta$ and data (with $\rho = \hat\rho$ fixed)  
%         & \eqref{eq:V_def_theta} \\ 
%         \hline
%         $\wt\vtheta_+, \wt\vtheta_-$ & 
%         statistics analogous to $\wt\vtheta$, only used in the proof of \cref{lem:theta_hat_z}
%         & \eqref{eq:wt_theta_pm} \\
%         \hline\hline
%     \end{tabular}
%     \caption{List of commonly used notations in \cref{append_sec:high_imb}.}
% \end{table}

\noindent
Consider a linear classifier based on $f(\xx) = \< \xx, \vbeta \> + \beta_0$ with $\norm{\vbeta}_2 = 1$. Denote projection matrices
\begin{equation*}
    \bP_{\vmu} := \frac{1}{\norm{\vmu}_2^2} \vmu \vmu^\top,
    \qquad 
    \bP_{\vmu}^\perp := \bI_d - \frac{1}{\norm{\vmu}_2^2} \vmu \vmu^\top,
\end{equation*}
where $\bP_{\vmu}$ is the orthogonal projection onto $\spann\{ \vmu \}$ and $\bP_{\vmu}^\perp$ is the orthogonal projection onto the orthogonal complement of $\spann\{ \vmu \}$. Then we define auxiliary parameters
\begin{equation}\label{eq:def-rho-theta}
    \rho := \left\< \vbeta, \frac{\vmu}{\norm{\vmu}_2} \right\>,
    \qquad
    \vtheta := 
    \begin{cases} 
        \, \dfrac{\bP_{\vmu}^\perp \vbeta}{\|\bP_{\vmu}^\perp \vbeta\|_2}
        = \dfrac{\bP_{\vmu}^\perp \vbeta}{\sqrt{1 - \rho^2}} , & \ \text{if} \ \abs{\rho} < 1, \\
        \, \vmu_\perp,         & \ \text{if} \ \abs{\rho} = 1, 
    \end{cases}
\end{equation}
where $\vmu_\perp \in \S^{d-1}$ is some deterministic vector such that $\vmu_\perp \perp \vmu$.
Therefore, we have the following decomposition:
\begin{equation*}
    \vbeta = \bP_{\vmu} \vbeta + \bP_{\vmu}^\perp \vbeta 
    = \rho\frac{\vmu}{\norm{\vmu}_2} + \sqrt{1 - \rho^2} \vtheta.
\end{equation*}
Note that $\norm{\vtheta}_2 = 1$, $\vtheta \perp \vmu$, and there exists a one-to-one correspondence\footnote{
    In fact, this one-to-one mapping $\vbeta \mapsto (\rho, \vtheta)$ is restricted to $\S^{d-1} \to \Theta_{\rho,\vtheta}$, where the range is $\Theta_{\rho,\vtheta} :=  \{(\rho, \vtheta): \rho \in (-1, 1), \norm{\vtheta}_2 = 1, \vtheta \perp \vmu \} \cup \{ (\rho, \vtheta): \rho = \pm1, \vtheta = \vmu_\perp \} $. However, for simplicity, we can expand the parameter space of $(\rho, \vtheta)$ into $\{(\rho, \vtheta): \rho \in [-1, 1], \norm{\vtheta}_2 = 1, \vtheta \perp \vmu \}$. This is because if $\rho = \pm 1$, we have $\bP_{\vmu}^\perp \vbeta = \bzero$, and $\sqrt{1 - \rho^2} \vtheta = \bzero$ for any $\vtheta$. We will see that $\vtheta$ always appears in the form of $\sqrt{1 - \rho^2} \vtheta$ (for example, in the decomposition of $\vbeta$, and the expression of $\kappa_i$ and $\kappa$). That also explains why we can take $\vmu_\perp$ arbitrarily in \cref{eq:def-rho-theta}.
} between $\vbeta$ and $(\rho, \vtheta)$. Therefore, the logit margin of $f(\xx)$ for the $i$-th data point $(\xx_i, y_i)$ can be reparametrized as
\begin{align}
        \kappa_i  & =  \kappa_i(\vbeta, \beta_0) 
        : = \wt y_i ( \< \xx_i, \vbeta \> + \beta_0 )  \nonumber \\
        & = s_i y_i \left( \Bigl\< y_i \bmu + \zz_i, \rho\frac{\vmu}{\norm{\vmu}_2} + \sqrt{1 - \rho^2}\vtheta  \Bigr\>  + \beta_0 \right) \nonumber \\
        & = s_i \left( \rho \norm{\vmu}_2 + y_i\beta_0 + \rho y_i g_i + \sqrt{1 - \rho^2} y_i \< \zz_i, \vtheta \> \right) 
        = : \kappa_i(\rho, \vtheta, \beta_0),
        \label{eq:logits}
\end{align}
where $\zz_i \sim \subGind(\bzero, \bI_n; K)$ according to \cref{def:subgauss}, $K > 0$ is some absolute constant, and
\begin{equation*}
    s_i := \begin{cases} 
    \, \tau^{-1}, & \ \text{if} \ y_i = +1, \\
    \, 1,         & \ \text{if} \ y_i = -1, \end{cases}
\qquad
    g_i := \left\langle \zz_i, \frac{\vmu}{\norm{\vmu}_2} \right\rangle,
\end{equation*}
where $\vg := (g_1, \dots, g_n)^\top \sim \subGind(\bzero, \bI_n; K)$ by \cref{lem:subG}\ref{lem:subG-b}. Therefore, the margin (in \cref{eq:margin}) of $f(\xx)$ can be viewed as function $(\vbeta, \beta_0) \mapsto \kappa$ or $(\rho, \vtheta, \beta_0) \mapsto \kappa$ based on different parametrization:
\begin{equation}\label{eq:margin_reparam}
    \begin{aligned}
        \kappa & = \mathmakebox[\widthof{$\kappa(\rho, \vtheta, \beta_0)$}][l]{\kappa(\vbeta, \beta_0)} = \min_{i \in [n]} \kappa_i(\vbeta, \beta_0) \\
        & = \kappa(\rho, \vtheta, \beta_0) = \min_{i \in [n]} \kappa_i(\rho, \vtheta, \beta_0).
    \end{aligned}
\end{equation}
As a consequence, the max-margin optimization problem \cref{eq:SVM} or \eqref{eq:SVM-m-reb} can be expressed as
\begin{equation}
	\label{eq:SVM-rho_theta}
    \begin{array}{rl}
    \maximize\limits_{ \rho, \beta_0 \in \R, \vtheta \in \R^{d} } & % \kappa(\rho, \vtheta, \beta_0) = 
    % \min\limits_{i \in [n]} s_i \bigl( \rho \norm{\vmu}_2 + y_i\beta_0 + \rho y_i g_i + \sqrt{1 - \rho^2} y_i \< \zz_i, \vtheta \> \bigr) = : 
    \kappa(\rho, \vtheta, \beta_0), \\
    \text{subject to} 
	& \rho \in [-1, 1],  
    \vphantom{\maximize\limits_{ \rho }}  \\ 
        & \norm{\vtheta}_2 = 1,  \ \  
    \vtheta \perp \vmu,
    \end{array}
\end{equation}
where
\begin{equation*}
    \kappa(\rho, \vtheta, \beta_0) = 
    \min\limits_{i \in [n]} s_i \left( \rho \norm{\vmu}_2 + y_i\beta_0 + \rho y_i g_i + \sqrt{1 - \rho^2} y_i \< \zz_i, \vtheta \> \right).
\end{equation*}
Recall that $(\hat\vbeta$, $\hat\beta_0)$ is the max-margin solution to \cref{eq:SVM}, and the maximum margin is given by
\begin{equation}\label{eq:max-margin}
    \hat\kappa = \kappa(\hat\vbeta, \hat\beta_0) = \min_{i \in [n]} \kappa_i(\hat\vbeta, \hat\beta_0).
\end{equation}
Similarly, we can also reparametrize $\hat\vbeta$ as in \cref{eq:def-rho-theta}:
\begin{equation}\label{eq:def-rho-theta_hat}
    \hat\rho := \left\< \hat\vbeta, \frac{\vmu}{\norm{\vmu}_2} \right\>,
    \qquad
    \hat\vtheta := 
    \begin{cases} 
        \, \dfrac{\bP_{\vmu}^\perp \hat\vbeta}{\|\bP_{\vmu}^\perp \hat\vbeta\|_2}
        = \dfrac{\bP_{\vmu}^\perp \hat\vbeta}{\sqrt{1 - \hat\rho^2}} , & \ \text{if} \ |\hat\rho| < 1, \\
        \, \vmu_\perp ,         & \ \text{if} \ |\hat\rho| = 1.
    \end{cases}
\end{equation}
Then, $(\hat\rho, \hat\vtheta, \hat\beta_0)$ is the optimal solution to \cref{eq:SVM-rho_theta}\footnote{
    According to \cref{eq:def-rho-theta_hat} and \cref{prop:SVM_tau_relation}, for linearly separable data, $(\hat\rho, \hat\beta_0)$ is the unique solution to \cref{eq:SVM-rho_theta}. If $|\hat\rho| < 1$, then $\hat\vtheta$ is also the unique solution to \cref{eq:SVM-rho_theta}. Otherwise, if $\hat\rho = \pm 1$, then $\sqrt{1 - \hat\rho^2} y_i \< \zz_i, \vtheta \> \equiv 0$ and thus any feasible $\vtheta$ could solve \cref{eq:SVM-rho_theta}.
}. Combining \cref{eq:margin_reparam} and \eqref{eq:max-margin}, the maximum margin can be rewritten as
\begin{equation}\label{eq:max-margin1}
    \hat\kappa = \kappa(\hat\rho, \hat\vtheta, \hat\beta_0) = \min_{i \in [n]} \kappa_i(\hat\rho, \hat\vtheta, \hat\beta_0),
\end{equation}
which is also the optimal objective value of \cref{eq:SVM-rho_theta}. Finally, we define a few quantities:
\begin{gather*}
    \bar g_{+} :=  \frac{1}{n_+}\sum_{i \in \mathcal{I}_+} g_i, 
    \qquad
    \bar g_{-} :=  \frac{1}{n_-}\sum_{i \in \mathcal{I}_-} g_i, 
    \qquad
    \wt g := \frac{\bar g_{+} - \bar g_{-}}{2},
    \\
    \bar \zz_{+} := \frac{1}{n_+}\sum_{i \in \mathcal{I}_+} \zz_i,     
    \qquad
    \bar \zz_{-} := \frac{1}{n_-}\sum_{i \in \mathcal{I}_-} \zz_i,
    \qquad
    \wt\zz  := \frac{\bar\zz_{+} - \bar\zz_{-}}{2}.
\end{gather*}
The proof structure of \cref{thm:main_high-imbal} is as follows:
\begin{enumerate}
    \item In \cref{subsec:highimb_upper}, we provide a (stochastic) tight upper bound for the maximum margin $\hat\kappa$, and a constructed solution $(\wt\rho, \wt\vtheta, \wt\beta_0)$ which approximates $(\hat\rho, \hat\vtheta, \hat\beta_0)$ well.
    \item In \cref{subsec:highimb_asymp}, we derive the asymptotic orders of $(\hat\rho, \hat\vtheta, \hat\beta_0)$ by using $(\wt\rho, \wt\vtheta, \wt\beta_0)$.
    \item In \cref{subsec:highimb_err}, we use these asymptotics to analyze test errors and conclude \cref{thm:main_high-imbal}.
\end{enumerate}




\subsection{A tight upper bound on maximum margin: Proof of \cref{lem:upper_bound}}
\label{subsec:highimb_upper}

The following Lemma provides a data-dependent upper bound on the margin $\kappa(\vbeta, \beta_0)$ which holds for all linear classifiers with $\norm{\vbeta}_2 = 1$. The bound is tight in sense that it can be (almost) achieved by a constructed solution. Therefore, such tightness ensures the optimal margin $\hat\kappa$ should have the same asymptotics given by its upper bound, which also deduces the data is linearly separable with probability tending to one (as $d \to \infty$). 
Notably, \cref{prop:SVM_tau_relation} implies that $\tau$ has no effect on $\hat\vbeta$, and $\hat\kappa \propto (1 + \tau)^{-1}$ in a fixed dataset. Hence, $\tau$ simply scales the magnitude of $\hat\kappa$, and it suffices to consider $\tau = 1$ in the following lemma.
\begin{lem} \label{lem:upper_bound}
    Fix $\tau = 1$. Denote
\begin{equation} \label{eq:kappa_upper}
    \bar\kappa := \sqrt{ ( \norm{\vmu}_2 + \wt g )^2 + \| \bP_{\vmu}^\perp \wt\zz \|_2^2 }.
\end{equation}
    \begin{enumerate}[label=(\alph*)]
        \item 
        \label{lem:upper_bound_a}
        (Upper bound) $\kappa(\rho, \vtheta, \beta_0) \le \bar\kappa$, for any $\rho \in [-1, 1]$, $\vtheta \in \S^{d-1}$, $\vtheta \perp \vmu$, $\beta_0 \in \R$. Moreover,
        \begin{equation*}
            \bar\kappa = \bigl(1 + o_\P(1)\bigr) \sqrt{  d^{b} +  \frac{1}{4} d^{a-c} },
        \end{equation*}
        as $d \to \infty$.
        \item
        \label{lem:upper_bound_b}
        (Tightness) $\kappa(\wt\rho, \wt\vtheta, \wt\beta_0) \ge \bar\kappa - \wt O_{\P}(1)$, where
        \begin{equation} \label{eq:param_star}
            \begin{gathered}
                \wt\rho := \dfrac{ \norm{\vmu}_2 + \wt g }{\sqrt{ ( \norm{\vmu}_2 + \wt g )^2 + \| \bP_{\vmu}^\perp \wt\zz \|_2^2 } },
                \qquad
                \wt\vtheta := \frac{\bP_{\vmu}^\perp \wt\zz}{\| \bP_{\vmu}^\perp \wt\zz \|_2},
                \\
                \wt\beta_0 := - \wt\rho \cdot \frac{\bar g_+ + \bar g_-}{2} - \sqrt{1 - \wt\rho^2} \cdot \left\< \frac{\bar\zz_{+} + \bar\zz_{-}}{2}, \wt\vtheta \right\>
            \end{gathered}
        \end{equation}
        is a feasible solution to \cref{eq:SVM-rho_theta}.
        \item
        \label{lem:upper_bound_c}
        (Asymptotics of $\hat\kappa$) As a consequence, the data is linearly separable with high probability, and the maximum margin satisfies
        $\bar\kappa - \wt O_{\P}(1) \le \hat\kappa \le \bar\kappa$.
    \end{enumerate}
\end{lem}
\begin{proof}
    \textbf{\ref{lem:upper_bound_a}:}
    % It suffices to consider $\norm{\vbeta}_2 = 1$ (otherwise, in the linearly separable case, any classifier with $\norm{\vbeta}_2 < 1$ must satisfy $\kappa < \hat\kappa$; or, in the non-separable case, we always have $\kappa \le \hat\kappa = 0$). 
    We reparametrize $\kappa(\vbeta, \beta_0) = \kappa(\rho, \vtheta, \beta_0)$ by using \cref{eq:def-rho-theta} and \eqref{eq:logits}.
    Then, the upper bound is established by calculating the \emph{average logit margin} for each class. Let
    \begin{equation}
            \label{eq:kappa_pm}
            \begin{aligned}
                \bar\kappa_{+}(\rho, \vtheta, \beta_0) & := \frac{1}{n_+}\sum_{i \in \mathcal{I}_+} \kappa_i(\rho, \vtheta, \beta_0)  =
               \rho \norm{\vmu}_2 + \beta_0 + \rho \bar g_{+} + \sqrt{1 - \rho^2} \< \bar\zz_{+}, \vtheta \>,
                \\
                \bar\kappa_{-}(\rho, \vtheta, \beta_0) & := \frac{1}{n_-}\sum_{i \in \mathcal{I}_-} \kappa_i(\rho, \vtheta, \beta_0)  =
                \rho \norm{\vmu}_2 - \beta_0 - \rho \bar g_{-} - \sqrt{1 - \rho^2} \< \bar\zz_{-}, \vtheta \>.
            \end{aligned}
    \end{equation}
    Clearly, $\kappa(\rho, \vtheta, \beta_0) \le \bar\kappa_+(\rho, \vtheta, \beta_0)$ and $\kappa(\rho, \vtheta, \beta_0) \le \bar\kappa_-(\rho, \vtheta, \beta_0)$. By averaging these two bounds,
    \begin{align}
        \kappa(\rho, \vtheta, \beta_0) & \le \frac{\bar\kappa_{+}(\rho, \vtheta, \beta_0) + \bar\kappa_{-}(\rho, \vtheta, \beta_0)}{2} \nonumber \\
        & = 
        \rho \norm{\vmu}_2 + \rho \cdot \frac{\bar g_+ - \bar g_-}{2} + \sqrt{1 - \rho^2} \cdot \left\< \frac{\bar\zz_{+} - \bar\zz_{-}}{2}, \vtheta \right\>
        \nonumber \\
        & = \rho \left( \norm{\vmu}_2 + \wt g \right) + \sqrt{1 - \rho^2} \left< \wt\zz, \vtheta \right\> 
        \nonumber \\
        & \overset{\mathmakebox[0pt][c]{\text{(i)}}}{\le}  \rho \left( \norm{\vmu}_2 +  \wt g \right) + \sqrt{1 - \rho^2}  \| \bP_{\vmu}^\perp \wt\zz \|_2 
        \label{eq:F_AB} \\
        & \overset{\mathmakebox[0pt][c]{\text{(ii)}}}{\le} \sqrt{ ( \norm{\vmu}_2 + \wt g )^2 + \| \bP_{\vmu}^\perp \wt\zz \|_2^2 }
        = \bar\kappa,
        \nonumber
\end{align}
which leads to $\bar\kappa$ defined in \cref{eq:kappa_upper}. Here, (i) is based on the fact that
\begin{equation}\label{eq:theta_min}
    \argmax_{ \vtheta \in \R^d :  \substack{ \| \vtheta \|_2 = 1 \\ \langle \vmu, \vtheta  \rangle = 0 } } \ \langle \wt\zz, \vtheta \rangle = \frac{\bP_{\vmu}^\perp \wt\zz}{\| \bP_{\vmu}^\perp \wt\zz \|_2},
    \qquad
    \max_{ \vtheta \in \R^d :  \substack{ \| \vtheta \|_2 = 1 \\ \langle \vmu, \vtheta  \rangle = 0 } }   \langle \wt\zz, \vtheta \rangle = \| \bP_{\vmu}^\perp \wt\zz \|_2,
\end{equation}
and recall that the optimal $\vtheta$ equals $\wt\vtheta$ defined in \cref{eq:param_star}. Moreover, (ii) is a consequence of Cauchy--Schwarz inequality ($A \in \R$, $B > 0$)
\begin{equation} \label{eq:F_AB_optim}
    \begin{aligned}
        \max_{\rho \in [-1, 1]} \left\{  \rho A + \sqrt{1 - \rho^2} B  \right\}
    & = \max_{\rho \in [-1, 1]} \left\< \begin{pmatrix}
        \rho \\ \sqrt{1-\rho^2}
    \end{pmatrix}, 
    \begin{pmatrix}
        A \\ B
    \end{pmatrix} \right\> = \sqrt{A^2 + B^2}, \\
    \argmax_{\rho \in [-1, 1]} \left\{  \rho A + \sqrt{1 - \rho^2} B \right\}
    & = \frac{A}{\sqrt{A^2 + B^2}},
    \end{aligned}
\end{equation}
and also note that the optimal $\rho$ in (ii) equals $\wt\rho$ defined in \cref{eq:param_star}.

~\\
\noindent
To study the asymptotics of $\bar\kappa$, recall that $\pi = n_+/n = o(1)$, $n_- = n - n_+ = n(1 - o(1))$. Then
\begin{equation*}
    \frac{1}{n_+} + \frac{1}{n_-} = \frac{1}{\pi n } + \frac{1}{n(1 - o(1))} = \frac{1}{\pi n}\bigl(1 + o(1)\bigr).
\end{equation*}
Denote
\begin{equation}\label{eq:alpha_d}
    \alpha_d := \frac12 \sqrt{ \frac{1}{n_+} + \frac{1}{n_-} }
    = \frac{1}{2\sqrt{\pi n}}\bigl(1 + o(1)\bigr).
\end{equation}
\cref{lem:subG}\ref{lem:subG-b} implies $\wt\zz/\alpha_d \sim \subGind(\bzero, \bI_d; K)$. Then according to \cref{lem:subG_concentrate}\ref{lem:subG-Hanson-Wright-II},
\begin{equation*}
    \P\biggl( \biggl| \frac{\| \bP_{\vmu}^\perp \wt\zz \|_2}{ \alpha_d \| \bP_{\vmu}^\perp \|_\mathrm{F}} - 1 \biggr| > t \biggr)
    \le 2 \exp\biggl( -\frac{ct^2}{K^4}\frac{\| \bP_{\vmu}^\perp \|_\mathrm{F}^2}{\| \bP_{\vmu}^\perp \|_{\mathrm{op}}^2} \biggr)
    = 2 \exp\biggl( -\frac{ct^2(d-1)}{K^4} \biggr) ,
\end{equation*}
where $\| \bP_{\vmu}^\perp \|_\mathrm{F} = \sqrt{d - 1}$, $\| \bP_{\vmu}^\perp \|_{\mathrm{op}} = 1$, and $c$ is an absolute constant. Therefore,
\begin{equation} \label{eq:P_wtz}
    \| \bP_{\vmu}^\perp \wt\zz \|_2 = \alpha_d \| \bP_{\vmu}^\perp \|_\mathrm{F} \bigl(1 + o_\P(1)\bigr)
    = \frac{1}{2\sqrt{\pi n}}\big(1 + o(1) \big) \cdot \sqrt{d - 1}\bigl(1 + o_\P(1)\bigr) 
    = \frac12 \sqrt{\frac{d}{\pi n}}\big(1 + o_\P(1) \big).
\end{equation}
In addition, by \cref{lem:subG_concentrate}\ref{lem:subG-Hoeffding},
\begin{equation*}
    % \|\wt\zz\|^2_2 = \alpha_d^2 d \bigl(1 + o_\P(1)\bigr) = \frac{d}{4\pi n} \bigl(1 + o_\P(1)\bigr),
    % \qquad
    \wt g = O_{\P}(\alpha_d) = O_\P\biggl( \frac{1}{\sqrt{\pi n}} \biggr).
\end{equation*}
Recall that $a - c - 1 < 0$. Finally, we have
\begin{align}
        \bar\kappa 
    & = \sqrt{ ( \norm{\vmu}_2 + \wt g )^2 + \| \bP_{\vmu}^\perp \wt\zz \|_2^2 } \nonumber \\
    & = \sqrt{ \left( \norm{\vmu}_2 + O_\P(1/\sqrt{\pi n}) \right)^2 + \frac{d}{4\pi n}\bigl(1 + o_\P(1)\bigr) }
    \nonumber \\
    & =  \sqrt{ \left( d^{b/2} + O_\P\bigl(d^{(a-c-1)/2}\bigr) \right)^2 + \frac{1}{4} d^{a - c} \bigl(1 + o_\P(1)\bigr) }
    \nonumber \\
    & = \sqrt{ d^{b} +  \frac{1}{4} d^{a-c} } \bigl(1 + o_\P(1)\bigr).
    \label{eq:margin_upper}
\end{align}
This concludes the proof of part \ref{lem:upper_bound_a}.

\vspace{0.5\baselineskip}
\noindent
\textbf{\ref{lem:upper_bound_b}:}
Next we show that the upper bound $\bar\kappa$ is nearly attainable, by a constructed solution $(\wt\rho, \wt\vtheta, \wt\beta_0)$ defined in \cref{eq:param_star}. Clearly, $(\wt\rho, \wt\vtheta, \wt\beta_0)$ satisfies the constraints in \cref{eq:SVM-rho_theta}. This candidate solution is motivated by the optimal $(\rho, \vtheta)$ that makes (i) and (ii) equal in \cref{eq:F_AB}, i.e.,
\begin{equation*}
    \bar\kappa = \wt\rho \left(\norm{\vmu}_2 +  \wt g \right) + \sqrt{1 - \wt\rho^2} \< \wt\zz, \wt\vtheta \>, 
\end{equation*}
and $\beta_0$ that balances the magnitude of average logit margins from the two classes, i.e., we choose $\beta_0$ such that $\bar\kappa_+ = \bar\kappa_-$ in \cref{eq:kappa_pm}. Substituting $(\wt\rho, \wt\vtheta, \wt\beta_0)$ back into \cref{eq:logits}, we obtain
\begin{equation*}
        \begin{aligned}
        \kappa_i(\wt\rho, \wt\vtheta, \wt\beta_0) & = \wt\rho \norm{\vmu}_2 + y_i\wt\beta_0 + \wt\rho y_i g_i + \sqrt{1 - \wt\rho^2} y_i \< \zz_i, \wt\vtheta \> \\
        & = 
        \wt\rho \left( \norm{\vmu}_2 + y_i g_i - y_i\frac{\bar g_{+} + \bar g_{-}}{2} \right) + \sqrt{1 - \wt\rho^2} \left\< y_i \zz_i - y_i \frac{\bar\zz_{+} + \bar\zz_{-}}{2}, \wt\vtheta \right\>.
    \end{aligned}
\end{equation*}
Therefore, the difference between each logit margin and the upper bound can be expressed as
\begin{align}
        \bar\kappa - \kappa_i(\wt\rho, \wt\vtheta, \wt\beta_0)
        & = \wt\rho \left( \wt g + y_i\frac{\bar g_{+} + \bar g_{-}}{2} - y_i g_i \right) + \sqrt{1 - \wt\rho^2} \left\< \wt\zz + y_i \frac{\bar\zz_{+} + \bar\zz_{-}}{2} - y_i \zz_i , \wt\vtheta \right\>  \nonumber  \\
        & = 
        \begin{cases} 
            \,  \wt\rho (\bar g_+ - g_i) +  \sqrt{1 - \wt\rho^2} \< \bar\zz_+ - \zz_i, \wt\vtheta \> , & \ \text{if} \ y_i = +1, \\
            \,  \wt\rho (g_i - \bar g_-) +  \sqrt{1 - \wt\rho^2} \< \zz_i - \bar\zz_-, \wt\vtheta \> , & \ \text{if} \ y_i = -1, \end{cases}        \label{eq:logit_diff}
\end{align}
where the leading terms $\rho\norm{\vmu}_2$, $\< \bar\zz_-, \wt\vtheta\>$ (for $i = +1$), $\< \bar\zz_+, \wt\vtheta\>$ (for $i = -1$) are all cancelled out. 
Our goal is to bound the maximum difference over all data points. Note that
\begin{equation}\label{eq:max_diff}
    \begin{aligned}
        & \max_{i \in [n]} \abs{\bar\kappa - \kappa_i(\wt\rho, \wt\vtheta, \wt\beta_0)}
        = 
        \max_{i \in \mathcal{I}_+} \abs{\bar\kappa - \kappa_i(\wt\rho, \wt\vtheta, \wt\beta_0)}
         \vee
        \max_{i \in \mathcal{I}_-} \abs{\bar\kappa - \kappa_i(\wt\rho, \wt\vtheta, \wt\beta_0)}  \\
        \le {} & 
        \max_{i \in \mathcal{I}_+} \left\{ \bigl|g_i - \bar g_+\bigr| + \bigl|\< \zz_i - \bar\zz_+, \wt\vtheta \>\bigr| \right\}
        \vee
        \max_{i \in \mathcal{I}_-} \left\{ \bigl|g_i - \bar g_-\bigr| + \bigl|\< \zz_i - \bar\zz_-, \wt\vtheta \>\bigr| 
        \right\} 
        \\
        \le {} & \left\{ 
            \max_{i \in \mathcal{I}_+} \bigl|g_i - \bar g_+\bigr|
            + \max_{i \in \mathcal{I}_+} \bigl|\< \zz_i - \bar\zz_+, \wt\vtheta \>\bigr|
        \right\}
        \vee
        \left\{ 
            \max_{i \in \mathcal{I}_-} \bigl|g_i - \bar g_-\bigr|
            + \max_{i \in \mathcal{I}_-} \bigl|\< \zz_i - \bar\zz_-, \wt\vtheta \>\bigr|
        \right\}.
    \end{aligned}
\end{equation}
For the first term involving $g_i$'s, recall that $\max_{i \in [n]}\norm{g_i}_{\psi_2} \lesssim K$. Therefore, as per \cref{lem:subG}\ref{lem:subG-c} and \cref{lem:subG_concentrate}\ref{lem:subG-Hoeffding}, $g_i, \bar g_\pm$ are sub-gaussian, and
\begin{equation}\label{eq:max_g}
    \begin{aligned}
        \max_{i \in \mathcal{I}_+} \bigl|g_i - \bar g_+\bigr|
        & \le \max_{i \in \mathcal{I}_+} \abs{g_i} + \abs{\bar g_+} 
        = O_\P(\sqrt{\log n_+}) + O_\P\biggl( \frac{1}{\sqrt{n_+}} \biggr)
        = O_\P(\sqrt{\log d}), \\
        \max_{i \in \mathcal{I}_-} \bigl|g_i - \bar g_-\bigr|
        & \le \max_{i \in \mathcal{I}_-} \abs{g_i} + \abs{\bar g_-} 
        = O_\P(\sqrt{\log n_-}) + O_\P\biggl( \frac{1}{\sqrt{n_-}} \biggr)
        = O_\P(\sqrt{\log d}).
    \end{aligned}
\end{equation}
For the second term involving $\zz_i$'s, note that
\begin{equation}\label{eq:maxmax_z}
    \begin{aligned}
        \max_{i \in \mathcal{I}_+} \bigl| \< \zz_i - \bar\zz_+, \wt\vtheta \> \bigr|
        & \le \max_{i \in \mathcal{I}_+} \frac{1}{n_+} \sum_{j \in \mathcal{I}_+} \bigl| \< \zz_i - \zz_j, \wt\vtheta \> \bigr|
        \le \frac{1}{\| \bP_{\vmu}^\perp \wt\zz \|_2} \max_{i,j \in \mathcal{I}_+} \bigl| \< \zz_i - \zz_j, \bP_{\vmu}^\perp \wt\zz \>  \bigr|,  \\
        \max_{i \in \mathcal{I}_-} \bigl| \< \zz_i - \bar\zz_-, \wt\vtheta \> \bigr|
        & \le \max_{i \in \mathcal{I}_-} \frac{1}{n_-} \sum_{j \in \mathcal{I}_-} \bigl| \< \zz_i - \zz_j, \wt\vtheta \> \bigr|
        \le \frac{1}{\| \bP_{\vmu}^\perp \wt\zz \|_2} \max_{i, j \in \mathcal{I}_-} \bigl| \< \zz_i - \zz_j, \bP_{\vmu}^\perp \wt\zz \>  \bigr|.
    \end{aligned}
\end{equation}
So it remains to bound $\< \zz_i - \zz_j, \bP_{\vmu}^\perp \wt\zz \>$ uniformly. We decompose it as
\begin{equation*}
    \< \zz_i - \zz_j, \bP_{\vmu}^\perp \wt\zz \>
    = 
    \< \zz_i - \zz_j, \wt\zz \> - \left\< \zz_i - \zz_j, \frac{\vmu}{\norm{\vmu}_2} \right\>
            \left\< \wt\zz, \frac{\vmu}{\norm{\vmu}_2} \right\>
    := I -  I\!I.
\end{equation*}
We will show that both $I$ and $I\!I$ are sub-exponential. To bound $\norm{I}_{\psi_1}$ via \cref{lem:subExp}\ref{lem:subExp-b}, we claim the inner product
\begin{equation*}
    I = \< \zz_i - \zz_j, \wt\zz \> = \sum_{k=1}^d (\zz_i - \zz_j)_k (\wt\zz)_k
\end{equation*}
is the sum of $d$ mean-zero random variables, i.e., $\E[(\zz_i - \zz_j)_k (\wt\zz)_k] = 0$, $\forall\, k \in [d]$, where we write $(\ba)_k$ as the $k$-th entry of vector $\ba$. To see this, we decompose $\wt\zz$ into terms that are independent or dependent of $(\zz_i, \zz_j)$.
\begin{itemize}
    \item If $y_i = y_j = +1$ and $i \not= j$, then
    \begin{align*}
        \E[ (\zz_i - \zz_j) \odot \wt\zz ]
        & =  \E\biggl[(\zz_i - \zz_j) \odot 
        \biggl( \underbrace{  \frac1{2 n_+} \!\!\! \sum_{k: \substack{k \not= i, j \\ y_k = +1}} \zz_k + \frac{\bar\zz_-}2  }_{ =: \wt\zz_{-ij}^+ }
        \biggr)
        \biggr] 
        + \frac{1}{2 n_+} \E[ (\zz_i - \zz_j) \odot (\zz_i + \zz_j) ] \\
        & = \E[\zz_i - \zz_j] \odot \E[ \wt\zz_{-ij}^+ ]
        + \frac{1}{2 n_+} \left( \E[\zz_i \odot \zz_i] - \E[\zz_j \odot \zz_j] \right) 
        \qquad (\wt\zz_{-ij}^+  \indep  \zz_i, \zz_j) \\
        & = \bzero \odot \bzero + \frac{1}{2 n_+} ( \bone - \bone ) = \bzero.
    \end{align*}
    \item If $y_i = y_j = -1$ and $i \not= j$, similarly
    \begin{align*}
        \E[ (\zz_i - \zz_j) \odot \wt\zz ]
        & =  \E\biggl[(\zz_i - \zz_j) \odot 
        \biggl( \underbrace{  \frac1{2 n_-} \!\!\! \sum_{k: \substack{k \not= i, j \\ y_k = -1}} \zz_k + \frac{\bar\zz_+}2  }_{ =: \wt\zz_{-ij}^- }
        \biggr)
        \biggr] 
        + \frac{1}{2 n_-} \E[ (\zz_i - \zz_j) \odot (\zz_i + \zz_j) ] \\
        & = \E[\zz_i - \zz_j] \odot \E[ \wt\zz_{-ij}^- ]
        + \frac{1}{2 n_-} \left( \E[\zz_i \odot \zz_i] - \E[\zz_j \odot \zz_j] \right) 
        \qquad (\wt\zz_{-ij}^-  \indep  \zz_i, \zz_j) 
        \\
        & = \bzero.
    \end{align*}
\end{itemize}
Therefore, when $d$ is large enough, we have
\begin{align*}
    \norm{I}_{\psi_1} & = \norm{\< \zz_i - \zz_j, \wt\zz \>}_{\psi_1}
    = \norm{ \sum_{k=1}^d (\zz_i - \zz_j)_k (\wt\zz)_k }_{\psi_1} \\
    & \overset{\mathmakebox[0pt][c]{\smash{\text{(i)}}}}{\lesssim} \sqrt{d} \max_{1 \le k \le d} \norm{ (\zz_i - \zz_j)_k (\wt\zz)_k }_{\psi_1} \\
    & \overset{\mathmakebox[0pt][c]{\smash{\text{(ii)}}}}{\le} \sqrt{d} \max_{1 \le k \le d} \norm{ (\zz_i - \zz_j)_k }_{\psi_2}
    \max_{1 \le k \le d} \norm{ (\wt\zz)_k }_{\psi_2} \\
    & \overset{\mathmakebox[0pt][c]{\smash{\text{(iii)}}}}{\lesssim} \sqrt{d} K \cdot \alpha_{d} K 
    \lesssim \sqrt{\frac{d}{\pi n}} K^2,
\end{align*}
where (i) results from coordinate independence and \cref{lem:subExp}\ref{lem:subExp-b}, (ii) is from \cref{lem:subExp}\ref{lem:subExp-d}, and (iii) is based on $\wt\zz/\alpha_{d} , (\zz_i - \zz_j)/\sqrt{2} \sim \subGind(\bzero, \bI_d; K)$. For the term $I\!I$, we have
\begin{align*}
    \norm{I\!I}_{\psi_2} & = \norm{\left\< \zz_i - \zz_j, \frac{\vmu}{\norm{\vmu}_2} \right\>
    \left\< \wt\zz, \frac{\vmu}{\norm{\vmu}_2} \right\>}_{\psi_1} \\
    & \overset{\mathmakebox[0pt][c]{\smash{\text{(i)}}}}{\le} \norm{\left\< \zz_i - \zz_j, \frac{\vmu}{\norm{\vmu}_2} \right\>}_{\psi_2}
    \norm{\left\< \wt\zz, \frac{\vmu}{\norm{\vmu}_2} \right\>}_{\psi_2} \\
    & \overset{\mathmakebox[0pt][c]{\smash{\text{(ii)}}}}{\lesssim}  \max_{1 \le k \le d} \norm{ (\zz_i - \zz_j)_k }_{\psi_2}
    \max_{1 \le k \le d} \norm{ (\wt\zz)_k }_{\psi_2} \\
    & \lesssim K \cdot \alpha_{d} K 
    \lesssim \frac{1}{\sqrt{\pi n}} K^2,
\end{align*}
where (i) is from \cref{lem:subExp}\ref{lem:subExp-d}, and (ii) is from \cref{lem:subG}\ref{lem:subG-b}. Hence,
\begin{equation*}
    \norm{\frac{\< \zz_i - \zz_j, \bP_{\vmu}^\perp \wt\zz \>}{\sqrt{d/\pi n}}}_{\psi_1}
    \le \sqrt{\frac{\pi n}{d}} \bigl( \norm{I}_{\psi_1} + \norm{I\!I}_{\psi_1} \bigr)
    \lesssim K^2.
\end{equation*}
Substituting this back into \cref{eq:maxmax_z}, referring to \cref{eq:P_wtz} and \cref{lem:subExp}\ref{lem:subExp-c}, we obtain
\begin{align}
    \!\!\! \max_{i \in \mathcal{I}_+} \bigl| \< \zz_i - \bar\zz_+, \wt\vtheta \> \bigr|
        & \le \frac{1}{\| \bP_{\vmu}^\perp \wt\zz \|_2} \max_{\substack{i \in \mathcal{I}_+ \\ j \in \mathcal{I}_+}} \bigl| \< \zz_i - \zz_j, \bP_{\vmu}^\perp \wt\zz \>  \bigr|
    = \bigl(1 + o_\P(1)\bigr) \max_{\substack{i \in \mathcal{I}_+ \\ j \in \mathcal{I}_+}} \abs{\frac{\< \zz_i - \zz_j, \bP_{\vmu}^\perp \wt\zz \>}{\sqrt{d/\pi n}}}
    \notag \\
    & =  O_{\P}(\log n_+^2) = O_{\P}(\log d), \notag \\
    \!\!\! \max_{i \in \mathcal{I}_-} \bigl| \< \zz_i - \bar\zz_-, \wt\vtheta \> \bigr|
        & \le \frac{1}{\| \bP_{\vmu}^\perp \wt\zz \|_2} \max_{\substack{i \in \mathcal{I}_- \\ j \in \mathcal{I}_-}} \bigl| \< \zz_i - \zz_j, \bP_{\vmu}^\perp \wt\zz \>  \bigr|
    = \bigl(1 + o_\P(1)\bigr) \max_{\substack{i \in \mathcal{I}_- \\ j \in \mathcal{I}_-}} \abs{\frac{\< \zz_i - \zz_j, \bP_{\vmu}^\perp \wt\zz \>}{\sqrt{d/\pi n}}} \notag \\
    & =  O_{\P}(\log n_-^2) = O_{\P}(\log d).
    \label{eq:max_z}
\end{align}
Finally, incorporating \cref{eq:max_g} and \cref{eq:max_z} into \cref{eq:max_diff}, we have
\begin{align*}
        & \max_{i \in [n]} \abs{\bar\kappa - \kappa_i(\wt\rho, \wt\vtheta, \wt\beta_0)}
        \\
        \le {} & \left\{ 
            \max_{i \in \mathcal{I}_+} \bigl|g_i - \bar g_+\bigr|
            + \max_{i \in \mathcal{I}_+} \bigl|\< \zz_i - \bar\zz_+, \wt\vtheta \>\bigr|
        \right\}
        \vee
        \left\{ 
            \max_{i \in \mathcal{I}_-} \bigl|g_i - \bar g_-\bigr|
            + \max_{i \in \mathcal{I}_-} \bigl|\< \zz_i - \bar\zz_-, \wt\vtheta \>\bigr|
        \right\} \\
        \le {} & \left\{ O_\P(\sqrt{\log d}) + O_\P(\log d) \right\} \vee \left\{ O_\P(\sqrt{\log d}) + O_\P(\log d) \right\} 
        = O_\P(\log d).
\end{align*}
%%%%%%%%%%%%%%%%%%%%%%%%%%%%%%%%%%%%%%%%%%%%%%%%%%%%%%%%%%%%%%%%%%%%%%%%%%%%%%%%%%%
% Old ver.: split \wt\zz into indep parts (maybe useful in non-indep coordinates) %
%%%%%%%%%%%%%%%%%%%%%%%%%%%%%%%%%%%%%%%%%%%%%%%%%%%%%%%%%%%%%%%%%%%%%%%%%%%%%%%%%%%
% \begin{itemize}
%     \item If $y_i = y_j = +1$ and $i \not= j$, we have the following decomposition:
%     \begin{equation*}
%         \begin{aligned}
%             & \< \zz_i - \zz_j, \bP_{\vmu}^\perp \wt\zz \>
%             = \< \zz_i - \zz_j, \wt\zz \> - \left\< \zz_i - \zz_j, \frac{\vmu}{\norm{\vmu}_2} \right\>
%             \left\< \wt\zz, \frac{\vmu}{\norm{\vmu}_2} \right\> \\
%             = {} & \biggl\< \zz_i - \zz_j, 
%             \underbrace{  \frac1{2 n_+} \!\!\! \sum_{k: \substack{k \not= i, j \\ y_k = +1}} \zz_k + \frac{\bar\zz_-}2  }_{ =: \wt\zz_{-ij}^+ }
%             \biggr\>
%             + \frac{1}{2 n_+} \< \zz_i - \zz_j, \zz_i + \zz_j \>
%             - \left\< \zz_i - \zz_j, \frac{\vmu}{\norm{\vmu}_2} \right\>
%             \left\< \wt\zz, \frac{\vmu}{\norm{\vmu}_2} \right\> \\
%             := {} & I +  I\!I - I\!I\!I,
%         \end{aligned}
%     \end{equation*}
%     where $\wt\zz$ is decomposed into terms independent ($I$) and dependent ($I\!I$) of $(\zz_i, \zz_j)$. That is, 
%     \begin{equation}\label{eq:zij_p}
%         \wt\zz_{-ij}^+  \indep  (\zz_i, \zz_j),
%         \quad
%         \alpha_{d,-ij}^+ := \frac12 \sqrt{\frac{n_+ - 2}{n_+^2} + \frac{1}{n_-}},
%         \quad
%         \frac{\wt\zz_{-ij}^+}{\alpha_{d,-ij}^+} , \frac{\zz_j - \zz_i}{\sqrt{2}}  \sim \subGind(\bzero, \bI_d; K).
%     \end{equation}
%     For the first term with $d$ large enough, we have
%     \begin{align*}
%             \norm{I}_{\psi_1} & = \norm{\< \zz_i - \zz_j, \wt\zz_{-ij}^+ \>}_{\psi_1}
%             = \norm{ \sum_{k=1}^d (\zz_i - \zz_j)_k (\wt\zz_{-ij}^+)_k }_{\psi_1} \\
%             & \overset{\mathmakebox[0pt][c]{\smash{\text{(i)}}}}{\lesssim} \sqrt{d} \max_{1 \le k \le d} \norm{ (\zz_i - \zz_j)_k (\wt\zz_{-ij}^+)_k }_{\psi_1} \\
%             & \overset{\mathmakebox[0pt][c]{\smash{\text{(ii)}}}}{\le} \sqrt{d} \max_{1 \le k \le d} \norm{ (\zz_i - \zz_j)_k }_{\psi_2}
%             \max_{1 \le k \le d} \norm{ (\wt\zz_{-ij}^+)_k }_{\psi_2} \\
%             & \overset{\mathmakebox[0pt][c]{\smash{\text{(iii)}}}}{\lesssim} \sqrt{d} \alpha_{d,-ij}^+ K^2 
%             \lesssim \sqrt{\frac{d}{\pi n}} K^2,
%     \end{align*}
%     where (i) is a consequence of \cref{eq:zij_p}, {\color{blue}coordinate independence}, and \cref{lem:subExp}\ref{lem:subExp-b}, (ii) is from \cref{lem:subExp}\ref{lem:subExp-d}, and (iii) is from \cref{eq:zij_p}. 
%     For the second term, we have
%     \begin{align*}
%         \norm{I\!I}_{\psi_1} & = \frac{1}{2 n_+} \norm{\< \zz_i - \zz_j, \zz_i + \zz_j \>}_{\psi_1}
%         \le \frac{1}{2 n_+}  \norm{ \norm{\zz_i}_2^2 - \norm{\zz_j}_2^2 }_{\psi_1} 
%     \end{align*}
% \end{itemize}
%%%%%%%%%%%%%%%%%%%%%%%%%%%%%%%%%%%%%%%%%%%%%%%%%%%%%%%%%%%%%%%
% Old ver. with mistakes: w/o considering uniform convergence %
%%%%%%%%%%%%%%%%%%%%%%%%%%%%%%%%%%%%%%%%%%%%%%%%%%%%%%%%%%%%%%%
% We claim the terms involving $g_i$'s are negligible.
% By \cref{lem:subG_concentrate}\ref{lem:subG-Hoeffding},
% \begin{equation*}
%     \bar g_+ = O_\P\biggl( \frac{1}{\sqrt{n_+}} \biggr) = O_\P\biggl( \frac{1}{\sqrt{\pi n}} \biggr) = o_{\P}(1),
%     \qquad
%     \bar g_- = O_\P\biggl( \frac{1}{\sqrt{n_-}} \biggr) = O_\P\biggl( \frac{1}{\sqrt{n}} \biggr) = o_{\P}(1).
% \end{equation*}
% Thus, the first term in \cref{eq:logit_diff} $\abs{\bar g_\pm - g_i} = O_\P(1)$ is bounded for both cases. 
% ~\\
% \noindent
% Now consider the second term \update{involving $\zz_i$'s}. For each $i$ such that $y_i = +1$,
% \begin{align*}
%         & \< \bar\zz_+ - \zz_i, \wt\vtheta \> \\
%         = {} & \frac{1}{n_+} \sum_{j \in \mathcal{I}_+} \< \zz_j - \zz_i, \wt\vtheta \> \\
%         = {} & \frac{1}{n_+} \sum_{j: \substack{j \not= i \\ y_j = +1}} \frac{1}{\| \bP_{\vmu}^\perp \wt\zz \|_2} \< \zz_j - \zz_i, \bP_{\vmu}^\perp \wt\zz \> \\
%         = {} & \frac{1}{n_+} \sum_{j: \substack{j \not= i \\ y_j = +1}} \frac{1}{\| \bP_{\vmu}^\perp \wt\zz \|_2} \Biggl\{ \biggl\< \zz_j - \zz_i, \bP_{\vmu}^\perp  
%         \biggl( \,
%             \underbrace{  \frac1{2 n_+} \sum_{k: \substack{k \not= i, j \\ y_k = +1}} \zz_k + \frac{\bar\zz_-}2  }_{ =: \wt\zz_{-ij}^+ }
%         \, \biggr) \biggr\> + \frac{1}{2 n_+}\< \zz_j - \zz_i,  \bP_{\vmu}^\perp (\zz_j + \zz_i)  \>  \Biggr\}.
% \end{align*}
% In the last line, $\wt\zz$ is decomposed into two terms, such that for the first term
% \begin{equation*}
%     \wt\zz_{-ij}^+  \indep  (\zz_i, \zz_j),
%     \qquad
%     \alpha_{d,-ij}^+ := \frac12 \sqrt{\frac{n_+ - 2}{n_+^2} + \frac{1}{n_-}},
%     \qquad
%     \frac{\wt\zz_{-ij}^+}{\alpha_{d,-ij}^+} , \frac{\zz_j - \zz_i}{\sqrt{2}}  \sim \subGind(\bzero, \bI_d; K).
% \end{equation*}
% According to \cref{lem:subG_concentrate}\ref{lem:subG-Bernstein},
% \begin{equation} \label{eq:ind_term+}
%     \< \zz_j - \zz_i, \bP_{\vmu}^\perp  \wt\zz_{-ij}^+ \> 
%     = 
%     \sqrt{2} \alpha_{d,-ij}^+  O_{\P}(\| \bP_{\vmu}^\perp \|_\mathrm{F})
%     = \frac{1}{\sqrt{2 \pi n}}\bigl(1 + o(1)\bigr) O_\P(\sqrt{d})
%     = O_\P\biggl( \sqrt{\frac{d}{\pi n}} \biggr).
% \end{equation}
% For the other term, let $\bar\xx = \begin{pmatrix}
%     \zz_i \\ \zz_j
% \end{pmatrix}$, $\bar\bA = \begin{pmatrix}
%     -\bP_{\vmu}^\perp & \bzero \\
%     \bzero & \bP_{\vmu}^\perp
% \end{pmatrix}$, then 
% \begin{equation*}
%     \begin{aligned}
%         \< \zz_j - \zz_i,  \bP_{\vmu}^\perp (\zz_j + \zz_i)  \> 
%     & = \bar\xx^\top \bar\bA \bar\xx,  \\
%     \E[\< \zz_j - \zz_i,  \bP_{\vmu}^\perp (\zz_j + \zz_i)  \> ]
%     & = 
%     \mathrm{tr}(\bP_{\vmu}^\perp \E[(\zz_j - \zz_i)(\zz_j + \zz_i)^\top]) 
%     = 0.
%     \end{aligned}
% \end{equation*}
% Note $\| \bar\bA \|_\mathrm{F} = \sqrt{2}\| \bP_{\vmu}^\perp \|_\mathrm{F} = \sqrt{2(d-1)}$, $\| \bar\bA \|_2 = \| \bP_{\vmu}^\perp \|_2 = 1$. Applying \cref{lem:subG_concentrate}\ref{lem:subG-Hanson-Wright-I} to $\bar\xx^\top \bar\bA \bar\xx$, for any fixed $t > 0$ and $d$ large enough,
% \begin{equation*}
%     \P\biggl( \frac{ | \bar\xx^\top \bar\bA \bar\xx | }{ \| \bar\bA \|_\mathrm{F} }  > t \biggr)
%     \le 2 \exp\biggl( -c \min \biggl\{ \frac{t^2}{K^4} , \frac{ t \| \bar\bA \|_\mathrm{F}}{ K^2 \| \bar\bA \|_2 } \biggr\} \biggr) =  2 \exp(-ct^2/K^4), 
% \end{equation*}
% where $c$ is an absolute constant. Therefore,
% \begin{equation} \label{eq:nonind_term}
%     \< \zz_j - \zz_i,  \bP_{\vmu}^\perp (\zz_j + \zz_i)  \>
%     = O_\P(\| \bar\bA \|_\mathrm{F}) = O_\P(\sqrt{d}).
% \end{equation}
% Hence, combining \cref{eq:P_wtz}, \eqref{eq:ind_term+} and \eqref{eq:nonind_term},
%     \begin{align}
%         \< \bar\zz_+ - \zz_i, \wt\vtheta \> 
%         & = \frac{1}{n_+} \sum_{j: \substack{j \not= i \\ y_j = +1}} \frac{1}{\| \bP_{\vmu}^\perp \wt\zz \|_2} \left\{ 
%             \< \zz_j - \zz_i,  \bP_{\vmu}^\perp \wt\zz_{-ij}^+ \> + 
%             \frac{1}{2 n_+}\< \zz_j - \zz_i,  \bP_{\vmu}^\perp (\zz_j + \zz_i)  \>
%         \right\}  \nonumber \\
%         & = \frac{1}{n_+} \sum_{j: \substack{j \not= i \\ y_j = +1}}
%         2 \sqrt{\frac{\pi n}{d}}\big(1 + o_\P(1) \big) \left\{ O_\P\biggl( \sqrt{\frac{d}{\pi n}} \biggr) + \frac{1}{2\pi n} O_\P(\sqrt{d}) \right\}  \nonumber \\
%         & = \frac{1}{n_+} \sum_{j: \substack{j \not= i \\ y_j = +1}} O_{\P}(1)
%         = \frac{n_+ - 1}{n_+} \cdot O_{\P}(1) = O_{\P}(1).  \label{eq:z_theta_diff+}
%     \end{align}
% Similarly, For each $i$ such that $y_i = -1$,
% \begin{align*}
%         & \< \bar\zz_- - \zz_i, \wt\vtheta \> \\
%         = {} & \frac{1}{n_-} \sum_{j \in \mathcal{I}_-} \< \zz_j - \zz_i, \wt\vtheta \> \\
%         = {} & \frac{1}{n_-} \sum_{j: \substack{j \not= i \\ y_j = -1}} \frac{1}{\| \bP_{\vmu}^\perp \wt\zz \|_2} \< \zz_j - \zz_i, \bP_{\vmu}^\perp \wt\zz \> \\
%         = {} & \frac{1}{n_-} \sum_{j: \substack{j \not= i \\ y_j = -1}} \frac{1}{\| \bP_{\vmu}^\perp \wt\zz \|_2} \Biggl\{ \biggl\< \zz_j - \zz_i, \bP_{\vmu}^\perp  
%         \biggl( \,
%             \underbrace{  \frac1{2 n_-} \sum_{k: \substack{k \not= i, j \\ y_k = -1}} \zz_k + \frac{\bar\zz_+}2  }_{ =: \wt\zz_{-ij}^- }
%         \, \biggr) \biggr\> + \frac{1}{2 n_-}\< \zz_j - \zz_i,  \bP_{\vmu}^\perp (\zz_j + \zz_i)  \>  \Biggr\}.
% \end{align*}
% Again, $\wt\zz$ is decomposed into two terms, such that for the first term
% \begin{equation*}
%     \wt\zz_{-ij}^-  \indep  (\zz_i, \zz_j),
%     \qquad
%     \alpha_{d,-ij}^- := \frac12 \sqrt{\frac{n_- - 2}{n_-^2} + \frac{1}{n_+}},
%     \qquad
%     \frac{\wt\zz_{-ij}^-}{\alpha_{d,-ij}^-} , \frac{\zz_j - \zz_i}{\sqrt{2}}  \sim \subGind(\bzero, \bI_d; K).
% \end{equation*}
% According to \cref{lem:subG_concentrate}\ref{lem:subG-Bernstein},
% \begin{equation} \label{eq:ind_term-}
%     \< \zz_j - \zz_i, \bP_{\vmu}^\perp  \wt\zz_{-ij}^- \> 
%     = 
%     \sqrt{2} \alpha_{d,-ij}^-  O_{\P}(\| \bP_{\vmu}^\perp \|_\mathrm{F})
%     = \frac{1}{\sqrt{2 \pi n}}\bigl(1 + o(1)\bigr) O_\P(\sqrt{d})
%     = O_\P\biggl( \sqrt{\frac{d}{\pi n}} \biggr).
% \end{equation}
% Hence, combining \cref{eq:P_wtz}, \eqref{eq:ind_term-} and \eqref{eq:nonind_term},
%     \begin{align}
%         \< \bar\zz_- - \zz_i, \wt\vtheta \> 
%         & = \frac{1}{n_-} \sum_{j: \substack{j \not= i \\ y_j = -1}} \frac{1}{\| \bP_{\vmu}^\perp \wt\zz \|_2} \left\{ 
%             \< \zz_j - \zz_i,  \bP_{\vmu}^\perp \wt\zz_{-ij}^- \> + 
%             \frac{1}{2 n_-}\< \zz_j - \zz_i,  \bP_{\vmu}^\perp (\zz_j + \zz_i)  \>
%         \right\}  \nonumber  \\
%         & = \frac{1}{n_-} \sum_{j: \substack{j \not= i \\ y_j = -1}}
%         2 \sqrt{\frac{\pi n}{d}}\big(1 + o_\P(1) \big) \left\{ O_\P\biggl( \sqrt{\frac{d}{\pi n}} \biggr) + \frac{1}{2 n} O_\P(\sqrt{d}) \right\}   \nonumber  \\
%         & = \frac{1}{n_-} \sum_{j: \substack{j \not= i \\ y_j = -1}} O_{\P}(1)
%         = \frac{n_- - 1}{n_-} \cdot O_{\P}(1) = O_{\P}(1).  \label{eq:z_theta_diff-}
%     \end{align}
% Finally, substitute these back to \cref{eq:logit_diff}, for each $i$,
% \begin{equation*}
%     \begin{aligned}
%         \abs{\bar\kappa - \kappa_i(\wt\rho, \wt\vtheta, \wt\beta_0)}
%         & = \abs{ \wt\rho (\bar g_\pm - g_i) +  \sqrt{1 - \wt\rho^2} \< \bar\zz_\pm - \zz_i, \wt\vtheta \> } \\
%         & \le  \abs{\bar g_\pm - g_i} + \abs{\< \bar\zz_\pm - \zz_i, \wt\vtheta \>} = O_{\P}(1).
%     \end{aligned}
% \end{equation*}
% This implies the logits for all the data points have almost the same asymptotics as the upper bound $\bar\kappa$. As the result,
Therefore, the difference between the margin of classifier characterized by $(\wt\rho, \wt\vtheta, \wt\beta_0)$ and its upper bound $\bar\kappa$ is bounded by
\begin{equation*}
    \bar\kappa - \kappa(\wt\rho, \wt\vtheta, \wt\beta_0) 
    = 
     \bar\kappa - \min_{i \in [n]} \kappa_i(\wt\rho, \wt\vtheta, \wt\beta_0) 
    = \max_{i \in [n]} \abs{\bar\kappa - \kappa_i(\wt\rho, \wt\vtheta, \wt\beta_0)} = O_\P(\log d) = \wt O_\P(1).
\end{equation*}
This concludes the proof of part \ref{lem:upper_bound_b}.


\vspace{0.5\baselineskip}
\noindent
\textbf{\ref{lem:upper_bound_c}:}
According to max-margin optimization problem \cref{eq:SVM-rho_theta}, note that
\begin{equation*}
    \hat\kappa = 
    \max_{ \substack{ \rho \in [-1, 1],  \beta_0 \in \R
    \\
    \vtheta \in \S^{d-1}, \vtheta \perp \vmu } } \kappa(\rho, \vtheta, \beta_0)
    \ge \kappa(\wt\rho, \wt\vtheta, \wt\beta_0),
\end{equation*}
hence the asymptotics of $\hat\kappa$ is followed by (a) and (b). As $d \to \infty$, note that
\begin{equation*}
    \hat\kappa \ge \bar\kappa - \wt O_\P(1)
    = \bigl(1 + o_\P(1)\bigr) \sqrt{  d^{b} +  \frac{1}{4} d^{a-c} },
    \qquad
    \sqrt{  d^{b} +  \frac{1}{4} d^{a-c} } \ge d^{b/2} \to + \infty,
\end{equation*}
which implies $\hat\kappa$ diverges with high probability, i.e., $\lim_{d \to \infty} \P(\hat\kappa > C) = 1$, $\forall\, C \in \R$. As the result, $\P\{\text{linearly separable}\} = \P(\hat\kappa > 0) \to 1$ as $d \to \infty$, deducing $\|\hat\vbeta\|_2 = 1$ with high probability. This concludes the proof of part \ref{lem:upper_bound_c}.
\end{proof}


\subsection{Asymptotics of optimal parameters: Proofs of \cref{lem:rho_hat}, \ref{lem:theta_hat_z}, \ref{lem:beta0_asymp}}
\label{subsec:highimb_asymp}

Followed by tightness of the upper bound $\bar\kappa$, we show that the optimal parameters $\hat\rho, \hat\vtheta$ should be very ``close'' to the constructed solution $\wt\rho, \wt\vtheta$ defined in \cref{eq:param_star} in some sense. On the event that the data is linearly separable, we have showed that $\hat\vbeta$, and therefore both $\hat\rho$ and $\hat\vtheta$, do not depend on $\tau$ in \cref{prop:SVM_tau_relation}. Hence, it still suffices to consider $\tau = 1$ in our proof.


\subsubsection{Asymptotic order of $\hat\rho$: Proofs of \cref{lem:rho_hat}}

The following technical Lemma is important for deriving the asymptotics of $\hat\rho$, which introduces a function of $\rho$ used implicitly in \cref{eq:F_AB} and \eqref{eq:F_AB_optim} for optimization.

\begin{lem} \label{lem:F_AB}
    Define $F_{A, B}(\rho) = \rho A + \sqrt{1 - \rho^2} B$, $\rho \in [-1, 1]$, with $A \in \R$, $B > 0$. Then
\begin{equation*}
    F_{A, B}'(\rho) = A - \frac{\rho}{\sqrt{1 - \rho^2}} B,
    \qquad
    F_{A, B}''(\rho) = -\frac{1}{(1 - \rho^2)^{3/2}} B,
\end{equation*}
which implies $F_{A,B}$ is $B$-strongly concave, that is, for all $\rho_1, \rho_2 \in [-1, 1]$,
\begin{equation*}
    F_{A,B}(\rho_2) \le F_{A,B}(\rho_1) + F'_{A,B}(\rho_1)(\rho_2 - \rho_1) - \frac12 B (\rho_2 - \rho_1)^2.
\end{equation*} 
Moreover,
\begin{equation*}
    \argmax_{\rho \in [-1, 1]} F_{A,B}(\rho) = \frac{A}{\sqrt{A^2 + B^2}},  \qquad \max_{\rho \in [-1, 1]} F_{A,B}(\rho) = \sqrt{A^2 + B^2}.
\end{equation*}
\end{lem}
\begin{proof}
    Strongly concavity is given by direct calculation and the fact that
    \begin{equation*}
        \sup_{\rho \in [-1, 1]} F''_{A,B}(\rho) = -B.
    \end{equation*}
    The optimality condition is already derived in \cref{eq:F_AB_optim}. This concludes the proof.
\end{proof}

\noindent
In the rest of this section, the (stochastic) parameters $A, B$ are defined as
\begin{equation} \label{eq:AB_def}
    A := \norm{\vmu}_2 + \wt g
    = d^{b/2} \bigl(1 + o_{\P}(1)\bigr),
    \qquad
    B := \| \bP_{\vmu}^\perp \wt\zz \|_2
    = \frac12 d^{(a-c)/2} \bigl(1 + o_{\P}(1)\bigr).
\end{equation}
Then followed by \cref{lem:F_AB}, we have $\wt\rho = \argmax_{\rho \in [-1, 1]} F_{A,B}(\rho)$ and $F'_{A,B}(\wt\rho) = 0$, where $\wt\rho$ is defined in \cref{eq:param_star}. The following Lemma describes the asymptotics of $\hat\rho$ with respect to $\wt\rho$.

\begin{lem}[Asymptotics of $\hat\rho$ and $\wt\rho$] \label{lem:rho_hat}
    Suppose that $a < c + 1$.
    \begin{enumerate}[label=(\alph*)]
        \item \label{lem:rho_hat(a)} If $a < b + c$, then $\wt\rho = 1 - o_\P(1)$, $\hat\rho = 1 - o_\P(1)$, and
        \begin{equation*}
            \sqrt{1 - \wt\rho^2} = \frac12 d^{(a-b-c)/2}\bigl( 1 + o_\P(1) \bigr).
        \end{equation*}
        Moreover, we further assume:
        \begin{itemize}
            \item[i.] If $a > \frac{b}{2} + c$, then $\sqrt{1 - \hat\rho^2} = \sqrt{1 - \wt\rho^2} \bigl(1 + o_\P(1)\bigr)$.
            \item[ii.] If $a \le \frac{b}{2} + c$, then $\sqrt{1 - \hat\rho^2} = \wt O_\P(d^{-b/4})$ and thus $\sqrt{1 - \hat\rho^2}\sqrt{d/\pi n} = \wt O_\P(1)$.
        \end{itemize}
        \item \label{lem:rho_hat(b)} If $a > b + c$, then $\wt\rho = o_\P(1)$, $\hat\rho = o_\P(1)$, and
        \begin{equation*}
            \wt\rho = 2d^{(b-a+c)/2}\bigl(1 + o_\P(1)\bigr).
        \end{equation*}
        Moreover, we further assume:
        \begin{itemize}
            \item[i.] If $a < 2b + c$, then $\hat\rho = \wt\rho \bigl(1 + o_\P(1)\bigr)$.
            \item[ii.] If $a > 2b + c$, then $\hat \rho = \wt O_\P(d^{-(a-c)/4})$ and thus $\hat \rho \norm{\vmu}_2 = o_\P(1)$.
        \end{itemize}
    \end{enumerate}
\end{lem}
\begin{proof}
According to \cref{eq:param_star} and \eqref{eq:AB_def}, an explicit expression of $\wt\rho$ is given by
\begin{equation}\label{eq:rho_star}
    \wt\rho = 
    \frac{A}{\sqrt{A^2 + B^2}}
    =
    \dfrac{ \norm{\vmu}_2 + \wt g }{\sqrt{ ( \norm{\vmu}_2 + \wt g )^2 + \| \bP_{\vmu}^\perp \wt\zz \|_2^2 } }
    = \frac{d^{b/2}}{\sqrt{ d^{b} + \frac14 d^{a-c} }}  \bigl(1 + o_{\P}(1)\bigr).
\end{equation}
In order to connect $\hat\rho$ with $\wt\rho$, recall \cref{eq:F_AB} that
\begin{equation*}
    \kappa(\rho, \vtheta, \beta_0) \le F_{A,B}(\rho) \le F_{A,B}(\wt\rho) = \bar\kappa,
    \qquad \forall\, \rho \in [-1, 1], \  \vtheta \in \S^{d-1}, \vtheta \perp \vmu, \  \beta_0 \in \R.
\end{equation*}
Apply this to $\hat\kappa = \kappa(\hat\rho, \hat\vtheta, \hat\beta_0)$ and use \cref{lem:upper_bound}, we have
\begin{equation}\label{eq:F_AB_diff}
    0 \le F_{A,B}(\wt\rho) - F_{A,B}(\hat\rho) \le \wt O_\P(1).
\end{equation}
Since $\wt O_\P(1)/F_{A,B}(\wt\rho) = \wt O_\P(1)/\sqrt{A^2 + B^2} \le \wt O_\P(d^{-b/2}) = o_\P(1)$, it implies
\begin{equation*}
    1 - o_\P(1)  =  \frac{F_{A,B}(\hat\rho)}{F_{A,B}(\wt\rho)} 
    = \frac{\hat\rho A }{\sqrt{A^2 + B^2}} + \frac{ \sqrt{1 - \hat\rho^2} B}{\sqrt{A^2 + B^2}}
    = \hat\rho \wt\rho + \sqrt{1 - \hat\rho^2}\sqrt{1 - \wt\rho^2}.
\end{equation*}
Therefore,
\begin{equation}\label{eq:rho_star_to_hat}
    \begin{aligned}
        \wt\rho = 1 - o_\P(1) & \ \Longrightarrow & \hat\rho = 1 - o_\P(1)
        \\
        \wt\rho = o_\P(1)     & \ \Longrightarrow & \hat\rho = o_\P(1)
    \end{aligned}
\end{equation}

\vspace{0.5\baselineskip}
\noindent
\textbf{\ref{lem:rho_hat(a)}:}
If $a - c < b$, then $d^{b/2} \gg d^{(a-c)/2}$ and by \cref{eq:rho_star} and \eqref{eq:rho_star_to_hat} we have $\wt\rho, \hat\rho = 1 - o_\P(1)$. Also,
\begin{equation*}
    \sqrt{1 - \wt\rho^2} = \frac{B}{\sqrt{A^2 + B^2}}
    = \frac{\frac12 d^{(a-b-c)/2}}{\sqrt{ 1 + \frac14 d^{a-b-c} }}\bigl(1 + o_{\P}(1)\bigr)
    = \frac12 d^{(a-b-c)/2}\bigl(1 + o_{\P}(1)\bigr).
\end{equation*}
To derive the precise order of $\sqrt{1 - \hat\rho^2}$, we define $r := \sqrt{1 - \rho^2}$ and $F_{B,A}(r) := rB + \sqrt{1 - r^2}A$. Then $F_{A,B}(\rho) = F_{B,A}(r)$ for any $\rho \in [0, 1]$. We similarly define $\hat r := \sqrt{1 - \hat\rho^2}$ and $\wt r := \sqrt{1 - \wt\rho^2}$. On the event $\mathcal{E} = \{ A > 0, \wt\rho >0, \hat\rho > 0 \}$, by \cref{lem:F_AB}, we have
\begin{equation*}
    F_{A,B}(\hat\rho) - F_{A,B}(\wt\rho) 
    = F_{B,A}(\hat r) - F_{B,A}(\wt r) 
    \le -\frac12 A (\hat r - \wt r)^2.
\end{equation*}
Combined with \cref{eq:F_AB_diff}, it implies
\begin{equation*}
    (\hat r - \wt r)^2 \le  \frac{2}{A}\bigl( F_{A,B}(\wt\rho) - F_{A,B}(\hat\rho) \bigr) \le \wt O_\P(d^{-b/2}),
\end{equation*}
so $|\hat r - \wt r| = \wt O_\P(d^{-b/4})$. Now consider different scenarios. Recall that $\wt r = \frac12 d^{(a-b-c)/2}\bigl(1 + o_{\P}(1)\bigr)$.
\begin{itemize}
    \item If $a - c > b/2$, then $|\hat r - \wt r|/\wt r = \wt O_\P(d^{(-2a + b + 2c)/4}) = o_\P(1)$, deduces $\hat r = \wt r \bigl(1 + o_{\P}(1)\bigr)$.
    \item If $a - c \le b/2$, then we only get $\hat r = \wt O_\P(d^{-b/4})$, and $\hat r \sqrt{d/\pi n} = \wt O_\P(d^{(2a - b-2c)/4})  \le \wt O_\P(1)$.
\end{itemize}
Recall that these hold on event $\mathcal{E}$. Since $\P(\mathcal{E}) \to 1$ as $d \to \infty$, these asymptotic results involving $o_\P(\,\cdot\,)$ and $\wt O_\P(\,\cdot\,)$ also hold on the whole sample space $\Omega$. This concludes the proof of part \ref{lem:rho_hat(a)}.



\vspace{0.5\baselineskip}
\noindent
\textbf{\ref{lem:rho_hat(b)}:}
If $a - c > b$, then $d^{b/2} \ll d^{(a-c)/2}$ and by \cref{eq:rho_star} and \eqref{eq:rho_star_to_hat} we have $\wt\rho, \hat\rho = o_\P(1)$. Also,
\begin{equation*}
    \wt\rho = 
    \frac{A}{\sqrt{A^2 + B^2}}
    = \frac{d^{(b-a+c)/2}}{\sqrt{ d^{b-a+c} + \frac14 }}  \bigl(1 + o_{\P}(1)\bigr)
    = 2d^{(b-a+c)/2} \bigl(1 + o_{\P}(1)\bigr).
\end{equation*}
Again, by \cref{lem:F_AB},
\begin{equation*}
    F_{A,B}(\hat\rho) - F_{A,B}(\wt\rho) \le -\frac12 B(\hat\rho - \wt\rho)^2.
\end{equation*}
Combined with \cref{eq:F_AB_diff}, it implies
\begin{equation*}
    (\hat\rho - \wt\rho)^2 \le  \frac{2}{B}\bigl( F_{A,B}(\wt\rho) - F_{A,B}(\hat\rho) \bigr) \le \wt O_\P(d^{-(a-c)/2}),
\end{equation*}
so $|\hat\rho - \wt\rho| = \wt O_\P(d^{-(a-c)/4})$. Now consider different scenarios.
\begin{itemize}
    \item If $a - c < 2b$, then $|\hat \rho - \wt\rho|/\wt\rho = \wt O_\P(d^{(a - 2b -c)/4}) = o_\P(1)$, deduces $\hat \rho = \wt\rho \bigl(1 + o_{\P}(1)\bigr)$.
    \item If $a - c > 2b$, then we only get $\hat \rho = \wt O_\P(d^{-(a-c)/4})$, and $\hat \rho \norm{\vmu}_2 = \wt O_\P(d^{(2b-a+c)/4}) = o_\P(1)$.
\end{itemize}
This concludes the proof of part \ref{lem:rho_hat(b)}.
\end{proof}
\begin{rem}
    In each part i. of \cref{lem:rho_hat}\ref{lem:rho_hat(a)} and \ref{lem:rho_hat(b)}, we can derive the precise asymptotic of $\hat\rho$, which is same as $\wt\rho$. It is difficult to do so in part ii. of \ref{lem:rho_hat(a)} and \ref{lem:rho_hat(b)}. However, as we will show in \cref{lem:theta_hat_z} and \ref{lem:beta0_asymp}, in case ii. the corresponding term ($\sqrt{1 - \hat\rho}$ or $\hat\rho$) is negligible, which won't affect the asymptotics of test errors.
\end{rem}


\subsubsection{Asymptotic order of $\<\zz_i, \hat\vtheta \>$'s on the margin: Proof of \cref{lem:theta_hat_z}}


Next, we discuss the asymptotics of $\hat\vtheta$ and $\wt\vtheta$. In fact, it suffices to consider the magnitude of their projection on some ``important'' $\zz_i$, which is related to the \emph{support vectors}, defined in \cref{eq:SV_def}. As we mentioned, $\mathcal{SV}_+(\vbeta), \mathcal{SV}_-(\vbeta)$ only depend on $\vbeta$ and $(\XX, \yy)$, not $\beta_0$ or $\tau$. If we fix $\rho = \hat\rho$, then the dependency of $\mathcal{SV}_\pm$ on $\vbeta$ only comes from $\vtheta$. So, recalling \cref{eq:logits}, 
\[ \kappa_i(\rho, \vtheta, \beta_0) = s_i \left( \rho \norm{\vmu}_2 + y_i\beta_0 + \rho y_i g_i + \sqrt{1 - \rho^2} y_i \< \zz_i, \vtheta \> \right), \]
we can rewrite \cref{eq:SV_def} in terms of $\vtheta$:
\begin{equation}\label{eq:SV_def_theta}
	\begin{aligned}
		\mathcal{SV}_+ = \mathcal{SV}_+(\vtheta) & :=  \argmin_{i \in \mathcal{I}_+} \kappa_i(\hat\rho, \vtheta, \beta_0)
	= \argmin_{i \in \mathcal{I}_+} \left\{ \phantom{-} \hat\rho g_i + \sqrt{1 - \hat\rho^2} \< \zz_i, \vtheta \> \right\} , \\
		\mathcal{SV}_- = \mathcal{SV}_-(\vtheta) & :=  \argmin_{i \in \mathcal{I}_-} \kappa_i(\hat\rho, \vtheta, \beta_0)
	= \argmin_{i \in \mathcal{I}_-} \left\{  - \hat\rho g_i - \sqrt{1 - \hat\rho^2} \< \zz_i, \vtheta \> \right\} . \\
	\end{aligned}
\end{equation}
As before, let $\mathsf{sv}_+(\vtheta), \mathsf{sv}_-(\vtheta)$ be (the indices of) any positive and negative support vectors, i.e.,
\begin{equation*}
	\mathsf{sv}_+(\vtheta) \in \mathcal{SV}_+(\vtheta),
	\qquad
	\mathsf{sv}_-(\vtheta) \in \mathcal{SV}_-(\vtheta).
\end{equation*} 
Now, recall that whenever a slope parameter $\vbeta$ is given, the optimal intercept $\check\beta_0 := \check\beta_0(\vbeta)$ (defined in \cref{eq:beta0_optim}) must satisfy the \emph{margin-balancing} condition \cref{eq:margin-bal}, according to \cref{lem:indep_tau}. Hence, fixing $\rho = \hat\rho$ and considering arbitrary $\vtheta$, we can rewrite \cref{eq:margin_pm} and \eqref{eq:margin-bal} as
\begin{equation}\label{eq:svm_sv_bal}
    \begin{aligned}
       \kappa(\hat\rho, \vtheta, \check\beta_0) & = \kappa_{\mathsf{sv}_+(\vtheta)}(\hat\rho, \vtheta, \check\beta_0) = 
       \hat\rho \norm{\vmu}_2 + \check\beta_0 + \hat\rho g_{\mathsf{sv}_+(\vtheta)} + \sqrt{1 - \hat\rho^2} \< \zz_{\mathsf{sv}_+(\vtheta)}, \vtheta \> \\
       & = \kappa_{\mathsf{sv}_-(\vtheta)}(\hat\rho, \vtheta, \check\beta_0) = 
       \hat\rho \norm{\vmu}_2 - \check\beta_0 - \hat\rho g_{\mathsf{sv}_-(\vtheta)} - \sqrt{1 - \hat\rho^2} \< \zz_{\mathsf{sv}_-(\vtheta)}, \vtheta \>.
    \end{aligned}
\end{equation}
In particular, if $\vtheta = \hat\vtheta$, we denote $\mathsf{sv}_+(\hat\vtheta) \in \mathcal{SV}_+(\hat\vtheta)$, $\mathsf{sv}_-(\hat\vtheta) \in \mathcal{SV}_-(\hat\vtheta)$ as the support vectors of max-margin classifier. The Lemma below describes the magnitude of $\< \zz_{\mathsf{sv}_+(\hat\vtheta)}, \hat\vtheta \>$ and $\< \zz_{\mathsf{sv}_-(\hat\vtheta)}, \hat\vtheta \>$.

%\update{
\begin{lem}[Asymptotics of $\< \zz_{i}, \hat\vtheta \>$'s for support vectors] \label{lem:theta_hat_z}
    Suppose that $a < c + 1$.
    \begin{enumerate}[label=(\alph*)]
        \item \label{lem:theta_hat_z(a)}
        If $a < b + c$, then 
        \begin{equation*}
            \sqrt{1 - \hat\rho^2} \< \zz_{\mathsf{sv}_+(\hat\vtheta)}, \hat\vtheta \>  =  \wt O_\P(d^{a-\frac{b}{2}-c} \vee 1),
            \qquad
            \sqrt{1 - \hat\rho^2} \< \zz_{\mathsf{sv}_-(\hat\vtheta)}, \hat\vtheta \>  =  \wt O_\P(1).
        \end{equation*}
        \item \label{lem:theta_hat_z(b)}
        If $a > b + c$, then 
        \begin{equation*}
            \< \zz_{\mathsf{sv}_+(\hat\vtheta)}, \hat\vtheta \> = \sqrt{\frac{d}{\pi n}}\bigl(1 + o_{\P}(1)\bigr),
            \qquad
            \< \zz_{\mathsf{sv}_-(\hat\vtheta)}, \hat\vtheta \> = \wt O_\P(1).
        \end{equation*}
    \end{enumerate}
\end{lem}
\begin{proof}
    $\mathcal{SV}_\pm(\vtheta)$ may not be tractable, since it involves a nuisance term $\hat\rho g_i$ as defined in \cref{eq:SV_def_theta}. Therefore, we introduce a proxy of support vectors, which is easier to work with. Formally, let
    \begin{equation}\label{eq:V_def_theta}
        \begin{aligned}
            \mathcal{V}_+ = \mathcal{V}_+(\vtheta) & :=  
        \argmin_{i \in \mathcal{I}_+}    +  \< \zz_i, \vtheta \>  , \\
            \mathcal{V}_- = \mathcal{V}_-(\vtheta) & :=  
        \argmin_{i \in \mathcal{I}_-}    -  \< \zz_i, \vtheta \>  , \\
        \end{aligned}
    \end{equation}
    where $\mathcal{V}_+, \mathcal{V}_-$ are sets of (the indices of) the smallest $y_i \< \zz_{i}, \vtheta \>$ from each class. Similarly, let
    \begin{equation*}
        \mathsf{v}_+(\vtheta) \in  \mathcal{V}_+(\vtheta),
        \qquad
        \mathsf{v}_-(\vtheta) \in  \mathcal{V}_-(\vtheta),
    \end{equation*}
    which are arbitrary elements in $\mathcal{V}_+(\vtheta)$ and $\mathcal{V}_-(\vtheta)$. Note that $\mathcal{V}_\pm$ is simply $\mathcal{SV}_\pm$ but ignoring term $\hat\rho g_i$. Indeed, as we will show later, the impact of $\hat\rho g_i = O_\P(1)$ is almost negligible.

    We are going to prove \cref{lem:theta_hat_z} by deriving tight upper bounds and lower bounds for both $\pm \sqrt{1 - \hat\rho^2} \< \zz_{\mathsf{sv}_\pm(\hat\vtheta)}, \hat\vtheta \>$. Then we conclude the precise asymptotics by verifying the upper and lower bounds are matched.
    % ~\\
    % \noindent
    % We restrict our analysis on the event $\{ \hat \kappa > 0\}$, i.e., $(\XX, \yy)$ is linearly separable.

    \paragraph{Upper bounds}
    Applying the same idea as \cref{eq:kappa_pm}, we can bound $\pm \< \zz_{\mathsf{v}_\pm(\vtheta)}, \vtheta \>$ via averaging:
    \begin{equation}
        \label{eq:zV_upper}
            \< \zz_{\mathsf{v}_+(\vtheta)}, \vtheta \> \le \< \bar\zz_{+}, \vtheta \> \le  \| \bP_{\vmu}^\perp \bar\zz_+ \|_2,
            \qquad
            - \< \zz_{\mathsf{v}_-(\vtheta)}, \vtheta \> \le - \< \bar\zz_{-}, \vtheta \> \le  \| \bP_{\vmu}^\perp \bar\zz_- \|_2,
    \end{equation}
    where the second inequality for each comes from \cref{eq:theta_min}. 
    To create a connection between $\mathsf{sv}_\pm(\hat\vtheta)$ and $\mathsf{v}_\pm(\hat\vtheta)$, note that by definition \cref{eq:SV_def_theta}
\begin{equation*}
    \begin{aligned}
        \phantom{-} \hat\rho g_{\mathsf{sv}_+(\hat\vtheta)} + \sqrt{1 - \hat\rho^2} \< \zz_{\mathsf{sv}_+(\hat\vtheta)}, \hat\vtheta \> 
        & \le \phantom{-} \hat\rho g_{\mathsf{v}_+(\hat\vtheta)} + \sqrt{1 - \hat\rho^2} \< \zz_{\mathsf{v}_+(\hat\vtheta)}, \hat\vtheta \>,
        % \quad \ 
        % \phantom{-} \<\zz_{\mathsf{v}_+(\hat\vtheta)}, \hat\vtheta \> \le \phantom{-} \<\zz_{\mathsf{sv}_+(\hat\vtheta)}, \hat\vtheta \>,
        \\
        - \hat\rho g_{\mathsf{sv}_-(\hat\vtheta)} - \sqrt{1 - \hat\rho^2} \< \zz_{\mathsf{sv}_-(\hat\vtheta)}, \hat\vtheta \> 
        & \le - \hat\rho g_{\mathsf{v}_-(\hat\vtheta)} - \sqrt{1 - \hat\rho^2} \< \zz_{\mathsf{v}_-(\hat\vtheta)}, \hat\vtheta \>.
        % \quad \ 
        % - \<\zz_{\mathsf{v}_-(\hat\vtheta)}, \hat\vtheta \> \le - \<\zz_{\mathsf{sv}_-(\hat\vtheta)}, \hat\vtheta \>.
    \end{aligned}
\end{equation*}
Using \cref{eq:zV_upper}, therefore we obtain the following non-asymptotic upper bounds on $\< \zz_{\mathsf{sv}_\pm(\hat\vtheta)}, \hat\vtheta \>$:
\begin{equation}
    \label{eq:zSV_upper}
    \begin{aligned}
        \phantom{-} \sqrt{1 - \hat\rho^2} \< \zz_{\mathsf{sv}_+(\hat\vtheta)}, \hat\vtheta \> \ \ 
        & \le   \phantom{-} \sqrt{1 - \hat\rho^2} \< \zz_{\mathsf{v}_+(\hat\vtheta)}, \hat\vtheta \> + \hat\rho \bigl(g_{\mathsf{v}_+(\hat\vtheta)} - g_{\mathsf{sv}_+(\hat\vtheta)} \bigr)  \\
        & \le  \phantom{-} \sqrt{1 - \hat\rho^2} \| \bP_{\vmu}^\perp \bar\zz_+ \|_2 + 
        2 \hat\rho \max_{i \in [n]} \abs{g_i} , \\
        - \sqrt{1 - \hat\rho^2} \< \zz_{\mathsf{sv}_-(\hat\vtheta)}, \hat\vtheta \> \ \ 
        & \le  - \sqrt{1 - \hat\rho^2} \< \zz_{\mathsf{v}_-(\hat\vtheta)}, \hat\vtheta \> - \hat\rho \bigl(g_{\mathsf{v}_-(\hat\vtheta)} - g_{\mathsf{sv}_-(\hat\vtheta)} \bigr) \\
        & \le  \phantom{-} \sqrt{1 - \hat\rho^2} \| \bP_{\vmu}^\perp \bar\zz_- \|_2 + 
        2 \hat\rho \max_{i \in [n]} \abs{g_i} .
    \end{aligned}
\end{equation}
To compute its asymptotics, recall that $\sqrt{n_+} \cdot \bar\zz_+ \sim \subGind(\bzero, \bI_d; K)$, $\sqrt{n_-} \cdot \bar\zz_- \sim \subGind(\bzero, \bI_d; K)$, and $\| \bP_{\vmu}^\perp \|_\mathrm{F} = \sqrt{d - 1}$. Then by \cref{lem:subG_concentrate}\ref{lem:subG-Hanson-Wright-II},
\begin{align}
        \| \bP_{\vmu}^\perp \bar\zz_+ \|_2 & = \frac{1}{\sqrt{n_+}} \| \bP_{\vmu}^\perp \|_\mathrm{F} \bigl(1 + o_\P(1)\bigr)
    = \sqrt{\frac{d}{\pi n}}\big(1 + o_\P(1) \big), \notag \\
        \| \bP_{\vmu}^\perp \bar\zz_- \|_2 & = \frac{1}{\sqrt{n_-}} \| \bP_{\vmu}^\perp \|_\mathrm{F} \bigl(1 + o_\P(1)\bigr)
    = \mathmakebox[\widthof{$\displaystyle \sqrt{\frac{d}{\pi n}}\big(1 + o_\P(1) \big)$}][r]{\sqrt{\frac{d}{n}}\big(1 + o_\P(1) \big)} = o_\P(1).
    \label{eq:Pzpm_asymp}
\end{align}
While, by maximal inequality \cref{lem:subG}\ref{lem:subG-c} or \cref{eq:max_g}, we have
\begin{equation}
    \label{eq:g_asymp}
    \max_{i \in [n]} \abs{g_i} = O_\P(\log n) = \wt O_\P(1),
\end{equation}
Plugging \cref{eq:Pzpm_asymp} and \eqref{eq:g_asymp} into \cref{eq:zSV_upper} gives the asymptotic upper bounds (involving $\hat\rho$):
\begin{equation}
    \label{eq:zSV_upper_asymp}
    \begin{aligned}
        \phantom{-} \sqrt{1 - \hat\rho^2} \< \zz_{\mathsf{sv}_+(\hat\vtheta)}, \hat\vtheta \>  
        & \le  \sqrt{1 - \hat\rho^2} \sqrt{\frac{d}{\pi n}}\big(1 + o_\P(1) \big) + 
        \hat\rho \cdot \wt O_\P(1) , \\
        - \sqrt{1 - \hat\rho^2} \< \zz_{\mathsf{sv}_-(\hat\vtheta)}, \hat\vtheta \> 
        & \le  \sqrt{1 - \hat\rho^2} \cdot o_\P(1) + 
        \hat\rho \cdot \wt O_\P(1) .
    \end{aligned}
\end{equation}

\paragraph{Lower bounds} Similar as the proof of \cref{lem:upper_bound}, a lower bound can be obtained by plugging our constructed solution $\wt\vtheta = \bP_{\vmu}^\perp \wt\zz/\| \bP_{\vmu}^\perp \wt\zz \|_2$, which can be a good ``proxy'' of $\vtheta$. Again, by margin-balancing condition \cref{eq:svm_sv_bal}, we can express the optimal $\vtheta$ as\footnote{
    Notice that if $|\hat\rho| < 1$, then $\argmax_{ \vtheta \in \S^{d-1}, \vtheta \perp \vmu } \kappa(\hat\rho, \vtheta, \hat\beta_0)$ is unique (on the event of $\{ \hat\kappa > 0\}$), and we could write $\hat\vtheta = \argmax_{ \vtheta \in \S^{d-1}, \vtheta \perp \vmu } \kappa(\hat\rho, \vtheta, \hat\beta_0)$. However, if $|\hat\rho| = 1$, then according to our construction \cref{eq:def-rho-theta}, the arguments of the maxima can be any $\vtheta \in \S^{d-1}$ such that $\vtheta \perp \vmu$, while $\hat\vtheta = \vmu_\perp$ as defined in \cref{eq:def-rho-theta_hat}.
}
\begin{equation*}
    \begin{aligned}
        \hat\vtheta & \in \argmax_{ \vtheta \in \S^{d-1}, \vtheta \perp \vmu } \kappa(\hat\rho, \vtheta, \hat\beta_0)
    = \argmax_{ \vtheta \in \S^{d-1}, \vtheta \perp \vmu } \frac{\kappa_{\mathsf{sv}_+(\vtheta)}(\hat\rho, \vtheta, \hat\beta_0) + \kappa_{\mathsf{sv}_-(\vtheta)}(\hat\rho, \vtheta, \hat\beta_0)}{2} \\
    & = \argmax_{ \vtheta \in \S^{d-1}, \vtheta \perp \vmu } 
    \left\{ \hat\rho \norm{\vmu}_2 + \hat\rho \frac{g_{\mathsf{sv}_+(\vtheta)} - g_{\mathsf{sv}_-(\vtheta)} }{2} + \sqrt{1 - \hat\rho^2}  \frac{ \< \zz_{\mathsf{sv}_+(\vtheta)}, \vtheta \>  -  \< \zz_{\mathsf{sv}_-(\vtheta)}, \vtheta \>  }{2} \right\}   \\
    & = \argmax_{ \vtheta \in \S^{d-1}, \vtheta \perp \vmu } \left\{  \hat\rho \left(  g_{\mathsf{sv}_+(\vtheta)} - g_{\mathsf{sv}_-(\vtheta)} \right) +  \sqrt{1 - \hat\rho^2} \left( \< \zz_{\mathsf{sv}_+(\vtheta)}, \vtheta \>  +  \< \zz_{\mathsf{sv}_-(\vtheta)}, \vtheta \> \right)  \right\}.
    \end{aligned}
\end{equation*}
Therefore, recalling \cref{eq:V_def_theta}, we have
\begin{align*}
        & \sqrt{1 - \hat\rho^2}  \bigl( \< \zz_{\mathsf{sv}_+(\hat\vtheta)}, \hat\vtheta \>  -  \< \zz_{\mathsf{sv}_-(\hat\vtheta)}, \hat\vtheta \>  \bigr) \\
        \ge {} & \sqrt{1 - \hat\rho^2}  \bigl( \< \zz_{\mathsf{sv}_+(\wt\vtheta)}, \wt\vtheta \>  -  \< \zz_{\mathsf{sv}_-(\wt\vtheta)}, \wt\vtheta \>  \bigr) + \hat\rho\bigl( g_{\mathsf{sv}_+(\wt\vtheta)} - g_{\mathsf{sv}_-(\wt\vtheta)} -  g_{\mathsf{sv}_+(\hat\vtheta)} + g_{\mathsf{sv}_-(\hat\vtheta)} \bigr) \\
        \ge {} & \sqrt{1 - \hat\rho^2}   \bigl( \< \zz_{\mathsf{v}_+(\wt\vtheta)}, \wt\vtheta \>  -  \< \zz_{\mathsf{v}_-(\wt\vtheta)}, \wt\vtheta \>  \bigr) 
        - 4 \hat\rho \max_{i \in [n]} \abs{g_i}.
        % \min_{i \in \mathcal{I}_+} \< + \zz_i, \wt\vtheta \> 
        % + \sqrt{1 - \hat\rho^2}  \min_{i \in \mathcal{I}_-} \< - \zz_i, \wt\vtheta \>
\end{align*}
Combining it with \cref{eq:zSV_upper}, we can obtain a lower bound for each term using $\wt\vtheta$:
\begin{equation}\label{eq:zSV_lower}
    \begin{aligned}
        \phantom{-} \sqrt{1 - \hat\rho^2} \< \zz_{\mathsf{sv}_+(\hat\vtheta)}, \hat\vtheta \>
        & \ge \sqrt{1 - \hat\rho^2}   \bigl( \< \zz_{\mathsf{v}_+(\wt\vtheta)}, \wt\vtheta \>  -  \< \zz_{\mathsf{v}_-(\wt\vtheta)}, \wt\vtheta \>  \bigr) 
        - 4 \hat\rho \max_{i \in [n]} \abs{g_i}
        + \sqrt{1 - \hat\rho^2} \< \zz_{\mathsf{sv}_-(\hat\vtheta)}, \hat\vtheta \> \\
        & \ge \sqrt{1 - \hat\rho^2} \bigl( 
        - \< \zz_{\mathsf{v}_-(\wt\vtheta)}, \wt\vtheta \> - \| \bP_{\vmu}^\perp \bar\zz_- \|_2 \bigr)
        - 6 \hat\rho \max_{i \in [n]} \abs{g_i}
        + \sqrt{1 - \hat\rho^2} \< \zz_{\mathsf{v}_+(\wt\vtheta)}, \wt\vtheta \>, 
        \\
        - \sqrt{1 - \hat\rho^2} \< \zz_{\mathsf{sv}_-(\hat\vtheta)}, \hat\vtheta \>
        & \ge \sqrt{1 - \hat\rho^2}   \bigl( \< \zz_{\mathsf{v}_+(\wt\vtheta)}, \wt\vtheta \>  -  \< \zz_{\mathsf{v}_-(\wt\vtheta)}, \wt\vtheta \>  \bigr) 
        - 4 \hat\rho \max_{i \in [n]} \abs{g_i}
        - \sqrt{1 - \hat\rho^2} \< \zz_{\mathsf{sv}_+(\hat\vtheta)}, \hat\vtheta \> \\
        & \ge \sqrt{1 - \hat\rho^2} \bigl( + \< \zz_{\mathsf{v}_+(\wt\vtheta)}, \wt\vtheta \>
        - \| \bP_{\vmu}^\perp \bar\zz_+ \|_2  \bigr)
        - 6 \hat\rho \max_{i \in [n]} \abs{g_i}
        - \sqrt{1 - \hat\rho^2} \< \zz_{\mathsf{v}_-(\wt\vtheta)}, \wt\vtheta \>.
    \end{aligned}
\end{equation}
To derive its asymptotic order, we first define two statistics that are closely related to $\wt\vtheta$: 
\begin{equation}\label{eq:wt_theta_pm}
    \wt\vtheta_+ := \frac{\bP_{\vmu}^\perp \bar\zz_+}{\| \bP_{\vmu}^\perp \bar\zz_+ \|_2},
    \qquad
    \wt\vtheta_- := \frac{-\bP_{\vmu}^\perp \bar\zz_-}{\| \bP_{\vmu}^\perp \bar\zz_- \|_2}.
\end{equation}
Then, the difference terms inside the parentheses in \cref{eq:zSV_lower} can be expressed as
\begin{equation}\label{eq:zV_diff}
    \begin{aligned}
        \phantom{+} \< \zz_{\mathsf{v}_+(\wt\vtheta)}, \wt\vtheta \> - \| \bP_{\vmu}^\perp \bar\zz_+ \|_2
        & = \min_{i \in \mathcal{I}_+} \< + \zz_i, \wt\vtheta \> - \< \bar\zz_+, \wt\vtheta_+ \>
        = \min_{i \in \mathcal{I}_+} \< \zz_i - \bar\zz_+, \wt\vtheta \> + \< \bar\zz_+, \wt\vtheta - \wt\vtheta_+ \>,
        \\
        - \< \zz_{\mathsf{v}_-(\wt\vtheta)}, \wt\vtheta \> - \| \bP_{\vmu}^\perp \bar\zz_- \|_2
        & = \min_{i \in \mathcal{I}_-} \< - \zz_i, \wt\vtheta \> + \< \bar\zz_-, \wt\vtheta_- \>
        = \min_{i \in \mathcal{I}_-} \< \bar\zz_- - \zz_i, \wt\vtheta \> - \< \bar\zz_-, \wt\vtheta - \wt\vtheta_- \>.
    \end{aligned}
\end{equation}
Now we study the two terms on the R.H.S. of \cref{eq:zV_diff}. For the first term, based on \cref{eq:max_z},
\begin{equation}\label{eq:zV_diff_1}
    \begin{aligned}
        \min_{i \in \mathcal{I}_+} \< \zz_i - \bar\zz_+, \wt\vtheta \>
        \ge -\max_{i \in \mathcal{I}_+} \bigl| \< \zz_i - \bar\zz_+, \wt\vtheta \> \bigr|
        & = \wt O_\P(1), \\
        \min_{i \in \mathcal{I}_-} \< \bar\zz_- - \zz_i, \wt\vtheta \>
        \ge -\max_{i \in \mathcal{I}_-} \bigl| \< \zz_i - \bar\zz_-, \wt\vtheta \> \bigr|
        & = \wt O_\P(1).
    \end{aligned}
\end{equation}
For the second term,
\begin{equation}\label{eq:zV_diff_2+}
    \begin{aligned}
        \< \bar\zz_+, \wt\vtheta - \wt\vtheta_+ \>
        & =
        \frac{1}{\| \bP_{\vmu}^\perp \wt\zz \|_2} \< \bar\zz_+, \bP_{\vmu}^\perp \wt\zz \> 
        - \frac{1}{\| \bP_{\vmu}^\perp \bar\zz_+ \|_2} \< \bar\zz_+, \bP_{\vmu}^\perp \bar\zz_+ \> \\
        & = 
        \frac{1}{\| \bP_{\vmu}^\perp \wt\zz \|_2} 
        \left\{  \< \bar\zz_+, \bP_{\vmu}^\perp \wt\zz \>  -  \< \bar\zz_+, \bP_{\vmu}^\perp \bar\zz_+ \> \cdot \frac{\| \bP_{\vmu}^\perp \wt\zz \|_2}{\| \bP_{\vmu}^\perp \bar\zz_+ \|_2} \right\}
        \\
        & \overset{\mathmakebox[0pt][c]{\text{(i)}}}{\ge}
        \frac{1}{\| \bP_{\vmu}^\perp \wt\zz \|_2} 
        \left\{  \< \bar\zz_+, \bP_{\vmu}^\perp \wt\zz \> 
        - \< \bar\zz_+, \bP_{\vmu}^\perp \bar\zz_+ \> \cdot 
        \frac12 \biggl( 1 +  \frac{\| \bP_{\vmu}^\perp \bar\zz_- \|_2}{\| \bP_{\vmu}^\perp \bar\zz_+ \|_2} \biggr) 
        \right\} \\
        & \overset{\mathmakebox[0pt][c]{\text{(ii)}}}{=} - \frac{1}{2\| \bP_{\vmu}^\perp \wt\zz \|_2} \left( 
            \< \bar\zz_+, \bP_{\vmu}^\perp \bar\zz_-  \> + \| \bP_{\vmu}^\perp \bar\zz_+ \|_2 \| \bP_{\vmu}^\perp \bar\zz_- \|_2
         \right) \\
        & \overset{\mathmakebox[0pt][c]{\text{(iii)}}}{=}
         - \sqrt{\frac{\pi n}{d}}\big(1 + o_\P(1) \big) \left\{ 
            O_\P\biggl(  \sqrt{\frac{d}{\pi n^2}} \biggr) + \sqrt{\frac{d}{\pi n}}\sqrt{\frac{d}{n}} \big(1 + o_\P(1) \big)
         \right\} \\
        & = - \sqrt{\frac{d}{n}}\big(1 + o_\P(1) \big)
        = o_\P(1),
    \end{aligned}
\end{equation}
where (i) is from triangular inequality $2\| \bP_{\vmu}^\perp \wt\zz \|_2 \le \| \bP_{\vmu}^\perp \bar\zz_+ \|_2 + \| \bP_{\vmu}^\perp \bar\zz_- \|_2$, (ii) uses $2 \wt\zz - \bar z_+ = -\bar z_-$, and (iii) applies the asymptotic results \cref{eq:P_wtz}, \eqref{eq:Pzpm_asymp}, and the fact that $\bar\zz_+ \indep \bar\zz_-$, 
\begin{equation*}
    \< \bar\zz_{+}, \bP_{\vmu}^\perp \bar\zz_{-} \> = \frac{1}{\sqrt{n_+ n_-}} O_{\P}(\| \bP_{\vmu}^\perp \|_\mathrm{F}) = O_\P\biggl( \sqrt{\frac{d}{\pi n^2}}  \biggr),
\end{equation*}
by \cref{lem:subG_concentrate}\ref{lem:subG-Bernstein}. Similarly, we also have
\begin{equation}\label{eq:zV_diff_2-}
    \begin{aligned}
        - \< \bar\zz_-, \wt\vtheta - \wt\vtheta_- \>
        & =
        - \frac{1}{\| \bP_{\vmu}^\perp \wt\zz \|_2} \< \bar\zz_-, \bP_{\vmu}^\perp \wt\zz \> 
        - \frac{1}{\| \bP_{\vmu}^\perp \bar\zz_- \|_2} \< \bar\zz_-, \bP_{\vmu}^\perp \bar\zz_- \> \\
        % & = 
        % - \frac{1}{\| \bP_{\vmu}^\perp \wt\zz \|_2} 
        % \left\{  \< \bar\zz_-, \bP_{\vmu}^\perp \wt\zz \>  +  \< \bar\zz_-, \bP_{\vmu}^\perp \bar\zz_- \> \cdot \frac{\| \bP_{\vmu}^\perp \wt\zz \|_2}{\| \bP_{\vmu}^\perp \bar\zz_- \|_2} \right\}
        % \\
        % & \ge
        % -\frac{1}{\| \bP_{\vmu}^\perp \wt\zz \|_2} 
        % \left\{  \< \bar\zz_-, \bP_{\vmu}^\perp \wt\zz \> 
        % + \< \bar\zz_-, \bP_{\vmu}^\perp \bar\zz_- \> \cdot 
        % \frac12 \biggl( 1 +  \frac{\| \bP_{\vmu}^\perp \bar\zz_+ \|_2}{\| \bP_{\vmu}^\perp \bar\zz_- \|_2} \biggr) 
        % \right\} \\
        & \ge - \frac{1}{2\| \bP_{\vmu}^\perp \wt\zz \|_2} \left( 
            \< \bar\zz_+, \bP_{\vmu}^\perp \bar\zz_-  \> + \| \bP_{\vmu}^\perp \bar\zz_+ \|_2 \| \bP_{\vmu}^\perp \bar\zz_- \|_2
         \right) \\
        & = o_\P(1).
    \end{aligned}
\end{equation}
Substituting \cref{eq:zV_diff_1}, \eqref{eq:zV_diff_2+}, and \eqref{eq:zV_diff_2-} into \cref{eq:zV_diff}, we get
\begin{equation}
    \label{eq:zv_diff_asymp}
    \< \zz_{\mathsf{v}_+(\wt\vtheta)}, \wt\vtheta \> - \| \bP_{\vmu}^\perp \bar\zz_+ \|_2
    \ge \wt O_\P(1),
    \qquad
    - \< \zz_{\mathsf{v}_-(\wt\vtheta)}, \wt\vtheta \> - \| \bP_{\vmu}^\perp \bar\zz_- \|_2
    \ge \wt O_\P(1).
\end{equation}
And combining this with \cref{eq:Pzpm_asymp}, we have
\begin{equation}
    \label{eq:zv_wt_asymp}
    \< \zz_{\mathsf{v}_+(\wt\vtheta)}, \wt\vtheta \> 
    \ge \sqrt{\frac{d}{\pi n}}\big(1 + o_\P(1) \big) + \wt O_\P(1),
    \qquad
    - \< \zz_{\mathsf{v}_-(\wt\vtheta)}, \wt\vtheta \> 
    \ge \wt O_\P(1).
\end{equation}
Plugging \cref{eq:zv_diff_asymp}, \eqref{eq:zv_wt_asymp}, and \eqref{eq:g_asymp} into \cref{eq:zSV_lower} gives the asymptotic lower bounds (involving $\hat\rho$):
\begin{equation}
    \label{eq:zSV_lower_asymp}
    \begin{aligned}
        \phantom{-} \sqrt{1 - \hat\rho^2} \< \zz_{\mathsf{sv}_+(\hat\vtheta)}, \hat\vtheta \>  
        & \ge  \sqrt{1 - \hat\rho^2} \biggl( \sqrt{\frac{d}{\pi n}}\big(1 + o_\P(1) \big) 
        + \wt O_\P(1)
        \biggr)
        + 
        \hat\rho \cdot \wt O_\P(1) , \\
        - \sqrt{1 - \hat\rho^2} \< \zz_{\mathsf{sv}_-(\hat\vtheta)}, \hat\vtheta \> 
        & \ge  \sqrt{1 - \hat\rho^2} \cdot \wt O_\P(1) + 
        \hat\rho \cdot \wt O_\P(1) .
    \end{aligned}
\end{equation}


~\\
\noindent
Finally, combining upper bounds \cref{eq:zSV_upper_asymp} and lower bounds \cref{eq:zSV_lower_asymp}, we obtain the exact order
\begin{equation*}
    \begin{aligned}
        \phantom{-} \sqrt{1 - \hat\rho^2} \< \zz_{\mathsf{sv}_+(\hat\vtheta)}, \hat\vtheta \>  
        & =  \sqrt{1 - \hat\rho^2} \biggl( \sqrt{\frac{d}{\pi n}}\big(1 + o_\P(1) \big) 
        + \wt O_\P(1)
        \biggr)
        + 
        \hat\rho \cdot \wt O_\P(1) \\
        & = \sqrt{1 - \hat\rho^2} \sqrt{\frac{d}{\pi n}}\big(1 + o_\P(1) \big) 
        + \wt O_\P(1),
        \\
        - \sqrt{1 - \hat\rho^2} \< \zz_{\mathsf{sv}_-(\hat\vtheta)}, \hat\vtheta \> 
        & =  \sqrt{1 - \hat\rho^2} \cdot \wt O_\P(1) + 
        \hat\rho \cdot \wt O_\P(1) \\
        & = \wt O_\P(1).
    \end{aligned}
\end{equation*}


\paragraph{\ref{lem:theta_hat_z(a)}:}
If $a < b + c$, according to \cref{lem:rho_hat}\ref{lem:rho_hat(a)}, $\hat\rho = 1 - o_\P(1)$. It is clear that \cref{lem:theta_hat_z} holds for $\hat\rho = \pm 1$. Now, restrict on the event $\{ |\hat\rho| < 1 \}$.
\begin{itemize}
    \item If $a > \frac{b}{2} + c$, then $\sqrt{1 - \hat\rho^2} = \frac12 d^{(a-b-c)/2}\bigl( 1 + o_\P(1) \bigr)$, hence
    \begin{equation*}
        \sqrt{1 - \hat\rho^2} \< \zz_{\mathsf{sv}_+(\hat\vtheta)}, \hat\vtheta \>
        = \frac12 d^{a - \frac{b}{2} - c}\big(1 + o_\P(1) \big).
    \end{equation*}
    \item If $a \le \frac{b}{2} + c$, then $\sqrt{1 - \hat\rho^2}\sqrt{d/\pi n} = \wt O_\P(1)$, hence
    \begin{equation*}
        \sqrt{1 - \hat\rho^2} \< \zz_{\mathsf{sv}_+(\hat\vtheta)}, \hat\vtheta \>
        = \wt O_\P(1).
    \end{equation*}
\end{itemize}
\paragraph{\ref{lem:theta_hat_z(b)}:}
If $a > b + c$, according to \cref{lem:rho_hat}\ref{lem:rho_hat(b)}, $\hat\rho = o_\P(1)$. Hence, on the event $\{ |\hat\rho| < 1 \}$,
\begin{equation*}
    \< \zz_{\mathsf{sv}_+(\hat\vtheta)}, \hat\vtheta \>
    = \sqrt{\frac{d}{\pi n}}\big(1 + o_\P(1) \big).
\end{equation*}
This also holds regardless of $\hat\rho$, since $\P( |\hat\rho| < 1 ) \to 1$ as $d \to \infty$. Then we complete the proof.
% ~\\
% \noindent
% At last, remind that all the results above hold on event $\{ \hat\kappa > 0\}$. Since $\P( \hat\kappa > 0) \to 1$ as $d \to \infty$, these asymptotic results involving $o_\P(\,\cdot\,)$ and $\wt O_\P(\,\cdot\,)$ also hold on the whole sample space $\Omega$.
\end{proof}


%%%%%%%%%%%%%%%%%%%%%%%%%%%%%%%%%%%%%%%%%%%%%%%%%%%%%%%%%%%%%%%%%%%%%%%%%%%%%%%%%%%%%%
% Old ver. with mistakes: w/o considering lower bound, uniform convergence, \rho = 1 %
%%%%%%%%%%%%%%%%%%%%%%%%%%%%%%%%%%%%%%%%%%%%%%%%%%%%%%%%%%%%%%%%%%%%%%%%%%%%%%%%%%%%%%
% (a): \update{Clearly, the conclusion holds for $\hat\rho = 1$.} If $a < b + c$, by \cref{lem:rho_hat}\ref{lem:rho_hat(a)} and \cref{eq:z_v_bound},
% \begin{equation*}
%     \begin{aligned}
%     &   \sqrt{1 - \hat\rho^2} \< \zz_{\mathsf{sv}_+(\hat\vtheta)}, \hat\vtheta \>
%     \le  \sqrt{1 - \hat\rho^2} \< \zz_{\mathsf{v}_+(\hat\vtheta)}, \hat\vtheta \>
%     + \hat\rho ( g_{\mathsf{v}_+(\hat\vtheta)} - g_{\mathsf{sv}_+(\hat\vtheta)}) \\
%     \le {} &  \sqrt{1 - \hat\rho^2}\sqrt{\frac{d}{\pi n}}\big(1 + o_\P(1) \big) + o_\P(1) \\
%     \le {} &  \begin{cases} 
%         \,  d^{(a-b-c)/2} \cdot d^{(a-c)/2} \big(1 + o_\P(1) \big) + o_\P(1)
%         =  d^{a-\frac{b}{2}-c} \big(1 + o_\P(1) \big) , & \ \text{if} \ a > \frac{b}{2} + c, \\
%         \,  O_\P(1)\big(1 + o_\P(1) \big) + o_\P(1) = O_\P(1) , & \ \text{if} \ a \le \frac{b}{2} + c, \end{cases}
%     \end{aligned}
% \end{equation*}
% and
% \begin{equation*}
%     \begin{aligned}
%     &   \sqrt{1 - \hat\rho^2} \< \zz_{\mathsf{sv}_-(\hat\vtheta)}, \hat\vtheta \>
%     \le  \sqrt{1 - \hat\rho^2} \< \zz_{\mathsf{v}_-(\hat\vtheta)}, \hat\vtheta \>
%     + \hat\rho ( g_{\mathsf{v}_-(\hat\vtheta)} - g_{\mathsf{sv}_-(\hat\vtheta)}) \\
%     \le {} &  \sqrt{1 - \hat\rho^2}\sqrt{\frac{d}{n}}\big(1 + o_\P(1) \big) + o_\P(1) \\
%     \le {} &  \begin{cases} 
%         \,  d^{(a-b-c)/2} \cdot d^{-c/2} \big(1 + o_\P(1) \big) + o_\P(1)
%          , & \ \text{if} \ a > \frac{b}{2} + c, \\
%         \,  O_\P(d^{-a/2})\big(1 + o_\P(1) \big) + o_\P(1)  , & \ \text{if} \ a \le \frac{b}{2} + c. \end{cases} \\
%     = {} & o_\P(1).
%     \end{aligned}
% \end{equation*}
% (b): If $a > b + c$, which implies $a - c > 0$. By \cref{lem:rho_hat}\ref{lem:rho_hat(b)}, $\hat\rho = o_\P(1)$, then
% \begin{equation}\label{eq:z_sv_bound}
%     \begin{aligned}
%         \< \zz_{\mathsf{sv}_+(\hat\vtheta)}, \hat\vtheta \> & \le  \< \zz_{\mathsf{v}_+(\hat\vtheta)}, \hat\vtheta \> + \frac{\hat\rho(g_{\mathsf{v}_+(\hat\vtheta)} - g_{\mathsf{sv}_+(\hat\vtheta)})}{\sqrt{1 - \hat\rho^2}}  = \< \zz_{\mathsf{v}_+(\hat\vtheta)}, \hat\vtheta \> + o_\P(1)
%         \le \sqrt{\frac{d}{\pi n}}\big(1 + o_\P(1) \big), \\
%         \< \zz_{\mathsf{sv}_-(\hat\vtheta)}, \hat\vtheta \> & \le  \< \zz_{\mathsf{v}_-(\hat\vtheta)}, \hat\vtheta \> + \frac{\hat\rho(g_{\mathsf{v}_-(\hat\vtheta)} - g_{\mathsf{sv}_-(\hat\vtheta)})}{\sqrt{1 - \hat\rho^2}}  = \< \zz_{\mathsf{v}_-(\hat\vtheta)}, \hat\vtheta \> + o_\P(1)
%         \le O_\P(1).
%     \end{aligned}
% \end{equation}
% Now we show that these bounds are nearly attained by $\wt\vtheta$. Since $\bar\zz_+ \indep \bar\zz_-$, by \cref{lem:subG_concentrate}\ref{lem:subG-Bernstein},
% \begin{equation*}
%     \< \bar\zz_{+}, \bP_{\vmu}^\perp \bar\zz_{-} \> = \< \bar\zz_{-}, \bP_{\vmu}^\perp \bar\zz_{+} \> = \frac{1}{\sqrt{n_+ n_-}} O_{\P}(\| \bP_{\vmu}^\perp \|_\mathrm{F}) = O_\P\biggl( \sqrt{\frac{d}{\pi n^2}}  \biggr).
% \end{equation*}
% Recall that $\wt\vtheta$ is defined in \cref{eq:param_star}. Therefore, combining with \cref{eq:P_wtz},
% \begin{equation*}
%     \begin{aligned}
%         \<  \bar\zz_{+}, \wt\vtheta \> & = \frac{\| \bP_{\vmu}^\perp \bar\zz_+ \|_2^2}{2 \| \bP_{\vmu}^\perp \wt\zz \|_2} + \frac{\< \bar\zz_{+}, \bP_{\vmu}^\perp \bar\zz_{-} \>}{2 \|\bP_{\vmu}^\perp \wt\zz \|_2} = \sqrt{\frac{d}{\pi n}}\big(1 + o_\P(1) \big) + O_\P\biggl( \frac{1}{\sqrt{n}} \biggr)
%         = \sqrt{\frac{d}{\pi n}}\big(1 + o_\P(1) \big), \\
%         \<  \bar\zz_{-}, \wt\vtheta \> & = \frac{\| \bP_{\vmu}^\perp \bar\zz_- \|_2^2}{2 \| \bP_{\vmu}^\perp \wt\zz \|_2} + \frac{\< \bar\zz_{-}, \bP_{\vmu}^\perp \bar\zz_{+} \>}{2 \|\bP_{\vmu}^\perp \wt\zz \|_2} = \sqrt{\frac{\pi d}{n}}\big(1 + o_\P(1) \big) + O_\P\biggl( \frac{1}{\sqrt{n}} \biggr)
%         =  O_{\P}(1). \\
%     \end{aligned}
% \end{equation*}
% Finally, use the results from \cref{eq:z_theta_diff+} and \eqref{eq:z_theta_diff-}, we have
% \begin{equation}\label{eq:z_theta_star}
%     \begin{aligned}
%         \< \zz_i, \wt\vtheta \> & = \<  \bar\zz_{+}, \wt\vtheta \> + \< \zz_i - \bar\zz_+, \wt\vtheta \>
%         = \sqrt{\frac{d}{\pi n}}\big(1 + o_\P(1) \big) + O_\P(1), 
%         &  \forall\, i \in \mathcal{I}_+,  \\
%         \< \zz_i, \wt\vtheta \> & = \<  \bar\zz_{-}, \wt\vtheta \> + \< \zz_i - \bar\zz_-, \wt\vtheta \>
%         = O_\P(1), 
%         &  \forall\, i \in \mathcal{I}_-. \\
%     \end{aligned}
% \end{equation}
% Note that these magnitudes match those in \cref{eq:z_sv_bound}. Recall that by definition and \cref{eq:svm_sv_bal},
% \begin{equation*}
%     \begin{aligned}
%         \hat\vtheta & = \argmax_{ \vtheta \in \S^{d-1}, \vtheta \perp \vmu } \kappa(\hat\rho, \vtheta, \hat\beta_0)
%     = \argmax_{ \vtheta \in \S^{d-1}, \vtheta \perp \vmu } \frac{\kappa_{\mathsf{sv}_+(\vtheta)}(\hat\rho, \vtheta, \hat\beta_0) + \kappa_{\mathsf{sv}_-(\vtheta)}(\hat\rho, \vtheta, \hat\beta_0)}{2} \\
%     & = \argmax_{ \vtheta \in \S^{d-1}, \vtheta \perp \vmu } 
%     \left\{ \hat\rho \norm{\vmu}_2 + \hat\rho \frac{g_{\mathsf{sv}_+(\vtheta)} + g_{\mathsf{sv}_-(\vtheta)} }{2} + \sqrt{1 - \hat\rho^2}  \frac{ \< \zz_{\mathsf{sv}_+(\vtheta)}, \vtheta \>  +  \< \zz_{\mathsf{sv}_-(\vtheta)}, \vtheta \>  }{2} \right\}   \\
%     & = \argmax_{ \vtheta \in \S^{d-1}, \vtheta \perp \vmu } \left\{  o_\P(1) + \left( \< \zz_{\mathsf{sv}_+(\vtheta)}, \vtheta \>  +  \< \zz_{\mathsf{sv}_-(\vtheta)}, \vtheta \> \right)  \right\},
%     \end{aligned}
% \end{equation*}
% then by using \cref{eq:z_theta_star}, we obtain a lower bound
% \begin{equation*}
%     \< \zz_{\mathsf{sv}_+(\hat\vtheta)}, \hat\vtheta \>  +  \< \zz_{\mathsf{sv}_-(\hat\vtheta)}, \hat\vtheta \> 
%     \ge \< \zz_{\mathsf{sv}_+(\wt\vtheta)}, \wt\vtheta \>  +  \< \zz_{\mathsf{sv}_-(\wt\vtheta)}, \wt\vtheta \> + o_\P(1)
%     \ge \sqrt{\frac{d}{\pi n}}\big(1 + o_\P(1) \big).
% \end{equation*}
% In the end, combining it with the upper bound \cref{eq:z_sv_bound}, we reach to conclusion
% \begin{equation*}
%     \< \zz_{\mathsf{sv}_+(\hat\vtheta)}, \hat\vtheta \> = \sqrt{\frac{d}{\pi n}}\bigl(1 + o_{\P}(1)\bigr).
% \end{equation*}

\subsubsection{Asymptotic expression of $\hat\beta_0$: Proof of \cref{lem:beta0_asymp}}

Finally, we consider arbitrary $\tau \ge 1$ and give an explicit expression for $\hat\beta_0$ with its asymptotics. Be aware that $\tau = \tau_d$ may depend on $d$.
\begin{lem}[Asymptotics of $\hat\beta_0$] \label{lem:beta0_asymp}
    Suppose that $a < c + 1$ and $\tau \ge 1$. Then we have
    \begin{equation*}
        \begin{aligned}
            \hat\beta_0 & = \left(1 - \frac{2}{\tau + 1}\right) \hat\rho\norm{\vmu}_2 
            - \hat\rho \frac{\tau g_{\mathsf{sv}_-(\hat\vtheta)} + g_{\mathsf{sv}_+(\hat\vtheta)} }{\tau + 1}
            - \sqrt{1 - \hat\rho^2} \frac{ \tau \< \zz_{\mathsf{sv}_-(\hat\vtheta)}, \hat\vtheta \> + \< \zz_{\mathsf{sv}_+(\hat\vtheta)}, \hat\vtheta \> }{\tau + 1} \\
            & = \left(1 - \frac{2}{\tau + 1}\right) \hat\rho\norm{\vmu}_2 
            - \frac{1}{\tau + 1} \sqrt{1 - \hat\rho^2}\< \zz_{\mathsf{sv}_+(\hat\vtheta)}, \hat\vtheta \> + \wt O_\P(1).
        \end{aligned}
    \end{equation*}
    \begin{enumerate}[label=(\alph*)]
        \item \label{lem:beta0_asymp(a)}
        If $a < b + c$, then
        \begin{equation*}
            \begin{aligned}
                \hat\beta_0 & = \left(1 - \frac{2}{\tau + 1}\right) \hat\rho\norm{\vmu}_2 
            - \frac{1}{\tau + 1} \wt O_\P(d^{a-\frac{b}{2}-c} \vee 1) 
            + \wt O_\P(1) \\
            & = \left(1 - \frac{2}{\tau + 1}\right) d^{b/2}\bigl(1 + o_{\P}(1)\bigr) 
            - \frac{1}{\tau + 1} \wt O_\P(d^{a-\frac{b}{2}-c} \vee 1)
            + \wt O_\P(1).
            \end{aligned}
    \end{equation*}
        \item \label{lem:beta0_asymp(b)}
        If $a > b + c$, then
        \begin{equation*}
            \begin{aligned}
            & \hat\beta_0  = \left(1 - \frac{2}{\tau + 1}\right) \hat\rho\norm{\vmu}_2 
            - \frac{1}{\tau + 1}\sqrt{\frac{d}{\pi n}}\bigl(1 + o_{\P}(1)\bigr) 
            + \wt O_\P(1) \\
            = {} & 
            \begin{cases} 
                \,  \displaystyle \left(1 - \frac{2}{\tau + 1}\right) 2d^{(2b-a+c)/2} \bigl(1 + o_{\P}(1)\bigr) 
                - \frac{1}{\tau + 1} d^{(a-c)/2} \bigl(1 + o_{\P}(1)\bigr) 
                + \wt O_\P(1) , & \ \text{if} \ a < 2b + c, \\
                \,  \displaystyle \phantom{\left(1 - \frac{2}{\tau + 1}\right) 2d^{(2b-a+c)/2} \bigl(1 + o_{\P}(1)\bigr) 
                }
                - \frac{1}{\tau + 1} d^{(a-c)/2} \bigl(1 + o_{\P}(1)\bigr) 
                + \wt O_\P(1) , & \ \text{if} \ a > 2b + c. \end{cases}
        \end{aligned}
    \end{equation*}
    \end{enumerate}
\end{lem}
\begin{proof}
    We rewrite the \emph{margin-balancing} condition \cref{eq:margin_pm}, \eqref{eq:margin-bal} in terms of $\hat\rho, \hat\vtheta, \hat\beta_0$, which generalizes \cref{eq:svm_sv_bal} to arbitrary $\tau \ge 1$:
    \begin{equation*}
        \begin{aligned}
           \kappa(\hat\rho, \hat\vtheta, \hat\beta_0) & = \kappa_{\mathsf{sv}_+(\hat\vtheta)}(\hat\rho, \hat\vtheta, \hat\beta_0) = 
           \tau^{-1} \Bigl(
           \hat\rho \norm{\vmu}_2 + \hat\beta_0 + \hat\rho g_{\mathsf{sv}_+(\hat\vtheta)} + \sqrt{1 - \hat\rho^2} \< \zz_{\mathsf{sv}_+(\hat\vtheta)}, \hat\vtheta \> 
           \Bigr)
           \\
           & = \kappa_{\mathsf{sv}_-(\hat\vtheta)}(\hat\rho, \hat\vtheta, \hat\beta_0) = 
           \phantom{ \tau^{-1} \Bigl( }
           \hat\rho \norm{\vmu}_2 - \hat\beta_0 - \hat\rho g_{\mathsf{sv}_-(\hat\vtheta)} - \sqrt{1 - \hat\rho^2} \< \zz_{\mathsf{sv}_-(\hat\vtheta)}, \hat\vtheta \>
           .
        \end{aligned}
    \end{equation*}
    Then we can solve the expression for $\hat\beta_0$ (this equals \cref{eq:beta0_hat} in \cref{lem:indep_tau} with parametrization \cref{eq:def-rho-theta_hat}). Its asymptotic simplification is followed by \cref{eq:g_asymp}:
    \begin{equation*}
        \abs{\hat\rho \frac{\tau g_{\mathsf{sv}_-(\hat\vtheta)} + g_{\mathsf{sv}_+(\hat\vtheta)} }{\tau + 1} } 
        \le \abs{\hat\rho} \frac{\tau |g_{\mathsf{sv}_-(\hat\vtheta)}| + |g_{\mathsf{sv}_+(\hat\vtheta)}| }{\tau + 1}
        \le \max_{i \in [n]} \abs{g_i} = \wt O_\P(1),
    \end{equation*}
    and \cref{lem:theta_hat_z}:
    \begin{equation*}
        \abs{
        \frac{\tau}{\tau + 1} \sqrt{1 - \hat\rho^2}\< \zz_{\mathsf{sv}_-(\hat\vtheta)}, \hat\vtheta \>
        } = \wt O_\P(1).
    \end{equation*}
    For \textbf{\ref{lem:beta0_asymp(a)}}, plugging $\hat\rho = 1 - o_\P(1)$ by \cref{lem:rho_hat}\ref{lem:rho_hat(a)} and asymptotics of $\< \zz_{\mathsf{sv}_+(\hat\vtheta)}, \hat\vtheta \>$ by \cref{lem:theta_hat_z}\ref{lem:theta_hat_z(a)}. For \textbf{\ref{lem:beta0_asymp(b)}}, plugging $\hat\rho = 2d^{(b-a+c)/2}\bigl(1 + o_\P(1)\bigr)$ by \cref{lem:rho_hat}\ref{lem:rho_hat(b)} from i., while $\hat\rho\norm{\bmu}_2 = o_\P(1)$ from ii., and asymptotics of $\< \zz_{\mathsf{sv}_+(\hat\vtheta)}, \hat\vtheta \>$ by \cref{lem:theta_hat_z}\ref{lem:theta_hat_z(b)}. This completes the proof.
\end{proof}


\subsection{Classification error: Completing the proof of \cref{thm:main_high-imbal}}
\label{subsec:highimb_err}

\begin{proof}[\textbf{Proof of \cref{thm:main_high-imbal}}]
Let $(\xx_\mathrm{new}, y_\mathrm{new})$ be a test data point independent of the training set $\{(\xx_i, y_i)\}_{i=1}^n$, such that $\xx_\mathrm{new} = y_\mathrm{new} \bmu + \zz_\mathrm{new}$, and $\zz_\mathrm{new} \sim \subGind(\bzero, \bI_d; K)$. Recall $\hat f(\xx) = \< \xx, \hat\vbeta \> + \hat\beta_0$. Following the same decomposition as \cref{eq:logits},
\begin{equation*}
    \begin{aligned}
        y_\mathrm{new} \hat f(\xx_\mathrm{new}) & = y_\mathrm{new} (\< \xx_\mathrm{new}, \hat\vbeta \> + \hat\beta_0) \\
        & = \hat\rho \norm{\vmu}_2 + y_\mathrm{new} \hat\beta_0 + 
        y_\mathrm{new} \bigl(  \hat\rho g_\mathrm{new} + \sqrt{1 - \hat\rho^2} \< \zz_\mathrm{new}, \hat\vtheta \> \bigr) \\
        & = \hat\rho \norm{\vmu}_2 + y_\mathrm{new} \hat\beta_0 + y_\mathrm{new} G_d,
    \end{aligned}
\end{equation*}
where 
\begin{equation*}
g_\mathrm{new} := \left\< \zz_\mathrm{new}, \frac{\vmu}{\norm{\vmu}_2} \right\>,
\qquad 
G_d := \hat\rho g_\mathrm{new} + \sqrt{1 - \hat\rho^2} \< \zz_\mathrm{new}, \hat\vtheta \>.
\end{equation*}
Therefore, the minority and majority test errors are
\begin{equation*}
    \begin{aligned}
        \Err_+ & = \P\left( \hat f(\xx_\mathrm{new}) \le 0 \,\big|\, y_\mathrm{new} = +1 \right)
        = \P\left( \hat\rho \norm{\vmu}_2 + \hat\beta_0 + G_d \le 0 \right), \\
        \Err_- & = \P\left( \hat f(\xx_\mathrm{new}) > 0 \,\big|\, y_\mathrm{new} = -1 \right)
        = \P\left( \hat\rho \norm{\vmu}_2 - \hat\beta_0 - G_d < 0 \right). \\
    \end{aligned}
\end{equation*}
By \cref{lem:subG_concentrate}\ref{lem:subG-Hoeffding}, we have $\norm{g_\mathrm{new}}_{\psi_2}, \| \< \zz_\mathrm{new}, \hat\vtheta \> \|_{\psi_2} \lesssim K$, since $\zz_\mathrm{new} \indep (\hat\rho, \hat\vtheta)$ and then $\forall\, t > 0$,
\begin{equation*}
     \P\left( \bigl| \< \zz_\mathrm{new}, \hat\vtheta \> \bigr| > t \right)
     =  \E\left[ \P\left( \bigl| \< \zz_\mathrm{new}, \hat\vtheta \> \bigr| > t \,\big|\, \hat\vtheta \right) \right]
     \le 2 e^{-ct^2/K^2}, \qquad \text{for some} ~ c > 0.
\end{equation*}
Then by \cref{lem:subG}\ref{lem:subG-a},
\begin{equation*}
     \norm{G_d}_{\psi_2} \le 
     \norm{\hat\rho g_\mathrm{new}}_{\psi_2} + \| \sqrt{1 - \hat\rho^2} \< \zz_\mathrm{new}, \hat\vtheta \>\|_{\psi_2}
     \le \norm{g_\mathrm{new}}_{\psi_2} + \| \< \zz_\mathrm{new}, \hat\vtheta \> \|_{\psi_2} \lesssim K,
\end{equation*}
which implies $G_d = O_\P(1)$. 


\vspace{0.5\baselineskip}
\noindent
\textbf{\ref{thm:high-imb_high}. High signal:}
If $a < b + c$, then we have $\hat\rho = 1 - o_\P(1)$ by \cref{lem:rho_hat}\ref{lem:rho_hat(a)}. Therefore, according to \cref{lem:beta0_asymp}\ref{lem:beta0_asymp(a)}, for all $\tau_d \ge 1$, we have
\begin{align*}
        \hat\rho \norm{\vmu}_2 + \hat\beta_0
        & = \left(2 - \frac{2}{\tau_d + 1}\right) \hat\rho\norm{\vmu}_2 
        - \frac{1}{\tau_d + 1} \wt O_\P(d^{a-\frac{b}{2}-c} \vee 1) 
        + \wt O_\P(1) \\
        & \ge  d^{b/2}\bigl(1 + o_{\P}(1)\bigr)
        - \wt O_\P(d^{a-\frac{b}{2}-c} \vee 1)  \\
        & \overset{\mathmakebox[0pt][c]{\text{(i)}}}{=} d^{b/2}\bigl(1 + o_{\P}(1)\bigr),
        \qquad  \lim_{d \to \infty} d^{b/2} = +\infty,
\end{align*}
where (i) is because $d^{b/2} \gg d^{a-\frac{b}{2}-c}$, as $d \to \infty$. If $1 \le \tau_d \ll d^{b/2}$, we also have
\begin{align*}
    \hat\rho \norm{\vmu}_2 - \hat\beta_0
    & = \frac{2}{\tau_d + 1} \hat\rho\norm{\vmu}_2 
    + \frac{1}{\tau_d + 1} \wt O_\P(d^{a-\frac{b}{2}-c} \vee 1) + \wt O_\P(1) \\
    & = \frac{2}{\tau_d + 1} d^{b/2} 
    + \frac{1}{\tau_d + 1} \wt O_\P(d^{a-\frac{b}{2}-c} \vee 1) + \wt O_\P(1) \\
    & \overset{\mathmakebox[0pt][c]{\text{(ii)}}}{=} \frac{2}{\tau_d + 1} d^{b/2} \bigl(1 + o_{\P}(1)\bigr) + \wt O_\P(1) \\
    & \ge \tau_d^{-1} d^{b/2} \bigl(1 + o_{\P}(1)\bigr) + \wt O_\P(1),
    \qquad
    \lim_{d \to \infty} \tau_d^{-1} d^{b/2} = +\infty,
\end{align*}
where (ii) is because $(\tau_d + 1)^{-1}d^{b/2} \gg (\tau_d + 1)^{-1}d^{a-\frac{b}{2}-c}$ and $(\tau_d + 1)^{-1}d^{b/2} \gg (\log d)^k$, $\forall\, k \ge 0$, as $d \to \infty$. Under these conditions, both $\hat\rho \norm{\vmu}_2 \pm \hat\beta_0$ diverges to $+\infty$ with high probability, i.e.,
\begin{equation*}
    \lim_{d \to \infty} \P\left(\hat\rho \norm{\vmu}_2 + \hat\beta_0 + G_d > C\right) 
    =
    \lim_{d \to \infty} \P\left(\hat\rho \norm{\vmu}_2 - \hat\beta_0 - G_d > C\right) 
    = 1,
    \qquad
    \forall\, C \in \R.
\end{equation*}
% Conditioning on $\zz_\mathrm{new} = \bz_h$, by Cauchy--Schwarz inequality, we can show that $G_d$ is bounded:
% \begin{equation*}
%     \abs{G_d} \le \sqrt{ \abs{\left\< \bz_h, \frac{\vmu}{\norm{\vmu}_2} \right\>}^2 
%     +
%     \abs{ \< \bz_h, \hat\vtheta \>}^2 
%     }
%     \le \sqrt{2}\norm{\bz_h}_2.
% \end{equation*}
% Since $\zz_\mathrm{new} \indep (\hat\rho, \hat\vtheta, \hat\beta_0)$, then for each $\bz_h$, we have
% \begin{equation*}
%     \lim_{d \to \infty} \P\left(\hat\rho \norm{\vmu}_2 + \hat\beta_0 + G_d \le 0 \,\big|\, \zz_\mathrm{new} = \bz_h \right) 
%     =
%     \lim_{d \to \infty} \P\left(\hat\rho \norm{\vmu}_2 - \hat\beta_0 - G_d \le 0 \,\big|\, \zz_\mathrm{new} = \bz_h \right) 
%     = 0.
% \end{equation*}
% Finally, taking expection on both sides, by bounded convergence theorem, we obtain
Hence
\begin{equation*}
    \Err_+ = o(1), \qquad \Err_- = o(1).
\end{equation*}
This concludes the proof for high signal regime.


\vspace{0.5\baselineskip}
\noindent
\textbf{\ref{thm:high-imb_moderate}. Moderate signal:}
If $b + c < a < 2b + c$, then $\hat\rho = 2d^{(b-a+c)/2}\bigl(1 + o_\P(1)\bigr)$ by \cref{lem:rho_hat}\ref{lem:rho_hat(b)}. Therefore, according to \cref{lem:beta0_asymp}\ref{lem:beta0_asymp(b)}, if $\tau_d \gg d^{a-b-c}$, then
\begin{align*}
        \hat\rho \norm{\vmu}_2 + \hat\beta_0 
        & = \left(2 - \frac{2}{\tau_d + 1}\right) \hat\rho\norm{\vmu}_2 
        - \frac{1}{\tau_d + 1}\sqrt{\frac{d}{\pi n}}\bigl(1 + o_{\P}(1)\bigr) 
        + \wt O_\P(1)  \\
        & = 4 d^{(2b-a+c)/2}  \bigl(1 + o_{\P}(1)\bigr) - \tau_d^{-1} d^{(a-c)/2} \bigl(1 + o_{\P}(1)\bigr) + \wt O_\P(1) \\
        & \overset{\mathmakebox[0pt][c]{\text{(iii)}}}{=} 4 d^{(2b-a+c)/2}  \bigl(1 + o_{\P}(1)\bigr),
        \qquad
    \lim_{d \to \infty} d^{(2b-a+c)/2} = +\infty,
\end{align*}
where (iii) is because $d^{(2b-a+c)/2} \gg \tau_d^{-1} d^{(a-c)/2}$ and $d^{(2b-a+c)/2} \gg (\log d)^k$, $\forall\, k \ge 0$, as $d \to \infty$. If $1 \le \tau_d \ll d^{(a-c)/2}$, we also have
\begin{align*}
        \hat\rho \norm{\vmu}_2 - \hat\beta_0 
        & = \frac{2}{\tau_d + 1} \hat\rho\norm{\vmu}_2 
        + \frac{1}{\tau_d + 1}\sqrt{\frac{d}{\pi n}}\bigl(1 + o_{\P}(1)\bigr) 
        + \wt O_\P(1)  \\
        & = \frac{4}{\tau_d + 1} d^{(2b-a+c)/2} \bigl(1 + o_{\P}(1)\bigr) + \frac{1}{\tau_d + 1} d^{(a-c)/2} \bigl(1 + o_{\P}(1)\bigr) + \wt O_\P(1) \\
        & \overset{\mathmakebox[0pt][c]{\text{(iv)}}}{=} \frac{1}{\tau_d + 1} d^{(a-c)/2} \bigl(1 + o_{\P}(1)\bigr) + \wt O_\P(1) \\
        & \ge \frac12 \tau_d^{-1} d^{(a-c)/2} \bigl(1 + o_{\P}(1)\bigr) + \wt O_\P(1),
        \qquad 
        \lim_{d \to \infty} \tau_d^{-1} d^{(a-c)/2} = +\infty,
\end{align*}
where (iv) is from $(\tau_d + 1)^{-1} d^{(2b-a+c)/2} \ll (\tau_d + 1)^{-1} d^{(a-c)/2}$. Under these conditions on $\tau_d$, both $\hat\rho \norm{\vmu}_2 \pm \hat\beta_0 $ diverges to $+\infty$ with high probability. Using the same approach, we can show that
\begin{equation*}
    \Err_+ = o(1), \qquad \Err_- = o(1).
\end{equation*}

\vspace{0.5\baselineskip}
\noindent
Now suppose $\tau_d \asymp 1$, then again $\hat\rho \norm{\vmu}_2 - \hat\beta_0 \to + \infty$ and hence $\Err_- = o_\P(1)$ still holds. However,
\begin{align*}
        \hat\rho \norm{\vmu}_2 + \hat\beta_0 
        & = \left(2 - \frac{2}{\tau_d + 1}\right) \hat\rho\norm{\vmu}_2 
        - \frac{1}{\tau_d + 1}\sqrt{\frac{d}{\pi n}}\bigl(1 + o_{\P}(1)\bigr) 
        + \wt O_\P(1)  \\
        & \le 2 d^{(2b-a+c)/2} \bigl(1 + o_{\P}(1)\bigr) - C d^{(a-c)/2} \bigl(1 + o_{\P}(1)\bigr) + \wt O_\P(1), \\
        & \overset{\mathmakebox[0pt][c]{\text{(v)}}}{=} - C d^{(a-c)/2} \bigl(1 + o_{\P}(1)\bigr),
        \qquad 
        \lim_{d \to \infty} -d^{(a-c)/2} = -\infty,
\end{align*}
where (v) is because $d^{(2b-a+c)/2} \ll d^{(a-c)/2}$, and $C \in (0, \infty)$ is an absolute constant. As the result, $- \hat\rho \norm{\vmu}_2 - \hat\beta_0 $ diverges to $+\infty$ with high probability. Using the same approach, we have
\begin{equation*}
    \Err_+ = 1 - o(1).
\end{equation*}
This concludes the proof for moderate signal regime.




\vspace{0.5\baselineskip}
\noindent
\textbf{\ref{thm:high-imb_low}. Low signal:}
If $a > 2b + c$, then $\hat\rho \norm{\vmu}_2 = o_\P(1) > 0$ by \cref{lem:rho_hat}\ref{lem:rho_hat(b)}. Therefore,
\begin{align*}
        \Err_+ + \Err_- & 
        = \P\left( \hat\rho \norm{\vmu}_2 + \hat\beta_0 + G_d \le 0 \right) 
        + \P\left( \hat\rho \norm{\vmu}_2 - \hat\beta_0 - G_d < 0 \right) \\
        & = 
        1 - \P\left( - \hat\rho \norm{\vmu}_2 \le \hat\beta_0 + G_d < \hat\rho \norm{\vmu}_2 \right)
        \\
        % & = \P\left( o_\P(1) + \hat\beta_0 + G_d \le 0 \right) 
        % + \P\left( o_\P(1) + \hat\beta_0 + G_d > 0 \right) \\
        % & = \P\left( \hat\beta_0 + G_d \le 0 \right) 
        % + \P\left( \hat\beta_0 + G_d > 0 \right) + o_\P(1) \\
        & = 1 - o(1).
\end{align*}
Hence, we have $\Err_\mathrm{b} \ge \frac12 - o(1)$. This concludes the proof for low signal regime.


Finally, we complete the proof of \cref{thm:main_high-imbal}.
\end{proof}


% \newpage
% \noindent
% Let $\vtheta = \bP_{\vmu}^\perp \wt\zz / \| \bP_{\vmu}^\perp \wt\zz \|_2$, then
% \begin{equation*}
%     \langle \zz_i, \vtheta \rangle
%     = \frac{1}{\| \bP_{\vmu}^\perp \wt\zz \|_2}\langle \zz_i, \bP_{\vmu}^\perp \wt\zz \rangle
%     = \frac{1}{\| \bP_{\vmu}^\perp \wt\zz \|_2} \biggl(  \langle \zz_i, \wt \zz \rangle - \frac1{\norm{\vmu}_2^2 } \langle \zz_i, \vmu \rangle \langle \wt\zz, \vmu \rangle \biggr).
% \end{equation*}
% To bound $\langle \zz_i, \vtheta \rangle$, we use the following lemma.
% \begin{lem} \label{lem:zi-theta} \mbox{}
%     \begin{enumerate}
%         \item[(a)] $\displaystyle \| \bP_{\vmu}^\perp \wt\zz \|_2 = \frac12 \sqrt{\frac{d}{\pi n}}\big(1 + o_\P(1) \big)$.
%         \item[(b)] $\displaystyle \langle \zz_i, \wt \zz \rangle = \ind_{y_i = +1}\frac{d}{2\pi n}\big(1 + o_\P(1) \big) + O_\P\biggl( \sqrt{\frac{d}{\pi n}} \biggr)$.
%         \item[(c)] $\displaystyle \frac1{\norm{\vmu}_2^2 } \langle \zz_i, \vmu \rangle \langle \wt\zz, \vmu \rangle = O_\P\biggl( \frac{1}{\sqrt{\pi n}} \biggr)$.
%     \end{enumerate}
% \end{lem}
% \begin{proof}
%     \begin{enumerate}
%         \item[(a)] Recall $\wt\zz$ and $\alpha_d$ defined in Eq.~\eqref{eq:alpha_d}. 
        
%         Denote $\wt G :=  2 ( n_+^{-1} + n_-^{-1} )^{-1/2} \langle \wt\zz, \vmu/\norm{\vmu} \rangle  \sim \normal(0, 1)$, then
%         \begin{equation*}
%             \begin{aligned}
%                 \| \bP_{\vmu}^\perp \wt\zz \|^2 & =  \|\wt\zz\|^2 - \norm{\left\langle \wt\zz, \frac{\vmu}{\norm{\vmu}} \right\rangle}^2 
%                  = \frac14 \left( \frac{1}{n_+} + \frac{1}{n_-} \right) \left(  \norm{\zz}^2 - \wt G^2 \right) \\
%                 & = \frac{1}{4\pi n}\bigl(1 + o(1)\bigr) \left( d(1 + o_\P(1)) - \wt G^2 \right)
%                 = \frac{d}{4\pi n}\big(1 + o_\P(1) \big).
%             \end{aligned}
%         \end{equation*} 
        
%         \item[(b)] Let
%         \begin{equation*}
%             \begin{aligned}
%                 \wt\zz_+ := \bar \zz_{-} + \frac{1}{n_+}\sum_{j \not= i, y_j = +1} \zz_j,
%                 \qquad & \alpha_{+d} := \sqrt{ \frac{1}{n_-} + \frac{n_+ - 1}{n_+^2} }
%                 = \frac{1}{\sqrt{\pi n}}\bigl(1 + o(1)\bigr), \\
%                 \wt\zz_- := \bar \zz_{+} + \frac{1}{n_-}\sum_{j \not= i, y_j = -1} \zz_j,
%                 \qquad & \alpha_{-d} := \sqrt{ \frac{1}{n_+} + \frac{n_- - 1}{n_-^2} }
%                 = \frac{1}{\sqrt{\pi n}}\bigl(1 + o(1)\bigr). \\
%             \end{aligned}
%         \end{equation*}
%         Again, by Lemma \ref{lem:subG}\ref{lem:subG-b}, $\zz_i$, $\wt\zz_{0+} := \alpha_{+d}^{-1} \wt\zz_+$, and $\wt\zz_{0-} := \alpha_{-d}^{-1} \wt\zz_-$ are all isotropic mean-zero sub-gaussian, each with independent coordinates. According to Lemma \ref{lem:subG_concentrate}\ref{lem:subG-Hanson-Wright-II}\ref{lem:subG-Bernstein},
%         \begin{equation*}
%             \norm{\zz_i}_2^2 = d\bigl(1 + o_\P(1)\bigr),
%             \qquad \langle \zz_i, \wt\zz_{0+} \rangle = O_\P(\sqrt{d}),
%             \qquad \langle \zz_i, \wt\zz_{0-} \rangle = O_\P(\sqrt{d}).
%         \end{equation*}
%         Therefore, if $y_i = +1$,
%         \begin{equation*}
%             \begin{aligned}
%                 \langle \zz_i, \wt \zz \rangle  & =   \frac{1}{2n_+}\norm{\zz_i}_2^2 + \frac12 \langle \zz_i,  \wt\zz_+   \rangle 
%                 = \frac{1}{2n_+}\norm{\zz_i}_2^2 + \frac{ \alpha_{+d}}2 \langle \zz_i,  \wt\zz_{0+}  \rangle \\
%                 & = \frac{d}{2\pi n}\bigl(1 + o_{\P}(1)\bigr) + \frac{1}{2 \sqrt{\pi n} }\bigl(1 + o(1)\bigr)  \cdot O_{\P}(\sqrt{d}) \\
%                 & = \frac{d}{2\pi n}\bigl(1 + o_{\P}(1)\bigr) + O_\P\biggl( \sqrt{\frac{d}{\pi n}} \biggr).
%             \end{aligned}
%         \end{equation*}
%         Similarly, if $y_i = -1$,
%         \begin{equation*}
%             \begin{aligned}
%                 \langle \zz_i, \wt \zz \rangle  & =   \frac{1}{2n_-}\norm{\zz_i}_2^2 + \frac12 \langle \zz_i,  \wt\zz_-   \rangle 
%                 = \frac{1}{2n_-}\norm{\zz_i}_2^2 + \frac{ \alpha_{-d}}2 \langle \zz_i,  \wt\zz_{0-}  \rangle \\
%                 & = \frac{d}{2n}\bigl(1 + o_{\P}(1)\bigr) + \frac{1}{2 \sqrt{\pi n} }\bigl(1 + o(1)\bigr)  \cdot O_{\P}(\sqrt{d}) \\
%                 & = O_\P\biggl( \sqrt{\frac{d}{\pi n}} \biggr),
%             \end{aligned}
%         \end{equation*}
%         where 
%         \[ \frac{d}{n} = d^{-c} \lesssim \sqrt{\frac{d}{\pi n}} = d^{(a-c)/2}  \]
        
%         \item[(c)] According to Lemma \ref{lem:subG}, both $g_i = \langle \zz_i, \vmu/\norm{\vmu}_2 \rangle$ and $\wt g_0 := \alpha_d^{-1} \langle \wt\zz, \vmu/\norm{\vmu}_2 \rangle$ are sub-gaussian with $\norm{g_i}_{\psi_2} \le CK$, $\|\wt g_0\|_{\psi_2} \le CK$, where $C$ is an absolute constant. Then
%         \begin{equation*}
%             \frac1{\norm{\vmu}_2^2 } \langle \zz_i, \vmu \rangle \langle \wt\zz, \vmu \rangle
%             = \left\langle \zz_i, \frac{\vmu}{\norm{\vmu}_2} \right\rangle
%               \left\langle \wt\zz, \frac{\vmu}{\norm{\vmu}_2} \right\rangle
%             = \alpha_d (g_i \wt g_0)
%             = O_\P(\alpha_d)
%             = O_\P\biggl( \frac{1}{\sqrt{\pi n}} \biggr).
%         \end{equation*}
%     \end{enumerate}
% \end{proof}
% Therefore, by Lemma \ref{lem:zi-theta},
% \begin{equation*}
%     \begin{aligned}
%         \langle \zz_i, \vtheta \rangle
%     & = \frac{1}{\| \bP_{\vmu}^\perp \wt\zz \|_2} \biggl(  \langle \zz_i, \wt \zz \rangle - \frac1{\norm{\vmu}_2^2 } \langle \zz_i, \vmu \rangle \langle \wt\zz, \vmu \rangle \biggr) \\
%     & = 2 \sqrt{\frac{\pi n}{d}}\big(1 + o_\P(1) \big)\biggl( \ind_{y_i = +1}\frac{d}{2\pi n}\big(1 + o_\P(1) \big) + O_\P\biggl( \sqrt{\frac{d}{\pi n}} \biggr) + O_\P\biggl( \frac{1}{\sqrt{\pi n}} \biggr) \biggr) \\
%     & = \ind_{y_i = +1}\sqrt{\frac{d}{\pi n}}\big(1 + o_\P(1) \big) + O_{\P}(1).
%     \end{aligned}
% \end{equation*}



\section{Confidence estimation and calibration: Proofs for \cref{sec:calibration}}
\label{append_sec:calib}

\subsection{Proof of \cref{prop:conf}}

The following preliminary result summarizes the precise asymptotics of three quantities: $\hat p(\xx)$ (max-margin confidence), $p^*(\xx)$ (Bayes optimal probability), and $\hat p_0(\xx)$ (true posterior probability).

% When data is generated from 2-GMM with proportional asymptotics $n/d \to \delta \in (0, \infty)$, we have closed-forms for the quantities above.
\begin{lem}\label{lem:conf_limit}
    Consider 2-GMM and proportional settings in \cref{sec:logit_SVM} on separable dataset ($\delta < \delta^*(0)$). Let $(\rho^*, \beta_0^*)$ be defined as per \cref{thm:SVM_main}, and $(Y, G, H) \sim P_y \times \normal(0,1) \times \normal(0,1)$. Let $G' := \rho^* G + \sqrt{1 - \rho^{*2}} H$. Then for any test point $(\xx, y) \sim P_{\xx, y}$ independent of $\hat p$, as $n \to \infty$,
    \begin{equation}
    \label{eq:p(x)_asymp}
        \begin{pmatrix}
        y 
        \vphantom{\left( \log\frac{\pi}{1 - \pi} \right)} 
        \\
        \hat p(\xx)
        \vphantom{\left( \log\frac{\pi}{1 - \pi} \right)}
        \\
        p^*(\xx) 
        \vphantom{\left( \log\frac{\pi}{1 - \pi} \right)}
        \\
        \hat p_0(\xx)
        \vphantom{\left( \log\frac{\pi}{1 - \pi} \right)}
        \end{pmatrix}
        \cond
        \begin{pmatrix}
            Y 
            \vphantom{\left( \log\frac{\pi}{1 - \pi} \right)} 
            \\
            \sigma\left( \rho^*\|\vmu\|_2 Y + G + \beta_0^* \right) 
            \vphantom{\left( \log\frac{\pi}{1 - \pi} \right)} 
            \\
            \sigma \left( 2 \|\bmu\|_2 (\|\bmu\|_2 Y + G') + \log\frac{\pi}{1 - \pi} \right) \\
            \sigma \left( 2 \rho^* \|\bmu\|_2 (\rho^* \|\bmu\|_2 Y + G) + \log\frac{\pi}{1 - \pi} \right)
        \end{pmatrix}.
    \end{equation}
    % \begin{align}
    %     \hat p(\xx) & \cond \sigma\left( \rho^*\|\vmu\|_2 Y + G + \beta_0^* \right),
    %     \\
    %     p^*(\xx) & 
    %     \mathmakebox[\widthof{${} \cond {}$}][c]{\overset{\mathrm{d}}{=}} 
    %     \sigma \left( 2 \|\bmu\|_2 (\|\bmu\|_2 Y + G) + \log\frac{\pi}{1 - \pi} \right),
    %     \\
    %     \hat p_0(\xx) & \cond \sigma \left( 2 \rho^* \|\bmu\|_2 (\rho^* \|\bmu\|_2 Y + G) + \log\frac{\pi}{1 - \pi} \right).
    % \end{align}
\end{lem}
\begin{proof}
    Rewrite $\xx = y\bmu + \zz$ where $\zz \sim \normal(\bzero, \bI_d)$. By direct calculation, the three quantities $\hat p(\xx)$, $p^*(\xx)$, and $\hat p_0(\xx)$ can be expressed by
    \begin{align}
        \hat p(\xx) = \sigma\bigl( \< \xx, \hat\vbeta \> + \hat\beta_0 \bigr) 
        & = \sigma\left( \hat \rho \norm{\bmu}_2 y + \< \zz, \hat\vbeta \> +\beta_0 \right),
        \label{eq:p_hat_exp}
        \\
        p^*(\xx) = \P( y = 1 \,|\, \xx )
        & = \frac{\pi e^{-\frac12 \| \xx - \bmu \|_2^2} }{\pi e^{-\frac12 \| \xx - \bmu \|_2^2} + (1 - \pi) e^{-\frac12 \| \xx + \bmu \|_2^2}} 
        \label{eq:p_star_bayes}
        \\
        & = \sigma \left( 2\< \xx, \bmu \> + \log\frac{\pi}{1 - \pi} \right)  \notag \\
        & = \sigma \left( 2 \|\bmu\|_2 \Bigl( \norm{\bmu}_2 y + \< \zz, \bmu/\|\bmu\|_2 \> \Bigr)  + \log\frac{\pi}{1 - \pi} \right),
        \label{eq:p_star_exp}
        \\
        \hat p_0(\xx) = \P \bigl( y = 1 \,|\, \hat p(\xx) \bigr)
        & = \frac{\pi e^{-\frac12 (\hat f(\xx) - \hat\rho\|\bmu\|_2 - \hat\beta_0)^2} }{\pi e^{-\frac12 (\hat f(\xx) - \hat\rho\|\bmu\|_2 - \hat\beta_0)^2} + (1 - \pi) e^{-\frac12 (\hat f(\xx) + \hat\rho\|\bmu\|_2 - \hat\beta_0)^2} } 
        \label{eq:p0_bayes}
        \\
        & = \sigma \left( 2 \, \hat\rho \, \|\bmu\|_2 \< \xx, \hat\bbeta \> + \log\frac{\pi}{1 - \pi} \right) \notag \\
        & = \sigma \left( 2 \, \hat\rho \, \|\bmu\|_2 \left(\hat \rho \norm{\bmu}_2 y + \< \zz, \hat\vbeta \> \right) + \log\frac{\pi}{1 - \pi} \right),
        \label{eq:p0_exp}
    \end{align}
    where the Bayes' law is applied in \cref{eq:p_star_bayes} and \eqref{eq:p0_bayes}.

    Next, it suffices to obtain the joint asymptotics of $\< \zz, \hat\vbeta \>$ and $\< \zz, \bmu/\|\bmu\|_2 \>$, which appear in the expressions of \cref{eq:p_hat_exp}, \eqref{eq:p_star_exp}, \eqref{eq:p0_exp}. Note that $\< \zz, \hat\vbeta \> \cond \normal(0, 1)$ (since $\zz \indep \hat\vbeta$, $\P(\| \hat\vbeta \|_2 = 1) \to 1$), $\< \zz, \bmu/\|\bmu\|_2 \> \sim \normal(0, 1)$. Moreover, $\E[\< \zz, \hat\vbeta \> \< \zz, \bmu/\|\bmu\|_2 \> ] = \E[ \hat\rho ] \to \rho^*$ by \cref{thm:SVM_main} and bounded convergence. These implies
    \begin{equation*}
        \begin{pmatrix}
            \< \zz, \hat\vbeta \> \\
            \< \zz, \bmu/\|\bmu\|_2 \>
        \end{pmatrix}
        \cond
        \normal\left(
        \begin{pmatrix}
            0 \\
            0
        \end{pmatrix},
        \begin{pmatrix}
            1 & \rho^* \\
            \rho^* & 1
        \end{pmatrix}
        \right) \overset{\mathrm{d}}{=} 
        \begin{pmatrix}
            G \\
            G'
        \end{pmatrix}.
    \end{equation*}
    Since $y \indep (\zz, \hat\vbeta)$ and $(\hat\rho, \hat\beta_0) \conp (\rho^*, \beta_0^*)$, we conclude \cref{eq:p(x)_asymp} by \cref{eq:p_hat_exp}, \eqref{eq:p_star_exp}, \eqref{eq:p0_exp} and then using the Slutsky's theorem. This completes the proof.
\end{proof}


The proof of \cref{prop:conf} is primarily based on asymptotics in \cref{lem:conf_limit}.

\begin{proof}[\textbf{Proof of \cref{prop:conf}}]
% \vspace{0.5\baselineskip}
% \noindent
\textbf{\ref{prop:conf_asymp}:} For $\mathrm{MSE}$, by directly using the asymptotics in \cref{lem:conf_limit} and bounded convergence theorem, we have
\begin{align*}
    \lim_{n \to \infty} \mathrm{MSE}(\hat p) & = \lim_{n \to \infty} \E\left[ \bigl( \mathbbm{1}\{ y = 1 \} - \hat p(\xx) \bigr)^2 \right] \\
    & = \E\left[ \bigl( \mathbbm{1}\{ Y = 1 \} -  \sigma\left( \rho^*\|\vmu\|_2 Y + G + \beta_0^* \right)  \bigr)^2 \right] \\
    & = \E \left[ \sigma \bigl( -\rho^* \norm{\bmu}_2 - \beta_0^* Y + G \bigr)^2 \right] = \mathrm{MSE}^*, 
    \\
    \lim_{n \to \infty} \mathrm{mMSE}(\hat p) & = \mathrm{MSE}^*
    - \var\, \bigl[\mathbbm{1}\{ y = 1 \} \bigr] = \mathrm{MSE}^* - \pi(1 - \pi).
\end{align*}
For $\mathrm{CalErr}$, we similarly get
\begin{align*}
    & \lim_{n \to \infty} \mathrm{CalErr}(\hat p)
    = \lim_{n \to \infty} \E\left[ \bigl( \hat p(\xx) - \hat p_0(\xx) \bigr)^2 \right] \\
    = {} & \E\left[ \left(  \sigma\bigl( \rho^* \norm{\bmu}_2 Y + G + \beta_0^* \bigr)
    - \sigma\Bigl( 2\rho^* \norm{\bmu}_2 (\rho^* \norm{\bmu}_2 Y + G) + \log\frac{\pi}{1-\pi} \Bigr)
     \right)^2 \right]
     = \mathrm{CalErr}^*.
\end{align*}
For $\mathrm{ConfErr}$, we can first obtain
\begin{align*}
    \lim_{n \to \infty} \E \left[ p^*(\xx) \bigl( 1 - p^*(\xx) \bigr) \right]
    & = \lim_{n \to \infty} \E \, \Bigl[ \var\, \bigl[\mathbbm{1}\{ y = 1 \} \,|\, \xx \bigr] \Bigr] \\
    & = \lim_{n \to \infty} \E\left[ \bigl( \mathbbm{1}\{ y = 1 \} - p^*(\xx) \bigr)^2 \right] \\
    & = \E\left[ \left( \mathbbm{1}\{ Y = 1 \} - \sigma \Bigl( 2 \|\bmu\|_2 (\|\bmu\|_2 Y + G) + \log\frac{\pi}{1 - \pi} \Bigr) \right)^2 \right] \\
    & = \E\left[ \sigma\Bigl( -2\norm{\bmu}_2( \norm{\bmu}_2 + G ) - \log\frac{\pi}{1-\pi} Y \Bigr)^2 \right]
    = V_{y|\xx}^*,
\end{align*}
and then by relation between $\mathrm{ConfErr}$ and $\mathrm{MSE}$ \cref{eq:MSE_vs_ConfErr}
\begin{equation*}
    \lim_{n \to \infty} \mathrm{ConfErr}(\hat p)
    = \lim_{n \to \infty} \mathrm{MSE}(\hat p) - \lim_{n \to \infty} \E \left[ p^*(\xx) \bigl( 1 - p^*(\xx) \bigr) \right]
    = \mathrm{MSE}^* - V_{y|\xx}^*.
\end{equation*}
This concludes the proof of part \ref{prop:conf_asymp}.

\vspace{0.5\baselineskip}
\noindent
\textbf{\ref{prop:conf_mono}:}
When $\tau = \tau^\mathrm{opt}$, by \cref{prop:tau_opt} $\beta_0^* = 0$. Then we can simplify
    \begin{equation*}
        % \begin{aligned}
            \mathrm{MSE}^* =  \E\left[ \bigl(  1 + \exp( \rho^* \norm{\bmu}_2 + G) \bigr)^{-2} \right] .
            % \\
            % \mathrm{TMSE}^* & = \E\left[ \left( \mathbbm{1}\{ \rho^* \norm{\bmu}_2 + G  >  0 \} - \frac{1}{1 + \exp( - \abs{\rho^* \norm{\bmu}_2 + G} ) } \right)^{2} \right] \\
            % \mathrm{V}_\mathrm{Err}^* & = \Err(1 - \Err).
        % \end{aligned}
    \end{equation*}
    According to \cref{lem:rho_mono}, we know that $\rho^* \norm{\bmu}_2$ is increasing in $\pi \in (0, \frac12)$, $\norm{\bmu}_2$, and $\delta$. It suffices to show that $\mathrm{MSE}^*$ is decreasing in $\rho^* \norm{\bmu}_2$, which is obvious by noticing $t \mapsto ( 1 + \exp(t) )^{-2}$ is a strictly decreasing function.

    For $\mathrm{mMSE}^*$, note that $\pi(1 - \pi)$ is a increasing function of $\pi \in (0, \frac12)$, and it does not depend on $\norm{\bmu}_2$, $\delta$. These shows the monotonicity of $\mathrm{mMSE}^* = \mathrm{MSE}^* - \pi(1 - \pi)$.

    For $\mathrm{ConfErr}^*$, note that $V_{y|\xx}^*$ does not depend on $\delta$. This implies that $\mathrm{ConfErr}^*$ has the same monotonicity in $\delta$ as $\mathrm{MSE}^*$, which concludes the proof of part \ref{prop:conf_mono}.
\end{proof}



\subsection{Verification of \cref{claim:conf}}

The analytical dependence of $\mathrm{CalErr}^*$ and $\mathrm{ConfErr}^*$ on model parameters is more complicated. We provide a numerical verification of \cref{claim:conf}.

\begin{proof}[\textbf{Verification of \cref{claim:conf}}]
    For $\mathrm{CalErr}^*$, denote
            \begin{align*}
                h_1(t) & := \E\left[ \Bigl( \sigma\bigl( 2 t (G+t) + c \bigr) - \sigma( G+t ) \Bigr)^2 \right]
                \\
                h_2(t) & := \E\left[ \Bigl( \sigma\bigl( 2 t (G-t) + c \bigr) - \sigma( G-t ) \Bigr)^2 \right]
            \end{align*}
            where $c < 0$ is a constant. When $\tau = \tau^\mathrm{opt}$, we have $\beta_0^* = 0$ and
        \begin{equation*}
            \mathrm{CalErr}^*
            = \pi h_1( \rho^* \norm{\bmu}_2 ) + (1 - \pi) h_2( \rho^* \norm{\bmu}_2 ),
            \qquad \text{where} ~ c = \log\frac{\pi}{1 - \pi}.
    \end{equation*}
    According to \cref{fig:mono_fun}, we can numerically show that $h(t) :=  \pi h_1(t) + (1 - \pi) h_2(t)$ is a decreasing function when $\pi \le \overline{\pi} \approx 0.25$ is fixed. Under this condition, $\mathrm{CalErr}^*$ is decreasing in $\rho^* \norm{\bmu}_2$. Then by using \cref{lem:rho_mono} and similar arguments in the proof of \cref{prop:conf}\ref{prop:conf_mono}, we can conclude the monotonicity of $\mathrm{CalErr}^*$ in $\norm{\bmu}_2$ and $\delta$.

    \begin{figure}[h!]
    \centering
    \includegraphics[width=0.32\textwidth]{Figs/cal_proof_h_plot_1.pdf}
    \includegraphics[width=0.32\textwidth]{Figs/cal_proof_h_plot_2.pdf}
    \includegraphics[width=0.32\textwidth]{Figs/cal_proof_Vyx_plot.pdf}
    \caption{
    \textbf{Monotonicity of $x \mapsto h(x)$ and $\pi \mapsto V_{y|\xx}^*$}. \textbf{Left:} $h$ is not monotone when $\pi > \overline{\pi} \approx 0.25$. \textbf{Middle:} $h$ is monotone decreasing when $\pi \le \overline{\pi} \approx 0.25$. \textbf{Right:} $V_{y|\xx}^*$ is monotone increasing in $\pi$ for different values of $\norm{\bmu}_2$.
    }
    \label{fig:mono_fun}
\end{figure}

For $\mathrm{ConfErr}^*$, in \cref{fig:mono_fun} we numerically show that $V_{y|\xx}^*$ is increasing in $\pi$ when $\norm{\bmu}_2$ is fixed. Since $\mathrm{ConfErr}^* = \mathrm{MSE}^* - V_{y|\xx}^*$ and we have shown in \cref{prop:conf}\ref{prop:conf_mono} that $\mathrm{MSE}^*$ is decreasing in $\pi$, we conclude $\mathrm{ConfErr}^*$ is also decreasing in $\pi$.
\end{proof}

    % (a): Recall $\xx_\text{new} = y_\text{new} \bmu + \zz_\text{new}$, $\hat f(\xx_\text{new}) = y_\text{new}\hat\rho\norm{\bmu}_2 + \< \zz_\text{new}, \hat\bbeta \> + \hat\beta_0$ and $\< \zz_\text{new}, \hat\bbeta \> \sim \normal(0, 1)$.
    % \begin{equation*}
    %     \begin{aligned}
    %          \mathrm{MSE}(\hat p) 
    %          & 
    %          = \E\left[ \bigl( \mathbbm{1}\{ y_\text{new} = 1 \} - \hat p(\xx_\text{new}) \bigr)^2  \right]
    %          \\
    %          & = \E\left[ \biggl( \mathbbm{1}\{ y_\text{new} = 1 \} - 
    %          \frac{1}{1 + \exp\bigl( - y_\text{new}\hat\rho\norm{\bmu}_2 - \< \zz_\text{new}, \hat\bbeta \> - \hat\beta_0 \bigr)}
    %          \biggr)^2  \right]
    %     \end{aligned}
    % \end{equation*}
    % Note $( y_\text{new}, \< \zz_\text{new}, \hat\bbeta \>, \hat\rho, \hat\beta_0) \cond (y_\text{new}, G, \rho^*, \beta_0^*)$ where $G \sim \normal(0, 1)$ and $G \indep y_\text{new}$. By DCT,
    % \begin{equation*}
    %     \begin{aligned}
    %          \lim_{n \to \infty} \mathrm{MSE}(\hat p) 
    %          & 
    %          = \E\left[ \biggl( \mathbbm{1}\{ y_\text{new} = 1 \} - 
    %          \frac{1}{1 + \exp\bigl( - y_\text{new} \rho^*\norm{\bmu}_2 - G - \beta_0^* \bigr)}
    %          \biggr)^2  \right] \\
    %          & 
    %          =  \pi \E\left[ \biggl( 1 - 
    %          \frac{1}{1 + \exp\bigl( -\rho^*\norm{\bmu}_2 - G - \beta_0^* \bigr)}
    %          \biggr)^2  \right]
    %          + (1 - \pi) \E\left[ \biggl(  
    %          \frac{1}{1 + \exp\bigl( \rho^*\norm{\bmu}_2 - G - \beta_0^* \bigr)}
    %          \biggr)^2  \right]
    %          \\
    %          &
    %          =
    %          \pi \E\left[ \bigl(  1 + \exp( \rho^* \norm{\bmu}_2 + G + \beta_0^*) \bigr)^{-2} \right] 
    %             + (1 - \pi) \E\left[ \bigl(  1 + \exp( \rho^* \norm{\bmu}_2 + G - \beta_0^* ) \bigr)^{-2} \right].
    %     \end{aligned}
    % \end{equation*}
    % For CE,
    % \begin{equation*}
    %     \mathrm{CalErr}(\hat p) = \E\left[  \Bigl(  
    %     \E\left[ \P \bigl( y_\text{new} = 1 \,|\,  \hat f(\xx_\text{new}), \XX, \yy \bigr) \, \big| \, \hat f(\xx_\text{new}) \right] - \hat p(\xx_\text{new})
    %     \Bigr)^2 \right]. 
    % \end{equation*}
    % Recall that $\hat f(\xx_\text{new}) \,|\, (y_\text{new}, \XX, \yy) \sim \normal( y_\text{new} \hat\rho \norm{\bmu}_2 + \hat\beta_0,  1)$. Then by Bayesian formula,
    % \begin{equation*}
    %     \begin{aligned}
    %         & \P \bigl( y_\text{new} = 1 \,|\,  \hat f(\xx_\text{new}), \XX, \yy \bigr)  \\
    %         = {} & 
    %         \frac{\P(y_\text{new} = 1) \cdot \texttt{p}(\hat f(\xx_\text{new}) \,|\, y_\text{new} = 1, \XX, \yy)}{
    %         \P(y_\text{new} = 1) \cdot \texttt{p}(\hat f(\xx_\text{new}) \,|\, y_\text{new} = 1, \XX, \yy)
    %         + \P(y_\text{new} = -1) \cdot \texttt{p}(\hat f(\xx_\text{new}) \,|\, y_\text{new} = -1, \XX, \yy)
    %         }
    %         \\
    %         = {} &
    %         \frac{\pi \exp\bigl\{ -\frac12 (\hat f(\xx_\text{new}) - \hat\rho \norm{\bmu}_2 - \hat\beta_0)^2 \bigr\}}{
    %         \pi \exp\bigl\{ -\frac12 (\hat f(\xx_\text{new}) - \hat\rho \norm{\bmu}_2 - \hat\beta_0)^2 \bigr\}
    %         + (1 - \pi) \exp\bigl\{ -\frac12 ( \hat f(\xx_\text{new}) + \hat\rho \norm{\bmu}_2 - \hat\beta_0 )^2 \bigr\}
    %         }
    %         \\
    %         = {} &
    %         \frac{1}{
    %         1 
    %         + \frac{1 - \pi}{\pi} \exp \bigl\{ -\frac12 \bigl[ ( \hat f(\xx_\text{new}) + \hat\rho \norm{\bmu}_2 - \hat\beta_0 )^2
    %         - ( \hat f(\xx_\text{new}) - \hat\rho \norm{\bmu}_2 - \hat\beta_0 )^2 \bigr]
    %         \bigr\}
    %         }
    %         \\
    %         = {} &
    %         \frac{1}{
    %         1 
    %         + \exp \bigl\{ - 2 \hat\rho \norm{\bmu}_2 ( \hat f(\xx_\text{new}) - \hat\beta_0)
    %         - \log \frac{\pi}{1-\pi}
    %         \bigr\}
    %         }
    %     \end{aligned}
    % \end{equation*}
    % By DCT of conditional expectation,
    % \begin{equation*}
    %     \E\left[ \P \bigl( y_\text{new} = 1 \,|\,  \hat f(\xx_\text{new}), \XX, \yy \bigr) \, \big| \, \hat f(\xx_\text{new}) \right]
    %     \xrightarrow{n \to \infty} 
    %     \frac{1}{
    %         1 
    %         + \exp \bigl\{ - 2 \rho^* \norm{\bmu}_2 ( \hat f(\xx_\text{new}) - \beta_0^*)
    %         - \log \frac{\pi}{1-\pi}
    %         \bigr\}
    %         }.
    % \end{equation*}
    % Note that $\hat f(\xx_\text{new}) = y_\text{new}\hat\rho\norm{\bmu}_2 + \< \zz_\text{new}, \hat\bbeta \> + \hat\beta_0$ and $\< \zz_\text{new}, \hat\bbeta \> \sim \normal(0, 1)$. Therefore, by DCT again,
    % \begin{equation*}
    %     \begin{aligned}
    %          & \lim_{n \to \infty} \mathrm{CalErr}(\hat p) \\
    %         = {} & 
    %         \E\left[  \lim_{n \to \infty} \Bigl(  
    %     \E\left[ \P \bigl( y_\text{new} = 1 \,|\,  \hat f(\xx_\text{new}), \XX, \yy \bigr) \, \big| \, \hat f(\xx_\text{new}) \right] - \hat p(\xx_\text{new})
    %     \Bigr)^2 \right]. 
    %     \\
    %         = {} & 
    %         \E\left[ \biggl(
    %          \frac{1}{
    %         1 
    %         + \exp \bigl\{ - 2 \rho^* \norm{\bmu}_2 ( y_\text{new} \rho^*\norm{\bmu}_2 + G)
    %         - \log \frac{\pi}{1-\pi}
    %         \bigr\}
    %         }
    %         -
    %         \frac{1}{1 + \exp( - y_\text{new} \rho^*\norm{\bmu}_2 - G - \beta_0^* )}
    %         \biggr)^2 \right],
    %     \end{aligned}
    % \end{equation*}
    % and the conclusion is followed by conditioning on $y_\text{new}$.
\section{Technical Lemmas}
\label{append_sec:tech}

\subsection{Properties of Gaussian random variables}

We need the following variant of Gordon's comparison theorem for Gaussian processes.
\begin{lem}[CGMT]
    \label{lem:CGMT}
    Let $D_{\bu} \subset \R^{n_1 + n_2}$, $D_{\bv} \subset \R^{m_1 + m_2}$ be compact sets and let $Q: D_{\bu} \times D_{\bv} \to \R$ be a continuous function. Let $\GG = (G_{i,j}) \iidsim \normal(0, 1)$, $\vg \sim \normal(\bzero, \bone_{n_1})$, $\hh \sim \normal(\bzero, \bone_{m_1})$ be independent standard Gaussian vectors. For any $\bu \in \R^{n_1 + n_2}$ and $\bv \in \R^{m_1 + m_2}$ we define $\wt\bu = (u_1, \dots, u_{n_1})$ and $\wt\bv = (v_1, \dots, v_{m_1})$. Define
    \begin{equation*}
        \begin{aligned}
            C^*(\GG)      & = \min_{\bu \in D_{\bu}} \max_{\bv \in D_{\bv}}  \wt\bv^\top \GG \wt\bu + Q(\bu, \bv), \\
            L^*(\vg, \hh) & = \min_{\bu \in D_{\bu}} \max_{\bv \in D_{\bv}}  \| \wt\bv \|_2 \vg^\top \wt\bu
            + \| \wt\bu \|_2 \hh^\top \wt\bv + Q(\bu, \bv).
        \end{aligned}
    \end{equation*}
    Then we have:
    \begin{enumerate}[label=(\alph*)]
        \item \label{lem:CGMT(a)}
        For all $t \in \R$,
        \begin{equation*}
            \P\left( C^*(\GG) \le t \right) \le 2 \, \P\left( L^*(\vg, \hh) \le t \right).
        \end{equation*}
        \item \label{lem:CGMT(b)}
        If $D_{\bu}$ and $D_{\bv}$ are convex and if $Q$ is convex concave, then for all $t \in \R$,
        \begin{equation*}
            \P\left( C^*(\GG) \ge t \right) \le 2 \, \P\left( L^*(\vg, \hh) \ge t \right).
        \end{equation*}
    \end{enumerate}
\end{lem}
\begin{proof}
    See \cite[Corollary G.1]{miolane2018distributionlassouniformcontrol}.
\end{proof}



\subsection{Properties of sub-gaussian and sub-exponential random variables}

\begin{defn}[Sub-gaussianity]
    \label{def:subgauss}
    The sub-gaussian norm of random variable $X$ is defined as
    \begin{equation*}
        \norm{X}_{\psi_2} := \inf\left\{ K > 0: \E[\exp(X^2/K^2)] \le 2 \right\}.
    \end{equation*}
    \begin{itemize}
        \item A random variable $X \in \R$ is called sub-gaussian if $\norm{X}_{\psi_2} < \infty$.
        \item A random vector $\xx = (X_1, \dots, X_d)^\top \in \R^d$ is called sub-gaussian if $\sup_{\bv \in \S^{d-1}}\norm{\langle \xx, \bv \rangle}_{\psi_2} < \infty$. Specifically, write $\xx \sim \subGind(\bzero, \bI_d; K)$ if
        $X_1, \dots, X_d$ are independent random variables with $\E[X_i] = 0$, $\var(X_i) = 1$,
        % $\xx$ is isotropic mean-zero ($\E[\xx] = \bzero$, $\cov(\xx) = \bI_d$) with independent coordinates, 
        and $\max_{1 \le i \le d} \norm{X_i}_{\psi_2} \lesssim K$. 
    \end{itemize}   
\end{defn}

\noindent
\cref{lem:subG} and \ref{lem:subG_concentrate} summarize some basic facts and concentration inequalities about sub-gaussian random variables and vectors.

\begin{lem}\label{lem:subG}
    Some facts about sub-gaussian 
    % and sub-exponential 
    random variables.
    \begin{enumerate}[label=(\alph*)]
        \item \label{lem:subG-a} $\norm{\, \cdot \,}_{\psi_2}$ is a norm on the space of sub-gaussian random variables.
        \item \label{lem:subG-b} Let $X_1, \ldots, X_N$ be independent mean-zero sub-gaussian random variables. Then $\sum_{i=1}^N X_i$ is also a sub-gaussian random variable, and
        \[ \norm{\sum_{i=1}^N X_i}_{\psi_2}^2\le C \sum_{i=1}^N \norm{X_i}^2_{\psi_2}, \]
        where $C$ is an absolute constant.
        \item \label{lem:subG-c} (Maximum) Let $X_1, \ldots, X_N$ be sub-gaussian random variables (not necessarily independent) with $K := \max_{1 \le i \le N} \norm{X_i}_{\psi_2}$. Then
        \[ \E\left[ \max_{1 \le i \le N} \abs{X_i} \right] \le  C K \sqrt{\log N},
        \qquad (N \ge 2), \]
        where $C$ is an absolute constant.
    \end{enumerate}
\end{lem}
\begin{proof}
    See \cite[Exercise 2.5.7, Proposition 2.6.1, Exercise 2.5.10]{vershynin2018high}.
    % , Lemma 2.7.7
\end{proof}

\begin{lem}[Concentration]\label{lem:subG_concentrate}
    Suppose $\xx, \yy \sim \subGind(\bzero, \bI_d; K)$ and $\xx \indep \yy$.
    \begin{enumerate}[label=(\alph*)]
        \item \label{lem:subG-Hanson-Wright-I} (Hanson-Wright inequality I)  \  Let $\bA \in \R^{d \times d}$ be a matrix. Then, for every $t \ge 0$,
        \begin{equation*}
           \P\left( \bigl| \xx^\top \bA \xx - \E[\xx^\top \bA \xx] \bigr| \ge t \right)
           \le 2 \exp\biggl( -c \min \biggl\{ \frac{t^2}{K^4 \norm{\bA}_\mathrm{F}^2} , \frac{ t }{ K^2 \| \bA \|_{\mathrm{op}} } \biggr\} \biggr),
        \end{equation*}
        where $c$ is an absolute constant.
        \item \label{lem:subG-Hanson-Wright-II} (Hanson-Wright inequality II) \  Let $\bB \in \R^{d' \times d}$ be a matrix. Then, for every $t \ge 0$,
        \begin{equation*}
            \P\biggl( \biggl| \frac{\| \bB \xx \|_2}{ \| \bB \|_\mathrm{F}} - 1 \biggr| > t \biggr)
            \le 2 \exp\biggl( -\frac{ct^2 \| \bB \|_\mathrm{F}^2 }{K^4 \| \bB \|_{\mathrm{op}}^2 } \biggr),
        \end{equation*}
        where $c$ is an absolute constant. In particular, when $\bB = \bI_d$,
        \begin{equation*}
            \P\biggl( \biggl| \frac{\| \xx \|_2}{ \sqrt{d} } - 1 \biggr| > t \biggr)
            \le 2 \exp\biggl( -\frac{ct^2 d}{K^4} \biggr).
        \end{equation*}

        \item \label{lem:subG-Hoeffding} (Hoeffding's inequality) \ Let $\ba \in \R^{d}$ be a vector. Then, for every $t \ge 0$,
        \begin{equation*}
            \P\biggl( \frac{ \abs{\< \xx, \ba \>} }{ \norm{\ba}_2 }  > t \biggr)
            \le 2 \exp\biggl( -\frac{ct^2}{K^2} \biggr),
        \end{equation*}
        where $c$ is an absolute constant.

        \item \label{lem:subG-Bernstein} (Bernstein's inequality) \ Let $\bB \in \R^{d \times d}$ be a matrix. Then, for every $t \ge 0$,
        \begin{equation*}
            \P\biggl( \frac{ | \xx^\top \bB \yy | }{ \| \bB \|_\mathrm{F} }  > t \biggr)
            \le 2 \exp\biggl( -c \min \biggl\{ \frac{t^2}{K^4} , \frac{ t \| \bB \|_\mathrm{F}}{ K^2 \| \bB \|_{\mathrm{op}} } \biggr\} \biggr),
        \end{equation*}
        where $c$ is an absolute constant. In particular, when $\bB = \bI_d$,
        \begin{equation*}
            \P\biggl( \frac{ \abs{\< \xx, \yy \>} }{ \sqrt{d} }  > t \biggr)
            \le 2 \exp\biggl( -c \min \biggl\{ \frac{t^2}{K^4} , \frac{t\sqrt{d}}{K^2} \biggr\} \biggr).
        \end{equation*}
    \end{enumerate}
\end{lem}
\begin{proof}
    For \ref{lem:subG-Hanson-Wright-I}, \ref{lem:subG-Hanson-Wright-II} and \ref{lem:subG-Hoeffding}, see \cite[Theorem 6.2.1, Theorem 6.3.2, Theorem 2.6.3]{vershynin2018high}. For \ref{lem:subG-Bernstein},
    let $\bar\xx = \begin{pmatrix}
        \xx \\ \yy
    \end{pmatrix}$, $\bar\bA = \dfrac12\begin{pmatrix}
        \bzero & \bB \\
        \bB & \bzero
    \end{pmatrix}$, then apply $\bar\xx^\top \bar\bA \bar\xx = \xx^\top \bB \yy$ to \ref{lem:subG-Hanson-Wright-I} and simplify.
    % notice $X_i Y_i$'s are independent mean-zero sub-exponential random variable with $\max_{1 \le i \le d} \norm{X_i Y_i}_{\psi_1} \le KL$ (by Lemma \ref{lem:subG}\ref{lem:subG-c}), then apply Bernstein's inequality \cite[Theorem 2.8.2]{vershynin2018high}.
\end{proof}

\begin{defn}[Sub-exponentiality]
    The sub-exponential norm of random variable $X$ is defined as
    \begin{equation*}
        \norm{X}_{\psi_1} = \inf\left\{ K > 0: \E[\exp(\abs{X}/K)] \le 2 \right\}.
    \end{equation*}
    \begin{itemize}
        \item A random variable $X \in \R$ is called sub-exponential if $\norm{X}_{\psi_1} < \infty$.
    \end{itemize}   
\end{defn}

\noindent
\cref{lem:subExp} summarizes some basic facts about sub-exponential random variables.
\begin{lem}\label{lem:subExp}
    Some facts about sub-exponential random variables.
    \begin{enumerate}[label=(\alph*)]
        \item \label{lem:subExp-a} $\norm{\, \cdot \,}_{\psi_1}$ is a norm on the space of sub-exponential random variables.
        \item \label{lem:subExp-b} Let $X_1, \ldots, X_N$ be independent mean-zero sub-exponential random variables. Then $\sum_{i=1}^N X_i$ is also a sub-exponential random variable. If $K := \max_{1 \le i \le N} \norm{X_i}_{\psi_1}$ and $N \ge C$, then
        \[ 
            \norm{\sum_{i=1}^N X_i}_{\psi_1} \le C' K \sqrt{N},   
        \]
        where $C, C'$ are absolute constants.
        % \item \label{lem:subExp-c} (Centering) If X is a sub-exponential random variable, then
        % \[ \norm{X - \E[X]}_{\psi_1} \le C \norm{X}_{\psi_1}, \]
        % where $C$ is an absolute constant.
        \item \label{lem:subExp-c} (Maximum) Let $X_1, \ldots, X_N$ be sub-exponential random variables (not necessarily independent) with $K := \max_{1 \le i \le N} \norm{X_i}_{\psi_1}$. Then
        \[ \E\left[ \max_{1 \le i \le N} \abs{X_i} \right] \le  C K \log N,
        \qquad (N \ge 2), \]
        where $C$ is an absolute constant.
        \item \label{lem:subExp-d} Let $X$ and $Y$ be sub-gaussian random variables. Then $XY$ is sub-exponential. Moreover,
        \[ \norm{XY}_{\psi_1} \le \norm{X}_{\psi_2} \norm{Y}_{\psi_2}. \]
        In particular, $X^2$ is sub-exponential, and
        \[ \| X^2 \|_{\psi_1} \le \norm{X}_{\psi_2}^2. \] 
    \end{enumerate}
\end{lem}
\begin{proof}
    For \ref{lem:subExp-a} and \ref{lem:subExp-d}, see \cite[Exercise 2.7.11, Lemma 2.7.6, Lemma 2.7.7]{vershynin2018high}. For (b), the proof is analogous to \cite[Proposition 2.6.1]{vershynin2018high}. For any $\abs{\lambda} \le 1/K$, we have
    \begin{equation*}
            \E\biggl[ \exp\biggl(\lambda \sum_{i=1}^N X_i \biggr) \biggr]
             = \prod_{i=1}^{N} \E[\exp(\lambda X_i)]
            \le \prod_{i=1}^{N} \exp\bigl( C \lambda^2 \norm{X_i}_{\psi_1}^2 \bigr)
             \le \exp\bigl( C \lambda^2 N K^2 \bigr),
    \end{equation*}
    where sub-exponential properties \cite[Proposition 2.7.1 (iv)(v)]{vershynin2018high} are used, and $C$ is an absolute constant. If $N \ge 1/C$, then $1/\sqrt{CNK^2} \le 1/K $ and therefore
    \[ \E\biggl[ \exp\biggl(\lambda \sum_{i=1}^N X_i \biggr) \biggr] \le \exp\bigl( \lambda^2 C N K^2 \bigr),
    \quad \text{for all $\lambda$ such that }  \abs{\lambda} \le \frac{1}{\sqrt{CN} K}.  \]
    Then the proof is completed by using \cite[Proposition 2.7.1 (iv)(v)]{vershynin2018high} again.

    For (d), the proof is analogous to \cite[Exercise 2.5.10]{vershynin2018high}. By \cite[Proposition 2.7.1 (i)(iv)]{vershynin2018high}, $\P(\abs{X_i} \ge t) \le 2\exp(-ct/\norm{X}_{\psi_1}) \le 2\exp(-ct/K)$, $\forall\, t \ge 0$, where $c$ is an absolute constant. Denote $t_0 := 2K/c$, then
	\begin{align*}
	& \E\left[ \max_{i \ge 1} \frac{\abs{X_i}}{1 + \log i} \right]
	\le t_0 + \int_{t_0}^\infty \P\biggl( \max_{i \ge 1} \frac{\abs{X_i}}{1 + \log i} > t \biggr) \d t 
	\le \frac{2K}{c} + \int_{t_0}^\infty  \sum_{i=1}^\infty  \P\biggl( \frac{\abs{X_i}}{1 + \log i} > t \biggr) \d t
	\\
	= {} & \frac{2K}{c} + \sum_{i=1}^\infty \int_{t_0}^\infty  \P\bigl( \abs{X_i} > t (1 + \log i) \bigr) \d t
	   \le \frac{2K}{c} + \sum_{i=1}^\infty \int_{t_0}^\infty 2\exp\bigl(-ct(1 + \log i)/K\bigr) \d t\\
	\le {} & 
    \frac{2K}{c} + \sum_{i=1}^\infty \int_{t_0}^\infty  \exp\bigl(- (\log i) ct_0/K \bigr) \cdot  2\exp(-ct/K) \d t
        \le 
    \frac{2K}{c} + \sum_{i=1}^\infty i^{-2} \int_{0}^\infty 2\exp(-ct/K) \d t \\
	= {} & \frac{2K}{c} + C_0 \cdot \frac{2K}{c} 
        \le C K,
	\end{align*}
    where $C_0, C$ are absolute constants. Hence, for any $N \ge 2$,
	\[ \E\left[\max_{1 \le i \le N} \abs{X_i}\right] 
	\le (1+\log N) \cdot \E\left[ \max_{1 \le i \le N} \frac{\abs{X_i}}{1 + \log i} \right] \lesssim K \log N.
	\]
    This concludes the proof.
\end{proof}




\subsection{Properties of the Moreau envelope and proximal operator}
\label{append_subsec_Moreau}

Let $\ell: \R \to \R_{\ge 0}$ be a continuous convex function. For any $x \in \R$ and $\lambda > 0$, the Moreau envelope of $\ell$ is defined as
\begin{equation}\label{eq:envelope}
    \envelope_\ell(x; \lambda) = \envelope_{\lambda\ell}(x)
    := \min_{t \in \R} \left\{  \ell(t) +  \frac1{2\lambda} (t - x)^2 \right\},
\end{equation}
and the proximal operator of $\ell$ is defined as
\begin{equation*}
    \prox_{\ell}(x; \lambda) =
    \prox_{\lambda \ell}(x) := \argmin_{t \in \R} \left\{ \ell(t) +  \frac1{2\lambda} (t - x)^2 \right\}.
\end{equation*}
\begin{lem} \label{lem:prox}
For any $x \in \R, \lambda > 0$, $\prox_{\ell}(x; \lambda)$ is uniquely determined by stationarity condition
    \begin{equation*}
        \prox_{\ell}(x; \lambda) + \lambda \ell'\bigl( \prox_{\ell}(x; \lambda) \bigr) - x = 0.
    \end{equation*}
    \begin{enumerate}[label=(\alph*)]
        \item \label{lem:prox(a)}
        $\envelope_\ell(x; \lambda)$ is continuous and convex in $(x, \lambda)$. If $\ell$ is differentiable, then $\envelope_\ell(x; \lambda)$ is also differentiable in its domain, with partial derivatives
        \begin{equation*}
            \begin{aligned}
                \frac{\partial \envelope_\ell(x; \lambda)}{\partial x}
            & = 
            \mathmakebox[\widthof{$\displaystyle -\frac{1}{2\lambda^2} \big( x - \prox_{\ell}(x; \lambda) \big)^2$}][r]{\frac{1}{\lambda} \big( x - \prox_{\ell}(x; \lambda) \big)\phantom{^2}}
            = 
            \mathmakebox[\widthof{$\displaystyle -\frac12 \bigl(\ell'(z) \bigr)^2 \big|_{z = \prox_{\ell}(x; \lambda) }$}][r]{
            \ell'(z) \phantom{^2} \big|_{z = \prox_{\ell}(x; \lambda) }
            },
            \\
                \frac{\partial \envelope_\ell(x; \lambda)}{\partial \lambda}
            & = -\frac{1}{2\lambda^2} \big( x - \prox_{\ell}(x; \lambda) \big)^2
            = -\frac12 \bigl(\ell'(z) \bigr)^2 \big|_{z = \prox_{\ell}(x; \lambda) }.
            \end{aligned}
        \end{equation*}
        Moreover, $\envelope_\ell(x; \lambda)$ is non-increasing in $\lambda$ and $\envelope_\ell(x; \lambda) \to \ell(x)$ when $\lambda \to 0^+$.
        \item \label{lem:prox(b)}
        $\prox_{\ell}(x; \lambda)$ is continuous in $(x, \lambda)$. If $\ell$ is twice differentiable, then $\prox_\ell(x; \lambda)$ is also differentiable in its domain, with partial derivatives
        \begin{equation*}
            \frac{\partial \prox_{\ell}(x; \lambda)}{\partial x}
            =  \frac{1}{1 + \lambda \ell''(z)} \bigg|_{z = \prox_{\ell}(x; \lambda)}
            \qquad
            \frac{\partial \prox_{\ell}(x; \lambda)}{\partial \lambda}
            =  -\frac{\ell'(z)}{1 + \lambda \ell''(z)} \bigg|_{z = \prox_{\ell}(x; \lambda)} .
        \end{equation*}
        Moreover, $\prox_\ell(x; \lambda) \to x$ when $\lambda \to 0^+$.
    \end{enumerate}
\end{lem}
\begin{proof}
See \cite[Lemma 15]{thrampoulidis2018precise}, \cite[Proposition A.1]{donoho2016high}, \cite[Lemma 2, Lemma 4]{salehi2019impact}, and relevant references therein.
\end{proof}


\section{Miscellaneous}

Let $\hat\kappa$ be the optimal objective value in \cref{eq:SVM}, which is the \emph{maximum margin} for data $(\XX, \yy)$. Moreover, $(\hat\vbeta, \hat\beta_0, \hat\kappa)$ is also the optimal solution to \cref{eq:SVM-m-reb}. Notice $\hat\kappa \ge 0$ always holds (by taking $\vbeta = 0$, $\beta_0 = 0$ in \cref{eq:SVM}), and we can observe the following relation.
\begin{equation*}
	\begin{aligned}
		\text{(linearly separable)} 
		\ \ & \text{$\exists\, \vbeta\not=\bzero$, $\beta_0 \in \R$, such that $y_i ( \< \xx_i, \bbeta \> + \beta_0 ) > 0$, $\forall\, i \in [n]$,} \\
		& \Longleftrightarrow \quad \hat\kappa > 0, \quad \Longrightarrow \quad \|\hat\vbeta\|_2 = 1, \\
		\text{(not linearly separable)} 
		\ \ & \text{$\forall\, \vbeta\not=\bzero$, $\beta_0 \in \R$, such that $y_i ( \< \xx_i, \bbeta \> + \beta_0 ) \overset{\mathmakebox[0pt][c]{\smash{(*)}}}{\le}  0$, $\forall\, i \in [n]$,} \\
		& \Longleftrightarrow \quad \hat\kappa = 0, \quad \Longrightarrow \quad \hat\vbeta = \bzero, \ \hat\beta_0 = 0 \text{ is a solution.\footnotemark}
	\end{aligned}
    \footnotetext{
	If $(*)$ is strict ($<$), then $\hat\vbeta = \bzero$, $\hat\beta_0 = 0$ is the \emph{unique} solution.
}
\end{equation*}
When data is linearly separable, it turns out \cref{eq:SVM-m-reb} also has the following equivalent form:
\begin{equation}
	\label{eq:SVM-1}
    \begin{array}{rl}
    \minimize\limits_{\bw \in \R^d, \, w_0 \in \R} & \norm{\bw}_2^2, \\
    \text{subject to} &  \wt y_i(\< \xx_i, \bw \> + w_0) \ge 1, \quad \forall\, i \in [n].
    \end{array}
\end{equation}
The parameters in \cref{eq:SVM-m-reb} and \eqref{eq:SVM-1} have one-to-one relation $(\kappa, \vbeta, \beta_0) = (1, \bw, w_0)/\|\bw\|_2$. Notably, \cref{eq:SVM-1} is known as the hard-margin SVM \cite{vapnik1998statistical} if $\tau = 1$. 



\newpage
%\bibliographystyle{alpha}
%\bibliographystyle{plain}
\bibliographystyle{unsrt}
\bibliography{refs}
\end{document}
