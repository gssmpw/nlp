\section{Introduction}
The estimation of parameters for portfolio optimization has long been recognized as one of the most challenging aspects of implementing modern portfolio theory \citep{michaud1989markowitz, demiguel2009optimal}. While Markowitz's mean-variance framework \citep{Markowitz1952} provides an elegant theoretical foundation for portfolio selection, its practical implementation has been persistently undermined by estimation errors in input parameters. \citep{chopra1993effect, chung2022effects} demonstrate that small changes in estimated expected returns can lead to dramatic shifts in optimal, while \citep{ledoit2003improved, ledoit2004well} show that traditional sample covariance estimates become unreliable as the number of assets grows relative to the sample size.

These estimation challenges are exacerbated by a fundamental limitation in the conventional approach to portfolio optimization: the reliance on a sequential, two-stage process where parameters are first estimated from historical data, and these estimates are then used as inputs in the optimization problem. This methodology, while computationally convenient, creates a profound disconnect between prediction accuracy and decision quality. Recent evidence suggests that even substantial improvements in predictive accuracy may not translate to better investment decisions. For example, \citep{gu2020empirical,cenesizoglu2012return} demonstrate that while machine learning methods can significantly improve the prediction of asset returns, these improvements do not consistently translate into superior portfolio performance. Similarly, \cite{elmachtoub2022smart} show that this disconnect can lead to substantially suboptimal investment decisions, even when the parameter estimates appear highly accurate by traditional statistical measures. This disconnect raises a profound question: \textit{Are we optimizing for the wrong objective?} The conventional wisdom of training models to minimize prediction error(commonly measured in mean squared error), while ignoring how these predictions influence downstream investment decisions, may be fundamentally flawed.

The significance of this prediction-decision gap has become increasingly acute in modern financial markets, characterized by growing complexity, non-stationary relationships \citep{bekaert2002dynamics}, and regime shifts \citep{guidolin2007asset}. Due to these market dynamics, traditional estimation methods often fail to capture the characteristics of evolving financial relationships. Even modern machine learning approaches struggle to incorporate the complex interplay between macro variables and asset returns \citep{kelly2019characteristics, hwang2024temporal}. The limitations of conventional approaches have created an urgent need for more sophisticated methodologies capable of capturing the intricate dynamics of contemporary financial markets. Recent advances in Large Language Models (LLMs) have introduced promising new directions for addressing these challenges by potentially capturing complex market relationships and incorporating unstructured information. While these approaches have shown remarkable success in time series forecasting across various domains \citep{jin2023time, ansari2024chronos}, their application to financial parameter estimation remains largely unexplored. Moreover, existing attempts to leverage LLMs for financial forecasting continue to follow the traditional sequential approach, focusing solely on prediction accuracy without considering the downstream impact on portfolio decisions \citep{romanko2023chatgpt, nie2024survey}. Despite their enhanced capabilities in capturing patterns in data, LLMs alone do not address the core problem of bridging the gap between predictive modeling and optimal portfolio construction.

A more comprehensive approach is needed to integrate prediction and optimization stages in portfolio management. A more promising direction lies in decision-focused learning frameworks \citep{mandi2024decision}, which represent a significant departure from traditional approaches by directly integrating the prediction and optimization stages. The emergence of decision-focused learning frameworks represents a significant departure from this approach by integrating the prediction and optimization stages. While these frameworks have shown promise in combinatorial optimization problems \citep{amos2017optnet, cvxpylayers2019}, their application to portfolio optimization has been limited. Early attempts in finance have primarily focused on simple linear models \citep{butler2023integrating}, leaving the potential of decision-focused learning in complex portfolio optimization largely untapped. For instance, \cite{elmachtoub2022smart} demonstrate the benefits of integration in linear optimization problems, but extending these insights to the non-linear, dynamic nature of portfolio optimization remains a significant challenge. The development of techniques for differentiating through convex optimization problems \citep{amos2017optnet, cvxpylayers2019} has created theoretical possibilities for more sophisticated applications, yet their practical implementation in portfolio management continues to face substantial computational and methodological challenges.

This paper proposes a framework that bridges the gap between advanced representation learning and decision-focused optimization in portfolio management. Our approach integrates the representational power of LLMs with the principles of decision-focused learning, creating an decision informed neural network architecture that simultaneously captures complex market relationships and optimizes for portfolio decisions. By developing attention mechanisms that incorporate both cross-sectional asset relationships and temporal dependencies, we enable the model to learn representations that are both predictively accurate and decision-aware. Furthermore, our proposed loss function simultaneously optimizes for statistical accuracy and portfolio performance, ensuring the model's predictions directly translate into better investment decisions.

The main contributions of this paper are as follows:
\begin{itemize}
\item We develop a decision-informed neural network framework that integrates LLMs for portfolio optimization. Our approach carefully considers the unique characteristics of financial markets by incorporating multiple data dimensions: cross-sectional relationships between assets, temporal market dynamics, and the influence of macroeconomic variables. This comprehensive modeling approach ensures that the LLM's powerful representation capabilities are properly adapted to the specific challenges of portfolio optimization.
\item We introduce an attention mechanism that selectively processes three crucial aspects of financial markets through learned representations from Large Language Models (LLMs): asset-to-asset relationships, temporal dependencies, and external macro variables. This mechanism implements an efficient filtering strategy that identifies and extracts only the most relevant information from the rich LLM representations, significantly reducing computational overhead while preserving essential market insights.  By selectively attending to the most relevant features within each aspect, our model achieves both superior computational efficiency and enhanced interpretability in capturing complex market interactions.
\item We propose a hybrid loss function that bridges the gap between statistical prediction accuracy and decision-focused learning for portfolio optimization objectives. This function combines traditional prediction metrics with portfolio performance measures, ensuring that the model learns parameters that are both statistically sound and economically meaningful. Our approach directly addresses the prediction-decision gap while maintaining the model's ability to capture complex market relationships. To the best of our knowledge, this is the first study to combine LLM and DFL from a portfolio optimization perspective.

\end{itemize}


% https://arxiv.org/pdf/2107.04636 Check 