\section{Conclusion} This paper addresses a longstanding challenge in quantitative finance: bridging the gap between more accurate forecasts of financial variables and truly optimal portfolio decisions. While improved prediction accuracy is often cited as the path to superior investment returns, our empirical and theoretical findings demonstrate that purely predictive approaches can fail to yield the best portfolio outcomes. Drawing on recent developments in decision-focused learning, we proposed the \texttt{\texttt{DINN}} (Decision-Informed Neural Network) framework, which not only advances the state of the art in financial forecasting by incorporating large language models (LLMs) but also directly integrates a portfolio optimization layer into the end-to-end training process.

From an empirical standpoint, the experiments conducted on two representative equity datasets (S\&P 100 and DOW 30) suggest three key findings. First, \texttt{DINN} delivers systematically stronger performance across a broad set of metrics—including annualized return, Sharpe ratio, and terminal wealth—when compared to standard deep learning baselines, such as Transformer variants and other LLM-based architectures that rely solely on traditional prediction losses. Second, the inclusion of a prob-sparse attention mechanism may helps the model identify and emphasize a smaller subset of assets critical to replicating market dynamics under a variety of macroeconomic conditions. This mechanism not only focuses the model on economically significant information but also yields portfolios with lower drawdowns and higher risk-adjusted performance during stress regimes (e.g., the COVID-19 crisis and spikes in jobless claims). Third, the gradient-based sensitivity analyses provide a theoretical framework through which to interpret \texttt{DINN}’s asset allocations: \textit{high-sensitivity assets}, which would inflict larger “regret” if incorrectly predicted, exhibit lower mean-squared errors than less-sensitive assets. This finding underscores that \texttt{DINN} “learns what matters” by adjusting its representational power to minimize precisely those errors most detrimental to the ultimate portfolio objective.

Methodologically, the paper makes several contributions that advance the intersection of machine learning and portfolio optimization. It develops a rigorous pipeline to incorporate LLM representations of both inter-asset relationships (e.g., sector-level textual prompts) and macroeconomic data (e.g., summary embeddings of irregularly sampled indicators), thereby enriching the model’s feature space without overwhelming it with noise. By differentiating directly through a convex optimization layer, \texttt{DINN} closes the prediction-decision gap: improving return forecasts is no longer an end in itself but a means to more robust investment decisions.

Looking ahead, three avenues of future research emerge. First, while the current formulation centers on a mean-variance objective with convex risk constraints, extending decision-focused learning to alternative objectives—such as value-at-risk or expected shortfall—may further enhance real-world robustness. Second, although LLM-driven embeddings capture textual and structured macroeconomic signals, ongoing advances in multimodal data ingestion (e.g., satellite imagery or social media feeds) could further refine how markets’ evolving information sets are integrated into portfolio weights. Lastly, large-scale empirical analyses across broader asset classes, such as fixed income or commodities, would help validate and generalize the \texttt{DINN} approach beyond equity-centric portfolios.
In conclusion, this paper shows that bridging the divide between forecasting and portfolio choice requires going beyond optimizing for statistical accuracy alone. By merging representation learning and end-to-end differentiable optimization, \texttt{DINN} offers a systematic way to ensure that improvements in predictive modeling directly translate into meaningful gains in investment decisions. We hope this framework will serve as a foundation for future work in decision-focused learning for finance, spurring both theoretical advances in differentiable optimization techniques and innovative empirical applications across various market settings.