\section*{Appendix}
\section*{Appendix A.1.: Detailed Proofs and Derivations} \label{Appendix:A1}

\noindent\textbf{Proof of Theorem 1.} \label{Appendix:A1}

\textbf{Proof: } Consider the problem \Cref{eq:modified_problem}. Introducing Lagrange multipliers $\eta$ and $\gamma$ for the risk and budget constraints, respectively, the Lagrangian is here: 
\begin{equation}
    \mathcal{L}(w,s_{t+h},\eta,\gamma)=\lambda s_{t+h}^2 - \hat{\mu}_{t+h}^{\top} \hat{w}_{t+h} + \eta(\hat{w}_{t+h}^{\top}\hat{\Sigma}_{t+h} \hat{w}_{t+h} - s_{t+h}) + \gamma(\mathbf{1}^{\top} \hat{w}_{t+h} -1)
\end{equation}
Differentiating with respect to $s_{t+h}$ gives $2\lambda s_{t+h}-\eta =0 \implies \eta=2\lambda s_{t+h}.$ Differentiation with respect to $w$ then yields $-\hat{\mu}_{t+h} + 2\lambda \hat{\Sigma}_{t+h} \hat{w}_{t+h} + \gamma \mathbf{1}=0.$ Hence $\hat{\mu}_{t+h} = 2\lambda \hat{\Sigma}_{t+h} \hat{w}_{t+h} + \gamma \mathbf{1}$. Since $\mathbf{1}^{\top}w=1,$ we have $w = \tfrac{1}{2\lambda}\hat{\Sigma}_{t+h}^{-1}(\hat{\mu}_{t+h}-\gamma\mathbf{1}).$ Multiplying by $\mathbf{1}^{\top}$ and solving for $\gamma$ gives $\gamma = \tfrac{\mathbf{1}^{\top}\hat{\Sigma}_{t+h}^{-1}\hat{\mu}_{t+h}-2\lambda}{\mathbf{1}^{\top}\hat{\Sigma}_{t+h}^{-1}\mathbf{1}}.$  Substituting back into the expression for $w$ and setting $2\lambda=1$ (which does not affect the structure of the solution) yields
\begin{equation}
    w = \hat{\Sigma}_{t+h}^{-1}\hat{\mu}_{t+h}-\hat{\Sigma}_{t+h}^{-1}\frac{\mathbf{1}^{\top}\hat{\Sigma}_{t+h}^{-1}\hat{\mu}_{t+h}-1}{\mathbf{1}^{\top}\hat{\Sigma}_{t+h}^{-1}\mathbf{1}}\mathbf{1}
\end{equation}
Differentiating this with respect to $\hat{\mu}_{t+h}$ and simplifying leads to
\begin{equation}
    \frac{\partial \hat{w}_{t+h}}{\partial \hat{\mu}_{t+h}} = \hat{\Sigma}_{t+h}^{-1}-\frac{\hat{\Sigma}_{t+h}^{-1}\mathbf{1}\mathbf{1}^{\top}\hat{\Sigma}_{t+h}^{-1}}{\mathbf{1}^{\top}\hat{\Sigma}_{t+h}^{-1}\mathbf{1}},
\end{equation}
as claimed $\blacksquare$.

\noindent\textbf{Proof of Theorem 2.} \label{Appendix:A2}

\textbf{Proof: } Starting from the expression derived in Theorem 1 under the normalization $2\lambda=1$, the optimal portfolio weights can be written as  $\hat{w}_{t+h} = \hat{\Sigma}_{t+h}^{-1}\hat{\mu}_{t+h} - p\,\hat{\Sigma}_{t+h}^{-1}\mathbf{1}$, where  $p = \frac{\mathbf{1}^{\top}\hat{\Sigma}_{t+h}^{-1}\hat{\mu}_{t+h} - 1}{\mathbf{1}^{\top}\hat{\Sigma}_{t+h}^{-1}\mathbf{1}}$.  
In this formulation, $\hat{\Sigma}_{t+h}^{-1}$ depends on $\hat{L}_{t+h}$ through the relation $\hat{\Sigma}_{t+h} = \hat{L}_{t+h}\hat{L}_{t+h}^{\top}$. Thus, the chain rule of differentiation implies that understanding $\partial \hat{w}_{t+h}/\partial \hat{\Sigma}_{t+h}$ allows recovery of $\partial \hat{w}_{t+h}/\partial \hat{L}_{t+h}$.

Differentiating $\hat{w}_{t+h}$ with respect to $\hat{\Sigma}_{t+h}$ first requires considering the terms $\hat{\Sigma}_{t+h}^{-1}\hat{\mu}_{t+h}$ and $\hat{\Sigma}_{t+h}^{-1}\mathbf{1}$. From standard matrix calculus \citep{petersen2008matrix}, one has $\partial \hat{\Sigma}^{-1}/\partial \hat{\Sigma} = -\hat{\Sigma}^{-1}(\cdot)\hat{\Sigma}^{-1}$. Applying this to $\hat{\Sigma}_{t+h}^{-1}\hat{\mu}_{t+h}$ yields $\frac{\partial (\hat{\Sigma}_{t+h}^{-1}\hat{\mu}_{t+h})}{\partial \hat{\Sigma}_{t+h}} = -\hat{\Sigma}_{t+h}^{-1}\hat{\mu}_{t+h}\hat{\Sigma}_{t+h}^{-1}$.

Similarly, the term involving $p$ is more involved since $p$ itself depends on $\hat{\Sigma}_{t+h}^{-1}$. Letting $g=\mathbf{1}^{\top}\hat{\Sigma}_{t+h}^{-1}\hat{\mu}_{t+h}$ and $z=\mathbf{1}^{\top}\hat{\Sigma}_{t+h}^{-1}\mathbf{1}$, one has $p=(g-1)/z$. Differentiating $g$ and $z$ with respect to $\hat{\Sigma}_{t+h}$ gives  
\begin{equation}
    \frac{\partial g}{\partial \hat{\Sigma}_{t+h}} = -\hat{\Sigma}_{t+h}^{-1}\mathbf{1}\hat{\mu}_{t+h}^{\top}\hat{\Sigma}_{t+h}^{-1}, \quad
\frac{\partial z}{\partial \hat{\Sigma}_{t+h}} = -\hat{\Sigma}_{t+h}^{-1}\mathbf{1}\mathbf{1}^{\top}\hat{\Sigma}_{t+h}^{-1}.
\end{equation}

Applying the quotient rule to differentiate $p=(g-1)/z$ yields  
\begin{equation}
    \frac{\partial p}{\partial \hat{\Sigma}_{t+h}}
= \frac{-\hat{\Sigma}_{t+h}^{-1}\mathbf{1}\hat{\mu}_{t+h}^{\top}\hat{\Sigma}_{t+h}^{-1}z + (g-1)\hat{\Sigma}_{t+h}^{-1}\mathbf{1}\mathbf{1}^{\top}\hat{\Sigma}_{t+h}^{-1}}{z^2}
\end{equation}

Combining these results, the derivative of $p\,\hat{\Sigma}_{t+h}^{-1}\mathbf{1}$ with respect to $\hat{\Sigma}_{t+h}$ is  
\begin{equation}
    \frac{\partial (p\,\hat{\Sigma}_{t+h}^{-1}\mathbf{1})}{\partial \hat{\Sigma}_{t+h}}
= \frac{\partial p}{\partial \hat{\Sigma}_{t+h}}\hat{\Sigma}_{t+h}^{-1}\mathbf{1} - p\,\hat{\Sigma}_{t+h}^{-1}\mathbf{1}\hat{\Sigma}_{t+h}^{-1}
\end{equation}

Subtracting this from $-\hat{\Sigma}_{t+h}^{-1}\hat{\mu}_{t+h}\hat{\Sigma}_{t+h}^{-1}$ and rearranging terms leads to  
\begin{equation}
    \frac{\partial \hat{w}_{t+h}}{\partial \hat{\Sigma}_{t+h}}
= -\hat{\Sigma}_{t+h}^{-1}(\hat{\mu}_{t+h}-p\mathbf{1})\hat{\Sigma}_{t+h}^{-1} - \frac{\partial p}{\partial \hat{\Sigma}_{t+h}}\hat{\Sigma}_{t+h}^{-1}\mathbf{1}
\end{equation}

Since $\hat{\Sigma}_{t+h} = \hat{L}_{t+h}\hat{L}_{t+h}^{\top}$, differentiating with respect to $\hat{L}_{t+h}$ involves applying the chain rule. Under appropriate vectorization, symmetry assumptions, the lower-triangular structure of $\hat{L}_{t+h}$, and considering only independent parameters, the derivative $\partial \hat{\Sigma}_{t+h}/\partial \hat{L}_{t+h}$ can be simplified to contribute a factor of $2\hat{L}_{t+h}$. Substituting back, the final result is  
\begin{equation}
\begin{aligned}
\frac{\partial \hat{w}_{t+h}}{\partial \hat{L}_{t+h}} 
&= \frac{\partial \hat{w}_{t+h}}{\partial \hat{\Sigma}_{t+h}}\frac{\partial \hat{\Sigma}_{t+h}}{\partial \hat{L}_{t+h}}, \\
&= -2\,\hat{\Sigma}_{t+h}^{-1}(\hat{\mu}_{t+h}-p\mathbf{1})\hat{\Sigma}_{t+h}^{-1}\hat{L}_{t+h} 
- 2\,\hat{\Sigma}_{t+h}^{-1}\left(\frac{\partial p}{\partial \hat{\Sigma}_{t+h}}\mathbf{1}\right)\hat{L}_{t+h}
\end{aligned}
\end{equation} $\blacksquare$.


