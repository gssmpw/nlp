%\section*{Appendix A.2.: Detailed Proofs and Derivations} \label{Appendix:A1}
%
%\textbf{Proof: } Consider the two-asset setting $(n = 2)$ with $\Sigma = I_2$ and $\lambda > 0$. Let the feasible set be
%\begin{equation}
%    \mathcal{W} = \{(w_1, w_2) : w_1 + w_2 = 1,  w_1, w_2 \ge 0 \}.
%\end{equation}
%Suppose the true return vector is ${\mu} = (\mu_1, \mu_2)^\top$ with $\mu_1 > \mu_2$.  Then the mean-variance optimization problem reduces to
%\begin{equation}
%    {w}^{\star} =\mathrm{arg\,max}_{\,w_1 + w_2 = 1,\;w_1,w_2 \ge 0}
%\Bigl[
%    w_1\,\mu_1+ w_2\,\mu_2 -\lambda\,\bigl(w_1^2 + w_2^2\bigr)
%\Bigr].
%\end{equation}
%
%Since $w_2 = 1 - w_1$, define $   L(w_1) =  w_1\mu_1 + (1 - w_1)\mu_2 -\lambda \Bigl[w_1^2 + (1 - w_1)^2 \Bigr]$. Then, differentiating $L(w_1)$ and setting it to zero gives 
%\begin{equation}
%      \frac{dL}{dw_1} = \mu_1 -\mu_2 - \lambda\,\bigl[\,4\,w_1 - 2\bigr] = 0
%\end{equation}
%
%Solving for $w_1$ yields
%\begin{equation}
%    \mu_1 -\mu_2 =  \lambda\,[4w_1^\star - 2] \quad\Longrightarrow\quad w_1^\star =  \frac{1}{2} + \frac{\mu_1 - \mu_2}{4 \lambda}.
%\end{equation}
%Consequently, $w_2^\star = 1- w^{\star}_1$.
%
%We then construct a specific sequence $\tilde{{\mu}}^{(k)}$ that converges to ${\mu}$ but whose induced portfolio weights fail to converge to ${w}^\star$. Set $\tilde{{\mu}}^{(k)} = \begin{pmatrix}  \mu_1 - \delta + \tfrac{1}{k}\\  \mu_2 \end{pmatrix}$ Where $\delta = \frac{\mu_1 - \mu_2}{2} > 0$. Clearly, as $k \to \infty$, $\tilde{{\mu}}^{(k)} \to {\mu}$ in $\ell_2$-norm because $\|\tilde{{\mu}}^{(k)} - {\mu}\|_2$ can be made arbitrarily small.
%
%To show that the induced weights do not converge to ${w^{\star}}$, substitute $\tilde{{\mu}}^{(k)}$ into the same mean-variance formula. the optimal weight $w_{1}^{(k)}$ solves 
%\begin{equation}
%     (\mu_1 - \delta + \tfrac{1}{k}) - \mu_2 =    \lambda\ [ 4w_1^{(k)}- 2]
%\end{equation}
%
%Hence, $   w_1^{(k)} = \frac{1}{2} + \frac{(\mu_1 - \mu_2) - \delta +  \tfrac{1}{k}}{4\lambda}$.  By definition $\delta = \tfrac{\mu_1 - \mu_2}{2}$, we get
%\begin{equation}
%\begin{aligned}
%w_1^{(k)} &= \frac{1}{2} + \frac{\tfrac{\mu_1 - \mu_2}{2} + \tfrac{1}{k}}{4\lambda},\\
%          &= \frac{1}{2} + \frac{\mu_1 - \mu_2}{8\lambda} + \frac{1}{4\lambda k}.
%\end{aligned}
%\end{equation}
%Meanwhile, the true optimum $w_1^\star$ is $w_1^\star = \frac{1}{2} + \frac{\mu_1 - \mu_2}{4\lambda}$. Observe that $\frac{\mu_1 - \mu_2}{8\lambda} + \frac{1}{4\lambda k} \neq \frac{\mu_1 - \mu_2}{4\lambda}$ So, $\lim_{k\to\infty} w_1^{(k)} \neq w_1^\star$ $\blacksquare$.
