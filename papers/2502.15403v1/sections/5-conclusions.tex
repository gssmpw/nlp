\section{Conclusions and Future Work}
\label{sec:Conclusions}
In this work, we introduced the Quality Gap Estimator (\texttt{QGE}), designed to compare the quality of an explanation against alternative explanations, aiding practitioners in determining the need to seek better alternatives. \texttt{QGE} is computationally efficient and can be used with most quality metrics commonly used in XAI, improving their informativeness.

By conceptualizing the challenge of achieving a relative quality measurement as a sampling issue, we demonstrated that \texttt{QGE} is more sample-efficient than the conventional method of comparing with a single random explanation. Extensive testing across various datasets, models, and quality metrics has consistently shown that employing \texttt{QGE} is advantageous over the traditional approach.

Additionally, the transformation implemented by \texttt{QGE} results in quality metrics with enhanced statistical significance, suggesting its utility even in scenarios where relative comparisons are not the primary objective.

For future work, we aim to enhance \texttt{QGE}'s performance with metrics that are inherently unstable, where it currently does not offer a significant improvement over the comparison with a single random sample. Further, we are interested in exploring the potential of employing a similar strategy to also improve the explanations, extending the utility of \texttt{QGE} beyond mere quality measurement.