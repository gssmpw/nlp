\section{Related Works}
Le et al. \cite{le2022survey} provide a comprehensive overview of data-driven software vulnerability (SV) assessment and prioritization, focusing on the use of ML, deep learning (DL), and NLP techniques to automate tasks in the SV management lifecycle. However, their scope is limited to the phases between SV discovery and remediation, excluding studies that rely solely on manual analysis or descriptive statistics.

Elder et al. \cite{elder2024survey} focus on methods for assessing the exploitability of vulnerabilities, categorizing them into manual CVSS-based assessments, automated deterministic assessments, and automated probabilistic assessments.

In contrast, our work introduces a broader taxonomy that includes compliance and contextual metrics, which have received limited attention in prior surveys. Furthermore, we analyze real-world challenges, such as explainability and vulnerability data quality, which extend beyond the data-driven focus of \cite{le2022survey}. By addressing these gaps, our study provides actionable insights into improving vulnerability prioritization frameworks for both research and industrial applications.