\section{Related Work}
Low-light image enhancement remains a significant challenge in computer vision and image processing, despite advancements in camera technology. Researchers have developed various solutions to address this issue, ranging from traditional methods to cutting-edge deep learning techniques, Retinex theory, and optimization algorithms. An enhanced Vector Wiener Filter (VWF) is introduced in \cite{ford1997reconstruction} that leverages photon noise correlation and Fourier domain signal-to-noise ratio enhancement, resulting in remarkable image quality and super-resolution capability.

Building on these foundations, further innovations have emerged that resulted in Histogram Equalization (HE) with adaptive illumination adjustment \cite{banik2018contrast}, an adjustable contrast stretching technique to enhance color image contrast \cite{al2018contrast}. Additionally, researchers have explored the use of deep learning techniques, such as Generative Diffusion Prior (GDP) \cite{fei2023generative} and light-effects suppression networks \cite{li2023pixel,lu2022progressive,zhou2023fusion}, to address uneven light distribution and over-enhancement. These advancements have significantly improved low-light image enhancement, but ongoing research is still needed to overcome the remaining challenges.

Recent advancements in deep learning have transformed low-light image enhancement, with state-of-the-art network architectures pushing the limits of image restoration and quality. Researchers have proposed various innovative methods, including unsupervised image de-noising using GANs \cite{lin2023unsupervised}, zero-reference approaches for noise mitigation and image enhancement \cite{cao2024zero,li2023zero1}, and networks leveraging RAW image data, Channel Guidance Net \cite{fu2023raw}, Fourier-based transforms \cite{wang2023fourllie}, UNet-based \cite{li2024color}, Two-stage Single Image De-hazing Network (TSID Net) \cite{wang2024tsid} and Adaptive Illumination Estimation Network (AIE Net) \cite{yu2024joint} architectures. Additionally, techniques such as style transfer-based data generation, teacher networks, and self-supervised approaches have been explored to address challenges like overexposure, underexposure, and noise removal, ultimately leading to significant improvements in low-light image enhancement.

Researchers have proposed various Retinex theory-based methods for low-light image enhancement, leveraging diverse datasets, multi-metric evaluations, and deep learning techniques. These methods decouple illumination and reflectance components to adjust lighting, suppress noise, and revive colors, restoring image clarity and visual fidelity. Recent approaches include combining Retinex theory with self-supervised learning \cite{wang2021seeing,rasheed2022empirical,fu2023learning}, zero-shot learning-based Retinex decomposition method (ZERRINNet) \cite{li2023zero}, and deep neural networks such as Decom-Net, Denoise-Net, Relight-Net, Diff-Retinex, DICNet, machine learning, and CNNs \cite{hai2023r2rnet,yi2023diff,li2024dark,pan2024dicnet} to achieve superior image restoration and enhancement outcomes in low-light environments.