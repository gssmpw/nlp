%\PassOptionsToPackage{demo}{hyperref}
%\RequirePackage[linesnumbered,lined,ruled,noend,vlined]{algorithm2e}
\documentclass[a4paper,UKenglish,cleveref, autoref, thm-restate]{lipics-v2021}
%This is a template for producing LIPIcs articles. 
%See lipics-v2021-authors-guidelines.pdf for further information.
%for A4 paper format use option "a4paper", for US-letter use option "letterpaper"
%for british hyphenation rules use option "UKenglish", for american hyphenation rules use option "USenglish"
%for section-numbered lemmas etc., use "numberwithinsect"
%for enabling cleveref support, use "cleveref"
%for enabling autoref support, use "autoref"
%for anonymousing the authors (e.g. for double-blind review), add "anonymous"
%for enabling thm-restate support, use "thm-restate"
%for enabling a two-column layout for the author/affilation part (only applicable for > 6 authors), use "authorcolumns"
%for producing a PDF according the PDF/A standard, add "pdfa"


%FULL OR CONF VERSION
\newif\ifFull
\newif\ifConf

\Fulltrue
\Conffalse

\ifFull
    \pdfoutput=1 %uncomment to ensure pdflatex processing (mandatatory e.g. to submit to arXiv)
    \hideLIPIcs  %uncomment to remove references to LIPIcs series (logo, DOI, ...), e.g. when preparing a pre-final version to be uploaded to arXiv or another public repository
\fi

\nolinenumbers

%\graphicspath{{./graphics/}}%helpful if your graphic files are in another directory

\bibliographystyle{plainurl}% the mandatory bibstyle

\title{On the Complexity of Minimising the Moving Distance for Dispersing Objects} %TODO Please add

%\titlerunning{Dummy short title} %TODO optional, please use if title is longer than one line

%%%%%
\author{Nicol{\'a}s {Honorato-Droguett}}{Nagoya University, Japan \and \url{https://sites.google.com/view/nicolas-honorato-droguett}}{honorato.droguett.nicolas.n7@s.mail.nagoya-u.ac.jp}{https://orcid.org/0009-0005-1969-3649}{}
%
\author{Kazuhiro {Kurita}}{Nagoya University, Japan \and \url{https://sites.google.com/view/kazuhirokurita} }{kurita@i.nagoya-u.ac.jp}{https://orcid.org/0000-0002-7638-3322}{This work is partially supported by JSPS KAKENHI Grant Numbers 
JP21K17812, %(kurita, wakate)
JP22H03549, %(kurita, kiban B(Horiyama))
and JST ACT-X Grant Number JPMJAX2105. %(kurita, ACT-X)
}
%
\author{Tesshu {Hanaka}}{Kyushu University, Japan \and \url{https://sites.google.com/view/tesshu-hanaka/home} }{hanaka@inf.kyushu-u.ac.jp}{https://orcid.org/0000-0001-6943-856X}{This work is partially supported by JSPS KAKENHI Grant Numbers 
%JP21K17812, %(kurita, wakate)
%JP22H03549, %(kurita, kiban B(Horiyama))
%and JST ACT-X Grant Number JPMJAX2105. %(kurita, ACT-X)
%JP20H05967, %(ono, gakuhen(Makino))
%JP21K19765, %(ono, hoga)
%JP22H00513, %(ono, kibanA) 
JP21K17707, %(hanaka, wakate) JP22H00513,
JP23H04388, %(hanaka, gakuhenA)
JST CRONOS Grant Number JPMJCS24K2. %(ono, hanaka, Izumi CRONOS),
}
%
\author{Hirotaka {Ono}}{Nagoya University, Japan \and \url{https://researchmap.jp/hono} }{ono@i.nagoya-u.ac.jp}{https://orcid.org/0000-0003-0845-3947}{This work is partially supported by JSPS KAKENHI Grant Numbers
%JP21K17812, %(kurita, wakate)
%JP22H03549, %(kurita, kiban B(Horiyama))
%and JST ACT-X Grant Number JPMJAX2105. %(kurita, ACT-X)
JP20H05967, %(ono, gakuhen(Makino))
JP21K19765, %(ono, hoga)
JP22H00513, %(ono, kibanA) 
%JP21K17707, %(hanaka, wakate) JP22H00513,
%JP23H04388, %(hanaka, gakuhenA)
JST CRONOS Grant Number JPMJCS24K2. %(ono, hanaka, Izumi CRONOS),
}
%
%%%%%

\authorrunning{N. {Honorato-Droguett} and K. Kurita and T. Hanaka and H. Ono} %TODO mandatory. First: Use abbreviated first/middle names. Second (only in severe cases): Use first author plus 'et al.'

\Copyright{Nicol{\'a}s {Honorato-Droguett} and Kazuhiro Kurita and Tesshu Hanaka and Hirotaka Ono} %TODO mandatory, please use full first names. LIPIcs license is "CC-BY";  http://creativecommons.org/licenses/by/3.0/

\ccsdesc[500]{Theory of computation~Mathematical optimization} %TODO mandatory: Please choose ACM 2012 classifications from https://dl.acm.org/ccs/ccs_flat.cfm 

\keywords{Intersection graphs, Optimisation, Graph modification} %TODO mandatory; please add comma-separated list of keywords

\category{} %optional, e.g. invited paper

\ifConf
    \relatedversiondetails[cite=]{Full Version}{https://arxiv.org} %optional, e.g. full version hosted on arXiv, HAL, or other respository/website
\fi

%\relatedversiondetails[linktext={opt. text shown instead of the URL}, cite=DBLP:books/mk/GrayR93]{Classification (e.g. Full Version, Extended Version, Previous Version}{URL to related version} %linktext and cite are optional

%\supplement{}%optional, e.g. related research data, source code, ... hosted on a repository like zenodo, figshare, GitHub, ...
%\supplementdetails[linktext={opt. text shown instead of the URL}, cite=DBLP:books/mk/GrayR93, subcategory={Description, Subcategory}, swhid={Software Heritage Identifier}]{General Classification (e.g. Software, Dataset, Model, ...)}{URL to related version} %linktext, cite, and subcategory are optional

%\funding{(Optional) general funding statement \dots}%optional, to capture a funding statement, which applies to all authors. Please enter author specific funding statements as fifth argument of the \author macro.

%\acknowledgements{I want to thank \dots}%optional

%\nolinenumbers %uncomment to disable line numbering



%Editor-only macros:: begin (do not touch as author)%%%%%%%%%%%%%%%%%%%%%%%%%%%%%%%%%%
\EventEditors{John Q. Open and Joan R. Access}
\EventNoEds{2}
\EventLongTitle{42nd Conference on Very Important Topics (CVIT 2016)}
\EventShortTitle{CVIT 2016}
\EventAcronym{CVIT}
\EventYear{2016}
\EventDate{December 24--27, 2016}
\EventLocation{Little Whinging, United Kingdom}
\EventLogo{}
\SeriesVolume{42}
\ArticleNo{23}
%%%%%%%%%%%%%%%%%%%%%%%%%%%%%%%%%%%%%%%%%%%%%%%%%%%%%%
%LAYOUT SIZES
%    \usepackage{layout}
%    \usepackage{showframe}
%
\usepackage{amsmath}
%qed at the end of proofs (for LNCS)
%\let\proof\relax\let\endproof\relax
\usepackage{amsthm}

%\usepackage{txfonts}  %英文をTimes Romanのようなフォントにする.
%\usepackage[top=30truemm,bottom=30truemm,left=25truemm,right=25truemm]{geometry} %余白の設定
%\usepackage{algpseudocode,algorithm}
%hyperref
%    \usepackage[unicode=true,psdextra]{hyperref}
%bm command
    \usepackage{bm}
%itembox
    \usepackage{ascmac}
%for mathbb
%    \usepackage{amssymb}
%    \usepackage{stmaryrd}
% default mathcal
    \DeclareMathAlphabet{\mathcal}{OMS}{cmsy}{m}{n}
%Tables
    \usepackage{booktabs}
    \usepackage{array}
        \setlength{\extrarowheight}{-1pt}
    \newcolumntype{C}[1]{>{\centering\let\newline\\\arraybackslash\hspace{0pt}}m{#1}}
    \usepackage{float}
    \usepackage{multicol}
    \usepackage{multirow} % Required for multirows
%layout sizes
    \usepackage{layout}
%colors
    \usepackage{xcolor}
    \usepackage{colortbl}
%Description options
%    \usepackage{enumitem}
% appendix
    \usepackage[title,titletoc]{appendix}
% maxsizebox
    \usepackage{adjustbox}
%line number for revision
    \usepackage{lineno}
%arrangement: spaces, etc
%    \usepackage[final]{microtype}
%for cites
    \usepackage{cite}
% Complexity terms
    \usepackage[full]{complexity}
%algorithm
    \usepackage[linesnumbered,lined,ruled,noend,vlined]{algorithm2e}
    \makeatletter
    \@ifpackageloaded{algorithm2e}{
    \SetKwProg{Fn}{Function}{}{}
    \SetKwProg{Procedure}{Procedure}{}{}
    \SetKwProg{Subprocedure}{Subprocedure}{}{}
    \SetKwComment{tcc}{//}{}
    \SetKwFunction{Output}{Output}%
    \SetKw{Continue}{continue} 
    \SetKwInOut{AlgInput}{Input}
    \SetKwInOut{AlgOutput}{Output}
    \SetKwInOut{AlgPrecondition}{Pre-conditions}
    \SetKwInOut{AlgInvariant}{Invariants}
    }{}
    \makeatother
%cref
%    \usepackage[capitalise]{cleveref}
    \crefname{problem}{Problem}{Problems}
    \crefname{itembox}{Problem}{Problems}
    \crefname{algocf}{algorithm}{algorithms}
    \Crefname{algocf}{Algorithm}{Algorithms}
    %\crefalias{AlgoLine}{line}
    \crefname{AlgoLine}{line}{lines}
    \Crefname{AlgoLine}{Line}{Lines}
    \crefname{observation}{observation}{observations}
    \Crefname{observation}{Observation}{Observations}
    \crefname{mtheorem}{theorem}{theorems}
    \Crefname{mtheorem}{Theorem}{Theorems}
    \crefname{mlemma}{lemma}{lemmas}
    \Crefname{mlemma}{Lemma}{Lemmas}
    \crefname{mcorollary}{corollary}{corollaries}
    \Crefname{mcorollary}{Corollary}{Corollaries}

%restate theorems
%for copying text
%    \usepackage{clipboard}
%    \newtheorem{rdef}{Definition}
%    \newtheorem{rlemma}{Lemma}
%    \newtheorem{rthm}{Theorem}
%    \newenvironment{repdefinition}[1]
%      {\renewcommand\therdef{\ref*{#1}}\rdef}
%      {\endrdef}
%    \newenvironment{replemma}[1]
%      {\renewcommand\therlemma{\ref*{#1}}\rlemma}
%      {\endrlemma}
%    \newenvironment{reptheorem}[1]
%      {\renewcommand\therthm{\ref*{#1}}\rthm}
%      {\endrthm}


% send proofs to appendix
% appendix
    \usepackage[title]{appendix}
    %\renewcommand{\thechapter}{1}
    %used to move proofs to the appendix.
    \ifConf
         \usepackage{apxproof}
    \fi
    %used to move proofs to the main sections
    \ifFull
       \usepackage[appendix=inline]{apxproof}
    \fi
    \renewenvironment{proofsketch}{\begin{axp@oldproof}[Sketch of Proof]}{\end{axp@oldproof}}
    %used to hide the appendix
    %    \usepackage[appendix=strip]{apxproof}
    \newtheoremrep{mtheorem}[theorem]{\ifConf$\star\,$\fi Theorem}
    \newtheoremrep{mlemma}[lemma]{\ifConf$\star\,$\fi Lemma}
    \newtheoremrep{mcorollary}[corollary]{\ifConf$\star\,$\fi Corollary}
    %format the appendix bibliography
        \renewcommand{\appendixbibliographystyle}{plainurl}
    % changes the counter of appendix section from \Alph to \arabic
        \renewcommand{\appendixprelim}{\renewcommand{\thesection}{Appendix \arabic{section}}\setcounter{section}{0}\renewcommand{\appendix}{}\clearpage\onecolumn}
        \renewcommand{\appendixbibliographyprelim}{\clearpage\onecolumn}
        %\renewcommand{\appendixrefname}{Appendix}

%To prevent placing floats before a section.
    \usepackage[section]{placeins}

    
%%%%%%%%%%%---SETME-----%%%%%%%%%%%%%
%replace @@ with the submission number submission site.
\newcommand{\thiswork}{INF$^2$\xspace}
%%%%%%%%%%%%%%%%%%%%%%%%%%%%%%%%%%%%


%\newcommand{\rev}[1]{{\color{olivegreen}#1}}
\newcommand{\rev}[1]{{#1}}


\newcommand{\JL}[1]{{\color{cyan}[\textbf{\sc JLee}: \textit{#1}]}}
\newcommand{\JW}[1]{{\color{orange}[\textbf{\sc JJung}: \textit{#1}]}}
\newcommand{\JY}[1]{{\color{blue(ncs)}[\textbf{\sc JSong}: \textit{#1}]}}
\newcommand{\HS}[1]{{\color{magenta}[\textbf{\sc HJang}: \textit{#1}]}}
\newcommand{\CS}[1]{{\color{navy}[\textbf{\sc CShin}: \textit{#1}]}}
\newcommand{\SN}[1]{{\color{olive}[\textbf{\sc SNoh}: \textit{#1}]}}

%\def\final{}   % uncomment this for the submission version
\ifdefined\final
\renewcommand{\JL}[1]{}
\renewcommand{\JW}[1]{}
\renewcommand{\JY}[1]{}
\renewcommand{\HS}[1]{}
\renewcommand{\CS}[1]{}
\renewcommand{\SN}[1]{}
\fi

%%% Notion for baseline approaches %%% 
\newcommand{\baseline}{offloading-based batched inference\xspace}
\newcommand{\Baseline}{Offloading-based batched inference\xspace}


\newcommand{\ans}{attention-near storage\xspace}
\newcommand{\Ans}{Attention-near storage\xspace}
\newcommand{\ANS}{Attention-Near Storage\xspace}

\newcommand{\wb}{delayed KV cache writeback\xspace}
\newcommand{\Wb}{Delayed KV cache writeback\xspace}
\newcommand{\WB}{Delayed KV Cache Writeback\xspace}

\newcommand{\xcache}{X-cache\xspace}
\newcommand{\XCACHE}{X-Cache\xspace}


%%% Notions for our methods %%%
\newcommand{\schemea}{\textbf{Expanding supported maximum sequence length with optimized performance}\xspace}
\newcommand{\Schemea}{\textbf{Expanding supported maximum sequence length with optimized performance}\xspace}

\newcommand{\schemeb}{\textbf{Optimizing the storage device performance}\xspace}
\newcommand{\Schemeb}{\textbf{Optimizing the storage device performance}\xspace}

\newcommand{\schemec}{\textbf{Orthogonally supporting Compression Techniques}\xspace}
\newcommand{\Schemec}{\textbf{Orthogonally supporting Compression Techniques}\xspace}



% Circular numbers
\usepackage{tikz}
\newcommand*\circled[1]{\tikz[baseline=(char.base)]{
            \node[shape=circle,draw,inner sep=0.4pt] (char) {#1};}}

\newcommand*\bcircled[1]{\tikz[baseline=(char.base)]{
            \node[shape=circle,draw,inner sep=0.4pt, fill=black, text=white] (char) {#1};}}
%
\begin{document}

\maketitle

%TODO mandatory: add short abstract of the document
\begin{abstract}
    We study {\gged} (GGED), a graph-editing model to compute the minimum edit distance of intersection graphs that uses moving objects as an edit operation.
    We first show an $O(n\log n)$-time algorithm that minimises the total moving distance to disperse unit intervals.
    This algorithm is applied to render a given unit interval graph (i) edgeless, (ii) acyclic and (iii) $k$-clique-free. % and (iv) bipartite graphs.
    We next show that GGED becomes strongly \NP-hard when rendering a weighted interval graph (i) edgeless, (ii) acyclic and (iii) $k$-clique-free. % and (iv) $k$-degree graphs.
    Lastly, we prove that minimising the maximum moving distance for rendering a unit disk graph edgeless is strongly \NP-hard over the $L_1$ and $L_2$ distances.
\end{abstract}

\section{Introduction}


\begin{figure}[t]
\centering
\includegraphics[width=0.6\columnwidth]{figures/evaluation_desiderata_V5.pdf}
\vspace{-0.5cm}
\caption{\systemName is a platform for conducting realistic evaluations of code LLMs, collecting human preferences of coding models with real users, real tasks, and in realistic environments, aimed at addressing the limitations of existing evaluations.
}
\label{fig:motivation}
\end{figure}

\begin{figure*}[t]
\centering
\includegraphics[width=\textwidth]{figures/system_design_v2.png}
\caption{We introduce \systemName, a VSCode extension to collect human preferences of code directly in a developer's IDE. \systemName enables developers to use code completions from various models. The system comprises a) the interface in the user's IDE which presents paired completions to users (left), b) a sampling strategy that picks model pairs to reduce latency (right, top), and c) a prompting scheme that allows diverse LLMs to perform code completions with high fidelity.
Users can select between the top completion (green box) using \texttt{tab} or the bottom completion (blue box) using \texttt{shift+tab}.}
\label{fig:overview}
\end{figure*}

As model capabilities improve, large language models (LLMs) are increasingly integrated into user environments and workflows.
For example, software developers code with AI in integrated developer environments (IDEs)~\citep{peng2023impact}, doctors rely on notes generated through ambient listening~\citep{oberst2024science}, and lawyers consider case evidence identified by electronic discovery systems~\citep{yang2024beyond}.
Increasing deployment of models in productivity tools demands evaluation that more closely reflects real-world circumstances~\citep{hutchinson2022evaluation, saxon2024benchmarks, kapoor2024ai}.
While newer benchmarks and live platforms incorporate human feedback to capture real-world usage, they almost exclusively focus on evaluating LLMs in chat conversations~\citep{zheng2023judging,dubois2023alpacafarm,chiang2024chatbot, kirk2024the}.
Model evaluation must move beyond chat-based interactions and into specialized user environments.



 

In this work, we focus on evaluating LLM-based coding assistants. 
Despite the popularity of these tools---millions of developers use Github Copilot~\citep{Copilot}---existing
evaluations of the coding capabilities of new models exhibit multiple limitations (Figure~\ref{fig:motivation}, bottom).
Traditional ML benchmarks evaluate LLM capabilities by measuring how well a model can complete static, interview-style coding tasks~\citep{chen2021evaluating,austin2021program,jain2024livecodebench, white2024livebench} and lack \emph{real users}. 
User studies recruit real users to evaluate the effectiveness of LLMs as coding assistants, but are often limited to simple programming tasks as opposed to \emph{real tasks}~\citep{vaithilingam2022expectation,ross2023programmer, mozannar2024realhumaneval}.
Recent efforts to collect human feedback such as Chatbot Arena~\citep{chiang2024chatbot} are still removed from a \emph{realistic environment}, resulting in users and data that deviate from typical software development processes.
We introduce \systemName to address these limitations (Figure~\ref{fig:motivation}, top), and we describe our three main contributions below.


\textbf{We deploy \systemName in-the-wild to collect human preferences on code.} 
\systemName is a Visual Studio Code extension, collecting preferences directly in a developer's IDE within their actual workflow (Figure~\ref{fig:overview}).
\systemName provides developers with code completions, akin to the type of support provided by Github Copilot~\citep{Copilot}. 
Over the past 3 months, \systemName has served over~\completions suggestions from 10 state-of-the-art LLMs, 
gathering \sampleCount~votes from \userCount~users.
To collect user preferences,
\systemName presents a novel interface that shows users paired code completions from two different LLMs, which are determined based on a sampling strategy that aims to 
mitigate latency while preserving coverage across model comparisons.
Additionally, we devise a prompting scheme that allows a diverse set of models to perform code completions with high fidelity.
See Section~\ref{sec:system} and Section~\ref{sec:deployment} for details about system design and deployment respectively.



\textbf{We construct a leaderboard of user preferences and find notable differences from existing static benchmarks and human preference leaderboards.}
In general, we observe that smaller models seem to overperform in static benchmarks compared to our leaderboard, while performance among larger models is mixed (Section~\ref{sec:leaderboard_calculation}).
We attribute these differences to the fact that \systemName is exposed to users and tasks that differ drastically from code evaluations in the past. 
Our data spans 103 programming languages and 24 natural languages as well as a variety of real-world applications and code structures, while static benchmarks tend to focus on a specific programming and natural language and task (e.g. coding competition problems).
Additionally, while all of \systemName interactions contain code contexts and the majority involve infilling tasks, a much smaller fraction of Chatbot Arena's coding tasks contain code context, with infilling tasks appearing even more rarely. 
We analyze our data in depth in Section~\ref{subsec:comparison}.



\textbf{We derive new insights into user preferences of code by analyzing \systemName's diverse and distinct data distribution.}
We compare user preferences across different stratifications of input data (e.g., common versus rare languages) and observe which affect observed preferences most (Section~\ref{sec:analysis}).
For example, while user preferences stay relatively consistent across various programming languages, they differ drastically between different task categories (e.g. frontend/backend versus algorithm design).
We also observe variations in user preference due to different features related to code structure 
(e.g., context length and completion patterns).
We open-source \systemName and release a curated subset of code contexts.
Altogether, our results highlight the necessity of model evaluation in realistic and domain-specific settings.






% !TEX root =  ../main.tex
\section{Background on causality and abstraction}\label{sec:preliminaries}

This section provides the notation and key concepts related to causal modeling and abstraction theory.

\spara{Notation.} The set of integers from $1$ to $n$ is $[n]$.
The vectors of zeros and ones of size $n$ are $\zeros_n$ and $\ones_n$.
The identity matrix of size $n \times n$ is $\identity_n$. The Frobenius norm is $\frob{\mathbf{A}}$.
The set of positive definite matrices over $\reall^{n\times n}$ is $\pd^n$. The Hadamard product is $\odot$.
Function composition is $\circ$.
The domain of a function is $\dom{\cdot}$ and its kernel $\ker$.
Let $\mathcal{M}(\mathcal{X}^n)$ be the set of Borel measures over $\mathcal{X}^n \subseteq \reall^n$. Given a measure $\mu^n \in \mathcal{M}(\mathcal{X}^n)$ and a measurable map $\varphi^{\V}$, $\mathcal{X}^n \ni \mathbf{x} \overset{\varphi^{\V}}{\longmapsto} \V^\top \mathbf{x} \in \mathcal{X}^m$, we denote by $\varphi^{\V}_{\#}(\mu^n) \coloneqq \mu^n(\varphi^{\V^{-1}}(\mathbf{x}))$ the pushforward measure $\mu^m \in \mathcal{M}(\mathcal{X}^m)$. 


We now present the standard definition of SCM.

\begin{definition}[SCM, \citealp{pearl2009causality}]\label{def:SCM}
A (Markovian) structural causal model (SCM) $\scm^n$ is a tuple $\langle \myendogenous, \myexogenous, \myfunctional, \zeta^\myexogenous \rangle$, where \emph{(i)} $\myendogenous = \{X_1, \ldots, X_n\}$ is a set of $n$ endogenous random variables; \emph{(ii)} $\myexogenous =\{Z_1,\ldots,Z_n\}$ is a set of $n$ exogenous variables; \emph{(iii)} $\myfunctional$ is a set of $n$ functional assignments such that $X_i=f_i(\parents_i, Z_i)$, $\forall \; i \in [n]$, with $ \parents_i \subseteq \myendogenous \setminus \{ X_i\}$; \emph{(iv)} $\zeta^\myexogenous$ is a product probability measure over independent exogenous variables $\zeta^\myexogenous=\prod_{i \in [n]} \zeta^i$, where $\zeta^i=P(Z_i)$. 
\end{definition}
A Markovian SCM induces a directed acyclic graph (DAG) $\mathcal{G}_{\scm^n}$ where the nodes represent the variables $\myendogenous$ and the edges are determined by the structural functions $\myfunctional$; $ \parents_i$ constitutes then the parent set for $X_i$. Furthermore, we can recursively rewrite the set of structural function $\myfunctional$ as a set of mixing functions $\mymixing$ dependent only on the exogenous variables (cf. \cref{app:CA}). A key feature for studying causality is the possibility of defining interventions on the model:
\begin{definition}[Hard intervention, \citealp{pearl2009causality}]\label{def:intervention}
Given SCM $\scm^n = \langle \myendogenous, \myexogenous, \myfunctional, \zeta^\myexogenous \rangle$, a (hard) intervention $\iota = \operatorname{do}(\myendogenous^{\iota} = \mathbf{x}^{\iota})$, $\myendogenous^{\iota}\subseteq \myendogenous$,
is an operator that generates a new post-intervention SCM $\scm^n_\iota = \langle \myendogenous, \myexogenous, \myfunctional_\iota, \zeta^\myexogenous \rangle$ by replacing each function $f_i$ for $X_i\in\myendogenous^{\iota}$ with the constant $x_i^\iota\in \mathbf{x}^\iota$. 
Graphically, an intervention mutilates $\mathcal{G}_{\mathsf{M}^n}$ by removing all the incoming edges of the variables in $\myendogenous^{\iota}$.
\end{definition}

Given multiple SCMs describing the same system at different levels of granularity, CA provides the definition of an $\alpha$-abstraction map to relate these SCMs:
\begin{definition}[$\abst$-abstraction, \citealp{rischel2020category}]\label{def:abstraction}
Given low-level $\mathsf{M}^\ell$ and high-level $\mathsf{M}^h$ SCMs, an $\abst$-abstraction is a triple $\abst = \langle \Rset, \amap, \alphamap{} \rangle$, where \emph{(i)} $\Rset \subseteq \datalow$ is a subset of relevant variables in $\mathsf{M}^\ell$; \emph{(ii)} $\amap: \Rset \rightarrow \datahigh$ is a surjective function between the relevant variables of $\mathsf{M}^\ell$ and the endogenous variables of $\mathsf{M}^h$; \emph{(iii)} $\alphamap{}: \dom{\Rset} \rightarrow \dom{\datahigh}$ is a modular function $\alphamap{} = \bigotimes_{i\in[n]} \alphamap{X^h_i}$ made up by surjective functions $\alphamap{X^h_i}: \dom{\amap^{-1}(X^h_i)} \rightarrow \dom{X^h_i}$ from the outcome of low-level variables $\amap^{-1}(X^h_i) \in \datalow$ onto outcomes of the high-level variables $X^h_i \in \datahigh$.
\end{definition}
Notice that an $\abst$-abstraction simultaneously maps variables via the function $\amap$ and values through the function $\alphamap{}$. The definition itself does not place any constraint on these functions, although a common requirement in the literature is for the abstraction to satisfy \emph{interventional consistency} \cite{rubenstein2017causal,rischel2020category,beckers2019abstracting}. An important class of such well-behaved abstractions is \emph{constructive linear abstraction}, for which the following properties hold. By constructivity, \emph{(i)} $\abst$ is interventionally consistent; \emph{(ii)} all low-level variables are relevant $\Rset=\datalow$; \emph{(iii)} in addition to the map $\alphamap{}$ between endogenous variables, there exists a map ${\alphamap{}}_U$ between exogenous variables satisfying interventional consistency \cite{beckers2019abstracting,schooltink2024aligning}. By linearity, $\alphamap{} = \V^\top \in \reall^{h \times \ell}$ \cite{massidda2024learningcausalabstractionslinear}. \cref{app:CA} provides formal definitions for interventional consistency, linear and constructive abstraction.

%\section{Satisfying \texorpdfstring{$\Pi_{\texttt{edgeless}}$}{} on Interval Graphs}\label{sec:edg_interval}
%%%%%%%%%%%%%%%%%%%%%%%%%%%%%%%%%%%%%%%%%%%%%%%%%%%%%%%%%%%%%%%%%%%%%%%%%%%%%%%%%%%%%%%%%%%%%%%%%
\section{Satisfying \texorpdfstring{$\Pi_{\texttt{edgeless}}$}{} on Unit Interval Graphs in \texorpdfstring{$O(n\log n)$}{} time}\label{sec:edg_uig}

We show that $\Pi_{\texttt{edgeless}}$ can be satisfied in $O(n\log n)$ time given a unit interval graph of $n$ intervals.
%In particular, we design an algorithm to disperse unit intervals by a distance $s \ge 1$.
We start by defining a problem that we call {\idisp} and then use the algorithm designed to satisfy the properties $\Pi_{\texttt{edgeless}}$, $\Pi_{\texttt{acyc}}$ and $\overline{\Pi_{k\texttt{-clique}}}$. %and $\Pi_{\texttt{bipar}}$. 
\ifConf
{\idisp} receives as input a collection $\I$ of $n$ intervals and a real $s \ge 1$, and asks for the minimum value of the total moving distance to obtain a collection $\I'$ that satisfies $c(I'_j)-c(I'_i) \ge s$ for each $I'_i, I'_j \in \I'$, $i<j$.
\fi
\ifFull
The problem is defined as follows:

\begin{itembox}[l]{\idisp}
    \begin{description}%[itemsep=0pt,align=left,leftmargin=50pt,labelindent=5pt,style=multiline]
        \item[Input:] A collection $\I$ of $n$ intervals and a real $s \ge 1$.
        \item[Output:] The minimum value of the total moving distance for obtaining a collection $\I'$ that satisfies $c(I'_j)-c(I'_i) \ge s$ for each $I'_i, I'_j \in \I'$, $i<j$.
    \end{description}
\end{itembox}
\fi
When $s = 1$, {\idisp} is equivalent to {\gged} on unit interval graphs for satisfying $\Pi_{\texttt{edgeless}}$.
%We assume that the intervals $I \in \I$ move over the $L_2$ distance. 
%Definitions-%%%%%%%%%%%%%%%%%%%%%%%%%%%%%%%%%%%%%%%%%%%%%%%%%%%%%%%%%%%%%%%%%%%%%%%%%%%%%%%%
%\paragraph*{Preliminaries} 
%
%For simplicity, it is assumed that intervals are open. This assumption helps to avoid dealing with the infinitesimal small distance needed to be applied to separate closed intervals. 
%
For simplicity, the intervals are assumed to be open. This avoids the need to address infinitesimally small distances required to separate closed intervals.
%
We must first introduce some basic definitions and notation to describe the algorithm.
Given a collection of $n$ intervals $\I= \set{I_1,\ldots,I_{n}}$, let $D = (d_1,\ldots,d_{n})$ be a vector such that $d_i$ is the moving distance applied to $I_i$. 
We denote by $\I^D = \{I^D_1\ldots, I^D_{n}\}$ the collection of intervals such that $c(I^D_i) = c(I_i) +d_i$.
The set $\D(\I) \subseteq \mathbb{R}^n$ is the set of vectors that describe the moving distance applied to intervals such that the condition of {\idisp} is satisfied. 
In other words, for all $D = (d_1,\ldots,d_n) \in \D(\I)$, $c(I^D_{j})+c(I^D_{i}) \ge s$ holds for $i < j$.
We use $\D^{\mathit{opt}}(\I) \subseteq \D(\I)$ to denote the subset of vectors in $\D(\I)$ that minimises the total moving distance applied to intervals; i.e. $\D^{\mathit{opt}}(\I) = \set{D=(d_1,\ldots,d_n) \in \D(\I)\mid \sum_{1\le i \le n} |d_i| = \min_{D' = (d'_1,\ldots,d'_n) \in \D(\I)}{\sum_{1\le i \le n}|d'_i|}}$.

%%%%%%%%%%%%%%%%%%%%%%%%%%%%%%%%%%%%%%%%%%%%%%%%%%%%%%%%%%%%%%%%%%%%%%%%%%%%%%%%%%%%%%%%%%%%%
%We are now ready to introduce some properties used in the algorithm.
Intuitively, we aim to find a vector $D \in \D^{\mathit{opt}}(\I)$ to move each interval so that the distance between each pair of intervals is at least $s$.
Given an arbitrary $D \in \D^{\mathit{opt}}(\I)$, the order of $\I^D$ may be different from the order of $\I$.
However, it was previously shown~\cite{HonoratoDroguett2024} that the there are always a vector $D \in \D^{\mathit{opt}}(\I)$ such that the order of $\I^D$ preserves the order of $\I$.
%However, the following lemma ensures that the order of $\I$ does not change for an arbitrary property $\Pi$.
%
%\begin{mlemmarep}\label{lem:edgeless:same}
%    Let $D = (d_1, \ldots, d_n)$ be a vector that describes the distances applied to each unit interval.
%    If $\I$ has a pair of unit intervals $I$ and $J$ that satisfy $c(I) \le c(J)$ and $c(I) + d_i > c(J) + d_j$, then the same collection of unit intervals exists such that the total moving distance is at most $\sum_{d_i \in D} |d_i|$.
%\end{mlemmarep}
%\begin{proof}
%    Let $I$ and $J$ be two intervals in $\I$.
%    Without loss of generality, assume that $c(I) \le c(J)$.
%    Suppose that $c(I) + d_i \ge c(J) + d_j$.
%    Since $c(I) \le c(J)$, $d_i \ge d_j$.
%    In this case, by replacing $d_i$ and $d_j$ with $c(J) - c(I) + d_j$ and $c(I) - c(J) + d_i$,
%    we obtain $c(I) + (c(J) - c(I) + d_j) = c(J) + d_j$ and $c(J) + (c(I) - c(J) + d_i) = c(I) + d_i$, respectively.
%    Hence the collection of intervals is the same.
%
%    We show that $|d_i| + |d_j| \ge |c(I) - c(J) + d_i| + |c(J) - c(I) + d_j|$.
%    We consider the following cases.
%    Suppose that $d_i \ge c(J) - c(I)$.
%    In this case, $|c(I) - c(J) + d_i| = (c(I) - c(J) + d_i)$.
%    If $c(J) - c(I) + d_j \ge 0$, then
%    \begin{align*}
%        |c(I) - c(J) + d_i| + |c(J) - c(I) + d_j| = d_i + d_j \le |d_i| + |d_j|.
%    \end{align*}
%    If $c(J) - c(I) + d_j  <  0$, then
%    \begin{align*}
%        |c(I) - c(J) + d_i| + |c(J) - c(I) + d_j| = d_i - d_j + 2(c(I) - c(J)).
%    \end{align*}
%    Since $c(J) \ge c(I)$, $d_j < 0$.
%    Therefore, $|d_i| + |d_j| = d_i - d_j$ and $d_i - d_j \ge d_i - d_j + 2(c(I) - c(J))$.
%
%    Suppose that $d_i < c(J) - c(I)$.
%    If $c(J) - c(I) + d_j < 0$, then 
%    \begin{align*}
%        |c(I) - c(J) + d_i| + |c(J) - c(I) + d_j| = -d_i - d_j \le |d_i| + |d_j|.
%    \end{align*}
%    If $c(J) - c(I) + d_j \ge 0$, then 
%    \begin{align*}
%        |c(I) - c(J) + d_i| + |c(J) - c(I) + d_j| = 2(c(J) - c(I)) - d_i + d_j.
%    \end{align*}
%    Since $d_j \ge c(J) - c(I)$, $|d_i| + |d_j| = |d_i| - d_j$.
%    Therefore, 
%    \begin{align*}
%        |d_i| - d_j + 2(c(J) - c(I)) - d_i + d_j & = |d_i| - d_i + 2(c(J) - c(I)).
%    \end{align*}
%    If $d_i \ge 0$, then the statement is true since $|d_i| - d_i = 0$ and $c(J) \ge c(I)$.
%    If $d_i  <  0$, then $-2(d_i - (c(J) - c(I)))$.
%    Since $d_i < c(J) - c(I)$, $-2(d_i - (c(J) - c(I))) < 0$ holds.
%\end{proof}
%%%%%%%%%%%%%%%%%%%%%%%%%%%%%%%%%%%%%%%%%%%%%%%%%%%%%%%%%%%%%%%%%%%%%%%%%%%%%%%%%%%%%%%%%%%%%
%
%\Cref{lem:edgeless:same} 
This implies that there always exists an optimal solution of {\idisp} for which checking the inequality $(c(I_{i+1})+d_{i+1}) - (c(I_{i})+d_{i}) \ge s \text{ for } \leq i \leq n-1$ is sufficient.

%
%We now define a function to move intervals so that the distance between the centres is \emph{exactly} equal to $s$ following the order induced by the centre of intervals. We call this function \emph{equispace function}.
%
We now define the \emph{equispace function}, which moves intervals so that the distance between their centres is exactly $s$, maintaining the order induced by interval centres.
%
\begin{definition}[\textit{Equispace function}]\label{def:tmd_equispace}
    Let $(\I,s)$ be an instance of {\idisp} where $\I$ is a collection of unit intervals. The \emph{equispace function} of $\I$ to a point $x$ is a function $E: \I\times \mathbb{R} \rightarrow \mathbb{R}$ defined as:
    \begin{align*}
        E(\I,x) = \sum_{i = 1}^n f_i(x),\quad f_i(x) = |x-c(I_i) - (n-i)s|.
    \end{align*}
    The vector that describes the moving distances given by $E(\I,x)$ is defined as $E_x(\I) = (e_1,\ldots,e_n)= \left(\alpha_1 f_1(x),\ldots,\alpha_n f_n(x)\right)$ where $\alpha_i = 1$ if $x\ge c(I_i)+(n-i)s$ and $\alpha_i = -1$ otherwise, for $1\le i \le n$.
    We also denote by $\I^{E_x(\I)} = \{I^{E_x(\I)}_1\ldots, I^{E_x(\I)}_{n}\}$ the collection of intervals where $c(I^{E_x(\I)}_i) = c(I_i) +\alpha_if_i(x)$ for $1\le i \le n$.
\end{definition}
%\Cref{fig:equispace_function} illustrates the movement represented by the equispace function. In the figure, a collection $\I = \set{I_1,\ldots,I_6}$ is moved to the point $x$ by the equispace function. In particular, the interval $I_6$ is centred at $x$ and subsequent intervals are positioned behind $I_6$ such that the distance between consecutive intervals is exactly $s$, while maintaining the order of $\I$. 
%\begin{figure}[bt]
%    \centering
%    \includegraphics[scale=1,page=1]{tex/Dispersal/fig/dispersal.pdf}
%    \caption{Equispace function: A collection $\I = \set{I_1,\ldots,I_6}$ is moved to the point $x$ by the equispace function. After moving the intervals, the distance between consecutive intervals is exactly $s$.}
%    \label{fig:equispace_function}
%\end{figure}
By the above, $E_x(\I) \in \D(\I)$ for all $x \in \mathbb{R}$.
Moreover, $c(I^{E_x(\I)}_{i+1}) - c(I^{E_x(\I)}_{i}) = s$ for all $1 \le i \le n-1$.
%As we shall prove, the advantage of using the equispace function is that its optimal solution can be found efficiently.
We first prove that for certain collections of intervals, minimising $E$ gives a vector contained in $\D^{\mathit{opt}}(\I)$.
\begin{mlemmarep}\label{lem:tmd_convex}
    The equispace function $E(\I,x)$ is a piecewise-linear convex function.
\end{mlemmarep}
\begin{proof}
    By \Cref{def:tmd_equispace}, the function $E(\I,x)$ is a function of the form $f(x) = f_n(x)+\cdots+ f_1(x) = |x - c(I_n)| + |x - c(I_{n-1}) - s| + \cdots + |x - c(I_1) - (n-1)s|$. 
    The absolute function is convex; hence each $f_i$ is convex.
    Consequently, $f(x)$ is also convex as it is a sum of convex functions.
    On the other hand, the piecewise linearity of $E$ is given by the fact that each absolute function is piecewise linear.
    Therefore, $E$ is a piecewise-linear convex function.
\end{proof}

%Let $\I = \set{I_1,\ldots,I_n}$ be a collection of $n$ unit intervals. 
We define the \emph{set of breakpoints of $E(\I,x)$} to be the set $B_\I =\set{b_1^\I,\ldots,b_n^\I} = \set{c(I_i) + (n-i)s\mid I_i \in \I,\: 1\le i\le n}$.
Given a collection of intervals $\I$, we define the equispace function $E(\I, x)$ as a sequence of linear functions $E_1(\I, x), \ldots, E_{\size{\I}+1}(\I, x)$. 
The slope of $E_i(\I, x)$ is less than the slope of $E_j(\I, x)$ for $1 \le i < j \le \size{\I}$.
Since the equispace function is convex and piecewise linear, the points that minimise $E$ are located within a range $b_\ell \leq x \leq b_r$, where $b_{\ell} \le b_r$ and $b_\ell, b_r \in B_\I$.
We prove that $b_\ell$ and $b_r$ can be easily found.

\begin{mlemmarep}\label{lem:tmd_opt}
     %Given an arbitrary collection $\I$ of $n$ unit intervals, t
     The minimum value of $E(\I,x)$ is given by the breakpoint $b^{\I}_{(n+1)/2}$ if $n$ is odd, and by breakpoints $b^{\I}_{n/2}$ and $b^{\I}_{(n/2)+1}$ otherwise.
\end{mlemmarep}
\begin{proof}
    Let $s_i$ be the slope of $E_i(\I,x)$.
    The function $E(\I,x)$ is a function of the form $f_1(x)+\cdots+f_n(x)$ where $f_i(x) = |x-c(I_i) -(n-i)s|$.
    The slope of $f_i$ is equal to $1$ if $x\ge c(I_i) +(n-i)s$ and $-1$ otherwise, which implies that $s_i = -n + 2(i-1)$ for $1\le i \le n +1$ and thus $s_{i+1}-s_i = 2$ for $1\le i \le n$.
    Suppose first that $n$ is odd.
    Then, 
    \begin{align*}
        s_{(n+1)/2} & = -n + 2((n+1)/2 - 1) = -n +n+1 - 2 = -1\text{ and}\\ 
        s_{(n+1)/2 + 1} &= -n + 2((n+1)/2 + 1 - 1) = -n+n+1 = 1.
    \end{align*}
    Hence, it follows that the $((n+1)/2)$th breakpoint minimises $E$.
    Suppose that $n$ is even.
    In this case, we have that
    \begin{align*}
        s_{n/2} & = -n + 2(n/2 - 1) = -n +n - 2 = -2,\\ 
        s_{n/2 + 1} &= -n + 2(n/2 + 1 - 1) = -n+n = 0\text{ and},\\
        s_{n/2 + 2} &= -n + 2(n/2+2 - 1) = -n + n +4 -2 = 2.
    \end{align*}
    Hence, any point $x\in \mathbb{R}$ such that $b^{\I}_{n/2} \le x \le b^{\I}_{n/2+1}$ minimises $E$, implying that $b^{\I}_{n/2}$ and $b^{\I}_{(n/2)+1}$ also minimise $E$.
    This concludes the proof.
\end{proof}

By \Cref{lem:tmd_opt}, the minimum value of $E$ for an arbitrary collection of intervals $\I$ is given by the median value(s) of $B_{\I}$.
%
%We now show for which collections of intervals it is sufficient to minimise $E$ to find a vector in $\D^{\mathit{opt}}(\I)$. We characterise such collections as follows.
%
We now show which collections allow minimising $E$ to obtain a vector in $\D^{\mathit{opt}}(\I)$, characterised as follows:
%
\begin{definition}[\textit{Optimally Equispaceable Collections}]\label{def:equispace_collections}
    Given a collection of intervals $\I$, we say that $\I$ is \emph{optimally equispaceable} if there exists a $D \in \D^{\mathit{opt}}(\I)$ such that $D = E_{x^*}(\I)$ and $x^* \in \argmin_{x \in \mathbb{R}} E(\I,x)$. Equivalently, $\I$ is optimally equispaceable if $E_{x^*}(\I) \in \D^{\mathit{opt}}(\I)$ for all $x^* \in \argmin_{x \in \mathbb{R}} E(\I,x)$.
\end{definition}

\begin{lemma}\label{lem:partition_opt}
    Let $\I = \set{I_1,\ldots,I_n}$ be a collection of unit intervals such that $c(I_{i+1}) - c(I_{i}) \le s$ for $1\le i \le n-1$.
    %and $D = (d_1,\ldots,d_n) \in \D(\I)$. 
    Then $\I$ is optimally equispaceable.
    Moreover, there exists a $D \in \D^{\mathit{opt}}(\I)$ such that $c(I^D_{i+1}) -c(I^D_i) = s$ holds for all $1 \le i \le n-1$.
\end{lemma}
\begin{proof}
    We only prove the latter, as the existence of $D$ in $\D^{\mathit{opt}}(\I)$ directly implies the optimal equispaceability of $\I$.
    That is, we show that $\I^D$ satisfies $c(I^D_{i+1}) - c(I^D_i) = s$, for $1 \le i \le n-1$.
    By the definition of {\idisp}, we have $c(I^D_{i+1}) \ge c(I^D_{i})$ and $c(I^D_{i+1}) - c(I^D_{i}) \ge s$ for $1\le i \le n-1$.
    Suppose that there exists a pair of intervals $I_i$ and $I_{i+1}$ that satisfies $c(I^D_{i+1}) - c(I^D_{i}) > s$. 
    Let $s' = c(I^D_{i+1}) - c(I^D_{i})$ and $\delta = s'-s$.
    We show how to obtain a total moving distance $D'$ such that $\sum_{d\in D'} |d|< \sum_{d\in D} |d|$ and $c(I^{D'}_{i+1}) - c(I^{D'}_{i}) = s$.
    
    We divide the proof into three cases: (i) $d_i \ge 0$, (ii) $d_{i+1} \le 0$ and (iii) $d_i\le 0$ and $d_{i+1}\ge 0$. 
    For case (i), it follows that $d_j \ge d_{j-1}\ge 0$ for $i+1\le j \le n$ and $(c(I^D_{i+1})-\delta) - c(I^D_{i})  = c(I_{i+1})+(d_{i+1}-\delta) - (c(I_i) + d_i) = s$ holds.
    Let $D' (d'_1,\ldots,d'_n) = (d_1,\ldots,d_i,d_{i+1}-\delta,\ldots,d_{n}-\delta)$.
    The dispersal condition is satisfied by $\I^{D'}$.
    Furthermore, since $\delta > 0$, the total moving distance satisfies 
    $\sum_{d\in D'} |d| = \sum_{j=1}^i |d_j| + \sum_{j=i+1}^n d_j - \delta < \sum_{d\in D} |d|$, which contradicts the optimality of $D$.
    
    For case (ii), $d_j \le d_{j+1}$ for $1\le j \le i$ holds, and the argument for case (i) applies analogously for $D' = (d'_1,\ldots,d'_n) = (d_1+\delta,\ldots,d_i+\delta,d_{i+1},\ldots,d_{n})$.
    
    We only need to prove case (iii).
    Let $\delta = s' - s$ as in the previous cases.
    If $\delta \le d_{i+1}$, then we move the intervals as in the first case. If $\delta \le -d_{i}$, then we move intervals as in the second case. 
    In both cases, the same argument applies and the total moving distance contradicts the optimality of $D$.
    Thus we assume that $\delta > d_{i+1}, -d_i$ holds.
    Without loss of generality, we move intervals $I_j$ for $i+1\le j \le n$ by $d_{i+1}$ to the left by $\delta' = d_{i+1}$ and intervals $I_j$ for $1\le j \le i$ to the right by $\delta'' = (c(I^D_{i+1})-\delta')- c(I^D_i) - s$. 
    Then $(c(I^D_{i+1})-\delta') - (c(I^D_i) +\delta'') = s$ holds since $d_{i+1}-\delta' = 0$. 
    Let $D' = (d'_1,\ldots,d'_n) = (d_1+\delta'',\ldots,d_i+\delta'',d_{i+1}-\delta',\ldots,d_n-\delta')$. 
    The inequality $\sum_{d\in D'} |d| = \sum_{j=1}^i d_j+\delta'' + \sum_{j=i+1}^n d_j - \delta' < \sum_{d\in D} |d|$ holds since $\delta',\delta''>0$, which contradicts the optimality of $D$.
    Therefore, in an optimal solution, $\I$ must satisfy $c(I_{i+1})+d_{i+1} - (c(I_i) + d_i) = c(I^D_{i+1})-c(I^D_i) = s$, for $1 \le i \le n-1$.
\end{proof}

Let $\I = \set{I_1,\ldots,I_n}$ and $\J = \set{J_1,\ldots,J_m}$ be two collections of unit intervals and let $x_1,x_2\in \argmin_{x\in\mathbb{R}} E(\I,x)$, $x_1\le x_2$, and $y_1,y_2\in \argmin_{x\in\mathbb{R}} E(\J,x)$, $y_1\le y_2$, be the breakpoints that minimise $E$ for $\I$ and $\J$, respectively.
We say that \emph{$\I$ and $\J$ intersect when equispaced} when $y_1 \le x_2 + \size{\J}s$.
In other words, $\I$ and $\J$ intersect when equispaced whenever there exist points $x_1\le x \le x_2$ and $y_1 \le y \le y_2$ such that there exist $I \in \I^{E_x(\I)}$ and $I \in \J^{E_y(\J)}$ for which $c(J) - c(I) < s$.

\begin{mlemmarep}\label{lem:tmd_opt_of_opts}
    %Let $\I = \set{I_1,\ldots,I_n}$ and $\J = \set{J_1,\ldots,J_m}$ be two optimally equispaceable collections of intervals and let $x_1,x_2\in \argmin_{x\in\mathbb{R}} E(\I,x)$, $x_1\le x_2$, and $y_1,y_2\in \argmin_{x\in\mathbb{R}} E(\J,x)$, $y_1\le y_2$, be the (at most) two breakpoints that minimise $E$ for $\I$ and $\J$, respectively. 
    Given that $\I \cup \J = \set{I_1,\ldots,I_n,J_1,\ldots,J_m}$, $\I \cup \J$ is optimally equispaceable if and only if $y_1 \le x_2 + \size{\J}s$.
\end{mlemmarep}
\begin{proof}
    Let $\H = \I \cup \J = \set{I_1,\ldots,I_n,J_1,\ldots,J_m}= \set{I_1,\ldots,I_{n+m}}$ and assume that $\H$ optimally equispaceable.
    In other words, there exists a vector $D = (d_1,\ldots,d_{n+m}) \in \D^{\mathit{opt}}(\H)$ such that $c(I^D_{i+1}) - c(I^D_{i}) = s$ for $1\le i\le n+m-1$.
    Suppose first that $y_1 > x_2 + \size{\J} s$.
    We show that this assumption contradicts the optimality of the vector $D$.
    First, assume that $c(I^D_{n+m}) \ge y_1$.
    Then, it follows that $x_2 < c(I^D_n)$ since $c(I^D_{n+m}) - c(I^D_n) = \size{\J}s$.
    Moreover, $\sum_{i=1}^n |d_i| = E(\I,c(I^D_{n}))$ holds.
    By \Cref{lem:tmd_convex}, $E(\I,x_2) < E(\I,c(I^D_{n}))$ also holds, which contradicts the optimality of the vector $D$.
    Moreover, $c(I^D_{n+1})-x_2 >s$ holds, and hence the inequality $c(I^D_{i+1}) - c(I^D_{i}) = s$ fails when $i=n$.
    Suppose instead that $c(I^D_{n+m}) < y_1$.
    In this case, \Cref{lem:tmd_convex} tells us that $\sum_{i=n}^{n+m} |d_i| = E(\J,c(I^D_{n+m})) > E(\J,y_1)$, which contradicts the optimality of $D$.
    Moreover, if $\delta = c(I^D_{n+m})-y_1$ is the value with which $\J$ is moved from $c(I^D_{n+m})$ to $y_1$, then $(c(I_{n+1}) + \delta) - c(I^D_n) > s$ holds.
    Consequently, $c(I^D_{i+1}) - c(I^D_{i}) = s$ fails when $i = n$.
    In both cases, the initial assumption that $D$ is an optimal solution is contradicted.
    Therefore, $y_1 \le x_2 + \size{\J}s$ must hold.

    In the other direction, assume that $y_1 \le x_2 + \size{\J}s$.
    We show that $\I\cup \J$ is optimally equispaceable.
    That is, we prove the existence of a vector $D = (d_1,\ldots,d_{n+m}) \in \D^{\mathit{opt}}(\I\cup \J)$ such that $c(I^D_{i+1}) - c(I^D_{i}) = s$ for $1\le i\le n+m-1$.
    Let $D' = (d'_1,\ldots,d'_{n+m}) \in \D^{\mathit{opt}}(\I\cup \J)$ be a vector for which the inequality $c(I^D_{i+1}) - c(I^D_{i}) = s$ does not hold for an $1\le i \le n+m-1$.
    We know that $\I$ and $\J$ are optimally equispaceable, thus we can assume that $c(I^{D'}_{i+1}) - c(I^{D'}_{i}) = s$ fails for $D'$ when $i=n$ without loss of generality.
    Let $x = c(I^{D'}_n)$ and $y=c(I^{D'}_{n+m})$.
    We have $x\le x_2$, $y\ge y_1$, and $\sum_{i=1}^{n+m} |d'_i| = E(\I,x) + E(\J,y)$.
    We show that $\sum_{i=1}^{n+m} |d_i| \le \sum_{i=1}^{n+m} |d'_i|$ for all values of $x$ and $y$.
    
    First, suppose that $x = x_2$.
    We have that $y_1 \le x+\size{\J}s$ and $x+\size{\J}s \le y$, otherwise $I^{D'}_n$ intersects with $I^{D'}_{n+1}$.
    Moreover, $x+\size{\J}s < y$, otherwise $c(I^{D'}_{i+1}) - c(I^{D'}_{i}) = s$ holds when $i=n$.
    Hence $E(\J,x+\size{\J}s) \le E(\J,y)$ holds by \Cref{lem:tmd_convex}.
    If we set $(d_n,\ldots,d_{n+m}) = E_{x+\size{\J}s}(\J)$ and $(d_1,\ldots,d_n) = (d'_1,\ldots,d'_n)$, then $c(I^D_{i+1}) - c(I^D_{i}) = s$ and $\sum_{i=1}^{n+m} |d_i| \le \sum_{i=1}^{n+m} |d'_i|$ hold.

    Suppose now that $y = y_1$.
    Analogously, $x< y - \size{\J}s$, otherwise $I^{D'}_n$ intersects with $I^{D'}_{n+1}$ or $c(I^{D'}_{i+1}) - c(I^{D'}_{i}) = s$ holds when $i=n$.
    Again, $E(\J,y-\size{\J}s) \le E(\J,x)$ holds by \Cref{lem:tmd_convex}.
    If we set $(d_1,\ldots,d_n) = E_{y-\size{\J}s}(\I)$  and $(d_n,\ldots,d_{n+m}) = (d'_n,\ldots,d'_{n+m})$, then $c(I^D_{i+1}) - c(I^D_{i}) = s$ and $\sum_{i=1}^{n+m} |d_i| \le \sum_{i=1}^{n+m} |d'_i|$ hold.

    Lastly, suppose that $x<x_2$ and $y>y_1$.
    Let $x',y'$ be two arbitrary points such that $x\le x' \le x_2$, $y_1 \le y' \le y$ and $y' = x'+\size{\J}s$.
    These points exist since $y > x+\size{\J}s$ and $y_1 \le x_2+\size{\J}s$.
    If we set $(d_1,\ldots,d_n) = E_{x'}(\I)$ and $(d_n,\ldots,d_{n+m}) = E_{y'}(\J)$, then $\sum_{i=1}^{n+m} |d_i| \le \sum_{i=1}^{n+m} |d'_i|$ holds by \Cref{lem:tmd_convex}.
    Moreover, $c(I^D_{i+1}) - c(I^D_{i}) = s$ also holds since $y' = x' +\size{\J}s$. %by the definition of $x'$ and $y'$.
    
    In all cases, we obtained a vector $D = (d_1,\ldots,d_{n+m})$ in $\D^{\mathit{opt}}(\I)$ such that $c(I^D_{i+1}) - c(I^D_{i}) = s$ for $1\le i\le n+m-1$.
    Consequently there exists a $D \in \D^{\mathit{opt}}(\I\cup \J)$ such that $D = E_{x^*}(\I\cup \J)$ and $x^* \in \argmin_{x \in \mathbb{R}} E(\I\cup\J,x)$.
    Therefore, $\I \cup \J$ is optimally equispaceable if and only if $y_1 \le x_2 + \size{\J}s$.
\end{proof}

\Cref{cor:tmd_opt_no_intersect} is directly implied by \Cref{lem:tmd_opt_of_opts}.

\begin{corollary}\label{cor:tmd_opt_no_intersect}
    %Let $\I = \set{I_1,\ldots,I_n}$ and $\J = \set{J_1,\ldots,J_m}$ be two optimally equispaceable collections of intervals and let $x_1,x_2\in \argmin_{x\in\mathbb{R}} E(\I,x)$, $x_1\le x_2$, and $y_1,y_2\in \argmin_{x\in\mathbb{R}} E(\J,x)$, $y_1\le y_2$, be the (at most) two breakpoints that minimise $E$ for $\I$ and $\J$, respectively. 
    If $y_1 > x_2 + \size{\J} s$, then $\I \cup \J$ is not optimally equispaceable.
    Moreover, the minimum total moving distance for dispersing $\I \cup \J$ is equal to $E(\I,x) + E(\J,y)$ for arbitrary $x_1\le x\le x_2$ and $y_1\le y\le y_2$.
\end{corollary}

Given a collection $\I$ of $n$ unit intervals, we note that $\I$ can be partitioned into $m\le n$ subcollections $\I_{a_1,b_1},\ldots,\I_{a_{m},b_{m}}$ such that for all $1\le i \le m$, $c(I_{j+1})-c(I_{j}) \le s$ for $a_i \le j \le b_i -1$.
By \Cref{lem:partition_opt}, each $\I_{a_i,b_i}$ is an optimally equispaceable collection.
We use \Cref{lem:tmd_opt_of_opts} and prove the statement of \Cref{lem:consec_partition_opt}.

\begin{mlemmarep}\label{lem:consec_partition_opt}
    Let $\I = \set{I_1,\ldots,I_n} = \I_{a_1,b_1}\cup\cdots\cup\I_{a_{m},b_{m}}$ be a collection of $n$ unit intervals partitioned as above.
    %Let $\I$ be a collection of $n$ unit intervals partitioned into $m\le n$ subcollections $\I_{a_1,b_1},\ldots,\I_{a_{m},b_{m}}$ such that for all $1\le i \le m$, $c(I_{j+1})-c(I_{j}) \le s$ for $a_i \le j \le b_i -1$.
    %Let $\S= \C_{a_1,b_1},\ldots,\C_{a_{m},b_{m}}$ be a family of $m$ consecutive clusters. 
    If there exist integers $\alpha_1, \ldots, \alpha_k$ such that $\I_{a_{\alpha_i},b_{\alpha_i}}$ and $\I_{a_{\alpha_i + 1},b_{\alpha_i + 1}}$ intersect when equispaced,
    then there exists an optimal solution for dispersing $\I$ that disperses the intervals in a way that $c(I_{j+1})+d_{j+1} - (c(I_j) + d_j) = s$ holds for $1 \le i \le k$ and $a_{\alpha_i} \le j < b_{\alpha_i + 1}$.
\end{mlemmarep}
\begin{proof}
    Consider the subcollections $\I_{a_{\alpha_{i}},b_{\alpha_{i}}}$ and $\I_{a_{\alpha_{i}+1},b_{\alpha_{i}+1}}$.
    These subcollections intersect when equispaced, hence $\I_{a_{\alpha_{i}},b_{\alpha_{i}}} \cup \I_{a_{\alpha_{i}+1},b_{\alpha_{i}+1}}$ is optimally equispaceable by \Cref{lem:tmd_opt_of_opts}.
    %Let $L_{x^*}$ be the vector given by $D(\C_{a_{\alpha_{i}},b_{\alpha_{i}}} \cup \C_{a_{\alpha_{i}+1},b_{\alpha_{i}+1}},x^*)$, where $x^* \in \argmin_{x\in\mathbb{R}} D(\C_{a_{\alpha_{i}},b_{\alpha_{i}}} \cup \C_{a_{\alpha_{i}+1},b_{\alpha_{i}+1}},x)$.
    Let $D^* \in \D^{\mathit{opt}}(\I_{a_{\alpha_{i}},b_{\alpha_{i}}} \cup \I_{a_{\alpha_{i}+1},b_{\alpha_{i}+1}})$.
    By the definition of $E$, we have that $c(I^{D^*}_{j+1}) - c(I^{D^*}_{j}) = s$ holds for $a_{\alpha_i} \le j < b_{\alpha_i + 1}$, which implies that the intervals are dispersed such that $c(I_{j+1})+d_{j+1} - \left(c(I_j) + d_j\right) = s$ holds for $1 \le i \le k$ and $a_{\alpha_i} \le j < b_{\alpha_i + 1}$.
    We only need to show that this dispersal is part of an optimal solution for dispersing $\I$.
    Without loss of generality, suppose that $\I_{a_{\alpha_{i}},b_{\alpha_{i}}} \cup \I_{a_{\alpha_{i}+1},b_{\alpha_{i}+1}}$ and $\I_{a_{\alpha_{i}+2},b_{\alpha_{i}+2}}$ intersect when equispaced. 
    Then, \Cref{lem:tmd_opt_of_opts} ensures that $\I_{a_{\alpha_{i}},b_{\alpha_{i}}} \cup \I_{a_{\alpha_{i}+1},b_{\alpha_{i}+1}} \cup \I_{a_{\alpha_{i}+2},b_{\alpha_{i}+2}}$ is optimally equispaceable.
    Applying \Cref{lem:tmd_opt_of_opts} recursively results in $k\le m$ partitions of $\I = \I_1,\ldots,\I_k$, where each $\I_i$ is dispersed to a point $x_i \in \argmin_{x\in \mathbb{R}} E(\I_i,x)$ and there exists no pair $\I_i,\I_{j}$, $i\neq j$, such that $\I_i$ and $\I_j$ intersect when equispaced.
    Moreover, the minimum total moving distance is given by $\sum_{i=1}^k E(\I_i,x_i)$ by \Cref{cor:tmd_opt_no_intersect}.
    In other words, $(x_1,\ldots,x_k)$ describes an optimal solution to disperse $\I$.
    Therefore, there exists an optimal solution to disperse $\I$ that disperses the intervals in a way that $c(I_{j+1})+d_{j+1} - \left(c(I_j) + d_j\right) = s$ holds for $1 \le i \le k$ and $a_{\alpha_i} \le j < b_{\alpha_i + 1}$.
\end{proof}

\paragraph*{Outline of \Cref{alg:dispersing-intervals}} 
%We are now ready to show the outline of the algorithm, illustrated in .
Given a collection of unit intervals $\I$ and a dispersal value $s\ge1$, the algorithm starts by sorting and partitioning $\I$ into $m\le n$ disjoint subcollections $\I_{a_1,b_1},\ldots,\I_{a_{m},b_{m}}$ such that each $\I_{a_{i},b_{i}}$ satisfies \Cref{lem:partition_opt}.
Subsequently, the optimal breakpoints are determined for each $E(\I_{a_{i},b_{i}},x)$.
Whenever there exist two subcollections $\I_{a_{i},b_{j}},\:i\le j$ and $\I_{a_{k},b_{\ell}},\:k\le \ell$ that intersect when equispaced, the algorithm considers both subcollections as a unique subcollection $\I_{a_i,b_{\ell}} = \I_{a_i,b_{j}} \cup \I_{a_k,b_{\ell}}$ and recursively determines the optimal breakpoints of $E(\I_{a_i,b_{\ell}},x)$ using the breakpoint sets of $E(\I_{a_i,b_{j}},x)$ and $E(\I_{a_k,b_{\ell}},x)$.
%
%It is ensured by \Cref{lem:consec_partition_opt} that this recursion gives a partition of $\I$ where each pair of subcollections does not intersect when equispaced.
%
\Cref{lem:consec_partition_opt} ensures that this recursion partitions $\I$ into non-intersecting subcollections when equispaced.
%
Lastly, the algorithm returns the total moving distance, which is calculated as the sum of the optimal values of $E$ for each subcollection.
%The \Cref{fig:alg_edg_outline} illustrates the overview of the algorithm for a collection $\I = \I_{1,2}\cup \I_{3,5}\cup \I_{6,8}$.
%After dispersing each subcollection using $E$, $\I_{1,2}$ and $\I_{3,5}$ intersect when equispaced. Hence, the subcollection $\I_{1,5} = \I_{1,2}\cup \I_{3,5}$ is obtained and the optimal breakpoints of $E(\I_{1,5},x)$ are determined. The optimal breakpoints of $E(\I_{1,5},x)$ and $E(\I_{6,8},x)$ give the solution to disperse $\I$ with minimum total moving distance.

%\begin{figure}[bt]
%    \centering
%    \includegraphics[scale=1,page=2]{tex/Dispersal/fig/dispersal.pdf}
%    \caption{Overview of the algorithm: (a) The given collection $I$ is partitioned into three sets $\I = \I_{1,2}$, $ \I_{3,5}$ and $\I_{6,8}$ according to \Cref{lem:partition_opt}; (b) The movement is represented by moving each subcollection to its optimal breakpoint using $E$. The subcollections $\I_{1,2}$ and $\I_{3,5}$ intersect when equispaced; (c) The subcollections $\I_{1,2}$ and $\I_{3,5}$ are merged and then the optimal breakpoints of $E(\I_{1,5},x)$ are determined. The resulting collection is dispersed with minimum total moving distance.}
%    \label{fig:alg_edg_outline}
%\end{figure}

Before showing the complexity of the algorithm, we must characterise the set of breakpoints further.
%Given a collection of unit intervals $\I =\set{I_1,\ldots,I_n}$, 
%observe that the set of breakpoints of $E(\I,x)$ is equal to $B_\I = \set{c(I_i)+(n-i)s\mid I_i\in \I,\:1\le i \le n}$.
When a collection of unit intervals $\I =\set{I_1,\ldots,I_n}$ is partitioned into $m$ disjoint subcollections $\I_{a_1,b_1},\ldots,\I_{a_{m},b_{m}}$ of intervals that satisfy \Cref{lem:partition_opt}, %the set of breakpoints $B_{\I}$ can be reformulated as
%\begin{align*}
%B_\I = \set{& c(I_1)+(\size{\I_{a_1,b_1}}+\cdots+\size{\I_{a_m,b_m}}-1)s,\ldots,\\
%& c(I_{a_i})+(\size{\I_{a_{i},b_{i}}}+\cdots+\size{\I_{a_m,b_m}}-1)s,\ldots,c(I_n)}.
%\end{align*}
%On the other hand, 
the set of breakpoints $B_{\I_{a_i,b_i}}$ is equal to $\set{c(I_j)+(\size{\I_{a_i,b_i}}-j)s\mid I_i\in \I,\:a_i\le j \le b_i}$ for each $1\le i \le m$.
%\begin{align*}
%B_{\I_{a_i,b_i}}& = \set{c(I_{a_i})+(\size{\I_{a_i,b_i}}-1)s,c(I_{a_{i+1}})+(\size{\I_{a_i,b_i}}-2)s,\ldots,c(I_{b_i})}\\
%&= \set{c(I_j)+(\size{\I_{a_i,b_i}}-j)s\mid I_i\in \I,\:a_i\le j \le b_i},  
%\end{align*}
%for each $1\le i \le m$.
Consequently, $B_\I$ can be reformulated as follows:
%\[
\begin{gather*}
    B_\I = \left\{c(I_j) + \left(\size{\I_{a_i,b_i}}-j + \sum_{k=i+1}^m \size{\I_{a_k,b_k}}\right)s\mid 1\le i \le m,\: a_i\le j \le b_i\right\}.
\end{gather*}
%\]
%By the above, if the index $i$ of the subcollection of an interval $I\in \I_i$ used to calculate a breakpoint $b$ of $B_\I$ is known, the value of the breakpoint $b'$ of $I$ in $B_{\I_{a_i,b_i}}$ can be directly calculated from $B_\I$. 
As a result, if $b$ and $b'$ are the breakpoints for $I$ in $B_{\I_{a_i,b_i}}$ and $B_\I$, respectively, then $b' = b -\sum_{j=i+1}^m \size{\I_{a_j,b_j}}$ holds. 
Moreover, the breakpoints of any union of subcollections $\I_{a_i,b_j} = \I_{a_i,b_i}\cup\cdots\cup\I_{a_j,b_j}$ can be calculated in the same way by subtracting $\sum_{k=j+1}^m \size{\I_{a_k,b_k}}$ from any breakpoint $b\in B_\I$ calculated using an interval $I \in \I_{a_i,b_j}$.
It follows that the order of $B_{\I_{a_i,b_j}}$ is the same as the order of the corresponding breakpoints in $B_\I$.

The above implies that the breakpoints of any (union of) subcollection(s) can be obtained from $B_\I$.
We denote the set $\bigcup_{i\le k \le j}\left\{b+s\sum_{l=k+1}^m \size{\I_{a_l,b_l}}\mid b \in B_{\I_{a_k,b_k}}\right\}$ by $B^*_{\I_{a_i,b_j}}$ and call it the \emph{cumulative set of breakpoints of $B_{\I_{a_i,b_j}}$}.
%In other words, $B^*_{\I_{a_i,b_j}}$ is simply the subset of elements of $B_\I$ that correspond to elements of $B_{\I_{a_i,b_j}}$.
We prove that $B^*_{\I_{a_1,b_1}},\ldots,B^*_{\I_{a_m,b_m}}$ can be found in $O(n\log n)$ time.

%\setlength{\intextsep}{1\baselineskip}
%\DontPrintSemicolon
\begin{algorithm}[tb]
    \caption{Dispersing $n$ unit intervals in $O(n\log n)$ time.}
    \label{alg:dispersing-intervals}
    \Procedure{\rm{DispersingIntervals}($\I$,$s$)}{
        Sort and partition $\I$ into $m\le n$ subcollections $\I_{a_1,b_1},\ldots,\I_{a_{m},b_{m}}$ such that for all $1\le i \le m$, $c(I_{j+1})-c(I_{j}) \le s$ for $a_i \le j \le b_i -1$.\;\nllabel{alg:disp-sort}
        Compute and sort $B^*_{\I_{a_i,b_i}}$ for all $1\le i \le n$.\;\nllabel{alg:breakpoint_sort}
        $x^1_{a_i,b_{i}},x^2_{a_i,b_{i}}$ are the breakpoint $b^{\I_{a_i,b_{i}}}_{(n+1)/2}$ if $\size{\I_{a_i,b_{i}}}$ is odd and $b^{\I_{a_i,b_{i}}}_{n/2}$ and $b^{\I_{a_i,b_{i}}}_{(n/2)+1}$ otherwise.\;\nllabel{alg:breakpoint_1}
        $D^{\mathit{opt}} \gets \bigcup_{1\le i \le n} \set{(B^*_{\I_{a_i,b_i}},x^1_{a_i,b_{i}},x^2_{a_i,b_{i}})}$\;
        \While{$x^1_{a_{k},b_{\ell}}\le x^2_{a_{i},b_{j}}+\size{\I_{a_{k},b_{\ell}}}s$, $1\le i\le j < k\le \ell \le n$}{\nllabel{alg:disp-while}
            %$\I_{a_i,b_{\ell}} \gets \I_{a_i,b_{j}} \cup \I_{a_k,b_{\ell}}$\;
            $B^*_{\I_{a_i,b_l}} \gets \mathit{merge}(B^*_{\I_{a_i,b_j}},B^*_{\I_{a_k,b_\ell}})$\;\nllabel{alg:merge}
            $x^1_{a_i,b_{\ell}}, x^2_{a_i,b_{\ell}} \gets b^{\I_{a_i,b_{\ell}}}_{(n+1)/2}$ if $\size{B^*_{\I_{a_i,b_{\ell}}}}$ is odd and $x^1_{a_i,b_{\ell}} \gets b^{\I_{a_i,b_{\ell}}}_{n/2}$,  $x^2_{a_i,b_{\ell}} \gets b^{\I_{a_i,b_{\ell}}}_{(n/2)+1}$ otherwise.\;
            $D^{\mathit{opt}} \gets \left(D^{\mathit{opt}} \setminus \left\{(B^*_{\I_{a_i,b_j}},x^1_{a_i,b_{j}},x^2_{a_i,b_{j}}),(B^*_{\I_{a_k,b_l}},x^1_{a_k,b_{\ell}},x^2_{a_k,b_{\ell}})\right\}\right) \cup \left\{(B^*_{\I_{a_i,b_\ell}},x^1_{a_i,b_{\ell}},x^2_{a_i,b_{\ell}})\right\}$\;\nllabel{alg:add_delete}
        }
        \Return $\sum_{(B^*_{\I_{a_i,b_j}},x_1,x_2) \in D^{\mathit{opt}}} E(\I_{a_i,b_i}\cup\cdots\cup \I_{a_j,b_j},x_1)$\;\nllabel{alg:disp_tmd_calc}
    }
\end{algorithm}

\begin{mlemmarep}\label{lem:breakpoint_sort}
    Let $\I = \set{I_1,\ldots,I_n} = \I_{a_1,b_1}\cup\cdots\cup\I_{a_{m},b_{m}}$ be a collection of $n$ unit intervals partitioned as above. %If $\I$ is 
    %sorted by centres of intervals and 
    %partitioned into $m\le n$ subcollections $\I_{a_1,b_1},\ldots,\I_{a_{m},b_{m}}$ such that for all $1\le i \le m$, $c(I_{j+1})-c(I_{j}) \le s$ for $a_i \le j \le b_i -1$, 
    Then the cumulative sets of breakpoints $\B^*_{\I_{a_1,b_1}},\ldots,\B^*_{\I_{a_m,b_m}}$ such that each $B^*_{\I_{a_i,b_i}}$ is sorted can be obtained in $O(n\log n)$ total time.
\end{mlemmarep}
\begin{proof}
    Let $n_i$ be the size $\size{\I_{a_i,b_i}}$ and $n_{i,j}$ be the cumulative sum of sizes $n_i+\cdots+n_j$, for $i\le j$.
    We observe that $n_1+\cdots+n_m = n$ since the given partitions are disjoint.
    We first determine $n_{i,m}$ for each $2\le i \le m$ in $O(m)$ time.
    For each $1\le i \le m$, we compute $\B^*_{\I_{a_i,b_i}} = \set{c(I_j)+(n_i-j + n_{i+1,m})s\mid a_i\le j \le b_i}$ in $O(n_i)$ time.
    The total running time of this procedure is $O(n_1+\cdots+n_m) = O(n)$ time.
    We then sort $B^*_{\I_{a_i,b_i}}$ for each $1\le i \le m$ in $O(n_i \log n_i)$ time. The total running time $T(n_1,\ldots,n_m)$ is given as follows:
    \begin{align*}
        T(n_1,\ldots,n_m) & = n_1 \log n_1 + \cdots + n_m\log n_m\le n_1\log{n} + \cdots +n_m\log{n}\\
        &= (n_1+\cdots+n_m)\log{n}= n \log{n}.
    \end{align*}
    Therefore, obtaining sets $\B^*_{\I_{a_1,b_1}},\ldots,\B^*_{\I_{a_m,b_m}}$ such that each $\B^*_{\I_{a_i,b_i}}$ is sorted can be done in $O(n\log n)$ total time.
\end{proof}
\begin{mlemmarep}\label{lem:total_merge_complexity}
    Let $\I = \set{I_1,\ldots,I_n} = \I_{a_1,b_1}\cup\cdots\cup\I_{a_{m},b_{m}}$ be a collection of $n$ unit intervals partitioned as above.
    %Let $\I = \set{I_1,\ldots,I_n}$ be a collection of $n$ unit intervals partitioned into $m\le n$ disjoint subcollections $\I_{a_1,b_1},\ldots,\I_{a_{m},b_{m}}$ such that for all $1\le i \le m$, $c(I_{j+1})-c(I_{j}) \le s$ for $a_i \le j \le b_i -1$.
    If cumulative breakpoint sets $B^*_{\I_{a_1,b_1}},\ldots,B^*_{\I_{a_m,b_m}}$ are given so that each $B^*_{\I_{a_i,b_i}}$ is sorted, then merging them into one sorted set can be done in $O(n\log n)$ total time.
\end{mlemmarep}
\begin{proof}
    We proceed using an unbalanced merge sort approach.
    Given two sorted sets $A$ and $B$ of numbers, it is known that $A$ and $B$ can be merged into one sorted set in $O(\size{A} \log{(\size{B}/\size{A})})$ time, assuming that $\size{A}\le \size{B}$~\cite{Brown1979}.
    We use this algorithm and show that the sets $B^*_{\I_{a_1,b_1}},\ldots,B^*_{\I_{a_m,b_m}}$ can be merged in $O(n \log n)$ total time.
    Let $n_i$ be the size $\size{\I_{a_i,b_i}}$ and $n_{i,j}$, $i\le j$, be the cumulative sum of sizes $n_i+\cdots+n_j$.
    We prove the lemma by induction on $m$.
    Given $m = 1$, no sort is performed since each $B^*_{\I_{a_i,b_i}}$ is already sorted, implying that the time is bounded by $n \log n$.
    Thus, we assume that the lemma holds for $1 < p \le m -1$ sets and prove that it also holds for $p = m$.
    Without loss of generality, assume that the merging of $B^*_{\I_{a_1,b_1}},\ldots,B^*_{\I_{a_m,b_m}}$ is done by merging two already merged sets $B^*_{\I_{a_1,b_i}} = B^*_{\I_{a_1,b_1}}\cup \cdots\cup B^*_{\I_{a_i,b_i}}$ and $B^*_{\I_{a_{i+1},b_m}} = B^*_{\I_{a_{i+1},b_{i+1}}}\cup \cdots \cup B^*_{\I_{a_m,b_m}}$, for an arbitrary $1\le i \le m-1$.
    Thus, we have $\size{B^*_{\I_{a_1,b_i}}},\size{B^*_{\I_{a_{i+1},b_m}}}\le m-1$ and $n = n_{1,i} + n_{i+1,m}$.
    Let $T(a,b)$ denote the number of steps necessary to merge two sets of size $a$ and $b$, $n_{1,i}$ be $ n_{1,j}+n_{j+1,i}$ and $n_{i+1,m}$ be $n_{i+1,k}+n_{k+1,m}$ for arbitrary $1\le j \le i$ and $i+1\le k \le m$.
    Without loss of generality, assume that $n_{1,i} \le n_{i+1,m}$.
    The value of $T(n_{1,i},n_{i+1,m})$ is bounded as follows:
    \begin{align*}
    T(n_{1,i}&,n_{i+1,m}) =\\
    & = T(n_{1,j},n_{j+1,i})+T(n_{i+1,k},n_{k+1,m}) + n_{1,i} \log{(n_{i+1,m}/n_{1,i})}\\
    & \le n_{1,i} \log n_{1,i} + n_{i+1,m} \log n_{i+1,m} + n_{1,i} \log{(n_{i+1,m}/n_{1,i})} & \text{(IH)}\\
    & = n_{1,i} \log n_{1,i} + n_{i+1,m} \log n_{i+1,m} + n_{1,i}\log{n_{i+1,m}} - n_{1,i} \log{n_{1,i}}\\
    & = n_{i+1,m} \log n_{i+1,m} + n_{1,i}\log{n_{i+1,m}}= (n_{1,i} + n_{i+1,m})\log{n_{i+1,m}}\\
    & \le n\log n.
    \end{align*}
    We have proved that if the lemma is true for $p \le m -1$ sets, then the lemma is also true for $p =m$.
    This concludes the induction and lemma proof.
\end{proof}
%\ifConf
%\begin{proofsketch}
    %We proceed using an unbalanced merge sort approach.
%    Given two sorted sets $A$ and $B$ of numbers, it is known that $A$ and $B$ can be merged into one sorted set in $O(\size{A} \log{(\size{B}/\size{A})})$ time, assuming that $\size{A}\le \size{B}$~\cite{Brown1979}.
%    We prove that this procedure allows us to merge the $m$ sets $B^*_{\I_{a_1,b_1}},\ldots,B^*_{\I_{a_m,b_m}}$ in total time $O(n\log n)$ time.
%\end{proofsketch}
%\fi

\begin{theorem}\label{thm:interval_dispersion}
    Given a collection of unit intervals $\I$ and a value $s\ge 1$, {\idisp} can be solved in $O(n\log n)$ time.
\end{theorem}
\begin{proof}
    We show the complexity of \Cref{alg:dispersing-intervals}. 
    Line~\ref{alg:disp-sort} can be done in $O(n\log n)$ time for sorting and $O(n)$ time to determine the initial $m$ partitions.
    Similarly, line~\ref{alg:breakpoint_sort} can be done in $O(n\log n)$ time by \Cref{lem:breakpoint_sort}.
    Given that each $\B^*_{\I_{a_i,b_i}}$ is sorted, the $((\size{\I_{a_i,b_i}}+1)/2)$th element (resp. $(\size{\I_{a_i,b_i}}/2)$th and $((\size{\I_{a_i,b_i}}/2)+1)$th element) can be calculated in $O(\log \size{\I_{a_i,b_i}})$ time using binary search on $\B^*_{\I_{a_i,b_i}}$.
    This ensures that line~\ref{alg:breakpoint_1} is done for all $1\le i \le m$ in $O(m\log n)$ total time.
    %The values of \cref{alg:disp-indp} can be calculated in $O(n)$ time by using the breakpoints obtained in the previous step.
    We initialise $D^{\mathit{opt}}$ as a doubly linked list where each node $i$ contains the information of $(B^*_{\I_{a_i,b_i}},x^1_{a_i,b_{i}},x^2_{a_i,b_{i}})$.
    We show the complexity of the loop in line~\ref{alg:disp-while}.
    We merge both $B^*_{\I_{a_i,b_j}}$ and $B^*_{\I_{a_k,b_\ell}}$ to obtain a sorted $B^*_{\I_{a_i,b_\ell}}$.
    Hence, the median value(s) of $B^*_{\I_{a_i,b_\ell}}$ can be calculated in $O(\log n)$ time by binary search.
    At each execution of line~\ref{alg:merge}, two partitions are merged; thus the number of partitions is reduced by one unit at each iteration.
    Initially, there exist $m$ partitions, and hence the loop of line~\ref{alg:disp-while} iterates at most $m-1$ times.
    Moreover, merging $m$ cumulative sets of breakpoints into one sorted set can be done in $O(n\log n)$ time by \Cref{lem:total_merge_complexity}, which implies that any partial merge of these sets is also bounded by $O(n\log n)$.
    Consequently, the total running time of line~\ref{alg:disp-while} is $O(n\log n)$ time.
    Lastly, in line~\ref{alg:add_delete} the two merged sets are deleted and the new one is added. 
    Since $D^{\mathit{opt}}$ is a doubly linked list, this can be done in $O(1)$ time by connecting the previous and next node of $B^*_{\I_{a_i,b_j}}$ and $B^*_{\I_{a_k,b_\ell}}$ to a new node containing $B^*_{\I_{a_i,b_\ell}}$, respectively.
    Once there is no pair of subcollections left to merge, the total moving distance is calculated in $O(n)$ time in line~\ref{alg:disp_tmd_calc} following the definition of cumulative set of breakpoints, which concludes that the total running time of \Cref{alg:dispersing-intervals} is $O(n\log n)$ time.
\end{proof}

%%%%%%%%%%%%%%%%%%%%%%%%%%%%%%%%%%%%%%%%%%%%%%%%%%%%%%%%%%%%%%%%%%%%%%%%%%

\Cref{thm:interval_dispersion} implies the following result for satisfying $\Pi_{\texttt{edgeless}}$ on unit interval graphs when $s=1$.

\begin{mcorollaryrep}
    Given a unit interval graph $(G,\I)$, {\gged} can be solved in $O(n\log n)$ time for satisfying $\Pi_{\texttt{edgeless}}$.
    %Given a collection of unit intervals $\I$, a solution where the total moving distance is minimum can be found in $O(n\log n)$ time for satisfying $\Pi_{\texttt{edgeless}}$.
\end{mcorollaryrep}
\begin{proof}
    %Let $\I = \{I_1,\ldots,I_n\}$ be a unit interval graph. 
    If $G$ is edgeless, then there exists an optimal solution that satisfies $c(I_{i+1}) - c(I_i) \ge 1$ for $1\le i \le n-1$. %by \Cref{lem:edgeless:same}. 
    Dispersing the intervals using $s = 1$ results in intervals separated by a distance of at least one unit. 
    That is, the resulting unit interval graph satisfies the edgeless condition. Therefore, \Cref{thm:interval_dispersion} works for satisfying $\Pi_{\texttt{edgeless}}$ on unit intervals graphs when $s=1$. 
\end{proof}

\subsection{Satisfying \texorpdfstring{$\Pi_{\texttt{acyc}}$}{} and \texorpdfstring{$\overline{\Pi_{k\texttt{-clique}}}$}{} on Unit Interval Graphs}
\label{ssec:acyc_kclique_uig}

This section shows how to use \Cref{alg:dispersing-intervals} for satisfying $\Pi_{\texttt{acyc}}$ and $\overline{\Pi_{k\texttt{-clique}}}$ on unit interval graphs. 
%A unit interval graph $(G,\I)$ is acyclic if the following inequality holds: $c(I_{i+2}) - c(I_i) \ge 1,\: 1\le i \le n-2$.
%Let $\I_{\texttt{odd}} = \{I_i \in \I\mid i\pmod{2} \neq 0 \}$ and $\I_{\texttt{even}} = \{I_i \in \I\mid i\pmod{2} = 0 \}$. It follows that $\I = \I_{\texttt{odd}}\,\cup\,\I_{\texttt{even}}$. The above inequality can be decomposed into the inequalities
%\begin{gather*}
%    $c(I_{i+2}) - c(I_i) \ge 1,\: 1\le i \le n-2,\: i\bmod{2} \neq 0$ and 
%    $c(I_{i+2}) - c(I_i) \ge 1,\: 1\le i \le n-2,\: i\bmod{2} = 0$.
%\end{gather*}
%In other words, $\I$ is contained in $\Pi_{\texttt{acyc}}$ if $\I_{\texttt{odd}}$ and $\I_{\texttt{even}}$ are contained in $\I_{\texttt{edgeless}}$. Consequently, \Cref{alg:dispersing-intervals} can be applied to $\I_{\texttt{odd}}$ and $\I_{\texttt{even}}$ independently for $s = 1$ to obtain optimal solutions for satisfying $\I_{\texttt{edgeless}}$ in the two subcollections, and combining both solutions gives the optimal solution for satisfying $\Pi_{\texttt{acyc}}$ in $\I$.
%\begin{corollary}
%    Given a unit interval graph $(G,\I)$, {\gged} can be solved in $O(n\log n)$ time for satisfying $\Pi_{\texttt{acyc}}$.
    %Given a collection of unit intervals $\I$, a solution where the total moving distance is minimum can be found in $O(n\log n)$ time for satisfying $\Pi_{\texttt{acyc}}$.
%\end{corollary}
%\subsection{Satisfying \texorpdfstring{$\overline{\Pi_{k\texttt{-clique}}}$}{} on Unit Interval Graphs}
We first show how to satisfy $\overline{\Pi_{k\texttt{-clique}}}$. %in $O(n\log n)$ time.


It is shown in~\cite{HonoratoDroguett2024} that given a unit interval graph $(G,\I)$, $G$ does not contain a $k$-clique if and only if $c(I_{i+k-1}) - c(I_i) \ge  1$ for all $1 \le i \le n-k+1$.
%
%We first give the following lemma regarding $k$-clique-free unit interval graphs.
%
%\begin{mlemmarep}
%    Given a unit interval graph $(G,\I)$ such that $\I = \{I_1,\ldots,I_n\}$, a $k$-clique exists in $G$ if and only if $c(I_{i+k-1}) - c(I_i) \le 1$ for $1 \le i \le n-k+1$.
%    \label{lem:kcliqueiff_updt}
%\end{mlemmarep}
%\begin{proof}
%    Suppose that there exists a $k$-clique $K_k$ in $G$. 
%    The clique $K_k$ consists of $k$ vertices, so the $k$ intervals intersect in $\I$. 
%    Let $I$ and $J$ be the leftmost and rightmost intervals that correspond to the vertices of $K_k$, respectively. 
%    Assume also that $I$ and $J$ are the $i$-th interval and the $j$-th interval in $\I$, respectively.
%    The inequality $c(J) - c(I) \le 1$ holds as $I$ and $J$ intersect. 
%    Since $\I$ is a collection of unit intervals, all intervals between $I$ and $J$ also intersect with $I$ and $J$.
%    Therefore, $j = i + k - 1$.
%
%   In the other direction, we assume that $c(I_{i+k-1}) - c(I_i) \le 1$ holds for $1 \le i \le n-k+1$.
%   By the index of both intervals, there exist other $k-2$ intervals between $I_{i}$ and $I_{i+k-1}$. Since $c(I_{i+k-1}) - c(I_i) \le 1$, the distance between all intervals $I_{i},I_{i+1},\ldots,I_{i+k-1}$ is also at most $1$.
%   It follows that this sequence forms a $k$-clique.
%\end{proof}
%
This inequality can be decomposed into $k-1$ inequalities of the following form:
%\begin{gather*}
    for each $0\le r \le k-2$, $c(I_{i+k-1}) - c(I_i) \ge 1$ for all $1\le i \le n-k+1$ such that $i\bmod{k-1} = r$.
%    \dots\\
%    c(I_{i+k-1}) - c(I_i) \ge 1,\: 1\le i \le n-k+1,\: i\bmod{k-1} = k-2
%\end{gather*}
If $\I$ is decomposed into $k-1$ subcollections such that $\I = \bigcup_{1\le i \le k-1} \I_i$, $\I_i = \{I_j \in \I\mid 1\le j \le n,\: j\pmod{k-1}=i\}$, then \Cref{alg:dispersing-intervals} can be applied to each $\I_i$ independently for $s = 1$ to satisfy the above inequalities.
Since unit interval graphs are chordal, $G$ is acyclic if it is triangle-free; i.e. $G$ is contained in $\overline{\Pi_{3\texttt{-clique}}}$.
Consequently $\Pi_{\texttt{acyc}}$ can be satisfied by satisfying $\overline{\Pi_{k\texttt{-clique}}}$ when $k = 3$.
%On the other hand, a unit interval graph $(G,\I)$ is acyclic if the following inequality holds: $c(I_{i+2}) - c(I_i) \ge 1,\: 1\le i \le n-2$.
The above ideas imply \Cref{cor:nokclique}.

\begin{corollary}\label{cor:nokclique}
    Given an interval graph $(G,\I)$, {\gged} can be solved in $O(n\log n)$ time for satisfying $\Pi_{\texttt{acyc}}$ and $\overline{\Pi_{k\texttt{-clique}}}$.
    %Given a collection of unit intervals $\I$, a solution where the total moving distance is minimum can be found in $O(n\log n)$ time for satisfying $\overline{\Pi_{k\texttt{-clique}}}$.
\end{corollary}

\ifFull
An interval graph $G$ is bipartite if $G$ does not contain an odd cycle. Since interval graphs are chordal, any existence of a cycle implies also the existence of an odd cycle. Thus it is sufficient to remove all cycles to obtain a bipartite graph. 
\begin{corollary}
    Given a unit interval graph $(G,\I)$, {\gged} can be solved in $O(n\log n)$ time for satisfying $\Pi_{\texttt{bipar}}$.
    %Given a collection of unit intervals $\I$, a solution where the total moving distance is minimum can be found in $O(n\log n)$ time for satisfying $\Pi_{\texttt{bipar}}$.
\end{corollary}
\fi
%%%%%%%%%%%%%%%%%%%%%%%%%%%%%%%%%%%%%%%%%%%%%%




%%%%%%%%%%%%%%%%%%%%%%%%%%%%%%%%%%%%%%%%%%%%%%%%%%%%%%%%%%%%%%%%%%%%%%%%%%%%%%%%%%%%%%%%%%%%%%%%%
\section{Minimising the Total Moving Distance for Satisfying \texorpdfstring{$\Pi_{\texttt{edgeless}}$}{} on Weighted Interval Graphs is Hard}\label{sec:edg_ig}

In this section we show that {\gged} is strongly \NP-hard on weighted interval graphs for satisfying $\Pi_{\texttt{edgeless}}$. We show a reduction from {\threepartition}~\cite{garey1979}.
\ifConf
{\threepartition} receives as input a set $A$ of $3m$ elements, a bound $B \in \mathbb{Z}^+$ and a size $s(a) \in \mathbb{Z}^+$ such that $B/4 <s(a) < B/2$ and $\sum_{a \in A} s(a) = mB$, and the task is to decide whether $A$ can be partitioned into $m$ disjoint sets $A_1,\ldots,A_m$ such that for $1\le i \le m$, $\size{A_i} = 3$ and $\sum_{a\in A_i} s(a) = B$.
\fi
\ifFull
\begin{itembox}[l]{{\threepartition}~\cite{garey1979}}
    \begin{description}%[itemsep=0pt,align=left,leftmargin=43pt,labelindent=5pt,style=multiline]
        \item[Input:] Set $A$ of $3m$ elements; a bound $B \in \mathbb{Z}^+$; a size $s(a) \in \mathbb{Z}^+$ such that $B/4 <s(a) < B/2$ and $\sum_{a \in A} s(a) = mB$.
        \item[Task:] Decide whether $A$ can be partitioned into $m$ disjoint sets $A_1,\ldots,A_m$ such that for $1\le i \le m$, $\size{A_i} = 3$ and $\sum_{a\in A_i} s(a) = B$.
    \end{description}
\end{itembox}
\fi

Given an instance $(A,B,s)$ of {\threepartition}, we construct a collection of intervals $\I_A$ and show that $A$ can be partitioned if and only if $\Pi_{\texttt{edgeless}}$ can be satisfied on $\I_A$ with at most total moving distance $T$.
Given two intervals $I,I'$ such that $c(I)\le c(I')$, we say that $I$ and $I'$ intersect if $c(I')-c(I) < (\len{I'}+\len{I})/2$.

We show the construction of $\I_A$ (see \Cref{fig:reduction_overview_ig_hard}).
We define $\I_A$ as the collection $\I\cup \I^s \cup \I^b$ where $\I = \{I_1,\ldots,I_{3m}\},\: \I^s = \{I^s_1,\ldots,I^s_{m-1}\},\: \I^b =  \{I_\ell,I_r\}$ and,
\begin{description}%[itemsep=0pt,align=left,leftmargin=25pt,labelindent=5pt,style=multiline]
    \item[(i)] for $1\le i\le 3m$, $I_i$ is an interval such that $\len{I_i} = s(a_i)$ and $c(I_i) = -s(a_i)/2$ (that is, $r(I_i) = 0$),
    \item[(ii)] for $1\le i \le m-1$, $I^s_i$ is an interval such that $\len{I^s_i} = B$ and $c(I^s_i) = (2i-1)B + B/2$ and
    \item[(iii)] $I_\ell$ and $I_r$ are intervals such that $\len{I_\ell} = \len{I_r} = 3Bm^2 + \max_{a\in A}{s(a)}$, $c(I_\ell) = -3Bm^2/2$ and $c(I_r) = (2m-1)B+3Bm^2/2$.
\end{description}
%\begin{toappendix}
%    \Cref{fig:reduction_overview_ig_hard} shows the construction of $\I_A$.
    
    \begin{figure}[!b]
        \centering
        \includegraphics[scale=1,page=1]{media/IG_hard.pdf}
        %\includesvg[width=0.5\textwidth]{media/definitions.svg}
        \caption{Reduction Overview}
        \label{fig:reduction_overview_ig_hard}
    \end{figure}  
    
%\end{toappendix}
For an interval $I\in \I_A$, we define the moving distance function $d_I:\mathbb{R}\rightarrow\mathbb{R}$ as:
\begin{align*}
    d_I(x) = \begin{cases}
        |c(I)-x|,&\quad I \in \I,\\
        12Bm^2|c(I)-x|,& \quad I \in \I^s \cup \I^b.
    \end{cases}
\end{align*}
Given an instance $(A,B,s)$ of {\threepartition}, we show the following properties.
\begin{mlemmarep}\label{lem:cumulative_sum_bound}
    Given an arbitrary partition of $A$ of $m$ disjoint sets $A_1,\ldots,A_m$ such that $A_i =\set{a^i_1,a^i_2,a^i_3}$ for $1\le i \le m$, $\sum_{i=1}^m 6(i-1)B + \sum^m_{i=1} (3a^i_1+2a^i_2+a^i_3) < 3Bm^2$ holds.
\end{mlemmarep}
\begin{proof}
    We first simplify the first sum:
    \begin{align*}
        \sum_{i=1}^m 6(i-1)B & = 6B\left(\sum_{i=1}^m i-1\right) = 6B\left(\sum_{i=1}^m i - \sum_{i=1}^m 1\right)\\
        & = 6B\left(\frac{m(m+1)}{2} - m\right) = 6B\left(\frac{m(m-1)}{2}\right).
    \end{align*}
    We obtain the upper bound using the fact that $s(a)< B/2$ for any $a \in A$:
    \begin{align*} \sum^m_{i=1} (3a^i_1+2a^i_2+a^i_3) <\sum_{i=1}^m 3\frac{B}{2}+2\frac{B}{2}+\frac{B}{2} = \sum_{i=1}^m 6\frac{B}{2} = \sum_{i=1}^m 3B = 3mB.
    \end{align*}
    Now we have that
    \begin{align*}
        \sum_{i=1}^m 6(i-1)B + \sum^m_{i=1} (3a^i_1+2a^i_2+a^i_3) & < 6B\left(\frac{m(m-1)}{2}\right) + 3mB\\
        & = 3Bm^2 -3mB +3mB = 3Bm^2.
    \end{align*}
    Therefore, the lemma statement is true.
\end{proof}
We note that \Cref{lem:cumulative_sum_bound} works for any partition of $A$ as described above, even without the restrictions of the {\threepartition} output.
\begin{mlemmarep}\label{lem:3p_iff_gged_edgeless}
    Given an instance $(A,B,s)$ of {\threepartition}, $A$ can be partitioned into $m$ disjoint sets $A_1,\ldots,A_m$ such that for $1\le i \le m$ $A_i = \set{a^i_1,a^i_2,a^i_3}$, $\size{A_i} = 3$ and $\sum_{a\in A_i} s(a) = B$ if and only if $\Pi_{\texttt{edgeless}}$ can be satisfied on $\I_A$ with total moving distance of at most $3Bm^2$.
\end{mlemmarep}
\begin{proof}
    Assume that $A$ can be partitioned into $m$ disjoint sets $A_1,\ldots,A_m$ such that for $1\le i \le m$, $\size{A_i} = 3$ and $\sum_{a\in A_i} s(a) = B$.
    Let $D = (d_1,\ldots,d_{3m}),\:D^s = (d^s_{1},\ldots, d^s_{m-1}),\:D^b = (d^b_\ell, d^b_r)$ be vectors that describe the moving distances of $\I$, $\I^s$ and $\I^b$ for satisfying $\Pi_{\texttt{edgeless}}$, respectively.
    We show that $D,D^s,D^b$ exist such that $\sum_{d\in D\cup D^s \cup D^b} |d| \le 3Bm^2$.
    
    Without loss of generality, we assume that the first three elements of $\I$ correspond to $A_1$, the next three elements to $A_2$, and so forth.
    We start by setting $D^s = 0$ and $D^b = 0$.
    We set $d_1 = a^1_1 $, $d_2 = a^1_1 + a^1_2 $ and $d_3 = a^1_1 + a^1_2 + a^1_3$.
    The centre of $I^D_1$ is equal to $c(I^D_1) = c(I_1) + d_1 = -a^1_1/2 + a^1_1 = a^1_1/2$.
    Similarly, $c(I^D_2) = a^1_1 + a^1_2/2$ and $c(I^D_3) = a^1_1 + a^1_2 + a^1_3/2$.
    Furthermore, it holds that $c(I^D_{i+1}) - c(I^D_i) = (\len{I^D_{i+1}}+\len{I^D_i})/2$ for $i\in \set{1,2}$, hence $I^D_1,I^D_2,I^D_3$ do not intersect each other.
    %(\len{}+\len{})/2
    Moreover, $c(I^D_1) - c(I_\ell) = (\len{I^D_1}+\len{I_\ell})/2$ and $c(I^s_1) - c(I^D_3) = (\len{I^s_1}+\len{I^D_3})/2$.
    That is, $I_1,I_2,I_3$ were moved to the area of length $B$ between $I_\ell$ and $I^s_1$ without introducing new intersections.
    For $2 \le i \le m$, we set $d_{3i-2} = 2(i-1)B + a^i_1$, $d_{3i-1} = 2(i-1)B + a^i_1 + a^i_2$ and $d_{3i-1} = 2(i-1)B + a^i_1 + a^i_2 + a^i_3$.
    Analogous to $I_1,I_2,I_3$, it is easy to see that $I^D_{3i-2}, I^D_{3i-1}, I^D_{3i}$ are moved to the area of length $B$ between $I^s_{i-1}$ and $I^s_{i}$ without introducing intersections.
    This implies that $D$, $D^s$, and $D^b$ describe moving distances to make $\I_A$ satisfy $\Pi_{\texttt{edgeless}}$.
    For $1\le i \le m$, the total moving distance for moving $I_{3i-2},I_{3i-1},I_{3i}$ as described is given by $d_{3i-2}+d_{3i-1}+d_{3i} = 6B(i-1) + 3a^i_1+2a^i_2+a^i_3$.
    Consequently, $T = \sum D + \sum D^s + \sum D^b = \sum D = \sum_{i=1}^m 6B(i-1) + 3a^i_1+2a^i_2+a^i_3$.
    By \Cref{lem:cumulative_sum_bound}, $T< 3Bm^2$ holds.
    Therefore, $\Pi_{\texttt{edgeless}}$ can be satisfied on $\I_A$ with total moving distance of at most $3Bm^2$.

    In the other direction, assume that $\Pi_{\texttt{edgeless}}$ can be satisfied on $\I_A$ with total moving distance of at most $3Bm^2$.
    We let $D,D^s,D^b$ be the vectors that describe such a solution.
    We show that $A$ can be partitioned into $m$ disjoint sets $A_1,\ldots,A_m$ such that for $1\le i \le m$, $A_i = \set{a^i_1,a^i_2,a^i_3}$, $\size{A_i} = 3$ and $\sum_{a\in A_i} s(a) = B$.
    Observe that if an interval $I\in \I$ is moved to a point $x \le \ell(I_\ell)$, then $d_I(x) \ge \len{I_\ell}-\len{I}/2 = 3Bm^2 + \max_{a\in A}{s(a)} - \len{I}/2> 3Bm^2$.
    Thus no interval is moved to the left side of $I_\ell$.
    Analogously, no interval is moved to the right side of $I_r$.
    Moreover, for $I\in \I^b$, $c(I)-1/4\le c(I^{D^b})\le c(I)+1/4$ holds, otherwise $d_I(x) > 3Bm^2$ by the definition of the moving distance function.
    This argument also holds for any $I \in \I^s$, since the moving distance is the same.
    Hence the intervals in $\I^D$ must be between $r(I_\ell)$ and $\ell(I_r)$.
    
    There exist $m$ areas of length $B$ between $r(I_{\ell})$ and $\ell(I_r)$ divided by the $m-1$ intervals of $\I^s$. 
    By definition of $\I_A = \I \cup \I^s\cup \I^b$, $\sum_{I\in \I } \len{I} = mB$ and $\size{\I} = 3m$. 
    For any subcollection of intervals $\I' \subseteq \I$ such that $\size{\I'} \ge 4$, the constraint $B/4 < s(a) < B/2$ ensures that $\sum_{I \in \I'} \len{I} > B$ holds.
    Consequently, $\I$ must admit a partition into $m$ subcollections $\I_1,\dots,\I_m$ of three intervals such that $\sum_{I \in \I_i} \len{I} = B$ for each $i$.
    Otherwise the intervals do not fit into the $m$ areas divided intervals of $\I^s$.
    On the other hand, it holds that the length of an area is in the range of $B\pm 1/2$ since $c(I)-1/4\le c(I^{D^b})\le c(I)+1/4$ holds for $I \in \I^s\cup \I^b$.
    Given that $s(a) \in \mathbb{Z}^+$ for all $a \in A$, this ensures that regardless of the movement of intervals in $\I^s$, the three intervals in each area must satisfy $\sum_{I \in \I_i} \len{I} = B$ for each $i$.
    Without loss of generality, assume that $\I$ is moved such that $c(I^D_{i+1}) \ge c(I^D_i)$ for $1\le i \le 3m-1$ and that $\len{I^D_i} = s(a'_i)$ for $a'_i \in A$.
    The $m$ disjoint subcollections $\set{I_{3i-2},I_{3i-1},I_{3i}}$, $1\le i \le m$, satisfy $\len{I_{3i-2}}+\len{I_{3i-1}}+\len{I_{3i}} = B$. That is, this partition gives $m$ disjoint subsets of the form $A_i = \set{a'_{3i-2},a'_{3i-1},a'_{3i}}$ such that $\sum_{a\in A_i} s(a)  = B$.
    Therefore, $A$ can be partitioned into $m$ disjoint sets $A_1,\ldots,A_m$ such that for $1\le i \le m$ $A_i = \set{a^i_1,a^i_2,a^i_3}$, $\size{A_i} = 3$ and $\sum_{a\in A_i} s(a) = B$.
\end{proof}
Lastly, we remark that the polynomial construction of $\I_A$ is straightforward by iterating over $A$ and following the definitions given at the beginning of the section. We summarise the main result of this section as follows:
\begin{theorem}\label{thm:ig_nphard_edgeless}
    {\gged} is strongly \NP-hard on weighted interval graphs for satisfying $\Pi_{\texttt{edgeless}}$.
\end{theorem}

\ifFull
Let $\Pi_{k\texttt{-deg}}$ be the class of graphs with maximum degree $k$. 
%We have $\Pi_{\texttt{edgeless}} = \Pi_{0-\texttt{deg}}$ and $\Pi_{\texttt{acyc}} = \Pi_{1-\texttt{deg}}$.
We slightly modify the reduction of \Cref{thm:ig_nphard_edgeless} and show that {\gged} is also strongly \NP-hard for satisfying $\Pi_{k-\texttt{deg}}$.

\begin{mtheoremrep}\label{thm:ig_nphard_kdeg}
    {\gged} is strongly \NP-hard on weighted interval graphs for satisfying $\Pi_{k-\texttt{deg}}$.
\end{mtheoremrep}
\begin{proof}
    We extend the reduction from {\threepartition} of \Cref{thm:ig_nphard_edgeless}.
    Given an instance $(A,B,s)$ of {\threepartition} and $\I_A = \I \cup \I^s \cup \I^b$ as defined above, we construct an instance $\J_A = \J \cup \J^s \cup \J^b \cup \J^f$ such that (i) $\J = \I$, (ii) for each $I^s_i \in \I^s$, $\J$ contains $k+1$ intervals $J_1,\ldots,J_{k+1}$ such that $\len{J_j} = \len{I^s_i}$ and $c(J_j) = c(I^s_i)$ for $1\le j \le k+1$, and (iii) $\J^b = \set{J_\ell^1,\ldots,J_\ell^{k+1}}\cup \set{J_r^1,\ldots,J_r^{k+1}}$ such that $\len{J_\ell^j} = \len{I_\ell}$ and $c(J^j_\ell) = c(I_\ell)$ for $1\le j \le k+1$, and $\len{J_r^j} = \len{I_r}$ and $c(J^j_r) = c(I_r)$ for $1\le j \le k+1$.
    We define the moving distance function analogously for $\J$ and $\J^s \cup \J^b$.
    Let $c_1,\ldots,c_m = (\ell(I^s_1)-r(I_\ell))/2,(\ell(I^s_2)-r(I^s_1))/2,\ldots, (\ell(I^s_{m-1})-r(I^s_{m-2}))/2,(\ell(I_r)-r(I^s_{m-1}))/2$.
    For each $1\le i \le m$, the subcollection $\J^f$ contains $k$ intervals $J_1,\ldots,J_k$ such that $c(J_j) = c_i$, $\len{J_j} = B$ and $d_{J_j}(x) = 12Bm^2|c(J_j)-x|$ for all $1\le j \le k$.
    %Duplicate $\I^s$ and $\I^b$ $k+1$ times in $\I_A$ and add $k$ intervals of size $B$ to each area of size $B$. 
    The resulting collection consists of a $(k+1)$-clique for each copy of an interval $I \in \I^s\cup \I^b$ and $k$-cliques of intervals centred at $c_1,\ldots,c_m$ in the areas of size $B$.
    We set the maximum total moving distance to $T = 3Bm^2$ as above.
    Analogously to $\I_A$, the intervals in $\J$ must be placed in the areas of the $k$-cliques; otherwise, a $(k+2)$-clique is formed by intersecting with the intervals in $\J^s \cup \J^b$ or the total moving distance is greater than $3Bm^2$.
    This implies that the proof of \Cref{lem:3p_iff_gged_edgeless} also shows that $\Pi_{k-\texttt{deg}}$ can be satisfied on $\J_a$ with total moving distance of at most $3Bm^2$ if and only if $(A,B,s)$ is a yes-instance of {\threepartition}.
    Lastly, $\J_A$ is constructed in polynomial time by iterating $A$ since $k$ is bounded by $n$.
    Therefore, the theorem statement is true.
\end{proof}

We notice that \Cref{thm:ig_nphard_kdeg} can be used to show that the cases for properties $\Pi_{\texttt{acyc}}$ and $\overline{\Pi_{k\texttt{-clique}}}$ are also strongly \NP-hard.
In particular, for $\Pi_{1\texttt{-deg}}$, the intervals $I,J^1_\ell,J^2_\ell$ form a cycle in $\J_A$ for any $I \in \J$. Consequently, moving the intervals of $\J$ with total moving distance of at most $3Bm^2$ is equivalent to removing all cycles from $\J_A$ with total moving distance of at most $3Bm^2$.
Similarly, for any $I\in \J$, the intervals $I,J^1_\ell,\ldots,J^k_\ell$ form a $k$-clique in $\I_A$ for satisfying $\Pi_{(k-1)\texttt{-deg}}$. Consequently, moving the intervals of $\J$ with total moving distance of at most $3Bm^2$ is equivalent to removing all $k$-cliques from $\J_A$ with total moving distance of at most $3Bm^2$.
\fi
\ifConf
We notice that \Cref{thm:ig_nphard_edgeless} can be extended to show that satisfying $\Pi_{\texttt{acyc}}$ and $\overline{\Pi_{k\texttt{-clique}}}$ is also strongly \NP-hard.
%In particular, when satisfying $\Pi_{\texttt{acyc}}$, we make an overlapping copy of intervals in $\I^s\cup\I^b$ and add one interval of size $B$ into the spaces between intervals of  $\I^s\cup\I^b$ with the same moving distance function.
%By doing this, an arbitrary interval of $\I$ forms a cycle with $I_\ell$ and its overlapping copy.
%Consequently, moving the intervals of $\I$ with total moving distance of at most $3Bm^2$ is equivalent to removing all cycles from $\I_A$ with at most the same distance.
In particular, when satisfying $\overline{\Pi_{k\texttt{-clique}}}$, we create $k-1$ overlapping copies of the intervals in $\I^s\cup\I^b$ and add $k-1$ overlapping intervals of size $B$ into the spaces between intervals of $\I^s\cup\I^b$ with the same moving distance function.
Any interval forms a $k$-clique with the $k$ copies of overlapping intervals.
Consequently, moving the intervals of $\I$ with total moving distance of at most $3Bm^2$ is equivalent to removing all $k$-cliques from $\I_A$ with at most the same distance.
Moreover, by the chordality of interval graphs, it is sufficient to satisfy $\overline{\Pi_{k\texttt{-clique}}}$ when $k =3$ to satisfy $\Pi_{\texttt{acyc}}$.
\fi
%\warning{Add explanation of the corollaries.} 
As a result, \Cref{cor:ig_nphard_acyc_and_nokclique} is obtained.

\begin{corollary}\label{cor:ig_nphard_acyc_and_nokclique}
    {\gged} is strongly \NP-hard on weighted interval graphs for satisfying $\Pi_{\texttt{acyc}}$ and $\overline{\Pi_{k\texttt{-clique}}}$.
\end{corollary}
%\begin{corollary}
%    {\gged} is strongly \NP-hard on weighted interval graphs for satisfying $\overline{\Pi_{k\texttt{-clique}}}$.
%\end{corollary}

\section{Minimising the Maximum Moving Distance for \texorpdfstring{$\Pi_{\texttt{edgeless}}$}{} on Unit Disk Graphs is Hard}\label{sec:disk_edgeless}

%%%%
    \newcommand{\bdisk}[1]{B\langle #1\rangle }
    \newcommand{\ldisk}[1]{L\langle #1\rangle }
    \newcommand{\hdisk}[1]{H\langle #1\rangle }
    \newcommand{\cstate}{\D_\Phi}
    \newcommand{\istate}{\D^{(i)}_\Phi}
    \newcommand{\mstate}{\D^{(m)}_\Phi}
    \newcommand{\mstatefinal}[1]{(#1)^{(\mathit{moved})}}
    %\newcommand{\init}[1]{(#1)^{(i)}}
    %\newcommand{\moved}[1]{(#1)^{(m)}}
    \newcommand{\ismoved}[1]{\X(#1)}
    \newcommand{\movedpos}[1]{\X_{\mathit{pos}}(#1)}
%%%%
%
%We reformulate {\gged} as a minimax problem.
%That is, we aim to minimise each object's maximum moving distance, rather than the total moving distance of the collection.
%
In this section, we deal with the minimax version of {\gged}, defined as follows:
%We reformulate {\gged} as a minimax problem, to minimise the maximum moving distance of each object, rather than the total moving distance of the collection.
%
\begin{itembox}[l]{{\ggedmm}}\label{pro:edg_disk}
    \begin{description}%[itemsep=0pt,align=left,leftmargin=50pt,labelindent=5pt,style=multiline]
        \item[Input:] An intersection graph $(G,\S)$, a graph property $\Pi$ and a real $K>0$.
        \item[Task:] Decide whether $\Pi$ can be satisfied by moving objects such that for all $S\in \S$, the moving distance of $S$ is at most $K$.
    \end{description}
\end{itembox}

We show that {\ggedmm} is strongly {\NP-hard} on unit disk graphs for satisfying $\Pi = \Pi_{\texttt{edgeless}}$ over the $L_1$ and $L_2$ distances by reducing from {\pthreesat}.
Specifically, we show a proof for \Cref{thm:edgeless_np_hard}.
%\begin{theorem}\label{thm:edgeless_np_hard}
%    \Copy{edgeless_np_hard}{{\ggedmm} is strongly {\NP-hard} on unit disk graphs for satisfying $\Pi_{\texttt{edgeless}}$ over the $L_1$ and $L_2$ distances.}
%\end{theorem}

\begin{restatable}{theorem}{edgelessNPHard}\label{thm:edgeless_np_hard}
    {\ggedmm} is strongly {\NP-hard} on unit disk graphs for satisfying $\Pi_{\texttt{edgeless}}$ over the $L_1$ and $L_2$ distances.
\end{restatable}

\ifConf
Due to space constraints, we only give an overview of the reduction. The complete reduction and proofs can be found in the full-version of the paper~[].
\fi
\subsection[Proof Overview of \texorpdfstring{\Cref{thm:edgeless_np_hard}}{}]{Proof Overview of \texorpdfstring{\Cref{thm:edgeless_np_hard}}{}: Reducing {\pthreesat} to {\ggedmm}}
%We start by defining the problem used in the reduction.
\ifConf
    We show a reduction from the following \NP-complete variation of {\pthreesat}~\cite{Lichtenstein1982,Knuth1992,Tovey1984}.
    Given CNF formula $\Phi$ equipped with a planar rectilinear embedding $G_\Phi$, a set $X$ of $n$ variables, a set $C$ of $m$ clauses over $X$ such that each $c \in C$ has length $|c| \le 3$, each variable $x \in X$ appears in at most three clauses, and $\Phi = \bigwedge_{c\in C} c$, {\pthreesat} asks whether $\Phi$ is satisfiable.
    %An instance of {\pthreesat} can always be described using a rectilinear embedding~\cite{Knuth1992}.
%    It is known that this problem is.
\fi
\ifFull
Given a boolean formula $\Phi$ and its planar incidence graph $G_{\Phi}$, {\pthreesat}~\cite{Lichtenstein1982} asks whether $\Phi$ is satisfiable. 
An instance of {\pthreesat} can always be described using a rectilinear embedding \cite{Knuth1992}.
Moreover, this problem is \NP-complete even if the appearance of variables in clauses is restricted to at most three \cite{Tovey1984}.
We use these restrictions in the reduction and define the problem as follows.
\begin{itembox}[l]{{\pthreesat}}
    \begin{description}%[itemsep=0pt,align=left,leftmargin=50pt,labelindent=5pt,style=multiline]
        \item[Input:] A CNF formula $\Phi$ equipped with a planar rectilinear embedding $G_\Phi$. Set $X$ of $n$ variables, set $C$ of $m$ clauses over $X$ such that each $c \in C$ has length $|c| \le 3$, each variable $x \in X$ appears in at most three clauses, and $\Phi = \bigwedge_{c\in C} c$.
        \item[Task:] Decide whether $\Phi$ is satisfiable.
    \end{description}
\end{itembox}
\fi
%%--------------------------------------------------------------
\ifFull
\paragraph*{Reduction Overview} 
\fi
We give a simplified overview of the reduction. The idea is to emulate each component (clauses, variables and connectors) of $G_{\Phi}$ using \emph{disk gadgets} and construct a collection of disks $\D_\Phi$ equivalent to $G_{\Phi}$. 
That is, our objective is to construct a $\D_\Phi$ such that $\Phi$ is satisfiable if and only if $\D_\Phi$ is a yes-instance of {\ggedmm} for $\Pi_{\texttt{edgeless}}$.
%
To do this, we emulate the truth assignment using a proper movement of disks. 
To force the disk movement, we deliberately insert intersecting disks in $\D_\Phi$. 
In particular, we insert intersecting disks in clause gadgets and restrict the movement of such disks to moving a sequence of disks such that a \emph{free slot} of a variable gadget is used.
To allow the removal of the intersection, the gadgets are connected following the structure of $G_\Phi$ using consecutive disks separated by distance $K$.
%
%We replicate the variable truth assignment with the strategic movement of disks. To compel disk movement, we deliberately place intersecting disks in $\D_\Phi$. Specifically, intersecting disks are placed within the clause gadget, and their movement is constrained by leveraging a variable gadget's \emph{free slot} linked to the clause gadget. To enable the resolution of these intersections, the gadgets are connected via connectors structured according to $G_\Phi$.
%
For example, consider the boolean formula $\Phi$ 
%= (x_1 \lor \overline{x_2} \lor x_4) \land (\overline{x_1} \lor x_2 \lor \overline{x_3}) \land (x_2 \lor \overline{x_3} \lor \overline{x_4})$ 
and its rectilinear embedding $G_\Phi$, illustrated in \Cref{fig:reduction_overview_a}. 
\begin{figure}[!bt]
    \centering
    \includegraphics[scale=1,page=32]{media/disk_edgeless.pdf}
    %\includesvg[width=0.5\textwidth]{media/definitions.svg}
    \caption{Reduction Overview: An arbitrary instance $\Phi$ of {\pthreesat} with its rectilinear embedding $G_\Phi$.}
    \label{fig:reduction_overview_a}
\end{figure}
A skeleton of the reduction is shown in \Cref{fig:reduction_overview_bc}(a), where representations of clause and variable gadgets are connected following $G_\Phi$. 
\begin{figure}[!tb]
    \centering
    \includegraphics[scale=1,page=33]{media/disk_edgeless.pdf}
    %\includesvg[width=0.5\textwidth]{media/definitions.svg}
    \caption{Reduction Overview: (a) The skeleton given by the instance $(\Phi, G_\Phi)$ of \Cref{fig:reduction_overview_a}; (b) The intersection of the gadget for $c = (x_1 \lor \overline{x_2} \lor x_4)$ is removed by moving disks in a way that a free slot of the gadget for $x_2$ is used. Since $c = \mathit{true}$ when $x_2 = \mathit{false}$, the free slots for the other two gadgets become blocked, being unable to remove their intersection using the variable gadget for $x_2$.}
    \label{fig:reduction_overview_bc}
\end{figure}
\ifFull
We remark that the gadgets in the figure are solely representations, and we shall show their detailed construction using disks later.
\fi
Let $c = (x_1 \lor \overline{x_2} \lor x_4)$ and suppose that $x_2$ is assigned to $\mathit{false}$.
This assignment implies a movement of disks that (i) removes the intersections in the clause gadget for $c$ and (ii) blocks the truth value of the variable gadget for $x_2$ (see \Cref{fig:reduction_overview_bc}(b)).
%
%Notice that we must block the truth value of the variable gadget to prevent another clause gadget $c'$ from using a free slot available in the variable gadget of $x_2$ when $x_2 = \mathit{true}$.
%
We must block the truth value of the variable gadget so that another clause gadget $c'$ does not use the free slot in the variable gadget for $x_2$ when $x_2 = \mathit{true}$.
%
%In \Cref{fig:reduction_overview}(c), it is shown that once the intersection of clause $c$ is removed using the gadget of $x_2$, the other two intersections cannot be removed using the same gadget.
Consequently, their intersections must be removed using other gadgets.
It can be shown that removing all intersections in this way is equivalent to a valid assignment of variables for which $\Phi = \mathit{true}$. 
\ifFull
\paragraph*{Reduction Overview: Moving disks} 
\fi
The disks are \emph{moved} by assigning a new location, and the distance is calculated using a function that we call \emph{moving distance function}, which is the $L_1$ or $L_2$ distance metric multiplied by a \emph{distance weight}.
We employ two types of disks classified by their distance weight, called \emph{transition disk} and \emph{heavy disk}. 
The transition disks are the disks that we aim to move, whereas heavy disks are used to restrict the movement of transition disks.
%
%The intuitive idea behind the definition of a heavy disk is that whenever a heavy disk is moved by a value large enough to alter the construction, the moving distance is greater than $K$ (hence its name).
%By this, we shall prove that any solution that removes all the intersections of $\D_\Phi$ with minimum maximum moving distance $K$ implies that the solution depends exclusively on how the transition disks were moved.
%
The moving distance function of a heavy disk is intuitively defined such that any significant movement that alters the construction exceeds a distance of $K$. We show that a solution that allows removing all intersections from $\D_\Phi$ with minimum maximum moving distance $K$ exclusively relies on the movement of transition disks.
%
We remark that, although heavy disks can move, their movement is negligible. 
Combining this condition and the above construction, it can be shown that $\Phi$ is satisfiable if and only if $\Pi_{\texttt{edgeless}}$ can be satisfied in $\cstate$ using minimum maximum moving distance $K$.%, which depends exclusively on the movement of the transition disks.
%%-----------------------------------------------------------------------

\ifConf
\begin{toappendix}
\section{Definitions and Proofs of \Cref{sec:disk_edgeless}}
\else
\begin{toappendix}
\fi
%\begin{toappendix}
In the subsequent sections, we formally define the disks and gadgets of the reduction.
The gadgets are based on the gadgets presented in~\cite{Breu1998} and their coordinates are given in \ref{apx:coordinates}.
All coordinates are rational numbers; thus, all centres of disks can be described using a finite number of bits.
We also consider instances for which $K=1$ exclusively.
Before presenting the details in the following subsections, we make some remarks to aid in understanding the reduction.
\begin{itemize}
    \item In the following figures, we omit heavy disks that are properly inserted into the blank spaces to highlight the shape of the gadgets.
    \item When a collection of disks representing a gadget is given, it is sometimes conveniently assumed that omitted heavy disks are contained in the collection.
    \item In the following figures, the distance between consecutive transition disks is highlighted using shaded concentric circles with radius $3$ for the $L_1$ and $L_2$ distances.
\end{itemize}


\subsubsection{General Definitions}

Given an arbitrary instance $\Phi$ of {\pthreesat} with $n$ variables and $m$ clauses, we denote the collection of $\n$ unit disks produced by the reduction as $\cstate$ where $\n = f(n,m)$ is a polynomial of $n$ and $m$.
%%%Preliminaries
%Given a radius $r>0$, a \emph{disk} $D$ centred at $p$ is the set $D = \set{x\in \mathbb{R}^2\mid \lVert x,p \rVert_\alpha \le r}$ for $m\in \set{1,2}$. 
%An \emph{open disk} $D$ is a disk without its boundary circle; that is, $D = \set{x\in \mathbb{R}^2\mid \lVert x,p \rVert_\alpha < r}$ for $m\in \set{1,2}$.
%A \emph{unit disk} is a disk with $r = 1/2$.
%The \emph{minimax centre} $p$ of a set of points $P \subseteq \mathbb{R}^2$ is the centre of the smallest circle that contains $P$, which is the point that minimises $\max_{p'\in P} \lVert p,p'\rVert_\alpha$ for $\alpha \in \set{1,2}$.
%The \emph{diameter} $\diam{P}$ of a set of points $P \subseteq \mathbb{R}^2$ is the distance of the farthest pair of points in $P$.
%Alternatively, the diameter $\diam{S}$ of a convex polygon $\S$ is the diameter of its edges~\cite{Preparata1985}.
%A set $C \subseteq \mathbb{R}^n$ is \emph{convex} if the line segment between two arbitrary points in $C$ is completely in $C$. Such a set is called a \emph{convex set}.
%Given a set of points $P$, the \emph{convex hull} of $P$, $\C(P)$, is the smallest convex set containing $P$.
%%%%
The \emph{moving distance function} of an arbitrary disk $D \in \D_\Phi$ is a function of the form $d_{D}:\mathbb{R}^2 \rightarrow \mathbb{R}$ such that $d_{D}(p) = w_D\lVert c(D),p \rVert_\alpha$ for $\alpha \in \set{1,2}$. The real $w_D >0$ is the \emph{moving weight} of $D$.
We differentiate disks according to the value of $w_D$.
The disk $D$ is called \emph{transition disk} when $w_D = K/3$.
The disk $D$ is a \emph{$k$-heavy disk} when $w_D = 2^kK$, for $k\ge 1$.
We sometimes identify the heavy disk centred at an arbitrary point $p$ as $\hdisk{p}$.
If such a heavy disk does not exist, $\hdisk{p} = \emptyset$.
Given disks $D,D' \in \cstate$, we say that \emph{$D$ is consecutive to $D'$} if $d_D(c(D')) \le K$.
Lastly, we also refer to a transition disk concentric with a heavy disk as \emph{intersection disk}.

The transition and heavy disks used in the reduction are shown in \Cref{fig:md_functions} with their corresponding moving distance functions.

%%%---fig:md_functions
\begin{figure}[!htb]
    \centering
    \includegraphics[scale=1,page=3]{media/disk_edgeless.pdf}
    %\includesvg[width=0.6\textwidth]{media/md_functions.svg}
    \caption{Disks used in the reduction: Transition disk $D$ and $k$-heavy disks, $k \in\set{1,2,6}$, with their corresponding moving distance function.}
    \label{fig:md_functions}
\end{figure}

We formally define the movement of the disks in $\cstate$. 
Let $\X : \cstate \to \set{0,1}$ be an indicator function that tells whether a disk in $\cstate$ was moved. If $D \in \cstate$ is has not been moved, then $\ismoved{D} = 0$, otherwise $\ismoved{D} = 1$. 
Let also $\X_{\mathit{pos}}: \cstate \to \mathbb{R}\times \mathbb{R}$ be a function that returns the position of a disk.
If $\ismoved{D} = 0$, then $\movedpos{D} = c(D)$ for any disk $D \in \cstate$.
Given a disk $D \in \cstate$ such that $\ismoved{D} = 0$ and a point $p \in \mathbb{R}$, we say that \emph{$D$ is moved to $p$} to refer to setting $\ismoved{D} = 1$ and $\movedpos{D} = p$.
We then define $\X_{\mathit{pos}}$ as follows:
\[
    \movedpos{D} = \begin{cases}
        c(D),&\quad \ismoved{D} = 0,\\
        p, &\quad \ismoved{D} = 1.
    \end{cases}
\]
We also define $\istate = \set{D \in \cstate\mid \ismoved{D} = 0}$ and $\mstate = \set{D \in \cstate\mid \ismoved{D} = 1}$ as the subcollections of disks that represent unmoved and moved disks, respectively.
By the definition of movement described above, we see that $\istate$ and $\mstate$ form a partition of $\cstate$. That is, $\cstate = \istate \cup \mstate$.

%%%%%%%%%%%%%%%%%%%%%%%%%%%%%%%%%%%%%%%%%%%%
%We require additional definitions to show the correctness of the movements of transition disks.
%実現可能
Given an arbitrary disk $D \in \cstate$, the \emph{range of movement of $D$}, denoted by $\A_D \subseteq \mathbb{R}^2$, is the set of points where $D$ can be moved with minimum maximum moving distance $K$. That is, 
\[
\A_D = \begin{cases}
        \set{p \in \mathbb{R}^2\mid d_D(p) \le K}, & \quad \ismoved{D} = 0,\\
        \emptyset, & \quad \ismoved{D} = 1.
    \end{cases}
\]
%However, with a slight abuse of notation, we denote $\A_D$ for a disk $D \in \cstate$ to refer to $\A_{\init{D}}$.
A \emph{blocked zone by} $D \in \cstate$, denoted by $\B_D$, is the zone where an arbitrary disk $D'\in \istate$, $D'\neq D$, cannot be moved avoiding intersecting $D$ even after moving $D$ to a point $p\in \A_D$. In particular,
\[
    \B_D  = \begin{cases}
        \emptyset,&\: \text{$D$ is a transition disk and $\ismoved{D} = 0$} ,\\
        \{p \in \mathbb{R}^2\mid \lVert c(D),p\rVert_\alpha < \frac{2^{k}-1}{2^k}\},&\: \text{$D$ is a $k$-heavy disk and $\ismoved{D} = 0$},\\
        \{p \in \mathbb{R}^2\mid \lVert \movedpos{D},p\rVert_\alpha  < 1\},&\: \ismoved{D} = 1.\\
    \end{cases}
\]
%If the moving weight of $D$ is at most $1$, then $\B_D = \emptyset$.
Let $r(D)$ be the radius of $D$.
Equivalently, the blocked zone of a $k$-heavy disk $D \in \istate$ is an open disk centred at $c(D)$ with radius $r(D) =(2^{k}-1)/2^k$.
The blocked zone of a disk $D \in \mstate$ is an open disk centred at $\movedpos{D}$ with radius $r(D) = 1$.

Let $D \in \cstate$ be a disk.
A point $p$ is a \emph{feasible position of movement} or simply \emph{feasible position} of $D$ if $D$ can be moved to $p$ such that $p \in \A_D$ and intersections can be removed with minimum maximum moving distance $K$ after moving $D$.
That is,
%\[
    $p \in \A_D \setminus \cup_{D' \in \left(\cstate\setminus\set{D}\right)}\B_{D'}$.
%\]
A \emph{feasible area of movement} or simply \emph{feasible area} $\F_D$ of $D$ is the union of subsets of $\mathbb{R}^2$ such that for any $p \in \F_D$, $p$ is a feasible position of $D$ (see \Cref{fig:feasible_area}).
In particular, 
\[
        \F_D =  \A_D \setminus \cup_{D' \in \left(\cstate\setminus\set{D}\right)}\B_{D'}\:.
\]


%%%---fig:feasible_area
\begin{figure}[!htb]
    \centering
    \includegraphics[scale=1,page=9]{media/disk_edgeless.pdf}
    %\includesvg[width=0.5\textwidth]{media/definitions.svg}
    \caption{Range of movement and feasible area: (a) illustration of a collection of disks $\D = \set{D,D_1,D_2,D_3} \subseteq \istate$ and the range of movement $\A_D$; (b) the feasible area of $D$, $\F_D$. In particular, $\F_D = \A_D \setminus \{\B_{D_1} \cup \B_{D_2} \cup \B_{D_3}\}$ is the region marked with a bold dotted line. The disk $D$ can be moved to an arbitrary point $p$ contained in $\F_D$. On the other hand, $D$ cannot be moved to the point $q \in \A_D$ without exceeding the minimum moving distance $K$ even if $d_D(q) \le K$ holds.}
    \label{fig:feasible_area}
\end{figure}

%%%---sum_of_notation
\Cref{tab:summary_of_notation} summarises the main notation used throughout \Cref{sec:disk_edgeless}.
\begin{table}[!hbt]
    \centering
    \caption{Summary of Notation}
    \begin{tabular}{C{0.15\textwidth}p{0.79\textwidth}}
        \textbf{Symbol} & \textbf{Explanation} \\\midrule
        \multicolumn{2}{c}{Reduction instance}\\\midrule
        $\cstate$       & Collection of disks constructed by using $\Phi$. \\
        $d_D$ & Moving distance function of the disk $D$.\\
        $W_D$ & Moving weight of the disk $D$.\\
        $\ismoved{D}$ & Indicator function that returns $1$ if the disk $D \in \cstate$ was moved and $0$ otherwise.\\
        $\movedpos{D} $ & Function that returns the position of $D \in \cstate$.\\
        $\istate$ & The disks $D \in \cstate$ for which $\ismoved{D} = 0$ (unmoved disks).\\
        $\mstate$ & The disks $D \in \cstate$ for which $\ismoved{D} = 1$ (moved disks).\\
        $\mstatefinal{\D}$ & The collection of disks in which there exists a disk $D' \in \mstatefinal{D}$, $c(D') = p$ for a disk $D \in \D$ such that $\movedpos{D} = p$, given that for all $D \in \D$, $\ismoved{D} = 1$.\\
        \midrule
        \multicolumn{2}{c}{Cell Gadget, Clause Gadget and Clause Component}\\\midrule
        $\G_{(x,y)}$ & Cell gadget with its transition disk centred at $(x,y)$.\\
        $(\G)^{(\mathit{heavy})}$ & The subcollection of the heavy disks contained in $\G$.\\
        $\H_{(x,y)},\H_{c(D)}$ & Interior hole of the cell gadget $\G_{(x,y)}$ with transition disk $D$.\\
        $\G_c$ & Clause gadget of the clause $c$.\\
        $T_c$ & Intersection disk of $\G_c$.\\
        $\G^c_{i,j,k}$ & Clause component composed of clause gadget $\G_c$ and variable gadgets $\G_{x_i}$, $\G_{x_j}$ and $\G_{x_k}$.\\
        \midrule
        \multicolumn{2}{c}{Variable Gadget}\\\midrule
        $\G_x$ & Variable gadget of the variable $x$.\\
        $S_x$ & Truth setter disk of $\G_x$.\\
        $\S^x_{t,i},s^x_{t,i}$ & Truth slot of the true ($i = 1$) and false side ($i = 2$) of $\G_x$ and its centre.\\
        $\bdisk{\S^x_{t,i}}$ &  Blocking disk of the true ($i = 1$) and false side ($i = 2$).\\
        $\S^x_{c,i}, s^x_{c,i}$ & Free space to position the disk coming from the clause gadget of $c_i$ and its centre.\\
        $D_{c_i}$ & Transition disk coming from clause gadget $c_i$.\\
        $\bdisk{\S^x_{c,i}}$ & Blocking disk for $c_i$.\\
        $\ldisk{\S^x_{c,i}}$ & Link disk for $c_i$.\\
        \midrule
        \multicolumn{2}{c}{General}\\\midrule
        $r(D)$ & Radius of disk $D \subseteq \mathbb{R}^2$.\\
        %$H^k$ & Type of heavy disk for which the moving weight is $2^kK$.\\
        $\hdisk{p}$ & Heavy disk centred at $p \in \mathbb{R}^2$.\\
        $\A_D$ & Range of movement of disk $D \in \cstate$.\\
        $\B_D$ & Blocked zone by disk $D \in \cstate$.\\
        $\F_D$ & Feasible area of disk $D \in \cstate$.\\
        \bottomrule
    \end{tabular}
    \label{tab:summary_of_notation}
\end{table}

%%%%%%%%%%%%%%%%%%%%%%%%%%%%%%%%%%%%%%%%%%%%

Given a subcollection $\D \subseteq \cstate$ such that for all $D \in \D, \ismoved{D} = 1$ (that is, all disks in $\D$ were moved), we denote by $\mstatefinal{\D}$ the collection of disks in which there exists a disk $D' \in \mstatefinal{\D}$ such that $c(D') = \movedpos{D}$ for a disk $D \in \D$.
If for all $D \in \cstate$, $\ismoved{D} = 1$, $d_D(\movedpos{D}) \le K$ and $\mstatefinal{\cstate}$ is contained in $\Pi$, then $(\cstate,\Pi)$ is a yes instance of {\ggedmm}.
In the reduction, we construct $\cstate$ using $\Phi$ and $G_\Phi$ and show that there exists a $\mstatefinal{\cstate}$ such that $\mstatefinal{\cstate}$ is in $\Pi_{\texttt{edgeless}}$ if and only if $\Phi$ is satisfiable.

\subsubsection{Cell Gadgets}

%%DEFINITIONS
A \emph{cell gadget} $\G_{(x,y)}\subseteq \cstate$ consists of a transition disk centred at $(x,y)$ surrounded by $6$-heavy disks centred at points $\{(x+i,y+j)\mid i,j\in \{-1,0,1\}\}\setminus\{(x,y)\}$ (see \Cref{fig:cell_gadget}). We denote the subcollection that contains these heavy disks of $\G_{(x,y)}$ by $\G_{(x,y)}^{(\mathit{heavy})}$.

%%%---fig:cell_gadget
\begin{figure}[!b]
    \centering
    \includegraphics[scale=1,page=26]{media/disk_edgeless.pdf}
    %\includesvg[width=0.5\textwidth]{media/definitions.svg}
    \caption{Cell gadget: An arbitrary cell gadget (left); two connected cell gadgets such that their transition disks are consecutive. The faded heavy disks are the omitted heavy disks to show the shape of gadgets.}
    \label{fig:cell_gadget}
\end{figure}

%%PROPERTIES 
Given a collection of disks $\D$, we denote the convex hull of the set $\set{c(D)\mid D\in \D}$ by $\C(\D)$.
Let $\D$ be a collection of $k$-heavy disks. If $\C(\D)$ (i) is a $|\D|$-gon and (ii) $\B_{D}\cap \B_{D'} \neq \emptyset$ for any pair of disks $D,D' \in \D$ such that $c(D)$ and $c(D')$ share an edge in $\C(\D)$, then the set $\cup_{D\in\D} \B_D $ is called \emph{blocked enclosure}.
For instance, the union of blocked zones of heavy disks in a cell gadget forms a blocked enclosure.
If the blocked zones of $\D$ form a blocked enclosure, the \emph{interior hole} of $\D$ is the region given by $\C(\D) \setminus \cup_{D\in\D} \B_D$.

Given a cell gadget $\G_{(x,y)}$ with transition disk $D$, we denote its interior hole $\C(\G_{(x,y)}^{(\mathit{heavy})})\setminus (\cup_{D' \in \G_{(x,y)}^{(\mathit{heavy})}} \B_{D'})$ by $\H_{(x,y)}$ or $\H_{c(D)}$ indistinctly. \Cref{fig:interior_holes} shows an example of interior holes for heavy disks of an arbitrary cell gadget.

\begin{mlemmarep}\label{lem:no_holes_hd_squares}
    If $\D = \set{H_1,\ldots,H_4} \subseteq \istate$ is a collection of $6$-heavy disks such that $c(H_1) = (x,y)$, $c(H_2) = (x-1,y)$, $c(H_3) = (x,y-1)$ and $c(H_4) = (x-1,y-1)$, then the interior hole of $\D$ is empty.
\end{mlemmarep}
\begin{proof}
    Notice that the minimax centre of $c(H_1),\ldots,c(H_4)$ is given by the centre of the smallest circle that contains them. 
    This point is $c = (x-1/2,y-1/2)$.
    Moreover, the point $c$ is the farthest point from any $c(H_i)$ for $i\in \set{1,\ldots,4}$.
    We have $\lVert c, c(H_i)\rVert_2 = \sqrt{2}/2 < (2^6 - 1 )/2^6 = 63/64$ for any $i \in \set{1,\ldots,4}$.
    Therefore, any other point enclosed by $c(H_1),\ldots,c(H_4)$ is contained in at least one blocked zone, implying that the interior hole is empty.
\end{proof}

\begin{mlemmarep}\label{lem:holes_in_cell_gadgets}
    The interior hole $\H_{(x,y)}$ of an arbitrary cell gadget $\G_{(x,y)}$ with transition disk $D$ is a non-empty set.
    Moreover, $c(D) \in \H_{(x,y)}$.
\end{mlemmarep}
\begin{proof}
    Let $H_1,\ldots,H_8 \in \G_{(x,y)}^{(\mathit{heavy})}$ be the eight heavy disks of $\G_{(x,y)}$ such that the centres $(c(H_1),\ldots,c(H_8))$ are equal to $ ((x-1,y+1),(x,y+1),(x+1,y+1),(x-1,y),(x+1,y),(x-1,y-1),(x,y-1),(x+1,y-1)$.
    By definition, $\lVert (x,y), c(H_i)\rVert_2 \ge 1 > (2^6-1)/ 2^6 = 63/64$ holds for $i \in \set{1,\ldots,8}$.
    Therefore $c(D) \in \H_{(x,y)}$ and $\H_{(x,y)} \neq \emptyset$.
\end{proof}

\begin{observation}\label{obs:convex_polygon_one_disk}
    Let $\S$ be a convex polygon such that $\diam{\S} < 1$. The region delimited by $\S$ admits exactly one disk centred within it.
\end{observation}
\begin{proof}
    Let $p_1,p_2$ be the farthest pair of points in $\S$.
    Without loss of generality, assume that an arbitrary disk $D$ is moved to $p_1$.
    It gives $r(\B_D) = 1$ and thus $p_2 \in \B_D$ since $\diam{\S} < 1$.
    Since $p_2$ is the farthest point from $p_1$, all other points in $\S$ are also blocked by $\B_D$.
    Therefore, $\S$ admits exactly one disk centred within it.
\end{proof}


\begin{mlemmarep}\label{lem:holes_one_disk}
    The interior hole $\H_{(x,y)}$ of an arbitrary cell gadget $\G_{(x,y)}$ admits exactly one disk centred within it.
\end{mlemmarep}
\begin{proof}
    Let $\S$ be the square formed by points $(p_1,\ldots,p_4) = ((x-1/64,y+1/64),(x+1/64,y+1/64),(x-1/64,y-1/64),(x+1/64,y-1/64))$.
    The square $\S$ is a rectangle such that $\lVert p_1,p_4\rVert_2,\lVert p_2,p_3\rVert_2<1$. Consequently, $\S$ admits exactly one disk centred within it by \Cref{obs:convex_polygon_one_disk}.
    Moreover, $\H_{(x,y)} \subseteq \S \setminus \cup_{D' \in \G_{(x,y)}^{(\mathit{heavy})}} \B_{D'}$.
    Therefore, the lemma statement is true.
\end{proof}

%%%---fig:interior_holes
\begin{figure}[!b]
    \centering
    \includegraphics[scale=1,page=10]{media/disk_edgeless.pdf}
    %\includesvg[width=1\textwidth]{media/interior_holes.svg}
    \caption{Interior holes: (a) A square of four disks with no interior hole (\Cref{lem:no_holes_hd_squares}); (b) union of blocked zones of heavy disks of a cell gadget (\Cref{lem:holes_in_cell_gadgets}), where the small region at the centre of the figure not covered by blocked zones is the interior hole.}
    \label{fig:interior_holes}
\end{figure}

%%PROPERTIES 2
We now show that the feasible area of a transition disk is restricted to the areas used by consecutive transition disks. %\warning{formalise the concept of holes}

\begin{mlemmarep}\label{lem:cg_two_subsets}
    Let $\G_{(x,y)}$ and $\G_{(x',y')}$ be two cell gadgets with transition disks $D$ and $D'$, respectively.
    If $D$ and $D'$ are consecutive, then the feasible area $\F_{D}$ is equal to $(A_D \cap \H_{(x,y)}) \cup (A_{D} \cap \H_{(x',y')})$.
    Moreover, $A_D \cap \H_{(x,y)}$ and $A_{D'} \cap \H_{(x',y')}$ are non-empty and disjoint.
\end{mlemmarep}
\begin{proof}
    First, $\H_{(x,y)}$ and $\H_{(x',y')}$ are disjoint, so their intersection with $\A_D$ is also disjoint.
    By \Cref{lem:holes_in_cell_gadgets}, $\H_{(x,y)} \neq \H_{(x',y')} \neq \emptyset$ holds.
    By definition of $\G_{(x,y)}$, $c(D) \in \H_{(x,y)}$ and $\B_D = \emptyset$ hold, so $\A_D \cap \H_{(x,y)} \neq \emptyset$ also holds.
    Since $D$ is consecutive to $D'$, $d_D(c(D')) \le K$ and $c(D') $ is contained in $ \A_D$.
    It follows that $\A_D \cap \H_{(x',y')} \neq \emptyset$.
    %Hence, $(\A_D \cap \H_{(x,y)}) \cup (\A_D \cap \H_{(x',y')}) \subseteq \F_D$ holds.

    We prove that $\F_{D}$ is equal to $(A_D \cap \H_{(x,y)}) \cup (A_{D} \cap \H_{(x',y')})$.
    Recall that $\H_{(x,y)} = \C(\G_{(x,y)}^{(\mathit{heavy})}) \setminus \cup_{O \in \G_{(x,y)}^{(\mathit{heavy})}} \B_{O}$ and $\H_{(x',y')} = \C(\G_{(x',y')}^{(\mathit{heavy})}) \setminus \cup_{O \in \G_{(x',y')}^{(\mathit{heavy})}} \B_{O}$. By \Cref{lem:holes_in_cell_gadgets} and the definition of $\cstate$, $\H_{(x,y)}$ and $\H_{(x',y')}$ do not intersect with any blocked zone. Hence we can define
    \begin{align*}
        \H_{(x,y)} & = \C(\G_{(x,y)}^{(\mathit{heavy})}) \setminus \cup_{O \in \cstate} \B_{O} = \C(\G_{(x,y)}^{(\mathit{heavy})}) \cap (\cup_{O \in \cstate} \B_{O})^c\\
        \H_{(x',y')} & = \C(\G_{(x',y')}^{(\mathit{heavy})}) \setminus \cup_{O \in \cstate} \B_{O} = \C(\G_{(x',y')}^{(\mathit{heavy})}) \cap (\cup_{O \in \cstate} \B_{O})^c \:.
    \end{align*}
    
    On the other hand, $\F_D = \A_D \setminus \cup_{D' \in \cstate} \B_{D'} = \A_D \cap (\cup_{O \in \cstate} \B_{O})^c$. We now have that
    \begin{align*}
        (A_D \cap \H_{(x,y)}) \cup (A_D \cap \H_{(x',y')}) & = \left(\A_D \cap \left(\C(\G_{(x,y)}^{(\mathit{heavy})}) \cap (\cup_{O \in \cstate} \B_{O})^c\right)\right)\\
        &\phantom{=}\cup \left(\A_D \cap \left(\C(\G_{(x',y')}^{(\mathit{heavy})}) \cap (\cup_{O \in \cstate} \B_{O})^c\right)\right) \\
        & = \left(\A_D \cap \C(\G_{(x,y)}^{(\mathit{heavy})}) \cap (\cup_{O \in \cstate} \B_{O})^c\right)\\
        &\phantom{=} \cup \left(\A_D \cap \C(\G_{(x',y')}^{(\mathit{heavy})}) \cap (\cup_{O \in \cstate} \B_{O})^c\right)\\
        & =(\F_D \cap \C(\G_{(x,y)}^{(\mathit{heavy})}))\cup (\F_D \cap \C(\G_{(x',y')}^{(\mathit{heavy})})).
    \end{align*}
    It holds that $\C(\G_{(x,y)}^{(\mathit{heavy})})) \subset \A_D$, thus
    \begin{align*}
        (A_D \cap \H_{(x,y)}) \cup (A_D \cap \H_{(x',y')}) & =\F_D \cup (\F_D \cap \C(\G_{(x',y')}^{(\mathit{heavy})}))\\
        & = \F_D.
    \end{align*}
    This concludes the proof.
\end{proof}

\subsubsection{Clause Gadgets}
%%DEFINITIONS

The \emph{clause gadget} $\G_c \subseteq \cstate$ for an arbitrary clause $c$ consists of an intersection disk $T_c$ centred at an arbitrary point $(x,y)$ surrounded by $6$-heavy disks centred at points $\{(x+i,y+j)\mid i,j\in \{-1,0,1\}\}\setminus\{(x,y)\}$.
It also contains three cell gadgets $\G_{(x-3,y)}$, $\G_{(x,y-3)}$ and $\G_{(x+3,y)}$ representing the three literals of $c$ (see \Cref{fig:clause_gadget}).
We interpret the movement of $T_c$ to one of the arms as the assignment of truth value to the clause by the literal corresponding to the arm. 

%%%---fig:clause_gadget
\begin{figure}[!b]
    \centering
    \includegraphics[scale=1,page=27]{media/disk_edgeless.pdf}
    %\includesvg[width=0.5\textwidth]{media/definitions.svg}
    \caption{Clause Gadget: Clause gadget for an arbitrary clause $c$.}
    \label{fig:clause_gadget}
\end{figure}


%OBS 

%%PROPERTIES
\Cref{lem:cg_two_subsets} implies that if a transition disk is moved from its position (that is, it is moved outside $\H_{(x,y)}$), then its new position must be in $\H_{(x',y')}$ for an arbitrary $(x',y')\neq (x,y)$.
We now prove a similar property of clause gadgets.

\begin{mlemmarep}\label{lem:tc_three_subsets}
    Let $\G_{c}$ be a clause gadget such that $c(T_c) = (x,y)$.
    The feasible area $\F_{T_c}$ is equal to $(\A_{T_c} \cap \H_{(x-3,y)}) \cup (\A_{T_c} \cap \H_{(x,y-3)}) \cup (\A_{T_c} \cap \H_{(x+3,y)})$.
    Moreover, $\A_{T_c} \cap \H_{(x-3,y)}$, $\A_{T_c} \cap \H_{(x,y-3)}$ and $\A_{T_c} \cap \H_{(x+3,y)}$ are non-empty and disjoint between each other.
%    The feasible area $\F_{T_c}$ can be partitioned into exactly three non-empty subsets.
\end{mlemmarep}
\begin{proof}
    First, $\H_{(x-3,y)}$, $\H_{(x,y-3)}$ and $\H_{(x+3,y)}$ are disjoint between each other, so their intersections with $\A_{T_c}$ must also be disjoint.
    By \Cref{lem:holes_in_cell_gadgets}, $\H_{(x-3,y)} \neq \H_{(x,y-3)} \neq \H_{(x+3,y)} \neq \emptyset$.
    Moreover, $d_{T_c}((x-3,y)),d_{T_c}((x,y-3)),d_{T_c}((x+3,y)) \le K$ holds, so $\A_{T_c} \cap \H_{(x-3,y)}$, $\A_{T_c} \cap \H_{(x,y-3)}$ and $\A_{T_c} \cap \H_{(x+3,y)}$ are non-empty.

    We prove that $\F_{T_c}$ is equal to $(\A_{T_c} \cap \H_{(x-3,y)}) \cup (\A_{T_c} \cap \H_{(x,y-3)}) \cup (\A_{T_c} \cap \H_{(x+3,y)})$.
    Recall that we can define $\H_{(x-3,y)}$, $\H_{(x,y-3)}$ and $\H_{(x+3,y)}$ as follows:
    \begin{align*}
        \H_{(x-3,y)} & = \C(\G_{(x-3,y)}^{(\mathit{heavy})}) \setminus \cup_{D' \in \cstate} \B_{D'} = \C(\G_{(x-3,y)}^{(\mathit{heavy})}) \cap (\cup_{D' \in \cstate} \B_{D'})^c\\
        \H_{(x,y-3)} & = \C(\G_{(x,y-3)}^{(\mathit{heavy})}) \setminus \cup_{D' \in \cstate} \B_{D'} = \C(\G_{(x,y-3)}^{(\mathit{heavy})}) \cap (\cup_{D' \in \cstate} \B_{D'})^c\\
        \H_{(x+3,y)} & = \C(\G_{(x+3,y)}^{(\mathit{heavy})}) \setminus \cup_{D' \in \cstate} \B_{D'} = \C(\G_{(x+3,y)}^{(\mathit{heavy})}) \cap (\cup_{D' \in \cstate} \B_{D'})^c.
    \end{align*}
    On the other hand, $\F_{T_c} = \A_{T_c} \setminus \cup_{D' \in \cstate} \B_{D'} = \A_{T_c} \cap (\cup_{D' \in \cstate} \B_{D'})^c$. Let $\H = (\H_{(x-3,y)} \cup \H_{(x,y-3)} \cup \H_{(x+3,y)})$. Then, 
    \begin{align*}
        (\A_{T_c} \cap \H_{(x-3,y)}) &\cup (\A_{T_c} \cap \H_{(x,y-3)}) \cup (\A_{T_c} \cap \H_{(x+3,y)}) = \\
        &= \A_{T_c} \cap (\H_{(x-3,y)} \cup \H_{(x,y-3)} \cup \H_{(x+3,y)})\\
        & = \A_{T_c} \cap \H\\
        & = \A_{T_c} \cap \H \cap (\cup_{D' \in \cstate} \B_{D'})^c\\
        & = \F_{T_c} \cap \H.
    \end{align*}
    For all $x \in \A_{T_c} \cap \H$, it holds that $x \in \F_{T_c}$ by the above equation. Hence, $\A_{T_c} \cap \H \subseteq F_{T_c}$.
    In contrast, for all $x \in \F_{T_c}$, we have $x \in \A_{T_c}$ since $\F_{T_c} \subseteq \A_{T_c}$.
    Moreover, it follows from \Cref{lem:no_holes_hd_squares} and the definition of clause gadgets that there exists no point $p\in \A_D$ such that $p \notin \cup_{D' \in \cstate} \B_{D'}$ and $p \notin \H$.
    Thus $\F_{T_c} \subseteq \H_{(x-3,y)}\cup \H_{(x,y-3)}\cup \H_{(x+3,y)}$ holds, which implies that $x \in \H$.
    Consequently, $\F_{T_c} \subseteq A_{T_c} \cap \H$ holds.
    Therefore $\F_{T_c} = \A_{T_c} \cap \H = (\A_{T_c} \cap \H_{(x-3,y)}) \cup (\A_{T_c} \cap \H_{(x,y-3)}) \cup (\A_{T_c} \cap \H_{(x+3,y)})$ holds.
\end{proof}

\subsubsection{Variable Gadgets}
%%DEFINITIONS

Lastly, the \emph{variable gadget} $\G_x \subseteq \cstate$ for an arbitrary variable $x$ is depicted in \Cref{fig:variable_gadget}. 
The variable gadget consists of two truth sides (called true and false sides) with three free slots and an intersection disk. 
There exists space for moving the truth setter disk in both truth sides.

%%%---fig:variable_gadget
\begin{figure}[!htb]
    \centering
    \includegraphics[scale=1,page=28]{media/disk_edgeless.pdf}
    \caption{Variable gadget: Variable gadget for variable $S_x$ with three arms connected to the left side. The dashed region is the central part of the variable gadget. Shaded concentric circles are circles of radius $K$ for the $L_1$ and $L_2$ distances highlighting the distance between consecutive transition disks.}
    \label{fig:variable_gadget}
\end{figure}

The collection of disks enclosed by the yellow dashed region is the \emph{central part} of the gadget. 
\Cref{fig:variable_names} shows the central part of the gadget for an arbitrary variable $x$.
The disk ${S}_x$ is called \emph{truth setter disk}. 

%%%---fig:variable_names
\begin{figure}[!htb]
    \centering
    \includegraphics[scale=1,page=2]{media/disk_edgeless.pdf}
    %\includesvg[width=0.8\textwidth]{media/variable_names.svg}
    \caption{Central part of the gadget for an arbitrary variable $x$.}
    \label{fig:variable_names}
\end{figure}

For convenience, we show the definitions assuming that $c(S_x) = (0,0)$.
The central part consists of consecutive $6$-heavy disks surrounding $S_x$ and two \emph{truth sides} called \emph{true side} and \emph{false side}.
As we show in \Cref{lem:block_disk_restricted}, $S_x$ can be moved to free spaces $\S_{t,1}^x$ or $\S_{t,2}^x$ called \emph{truth slots}, representing the false or true value given $x$, respectively.
The centres of $\S_{t,1}^x$ and $\S_{t,2}^x$ are denoted by $s^x_{t,1} = (-3,0)$ and $s^x_{t,2} = (3,0)$, respectively.
The \emph{blocking disk of a truth side} is a $1$-heavy disk that possibly blocks $\S_{t,i}^x$ and is denoted by $\bdisk{\S^x_{t,i}}$, where $i = 1$ for the true side and $i=2$ for the false side.
Each truth side can be connected to at most three clauses, denoted in the figure by $c_1,c_2,c_3$ for the true side and $c_4,c_5,c_6$ for the false side.
For $i \in \set{1,\ldots,6}$, clause $c_i$ contains a free space $\S_{c,i}^x$ to position the transition disk moved from clause gadget $c_i$, with centre $s_{c,i}^x$.
We denote the transition disk moved from $c_i$ by $D_{c_i}$.
\Cref{fig:move_of_disks} shows the transition disks $D_{c_1},D_{c_2},D_{c_3}$ for $c_1,c_2,c_3$, respectively.

%%%---fig:move_of_disks
\begin{figure}[!hbt]
    \centering
    \includegraphics[scale=1,page=5]{media/disk_edgeless.pdf}
    \caption{Left: The truth setter disk is blocking the right side, letting transition disks $D_{c_1},D_{c_2},D_{c_3}$ to be moved to free slots. Right: The truth setter disk is blocking free slots to move $D_{c_1},D_{c_2},D_{c_3}$ into the gadget.}
    \label{fig:move_of_disks}
\end{figure}

The \emph{blocking disk for $c_i$} is a $1$-heavy disk that possibly blocks $\S_{c,i}^x$ and is denoted by $\bdisk{\S^x_{c,i}}$.
The \emph{link disk for $c_i$} is a $2$-heavy disk moved close to $\bdisk{\S^x_{c,i}}$ depending on the position of the blocking disk of the truth side.
The link disk is denoted by $\ldisk{\S^x_{c,i}}$.

Each variable is connected to at most three clauses by the arms depicted in \Cref{fig:variable_gadget}connected to $c_1$, $c_2$ and $c_3$.
The central part of the gadget can be mirrored horizontally and arms can be mirrored vertically and horizontally.
%OBS 
We make some observations used in the subsequent lemmas.

\begin{observation}[Vertical condition of blocked zones]\label{obs:vertical_condition}
    Let $H_1$ and $H_2$ be two $k$-heavy disks centred at $(x,y)$ and $(x+1,y)$, respectively. A point $(x',y')$ such that the inequality $y - \sqrt{((2^k-1)/2^k)^2 - 1/4} \le y' \le y + \sqrt{((2^k-1)/2^k)^2 - 1/4}$ holds is in $\B_{H_1} \cup \B_{H_2}$ for any $x-1/2\le x' \le x+3/2 $.
\end{observation}
\begin{proof}
    We show that $\B_{H_1} $ and $ \B_{H_2}$ intersect at $(x+1/2,y + \sqrt{((2^k-1)/2^k)^2 - 1/4})$. 
    The proof is analogous for the lower bound.
    Let $p = (x'',y'')$ be the point where $\B_{H_1} $ and $ \B_{H_2}$ intersect.
    Straightforwardly, $x'' = x+1/2$.
    The triangle formed by points $p$, $(x,y)$ and $(x,y+1)$ is an isosceles triangle of base length $1$ and sides of length $2^k-1 /2^k$.
    Moreover, $y'' = y + h$ where $h$ is the height of the triangle.
    The height is equal to $h = \sqrt{((2^k-1)/2^k)^2 - 1/4}$.
    Hence $p = (x+1/2,y + \sqrt{((2^k-1)/2^k)^2 - 1/4})$.

    Furthermore, notice that $(x-1/2,y'') \in \B_{H_1}$ and $(x+3/2,y'') \in \B_{H_2}$.
    Therefore, for any $x-1/2\le x' \le x+3/2 $, $(x',y + \sqrt{((2^k-1)/2^k)^2 - 1/4}) \in \B_{H_1} \cup \B_{H_2}$.
\end{proof}

\begin{observation}[Horizontal condition of blocked zones]\label{obs:horizontal_condition}
    Let $H_1$ and $H_2$ be two $k$-heavy disks centred at $(x,y)$ and $(x,y+1)$, respectively. A point $(x',y')$ such that $x - \sqrt{((2^k-1)/2^k)^2 - 1/4} \le x' \le x + \sqrt{((2^k-1)/2^k)^2 - 1/4}$ is in $\B_{H_1} \cup \B_{H_2}$ for any $y-1/2\le y' \le y+3/2 $.
\end{observation}
    
\begin{figure}[tb]
    \centering
    \includegraphics[scale=1,page=14]{media/disk_edgeless.pdf}
    %\includesvg[width=0.8\textwidth]{media/vert_hor_condition.svg}
    \caption{Illustration of \Cref{obs:vertical_condition} (left) and \Cref{obs:horizontal_condition} (right).}
\end{figure}


%%PROPERTIES
\Cref{lem:tc_three_subsets} defines the removal of the intersection of $T_c$. In particular, $T_c$ must be moved to one of the three non-empty disjoint subsets given by $(\A_{T_c} \cap \H_{(x-3,y)})$, $(\A_{T_c} \cap \H_{(x,y-3)})$ and $(\A_{T_c} \cap \H_{(x+3,y)})$.
%In the following lemmas, it is assumed that $K = 1$.

\begin{mlemmarep}\label{lem:sx_two_subsets}
    Let $\G_{x}$ be a variable gadget such that $c(S_x) = (x,y)$ and $H_1^{(i)},\ldots,H_5^{(i)}$ the $6$-heavy disks surrounding $s^x_{t,i}$ for $i \in \set{1,2}$.
    The feasible area $\F_{S_x}$ is equal to $\F_{S_x}^{(1)} \cup \F_{S_x}^{(2)}$, where $\F_{S_x}^{(i)} = (\A_{S_x}\cap \S^x_{t,i})\setminus \cup_{j=1}^5 \B_{H_j^{(i)}}$ for $i \in \set{1,2}$.
    Moreover, $\F_{S_x}^{(t)}$ and $\F_{S_x}^{(f)}$ are non-empty and disjoint.
\end{mlemmarep}
\begin{proof}
    %Let $\A_t$ and $\A_f$ be two feasible areas of $S_x$ for the true and false side, respectively.
    It holds that $\A_{S_x}\cap \S^x_{t,i} \neq \emptyset$ since $d_{S_x}(s^x_{t,1}) \le K$.
    The region $\S^x_{t,i}$ is partially blocked by $H_1^{(1)},\ldots,H_5^{(1)}$.
    In particular, $s^x_{t,1} = (x-3,y)$ and the centres of $H_1^{(1)},\ldots,H_5^{(1)}$ are given by $c(H_1^{(1)}) = (x-3,y+1)$, $c(H_2^{(1)}) = (x-2,y+1)$, $c(H_3^{(1)}) = (x-2,y)$, $c(H_4^{(1)}) = (x-2,y-1)$ and $c(H_5^{(1)}) = (x-3,y-1)$.
    %By \Cref{obs:unreachable_disk}, we only need to consider the blocked zones by $H_1,\ldots,H_5$.
    For any $i \in \set{1,\ldots,5}$, $d_{H_i^{(1)}}(s_{t,1}^x) \ge 1$ holds, implying that $s_{t,1}^x \notin \cup_{i=1}^5 \B_{H_i^{(1)}}$.
    It follows that $\F_{S_x}^{(i)} = (\A_{S_x}\cap \S^x_{t,i})\setminus \cup_{j=1}^5 \B_{H_j^{(i)}}$ is non-empty for $i = 1$.
    Moreover, by the definition of $\G_x$, there exists no other heavy disk close to $\S^x_{t,i}$.
    Consequently, $\F_{S_x}^{(i)} = (\A_{S_x}\cap \S^x_{t,i})\setminus \cup_{j=1}^5 \B_{H_j^{(i)}} = \cup_{D'\in\cstate-\{S_x\}}B_{D'}$.
    The proof is analogous for $i = 2$.
    Moreover, $\S^x_{t,1}$ and $\S^x_{t,2}$ are disjoint by the definition of $\G_x$, so $\F_{S_x}^{(1)}$ and $\F_{S_x}^{(2)}$ must also be disjoint.

    We have shown that $\F_{S_x}^{(i)} \neq \emptyset$ for $i \in \set{1,2}$. 
    We only need to prove that $\F_{S_x} \setminus (\F_{S_x}^{(1)} \cup \F_{S_x}^{(2)}) = \emptyset$ holds.
    By the definition of the variable gadget, $S_x$ is surrounded by $6$-heavy disks $\cup_{i,j\in \set{-3,\ldots,3}} \hdisk{(i,j)}$ where $\hdisk{(-3,0)}= \hdisk{s_{t,1}^x} = \emptyset$ and $\hdisk{(3,0)}= \hdisk{s_{t,2}^x} = \emptyset$.
    %Notice that $\hdisk{(-3,0)}= \hdisk{s_{t,1}^x} = \emptyset$ and $\hdisk{(3,0)}= \hdisk{s_{t,2}^x} = \emptyset$.
    Suppose instead that there exist such disks at $s_{t,1}^x$ and $s_{t,2}^x$.
    Let $z = \sqrt{(63/64)^2-1/4}$.
    By \Cref{obs:vertical_condition,obs:horizontal_condition}, any point $(x',y')$ such that $x-3-z \le x' \le x+3+z$ and $y-3 - z \le y' \le y+3 + z$ is in $\cup_{i,j\in \set{-3,\ldots,3}} \B_{\hdisk{(i,j)}}$.
    This would imply $\F_{S_x} = \emptyset$ as $r(\A_{S_x}) = 3$.
    %If $\hdisk{s_{t,1}^x}$ and $\hdisk{s_{t,2}^x}$ are deleted, at most two connected subsets might appear in $\A_{S_x}$.
    We have already shown that if there exists no heavy disk centred at $s_{t,1}^x$ and $s_{t,2}^x$, then there exist two non-empty disjoint subsets in $\F_{S_x}$, namely $\F_{S_x}^{(1)}$ and $\F_{S_x}^{(2)}$.
    Therefore, $\F_{S_x} \setminus (\F_{S_x}^{(1)} \cup \F_{S_x}^{(2)}) = \emptyset$ holds.
\end{proof}
\Cref{lem:sx_two_subsets} implies that the movement of $S_x$ is restricted to the two subsets $\F_{S_x}^{(1)}$ and $\F_{S_x}^{(2)}$. 
We now show that depending on where $S_x$ is moved, the feasible areas of $1$- and $2$-heavy disks of the variable gadget become restricted.

\begin{mlemmarep}\label{lem:block_disk_restricted}
    Let $\G_{x}$ be a variable gadget such that $c(S_x) = (x,y)$ and $H_1^{(i)},\ldots,H_5^{(i)}$ the $6$-heavy disks surrounding $s^x_{t,i}$ for $i \in \set{1,2}$.
    If $S_x$ is moved to $\F_{S_x}^{(1)}$, then $\bdisk{S^x_{t,1}}$ must be moved to a point $p$ that makes $\bdisk{S^x_{t,1}}$ intersect with disks $\ldisk{S^x_{c,1}}$, $\ldisk{S^x_{c,2}}$ and $\ldisk{S^x_{c,3}}$.
\end{mlemmarep}
\begin{proof}
    We need to consider the feasible areas of disks to show the statement.
    For simplicity, we assume that feasible areas of the involved disks are properly defined rectangles.
    These rectangles completely contain the actual feasible areas of disks.
    We then show that the statement is true for the rectangles (and hence also true for the actual feasible areas). 
    In particular, we describe the movement of disks and show that whenever we move the disks to any point of the rectangles, the statement holds true.
    An illustration of the proof is shown in \Cref{fig:block_disk_restricted}.
    
    By \Cref{lem:sx_two_subsets}, $\F_{S_x}^{(1)} = (\A_{S_x}\cap \S^x_{t,1})\setminus \cup_{j=1}^5 \B_{H_j^{(1)}}$. %where
    Without loss of generality, assume that $(x,y) = 0$.
    Recall that disks $H_1^{(1)},\ldots,H_5^{(1)}$ are the five $6$-heavy disks surrounding $\S^x_{t,1}$ with centres $c(H_1^{(1)}) = (-3,1)$, $c(H_2^{(1)}) = (-2,1)$, $c(H_3^{(1)}) = (-2,0)$, $c(H_4^{(1)}) = (-2,-1)$ and $c(H_5^{(1)}) = (-3,-1)$.
    We prove the statement for a rectangle $\R$ such that $\F_{S_x}^{(1)} \subseteq \R$. %defined by points where $\A_{S_x}$ and $(\cup_{i=1}^5 \B_{H_i})^c$ intersect. 
    In particular, let $\R$ be the rectangle defined by points $(-3,1/50),(-3+1/50,1/50),(-3+1/50,-1/50), (-3,1/50)$.
    It can be checked that $\F_{S_x}^{(1)}$ is contained in $\R$ by calculating the distances to $c(H_1^{(1)})$, $c(H_3^{(1)})$ and $c(H_5^{(1)})$.
    Let $q = (q_x,q_y) \in \R$ be the point where $S_x$ is moved.
    We show the feasible area of $\bdisk{S_{t,1}^x}$ after moving $S_x$ to $q$.
    Let $p = (p_x, p_y)$ be an arbitrary point of $\F_{\bdisk{S_{t,1}^x}}$.
    Recall that if $S_x$ is moved, then $\ismoved{S_x} = 1$ and $r(\B_{S_x}) = 1$.
    Thus $p_x \le q_x - 1 = -4 + 1/50$ holds since $\bdisk{S_{t,1}^x}$ is initially placed at $(-3-1/2,0)$ and $q_x \le -3 +1/50$ by the definition of $\R$.
    Moreover, we have that $c(\ldisk{S^x_{c,1}}) = (-4,3/4)$, $c(\ldisk{S^x_{c,2}}) = (-4-3/4,0)$ and $c(\ldisk{S^x_{c,3}}) = (0,-3/4)$.
    Consequently $-4 \le p_x$ also holds since $r(\ldisk{S_{c,2}^x}) = 3/4$.
    
    By the above argument, the inequality $-4 \le p_x \le -4+1/50$ holds. 
    We use this range of values for $p_x$ and define a rectangle $\R'$ that contains $\F_{\bdisk{S_{t,1}^x}}$ and thus $p$.
    Notice that $\F_{\bdisk{S_{t,1}^x}} = \A_{\bdisk{S_{t,1}^x}} \setminus (\cup_{i=1}^3 \B_{\ldisk{S^x_{c,i}}} \cup \B_{S_x})$. %where $S_x \in \mstate$. 
    We describe $\R'$ by using the limits of $\F_{\bdisk{S_{t,1}^x}}$. 
    First, let $p' = (-4+1/50,y')$ be a point. 
    We set $y'$ such that $\lVert c(\ldisk{S^x_{c,1}}),(-4+1/50,y')\rVert_\alpha = 3/4$.
    The point $p'$ is on the boundary of $\B_{\ldisk{S^x_{c,1}}}$.
    It can be checked that $y' < 1/100$, thus we conveniently set $y' = 1/100$ as an upper bound for $\R'$.
    The same argument can be used for $\ldisk{S^x_{c,3}}$ and a $y' = -1/100$ as a lower bound for $\R'$.
    On the other side, note that $\cap_{i \in \set{1,2,3}} \B_{\ldisk{S^x_{c,i}}} = \set{(-4,0)}$. 
    In other words, $\B_{\bdisk{S^x_{t,1}}}$ cannot be moved to a point $(-4,y')$ for $y' \neq 0$.
    Hence, we can use a rectangle $\R'$ defined by points $(-4,1/100)$, $(-4+1/50,1/100)$, $(-4+1/50,-1/100)$ and $(-4,-1/100)$ for which $F_{\bdisk{S^x_{t,1}}}$ is a subset.
    It can be checked that for any point in $p \in \R'$, the distance $\lVert p,c(\ldisk{S^x_{c,i}})\rVert_\alpha < 1$ for $i \in \set{1,2,3}$.
    In other words, whenever $\bdisk{S_{t,1}^x}$ is moved to a point $p \in \R'$, it intersects with link disks $\ldisk{S^x_{c,1}}$, $\ldisk{S^x_{c,2}}$ and $\ldisk{S^x_{c,3}}$.
    Therefore, $\bdisk{S^x_{t,1}}$ must be moved to a point $p$ that makes $\bdisk{S^x_{t,1}}$ intersect with disks $\ldisk{S^x_{c,1}}$, $\ldisk{S^x_{c,2}}$ and $\ldisk{S^x_{c,3}}$.
\end{proof}

\begin{figure}[bt]
    \centering
    \includegraphics[scale=1,page=11]{media/disk_edgeless.pdf}
    %\includesvg[width=0.8\textwidth]{media/block_disk_restricted.svg}
    \caption{Illustration of \Cref{lem:block_disk_restricted}. The orange regions is the union of blocked zones of disks $\ldisk{S^x_{c,i}}$ for $i\in \set{1,2,3}$, whereas the grey region is the blocked zone of $S_x$. The feasible area $\F_{\bdisk{S^x_{t,1}}}$ is confined to the rectangle $\R'$.}
    \label{fig:block_disk_restricted}
\end{figure} 


\begin{mlemmarep}\label{lem:feasible_areas_blocked}
    If $S_x$ is moved to $\F_{S_x}^{(1)}$, then $\F_{D_{c_i}} = \emptyset$ for $i \in \set{1,2,3}$.
\end{mlemmarep}
\begin{proof}
    Let $\R'$ be the rectangle defined by points $(-4,1/100)$, $(-4+1/50,1/100)$, $(-4+1/50,-1/100)$ and $(-4,-1/100)$.
    As proven in \Cref{lem:block_disk_restricted}, the disk $\bdisk{S_{t,1}^x}$ intersects with link disks $\ldisk{S^x_{c,1}}$, $\ldisk{S^x_{c,2}}$ and $\ldisk{S^x_{c,3}}$ when moved to a point $p \in \R'$ (see \Cref{fig:feasible_areas_blocked_1}).
    Suppose that $\bdisk{S_{t,1}^x}$ is actually moved to a point $p \in \R'$.
    We show the statement in the same fashion as in \Cref{lem:block_disk_restricted} using rectangles $\R^1,\R^2,\R^3$ containing $\F_{\ldisk{S^x_{c,1}}}$, $\F_{\ldisk{S^x_{c,2}}}$ and $\F_{\ldisk{S^x_{c,3}}}$, respectively.
    
    We start by defining $\R^1$ (see \Cref{fig:feasible_areas_blocked_2}).
    Let $q$ be a point such that $q \in \F_{\ldisk{S^x_{c,1}}}$.
    We have $q_y \le 1$ since $r(\A_{\ldisk{S^x_{c,1}}}) = 1/4$ and $c(\ldisk{S^x_{c,1}}) = (-4,3/4)$.
    The farthest point in $\R'$ from $c(\ldisk{S^x_{c,1}})$ is the bottom right corner $(-4+1/50,-1/100)$.
    We show a lower bound for $q_y$ assuming that $p = (-4+1/50,-1/100)$.
    Let $q'$ be the point with the lowest $y$-axis coordinate value that is in $\F_{\ldisk{S^x_{c,1}}}$. 
    The values of $q'$ are given by the intersection of the boundaries of $\B_{\bdisk{S_{t,1}^x}}$ and $\B_{\hdisk{(-5,1)}}$ (or $\B_{\hdisk{(-3,1)}}$).
    %, where $\bdisk{S_{t,1}^x} \in \mstate$ and $\hdisk{(-5,1)} \in \istate$.
    It can be checked that $q'_y$ approximately equals $0.9893$. Thus, we reasonably set the lower bound of $q_y$ to $1-1/50$.
    % sqrt((x+5)^2 + (y-1)^2) = 63/64
    % sqrt((x-(-4+1/50))^2 + (y-(-1/100))^2) = 1
    Given that $1-1/50 \le q_y\le 1$, the range of $q_x$ is given by the intersection of boundaries of blocked zones $\B_{\hdisk{(-5,1)}}$ and $\B_{\hdisk{(-3,1)}}$ with the boundary of $\B_{\bdisk{S_{t,1}^x}}$.
    % sqrt((-5-(x))^2+(1-(1-1/50))^2)=63/64 for (-5,1)
    % sqrt((-3-(x))^2+(1-(1-1/50))^2)=63/64 for (-3,1)
    It can be checked that $q_x$ has a value $-4\pm 0.015\dots$, thus we reasonably set the bound for $q_x$ to $-4 - 1/50 \le q_x \le -4+1/50$.

    We have the points that define $\R^1$. 
    Now we use $\R^1$ and show that if $\ldisk{S^x_{c,1}}$ is moved to any point $q\in \R^1$, then $\bdisk{S^x_{c,1}}$ must be moved to a point $p^1 \in \F_{\bdisk{S^x_{c,1}}}$ that makes $\F_{D_{c_1}} = \emptyset$.
    When $\ldisk{S^x_{c,1}}$ is moved to $q$, $\ismoved{\ldisk{S^x_{c,1}}} = 1$ and $r(\B_{\ldisk{S^x_{c,1}}}) =1$. 
    So the point $p^1$ must satisfy $\lVert p^1,q\rVert_\alpha \ge 1$.
    Moreover, $\B_{\ldisk{S^x_{c,1}}}$ intersects $\A_{\bdisk{S^x_{c,1}}}$, thus the lowest possible values for $p^1_y$ are given by the intersection of $\B_{\ldisk{S^x_{c,1}}}$ and $\B_{\hdisk{(-5,2)}}$ when $q = (-4+1/50,1-1/50)$ and the intersection of $\B_{\ldisk{S^x_{c,1}}}$ and $\B_{\hdisk{(-3,2)}}$ when $q = (-4-1/50,1-1/50)$.
    % c(\bdisk{S^x_{c,1}}) = (-4,2+1/4)
    % for (-4+1/50,1-1/50) → x≈-4.01584 ∧ y≈1.97936
    %   sqrt((x-(-4+1/50))^2+(y-(1-1/50))^2) = 1
    %   sqrt((x-(-5))^2+(y-(2))^2) = 63/64
    % for (-4-1/50,1-1/50) → x≈-3.98416 ∧ y≈1.97936
    %   sqrt((x-(-4-1/50))^2+(y-(1-1/50))^2) = 1
    %   sqrt((x-(-3))^2+(y-(2))^2) = 63/64
    In particular, $p^1_y \ge 1.979\ldots > 2-3/100$.
    The $x$-axis values of the points on the boundary of $\B_{\hdisk{(-5,2)}}$ and $\B_{\hdisk{(-3,2)}}$ for which $y = 2-3/100$ are $-4- 0.0161\ldots$ and $-4+0.0161\ldots$, which are bounded by $-4- 1/50$ and $-4+ 1/50$, respectively.
    % for (-5,2) → x≈-4.0161
    %   sqrt((x-(-5))^2+(2-3/100-(2))^2) = 63/64
    % for (-3,2) → x≈-3.9839
    %   sqrt((x-(-3))^2+(2-3/100-(2))^2) = 63/64
    These two points are the farthest point from $s^x_{c,1}$ for which $\bdisk{S^x_{c,1}}$ can be relocated.
    \Cref{fig:feasible_areas_blocked_3} illustrates the range of values for $p^1$.
    Before moving $\bdisk{S^x_{c,1}}$, the closest points in $\F_{D_{c_1}}$ to $D_{c_1}$ are the intersection points of $\B_{\hdisk{(-5,3)}}$ and $\B_{\hdisk{(-3,3)}}$ with $\B_{\hdisk{(-4,4-1/4)}}$. 
    In particular, these points are $(-4.04\ldots,2.76\ldots)$ and $(-3.95\ldots,2.76\ldots)$.
    We reasonably round these points to $(-4.05,2.77)$ and $(-3.95,2.77)$, respectively.
    % (-5,3) and (-4,4-1/4) → x≈-4.0437 ∧ y≈2.7666
    %   sqrt((x-(-5))^2+(y-(3))^2) = 63/64
    %   sqrt((x-(-4))^2+(y-(4-1/4))^2) = 63/64
    % (-3,3) and (-4,4-1/4) → x≈-3.9563 ∧ y≈2.7666
    %   sqrt((x-(-3))^2+(y-(3))^2) = 63/64
    %   sqrt((x-(-4))^2+(y-(4-1/4))^2) = 63/64
    When $\bdisk{S^x_{c,1}}$ is moved to $(-4\pm 1/50, 2-3/100)$, 
    $\ismoved{\bdisk{S^x_{c,1}}} = 1$ and thus $r(\B_{\bdisk{S^x_{c,1}}}) = 1$. 
    It can be checked that both $(-4.05,2.77)$ and $(-3.95,2.77)$ are contained in $\B_{\bdisk{S^x_{c,1}}}$.
    % sqrt((-4-1/50-(-4.05))^2+(2-3/100-(2.77))^2) = 0.8005623023850175
    % sqrt((-4-1/50-(-3.95))^2+(2-3/100-(2.77))^2) = 0.8030566605
    % sqrt((-4+1/50-(-4.05))^2+(2-3/100-(2.77))^2) = 0.8030566605165542
    % sqrt((-4+1/50-(-3.95))^2+(2-3/100-(2.77))^2) = 0.8005623024
    See \Cref{fig:feasible_areas_blocked_3} for the case when $\bdisk{S^x_{c,1}}$ is moved to $(-4+ 1/50, 2-3/100)$ and \Cref{fig:feasible_areas_blocked_4} to see that the points defined are contained in $\B_{\bdisk{S^x_{c,1}}}$.
    Moreover, both points are the farthest points to $c(\bdisk{S^x_{c,1}})$ in $\F_{D_{c_1}}$ before moving $\bdisk{S^x_{c,1}}$. 
    Consequently, we conclude that $\F_{D_{c_1}} = \emptyset$.
    The proof is analogous for $\R^2$ and $\R^3$ by rotating the given coordinates by $\pi/2$ and $\pi$ degrees, respectively.
    We started by moving $\bdisk{S_{t,1}^x}$ to a point $p \in \R'$, which is given by moving $S_x$ to $\F_{S_x}^{(1)}$ by \Cref{lem:block_disk_restricted}.
    Therefore, if $S_x$ is moved to $\F_{S_x}^{(1)}$, then $\F_{D_{c_i}} = \emptyset$ for $i \in \set{1,2,3}$.
\end{proof}

\begin{figure}[!htb]
    \centering
    \includegraphics[scale=1,page=18]{media/disk_edgeless.pdf}
    %\includesvg[width=0.75\textwidth]{media/feasible_areas_blocked_1.svg}
    \caption{The disk $\bdisk{S_{t,1}^x}$ is moved to the point $p= (-4+1/50,-1/100) \in \R'$.}
    \label{fig:feasible_areas_blocked_1}
\end{figure}
\begin{figure}[!htb]
    \centering
    \includegraphics[scale=1,page=19]{media/disk_edgeless.pdf}
    %\includesvg[width=0.75\textwidth]{media/feasible_areas_blocked_2.svg}
    \caption{The disk $\ldisk{S_{c,1}^x}$ is moved to the point $(-4-1/50,1-1/50)$ contained in the rectangle $\R^1$ defined by points $(x,y)$ such that $-4-1/50 \le x \le -4+1/50$ and $1-1/50 \le y \le 1$.}
    \label{fig:feasible_areas_blocked_2}
\end{figure}
\begin{figure}[!htb]
    \centering
    \includegraphics[scale=1,page=20]{media/disk_edgeless.pdf}
    %\includesvg[width=0.75\textwidth]{media/feasible_areas_blocked_3.svg}
    \caption{Lowest value for $p_y^1$ and range of values for $p_x^1$ assuming that $\ldisk{S_{c,1}^x}$ was moved to the point $(-4-1/50,1-1/50)$. The disk $\bdisk{S^x_{c,1}}$ is moved to the point $(-4+1/50,2-3/100)$.}
    \label{fig:feasible_areas_blocked_3}
\end{figure}
\begin{figure}[!htb]
    \centering
    \includegraphics[scale=1,page=21]{media/disk_edgeless.pdf}
    %\includesvg[width=0.75\textwidth]{media/feasible_areas_blocked_4.svg}
    \caption{The closest points to $D_{c_1}$ in $\F_{D_{c_1}}$ before moving $\bdisk{S_{c,1}^x}$. When $\bdisk{S_{c,1}^x}$ is moved to the point $(-4\pm 1/50, 2-3/100)$, both points are contained in $\B_{\bdisk{S_{c,1}^x}}$ and $\F_{D_{c_1}}$ becomes $\emptyset$.}
    \label{fig:feasible_areas_blocked_4}
\end{figure}    



Lastly, we show that if $S_x$ is moved to $\F_{S_x}^{(2)}$ (resp. $\F_{S_x}^{(1)}$), then there exist three spaces for moving $D_{c_i}$ to the true side for $i\in \set{1,2,3}$ (resp. the false side for $i\in \set{4,5,6}$).
\begin{mlemmarep}\label{lem:feasible_areas_available}
    If $S_x$ is moved to $\F_{S_x}^{(2)}$, then $\F_{D_{c_i}} \neq \emptyset$ for $i \in \set{1,2,3}$.
    Moreover, $\F_{\bdisk{S}^x_{t,1}} \neq \emptyset$.
\end{mlemmarep}

\begin{proof}
    Let $\G_{x}$ be the variable gadget of an arbitrary variable $x$.
    We show that the disks in $\G_{x}$ can be moved such that $\F_{D_{c_i}} \neq \emptyset$ for $i \in \set{1,2,3}$.
    Recall that the disks $H_1^{(1)},\ldots,H_5^{(1)}$ are the five $6$-heavy disks surrounding $\S^x_{t,1}$.
    First, we have $\S^x_{t,1} \setminus \cup_{j=1}^5 \B_{H_j^{(1)}} \neq \emptyset$ since $S_x$ was moved to $\F_{S_x}^{(2)}$.
    Moreover, $d_{\bdisk{S^x_{t,1}}}(s^x_{t,1}) = K$ holds, thus we move $\bdisk{S^x_{t,1}}$ to $s^x_{t,1}$.
    This implies $\F_{\bdisk{S}^x_{t,1}} \neq \emptyset$.
    The intersection between $\bdisk{S^x_{t,1}}$ and disks $\ldisk{S^x_{c,1}},\ldisk{S^x_{c,3}}$ is removed, so these disks can remain unmoved as well as $\ldisk{S^x_{c,2}}$.
    Notice that $s^x_{c,i}\notin \B_{\bdisk{S^x_{c,i}}}$ holds for $i\in \set{1,2,3}$.
    Moreover, $s^x_{c,i}$ is not contained in the zones blocked by heavy disks surrounding $\S^x_{c,i}$ by \Cref{obs:vertical_condition,obs:horizontal_condition}.
    We also know that $s^x_{c,i} \in \F_{D_{c_i}}$ since $d_{D_{c_i}}(s^x_{c,i}) \le K$.
    Therefore $\F_{D_{c_i}} \neq \emptyset$ for $i \in \set{1,2,3}$.
\end{proof}

\subsubsection{Clause Components}
%%DEFINITIONS
\Cref{lem:cg_two_subsets,lem:tc_three_subsets,lem:block_disk_restricted,lem:feasible_areas_blocked} ensure that any undesired movement of the disks does not significantly alter the correctness of the reduction, whereas \Cref{lem:feasible_areas_available} provides a valid way to move the disks into the free slots of the variable gadget.
We are now ready to introduce the \emph{clause component} and show how the gadgets are connected to each other.

A \emph{clause component} $\G^c_{i,j,k} \subseteq \cstate$ is a collection of disks that represent the clause gadget $\G_c$ for a clause $c$ formed by variables $x_i,x_j,x_k$ connected to three variable gadgets $\G_{x_i},\G_{x_j},\G_{x_k}$ by consecutive cell gadgets.
As we mentioned earlier, the gadgets are connected by using arms, as depicted in \Cref{fig:variable_gadget}.
Arms are also formed by consecutive cell gadgets, but there exist cells that do not follow this definition.
We call these cells \emph{irregular cell gadgets} and their holes \emph{irregular interior holes}.
\Cref{fig:irreg_interior_holes} shows the three irregular cell gadgets present in the arms with their interior holes.


%%%---fig:irreg_interior_holes
\begin{figure}[!htb]
    \centering
    \includegraphics[scale=1,page=30]{media/disk_edgeless.pdf}
    \caption{Top: Irregular cell gadgets of arms; Bottom: Irregular interior holes of their respective cell gadget. The marked points are the farthest pair of points for each interior hole.}
    \label{fig:irreg_interior_holes}
\end{figure}

We show that \Cref{lem:cg_two_subsets} can be extended to irregular cell gadgets.
%OBS 

%%PROPERTIES
\begin{mlemmarep}\label{lem:irreg_holes_one_disk}
    The interior hole $\H_{(x,y)}$ of an arbitrary irregular cell gadget $\G_{(x,y)}$ of an arm admits exactly one disk centred within it.
\end{mlemmarep}
\begin{proof}
    We prove the statement for each irregular cell gadget.
    Let $\G_{(x,y)}$ be the irregular cell gadget in \Cref{fig:irreg_interior_holes}(a).
    The disks $H_1,\ldots,H_6$ surrounding $D$ are $6$-heavy disks such that $c(H_1) = (x-1/2,y+1)$, $c(H_2) = (x+1/2,y+1)$, $c(H_3) = (x+1,y)$, $c(H_4) = (x+1/2,y-1)$, $c(H_5) = (x-1/2,y-1)$ and $c(H_6) = (x-1,y)$.
    Let $\S$ be the convex polygon containing $\H_{(x,y)}$ defined by the intersection points of the boundaries of the blocked zones contained in $\H_{(x,y)}$.
    We check that $\S$ satisfies $\diam{S} < 1$.
    To aim for simplicity, we only give the farthest pair of points of $\S$.
    The coordinates of the cell gadget can be checked in \Cref{apx:coordinates}.
    Let $p,p' \in \H_{(x,y)}$ be the intersection points of boundaries of $(\B_{H_1},\B_{H_2})$ and $(\B_{H_4},\B_{H_5})$, respectively.
    It can be checked that the farthest pair of points of $\S$ is $p$ and $p'$, for which $\lVert p,p'\rVert_2$ equals $0.304\ldots < 0.31$.
    % for H1 - H2 x = 0 ∧ y≈0.152065
    %   sqrt( ( (x) - (-1/2) )^2 + ( (y) - (1) )^2) = 63/64
    %   sqrt( ( (x) - (1/2) )^2 + ( (y) - (1) )^2) = 63/64
    %for H4 - H5 x = 0 ∧ y≈-0.152065
    %   sqrt( ( (x) - (1/2) )^2 + ( (y) - (-1) )^2) = 63/64
    %   sqrt( ( (x) - (-1/2) )^2 + ( (y) - (-1) )^2) = 63/64
    % distance between (0,0.152065) and (0,-0.152065) = 0.30413
    That is, $\diam{\S} < 1$.
    By \Cref{obs:convex_polygon_one_disk}, $\S$ admits exactly one disk centred within it.
    Consequently, $\H_{(x,y)}$ also admits exactly one disk centred within it, since $\H_{(x,y)} = \S \setminus (\B_{H_1} \cup \cdots \cup \B_{H_6})$.
    
    Let $\G_{(x,y)}$ be the irregular cell gadget in \Cref{fig:irreg_interior_holes}(b) and $\S$ the convex polygon for $\H_{(x,y)}$.
    Let $p,p' \in \H_{(x,y)}$ be the intersection points of boundaries of $(\B_{H_2},\B_{H_4})$ and $(\B_{H_6},\B_{H_8})$, respectively.
    It can be checked that the farthest pair of points of $\S$ is $p$ and $p'$, for which $\lVert p,p'\rVert_2$ equals $0.101\ldots < 0.11$.
    % for H2 - H4 x≈0.267028 ∧ y≈0.0525348  at origin
    %   sqrt( ( (x) - (0) )^2 + ( (y) - (1) )^2) = 63/64
    %   sqrt( ( (x) - (1+1/4) )^2 + ( (y) - (0) )^2) = 63/64
    %for H6 - H8 x≈0.233405 ∧ y≈-0.0436965 at origin
    %   sqrt( ( (x) - (0) )^2 + ( (y) - (-1) )^2) = 63/64
    %   sqrt( ( (x) - (-3/4) )^2 + ( (y) - (0) )^2) = 63/64
    % distance between (0.267028,0.0525348) and (0.233405,-0.0436965) = 0.101936
    Hence, $\S$ admits exactly one disk centred within it by \Cref{obs:convex_polygon_one_disk}.
    Consequently, $\H_{(x,y)}$ also admits exactly one disk centred within it.

    Lastly, $\G_{(x,y)}$ be the irregular cell gadget in \Cref{fig:irreg_interior_holes}(c) and $\S$ the convex polygon for $\H_{(x,y)}$.
    Let $p,p' \in \H_{(x,y)}$ be the intersection points of boundaries of $(\B_{H_3},\B_{H_{4}})$ and $(\B_{H_6},\B_{H_8)}$, respectively.
    It can be checked that the farthest pair of points of $\S$ is $p$ and $p'$, for which $\lVert p,p'\rVert_2$ equals $0.562\ldots < 0.57$.
    % for H3 - H4 x≈0.525296 ∧ y≈0.137648  at origin
    %   sqrt( ( (x) - (1) )^2 + ( (y) - (1) )^2) = 63/64
    %   sqrt( ( (x) - (3/2) )^2 + ( (y) - (0) )^2) = 63/64
    %for H6 - H8 x≈-0.015751 ∧ y≈-0.015751 at origin
    %   sqrt( ( (x) - (0) )^2 + ( (y) - (-1) )^2) = 63/64
    %   sqrt( ( (x) - (-1) )^2 + ( (y) - (0) )^2) = 63/64
    % distance between (0.525296,0.137648) and (-0.015751,-0.015751) = 0.562373
    Hence, $\S$ admits exactly one disk centred within it by \Cref{obs:convex_polygon_one_disk}.
    Consequently, $\H_{(x,y)}$ also admits exactly one disk centred within it.

    Notice that the irregular cell gadgets in \Cref{fig:interior_holes} are present in the arm for $D_{c_3}$ (see \Cref{fig:variable_gadget}).
    The rest arms contains the gadget of \Cref{fig:interior_holes}(b) rotated.
    Therefore, all irregular gadgets admits one centre of disk within them.
\end{proof}

Observe that even if a cell gadget is irregular, the interior holes can be used as for normal cell gadgets.
This allows us to define \Cref{cor:cg_two_subsets}, which is a slight extension of \Cref{lem:cg_two_subsets} to characterise irregular cell gadgets.

\begin{corollary}\label{cor:cg_two_subsets}
    Let $\G_{(x,y)}$ and $\G_{(x',y')}$ be two (possibly irregular) cell gadgets with transition disks $D$ and $D'$, respectively.
    If $D$ and $D'$ are consecutive, then the feasible area $\F_{D}$ is equal to $(A_D \cap \H_{(x,y)}) \cup (A_{D} \cap \H_{(x',y')})$.
    Moreover, $A_D \cap \H_{(x,y)}$ and $A_{D'} \cap \H_{(x',y')}$ are non-empty and disjoint.
\end{corollary}

For any variable $x \in \set{x_i,x_j,x_k}$, when $x$ appears as a positive (negative) literal in $c$, $\G_{x}$ is connected to the true (false) side of $\G_c$.
\Cref{fig:component}(c) shows an example of a clause gadget for clause $c = (\overline{x_1} \lor x_2 \lor x_3)$.

%%%---fig:component
\begin{figure}[!htb]
    \centering
    \includegraphics[scale=1,page=24]{media/disk_edgeless.pdf}
    %\includesvg[width=0.95\textwidth]{media/clause_component.svg}
    \caption{Example of a clause component for clause $c = (\overline{x_1} \lor x_2 \lor x_3)$.}
    \label{fig:component}
\end{figure}

We say \emph{removing the intersection of} $\G^c_{i,j,k}$ to refer to moving the intersection disk of $\G_c$ so that a free slot of $\G_x$ for $x\in \set{x_i,x_j,x_k}$ is occupied, under the condition that the minimum maximum moving distance is $K$.
Removing the intersection of $\G_c$ using a free slot of $\G_x$ is equivalent to assigning a truth value to $x$ that satisfies $c$.
\Cref{lem:clause_satisfiable_md_k} formalises the idea of removing an intersection.

\begin{mlemmarep}\label{lem:clause_satisfiable_md_k}
    Given an arbitrary clause $c$ and its clause component $\G^c_{i,j,k} \subseteq \istate$, $c$ is satisfiable if and only if removing the intersection of $\G^c_{i,j,k}$ can be done with minimum maximum moving distance $K$ for the $L_1$ and $L_2$ distances.
\end{mlemmarep}

\begin{proof}
    Without loss of generality, suppose that $x_i$ appears as a positive literal in $c$ and $c$ is satisfied by $x_i$.
    We show that $\mstatefinal{\G^c_{i,j,k}}$ without intersections can be obtained by moving disks with minimum maximum moving distance $K$.
    Let $\{D_1,\ldots,D_k\}$ be the collection of consecutive transition disks that connect $\G_c$ and $\G_{x_i}$ such that $D_1 = T_c$ and $D_k = D_{c_1}$.
    Since $x_i$ satisfies $c$ when assigned to true, $S_{x_i}$ is moved to $\F_{S_x}^{(2)}$.
    By \Cref{lem:feasible_areas_available}, we have $\F_{D_{c_1}}\neq \emptyset$.
    Thus, we move $D_k$ to $s^{x_i}_{c,1}$ with moving distance $K$.
    Since $\H_{c(D_k)} \setminus \B_{D_k} = \H_{c(D_k)}$,
    %Since there is no disk at the previous position of $D_k$, 
    $D_{k-1}$ is moved to $c(D_k)$ with moving distance $K$.
    This procedure is repeated until $D_1 = T_c$ is moved to the previous position of $D_2$, which is a consecutive transition disk to $T_c$ in the clause gadget.
    \Cref{cor:cg_two_subsets} and \Cref{lem:tc_three_subsets} ensure that the procedure can be performed.
    Moreover, the procedure meets the conditions of \Cref{lem:holes_one_disk,lem:irreg_holes_one_disk}.
    The disks left in $\G^c_{i,j,k}$ are moved to their centres and consequently $\mstatefinal{\G^c_{i,j,k}}$ does not contain intersections.
    Therefore removing the intersection of $\G_c$ can be done with minimum maximum moving distance $K$.

    In the other direction, assume that removing the intersection of $\G^c_{i,j,k}$ can be done with minimum maximum moving distance of $K$.
    We show that the removal is equal to an assignment of variables that satisfies $c$.
    By \Cref{lem:tc_three_subsets}, $T_c$ must be moved to one of the three positions of its consecutive transition disks in $\G_c$.
    Without loss of generality, suppose that it was moved to the arm connected to the true side of $\G_{x_i}$ through a set $\{D_1,\ldots,D_k\}$ of consecutive transition disks as described before.
    Now $\movedpos{T_c} \in \H_{c(D_2)}$ holds, so $D_2$ must be moved outside $\H_{c(D_2)}$.
    By \Cref{lem:cg_two_subsets}, $\F_{D_2}$ is only the non-empty intersection of $\A_{D_2}$ and the interior hole $\H_{c(D_3)}$, as $\H_{c(D_2)} \setminus \B_{T_c} = \emptyset$.
    The same reasoning can be applied to disks $D_{3},\ldots,D_{k}$ by \Cref{lem:cg_two_subsets,cor:cg_two_subsets}.
    Assume that $D_k = D_{c_1}$.
    The feasible area of $D_k$ consists only of the non-empty intersection of $\A_{D_k}$ and the area of $\S^{x_i}_{c,1}$ that does not intersect blocked zones.
    Hence $D_k$ is moved to $s^{x_i}_{c,1}$.
    Since $\G_c$ is connected to the true side of $\G_{x_i}$, $S_{x_i}$ must be moved to $\F_{S_x}^{(2)}$ by \Cref{lem:feasible_areas_blocked} and $x_i$ appears as a positive literal in $c$ by the definition of clause gadgets.
    Therefore, $c$ is satisfied by an assignment of variables in which $x_i = 1$.
\end{proof}
%%%%%%%%%%%%%%%%%%%%%%%%%%%%%%%%%%%%%%%%%%%%%%%%%%%%

\subsubsection{Reduction Correctness and Running Time}

We are now ready to characterise the reduction from {\pthreesat}. We first show the correctness of the reduction and then show that it can be obtained in polynomial time.

\begin{mlemmarep}\label{lem:3sat_edgeless_equiv}
    Given an instance $(\Phi,G_\Phi)$ of {\pthreesat}, the minimum maximum moving distance for satisfying $\Pi_{\texttt{edgeless}}$ in $\cstate$ is at most $K$ if and only if $\Phi$ is satisfiable.
\end{mlemmarep}

\begin{proof}
    Assume first that $\Phi$ is satisfiable by values $t_1,\ldots,t_n$ and the number of clauses is $m$.
    The collection of disks $\cstate$ can be partitioned into $m$ clause components that possibly share variable gadgets.
    By \Cref{lem:clause_satisfiable_md_k}, removing the intersections of all clause gadgets in $\cstate$ can be done with minimum maximum moving distance $K$.
    In particular, for each $t_i$, $S_{x_i}$ is moved to $\F_{S_{x_i}}^{(1)}$ if $t_i = 0$ and $\F_{S_{x_i}}^{(2)}$ otherwise.
    \Cref{lem:clause_satisfiable_md_k} ensures that by moving the truth setter disks in this way allows moving the intersection disk of the $m$ clauses to a free space of variable gadgets
    That is, it produces a $\mstatefinal{\D}$ without intersections.
    Therefore $\Pi_{\texttt{edgeless}}$ is satisfied with minimum maximum moving distance $K$.

    In the other direction, assume that $\Pi_{\texttt{edgeless}}$ is satisfied with minimum maximum moving distance $K$ in $\cstate$.
    That is, $\mstatefinal{\D}$ can be constructed by moving disks with minimum maximum moving distance $K$.
    By \Cref{lem:clause_satisfiable_md_k}, it implies that each clause is satisfied by an assignment of a variable of $\Phi$.
    Let $c,c' \in \Phi$ be two arbitrary clauses and $x$ be a variable such that $x$ appears in $c$ and $c'$.
    If $x$ appears in both variables as a positive (negative) literal, $\G_c$ and $\G_{c'}$ are connected to the true (false) side of $\G_{x}$ and \Cref{lem:feasible_areas_blocked,lem:feasible_areas_available} ensure that removing the intersection of both clause gadgets is given by the movement of $S_x$.
    Suppose instead that $x$ appears as a true literal in $c$ and as a false literal in $c'$.
    The gadgets $\G_c$ and $\G_{c'}$ are connected to the true and false side of $\G_x$, respectively.
    Again, \Cref{lem:feasible_areas_blocked,lem:feasible_areas_available} ensure that removing the intersection of $\G_{c}$ using $\G_x$ is not possible whenever the intersection of $\G_{c'}$ is removed using $\G_{x}$, or vice versa.
    By the definition of clause gadgets, it means that satisfying clauses $c$ and $c'$ by a variable $x$ such that $x$ appears as a true literal in $c$ and as a false literal in $c'$ (or the contrary) at the same time is not possible.
    That is, the side not blocked by $S_x$ decides the truth assignment of $x$.
    Consequently, the movement of the truth setter disks of the variable gadgets in $\cstate$ describes a feasible solution for satisfying $\Phi$.
    This concludes the proof.
\end{proof}

The last lemma shows that the construction of the instance can be done in polynomial time.

\begin{mlemmarep}\label{lem:reduction_poly_time}
    Given an instance $(\Phi,G_{\Phi})$ of {\pthreesat} with $n$ variables and $m$ clauses, the instance ($\cstate,K$) of {\ggedmm} can be obtained in $\poly(\eta)$ where $\eta = f(n,m)$.
\end{mlemmarep}

\begin{proof}
    By \Cref{lem:3sat_edgeless_equiv}, there exists a reduction from {\pthreesat} that converts an instance $\Phi$ to a collection of disks $\cstate$ such that $\Phi$ is satisfiable if and only if the minimum maximum moving distance for satisfying $\Pi_{\texttt{edgeless}}$ in $\cstate$ is $K$.
    Thus we only need to prove that $\cstate$ can be obtained in polynomial time from $\Phi$ and $G_{\Phi}$.
    An arbitrary $\cstate$ consists of horizontally aligned variable gadgets with clause gadgets vertically connected to them from up and down.
    The positions and connections of clause components are given by the representation of $G_{\Phi}$.
    All disks are in a unique position and do not intersect any other disk except for the $n + m$ intersection disks and the $n$ blocking disks intersecting link disks.
    We define a grid $\G$ of area $A_{\G} = W_{\G}H_{\G}$ that contains $\cstate$, where $W_{\G}$ and $H_{\G}$ denote the width and height of $\G$ expressed by number of disks, respectively. We show that $\eta = f(n,m)$ polynomially by the size of the input $n+m$.
    
    The maximum area occupied by a variable gadget is a constant value, denoted by $A_v = W_vH_v$.
    By the definition of the variable gadget, $W_v \le 35$ and $H_v \le 19$ hold.
    We add a separation of $10$ between variable gadgets. Consequently, $W_{\G} \le n(W_v+S)\le 35n\times 10$ holds. %, where the value $10$ is the separation we add between gadgets.
    %The height of the grid is limited by $m$ given that the maximum number of clause components `enclosed' by an arbitrary clause component is $m-1$.
    A clause gadget $c$ occupies an area of $A_g = W_gH_g = 9\times 6$. %, so $B$ is a constant value. 
    The number of `nested' clause components in $\G$ is limited by $m$.
    If variable gadgets are aligned at $y = 0$, we locate clause gadgets at $y=20$ for non-nested components and $y = 20+9m'$ for components enclosing $m'-1$ components, where $m' < m$.
    Consequently, $H_{\G}$ is bounded by $2(20+9m)$ and hence $A_{\G} \le 350n \times (40+18m)$.
    %In other words, $H_{\G} \le 2Bm$, where $B$ is a constant value that describes the vertical space necessary to connect the clause gadget to variables.
    This implies that the number of disks required to construct $\cstate$ is also polynomially limited by $n$ and $m$.
    That is, $\eta = f(n,m)$ describes a polynomial over $n$ and $m$.
    Therefore, $\cstate$ can be constructed in polynomial time.
\end{proof}

With \Cref{lem:reduction_poly_time}, we restate \Cref{thm:edgeless_np_hard} below to conclude this section.
We remark that the strongly \NP-hardness comes from assuming $K=1$, which implies that the distance weights used are constant values. Therefore the values of the instance can be bounded by the input size.

\edgelessNPHard*
%\begin{reptheorem}{thm:edgeless_np_hard}
%    \Paste{edgeless_np_hard}
%\end{reptheorem}

\end{toappendix}

%\begin{toappendix}
%    \input{tex/apx_misc_edg_disk}
%\end{toappendix}

\ifConf
    \begin{toappendix}
        %\begin{subappendices}
%\renewcommand{\thesection}{\arabic{section}}%
\clearpage
\section{Coordinates of gadgets}\label{apx:coordinates}

\begin{table}[!bht]
\caption{Coordinates for a cell gadget $\G_{(0,0)}$}
\label{tab:cell_gadget_coords}
\centering
\maxsizebox{\textwidth}{\textheight}{%
\begin{tabular}{@{}ccccccccccccccc@{}}
\toprule
\multicolumn{15}{l}{\textbf{Cell Gadget}} \\ \midrule
\multirow{2}{*}{\textbf{$x$}} & \multirow{2}{*}{\textbf{$y$}} & \multirow{2}{*}{\textbf{Type}} & \multirow{2}{*}{\textbf{$x$}} & \multirow{2}{*}{\textbf{$y$}} & \multirow{2}{*}{\textbf{Type}} & \multirow{2}{*}{\textbf{$x$}} & \multirow{2}{*}{\textbf{$y$}} & \multirow{2}{*}{\textbf{Type}} & \multirow{2}{*}{\textbf{$x$}} & \multirow{2}{*}{\textbf{$y$}} & \multirow{2}{*}{\textbf{Type}} & \multirow{2}{*}{\textbf{$x$}} & \multirow{2}{*}{\textbf{$y$}} & \multirow{2}{*}{\textbf{Type}} \\
 &  &  &  &  &  &  &  &  &  &  &  &  &  &  \\
$-1$ & $1$ & \multicolumn{1}{c|}{$6$-heavy} & $0$ & $1$ & \multicolumn{1}{c|}{$6$-heavy} & $1$ & $1$ & \multicolumn{1}{c|}{$6$-heavy} & $-1$ & $0$ & \multicolumn{1}{c|}{$6$-heavy} & $0$ & $0$ & Transition \\
$1$ & $0$ & \multicolumn{1}{c|}{$6$-heavy} & $-1$ & $-1$ & \multicolumn{1}{c|}{$6$-heavy} & $0$ & $-1$ & \multicolumn{1}{c|}{$6$-heavy} & $1$ & $-1$ & \multicolumn{1}{c|}{$6$-heavy} &  &  &  \\ \bottomrule
\end{tabular}%
}
\end{table}

\begin{table}[!bht]
\caption{Coordinates for a clause gadget $\G_{c}$ such that $c(T_c) = (0,0)$.}
\label{tab:clause_gadget_coords}
\centering
\maxsizebox{\textwidth}{\textheight}{%
\begin{tabular}{@{}ccccccccccccccc@{}}
\toprule
\multicolumn{15}{l}{\textbf{Clause Gadget}} \\ \midrule
\multirow{2}{*}{\textbf{$x$}} & \multirow{2}{*}{\textbf{$y$}} & \multirow{2}{*}{\textbf{Type}} & \multirow{2}{*}{\textbf{$x$}} & \multirow{2}{*}{\textbf{$y$}} & \multirow{2}{*}{\textbf{Type}} & \multirow{2}{*}{\textbf{$x$}} & \multirow{2}{*}{\textbf{$y$}} & \multirow{2}{*}{\textbf{Type}} & \multirow{2}{*}{\textbf{$x$}} & \multirow{2}{*}{\textbf{$y$}} & \multirow{2}{*}{\textbf{Type}} & \multirow{2}{*}{\textbf{$x$}} & \multirow{2}{*}{\textbf{$y$}} & \multirow{2}{*}{\textbf{Type}} \\
 &  &  &  &  &  &  &  &  &  &  &  &  &  &  \\
$1$ & $1$ & \multicolumn{1}{c|}{$6$-heavy} & $2$ & $1$ & \multicolumn{1}{c|}{$6$-heavy} & $3$ & $1$ & \multicolumn{1}{c|}{$6$-heavy} & $4$ & $1$ & \multicolumn{1}{c|}{$6$-heavy} & $-4$ & $0$ & $6$-heavy \\
$-3$ & $0$ & \multicolumn{1}{c|}{Transition} & $-2$ & $0$ & \multicolumn{1}{c|}{$6$-heavy} & $-1$ & $0$ & \multicolumn{1}{c|}{$6$-heavy} & $0$ & $0$ & \multicolumn{1}{c|}{Transition} & $0$ & $0$ & $6$-heavy \\
$1$ & $0$ & \multicolumn{1}{c|}{$6$-heavy} & $2$ & $0$ & \multicolumn{1}{c|}{$6$-heavy} & $3$ & $0$ & \multicolumn{1}{c|}{Transition} & $4$ & $0$ & \multicolumn{1}{c|}{$6$-heavy} & $-4$ & $-1$ & $6$-heavy \\
$-3$ & $-1$ & \multicolumn{1}{c|}{$6$-heavy} & $-2$ & $-1$ & \multicolumn{1}{c|}{$6$-heavy} & $-1$ & $-1$ & \multicolumn{1}{c|}{$6$-heavy} & $0$ & $-1$ & \multicolumn{1}{c|}{$6$-heavy} & $1$ & $-1$ & $6$-heavy \\
$2$ & $-1$ & \multicolumn{1}{c|}{$6$-heavy} & $3$ & $-1$ & \multicolumn{1}{c|}{$6$-heavy} & $4$ & $-1$ & \multicolumn{1}{c|}{$6$-heavy} & $-1$ & $-2$ & \multicolumn{1}{c|}{$6$-heavy} & $0$ & $-2$ & $6$-heavy \\
$1$ & $-2$ & \multicolumn{1}{c|}{$6$-heavy} & $-1$ & $-3$ & \multicolumn{1}{c|}{$6$-heavy} & $0$ & $-3$ & \multicolumn{1}{c|}{Transition} & $1$ & $-3$ & \multicolumn{1}{c|}{$6$-heavy} & $-1$ & $-4$ & $6$-heavy \\
$0$ & $-4$ & \multicolumn{1}{c|}{$6$-heavy} & $1$ & $-4$ & \multicolumn{1}{c|}{$6$-heavy} &  &  & \multicolumn{1}{c|}{} &  &  & \multicolumn{1}{c|}{} &  &  &  \\ \bottomrule
\end{tabular}%
}
\end{table}

\begin{table}[!bht]
\caption{Coordinates for the central part of a variable gadget $\G_{x}$ such that $c(S_x) = (0,0)$.}
\label{tab:variable_gadget_central_coords}
\centering
\maxsizebox{\textwidth}{\textheight}{%
\begin{tabular}{@{}ccccccccccccccc@{}}
\toprule
\multicolumn{15}{l}{\textbf{Variable Gadget (central part)}} \\ \midrule
\multirow{2}{*}{\textbf{$x$}} & \multirow{2}{*}{\textbf{$y$}} & \multirow{2}{*}{\textbf{Type}} & \multirow{2}{*}{\textbf{$x$}} & \multirow{2}{*}{\textbf{$y$}} & \multirow{2}{*}{\textbf{Type}} & \multirow{2}{*}{\textbf{$x$}} & \multirow{2}{*}{\textbf{$y$}} & \multirow{2}{*}{\textbf{Type}} & \multirow{2}{*}{\textbf{$x$}} & \multirow{2}{*}{\textbf{$y$}} & \multirow{2}{*}{\textbf{Type}} & \multirow{2}{*}{\textbf{$x$}} & \multirow{2}{*}{\textbf{$y$}} & \multirow{2}{*}{\textbf{Type}} \\
 &  &  &  &  &  &  &  &  &  &  &  &  &  &  \\
$4$ & $3+3/4$ & \multicolumn{1}{c|}{$6$-heavy} & $-5$ & $3$ & \multicolumn{1}{c|}{$6$-heavy} & $-3$ & $3$ & \multicolumn{1}{c|}{$6$-heavy} & $3$ & $3$ & \multicolumn{1}{c|}{$6$-heavy} & $5$ & $3$ & $6$-heavy \\
$-4$ & $2+1/4$ & \multicolumn{1}{c|}{$1$-heavy} & $4$ & $2+1/4$ & \multicolumn{1}{c|}{$1$-heavy} & $-5$ & $2$ & \multicolumn{1}{c|}{$6$-heavy} & $-3$ & $2$ & \multicolumn{1}{c|}{$6$-heavy} & $-2$ & $2$ & $6$-heavy \\
$-1$ & $2$ & \multicolumn{1}{c|}{$6$-heavy} & $0$ & $2$ & \multicolumn{1}{c|}{$6$-heavy} & $1$ & $2$ & \multicolumn{1}{c|}{$6$-heavy} & $2$ & $2$ & \multicolumn{1}{c|}{$6$-heavy} & $3$ & $2$ & $6$-heavy \\
$5$ & $2$ & \multicolumn{1}{c|}{$6$-heavy} & $-8$ & $1$ & \multicolumn{1}{c|}{$6$-heavy} & $-7$ & $1$ & \multicolumn{1}{c|}{$6$-heavy} & $-6$ & $1$ & \multicolumn{1}{c|}{$6$-heavy} & $-5$ & $1$ & $6$-heavy \\
$-3$ & $1$ & \multicolumn{1}{c|}{$6$-heavy} & $-2$ & $1$ & \multicolumn{1}{c|}{$6$-heavy} & $-1$ & $1$ & \multicolumn{1}{c|}{$6$-heavy} & $0$ & $1$ & \multicolumn{1}{c|}{$6$-heavy} & $1$ & $1$ & $6$-heavy \\
$2$ & $1$ & \multicolumn{1}{c|}{$6$-heavy} & $3$ & $1$ & \multicolumn{1}{c|}{$6$-heavy} & $5$ & $1$ & \multicolumn{1}{c|}{$6$-heavy} & $6$ & $1$ & \multicolumn{1}{c|}{$6$-heavy} & $7$ & $1$ & $6$-heavy \\
$8$ & $1$ & \multicolumn{1}{c|}{$6$-heavy} & $-4$ & $3/4$ & \multicolumn{1}{c|}{$2$-heavy} & $4$ & $3/4$ & \multicolumn{1}{c|}{$2$-heavy} & $-7-3/4$ & $0$ & \multicolumn{1}{c|}{$6$-heavy} & $-6$ & $0$ & $1$-heavy \\
$-4-3/4$ & $0$ & \multicolumn{1}{c|}{$2$-heavy} & $-3-1/2$ & $0$ & \multicolumn{1}{c|}{$1$-heavy} & $-2$ & $0$ & \multicolumn{1}{c|}{$6$-heavy} & $-1$ & $0$ & \multicolumn{1}{c|}{$6$-heavy} & $0$ & $0$ & Transition \\
$0$ & $0$ & \multicolumn{1}{c|}{$6$-heavy} & $1$ & $0$ & \multicolumn{1}{c|}{$6$-heavy} & $2$ & $0$ & \multicolumn{1}{c|}{$6$-heavy} & $3+1/2$ & $0$ & \multicolumn{1}{c|}{$1$-heavy} & $4+3/4$ & $0$ & $2$-heavy \\
$6$ & $0$ & \multicolumn{1}{c|}{$1$-heavy} & $7+3/4$ & $0$ & \multicolumn{1}{c|}{$6$-heavy} & $-4$ & $-3/4$ & \multicolumn{1}{c|}{$2$-heavy} & $4$ & $-3/4$ & \multicolumn{1}{c|}{$2$-heavy} & $-8$ & $-1$ & $6$-heavy \\
$-7$ & $-1$ & \multicolumn{1}{c|}{$6$-heavy} & $-6$ & $-1$ & \multicolumn{1}{c|}{$6$-heavy} & $-5$ & $-1$ & \multicolumn{1}{c|}{$6$-heavy} & $-3$ & $-1$ & \multicolumn{1}{c|}{$6$-heavy} & $-2$ & $-1$ & $6$-heavy \\
$-1$ & $-1$ & \multicolumn{1}{c|}{$6$-heavy} & $0$ & $-1$ & \multicolumn{1}{c|}{$6$-heavy} & $1$ & $-1$ & \multicolumn{1}{c|}{$6$-heavy} & $2$ & $-1$ & \multicolumn{1}{c|}{$6$-heavy} & $3$ & $-1$ & $6$-heavy \\
$5$ & $-1$ & \multicolumn{1}{c|}{$6$-heavy} & $6$ & $-1$ & \multicolumn{1}{c|}{$6$-heavy} & $7$ & $-1$ & \multicolumn{1}{c|}{$6$-heavy} & $8$ & $-1$ & \multicolumn{1}{c|}{$6$-heavy} & $-5$ & $-2$ & $6$-heavy \\
$-3$ & $-2$ & \multicolumn{1}{c|}{$6$-heavy} & $-2$ & $-2$ & \multicolumn{1}{c|}{$6$-heavy} & $-1$ & $-2$ & \multicolumn{1}{c|}{$6$-heavy} & $0$ & $-2$ & \multicolumn{1}{c|}{$6$-heavy} & $1$ & $-2$ & $6$-heavy \\
$2$ & $-2$ & \multicolumn{1}{c|}{$6$-heavy} & $3$ & $-2$ & \multicolumn{1}{c|}{$6$-heavy} & $5$ & $-2$ & \multicolumn{1}{c|}{$6$-heavy} & $-4$ & $-2-1/4$ & \multicolumn{1}{c|}{$1$-heavy} & $4$ & $-2-1/4$ & $1$-heavy \\
$-5$ & $-3$ & \multicolumn{1}{c|}{$6$-heavy} & $-3$ & $-3$ & \multicolumn{1}{c|}{$6$-heavy} & $3$ & $-3$ & \multicolumn{1}{c|}{$6$-heavy} & $5$ & $-3$ & \multicolumn{1}{c|}{$6$-heavy} & $-4$ & $-3-3/4$ & $6$-heavy \\
$4$ & $-3-3/4$ & \multicolumn{1}{c|}{$6$-heavy} & $-5$ & $-4$ & \multicolumn{1}{c|}{$6$-heavy} & $-3$ & $-4$ & \multicolumn{1}{c|}{$6$-heavy} & $3$ & $-4$ & \multicolumn{1}{c|}{$6$-heavy} & $5$ & $-4$ & $6$-heavy \\ \bottomrule
\end{tabular}%
}
\end{table}


\begin{table}[!bht]
\caption{Coordinates of arm 1 for $D_{c_1}$ when $S_x$ is centred at $(0,0)$.}
\label{tab:variable_gadget_arm1_coords}
\centering
\maxsizebox{\textwidth}{\textheight}{%
\begin{tabular}{@{}ccccccccccccccc@{}}
\toprule
\multicolumn{15}{l}{\textbf{Arm 1 (when $S_x$ centred at $(0,0)$)}} \\ \midrule
\multirow{2}{*}{\textbf{$x$}} & \multirow{2}{*}{\textbf{$y$}} & \multirow{2}{*}{\textbf{Disk Type}} & \multirow{2}{*}{\textbf{$x$}} & \multirow{2}{*}{\textbf{$y$}} & \multirow{2}{*}{\textbf{Disk Type}} & \multirow{2}{*}{\textbf{$x$}} & \multirow{2}{*}{\textbf{$y$}} & \multirow{2}{*}{\textbf{Disk Type}} & \multirow{2}{*}{\textbf{$x$}} & \multirow{2}{*}{\textbf{$y$}} & \multirow{2}{*}{\textbf{Disk Type}} & \multirow{2}{*}{\textbf{$x$}} & \multirow{2}{*}{\textbf{$y$}} & \multirow{2}{*}{\textbf{Disk Type}} \\
 &  &  &  &  &  &  &  &  &  &  &  &  &  &  \\
$-5$ & $9$ & \multicolumn{1}{c|}{$6$-heavy} & $-4$ & $9$ & \multicolumn{1}{c|}{$6$-heavy} & $-3$ & $9$ & \multicolumn{1}{c|}{$6$-heavy} & $-5$ & $8$ & \multicolumn{1}{c|}{$6$-heavy} & $-4$ & $8$ & Transition \\
$-3$ & $8$ & \multicolumn{1}{c|}{$6$-heavy} & $-5$ & $7$ & \multicolumn{1}{c|}{$6$-heavy} & $-4$ & $7$ & \multicolumn{1}{c|}{$6$-heavy} & $-3$ & $7$ & \multicolumn{1}{c|}{$6$-heavy} & $-5$ & $6$ & $6$-heavy \\
$-3$ & $6$ & \multicolumn{1}{c|}{$6$-heavy} & $-4$ & $5+3/4$ & \multicolumn{1}{c|}{Transition} & $-5$ & $5$ & \multicolumn{1}{c|}{$6$-heavy} & $-3$ & $5$ & \multicolumn{1}{c|}{$6$-heavy} & $-4$ & $4+3/4$ & $6$-heavy \\ \bottomrule
\end{tabular}%
}
\end{table}

\begin{table}[!bht]
\caption{Coordinates of arm for $D_{c_2}$ when $S_x$ is centred at $(0,0)$.}
\label{tab:variable_gadget_arm2_coords}
\centering
\maxsizebox{\textwidth}{\textheight}{%
\begin{tabular}{@{}ccccccccccccccc@{}}
\toprule
\multicolumn{15}{l}{\textbf{Arm 2 (when $S_x$ centred at $(0,0)$)}} \\ \midrule
\multirow{2}{*}{\textbf{$x$}} & \multirow{2}{*}{\textbf{$y$}} & \multirow{2}{*}{\textbf{Disk Type}} & \multirow{2}{*}{\textbf{$x$}} & \multirow{2}{*}{\textbf{$y$}} & \multirow{2}{*}{\textbf{Disk Type}} & \multirow{2}{*}{\textbf{$x$}} & \multirow{2}{*}{\textbf{$y$}} & \multirow{2}{*}{\textbf{Disk Type}} & \multirow{2}{*}{\textbf{$x$}} & \multirow{2}{*}{\textbf{$y$}} & \multirow{2}{*}{\textbf{Disk Type}} & \multirow{2}{*}{\textbf{$x$}} & \multirow{2}{*}{\textbf{$y$}} & \multirow{2}{*}{\textbf{Disk Type}} \\
 &  &  &  &  &  &  &  &  &  &  &  &  &  &  \\
$-9$ & $8$ & \multicolumn{1}{c|}{$6$-heavy} & $-11$ & $7$ & \multicolumn{1}{c|}{$6$-heavy} & $-10$ & $7$ & \multicolumn{1}{c|}{$6$-heavy} & $-9$ & $7$ & \multicolumn{1}{c|}{$6$-heavy} & $-11$ & $6$ & $6$-heavy \\
$-10$ & $6$ & \multicolumn{1}{c|}{$6$-heavy} & $-9$ & $6$ & \multicolumn{1}{c|}{$6$-heavy} & $-11$ & $5$ & \multicolumn{1}{c|}{$6$-heavy} & $-10$ & $5$ & \multicolumn{1}{c|}{Transition} & $-9$ & $5$ & $6$-heavy \\
$-11$ & $4$ & \multicolumn{1}{c|}{$6$-heavy} & $-10$ & $4$ & \multicolumn{1}{c|}{$6$-heavy} & $-9$ & $4$ & \multicolumn{1}{c|}{$6$-heavy} & $-11$ & $3$ & \multicolumn{1}{c|}{$6$-heavy} & $-10$ & $3$ & $6$-heavy \\
$-9$ & $3$ & \multicolumn{1}{c|}{$6$-heavy} & $-11$ & $2$ & \multicolumn{1}{c|}{$6$-heavy} & $-10$ & $2$ & \multicolumn{1}{c|}{Transition} & $-9$ & $2$ & \multicolumn{1}{c|}{$6$-heavy} & $-11$ & $1$ & $6$-heavy \\
$-10$ & $1$ & \multicolumn{1}{c|}{$6$-heavy} & $-9$ & $1$ & \multicolumn{1}{c|}{$6$-heavy} & $-11$ & $0$ & \multicolumn{1}{c|}{$6$-heavy} & $-9-3/4$ & $0$ & \multicolumn{1}{c|}{Transition} & $-8-3/4$ & $0$ & $6$-heavy \\
$-11$ & $-1$ & \multicolumn{1}{c|}{$6$-heavy} & $-10$ & $-1$ & \multicolumn{1}{c|}{$6$-heavy} & $-9$ & $-1$ & \multicolumn{1}{c|}{$6$-heavy} &  &  & \multicolumn{1}{c|}{} &  &  & \\\bottomrule
\end{tabular}%
}
\end{table}

\begin{table}[!bht]
\caption{Coordinates of arm for $D_{c_3}$ when $S_x$ is centred at $(0,0)$.}
\label{tab:variable_gadget_arm3_coords}
\centering
\maxsizebox{\textwidth}{\textheight}{%
\begin{tabular}{@{}ccccccccccccccc@{}}
\toprule
\multicolumn{15}{l}{\textbf{Arm 3 (when $S_x$ centred at $(0,0)$)}} \\ \midrule
\multirow{2}{*}{\textbf{$x$}} & \multirow{2}{*}{\textbf{$y$}} & \multirow{2}{*}{\textbf{Disk Type}} & \multirow{2}{*}{\textbf{$x$}} & \multirow{2}{*}{\textbf{$y$}} & \multirow{2}{*}{\textbf{Disk Type}} & \multirow{2}{*}{\textbf{$x$}} & \multirow{2}{*}{\textbf{$y$}} & \multirow{2}{*}{\textbf{Disk Type}} & \multirow{2}{*}{\textbf{$x$}} & \multirow{2}{*}{\textbf{$y$}} & \multirow{2}{*}{\textbf{Disk Type}} & \multirow{2}{*}{\textbf{$x$}} & \multirow{2}{*}{\textbf{$y$}} & \multirow{2}{*}{\textbf{Disk Type}} \\
 &  &  &  &  &  &  &  &  &  &  &  &  &  &  \\
$-15$ & $8$ & \multicolumn{1}{c|}{$6$-heavy} & $-17$ & $7$ & \multicolumn{1}{c|}{$6$-heavy} & $-16$ & $7$ & \multicolumn{1}{c|}{$6$-heavy} & $-15$ & $7$ & \multicolumn{1}{c|}{$6$-heavy} & $-17$ & $6$ & $6$-heavy \\
$-16$ & $6$ & \multicolumn{1}{c|}{$6$-heavy} & $-15$ & $6$ & \multicolumn{1}{c|}{$6$-heavy} & $-17$ & $5$ & \multicolumn{1}{c|}{$6$-heavy} & $-16$ & $5$ & \multicolumn{1}{c|}{Transition} & $-15$ & $5$ & $6$-heavy \\
$-17$ & $4$ & \multicolumn{1}{c|}{$6$-heavy} & $-16$ & $4$ & \multicolumn{1}{c|}{$6$-heavy} & $-15$ & $4$ & \multicolumn{1}{c|}{$6$-heavy} & $-17$ & $3$ & \multicolumn{1}{c|}{$6$-heavy} & $-16$ & $3$ & $6$-heavy \\
$-15$ & $3$ & \multicolumn{1}{c|}{$6$-heavy} & $-17$ & $2$ & \multicolumn{1}{c|}{$6$-heavy} & $-16$ & $2$ & \multicolumn{1}{c|}{Transition} & $-15$ & $2$ & \multicolumn{1}{c|}{$6$-heavy} & $-17$ & $1$ & $6$-heavy \\
$-16$ & $1$ & \multicolumn{1}{c|}{$6$-heavy} & $-15$ & $1$ & \multicolumn{1}{c|}{$6$-heavy} & $-17$ & $0$ & \multicolumn{1}{c|}{$6$-heavy} & $-16$ & $0$ & \multicolumn{1}{c|}{$6$-heavy} & $-15$ & $0$ & $6$-heavy \\
$-17$ & $-1$ & \multicolumn{1}{c|}{$6$-heavy} & $-16$ & $-1$ & \multicolumn{1}{c|}{Transition} & $-15$ & $-1$ & \multicolumn{1}{c|}{$6$-heavy} & $-17$ & $-2$ & \multicolumn{1}{c|}{$6$-heavy} & $-16$ & $-2$ & $6$-heavy \\
$-15$ & $-2$ & \multicolumn{1}{c|}{$6$-heavy} & $-17$ & $-3$ & \multicolumn{1}{c|}{$6$-heavy} & $-16$ & $-3$ & \multicolumn{1}{c|}{$6$-heavy} & $-15$ & $-3$ & \multicolumn{1}{c|}{$6$-heavy} & $-17$ & $-4$ & $6$-heavy \\
$-16$ & $-4$ & \multicolumn{1}{c|}{Transition} & $-15$ & $-4$ & \multicolumn{1}{c|}{$6$-heavy} & $-4$ & $-4-3/4$ & \multicolumn{1}{c|}{$6$-heavy} & $-17$ & $-5$ & \multicolumn{1}{c|}{$6$-heavy} & $-16$ & $-5$ & $6$-heavy \\
$-15$ & $-5$ & \multicolumn{1}{c|}{$6$-heavy} & $-14$ & $-5$ & \multicolumn{1}{c|}{$6$-heavy} & $-13$ & $-5$ & \multicolumn{1}{c|}{$6$-heavy} & $-12$ & $-5$ & \multicolumn{1}{c|}{$6$-heavy} & $-11$ & $-5$ & $6$-heavy \\
$-10$ & $-5$ & \multicolumn{1}{c|}{$6$-heavy} & $-9$ & $-5$ & \multicolumn{1}{c|}{$6$-heavy} & $-8$ & $-5$ & \multicolumn{1}{c|}{$6$-heavy} & $-7$ & $-5$ & \multicolumn{1}{c|}{$6$-heavy} & $-6$ & $-5$ & $6$-heavy \\
$-5$ & $-5$ & \multicolumn{1}{c|}{$6$-heavy} & $-3$ & $-5$ & \multicolumn{1}{c|}{$6$-heavy} & $-4$ & $-5-3/4$ & \multicolumn{1}{c|}{Transition} & $-17$ & $-6$ & \multicolumn{1}{c|}{$6$-heavy} & $-15-1/2$ & $-6$ & Transition \\
$-14-1/2$ & $-6$ & \multicolumn{1}{c|}{$6$-heavy} & $-13-1/2$ & $-6$ & \multicolumn{1}{c|}{$6$-heavy} & $-12-1/2$ & $-6$ & \multicolumn{1}{c|}{Transition} & $-11-1/2$ & $-6$ & \multicolumn{1}{c|}{$6$-heavy} & $-10-1/2$ & $-6$ & $6$-heavy \\
$-9-1/2$ & $-6$ & \multicolumn{1}{c|}{Transition} & $-8-1/2$ & $-6$ & \multicolumn{1}{c|}{$6$-heavy} & $-7-1/2$ & $-6$ & \multicolumn{1}{c|}{$6$-heavy} & $-6-1/2$ & $-6$ & \multicolumn{1}{c|}{Transition} & $-5$ & $-6$ & $6$-heavy \\
$-3$ & $-6$ & \multicolumn{1}{c|}{$6$-heavy} & $-17$ & $-7$ & \multicolumn{1}{c|}{$6$-heavy} & $-16$ & $-7$ & \multicolumn{1}{c|}{$6$-heavy} & $-15$ & $-7$ & \multicolumn{1}{c|}{$6$-heavy} & $-14$ & $-7$ & $6$-heavy \\
$-13$ & $-7$ & \multicolumn{1}{c|}{$6$-heavy} & $-12$ & $-7$ & \multicolumn{1}{c|}{$6$-heavy} & $-11$ & $-7$ & \multicolumn{1}{c|}{$6$-heavy} & $-10$ & $-7$ & \multicolumn{1}{c|}{$6$-heavy} & $-9$ & $-7$ & $6$-heavy \\
$-8$ & $-7$ & \multicolumn{1}{c|}{$6$-heavy} & $-7$ & $-7$ & \multicolumn{1}{c|}{$6$-heavy} & $-6$ & $-7$ & \multicolumn{1}{c|}{$6$-heavy} & $-5$ & $-7$ & \multicolumn{1}{c|}{$6$-heavy} & $-4$ & $-7$ & $6$-heavy \\
$-3$ & $-7$ & \multicolumn{1}{c|}{$6$-heavy} &  &  & \multicolumn{1}{c|}{} &  &  & \multicolumn{1}{c|}{} &  &  & \multicolumn{1}{c|}{} &  &  & \\\bottomrule
\end{tabular}%
}
\end{table}

%\end{subappendices}
    \end{toappendix}
\fi

\section{Conclusion}
In this paper, we propose a novel approach that integrates conditional generative models with DAS for circuit generation. 
Our framework begins with the design of CircuitVQ, a circuit tokenizer trained using a Circuit AutoEncoder. 
Building on this, we develop CircuitAR, a masked autoregressive model that utilizes CircuitVQ as its tokenizer. 
CircuitAR can generate preliminary circuit structures directly from truth tables, which are then refined by DAS to produce functionally equivalent circuits. 
The scalability and capability emergence of CircuitAR highlights the potential of masked autoregressive modeling for circuit generation tasks, akin to the success of large models in language and vision domains. 
Extensive experiments demonstrate the superior performance of our method, underscoring its efficiency and effectiveness. 
This work bridges the gap between probabilistic generative models and precise circuit generation, offering a robust and automated solution for logic synthesis.


\bibliography{ref}

\clearpage

\appendix

\renewcommand{\thesubsection}{\arabic{section}}
\ifFull
    %\begin{subappendices}
%\renewcommand{\thesection}{\arabic{section}}%
\clearpage
\section{Coordinates of gadgets}\label{apx:coordinates}

\begin{table}[!bht]
\caption{Coordinates for a cell gadget $\G_{(0,0)}$}
\label{tab:cell_gadget_coords}
\centering
\maxsizebox{\textwidth}{\textheight}{%
\begin{tabular}{@{}ccccccccccccccc@{}}
\toprule
\multicolumn{15}{l}{\textbf{Cell Gadget}} \\ \midrule
\multirow{2}{*}{\textbf{$x$}} & \multirow{2}{*}{\textbf{$y$}} & \multirow{2}{*}{\textbf{Type}} & \multirow{2}{*}{\textbf{$x$}} & \multirow{2}{*}{\textbf{$y$}} & \multirow{2}{*}{\textbf{Type}} & \multirow{2}{*}{\textbf{$x$}} & \multirow{2}{*}{\textbf{$y$}} & \multirow{2}{*}{\textbf{Type}} & \multirow{2}{*}{\textbf{$x$}} & \multirow{2}{*}{\textbf{$y$}} & \multirow{2}{*}{\textbf{Type}} & \multirow{2}{*}{\textbf{$x$}} & \multirow{2}{*}{\textbf{$y$}} & \multirow{2}{*}{\textbf{Type}} \\
 &  &  &  &  &  &  &  &  &  &  &  &  &  &  \\
$-1$ & $1$ & \multicolumn{1}{c|}{$6$-heavy} & $0$ & $1$ & \multicolumn{1}{c|}{$6$-heavy} & $1$ & $1$ & \multicolumn{1}{c|}{$6$-heavy} & $-1$ & $0$ & \multicolumn{1}{c|}{$6$-heavy} & $0$ & $0$ & Transition \\
$1$ & $0$ & \multicolumn{1}{c|}{$6$-heavy} & $-1$ & $-1$ & \multicolumn{1}{c|}{$6$-heavy} & $0$ & $-1$ & \multicolumn{1}{c|}{$6$-heavy} & $1$ & $-1$ & \multicolumn{1}{c|}{$6$-heavy} &  &  &  \\ \bottomrule
\end{tabular}%
}
\end{table}

\begin{table}[!bht]
\caption{Coordinates for a clause gadget $\G_{c}$ such that $c(T_c) = (0,0)$.}
\label{tab:clause_gadget_coords}
\centering
\maxsizebox{\textwidth}{\textheight}{%
\begin{tabular}{@{}ccccccccccccccc@{}}
\toprule
\multicolumn{15}{l}{\textbf{Clause Gadget}} \\ \midrule
\multirow{2}{*}{\textbf{$x$}} & \multirow{2}{*}{\textbf{$y$}} & \multirow{2}{*}{\textbf{Type}} & \multirow{2}{*}{\textbf{$x$}} & \multirow{2}{*}{\textbf{$y$}} & \multirow{2}{*}{\textbf{Type}} & \multirow{2}{*}{\textbf{$x$}} & \multirow{2}{*}{\textbf{$y$}} & \multirow{2}{*}{\textbf{Type}} & \multirow{2}{*}{\textbf{$x$}} & \multirow{2}{*}{\textbf{$y$}} & \multirow{2}{*}{\textbf{Type}} & \multirow{2}{*}{\textbf{$x$}} & \multirow{2}{*}{\textbf{$y$}} & \multirow{2}{*}{\textbf{Type}} \\
 &  &  &  &  &  &  &  &  &  &  &  &  &  &  \\
$1$ & $1$ & \multicolumn{1}{c|}{$6$-heavy} & $2$ & $1$ & \multicolumn{1}{c|}{$6$-heavy} & $3$ & $1$ & \multicolumn{1}{c|}{$6$-heavy} & $4$ & $1$ & \multicolumn{1}{c|}{$6$-heavy} & $-4$ & $0$ & $6$-heavy \\
$-3$ & $0$ & \multicolumn{1}{c|}{Transition} & $-2$ & $0$ & \multicolumn{1}{c|}{$6$-heavy} & $-1$ & $0$ & \multicolumn{1}{c|}{$6$-heavy} & $0$ & $0$ & \multicolumn{1}{c|}{Transition} & $0$ & $0$ & $6$-heavy \\
$1$ & $0$ & \multicolumn{1}{c|}{$6$-heavy} & $2$ & $0$ & \multicolumn{1}{c|}{$6$-heavy} & $3$ & $0$ & \multicolumn{1}{c|}{Transition} & $4$ & $0$ & \multicolumn{1}{c|}{$6$-heavy} & $-4$ & $-1$ & $6$-heavy \\
$-3$ & $-1$ & \multicolumn{1}{c|}{$6$-heavy} & $-2$ & $-1$ & \multicolumn{1}{c|}{$6$-heavy} & $-1$ & $-1$ & \multicolumn{1}{c|}{$6$-heavy} & $0$ & $-1$ & \multicolumn{1}{c|}{$6$-heavy} & $1$ & $-1$ & $6$-heavy \\
$2$ & $-1$ & \multicolumn{1}{c|}{$6$-heavy} & $3$ & $-1$ & \multicolumn{1}{c|}{$6$-heavy} & $4$ & $-1$ & \multicolumn{1}{c|}{$6$-heavy} & $-1$ & $-2$ & \multicolumn{1}{c|}{$6$-heavy} & $0$ & $-2$ & $6$-heavy \\
$1$ & $-2$ & \multicolumn{1}{c|}{$6$-heavy} & $-1$ & $-3$ & \multicolumn{1}{c|}{$6$-heavy} & $0$ & $-3$ & \multicolumn{1}{c|}{Transition} & $1$ & $-3$ & \multicolumn{1}{c|}{$6$-heavy} & $-1$ & $-4$ & $6$-heavy \\
$0$ & $-4$ & \multicolumn{1}{c|}{$6$-heavy} & $1$ & $-4$ & \multicolumn{1}{c|}{$6$-heavy} &  &  & \multicolumn{1}{c|}{} &  &  & \multicolumn{1}{c|}{} &  &  &  \\ \bottomrule
\end{tabular}%
}
\end{table}

\begin{table}[!bht]
\caption{Coordinates for the central part of a variable gadget $\G_{x}$ such that $c(S_x) = (0,0)$.}
\label{tab:variable_gadget_central_coords}
\centering
\maxsizebox{\textwidth}{\textheight}{%
\begin{tabular}{@{}ccccccccccccccc@{}}
\toprule
\multicolumn{15}{l}{\textbf{Variable Gadget (central part)}} \\ \midrule
\multirow{2}{*}{\textbf{$x$}} & \multirow{2}{*}{\textbf{$y$}} & \multirow{2}{*}{\textbf{Type}} & \multirow{2}{*}{\textbf{$x$}} & \multirow{2}{*}{\textbf{$y$}} & \multirow{2}{*}{\textbf{Type}} & \multirow{2}{*}{\textbf{$x$}} & \multirow{2}{*}{\textbf{$y$}} & \multirow{2}{*}{\textbf{Type}} & \multirow{2}{*}{\textbf{$x$}} & \multirow{2}{*}{\textbf{$y$}} & \multirow{2}{*}{\textbf{Type}} & \multirow{2}{*}{\textbf{$x$}} & \multirow{2}{*}{\textbf{$y$}} & \multirow{2}{*}{\textbf{Type}} \\
 &  &  &  &  &  &  &  &  &  &  &  &  &  &  \\
$4$ & $3+3/4$ & \multicolumn{1}{c|}{$6$-heavy} & $-5$ & $3$ & \multicolumn{1}{c|}{$6$-heavy} & $-3$ & $3$ & \multicolumn{1}{c|}{$6$-heavy} & $3$ & $3$ & \multicolumn{1}{c|}{$6$-heavy} & $5$ & $3$ & $6$-heavy \\
$-4$ & $2+1/4$ & \multicolumn{1}{c|}{$1$-heavy} & $4$ & $2+1/4$ & \multicolumn{1}{c|}{$1$-heavy} & $-5$ & $2$ & \multicolumn{1}{c|}{$6$-heavy} & $-3$ & $2$ & \multicolumn{1}{c|}{$6$-heavy} & $-2$ & $2$ & $6$-heavy \\
$-1$ & $2$ & \multicolumn{1}{c|}{$6$-heavy} & $0$ & $2$ & \multicolumn{1}{c|}{$6$-heavy} & $1$ & $2$ & \multicolumn{1}{c|}{$6$-heavy} & $2$ & $2$ & \multicolumn{1}{c|}{$6$-heavy} & $3$ & $2$ & $6$-heavy \\
$5$ & $2$ & \multicolumn{1}{c|}{$6$-heavy} & $-8$ & $1$ & \multicolumn{1}{c|}{$6$-heavy} & $-7$ & $1$ & \multicolumn{1}{c|}{$6$-heavy} & $-6$ & $1$ & \multicolumn{1}{c|}{$6$-heavy} & $-5$ & $1$ & $6$-heavy \\
$-3$ & $1$ & \multicolumn{1}{c|}{$6$-heavy} & $-2$ & $1$ & \multicolumn{1}{c|}{$6$-heavy} & $-1$ & $1$ & \multicolumn{1}{c|}{$6$-heavy} & $0$ & $1$ & \multicolumn{1}{c|}{$6$-heavy} & $1$ & $1$ & $6$-heavy \\
$2$ & $1$ & \multicolumn{1}{c|}{$6$-heavy} & $3$ & $1$ & \multicolumn{1}{c|}{$6$-heavy} & $5$ & $1$ & \multicolumn{1}{c|}{$6$-heavy} & $6$ & $1$ & \multicolumn{1}{c|}{$6$-heavy} & $7$ & $1$ & $6$-heavy \\
$8$ & $1$ & \multicolumn{1}{c|}{$6$-heavy} & $-4$ & $3/4$ & \multicolumn{1}{c|}{$2$-heavy} & $4$ & $3/4$ & \multicolumn{1}{c|}{$2$-heavy} & $-7-3/4$ & $0$ & \multicolumn{1}{c|}{$6$-heavy} & $-6$ & $0$ & $1$-heavy \\
$-4-3/4$ & $0$ & \multicolumn{1}{c|}{$2$-heavy} & $-3-1/2$ & $0$ & \multicolumn{1}{c|}{$1$-heavy} & $-2$ & $0$ & \multicolumn{1}{c|}{$6$-heavy} & $-1$ & $0$ & \multicolumn{1}{c|}{$6$-heavy} & $0$ & $0$ & Transition \\
$0$ & $0$ & \multicolumn{1}{c|}{$6$-heavy} & $1$ & $0$ & \multicolumn{1}{c|}{$6$-heavy} & $2$ & $0$ & \multicolumn{1}{c|}{$6$-heavy} & $3+1/2$ & $0$ & \multicolumn{1}{c|}{$1$-heavy} & $4+3/4$ & $0$ & $2$-heavy \\
$6$ & $0$ & \multicolumn{1}{c|}{$1$-heavy} & $7+3/4$ & $0$ & \multicolumn{1}{c|}{$6$-heavy} & $-4$ & $-3/4$ & \multicolumn{1}{c|}{$2$-heavy} & $4$ & $-3/4$ & \multicolumn{1}{c|}{$2$-heavy} & $-8$ & $-1$ & $6$-heavy \\
$-7$ & $-1$ & \multicolumn{1}{c|}{$6$-heavy} & $-6$ & $-1$ & \multicolumn{1}{c|}{$6$-heavy} & $-5$ & $-1$ & \multicolumn{1}{c|}{$6$-heavy} & $-3$ & $-1$ & \multicolumn{1}{c|}{$6$-heavy} & $-2$ & $-1$ & $6$-heavy \\
$-1$ & $-1$ & \multicolumn{1}{c|}{$6$-heavy} & $0$ & $-1$ & \multicolumn{1}{c|}{$6$-heavy} & $1$ & $-1$ & \multicolumn{1}{c|}{$6$-heavy} & $2$ & $-1$ & \multicolumn{1}{c|}{$6$-heavy} & $3$ & $-1$ & $6$-heavy \\
$5$ & $-1$ & \multicolumn{1}{c|}{$6$-heavy} & $6$ & $-1$ & \multicolumn{1}{c|}{$6$-heavy} & $7$ & $-1$ & \multicolumn{1}{c|}{$6$-heavy} & $8$ & $-1$ & \multicolumn{1}{c|}{$6$-heavy} & $-5$ & $-2$ & $6$-heavy \\
$-3$ & $-2$ & \multicolumn{1}{c|}{$6$-heavy} & $-2$ & $-2$ & \multicolumn{1}{c|}{$6$-heavy} & $-1$ & $-2$ & \multicolumn{1}{c|}{$6$-heavy} & $0$ & $-2$ & \multicolumn{1}{c|}{$6$-heavy} & $1$ & $-2$ & $6$-heavy \\
$2$ & $-2$ & \multicolumn{1}{c|}{$6$-heavy} & $3$ & $-2$ & \multicolumn{1}{c|}{$6$-heavy} & $5$ & $-2$ & \multicolumn{1}{c|}{$6$-heavy} & $-4$ & $-2-1/4$ & \multicolumn{1}{c|}{$1$-heavy} & $4$ & $-2-1/4$ & $1$-heavy \\
$-5$ & $-3$ & \multicolumn{1}{c|}{$6$-heavy} & $-3$ & $-3$ & \multicolumn{1}{c|}{$6$-heavy} & $3$ & $-3$ & \multicolumn{1}{c|}{$6$-heavy} & $5$ & $-3$ & \multicolumn{1}{c|}{$6$-heavy} & $-4$ & $-3-3/4$ & $6$-heavy \\
$4$ & $-3-3/4$ & \multicolumn{1}{c|}{$6$-heavy} & $-5$ & $-4$ & \multicolumn{1}{c|}{$6$-heavy} & $-3$ & $-4$ & \multicolumn{1}{c|}{$6$-heavy} & $3$ & $-4$ & \multicolumn{1}{c|}{$6$-heavy} & $5$ & $-4$ & $6$-heavy \\ \bottomrule
\end{tabular}%
}
\end{table}


\begin{table}[!bht]
\caption{Coordinates of arm 1 for $D_{c_1}$ when $S_x$ is centred at $(0,0)$.}
\label{tab:variable_gadget_arm1_coords}
\centering
\maxsizebox{\textwidth}{\textheight}{%
\begin{tabular}{@{}ccccccccccccccc@{}}
\toprule
\multicolumn{15}{l}{\textbf{Arm 1 (when $S_x$ centred at $(0,0)$)}} \\ \midrule
\multirow{2}{*}{\textbf{$x$}} & \multirow{2}{*}{\textbf{$y$}} & \multirow{2}{*}{\textbf{Disk Type}} & \multirow{2}{*}{\textbf{$x$}} & \multirow{2}{*}{\textbf{$y$}} & \multirow{2}{*}{\textbf{Disk Type}} & \multirow{2}{*}{\textbf{$x$}} & \multirow{2}{*}{\textbf{$y$}} & \multirow{2}{*}{\textbf{Disk Type}} & \multirow{2}{*}{\textbf{$x$}} & \multirow{2}{*}{\textbf{$y$}} & \multirow{2}{*}{\textbf{Disk Type}} & \multirow{2}{*}{\textbf{$x$}} & \multirow{2}{*}{\textbf{$y$}} & \multirow{2}{*}{\textbf{Disk Type}} \\
 &  &  &  &  &  &  &  &  &  &  &  &  &  &  \\
$-5$ & $9$ & \multicolumn{1}{c|}{$6$-heavy} & $-4$ & $9$ & \multicolumn{1}{c|}{$6$-heavy} & $-3$ & $9$ & \multicolumn{1}{c|}{$6$-heavy} & $-5$ & $8$ & \multicolumn{1}{c|}{$6$-heavy} & $-4$ & $8$ & Transition \\
$-3$ & $8$ & \multicolumn{1}{c|}{$6$-heavy} & $-5$ & $7$ & \multicolumn{1}{c|}{$6$-heavy} & $-4$ & $7$ & \multicolumn{1}{c|}{$6$-heavy} & $-3$ & $7$ & \multicolumn{1}{c|}{$6$-heavy} & $-5$ & $6$ & $6$-heavy \\
$-3$ & $6$ & \multicolumn{1}{c|}{$6$-heavy} & $-4$ & $5+3/4$ & \multicolumn{1}{c|}{Transition} & $-5$ & $5$ & \multicolumn{1}{c|}{$6$-heavy} & $-3$ & $5$ & \multicolumn{1}{c|}{$6$-heavy} & $-4$ & $4+3/4$ & $6$-heavy \\ \bottomrule
\end{tabular}%
}
\end{table}

\begin{table}[!bht]
\caption{Coordinates of arm for $D_{c_2}$ when $S_x$ is centred at $(0,0)$.}
\label{tab:variable_gadget_arm2_coords}
\centering
\maxsizebox{\textwidth}{\textheight}{%
\begin{tabular}{@{}ccccccccccccccc@{}}
\toprule
\multicolumn{15}{l}{\textbf{Arm 2 (when $S_x$ centred at $(0,0)$)}} \\ \midrule
\multirow{2}{*}{\textbf{$x$}} & \multirow{2}{*}{\textbf{$y$}} & \multirow{2}{*}{\textbf{Disk Type}} & \multirow{2}{*}{\textbf{$x$}} & \multirow{2}{*}{\textbf{$y$}} & \multirow{2}{*}{\textbf{Disk Type}} & \multirow{2}{*}{\textbf{$x$}} & \multirow{2}{*}{\textbf{$y$}} & \multirow{2}{*}{\textbf{Disk Type}} & \multirow{2}{*}{\textbf{$x$}} & \multirow{2}{*}{\textbf{$y$}} & \multirow{2}{*}{\textbf{Disk Type}} & \multirow{2}{*}{\textbf{$x$}} & \multirow{2}{*}{\textbf{$y$}} & \multirow{2}{*}{\textbf{Disk Type}} \\
 &  &  &  &  &  &  &  &  &  &  &  &  &  &  \\
$-9$ & $8$ & \multicolumn{1}{c|}{$6$-heavy} & $-11$ & $7$ & \multicolumn{1}{c|}{$6$-heavy} & $-10$ & $7$ & \multicolumn{1}{c|}{$6$-heavy} & $-9$ & $7$ & \multicolumn{1}{c|}{$6$-heavy} & $-11$ & $6$ & $6$-heavy \\
$-10$ & $6$ & \multicolumn{1}{c|}{$6$-heavy} & $-9$ & $6$ & \multicolumn{1}{c|}{$6$-heavy} & $-11$ & $5$ & \multicolumn{1}{c|}{$6$-heavy} & $-10$ & $5$ & \multicolumn{1}{c|}{Transition} & $-9$ & $5$ & $6$-heavy \\
$-11$ & $4$ & \multicolumn{1}{c|}{$6$-heavy} & $-10$ & $4$ & \multicolumn{1}{c|}{$6$-heavy} & $-9$ & $4$ & \multicolumn{1}{c|}{$6$-heavy} & $-11$ & $3$ & \multicolumn{1}{c|}{$6$-heavy} & $-10$ & $3$ & $6$-heavy \\
$-9$ & $3$ & \multicolumn{1}{c|}{$6$-heavy} & $-11$ & $2$ & \multicolumn{1}{c|}{$6$-heavy} & $-10$ & $2$ & \multicolumn{1}{c|}{Transition} & $-9$ & $2$ & \multicolumn{1}{c|}{$6$-heavy} & $-11$ & $1$ & $6$-heavy \\
$-10$ & $1$ & \multicolumn{1}{c|}{$6$-heavy} & $-9$ & $1$ & \multicolumn{1}{c|}{$6$-heavy} & $-11$ & $0$ & \multicolumn{1}{c|}{$6$-heavy} & $-9-3/4$ & $0$ & \multicolumn{1}{c|}{Transition} & $-8-3/4$ & $0$ & $6$-heavy \\
$-11$ & $-1$ & \multicolumn{1}{c|}{$6$-heavy} & $-10$ & $-1$ & \multicolumn{1}{c|}{$6$-heavy} & $-9$ & $-1$ & \multicolumn{1}{c|}{$6$-heavy} &  &  & \multicolumn{1}{c|}{} &  &  & \\\bottomrule
\end{tabular}%
}
\end{table}

\begin{table}[!bht]
\caption{Coordinates of arm for $D_{c_3}$ when $S_x$ is centred at $(0,0)$.}
\label{tab:variable_gadget_arm3_coords}
\centering
\maxsizebox{\textwidth}{\textheight}{%
\begin{tabular}{@{}ccccccccccccccc@{}}
\toprule
\multicolumn{15}{l}{\textbf{Arm 3 (when $S_x$ centred at $(0,0)$)}} \\ \midrule
\multirow{2}{*}{\textbf{$x$}} & \multirow{2}{*}{\textbf{$y$}} & \multirow{2}{*}{\textbf{Disk Type}} & \multirow{2}{*}{\textbf{$x$}} & \multirow{2}{*}{\textbf{$y$}} & \multirow{2}{*}{\textbf{Disk Type}} & \multirow{2}{*}{\textbf{$x$}} & \multirow{2}{*}{\textbf{$y$}} & \multirow{2}{*}{\textbf{Disk Type}} & \multirow{2}{*}{\textbf{$x$}} & \multirow{2}{*}{\textbf{$y$}} & \multirow{2}{*}{\textbf{Disk Type}} & \multirow{2}{*}{\textbf{$x$}} & \multirow{2}{*}{\textbf{$y$}} & \multirow{2}{*}{\textbf{Disk Type}} \\
 &  &  &  &  &  &  &  &  &  &  &  &  &  &  \\
$-15$ & $8$ & \multicolumn{1}{c|}{$6$-heavy} & $-17$ & $7$ & \multicolumn{1}{c|}{$6$-heavy} & $-16$ & $7$ & \multicolumn{1}{c|}{$6$-heavy} & $-15$ & $7$ & \multicolumn{1}{c|}{$6$-heavy} & $-17$ & $6$ & $6$-heavy \\
$-16$ & $6$ & \multicolumn{1}{c|}{$6$-heavy} & $-15$ & $6$ & \multicolumn{1}{c|}{$6$-heavy} & $-17$ & $5$ & \multicolumn{1}{c|}{$6$-heavy} & $-16$ & $5$ & \multicolumn{1}{c|}{Transition} & $-15$ & $5$ & $6$-heavy \\
$-17$ & $4$ & \multicolumn{1}{c|}{$6$-heavy} & $-16$ & $4$ & \multicolumn{1}{c|}{$6$-heavy} & $-15$ & $4$ & \multicolumn{1}{c|}{$6$-heavy} & $-17$ & $3$ & \multicolumn{1}{c|}{$6$-heavy} & $-16$ & $3$ & $6$-heavy \\
$-15$ & $3$ & \multicolumn{1}{c|}{$6$-heavy} & $-17$ & $2$ & \multicolumn{1}{c|}{$6$-heavy} & $-16$ & $2$ & \multicolumn{1}{c|}{Transition} & $-15$ & $2$ & \multicolumn{1}{c|}{$6$-heavy} & $-17$ & $1$ & $6$-heavy \\
$-16$ & $1$ & \multicolumn{1}{c|}{$6$-heavy} & $-15$ & $1$ & \multicolumn{1}{c|}{$6$-heavy} & $-17$ & $0$ & \multicolumn{1}{c|}{$6$-heavy} & $-16$ & $0$ & \multicolumn{1}{c|}{$6$-heavy} & $-15$ & $0$ & $6$-heavy \\
$-17$ & $-1$ & \multicolumn{1}{c|}{$6$-heavy} & $-16$ & $-1$ & \multicolumn{1}{c|}{Transition} & $-15$ & $-1$ & \multicolumn{1}{c|}{$6$-heavy} & $-17$ & $-2$ & \multicolumn{1}{c|}{$6$-heavy} & $-16$ & $-2$ & $6$-heavy \\
$-15$ & $-2$ & \multicolumn{1}{c|}{$6$-heavy} & $-17$ & $-3$ & \multicolumn{1}{c|}{$6$-heavy} & $-16$ & $-3$ & \multicolumn{1}{c|}{$6$-heavy} & $-15$ & $-3$ & \multicolumn{1}{c|}{$6$-heavy} & $-17$ & $-4$ & $6$-heavy \\
$-16$ & $-4$ & \multicolumn{1}{c|}{Transition} & $-15$ & $-4$ & \multicolumn{1}{c|}{$6$-heavy} & $-4$ & $-4-3/4$ & \multicolumn{1}{c|}{$6$-heavy} & $-17$ & $-5$ & \multicolumn{1}{c|}{$6$-heavy} & $-16$ & $-5$ & $6$-heavy \\
$-15$ & $-5$ & \multicolumn{1}{c|}{$6$-heavy} & $-14$ & $-5$ & \multicolumn{1}{c|}{$6$-heavy} & $-13$ & $-5$ & \multicolumn{1}{c|}{$6$-heavy} & $-12$ & $-5$ & \multicolumn{1}{c|}{$6$-heavy} & $-11$ & $-5$ & $6$-heavy \\
$-10$ & $-5$ & \multicolumn{1}{c|}{$6$-heavy} & $-9$ & $-5$ & \multicolumn{1}{c|}{$6$-heavy} & $-8$ & $-5$ & \multicolumn{1}{c|}{$6$-heavy} & $-7$ & $-5$ & \multicolumn{1}{c|}{$6$-heavy} & $-6$ & $-5$ & $6$-heavy \\
$-5$ & $-5$ & \multicolumn{1}{c|}{$6$-heavy} & $-3$ & $-5$ & \multicolumn{1}{c|}{$6$-heavy} & $-4$ & $-5-3/4$ & \multicolumn{1}{c|}{Transition} & $-17$ & $-6$ & \multicolumn{1}{c|}{$6$-heavy} & $-15-1/2$ & $-6$ & Transition \\
$-14-1/2$ & $-6$ & \multicolumn{1}{c|}{$6$-heavy} & $-13-1/2$ & $-6$ & \multicolumn{1}{c|}{$6$-heavy} & $-12-1/2$ & $-6$ & \multicolumn{1}{c|}{Transition} & $-11-1/2$ & $-6$ & \multicolumn{1}{c|}{$6$-heavy} & $-10-1/2$ & $-6$ & $6$-heavy \\
$-9-1/2$ & $-6$ & \multicolumn{1}{c|}{Transition} & $-8-1/2$ & $-6$ & \multicolumn{1}{c|}{$6$-heavy} & $-7-1/2$ & $-6$ & \multicolumn{1}{c|}{$6$-heavy} & $-6-1/2$ & $-6$ & \multicolumn{1}{c|}{Transition} & $-5$ & $-6$ & $6$-heavy \\
$-3$ & $-6$ & \multicolumn{1}{c|}{$6$-heavy} & $-17$ & $-7$ & \multicolumn{1}{c|}{$6$-heavy} & $-16$ & $-7$ & \multicolumn{1}{c|}{$6$-heavy} & $-15$ & $-7$ & \multicolumn{1}{c|}{$6$-heavy} & $-14$ & $-7$ & $6$-heavy \\
$-13$ & $-7$ & \multicolumn{1}{c|}{$6$-heavy} & $-12$ & $-7$ & \multicolumn{1}{c|}{$6$-heavy} & $-11$ & $-7$ & \multicolumn{1}{c|}{$6$-heavy} & $-10$ & $-7$ & \multicolumn{1}{c|}{$6$-heavy} & $-9$ & $-7$ & $6$-heavy \\
$-8$ & $-7$ & \multicolumn{1}{c|}{$6$-heavy} & $-7$ & $-7$ & \multicolumn{1}{c|}{$6$-heavy} & $-6$ & $-7$ & \multicolumn{1}{c|}{$6$-heavy} & $-5$ & $-7$ & \multicolumn{1}{c|}{$6$-heavy} & $-4$ & $-7$ & $6$-heavy \\
$-3$ & $-7$ & \multicolumn{1}{c|}{$6$-heavy} &  &  & \multicolumn{1}{c|}{} &  &  & \multicolumn{1}{c|}{} &  &  & \multicolumn{1}{c|}{} &  &  & \\\bottomrule
\end{tabular}%
}
\end{table}

%\end{subappendices}
\fi
%\section{Styles of lists, enumerations, and descriptions}\label{sec:itemStyles}
%%
%% Bibliography
%%

%% Please use bibtex, 


\end{document}
