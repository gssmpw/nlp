\section{Preliminaries}\label{sec:preliminaries}
%用語の定義等をする.
    This section provides the main definitions used in the paper, referencing geometry, graph, and convexity terminology from textbooks~\cite{Preparata1985,cormen2009,Diestel2017,boyd2004}.
\ifConf
\begin{toappendix}
\section{Auxiliary Definitions}
\else
\begin{toappendix}
\fi
\ifFull
\paragraph*{Convexity}
\fi
    A set $C$ is \emph{convex} if the line segment between any two points in $C$ lies entirely in $C$. Such a set is called \emph{convex set}. 
    A function $f:\mathbb{R}^n \rightarrow \mathbb{R}$ is \emph{convex} if its domain $\mathrm{dom} f$ is a convex set, and if for all $x,y \in \mathrm{dom} f$ and $\theta \in [0,1]$, the inequality
    $f(\theta x + (1-\theta)y) \leq \theta f(x) + (1-\theta)f(y)$ holds. 
    A function satisfying the above is called a \emph{convex function}. 
    The \emph{convex hull} of a finite set of points $P \subset \mathbb{R}^2$ is the smallest convex set that contains $P$ and is denoted by $\C(P)$.
    %
    A \emph{polygon} is defined by a finite set of segments such that every segment endpoint is shared by exactly two edges, and no subset of edges has the same property. 
    The segments are the \emph{edges} and their endpoints are the \emph{vertices} of the polygon. 
    An $n$-vertex polygon is called an $n$-gon.
    A polygon is \emph{simple} if no pair of non-consecutive edges that share a point exists.
    A simple polygon is \emph{convex} if its interior is a convex set.
    Similarly, the \emph{boundary} of a convex hull is a convex polygon.
    %

\end{toappendix}

\paragraph*{Objects}
    An \emph{interval} $I$ is a line segment on the real line of length $\mathrm{len}(I) \in \mathbb{R}^+$.
    Intervals are assumed to be open, unless explicitly stated otherwise.
    An interval such that $\mathrm{len}(I) = 1$ is called \emph{unit interval}.
    The \emph{left endpoint} $\ell(I)$ of an interval $I$ is the point that satisfies $\ell(I) \le y$ for any $y \in I$. 
    Similarly, the \emph{right endpoint} $r(I)$ of $I$ is the point that satisfies $y \le r(I)$ for any $y \in I$.
    The \emph{centre} $c(I)$ of $I$ is the point $c(I) = (r(I) - \ell(I))/2$.
%
\begin{toappendix}
    The \emph{left interval set} $L(\mathcal{I}, x)$ of a collection of intervals $\mathcal{I}$ is the subcollection of intervals to the `left' of a given point $x$.
    That is, $L(\mathcal{I}, x) = \{I \in \mathcal{I}:\: r(I) < x\}$. 
    Similarly, the \emph{right interval set} $R(\mathcal{I},x)$ is defined as $R(\mathcal{I},x) = \{I\in \mathcal{I}:\:\ell(I) > x\}$.
    Additionally, the \emph{leftmost endpoint} $\ell(\mathcal I)$ of a collection of intervals $\mathcal{I}$ is defined as $\min_{I \in \mathcal I} \ell(I)$.
    Similarly, the \emph{rightmost endpoint} $r(\mathcal I)$ of $\mathcal{I}$ is defined as $\max_{I \in \mathcal I} r(I)$.
\end{toappendix}
    %
    Throughout the paper, we assume that the indices of a collection of intervals $\I = \set{I_1,\ldots,I_n}$ follow the order given by centres of intervals.
    That is, $c(I_{i}) \le c(I_{i+1})$ for all $1\le i\le n-1$.
    However, it is not assumed that collections are ordered when given as the input graph.
    %The \emph{set of endpoints} $\mathcal{E}(\mathcal{I})$ of $\mathcal I$ is defined as $\mathcal{E}(\mathcal{I}) = \bigcup_{I \in \mathcal I} (\ell(I) \cup r(I))$.
    %The subcollection $\mathcal{I}(i,k)$ is the subcollection $\{I_i,I_{i+1},\ldots, I_{i+k-1}\}$.
    %The endpoint $x_k^{\mathcal{I}}$ is the $k$th endpoint of $\mathcal{E}(\mathcal{I})$. 
    %For an arbitrary point $x$, we define two subsets of intervals $\mathcal{E}_{\ell}(\mathcal{I},x) = \left\{I \in \mathcal{I}:\: \ell(I) = x\right\}$ and $\mathcal{E}_r(\mathcal{I},x) = \left\{I \in \mathcal{I}:\: r(I) = x\right\}$  to denote the subcollections of intervals of $\mathcal{I}$ having $x$ as the left and right endpoint, respectively.
    %
    %A unit $d$-cube is a $d$-dimensional hypercube where each side has a length of one unit and its centre is positioned at $(u_{x_1},\ldots,u_{x_d}) \in \mathbb{R}^d$.
    %Throughout the paper, $d$-cubes are assumed to be closed unless explicitly specified. 
    %A unit $2$-cube is called \emph{unit square} and its centre is denoted by $(s_x,s_y)$.
    %Given a radius $r>0$ and an integer $d>0$, a $d$-\emph{ball} $B$ centred at $p$ is the set $B = \set{x\in \mathbb{R}^d\mid \lVert x,p \rVert_2 \le r}$.
    %A $2$-ball is called a \emph{disk}.
    Given a radius $r>0$ and a point $p\in \mathbb{R}$, a \emph{disk} $D$ centred at $p$ is the set $D = \set{x\in \mathbb{R}^2\mid \lVert x,p \rVert_2 \le r}$.
    An \emph{open disk} $D$ is a disk without its boundary circle; that is, $D = \set{x\in \mathbb{R}^2\mid \lVert x,p \rVert_2 < r}$.
    We assume that the disks are open, unless we mention the contrary.
    A \emph{unit disk} is a disk of radius $r = 1/2$.
    %
    \begin{toappendix}
    The \emph{minimax centre} $p$ of a finite set of points $P \subset \mathbb{R}^2$ is the centre of the smallest circle that contains $P$, which is the point that minimises $\max_{p'\in P} \lVert p,p'\rVert_m$ for $m \in \set{1,2}$.
    The \emph{diameter} $\diam{P}$ of a finite set of points $P \subset \mathbb{R}^2$ is the distance of the farthest pair of points in $P$.
    Similarly, the diameter $\diam{S}$ of a convex polygon $\S$ is the diameter of its vertices.
    \end{toappendix}
    %
    The \emph{$L_m$ distance} for a $m\ge 1$ defines the distance between two points $p = (p_1,\ldots,p_d)$ and $q = (q_1,\ldots,q_d)$ in $\mathbb{R}^d$ as $\lVert p,q\rVert_m = \sqrt[m]{(p_1-q_1)^m+\cdots + (p_d - q_d)^m}$.
    In all subsequent sections, we use the $L_2$ distance (also known as the Euclidean distance) and the $L_1$ distance (also known as the Manhattan distance).

\paragraph*{Graphs}
    Throughout the paper, a graph $G = (V,E)$ is assumed to be a simple, finite, and undirected graph with vertex set $V$ and edge set $E$.
    An \emph{edgeless graph} is a graph $G = (V, E)$ such that $E = \emptyset$.
    %A \emph{complete graph} is a graph $G = (V, E)$ such that $E = \binom{V}{2}$.
    A $k$\emph{-clique} of a graph $G = (V, E)$ is a subset $W\subseteq V$ such that $|W| = k$ and for all $u,v \in W,\: u\neq v$, $\{u,v\} \in E$, for $k \le n$.
    If such $W$ exists in $V$, we say that $G$ \emph{contains a $k$-clique}.
    An \emph{interval graph} is an intersection graph $G = (V,E)$ where the vertex set $V = \{v_1,\ldots,v_n\}$ corresponds to a collection of intervals $\mathcal{I} = \{I_1,\ldots,I_n\}$ and an edge $\{v_i,v_j\} \in E$ exists if and only if $I_i \cap I_j \neq \emptyset$, for any $1\leq i,j \leq n,\: i \neq j$. 
    An interval graph is called \emph{unit interval graph} if $\len{I} = 1$ for all $I \in \mathcal{I}$.
    %A \emph{unit $d$-cube graph} is an intersection graph $G = (V,E)$ where the vertex set $V = \{v_1,\ldots,v_n\}$ corresponds to a collection of unit $d$-cubes $\mathcal{U} = \{U_1,\ldots,U_n\}$. An edge $\{v_i,v_j\} \in E$ exists if and only if $U_i \cap U_j \neq\emptyset$, for any $1\leq i,j \leq n,\: i \neq j$.
    %A unit $2$-cube graph is called \emph{unit square graph}.
    %Similarly, a \emph{$d$-ball graph} is an intersection graph $G = (V,E)$ where the vertex set $V = \{v_1,\ldots,v_n\}$ corresponds to a $d$-ball collection $\mathcal{B} = \{B_1,\ldots,B_n\}$. An edge $(v_i,v_j) \in E$ exists if and only if $B_i \cap B_j \neq \emptyset$, for any $1\leq i,j \leq n,\: i \neq j$. 
    %A $2$-ball graph is called \emph{disk graph} and a \emph{unit disk graph} is a disk graph in which the collection contains exclusively unit disks.
    Similarly, a \emph{disk graph} is an intersection graph $G = (V,E)$ where the vertex set $V = \{v_1,\ldots,v_n\}$ corresponds to a disk collection $\mathcal{D} = \{D_1,\ldots,D_n\}$. An edge $(v_i,v_j) \in E$ exists if and only if $D_i \cap D_j \neq \emptyset$, for any $1\leq i,j \leq n,\: i \neq j$.
    A \emph{unit disk graph} is a disk graph in which the collection contains exclusively unit disks.
    Unless stated otherwise, all intersection graphs are assumed to be \emph{unweighted}. 
    %A \emph{weighted intersection graph} is an intersection graph in which each object has a multiplicative weight associated with its moving distance function, referred to as \emph{distance weight}.
    %The distance weight is formally defined in subsequent sections when required.
    A \emph{weighted intersection graph} assigns a multiplicative weight, called the \emph{distance weight}, to the moving distance function of each object. 
    The formal definition of distance weight appears in later sections when required.
    %
    An (infinite) set of graphs $\Pi$ is a \emph{graph property} (or simply a property), and we say that \emph{$G$ satisfies $\Pi$} if $G \in \Pi$.
    A graph property $\Pi$ is \emph{non-trivial} if infinitely many graphs belong to $\Pi$ and infinitely many graphs do not belong to $\Pi$.
    In this paper, we deal with the following non-trivial properties:
    %(i)   $\Pi_{\texttt{comp}} = \{G :G\text{ is a complete graph.}\}$,
    (i)  $\Pi_{\texttt{edgeless}} = \{G :G\text{ is an edgeless graph.}\}$,
    (ii) $\Pi_{\texttt{acyc}} = \{G :G\text{ is an acyclic graph.}\}$,  
    (iii)  $\Pi_{k\texttt{-clique}} = \{G :G\text{ contains a $k$-clique.}\}$ and
    (iv)   $\overline{\Pi_{k\texttt{-clique}}} = \{G :G \not\in \Pi_{k\texttt{-clique}}\}$.
    %(v)  $\Pi_{k\texttt{-conn}} = \{G :G\text{ is a $k$-connected graph.}\}$ and
%    (vi)  $\Pi_{\texttt{bipar}} = \{G :G$ is a bipartite graph.$\}$ and
%    (vii) $\Pi_{k\texttt{-deg}} = \{G:G\text{ has degree at most $k$}\}$.