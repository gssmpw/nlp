\section{Concluding Remarks}\label{sec:conclu}

The main contribution of this paper is two-fold.
First, we continued the study of {\gged} originally presented in~\cite{HonoratoDroguett2024}, showing complexity results for satisfying several properties for sparse graphs on interval graphs.
%In particular, we introduced {\idisp} and show that it can be solved in $O(n\log n)$ time for unit interval graphs.
In particular, we showed that satisfying properties $\Pi_{\texttt{edgeless}}$, $\Pi_{\texttt{acyc}}$ and $\overline{\Pi_{k\texttt{-clique}}}$ is solvable in $O(n\log n)$ time on unit interval graphs.
In contrast, we showed that the problem becomes strongly \NP-hard on weighted interval graphs for satisfying the same properties.
Second, we defined {\ggedmm} as a variation of the above problem and showed that it is strongly {\NP-hard} for satisfying $\Pi_{\texttt{edgeless}}$ on weighted unit disk graphs over the $L_1$ and $L_2$ distances.
%
%First, we extend the {\gged} model originally presented in~\cite{HonoratoDroguett2024}, called {\ggedmm}, to focus on minimising the maximum moving distance.
%Second, several algorithmic results are presented to solve cases of fundamental graph properties in the model, as summarised in Table \ref{tab:summary}.
%In \Cref{sec:edg_uig,sec:edg_ig} we give the complexity results of {\gged} for various graph properties.
%We introduce the problem {\idisp} and show that it can be solved in $O(n\log n)$ time for unit interval graphs.
%The solution can be used to satisfy properties $\Pi_{\texttt{edgeless}}$, $\Pi_{\texttt{acyc}}$, $\overline{\Pi_{k\texttt{-clique}}}$ and $\Pi_{\texttt{bipar}}$ in the same running time.
%In contrast, we showed that {\gged} becomes strongly \NP-hard for satisfying properties $\Pi_{\texttt{edgeless}}$, $\Pi_{\texttt{acyc}}$, $\overline{\Pi_{k\texttt{-clique}}}$ and $\Pi_{\texttt{bipar}}$ when the input is a weighted interval graph.
%Lastly, the problem {\ggedmm} is defined in \Cref{sec:disk_edgeless} and shown that it is {\NP-hard} for satisfying $\Pi_{\texttt{edgeless}}$ on unit disk graphs over $L_1$ and $L_2$ distances.
%We conclude the paper by showing that several graph properties can be achieved through linear-size linear optimisation formulations in the {\ggedmm} model.

There are several directions for further research.
%Given our results, a natural extension is to study other graph properties over different distance metrics.
Our results provide a comprehensive picture of the complexity of {\gged} on interval graphs. 
In particular, we showed that the problem becomes hard even in lower dimensions when the input is not restricted by interval size and distance weight.
As a result, a potential future work is to study the complexity when exclusively one of the restrictions is applied.
%We also show that {\ggedmm} becomes {\NP-hard} on unit disk graphs for satisfying $\Pi_{\texttt{edgeless}}$.
Another interesting direction is to study the model for satisfying $\Pi_{\texttt{edgeless}}$ in higher dimensions.
Related works \cite{fomin2023,Fomin2025,Fiala2005} suggest that our model on more complex intersection graphs becomes intractable for some of the properties presented in this work. 
In general, we deal with the edit operation that moves the objects of the given intersection graph. 
However, the model is not restricted to this operation. 
Determining {\gged} using other geometric edit operations (such as shrinking or rotating objects) is left for future research for all intersection graphs and graph properties presented in this work.