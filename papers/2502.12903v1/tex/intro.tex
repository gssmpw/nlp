%%%%% 
\section{Introduction}\label{sec:intro}
Graph modification is a fundamental topic to address graph similarity and dissimilarity, where a given graph is deformed by adding or deleting vertices or edges to satisfy a specific non-trivial graph property, while minimising the cost of edit operations. The problem of determining this cost is commonly known as \emph{graph modification problem} (GMP) and has applications in various disciplines, such as computer vision~\cite{Chung1994}, network interdiction~\cite{Hoang2023}, and molecular biology~\cite{Hellmuth2020}. GMPs are often categorised into vertex and edge modification problems, with edit operations restricted to the vertex and edge sets, respectively.

The cost of a single edit operation in a GMP is often determined by the specific application. In theoretical studies, a unit-cost model is often assumed, where each addition or deletion of a vertex or edge has a uniform cost. However, for such models, it is known that determining whether a graph can be modified to satisfy a given property is \NP-hard for a wide range of graph classes and properties~\cite{Lewis1980,Burzyn2006,Fomin2015,Sritharan2016}. These negative bounds of GMPs motivate alternative formulations for graph editing that consider domain-specific constraints and cost measures.%, such as geometric  in intersection graphs.

The choice of edit operations and their associated costs is a crucial aspect of GMPs, as different formulations capture different structural properties and computational challenges. Analogous to string similarity analysis, where modifications are based on biologically significant operations such as DNA mutations and repeats~\cite{Li2009}, graph modification problems should reflect the inherent constraints and structural properties of the graphs being studied. In particular, \emph{geometric intersection graphs} (hereafter intersection graphs) provide a suitable framework for studying GMPs for scenarios where graphs represent spatial relationships (see, e.g.,~\cite{Panolan2024,Berg2019,fomin2023}). Given a collection of geometric objects $\S$, an \emph{intersection graph} $(G,\S)$ is a graph where there is a one-to-one correspondence between the vertex set $V(G)$ and $\S$, and two vertices are adjacent if and only if their corresponding objects intersect. This model includes many well-known graph classes, such as interval graphs and disk graphs. These graphs can be frequently found in real-world applications such as network modelling and bioinformatics~\cite{McKee1999}.

Motivated by this context, this paper investigates GMPs for intersection graphs. In this context, two natural questions arise: \begin{enumerate} \item Are standard graph edit operations suitable for modifying intersection graphs? \item How can the geometric properties of objects be exploited to overcome the hardness of GMPs? \end{enumerate} To answer these questions, we introduce {\gged}, a model for modifying intersection graphs from a geometric perspective.

In the intersection graph model, a natural edit operation is to move the objects in $\S$. We treat this movement as a graph edit operation and focus on minimising the cost required to modify an intersection graph to satisfy a specific graph property. The cost is quantified by the total moving distance, which is the sum of the distances by which objects in $\S$ are moved. More precisely, we define the problem as follows:

\begin{itembox}[l]{{\gged}}\label{pro:uig_general} \begin{description} \item[Input:] An intersection graph $(G,\S)$ and a graph property $\Pi$. \item[Output:] The minimum total moving distance of the objects in $\S$ so that the resulting intersection graph satisfies $\Pi$. \end{description} \end{itembox}

We assume that $\Pi$ is given by an oracle, i.e. we have an algorithm to determine whether a given intersection graph $(G, \S)$ satisfies $\Pi$.

\subsection*{Related work}
Numerous GMPs are known to be computationally hard. In the early 1980s, Lewis and Yannakakis~\cite{Lewis1980} showed that vertex-deletion problems are \NP-complete for any hereditary graph property. Similarly, many edge modification problems have been shown to be \NP-complete, such as transforming a graph into a perfect, chordal, or interval graph~\cite{Burzyn2006}. As a result, the past decade has seen a growing interest in addressing these problems from the perspective of parameterised complexity. The recent survey by Crespelle et al.\cite{Crespelle2023} provides a comprehensive overview of this subject (see also\cite{Drange2015}).

Although classical GMPs focus on structural modifications of graphs, recent studies have explored models that include geometric constraints. Honorato-Droguett et al.~\cite{HonoratoDroguett2024} introduced the above geometric approach to graph modification, demonstrating that certain properties, such as graph completeness and the existence of a $k$-clique, can be efficiently satisfied on interval graphs. 
Their work highlights how the underlying geometric properties of intersection graphs can be exploited to design appropriate modification models.

In a similar vein, Fomin et al.~\cite{fomin2023} studied the \emph{disk dispersal} problem, where a set $\S$ of $n$ disks, an integer $k \geq 0$, and a real number $d \geq 0$ are given, and the goal is to determine whether an edgeless disk graph can be realised by moving at most $k$ disks by at most $d$ distance each. They proved that this problem is {\NP-hard} when $d=2$ and $k = n$ and also {\FPT} when parameterised by $k+d$. Furthermore, they showed that the problem becomes \W[1]-hard when parameterised by $k$ when disk movement is restricted to rectilinear directions.

Expanding on this line of research, Fomin et al.~\cite{Fomin2025} conducted a parameterised complexity study of edge modification problems where \emph{scaling} objects is considered as the edit operation. Their results illustrate how alternative edit operations in geometric intersection graphs can impact computational complexity, enabling further study of geometric modification graph models.
In particular, their work includes several {\FPT} results to achieve independence, acyclity and connectivity on disk graphs.

Our work continues these developments by introducing {\gged}, a model that considers object movement as an edit operation to modify intersection graphs. Unlike prior studies that focus on vertex and edge modifications or object scaling, our approach explicitly considers movement costs by quantifying the total moving distance required to satisfy a given graph property. This approach enables the exploration of new algorithmic and complexity-theoretic questions in the context of geometric intersection graphs.

\subsection*{Our contribution}
Our results are mainly focused on interval graphs and summarised in \Cref{tab:summary}. In this paper, we deal with the following graph properties: \begin{itemize} 
\item $\Pi_{\texttt{edgeless}}$ (edgeless graphs), 
\item $\Pi_{\texttt{acyc}}$ (acyclic graphs), and 
\item $\overline{\Pi_{k\texttt{-clique}}}$ ($k$-clique-free graphs).
\end{itemize} 

%\begin{tabular}{@{}c@{}}{\bfseries Problem} \\ {\bfseries Type}\end{tabular}
%\begin{tabular}{@{}c@{}}{\bfseries Target Graph} \\ {\bfseries Property}\end{tabular}
\begin{table}[!b]
    \setlength\extrarowheight{2pt}
    \centering
    \caption{Summary of our results. In this table, $L_1$ and $L_2$ are the Manhattan and Euclidean distances, respectively. The terms IG, UIG and UDG are abbreviations of interval graphs, unit interval graphs and unit disk graphs, respectively.}\label{tab:summary}
    \begin{tabular}{|c|c|c|c|c|c|}
        \hline
        {\bfseries Problem} &  &  &  & {\bfseries Target Graph} & \\
        {\bfseries Type} & \multirow{-2}{*}{{\bf Metric}} & \multirow{-2}{*}{{\bf Weighted}} & \multirow{-2}{*}{\textbf{Graph}} & {\bfseries Property} & \multirow{-2}{*}{\textbf{Complexity}} \\
        \hline
        \multirow{6}{*}{\texttt{minsum}} & \multirow{7}{*}{$L_2(=L_1)$}  & \multirow{4}{*}{Yes} & \multirow{4}{*}{\textbf{IG}} &  $\Pi_{\texttt{edgeless}}$ & strongly \NP-hard\\\cline{5-6}
        & &  &  &   $\Pi_{\texttt{acyc}}$  & strongly \NP-hard \\\cline{5-6}
        & &  &  &   \multirow{2}{*}{$\overline{\Pi_{k\texttt{-clique}}}$} & {strongly \NP-hard} \\
        & & & & & {for any $1\le k \le n$} \\\cline{3-6}
        %& &    &   $\Pi_{k\texttt{-deg}}$: at most degree $k$ & strongly \NP-hard \\
        %\cline{3-5}
        &  & \multirow{3}{*}{No} &\multirow{3}{*}{\textbf{UIG}}  &  $\Pi_{\texttt{edgeless}}$ & $O(n\log n)$ \\\cline{5-6}
        & &  &  &   $\Pi_{\texttt{acyc}}$  & $O(n\log n)$ \\\cline{5-6}
        & &  &  &   $\overline{\Pi_{k\texttt{-clique}}}$ & $O(n\log n)$ \\%\cline{4-5}
        %& &    &   $\Pi_{\texttt{bipar}}$: bipartite & $O(n\log n)$ \\
       \hline
        \multirow{1}{*}{\texttt{minimax}} & \multirow{1}{*}{$L_1$, $L_2$} & Yes &\multirow{1}{*}{\textbf{UDG}} &   \multirow{1}{*}{$\Pi_{\texttt{edgeless}}$} & \multirow{1}{*}{strongly \NP-hard} \\\hline%\cline{1-3}
        %\cline{2-2}
        %&  & &    &  \\\hline
    \end{tabular}
\end{table}





%\Cref{tab:summary} summarises our results. 
In~\cite{HonoratoDroguett2024}, the model presented is studied mainly for properties of dense graphs.
This inspires the present paper as a subsequent work, where we instead focus on properties for sparse graphs.
As two fundamental classes of sparse graphs, we consider edgeless graphs ($\Pi_{\texttt{edgeless}}$) and acyclic graphs ($\Pi_{\texttt{acyc}}$). These properties have also been studied in related work on geometric intersection graphs\cite{fomin2023,Fomin2025}.

As we shall detail, $\Pi_{\texttt{acyc}}$ is contained in $\overline{\Pi_{k\texttt{-clique}}}$ in our context.
As a result, one might argue that the distinction of both properties is irrelevant.
However, we still consider them distinctively, as forests are a well-known class of graphs.
Our analysis highlights the computational complexity of modifying intersection graphs while considering movement-based edit operations, a perspective distinct from prior work that focuses on exclusively modifying the graph structure.

\subsection*{Paper Organisation}
\Cref{sec:preliminaries} formally describes the definitions needed to address the above ideas. \Cref{sec:edg_uig} presents the problem {\idisp} and shows that it can be solved in $O(n\log n)$ time. Using this algorithm, we establish that {\gged} can also be solved in $O(n\log n)$ time for satisfying $\Pi_{\texttt{edgeless}}$, $\Pi_{\texttt{acyc}}$, and $\overline{\Pi_{k\texttt{-clique}}}$ on unit interval graphs.
\Cref{sec:edg_ig} demonstrates that {\gged} becomes strongly \NP-hard on weighted interval graphs for satisfying $\Pi_{\texttt{edgeless}}$, $\Pi_{\texttt{acyc}}$, and $\overline{\Pi_{k\texttt{-clique}}}$.
\Cref{sec:disk_edgeless} shows that the minimax version of {\gged} is strongly \NP-hard on weighted unit disk graphs when satisfying $\Pi_{\texttt{edgeless}}$ under both the $L_1$ and $L_2$ distance metrics.
\Cref{sec:conclu} concludes with remarks on our results and potential future directions.


\ifConf
Due to space restrictions, we omit in-depth explanations and all full proofs of statements with a $\star$-mark. The reader is referred to the full version of this paper~[] for these details.
\fi
