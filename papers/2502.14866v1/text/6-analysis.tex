\section{Analysis}
\label{sect:analysis}

\begin{figure}[t]
    \centering
    \includegraphics[width=0.86\linewidth]{figure/6-analysis/prefilling_attn_kernel.pdf}
    \caption{Prefilling Stage Attention Kernel Evaluation.} 
        
    \label{fig:ana:prefilling_attention}
\end{figure}

\begin{figure}[t]
    \centering
    \includegraphics[width=\linewidth]{figure/6-analysis/hierarchy_NIAH.pdf}
    \caption{\textbf{Hierarchical paging} enables \system to preserve the long-context retrieval capabilities of the original model without increasing the key-value (KV) token budget. We use Llama-3-8B for the ablation.}

    
    \label{fig:ana:our_larger_page}
\end{figure}

\begin{figure}[t]
    \centering
    \includegraphics[width=\linewidth]{figure/6-analysis/selector_overhead.pdf}
    \caption{\textbf{Effect of Reusable Page Selection}. The overhead of the dynamic page selector is significant, as its complexity increases linearly with input sequence length. Our \textit{Reusable Page Selection} effectively mitigates this issue. The latency breakdown is evaluated on Llama-3-8B.} 
    \label{fig:ana:selector_overhead}
\end{figure}


In this section, we present in-depth analysis on our design choices in the \system system from both the accuracy and the efficiency perspective. We also scrutinize the sources of performance gains in \sect{sect:results}.

\subsection{Prefilling Stage Sparse Attention Kernel}

We benchmark the performance of our block sparse attention kernel for the prefilling stage in Figure~\ref{fig:ana:prefilling_attention}. Compared with the implementation by MInference~\cite{jiang2024minference}, our kernel consistently achieves 1.3$\times$ speedup at the same sparsity level. Oracle stands for the theoretical upper-bound speedup ratio: $\text{Latency}_{\text{oracle}} = \text{Latency}_{\text{dense}} * (1-\text{sparsity})$.


\subsection{Effectiveness of Hierarchical Paging}



We use the Needle-in-a-Haystack ~\cite{LLMTest_NeedleInAHaystack} test to demonstrate that the hierarchical paging design effectively maintains the model's long-context capability on larger page blocks without increasing the token budget. In contrast to the performance drop observed with increased page granularity in Figure~\ref{fig:ana:naive-larger-page}, \system leverages a hierarchical page structure to decouple the pruning algorithm’s page granularity from the physical memory layout of the KV cache. This approach enables our sparse attention mechanism to remain both accurate and hardware-efficient. Figure~\ref{fig:ana:our_larger_page} highlights this improvement: with a page size of 64 and the same token budget, \system achieves accuracy comparable to the baseline algorithm~\cite{tang2024quest}, which prunes history tokens at a granularity of 16.

\subsection{Mitigating Page Selection Overhead}



\begin{table}[t]
\centering
\caption{The reusable page selector in \system preserves the model's long-context accuracy while significantly reducing selection overhead by \textbf{4$\times$} with a reuse interval of 4. We evaluate Llama-3-8B on RULER~\cite{nvidia_ruler} at a sequence length of 64K. LServe-$N$ denotes that the token budget for dynamic sparsity is $N$.}

\footnotesize
\scalebox{0.95}{
\begin{tabular}{ccccccc}


\toprule
Reuse Interval & Dense    & 1    & 2    & 4    & 8    & 16   \\ 
\midrule
LServe-4096 & 86.8 & 86.2 & 85.6 & 85.6 & 84.8 & 83.2 \\ 
\midrule			
LServe-8192 & 86.8 & 86.1 & 85.8 & 85.5 & 85.6 & 84.8\\ 
\bottomrule
\end{tabular}
}
\label{tab:ana:reusable_accuracy}
\vspace{10pt}
\end{table}


\paragraph{Reusable Page Selection.} During decoding, although the attention kernel maintains constant complexity due to a capped number of historical KV tokens, the complexity of the page selector still scales linearly with sequence length. As illustrated in Figure~\ref{fig:ana:selector_overhead}, for a sequence length of 128K and a 4K token budget for sparse attention, the page selector (0.24 ms) is already twice as slow as the sparse attention kernel (0.12 ms). With our reusable page selector, however, \system significantly reduces page selection overhead by a factor of $C$, where $C$ is the reuse interval. We further show that \system is resilient to different reuse interval choices. Table~\ref{tab:ana:reusable_accuracy} demonstrates no significant performance degradation until the reuse interval exceeds 8, so we set it to 4 by default in \system.

\paragraph{Context Pooling Overhead.} To enable page selection during decoding, we must calculate representative features using min-max pooling in the prefilling stage. It is important to note that a single pooling kernel executes under \textbf{1 ms}, while the entire prefilling stage completes in approximately 17 seconds with 128K context length. Consequently, the context pooling overhead is negligible.

\subsection{Sparse Attention Kernel for Decoding Stage}

\begin{figure}[t]
    \centering
    \includegraphics[width=\linewidth]{figure/6-analysis/decoding_attn_kernel.pdf}
    \caption{\textbf{Efficiency gains from static and dynamic sparsity in \system}. These sparsity patterns contribute to a compound speedup effect, with static sparsity being more effective at shorter contexts, and dynamic sparsity offering greater benefits at longer contexts. We report the latency of a single attention layer in Llama-2-7B.}
    \label{fig:ana:decoding_attn_kernel}
\end{figure}



We analyze the effectiveness of different sparsity patterns in decoding attention. In Figure~\ref{fig:ana:decoding_attn_kernel}, we apply \textit{static} sparsity by converting 50\% of attention heads to streaming heads, achieving a \textbf{1.3-1.7$\times$} speedup across various input sequence lengths. Additionally, we introduce dynamic sparsity with a fixed KV budget of 4096 tokens, which bounds the computational complexity of decoding attention to a \textbf{constant}, delivering a \textbf{30$\times$} speedup over the dense baseline for an input length of 256K.  Although sparsity selection introduces minor overhead for shorter sequences, this is mitigated by reusable page selection. Additionally, we also perform the end-to-end ablation study in Section \ref{sect:End-to-End Ablation}.



\subsection{End-to-End Speedup Breakdown}
\label{sect:End-to-End Ablation}



\begin{table*}[htbp]
    \centering
    \small
    \begin{tabular}{p{14cm}}
     \toprule
\#\#\#  Objective: \\
Generate a 5-day family travel itinerantry that satisfies all specified requirements while adhering to highly fine-grained constraints. The generated itinerary should balance real-time adaptability, strict hard attributes, and semantic soft attributes. \\

\#\#\# User Profile: \\
 - Travelers: 2 adults + 1 child (age 8) \\
 - Budget: $<=$ \$300/day (total \$1,500 for the trip) \\
 - Activity Balance: 70\% educational/cultural experiences, 20\% relaxation, 10\% family-friendly shopping. \\

\#\#\# Hard Attributes: \\
- Activity Scheduling: \\
\quad- Each activity must have a defined start and end time, ensuring there is no overlap between activities. \\
\quad- A break period from 13:00-14:30 is mandatory daily. \\
\quad- Each activity must fit within a 2-hour window unless otherwise specified. \\

- Budget Requirements: \\
\quad- Each day’s total cost (including transportation, food, and activities) must not exceed \$300. \\
\quad- Transportation is limited to metro and walking only, with a maximum of 3 metro rides per day. \\

- Location Constraints: \\
\quad- Must-visit locations: City Zoo (Day 1) and Science Museum (Day 3). \\
\quad- Activities must occur in geographically adjacent areas to minimize walking distance. \\

- Keyword Requirements: \\
\quad- Each day’s description must include specific keywords. For example: \\
\quad- Day 1: “wildlife,” “exploration,” and “interactive learning.” \\
\quad- Day 3: “science,” “innovation,” and “hands-on exhibits.” \\

- Structure Constraints: \\
\quad- Each day’s itinerary must consist of 4 sections: \\
\quad\quad- Morning activity \\
\quad\quad- Break/lunch period \\ 
\quad\quad- Afternoon activity \\
\quad\quad- Evening summary (limited to 50 words) \\

\#\#\# Soft Attributes \\
- Tone and Emotion: \\ 
\quad- Day 1: Use a tone that conveys “excitement and discovery.” \\ 
\quad- Day 3: Use a tone that conveys “curiosity and wonder.” \\
- Language Style: \\ 
\quad- Use descriptive, vivid, and family-friendly language throughout. \\
\quad- Include at least one metaphor or simile per day (e.g., "The Science Museum felt like stepping into the future!"). \\
- Visual Details: \\
\quad- Each activity must include specific sensory details (e.g., "the bright colors of the parrots at the zoo" or "the tinkling sound of water fountains at the park").

- Adaptive Adjustments (Real-time Constraints): \\
\quad- Weather Sensitivity: \\
\quad\quad- If the rain forecast exceeds 60\%, replace outdoor activities with indoor alternatives while keeping the overall tone and keywords intact. \\ 
\quad- Physical Endurance: \\
\quad\quad- If a day’s total walking distance exceeds 10 kilometers, the next day’s activities must reduce walking by 30\%. \\
\quad- Health Responsiveness: \\
\quad\quad- If a health-related issue arises (e.g., fatigue or illness), adjust the itinerary dynamically to: \\
\quad\quad- Reduce activity duration to half. \\ 
\quad\quad- Substitute the activity with a more relaxing or passive option. \\
\bottomrule
    \end{tabular}
    \caption{The complete travel planner case study.}
    \label{tab:travel_planner_case}
\end{table*}
In Figure~\ref{fig:ana:case_study}, we highlight the sources of performance improvement in \system. By leveraging static sparsity, \system achieves end-to-end speedups of up to \textbf{1.7$\times$} over the dense baseline. Additionally, dynamic sparsity, aided by a reusable page selector, significantly reduces generation latency, yielding a \textbf{7.7$\times$} speedup for sequence lengths of 256K. Lastly, \system configures sparse patterns through offline profiling, effectively avoiding slowdowns from dynamic sparsity at shorter context lengths.

























