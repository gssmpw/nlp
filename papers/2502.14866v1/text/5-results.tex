\section{Evaluation}
\label{sect:results}




\subsection{Evaluation Setup}

\textbf{Implementation}. We implement \system in CUDA and PTX assembly on the basis of QServe~\cite{lin2024qserve} and TensorRT-LLM~\cite{trtllm} system. The specialized CUDA kernels are compiled into PyTorch extensions for better flexibility and compatibility with the purely PyTorch-based serving interface. 


\textbf{Testbed}. 
Our primary experiments are conducted on a server equipped with 8 NVIDIA A100 80GB GPUs, 2 AMD EPYC 7763 CPUs (128 cores), and 2TB of memory. Unless explicitly stated, all experiments utilize the A100 GPUs. Additionally, we perform some evaluations on a cloud instance with a single NVIDIA L40S 48GB GPU to assess system performance across different GPU architectures. All evaluations use PyTorch 2.5.0 with CUDA 12.4 and cuDNN 9.2.0.
\begin{table}[t]
\centering
\small
\caption{\textbf{Accuracy evaluation on LongBench}~\cite{bai2023longbench}. We compare our method with vanilla dense attention on 2 models and 10 different benchmarks.}
\scalebox{1.0}
{
\begin{tabular}{ccccc}
\toprule
Model & \multicolumn{2}{c}{Llama-3-8B} & \multicolumn{2}{c}{Llama-2-7B} \\
\midrule
Benchmark & Dense & \system & Dense & \system \\
\midrule
2WikiMQA  & 30.3 & 31.6 & 35.4 & 35.1 \\ 
DuReader  & 30.3 & 30.8 & 25.4 & 24.7 \\ 
HotpotQA  & 41.7 & 42.7 & 47.4 & 49.6 \\ 
MultiNews & 27.7 & 27.7 & 26.6 & 26.6 \\ 
Qasper    & 31.7 & 29.3 & 32.6 & 29.5 \\ 
QMSum     & 23.8 & 24.0 & 21.0 & 21.3 \\  
SamSum   & 41.2 & 39.3 & 41.8 & 41.5 \\ 
TriviaQA  & 84.9 & 83.7 & 86.2 & 86.5 \\ \midrule
\textbf{Average}   & \textbf{38.9} & \textbf{38.6} & \textbf{39.5} & \textbf{39.4}
\\
\bottomrule
\end{tabular}
}
\label{tab:results:long_bench}
\end{table}


\begin{figure}[t]
    \centering
    \includegraphics[width=\linewidth]{figure/5-results/main_NIAH.pdf}
    \caption{\textbf{Accuracy evaluation on Needle-in-a-Haystack}. 
    }%
    \label{fig:evaluation:main_niah}
    \vspace{10pt}
\end{figure}

\begin{table}[t]
\centering
\caption{\textbf{Accuracy evaluation on RULER}~\cite{nvidia_ruler}. We evaluate the accuracy of Llama-3-8B on RULER benchmarks, including challenging tasks such as multi-hop tracing and aggregation to test behaviors beyond searching from context. LServe-$N$ denotes that the token budget for dynamic sparsity is $N$. Note that for long-context inputs, latency is not dominated by attention alone in \system, with page selector and GEMM also contributing to it. Experiments reveal that LServe-8192 is only up to 6\% slower than LServe-4096 when the sequence length exceeds 128K.}
\scalebox{0.9}
{
\begin{tabular}{ccccccc}
\toprule
Llama-3-8B  & 32K  & 64K  & 128K & 160K	& 192K & 256K \\ \midrule
Dense       & 90.5 & 86.8 & 83.8 & 79.3 & 79.6 & 79.4 \\ \midrule
LServe-4096 & 91.0 & 85.6 & 81.0 & 79.0 & 76.1 & 75.7 \\ \midrule
LServe-8192 & 91.8 & 86.1 & 81.7 & 81.2 & 79.7 & 79.1 \\
\bottomrule
\end{tabular}
}
\label{tab:results:main_ruler}
\end{table}





\textbf{Models}. 
To comprehensively assess system performance across various LLM architectures, we utilize the widely adopted GQA-based model Llama-3-8B \cite{dubey2024llama}, the MHA-based model Llama-2-7B \cite{touvron2023llama2}, and the smaller-scale model Minitron-4B \cite{Minitron}. Additionally, to support long-context inference, we employ the context-extended Llama-3-8B version Gradient \cite{gradientlongcontextllama3}.


\textbf{Metrics}.
Our primary focus is on serving throughput. For the prefilling stage, we use \emph{time-to-first-token} (TTFT) as a key metric, while for the decoding stage, we emphasize minimizing the \emph{per-token generation latency}.



\textbf{Baselines}. 
We consider the following systems as baselines, using their latest versions\footnote{vLLM 0.6.3} to ensure a fair comparison. We activated W8A8 precision for baselines if available.

\begin{itemize}[leftmargin=*, itemsep=-3pt, topsep=-5pt, ]
  \item \emph{vLLM}~\cite{vllm}, one of the most popular LLM serving systems featuring PagedAttention.
  
  \item \emph{QServe}~\cite{lin2024qserve}, efficient LLM serving system featuring W4A8KV4 quantization.
  
  \item \emph{MInference} \cite{jiang2024minference}, the state-of-the-art long-context prefilling stage acceleration system.
  
  \item \emph{DuoAttention} \cite{xiao2024duoattention}, a strong long-sequence LLM inference framework with static sparse attention.
  
\end{itemize}



Additionally, we compare our approach with the state-of-the-art long-context decoding stage acceleration system, \emph{Quest} \cite{tang2024quest}. Since Quest only supports MHA models, we conduct and discuss this comparison in Table~\ref{tab:results:e2e_quest}.



\subsection{End-to-end Accuracy}
\label{sect:results:acc_eval}






We evaluate the accuracy of our hybrid block-sparse mechanism with LongBench~\cite{bai2023longbench} tasks, the Needle-in-a-Haystack (NIAH)~\cite{LLMTest_NeedleInAHaystack} pressure tests, as well as the challenging RULER~\cite{nvidia_ruler} benchmarks.  Table~\ref{tab:results:long_bench} compares the LongBench accuracy between \system and dense baseline. Results show that \system well preserves the performance of two models across different test sets. Figure~\ref{fig:evaluation:main_niah} showcases the NIAH evaluation results of our system, where \system also achieves the same level of accuracy compared to the dense baseline. In Table~\ref{tab:results:main_ruler}, we test \system with RULER benchmarks. Unless otherwise specified, we convert half of the attention heads into streaming heads and keep token budget for dynamic sparsity to 4096 for the benchmarks.



% Please add the following required packages to your document preamble:

% Beamer presentation requires \usepackage{colortbl} instead of \usepackage[table,xcdraw]{xcolor}
\begin{table*}[t]
\centering
\caption{Main Results. Eurus-2-7B-PRIME demonstrates the best reasoning ability.}
\label{tab:main_results}
\resizebox{\textwidth}{!}{
\begin{tabular}{lcccccc}
\toprule
\textbf{Model}                     & \textbf{AIME 2024}                           & \textbf{MATH-500} & \textbf{AMC}          & \textbf{Minerva Math} & \textbf{OlympiadBench} & \textbf{Avg.}          \\ \midrule
\textbf{GPT-4o}                    & 9.3                                          & 76.4              & 45.8                  & 36.8                  & \textbf{43.3}          & 43.3                   \\
\textbf{Llama-3.1-70B-Instruct}    & 16.7                                         & 64.6              & 30.1                  & 35.3                  & 31.9                   & 35.7                   \\
\textbf{Qwen-2.5-Math-7B-Instruct} & 13.3                                         & \textbf{79.8}     & 50.6                  & 34.6                  & 40.7                   & 43.8                   \\
\textbf{Eurus-2-7B-SFT}            & 3.3                                          & 65.1              & 30.1                  & 32.7                  & 29.8                   & 32.2                   \\
\textbf{Eurus-2-7B-PRIME}          & \textbf{26.7 {\color[HTML]{009901} (+23.3)}} & 79.2 {\color[HTML]{009901}(+14.1)}      & \textbf{57.8 {\color[HTML]{009901}(+27.7)}} & \textbf{38.6 {\color[HTML]{009901}(+5.9)}}  & 42.1 {\color[HTML]{009901}(+12.3) }          & \textbf{48.9 {\color[HTML]{009901}(+ 16.7)}} \\ \bottomrule
\end{tabular}
}
\end{table*}

\begin{figure}[t]
    \centering
    \includegraphics[width=\linewidth]{figure/5-results/main_speed_ctx_main_wo_trt.pdf}
    \caption{\textbf{Prefilling Speed Evaluation}. Performance comparison of long-sequence prefilling across different serving frameworks, normalized to \system's speed.
    }%
    \label{fig:evaluation:main_speed_prefill}
\end{figure}


\subsection{End-to-end Efficiency}

\textbf{Decoding Efficiency}. Figure~\ref{fig:evaluation:main_speed} presents the efficiency benchmarking results for the decoding stage. We use the same sparsity configurations as in \sect{sect:results:acc_eval}. Compared with the state-of-the-art serving systems, \system demonstrates significant and consistent efficiency improvements across different GPU platforms and model architectures. On Llama-3-8B and Minitron-4B, \system achieves 1.5$\times$ average speedup over vLLM. For MHA-based model Llama-2-7B, \system runs more than 2.0$\times$ faster than baselines on average. Additionally, we demonstrate that \system also functions well on other GPU devices such as L40S with Ada Lovelace Architecture. \system achieves up to 1.7$\times$ speedup over vLLM.

\textbf{Prefilling Efficiency}. In Figure~\ref{fig:evaluation:main_speed_prefill}, we compare the prefilling speed of \system against 4 baselines on Llama-3-8B and Llama-2-7B. \system maintains superior prefilling throughput across different sequence lengths. For instance, on Llama-2-7B, \system achieves an average of 1.8$\times$ higher prefilling throughput over vLLM. \system is also compatible with the prefilling dynamic sparsity in MInference, which we activated after 128K sequence length. 


\begin{table}[t]
\centering
\caption{\system achieves lower latency over Quest~\cite{tang2024quest} system in both prefilling stage and decoding stage. We benchmark the two systems on Llama-2-7B model, since Quest does not support GQA~\cite{ainslie2023gqa} architecture. }
\scalebox{0.82}
{
\begin{tabular}{cccccccccc}
\toprule
\multirow{2.5}{*}{Stage}      & \multirow{2.5}{*}{System} & \multicolumn{5}{c}{Sequence Length} \\ \cmidrule{3-7} 
 & & 4K & 8K & 16K & 32K & 64K \\
\midrule
\multirow{4}{*}{\makecell[l]{Prefilling \\ Latency (s)}} & Quest & 0.51    & 0.82   & 1.62  & 3.61    & OOM      \\ \cmidrule{2-7}
& \system  & 0.24     & 0.49     & 1.08    & 2.32    & 5.27    
 \\ \cmidrule{2-7}
     
& Speedup & 2.1 $\times$      & 1.7$\times$       & 1.5$\times$     & 1.6$\times$     & /        \\ \midrule

\multirow{4}{*}{\makecell[l]{Decoding \\ Latency (ms)}}   & Quest & 13.13   & 13.58   & 14.08  & 14.86  & OOM      \\ \cmidrule{2-7}
& \system  & 10.02   & 10.29    & 10.22  & 10.24  & 11.54    \\ \cmidrule{2-7}
& Speedup    & 1.3$\times$      & 1.3$\times$       & 1.4$\times$     & 1.5$\times$     & /        \\ 
\bottomrule
\end{tabular}
}
\label{tab:results:e2e_quest}
\end{table}


\subsection{End-to-End Comparison with Quest}

We also compares our system against Quest~\cite{tang2024quest} in Table~\ref{tab:results:e2e_quest}. Across different sequence lengths, \system consistently outperforms Quest in both prefilling (1.6-2.1$\times$ speedup) and decoding stages (1.3-1.5$\times$ speedup). 

