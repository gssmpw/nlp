\section{Introduction}
\label{sec:intro}

Graph sampling is commonly used in a variety of fields that are concerned with large scale graph data such as the Web graph or real-world social networks. In many cases, accessing or processing the full graph is impractical or impossible, and one has to effectively reduce the size of the graph (often through sampling the nodes) so that sophisticated data science methods can be practically deployed. For example, Web crawlers travel the graph of Web pages to model the Internet~\cite{boldi2004your}, epidemiologists sample a population to model the spread of diseases~\cite{cohen2003efficient}, and biologists sample cell interactions to understand behavior of cellular networks~\cite{aittokallio2006graph}. Sampling can also be used to simplify complex graph structures that would otherwise be difficult to work with~\cite{kurant2012coarse}. In particular, when dealing with large social networks, sampling reduces the size and structural complexity of the graph and allows for easier analysis~\cite{wang2011understanding}. 

However, graph sampling is not limited to selecting nodes at random, even though this approach may be appropriate in some cases. Ideally, the sampling process should be tailored to the specific research question while minimizing any unwanted bias. For example, sampling hard-to-reach communities poses unique challenges in that those who are successfully sampled may not be representative of the larger population~\cite{salganik2004sampling}. In addition, samples would ideally retain relevant properties of the original graph, but retention can be difficult~\cite{stumpf2005subnets}. Poor sampling practices can introduce unwanted bias~\cite{maiya2011benefits}, especially in more complex graphs where the importance of nodes is not necessarily equal~\cite{stutzbach2006sampling}. Carefully choosing a sampling method can help address these representation concerns.

A crucial way to understand the representativeness of graph sampling methods is through the perspective of the distribution of shortest-path distances (DSPD) from other nodes to a sample.
The DSPD provides valuable insights into how well the sample covers the graph, a measure of representativeness known as network reach~\cite{maiya2011benefits}.
Additionally, in graph neural networks~\cite{scarselli2008graph}, distance from sample has been shown to be an indicator of fairness~\cite{ma2021subgroup}, as machine learning models perform better on nodes closer to the labeled nodes than on those further away.
When dealing with hard-to-reach communities, models perform worse on those nodes further away from the sample~\cite{heckathorn2017network}. However, determining the DSPD can be difficult: the straightforward way to perform sampling then empirically calculate the shortest-path distances can be computationally expensive and requires full knowledge of the graph structure.

This paper provides, to our knowledge, the first framework for efficiently and accurately estimating the DSPD to a sample without accessing the full graph structure. Such a framework will help inform graph sampling decisions and enable future research into the DSPD to a sample. We demonstrate that our framework is more efficient compared to naive methods even assuming full knowledge of the graph, and that it is highly accurate for graphs without community structures. While we observe reduced accuracy for graphs with community structures, we show that that our framework remains highly reliable for downstream comparison tasks---a common use case when using our framework to inform sampling decisions.

The rest of the paper is organized as follows: Section \ref{sec:related} surveys existing works related to graph sampling and shortest-path distances. Section \ref{sec:framework} describes our framework for estimating the DSPD. Section \ref{sec:exp} describes the experiments we run to verify the accuracy and efficiency of our framework. Section \ref{sec:results} presents the results of our experiments. Section \ref{sec:conclusion} concludes the paper.
