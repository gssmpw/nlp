\section{Related Work}
\label{sec:related}

\subsection{Graph Sampling}

Graph sampling is a common technique used in cases where working with a full graph is impractical or impossible. These cases can arise when collecting data from the full network or working with the full network is computationally infeasible, or in social science contexts when working with a hard-to-reach communities, to name a few.
\citet{maiya2011benefits} discusses how well a sample covers a graph through \textit{network reach}, measuring it through \textit{discovery quotient}, which is the proportion of nodes that are in the sample or is a neighbor of a sample. However, we examine a more general idea via the full distribution of distances.

The effect of sampling methods on various metrics is also well-studied.
\citet{hu2013survey} conduct a survey of graph sampling techniques and provide ad hoc analysis of the impacts of the sampling techniques.
\citet{costenbader2003stability} look at the stability of centrality metrics under sampling. \citet{stumpf2005subnets} conclude that the scale-free property of a graph would not be preserved under random sampling.
% \citet{maiya2011benefits} present a more general study of the structural representativeness of graph samples and how biases in these samples may be desirable in certain contexts.
These studies all look at whether certain properties of the original graph would be preserved by the sampled graph. This paper instead examines the representativeness of nodes in a sample using the DSPD of other nodes to that sample. This property is not intrinsic to the original graph, but depends on both the graph and the sample, and therefore is not something to be preserved ``before and after'' sampling.

\subsection{Shortest-Path Distance}
Classical network science literature measures the mean of shortest-path distances in a graph and uses it to categorize networks \cite{watts1998collective}. 
A few studies look at the DSPD beyond the mean, including \citet{katzav2018distribution}, 
\citet{ventrella2018modeling}, and \citet{katzav2015analytical}. \citet{katzav2018distribution} obtain an analytical expression for the DSPD between nodes in subcritical Erd\"os-R\'enyi graphs, while \citet{katzav2015analytical} do so for Erd\"os-R\'enyi graphs in general. \citet{ventrella2018modeling} propose models that can be used to find the DSPD for scale-free networks. These studies all model the DSPD in the full graph, while we do so for the DSPD \textit{to a sample of nodes} selected from the full graph.

Recent work has made note of the importance of the DSPD to a sample. \citet{ma2021subgroup} find that the distance of an unlabeled node to the subgroup of labeled nodes in a graph is a good indicator of the performance of a graph neural network on that node, which helps motivate our investigation of the DSPD.
