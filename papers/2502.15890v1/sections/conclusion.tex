\section{Conclusion}
\label{sec:conclusion}

We propose a framework to accurately and efficiently estimate the DSPD to a sample in a graph by extending an existing framework that estimates the DSPD to a single random node. This framework does not require access to the full graph structure or the sample; rather, it only requires a model of the graph and the sampling method to obtain relevant degree distributions.

We evaluate the accuracy and efficiency of our framework by applying it to graphs with and without community structures and to random and snowball sampling. Our evaluation shows that the proposed framework is generally accurate, achieving high accuracy compared to empirically obtained distributions on graphs without community structures and large graphs with community structures. Even when accuracy decreases due to the complexity of handling community structures in smaller graphs, the framework remains useful for downstream tasks, consistently achieving perfect comparisons of mean distances across different sampling methods. Moreover, our framework is highly efficient. It eliminates the need to generate graphs and samples, and even when empirical methods have access to them, our framework can be up to an order of magnitude faster. While empirical methods scale linearly with graph size, our framework scales quadratically with sample size, making it especially effective for large graphs and small sample sizes. These advantages suggest that the proposed framework could be used to evaluate and select sampling methods for very large graphs without needing to perform the sampling itself. Future work will validate this approach using real-world graph datasets.
