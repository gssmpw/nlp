\section{Theoretical Framework}
\label{sec:framework}

In this section, we describe our framework for efficiently estimating the DSPD. Our framework works primarily within the configuration model~\citep{newman2003structure} setting, where the degree distribution of the nodes in the graph is specified. Notably, the configuration model does not account for community structure as it assumes the degree of each node is drawn independently of all the others from the specified distribution. We thus also present a method to account for community structure using our framework within the Stochastic Block Model (SBM)~\cite{holland1983stochastic} setting.

\subsection{Preliminaries}
\label{sec:preliminaries}

As preliminaries, we begin with a simpler question: the DSPD to a \textit{single random node} in a graph. This is equivalent to determining the DSPD in the entire graph. \citet{nitzan2016distance} derive such a distribution analytically in the configuration model setting.

Intuitively, the distribution is obtained by looking at ``shells'' around a single node recursively. The probability that the shortest path distance between two random points $i$ and $j$ is greater than $\ell$ can be calculated recursively with respect to $\ell$ as the probability that shortest-path distance between every neighbor of $i$ and $j$ is greater than $\ell-1$. This can be determined from the probability that the distance between any two nodes is greater than $\ell-1$ and the distribution of the number of neighbors of $i$. An additional complication is that the degree distribution of a randomly selected node is different from the degree distribution of a node known to have a neighbor, and this is resolved by maintaining two recursive equations.

Begin with the chain rule: for graphs with $N$ nodes,
\[P(d > \ell) = P(d > 0) \prod_{\ell'=1}^\ell m_{N, \ell'}, \]
where $d$ is the shortest-path distance between a pair of nodes $i, j$ and $m_{N, \ell} =  P(d > \ell | d > \ell-1)$ in a graph with $N$ nodes.

First,
\[m_{n, \ell} = \sum_{k=1}^{n-2} p(k) (\tilde{m}_{n-1,\ell-1})^k.\]
Here $p(k)$ is the degree distribution. For a pair of nodes $i,j$, let $r$ be a neighbor of $i$. $\tilde{m}_{n-1,\ell-1}$ is the probability that in a graph of size $n-1$ (i.e., the graph on the LHS but excluding $i$), the shortest-path distance from $r$ to $j$ is greater than $\ell-1$ given that the distance is greater than $\ell - 2$.

Intuitively, we are given that the distance from $i$ to $j$ is greater than $\ell-1$, so we know the distance from every neighbor of $i$ to $j$ is greater than $\ell-2$. To find the probability that the distance from $i$ to $j$ is greater than $\ell$ we thus additionally require that the distance from every neighbor of $i$ to $j$ to be greater than $\ell-1$.

Next,
\[\tilde{m}_{n, \ell} = \sum_{k=1}^{n-2} \frac{k}{c} p(k) (\tilde{m}_{n-1, \ell-1})^{k-1},\]
where $c = \sum_{k=1}^\infty kp(k)$ is a normalizing constant. Here the distribution $\frac{k}{c} p(k)$ is used as the degree distribution of $i$'s neighbor $r$ is not drawn from $p(k)$: the probability of a node being a neighbor of $i$ is proportional to its degree. The exponent is $k-1$ as one of $r$'s neighbors is $i$.

The base cases are
\[m_{n,1} = \sum_{k=1}^{n-1} p(k) \left(1 - \frac{1}{n-1}\right)^k,\]
and
\[\tilde{m}_{n,1} = \sum_{k=1}^{n-1} \frac{k}{c}p(k) \left(1 - \frac{1}{n-1}\right)^{k-1}.\]

We may now recursively solve for $m_{N,\ell'}$ for $\ell' = 1,2,\cdots,\ell$ and calculate $P(d > \ell)$.

\subsection{Multi-Node Sample Framework}

Our proposed framework extends the analysis above to the DSPD to a \emph{collection} of sampled nodes, rather than a single random node. For a sample size of $s$, the problem is more complex than simply considering the distribution of the minimum among $s$ samples from the single node DSPD, as the distances to each sample node are not necessarily independent. Additionally, our framework must accommodate cases where sample nodes are not selected randomly.

To overcome these challenges, we retain the idea of looking at shells centered around a single node, but view the center of the shell as a \textit{supernode} consisting of all the nodes in the sample contracted into one. More formally, let a graph be $G = (V, E)$, where $V$ is the set of nodes and $E\subseteq V\times V$ is the set of edges. Denote a set of sample nodes as $S\subseteq V$, all the edges of $G$ involving a node in $S$ as $E_S \subseteq E$. Let $\mathcal{N} := \{v\in V\setminus S \mid (u, v) \in E\}$, which is the set of neighbors of nodes in $S$ that are outside $S$. Then the graph (which we call a \textit{contracted graph}) after contracting the sample $S$ into a supernode $u_S$ is $G' = (V', E')$, where 
\[V' := V \setminus S \cup \{u_S\}, E' := E \setminus E_S \cup \{(u_S, v) \mid v\in N\}).\]

We then need to adjust the formula for $m_{n,\ell}$. In particular, instead of the degree distribution $p(k)$ in the original graph $G$, we draw from the degree distribution of the supernode $u_S$ in the contracted graph $G'$ (i.e., the distribution of the degree of a supernode in a contracted graph), denoted as $p_S(k)$. Additionally, let $N'$ be the number of nodes in the contracted graph. Then,
\[
P(d > \ell) = P(d > 0) \prod_{\ell'=1}^\ell m_{N', \ell'},
\]
as before. However, the recursions are now
\[
    m_{n, \ell} = \sum_{k=1}^{n-2} p_S(k) (\tilde{m}_{n-1,\ell-1})^k,
\]
and
\[
    \tilde{m}_{n, \ell} = \sum_{k=1}^{n-2} \frac{k}{c} p(k) (\tilde{m}_{n-1, \ell-1})^{k-1}.
\]
The base cases are
\[
    m_{n,1} = \sum_{k=1}^{n-1} p_S(k) \left(1 - \frac{1}{n-1}\right)^k,
\]
and
\[
    \tilde{m}_{n,1} = \sum_{k=1}^{n-1} \frac{k}{c}p(k) \left(1 - \frac{1}{n-1}\right)^{k-1}.
\]

It then remains to determine $p_S$, which depends on the structure of $G$ and the sampling method used to draw $S$. Examples of such determinations for the graphs and sampling methods involved in our experiments are presented in Section \ref{sec:sampling_methods}. Finally we may solve the recursion to obtain the DSPD to the collection of sampled nodes.

\subsection{Community Structure}

Note that our framework as presented above is not able to account for community structures: it assumes that the degree of a node is independent of all other nodes in the graph. We now present an extension to our framework that is able to account of community structure under the SBM setting. Here, graphs are split into blocks, with each edge within a block appearing with probability $p_1$ and each edge across two blocks appearing with probability $p_2$. Each block then models a community, and with $p_1 > p_2$ the connections are denser within a community than across communities. We leave applications on other community structure models as future work.

Challenges thus arise as there are two degree distributions to track: the degree within a block and the degree across a block. This is important as in our recursion we examine the degree distribution of a neighbor node, and now that distribution is different depending on whether that neighbor relation is via an within-block or across-block edge. To resolve this we keep an additional set of recursions. For an SBM graph $G$, contracted to $G'$ with $N'$, denote $p_w$ as the within-block degree distribution (i.e., degree distribution counting only within-block edges) and $p_a$ as the across-block degree distribution.

As before,
\[
P(d > \ell) = P(d > 0) \prod_{\ell'=1}^\ell m_{N', \ell'}.
\]
However, now
\begin{align*}
    m_{n, \ell} = \sum_{k_w + k_a < n-1} &p_{Sw}(k_w) p_{Sa}(k_a) \\
    &\left(\tilde{m}_{n-1, \ell-1}^{w}\right)^{k_w} \left(\tilde{m}_{n-1, \ell-1}^{a}\right)^{k_a},
\end{align*}
where $p_{Sw}$ is the sample supernode degree distribution counting only within-block edges in $G'$, $p_{Sa}$ is the sample supernode degree distribution counting only outside-block edges in $G'$, $\tilde{m}_{n-1, \ell-1}^w$ is the probability that a node in the next shell reached via a within-block edge is greater than $\ell-1$ away from the node given that it is greater than $\ell - 2$ away, and $\tilde{m}_{n-1, \ell-1}^a$ is the probability that a node in the next shell reached via an across-block edge is greater than $\ell-1$ away from the node given that it is greater than $\ell - 2$ away. The recursions for $\tilde{m}^w$ and $\tilde{m}^a$ are
\begin{align*}
    \tilde{m}_{n, \ell}^{w} = \sum_{k_w + k_a < n-1} &\frac{k_w p_w(k_w)}{c_w} p_a(k_a) \\
    &\left(\tilde{m}_{n-1, \ell-1}^{w}\right)^{k_w-1} \left(\tilde{m}_{n-1, \ell-1}^{a}\right)^{k_a}, \\
    \tilde{m}_{n, \ell}^{o} = \sum_{k_w + k_a < n-1} &p_w(k_w) \frac{k_a p_a(k_a)}{c_a} \\
    &\left(\tilde{m}_{n-1, \ell-1}^{w}\right)^{k_w} \left(\tilde{m}_{n-1, \ell-1}^{a}\right)^{k_a-1},
\end{align*}
with $c_w$ and $c_a$ being $\sum_{0 < k < n-1}kp_w(k)$ and $\sum_{0 < k < n-1}kp_a(k)$, the normalizing constants after weighting $p_w$ and $p_a$ by the degree. Depending on whether the node in the next shell was arrived at via a within-block edge or across-block edge, the within-block or across-block degree distribution is the weighted weighted version rather than the unweighted degree distribution, respectively. The base cases are then
\begin{align*}
    m_{n, 1} &= \sum_{k_w + k_a < n} p_{Sw}(k_w) p_{Sa}(k_a) \left(1 - \frac{1}{Nn-1}\right)^{k_w + k_a}, \\
    \tilde{m}_{n,1}^{w} &= \sum_{k_w + k_a < n} \frac{k_w p_w(k_w)}{c_w} p_a(k_a) \left(1 - \frac{1}{n-1}\right)^{k_w + k_a - 1}, \\
    \tilde{m}_{n,1}^{a} &= \sum_{k_w + k_a < N} p_w(k_w) \frac{k_a p_a(k_a)}{c_a} \left(1 - \frac{1}{n-1}\right)^{k_w + k_a - 1}.
\end{align*}

Similarly, it remains to determine $p_{Sw}$ and $p_{Sa}$, which depends on $p_1, p_2$ and the sampling method used to draw $S$. Details are presented in Section \ref{sec:sampling_methods}. Finally we may solve the recursion to get the results.\\

Notably, these recursions only require the degree distribution of the original graph and the distribution of the degree of the sample supernode in the contracted graph. Such distributions can be derived from assumptions on the generative model of the graph and the sampling method, without access to the full graph structure or sample necessary for empirically calculating the DSPD.

% \textcolor{red}{Add remarks here about what needs to be known to estimate the DSPD? Degree distribution of the original graph; degree distribution of the contracted graph? What else? These distributions can be estimated given assumptions about the generative model of the graph, without actual knowing the full graph structure or the sample. This is contrast to empirically computing the DSPD which needs to full graph structure as well as the sample. }