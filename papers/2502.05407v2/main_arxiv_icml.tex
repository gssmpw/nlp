\documentclass[letterpaper,11pt]{article}
\usepackage{graphicx} % Required for inserting images
%\usepackage{icml2025}

%
\setlength\unitlength{1mm}
\newcommand{\twodots}{\mathinner {\ldotp \ldotp}}
% bb font symbols
\newcommand{\Rho}{\mathrm{P}}
\newcommand{\Tau}{\mathrm{T}}

\newfont{\bbb}{msbm10 scaled 700}
\newcommand{\CCC}{\mbox{\bbb C}}

\newfont{\bb}{msbm10 scaled 1100}
\newcommand{\CC}{\mbox{\bb C}}
\newcommand{\PP}{\mbox{\bb P}}
\newcommand{\RR}{\mbox{\bb R}}
\newcommand{\QQ}{\mbox{\bb Q}}
\newcommand{\ZZ}{\mbox{\bb Z}}
\newcommand{\FF}{\mbox{\bb F}}
\newcommand{\GG}{\mbox{\bb G}}
\newcommand{\EE}{\mbox{\bb E}}
\newcommand{\NN}{\mbox{\bb N}}
\newcommand{\KK}{\mbox{\bb K}}
\newcommand{\HH}{\mbox{\bb H}}
\newcommand{\SSS}{\mbox{\bb S}}
\newcommand{\UU}{\mbox{\bb U}}
\newcommand{\VV}{\mbox{\bb V}}


\newcommand{\yy}{\mathbbm{y}}
\newcommand{\xx}{\mathbbm{x}}
\newcommand{\zz}{\mathbbm{z}}
\newcommand{\sss}{\mathbbm{s}}
\newcommand{\rr}{\mathbbm{r}}
\newcommand{\pp}{\mathbbm{p}}
\newcommand{\qq}{\mathbbm{q}}
\newcommand{\ww}{\mathbbm{w}}
\newcommand{\hh}{\mathbbm{h}}
\newcommand{\vvv}{\mathbbm{v}}

% Vectors

\newcommand{\av}{{\bf a}}
\newcommand{\bv}{{\bf b}}
\newcommand{\cv}{{\bf c}}
\newcommand{\dv}{{\bf d}}
\newcommand{\ev}{{\bf e}}
\newcommand{\fv}{{\bf f}}
\newcommand{\gv}{{\bf g}}
\newcommand{\hv}{{\bf h}}
\newcommand{\iv}{{\bf i}}
\newcommand{\jv}{{\bf j}}
\newcommand{\kv}{{\bf k}}
\newcommand{\lv}{{\bf l}}
\newcommand{\mv}{{\bf m}}
\newcommand{\nv}{{\bf n}}
\newcommand{\ov}{{\bf o}}
\newcommand{\pv}{{\bf p}}
\newcommand{\qv}{{\bf q}}
\newcommand{\rv}{{\bf r}}
\newcommand{\sv}{{\bf s}}
\newcommand{\tv}{{\bf t}}
\newcommand{\uv}{{\bf u}}
\newcommand{\wv}{{\bf w}}
\newcommand{\vv}{{\bf v}}
\newcommand{\xv}{{\bf x}}
\newcommand{\yv}{{\bf y}}
\newcommand{\zv}{{\bf z}}
\newcommand{\zerov}{{\bf 0}}
\newcommand{\onev}{{\bf 1}}

% Matrices

\newcommand{\Am}{{\bf A}}
\newcommand{\Bm}{{\bf B}}
\newcommand{\Cm}{{\bf C}}
\newcommand{\Dm}{{\bf D}}
\newcommand{\Em}{{\bf E}}
\newcommand{\Fm}{{\bf F}}
\newcommand{\Gm}{{\bf G}}
\newcommand{\Hm}{{\bf H}}
\newcommand{\Id}{{\bf I}}
\newcommand{\Jm}{{\bf J}}
\newcommand{\Km}{{\bf K}}
\newcommand{\Lm}{{\bf L}}
\newcommand{\Mm}{{\bf M}}
\newcommand{\Nm}{{\bf N}}
\newcommand{\Om}{{\bf O}}
\newcommand{\Pm}{{\bf P}}
\newcommand{\Qm}{{\bf Q}}
\newcommand{\Rm}{{\bf R}}
\newcommand{\Sm}{{\bf S}}
\newcommand{\Tm}{{\bf T}}
\newcommand{\Um}{{\bf U}}
\newcommand{\Wm}{{\bf W}}
\newcommand{\Vm}{{\bf V}}
\newcommand{\Xm}{{\bf X}}
\newcommand{\Ym}{{\bf Y}}
\newcommand{\Zm}{{\bf Z}}

% Calligraphic

\newcommand{\Ac}{{\cal A}}
\newcommand{\Bc}{{\cal B}}
\newcommand{\Cc}{{\cal C}}
\newcommand{\Dc}{{\cal D}}
\newcommand{\Ec}{{\cal E}}
\newcommand{\Fc}{{\cal F}}
\newcommand{\Gc}{{\cal G}}
\newcommand{\Hc}{{\cal H}}
\newcommand{\Ic}{{\cal I}}
\newcommand{\Jc}{{\cal J}}
\newcommand{\Kc}{{\cal K}}
\newcommand{\Lc}{{\cal L}}
\newcommand{\Mc}{{\cal M}}
\newcommand{\Nc}{{\cal N}}
\newcommand{\nc}{{\cal n}}
\newcommand{\Oc}{{\cal O}}
\newcommand{\Pc}{{\cal P}}
\newcommand{\Qc}{{\cal Q}}
\newcommand{\Rc}{{\cal R}}
\newcommand{\Sc}{{\cal S}}
\newcommand{\Tc}{{\cal T}}
\newcommand{\Uc}{{\cal U}}
\newcommand{\Wc}{{\cal W}}
\newcommand{\Vc}{{\cal V}}
\newcommand{\Xc}{{\cal X}}
\newcommand{\Yc}{{\cal Y}}
\newcommand{\Zc}{{\cal Z}}

% Bold greek letters

\newcommand{\alphav}{\hbox{\boldmath$\alpha$}}
\newcommand{\betav}{\hbox{\boldmath$\beta$}}
\newcommand{\gammav}{\hbox{\boldmath$\gamma$}}
\newcommand{\deltav}{\hbox{\boldmath$\delta$}}
\newcommand{\etav}{\hbox{\boldmath$\eta$}}
\newcommand{\lambdav}{\hbox{\boldmath$\lambda$}}
\newcommand{\epsilonv}{\hbox{\boldmath$\epsilon$}}
\newcommand{\nuv}{\hbox{\boldmath$\nu$}}
\newcommand{\muv}{\hbox{\boldmath$\mu$}}
\newcommand{\zetav}{\hbox{\boldmath$\zeta$}}
\newcommand{\phiv}{\hbox{\boldmath$\phi$}}
\newcommand{\psiv}{\hbox{\boldmath$\psi$}}
\newcommand{\thetav}{\hbox{\boldmath$\theta$}}
\newcommand{\tauv}{\hbox{\boldmath$\tau$}}
\newcommand{\omegav}{\hbox{\boldmath$\omega$}}
\newcommand{\xiv}{\hbox{\boldmath$\xi$}}
\newcommand{\sigmav}{\hbox{\boldmath$\sigma$}}
\newcommand{\piv}{\hbox{\boldmath$\pi$}}
\newcommand{\rhov}{\hbox{\boldmath$\rho$}}
\newcommand{\upsilonv}{\hbox{\boldmath$\upsilon$}}

\newcommand{\Gammam}{\hbox{\boldmath$\Gamma$}}
\newcommand{\Lambdam}{\hbox{\boldmath$\Lambda$}}
\newcommand{\Deltam}{\hbox{\boldmath$\Delta$}}
\newcommand{\Sigmam}{\hbox{\boldmath$\Sigma$}}
\newcommand{\Phim}{\hbox{\boldmath$\Phi$}}
\newcommand{\Pim}{\hbox{\boldmath$\Pi$}}
\newcommand{\Psim}{\hbox{\boldmath$\Psi$}}
\newcommand{\Thetam}{\hbox{\boldmath$\Theta$}}
\newcommand{\Omegam}{\hbox{\boldmath$\Omega$}}
\newcommand{\Xim}{\hbox{\boldmath$\Xi$}}


% Sans Serif small case

\newcommand{\Gsf}{{\sf G}}

\newcommand{\asf}{{\sf a}}
\newcommand{\bsf}{{\sf b}}
\newcommand{\csf}{{\sf c}}
\newcommand{\dsf}{{\sf d}}
\newcommand{\esf}{{\sf e}}
\newcommand{\fsf}{{\sf f}}
\newcommand{\gsf}{{\sf g}}
\newcommand{\hsf}{{\sf h}}
\newcommand{\isf}{{\sf i}}
\newcommand{\jsf}{{\sf j}}
\newcommand{\ksf}{{\sf k}}
\newcommand{\lsf}{{\sf l}}
\newcommand{\msf}{{\sf m}}
\newcommand{\nsf}{{\sf n}}
\newcommand{\osf}{{\sf o}}
\newcommand{\psf}{{\sf p}}
\newcommand{\qsf}{{\sf q}}
\newcommand{\rsf}{{\sf r}}
\newcommand{\ssf}{{\sf s}}
\newcommand{\tsf}{{\sf t}}
\newcommand{\usf}{{\sf u}}
\newcommand{\wsf}{{\sf w}}
\newcommand{\vsf}{{\sf v}}
\newcommand{\xsf}{{\sf x}}
\newcommand{\ysf}{{\sf y}}
\newcommand{\zsf}{{\sf z}}


% mixed symbols

\newcommand{\sinc}{{\hbox{sinc}}}
\newcommand{\diag}{{\hbox{diag}}}
\renewcommand{\det}{{\hbox{det}}}
\newcommand{\trace}{{\hbox{tr}}}
\newcommand{\sign}{{\hbox{sign}}}
\renewcommand{\arg}{{\hbox{arg}}}
\newcommand{\var}{{\hbox{var}}}
\newcommand{\cov}{{\hbox{cov}}}
\newcommand{\Ei}{{\rm E}_{\rm i}}
\renewcommand{\Re}{{\rm Re}}
\renewcommand{\Im}{{\rm Im}}
\newcommand{\eqdef}{\stackrel{\Delta}{=}}
\newcommand{\defines}{{\,\,\stackrel{\scriptscriptstyle \bigtriangleup}{=}\,\,}}
\newcommand{\<}{\left\langle}
\renewcommand{\>}{\right\rangle}
\newcommand{\herm}{{\sf H}}
\newcommand{\trasp}{{\sf T}}
\newcommand{\transp}{{\sf T}}
\renewcommand{\vec}{{\rm vec}}
\newcommand{\Psf}{{\sf P}}
\newcommand{\SINR}{{\sf SINR}}
\newcommand{\SNR}{{\sf SNR}}
\newcommand{\MMSE}{{\sf MMSE}}
\newcommand{\REF}{{\RED [REF]}}

% Markov chain
\usepackage{stmaryrd} % for \mkv 
\newcommand{\mkv}{-\!\!\!\!\minuso\!\!\!\!-}

% Colors

\newcommand{\RED}{\color[rgb]{1.00,0.10,0.10}}
\newcommand{\BLUE}{\color[rgb]{0,0,0.90}}
\newcommand{\GREEN}{\color[rgb]{0,0.80,0.20}}

%%%%%%%%%%%%%%%%%%%%%%%%%%%%%%%%%%%%%%%%%%
\usepackage{hyperref}
\hypersetup{
    bookmarks=true,         % show bookmarks bar?
    unicode=false,          % non-Latin characters in AcrobatÕs bookmarks
    pdftoolbar=true,        % show AcrobatÕs toolbar?
    pdfmenubar=true,        % show AcrobatÕs menu?
    pdffitwindow=false,     % window fit to page when opened
    pdfstartview={FitH},    % fits the width of the page to the window
%    pdftitle={My title},    % title
%    pdfauthor={Author},     % author
%    pdfsubject={Subject},   % subject of the document
%    pdfcreator={Creator},   % creator of the document
%    pdfproducer={Producer}, % producer of the document
%    pdfkeywords={keyword1} {key2} {key3}, % list of keywords
    pdfnewwindow=true,      % links in new window
    colorlinks=true,       % false: boxed links; true: colored links
    linkcolor=red,          % color of internal links (change box color with linkbordercolor)
    citecolor=green,        % color of links to bibliography
    filecolor=blue,      % color of file links
    urlcolor=blue           % color of external links
}
%%%%%%%%%%%%%%%%%%%%%%%%%%%%%%%%%%%%%%%%%%%


\usepackage[utf8]{inputenc}
%\usepackage[margin=.8in]{geometry}
\usepackage{tikz-3dplot}
\usepackage{algpseudocode}
\usepackage{varwidth}
\usepackage{xcolor} 
\usepackage{empheq}
\usepackage{tikz}
\usepackage{framed}
\definecolor{shadecolor}{gray}{0.9}
\def\eps{\varepsilon}
\newcommand{\curly}[1]{\left\{#1\right\}}
\newcommand{\set}[1]{\left\{#1\right\}}
\newcommand{\abs}[1]{\left|#1\right|}
\newcommand{\round}[1]{\left(#1\right)}
\newcommand{\inner}[1]{\left<#1\right>}
\newcommand{\norm}[1]{\left\|#1\right\|}
\newcommand{\red}[1]{\textcolor{red}{#1}}
\def\Banach{\mathcal{B}}
\def\Hilbert{\mathcal{H}}
\def\Real{\mathbb{R}}
\def\tran{^\top}
\def\alphavec{\bm{\alpha}}
\def\betavec{\bm{\beta}}
\def\x{\bm{x}}
\def\y{\bm{y}}
\def\r{\bm{r}}
\def\Kmat{\mathbf{K}}
\setlength{\parskip}{2ex}
\def\prox{\textsf{prox}}
\def\wb{\overline}
\usepackage{booktabs}

\newcommand{\dataset}{{\cal D}}
\newcommand{\fracpartial}[2]{\frac{\partial #1}{\partial  #2}}
\usepackage[margin=.8in]{geometry}
\usepackage{xcolor} 
\usepackage{empheq}
\usepackage{framed}
\definecolor{shadecolor}{gray}{0.9}
\def\eps{\varepsilon}
\def\Banach{\mathcal{B}}
\def\Hilbert{\mathcal{H}}
\def\Real{\mathbb{R}}
\def\tran{^\top}
\def\alphavec{\bm{\alpha}}
\def\betavec{\bm{\beta}}
\def\x{\bm{x}}
\def\y{\bm{y}}
\def\r{\bm{r}}
\def\Kmat{\mathbf{K}}
\setlength{\parskip}{2ex}
\def\prox{\textsf{prox}}
\def\wb{\overline}
\usepackage{booktabs}
%\usepackage{natbib}
%\hypersetup{hidelinks,colorlinks,citecolor=blue}

\def\mc{\mathcal}
\def\mf{\mathfrak}
\def\msf{\mathsf}
\def\mrm{\mathrm}

\usetikzlibrary{calc, shadings} 
\usetikzlibrary{positioning,arrows.meta}


\title{The Complexity of Learning Sparse Superposed Features with Feedback}
\author{Akash Kumar\\\small{Department of Computer Science and Engineering}\\ \small{University of California-San Diego}\\ \texttt{akk002@ucsd.edu}}
\date{}
\usepackage[round]{natbib}
%\hypersetup{hidelinks,colorlinks,citecolor=red}

\def\mc{\mathcal}
\def\mf{\mathfrak}
\def\msf{\mathsf}
\def\mrm{\mathrm}
\usepackage{verbatim}
% \documentclass{article}
% \usepackage[accepted]{icml2025}
% \usepackage{graphicx} % Required for inserting images
% 
%
\setlength\unitlength{1mm}
\newcommand{\twodots}{\mathinner {\ldotp \ldotp}}
% bb font symbols
\newcommand{\Rho}{\mathrm{P}}
\newcommand{\Tau}{\mathrm{T}}

\newfont{\bbb}{msbm10 scaled 700}
\newcommand{\CCC}{\mbox{\bbb C}}

\newfont{\bb}{msbm10 scaled 1100}
\newcommand{\CC}{\mbox{\bb C}}
\newcommand{\PP}{\mbox{\bb P}}
\newcommand{\RR}{\mbox{\bb R}}
\newcommand{\QQ}{\mbox{\bb Q}}
\newcommand{\ZZ}{\mbox{\bb Z}}
\newcommand{\FF}{\mbox{\bb F}}
\newcommand{\GG}{\mbox{\bb G}}
\newcommand{\EE}{\mbox{\bb E}}
\newcommand{\NN}{\mbox{\bb N}}
\newcommand{\KK}{\mbox{\bb K}}
\newcommand{\HH}{\mbox{\bb H}}
\newcommand{\SSS}{\mbox{\bb S}}
\newcommand{\UU}{\mbox{\bb U}}
\newcommand{\VV}{\mbox{\bb V}}


\newcommand{\yy}{\mathbbm{y}}
\newcommand{\xx}{\mathbbm{x}}
\newcommand{\zz}{\mathbbm{z}}
\newcommand{\sss}{\mathbbm{s}}
\newcommand{\rr}{\mathbbm{r}}
\newcommand{\pp}{\mathbbm{p}}
\newcommand{\qq}{\mathbbm{q}}
\newcommand{\ww}{\mathbbm{w}}
\newcommand{\hh}{\mathbbm{h}}
\newcommand{\vvv}{\mathbbm{v}}

% Vectors

\newcommand{\av}{{\bf a}}
\newcommand{\bv}{{\bf b}}
\newcommand{\cv}{{\bf c}}
\newcommand{\dv}{{\bf d}}
\newcommand{\ev}{{\bf e}}
\newcommand{\fv}{{\bf f}}
\newcommand{\gv}{{\bf g}}
\newcommand{\hv}{{\bf h}}
\newcommand{\iv}{{\bf i}}
\newcommand{\jv}{{\bf j}}
\newcommand{\kv}{{\bf k}}
\newcommand{\lv}{{\bf l}}
\newcommand{\mv}{{\bf m}}
\newcommand{\nv}{{\bf n}}
\newcommand{\ov}{{\bf o}}
\newcommand{\pv}{{\bf p}}
\newcommand{\qv}{{\bf q}}
\newcommand{\rv}{{\bf r}}
\newcommand{\sv}{{\bf s}}
\newcommand{\tv}{{\bf t}}
\newcommand{\uv}{{\bf u}}
\newcommand{\wv}{{\bf w}}
\newcommand{\vv}{{\bf v}}
\newcommand{\xv}{{\bf x}}
\newcommand{\yv}{{\bf y}}
\newcommand{\zv}{{\bf z}}
\newcommand{\zerov}{{\bf 0}}
\newcommand{\onev}{{\bf 1}}

% Matrices

\newcommand{\Am}{{\bf A}}
\newcommand{\Bm}{{\bf B}}
\newcommand{\Cm}{{\bf C}}
\newcommand{\Dm}{{\bf D}}
\newcommand{\Em}{{\bf E}}
\newcommand{\Fm}{{\bf F}}
\newcommand{\Gm}{{\bf G}}
\newcommand{\Hm}{{\bf H}}
\newcommand{\Id}{{\bf I}}
\newcommand{\Jm}{{\bf J}}
\newcommand{\Km}{{\bf K}}
\newcommand{\Lm}{{\bf L}}
\newcommand{\Mm}{{\bf M}}
\newcommand{\Nm}{{\bf N}}
\newcommand{\Om}{{\bf O}}
\newcommand{\Pm}{{\bf P}}
\newcommand{\Qm}{{\bf Q}}
\newcommand{\Rm}{{\bf R}}
\newcommand{\Sm}{{\bf S}}
\newcommand{\Tm}{{\bf T}}
\newcommand{\Um}{{\bf U}}
\newcommand{\Wm}{{\bf W}}
\newcommand{\Vm}{{\bf V}}
\newcommand{\Xm}{{\bf X}}
\newcommand{\Ym}{{\bf Y}}
\newcommand{\Zm}{{\bf Z}}

% Calligraphic

\newcommand{\Ac}{{\cal A}}
\newcommand{\Bc}{{\cal B}}
\newcommand{\Cc}{{\cal C}}
\newcommand{\Dc}{{\cal D}}
\newcommand{\Ec}{{\cal E}}
\newcommand{\Fc}{{\cal F}}
\newcommand{\Gc}{{\cal G}}
\newcommand{\Hc}{{\cal H}}
\newcommand{\Ic}{{\cal I}}
\newcommand{\Jc}{{\cal J}}
\newcommand{\Kc}{{\cal K}}
\newcommand{\Lc}{{\cal L}}
\newcommand{\Mc}{{\cal M}}
\newcommand{\Nc}{{\cal N}}
\newcommand{\nc}{{\cal n}}
\newcommand{\Oc}{{\cal O}}
\newcommand{\Pc}{{\cal P}}
\newcommand{\Qc}{{\cal Q}}
\newcommand{\Rc}{{\cal R}}
\newcommand{\Sc}{{\cal S}}
\newcommand{\Tc}{{\cal T}}
\newcommand{\Uc}{{\cal U}}
\newcommand{\Wc}{{\cal W}}
\newcommand{\Vc}{{\cal V}}
\newcommand{\Xc}{{\cal X}}
\newcommand{\Yc}{{\cal Y}}
\newcommand{\Zc}{{\cal Z}}

% Bold greek letters

\newcommand{\alphav}{\hbox{\boldmath$\alpha$}}
\newcommand{\betav}{\hbox{\boldmath$\beta$}}
\newcommand{\gammav}{\hbox{\boldmath$\gamma$}}
\newcommand{\deltav}{\hbox{\boldmath$\delta$}}
\newcommand{\etav}{\hbox{\boldmath$\eta$}}
\newcommand{\lambdav}{\hbox{\boldmath$\lambda$}}
\newcommand{\epsilonv}{\hbox{\boldmath$\epsilon$}}
\newcommand{\nuv}{\hbox{\boldmath$\nu$}}
\newcommand{\muv}{\hbox{\boldmath$\mu$}}
\newcommand{\zetav}{\hbox{\boldmath$\zeta$}}
\newcommand{\phiv}{\hbox{\boldmath$\phi$}}
\newcommand{\psiv}{\hbox{\boldmath$\psi$}}
\newcommand{\thetav}{\hbox{\boldmath$\theta$}}
\newcommand{\tauv}{\hbox{\boldmath$\tau$}}
\newcommand{\omegav}{\hbox{\boldmath$\omega$}}
\newcommand{\xiv}{\hbox{\boldmath$\xi$}}
\newcommand{\sigmav}{\hbox{\boldmath$\sigma$}}
\newcommand{\piv}{\hbox{\boldmath$\pi$}}
\newcommand{\rhov}{\hbox{\boldmath$\rho$}}
\newcommand{\upsilonv}{\hbox{\boldmath$\upsilon$}}

\newcommand{\Gammam}{\hbox{\boldmath$\Gamma$}}
\newcommand{\Lambdam}{\hbox{\boldmath$\Lambda$}}
\newcommand{\Deltam}{\hbox{\boldmath$\Delta$}}
\newcommand{\Sigmam}{\hbox{\boldmath$\Sigma$}}
\newcommand{\Phim}{\hbox{\boldmath$\Phi$}}
\newcommand{\Pim}{\hbox{\boldmath$\Pi$}}
\newcommand{\Psim}{\hbox{\boldmath$\Psi$}}
\newcommand{\Thetam}{\hbox{\boldmath$\Theta$}}
\newcommand{\Omegam}{\hbox{\boldmath$\Omega$}}
\newcommand{\Xim}{\hbox{\boldmath$\Xi$}}


% Sans Serif small case

\newcommand{\Gsf}{{\sf G}}

\newcommand{\asf}{{\sf a}}
\newcommand{\bsf}{{\sf b}}
\newcommand{\csf}{{\sf c}}
\newcommand{\dsf}{{\sf d}}
\newcommand{\esf}{{\sf e}}
\newcommand{\fsf}{{\sf f}}
\newcommand{\gsf}{{\sf g}}
\newcommand{\hsf}{{\sf h}}
\newcommand{\isf}{{\sf i}}
\newcommand{\jsf}{{\sf j}}
\newcommand{\ksf}{{\sf k}}
\newcommand{\lsf}{{\sf l}}
\newcommand{\msf}{{\sf m}}
\newcommand{\nsf}{{\sf n}}
\newcommand{\osf}{{\sf o}}
\newcommand{\psf}{{\sf p}}
\newcommand{\qsf}{{\sf q}}
\newcommand{\rsf}{{\sf r}}
\newcommand{\ssf}{{\sf s}}
\newcommand{\tsf}{{\sf t}}
\newcommand{\usf}{{\sf u}}
\newcommand{\wsf}{{\sf w}}
\newcommand{\vsf}{{\sf v}}
\newcommand{\xsf}{{\sf x}}
\newcommand{\ysf}{{\sf y}}
\newcommand{\zsf}{{\sf z}}


% mixed symbols

\newcommand{\sinc}{{\hbox{sinc}}}
\newcommand{\diag}{{\hbox{diag}}}
\renewcommand{\det}{{\hbox{det}}}
\newcommand{\trace}{{\hbox{tr}}}
\newcommand{\sign}{{\hbox{sign}}}
\renewcommand{\arg}{{\hbox{arg}}}
\newcommand{\var}{{\hbox{var}}}
\newcommand{\cov}{{\hbox{cov}}}
\newcommand{\Ei}{{\rm E}_{\rm i}}
\renewcommand{\Re}{{\rm Re}}
\renewcommand{\Im}{{\rm Im}}
\newcommand{\eqdef}{\stackrel{\Delta}{=}}
\newcommand{\defines}{{\,\,\stackrel{\scriptscriptstyle \bigtriangleup}{=}\,\,}}
\newcommand{\<}{\left\langle}
\renewcommand{\>}{\right\rangle}
\newcommand{\herm}{{\sf H}}
\newcommand{\trasp}{{\sf T}}
\newcommand{\transp}{{\sf T}}
\renewcommand{\vec}{{\rm vec}}
\newcommand{\Psf}{{\sf P}}
\newcommand{\SINR}{{\sf SINR}}
\newcommand{\SNR}{{\sf SNR}}
\newcommand{\MMSE}{{\sf MMSE}}
\newcommand{\REF}{{\RED [REF]}}

% Markov chain
\usepackage{stmaryrd} % for \mkv 
\newcommand{\mkv}{-\!\!\!\!\minuso\!\!\!\!-}

% Colors

\newcommand{\RED}{\color[rgb]{1.00,0.10,0.10}}
\newcommand{\BLUE}{\color[rgb]{0,0,0.90}}
\newcommand{\GREEN}{\color[rgb]{0,0.80,0.20}}

%%%%%%%%%%%%%%%%%%%%%%%%%%%%%%%%%%%%%%%%%%
\usepackage{hyperref}
\hypersetup{
    bookmarks=true,         % show bookmarks bar?
    unicode=false,          % non-Latin characters in AcrobatÕs bookmarks
    pdftoolbar=true,        % show AcrobatÕs toolbar?
    pdfmenubar=true,        % show AcrobatÕs menu?
    pdffitwindow=false,     % window fit to page when opened
    pdfstartview={FitH},    % fits the width of the page to the window
%    pdftitle={My title},    % title
%    pdfauthor={Author},     % author
%    pdfsubject={Subject},   % subject of the document
%    pdfcreator={Creator},   % creator of the document
%    pdfproducer={Producer}, % producer of the document
%    pdfkeywords={keyword1} {key2} {key3}, % list of keywords
    pdfnewwindow=true,      % links in new window
    colorlinks=true,       % false: boxed links; true: colored links
    linkcolor=red,          % color of internal links (change box color with linkbordercolor)
    citecolor=green,        % color of links to bibliography
    filecolor=blue,      % color of file links
    urlcolor=blue           % color of external links
}
%%%%%%%%%%%%%%%%%%%%%%%%%%%%%%%%%%%%%%%%%%%


% \newcommand{\theHalgorithm}{\arabic{algorithm}}
\usepackage{hyperref}
\hypersetup{hidelinks,colorlinks,citecolor=blue}
% \usepackage[utf8]{inputenc}
% %\usepackage[margin=1in]{geometry}
% \usepackage{tikz-3dplot}
% %\usepackage{algpseudocode}
% \usepackage{varwidth}
% \usepackage{xcolor} 
% \usepackage{empheq}
% \usepackage{tikz}
% \usepackage{framed}
% \definecolor{shadecolor}{gray}{0.9}
% \def\eps{\varepsilon}
% \newcommand{\curly}[1]{\left\{#1\right\}}
% \newcommand{\set}[1]{\left\{#1\right\}}
% \newcommand{\abs}[1]{\left|#1\right|}
% \newcommand{\round}[1]{\left(#1\right)}
% \newcommand{\inner}[1]{\left<#1\right>}
% \newcommand{\norm}[1]{\left\|#1\right\|}
% \newcommand{\red}[1]{\textcolor{red}{#1}}
% \def\Banach{\mathcal{B}}
% \def\Hilbert{\mathcal{H}}
% \def\Real{\mathbb{R}}
% \def\tran{^\top}
% \def\alphavec{\bm{\alpha}}
% \def\betavec{\bm{\beta}}
% \def\x{\bm{x}}
% \def\y{\bm{y}}
% \def\r{\bm{r}}
% \def\Kmat{\mathbf{K}}
% \setlength{\parskip}{2ex}
% \def\prox{\textsf{prox}}
% \def\wb{\overline}
% \usepackage{booktabs}

% \newcommand{\dataset}{{\cal D}}
% \newcommand{\fracpartial}[2]{\frac{\partial #1}{\partial  #2}}
% %\usepackage[margin=1in]{geometry}
% \usepackage{xcolor} 
% \usepackage{empheq}
% \usepackage{framed}
% \definecolor{shadecolor}{gray}{0.9}
% \def\eps{\varepsilon}
% \def\Banach{\mathcal{B}}
% \def\Hilbert{\mathcal{H}}
% \def\Real{\mathbb{R}}
% \def\tran{^\top}
% \def\alphavec{\bm{\alpha}}
% \def\betavec{\bm{\beta}}
% \def\x{\bm{x}}
% \def\y{\bm{y}}
% \def\r{\bm{r}}
% \def\Kmat{\mathbf{K}}
% \setlength{\parskip}{2ex}
% \def\prox{\textsf{prox}}
% \def\wb{\overline}
% \usepackage{booktabs}
% \usepackage{natbib}
% %\hypersetup{hidelinks,colorlinks,citecolor=blue}

% \def\mc{\mathcal}
% \def\mf{\mathfrak}
% \def\msf{\mathsf}
% \def\mrm{\mathrm}

% \usetikzlibrary{calc, shadings} 
% \usetikzlibrary{positioning,arrows.meta}


% %\title{On query complexity of learning a Mahalanobis distance function with a teacher}
% %\author{}
% %\date{}
% \usepackage{natbib}
% %\hypersetup{hidelinks,colorlinks,citecolor=red}

% \def\mc{\mathcal}
% \def\mf{\mathfrak}
% \def\msf{\mathsf}
% \def\mrm{\mathrm}
% \usepackage{verbatim}
% \usepackage[textsize=tiny]{todonotes}


% The \icmltitle you define below is probably too long as a header.
% Therefore, a short form for the running title is supplied here:
%\icmltitlerunning{Learning Sparse Superposed Features with Feedback}

\begin{document}
\maketitle
%\twocolumn[
%\icmltitle{The Complexity of Learning Sparse Superposed Features with Feedback}

% It is OKAY to include author information, even for blind
% submissions: the style file will automatically remove it for you
% unless you've provided the [accepted] option to the icml2025
% package.

% List of affiliations: The first argument should be a (short)
% identifier you will use later to specify author affiliations
% Academic affiliations should list Department, University, City, Region, Country
% Industry affiliations should list Company, City, Region, Country

% You can specify symbols, otherwise they are numbered in order.
% Ideally, you should not use this facility. Affiliations will be numbered
% in order of appearance and this is the preferred way.
% \icmlsetsymbol{equal}{*}

% \begin{icmlauthorlist}
% \icmlauthor{Akash Kumar}{yyy}
% %\icmlauthor{Firstname2 Lastname2}{equal,yyy,comp}
% \icmlauthor{Firstname3 Lastname3}{comp}
% %\icmlauthor{Firstname4 Lastname4}{sch}
% % \icmlauthor{Firstname5 Lastname5}{yyy}
% % \icmlauthor{Firstname6 Lastname6}{sch,yyy,comp}
% % \icmlauthor{Firstname7 Lastname7}{comp}
% %\icmlauthor{}{sch}
% % \icmlauthor{Firstname8 Lastname8}{sch}
% % \icmlauthor{Firstname8 Lastname8}{yyy,comp}
% \icmlauthor{}{sch}
% \icmlauthor{}{sch}
% \end{icmlauthorlist}

% \icmlaffiliation{yyy}{Department of Computer Science, University of California-San Diego, La Jolla, USA}
% \icmlaffiliation{comp}{Company Name, Location, Country}
% \icmlaffiliation{sch}{School of ZZZ, Institute of WWW, Location, Country}

% \icmlcorrespondingauthor{Firstname1 Lastname1}{first1.last1@xxx.edu}
% \icmlcorrespondingauthor{Firstname2 Lastname2}{first2.last2@www.uk}

% % You may provide any keywords that you
% % find helpful for describing your paper; these are used to populate
% % the "keywords" metadata in the PDF but will not be shown in the document
% \icmlkeywords{Machine Learning, ICML}

% \vskip 0.3in
% ]

% this must go after the closing bracket ] following \twocolumn[ ...

% This command actually creates the footnote in the first column
% listing the affiliations and the copyright notice.
% The command takes one argument, which is text to display at the start of the footnote.
% The \icmlEqualContribution command is standard text for equal contribution.
% Remove it (just {}) if you do not need this facility.

%\printAffiliationsAndNotice{}  % leave blank if no need to mention equal contribution
%\printAffiliationsAndNotice{\icmlEqualContribution} % otherwise use the standard text.

% \section{Problems: Learning a metric}
\begin{abstract}
%In machine learning, the predictive power of a method hinges on the extraction of meaningful features from the ambient sample space. 
The success of deep networks is crucially attributed to their ability to capture latent features within a representation space. In this work, we investigate whether the underlying learned features of a model can be efficiently retrieved through feedback from an agent, such as a large language model (LLM), in the form of relative \tt{triplet comparisons}. These features may represent various constructs, including dictionaries in LLMs or components of a covariance matrix of Mahalanobis distances. We analyze the feedback complexity associated with learning a feature matrix in sparse settings. Our results establish tight bounds when the agent is permitted to construct activations and demonstrate strong upper bounds in sparse scenarios when the agent's feedback is limited to distributional information. We validate our theoretical findings through experiments on two distinct applications: feature recovery from Recursive Feature Machine-trained models and dictionary extraction from sparse autoencoders trained on Large Language Models.%To complement our theoretical findings, we conduct experiments on feature retrieval from models trained via Recursive Feature Machines and dictionaries derived from sparse autoencoders (SAEs) trained for Large Language Models. 

%Deep networks' success fundamentally relies on their ability to capture latent features within representation spaces. We investigate whether these learned features can be efficiently recovered through relative triplet comparisons provided by an agent, such as a large language model (LLM). Our framework applies broadly to different feature representations, from LLM dictionaries to Mahalanobis distance matrices, unifying their treatment through feature matrix learning.
%We analyze the feedback complexity of learning feature matrices in sparse settings, establishing two key results. First, we prove tight bounds for scenarios where the agent can construct activations freely. Second, we demonstrate strong upper bounds when the agent must rely on distributional feedback under sparsity constraints. We validate our theoretical findings through experiments on two distinct applications: feature recovery from Recursive Feature Machine-trained models and dictionary extraction from sparse autoencoders trained on Large Language Models.
%\akash{put github if possible}
\end{abstract}
\section{Introduction}


\begin{figure}[t]
\centering
\includegraphics[width=0.6\columnwidth]{figures/evaluation_desiderata_V5.pdf}
\vspace{-0.5cm}
\caption{\systemName is a platform for conducting realistic evaluations of code LLMs, collecting human preferences of coding models with real users, real tasks, and in realistic environments, aimed at addressing the limitations of existing evaluations.
}
\label{fig:motivation}
\end{figure}

\begin{figure*}[t]
\centering
\includegraphics[width=\textwidth]{figures/system_design_v2.png}
\caption{We introduce \systemName, a VSCode extension to collect human preferences of code directly in a developer's IDE. \systemName enables developers to use code completions from various models. The system comprises a) the interface in the user's IDE which presents paired completions to users (left), b) a sampling strategy that picks model pairs to reduce latency (right, top), and c) a prompting scheme that allows diverse LLMs to perform code completions with high fidelity.
Users can select between the top completion (green box) using \texttt{tab} or the bottom completion (blue box) using \texttt{shift+tab}.}
\label{fig:overview}
\end{figure*}

As model capabilities improve, large language models (LLMs) are increasingly integrated into user environments and workflows.
For example, software developers code with AI in integrated developer environments (IDEs)~\citep{peng2023impact}, doctors rely on notes generated through ambient listening~\citep{oberst2024science}, and lawyers consider case evidence identified by electronic discovery systems~\citep{yang2024beyond}.
Increasing deployment of models in productivity tools demands evaluation that more closely reflects real-world circumstances~\citep{hutchinson2022evaluation, saxon2024benchmarks, kapoor2024ai}.
While newer benchmarks and live platforms incorporate human feedback to capture real-world usage, they almost exclusively focus on evaluating LLMs in chat conversations~\citep{zheng2023judging,dubois2023alpacafarm,chiang2024chatbot, kirk2024the}.
Model evaluation must move beyond chat-based interactions and into specialized user environments.



 

In this work, we focus on evaluating LLM-based coding assistants. 
Despite the popularity of these tools---millions of developers use Github Copilot~\citep{Copilot}---existing
evaluations of the coding capabilities of new models exhibit multiple limitations (Figure~\ref{fig:motivation}, bottom).
Traditional ML benchmarks evaluate LLM capabilities by measuring how well a model can complete static, interview-style coding tasks~\citep{chen2021evaluating,austin2021program,jain2024livecodebench, white2024livebench} and lack \emph{real users}. 
User studies recruit real users to evaluate the effectiveness of LLMs as coding assistants, but are often limited to simple programming tasks as opposed to \emph{real tasks}~\citep{vaithilingam2022expectation,ross2023programmer, mozannar2024realhumaneval}.
Recent efforts to collect human feedback such as Chatbot Arena~\citep{chiang2024chatbot} are still removed from a \emph{realistic environment}, resulting in users and data that deviate from typical software development processes.
We introduce \systemName to address these limitations (Figure~\ref{fig:motivation}, top), and we describe our three main contributions below.


\textbf{We deploy \systemName in-the-wild to collect human preferences on code.} 
\systemName is a Visual Studio Code extension, collecting preferences directly in a developer's IDE within their actual workflow (Figure~\ref{fig:overview}).
\systemName provides developers with code completions, akin to the type of support provided by Github Copilot~\citep{Copilot}. 
Over the past 3 months, \systemName has served over~\completions suggestions from 10 state-of-the-art LLMs, 
gathering \sampleCount~votes from \userCount~users.
To collect user preferences,
\systemName presents a novel interface that shows users paired code completions from two different LLMs, which are determined based on a sampling strategy that aims to 
mitigate latency while preserving coverage across model comparisons.
Additionally, we devise a prompting scheme that allows a diverse set of models to perform code completions with high fidelity.
See Section~\ref{sec:system} and Section~\ref{sec:deployment} for details about system design and deployment respectively.



\textbf{We construct a leaderboard of user preferences and find notable differences from existing static benchmarks and human preference leaderboards.}
In general, we observe that smaller models seem to overperform in static benchmarks compared to our leaderboard, while performance among larger models is mixed (Section~\ref{sec:leaderboard_calculation}).
We attribute these differences to the fact that \systemName is exposed to users and tasks that differ drastically from code evaluations in the past. 
Our data spans 103 programming languages and 24 natural languages as well as a variety of real-world applications and code structures, while static benchmarks tend to focus on a specific programming and natural language and task (e.g. coding competition problems).
Additionally, while all of \systemName interactions contain code contexts and the majority involve infilling tasks, a much smaller fraction of Chatbot Arena's coding tasks contain code context, with infilling tasks appearing even more rarely. 
We analyze our data in depth in Section~\ref{subsec:comparison}.



\textbf{We derive new insights into user preferences of code by analyzing \systemName's diverse and distinct data distribution.}
We compare user preferences across different stratifications of input data (e.g., common versus rare languages) and observe which affect observed preferences most (Section~\ref{sec:analysis}).
For example, while user preferences stay relatively consistent across various programming languages, they differ drastically between different task categories (e.g. frontend/backend versus algorithm design).
We also observe variations in user preference due to different features related to code structure 
(e.g., context length and completion patterns).
We open-source \systemName and release a curated subset of code contexts.
Altogether, our results highlight the necessity of model evaluation in realistic and domain-specific settings.





%\vspace{-3mm}
\section{Preliminaries}
\label{sec:preliminaries}
\subsection{Formulation of Collaborative Perception}
\label{sec:formulation}

% \vspace{-3mm}
In this section, we formulate collaborative perception and give the pipeline of our CP system. Specifically, let $\mathcal{X}^N$ denote the set of $N$ CAVs in the CP system. CAVs in $\mathcal{X}$ can be divided into two categories: the ego CAV and helping CAVs. The ego CAV is the one that needs to perceive its surrounding environment, while helping CAVs are the ones that send their complementary sensing information to the ego CAV to help it enhance its perception performance.
Thus, each CAV can be an ego one and helping one, depending on its role in a perception process. We assume that each CAV is equipped with a feature encoder $f_\mathtt{{encoder}}(\cdot)$, a feature aggregator $f_\mathtt{{agg}}(\cdot)$, and a feature decoder $f_\mathtt{{decoder}}(\cdot)$. For the $i$-th CAV in the set $\mathcal{X}$, the raw observation is denoted as $\mathbf{O}_i$ (such as camera images and LiDAR point clouds), and the final perception results are denoted as $\mathbf{Y}_i$. The CP pipeline of the $i$-th CAV can be described as follows.
\begin{enumerate}
    % \vspace{-3mm}
    \setlength{\itemsep}{0pt}
    \setlength{\parskip}{0pt}
    \setlength{\parsep}{0pt}
    \item \textit{Observation Encoding}: Each CAV encodes its raw observation $\mathbf{O}_j$ into an initial feature map $\mathbf{F}_j = f_\mathtt{{encoder}}(\mathbf{O}_j)$, where $j \in \mathcal{X}^N$.
    \item \textit{Intermediate Feature Transmission}: Helping CAVs transmit their intermediate features to the ego CAV: $\mathbf{F}_{j\rightarrow i}=\mathbf{\Gamma}_{j\rightarrow i}(\mathbf{F}_j),\  j\in \mathcal{X}^N, j\neq i,$
    where $\mathbf{\Gamma}_{j\rightarrow i}(\cdot)$ denotes a transmitter that conveys the $j$-th CAV's intermediate feature $\mathbf{F}_j$ to the ego CAV, while performing a spatial transformation. $\mathbf{F}_{j\rightarrow i}$ is the spatially aligned feature in the $i$-th CAV's coordinate.
    \item \textit{Feature Aggregation}: The ego CAV receives all the intermediate features and fuses them into a unified observational feature $\mathbf{F}_\mathtt{fused}=f_\mathtt{agg}(\mathbf{F}_{0\rightarrow i}, \{\mathbf{F}_{j\rightarrow i}\}_{j\neq i,\  j\in \mathcal{X}^N})$.
    \item \textit{Perception Decoding}: Finally, the ego CAV decodes the unified observational feature $\mathbf{F}_\mathtt{fused}$ into the final perception results $\mathbf{Y}=f_\mathtt{decoder}(\mathbf{F}_\mathtt{fused})$.
    % \vspace{-4mm}
\end{enumerate}


\begin{figure*}[t]
    % \vspace{-5mm}
    \centering
    % \fbox{\rule{0pt}{1.8in} \rule{0.9\linewidth}{0pt}}
    \includegraphics[width=.9\linewidth]{fig/CPDataGenerationPipeline.png}
    \vspace{-3mm}
    \caption{\textbf{Automatic Data Generation and Annotation Pipeline.} We first train a robust LiDAR collaborative object detector. Then, we discard the detection head and decoder and only keep the backbone as the intermediate feature generator. The data generation pipeline is shown in (a), (b), and (c), where (a) is the intermediate feature generation, (b) is the attack implementation, and (c) is the pair generation and saving.}
    \label{fig:data_generation}
    \vspace{-5mm} 
\end{figure*}



\subsection{Adversarial Threat Model}

Our focus is on the operation of an  intermediate-fusion collaboration scheme, where an attacker introduces designed adversarial perturbations into the intermediate features to mislead the perception of the ego CAV. Since an attacker participates in the collaborative system with local perception model installation, we assume they have white-box access to the model parameters. The attack procedure in each frame follows four sequential phases.
\begin{enumerate}
    \vspace{-4mm}
    \setlength{\itemsep}{0pt}
    \setlength{\parskip}{0pt}
    \setlength{\parsep}{0pt}
    \item \textit{Local Perception Phase}: All agents, including the malicious one, process their sensing data independently and extract intermediate features using feature encoders. 
    \vspace{-1mm}
    \begin{equation}
    \mathbf{F}_k = f_\mathtt{encoder}(\mathbf{O}_k), \quad k \in \mathcal{X}^N
    \end{equation}
    \vspace{-1mm}
    This phase operates in parallel without inter-agent communication.

    \item \textit{Feature Communication Phase}: All agents broadcast their extracted features through the network. Malicious agent $k$ collects feature information $\{\mathbf{F}_{j\rightarrow i}\}$ from other agents. Feature-level transmission ensures minimal communication overhead compared to raw sensor data exchange.

    \item \textit{Attack Generation Phase}: A malicious agent executes the attack by first perturbing its local features and then propagating them through the collaborative perception pipeline described in Section \ref{sec:formulation}.
    The attacker aims to optimize the perturbation $\delta$ through an iterative process. The optimization objective is formulated as:
    \vspace{-1mm}
    \begin{equation}
        \vspace{-2mm}
        \begin{aligned}
        \mathop{\arg\max}_{\delta} \mathcal{L}(\mathbf{Y}^\delta, \mathbf{Y}^\mathtt{gt}),
        \quad \mathtt{s.t.}\quad  \|\delta\|\leq \Delta
        \end{aligned}
    \end{equation}
    where $\Delta$ bounds the perturbation magnitude to maintain attack stealthiness. The total loss function is designed to aggregate adversarial losses over all object proposals, targeting both classification and localization aspects:
    \vspace{-1mm}
    \begin{equation}
        \vspace{-2mm}
        \mathcal{L}(\mathbf{Y}^\delta, \mathbf{Y}^\mathtt{gt}) = \sum_{p \in \mathbf{Y}^\delta} \mathcal{L}_\mathtt{adv}(p, p^\mathtt{gt})
    \end{equation}
    For each proposal $p$ with the highest confidence class $c = \mathop{\arg\max}\{p_i\}$, we leverage a class-specific adversarial loss following \citep{tuAdversarialAttacksMultiAgent2021}:
    \begin{equation*}
        % \vspace{-2mm}
        \mathcal{L}_\mathtt{adv}(p', p) = \begin{cases}
            -\log(1 - p'_c)\cdot\eta & c \neq k,\ p_c > \tau_1\\
            -\lambda p'_c\log(1 - p'_c) & c = k,\ p_c > \tau_2\\
            0 & \text{otherwise}
        \end{cases}
    \end{equation*}
    where $\eta$ represents the IoU between perturbed and original proposals to consider spatial accuracy, $\tau_1$ and $\tau_2$ are confidence thresholds for different attack scenarios, $\lambda$ balances the importance of different attack objectives, and $k$ denotes the background class.

    \item \textit{Defense and Final Perception Phase}: The ego vehicle integrates all received feature information, including potentially corrupted ones, to complete the final object detection task. Note that we focus exclusively on CP-specific vulnerabilities, excluding physical sensor attacks (e.g., LiDAR or GPS spoofing), which are general threats to CAVs. We also assume communication channels are secured with proper cryptographic protection.
\end{enumerate}

%\subsection{Does metric learning provably help in learning a task?}

%Several works have stipulated sample complexity for generalization bounds and excess risk for metric and similarity learning for classification tasks~\cite{Guo2013GuaranteedCV,Bellet2012SimilarityLF}
Understanding a question of the form: "if metric learning provably helps in classification" would require analyzing whether metric learning helps achieve 
 better bounds asymptotically. Ideally, this would require either providing a sharper bound for classification tasks compared to the known algorithms w/ a priori learning a metric or showing that it can be a method to beat lower bounds for learning specific tasks. Note, metric learning generally comes up with a computational overhead, for it is designed as a convex/non-convex objective. Now, even if there is an oracle that solves that objective for `free', beating a lower bound could be tricky for the general technique for achieving the lower bounds is showing statistical bottlenecks in terms of VC-dim of the hypothesis class or information theoretic arguments. 

 \subsection{Sample complexity of learning a metric}

Sample complexity for generalization bound has been established for both general metric learning problem and metric learning for classification: $k$NN in the imbalanced data setting~\cite{viola20,Gautheron2020MetricLF}, and linear classification~\cite{Bellet2012SimilarityLF}; Mahalanobis distance metric~\cite{Verma2015SampleCO}, non-linear metric learning in sparse and bounded amplification regimes (embeddings corresponding to neural networks )~\cite{Kozdoba2021DimensionFG}.
More recently, learning a Mahalanobis metric has been further explored under various settings, e.g. low dimensional metric learning~\cite{Mason2017LearningLM}, multitask metric learning~\cite{Wang2019MultitaskML}, under noise labels~\cite{Alishahi2023LinearDM}.

\subsubsection{Learning a metric for perceptron in the noisy setting}

\begin{shaded}
\noindent Setting: Given a sample space $\cX \in \reals^d$, with a distribution $P_X$. Label set $\cY = \curly{-1,1}$ with noise $P_{Y|X}$. $S^1$ and $S^{-1}$ denotes examples with two labels.\\

\noindent\textit{at time} $t = 1,2,\ldots$
\begin{enumerate}
    \item environment provides a labeled input $(x_t,y_t) \sim P_{X,Y}$
    \item learner updates $S^1_t:= S^1_{t-1} \cup \curly{x_t}$ or $S^{-1}_t:= S^{-1}_{t-1}  \cup \{x_t\}$ 
    \item learner updates $M_t \leftarrow \textsf{Gradient-Update}(M_{t-1}, S^1_t,S^{-1}_t)$ 
    \item if $w_{t-1}$ misclassifies $M_t^{1/2}x_t$ then learner updates
    \begin{enumerate}
        \item $w_t \leftarrow w_{t-1} + y_t(M_t^{1/2}x_t)$
    \end{enumerate}
    %learner updates the teaching set $\mathcal{T}_t := \mathcal{T}_{t-1} \cup \{(x_i^t,x_j^t,x_k^t)\}$
    %\item  learner makes updates $M_t \leftarrow M_{t-1} + \eta\cdot \frac{1}{t} \sum_{i=1}^t (x_i - x_k)(x_i - x_k)^{\top} - (x_i - x_k)(x_i - x_j)^{\top}$
    %\item learner updates $\hat{M_t} \leftarrow \textsf{EigenDecompose}(M_t)$
    %\item teacher receives $\ell(M_t, \mathcal{T}_t)$
\end{enumerate}
\end{shaded}
Here, for a given labeled example $(x,y)$,
\begin{align*}
\textsf{Gradient-Update}(M_{t-1}, S^1_t,S^{-1}_t) = M_{t-1} + \eta y\cdot\paren{ \sum_{x' \in S^1_t} (x - x')(x-x')^{\top} - \sum_{x' \in S^{+1}_t} (x - x')(x-x')^{\top}} 
\end{align*}
One problem of interest is if this approach achieves a better bound on the excess risk for some loss function $\ell$:
\begin{align*}
    \expctover{(x,y)\sim P_{X,Y}}{\ell(w_t(x),y)} - \expctover{(x,y)\sim P_{X,Y}}{\ell(f^*(x),y)}
\end{align*}


\section{RELATED WORK}
\label{sec:relatedwork}
In this section, we describe the previous works related to our proposal, which are divided into two parts. In Section~\ref{sec:relatedwork_exoplanet}, we present a review of approaches based on machine learning techniques for the detection of planetary transit signals. Section~\ref{sec:relatedwork_attention} provides an account of the approaches based on attention mechanisms applied in Astronomy.\par

\subsection{Exoplanet detection}
\label{sec:relatedwork_exoplanet}
Machine learning methods have achieved great performance for the automatic selection of exoplanet transit signals. One of the earliest applications of machine learning is a model named Autovetter \citep{MCcauliff}, which is a random forest (RF) model based on characteristics derived from Kepler pipeline statistics to classify exoplanet and false positive signals. Then, other studies emerged that also used supervised learning. \cite{mislis2016sidra} also used a RF, but unlike the work by \citet{MCcauliff}, they used simulated light curves and a box least square \citep[BLS;][]{kovacs2002box}-based periodogram to search for transiting exoplanets. \citet{thompson2015machine} proposed a k-nearest neighbors model for Kepler data to determine if a given signal has similarity to known transits. Unsupervised learning techniques were also applied, such as self-organizing maps (SOM), proposed \citet{armstrong2016transit}; which implements an architecture to segment similar light curves. In the same way, \citet{armstrong2018automatic} developed a combination of supervised and unsupervised learning, including RF and SOM models. In general, these approaches require a previous phase of feature engineering for each light curve. \par

%DL is a modern data-driven technology that automatically extracts characteristics, and that has been successful in classification problems from a variety of application domains. The architecture relies on several layers of NNs of simple interconnected units and uses layers to build increasingly complex and useful features by means of linear and non-linear transformation. This family of models is capable of generating increasingly high-level representations \citep{lecun2015deep}.

The application of DL for exoplanetary signal detection has evolved rapidly in recent years and has become very popular in planetary science.  \citet{pearson2018} and \citet{zucker2018shallow} developed CNN-based algorithms that learn from synthetic data to search for exoplanets. Perhaps one of the most successful applications of the DL models in transit detection was that of \citet{Shallue_2018}; who, in collaboration with Google, proposed a CNN named AstroNet that recognizes exoplanet signals in real data from Kepler. AstroNet uses the training set of labelled TCEs from the Autovetter planet candidate catalog of Q1–Q17 data release 24 (DR24) of the Kepler mission \citep{catanzarite2015autovetter}. AstroNet analyses the data in two views: a ``global view'', and ``local view'' \citep{Shallue_2018}. \par


% The global view shows the characteristics of the light curve over an orbital period, and a local view shows the moment at occurring the transit in detail

%different = space-based

Based on AstroNet, researchers have modified the original AstroNet model to rank candidates from different surveys, specifically for Kepler and TESS missions. \citet{ansdell2018scientific} developed a CNN trained on Kepler data, and included for the first time the information on the centroids, showing that the model improves performance considerably. Then, \citet{osborn2020rapid} and \citet{yu2019identifying} also included the centroids information, but in addition, \citet{osborn2020rapid} included information of the stellar and transit parameters. Finally, \citet{rao2021nigraha} proposed a pipeline that includes a new ``half-phase'' view of the transit signal. This half-phase view represents a transit view with a different time and phase. The purpose of this view is to recover any possible secondary eclipse (the object hiding behind the disk of the primary star).


%last pipeline applies a procedure after the prediction of the model to obtain new candidates, this process is carried out through a series of steps that include the evaluation with Discovery and Validation of Exoplanets (DAVE) \citet{kostov2019discovery} that was adapted for the TESS telescope.\par
%



\subsection{Attention mechanisms in astronomy}
\label{sec:relatedwork_attention}
Despite the remarkable success of attention mechanisms in sequential data, few papers have exploited their advantages in astronomy. In particular, there are no models based on attention mechanisms for detecting planets. Below we present a summary of the main applications of this modeling approach to astronomy, based on two points of view; performance and interpretability of the model.\par
%Attention mechanisms have not yet been explored in all sub-areas of astronomy. However, recent works show a successful application of the mechanism.
%performance

The application of attention mechanisms has shown improvements in the performance of some regression and classification tasks compared to previous approaches. One of the first implementations of the attention mechanism was to find gravitational lenses proposed by \citet{thuruthipilly2021finding}. They designed 21 self-attention-based encoder models, where each model was trained separately with 18,000 simulated images, demonstrating that the model based on the Transformer has a better performance and uses fewer trainable parameters compared to CNN. A novel application was proposed by \citet{lin2021galaxy} for the morphological classification of galaxies, who used an architecture derived from the Transformer, named Vision Transformer (VIT) \citep{dosovitskiy2020image}. \citet{lin2021galaxy} demonstrated competitive results compared to CNNs. Another application with successful results was proposed by \citet{zerveas2021transformer}; which first proposed a transformer-based framework for learning unsupervised representations of multivariate time series. Their methodology takes advantage of unlabeled data to train an encoder and extract dense vector representations of time series. Subsequently, they evaluate the model for regression and classification tasks, demonstrating better performance than other state-of-the-art supervised methods, even with data sets with limited samples.

%interpretation
Regarding the interpretability of the model, a recent contribution that analyses the attention maps was presented by \citet{bowles20212}, which explored the use of group-equivariant self-attention for radio astronomy classification. Compared to other approaches, this model analysed the attention maps of the predictions and showed that the mechanism extracts the brightest spots and jets of the radio source more clearly. This indicates that attention maps for prediction interpretation could help experts see patterns that the human eye often misses. \par

In the field of variable stars, \citet{allam2021paying} employed the mechanism for classifying multivariate time series in variable stars. And additionally, \citet{allam2021paying} showed that the activation weights are accommodated according to the variation in brightness of the star, achieving a more interpretable model. And finally, related to the TESS telescope, \citet{morvan2022don} proposed a model that removes the noise from the light curves through the distribution of attention weights. \citet{morvan2022don} showed that the use of the attention mechanism is excellent for removing noise and outliers in time series datasets compared with other approaches. In addition, the use of attention maps allowed them to show the representations learned from the model. \par

Recent attention mechanism approaches in astronomy demonstrate comparable results with earlier approaches, such as CNNs. At the same time, they offer interpretability of their results, which allows a post-prediction analysis. \par


\vspace{-3mm}
\section{Preliminaries}
\label{sec:preliminaries}
\subsection{Formulation of Collaborative Perception}
\label{sec:formulation}

% \vspace{-3mm}
In this section, we formulate collaborative perception and give the pipeline of our CP system. Specifically, let $\mathcal{X}^N$ denote the set of $N$ CAVs in the CP system. CAVs in $\mathcal{X}$ can be divided into two categories: the ego CAV and helping CAVs. The ego CAV is the one that needs to perceive its surrounding environment, while helping CAVs are the ones that send their complementary sensing information to the ego CAV to help it enhance its perception performance.
Thus, each CAV can be an ego one and helping one, depending on its role in a perception process. We assume that each CAV is equipped with a feature encoder $f_\mathtt{{encoder}}(\cdot)$, a feature aggregator $f_\mathtt{{agg}}(\cdot)$, and a feature decoder $f_\mathtt{{decoder}}(\cdot)$. For the $i$-th CAV in the set $\mathcal{X}$, the raw observation is denoted as $\mathbf{O}_i$ (such as camera images and LiDAR point clouds), and the final perception results are denoted as $\mathbf{Y}_i$. The CP pipeline of the $i$-th CAV can be described as follows.
\begin{enumerate}
    % \vspace{-3mm}
    \setlength{\itemsep}{0pt}
    \setlength{\parskip}{0pt}
    \setlength{\parsep}{0pt}
    \item \textit{Observation Encoding}: Each CAV encodes its raw observation $\mathbf{O}_j$ into an initial feature map $\mathbf{F}_j = f_\mathtt{{encoder}}(\mathbf{O}_j)$, where $j \in \mathcal{X}^N$.
    \item \textit{Intermediate Feature Transmission}: Helping CAVs transmit their intermediate features to the ego CAV: $\mathbf{F}_{j\rightarrow i}=\mathbf{\Gamma}_{j\rightarrow i}(\mathbf{F}_j),\  j\in \mathcal{X}^N, j\neq i,$
    where $\mathbf{\Gamma}_{j\rightarrow i}(\cdot)$ denotes a transmitter that conveys the $j$-th CAV's intermediate feature $\mathbf{F}_j$ to the ego CAV, while performing a spatial transformation. $\mathbf{F}_{j\rightarrow i}$ is the spatially aligned feature in the $i$-th CAV's coordinate.
    \item \textit{Feature Aggregation}: The ego CAV receives all the intermediate features and fuses them into a unified observational feature $\mathbf{F}_\mathtt{fused}=f_\mathtt{agg}(\mathbf{F}_{0\rightarrow i}, \{\mathbf{F}_{j\rightarrow i}\}_{j\neq i,\  j\in \mathcal{X}^N})$.
    \item \textit{Perception Decoding}: Finally, the ego CAV decodes the unified observational feature $\mathbf{F}_\mathtt{fused}$ into the final perception results $\mathbf{Y}=f_\mathtt{decoder}(\mathbf{F}_\mathtt{fused})$.
    % \vspace{-4mm}
\end{enumerate}


\begin{figure*}[t]
    % \vspace{-5mm}
    \centering
    % \fbox{\rule{0pt}{1.8in} \rule{0.9\linewidth}{0pt}}
    \includegraphics[width=.9\linewidth]{fig/CPDataGenerationPipeline.png}
    \vspace{-3mm}
    \caption{\textbf{Automatic Data Generation and Annotation Pipeline.} We first train a robust LiDAR collaborative object detector. Then, we discard the detection head and decoder and only keep the backbone as the intermediate feature generator. The data generation pipeline is shown in (a), (b), and (c), where (a) is the intermediate feature generation, (b) is the attack implementation, and (c) is the pair generation and saving.}
    \label{fig:data_generation}
    \vspace{-5mm} 
\end{figure*}



\subsection{Adversarial Threat Model}

Our focus is on the operation of an  intermediate-fusion collaboration scheme, where an attacker introduces designed adversarial perturbations into the intermediate features to mislead the perception of the ego CAV. Since an attacker participates in the collaborative system with local perception model installation, we assume they have white-box access to the model parameters. The attack procedure in each frame follows four sequential phases.
\begin{enumerate}
    \vspace{-4mm}
    \setlength{\itemsep}{0pt}
    \setlength{\parskip}{0pt}
    \setlength{\parsep}{0pt}
    \item \textit{Local Perception Phase}: All agents, including the malicious one, process their sensing data independently and extract intermediate features using feature encoders. 
    \vspace{-1mm}
    \begin{equation}
    \mathbf{F}_k = f_\mathtt{encoder}(\mathbf{O}_k), \quad k \in \mathcal{X}^N
    \end{equation}
    \vspace{-1mm}
    This phase operates in parallel without inter-agent communication.

    \item \textit{Feature Communication Phase}: All agents broadcast their extracted features through the network. Malicious agent $k$ collects feature information $\{\mathbf{F}_{j\rightarrow i}\}$ from other agents. Feature-level transmission ensures minimal communication overhead compared to raw sensor data exchange.

    \item \textit{Attack Generation Phase}: A malicious agent executes the attack by first perturbing its local features and then propagating them through the collaborative perception pipeline described in Section \ref{sec:formulation}.
    The attacker aims to optimize the perturbation $\delta$ through an iterative process. The optimization objective is formulated as:
    \vspace{-1mm}
    \begin{equation}
        \vspace{-2mm}
        \begin{aligned}
        \mathop{\arg\max}_{\delta} \mathcal{L}(\mathbf{Y}^\delta, \mathbf{Y}^\mathtt{gt}),
        \quad \mathtt{s.t.}\quad  \|\delta\|\leq \Delta
        \end{aligned}
    \end{equation}
    where $\Delta$ bounds the perturbation magnitude to maintain attack stealthiness. The total loss function is designed to aggregate adversarial losses over all object proposals, targeting both classification and localization aspects:
    \vspace{-1mm}
    \begin{equation}
        \vspace{-2mm}
        \mathcal{L}(\mathbf{Y}^\delta, \mathbf{Y}^\mathtt{gt}) = \sum_{p \in \mathbf{Y}^\delta} \mathcal{L}_\mathtt{adv}(p, p^\mathtt{gt})
    \end{equation}
    For each proposal $p$ with the highest confidence class $c = \mathop{\arg\max}\{p_i\}$, we leverage a class-specific adversarial loss following \citep{tuAdversarialAttacksMultiAgent2021}:
    \begin{equation*}
        % \vspace{-2mm}
        \mathcal{L}_\mathtt{adv}(p', p) = \begin{cases}
            -\log(1 - p'_c)\cdot\eta & c \neq k,\ p_c > \tau_1\\
            -\lambda p'_c\log(1 - p'_c) & c = k,\ p_c > \tau_2\\
            0 & \text{otherwise}
        \end{cases}
    \end{equation*}
    where $\eta$ represents the IoU between perturbed and original proposals to consider spatial accuracy, $\tau_1$ and $\tau_2$ are confidence thresholds for different attack scenarios, $\lambda$ balances the importance of different attack objectives, and $k$ denotes the background class.

    \item \textit{Defense and Final Perception Phase}: The ego vehicle integrates all received feature information, including potentially corrupted ones, to complete the final object detection task. Note that we focus exclusively on CP-specific vulnerabilities, excluding physical sensor attacks (e.g., LiDAR or GPS spoofing), which are general threats to CAVs. We also assume communication channels are secured with proper cryptographic protection.
\end{enumerate}

%\section{Feature learning, Linear representation hypothesis}
\akash{
\begin{itemize}
    \item Results stated with sparsity assumptions
    \item How does it differ from the work for Mahalanobis distances?
\end{itemize}
}

In the feature learning with a superposition net, we assume that
\begin{align*}
    \bx \approx \mathbf{\Phi}_{\ell}\cdot \alpha_\ell(\bx) + \epsilon_{\ell}(\bx),
\end{align*}
where $\bx \in \reals^p$, is a dictionary matrix $\mathbf{\Phi}_{\ell} \in \reals^{p \times r}$ (could be complete/incomplete), representation vectors $\alpha_\ell(x) \in \reals^r$ ($r$ could be larger or smaller than $p$ inducing superimposed/non-superimposed features) and error terms $\epsilon_\ell(x) \in \reals^p$.

Now the task is to learn $\mathbf{\Phi}$ up to normal transformation, i.e. $\mathbf{\Phi}^{\top}\mathbf{\Phi}$ using triplet queries $(0, \beta, y) \in (\reals^{r})^3$.
\begin{align*}
    \beta^{\top} \mathbf{\Phi}^{\top}\mathbf{\Phi}\beta = y^\top \mathbf{\Phi}^{\top}\mathbf{\Phi}y
    \implies \beta^\top \cD \beta = y^{\top}\cD y
\end{align*}
such that the sparsity coefficient of $\beta, y$ is high, i.e $\cS(\beta) = \bigO{1}$.

\akash{Now, all the results from the work on Mahalanobis distance function should work}

\begin{lemma} Consider a PSD, symmetric matrix $M \in \reals^{n^2}$. Define the following set of orthogonal Cholesky decompositions of $M$
\begin{align*}
    W_{\sf{CD}} = \curlybracket{U : M = UU^\top,\, U^\top U = \text{diag}(d_1, d_2, \ldots, d_n)}
\end{align*}
    Then, for any two $U,U' \in W_{\sf{CD}}$, the column vectors of U are equivalent upto a rotation and sign change.
\end{lemma}

% \section{Geometry of Feature Learning with Feedback: Reduction to Pairs}

% First, we note that using triplet comparisons alone learner can't find a fixed target matrix $\pphi^*$ because for any $\lambda > 0$ and $(x,y,z) \in \cV^3$
% \begin{gather*}
%     (x-y)^{\top}\pphi^*(x-y) \ge (x-z)^{\top}\pphi^*(x-z)\\ \implies (x-y)^{\top}(\lambda \pphi^*)(x-y) \ge (x-z)^{\top}(\lambda \pphi^*)(x-z)
% \end{gather*}

% Thus, we can only hope to find $\pphi^*$ up to a positive linear scaling. Thus, we define a linear scaling relation $\sim_{R_l}$ on the space $\cM_{\mathsf{F}}$ as follows:
% \begin{align*}
%    \textnormal{for any } d_M, d_{M'} \in \cM_{\mathsf{F}}, d_M \sim_{R_l} d_{M'} \textnormal{ iff }  M = \gamma \cdot M' \textnormal{ for } \gamma > 0
% \end{align*}
% In the rest of the discussion, we study the oblivious teaching complexity of metric learning of mahalanobis distance metric wrt to the linear scaling relation $\sim_{R_l}$. 

% In the rest of the section, we follow the notations as mentioned below:
%\sanjoy{I didn't get far in sections 2-3 because of high-level issues (see my comments). Can you fix these things and then I will go through these two sections more carefully.}
%\sanjoy{What is the formal model -- what problem is being solved?}
%For a given matrix $M \in \reals^{d\times d}$ and a vector $x \in \reals^d$, define an inner product $\inner{xx^{\top}, M} := x^{\top} Mx$. We denote the set of symmetric matrices as $\sf symm(\reals^{p \times p})$.

%\sanjoy{For a Mahalanobis metric it is essential for the matrix to be positive semidefinite. This should be made clear in the definitions and will be part of the proofs.}

%\begin{lemma}
%    Any symmetric matrix of dimension $p \times p$ of rank-1 such that $M_{1,1}$ is positive can be written as
%    \begin{align}
%        M = xx^{\top}
%    \end{align}
%    for $x \in \reals^n$.
%\end{lemma}
%\begin{proof}
%    Wlog assume that $M_{1,1}$ is positive otherwise
%    Since $M$ is rank-1 every column vector $M_i$ is a linear scaling of any other column $M_j$ for $i \neq j$, i.e. there exists $\lambda_{j} \in \reals$ such that $M_1 = \lambda_{j}M_j$. But $M$ is symmetric and thus $R_1 = \lambda_{j}R_j$ for each $R_i = C_i$.
%    \begin{align*}
%    M = \begin{bmatrix}
%  M_{1,1} & \lambda_2 M_{1,1}& \ldots & \lambda_n M_{1,1} \\
%  \lambda_2 M_{1,1} & \lambda_2^2 M_{1,1}& \ldots & \lambda_2 \lambda_n M_{1,1}\\
%  \vdots & \vdots & \vdots & \vdots\\
%  \lambda_n M_{1,1}& \lambda_n\lambda_2M_{1,1}& \ldots & \lambda_n^2 M_{1,1}
%\end{bmatrix}
%\end{align*}
%Now, $M_{1,1}$ is positive thus we can write $M_{1,1} = (\sqrt{M_{1,1}})^2$. Thus, we  can write
%\begin{align}
%    M = \paren{\sqrt{M_{1,1}}, \lambda_2\sqrt{M_{1,1}},\ldots, \lambda_n\sqrt{M_{1,1}}}^{\top} \paren{\sqrt{M_{1,1}}, \lambda_2\sqrt{M_{1,1}},\ldots, \lambda_n\sqrt{M_{1,1}}} =: xx^{\top}
%\end{align}
%\end{proof}

%\sanjoy{How is this related to the problem we are trying to solve? Should give a reduction.}
%Thus, if we think of $M$ as a vector then $y_iy_i^{\top} - z_iz_i^{\top}$ lies in the orthogonal complement as vectors.
\iffalse
\subsection{A negative result on $c$-approximate teaching}
\akash{the assumption of PSD might not hold for any feature matrix, e.g. dictionary learning\\
Another concern is "assuming LRH and the features are presented as a matrix, does it make sense to consider this interactive setting where a powerful teacher is providing samples to a learner/agent/system."}


Let's consider a $c$-approximate version of the learning problem, for $c \geq 1$. Let $d_M$ be the Mahalanobis distance induced by matrix $M$. If the target is $\pphi^*$ then the $c$-approximate problem is to find a matrix $\pphi \succeq 0$ such that $d_M(x,x') \geq d_M(x,x'')$ whenever $d_{\pphi^*}(x,x') \geq c \cdot d_{\pphi^*}(x,x'')$. What is the sample complexity of this relaxed learning objective? Here's one possible notion: we say $M, M'$ are $c$-close if there exists $\lambda > 0$ such that
$$ \frac{1}{c} \leq \lambda \cdot \frac{u^T M' u}{u^T \pphi u} \leq c$$
for all $u \neq 0$.

\textcolor{red}{Can simply argue that the triplet queries are information-theoretically weak for retrieval with non-orthogonal pairs of rank-1 matrices. This can be used to get rid of the discussion above.}
\fi


\iffalse
\begin{figure*}[t] % 'htbp' suggests LaTeX to place the figure here, top, bottom, or on a special page
    \centering % Centers the entire figure
    % First Subfigure
    \begin{subfigure}[b]{0.3\textwidth} % Adjust the width as needed
        \centering % Centers the subfigure within its minipage
        \includegraphics[width=\textwidth]{ICML'25/Images/ground_mono.png} % Includes the first image
        \caption{Matrix $\pphi^*$ trained with RFM} % Sub-caption for the first image
        \label{fig:sub1} % Label for referencing the first subfigure
    \end{subfigure}
    \qquad
    %\hfill % Adds horizontal space between the subfigures
    % Second Subfigure
    \begin{subfigure}[b]{0.3\textwidth}
        \centering
        \includegraphics[width=\textwidth]{ICML'25/Images/feedback_mono.png}
        \caption{Approximation with feedback}
        \label{fig:sub2}
    \end{subfigure}
    \caption{ In this setting, we consider monomial regression where we sample random vectors of dimension 10 $z \sim \cN(0, .5 \mathbb{I}_{10})$} of dimension 10 and compute a target function $f^*(z) = z_0z_1\mathbf{1}(z_5 > 0)$. We train a kernel machine using RF\pphi (recursive feature machine), of the form $\hat{f}_M(z) = \sum_{y_i \in \cD_{\sf{train}}} a_i \cdot K_M(y_i,z)$, for 5 iterations on 5000 sampled training points and obtain a feature matrix in (a) above. Each iteration of update for $M$ is based on gradient computation of the form $\pphi = \sum_{z \in \cD_{\sf{train}}} (\frac{\partial \hat{f}_{M}}{\partial z })(\frac{\partial \hat{f}_{M}}{\partial z})^\top$. 
    On the right-hand side in (b), the agent provides (at most) 10 constructive feedbacks (rank of $\pphi^*$ is 4) to teach $\pphi^*$. We visualize the oblivious solution learned as $\hat{M}$. MSE on 4000 test samples: with the ground truth matrix-\textbf{0.0022}, and with the feature matrix with feedback-\textbf{0.00219}.
    
    % Main caption for the entire figure
    \label{fig:main} % Label for referencing the entire figure
\end{figure*}
\fi 
\iffalse
\section{Sparse Feature Learning with Constructive Feedback}

In this section, we show how an agent can provide constructive feedbacks to teach a given feature matrix. We allow arbitrary construction of activations from the ambient space of activations, i.e. $\reals^p$.

\paragraph{Reduction to pairs}
Now, we would argue that the general triplet comparisons provided by the agent can be simplified to pair comparisons of the form $(0, y,z) \in \cV^3$, denoted as $(y,z)$ for simplification, for strict equality constraint in \algoref{alg: main}.%\eqnref{eq: sol}. 


\begin{lemma}\label{lem: reduction}
    Consider a representation space $\reals^p$ and a feature family $\cM_{\mathsf{F}}$. Fix a target feature matrix $\pphi^*$ for oblivious learning with feedbacks. If there exists a feedback set $\cF(\reals^p, \cM_{\mathsf{F}}) = \curlybracket{(x,y,z) \in \reals^{3p}\,|\, (x-y)^{\top}\pphi^*(x-y) \ge (x-z)^{\top}\pphi^*(x-z)} $, such that any $\pphi' \in \textsf{VS}(\cF, \cM_{\mathsf{F}})$ is feature equivalent to ${\pphi^*}$ then there exists a feedback set $\cF' = \curlybracket{(y',z') \in \reals^{2p}\,|\, y'^\top{\pphi^*}y' = z'^\top{\pphi^*}z'}$ of pairs such that $\pphi' \in \textsf{VS}(\cF',\cM_{\mathsf{F}})$.
\end{lemma}

\begin{proof}
    WLOG assume that for all $(x,y,z) \in \cF(\reals^{p}, \cM_{\mathsf{F}})$ $x \neq z$. Now, consider the following: if for $(x,y,z) \in \cF(\cV, \cM_{\mathsf{F}})$ $(x-y)^\top{\pphi^*}(x-y) = (x-z)^\top{\pphi^*}(x-z)$ then $(x-y, x-z)$ satisfy the same equality wrt $\pphi^*$.

    Furthermore, if for $(x,y,z) \in \cF(\cV, \cM_{\mathsf{F}})$ $(x-y)^\top{\pphi^*}(x-y) > (x-z)^\top{\pphi^*}(x-z)$ then there exists $\gamma > 0$ such that
       \begin{align*}
           &(x-y)^{\top}\pphi^*(x-y) = \gamma + (x-z)^{\top}\pphi^*(x-z)
       \end{align*}
       Since $x \neq z$ we can rewrite $\gamma$ as a scalar multiple of $(x-z)^{\top}\pphi^*(x-z)$ and thus 
       \begin{align*}
           (x-y)^{\top}\pphi^*(x-y) = (1 + \lambda) (x-z)^{\top}\pphi^*(x-z)
       \end{align*}
       Hence, $(x - y, (1 + \lambda)(x-z))$ satisfy the equality constraint wrt $\pphi^*$.
       
   So, we have provided a procedure to translate any triplet comparison in $\cF(\reals^{p}, \cM_{\mathsf{F}})$ to a form as specified in $\cF' = \curlybracket{(y',z') \in\cV^2\,|\, y'^\top{\pphi^*}y' = z'^\top{\pphi^*}z'}$. As the equations in $1.$ and $2.$ are agnostic to a positive linear scaling thus we have: if $\pphi' \in \cF(\reals^{p}, \cM_{\mathsf{F}})$ such that $\pphi'$ is equivalent to ${\pphi^*}$ then $\pphi' \in \textsf{VS}(\cF', \cM_{\mathsf{F}})$.
\end{proof}

\lemref{lem: reduction} implies that if there are triplet comparisons to \algoref{alg: main} then there are also pair comparisons that satisfy the following equation:
\begin{subequations}
\begin{gather}
    \pphi' = \lambda\cdot d_{\pphi^*}, \text{ for some }\lambda > 0\\
    \pphi'  \in \curlybracket{\pphi \in \cM_{\mathsf{F}}\,|\, \forall (y,z) \in \cF(\reals^{2p},\cM_{\mathsf{F}},\pphi^*),\, y^{\top}\pphi y = z^{\top}\pphi z} \label{eq: redsol}
\end{gather}
\end{subequations}
where $\cF(\reals^p,\cM_{\mathsf{F}},\pphi^*)$ is the feedback set that satisfy equality constraints with $\pphi^*$.

In the rest of this section, we consider the reformulation of oblivious learning of a feature matrix with pair comparisons.

\paragraph{Geometry of Feature Learning} For a given pair $(y,z)$ and a matrix $\pphi$, if we have equality constraint $y^{\top}\pphi y = z^{\top}\pphi z$ then with some rewriting we observe that
\begin{align*}
    y^{\top}\pphi y = z^{\top}\pphi z &\iff \inner{\pphi,\, yy^{\top} - zz^{\top}} = 0\\ &\iff \pphi \idot (yy^{\top} - zz^{\top}) = 0
\end{align*}
Thus, $(yy^{\top} - zz^{\top})$ is orthogonal to $\pphi $. So, we can pose the learning problem of \eqnref{eq: redsol} for a set of pairs of feedbacks $\cF(\cV,\cM_{\mathsf{F}},\pphi^*) := \curlybracket{(y_i,z_i)}_{i=1}^k $ corresponding to a target feature matrix $\pphi^*$ as follows:
\begin{align}
  \forall (y,z) \in \cF(\reals^p,\cM_{\mathsf{F}},\pphi^*), \quad \pphi \idot (yy^{\top} - zz^{\top})  = 0 \label{eq: orthosat}
\end{align}
Essentially, the learner obliviously picks a matrix $\pphi'$ that satisfy \eqnref{eq: orthosat}. Note that this has a nice geometric picture to this as shown in \figref{fig: geom} where $\mathcal{O}_{\pphi^*} := \curlybracket{S \in \reals^{p \times p} \,|\, \pphi^* \idot S = 0}$.
\fi
%-------------------------------------\\

%\akash{remove this if no fig}\figref{fig: geom} illustrates this geometric interpretation, where $yy^{\top} - zz^{\top} \}_{(y,z) \in \mathcal{F}}$ sits in the orthogonal complement $\mathcal{O}_{\pphi^*}$.
\section{Sparse Feature Learning with Constructive
Feedback}\label{sec: construct}
Here, we study the feedback complexity in the setting where agent is allowed to pick/construct any activation from $\reals^p$. 


\paragraph{Reduction to Pairwise Comparisons}
The general triplet feedbacks with potentially inequality constraints in \algoref{alg: main}
 can be simplified to pairwise comparisons with equality constraints with a simple manipulation as follows. %Specifically, triplet comparisons of the form $(0, y, z) \in \mathcal{V}^3$ can be denoted as $(y, z)$ for simplicity, particularly when enforcing strict equality constraints in \algoref{alg: main}.%\eqnref{eq:sol}.

\begin{lemma}\label{lem: reduction}
Let $\pphi^* \in \mathcal{M}_{\mathsf{F}}$ be a target feature matrix in representation space $\reals^p$ used for oblivious learning. Given a feedback set \vspace{-1.5mm}
\[
\mathcal{F} = \left\{ (x, y, z) \in \reals^{3p} \,\big|\, (x - y)^{\top} \pphi^* (x - y) \geq (x - z)^{\top} \pphi^* (x - z) \right\},
\]
such that any $\pphi' \in \textsf{VS}(\mathcal{F}, \mathcal{M}_{\mathsf{F}})$ is feature equivalent to $\pphi^*$, there exists a pairwise feedback set \vspace{-1.5mm}
\[
\mathcal{F}' = \left\{ (y', z') \in \reals^{2p} \,\big|\, y'^{\top} \pphi^* y' = z'^{\top} \pphi^* z' \right\}
\]
such that $\pphi' \in \textsf{VS}(\mathcal{F}', \mathcal{M}_{\mathsf{F}})$.\vspace{-3mm}
\end{lemma}
\begin{proof}
WLOG, assume $x \neq z$ for all $(x, y, z) \in \mathcal{F}$. For any triplet $(x, y, z) \in \mathcal{F}$: \textbf{Case} (i): If $(x - y)^{\top} \pphi^* (x - y) = (x - z)^{\top} \pphi^* (x - z)$, then $(x - y, x - z)$ satisfies the equality. \textbf{Case} (ii): If $(x - y)^{\top} \pphi^* (x - y) > (x - z)^{\top} \pphi^* (x - z)$, then for some $\lambda > 0$:
\[
(x - y)^{\top} \pphi^* (x - y) = (1 + \lambda)(x - z)^{\top} \pphi^* (x - z)
\]
implying $(x - y, \sqrt{1 + \lambda}(x - z))$ satisfies the equality.

Thus, each triplet in $\mathcal{F}$ maps to a pair in $\mathcal{F}'$, preserving feature equivalence under positive scaling.
\end{proof}
This implies that if triplet comparisons are used in \algoref{alg: main}, equivalent pairwise comparisons exist satisfying:\vspace{-5mm}
\begin{subequations}\label{eq: redsol}
\begin{align}
    \pphi' &= \lambda \cdot {\pphi^*}, \quad \lambda > 0, \label{eq:reduction_scaling} \\
    \pphi' &\in \left\{ \pphi \in \mathcal{M}_{\mathsf{F}} \,\big|\, \forall (y, z) \in \mathcal{F}', \, y^{\top} \pphi y = z^{\top} \pphi z \right\}. \label{eq:reduction_solution}
\end{align}
\end{subequations}
Now, we show a reformulation of the oblivious learning problem for a feature matrix using pairwise comparisons that provide a unique geometric interpretation.
\allowdisplaybreaks
%\subsection{Geometric Perspective on Feature Learning}
Consider a pair $(y, z)$ and a matrix $\pphi$. An equality constraint implies
\begin{align*}
    y^{\top} \pphi y = z^{\top} \pphi z &\iff \langle \pphi,\, yy^{\top} - zz^{\top} \rangle = 0 %\\
    %&\iff \pphi \cdot (yy^{\top} - zz^{\top}) = 0,
\end{align*}
where $\langle \cdot, \cdot \rangle$ denotes the Frobenius inner product. This equivalence indicates that the matrix $(yy^{\top} - zz^{\top})$ is orthogonal to $\pphi$ in the Frobenius inner product space. Now, given a set of pairwise feedbacks 
\[
\mathcal{F}(\mathbb{R}^{p}, \mathcal{M}_{\mathsf{F}}, \pphi^*) = \{ (y_i, z_i) \}_{i=1}^k
\]
corresponding to the target feature matrix $\pphi^*$, the learning problem defined by \eqnref{eq:reduction_solution} can be formulated as:
\begin{align}\label{eq: orthosat}
    \forall (y, z) \in \mathcal{F}(\mathbb{R}^p, \mathcal{M}_{\mathsf{F}}, \pphi^*), \quad \langle \pphi,\, yy^{\top} - zz^{\top} \rangle = 0. %\\
\end{align}
Essentially, the learner aims to select a matrix $\pphi$ that satisfies the condition in \eqnref{eq: orthosat}.
Geometrically, the condition in \eqnref{eq: orthosat} implies that any solution $\pphi$ should annihilate the subspace of the orthogonal complement that is spanned by the matrices $\{ yy^{\top} - zz^{\top} \}_{(y,z) \in \mathcal{F}}$. Formally, this complement is defined as:\vspace{-1mm}
\[
\mathcal{O}_{\pphi^*} := \left\{ S \in \symm \,\bigg|\, \langle\pphi^*, S\rangle = 0 \right\}.\vspace{-3mm}
\]
% \begin{figure}[!]
% \begin{center}
% \tdplotsetmaincoords{60}{120}
% \begin{tikzpicture}[tdplot_main_coords]
% % Draw the plane
% \fill[fill=blue!10] (0,0,0) -- (3,-1,0) -- (4,1,0) -- (1,2,0) -- cycle;
% % Define the center point on the plane
% \coordinate (center) at (2,0.5,0);
% % Draw the arrows
% \draw[->, line width=.5mm] (center) -- ++(0,0,2.5) node[above, anchor=south] {$\pphi^*$};
% %\draw[->, thick] (0,0) -- (3,-1) node[anchor=north west] {$\mathbb{R}^{p \times p}$};
% % \draw[->, thick] (0,0) -- (4,1);
% % Draw the dashed arrows from the center
% \draw[dashed, ->] (center) -- ++(-1,-1,0) node[above, anchor=south] {\tiny{$yy^{\top} - zz^{\top}$}};
% \draw[dashed, ->] (center) -- ++(.7,-.8,0) node[below, anchor=east] {\tiny{$y'y'^{\top} - z'z'^{\top}$}};
% % Add the label on the plane
% \node at (4.2,0.5,0) {$\mathcal{O}_{\pphi^*}$};
% \node at (3,3,3) {$\mathbb{R}^{p \times p}$};
% \end{tikzpicture}
% \end{center}
% \caption{Geometry of solving \eqnref{eq: redsol}}
%     \label{fig: geom}
% \end{figure}
\subsection{Constructive feedbacks: Worst-case lower bound}
To learn a symmetric PSD matrix, learner needs at most $p(p+1)/2$ constraints for linear programming corresponding to the number of degrees of freedom. So, the first question is are there pathological cases of feature matrices in $\cM_{\sf{F}}$ which would require at least $p(p+1)/2$ many triplet feedbacks in \algoref{alg: main}. This indeed is the case, if a target matrix $\pphi^* \in \symmp$ is full rank.

%In this subsection, we study the \tt{the minimal size of $\cF(\cV,\cM_{\mathsf{F}})$ that exactly fixes a target matrix $\pphi^*$ (upto linear scaling relation $\sim_{R_l}$) in \eqnref{eq: redsol}}.

In the following proposition proven in \appref{app: worstcase}, we show a strong lower bound on the worst-case $\pphi^*$ that turns out to be of order $\Omega(p^2)$. %Note that the $\dim(\symm) = \frac{p(p+1)}{2}$.

%\sanjoy{These results should incorporate the PSD constraint. They might need to be reworked slightly for that.}

\begin{proposition}\label{prop: worstcase} In the constructive setting, the worst-case feedback complexity of the class $\cM_{\sf{F}}$ with general activations
%oblivious teaching complexity of a teaching set $\cF(\cV, \cM_{\sf{F}})$ up to linear scaling relation $\sim_{R_l}$ for \eqnref{eq: redsol} 
is at the least $\paren{\frac{p(p+1)}{2} - 1}$.
\end{proposition}
\begin{proof}[Proof Outline]  As discussed in \eqnref{eq: redsol} and \eqnref{eq: orthosat}, for a full-rank feature matrix $\pphi^* \in \cM_{\sf{F}}$, the span of any feedback set $\cF$, i.e., $\sn{\curlybracket{xx^{\top} - yy^{\top}}_{(x,y) \in \cF}}$, must lie within the orthogonal complement $\mathcal{O}_{\pphi^*}$ of $\pphi^*$ in the space of symmetric matrices $\symm$. Conversely, if $\pphi^*$ has full rank, then $\mathcal{O}_{\pphi^*}$ is contained within this span. This necessary condition requires the feedback set to have a size of at least $\frac{p(p+1)}{2} - 1$, given that $\dim(\symm) = \frac{p(p+1)}{2}$.
\end{proof}
%Later in the section we will show that worst-case upper bound on oblivious teaching complexity is $\frac{p(p+1)}{2} - 1$, which is expected as there are $\frac{p(p+1)}{2}$ degree of freedom for any symmetric, postive semi-definite matrix. 
Since the worst-case bounds are pessimistic for oblivious learning of \eqnref{eq: redsol} a general question is how feedback complexity varies over the feature model $\maha$. In the following subsection, we study the feedback complexity for feature model based on the rank of the underlying matrix, showing that the bounds can be drastically reduced. 

\subsection{Feature learning of low-rank matrices}

As stated in \propref{prop: worstcase}, the learner requires at least $\frac{p(p+1)}{2} - 1$ feedback pairs to annihilate the orthogonal complement $\mathcal{O}_{\pphi^*}$. However, this requirement decreases with a lower rank of $\pphi^*$. We illustrate this in \figref{fig: monoconst} for a feature matrix $\pphi \in \reals^{10 \times 10}$ of rank 4 trained via Recursive Feature Machines~\citep{rfm}.

Consider an activation $\alpha \in \reals^p$ in the nullspace of $\pphi^*$. Since $\pphi^*\alpha = 0$, it follows that $\alpha^\top \pphi^* \alpha = 0$. Moreover, for another activation $\beta \notin \sn{\alpha}$ in the nullspace, any linear combination $a\alpha + b\beta$ satisfies
$$(a\alpha + b\beta)^\top \pphi^* (a\alpha + b\beta) = 0.\vspace{-4mm}$$

This suggests a strategy for designing effective feedback based on the kernel $\kernel{\pphi^*}$ and the null space $\nul{\pphi^*}$ of $\pphi^*$ (see \appref{app: notations} for table of notations). This intuition is formalized by the eigendecomposition of the feature matrix:\vspace{-3mm}
\begin{align}
    \pphi^* = \sum_{i=1}^r \lambda_i u_i u_i^\top, \label{eq: eigen} \vspace{-4mm}
\end{align}
where $\{\lambda_i\}$ are the eigenvalues and $\{u_i\}$ are the orthonormal eigenvectors.
Since $\pphi^* \succeq 0$ this decomposition is \tt{unique} with non-negative eigenvalues.

To teach $\pphi^*$, the agent can employ a dual approach: teaching the kernel associated with the eigenvectors in this decomposition and the null space separately. Specifically, the agent can provide feedbacks corresponding to the eigenvectors of $\pphi^*$'s kernel and extend the basis $\{u_i\}$ for the null space. We first present the following useful result (see proof in \appref{app: constub}).

\begin{lemma}\label{lem: basis}
    Let $\{v_i\}_{i=1}^r \subset \reals^p$ be a set of orthogonal vectors. Then, the set of rank-1 matrices
    \[
    \mathcal{B} := \left\{v_i v_i^{\top},\ (v_i + v_j)(v_i + v_j)^{\top}\ \bigg| \ 1 \leq i < j \leq r \right\}
    \]
    is linearly independent in the space symmetric matrices $\symm$.
\end{lemma}

Using this construction, the agent can provide feedbacks of the form $(u_i, \sqrt{c_i} y)$ for some $y \in \reals^p$ with $\pphi^* y \neq 0$ and $v_i^\top \pphi^* v_i = c_i y^\top \pphi^* y$ to teach the kernel of $\pphi^*$. For an orthogonal extension $\{u_i\}_{i=r+1}^p$ where $\pphi^* u_i = 0$ for all $i = r+1,\dots,p$, feedbacks of the form $(u_i, 0)$ suffice to teach the null space of $\pphi^*$.

This is the key idea underlying our study on feedback complexity in the general constructive setting that is stated below with the full proof deferred to \appref{app: constub} and \ref{app: constlb}.

\begin{theorem}[General Activations]\label{thm: constructgeneral}
    Let $\Phi^* \in \maha$ be a target feature matrix with $\rank{\pphi^*} = r$. Then, in the setting of constructive feedbacks with general activations, the feedback complexity has a tight bound of $\Theta\left(\frac{r(r+1)}{2} + (p - r) - 1\right)$ for \eqnref{eq: redsol}.
\end{theorem}

\begin{proof}[Proof Outline]
    As discussed above we decompose the feature matrix $\pphi^*$ into its eigenspace and null space, leveraging the linear independence of the constructed feedbacks to ensure that the span covers the necessary orthogonal complements. 
    The upper bound is established with a simple observation: $r(r+1)/2 - 1$ many pairs composed of $\cB$ are sufficient to teach $\pphi^*$ if the null space of $\pphi^*$ is known, whereas the agent only needs to provide $(p-r)$ many feedbacks corresponding to a basis extension to cover the null space, and hence the stated upper bound is achieved.
    
    The lower bound requires showing that a valid feedback set possesses two spanning properties of $\langle{xx^\top - yy^\top\rangle}$ for all $(x,y) \in \cF$: (1) it must include any $\pphi \in \mathcal{O}_{\pphi^*}$ whose column vectors are within the span of eigenvectors of $\pphi^*$, and (2) it must include any $vv^\top$ for some subset $U$ that spans the null space of $\pphi^*$ and $v \in U$.
    %if any $\pphi \in \mathcal{O}_{\pphi^*}$ and its column vectors are within the span of eigenvectors of $\pphi^*$, then $\pphi$ should be within the span $\langle{xx^\top - yy^\top\rangle}$ for $(x,y) \in \cF$, and (2) there exists a subset $U$ that spans the null space of $\pphi^*$ such that any rank-1 matrix $vv^\top$ is within $\langle{xx^\top - yy^\top\rangle}$ for $(x,y) \in \cF$ for any $v \in U$ 
    %This guarantees that the minimum number of feedback pairs meets the stated bound. 
    %\akash{complete the lower bound proof which is much more intricate}
\end{proof}

%In general, the input space $\cV \subset \reals^d$ exists in a small dimensional manifold. So, if there is a distance function which linearly transforms the space to capture the distance well, e.g. in the form of a matrix in Mahalanobis metric, we expect the rank of the matrix to be small as well. So, it is interesting to understand if the oblivious teaching can capture this small dimension with an optimistic teaching complexity. 
\iffalse
As stated in \propref{prop: worstcase}, we note that the learner needs at least $p(p+1)/2 - 1$ pairs of feedbacks to annihilate the orthogonal complement $\mathcal{O}_{\pphi^*}$. But the feedback requirement goes down if the rank is smaller. 

Consider an activation $\alpha \in \reals^p$ in the nullspace of $\pphi^*$. Note that $\pphi^*\alpha = 0$ implies $\alpha^\top \pphi^* \alpha = 0$. Furthermore, if $\beta \notin \sn{\alpha}$ is another activation in the null space then for any choice of scalars $a$ and $b$, 
$$(a\alpha + b\beta)^\top\pphi^*(a\alpha + b\beta) = 0.$$

This suggests a strategy in designing helpful feedback according to the kernel $\kernel{\pphi^*}$ and the null space $\nul{\pphi^*}$ of the matrix $\pphi^*$. 
%If we consider the orthogonal complement $\mathcal{O}_{\pphi^*}$, we might expect that if the rank of $\pphi^*$ is smaller than $p$, then the PSD cone projected onto it might be contained in a smaller subspace than $p(p+1)/2 - 1$. We show 
%\akash{provide the eigendecomposition method and also talk about the experiment!}
This intuition can be precisely captured by the eigendecomposition of a feature matrix:
\begin{align}
    \pphi^* = \sum_{i = 1}^r \lambda_i u_iu_i^\top, \label{eq: eigen}
\end{align}
where $\curlybracket{\lambda_i}$ are the set of eigenvalues and $\curlybracket{u_i}$ are the set of orthonormal eigenvectors. This decomposition is \tt{unique} for a symmetric matrix; furthermore all eigenvalues are non-negative as $\pphi^* \succeq 0$.

In order to teach $\pphi^*$, the agent could devise a dual technique of teaching the kernel corresponding to the eigenvectors in this decomposition, and the null space \tt{separately}.


We show that the agent could precisely provide feedbacks corresponding to eigenvectors for the kernel of $\pphi^*$, and an extension to the basis $\curlybracket{u_i}$ for the null space. First we note this useful result.
\begin{lemma}
    Assume $\curlybracket{v_i}_{i=1}^r \subset \reals^p$ be a set of orthogonal vectors, then the following set of rank-1 matrices
    \begin{gather*}
        \mathcal{B} := \{v_1v_1^{\top},v_2v_2^{\top}, (v_1 + v_2)(v_1 + v_2)^{\top},\ldots,\\v_rv_r^{\top}, (v_1 + v_r)(v_1 + v_r)^{\top},\ldots, (v_{r-1} + v_r)(v_{r-1} + v_r)^{\top}\}
    \end{gather*}
    are linearly independent in the space of symmetric matrices $\symm$.
\end{lemma}
%\akash{why can't we design wrt to any basis; maybe proof is not clear!!}
%\begin{lemma}
%    Let $M$ be a positive definite matrix. Then, the lower bound on the teaching complexity is $\frac{p(p+1)}{2} - 1$.
%\end{lemma}
Using this construction, agent can devise feedbacks of the form $(u_i, \sqrt{c_i} y)$ for some $y \in \reals^p$ such that $\pphi^*y \neq 0$ and $v_i^\top \pphi^*v_i = c_iy^\top \pphi^*y$ to teach the kernel of $\pphi^*$; whereas for any orthogonal extension $\curlybracket{u_i}_{i = r+1}^p$ such that for all $i = r+1, r+2,\ldots,p$ $\pphi^*u_i = 0$ feedbacks of the form $(u_i,0)$ suffice to teach the null space of the feature matrix $\pphi^*$.

We provide the  concrete statement on the feedback complexity in the general constructive setting below.

\begin{theorem}[general activations]\label{thm: constructgeneral}
    Assume that $\Phi^* \in \maha$ be a target feature matrix with $\rank{\pphi^*} = r$. %Suppose a superagent can design pairs of activations from the ambient space of representations $\reals^p$. 
    Then, in the setting of constructive feedbacks with general activations, the feedback complexity has a tight bound of $\Theta\paren{\frac{r(r+1)}{2} + (p - r) - 1}$ for \eqnref{eq: redsol}.
\end{theorem}
\begin{proof}[Proof Outline] The key aspect of the analysis is to decompose the 
    
\end{proof}
\fi
\paragraph{Learning with sparse activations} In the discussion above, we demonstrated a strategy for reducing the feedback complexity when general activations are allowed. Now, we aim to understand how this complexity changes when activations are $s$-sparse (see \defref{def: sparse}) for some $s < p$. Notably, there exists a straightforward construction of rank-1 matrices using a sparse set of activations.

Consider this sparse set of activations ${B}$ consisting of $\frac{p(p+1)}{2}$ items in $\reals^p$:
\begin{align}
  {B} = \{e_i \mid 1 \leq i \leq p\} \cup \{e_i + e_j \mid 1 \leq i < j \leq p\},  \label{eq: sparsebasis}
\end{align}
where $e_i$ is the $i$th standard basis vector. Using a similar argument to \lemref{lem: basis}, it is straightforward to show that the set of rank-1 matrices
$$\cB_{\sf{sparse}} := \curlybracket{uu^\top \mid u \in {B}}$$
is linearly independent in the space of symmetric matrices $\symm$ and forms a basis. Moreover, every activation in $B_{\sf{ext}}$ is at most 2-sparse (see \defref{def: sparse}). With this, we state the main result on learning with sparse constructive feedback here.

\begin{theorem}[Sparse Activations]\label{thm: constructsparse}
    Let $\pphi^* \in \maha$ be the target feature matrix. 
    If an agent can construct pairs of activations from a representation space $\reals^p$, then the feedback complexity of the feature model $\maha$ with 2-sparse activations is upper bounded by $\frac{p(p+1)}{2}$.
    %superagent can teach $\pphi^*$ up to a sign and rotational transformation.
\end{theorem}
%\looseness-1\vfill\vspace*{-10mm}
\allowdisplaybreaks
\tt{Remark}: While the lower bound from \thmref{thm: constructgeneral} applies here, sparse settings may require even more feedbacks. Consider a rank-1 matrix $\pphi^* = vv^\top$ with $\text{sparsity}(v) = p$. By the Pigeonhole principle, representing this using $s$-sparse activations requires at least $(p/s)^2$ rank-1 matrices. Thus, for constant sparsity $s = O(1)$, we need $\Omega(p^2)$ feedbacks—implying sparse representation of dense features might not exploit the low-rank structure to minimize feedbacks. %Furthmore, the upper bound
%$r = 1$,  \akash{check if something better can be said}
\iffalse
\begin{theorem}\label{thm: obv}
    Consider an input space $\cV \subset \reals^d$, and a metric model $\maha$.
    Consider a symmetric, positive semi-definite matrix $\pphi^*$ of rank $r \le d$. Assume that $d_{\pphi^*}$ be the target metric for oblivious teaching in \eqnref{eq: sol}. Then, the teaching complexity $TC_O(\cV, \maha, d_{\pphi^*})$ has the tight bound of $\paren{\frac{r(r+1)}{2} + (d - r) - 1}$.
\end{theorem}
\begin{theorem}\label{thm: constructive}
    Assume that $\Phi^* \in \reals^{d \times p}$ be a target feature matrix. 
    Consider a representation space $\cV \subseteq \reals^p$. Suppose a superagent can design pairs of activations from a representation space $\cV \subseteq \reals^p$. Then, we have the following guarantee on the number of feedbacks required:
    \begin{enumerate}
        \item \sf{w}/ sparsity: With a tight bound of $\paren{\frac{r(r+1)}{2} + (p - r) - 1}$, superagent can teach $\Phi^*$ up to rotational transformation.
        \item sparsity: With an upper bound of $\frac{p(p+1)}{2}$, a supergent can design pairs of sparse activations to teach $\Phi^*$ up to rotational transformation.
    \end{enumerate}
\end{theorem}

Now, it is clear that a teaching set of size $\frac{r(r+1)}{2} - 1 + (d -r)$ is sufficient to teach $\pphi^*$. In the following, we show the teaching set size of $\frac{r(r+1)}{2} - 1 + (d -r)$ is also necessary for oblivious teaching.

\begin{lemma}\label{lemma: lowerbound}
    Consider an input space $\cV \subset \reals^d$, and a metric model $\maha$.
    Consider a symmetric, positive semi-definite matrix $\pphi^*$ of rank $r \le d$. Assume that $d_{\pphi^*}$ be the target metric for oblivious teaching in \eqnref{eq: sol}. Then, the teaching complexity $TC_O(\cV, \maha, d_{\pphi^*})$ has a lower bound of $\paren{\frac{r(r+1)}{2} + (d - r) - 1}$.
\end{lemma}
\fi

\section{Sparse Feature Learning with Sampled Feedback}\label{sec: sample}

\begin{algorithm}[t]
\caption{Feature learning with sampled representations}
\label{alg: randmaha}
\textbf{Given}: Representation space $\cV  \subset \reals^p$, Distribution over representations $\cD_\cV$, Feature family $\cM_{\mathsf{F}}$. \vspace{2mm}\\
\textit{In batch setting:}%\vspace{1mm}
\begin{enumerate}
    \item Agent receives sampled representations $\cV_n \sim \cD_{\cV}$.\vspace{-2mm}
    \item Agent picks pairs $\cF(\cV_n,\pphi^*) =$ \vspace{-2mm}$$\curlybracket{(x,\sqrt{\lambda_{x}}y)\,|\, (x,y) \in \cV_n^2,\, x^{\top}\pphi^*x = \lambda_{x}\cdot y^{\top}\pphi^*y}\vspace{-2mm}$$ 
    \item Learner receives $\cF$; and obliviously picks a feature matrix $\pphi \in \cM_{\mathsf{F}}$ that satisfy the set of constraints in $\cF(\cV_n,\pphi^*)$\vspace{-2mm}
    \item Learner outputs $\pphi$.\vspace{-2mm}
\end{enumerate}
\end{algorithm}

%In the previous section, we assumed that the teaching agent can arbitrarily construct feature activations $\alpha \in \reals^p$ to provide comparisons. 
In general, the assumption of constructive feedback may not hold in practice, as ground truth samples from nature or induced representations from a model are typically independently sampled from the representation space. The literature on Mahalanobis distance learning and dictionary learning has explored distributional assumptions on the sample/activation space~\citep{Mason2017LearningLM,Gribonval2014SparseAS}.

In this section, we consider a more realistic scenario where the agent observes a set of representations/activations $\cV_n := \{\alpha_1, \alpha_2, \ldots, \alpha_n\} \sim \cD_{\cV}$, with $\cD_{\cV}$ being an unknown measure over the continuous space $\cV \subseteq \reals^p$. With these observations, the agent designs pairs of activations to teach a target feature matrix $\pphi^* \in \symmp$.

As demonstrated in \lemref{lem: reduction}, we can reduce inequality constraints with triplet comparisons to equality constraints with pairs in the constructive setting. However, when the agent is restricted to selecting activations from the sampled set $\cV_n$ rather than arbitrarily from $\cV$, this reduction no longer holds. This is evident from the following observation: let $\alpha, \beta \sim \text{iid} \ \cD_{\cV}$ and let $\pphi^* \neq 0$ be a non-degenerate feature matrix. Then,\vspace{-1.5mm}
\[
\alpha^{\top}\pphi^*\alpha = \beta^{\top}\pphi^*\beta \implies \sum_{i,j} (\alpha_i \alpha_j - \beta_i \beta_j) \pphi^*_{ij} = 0.\vspace{-1.5mm}
\]
This equation represents a non-zero polynomial. According to Sard's Theorem, the zero set of a non-zero polynomial has Lebesgue measure zero. Therefore,\vspace{-1.5mm}
\[
\cP_{(\alpha,\beta)}\left( \left\{ \alpha^{\top}\pphi^*\alpha = \beta^{\top}\pphi^*\beta \right\} \right) = 0.\vspace{-1mm}
\]
Given this, the agent cannot reliably construct pairs that satisfy the required equality constraints from independently sampled activations. Since a general triplet feedback only provides 3 bits of information, exact recovery up to feature equivalence is impossible. To address these limitations, we consider rescaling the sampled activations to enable the agent to design effective pairs for the target feature matrix $\pphi^* \in \cM_{\sf{F}}$.

%\akash{check if it is applicable}
%\begin{lemma}[Sard's Theorem]
%Let \( f: X \to Y \) be a smooth map of manifolds, and let \( C \) be the set of critical points of \( f \) in \( X \). Then \( f(C) \) has measure zero in \( Y \).
%\end{lemma}

\begin{figure*}[t] % 'htbp' suggests LaTeX to place the figure here, top, bottom, or on a special page
    \centering % Centers the entire figure
    % First Subfigure
    
        \includegraphics[width=1\textwidth]{sparsemono.png} % Includes the first image
    
    %\caption{ In this setting, we consider monomial regression where we sample random variables of dimension 10: $z \sim \cN(0, .5 \mathbb{I}_{10})$} and compute a target function $f^*(z) = z_0z_1\mathbf{1}(z_5 > 0)$. We train a kernel machine using RFM (recursive feature machine), of the form $\hat{f}_{\pphi}(z) = \sum_{y_i \in \cD_{\sf{train}}} a_i \cdot K_{\pphi}(y_i,z)$, for 5 iterations on 4000 sampled training points and obtain a feature matrix $\pphi^*$ in rightmost plot (Ground truth). Each iteration of RFM update for $\pphi^*$ is based on gradient computation of the form $\pphi = \sum_{z \in \cD_{\sf{train}}} (\frac{\partial \hat{f}_{\pphi}}{\partial z })(\frac{\partial \hat{f}_{\pphi}}{\partial z})^\top$. The rank of $\pphi^*$ is 4. In the next set of plots, we employ different feedback methods where a superagent/teacher provides feedbacks based on $\pphi^*$: (constructive) eigendecompostion and sparse construction; (sampling) random sampling with Gaussian distribution and sparse sampling where the probability of sparsity is .9 (sparsity: very high). Note that the three middle plots achieve the same MSE (mean squared error) as the ground truth feature matrix $\pphi^*$. But since the sparsity is pretty high, the approximated feature matrix with the same number of feedbacks: 55 is not sufficient and both the feature matrix and computed MSE are poor (see Theorem for discussion).}
    \caption{
    \textbf{Sparse sampling}: We consider the same setup as \figref{fig: monoconst} for the target function $f^*(z) = z_0 z_1 z_3 \mathbf{1}(z_5 > 0)$. In these plots, we employ sparse sampling feedback methods where an agent provides feedback based on \( \pphi^* \) with different sparsity probability (mu: probability of 0 being sampled). Thus, as mu decreases the theorized complexity of $p(p+1)/2 = 55$ obtains a close approximation of $\pphi^*$. But for $mu = .97$, the agent needs to sample more number of activations to approximate properly, i.e. from 55, 110, $\ldots$, and 1100 approximation gradually improves as shown in \thmref{thm: samplingsparse}.
    %Notably, the three central plots achieve the same Mean Squared Error (MSE) as the ground truth feature matrix \( \pphi^* \). However, due to the high sparsity, the approximated feature matrices with the same number of feedbacks (55) are insufficient, resulting in poor feature matrices and computed MSE (see Theorem for discussion).
}
    
    %On the right-hand side in (b), the teacher provides (at most) 10 constructive feedbacks (rank of $\pphi^*$ is 4) to teach $\pphi^*$. We visualize the oblivious solution learned as $\hat{M}$. MSE on 4000 test samples: with the ground truth matrix-\textbf{0.0022}, and with the feature matrix with feedback-\textbf{0.00219}.
    
    % Main caption for the entire figure
    \label{fig: monosparse} % Label for referencing the entire figure
\end{figure*}


\iffalse
\paragraph{Rescaled pairs} For a given matrix $\pphi \neq 0$ a sampled input $x \sim \cD_{\cV}$ is (almost) never orthogonal, i.e. (a.s.) $\pphi x \neq 0$. This can be used to rescale an input to construct pairs to satisfy equality constraints. In other words, there exists $\gamma, \lambda >0$ such that (assume wlog $x^{\top}\pphi x > y^{\top}\pphi y$)
\begin{align*}
    x^{\top}\pphi x = \gamma + y^{\top}\pphi y = \lambda\cdot y^{\top}\pphi y + y^{\top}\pphi y \\= (\sqrt{1 + \lambda})y^{\top}\pphi(\sqrt{1 + \lambda})y 
\end{align*}
Thus, $(x, (\sqrt{1 + \lambda})y)$ satisfy the equality constraints. With this understanding we provide a reformulation of \algoref{alg: main} into \algoref{alg: randmaha}. %\akash{In the rest of the section, we study the expected oblivious teaching complexity for  \algoref{alg: randmaha} under Lebesgue distribution $\cD_{\cV}$ over the input space $\cV$. We show tight bounds on the teaching complexity highlighting a gap between the constructive and sampled teaching scenarios.} 

%\paragraph{A noisy model of approximations} An adversarial teacher provides $(x,y,z; \delta)$ where $x,y,z \in \cV$ and $\delta \sim \cN(0, \sigma^2)$ with the equality label, i.e.
%\begin{align*}
%    (x-y)^\top \pphi^*(x-y) - (x-z)^\top \pphi^*(x-z) = \delta_{xyz} 
%\end{align*}
%In the constructive setting, we discussed concrete strategies for devising optimal teaching set, the number of triplet or pairs for oblivious teaching. In the sampled case, we want to study the size of samples $\cV_n$, both necessary and sufficient, so that teacher can devise rescaled pairs. Thus, the oblivious teaching complexity is defined as the size $|\cV_n|$ such that the learner finds $\pphi^*$ up to linear scaling relation $\sim_{R_l}$ in \algoref{alg: randmaha}, $\expctover{\cV_n}{}$
%\akash{Define teaching set }
%In the sampling case, we define the teaching complexity in terms of the size of the sampled set $\cV_n$ so that the teacher can devise a teaching set $\cF(\cV, \maha,\pphi^*)$ for a target matrix $\pphi^*$.
%\begin{definition}\label{defn: lebsample}
%    Consider an input space $\cV \subset \reals^d$. Let $\cD_{\cV}$ be a Lebesgue distribution over $\cV$ from which iid samples $\cV_n$ are selected.    Fix a constant $\epsilon > 0$ and a target matrix $\pphi^*$. We define the oblivious teaching complexity for $\epsilon$-accuracy for random samples is $n$ if the teacher provides a teaching set $\cF(\cV_n, \maha,\pphi^*)$ for \algoref{alg: randmaha} using the samples $\cV_n$ such that
    %\begin{align*}
     %   \cP_{\cV_n}(d_{M'} \in \textsf{VS}(\cF(\cV_n, \maha,\pphi^*), \maha) \textnormal{ such that } d_{M'} \sim_{R_l} d_{\pphi^*}) \ge \epsilon
    %\end{align*}
%\end{definition}
%First, we start with an upper bound on the worst-case (across the space $\maha$) oblivious teaching complexity that achieves $1$-accuracy.

%\begin{proposition}\label{prop: sampling}
%    Consider an input space $\cV \subset \reals^d$. Let $\cD_{\cV}$ be a Lebesgue distribution over $\cV$ from which iid samples $\cV_n$ are selected. Then, the worst-case oblivious teaching complexity for \algoref{alg: randmaha} has an upper bound of $O\paren{\frac{p(p+1)}{2}}$ that achieves 1-accuracy for teaching up to linear scaling relation $\sim_{R_l}$.
    
%    On the other hand, for any $\epsilon \in (0,1]$, the worst-case oblivious teaching complexity for \algoref{alg: randmaha} has a lower bound of $\Omega\paren{\frac{p(p+1)}{2}}$ that achieves $\epsilon$-accuracy for teaching up to linear scaling relation $\sim_{R_l}$.
%\end{proposition}

\begin{theorem}[General Sampled Activations]\label{thm: samplegeneral}
    Consider a representation space $\cV \subset \reals^p$. Assume that the agent receives activations sampled independently and identically (i.i.d) from a Lebesgue distribution $\cD_{\cV}$. Then, for any target feature matrix $\pphi^* \in \mathcal{M}_{\sf{F}}$, with a tight bound of $n = \Theta(\frac{p(p+1)}{2})$ on the feedback complexity, the oblivious learner (almost surely) learns $\pphi^*$ up to feature equivalence using the feedback set $\cF(\cV_n,\pphi^*)$, i.e.,
    \begin{align*}
        \cP_{\cV} \paren{\forall\,\, \pphi' \in \cF(\cV_n,\pphi^*),\, \exists \lambda > 0, \pphi' = \lambda\cdot \pphi^* } = 1.
    \end{align*}
    %\begin{align*}
    %    \cP_{\cV_P}\paren{ \dim \paren{{ span \inner{\curlybracket{vv^T : v \in \cV_P}}}}  = P} = 1
    %\end{align*}
    %where $\cV_P \sim \cD_{\cV}^P$ such that $P = \frac{p(p+1)}{2}$.
\end{theorem}


%\begin{theorem}\label{thm: samplingsparse}
%    Consider a representation space $\cV \subset \reals^p$. Let $\cD_{\cV}$ be a Lebesgue measure over $\cV$ from which iid representations $\cV_n$ are selected. Then, we can achieve the following guarantees on the expected teaching complexity:
%    \begin{enumerate}
%        \item No-sparsity: Worst-case bound of $\frac{p(p+1)}{2}$ that achieves 1-accuracy for feature learning.
%        \item With sparsity: Worst-case bound of $f(d)$ that achieves $g(d)$-accuracy for feature learning.
%    \end{enumerate}
%\end{theorem}

\begin{assumption}[Sparse-distribution]\label{ass: sparse}
    Each index of a sparse activation vector $\alpha \in \reals^p$ is sampled i.i.d from a sparse distribution defined as: for all $i$
\begin{equation*}
\exists p_i \neq 0,\quad   \cP(\alpha_i  = 0) = p_i, \alpha_i\, |\, \alpha_i \neq 0 \sim \textit{Lebesgue}((0,1]), 
%\textbf{Beta}(\alpha_i(x); \mu_i, \nu_i) = \alpha_i(x)^{\mu_i -1}(1 - \alpha_i(x))^{\nu_i-1}
\end{equation*}
i.e., any index equals zero with a given non-zero probability and is non-zero based on a Lebesgue distribution.
\end{assumption}


\begin{theorem}[Sparse Sampled Activations]\label{thm: samplingsparse}
    Consider a representation space $\cV \subset \reals^p$. Assume that the agent receives representations sampled independently and identically (i.i.d) from a sparse distribution $\cD_{\cV}$ as stated in \assref{ass: sparse}. Fix a threshold $\delta > 0$, and sparsity parameter $s < p$. Then, for any target feature matrix $\pphi^* \in \mathcal{M}_{\sf{F}}$, with a bound of $n = p^2 \paren{\frac{2}{p_{\sf{s}}^2} \log \frac{2}{\delta}}^{1/p^2}$ on the feedback complexity using $s$-sparse feedbacks, the oblivious learner learns $\pphi^*$ up to feature equivalence with high probability using the feedback set $\cF(\cV_n,\pphi^*)$, i.e.,
     \begin{align*}
        \cP_{\cV} \paren{\forall\,\, \pphi' \in \cF(\cV_n,\pphi^*),\, \exists \lambda > 0, \pphi' = \lambda\cdot \pphi^* } \ge (1 - \delta).
    \end{align*}
    % Then, the feedback complexity of learning any target feature matrix $\pphi^* \in \mathcal{M}_{\sf{F}}$ with $s$-sparse feedbacks has a bound of $N$ with ($1 - \delta$)-accuracy, i.e, alternately,
    % \begin{align*}
    %     \cP_{\cV_N}\paren{ \dim \paren{{ span \inner{\curlybracket{vv^T : v \in \cV_N}}}}  \ge P} \ge 1 - \delta
    % \end{align*}
    % where $\cV_N \sim \cD_{\cV}^N$ such that $N = g(\delta, s)$.
\end{theorem}
\fi
%Now, we show a lower bound on the worst-case oblivious teaching complexity in the sampling case even to achieve any non-zero accuracy.
%\begin{lemma}
%       Consider an input space $\cV \subset \reals^d$. Let $\cD_{\cV}$ be a Lebesgue distribution over $\cV$ from which iid samples $\cV_n$ are selected. For any $\epsilon \in (0,1]$, the worst-case oblivious teaching complexity for \algoref{alg: randmaha} has a lower bound of $\Omega\paren{\frac{p(p+1)}{2}}$ that achieves $\epsilon$-accuracy for teaching up to linear scaling relation $\sim_{R_l}$.
%\end{lemma}

\paragraph{Rescaled Pairs} For a given matrix \( \pphi \neq 0 \), a sampled input \( x \sim \mathcal{D}_{\mathcal{V}} \) is almost never orthogonal, i.e., almost surely \( \pphi x \neq 0 \). This property can be utilized to rescale an input and construct pairs that satisfy equality constraints. Specifically, there exist scalars \( \gamma, \lambda > 0 \) such that (assuming without loss of generality \( x^{\top}\pphi x > y^{\top}\pphi y \)),
\begin{align*}
    x^{\top}\pphi x = \lambda \cdot y^{\top}\pphi y + y^{\top}\pphi y = (\sqrt{1 + \lambda}) y^{\top}\pphi (\sqrt{1 + \lambda}) y.
\end{align*}
Thus, the pair \( (x, (\sqrt{1 + \lambda})y) \) satisfies the equality constraints. With this understanding, we reformulate \algoref{alg: main} into \algoref{alg: randmaha}. 

In this section, we analyze the feedback complexity in terms of the minimum number of sampled activations required for the agent to construct an effective feedback set achieving feature equivalence which is illustrated in \figref{fig: monosparse}. Our first result establishes complexity bounds for general activations (without sparsity constraints) sampled from a Lebesgue distribution, with the complete proof provided in \appref{app: samplegeneral}.
%\akash{a remark on how complexity is measured in this setting}
% \akash{In the rest of the section, we study the expected oblivious teaching complexity for \algoref{alg: randmaha} under Lebesgue distribution \( \mathcal{D}_{\mathcal{V}} \) over the input space \( \mathcal{V} \). We show tight bounds on the teaching complexity highlighting a gap between the constructive and sampled teaching scenarios.} 

% \begin{theorem}[Sard's Theorem]
% Let \( f: X \to Y \) be a smooth map of manifolds, and let \( C \) be the set of critical points of \( f \) in \( X \). Then \( f(C) \) has measure zero in \( Y \).
% \end{theorem}

% Now, there are some questions to study:
% \begin{enumerate}
%     \item In the oblivious teaching scenario, what is the teaching complexity?
%     \item If the answer to 1. leads to poor lower bound on the teaching complexity, what are some ways we can improve the complexity.
% \end{enumerate}

% Thus, we consider rescaling the sampled activations in order for the superagent/teacher to design pairs for a given target feature matrix \( \pphi^* \in \mathcal{M}_{\sf{F}} \).

% \begin{definition}\label{defn: lebsample}
%     Consider an input space \( \mathcal{V} \subset \mathbb{R}^d \). Let \( \mathcal{D}_{\mathcal{V}} \) be a Lebesgue distribution over \( \mathcal{V} \) from which iid samples \( \mathcal{V}_n \) are selected. Fix a constant \( \epsilon > 0 \) and a target matrix \( \pphi^* \). We define the oblivious teaching complexity for \( \epsilon \)-accuracy for random samples as \( n \) if the teacher provides a teaching set \( \mathcal{T}(\mathcal{V}_n, \mathcal{M}_{\sf{F}}, \pphi^*) \) for \algoref{alg: randmaha} using the samples \( \mathcal{V}_n \) such that
%     \[
%         \mathcal{P}_{\mathcal{V}} \left( d_{M'} \in \textsf{VS}(\mathcal{T}(\mathcal{V}_n, \mathcal{M}_{\sf{F}}, \pphi^*), \mathcal{M}_{\sf{F}}) \text{ such that } d_{M'} \sim_{R_l} d_{\pphi^*} \right) \geq \epsilon.
%     \]
% \end{definition}

% First, we start with an upper bound on the worst-case (across the space \( \mathcal{M}_{\sf{F}} \)) oblivious teaching complexity that achieves \( 1 \)-accuracy.

% \begin{proposition}\label{prop: sampling}
%     Consider an input space \( \mathcal{V} \subset \mathbb{R}^d \). Let \( \mathcal{D}_{\mathcal{V}} \) be a Lebesgue distribution over \( \mathcal{V} \) from which iid samples \( \mathcal{V}_n \) are selected. Then, the worst-case oblivious teaching complexity for \algoref{alg: randmaha} has an upper bound of \( O\left(\frac{p(p+1)}{2}\right) \) that achieves \( 1 \)-accuracy for teaching up to linear scaling relation \( \sim_{R_l} \).
    
%     On the other hand, for any \( \epsilon \in (0,1] \), the worst-case oblivious teaching complexity for \algoref{alg: randmaha} has a lower bound of \( \Omega\left(\frac{p(p+1)}{2}\right) \) that achieves \( \epsilon \)-accuracy for teaching up to linear scaling relation \( \sim_{R_l} \).
% \end{proposition}

\begin{theorem}[General Sampled Activations]\label{thm: samplegeneral}
    Consider a representation space \( \mathcal{V} \subseteq \mathbb{R}^p \). Assume that the agent receives activations sampled i.i.d from a Lebesgue distribution \( \mathcal{D}_{\mathcal{V}} \). Then, for any target feature matrix \( \pphi^* \in \mathcal{M}_{\sf{F}} \), with a tight bound of \( n = \Theta\left(\frac{p(p+1)}{2}\right) \) on the feedback complexity, the oblivious learner (almost surely) learns \( \pphi^* \) up to feature equivalence using the feedback set \( \mathcal{F}(\mathcal{V}_n,\pphi^*) \), i.e.,
    \[
        \mathcal{P}_{\mathcal{V}} \left( \forall\,\, \pphi' \in \mathcal{F}(\mathcal{V}_n,\pphi^*),\, \exists \lambda > 0, \pphi' = \lambda \cdot \pphi^* \right) = 1.
    \]
    %\begin{align*}
    %    \mathcal{P}_{\mathcal{V}_P}\left( \dim \left( \text{span} \left\{ vv^\top : v \in \mathcal{V}_P \right\} \right) = P \right) = 1
    %\end{align*}
    %where \( \mathcal{V}_P \sim \mathcal{D}_{\mathcal{V}}^P \) such that \( P = \frac{p(p+1)}{2} \).
\end{theorem}
\begin{proof}[Proof Outline] The key observation is that almost surely for any $n \le p(p+1)/2$ randomly sampled activations on a unit sphere $\mathbb{S}^{p-1}$ under Lebesgue measure, the corresponding rank-1 matrices are linearly independent. This is a direct application of Sard's theorem on the zero set of a non-zero polynomial equation, yielding the upper bound. For the lower bound, we use some key necessary properties of a feedback set as elucidated in the proof of \thmref{thm: constructgeneral}. This result essentially fixes certain activations that need to be spanned by a feedback set, but under a Lebesgue measure on a continuous domain, we know that the probability of sampling a direction is zero.
\end{proof}

We consider a fairly general distribution over sparse activations similar to the signal model in \cite{bachsparse}.

\begin{assumption}[Sparse-Distribution]\label{ass: sparse}
    Each index of a sparse activation vector \( \alpha \in \mathbb{R}^p \) is sampled i.i.d from a sparse distribution defined as: for all \( i \),
    \[
        \mathcal{P}(\alpha_i = 0) = p_i, \quad \alpha_i \, | \, \alpha_i \neq 0 \sim \text{Lebesgue}((0,1]).
    \]
    %i.e., any index equals zero with a given non-zero probability and is non-zero based on a Lebesgue distribution.
\end{assumption}
With this we state the main theorem of the section with the proof deferred to \appref{app: samplesparse}.
\begin{theorem}[Sparse Sampled Activations]\label{thm: samplingsparse} Consider a representation space \( \mathcal{V} \subseteq \mathbb{R}^p \).
    Assume that the agent receives representations sampled i.i.d from a sparse distribution \( \mathcal{D}_{\mathcal{V}} \). Fix a threshold \( \delta > 0 \), and sparsity parameter \( s < p \). Then, for any target feature matrix \( \pphi^* \in \mathcal{M}_{\sf{F}} \), with a bound of $n = O\paren{p^2(\frac{2}{p_{\sf{s}}^2} \log \frac{2}{\delta})^{1/p^2}}$
    % \[
    %     n = O\paren{p^2 \left( \frac{2}{p_{\sf{s}}^2} \log \frac{2}{\delta} \right)^{1/p^2}}
    % \]
    on the feedback complexity using \( s \)-sparse feedbacks,the oblivious learner learns \( \pphi^* \) up to feature equivalence with high probability using the feedback set \( \mathcal{F}(\mathcal{V}_n,\pphi^*) \), i.e.,
    \[
        \mathcal{P}_{\mathcal{V}} \left( \forall\,\, \pphi' \in \mathcal{F}(\mathcal{V}_n,\pphi^*),\, \exists \lambda > 0, \pphi' = \lambda \cdot \pphi^* \right) \geq (1 - \delta),
    \]
    where $p_{\sf{s}}$ depends on $\cD_\cV$, and sparasity parameter $s$.
    % Then, the feedback complexity of learning any target feature matrix \( \pphi^* \in \mathcal{M}_{\sf{F}} \) with \( s \)-sparse feedbacks has a bound of \( N \) with (\( 1 - \delta \))-accuracy, i.e,
    % \begin{align*}
    %     \mathcal{P}_{\mathcal{V}_N}\left( \dim \left( \text{span} \left\{ vv^\top : v \in \mathcal{V}_N \right\} \right) \geq P \right) \geq 1 - \delta
    % \end{align*}
    % where \( \mathcal{V}_N \sim \mathcal{D}_{\mathcal{V}}^N \) such that \( N = g(\delta, s) \).
\end{theorem}
    \begin{proof}[Proof Outline]Using the formulation of \eqnref{eq: orthosat}, we need to estimate the number of activations the agent needs to receive/sample before an induced set of $p(p+1)/2$ many rank-1 linearly independent matrices are found.
    To estimate this, first we generalize the construction of the set ${\cB}$ from the proof of \thmref{thm: constructsparse} to 
    \begin{align*}
         %{U}_g = &\{\lambda_i e_i: \lambda_i \neq 0, 1 \leq i \leq p\}\,\, \cup\, \\ &\{(\lambda_{iji} e_i + \lambda_{ijj}e_j): \lambda_{iji},\lambda_{ijj} \neq 0,  1 \leq i < j \leq p \}.\\
         \widehat{U}_g = \curlybracket{\lambda_i^2 e_ie_i^T: 1 \leq i \leq p} \cup  \curlybracket{(\lambda_{iji} e_i + \lambda_{ijj}e_j)(\lambda_{iji} e_i + \lambda_{ijj}e_j)^T: 1 \leq i < j \leq p}
    \end{align*}
    We then analyze a design matrix $\mathbb{M}$ of rank-1 matrices from sampled activations and compute the probability of finding columns with entries semantically similar to those in $\widehat{U}_g$ (as vectorizations), ensuring a non-trivial determinant. The final complexity bound is derived using Hoeffding's inequality and Sterling's approximation.
    %Then we consider a design matrix $\mathbb{D}$ of rank-1 matrices induced by sampled activations and compute the probability that there exists columns which have a semantically similar entries as corresponded by $U_g$ so that the determinant of $\mathbb{D}$ is not trivial zero. With this, the final expression is based on Hoeffding's inequality and Sterling's approximation.
    %We evaluate the probability that rank-1 matrices induced by $U_g$
    \end{proof}

%\subsection{Reduction to zero signal}



\section{Experiments}
\label{sec:experiments}
The experiments are designed to address two key research questions.
First, \textbf{RQ1} evaluates whether the average $L_2$-norm of the counterfactual perturbation vectors ($\overline{||\perturb||}$) decreases as the model overfits the data, thereby providing further empirical validation for our hypothesis.
Second, \textbf{RQ2} evaluates the ability of the proposed counterfactual regularized loss, as defined in (\ref{eq:regularized_loss2}), to mitigate overfitting when compared to existing regularization techniques.

% The experiments are designed to address three key research questions. First, \textbf{RQ1} investigates whether the mean perturbation vector norm decreases as the model overfits the data, aiming to further validate our intuition. Second, \textbf{RQ2} explores whether the mean perturbation vector norm can be effectively leveraged as a regularization term during training, offering insights into its potential role in mitigating overfitting. Finally, \textbf{RQ3} examines whether our counterfactual regularizer enables the model to achieve superior performance compared to existing regularization methods, thus highlighting its practical advantage.

\subsection{Experimental Setup}
\textbf{\textit{Datasets, Models, and Tasks.}}
The experiments are conducted on three datasets: \textit{Water Potability}~\cite{kadiwal2020waterpotability}, \textit{Phomene}~\cite{phomene}, and \textit{CIFAR-10}~\cite{krizhevsky2009learning}. For \textit{Water Potability} and \textit{Phomene}, we randomly select $80\%$ of the samples for the training set, and the remaining $20\%$ for the test set, \textit{CIFAR-10} comes already split. Furthermore, we consider the following models: Logistic Regression, Multi-Layer Perceptron (MLP) with 100 and 30 neurons on each hidden layer, and PreactResNet-18~\cite{he2016cvecvv} as a Convolutional Neural Network (CNN) architecture.
We focus on binary classification tasks and leave the extension to multiclass scenarios for future work. However, for datasets that are inherently multiclass, we transform the problem into a binary classification task by selecting two classes, aligning with our assumption.

\smallskip
\noindent\textbf{\textit{Evaluation Measures.}} To characterize the degree of overfitting, we use the test loss, as it serves as a reliable indicator of the model's generalization capability to unseen data. Additionally, we evaluate the predictive performance of each model using the test accuracy.

\smallskip
\noindent\textbf{\textit{Baselines.}} We compare CF-Reg with the following regularization techniques: L1 (``Lasso''), L2 (``Ridge''), and Dropout.

\smallskip
\noindent\textbf{\textit{Configurations.}}
For each model, we adopt specific configurations as follows.
\begin{itemize}
\item \textit{Logistic Regression:} To induce overfitting in the model, we artificially increase the dimensionality of the data beyond the number of training samples by applying a polynomial feature expansion. This approach ensures that the model has enough capacity to overfit the training data, allowing us to analyze the impact of our counterfactual regularizer. The degree of the polynomial is chosen as the smallest degree that makes the number of features greater than the number of data.
\item \textit{Neural Networks (MLP and CNN):} To take advantage of the closed-form solution for computing the optimal perturbation vector as defined in (\ref{eq:opt-delta}), we use a local linear approximation of the neural network models. Hence, given an instance $\inst_i$, we consider the (optimal) counterfactual not with respect to $\model$ but with respect to:
\begin{equation}
\label{eq:taylor}
    \model^{lin}(\inst) = \model(\inst_i) + \nabla_{\inst}\model(\inst_i)(\inst - \inst_i),
\end{equation}
where $\model^{lin}$ represents the first-order Taylor approximation of $\model$ at $\inst_i$.
Note that this step is unnecessary for Logistic Regression, as it is inherently a linear model.
\end{itemize}

\smallskip
\noindent \textbf{\textit{Implementation Details.}} We run all experiments on a machine equipped with an AMD Ryzen 9 7900 12-Core Processor and an NVIDIA GeForce RTX 4090 GPU. Our implementation is based on the PyTorch Lightning framework. We use stochastic gradient descent as the optimizer with a learning rate of $\eta = 0.001$ and no weight decay. We use a batch size of $128$. The training and test steps are conducted for $6000$ epochs on the \textit{Water Potability} and \textit{Phoneme} datasets, while for the \textit{CIFAR-10} dataset, they are performed for $200$ epochs.
Finally, the contribution $w_i^{\varepsilon}$ of each training point $\inst_i$ is uniformly set as $w_i^{\varepsilon} = 1~\forall i\in \{1,\ldots,m\}$.

The source code implementation for our experiments is available at the following GitHub repository: \url{https://anonymous.4open.science/r/COCE-80B4/README.md} 

\subsection{RQ1: Counterfactual Perturbation vs. Overfitting}
To address \textbf{RQ1}, we analyze the relationship between the test loss and the average $L_2$-norm of the counterfactual perturbation vectors ($\overline{||\perturb||}$) over training epochs.

In particular, Figure~\ref{fig:delta_loss_epochs} depicts the evolution of $\overline{||\perturb||}$ alongside the test loss for an MLP trained \textit{without} regularization on the \textit{Water Potability} dataset. 
\begin{figure}[ht]
    \centering
    \includegraphics[width=0.85\linewidth]{img/delta_loss_epochs.png}
    \caption{The average counterfactual perturbation vector $\overline{||\perturb||}$ (left $y$-axis) and the cross-entropy test loss (right $y$-axis) over training epochs ($x$-axis) for an MLP trained on the \textit{Water Potability} dataset \textit{without} regularization.}
    \label{fig:delta_loss_epochs}
\end{figure}

The plot shows a clear trend as the model starts to overfit the data (evidenced by an increase in test loss). 
Notably, $\overline{||\perturb||}$ begins to decrease, which aligns with the hypothesis that the average distance to the optimal counterfactual example gets smaller as the model's decision boundary becomes increasingly adherent to the training data.

It is worth noting that this trend is heavily influenced by the choice of the counterfactual generator model. In particular, the relationship between $\overline{||\perturb||}$ and the degree of overfitting may become even more pronounced when leveraging more accurate counterfactual generators. However, these models often come at the cost of higher computational complexity, and their exploration is left to future work.

Nonetheless, we expect that $\overline{||\perturb||}$ will eventually stabilize at a plateau, as the average $L_2$-norm of the optimal counterfactual perturbations cannot vanish to zero.

% Additionally, the choice of employing the score-based counterfactual explanation framework to generate counterfactuals was driven to promote computational efficiency.

% Future enhancements to the framework may involve adopting models capable of generating more precise counterfactuals. While such approaches may yield to performance improvements, they are likely to come at the cost of increased computational complexity.


\subsection{RQ2: Counterfactual Regularization Performance}
To answer \textbf{RQ2}, we evaluate the effectiveness of the proposed counterfactual regularization (CF-Reg) by comparing its performance against existing baselines: unregularized training loss (No-Reg), L1 regularization (L1-Reg), L2 regularization (L2-Reg), and Dropout.
Specifically, for each model and dataset combination, Table~\ref{tab:regularization_comparison} presents the mean value and standard deviation of test accuracy achieved by each method across 5 random initialization. 

The table illustrates that our regularization technique consistently delivers better results than existing methods across all evaluated scenarios, except for one case -- i.e., Logistic Regression on the \textit{Phomene} dataset. 
However, this setting exhibits an unusual pattern, as the highest model accuracy is achieved without any regularization. Even in this case, CF-Reg still surpasses other regularization baselines.

From the results above, we derive the following key insights. First, CF-Reg proves to be effective across various model types, ranging from simple linear models (Logistic Regression) to deep architectures like MLPs and CNNs, and across diverse datasets, including both tabular and image data. 
Second, CF-Reg's strong performance on the \textit{Water} dataset with Logistic Regression suggests that its benefits may be more pronounced when applied to simpler models. However, the unexpected outcome on the \textit{Phoneme} dataset calls for further investigation into this phenomenon.


\begin{table*}[h!]
    \centering
    \caption{Mean value and standard deviation of test accuracy across 5 random initializations for different model, dataset, and regularization method. The best results are highlighted in \textbf{bold}.}
    \label{tab:regularization_comparison}
    \begin{tabular}{|c|c|c|c|c|c|c|}
        \hline
        \textbf{Model} & \textbf{Dataset} & \textbf{No-Reg} & \textbf{L1-Reg} & \textbf{L2-Reg} & \textbf{Dropout} & \textbf{CF-Reg (ours)} \\ \hline
        Logistic Regression   & \textit{Water}   & $0.6595 \pm 0.0038$   & $0.6729 \pm 0.0056$   & $0.6756 \pm 0.0046$  & N/A    & $\mathbf{0.6918 \pm 0.0036}$                     \\ \hline
        MLP   & \textit{Water}   & $0.6756 \pm 0.0042$   & $0.6790 \pm 0.0058$   & $0.6790 \pm 0.0023$  & $0.6750 \pm 0.0036$    & $\mathbf{0.6802 \pm 0.0046}$                    \\ \hline
%        MLP   & \textit{Adult}   & $0.8404 \pm 0.0010$   & $\mathbf{0.8495 \pm 0.0007}$   & $0.8489 \pm 0.0014$  & $\mathbf{0.8495 \pm 0.0016}$     & $0.8449 \pm 0.0019$                    \\ \hline
        Logistic Regression   & \textit{Phomene}   & $\mathbf{0.8148 \pm 0.0020}$   & $0.8041 \pm 0.0028$   & $0.7835 \pm 0.0176$  & N/A    & $0.8098 \pm 0.0055$                     \\ \hline
        MLP   & \textit{Phomene}   & $0.8677 \pm 0.0033$   & $0.8374 \pm 0.0080$   & $0.8673 \pm 0.0045$  & $0.8672 \pm 0.0042$     & $\mathbf{0.8718 \pm 0.0040}$                    \\ \hline
        CNN   & \textit{CIFAR-10} & $0.6670 \pm 0.0233$   & $0.6229 \pm 0.0850$   & $0.7348 \pm 0.0365$   & N/A    & $\mathbf{0.7427 \pm 0.0571}$                     \\ \hline
    \end{tabular}
\end{table*}

\begin{table*}[htb!]
    \centering
    \caption{Hyperparameter configurations utilized for the generation of Table \ref{tab:regularization_comparison}. For our regularization the hyperparameters are reported as $\mathbf{\alpha/\beta}$.}
    \label{tab:performance_parameters}
    \begin{tabular}{|c|c|c|c|c|c|c|}
        \hline
        \textbf{Model} & \textbf{Dataset} & \textbf{No-Reg} & \textbf{L1-Reg} & \textbf{L2-Reg} & \textbf{Dropout} & \textbf{CF-Reg (ours)} \\ \hline
        Logistic Regression   & \textit{Water}   & N/A   & $0.0093$   & $0.6927$  & N/A    & $0.3791/1.0355$                     \\ \hline
        MLP   & \textit{Water}   & N/A   & $0.0007$   & $0.0022$  & $0.0002$    & $0.2567/1.9775$                    \\ \hline
        Logistic Regression   &
        \textit{Phomene}   & N/A   & $0.0097$   & $0.7979$  & N/A    & $0.0571/1.8516$                     \\ \hline
        MLP   & \textit{Phomene}   & N/A   & $0.0007$   & $4.24\cdot10^{-5}$  & $0.0015$    & $0.0516/2.2700$                    \\ \hline
       % MLP   & \textit{Adult}   & N/A   & $0.0018$   & $0.0018$  & $0.0601$     & $0.0764/2.2068$                    \\ \hline
        CNN   & \textit{CIFAR-10} & N/A   & $0.0050$   & $0.0864$ & N/A    & $0.3018/
        2.1502$                     \\ \hline
    \end{tabular}
\end{table*}

\begin{table*}[htb!]
    \centering
    \caption{Mean value and standard deviation of training time across 5 different runs. The reported time (in seconds) corresponds to the generation of each entry in Table \ref{tab:regularization_comparison}. Times are }
    \label{tab:times}
    \begin{tabular}{|c|c|c|c|c|c|c|}
        \hline
        \textbf{Model} & \textbf{Dataset} & \textbf{No-Reg} & \textbf{L1-Reg} & \textbf{L2-Reg} & \textbf{Dropout} & \textbf{CF-Reg (ours)} \\ \hline
        Logistic Regression   & \textit{Water}   & $222.98 \pm 1.07$   & $239.94 \pm 2.59$   & $241.60 \pm 1.88$  & N/A    & $251.50 \pm 1.93$                     \\ \hline
        MLP   & \textit{Water}   & $225.71 \pm 3.85$   & $250.13 \pm 4.44$   & $255.78 \pm 2.38$  & $237.83 \pm 3.45$    & $266.48 \pm 3.46$                    \\ \hline
        Logistic Regression   & \textit{Phomene}   & $266.39 \pm 0.82$ & $367.52 \pm 6.85$   & $361.69 \pm 4.04$  & N/A   & $310.48 \pm 0.76$                    \\ \hline
        MLP   &
        \textit{Phomene} & $335.62 \pm 1.77$   & $390.86 \pm 2.11$   & $393.96 \pm 1.95$ & $363.51 \pm 5.07$    & $403.14 \pm 1.92$                     \\ \hline
       % MLP   & \textit{Adult}   & N/A   & $0.0018$   & $0.0018$  & $0.0601$     & $0.0764/2.2068$                    \\ \hline
        CNN   & \textit{CIFAR-10} & $370.09 \pm 0.18$   & $395.71 \pm 0.55$   & $401.38 \pm 0.16$ & N/A    & $1287.8 \pm 0.26$                     \\ \hline
    \end{tabular}
\end{table*}

\subsection{Feasibility of our Method}
A crucial requirement for any regularization technique is that it should impose minimal impact on the overall training process.
In this respect, CF-Reg introduces an overhead that depends on the time required to find the optimal counterfactual example for each training instance. 
As such, the more sophisticated the counterfactual generator model probed during training the higher would be the time required. However, a more advanced counterfactual generator might provide a more effective regularization. We discuss this trade-off in more details in Section~\ref{sec:discussion}.

Table~\ref{tab:times} presents the average training time ($\pm$ standard deviation) for each model and dataset combination listed in Table~\ref{tab:regularization_comparison}.
We can observe that the higher accuracy achieved by CF-Reg using the score-based counterfactual generator comes with only minimal overhead. However, when applied to deep neural networks with many hidden layers, such as \textit{PreactResNet-18}, the forward derivative computation required for the linearization of the network introduces a more noticeable computational cost, explaining the longer training times in the table.

\subsection{Hyperparameter Sensitivity Analysis}
The proposed counterfactual regularization technique relies on two key hyperparameters: $\alpha$ and $\beta$. The former is intrinsic to the loss formulation defined in (\ref{eq:cf-train}), while the latter is closely tied to the choice of the score-based counterfactual explanation method used.

Figure~\ref{fig:test_alpha_beta} illustrates how the test accuracy of an MLP trained on the \textit{Water Potability} dataset changes for different combinations of $\alpha$ and $\beta$.

\begin{figure}[ht]
    \centering
    \includegraphics[width=0.85\linewidth]{img/test_acc_alpha_beta.png}
    \caption{The test accuracy of an MLP trained on the \textit{Water Potability} dataset, evaluated while varying the weight of our counterfactual regularizer ($\alpha$) for different values of $\beta$.}
    \label{fig:test_alpha_beta}
\end{figure}

We observe that, for a fixed $\beta$, increasing the weight of our counterfactual regularizer ($\alpha$) can slightly improve test accuracy until a sudden drop is noticed for $\alpha > 0.1$.
This behavior was expected, as the impact of our penalty, like any regularization term, can be disruptive if not properly controlled.

Moreover, this finding further demonstrates that our regularization method, CF-Reg, is inherently data-driven. Therefore, it requires specific fine-tuning based on the combination of the model and dataset at hand.
\section*{Acknowledgments}
Author thanks the National Science Foundation for support under grant IIS-2211386 in duration of this project. Author thanks Sanjoy Dasgupta (UCSD) for helping to develop the preliminary ideas of the work. Author thanks Geelon So (UCSD) for many extended discussions on the project. Author also thanks Mikhail (Misha) Belkin (UCSD) and Enric Boix-Adserà (MIT) for a helpful discussion during a visit to Simons Insitute. The general idea of writing was developed while the author was visiting Simons Insitute (UC Berkeley) for a workshop for which the travel was supported by the National Science Foundation (NSF) and the Simons Foundation for the Collaboration on the Theoretical Foundations of Deep Learning through awards DMS-2031883 and \#814639.
\newpage
%\section{Information bottleneck: Triplet comparisons versus Quadruple comparisons}

In this section, we will discuss the consistency up to which triplet and quadruple comparisons specify a given distance function. Consider a distance function $d$ on an arbitrary sample space $\cX$ (can be any well-defined object). First, we define quadruple comparison in the following.

\paragraph{Quadruple comparisons:} In Section 2, we defined the concept of oblivious teaching of a metric \( d \in \mathcal{M} \) using triplet comparisons (see \eqnref{eq: sol}). An extension to triplet comparisons is the notion of quadruple comparisons. For any given points \( v_1, v_2, v_3, v_4 \), a comparison of the form \( (v_1, v_2, v_3, v_4)_q \) denotes
\begin{align*}
 (v_1, v_2, v_3, v_4)_q
    = \begin{cases}
        = & \textnormal{if } d(v_1, v_2) = d(v_3, v_4)\\
        < & \textnormal{if } d(v_1, v_2) < d(v_3, v_4)\\
        > & \textnormal{if } d(v_1, v_2) > d(v_3, v_4)
    \end{cases}
\end{align*}
If \( v_1 = v_3 \), then the quadruple comparison reduces to a triplet comparison, thus generalizing the notion of triplet comparisons. It is straightforward that quadruple comparisons are at least as informative as triplet comparisons. In the following, we will demonstrate that they are strictly more informative for oblivious teaching of a graph distance function.

We want to understand whether two comparisons of interest, triplet, and quadruples, can fully specify a distance function $d$ up to monotonic transformation $f$ which is defined as follows:
%\sanjoy{I think it would be better to avoid using capital script for the transformation since we have otherwise been using it for classes. Maybe just $f$ instead of $\cF$?}

\begin{definition}\label{def: monotonic}
    Define a scalar transformation $f: \reals \to \reals$. We call $f$ to be strictly monotonically increasing if for all scalars $u < v$ we have $f(u) < f(v)$.

\end{definition}
%\sanjoy{We will actually need it to be strictly monotonic.}

Now, let $w: E \to \reals_{+}$ be the weights or distances between the nodes in $\cX_N$ as assigned by $d$.
We say a learner learns a distance function $d$ upto monotonic transformation if for all such $f$ and for any $w_i,w_j \in w$ $f(w'_{i}) \le f(w'_j)$ if $w_i \le w_j$ where $(d',w')$ is the learned distance function.
Note that a (strict) monotonic transformation on a real line is invertible. Thus, if a distance function $d$ is learnt up to monotonic transformation, then there exists $f$ such that
\begin{align*}
    f^{-1}(w') = w
\end{align*}
We are interested in understanding the information bottleneck of triplet versus quadruple comparisons in specifying a distance function. We study this as an equivalence property induced by distance specification up to monotonic transformation as defined in \defref{def: monotonic}.
\begin{definition}
    Consider two distance functions $d,d'$ on a sample space $\cX$. We say $d$ and $d'$ are ordinally equivalent, if they are equivalent up to strictly monotonic transformation, i.e.
    \begin{align*}
        \exists \textnormal{ a mono. trans. } f,\, s.t.\,\, \forall x,y \in \cX,  \quad d'(x,y) = f (d(x,y)).
    \end{align*}
\end{definition}

We are interested in understanding if triplet comparisons can specify $d$ up to ordinal equivalence. The answer is negative, i.e. triplet comparisons alone can't specify $d$ up to ordinal equivalence. The counterexample is a simple path distance function of length 3 as follows:

\begin{lemma}
    For a path metric $d$ on four points $a,b,c,d$ such that $d(a,b) = 3$, $d(b,c) = 4$ and $d(c,d)= 2$. Triplet comparisons can't specify $d$ up to ordinal equivalence.
\end{lemma}
\begin{figure}[ht!]
    \centering
    \begin{subfigure}[b]{0.8\textwidth}
        \centering
        \begin{tikzpicture}
            % First diagram
            \node (a) at (0,0) {a};
            \node (b) at (3,0) {b};
            \node (c) at (7,0) {c};
            \node (d) at (9,0) {d};

            \draw (a) -- (b) node[midway, above] {3};
            \draw (b) -- (c) node[midway, above] {4};
            \draw (c) -- (d) node[midway, above] {2};
        \end{tikzpicture}
        \caption{Correct metric $d$}
        \label{fig:correct_metric}
    \end{subfigure}
    
    \vspace{1cm} % Adjust the space between the diagrams as needed

    \begin{subfigure}[b]{0.8\textwidth}
        \centering
        \begin{tikzpicture}
            % Second diagram
            \node (a2) at (0,0) {a};
            \node (b2) at (2,0) {b};
            \node (c2) at (6,0) {c};
            \node (d2) at (9,0) {d};

            \draw (a2) -- (b2) node[midway, above] {2};
            \draw (b2) -- (c2) node[midway, above] {4};
            \draw (c2) -- (d2) node[midway, above] {3};
        \end{tikzpicture}
        \caption{Possible triplet comparisons induced metric}
        \label{fig:possible_metric}
    \end{subfigure}
    
\end{figure}
\begin{proof}
 Note that the path metric specified in \figref{fig:possible_metric}, denoted as $d'$ satisfies all possible triplet comparisons on the metric $d$ as shown in \figref{fig:correct_metric}:
 \begin{align*}
     (a,b,c), (a,b,d), (a,c,d), (b,a,c),(b,a,d), (b,c,d),\ldots, (d,c,b), (d,c,a), (d,b,a)
 \end{align*}
 But $d'(a,b) < d'(c,d)$ whereas $d(a,b) > d(c,d)$.
\end{proof}

This issue arises because triplet comparisons can't specify the ordering of the distances $d(a,b)$ and $d(c,d)$. Essentially, it only specifies the ordering with respect to a center (node) not when we compare distances without fixing a center. We can resolve this issue with \tt{quadruples}.

\begin{lemma}
    Consider $\cX$ to be an underlying sample space, and $d: \cX \to \cX \to \reals$ be a target distance function. Then, quadruplets specify $d$ up to ordinal equivalence.%, i.e. learner finds a distance function $d': \cX \times \cX \to \reals$ such that 
    %\begin{align*}
    %    \exists \textnormal{ a mono. trans. } f,\, s.t.\,\, \forall x,y \in \cX,  \quad d'(x,y) = f (d(x,y))
    %\end{align*}
\end{lemma}
%\sanjoy{We can henceforth call this \emph{ordinal equivalence}: that is, equivalence upto strictly monotonic transformation.}

\begin{proof}
    Assume that the learner has access to all possible quadruplets. It is clear that for any $x,y,u,v$,
    \begin{align}
        d(x,y) = (\textnormal{or} >)\,\, d(u,v) \implies d'(x,y) =(\textnormal{or} >)\,\, d'(u,v) \label{eq: disteq}
    \end{align}
    Now, consider this restricted transformation $f_{\lvert}$ as
    \begin{align*}
        f_{\lvert}(d(u,v)) := \frac{d'(u,v)}{d(u,v)} \cdot d(u,v)
    \end{align*}
    \eqnref{eq: disteq} implies $f_{\lvert}$ is well-defined on any scalar $r \in \reals$ as long as $r = d(u,v)$ for some $u,v \in \cX$. Note that on the restriction $f_{\lvert}$ is one-one and onto (over the domain of realizable distances). 
    
    Now, $f_{\lvert}$ can be extended to $f$ for all of $\reals$ as follows: if $r \in \reals$, s.t. $\forall u,v \in \cX$, $d(u,v) \neq r$, then define
    \begin{align}
         f(r) := \frac{1}{2}\paren{
    \limsup_{\substack{r' < r, \\ \exists u,v, d(u,v) = r'}} d'(u,v) + 
    \liminf_{\substack{r' > r, \\ \exists u,v, d(u,v) = r'}} d'(u,v)}
    \end{align}
    Thus, we have shown that there exists a monotonic transformation up to which $d$ is specified as $d'$.
\end{proof}
%\sanjoy{We can use the term \emph{triplet equivalence} when two distance functions assign the same labeling to all triplets. As we have seen, this is looser than ordinal equivalence.}


\section{General distance functions on finite spaces}
%\sanjoy{Our bounds apply to arbitrary distance functions on finite spaces; but talking about graphs constrains us to metrics. So it might be better to use graph terminology only for the lower bounds.}

In this section, we consider arbitrary distance functions on a set of discrete samples $\cX_N \subset \cX$, as shown in \figref{fig:graph_metric}. We denote the elements of $\cX_N$ as $v_1,v_2,\ldots,v_N$. 

\paragraph{Notations}: We alternatively use $\cX$ or $\cX$ (unless stated otherwise). For a distance function on $\cX$, we denote the distance function as $d_w$ (wrt the weights $w$ on any edge connecting two nodes in $\cX$). Nodes or vertices in $\cX$ are denoted as $u,v,t,x$.

\subsection{Bounds for triplet comparisons}
%In the previous section, we discussed the oblivious teaching complexity of Mahalanobis distance metrics. 

In this section, we discuss distance functions on a set of objects $\cX$ containing $n$ vertices. 

Here, we assume that the sample space is a finite set of objects; with the underlying distance function model denoted as $\fin$, i.e any $(\cX, w \in \fin)$ is a finite sample distance function on the nodes in $\cX$ and weighting $w: \cX \times \cX \to \reals_{+}$. 
%the graphs $G$ on $\cX$ are specific graph distance functions; with the underlying distance function model denoted as $\fin$, i.e any $(\cX, G, w) \in \fin$ is a graph distance function on vertices $\cX$ and weighting $w$.
%Here, we assume that the graphs $G$ on $\cX$ are specific trees; with the underlying metric model denoted as $\tree$, i.e any $(\cX, G, w) \in \tree$ is a tree metric on vertices $\cX$ with tree graph $G$ and weighting $w$.



\begin{algorithm}[t]
\caption{Teaching a general distance function on finite samples with triplet comparisons}
\label{alg: tree}
\textbf{Given}: Input space $\cX_N \sim \cD_{\cX}$, distance function model $\cM_{\mathsf{fin}}$\\
\vspace{1mm}
\textit{In batch setting:}\vspace{1mm}
\begin{enumerate}
    \item teacher picks pairs $\cT(\cX,d_{w^*}, \fin) =$ $\curlybracket{(t,u,v)_{\ell} \in \cX \,|\, \ell((t,u,v); d_{w^*}) = \sgn{w^*(e_{tu}) - w^*(e_{tv})}}$%\sum_{e' \in P_G^*(t, u)} w^*(e') \ge \sum_{e' \in P_G^*(u,v)} w^*(e')}$
    \item learner receives $\cT$; and obliviously picks a distance function $d \in \fin$ as per \eqnref{eq: sol}
\end{enumerate}
\end{algorithm}
%\documentclass{standalone}




First, we consider the case when the teacher can \tt{deterministically} select any node in $\cX$ to construct a triplet comparison. In this case, we show that the teacher needs to provide at least $\Omega(|\cX|^2)$ triplet comparisons in \algoref{alg: tree} in the worst-case scenario. We show a construction of a distance function that can't be sufficiently taught up to triplet equivalence in size less than a quadratic dependence on the size of $\cX$. But on the other hand, this dependence is sufficient, i.e. we can obliviously teach any finite sample distance function up to triplet equivalence with $O(|\cX|^2)$ comparisons. 

For a given sample $\cX$ of fixed set of nodes, distances $d_w$ and $d_{w'}$, are related up to triplet constraints, if
\begin{align*}
    \forall(t, u, v) \in \cX^3, \ell((t,u,v); d_{w}) = \ell((t,u,v); d_{w'})%\paren{w(e_{tu}) \ge w(e_{tv})} \Longleftrightarrow \paren{w'(e_{tu}) \ge w'(e_{tv})}%\paren{\sum_{e' \in P_G(t, u)} w(e') \ge \sum_{e' \in P_G(u,v)} w(e')} \Longleftrightarrow \paren{\sum_{e' \in P_G(t, u)} w'(e') \ge \sum_{e' \in P_G(u,v)} w'(e')}
\end{align*}
We denote this relation as $\sim_{triplet}$. We define this relation as triplet equivance as
\begin{definition}[triplet equivalence]
    On a given sample space $\cX$, we say distance functions $d$ and $d'$ are related up to triplet equivalence if for any triplet $(t,u,v) \in \cX^3$
    \begin{equation*}
        \ell((t,u,v); d) = \ell((t,u,v); d')
    \end{equation*}
\end{definition}
%This induces an equivalence class on the set of graph distance functions on a fixed graph $G$. 
%Later in the section, we will show that triplet comparisons have certain limitations in providing the exact ordering on the weights of the edges of the graph, i.e we can't hope to teach a graph distance function consistent up to exact ordering (we call this monotonic transformation) to an oblivious learner. This necessitates a much weaker notion of teaching in the form of the triplet constraints relation. 

First, we show a worst-case lower bound on the oblivious teaching complexity for \algoref{alg: tree} where we use a construction of a tree metric.

%\sanjoy{I think we should just focus on the finite-space setting here, rather than tree or graph (which can be saved for a different time). In this setting:}
%\begin{itemize}
%\item \sanjoy{Upper bound: If $|\cX| = N$, then any distance function can be taught upto triplet equivalence using $N(N-2)$ triplets (as in your construction).}
%\item \sanjoy{Upper bound: Likewise, any symmetric distance function can be taught upto ordinal equivalence using ${N \choose 2} -1$ quadruples.}
%\item \sanjoy{Lower bound: Any symmetric distance function requires at least ${N \choose 2}-1$ quadruples to teach upto ordinal equivalence. To see this, consider an undirected graph with ${N \choose 2}$ nodes, each corresponding to a pair of points from $\cX$. We place an edge between two nodes $\{u,v\}$ and $\{x,y\}$ if there is a quadruple involving those two pairs (e.g. $(u,v, x,y)$). Such a quadruple tells us about the relative size of $d(u,v)$ versus $d(x,y)$. If we have less than ${N \choose 2} -1$ quadruples, then this graph will necessarily be disconnected. This means that anything is possible for distance comparisons between pairs of points in different components.}
%\item \sanjoy{Lower bound: Any distance function requires at least ${N \choose 2}-1$ triples to teach upto triplet-equivalence. Roughly the same argument should work. We can discuss further.}
%\end{itemize}

\begin{proposition}\label{prop: treelower}
    For any distance $d$ on a set $\cX$ of size $N$, the teacher needs to necessarily provide $N(N-2)$ triplet comparisons for the learner to identify the target distance function up to triplet equivalence.
    %there exists a tree metric that requires at least $\Omega(n^2)$ triplet comparisons for oblivious teaching in \algoref{alg: tree}.
    %taught up to triplet constraints relation in 
    %For a tree metric on the pair $(\cX,E, w)$, given any target metric $d_T$ requires at the least %$\Omega(|\cX|^2)$ many teaching triplets.
\end{proposition}
\begin{proof}
    Consider a distance function $d'$ that is triplet equivalent to $d$. This implies that for any triplets $x,a,b \in \cX^3$, $\ell((x,a,b);d) = \ell((x,a,b);d')$. This implies that for any fixed center $x$, ordering of the distances $d(x,y)$ over $y \in \cX\setminus \curlybracket{x}$ should be taught. Note that any ordering 
    \begin{align*}
        d(x,y_1) \le d(x,y_2)\le \ldots \le d(x,y_{N-1})
    \end{align*}
    where $y_i \in \cX \setminus \curlybracket{x}$, can only be fully specified by $(N-2)$ pairs, otherwise $d$ and $d'$ would differ on at least one pair of samples. Thus, for any fixed $x$, it requires $(N-2)$ samples to specify $d$ up to triplet equivalence. Since the ordering of $d(x',\cdot)$ wrt $x' \neq x$ is independent of $d(x,\cdot)$, for oblivious teaching an arbitrary distance function on $\cX$ requires $N(N-2)$ (there are $N$ samples) triplet pairs to teach $d$ up to triplet equivalence. 
\end{proof}
\begin{comment}
\begin{proof}
     The key idea is to consider a target weighting $\boldsymbol{w}^*$ that alternates around a fixed scalar $\delta$ with perturbations on a tree with a single path (see \figref{fig:tree}). One can show that if the perturbations are small enough, a large number of triplet comparisons (at least $\Omega(n^2)$) is required to teach this metric.

    Consider a tree graph $G_p$ with $n$ ordered vertices $\curlybracket{v_1,v_2,\ldots,v_N}$ such that for each $i = 2,\ldots, n-1$ $v_i$ is connected to $\curlybracket{v_{i-1},v_{i+1}}$ and $v_1$ and $v_{n}$ are only connected to $v_2$ and $v_{n-1}$ respectively. For ease of notation, we use subscript $i$ of a weighting $w$ to denote the weight on edge connecting $v_{i-1}$ and $v_{i+1}$ (thus there are precisely $n-1$ coordinates in $\boldsymbol{w}$).
    
    Now, consider weighting $\boldsymbol{w}$: $\forall i$
    \begin{align}
    \boldsymbol{w}_i = \begin{cases}
        \delta + \epsilon_i,\, \epsilon_i > 0 & \textit{if}\quad  i \equiv 0 \mod 2 \\
        \delta - \epsilon_i,\, \epsilon_i > 0 & \textit{if}\quad  i \equiv 1 \mod 2
    \end{cases} \label{eq: mod1}\\
    \boldsymbol{w}_i + \boldsymbol{w}_{i+1} > \boldsymbol{w}_{i+2} + \boldsymbol{w}_{i+3}\quad \textit{if }\quad  i \equiv 1 \mod 4\label{eq: mod2}\\
    \boldsymbol{w}_i + \boldsymbol{w}_{i+1} < \boldsymbol{w}_{i+2} + \boldsymbol{w}_{i+3}\quad \textit{if }\quad  i \equiv 3 \mod 4 \label{eq: mod3}
    \end{align}
    Any large enough $\delta$ is sufficient for the analysis which will be clear from the construction below.    
    Furthermore, any 4-tuple sums to the left to be greater than one to the right as follows:
    \begin{align}
        \forall i,\, \boldsymbol{w}_i + \boldsymbol{w}_{i+1} + \boldsymbol{w}_{i+2} + \boldsymbol{w}_{i+3} > \boldsymbol{w}_{i+4} + \boldsymbol{w}_{i+5} + \boldsymbol{w}_{i+6} + \boldsymbol{w}_{i+7} \label{eq: mod4}
    \end{align}
    A schematic diagram for this weighting is shown in \figref{fig:tree}. Note that if for all $\epsilon_i = \epsilon_0$ (for some positive scalar $\epsilon_0$), then the conditions in \eqnref{eq: mod1}-\eqnref{eq: mod3} are trivially satisfied.
    
    Now, consider the assignment where for all $k,j = 0,1,\ldots$ %$\epsilon_{4i+1} = \epsilon_0 + .1$ and $\epsilon_{4i + 0/2/3} = \epsilon_0$. 
    \begin{gather*}
        \epsilon_{4k+1} = \epsilon_0 + .2 + 10^{-k}\\
        \epsilon_{4k+3} = \epsilon_0 + 10^{-k}\\
        \epsilon_{2j} = \epsilon_0 
    \end{gather*}
    Assume that $\epsilon_0> .2*n$. Note that this assignment ensures all the comparisons lead to (strict) inequalities. This gives a non-trivial weighting.
    
    %We assume that $n$ is at least $4$; it is only a restriction for the proof methodology not on the claim of the lower bound. 
    Note that there are $\binom{n}{3}$ or $\Omega(n^3)$ many triplet comparisons possible on a tree metric of size $n$. Now consider triplet comparisons 
    \begin{align}
       \cT' = \curlybracket{(v_{i_1},v_{i_2},v_{i_3})\,|\,i_3 > i_1 > i_2,\, i_3 - i_2 \equiv 0 \mod 4,\, i_3 - i_2 \not\equiv 0 \mod 8} 
    \end{align}
    Now, we would argue that there is at one such comparison that the teacher can't provide otherwise the total comparisons provided is at least $\Omega(n^2)$. Let $\cT(\cX, d_{\boldsymbol{w}},\fin)$ be a teaching set of triplet comparisons for oblivious teaching of target tree metric $(G,\boldsymbol{w})$. For the sake of contradiction assumed that $\cT = O(n^2)$.
    
    First, we note that the number of continuous segments on a path whose length is divisible by 4 but not by 8 is 
    \begin{align}
     (n - 4) + (n - 12) + \dots = \Omega(n^2) \implies |\cT'| = \Omega(n^2)   
    \end{align}
    Thus, there is one triplet comparison in $\cT'$ not provided by the teacher in the set $\cT$. In other words, there exists an odd number $j$ and non-negative integer $i$ such that the teacher didn't provide the triplet $(v_{2j + i}, v_{i}, v_{4j + i})$. WLOG assume
    \begin{align*}
        \sum_{k=i}^{2j+i} \boldsymbol{w}_k < \sum_{k=2j+i+1}^{4j+i} \boldsymbol{w}_k
    \end{align*}
    
    Now, we show that the learner maintains a version space $\sf{VS}(\cT)$
    of feasible tree metrics that satisfying the opposing triplet comparisons for $(v_{2j + i}, v_{i}, v_{4j + i})$, i.e. there exists $(G,\hat{w}), (G,\tilde{w}) \in \sf{VS}(\cT)$  such that $\hat{w}$ satifies $(v_{2j + i}, v_{i}, v_{4j + i})$ (with strict inequality) and $\tilde{w}$ satisfies $(v_{2j + i}, v_{4j + i}, v_{i})$. Alternatively,
    \begin{align*}
        \sum_{k=i}^{2j+i} \hat{w}_k < \sum_{k=2j+i+1}^{4j+i} \hat{w}_k\\
        \sum_{k=i}^{2j+i} \tilde{w}_k \ge \sum_{k=2j+i+1}^{4j+i} \tilde{w}_k
    \end{align*}
    It is clear that if we use the assignment: for all $i = 0,1,\ldots$ $\epsilon_{4i+1} = \epsilon_0 + .1$ and $\epsilon_{4i + 0/2/3} = \epsilon_0$ for some large enough $\delta$ then the version space contains a tree metric that satisfies the correct triplet comparison: $(v_{2j + i}, v_{i}, v_{4j + i})$. 

\begin{figure}[t]
 

\centering
\begin{tikzpicture}

% Define coordinates for the zigzag pattern
\coordinate (A) at (1,0.3);
\coordinate (B) at (3,1.3);
\coordinate (C) at (5,0.3);
\coordinate (D) at (7,1.3);
\coordinate (E) at (9,0.3);
\coordinate (F) at (11,1.3);
%\coordinate (G) at (13,0);

% Draw the zigzag pattern
\draw (A) -- (B) -- (C) -- (D) -- (E) -- (F);

% Draw circles with text "4.3" inside
\foreach \x in {0.5, 1.5, 2.5, 3.5, 4.5,5.5,6.5,7.5, 8.5, 9.5, 10.5, 11.5} {
    \node[circle, draw, minimum size=.5cm] at (\x, -0.5) {$\delta$};
}

\coordinate (A) at (-1.0,-.5);
\coordinate (B) at (.2,-.5);
\draw (A) -- (B);

\coordinate (A) at (11.8,-.5);
\coordinate (B) at (12.8,-.5);
\draw (A) -- (B);

% Define coordinates for the zigzag pattern
\coordinate (A) at (0.5,-2.3);
\coordinate (B) at (1.5,-1.3);
\coordinate (C) at (2.5,-2.3);
\coordinate (D) at (3.5,-1.3);
\coordinate (E) at (4.5,-2.3);
\coordinate (F) at (5.5,-1.3);
\coordinate (G) at (6.5,-2.3);
\coordinate (H) at (7.5,-1.3);
\coordinate (I) at (8.5,-2.3);
\coordinate (J) at (9.5,-1.3);
\coordinate (K) at (10.5,-2.3);
\coordinate (L) at (11.5,-1.3);
% Draw the zigzag pattern
\draw (A) -- (B) -- (C) -- (D) -- (E) -- (F) -- (G) -- (H) -- (I) -- (J) -- (K) -- (L);

% Draw purple rectangles above the circles
\foreach \x in {0.5, 1.0} {
    \fill[purple] (\x-0.0, 0.1) rectangle (\x+.5, 0.15);
}

\foreach \x in {2.5, 3.0} {
    \fill[purple] (\x-0.0, 0.1) rectangle (\x+.5, 0.15);
}

\foreach \x in {4.5, 5.0} {
    \fill[purple] (\x-0.0, 0.1) rectangle (\x+.5, 0.15);
}

\foreach \x in {6.5, 7} {
    \fill[purple] (\x-0.0, 0.1) rectangle (\x+.5, 0.15);
}

\foreach \x in {8.5, 9} {
    \fill[purple] (\x-0.0, 0.1) rectangle (\x+.5, 0.15);
}

\foreach \x in {10.5, 11} {
    \fill[purple] (\x-0.0, 0.1) rectangle (\x+.5, 0.15);
}


% Add small annotations and text values
\node at (0.5, -1.) {$-\epsilon_0$};
\node at (1.5, -1.) {$+\epsilon_{1}$};
\node at (2.5, -1.) {$-\epsilon_{2}$};
\node at (3.5, -1.) {$+\epsilon_{3}$};
\node at (4.5, -1.) {$-\epsilon_{4}$};
\node at (5.5, -1.) {$+\epsilon_{5}$};
\node at (6.5, -1.) {$-\epsilon_{6}$};
\node at (7.5, -1.) {$+\epsilon_{7}$};
\node at (8.5, -1.) {$-\epsilon_{8}$};
\node at (9.5, -1.) {$+\epsilon_{9}$};
\node at (10.5, -1.) {$-\epsilon_{10}$};
\node at (11.5, -1.) {$+\epsilon_{11}$};




\end{tikzpicture}
   \caption{A tree with alternating weight assignment.}
    \label{fig:tree}
\end{figure}
    
    
    Now, consider the assignment (except for $\tilde{w}_i$ and $\tilde{w}_{4j+i}$): $\forall k,j$ 
    \begin{gather*}
        \epsilon_{4k+1} = \epsilon_0 + .2 + 10^{-k}\\
        \epsilon_{4k+3} = \epsilon_0 + 10^{-k}\\
        \epsilon_{2j} = \epsilon_0 
    \end{gather*}
    As argued above this assignment ensures strict comparisons (inequalities). Now, let
    \begin{align*}
        \tilde{w}_i, \tilde{w}_{4j+i} = \epsilon_0 + .1 + 10^{-\frac{i}{4}}.
    \end{align*}
    This forces the weighting $\tilde{w}$ to satisfy the comparison $(v_{2j + i}, v_{4j + i}, v_{i})$. We need to argue that all the other comparisons are satisfied according to the scheme above (\eqnref{eq: mod1}-\eqref{eq: mod4}).

    Now, consider the following assignment for $\hat{w}$:

\begin{equation}
    \hat{w}_l =\begin{cases}
        \epsilon_0 + .1 + 10^{-k} & \textnormal{ if } l = 4k+1 \ge 4j+i \textnormal{ or } l = 4k+3 \le i\\
        \epsilon_0 + .2 + 10^{-k} & \textnormal{ if } l = 4k+1 \le i\\
        \epsilon_0 + 10^{-k} & \textnormal{ if } l = 4k+3 \ge 4j+i\\
        \epsilon_0& \textnormal{ if } l = 2k\\
    \end{cases}
\end{equation}

It is easy to check that the assigned weighting $\hat{w}$ satisfies all the inequalities in \eqnref{eq: mod1}-\eqref{eq: mod4}. But then the learner maintains both the weightings $\hat{w}$ and $\tilde{w}$ in the version space $\sf{VS}(\cT)$. Thus, if the comparison $(v_{2j + i}, v_{i}, v_{4j + i})$ is not provided by the teacher, the learner maintains tree metrics not related up to triplet constraints. This contradicts the validity of the oblivious teaching set $\cT$ for \algoref{alg: tree}. Hence, there doesn't exist an oblivious teaching set for the weighting $\boldsymbol{w}$ on the path tree $G_{p}$ of size $O(n^2)$.
\end{proof}
\end{comment}

In \propref{prop: treelower}, we showed a lower bound on the size of triplet comparisons for \algoref{alg: tree}. Now, we show that the worst-case oblivious teaching complexity is tight.


\begin{proposition}
        For the distance model $\fin$ on a set $\cX$ of size $N$, the teaching complexity for \algoref{alg: tree} is $N(N-2)$ for oblivious teaching up to triplet equivalence. 
\end{proposition}
\begin{proof}
    Let $d_{w^*} \in \fin$ be an arbitrary target distance function for oblivious teaching in \algoref{alg: tree}. We show a construction of a teaching set $\cT(d_{w^*}, \fin)$ of size $N(N-2)$.
    The set is composed of the following set of triplet comparisons: 
    \begin{enumerate}
        %\item $T_{\sf{pos}} = \curlybracket{(x,y,z) \,|\, e_{xy},e_{xz} \in E(G^*), d_{w^*}(x,y) \ge d_{w^*}(x,z)}$
        %\item $T_{\sf{comp}} = \bigcup_{v \in \cX} \bigcup_{P \in T_v} \bigcup_{u \in P} \bigcup_{(P' \in T_v\setminus P)} \curlybracket{(u,v,y), (u,v,z)\,|\, \not\exists\, x' \neq y,z \in P' \textnormal{ s.t. } d_{w^*}(v,y) < d_{w^*}(v,x') < d_{w^*}(v,z) } $ where $T_v$ denotes a tree rooted at the node $v$ and $P$ denotes a subtree of $T_v \setminus \{v\}$.
        \item  $T_{\sf{comp}} = \bigcup_{v \in \cX} \curlybracket{(v, x, y)_{\ell}\,|\,  d_{w^*}(v,x) \ge d_{w^*}(v,y), \not\exists z \in \cX,\, d_{w^*}(v,x) > d_{w^*}(v,z)\textnormal{ or }d_{w^*}(v,z) > d_{w^*}(v,y)}$
    \end{enumerate}
    The key idea is to fix a pivot (a center node) and teach all the distances wrt that node in the finite sample space $\cX$. This is exactly what the comparisons in $T_{\sf{comp}}$ specify. Since any triplet comparison is centered this set sufficiently teaches the distance function $d_{w^*}$ on $\cX$.
 %   We can interpret $T_{\sf{pos}}$ as the set of comparisons for each node and adjacent edges. We can show that this set is sufficient to retain the tree structure $G^*$ of the target metric with the exact node connectivity, i.e. for all distinct metrics $(G', w'), (G'', w'') \in \sf{\cXS}(T_{\sf{pos}})$ we have $G' \cong G''$. 
    
    %The proof follows from two straight-forward observations: i) if $u$ is on the path in $G^*$ connecting $x,y \in \cX$ and (wlog) $d_{w^*}(u,x) \ge d_{w^*}(u,y)$, then the triplets $(u,x,y)$ and $(y, x, u)$ sufficiently specify that, and ii) if $e_{uv} \in E(G^*)$ for vertices $u,v\in \cX$ then there is no other path from $u$ to $v$ containing non-trivial vertex $x \in \cX$. With a simple manipulation, it is clear that the first one holds trivially. The second observation follows from a standard property of a tree graph where two nodes have a unique path connecting them. The rest of the proof uses these observations to drive an induction argument on the number of vertices. For the case when $|\cX| = 3$, the proof statement holds trivially. Now, assume it holds up to $|\cX| = k-1$. For the case $|\cX| = k$, consider a leaf node $v$ and all the triplets in $T_{\sf{pos}}$ that compares other nodes to $v$, call it $\cT_v$. Excluding these triplets induction argument and observation i) implies $G^*\setminus \{v\}$ is correctly identified. Now, if $v$ is connected to $u$ in $G^*$ then all the triplets in $\cT_v$ is wrt $u$. So, there doesn't exist a tree in which $v$ is connected to some $u' \neq u \in \cX$ otherwise there exists $x$ such that (wlog) $(u,v,x) \in \cT_v$ and is on the path connecting $u$ and $v$. But this contradicts the assumption that $u$ is on the unique path connecting $x$ and $v$ as implied by the choice of $\cT_v$ as shown in the observation i). Thus, $T_{\sf{pos}}$ sufficiently retains the structure of the tree $G^*$. We can count the size of $T_{\sf{pos}}$ as follows:
    %\begin{align*}
    %    |T_{\sf{pos}}| = 2\sum_{v \in \cX} \binom{\deg^+(v)}{2} \le 2\binom{\sum_{v \in \cX} \deg^+(v)}{2} = 2\binom{2(n-1)}{2} = 2(n-1)(2n-3)
    %\end{align*}
    %where we have used the convexity of $\binom{t}{2}$ for $t > 0$.
    
    %$T_{\sf{comp}}$ is the set of triplet comparisons wrt each vertex $v \in \cX$; essentially capturing the ordering of the distances wrt $v$. In order words, order all the distances $\curlybracket{ \forall x\in \cX, w^*(v,x)}$ inducing a ranking $\sigma_v$ on $\cX \setminus \{v\}$, i.e. 
    %\begin{align*}
    %    w^*(v,\sigma_v(x)) \ge w^*(v,\sigma_v(y)) \Longleftrightarrow \sigma_v(x) \succeq \sigma_v(y)
    %\end{align*}
    %But teaching this ranking only requires $(n-2)$ triplet comparisons. Since there are $n$ vertices, we need at the most $(n-2)n$ triplet comparisons to specify the entire weighting $w^*$ on $G^*$ up to triplet constraints relation.

    %Adding the sizes of $T_{\sf{pos}}$ and $T_{\sf{comp}}$ gives the stated upper bound on the oblivious teaching set for \algoref{alg: tree}. 
\end{proof}

\subsection{Bounds for quadruple comparisons}

In this subsection, we study the oblivious teaching complexity with quadruple comparisons. First, we will show that with $\binom{N}{2} -1 $ quadruple comparisons, the teacher can specify any symmetric distance function $d_w$ up to ordinal equivalence.
\begin{figure}[h]
\centering
\begin{tikzpicture}
    % Parameters
    \def\radius{2.5cm} % Radius of the circle
    
    % Draw vertices in a circle and rename them
    \node[circle, draw, fill=white, inner sep=1.5pt] (v1) at (90:\radius) {$v_1$};
    \node[circle, draw, fill=white, inner sep=1.5pt] (v2) at (162:\radius) {$v_2$};
    \node[circle, draw, fill=white, inner sep=1.5pt] (v3) at (234:\radius) {$v_3$};
    \node[circle, draw, fill=white, inner sep=1.5pt] (v4) at (306:\radius) {$v_4$};
    \node[circle, draw, fill=white, inner sep=1.5pt] (vN_1) at (18:\radius) {$...$};
    \node[circle, draw, fill=white, inner sep=1.5pt] (vN_2) at (345:\radius) {$...$};
    \node[circle, draw, fill=white, inner sep=1.5pt] (v_N) at (52:\radius) {$v_N$};

    % Draw edges (adjust as needed)
    \draw (v1) -- (v2);
    \draw (v1) -- (v3);
    \draw (v2) -- (v3);
    \draw (v2) -- (v4);
    \draw (v3) -- (v4);

    \draw (v1) -- (v4);
    \draw (v_N) -- (v4);
    \draw (v_N) -- (v2);
    \draw (vN_2) -- (v3);
    \draw (vN_1) -- (v3);
    \draw (vN_2) -- (v1);
    \draw[dotted] (vN_2) -- (vN_1);
    \draw[dotted] (vN_2) -- (v4);
    \draw (v_N) -- (vN_1);
\end{tikzpicture}
\caption{A finite sample distance function}
        \label{fig:graph_metric}
\end{figure}
\begin{proposition} 
Consider a symmetric distance function $d_w$ on a finite sample $\cX$ of size $N$. With $\binom{N}{2} - 1$ many quadruple comparisons an oblivious learner can exactly learn $d_w$ up to ordinal equivalence.
\end{proposition}
\begin{proof}
    We can order the weight $w: \cX \times \cX \to \reals_{+}$ over any pair of vertices in $\cX$ as
    \begin{align}
        w_1 \le w_2 \le w_3 \le \ldots \le w_{\binom{N}{2}}   \label{eq: ordering}
    \end{align}
    (there are $\binom{N}{2}$ many pairs). Thus, if we provide the quadruple comparisons for each $i$ $w_i \le w_{i+1}$, i.e $(u_i, v_i , u_{i+1}, v_{i+1})_{\ell}$ where $(u_i, v_i)$ is the vertices pair for the distance $w_i$, similarly $(u_{i+1}, v_{i+1})$ for $w_{i+1}$. Thus, the set of quadruplets:
    \begin{align*}
        Q = \curlybracket{(u_i, v_i , u_{i+1}, v_{i+1})_{\ell} \,|\, i \in [n-1],\, d_w(u_i, v_i) = w_i}
    \end{align*}
    Note that $|Q| = \binom{N}{2} - 1$. $Q$ exactly specifies the ordering over all the distances, thus if the learner learns a distance function $d_{w'}$ over $\cX$ then the weights corresponding to the same pair of vertices in maintain the ordering of \eqnref{eq: ordering}. This implies that no matter the weights $w'$ assigned by the learner they are ordinally equivalent to $w$. Hence, the claim of the proposition is proven.
\end{proof}

\begin{figure}[h]
\centering
\begin{tikzpicture}
    % Parameters
    \def\radius{2.5cm} % Radius of the circle
    
    % Draw vertices in a circle and rename them
    \node[circle, draw, fill=white, inner sep=1.5pt] (v1) at (90:\radius) {};
    \node[circle, draw, fill=white, inner sep=1.5pt] (v2) at (162:\radius) {};
    \node[circle, draw, fill=white, inner sep=1.5pt] (v3) at (234:\radius) {$u_i$};
    \node[circle, draw, fill=white, inner sep=1.5pt] (v4) at (306:\radius) {$v_i$};
    \node[circle, draw, fill=white, inner sep=1.5pt] (vN_1) at (18:\radius) {$...$};
    \node[circle, draw, fill=white, inner sep=1.5pt] (vN_2) at (345:\radius) {$...$};
    \node[circle, draw, fill=white, inner sep=1.5pt] (v_N) at (52:\radius) {$v_N$};

    % Draw edges (adjust as needed)
    \draw[blue] (v1) -- (v2);
    \draw[blue] (v1) -- (v3);
    \draw[blue] (v2) -- (v3);
    \draw[blue] (v2) -- (v4);
    \draw[red] (v3) -- (v4) node[midway, below] {?};

    \draw[blue] (v1) -- (v4);
    \draw[blue] (v_N) -- (v4);
    \draw[blue] (v_N) -- (v2);
    \draw[blue] (vN_2) -- (v3);
    \draw[blue] (vN_1) -- (v3);
    \draw[blue] (vN_2) -- (v1);
    \draw[dotted] (vN_2) -- (vN_1);
    \draw[dotted] (vN_2) -- (v4);
    \draw[blue] (v_N) -- (vN_1);
\end{tikzpicture}
\caption{A finite sample distance function}
        \label{fig:counter_metric}
\end{figure}
Now, we will also provide a tight lower bound on the quadruple comparisons via showing a lower bound on oblivious teaching complexity for a finite-sample distance function, where we further assume that it satisfies the triangle inequality.

%\begin{lemma}
%    There exists a graph distance function $(G, d_G)$ such that the teacher needs to necessarily provide $\binom{n}{2} - 1$ quadruple comparisons in order for the learner to identify the target graph distance function. 
%\end{lemma}
%\begin{proof}
%    For the sake of contradiction, assume the contrary on the worst-case lower bound.   We will construct a counterexample based on a graph metric of size $n > 2$. We will show an assignment of weight $W_G$ which can't be taught in fewer than $\binom{n}{2} - 1$ many quadruple comparisons.

%    Lets fix a large positive scalar $\hat{w} > 0$ (sufficient choice would be clear from the discussion). Now, consider a weighting $W_G$ where for any $v_k, v_m \in G$, $d(v_k, v_m) = \hat{w} + \epsilon_{km}$ where $\epsilon_{km} > 0$. Assume that $\epsilon_{km}$ is within $(\epsilon, 2\epsilon)$ where $\epsilon > 0 $ is comparatively smaller than $\hat{w}$. 
%    Now, note that for any vertices $v_a, v_b,v_c \in \cX$, we obtain lower bounds and upper bounds on the distances within these vertices using the triangle inequality property of the metric $d$, i.e.
%    \begin{align*}
%        d(v_a, v_b) + d(v_b, v_c) \ge d(v_a, v_c) \ge |d(v_a, v_b) - d(v_b, v_c)|
%    \end{align*}
%    Thus, $d(v_a, v_c)$ can attain any value that satisfies these inequality and the distance function induced by these three vertices is a valid metric (other properties are trivially satisfied). We will use his observation to show that teacher needs to provide quadruple comparisons with respect to any pair $(v_i, v_j) \in G$.
%    Now, consider an ordering of weights in $W_G$ as shown in \eqnref{eq: ordering}. WLOG assume that for a weight $w_i$ corresponding to vertices $u_i$ and $v_i$ is not specified correctly in the ordering, i.e. for
%    \begin{align*}
%        w_{i-1} \le w_i \le w_{i+1}
%    \end{align*}
%    at least one of the quadruple comparisons---$(u_{i-1},v_{i-1},u_{i},v_{i})$ or $(u_{i},v_{i},u_{i+1},v_{i+1})$ are not provided by the teacher. Assume it is $(u_{i-1},v_{i-1},u_{i},v_{i})$. We can show that even if all the other quadruple comparisons are provided an oblivious learner can't find the exact ordering of $w_i$ wrt $w_{i-1}$. In fact, learner maintains a weight $w'_G$ in which $d'(u_i,v_i) < d'(u_{i-1},v_{i-1})$ but all the other quadruple comparisons are satisfied.

%    Fixing $u_i$ and $v_i$, consider all the possible triangle inequalities for any vertex $v \in \cX \setminus \{u_i,v_i\}$. Due to the choice of the target weighting $w_G$, $d(u_i,v_i)$ has to satisfy the following inequalities to be consistent with every distance corresponding to the vertex $v$:
%    \begin{align}
%        2 \epsilon \le d(u_i,v_i) \le 2\hat{w} + \epsilon \label{eq1}
%    \end{align}
%    But then,
%    \begin{align*}
%       \hat{w} + \epsilon \le  w_{i-1} = d(u_i,v_i) \le \hat{w} + 2\epsilon
%    \end{align*}
%    Since $\epsilon$ is comparatively smaller than $\hat{w}$, 
%    \begin{align}
%        2 \epsilon < w_{i-1} < 2\hat{w} + \epsilon \label{eq2}
%    \end{align}
%    Thus, using \eqnref{eq1} and \eqnref{eq2} the learner can pick/assign a weight for $d'(u_i,v_i)$ either in the interval $(2\epsilon, w_{i-1})$ or $(w_{i-1}, 2\hat{w} + \epsilon)$ and still satisfy all the quadruple comparisons given by the teacher for they are consistent with \eqnref{eq: ordering}. Hence, the teacher has to provide the comparison $(u_{i-1},v_{i-1},u_{i},v_{i})$. Similarly one can argue for the case when the teacher omits the comparison $(u_{i},v_{i},u_{i+1},v_{i+1})$. This completes the proof of the stated claim and thus we achieve a lower bound of $\binom{n}{2} -1$ on the number of quadruple comparisons for oblivious teaching.
%\end{proof}


\begin{lemma}
    For any symmetric distance function $d_w$ on the finite sample space $\cX$ the teacher needs to necessarily provide $\binom{N}{2} - 1$ quadruple comparisons for the learner to identify the target distance function up to ordinal equivalence. 
\end{lemma}

\begin{proof}
    For any symmetric distance function $d_w$, note that there are $\binom{N}{2}$ distances corresponding to any pair $\curlybracket{x,y}$ of vertices in $\cX$. Consider the ordering $\sigma_w$ of the weights:
    \begin{align*}
        w_1 \le w_2 \le \ldots \le w_{\binom{N}{2}}
    \end{align*}
    where each $w_i$ corresponds to a unique pair of vertices in $\cX$. A distance function $d_{w'}$ is ordinally equivalent to $d_w$ only if $w'$ assigns the same ordering $\sigma_{w'} = \sigma_w$ over the pairs of vertices. 

    WLOG assume that for a weight $w_i$ corresponding to vertices $u_i$ and $v_i$ is not specified correctly in the ordering, i.e. for
    \begin{align*}
        w_{i-1} \le w_i \le w_{i+1}
    \end{align*}
    at least one of the quadruple comparisons---$(u_{i-1},v_{i-1},u_{i},v_{i})_{\ell}$ or $(u_{i},v_{i},u_{i+1},v_{i+1})_{\ell}$ are not provided by the teacher. Assume it is $(u_{i-1},v_{i-1},u_{i},v_{i})_{\ell}$. We can show that even if all the other quadruple comparisons are provided an oblivious learner can't find the exact ordering of $w_i$ wrt $w_{i-1}$. In fact, learner maintains a weight $w'$ in which $d_{w'}(u_i,v_i) < d_{w'}(u_{i-1},v_{i-1})$ but all the other quadruple comparisons are satisfied. Since $w'_i$ is only required to be a positive scalar, the learner can maintain any ordering of $w'_i$ by assigning a value smaller than $w'_{i-1}$ inducing a different ordering than $\sigma_w$. A contradiction. 

    Hence, the teacher has to provide the comparison $(u_{i-1},v_{i-1},u_{i},v_{i})_{\ell}$. Similarly one can argue for the case when the teacher omits the comparison $(u_{i},v_{i},u_{i+1},v_{i+1})_{\ell}$ (assuming $w_i$ is not the largest weight in the ordering). This completes the proof of the stated claim and thus we achieve a lower bound of $\binom{N}{2} -1$ on the number of quadruple comparisons for oblivious teaching.

    
    %Fixing $u_i$ and $v_i$, consider all the possible triangle inequalities for any vertex $v \in \cX \setminus \{u_i,v_i\}$. Due to the choice of the target weighting $w_G$, $d(u_i,v_i)$ has to satisfy the following inequalities to be consistent with every distance corresponding to the vertex $v$:
    %\begin{align}
    %    2 \epsilon \le d(u_i,v_i) \le 2\hat{w} + \epsilon \label{eq1}
    %\end{align}
    %But then,
    %\begin{align*}
    %   \hat{w} + \epsilon \le  w_{i-1} = d(u_i,v_i) \le \hat{w} + 2\epsilon
    %\end{align*}
    %Since $\epsilon$ is comparatively smaller than $\hat{w}$, 
    %\begin{align}
    %    2 \epsilon < w_{i-1} < 2\hat{w} + \epsilon \label{eq2}
    %\end{align}
    %Thus, using \eqnref{eq1} and \eqnref{eq2} the learner can pick/assign a weight for $d'(u_i,v_i)$ either in the interval $(2\epsilon, w_{i-1})$ or $(w_{i-1}, 2\hat{w} + \epsilon)$ and still satisfy all the quadruple comparisons given by the teacher for they are consistent with \eqnref{eq: ordering}. Hence, the teacher has to provide the comparison $(u_{i-1},v_{i-1},u_{i},v_{i})$. Similarly one can argue for the case when the teacher omits the comparison $(u_{i},v_{i},u_{i+1},v_{i+1})$. This completes the proof of the stated claim and thus we achieve a lower bound of $\binom{n}{2} -1$ on the number of quadruple comparisons for oblivious teaching.
    
    
\end{proof}

\newpage
%\paragraph{Sampled triplet comparisons for a graph metric:} We can study the notion of oblivious teaching of a tree metric for a sampled triplet comparison from the set $\curlybracket{(u,x,y)\,|\, u,x,y \in G}$. We can study the teaching complexity for $\epsilon$-accuracy as in \defref{defn: lebsample} under a discrete distribution over the triplets. Denote the set of all possible triplet as $\cT_{\sf{all}}$ for a vertex set $\cX$ and metric model $\tree$.
%\begin{lemma}
%    Consider a distribution $\cD$ over $\cT_{\sf{all}}$. For any $\epsilon \in (0,1]$, the worst-case oblivious teaching complexity for \algoref{alg: tree} has a lower bound of $\Omega\paren{n^2}$ that achieves $\epsilon$-accuracy for teaching up to triplet constraints relation $\sim_{tree}$. Furthermore, the lower bound for $1$-accuracy is $\Omega\paren{n^3}$.
%\end{lemma}
%\begin{proof}
%    This is a straight-forward implication of the construction in \propref{prop: treelower}. 
%\end{proof}
%\begin{lemma}
%    Consider a distribution $\cD$ over $\cT_{\sf{all}}$. For any $\epsilon \in (0,1]$, the worst-case oblivious teaching complexity for \algoref{alg: tree} has a lower bound of $\Omega\paren{n^2}$ that achieves $\epsilon$-accuracy for teaching up to triplet constraints relation $\sim_{tree}$. Furthermore, the lower bound for $1$-accuracy is $\Omega\paren{n^3}$.
%\end{lemma}





%\input{teaching_graphmetric}
%\section{Smooth distance functions}
In this section, we study specific distance functions that satisfy the triangle inequality. One such family is the space of metrics on a given sample space $\cX$. We consider these specific distance functions and establish the framework for oblivious teaching, achieving an error bound of \(\epsilon > 0\) on all possible triplet comparisons. This is achieved under certain mild assumptions regarding the underlying distance function and the distribution over the input space \(\cX\), as outlined in the preceding discussion.

Consider general distance functions defined on a separable metric space \((\cX, d)\), where \(\cX \subset \mathbb{R}^k\) is a compact set, which means it is both closed and bounded within \(\mathbb{R}^k\). The distance function \(d: \cX \times \cX \to \mathbb{R}^{+}\) is assumed to be a \(C^2\)-map with respect to each argument (with symmetry of \(d\) implying this smoothness in both arguments). 

To effectively teach the distance function \(d\), the teacher employs a dual strategy that combines \texttt{local} and \texttt{global} distance approximations. The global approximation is achieved through a finite-sample distance function, while local approximations leverage Mahalanobis distance metrics, derived from the curvature information of a local neighborhood through its Hessian (see \figref{fig:bothfigures} for an illustration).

\paragraph{Local approximations:} Local approximation of the distance function $d$ is based on the taylor series expansion around the neighborhood at a given point. First, we state a high-dimensional version of Taylor's formula:
\begin{theorem}[Theorem 2.6.8~\citet{folland2002advanced}]\label{thm:taylorseries}
Suppose \( f : \mathbb{R}^k \rightarrow \mathbb{R} \) is of class \( C^{k+1} \) on an open convex set \( S \).
If \( a \in S \) and \( a + \mathbf{h} \in S \), then
\begin{equation}
f(a + \mathbf{h}) = \sum_{|\alpha| \leq k} \frac{\partial^\alpha f(a)}{\alpha!} \mathbf{h}^\alpha + R_{a,k}(\mathbf{h}),
\end{equation}
where
\begin{equation}
R_{a,k}(\mathbf{h}) = \sum_{|\alpha| = k+1} \frac{\partial^\alpha f(a + c\mathbf{h})}{\alpha!} \mathbf{h}^\alpha \quad \text{for some } c \in (0, 1). \label{eq: rem}
\end{equation}
%\begin{equation}
%R_{a,k}(\mathbf{h}) = k \sum_{|\alpha| = k} \frac{\mathbf{h}^\alpha}{\alpha!} \int_0^1 (1 - t)^{k-1} %\left[\partial^\alpha f(a + t\mathbf{h}) - \partial^\alpha f(a)\right] dt.
%\end{equation}
%If \( f \) is of class \( C^{k+1} \) on \( S \), we also have
%\begin{equation}
%R_{a,k}(\mathbf{h}) = (k + 1) \sum_{|\alpha| = k+1} \frac{\mathbf{h}^\alpha}{\alpha!} \int_0^1 (1 - t)^k \partial^\alpha f(a + t\mathbf{h}) dt,
%\end{equation}
%and
%\begin{equation}
%R_{a,k}(\mathbf{h}) = \sum_{|\alpha| = k+1} \frac{\partial^\alpha f(a + c\mathbf{h})}{\alpha!} \mathbf{h}^\alpha \quad \text{for some } c \in (0, 1). \label{eq: rem}
%\end{equation}
\end{theorem}

%\begin{theorem}[Taylor's expansion for Banach space, Page-%276,~\cite{Planitz_1980}]\label{thm:taylorseries}
%    Let $X, Y$ be two Banach spaces, $U$ open in $X$, and $x$ a point of $X$. Let $F: U \to Y$ be such that $\partial_x F$ is Frechet differentiable. Then for all $h \in U$, we have
%\[
%F(x + h) = F(x) + \frac{F'(x)h}{1!} + \frac{1}{2}\curlybracket{F''(x)h}h + o(||h||^2)
%\]

%\end{theorem}

\begin{figure}[t]
    \centering
    \begin{subfigure}[b]{0.45\textwidth}
        \centering
        \begin{tikzpicture}
            \begin{scope}[]

            
                 \fill[cyan!10] plot [smooth cycle, tension=0.8] coordinates {(1,3) (0,2) (-1,1) (0,-2) (2,-1) (3,1) (2,2)};
            
            % Draw the outer shape
            \draw[thick] plot [smooth cycle, tension=0.8] coordinates {(1,3) (0,2) (-1,1) (0,-2) (2,-1) (3,1) (2,2)};
        
            % Fill the i\sf{NN}er circle
            \fill[blue!30] (0.4, -1.0) circle (0.2);
            
            % Draw the i\sf{NN}er circle with edges
            \draw[thick] (0.4, -1.0) circle (0.2);
        
            % Draw and fill another circle without edges
            \fill[red!50, opacity=0.5] (0.4, -1.0) circle (0.3);
            
            % Define the coordinates of the vertices
            \coordinate (A) at (0, 0.5);
            \coordinate (B) at (1, 1.5);
            \coordinate (C) at (1.5, 0.5);
            \coordinate (D) at (2, 1.5);
            \coordinate (E) at (0.5, 2);
            \coordinate (F) at (1.5, -0.5);
            \coordinate (G) at (0.4, -1.0);
        
            % Draw the vertices
            \foreach \i in {A, B, C, D, E, F, G} {
                \fill (\i) circle (2pt);
            }
        
            % Draw the edges
            \foreach \i/\j in {A/B, A/C, A/D, A/E, A/F, B/C, B/D, B/E, B/F, B/G, C/D, C/E, C/F, C/G, D/E, D/F, D/G, E/F, E/G, F/G} {
                \draw (\i) -- (\j);
            }
            \draw (A) -- (G);
            \end{scope}
        \end{tikzpicture}
        \caption{Global approximation using a finite-sample distance function}
        \label{fig:figure1}
    \end{subfigure}
    %\hfill
    \begin{subfigure}[b]{0.45\textwidth}
        \centering
        \begin{tikzpicture}
            \begin{scope}[]
    % Define the coordinates of the outer shape
    %\path[name path=shape] plot [smooth cycle, tension=0.8] coordinates {(3,4) (2,3) (1,2) (2,-1) (4,0) (5,2) (4,3)};

    % Define the shading
    \begin{scope}
        \clip plot [smooth cycle, tension=0.8] coordinates {(3,4) (2,3) (1,2) (2,-1) (4,0) (5,2) (4,3)};
        \shade[inner color=cyan!50, outer color=cyan!20] (3,4) to[out=-45, in=135] (4,0) to[out=45, in=-135] (5,2) to[out=45, in=225] cycle;
    \end{scope}
    \fill[cyan!10,opacity=.6] plot [smooth cycle, tension=0.8] coordinates {(3,4) (2,3) (1,2) (2,-1) (4,0) (5,2) (4,3)};
    % Draw the outer shape outline
    \draw[thick] plot [smooth cycle, tension=0.8] coordinates {(3,4) (2,3) (1,2) (2,-1) (4,0) (5,2) (4,3)};

    % Define the coordinates of the parallelogram
    \coordinate (A) at (4,2);
    \coordinate (B) at (5,2);
    \coordinate (C) at (4,3);
    \coordinate (D) at (3,3);

    % Draw the parallelogram
    %\draw[thick] (A) -- (B) -- (C) -- (D) -- cycle;

    
    % Draw the tangent rectangle
    \fill[blue!30,opacity=.5] (A) -- (B) -- (C) -- (D) -- cycle;
    %\draw[thick] (3,2) rectangle (5,3);
    
    % Draw the point inside the rectangle
    \fill[black] (4, 2.5) circle (2pt);

    \fill[red!50, opacity=0.3, rotate=30] (4.7, 0.2) ellipse[x radius=0.2, y radius=0.3];

    \node at (3.25, 2.5) {$\sf{H}_x$};
    \node at (4, 2.3) {$x$};
\end{scope}
        \end{tikzpicture}
        \caption{Linear approximation at $x$: $\sf{H}_x$}
        \label{fig:figure2}
    \end{subfigure}
    \caption{Approximation of the metric $d$: Linear and Finite-sample approximations}
    \label{fig:bothfigures}
\end{figure}

We are interested in using \thmref{thm:taylorseries} to devise local linear approximation to $d$.
Consider the following:
\begin{align*}
    d(x,x') = d(x,x) + (x'-x)^{\top}\cdot\boldsymbol{\partial}_{x} d(x,\cdot) + \frac{1}{2!}(x'-x)^{\top} \sf{H}_x d(x,\cdot) (x'-x) + R_{x,3}(x'-x)
\end{align*}
Here, we assume that for all $\alpha \in [n]^3$, the partial derivative is bounded, i.e $|\partial^\alpha d(x,\cdot)| \le M$ for some constant $M > 0$. Thus, the remainder term $R_{a,k}(\mathbf{h})$ in \eqnref{eq: rem} can be bounded as 
\begin{align*}
 |R_{a,k}(\mathbf{h})| = \bigg\lvert\sum_{|\alpha| = k+1} \frac{\partial^\alpha f(a + c\mathbf{h})}{\alpha!} \mathbf{h}^\alpha\bigg\rvert \le \frac{M n^{\frac{k+1}{2}}}{(k+1)!} \cdot||\mathbf{h}||^{k+1}_2   
\end{align*}

Since $d$ is a metric it is straightforward to note that $\boldsymbol{\partial}_{x} d(x,\cdot) = 0$. Now, if $||x'-x|| \le \xi$, we have
\begin{subequations}\label{eq: 1}
\begin{align}
    d(x,x') &= \frac{1}{2!}(x'-x)^{\top} \sf{H}_x d(x,\cdot) (x'-x) + \Tilde{O}(||x'-x||_2^{2})\\
    \implies d(x,x') &\approx_{\xi} \frac{1}{2!}(x'-x)^{\top} \sf{H}_x d(x,\cdot) (x'-x)
\end{align}
\end{subequations}
So, for all distances centered at $x$ the inner product $(x'-x)^{\top} \sf{H}_x d(x,\cdot) (x'-x)$ is a tight linear approximation to $d$ with error at most $\xi^2$ within a ball of norm up to $\xi$. Note that since $d(x,\cdot)$ is a convex function at $x$ the hessian operator $\sf{H}_x$ is positive semi-definite. Since it is symmetric, we obtain a local Mahalanobis distance at $x$ in a $\xi$ neighborhood around $x$.

Now, we note a simple observation for this local approximation, i.e. 
\begin{align}
    (x'-x)^{\top} \sf{H}_x d(x,\cdot) (x'-x) \le \gamma_{\max}\cdot ||x-x'||_2^2 \label{eq: mahaapp}
\end{align}
Thus, if $\ell_2$ norm of two samples is small, so is the local approximation wrt the Mahalanobis distance function with matrix $\sf{H}_x$.

\paragraph{Global approximation:} Now, we will discuss the global approximation of the distance function $d$ using a finite sample distance function. Given that $\cX$ is a compact separable space, there is a finite $\epsilon$-cover of the space for any $\epsilon > 0$. Consider an $\epsilon$-cover $\cC_{\cX} \subset \cX$ (denoted $\cC$ for ease of notation) such that for all $x \in \cX$
\begin{align*}
    \exists c \in \cC, d(x,c) \le \epsilon %\cC_{\cX} =  \curlybracket{B_d(x,\epsilon) \,|\, x \in \cC}
\end{align*}
Note that, $\cX \subset {\bigcup}_{x \in \cC} B_d(x,\epsilon)$. We can similarly define an $\epsilon$ cover for $\ell_2$ distance as well. Later in the section we discuss an approximation method via $\ell_2$ cover of the space $\cX$.
%\sanjoy{This notation, which differentiates $\cC$ from $\cC$, is a bit cumbersome. Maybe we can just let $\cC$ be the centers of the covering balls (i.e. what $\cC$ is)?}

Note that the teacher can exactly specify the distances on $\cC$ wrt $d$ upto triplet comparisons as $d_G$, i.e.
\begin{align*}
    \forall x, x',x'' \in \cC,\,\, (x,x',x'')_{d_G} \,\,\textit{iff}\,\, (x,x',x'')_{d}
\end{align*}
For the study here, we assume the sample space is continuous. Thus, in order to analyze the performance of a taught metric $d'$ we define a notion of \tt{regret}, denoted as $\cR$, for a learner making mistakes in providing the correct relation on the comparisons on a given triplet $(x, x',x'')$ using $d'$ against the target metric $d$ as follows:
\begin{align*}
    \cR_{d'}(x, x',x'') = \begin{cases}
        0 & \textnormal{if } (x, x',x'')_{d'} = (x, x',x'')_{d}\\
        1 & \textnormal{o.w.}
    \end{cases}
\end{align*}
Here, we abuse the notation, alternately assume that $(x, x',x'')_{d'}$ means $d'$ assigns a comparison over $x,x',x"$, in addition to the usual understanding that $d'(x,x') \ge d'(x,x")$.

\textbf{Remark.} \tt{We assume that, in order to break ties among equal distances, the teacher could provide a triplet $(x,y,z)$ with the signal if there is equality or inequality, i.e. $d(x,y) = d(x,z)$ or $d(x,y) > d(x,z)$. We need this strong triplet comparison so that the learner can distinguish if the distances are equal or not, in the absence of which, it could choose a degenerate metric with all the distances equal.}

Now, we are interested in how good is $d'$ compared to $d$ which we measure in terms of $\epsilon$-error defined as follows: 
\begin{definition}
    For a given metric $d$, we say an approximation $d'$ achieves $\epsilon$-error against $d$ if 
    there exists a constant $C > 0$ such that
\begin{align*}
    \expctover{(x,y,z) \sim \cX^3}{\cR_{d'}(x,y,z)} \le C\cdot\epsilon
\end{align*}
    
\end{definition}
We assume the realizable case and thus $d$ has $0$ error on the sample space. 

Before we state our main result of the section, we consider a regularity condition of smoothness on the given distribution $\mu$ over the sample space $\cX$. 


\begin{definition}[Lipschitz smoothness of $\mu$]\label{def: lipschitz} We say a distribution $\mu$ over a metric space $(\cX,d)$ is $L$-Lipschitz smooth if there exists a parameter $\alpha > 0$ such that the following property holds on open normed balls:
\begin{align*}
    \forall x \in \cX,\,\, \forall r,r' > 0, \,\, |\mu(B_2(x,r)) - \mu(B_2(x,r'))| \le L \cdot |r - r'|^{\alpha}
\end{align*}
where $B_2(x,r) := \curlybracket{x' \in \cX\,|\, ||x - x'||_2 \le r}$.
\end{definition}
An implication of this definition is
\begin{align}
    &\liminf_{r_n \downarrow 0} |\mu(B_2(x,r)) - \mu(B_2(x,r_n))| \le \liminf_{r_n \downarrow 0} L \cdot |r - r_n| \nonumber\\
    \implies & \mu(B_2(x,r)) - \liminf_{r_n \downarrow 0} \mu(B_2(x,r_n)) \le L\cdot r^\alpha \nonumber\\
    \implies & \mu(B_2(x,r)) \le L\cdot r^\alpha \label{eq: rbound}
\end{align}

\subsection{Naive approximation of a smooth distance function via $\ell_2$ distance function}
%\sanjoy{I think there should be two results. The first should just say that any triplet $(x,y,z)$ is answered correctly if $d(x,y) - d(x,z) > \epsilon$. The second can be a distributional corollary. The theorem statement for the first should include all assumptions about the distance functions, e.g. that the third partial derivatives are uniformly bounded by $M$ and that all Hessians have maximum eigenvalue bounded by $\gamma$. The number of triplets needed can then be given explicitly in terms of an $\ell_2$ covering number of $\cX$, with radius in terms of $\epsilon, M, \gamma$ and whatever else.}
\paragraph{Distance Model:} In this section, we consider a model $\cM = \curlybracket{(\cC, d_{\cC}, \ell_2)}$ where $\cC \subset \cX$ of finite cardinality and $d_{\cC}: \cC \times \cC \to \reals_{+}$ is finite-sample distance function on $\cC$. Any distance function $d \in \cM$ has the following form:
\begin{equation*}
    d(x,y) = d(c(x), c(y)) 
\end{equation*}
where $c(x) = \argmin_{c \in \cC} ||c - x||_2$.

\begin{theorem}\label{thm: smoothl2} Consider a compact separable metric space $\cX \subset \reals^k$. Consider a distance function $d: \cX \times \cX \to \reals_{\ge 0}$ that is a $C^2$-map in the second argument. Assume that $d$ satisfies the following properties:
\begin{enumerate}
    \item it satisfies the triangle inequality\footnote{Since we needn't have symmetric $d$ note the directional version of this triangle inequality which is consistent with the parallelogram law of vector addition.} over $\cX$, i.e. for all $x,y,z \in \cX$, $d(x,y) + d(y,z) \ge d(x,z)$,
    \item it has uniformly bounded third partial derivative by a constant $M > 0$, and
    \item at any sample $x \in \cX$, $\lambda_{\max}(\sf{H}_x d(x,\cdot))$ is bounded by a constant $\gamma > 0$. 
\end{enumerate}

%Consider a distribution $\mu$ over $\cX$ which is $L$-lipschitz. %for some constant $\frac{1}{\lambda_{\textsf{diam}}}\ge L > 0$ where $\lambda_{\textsf{diam}} = \textsf{diam}(\cX)$. 
Then, for any error parameter $\epsilon > 0$, there exists a finite-sample distance function $d'$ (see \algoref{alg: smoothdistl2}) which can be taught with $\cN(\cX, \epsilon(\epsilon, M, \gamma), \ell_2)^2$ triplet comparisons such that for any $x,y,z \in \cX$, if $d(x,y) > d(x,z) + \epsilon$ then
\begin{align*}
    d'(x,y) > d'(x,z)
\end{align*}
where the covering accuracy $\epsilon(\epsilon, M, \gamma)$ denotes that the covering depends on $\epsilon$, and constants $M$ and $\gamma$.

%there exists a distance function $d'$ which can be taught with $\cN(\cX, \epsilon)^2$ triplet comparisons such that 

%achieving an $\epsilon$-error, i.e. there exists a constant $C > 0$ such that
%\begin{align*}
%    \expctover{(x,y,z) \sim \cX^3}{\cR_{d'}(x,y,z)} \le C\cdot\epsilon
%\end{align*}
    
\end{theorem}
\begin{algorithm}[t]
\caption{Teaching a smooth distance function with $\ell_2$ covering of the space}
\label{alg: smoothdistl2}
\textbf{Given}: Input space $\cX \sim \cC_{\cX}$, error threshold: $\epsilon$.\\
\vspace{1mm}
\textit{In batch setting:}\vspace{1mm}
\begin{enumerate}
    \item teacher provides triplet comparisons $\cT$ corresponding to finite-sample distance function $(\cC, d)$ for an $\epsilon$-cover $\cC_2(\cX,\epsilon,\ell_2)$.
\end{enumerate}
For a given sample $x$, learner computes:
\begin{itemize}
    \item $\cC_x := \curlybracket{c \in \cC: ||x - c ||_2 \le \epsilon}$ 
    \item $c_x := \argmin_{c \in \cC_x} \sqrt{(x-c)^{\top}(x-c)}$
\end{itemize}
\vspace{1mm}
To answer query on $(x,y,z)$, learner checks:
\begin{itemize}
    \item \textbf{If} $c_y = c_z$, then say $``d(x,y) \approx d(x,z)"$
    \item \textbf{Else}: answer according to the triplet $(c_x,c_y,c_z)$.
\end{itemize}
\end{algorithm}

\begin{proof}
    The proof requires a global approximation using an $\epsilon$-cover wrt $\ell_2$ distance.
    
   Consider an $\epsilon$-cover $\cC(\cX,\epsilon,\ell_2)$, denoted as $\cC_2(\cX)$ (in short) with centers $\cC$. The teacher provides triplet comparisons to teach a finite-sample distance function $(\cC, d)$, with the underlying distance function denoted as $d_G$ such that
        \begin{align*}
            x,y,z \in \cC,\quad (x,y,z)_{d_G} \textit{ iff } (x,y,z)_{d}
        \end{align*}
        For oblivious teaching,  $\cC_2(\cX)^2$ (square of covering number) many triplet comparisons are sufficient to teach $d_G$.
      %  \item[ii)] \tt{Local approximation:} For each center $x \in \cC$, consider local linear approximations $\sf{H}_x d$ with error bounded by $\xi$ where $\xi \le \frac{\epsilon}{6}$ (see \eqnref{eq: 1}). For each hessian operator $\sf{H}_x$, teacher provides triplets based on its eigendecompositon as discussed in previous sections. Each linear transformations requires at most $n^2$ (where $n$ is the ambient space dimension) many triplets. We use the notation $\hat{\sf{H}}_x$ for a taught matrix against the correct hessian matrix $\sf{H}_x$ for any $x \in \cX$. 

    Now, we discuss how a learner assigns comparisons for $d'$ for a given triplet $(x_1,x_2,x_3) \in \cX^3$. Denote by $\sf{NN}_2: \cX \to \cX$, a nearest neighbor map based on $\ell_2$ distances of any sample $x \in \cX$ (from the centers $\cC$). Thus, 
    \begin{align*}
        \sf{NN}_2(x) = \argmin_{x' \in \cC} \sqrt{(x-x')^{\top}(x-x')}
    \end{align*}
    We break the ties arbitrarily. This could potentially create some error in assigning the right comparison, which would at the worst only add a constant multiple of $\epsilon$ on the error. Denote by $\hat{x}$, the nearest neighbor of $x \in \cX$, i.e.  $\hat{x} \in \sf{NN}_2(x)$.
    $d'$ assigns the comparison on the triplet $\hat{x}_1, \hat{x}_2,\hat{x}_3$ as follows:
    %Now, the learner constructs a metric $d'$ to assign relation on any arbitrary triplet $(x_1,x_2,x_3) \in \cX^3$ as follows: %(this could be extended to quadruples similarly):
    %\begin{align}
    %    d'(x_1,x_2) \ge d'(x_1,x_3) = (\sf{NN}(x_1), \sf{NN}(x_2), \sf{NN}(x_3))_{d_G} \label{eq: assign}
    %\end{align}
    %Here, $\sf{NN}$ denotes the nearest neighbor based on local linear approximations. In the \eqnref{eq: assign} above, we assume that $\sf{NN}(x_2)$ is closer to $\sf{NN}(x_1)$ than $\sf{NN}(x_3)$, otherwise the relation could be written accordingly.
    %There are two possibilities for the 
    \begin{align}\label{eq: approxmetric}
        (x_1,x_2,x_3)_{d'} = \begin{cases}
            (\hat{x}_1,\hat{x}_2,\hat{x}_3)_{d_G} & \textit{ if } 
            \hat{x}_1 \neq \hat{x}_2 \textit{ or } \hat{x}_1 \neq \hat{x}_3\\
            d'(x_1,x_2) = d'(x_1,x_3)  & \textit{ o.w. }   %\hat{x}_1 = \hat{x}_2 = \hat{x}_3       \textit{ and } ({x}_1 -  {x}_2) \hat{\sf{H}}_{x_1}({x}_1 -  {x}_2)^{\top} \ge ({x}_1 -  {x}_3) \hat{\sf{H}}_{x_1}({x}_1 -  {x}_3)^{\top}
            %(\hat{x}_1,\Hat{x}_3,\hat{x}_2)_{d_{\sf{local}}}  & \textit{ if }   \hat{x}_1 = \hat{x}_2 = \hat{x}_3       \textit{ and } ({x}_1 -  {x}_3) \hat{\sf{H}}_{x_1}({x}_1 -  {x}_3)^{\top}  \ge   ({x}_1 -  {x}_2) \hat{\sf{H}}_{x_1}({x}_1 -  {x}_2)^{\top}
        \end{cases}
    \end{align}
    %where $d_{\sf{local}}$ is written as an abuse of notation for $d_{\hat{\sf{H}}_x}$ for any $x \in \cX$. 
    
     So, the learner answers: if ``$x_2$ is closer to $x_1$ than $x_3$'' by first finding the nearest neighbors of these points and then checking the corresponding relation to these in the distance function $(\cC, d_G)$. %and local Mahalanobis metrics induced by $\{\hat{\sf{H}}_x: x \in \cC\}$.
     Thus, the distance function $d'$ is defined as follows: for all $x,y \in \cX$
     \begin{equation}
         d'(x,y) := d_G(\hat{x}, \hat{y})
     \end{equation}
    Now, we will show the approximation guarantee for $d'$.
    
    First, we define the following maximum over the largest eigenvalues of the hessian operators $\curlybracket{\sf{H}_x}_{x \in \cC}$
\begin{align*}
    \hat{\gamma} := \max_{x \in \cC} \gamma_{\max}(\sf{H}_x)
\end{align*}
    
    Note, that the specific choice of the cover $\cC_2(\cX)$, and local approximations in \eqnref{eq: 1} and \eqnref{eq: mahaapp}
    %of the approximation constant $\xi$ for local approximation in ii), 
    ensures that if any points $y,z$ are far apart wrt a given $x$ then the distances are correctly captured, i.e. for any $x,y \in \cX$:
    \begin{align}
        d(x,y) - 2\hat{\gamma}\epsilon - 2\xi \le d(\hat{x}, \hat{y}) \le  d(x,y) + 2\hat{\gamma}\epsilon + 2\xi \label{eq: apdist}
    \end{align}
    We can show this as follows. Using triangle inequality we have
    \begin{align*}
        d(x,y) \le d(\hat{x},y) + d(\hat{x},x)\\
        d(\hat{x},y) \le d(\hat{x},\hat{y}) + d(\hat{y},y)
    \end{align*}
    Summing these 
    \begin{align}
        d(x,y) \le d(\hat{x},\hat{y}) + d(\hat{x},x) + d(\hat{y},y) \le d(\hat{x},\hat{y}) + 2\hat{\gamma}\epsilon + 2\xi
    \end{align}
    Similarly, we get the lower bounds
    \begin{align*}
       d(\hat{x},\hat{y}) - 2\hat{\gamma}\epsilon - 2\xi \le d(x,y)
    \end{align*}
    In the bounds, we have used the error in local approximation, i.e. for any $x,y$ such that $||x-y|| \le \xi$, $d(x,y) \approx (x-y)^{\top}\sf{H}_x(x-y) \pm \xi$.
    
    But since $\xi$ scales at the rate $\frac{Mn^{1.5}}{6}\epsilon^3$, we have 
    \begin{align}
        d(x,y) - 2\paren{\hat{\gamma} + \frac{Mn^{1.5}}{6}\epsilon^2}\epsilon \le d(\hat{x}, \hat{y}) \le  d(x,y) + 2\paren{\hat{\gamma} + \frac{Mn^{1.5}}{6}\epsilon^2}\epsilon
    \end{align}
    This in turn gives
    \begin{align*}
      \forall x,y,z \in \cX,\quad  d(x,y) > d(x,z) + 4\paren{\hat{\gamma} + \frac{Mn^{1.5}}{6}\epsilon^2}\epsilon \implies d(\hat{x},\hat{y}) > d(\hat{x},\hat{z}) \implies d'(\hat{x},\hat{y}) > d'(\hat{x},\hat{z}).
    \end{align*}
    Since $\hat{\gamma} \le \gamma$ this proofs the stated claim of the theorem.
\end{proof}
\begin{corollary}
    Consider the setting of \thmref{thm: smoothl2}. In addition, assume a distribution $\mu$ over $\cX$ which is $L$-lipschitz (see \defref{def: lipschitz}) for some constant $\alpha > 0$. %for some constant $\frac{1}{\lambda_{\textsf{diam}}}\ge L > 0$ where $\lambda_{\textsf{diam}} = \textsf{diam}(\cX)$. 
Then, the finite sample distance function $d'$ taught in \algoref{alg: smoothdistl2} achieves an $\epsilon$-error, i.e. there exists a constant $C > 0$ such that
\begin{align*}
    \expctover{(x,y,z) \sim \cX^3}{\cR_{d'}(x,y,z)} \le C\cdot\epsilon
\end{align*}
\end{corollary}
\begin{proof}
     Denote $\gamma'(\epsilon) := \paren{\hat{\gamma} + \frac{Mn^{1.5}}{6}\epsilon}$. So, $d'$ could make error in assigning a correct comparison on a triplet $x,y,z$ but only if $|d(x,y) - d(x,z)| \le 4\gamma'\epsilon$. In order to show the bound on the expected error of $\cR_{d'}$, we consider this case.
    %First, we note that for a given triplet $(x,y,z) \in \cX^3$, if  then \eqnref{eq: approxmetric} gives the correct relation as the comparison reduces to the graph metric. 
    
    
    Now, we will discuss the approximation of the error. First, note that by definition of conditional expectation
\begin{align*}
    \expctover{(x,y,z) \sim \cX^3}{\cR_{d'}(x,y,z)} = \expctover{x}{\expctover{(y,z\,|\,x)}{\cR_{d'}(x,y,z)}} 
\end{align*}
%\le \sum_{x \in \cC:  B(x,\epsilon)} \expctover{(y,z\,|\,x)}{\cR_{d'}(x,y,z)}\cdot \cP(B(x,\epsilon))

Now, we will analyze $\expctover{(y,z\,|\,x)}{\cR_{d'}(x,y,z)}$ for the cases where $|d(x,y) - d(x,z)| \le 4\gamma'\epsilon$. But this condition leads to two possibilities: either $y,z \in B(x,4\gamma'\epsilon)$, or $y,z \in B(x,r + 4\gamma'\epsilon)\setminus B(x,r)$ for some $r > 0$.


Thus, we can write:
\begin{align*}
    &\expctover{(y,z\,|\,x)}{\cR_{d'}(x,y,z)}\\
    &\le \underbrace{\expctover{(y,z\,|\,x)}{\mathds{1}[y,z \in B(x,4\gamma'\epsilon)]\cdot \cR_{d'}(x,y,z)}}_{(\star)} + \underbrace{\expctover{(y,z\,|\,x)}{\mathds{1}[\exists r>0, y,z \in B(x,r + 4\gamma'\epsilon)\setminus B(x,r)]\cdot\cR_{d'}(x,y,z)}}_{(\star \star)}
\end{align*}
But by definition, the first term can be bounded as follows:
\begin{align*}
    (\star) \le \cP_{(y,z|x)} (y,z \in B(x,4\gamma'\epsilon)) = \cP_{(y|x,z)} (y \in B(x,4\gamma'\epsilon)) \cP_{(z|x)} (z \in B(x,4\gamma'\epsilon)) \le L^2\cdot (4\gamma'\epsilon)^{2\alpha},
\end{align*}
where we have used the observation from \eqnref{eq: rbound} and independence of the variables $y$ and $z$. We can bound the second term as
\begin{center}
\begin{align*}
    (\star \star) &\le   \int_{4\gamma'\epsilon}^{\lambda_{\textsf{diam}}} \cP_{(y,z\,|\, x)}( y,z \in B(x,r+4\gamma'\epsilon))\setminus B(x,r)) d r\\
    %&=  \int_{2\epsilon}^{\lambda_{\textsf{diam}}} \cP_{(y\,|\, x,z)}( y \in B(x,r+2\epsilon))\setminus B(x,r))\cdot \cP_{(z\,|\, x)}( z \in B(x,r+2\epsilon))\setminus B(x,r)) dr\\
    & \le \int_{4\gamma'\epsilon}^{\lambda_{\textsf{diam}}} L^2\cdot (r + 4\gamma'\epsilon  - r )^{2\alpha} dr\\
    &= (\lambda_{\textsf{diam}} - 4\gamma'\epsilon)\cdot L^2(4\gamma'\epsilon)^{2\alpha}
    %&\le 4(1 + \lambda_{\textsf{diam}}L^2)\cdot \epsilon^2\\
    %&\le 5L\cdot \epsilon^2
\end{align*}
\end{center}
Adding the bounds for $(\star)$ and $(\star\star)$
\begin{align}
    (\star) + (\star \star) \le L^2\cdot (4\gamma'\epsilon)^{2\alpha} + (\lambda_{\textsf{diam}} - 4\gamma'\epsilon)\cdot L^2(4\gamma'\epsilon)^{2\alpha} = (\lambda_{\textsf{diam}} - 4\gamma'\epsilon + 1)\cdot L^2(4\gamma'\epsilon)^{2\alpha} %< (\lambda_{\textsf{diam}} + 1)L^2(4\gamma'\epsilon)^{2\alpha}
\end{align}
Thus,
\begin{align*}
    \expctover{(x,y,z) \sim \cX^3}{\cR_{d'}(x,y,z)} = \expctover{x}{\expctover{(y,z\,|\,x)}{\cR_{d'}(x,y,z)}} &\le \expctover{x}{ (\lambda_{\textsf{diam}} - 4\gamma'\epsilon + 1)\cdot L^2(4\gamma'\epsilon)^{2\alpha}}\\
    &=  (\lambda_{\textsf{diam}} - 4\gamma'\epsilon + 1)\cdot L^2(4\gamma'\epsilon)^{2\alpha}
\end{align*}
Since $\gamma' =  \paren{\hat{\gamma} + \frac{Mn^{1.5}}{6}\epsilon}$, we assume that $\epsilon$ must scale with $\min\{n^{-.75}, \hat{\gamma}\}$. This completes the proof of the corollary.

%Note, it takes $|\cC_2(\cX)|^2$ triplet comparisons to teach $d'$. This completes the proof of the theorem.
\end{proof}
%\usetikzlibrary{shapes.geometric, calc}

\subsection{(1 + $\epsilon$) approximation via Mahalanobis distance functions}

\begin{algorithm}[t]
\caption{Teaching a smooth distance function via local Mahalanobis distance functions}
\label{alg: smoothdistl2}
\textbf{Given}: Input space $\cX \sim \cC_{\cX}$, approximation threshold: $\epsilon$.\\
\vspace{1mm}
\textit{In batch setting:}\vspace{1mm}
\begin{enumerate}
    \item teacher provides triplet comparisons $\cT$ corresponding to finite distance function $(\cC_\Delta, d)$ for an $\paren{\frac{\gamma_m \epsilon^4}{16\gamma_M^2 (1 + \epsilon)}}$-cover $\cC_\Delta\paren{\cX,\paren{\frac{\gamma_m \epsilon^4}{16\gamma_M^2 (1 + \epsilon)}},\ell_2}$.
    %\item teacher provides triplet comparisons $\cT$ corresponding to finite distance function $(\cC_3, d)$ for an $\epsilon^3$-cover $\cC_3(\cX,\epsilon^3,d)$.
    \item teacher provides triplet comparisons $\{\cT_c\}_{c \in \cC_\Delta}$ to teach local Mahalanobis distance functions at the centers $\cC_\Delta$.
    %\item teacher provides triplet comparisons $\{\cT_x\}_{x \in \cC_3}$ to teach local Mahalanobis distance functions at the centers $\cC_3$
\end{enumerate}
For a given sample $x$, learner computes:
\begin{itemize}
    \item $\cC_2(x) := \curlybracket{c \in \cC_2: ||x - c ||^2 \le \epsilon^2}$ 
    \item $c_2(x) := \argmin_{c \in \cC_2(x)} \sqrt{(x-c)^{\top}\sf{H}_c(x-c)}$
\end{itemize}
\vspace{1mm}
To answer query on $(x,y,z)$, learner computes 
\begin{gather*}
  \ell_{xz} := (z - x)^{\top}\sf{H}_{c_2(x)}(z - x),\,\, \ell_{xy} := (y - x)^{\top}\sf{H}_{c_2(x)}(y - x)  \\
  \ell_{c_2(x)z} := (z - c_2(x))^{\top}\sf{H}_{c_2(x)}(z - c_2(x)),\,\, \ell_{c_2(x)y} := (y - c_2(x))^{\top}\sf{H}_{c_2(x)}(y - c_2(x))    
\end{gather*}
and checks:
\begin{itemize}
    \item \textbf{If} $\ell_{c_2(x)y}$ and $\ell_{c_2(x)z}$ $\ge 2\epsilon^3$: \\
        \qquad answer according to the triplet $(c_2(x),c_2(y),c_2(z))$
    \item \textbf{Else If} $\ell_{c_2(x)y}$ and $\ell_{c_2(x)z}$ $ \le  4\epsilon^3$:\\
        \qquad answer according to $\sgn{\ell_{xy} - \ell_{xz}}$
    \item \textbf{Else If} $\ell_{c_2(x)y} > 4\epsilon^3$ and $\ell_{c_2(x)z} < 2\epsilon^3$:\\
    \qquad answer $``d(x,y) > d(x,z)"$
    \item \textbf{Else If} $\ell_{c_2(x)z} > 4\epsilon^3$ and $\ell_{c_2(x)y} < 2\epsilon^3$:\\ 
    \qquad answer $``d(x,z) > d(x,y)"$
    
\end{itemize}
\end{algorithm}
\paragraph{Distance model:} In this section, we consider distance model defined wrt a threshold $\tau$ of the form $\cM(\tau) = \{((\cC, d_{\cC}), (\cU_c, d_c)_{c \in \cU_c})\}$ where $\cC \subset \cX$ is a finite subset of centers with a distance function $d_{\cC}: \cC \times \cC \to \reals_{+}$ and for each center $c \in \cC$ and a neighborhood $\cU_c \subset \cX$, there is a local distance function $d_c: \cU_c \times \cU_c \to \reals_{+}$. Any distance function $d \in \cM$ has the following form:
\begin{align*}
    d(x,y) = \begin{cases}
        d_{\cC}(c(x), c(y)) & \textnormal{ if } d_{c(x)}(c(x), y) \ge \tau\\
        d_{c(x)}(x,y)
    \end{cases}
\end{align*}
where $c(x) := c$ such that $x \in \cU_c$. In case, $x \in \cU_c \cap \cU_{c'}$, we break the ties with the lower $\ell_2$ norm.

The aim of this section is to approximate arbitrary smooth distacne functions with specific family of distance functions $\cM(\tau)$.

\begin{assumption}[hessian continuity] Consider a distance function $d: \cX \times \cX \to \reals_{+}$ which is $C^2$ in second argument. We say $d$ is hessian continuous if for all $x, x' \in \cX$ there exists a positive scalar $L > 0$ such that 
\begin{align*}
    ||\sf{H}_x - \sf{H}_{x'}||_{F} \le L\cdot ||x - x'||_2
\end{align*}
where $\sf{H}_x$ and $\sf{H}_{x'}$ are hessian matrices of $d(x,\cdot)$ and $d(x',\cdot)$ at $y = x$ and $y = x'$ respectively.
\end{assumption}

An immediate consequence of this result is that if $x$ and $x'$ are close to each other then the distances wrt sample $x'$ in the neighborhood of $x$ can be computed using the hessian information at $x$. To see this, assume that $d(x, \cdot)$ has uniformly bounded third partial derivatives as in \thmref{thm: smoothl2}. Now, we note that 
\begin{align}
    d(x', y') \approx_{\xi} (x' - y')^{\top}\sf{H}_{x'}(x' - y') \label{eq: 11}\\
    d(x, x - x' + y') \approx_{\xi} (x' - y')^{\top}\sf{H}_{x}(x' - y') \label{eq: 12}
\end{align}
But we can bound the difference assuming (wlog) \eqref{eq: 11} $\ge$ \eqref{eq: 12},
\begin{align*}
    (x' - y')^{\top}\sf{H}_{x'}(x' - y') - (x' - y')^{\top}\sf{H}_{x}(x' - y') &\le ||\sf{H}_x - \sf{H}_{x'}||_{F}\cdot ||x' -y'||^2_2%\\
    %& \le L\cdot ||x - x'||_2\cdot ||x' -y'||^2_2
\end{align*}
But using the hessian continuity of $d$ we get 
\begin{align*}
    (x' - y')^{\top}\sf{H}_{x'}(x' - y') - (x' - y')^{\top}\sf{H}_{x}(x' - y') \le L\cdot ||x - x'||_2\cdot ||x' -y'||^2_2 
\end{align*}
Now, if $x, x', y'$ are in close proximity the difference in Mahalanobis distances is small. Thus, to approximate $d(x', y')$, we can use curvature information at $x$ itself, in the form of the hessian matrix $\sf{H}_x$.

\begin{theorem}\label{thm: smoothl2} Consider a compact separable metric space $\cX \subset \reals^k$. Consider a distance function $d: \cX \times \cX \to \reals_{\ge 0}$ that is a $C^2$-map in the second argument. Assume that $d$ satisfies the following properties:
\begin{enumerate}
    \item it satisfies the triangle inequality over $\cX$, i.e. for all $x,y,z \in \cX$, $d(x,y) + d(y,z) \ge d(x,z)$,
    \item it has uniformly bounded third partial derivative by a constant $M > 0$, 
    \item at any sample $x \in \cX$, the smallest and largest eigenvalues $\lambda_{\min}(\sf{H}_x d(x,\cdot))$ and $\lambda_{\max}(\sf{H}_x d(x,\cdot))$ are lower bounded and upper bounded by the constants $\gamma_m, \gamma_M > 0$ respectively, 
    \item it is hessian continuous, and
    \item for any $\delta > 0$, $\liminf_{x,y \in \cX  } \{d(x,y) : ||x-y||_2 > \delta\} > 0$. 
\end{enumerate}

%Consider a distribution $\mu$ over $\cX$ which is $L$-lipschitz. %for some constant $\frac{1}{\lambda_{\textsf{diam}}}\ge L > 0$ where $\lambda_{\textsf{diam}} = \textsf{diam}(\cX)$. 
Consider the distance function $d'$ that combines a finite sample distance function and local Mahalanobis distance functions as shown in \algoref{alg: smoothdistl2}. Then, for any error parameter $\epsilon > 0$ such that
\begin{align*}
    \epsilon \le \min\curlybracket{\paren{\frac{\gamma_M \Delta}{2 \gamma_m}}^{\frac{1}{3}}, \left(\frac{\gamma_m^2}{16\paren{\frac{M k^{\frac{3}{2}}}{6}\paren{\frac{2}{\sqrt{\gamma_m}} + 1} + \frac{L}{\gamma_m}}}\right)^2}
\end{align*}
for any $x,y,z \in \cX$, if $d(x,y) > (1 + \epsilon)\cdot d(x,z)$ then
\begin{align*}
    d'(x,y) > d'(x,z).
\end{align*}
Furthermore, the teaching complexity of $d'$ is $\cN(\cX, \epsilon', d) (\cN(\cX, \epsilon', d) + k^2)$ where the covering accuracy $\epsilon' = \paren{\frac{\gamma_m \epsilon^4}{16\gamma_M^2 (1 + \epsilon)}}$.
\end{theorem}


\begin{proof}
 %   Consider covers of the following form:
  %  \begin{itemize}
   %     \item $\epsilon^2$-cover $\cC(\cX, \epsilon^2, d)$ denoted as $\cC_2(\cX)$ with centers $\cC_2$,
   %     %\item $\epsilon^3$-cover $\cC(\cX, \epsilon^3, d)$, denoted as $\cC_3(\cX)$ with centers $\cC_3$.
   % \end{itemize}
   % We need to adjust some constants for the cover which can be fixed appropriately from the later discussion. 
    
    Now, consider the following constant:
    \begin{align*}
        \Delta := \min_{x \in \cX} \curlybracket{d(x,x') : x' \in \cX \setminus B_d(x,\epsilon^2)}
    \end{align*}
    Using the condition 5. on $d$, we know that $\Delta > 0$. %We pick $\Delta_s := \min\{\frac{\Delta}{3}, \frac{\epsilon^2}{3\gamma_m}\}$.

    
    First, note that for any fixed sample $x \in \cX$, $d(x, y)$ is strongly convex for $y$ close to $x$:
    \begin{align*}
        d(x, x+ h) \ge \frac{1}{2}h^{\top}\sf{H}_x h - \frac{M k^{\frac{3}{2}}}{6} \cdot||h||^{3}_2
    \end{align*}
    Now, if $||h||_2$ is small enough, i.e. smaller than $\frac{4\gamma_m}{M k^{\frac{3}{2}}}$, then we have 
    \begin{align}
        d(x, x+ h) \ge \frac{\gamma_m}{2}||h||^2_2 - \frac{\gamma_m}{4}||h||^2_2 = \frac{\gamma_m}{4}||h||^2_2 \label{eq: strongconvex}
    \end{align}
    Thus, $d(x,\cdot)$ is $\frac{\gamma_m}{4}$-strongly convex at $x$. Since $\gamma_m$ is the smallest eigenvalue for the hessian at any arbitrary sample $x \in \cX$, this holds over $\cX$. On the other hand, the function $d(x,\cdot)$ is smooth, i.e.
    \begin{align}
        d(x, x+ h) \le \gamma_M ||h||^2_2 \label{eq: smoothconvex}
    \end{align}
    Now, we pick the threshold for deciding if the learner should use local Mahalanbis distance function or finite sample distance function on the centers of a cover as shown in \algoref{alg: smoothdistl2}. We pick this as follows:
    \begin{align*}
        \hat{\epsilon} := \min\curlybracket{\frac{\gamma_M\Delta}{2\gamma_m}, \epsilon^3}%\frac{4}{\gamma_m} \Delta_s
    \end{align*}
%    Consider an $\hat{\epsilon}$-cover of the space with the centers $\cC$.    
    In the proof, we use the threshold $\hat{\epsilon}$, assuming that $\epsilon$ is small enough of the order
    \begin{align*}
        \epsilon \le \min\curlybracket{\paren{\frac{\gamma_M \Delta}{2 \gamma_m}}^{\frac{1}{3}}, \paren{\frac{\gamma_m}{16\paren{\frac{M k^{\frac{3}{2}}}{6}\paren{\frac{2}{\sqrt{\gamma_m}} + 1} + \frac{L}{\gamma_m}}}}^2}
    \end{align*}
    Now, consider a $\paren{\frac{\gamma_m \hat{\epsilon}\epsilon}{16\gamma_M^2 (1 + \epsilon)}}$-cover with centers $\cC_\Delta$ wrt to $\ell_2$ distance. The teacher provides triplet comparisons to teach a finite sample distance function $(\cC_\Delta,d)$, denoted as $d_f$ (learner finds  $(\cC_\Delta,d)$ up to triplet equivalence as $d_f$). Simultaneously, the teacher also provides triplet comparisons to teach local Mahalanobis distance function $\curlybracket{\sf{H}_c}_{c \in \cC_\Delta}$.

    With this, we can define the approximated distance function $d'$ as follows: for all $x,y \in \cX$
    \begin{equation}
        d'(x,y) = \begin{cases}
            d_{f}(c_2(x), c_2(x)) & \textnormal{ if } (y - c_2(x))^{\top}\sf{H}_{c_2(x)}(y - c_2(x)) \ge 4\epsilon^3\\
             (y - x)^{\top}\sf{H}_{c_2(x)}(y - x) & \textnormal{ o.w. }
        \end{cases}
    \end{equation}

    In \algoref{alg: smoothdistl2}, learner first checks if $(y - c_2(x))^{\top}\sf{H}_{c_2(x)}(y - c_2(x))$ and $(z - c_2(x))^{\top}\sf{H}_{c_2(x)}(z - c_2(x))$ are greater than $2 \hat{\epsilon}$. In this case, the learner finds the centers $(c_2(x), c_2(y),c_2(z))$ for the samples $(x,y,z)$ and then uses finite sample distance function $(\cC_\Delta,d)$ to answer the query.
    
    We show that, under this condition, if $d(x,y) \ge (1 + \epsilon) d(x,z)$, then $d(c_2(x),c_2(y)) > d(c_2(x),c_2(z))$.

    In \eqnref{eq: strongconvex}, we showed tha if $||h||_2$ is smaller than $\frac{4\gamma_m}{Mk^{\frac{3}{2}}}$ then, $d(x,\cdot)$ is $\frac{\gamma_m}{4}$-strongly convex at $x$ in the second argument. Using this observation, we note that if $c \in \cC_2$ and $y \in \cX$ then on the boundary  $(c - y)^{\top}\sf{H}_c(c - y) = 2 \hat{\epsilon}$, we have
    \begin{align*}
        \frac{2\hat{\epsilon}}{\gamma_m}  \ge ||c - y||_2^2 \ge \frac{2\hat{\epsilon}}{\gamma_M}
    \end{align*}
    Using strong convexity and smoothness as shown in \eqnref{eq: strongconvex}-\eqnref{eq: smoothconvex},
    \begin{equation*}
         \frac{\gamma_M\cdot \hat{\epsilon}}{\gamma_m} \ge d(c,y) \ge \frac{\gamma_m\cdot \hat{\epsilon}}{2\gamma_M}
    \end{equation*}
    Now note that $\frac{\gamma_m\cdot \hat{\epsilon}}{2\gamma_M} < \Delta$, which implies if $(c - y)^{\top}\sf{H}_c(c - y) > 2\hat{\epsilon}$ then $d(c,y) > \frac{\gamma_m\cdot \hat{\epsilon}}{2\gamma_M}$.

    Now, if $||c_2(x) - x||^2_2 \le \paren{\frac{\gamma_m \hat{\epsilon}\epsilon}{16\gamma_M^2 (1 + \epsilon)}}$, using strong convexity of $d(c_2(x),\cdot)$
    \begin{align*}
        d(c_2(x), x) \le \gamma_M\cdot || c_2 (x) - x ||^2_2 \le \paren{\frac{\gamma_m \hat{\epsilon}\epsilon}{16\gamma_M (1 + \epsilon)}}
        \end{align*}
    Now, we will show how reducing $x,y,z$ to the closest centers in $\cC_{\Delta}$ leads to correct classification:
    \allowdisplaybreaks
    \begin{align}
        d(c_2(x), c_2(y)) &\ge d(c_2(x), y) - d(c_2(y), y)\\
                          & \ge d(x,y) - d(x, c_2(x)) - d(c_2(y), y)\\
                          & \ge (1+ \epsilon)\cdot d(x,z) - 2\paren{\frac{\gamma_m \hat{\epsilon}}{16\gamma_M}}\\
                          & \ge (1 + \epsilon)\cdot (d(c_2(x),c_2(z)) - d(x, c_2(x)) - d(c_2(z), z)) -  \paren{\frac{\gamma_m \hat{\epsilon}\epsilon}{8\gamma_M (1 + \epsilon)}}\\
                          & \ge d(c_2(x),c_2(z)) + \epsilon\cdot d(c_2(x),c_2(z)) - (1 + \epsilon) \paren{\frac{\gamma_m \hat{\epsilon}\epsilon}{8\gamma_M (1 + \epsilon)}} - \paren{\frac{\gamma_m \hat{\epsilon}}{8\gamma_M (1 + \epsilon)}}\\
                          & \ge d(c_2(x),c_2(z)) + \epsilon \cdot (d(c_2(x),z) - d(z, c_2(z))) - (1 + \epsilon) \paren{\frac{\gamma_m \hat{\epsilon}\epsilon}{8\gamma_M (1 + \epsilon)}} - \paren{\frac{\gamma_m \hat{\epsilon}}{8\gamma_M (1 + \epsilon)}}\\
                          & \ge d(c_2(x),c_2(z)) + \epsilon\cdot d(c_2(x),z) - \epsilon \paren{\frac{\gamma_m \hat{\epsilon}\epsilon}{16\gamma_M (1 + \epsilon)}}- (1 + \epsilon) \paren{\frac{\gamma_m \hat{\epsilon}\epsilon}{8\gamma_M (1 + \epsilon)}} - \paren{\frac{\gamma_m \hat{\epsilon}\epsilon}{8\gamma_M (1 + \epsilon)}}\\
                          & \ge d(c_2(x),c_2(z)) + \epsilon\cdot d(c_2(x),z) - (1 + \epsilon)\paren{\frac{\gamma_m \hat{\epsilon}\epsilon}{4\gamma_M (1 + \epsilon)}}\\
                          & \ge d(c_2(x),c_2(z)) + \epsilon \cdot \paren{\frac{\gamma_m \hat{\epsilon}}{2\gamma_M}} - \paren{\frac{\gamma_m \hat{\epsilon}\epsilon}{4\gamma_M}}\\
                          & > d(c_2(x),c_2(z))
                          %& \ge d(c_2(x),c_2(z)) + 4\epsilon \hat{\epsilon} - \epsilon \frac{M k^{\frac{3}{2}}}{6} d(c_2(x),z) - \epsilon \hat{\epsilon} - 2(1 + \epsilon) \hat{\epsilon} - \paren{\frac{\gamma_m \hat{\epsilon}}{4\gamma_M}}
    \end{align}
    %Note that for any point $x$ and a center $c \in \cC$ such that $||x - c||_2^2 = \hat{\epsilon}$, we have 
    %\begin{align*}
    %    d(x,c) \ge (x-c)^{\top}\sf{H}_c(x-c) - \frac{M k^{\frac{3}{2}}}{6} \cdot||x-c||^{3}_2 \ge 3 \Delta_s \ge \Delta 
    %\end{align*}
    
    %Now, we will show the proof of the algorithm case by case.

    In the second case, the learner checks if $(y - c_2(x))^{\top}\sf{H}_{c_2(x)}(y - c_2(x))$ and $(z - c_2(x))^{\top}\sf{H}_{c_2(x)}(z - c_2(x))$ are smaller than $4 \hat{\epsilon}$. Under this condition, learner decides the triplet comparison on $(x,y,z)$ by checking  $\sgn{(y - x)^{\top}\sf{H}_{c_2(x)}(y - x) - (z - x)^{\top}\sf{H}_{c_2(x)}(z - x)}$. We show that it is postive if $d(x,y) \ge (1 + \epsilon)d(x,z)$.

    First note that, 
    \begin{subequations}\label{eq: bound1}
    \begin{align}
        (y - x)^{\top}\sf{H}_{c_2(x)}(y - x) &\ge (y - x)^{\top}\sf{H}_{x}(y - x) - L\cdot ||x - c_2(x)||_2\cdot ||x - y||_2^2\\
        &\ge d(x,y) - \frac{M k^{\frac{3}{2}}}{6}||x - y||_2^3 - L\cdot ||x - c_2(x)||_2\cdot ||x - y||_2^2\\
        & = d(x,y) -\paren{\frac{M k^{\frac{3}{2}}}{6}||x - y||_2 + L\cdot ||x - c_2(x)||_2} ||x - y||_2^2
    \end{align}
    \end{subequations}
    We will first show that $\paren{\frac{M k^{\frac{3}{2}}}{6}||x - y||_2 + L\cdot ||x - c_2(x)||_2} $ is smaller than $\frac{\epsilon\gamma_m}{4}$. But this holds trivially for we have
    \begin{align*}
        ||c_2(x) - y||^2_2 \le \frac{4\hat{\epsilon}}{\gamma_m},\quad ||x - c_2(x)||^2_2 \le \hat{\epsilon}
    \end{align*}
    This implies that 
    \begin{align*}
        ||x - y||_2 \le \paren{\frac{2}{\sqrt{\gamma_m}} + 1}\sqrt{\hat{\epsilon}}
    \end{align*}
    
    Since $\epsilon$ is smaller than $\frac{1}{16}\paren{\frac{M k^{\frac{3}{2}}}{6}\paren{\frac{2}{\sqrt{\gamma_m}} + 1} + \frac{L}{\gamma_m}}^{-1}$ we have 
    \begin{align*}
        \paren{\frac{M k^{\frac{3}{2}}}{6}||x - y||_2 + L\cdot ||x - c_2(x)||_2} \le \paren{\frac{M k^{\frac{3}{2}}}{6}\paren{\frac{2}{\sqrt{\gamma_m}} + 1} + L}\cdot \sqrt{\hat{\epsilon}} \le \frac{\epsilon\gamma_m}{16}
    \end{align*}
    But this implies that 
    \begin{align*}
        (x-y)^{\top}\sf{H}_{x}(x-y) -\paren{\frac{M k^{\frac{3}{2}}}{6}||x - y||_2 + L\cdot ||x - c_2(x)||_2} ||x - y||_2^2 \ge \paren{1 - \frac{\epsilon}{16}}\cdot (x-y)^{\top}\sf{H}_{x}(x-y)\\
        (x-y)^{\top}\sf{H}_{c_2(x)}(x-y) -\paren{\frac{M k^{\frac{3}{2}}}{6}||x - y||_2 + L\cdot ||x - c_2(x)||_2} ||x - y||_2^2 \ge \paren{1 - \frac{\epsilon}{16}}\cdot (x-y)^{\top}\sf{H}_{c_2(x)}(x-y)
    \end{align*}
    Now note that 
    \begin{align*}
        |(x-y)^{\top}\sf{H}_{x}(x-y) - d(x,y)| \le \frac{M k^{\frac{3}{2}}}{6}||x - y||_2^3
    \end{align*}
    Since $||x - y||_2 \le \paren{\frac{2}{\sqrt{\gamma_m}} + 1}\sqrt{\hat{\epsilon}}$, we note that  
    \begin{align*}
        d(x,y) -\paren{\frac{M k^{\frac{3}{2}}}{6}||x - y||_2 + L\cdot ||x - c_2(x)||_2} ||x - y||_2^2 \ge \paren{1 - \frac{\epsilon}{8}}\cdot d(x,y)
    \end{align*}
    Since $y$ is chosen agnostic of $x$, this holds for the sample $z$ as well.
    
    Now, returning to \eqnref{eq: bound1},
    \begin{align}
        (y - x)^{\top}\sf{H}_{c_2(x)}(y - x) &\ge d(x,y) -\paren{\frac{M k^{\frac{3}{2}}}{6}||x - y||_2 + L\cdot ||x - c_2(x)||_2} ||x - y||_2^2 \\
        &\ge \paren{1 - \frac{\epsilon}{8}}\cdot d(x,y)\\
        &\ge \paren{1 - \frac{\epsilon}{8}} (1 + \epsilon)\cdot d(x,z)\\
        & \ge \paren{1 - \frac{\epsilon}{8}} (1 + \epsilon) \paren{(z - x)^{\top}\sf{H}_{c_2(x)}(z - x) - \frac{M k^{\frac{3}{2}}}{6}||x - z||_2^3 - L\cdot ||x - c_2(x)||_2\cdot ||x - z||_2^2}\\
        &\ge \paren{1 - \frac{\epsilon}{8}} (1 + \epsilon)\paren{1 - \frac{\epsilon}{16}} (z - x)^{\top}\sf{H}_{c_2(x)}(z - x)
    \end{align}
    Since $\paren{1 - \frac{\epsilon}{8}} (1 + \epsilon)\paren{1 - \frac{\epsilon}{16}} > 1$, we have 
    \begin{align*}
        (y - x)^{\top}\sf{H}_{c_2(x)}(y - x) > (z - x)^{\top}\sf{H}_{c_2(x)}(z - x)
    \end{align*}
    In the last case, assuming the first two fail, the learner checks if one of $(y - c_2(x))^{\top}\sf{H}_{c_2(x)}(y - c_2(x))$ and $(z - c_2(x))^{\top}\sf{H}_{c_2(x)}(z - c_2(x))$ is smaller $2 \hat{\epsilon}$ and the other greater than $4 \hat{\epsilon}$. It is straightforward to note that reducing the samples $x,y,z$ to the centers $(c_2(x),c_2(y),c_2(z))$ should correctly give the comparison as follows from the analysis for the second case.
\end{proof}
    

    \newpage
   
%\nocite*
\bibliographystyle{plainnat}
\bibliography{ref}
\appendix
\onecolumn
% \section{General Results on Space of Symmetric Matrices}

%  \begin{enumerate}
%         \item[\textcolor{blue}{S(1)}] if $\pphi \in \mathcal{O}_{\pphi^*}$ such that $span\inner{\col{\pphi}} \subset span\inner{V_{\bracket{r}}}$ then $\pphi \in span \inner{\cT}$.
%         \item[\textcolor{blue}{S(2)}] there exists vectors $U_{\bracket{d-r}} \subset \nul{\pphi^*}$ (of size $d - r $) such that $span \inner{U_{\bracket{d-r}} } = \nul{\pphi^*}$ and 
%         for any vector $v \in U_{\bracket{d-r}}$, $vv^{\top} \in span \inner{\cT}$.
%     \end{enumerate}
\section{Table of Contents}
Here, we provide the table of contents for the appendix of the supplementary.

\begin{itemize}
\item[-] \appref{app: notations} provides a comprehensive table of additional notations used throughout the paper and supplementary material.
\item[-] \appref{app:atom} contains the proof for \lemref{lem: ortho}, establishing conditions for recovering orthogonal representations.\vspace{2mm}

\item[-] \appref{app: worstcase} completes the proof of \propref{prop: worstcase}, establishing a worst-case lower bound on feedback complexity in the constructive setting.\vspace{2mm}

\item[-] \appref{app: constub} presents the proof for the upper bound in \thmref{thm: constructgeneral} for low-rank feature matrices.\vspace{2mm}

\item[-] \appref{app: constlb} establishes the proof for the lower bound in \thmref{thm: constructgeneral} for low-rank feature matrices.\vspace{2mm}

\item[-] \appref{app: samplegeneral} details the proof of \thmref{thm: samplegeneral} which asserts tight bounds on feedback complexity for general sampled activations.\vspace{2mm}

\item[-] \appref{app: samplesparse} demonstrates the proof of \thmref{thm: samplingsparse} establishing an upper bound on the feedback complexity for sparse sampled activations.\vspace{2mm}

\item[-] \appref{app: additional} provides supplementary experimental results validating our theoretical findings.\vspace{2mm}
\end{itemize}
\newpage

\section{Notations}\label{app: notations}
Here we provide the glossary of notations followed in the supplementary material.
\iffalse
For a given matrix $\pphi \in \reals^{p\times p}$ and indices $i,j \in \bracket{p}$ $\pphi_{ij}$ denotes the entry of $\pphi$ at $ith$ row and $jth$ column. Matrices are denoted as $\pphi,\pphi',\Sigma$. Unless stated otherwise, a target matrix (for teaching a mahalanobis metric) is denoted as $\pphi^*$. We denote that null set of a matrix $\pphi$, i.e. $\curlybracket{x \in \reals^p\,|\, \pphi x = 0}$, as $\nul{\pphi}$; whereas $
\kernel{\pphi}$ for the kernel of the matrix. For a matrix, we denote its eigenvalues as $\gamma,\lambda, \gamma_i$ or $\lambda_i$ and its eigenvectors (orthogonal vectors) as $\mu_i ,u_i$ or $v_i$. We define the element-wise product of two matrices $\pphi,\pphi'$ via an inner product $\inner{\pphi', \pphi} := \sum_{i,j} \pphi'_{ij}\pphi_{ij}$. 
We denote vectors in $\reals^p$ as $x,y$ or $z$.
Note, for any $x \in \reals^p$ $\inner{\pphi, xx^{\top}} = x^{\top}\pphi x$. For ease of notation, we also write the inner product as $\pphi \idot \pphi'$.

We denote the space of symmetric matrices in $\reals^{p \times p}$ as $\symm$, and similarly the space of symmetric, positive semi-definite matrices as $\symmp$.
Since the space of matrices on $\reals^{p\times p}$ is isomorphic to the Euclidean vector space $\reals^{p^2}$ for any matrix $\pphi$ we also call it a vector identified as an element of $\reals^{p^2}$. We say two matrices $\pphi,\pphi'$ are \tt{orthogonal}, denoted as $\pphi \bot \pphi'$, if $\pphi \idot \pphi' = 0$. For a set of vectors/matrices $\cC \subset \reals^{p\times p}$, the subspace induced by the elements in $\cC$ is denoted as $span \inner{\cC} := \curlybracket{a \pphi + b \pphi' \,|\, \pphi,\pphi' \in \cC,\, a,b \in \reals}$. Similarly, the set of columns of a matrix $\pphi$ is denoted as $\col{\pphi}$, and its span as $span \inner{\col{\pphi}}$.

\begin{table}[h]
\centering
\begin{tabular}{|c|c|c|}
\hline
\textbf{Symbol} & \textbf{Description}\\
\hline
$\pphi, \Sigma$ & Feature matrix\\
$\cV \subset \reals^p$ & Activation/Representation space\\
$\alpha, \beta, x,y,z$ & Activations\\
$\cX \subset \reals^p$ & Ground truth sample space\\
$d$ & Dimension of ground-truth sample space\\
$p$ & Dimension of representation space\\
$r$ & Rank of a feature matrix\\
$\curlybracket{v_1, v_2, \ldots, v_r}$ & A set of orthonormal vectors, typically eigenvectors of $\pphi^*$\\
$V_{\bracket{r}}$ & The set $\curlybracket{v_1, v_2, \ldots, v_r}$ \\
$V_{\bracket{p - r}}$ & The set $\curlybracket{v_{r+1}, \ldots, v_p}$, forming an orthogonal extension to $V_{\bracket{r}}$ \\
$V_{\bracket{p}}$ & The complete orthonormal basis $\curlybracket{v_1, v_2, \ldots, v_p}$ \\
$\mathcal{O}_{\pphi^*}$ & Orthogonal complement of $\pphi^*$ in $\symm$\\
$\dd$ & Dictionary matrix\\
$\cD, \cD_{\sf{sparse}}$ & Distributions over activations\\
$\sf{VS}(\cF, \maha)$ & Version space of $\maha$ wrt feedback set $\cF$\\ 
$\symm$ & Space of symmetric matrices\\
$\symmp$ & Space of PSD, symmetric matrices\\
\hline
\end{tabular}
\end{table}
\fi

% \begin{table}[h]
% \centering
% \begin{tabular}{|c|c|c|}
% \hline
% \textbf{Symbol} & \textbf{Description}\\
% \hline
% \parbox{3cm}{$\alpha, \beta, x,y,z$} & Activations\\
% $\col{\pphi}$ & Set of columns of matrix $\pphi$\vspace{1mm}\\
% $\cD, \cD_{\sf{sparse}}$ & Distributions over activations\vspace{1mm}\\
% $d$ & Dimension of ground-truth sample space\vspace{1mm}\\
% $\dd$ & Dictionary matrix\vspace{1mm}\\
% $\gamma,\lambda, \gamma_i, \lambda_i$ & Eigenvalues of a matrix\vspace{1mm}\\
% $\inner{\pphi', \pphi}$ & Element-wise product (inner product) of matrices\vspace{1mm}\\
% $\kernel{\pphi}$ & Kernel of matrix $\pphi$\vspace{1mm}\\
% $\mu_i ,u_i, v_i$ & Eigenvectors (orthogonal vectors)\vspace{1mm}\\
% $\nul{\pphi}$ & Null set of matrix $\pphi$\vspace{1mm}\\
% $\mathcal{O}_{\pphi^*}$ & Orthogonal complement of $\pphi^*$ in $\symm$\vspace{1mm}\\
% $p$ & Dimension of representation space\vspace{1mm}\\
% $\pphi, \Sigma$ & Feature matrix\vspace{1mm}\\
% $\pphi_{ij}$ & Entry at $i$th row and $j$th column of $\pphi$\vspace{1mm}\\
% $\pphi^*$ & Target feature matrix\vspace{1mm}\\
% $r$ & Rank of a feature matrix\vspace{1mm}\\
% $\symm$ & Space of symmetric matrices\vspace{1mm}\\
% $\symmp$ & Space of PSD, symmetric matrices\vspace{1mm}\\
% $\sf{VS}(\cF, \maha)$ & Version space of $\maha$ wrt feedback set $\cF$\vspace{1mm}\\
% $V_{\bracket{r}}$ & The set $\curlybracket{v_1, v_2, \ldots, v_r}$\vspace{1mm}\\
% $V_{\bracket{p - r}}$ & The set $\curlybracket{v_{r+1}, \ldots, v_p}$\vspace{1mm}\\
% $V_{\bracket{p}}$ & Complete orthonormal basis $\curlybracket{v_1, v_2, \ldots, v_p}$\vspace{1mm}\\
% $\cV \subset \reals^p$ & Activation/Representation space\vspace{1mm}\\
% $\cX \subset \reals^d$ & Ground truth sample space\vspace{1mm}\\
% \hline
% \end{tabular}
% \end{table}

\begin{table}[h]
\centering
\begin{tabular}{|c|c|c|}
\hline
\parbox{3cm}{\textbf{Symbol}} & \parbox{7cm}{\textbf{Description}}\\
\hline
\parbox{3cm}{$\alpha, \beta, x,y,z$} & \parbox{7cm}{Activations}\\
\parbox{3cm}{$\col{\pphi}$} & \parbox{7cm}{Set of columns of matrix $\pphi$}\\
\parbox{3cm}{$\cD, \cD_{\sf{sparse}}$} & \parbox{7cm}{Distributions over activations}\\
\parbox{3cm}{$d$} & \parbox{7cm}{Dimension of ground-truth sample space}\\
\parbox{3cm}{$\dd$} & \parbox{7cm}{Dictionary matrix}\\
\parbox{3cm}{$\gamma,\lambda, \gamma_i, \lambda_i$} & \parbox{7cm}{Eigenvalues of a matrix}\\
\parbox{3cm}{$\inner{\pphi', \pphi}$} & \parbox{7cm}{Element-wise product (inner product) of matrices}\\
\parbox{3cm}{$\kernel{\pphi}$} & \parbox{7cm}{Kernel of matrix $\pphi$}\\
\parbox{3cm}{$\mu_i ,u_i, v_i$} & \parbox{7cm}{Eigenvectors (orthogonal vectors)}\\
\parbox{3cm}{$\nul{\pphi}$} & \parbox{7cm}{Null set of matrix $\pphi$}\\
\parbox{3cm}{$\mathcal{O}_{\pphi^*}$} & \parbox{7cm}{Orthogonal complement of $\pphi^*$ in $\symm$}\\
\parbox{3cm}{$p$} & \parbox{7cm}{Dimension of representation space}\\
\parbox{3cm}{$\pphi, \Sigma$} & \parbox{7cm}{Feature matrix}\\
\parbox{3cm}{$\pphi_{ij}$} & \parbox{7cm}{Entry at $i$th row and $j$th column of $\pphi$}\\
\parbox{3cm}{$\pphi^*$} & \parbox{7cm}{Target feature matrix}\\
\parbox{3cm}{$r$} & \parbox{7cm}{Rank of a feature matrix}\\
\parbox{3cm}{$\symm$} & \parbox{7cm}{Space of symmetric matrices}\\
\parbox{3cm}{$\symmp$} & \parbox{7cm}{Space of PSD, symmetric matrices}\\
\parbox{3cm}{$\sf{VS}(\cF, \maha)$} & \parbox{7cm}{Version space of $\maha$ wrt feedback set $\cF$}\\
\parbox{3cm}{$V_{\bracket{r}}$} & \parbox{7cm}{The set $\curlybracket{v_1, v_2, \ldots, v_r}$}\\
\parbox{3cm}{$V_{\bracket{p - r}}$} & \parbox{7cm}{The set $\curlybracket{v_{r+1}, \ldots, v_p}$}\\
\parbox{3cm}{$V_{\bracket{p}}$} & \parbox{7cm}{Complete orthonormal basis $\curlybracket{v_1, v_2, \ldots, v_p}$}\\
\parbox{3cm}{$\cV \subset \reals^p$} & \parbox{7cm}{Activation/Representation space}\\
\parbox{3cm}{$\cX \subset \reals^d$} & \parbox{7cm}{Ground truth sample space}\\
\hline
\end{tabular}
\end{table}

\newpage

% \textbf{Vectors and Sets:}
% \begin{itemize}
%     \item $\curlybracket{v_1, v_2, \ldots, v_r}$: A set of orthonormal vectors, typically eigenvectors of $\pphi^*$.
%     \item $V_{\bracket{r}}$: The set $\curlybracket{v_1, v_2, \ldots, v_r}$.
%     \item $V_{\bracket{d - r}}$: The set $\curlybracket{v_{r+1}, \ldots, v_p}$, forming an orthogonal extension to $V_{\bracket{r}}$.
%     \item $V_{\bracket{d}}$: The complete orthonormal basis $\curlybracket{v_1, v_2, \ldots, v_p}$.
%     \item $\cX$: The domain from which feedback pairs $(y, z)$ are drawn.
%     \item $\cX^2$: The Cartesian product of $\cX$ with itself, representing pairs $(y, z) \in \cX \times \cX$.
% \end{itemize}

% \textbf{Feedback Sets:}
% \begin{itemize}
%     \item $\cF_{\sf{null}}$: A partial feedback set consisting of pairs $\curlybracket{(0, v_i)}_{i = r+1}^p$ used to teach the null space of $\pphi^*$.
%     \item $\mathcal{F}(\cX, \textsf{VS}(\maha, \cF_{\sf{null}}), \pphi^*)$: The feedback set formulated to operate within the version space $\textsf{VS}(\maha, \cF_{\sf{null}})$ for the target matrix $\pphi^*$.
% \end{itemize}

% \textbf{Version Space and Metrics:}
% \begin{itemize}
%     \item $\maha$: A metric space relevant to the version space formulation (specific definition to be provided based on context).
%     \item $\textsf{VS}(\maha, \cF_{\sf{null}})$: The version space restricted by the feedback set $\cF_{\sf{null}}$ within the metric space $\maha$.
% \end{itemize}

% \textbf{Relations and Operators:}
% \begin{itemize}
%     \item $\idot$: Inner product between two matrices, typically defined as $\pphi \idot \psi = \text{trace}(\pphi^{\top} \psi)$.
%     \item $\sim_{R_l}$: An equivalence relation denoting linear scaling, i.e., $\pphi \sim_{R_l} \pphi'$ if $\pphi' = c \pphi$ for some scalar $c > 0$.
%     \item $\succeq 0$: Denotes that a matrix is positive semidefinite (PSD), i.e., $\pphi \succeq 0$ means $\pphi$ is PSD.
% \end{itemize}

% \textbf{Indices and Sets:}
% \begin{itemize}
%     \item $\bracket{n}$: The set $\{1, 2, \ldots, n\}$.
%     \item $\curlybracket{v_{l_k}, v_{m_k}}$: Pairs of vectors used in linear combinations within feedback sets or basis constructions.
% \end{itemize}
%  \textbf{Other Symbols:}
% \begin{itemize}
%     \item $\mathds{1}[\cdot]$: The indicator function, where $\mathds{1}[P] = 1$ if predicate $P$ is true, and $0$ otherwise.
%     \item $\mathcal{B}$: A basis set of rank-1 symmetric matrices constructed from the eigenvectors $\curlybracket{v_i}$.
%     \item $\mathcal{O}_{\cB}$: A set of matrices derived from $\mathcal{B}$, adjusted by scaling with a vector $y$.
%     \item $\lambda_{ij}$: Scaling factors defined as $\lambda_{ii} = \frac{v_i \pphi^* v_i^{\top}}{y \pphi^* y^{\top}}$ and $\lambda_{ij} = \frac{(v_i + v_j) \pphi^* (v_i + v_j)^{\top}}{y \pphi^* y^{\top}}$ for $i \neq j$.
% \end{itemize}



\section{Proof of \lemref{lem: ortho}}\label{app:atom}
In this appendix we restate and provide the proof of \lemref{lem: ortho}. 
\begingroup
\renewcommand\thelemma{\ref{lem: ortho}} 
\begin{lemma}[Recovering orthogonal atoms]%\label{lem: ortho}
    Let \( \pphi \in \reals^{p \times p} \) be a symmetric positive semi-definite matrix. Define the set of orthogonal Cholesky decompositions of \( \pphi \) as
    \[
        \cW_{\sf{CD}} = \left\{ \textbf{U} \in \reals^{p \times r} \,\bigg|\, \pphi = \textbf{U} \textbf{U}^\top \text{ and } \textbf{U}^\top \textbf{U} = \text{diag}(\lambda_1,\ldots, \lambda_r) \right\},
    \]
    where \( r = \text{rank}(\pphi) \) and \( \lambda_1, \lambda_2, \ldots, \lambda_r \) are the eigenvalues of $\pphi$ in descending order. Then, for any two matrices \( \textbf{U}, \textbf{U}' \in \cW_{\sf{CD}} \), there exists an orthogonal matrix \( R \in \reals^{r \times r} \) such that
    \[
        \textbf{U}' = \textbf{U} \textbf{R},
    \]
    where \( \textbf{R} \) is block diagonal with orthogonal blocks corresponding to any repeated diagonal entries \( d_i \) in \( \textbf{U}^\top \textbf{U} \). Additionally, each column of \( \textbf{U}' \) can differ from the corresponding column of \( \textbf{U} \) by a sign change.
\end{lemma}
\endgroup


\begin{proof}
Let $\textbf{U}, \textbf{U}' \in \cW_{\sf{CD}}$ be two orthogonal Cholesky decompositions of $\pphi$. Define $\textbf{R} = \textbf{U}^\top \text{diag}(1/\lambda_1,\ldots,1/\lambda_r)\textbf{U}'$. We will show that this matrix satisfies our requirements through the following steps:

First, we show that $\textbf{R}$ is orthogonal. Note,
\begin{align*}
    \textbf{R}^\top \textbf{R} &= (\textbf{U}^\top \text{diag}(1/\lambda_1,\ldots,1/\lambda_r)\textbf{U}')^\top (\textbf{U}^\top \text{diag}(1/\lambda_1,\ldots,1/\lambda_r)\textbf{U}') \\
    &= \textbf{U}'^\top \text{diag}(1/\lambda_1,\ldots,1/\lambda_r)\textbf{U} \textbf{U}^\top \text{diag}(1/\lambda_1,\ldots,1/\lambda_r)\textbf{U}' \\
    &= \textbf{U}'^\top \text{diag}(1/\lambda_1,\ldots,1/\lambda_r)\pphi \text{diag}(1/\lambda_1,\ldots,1/\lambda_r)\textbf{U}' \\
    &= \textbf{U}'^\top \text{diag}(1/\lambda_1,\ldots,1/\lambda_r)\textbf{U}'\textbf{U}'^\top \text{diag}(1/\lambda_1,\ldots,1/\lambda_r)\textbf{U}' \\
    &= \textbf{U}'^\top \text{diag}(1/\lambda_1,\ldots,1/\lambda_r)\textbf{U}'\\%\text{diag}(\lambda_1,\ldots,\lambda_r) \\
    &= \textbf{I}_r
\end{align*}

Similarly,
\begin{align*}
    \textbf{R}\textbf{R}^\top &= \textbf{U}^\top \text{diag}(1/\lambda_1,\ldots,1/\lambda_r)\textbf{U}'(\textbf{U}')^\top \text{diag}(1/\lambda_1,\ldots,1/\lambda_r)\textbf{U} \\
    &= \textbf{U}^\top \text{diag}(1/\lambda_1,\ldots,1/\lambda_r)\pphi \text{diag}(1/\lambda_1,\ldots,1/\lambda_r)\textbf{U} \\
    &= \textbf{U}^\top \text{diag}(1/\lambda_1,\ldots,1/\lambda_r)\textbf{U}\textbf{U}^\top\textbf{U} \\
    &= \textbf{U}^\top \text{diag}(1/\lambda_1,\ldots,1/\lambda_r)\textbf{U}\text{diag}(\lambda_1,\ldots,\lambda_r) \\
    &= \textbf{I}_r
\end{align*}

Now we show that $\textbf{U}' = \textbf{U}\textbf{R}$. 
\begin{align*}
    \textbf{U}\textbf{R} &= \textbf{U}\textbf{U}^\top \text{diag}(1/\lambda_1,\ldots,1/\lambda_r)\textbf{U}' \\
    &= \pphi \text{diag}(1/\lambda_1,\ldots,1/\lambda_r)\textbf{U}' \\
    &= \textbf{U}'\textbf{U}'^\top\textbf{U}' \text{diag}(1/\lambda_1,\ldots,1/\lambda_r) \\
    &= \textbf{U}'\text{diag}(\lambda_1,\ldots,\lambda_r) \text{diag}(1/\lambda_1,\ldots,1/\lambda_r) \\
    &= \textbf{U}'
\end{align*}
 To show that \( \mathbf{R} \) is block diagonal with orthogonal blocks corresponding to repeated eigenvalues, consider the partitioning based on distinct eigenvalues. Let \( \mathcal{I}_k = \{i \mid \lambda_i = \gamma_k\} \) be the set of indices corresponding to the \( k \)-th distinct eigenvalue \( \gamma_k \) of \( \pphi \), for \( k = 1, \ldots, K \), where \( K \) is the number of distinct eigenvalues. Let \( m_k = |\mathcal{I}_k| \) denote the multiplicity of \( \gamma_k \).
    
    Define \( \mathbf{U}_k \) and \( \mathbf{U}'_k \) as the submatrices of \( \mathbf{U} \) and \( \mathbf{U}' \) consisting of columns indexed by \( \mathcal{I}_k \), respectively.
    
    Now, consider the block \( \mathbf{R}_{k\ell} \) of \( \mathbf{R} \) corresponding to eigenvalues \( \gamma_k \) and \( \gamma_\ell \). For \( k \neq \ell \),
    % \[
    %     \mathbf{U}_k^\top \mathbf{D} \mathbf{U}'_\ell = \mathbf{U}_k^\top \text{diag}\left(\frac{1}{\gamma_1}, \ldots, \frac{1}{\gamma_r}\right) \mathbf{U}'_\ell.
    % \]
     \( \mathbf{U}_k \) and \( \mathbf{U}'_\ell \) correspond to different eigenspaces (as \( \gamma_k \neq \gamma_\ell \)), and thus their inner product is zero. Hence,
    
    \[
        \mathbf{U}_k^\top \text{diag}\left(\frac{1}{\lambda_1}, \ldots, \frac{1}{\lambda_r}\right) \mathbf{U}'_\ell = \mathbf{0}_{m_k \times m_\ell}.
    \]
    
    This implies $\mathbf{R}_{k\ell} = \mathbf{0}_{m_k \times m_\ell} \quad \text{for} \quad k \neq \ell.$
    
    But then \( \mathbf{R} \) must be block diagonal:
    
    \[
        \mathbf{R} = \begin{bmatrix}
            \mathbf{R}_1 & \mathbf{0} & \cdots & \mathbf{0} \\
            \mathbf{0} & \mathbf{R}_2 & \cdots & \mathbf{0} \\
            \vdots & \vdots & \ddots & \vdots \\
            \mathbf{0} & \mathbf{0} & \cdots & \mathbf{R}_K \\
        \end{bmatrix},
    \]
    where each \( \mathbf{R}_k \in \mathbb{R}^{m_k \times m_k} \) is an orthogonal matrix. For eigenvalues with multiplicity one (\( m_k = 1 \)), the corresponding block \( \mathbf{R}_k \) is a \( 1 \times 1 \) orthogonal matrix. The only possibilities are:
    \[
        \mathbf{R}_k = [1] \quad \text{or} \quad \mathbf{R}_k = [-1],
    \]
    representing a sign change in the corresponding column of \( \mathbf{U} \). For eigenvalues with multiplicity greater than one (\( m_k > 1 \)), each block \( \mathbf{R}_k \) can be any \( m_k \times m_k \) orthogonal matrix. This allows for rotations within the eigenspace corresponding to the repeated eigenvalue \( \gamma_k \).
    
    Combining all steps, we have shown that:
    \[
        \mathbf{U}' = \mathbf{U} \mathbf{R},
    \]
    where \( \mathbf{R} \) is an orthogonal, block-diagonal matrix. Each block \( \mathbf{R}_k \) corresponds to a distinct eigenvalue \( \gamma_k \) of \( \pphi \) and is either a \( 1 \times 1 \) matrix with entry \( \pm 1 \) (for unique eigenvalues) or an arbitrary orthogonal matrix of size equal to the multiplicity of \( \gamma_k \) (for repeated eigenvalues). This completes the proof of the lemma.
    
%  To show that $\textbf{R}$ is block diagonal, let $\mathcal{I}_k = \{i : \lambda_i = \gamma_k\}$ be the set of indices corresponding to the $k$-th distinct eigenvalue $\gamma_k$ of $\pphi$. Let $\textbf{U}_k$ and $\textbf{U}'_k$ be the submatrices of $\textbf{U}$ and $\textbf{U}'$ consisting of columns indexed by $\mathcal{I}_k$.

% For any $i \in \mathcal{I}_k$ and $j \in \mathcal{I}_\ell$ where $k \neq \ell$:
% \begin{align*}
%     (\textbf{U}_k)^\top \text{diag}(1/\lambda_1,\ldots,1/\lambda_r)\textbf{U}'_\ell &= 0
% \end{align*}
% This follows because $\textbf{U}_k$ and $\textbf{U}'_\ell$ correspond to different eigenspaces of $\pphi$.

% Therefore, $\textbf{R}$ must be block diagonal, with blocks corresponding to each distinct eigenvalue.

% For the sign change property, note that when an eigenvalue $\lambda_i$ appears with multiplicity 1, the corresponding block in $\textbf{R}$ is $1 \times 1$ and must be $\pm 1$ since $\textbf{R}$ is orthogonal.

% Thus, we have shown that $\textbf{U}' = \textbf{U}\textbf{R}$ where $\textbf{R}$ is an orthogonal block diagonal matrix with the specified structure.
\end{proof}

% \begin{proof}
%     Let \( \mathbf{U}, \mathbf{U}' \in \cW_{\sf{CD}} \) be two orthogonal Cholesky decompositions of \( \pphi \). Define the diagonal matrix \( \mathbf{D} = \text{diag}\left(\frac{1}{\lambda_1}, \ldots, \frac{1}{\lambda_r}\right) \) and set
%     \[
%         \mathbf{R} = \mathbf{U}^\top \mathbf{D} \mathbf{U}'.
%     \]
    
%     \textbf{Step 1: Verifying Orthogonality of \( \mathbf{R} \)}
    
%     We first show that \( \mathbf{R} \) is orthogonal, i.e., \( \mathbf{R}^\top \mathbf{R} = \mathbf{I}_r \) and \( \mathbf{R} \mathbf{R}^\top = \mathbf{I}_r \).
    
%     \begin{align*}
%         \mathbf{R}^\top \mathbf{R} &= (\mathbf{U}^\top \mathbf{D} \mathbf{U}')^\top (\mathbf{U}^\top \mathbf{D} \mathbf{U}') \\
%         &= (\mathbf{U}')^\top \mathbf{D}^\top \mathbf{U} \mathbf{U}^\top \mathbf{D} \mathbf{U}' \\
%         &= (\mathbf{U}')^\top \mathbf{D} \mathbf{U} \mathbf{U}^\top \mathbf{D} \mathbf{U}' \quad (\mathbf{D} \text{ is diagonal and hence } \mathbf{D}^\top = \mathbf{D}) \\
%         &= (\mathbf{U}')^\top \mathbf{D} \pphi \mathbf{D} \mathbf{U}' \quad (\pphi = \mathbf{U} \mathbf{U}^\top = \mathbf{U}' \mathbf{U}'^\top) \\
%         &= (\mathbf{U}')^\top \mathbf{D} \mathbf{U}' \mathbf{U}'^\top \mathbf{D} \mathbf{U}' \\
%         &= (\mathbf{U}')^\top \mathbf{D} \mathbf{U}' (\mathbf{U}'^\top \mathbf{U}') \mathbf{D} \mathbf{U}' \quad (\pphi = \mathbf{U}' \mathbf{U}'^\top) \\
%         &= (\mathbf{U}')^\top \mathbf{D} \mathbf{U}' \mathbf{D} \mathbf{U}' \quad (\mathbf{U}'^\top \mathbf{U}' = \text{diag}(\lambda_1,\ldots,\lambda_r)) \\
%         &= \mathbf{I}_r \quad (\mathbf{D} = \text{diag}(1/\lambda_i) \text{ and } \mathbf{U}'^\top \mathbf{U}' = \text{diag}(\lambda_i)) \\
%     \end{align*}
    
%     Similarly,
    
%     \begin{align*}
%         \mathbf{R} \mathbf{R}^\top &= \mathbf{U}^\top \mathbf{D} \mathbf{U}' (\mathbf{U}^\top \mathbf{D} \mathbf{U}')^\top \\
%         &= \mathbf{U}^\top \mathbf{D} \mathbf{U}' \mathbf{U}'^\top \mathbf{D}^\top \mathbf{U} \\
%         &= \mathbf{U}^\top \mathbf{D} \pphi \mathbf{D} \mathbf{U} \quad (\pphi = \mathbf{U}' \mathbf{U}'^\top) \\
%         &= \mathbf{U}^\top \mathbf{D} \mathbf{U} \mathbf{U}^\top \mathbf{U} \mathbf{D} \mathbf{U} \\
%         &= \mathbf{U}^\top \mathbf{D} \mathbf{U} \text{diag}(\lambda_1,\ldots,\lambda_r) \mathbf{D} \mathbf{U} \quad (\mathbf{U}^\top \mathbf{U} = \text{diag}(\lambda_i)) \\
%         &= \mathbf{U}^\top \mathbf{D} \text{diag}(\lambda_i) \mathbf{D} \mathbf{U} \\
%         &= \mathbf{U}^\top \text{diag}\left(\frac{\lambda_1}{\lambda_1}, \ldots, \frac{\lambda_r}{\lambda_r}\right) \mathbf{U} \\
%         &= \mathbf{U}^\top \mathbf{U} \\
%         &= \text{diag}(\lambda_1, \ldots, \lambda_r) \\
%         &= \mathbf{I}_r \quad (\text{since } \mathbf{D} \mathbf{U}'^\top \mathbf{U}' \mathbf{D} = \mathbf{I}_r).
%     \end{align*}
    
%     Therefore, \( \mathbf{R} \) is orthogonal:
%     \[
%         \mathbf{R}^\top \mathbf{R} = \mathbf{R} \mathbf{R}^\top = \mathbf{I}_r.
%     \]
    
%     \textbf{Step 2: Showing \( \mathbf{U}' = \mathbf{U} \mathbf{R} \)}
    
%     \begin{align*}
%         \mathbf{U} \mathbf{R} &= \mathbf{U} \mathbf{U}^\top \mathbf{D} \mathbf{U}' \\
%         &= \pphi \mathbf{D} \mathbf{U}' \quad (\pphi = \mathbf{U} \mathbf{U}^\top) \\
%         &= \mathbf{U}' \mathbf{U}'^\top \mathbf{D} \mathbf{U}' \quad (\pphi = \mathbf{U}' \mathbf{U}'^\top) \\
%         &= \mathbf{U}' (\mathbf{U}'^\top \mathbf{U}') \mathbf{D} \mathbf{U}' \quad (\text{Associativity}) \\
%         &= \mathbf{U}' \text{diag}(\lambda_1,\ldots,\lambda_r) \mathbf{D} \mathbf{U}' \\
%         &= \mathbf{U}' \text{diag}\left(\frac{\lambda_1}{\lambda_1}, \ldots, \frac{\lambda_r}{\lambda_r}\right) \mathbf{U}' \\
%         &= \mathbf{U}' \mathbf{I}_r \\
%         &= \mathbf{U}'
%     \end{align*}
    
%     Thus, we have:
%     \[
%         \mathbf{U}' = \mathbf{U} \mathbf{R}.
%     \]
    
%     \textbf{Step 3: Establishing Block-Diagonal Structure of \( \mathbf{R} \)}
    
%     To show that \( \mathbf{R} \) is block diagonal with orthogonal blocks corresponding to repeated eigenvalues, consider the partitioning based on distinct eigenvalues.
    
%     Let \( \mathcal{I}_k = \{i \mid \lambda_i = \gamma_k\} \) be the set of indices corresponding to the \( k \)-th distinct eigenvalue \( \gamma_k \) of \( \pphi \), for \( k = 1, \ldots, K \), where \( K \) is the number of distinct eigenvalues. Let \( m_k = |\mathcal{I}_k| \) denote the multiplicity of \( \gamma_k \).
    
%     Define \( \mathbf{U}_k \) and \( \mathbf{U}'_k \) as the submatrices of \( \mathbf{U} \) and \( \mathbf{U}' \) consisting of columns indexed by \( \mathcal{I}_k \), respectively.
    
%     Consider the block \( \mathbf{R}_{k\ell} \) of \( \mathbf{R} \) corresponding to eigenvalues \( \gamma_k \) and \( \gamma_\ell \). For \( k \neq \ell \):
    
%     \[
%         \mathbf{U}_k^\top \mathbf{D} \mathbf{U}'_\ell = \mathbf{U}_k^\top \text{diag}\left(\frac{1}{\gamma_1}, \ldots, \frac{1}{\gamma_r}\right) \mathbf{U}'_\ell.
%     \]
    
%     Since \( \mathbf{U}_k \) and \( \mathbf{U}'_\ell \) correspond to different eigenspaces (as \( \gamma_k \neq \gamma_\ell \)), their inner product is zero:
    
%     \[
%         \mathbf{U}_k^\top \mathbf{U}'_\ell = \mathbf{0}_{m_k \times m_\ell}.
%     \]
    
%     Therefore:
    
%     \[
%         \mathbf{R}_{k\ell} = \mathbf{0}_{m_k \times m_\ell} \quad \text{for} \quad k \neq \ell.
%     \]
    
%     This implies that \( \mathbf{R} \) must be block diagonal:
    
%     \[
%         \mathbf{R} = \begin{bmatrix}
%             \mathbf{R}_1 & \mathbf{0} & \cdots & \mathbf{0} \\
%             \mathbf{0} & \mathbf{R}_2 & \cdots & \mathbf{0} \\
%             \vdots & \vdots & \ddots & \vdots \\
%             \mathbf{0} & \mathbf{0} & \cdots & \mathbf{R}_K \\
%         \end{bmatrix},
%     \]
%     where each \( \mathbf{R}_k \in \mathbb{R}^{m_k \times m_k} \) is an orthogonal matrix.
    
%     \textbf{Step 4: Sign Change for Unique Eigenvalues}
    
%     For eigenvalues with multiplicity one (\( m_k = 1 \)), the corresponding block \( \mathbf{R}_k \) is a \( 1 \times 1 \) orthogonal matrix. The only possibilities are:
%     \[
%         \mathbf{R}_k = [1] \quad \text{or} \quad \mathbf{R}_k = [-1],
%     \]
%     representing a **sign change** in the corresponding column of \( \mathbf{U} \).
    
%     \textbf{Step 5: Rotations for Repeated Eigenvalues}
    
%     For eigenvalues with multiplicity greater than one (\( m_k > 1 \)), each block \( \mathbf{R}_k \) can be any \( m_k \times m_k \) orthogonal matrix. This allows for **rotations within the eigenspace** corresponding to the repeated eigenvalue \( \gamma_k \).
    
%     \textbf{Conclusion}
    
%     Combining all steps, we have shown that:
%     \[
%         \mathbf{U}' = \mathbf{U} \mathbf{R},
%     \]
%     where \( \mathbf{R} \) is an orthogonal, block-diagonal matrix. Each block \( \mathbf{R}_k \) corresponds to a distinct eigenvalue \( \gamma_k \) of \( \pphi \) and is either a \( 1 \times 1 \) matrix with entry \( \pm 1 \) (for unique eigenvalues) or an arbitrary orthogonal matrix of size equal to the multiplicity of \( \gamma_k \) (for repeated eigenvalues).

%     This completes the proof of the lemma.
% \end{proof}


\iffalse
\begin{proof}
    Since \( \pphi \) is symmetric and positive semi-definite, it admits an eigen-decomposition:
\[
    \pphi = \mathbf{Q} \boldsymbol{\Lambda} \mathbf{Q}^\top,
\]
where:
\begin{itemize}
    \item \( \mathbf{Q} \in \mathbb{R}^{p \times p} \) is an orthogonal matrix (\( \mathbf{Q}^\top \mathbf{Q} = \mathbf{Q} \mathbf{Q}^\top = \mathbf{I}_p \)),
    \item \( \boldsymbol{\Lambda} = \text{diag}(\lambda_1, \lambda_2, \ldots, \lambda_p) \) with \( \lambda_1 \geq \lambda_2 \geq \ldots \geq \lambda_p \geq 0 \).
\end{itemize}
Let \( r = \text{rank}(\pphi) \), implying \( \lambda_1, \lambda_2, \ldots, \lambda_r > 0 \) and \( \lambda_{r+1} = \ldots = \lambda_p = 0 \).

\textbf{Step 2: Orthogonal Cholesky Decompositions}

Consider two matrices \( \mathbf{U}, \mathbf{U}' \in \cW_{\sf{CD}} \). By definition:
\[
    \pphi = \mathbf{U} \mathbf{U}^\top = \mathbf{U}' \mathbf{U}'^\top,
\]
and
\[    \mathbf{U}^\top \mathbf{U} = \mathbf{U}'^\top \mathbf{U}' = \text{diag}(\lambda_1, \lambda_2, \ldots, \lambda_r).\]
Both \( \mathbf{U} \) and \( \mathbf{U}' \) have full column rank \( r \).

\textbf{Step 3: Relating \( \mathbf{U} \) and \( \mathbf{U}' \) via an Orthogonal Matrix}

Since both \( \mathbf{U} \) and \( \mathbf{U}' \) provide a full-rank factorization of \( \pphi \), there exists an orthogonal matrix \( \mathbf{R} \in \mathbb{R}^{r \times r} \) such that:
\[
    \mathbf{U}' = \mathbf{U} \mathbf{R}.
\]
\textbf{Justification:}
Given that \( \mathbf{U} \) and \( \mathbf{U}' \) are both in \( \cW_{\sf{CD}} \), we can write:
\[
    \mathbf{U}' = \mathbf{U} \mathbf{R},
\]
where \( \mathbf{R} \) satisfies:
\[
    \mathbf{U}'^\top \mathbf{U}' = \mathbf{R}^\top \mathbf{U}^\top \mathbf{U} \mathbf{R} = \mathbf{R}^\top \text{diag}(\lambda_1, \ldots, \lambda_r) \mathbf{R} = \text{diag}(\lambda_1, \ldots, \lambda_r).
\]
Therefore, \( \mathbf{R} \) must satisfy:
\[
    \mathbf{R}^\top \text{diag}(\lambda_1, \ldots, \lambda_r) \mathbf{R} = \text{diag}(\lambda_1, \ldots, \lambda_r).
\]
This condition constrains \( \mathbf{R} \) to be block diagonal with orthogonal blocks corresponding to repeated eigenvalues.

\textbf{Step 4: Structure of \( \mathbf{R} \)}

To elucidate the structure of \( \mathbf{R} \), consider the multiplicities of the eigenvalues \( \lambda_i \):

\begin{itemize}
    \item Let there be \( k \) distinct eigenvalues among \( \lambda_1, \lambda_2, \ldots, \lambda_r \), with \( \mu_1, \mu_2, \ldots, \mu_k \).
    \item Let \( m_j \) denote the multiplicity of \( \mu_j \), such that \( \sum_{j=1}^k m_j = r \).
\end{itemize}

Rearrange \( \lambda_1, \lambda_2, \ldots, \lambda_r \) so that identical eigenvalues are consecutive. Thus, the diagonal matrix can be partitioned as:
\[
    \text{diag}(\lambda_1, \ldots, \lambda_r) = \begin{bmatrix}
        \mu_1 \mathbf{I}_{m_1} & & & \\
        & \mu_2 \mathbf{I}_{m_2} & & \\
        & & \ddots & \\
        & & & \mu_k \mathbf{I}_{m_k}
    \end{bmatrix}.
\]
Given this partitioning, the orthogonal matrix \( \mathbf{R} \) must preserve each block corresponding to a distinct eigenvalue. Therefore, \( \mathbf{R} \) can be expressed as a block diagonal matrix:
\[
    \mathbf{R} = \begin{bmatrix}
        \mathbf{R}_1 & & & \\
        & \mathbf{R}_2 & & \\
        & & \ddots & \\
        & & & \mathbf{R}_k
    \end{bmatrix},
\]
where each \( \mathbf{R}_j \in \mathbb{R}^{m_j \times m_j} \) is an orthogonal matrix (\( \mathbf{R}_j^\top \mathbf{R}_j = \mathbf{I}_{m_j} \)).

\textbf{Explanation:}

\begin{enumerate}
    \item **Unique Eigenvalues (\( m_j = 1 \)):**  
        For eigenvalues with multiplicity one, \( \mathbf{R}_j \) must be a \( 1 \times 1 \) orthogonal matrix. The only orthogonal \( 1 \times 1 \) matrices are \( [1] \) and \( [-1] \), corresponding to sign changes.
    
    \item **Repeated Eigenvalues (\( m_j > 1 \)):**  
        For eigenvalues with multiplicity greater than one, \( \mathbf{R}_j \) can be any orthogonal matrix of size \( m_j \times m_j \), allowing for rotations within the corresponding eigenspace.
\end{enumerate}

\textbf{Step 5: Conclusion}

Combining the above steps, we conclude that for any two orthogonal Cholesky decompositions \( \mathbf{U} \) and \( \mathbf{U}' \) of \( \pphi \), there exists an orthogonal matrix \( \mathbf{R} \) such that:
\[
    \mathbf{U}' = \mathbf{U} \mathbf{R},
\]
where \( \mathbf{R} \) is block diagonal with orthogonal blocks \( \mathbf{R}_j \) corresponding to the multiplicities of the eigenvalues \( \lambda_j \). Specifically:
\begin{itemize}
    \item Each \( \mathbf{R}_j \) for \( m_j > 1 \) allows for arbitrary rotations within the eigenspace corresponding to \( \mu_j \).
    \item Each \( \mathbf{R}_j \) for \( m_j = 1 \) allows for sign changes in the corresponding column of \( \mathbf{U} \).
\end{itemize}

Therefore, the lemma is proven.

\end{proof}
\fi

%\begin{proof}
%    Let \( U, U' \in W_{\sf{CD}} \). Since both \( U^\top U \) and \( U'^\top U' \) are diagonal matrices with the same entries, it follows that the columns of \( U \) and \( U' \) are orthogonal and scaled by the square roots of the diagonal entries \( d_i \).
%    Define \( R = U^\top U' \). Since \( U^\top U = U'^\top U' = \text{diag}(d_1, d_2, \ldots, d_r) \), it follows that \( R \) is an orthogonal matrix.
%    Therefore, \( U' = U R \). Additionally, to account for possible sign changes, \( R \) can be decomposed into a product of an orthogonal matrix and a diagonal matrix with \( \pm 1 \) entries, i.e., \( R = Q S \), where \( Q \) is orthogonal and \( S = \text{diag}(\epsilon_1, \epsilon_2, \ldots, \epsilon_r) \) with \( \epsilon_i \in \{+1, -1\} \).
%    Hence, \( U' = U Q S \), illustrating that the columns of \( U' \) are equivalent to those of \( U \) up to orthogonal transformations and sign changes.
%\end{proof}

\section{Worst-case bounds: Constructive case}\label{app: worstcase}

In this Appendix, we provide the proof of the lower bound as stated in \propref{prop: worstcase}. Before we prove this lower bound, we state a useful property of the sum of a symmetric, PSD matrix and a general symmetric matrix in $\symm$.
\begin{lemma}\label{lem: sum}
    Let $\pphi \in \symmp$ be a symmetric matrix with full rank, i.e., $\rank{\pphi} = p$. For any arbitrary symmetric matrix $\pphi' \in \symm$, there exists a positive scalar $\lambda > 0$ such that the matrix $(\pphi + \lambda \pphi')$ is positive semidefinite.
\end{lemma}

\begin{proof}
    Since $\pphi$ is symmetric and has full rank, it admits an eigendecomposition:
    \[
        \pphi = \sum_{i=1}^p \lambda_i u_i u_i^{\top},
    \]
    where $\{\lambda_i\}_{i=1}^p$ are the positive eigenvalues and $\{u_i\}_{i=1}^p$ are the corresponding orthonormal eigenvectors of $\pphi$.

    Define the constant $\gamma$ as the maximum absolute value of the quadratic forms of $\pphi'$ with respect to the eigenvectors of $\pphi$:
    \[
        \gamma := \max_{1 \leq i \leq p} \left| u_i^{\top} \pphi' u_i \right|.
    \]
    
    Let $\lambda$ be chosen as:
    \[
        \lambda := \frac{\min_{1 \leq i \leq p} \lambda_i}{\gamma}.
    \]
    
    For each eigenvector $u_i$, consider the quadratic form of $(\pphi + \lambda \pphi')$:
    \[
        u_i^{\top} (\pphi + \lambda \pphi') u_i = \lambda_i + \lambda u_i^{\top} \pphi' u_i \geq \lambda_i - \lambda \gamma = \lambda_i - \frac{\min \lambda_i}{\gamma} \gamma = \lambda_i - \min \lambda_i \geq 0.
    \]
    This shows that each eigenvector $u_i$ satisfies:
    \[
        u_i^{\top} (\pphi + \lambda \pphi') u_i \geq 0.
    \]
    
    Since $\{u_i\}_{i=1}^p$ forms an orthonormal basis for $\mathbb{R}^p$, for any vector $x \in \mathbb{R}^p$, we can express $x$ as $x = \sum_{i=1}^p a_i u_i$. Then:
    \[
        x^{\top} (\pphi + \lambda \pphi') x = \sum_{i=1}^p a_i^2 u_i^{\top} (\pphi + \lambda \pphi') u_i \geq 0,
    \]
    since each term in the sum is non-negative.

    Therefore, $(\pphi + \lambda \pphi')$ is positive semidefinite.
\end{proof}


Now, we provide the proof of \propref{prop: worstcase} in the following:
\iffalse
\begin{proof}[Proof of \propref{prop: worstcase}] 
    Consider a full rank matrix $\pphi^* \in \symmp$. For the sake of contradiction, let $\cF(\cV, \maha, \pphi^*)$ be a feedback set for \eqnref{eq: redsol} up to linear scaling relation with size strictly less than $\paren{\frac{p(p+1)}{2} - 1}$.
    
    Now, for any pair $(y,z) \in \cF$, $\pphi^*$ is orthogonal to $(yy^{\top} - zz^{\top})$. Thus, if we define $\mathcal{O}_{\pphi^*}$ as the orthogonal complement of $\pphi^*$ then for any $(y,z) \in \cF$ we have $(yy^{\top} - zz^{\top}) \in \mathcal{O}_{\pphi^*}$. Thus,
    \begin{align*}
        span\inner{\{yy^{\top} - zz^{\top}\}_{(y,z) \in \cF}} \subset \mathcal{O}_{\pphi^*}
    \end{align*}
    Hence,
    \begin{align*}
        \pphi^* \perp span\inner{\{yy^{\top} - zz^{\top}\}_{(y,z) \in \cF}}
    \end{align*}
    Since the feedback set $|\cF| < \paren{\frac{p(p+1)}{2} - 1}$ we note that $\dim (span\inner{(yy^{\top} - zz^{\top})}) < \paren{\frac{p(p+1)}{2} - 1}$. 
    
    Note $\pphi^*$ is a singleton vector in $\reals^{p \times p}$ the union  $\curlybracket{\pphi^*} \cup \{yy^{\top} - zz^{\top}\}_{(y,z) \in \cF}$ will only add an extra dimension in the space $\reals^{p \times p}$. This implies that
    \begin{align*}
        \dim(span\inner{\pphi^*\cup \{yy^{\top} - zz^{\top}\}_{(y,z) \in \cF}} \le \paren{\frac{p(p+1)}{2} - 1}
    \end{align*}
    Since $\symm$ is a vector space over $\reals$ and $\dim(\symm) = \frac{p(p+1)}{2}$ there is a symmetric matrix $\pphi'$ such that the following holds
    \begin{gather*}    
        \pphi' \in \mathcal{O}_{\pphi^*},\\
        \forall (y,z) \in \cF,\,  \pphi' \perp (yy^{\top} - zz^{\top})
    \end{gather*}
    But \lemref{lem: sum} implies there exists $\lambda > 0$ such that $\pphi^* + \lambda \pphi'$ is PSD and symmetric (sum of symmetric matrices is symmetric). Since $\pphi' \in \mathcal{O}_{\pphi^*}$, $\pphi'$ is not identical to $\pphi^*$ up to a linear scaling. This implies that there exists a matrix in the form $\pphi^* + \lambda \pphi'$ ($\,\not \sim_{R_l} \pphi^*$) that is orthogonal to all the matrices $(yy^{\top} - zz^{\top})$ for any pair $(y,z) \in \cF$.
    
    Thus, if the feedback set is smaller than $\frac{p(p+1)}{2} - 1$, we can find symmetric, PSD matrices not related up to linear scaling that satisfy \eqnref{eq: redsol}. This contradicts the assumption on $\cF$. This establishes the stated lower bound on the feedback complexity of the feedback set.
\end{proof}
\fi
\begin{proof}[Proof of \propref{prop: worstcase}] 
Assume, for contradiction, that there exists a feedback set $\cF(\cV, \maha, \pphi^*)$ for \eqnref{eq: redsol} with size $|\cF| < \left(\frac{p(p+1)}{2} - 1\right)$.

For each pair $(y,z) \in \cF$, $\pphi^*$ is orthogonal to $(yy^{\top} - zz^{\top})$, implying that $(yy^{\top} - zz^{\top}) \in \mathcal{O}_{\pphi^*}$, the orthogonal complement of $\pphi^*$. Therefore,
\[
\spn\inner{\{yy^{\top} - zz^{\top}\}_{(y,z) \in \cF}} \subset \mathcal{O}_{\pphi^*}.
\]
This leads to
\[
\pphi^* \perp \spn\inner{\{yy^{\top} - zz^{\top}\}_{(y,z) \in \cF}}.
\]
Since $|\cF| < \frac{p(p+1)}{2} - 1$, we have
\[
\dim \left( \spn\inner{ \{yy^{\top} - zz^{\top}\} } \right) < \frac{p(p+1)}{2} - 1.
\]
Adding $\pphi^*$ to this span increases the dimension by at most one:
\[
\dim \left( \spn\inner{ \pphi^* \cup \{yy^{\top} - zz^{\top}\}_{(y,z) \in \cF} } \right) \leq \frac{p(p+1)}{2} - 1.
\]
Since $\symm$ is a vector space with $\dim(\symm) = \frac{p(p+1)}{2}$, there exists a symmetric matrix $\pphi' \in \mathcal{O}_{\pphi^*}$ such that
\[
\pphi' \perp (yy^{\top} - zz^{\top}) \quad \forall \, (y,z) \in \cF.
\]
By \lemref{lem: sum}, there exists $\lambda > 0$ such that $\pphi^* + \lambda \pphi'$ is PSD and symmetric. Since $\pphi' \in \mathcal{O}_{\pphi^*}$ and $\pphi'$ is not a scalar multiple of $\pphi^*$, the matrix $\pphi^* + \lambda \pphi'$ is not related to $\pphi^*$ via linear scaling. However, it still satisfies \eqnref{eq: redsol}, contradicting the minimality of $\cF$.

Thus, any feedback set must satisfy
\[
|\cF| \geq \frac{p(p+1)}{2} - 1.
\]
This establishes the stated lower bound on the feedback complexity of the feedback set.
\end{proof}

\section{Proof of \thmref{thm: constructgeneral}: Upper bound}\label{app: constub}
%\section{Proof of \thmref{thm: obv}}
Below we provide proof of the upper bound stated in \thmref{thm: constructgeneral}. 


Consider the eigendecomposition of the matrix $\pphi^*$. There exists a set of orthonormal vectors $\curlybracket{v_1, v_2, \ldots, v_r}$ with corresponding eigenvalues $\curlybracket{\gamma_1, \gamma_2, \ldots, \gamma_r}$ such that
\begin{align}
    \pphi^* = \sum_{i=1}^r \gamma_i v_i v_i^{\top} \label{eq: target}
\end{align}
Denote the set of orthogonal vectors $\curlybracket{v_1, v_2, \ldots, v_r}$ as $V_{\bracket{r}}$.

%\subsection{Orthogonal Extension to a Basis}

Let $\curlybracket{v_{r+1}, \dots, v_p}$, denoted as $V_{\bracket{p - r}}$, be an orthogonal extension to the vectors in $V_{\bracket{r}}$ such that
\[
    V_{\bracket{r}} \cup V_{\bracket{p - r}} = \curlybracket{v_1, v_2, \ldots, v_p}
\]
forms an orthonormal basis for $\reals^p$. Denote the complete basis $\curlybracket{v_1, v_2, \ldots, v_p}$ as $V_{\bracket{p}}$.

Note that $\curlybracket{v_{r+1}, \ldots, v_p}$ precisely defines the null space of $\pphi^*$, i.e.,
\[
    \nul{\pphi^*} = \text{span}\inner{\curlybracket{v_{r+1}, \ldots, v_p}}.
\]

%\subsection{Strategy for Teaching the Null Space and Eigenvectors}

The key idea of the proof is to manipulate this null space to satisfy the feedback set condition in \eqnref{eq: orthosat} for the target matrix $\pphi^*$. Since $\pphi^*$ has rank $r \leq p$, the number of degrees of freedom is exactly $\frac{r(r+1)}{2}$. Alternatively, the span of the null space of $\pphi^*$, which has dimension exactly $p - r$, fixes the remaining entries in $\pphi^*$. 

Using this intuition, the teacher can provide pairs $(y, z) \in \cV^2$ to teach the null space and the eigenvectors $\curlybracket{v_1, v_2, \ldots, v_r}$ separately. However, it is necessary to ensure that this strategy is optimal in terms of sample efficiency. We confirm the optimality of this strategy in the next two lemmas.

\subsection{Feedback set for the null space of \texorpdfstring{$\pphi^*$}{phi*}}

Our first result is on nullifying the null set of $\pphi^*$ in the \eqnref{eq: orthosat}. Consider a partial feedback set 
\begin{align*}
    \cF_{\sf {null}} = \curlybracket{(0, v_{i})}_{i = r+1}^p
\end{align*}
\begin{lemma}\label{lem: nullset}
    If the teacher provides the set $\cF_{\sf{null}}$, then the null space of any PSD symmetric matrix $\pphi'$ that satisfies \eqnref{eq: orthosat} contains the span of $\{v_{r+1}, \ldots, v_p\}$, i.e.,
    \begin{equation*}
        \{v_{r+1}, \ldots, v_p\} \subseteq \nul{\pphi'}.
    \end{equation*}
    %If the teacher provides the set $\cF_{\sf{null}}$, then the null set of any psd symmetric matrix $\pphi'$ that satisfies \eqnref{eq: orthosat} contains the span of $\{v_{r+1},\ldots, v_p\}$, i.e.
    % \begin{align*}
    %    \{v_{r+1},\ldots, v_p\} \subset \nul{\pphi'}
    % \end{align*}
\end{lemma}
\begin{proof} Let $\pphi' \in \symmp$ be a matrix that satisfies \eqnref{eq: orthosat} (note that $\pphi^*$ satisfies \eqnref{eq: orthosat}). Thus, we have the following equality constraints:
\begin{equation*}
       \forall (0, v) \in \cF_{\sf{null}}, \quad v^{\top} \pphi' v = 0.
\end{equation*}
    Since $\curlybracket{v_{r+1}, \ldots, v_p}$ is a set of linearly independent vectors, it suffices to show that
    \begin{align}
        \forall v \in V_{\bracket{d - r}}, \quad v^{\top} \pphi' v = 0 \implies \pphi' v = 0. \label{eq: lemmain}
    \end{align}
    
    To prove \eqnref{eq: lemmain}, we utilize general properties of the eigendecomposition of a symmetric, positive semi-definite matrix. We express $\pphi'$ in its eigendecomposition as
    \[
        \pphi' = \sum_{i=1}^{s} \gamma_i' u_i u_i^{\top},
    \]
    where $\curlybracket{u_i}_{i=1}^{s}$ are the eigenvectors and $\curlybracket{\gamma_i'}_{i=1}^s$ are the corresponding eigenvalues of $\pphi'$. Assume that $x \neq 0 \in \reals^p$ satisfies
    \[
        x^{\top} \pphi' x = 0.
    \]
    Consider the decomposition $x = \sum_{i=1}^s a_iu_i + v'$ for scalars $a_i$ and $v' \bot \{u_i\}_{i=1}^s$ . Now, expanding the equation above, we get
    \allowdisplaybreaks
    \begin{align*}
       x^{\top}\pphi'x &= \paren{\sum_{i=1}^s a_iu_i + v'}^{\top}\pphi'\paren{\sum_{i=1}^s a_iu_i + v'}  \\
       & = \paren{\sum_{i=1}^s a_iu_i}^{\top}\pphi'\paren{\sum_{i=1}^s a_iu_i } + v'^{\top}\pphi'\paren{\sum_{i=1}^s a_iu_i} + \paren{\sum_{i=1}^s a_iu_i}\pphi'v' + v'^{\top}\pphi'v'\\
       & = \paren{\sum_{i=1}^s a_iu_i}^{\top}\paren{\sum_{i = 1}^{s} \gamma_i'u_iu_i^{\top}}\paren{\sum_{i=1}^s a_iu_i } + \underbrace{2v'^{\top}\paren{\sum_{i = 1}^{s} \gamma_i'u_iu_i^{\top}}\paren{\sum_{i=1}^s a_iu_i} + v'^{\top}\paren{\sum_{i = 1}^{s} \gamma_i'u_iu_i^{\top}}v'}_{ =\, 0 \textnormal{ as } v' \bot \curlybracket{u_i}} \\
       & = \sum_{i,j,k} a_i u_i^{\top} (\gamma_j'u_ju_j^{\top}) a_k u_k\\
       & = \sum_{i=1}^s a_i^2\gamma_i' = 0
    \end{align*}
    Since $\gamma_i' > 0$ for all $i = 1, \ldots, s$ (because $\pphi'$ is PSD), it follows that each $a_i = 0$. Therefore,
    \[
        \pphi' x = \pphi' v' = 0.
    \]
    This implies that $x \in \nul{\pphi'}$, thereby proving \eqnref{eq: lemmain}.
    
    Hence, if the teacher provides $\cF_{\sf{null}}$, any solution $\pphi'$ to \eqnref{eq: orthosat} must satisfy
    \[
        \{v_{r+1}, \ldots, v_p\} \subseteq \nul{\pphi'}.
    \]
\end{proof}

With this we will argue that the feedback setup in \eqnref{eq: orthosat} can be decomposed in two parts: first is teaching the null set $ \nul{\pphi^*}:= \text{span} \inner{\{v_i\}_{i=r+1}^n}$, and second is teaching $\mathcal{S}_{\pphi^*} = \text{span} \inner{\{v_i\}_{i=1}^r}$ in the form of $\pphi^* = \sum_{i=1}^r \gamma_i v_iv_i^{\top}$. 

\lemref{lem: nullset} implies that using a feedback set of the form $\cF_{\sf {null}}$ any solution $\pphi' \in \symmp$ to \eqnref{eq: orthosat} satisfies the property $V_{\bracket{d - r}} \subset \nul{\pphi'}$. Furthermore, $|\cF_{\sf {null}}| = p - r$. 

\subsection{Feedback set for the kernel of \texorpdfstring{$\pphi^*$}{phi*}}
Next, we discuss how to teach $V_{\bracket{r}}$, i.e. $V_{\bracket{r}}$ span the rows of any solution $\pphi' \in \symmp$ to \eqnref{eq: orthosat} with the corresponding eigenvalues $\curlybracket{\gamma_i}_{i=1}^r$. We show that if the search space of metrics in \eqnref{eq: orthosat} is the version space $\textsf{VS}(\maha,\cF_{\sf {null}})$  which is a restriction of the space $\maha$ to feedback set $\cF_{\sf {null}}$, then a feedback set of size at most $\frac{r(r+1)}{2} -1$ is sufficient to teach $\pphi^*$ up to feature equivalence. Thus, we consider the reformation of the problem in \eqnref{eq: orthosat} as 
\begin{align}
  \forall (y,z) \in \cF(\cX,\textsf{VS}(\maha,\cF_{\sf {null}}),\pphi^*), \quad \pphi \idot (yy^{\top} - zz^{\top})  = 0  \label{eq: redorthosat}
\end{align}
where the feedback set $\cF(\cX,\textsf{VS}(\maha,\cF_{\sf {null}}),\pphi^*)$ is devised to solve a smaller space $\textsf{VS}(\maha,\cF_{\sf {null}}) := \curlybracket{\pphi \in \maha \,|\, \pphi v = 0, \forall (0,v) \in \cF_{\sf {null}}}$. With this state the following useful lemma on the size of the restricted feedback set $\cF(\cX,\textsf{VS}(\maha,\cF_{\sf {null}}),\pphi^*)$.


%It is straight-forward that, since $\dim(\mathcal{N}_{\pphi^*}) = d - r$ one needs at least $(d-r)$ pairs to nullify $\mathcal{N}_{\pphi^*}$ for any psd symmetric matrix. On the other hand, using \lemref{lemma: nullset} we note that one can sufficiently nullify it with just $(d-r)$ pairs of the form $\curlybracket{(0, v_i)}_{i=r+1}^p$. But still one question remains if this set of pairs can be used in a different form. For that, we consider the following result.


\begin{lemma}\label{lem: orthoset}
    Consider the problem as formulated in \eqnref{eq: redorthosat} in which the null set $\nul{\pphi^*}$ of the target matrix $\pphi^*$ is known. Then, the teacher sufficiently and necessarily finds a set $\cF(\cX,\textsf{VS}(\cF_{\sf{null}}),\pphi^*)$ of size $\frac{r(r+1)}{2} - 1$ for oblivious learning up to feature equivalence.
\end{lemma}
\begin{proof}

    Note that any solution $\pphi'$ of \eqnref{eq: redorthosat} has its columns spanned exactly by $V_{\bracket{r}}$. Alternatively, if we consider the eigendecompostion of $\pphi'$ then the corresponding eigenvectors exists in $span \inner{V_{\bracket{r}}}$. Furthermore, note that $\pphi^*$ is of rank $r$ which implies there are only $\frac{r(r+1)}{2}$ degrees of freedom, i.e. entries in the matrix $\pphi^*$, that need to be fixed.

    Thus, there are exactly $r$ linearly independent columns of $\pphi^*$, indexed as $\{j_1,j_2,\ldots, j_r\}$. Now, consider the set of matrices
    \begin{align*}
        \curlybracket{\pphi^{(i,j)}\,|\, i \in \bracket{d}, j \in \{j_1,j_2,\ldots, j_r\}, \pphi^{(i,j)}_{i'j'} = \mathds{1}[i'\in \{i,j\}, j' \in \{i,j\}\setminus \{i'\}]}
    \end{align*}
    This forms a basis to generate any matrix with independent columns along the indexed set. Hence, the span of $\mathcal{S}_{\pphi^*}$ induces a subspace of symmetric matrices of dimension $\frac{r(r+1)}{2}$ in the vector space $\sf{symm}(\reals^p)$, i.e. the column vectors along the indexed set is spanned by elements of $\mathcal{S}_{\pphi^*}$. Thus, it is clear that picking a feedback set of size $\frac{r(r+1)}{2} -1$ in the orthogonal complement of $\pphi^*$, i.e. $\mathcal{O}_{\pphi^*}$ restricted by this span sufficiently teaches $\pphi^*$ if $\nul{\pphi^*}$ is known. One exact form of this set is proven in \lemref{lem: basis}. Since any solution $\pphi'$ is agnostic to the scaling of the target matrix $\pphi'$, we have shown that the sufficiency on the feedback complexity for $\pphi^*$ up to feature equivalence.

   Now, we show that the stated feedback set size is necessary. The argument is similar to the proof of \lemref{lem: sum}.
   
   For the sake of contradiction assume that there is a smaller sized feedback set $\cF_{\sf{small}}$. This implies that there is some matrix in $\textsf{VS}(\maha,\cF_{\sf {null}})$, a subspace induced by span $\mathcal{S}_{\pphi^*}$, orthogonal to $(\pphi^*)$ is not in the span of $\cF_{\sf{small}}$, denoted as $\pphi'$. If $\pphi'$ is PSD then it is a solution to \eqnref{eq: redorthosat} and $\pphi'$ is not a scalar multiple of $\pphi^*$. Now, if $\pphi'$ is not PSD we show that there exists scalar $\lambda > 0$ such that
    \begin{align*}
        \pphi^* + \lambda \pphi' \in \symmp,
    \end{align*}
     i.e. the sum is PSD. Consider the eigendecompostion of $\pphi'$ (assume $\rank{\pphi'} = r'$)
     \begin{align*}
         \pphi' = \sum_{i = 1}^{r'} \delta_i\mu_i\mu_i^{\top}
     \end{align*}
     for orthogonal eigenvectors $\curlybracket{\mu_i}_{i=1}^{r'}$ and the corresponding eigenvalues $\curlybracket{\delta_i}_{i=1}^{r'}$. Since (assume) $r_0 \le r'$ of the eigenvalues are negative we can rewrite $\pphi'$ as
     \begin{align*}
         \pphi' = \sum_{i=1}^{r_0} \delta_i \mu_i\mu_i^{\top} + \sum_{j=r_0 + 1}^{r'} \delta_j \mu_j\mu_j^{\top} 
     \end{align*}
     Thus, if we can regulate the values of $\mu^{\top}_i\pphi^*\mu_i$, for all $i = 1,2,\ldots,r_0$, noting they are positive, then we can find an appropriate scalar $\lambda > 0$. Let $m^* := \min_{i \in [r_0]} \mu_i^{\top}\pphi^*\mu_i$ and $\ell^* := \max_{i \in [r_0]} |\delta_i|$. Now, setting $\lambda \le \frac{m^*}{\ell^*}$ achieves the desired property of $\pphi^* + \lambda \pphi'$ as shown in the proof of \lemref{lem: sum}. 

     Consider that both $\pphi'$ and $\pphi^*$ are orthogonal to every element in the feedback set $\cF_{\sf{small}}$. This orthogonality implies that $\pphi^*$ is not a unique solution to equation \eqnref{eq: redorthosat} up to a positive scaling factor.

Therefore, we have demonstrated that when the null set $\nul{\pphi^*}$ of the target matrix $\pphi^*$ is known, a feedback set of size exactly $\frac{r(r+1)}{2} - 1$ is both \text{necessary} and \text{sufficient}.
\end{proof}

\subsection{Proof of \lemref{lem: basis} and construction of feedback set for \texorpdfstring{$\kernel{\pphi^*}$}{phi*}}


Up until this point we haven's shown how to construct this $\frac{r(r+1)}{2}-1$ sized feedback set. 
Consider the following union:
\begin{align*}
    \curlybracket{v_1v_1^{\top}} \cup \curlybracket{v_2v_2^{\top}, (v_2 + v_1)(v_2 + v_1)^{\top}} \cup \ldots \cup \curlybracket{v_rv_r^{\top}, (v_1 + v_r)(v_1 + v_r)^{\top},\ldots, (v_{r-1} + v_r)(v_{r-1} + v_r)^{\top}}
\end{align*}
We can show that this union is a set of linearly independent matrices of rank 1 as stated in \lemref{lem: basis} below. 
%First, note that if this set is not linearly independent then 
\begingroup
\renewcommand\thelemma{\ref{lem: basis}} 
\begin{lemma}
     Let $\{v_i\}_{i=1}^r \subset \reals^p$ be a set of orthogonal vectors. Then, the set of rank-1 matrices
    \[
    \mathcal{B} := \left\{v_i v_i^{\top},\ (v_i + v_j)(v_i + v_j)^{\top}\ \bigg| \ 1 \leq i < j \leq r \right\}
    \]
    is linearly independent in the space of symmetric matrices $\symm$.
\end{lemma}
\endgroup
\begin{proof}
    We prove the claim by considering two separate cases. For the sake of contradiction, suppose that the set $\cB$ is linearly dependent. This implies that there exists at least one matrix of the form $v_i v_i^{\top}$ or $(v_i + v_j)(v_i + v_j)^{\top}$ that can be expressed as a linear combination of the other matrices in $\cB$. We now examine these two cases individually.
    
    \textbf{Case 1}: First, we assume that for some $i \in [r]$, $v_iv_i^{\top}$ can be written as a linear combination. Thus, there exists scalars that satisfy the following property
    \begin{gather}
        v_iv_i^{\top} = \sum_{j = 1}^{r'} \alpha_{j}v_{i_j}v_{i_j}^{\top} + \sum_{k = 1}^{r''} \beta_{k}(v_{l_k} + v_{m_k})(v_{l_k} + v_{m_k})^{\top}\\
        \forall j,k,\quad \alpha_j, \beta_k > 0, i_j \neq i, l_k < m_k
    \end{gather}
    Now, note that we can write
    \begin{align*}
       \sum_{k = 1}^{r''} \beta_{k}(v_{l_k} + v_{m_k})(v_{l_k} + v_{m_k})^{\top} =  \sum_{k = 1, l_k = i}^{r''} \beta_{k}(v_{l_k} + v_{m_k})v_{l_k}^{\top} + \sum_{k = 1, l_k \neq i}^{r''} \beta_{k}(v_{l_k} + v_{m_k})v_{l_k}^{\top} + \sum_{k = 1}^{r''} \beta_{k}(v_{l_k} + v_{m_k})v_{m_k}^{\top}
    \end{align*}
    But the following sum 
    \begin{align*}
        \sum_{j = 1}^{r'} \alpha_{j}v_{i_j}v_{i_j}^{\top} + \sum_{k = 1, l_k \neq i}^{r''} \beta_{k}(v_{l_k} + v_{m_k})v_{l_k}^{\top} + \sum_{k = 1}^{r''} \beta_{k}(v_{l_k} + v_{m_k})v_{m_k}^{\top}
    \end{align*}
    doesn't span (as column vectors) a subspace that contains the column vector $v_i$ because $\curlybracket{v_i}_{i=1}^r$ is a set of orthogonal vectors. Thus, we can write
    \begin{align}
        v_iv_i^{\top} = \sum_{k = 1, l_k = i}^{r''} \beta_{k}(v_{l_k} + v_{m_k})v_{l_k}^{\top} = \paren{\sum_{k = 1, l_k = i}^{r''} \beta_k v_{l_k} + \sum_{k = 1, l_k = i}^{r''} \beta_k v_{m_k}}v_i^{\top} \label{eq: v1}
    \end{align}
    This implies that 
    \begin{align}
        \sum_{k = 1, l_k = i}^{r''} \beta_k v_{m_k} = 0 \implies \textnormal{ if } l_k = i, \beta_k = 0 \label{eq: v2}
    \end{align}
    Since not all $\beta_k = 0$ corresponding to $l_k = i$ (otherwise $\sum_{k = 1, l_k = i}^{r''} \beta_k v_{l_k} = 0$ ) we have shown that $v_iv_i^{\top}$ can not be written as a linear combination of elements in $\cB \setminus \curlybracket{v_iv_i^\top}$.

    \textbf{Case 2}: Now, we consider the second case where there exists some indices $i,j$ such that $(v_i + v_j)(v_i+v_j)^{\top}$ is a sum of linear combination of elements in $\cB$. Note that this linear combination can't have an element of type $v_kv_k^{\top}$ as it contradicts the first case. So, there are scalars such that
    \begin{gather}
        (v_i + v_j)(v_i+v_j)^{\top} = \sum_{k = 1}^{r''} \beta_{k}(v_{l_k} + v_{m_k})(v_{l_k} + v_{m_k})^{\top}\\
        \forall k,\quad l_k < m_k
    \end{gather}
    But we rewrite this as 
    \begin{align*}
        &(v_i + v_j)v_i^{\top} + (v_i + v_j)v_j^{\top}\\ = &\sum_{k = 1, l_k = i}^{r''} \beta_{k}(v_{i} + v_{m_k})v_{i}^{\top} + \sum_{k = 1, m_k = j}^{r''} \beta_{k}(v_{l_k} + v_{j})v_{j}^{\top} + \sum_{\substack{k = 1, l_k \neq i,\\ m_k \neq j}}^{r''} \beta_{k}(v_{l_k} + v_{m_k})(v_{l_k} + v_{m_k})^{\top}
    \end{align*}
    Note that if $l_k = i$ then the corresponding $m_k \neq j$ and vice versa. Since $\curlybracket{v_i}_{i=1}^r$ are orthogonal, the decomposition above implies
    \begin{gather}
        (v_i + v_j)v_i^{\top} = \sum_{k = 1, l_k = i}^{r''} \beta_{k}(v_{i} + v_{m_k})v_{i}^{\top} \label{eq: vplusv1}\\
        (v_i + v_j)v_j^{\top} =  \sum_{k = 1, m_k = j}^{r''} \beta_{k}(v_{l_k} + v_{j})v_{j}^{\top}\label{eq: vplusv2}\\
        \sum_{\substack{k = 1, l_k \neq i,\\ m_k \neq j}}^{r''} \beta_{k}(v_{l_k} + v_{m_k})(v_{l_k} + v_{m_k})^{\top} = 0
    \end{gather}
    But using the arguments in \eqnref{eq: v1} and \eqnref{eq: v2}, we can achieve \eqnref{eq: vplusv1} or \eqnref{eq: vplusv2}.

    Thus, we have shown that the set of rank-1 matrices as described in $\cB$ are linearly independent.
\end{proof}




In \lemref{lem: orthoset}, we discussed that in order to teach $\pphi^*$ sufficiently agent needs a feedback set of size $\frac{r(r+1)}{2} -1$ if the null set of $\pphi^*$ is known. We can establish this feedback set using the basis shown in \lemref{lem: basis}. We state this result in the following lemma.
\begin{lemma}\label{lem: orthocons}
    For a  given target matrix $\pphi^* = \sum_{i=1}^r \gamma_iv_iv_i^{\top}$ and basis set of matrices $\cB$ as shown in \lemref{lem: basis}, the following set spans a subspace of dimension $\frac{r(r+1)}{2} -1$ in $\symm$. 
\begin{equation*}
\mathcal{O}_{\cB} := \left\{
\begin{aligned}
&v_1v_1^{\top} - \lambda_{11}yy^{\top}, v_2v_2^{\top} - \lambda_{22}yy^{\top}, (v_1 + v_2)(v_1 + v_2)^{\top} - \lambda_{12}yy^{\top}, \ldots,\\
&v_rv_r^{\top} - \lambda_{rr}yy^{\top}, (v_1 + v_r)(v_1 + v_r)^{\top} - \lambda_{1r}yy^{\top}, \ldots, \\
&(v_{r-1} + v_r)(v_{r-1} + v_r)^{\top} - \lambda_{(r-1)r}yy^{\top}
\end{aligned}
\right\}
\end{equation*}

\begin{equation*}
y\pphi^*y^{\top} \neq 0
\end{equation*}

\begin{equation*}
\forall i,j,\quad \lambda_{ii} = \frac{v_i\pphi^*v_i^{\top}}{y\pphi^*y^{\top}}, \quad \lambda_{ij} = \frac{(v_i + v_j)\pphi^*(v_i+ v_j)^{\top}}{y\pphi^*y^{\top}} \quad (i \neq j)
\end{equation*}


\end{lemma}
\begin{proof}
    Since $\pphi^*$ has at least $r$ positive eigenvalues there exists a vector $y \in \reals^p$ such that $y\pphi^*y^{\top} \neq 0$. It is straightforward to note that $\mathcal{O}_{\cB}$ is orthogonal to $\pphi^*$. As $\mathcal{O}_{\cB} \subset \text{span}\langle \cB \rangle$ and $\pphi^* \bot \mathcal{O}_{\cB}$, $\dim(\text{span}\langle \mathcal{O}_{\cB} \rangle) = \frac{r(r+1)}{2} -1$. 
\end{proof}

Now, we will complete the proof of the main result of the appendix here.

\begin{proof}[Proof of \thmref{thm: constructgeneral}]
Combining the results from \lemref{lem: nullset}, \lemref{lem: orthoset}, and \lemref{lem: orthocons}, we conclude that the feedback setup in \eqnref{eq: orthosat} can be effectively decomposed into teaching the null space and the span of the eigenvectors of $\pphi^*$. The constructed feedback sets ensure that $\pphi^*$ is uniquely identified up to a linear scaling factor with optimal sample efficiency.    
\end{proof}

\newpage
%\section{Feature learning with feedbacks: Constructing general activations}
%In this appendix, we will provide the proof of \thmref{thm: constructgeneral}. We prove the result in two parts- lower bound and upper bound on the number of feedbacks. 

\section{Proof of \thmref{thm: constructgeneral}: Lower bound}\label{app: constlb}
In this appendix, we provide the proof of the lower bound as stated in \thmref{thm: constructgeneral}. We proceed by first showing some useful properties on a valid feedback set $\cF(\reals^p,\maha, \pphi^*)$ for a target feature matrix $\pphi^*$. They are stated in \lemref{lem: inclusion} and \lemref{lem: unique}.

First, we consider a basic spanning property of matrices $(xx^\top - yy^\top)$ for any pair $(x,y) \in \cF$ in the space of symmetric matrices $\symm$.

\begin{lemma}\label{lem: inclusion}
    If $\pphi \in \mathcal{O}_{\pphi^*}$ such that $\text{span}\inner{\col{\pphi}} \subset \text{span}\inner{V_{\bracket{r}}}$ then $\pphi \in span \inner{\cF}$.
\end{lemma}
\begin{proof}
     Consider an $\pphi \in \mathcal{O}_{\pphi^*}$ such that $\text{span}\inner{\col{\pphi}} \subset \text{span}\inner{V_{\bracket{r}}}$. Note that the eigendecompostion of $\pphi$ (assume $\rank{\pphi} = r' < r$)
     \begin{align*}
         \pphi = \sum_{i = 1}^{r'} \delta_i\mu_i\mu_i^{\top}
     \end{align*}
     for orthogonal eigenvectors $\curlybracket{\mu_i}_{i=1}^{r'}$ and the corresponding eigenvalues $\curlybracket{\delta_i}_{i=1}^{r'}$ has the property that $span \inner{\curlybracket{\mu_i}_{i=1}^{r'}} \subset \text{span} \inner{V_{\bracket{r}}}$. Using the arguments exactly as shown in the second half of the proof of \lemref{lem: orthoset} we can show there exists $\lambda > 0$ such that $\pphi^* + \lambda \pphi \in \sf{VS}(\cF, \maha)$. But then $\pphi$ is not feature equivalent to $\pphi^*$. But this contradicts the assumption of $\cF$ being a valid feedback set. 
     %But using \lemref{lem: orthoset} and \lemref{lem: orthocons} we know that the dimension of the span of matrices that satisfy the condition in \lemref{lem: inclusion} is at the least $\frac{r(r+1)}{2} -1$. We can use \lemref{lem: orthocons} where $y = \sum_{i = 1}^r v_r$ (note $\pphi^*v \neq 0$). Thus, any basis matrix in $\mathcal{O}_{\cB}$ satisfy the conditions in \lemref{lem: inclusion}.
\end{proof}


\begin{lemma}\label{lem: unique}
    There exists vectors $U_{\bracket{p-r}} \subset \nul{\pphi^*}$ (of size $p - r $) such that $\text{span} \inner{U_{\bracket{p-r}} } = \nul{\pphi^*}$ and 
        for any vector $v \in U_{\bracket{p-r}}$, $vv^{\top} \in \text{span} \inner{\cF}$.
\end{lemma}
\begin{proof}
    Assuming the contrary, there exists $v \in \text{span} \inner{\nul{\pphi^*}}$ such that $vv^{\top} \notin \text{span} \inner{\cF}$.

    Now if $vv^{\top}\, \bot\, \cF$, then for any scalar $\lambda > 0$, $\pphi^* + \lambda vv^{\top}$ is both symmetric and positive semi-definite and satisfies all the conditions in \eqnref{eq: redsol} wrt $\cF$ a contradiction as $\pphi^* + \lambda vv^{\top}$ is not feature equivalent to $\pphi^*$. 
    
    So, consider the case when $vv^{\top}\, \not\perp\, \cF$. Let $\curlybracket{v_{r+1},\ldots,v_{p-1}}$ be an orthogonal extension\footnote{the set is not trivially empty in which case the proof follows easily} of $v$ such that $\curlybracket{v_{r+1},\ldots,v_{p-1}, v}$ forms a basis of $\nul{\pphi^*}$, i.e., in other words 
    \begin{align*}
    v \bot \curlybracket{v_{r+1},\ldots,v_{p-1}}\quad \&\quad \text{span} \inner{\curlybracket{v_{r+1},\ldots,v_{p-1}, v}} = \nul{\pphi^*}.
    \end{align*}
    We will first show that there exists some $\pphi'$ $(\not = \lambda\pphi^*, \text{for some } \lambda > 0)$ $\in \symm$ orthogonal to $\cF$ and furthermore $\curlybracket{v_{r+1},\ldots,v_{p-1}} \subset \nul{\pphi'}$ . 
    
    
    Consider the intersection (in the space $\symm$) of the orthogonal complement of the matrices $\curlybracket{v_{r+1}v_{r+1}^{\top},\ldots,v_{p-1}v_{p-1}^{\top}}$, denote it as $\mathcal{O}_{\sf{rest}}$, i.e.,
    \begin{align*}
        \mathcal{O}_{\sf{rest}} := \bigcap_{i = r+1}^{p-1} \mathcal{O}_{v_iv_i^{\top}} 
    \end{align*}
    Note that %\akash{double check this}
    \begin{align*}
        \dim(\mathcal{O}_{\sf{rest}}) = p(p+1)/2 - p+ r
    \end{align*}
    Since $vv^{\top}$ is in $\mathcal{O}_{\sf{rest}}$ and $\dim(\mathcal{O}_{\sf{rest}}) > 1$ there exists some $\pphi'$ such that $\pphi' \perp \pphi^*$, and also orthogonal to elements in the feedback set $\cF$. Thus, $\pphi'$ has a null set which includes the subset $\curlybracket{v_{r+1},\ldots,v_{p-1}}$. 
    
    Now, the rest of the proof involves showing existence of some scalar $\lambda > 0$ such that $\pphi^* + \lambda \pphi'$ satisfies the conditions of \eqnref{eq: redsol} for the feedback set $\cF$. Note that if $v\pphi'v^{\top} = 0$ then the proof is straightforward as $ \text{span} \inner{\curlybracket{v_{r+1},\ldots,v_{p-1}, v}} \subset \nul{\pphi'}$, which implies $\text{span} \inner{\col{\pphi'}} \subset \text{span} \inner{V_{[r]}}$. But this is precisely the condition for \lemref{lem: inclusion} to hold. 
    
     
     Without loss of generality assume that $v\pphi'v^{\top} > 0$. First note that the eigendecomposition of $\pphi'$ has eigenvectors that are contained in $V_{[r]} \cup \curlybracket{v}$. Consider some arbitrary choice of $\lambda > 0$, we will fix a value later. It is straightforward that $\pphi^* + \lambda \pphi'$ is symmetric for $\pphi^*$ and $\pphi'$ are symmetric. In order to show it is positive semi-definite, it suffices to show that
     \begin{align}
         \forall u \in \reals^p, u^{\top}(\pphi^* + \lambda \pphi') u \ge 0 \label{eq: psd}
     \end{align}
    Since  $\curlybracket{v_{r+1},\ldots, v_{p-1}} \subset \paren{\nul{\pphi^*} \cap \nul{\pphi'}}$ we can simplify \eqnref{eq: psd} to
    \begin{align}
        \forall u \in \text{span}\inner{V_{[r]} \cup \curlybracket{v}}, u^{\top}(\pphi^* + \lambda \pphi') u \ge 0 \label{eq: repsd}
    \end{align}
    Consider the decomposition of any arbitrary vector $u \in \text{span}\inner{V_{[r]} \cup \curlybracket{v}}$ as follows:
    \begin{gather}
        u = u_{[r]} + v', \textnormal{ such that } u_{[r]} \in \text{span}\inner{V_{[r]}}, v' \in \text{span} \inner{\{v\}} \label{eq: decom1}\\
        u_{[r]} := \sum_{i =1}^r \alpha_i v_i,\;\; \forall i\; \alpha_i \in \reals \label{eq: decom2}
    \end{gather}
    From here on we assume that $u_{[r]} \neq 0$. The alternate case is trivial as $v'^{\top}\pphi'v' > 0$.
    
    Now, we write the vectors as scalar multiples of their corresponding unit vectors
    \begin{gather}
        u_{[r]} = \delta_r \cdot \hat{u}_r,\;\; \hat{u}_r := \frac{u_{[r]}}{||u_{[r]}||^2_{V_{[r]}}}, ||u_{[r]}||^2_{V_{[r]}} := \sum_{i =1}^r \alpha_i^2 \label{eq: scale1}\\
        v' = \delta_{v'}\cdot \hat{v},\;\; \hat{v} := \frac{v}{||v||_2^2} \label{eq: scale2}
    \end{gather}
    \underline{\tt{Remark}}: Although we have computed the norm of $ u_{[r]}$  as $||u_{[r]}||^2_{V_{[r]}}$ in the orthonormal basis $V_{[r]}$, note that the norm remains unchanged (same as the $\ell_2$). $\ell_2$ is used for ease of analysis later on.
    
    Using the decomposition in \eqnref{eq: decom1}-(\ref{eq: decom2}), we can write \eqnref{eq: repsd} as
    \begin{align}
        u^{\top}(\pphi^* + \lambda \pphi')u &= (u_{[r]} + v')^{\top}(\pphi^* + \lambda \pphi')(u_{[r]} + v') \nonumber\\
        &= u_{[r]}^{\top} \pphi^*u_{[r]} + \lambda (u_{[r]} + v')^{\top}\pphi'(u_{[r]} + v')\nonumber\\
        & = \delta_r^2 \cdot\hat{u}_r^{\top}\pphi^*\hat{u}_r + \lambda\big( \delta_r^2 \cdot\hat{u}_r^{\top}\pphi'\hat{u}_r + 2 \delta_r \delta_{v'}\cdot \hat{u}_r^{\top} \pphi' \hat{v} + \delta^2_{v'}\cdot \hat{v}^{\top}\pphi'\hat{v} \big) \label{eq: eq1}
    \end{align}
    Since we want $u^{\top}(\pphi^* + \lambda \pphi')u \ge 0$ we can further simplify \eqnref{eq: eq1} as 
    \begin{align}
        \hat{u}_r^{\top}\pphi^*\hat{u}_r + \lambda\paren{ \hat{u}_r^{\top}\pphi'\hat{u}_r + 2 \textcolor{gray}{\frac{\delta_r\delta_{v'}}{\delta_r^2 }} \cdot \hat{u}_r^{\top} \pphi' \hat{v} + \textcolor{gray}{\frac{\delta^2_{v'}}{\delta^2_r}}\cdot \hat{v}^{\top}\pphi'\hat{v} } \underset{?}{\ge} 0 \label{eq: equiv1}\\
        \Longleftrightarrow \underbrace{\hat{u}_r^{\top}\pphi^*\hat{u}_r}_{\textcolor{red}{(1)}} + \lambda\paren{ \underbrace{\hat{u}_r^{\top}\pphi'\hat{u}_r}_{\textcolor{violet}{(3)}} + \underbrace{2 \textcolor{gray}{\xi}\cdot \hat{u}_r^{\top} \pphi' \hat{v} + \textcolor{gray}{\xi^2}\cdot \hat{v}^{\top}\pphi'\hat{v} }_{\textcolor{blue}{(2)}}} \underset{?}{\ge} 0 \label{eq: equiv2}
    \end{align}
    where we have used $\xi = \frac{\delta_{v'}}{\delta_r}$. The next part of the proof we show that $\textcolor{red}{(1)}$ is lower bounded by a positive constant whereas $\textcolor{blue}{(2)}$ is upper bounded by a positive constant and there is a choice of $\lambda$ so that $\textcolor{blue}{(3)}$ is always smaller than $\textcolor{red}{(1)}$.
    
    Considering $\textcolor{red}{(1)}$ we note that $\hat{u}_r$ is a unit vector wrt the orthonormal set of basis $V_{[r]}$. Expanding using the eigendecomposition of \eqnref{eq: target}
    \begin{align*}
        \hat{u}_r^{\top}\pphi^*\hat{u}_r = \sum_{i=1}^r \frac{\alpha^2_i}{\sum_{i=1}^r \alpha_i^2}\cdot \gamma_i \ge \min_i \gamma_i > 0
    \end{align*}
    The last inequality follows as all the eigenvalues in the eigendecompostion are (strictly) positive. Denote this minimum eigenvalue as $\gamma_{\min} := \min_i \gamma_i$.
    
    Considering $\textcolor{blue}{(2)}$ note that only terms that are variable (i.e. could change value) is $\xi$ as $\hat{u}_r^{\top} \pphi' \hat{v}$ is 

    Note that $\hat{v}$ is a fixed vector and $\hat{u}_r$ has a fixed norm (using \eqnref{eq: scale1}-(\ref{eq: scale2})), so $|\hat{u}_r^{\top} \pphi' \hat{v}| \le C$ for some bounded constant $C > 0$ whereas $\hat{v}^{\top}\pphi'\hat{v}$ is already a constant. Now, $|2 \textcolor{gray}{\xi}\cdot \hat{u}_r^{\top} \pphi' \hat{v}|$ exceeds $\textcolor{gray}{\xi^2}\cdot \hat{v}^{\top}\pphi'\hat{v}$ only if
    \begin{align*}
        |2 \textcolor{gray}{\xi}\cdot \hat{u}_r^{\top} \pphi' \hat{v}| \ge |\textcolor{gray}{\xi^2}\cdot \hat{v}^{\top}\pphi'\hat{v}| %\ge |2 \textcolor{gray}{\xi}\cdot \hat{u}_r^{\top} \pphi' \hat{v} + \textcolor{gray}{\xi^2}\cdot \hat{v}^{\top}\pphi'\hat{v}|
        \Longleftrightarrow \frac{|\hat{u}_r^{\top} \pphi' \hat{v}|}{\hat{v}^{\top}\pphi'\hat{v}} \ge \textcolor{gray}{\xi} \implies \frac{C}{\hat{v}^{\top}\pphi'\hat{v}} \ge \textcolor{gray}{\xi}
    \end{align*}
    Rightmost inequality implies that $2 \textcolor{gray}{\xi}\cdot \hat{u}_r^{\top} \pphi' \hat{v} + \textcolor{gray}{\xi^2}\cdot \hat{v}^{\top}\pphi'\hat{v}$ is negative only for an $\textcolor{gray}{\xi}$ bounded from above by a positive constant. But since $\xi$ is non-negative 
    \begin{align*}
        |2 \textcolor{gray}{\xi}\cdot \hat{u}_r^{\top} \pphi' \hat{v} + \textcolor{gray}{\xi^2}\cdot \hat{v}^{\top}\pphi'\hat{v}| \le C' (\textnormal{bounded constant})
    \end{align*}
    Now using an argument similar to the second half of the proof of \lemref{lem: orthoset}, it is straight forward to show that there is a choice of $\lambda' > 0$ so that $\textcolor{violet}{(3)}$ is always smaller than $\textcolor{red}{(1)}$.

    Now, for $\lambda = \frac{\lambda'}{2\lceil C' \rceil \lambda''}$ where $\lambda''$ is chosen so that $\lambda_{\min} \ge \frac{\lambda'}{\lambda''}$, we note that
    \begin{align*}
        \hat{u}_r^{\top}\pphi^*\hat{u}_r + \lambda\paren{ \hat{u}_r^{\top}\pphi'\hat{u}_r + 2 \textcolor{gray}{\xi}\cdot \hat{u}_r^{\top} \pphi' \hat{v} + \textcolor{gray}{\xi^2}\cdot \hat{v}^{\top}\pphi'\hat{v} } \ge \lambda_{\min} + \frac{\lambda'}{2\lceil C' \rceil \lambda''} \hat{u}_r^{\top}\pphi'\hat{u}_r -\frac{\lambda'}{2\lambda''} >  0.
    \end{align*}
    Using the equivalence in \eqnref{eq: eq1}, \eqnref{eq: equiv1} and \eqnref{eq: equiv2}, we have a choice of $\lambda > 0$ such that $u^{\top}(\pphi^* + \lambda \pphi')u \ge 0$ for any arbitrary vector $u \in \text{span}\inner{V_{[r]} \cup \curlybracket{v}}$. Hence, we have achieved the conditions in \eqnref{eq: repsd}, which is the simplification of \eqnref{eq: psd}. This implies that $\pphi^* + \lambda \pphi'$ is positive semi-definite. 
    
    This implies that there doesn't exist a $v \in \text{span} \inner{\nul{\pphi^*}}$ such that $vv^{\top} \notin \text{span} \inner{\cF}$ otherwise the assumption on $\cF$ to be an oblivious feedback set for $\pphi^*$ is violated. Thus, the statement of \lemref{lem: unique} has to hold.
\end{proof}

%\begin{proof}[Proof of \lemref{lemma: lowerbound}]
%\akash{there are a number of things flying here. First, define some of the set of matrices carefully. Second, write down the statements below as lemmas at the start of the supplementary to highlight their use. They are being used at different places. If possible it would be good to include them in the main paper as well.}

    % The key idea of the proof is that any feedback set, say $\cF$ for the oblivious teaching in \eqnref{eq: sol} must have matrices that satisfy the following properties:
    % \begin{enumerate}
    %     \item[\lemref{lem: inclusion}] if $\pphi \in \mathcal{O}_{\pphi^*}$ such that $\text{span}\inner{\col{\pphi}} \subset \text{span}\inner{V_{\bracket{r}}}$ then $\pphi \in \text{span} \inner{\cF}$.
    %     \item[\lemref{lem: unique}] there exists vectors $U_{\bracket{d-r}} \subset \nul{\pphi^*}$ (of size $d - r $) such that $\text{span} \inner{U_{\bracket{d-r}} } = \nul{\pphi^*}$ and 
    %     for any vector $v \in U_{\bracket{d-r}}$, $vv^{\top} \in \text{span} \inner{\cF}$.
    % \end{enumerate}

    \subsection{Proof of lower bound in \thmref{thm: constructgeneral}}
    
    
    In the following, we provide proof of the main statement on the lower bound of the size of a feedback set.
    
    %\begin{proof}[Proof of lower bound in \thmref{thm: constructgeneral}]
    
    If any of the two lemmas (\ref{lem: inclusion}-\ref{lem: unique}) are violated, we can show there exists $\lambda > 0$ and $\pphi$ such that $\pphi^* + \lambda \pphi \in \sf{VS}(\cF,\maha)$. In  order to ensure these statements, the feedback set should have $\paren{\frac{r(r+1)}{2} + (d - r) - 1}$ many elements which proves the lower bound on $\cF$. 
    
    % Now, we argue the necessity of these statements.

    % Consider the first statement. Consider an $\pphi \in \mathcal{O}_{\pphi^*}$ such that $\text{span}\inner{\col{\pphi}} \subset \text{span}\inner{V_{\bracket{r}}}$. Note that the eigendecompostion of $\pphi$ (assume $\rank{\pphi} = r' < r$)
    %  \begin{align*}
    %      \pphi = \sum_{i = 1}^{r'} \delta_i\mu_i\mu_i^{\top}
    %  \end{align*}
    %  for orthogonal eigenvectors $\curlybracket{\mu_i}_{i=1}^{r'}$ and the corresponding eigenvalues $\curlybracket{\delta_i}_{i=1}^{r'}$ has the property that $\text{span} \inner{\curlybracket{\mu_i}_{i=1}^{r'}} \subset \text{span} \inner{V_{\bracket{r}}}$. Using the arguments exactly as shown in the second half of the proof of \lemref{lem: orthoset} we can show there exists $\lambda > 0$ such that $\pphi^* + \lambda \pphi \in \sf{VS}(\cF, \maha)$. But then $\pphi \not\sim_{R_l} \pphi^*$. But this contradicts the assumption on $\cF$ being a valid oblivious feedback set for \eqnref{eq: sol} up to linear scaling relation $\sim_{R_l}$. 
     
     
     But using \lemref{lem: orthoset} and \lemref{lem: orthocons} we know that the dimension of the \text{span} of matrices that satisfy the condition in \lemref{lem: inclusion} is at the least $\frac{r(r+1)}{2} -1$. We can use \lemref{lem: orthocons} where $y = \sum_{i = 1}^r v_r$ (note $\pphi^*v \neq 0$). Thus, any basis matrix in $\mathcal{O}_{\cB}$ satisfy the conditions in \lemref{lem: inclusion}.

    % Now, consider the second statement. Assuming the contrary, there exists $v \in \text{span} \inner{\nul{\pphi^*}}$ such that $vv^{\top} \notin \text{span} \inner{\cF}$.

    % Now if $vv^{\top}\, \bot\, \cF$, then for any scalar $\lambda > 0$, $\pphi^* + \lambda vv^{\top}$ is both symmetric and positive semi-definite and satisfies all the conditions in \eqnref{eq: redsol} wrt $\cF$ a contradiction as $\pphi^* + \lambda vv^{\top} \not\sim_{R_l} \pphi^*$. 
    
    % So, consider the case when $vv^{\top}\, \not\perp\, \cF$. Let $\curlybracket{v_{r+1},\ldots,v_{d-1}}$ be an orthogonal extension of $v$ such that $\curlybracket{v_{r+1},\ldots,v_{d-1}, v}$ forms a basis of $\nul{\pphi^*}$, i.e., in other words 
    % \begin{align*}
    % v \bot \curlybracket{v_{r+1},\ldots,v_{d-1}}\quad \&\quad \text{span} \inner{\curlybracket{v_{r+1},\ldots,v_{d-1}, v}} = \nul{\pphi^*}.
    % \end{align*}
    % We will first show that there exists some $\pphi'$ $(\not \sim_{R_l} \pphi^*)$ $\in \symm$ orthogonal to $\cF$ and furthermore $\curlybracket{v_{r+1},\ldots,v_{d-1}} \subset \nul{\pphi'}$ . 
    
    
    % Consider the intersection (in the space $\symm$) of the orthogonal complement of the matrices $\curlybracket{v_{r+1}v_{r+1}^{\top},\ldots,v_{d-1}v_{d-1}^{\top}}$, denote it as $\mathcal{O}_{rest}$, i.e.,
    % \begin{align*}
    %     \mathcal{O}_{rest} := \bigcap_{i = r+1}^{d-1} \mathcal{O}_{v_iv_i^{\top}} 
    % \end{align*}
    % Note that
    % \begin{align*}
    %     \dim(\mathcal{O}_{rest}) = D - d + r
    % \end{align*}
    % Since $vv^{\top}$ is in $\mathcal{O}_{rest}$ and $\dim(\mathcal{O}_{rest}) > 1$ there exists some $\pphi'$ such that $\pphi' \perp \pphi^*$, and also orthogonal to elements in the feedback set $\cF$. Thus, $\pphi'$ has a null set which includes the subset $\curlybracket{v_{r+1},\ldots,v_{d-1}}$. 
    
    % Now, the rest of the proof involves showing existence of some scalar $\lambda > 0$ such that $\pphi^* + \lambda \pphi'$ satisfies the conditions of \eqnref{eq: redsol} for the feedback set $\cF$. Note that if $v\pphi'v^{\top} = 0$ then the proof is straightforward as $ \text{span} \inner{\curlybracket{v_{r+1},\ldots,v_{d-1}, v}} \subset \nul{\pphi'}$, which implies $\text{span} \inner{\col{\pphi'}} \subset \text{span} \inner{V_{[r]}}$. But this is precisely the condition for \lemref{lem: inclusion} to hold. 
    
     
    %  Without loss of generality assume that $v\pphi'v^{\top} > 0$. First note that the eigendecomposition of $\pphi'$ has eigenvectors that are contained in $V_{[r]} \cup \curlybracket{v}$. Consider some arbitrary choice of $\lambda > 0$, we will fix a value later. It is straightforward that $\pphi^* + \lambda \pphi'$ is symmetric for $\pphi^*$ and $\pphi'$ are symmetric. In order to show it is positive semi-definite, it suffices to show that
    %  \begin{align}
    %      \forall u \in \reals^p, u^{\top}(\pphi^* + \lambda \pphi') u \ge 0 \label{eq: psd}
    %  \end{align}
    % Since  $\curlybracket{v_{r+1},\ldots, v_{d-1}} \subset \paren{\nul{\pphi^*} \cap \nul{\pphi'}}$ we can simplify \eqnref{eq: psd} to
    % \begin{align}
    %     \forall u \in \text{span}\inner{V_{[r]} \cup \curlybracket{v}}, u^{\top}(\pphi^* + \lambda \pphi') u \ge 0 \label{eq: repsd}
    % \end{align}
    % Consider the decomposition of any arbitrary vector $u \in \text{span}\inner{V_{[r]} \cup \curlybracket{v}}$ as follows:
    % \begin{gather}
    %     u = u_{[r]} + v', \textnormal{ such that } u_{[r]} \in \text{span}\inner{V_{[r]}}, v' \in \text{span} \inner{\{v\}} \label{eq: decom1}\\
    %     u_{[r]} := \sum_{i =1}^r \alpha_i v_i,\;\; \forall i\; \alpha_i \in \reals \label{eq: decom2}
    % \end{gather}
    % From here on we assume that $u_{[r]} \neq 0$. The alternate case is trivial as $v'^{\top}\pphi'v' > 0$.
    
    % Now, we write the vectors as scalar multiples of their corresponding unit vectors
    % \begin{gather}
    %     u_{[r]} = \delta_r \cdot \hat{u}_r,\;\; \hat{u}_r := \frac{u_{[r]}}{||u_{[r]}||^2_{V_{[r]}}}, ||u_{[r]}||^2_{V_{[r]}} := \sum_{i =1}^r \alpha_i^2 \label{eq: scale1}\\
    %     v' = \delta_{v'}\cdot \hat{v},\;\; \hat{v} := \frac{v}{||v||_2^2} \label{eq: scale2}
    % \end{gather}
    % \tt{Remark}: Although we have computed the norm of $ u_{[r]}$  as $||u_{[r]}||^2_{V_{[r]}}$ in the orthonormal basis $V_{[r]}$, note that the norm remains unchanged (same as the $\ell_2$). $\ell_2$ is used for ease of analysis later on.
    
    % Using the decomposition in \eqnref{eq: decom1}-(\ref{eq: decom2}), we can write \eqnref{eq: repsd} as
    % \begin{align}
    %     u^{\top}(\pphi^* + \lambda \pphi')u &= (u_{[r]} + v')^{\top}(\pphi^* + \lambda \pphi')(u_{[r]} + v') \nonumber\\
    %     &= u_{[r]}^{\top} \pphi^*u_{[r]} + \lambda (u_{[r]} + v')^{\top}\pphi'(u_{[r]} + v')\nonumber\\
    %     & = \delta_r^2 \cdot\hat{u}_r^{\top}\pphi^*\hat{u}_r + \lambda\big( \delta_r^2 \cdot\hat{u}_r^{\top}\pphi'\hat{u}_r + 2 \delta_r \delta_{v'}\cdot \hat{u}_r^{\top} \pphi' \hat{v} + \delta^2_{v'}\cdot \hat{v}^{\top}\pphi'\hat{v} \big) \label{eq: eq1}
    % \end{align}
    % Since we want $u^{\top}(\pphi^* + \lambda \pphi') \ge 0$ we can further simplify \eqnref{eq: eq1} as 
    % \begin{align}
    %     \hat{u}_r^{\top}\pphi^*\hat{u}_r + \lambda\paren{ \hat{u}_r^{\top}\pphi'\hat{u}_r + 2 \textcolor{gray}{\frac{\delta_r\delta_{v'}}{\delta_r^2 }} \cdot \hat{u}_r^{\top} \pphi' \hat{v} + \textcolor{gray}{\frac{\delta^2_{v'}}{\delta^2_r}}\cdot \hat{v}^{\top}\pphi'\hat{v} } \underset{?}{\ge} 0 \label{eq: equiv1}\\
    %     \Longleftrightarrow \underbrace{\hat{u}_r^{\top}\pphi^*\hat{u}_r}_{\textcolor{red}{(1)}} + \lambda\paren{ \underbrace{\hat{u}_r^{\top}\pphi'\hat{u}_r}_{\textcolor{violet}{(3)}} + \underbrace{2 \textcolor{gray}{\xi}\cdot \hat{u}_r^{\top} \pphi' \hat{v} + \textcolor{gray}{\xi^2}\cdot \hat{v}^{\top}\pphi'\hat{v} }_{\textcolor{blue}{(2)}}} \underset{?}{\ge} 0 \label{eq: equiv2}
    % \end{align}
    % where we have used $\xi = \frac{\delta_{v'}}{\delta_r}$. The next part of the proof we show that $\textcolor{red}{(1)}$ is lower bounded by a positive constant whereas $\textcolor{blue}{(2)}$ is upper bounded by a positive constant and there is a choice of $\lambda$ so that $\textcolor{blue}{(3)}$ is always smaller than $\textcolor{red}{(1)}$.
    
    % Considering $\textcolor{red}{(1)}$ we note that $\hat{u}_r$ is a unit vector wrt the orthonormal set of basis $V_{[r]}$. Expanding using the eigendecomposition of \eqnref{eq: target}
    % \begin{align*}
    %     \hat{u}_r^{\top}\pphi^*\hat{u}_r = \sum_{i=1}^r \frac{\alpha^2_i}{\sum_{i=1}^r \alpha_i^2}\cdot \gamma_i \ge \min_i \gamma_i > 0
    % \end{align*}
    % The last inequality follows as all the eigenvalues in the eigendecompostion are (strictly) positive. Denote this minimum eigenvalue as $\gamma_{\min} := \min_i \gamma_i$.
    
    % Considering $\textcolor{blue}{(2)}$ note that only terms that are variable (i.e. could change value) is $\xi$ as $\hat{u}_r^{\top} \pphi' \hat{v}$ is 

    % Note that $\hat{v}$ is a fixed vector and $\hat{u}_r$ has a fixed norm (using \eqnref{eq: scale1}-(\ref{eq: scale2})), so $|\hat{u}_r^{\top} \pphi' \hat{v}| \le C$ for some bounded constant $C > 0$ whereas $\hat{v}^{\top}\pphi'\hat{v}$ is already a constant. Now, $|2 \textcolor{gray}{\xi}\cdot \hat{u}_r^{\top} \pphi' \hat{v}|$ exceeds $\textcolor{gray}{\xi^2}\cdot \hat{v}^{\top}\pphi'\hat{v}$ only if
    % \begin{align*}
    %     |2 \textcolor{gray}{\xi}\cdot \hat{u}_r^{\top} \pphi' \hat{v}| \ge |\textcolor{gray}{\xi^2}\cdot \hat{v}^{\top}\pphi'\hat{v}| %\ge |2 \textcolor{gray}{\xi}\cdot \hat{u}_r^{\top} \pphi' \hat{v} + \textcolor{gray}{\xi^2}\cdot \hat{v}^{\top}\pphi'\hat{v}|
    %     \Longleftrightarrow \frac{|\hat{u}_r^{\top} \pphi' \hat{v}|}{\hat{v}^{\top}\pphi'\hat{v}} \ge \textcolor{gray}{\xi} \implies \frac{C}{\hat{v}^{\top}\pphi'\hat{v}} \ge \textcolor{gray}{\xi}
    % \end{align*}
    % Rightmost inequality implies that $2 \textcolor{gray}{\xi}\cdot \hat{u}_r^{\top} \pphi' \hat{v} + \textcolor{gray}{\xi^2}\cdot \hat{v}^{\top}\pphi'\hat{v}$ is negative only for an $\textcolor{gray}{\xi}$ bounded from above by a positive constant. But since $\xi$ is non-negative 
    % \begin{align*}
    %     |2 \textcolor{gray}{\xi}\cdot \hat{u}_r^{\top} \pphi' \hat{v} + \textcolor{gray}{\xi^2}\cdot \hat{v}^{\top}\pphi'\hat{v}| \le C' (\textnormal{bounded constant})
    % \end{align*}
    % Now using an argument similar to the second half of the proof of \lemref{lem: orthoset}, it is straight forward to show that there is a choice of $\lambda' > 0$ so that $\textcolor{violet}{(3)}$ is always smaller than $\textcolor{red}{(1)}$.

    % Now, for $\lambda = \frac{\lambda'}{2\lceil C' \rceil \lambda''}$ where $\lambda''$ is chosen so that $\lambda_{\min} \ge \frac{\lambda'}{\lambda''}$, we note that
    % \begin{align*}
    %     \hat{u}_r^{\top}\pphi^*\hat{u}_r + \lambda\paren{ \hat{u}_r^{\top}\pphi'\hat{u}_r + 2 \textcolor{gray}{\xi}\cdot \hat{u}_r^{\top} \pphi' \hat{v} + \textcolor{gray}{\xi^2}\cdot \hat{v}^{\top}\pphi'\hat{v} } \ge \lambda_{\min} + \frac{\lambda'}{2\lceil C' \rceil \lambda''} \hat{u}_r^{\top}\pphi'\hat{u}_r -\frac{\lambda'}{2\lambda''} >  0.
    % \end{align*}
    % Using the equivalence in \eqnref{eq: eq1}, \eqnref{eq: equiv1} and \eqnref{eq: equiv2}, we have a choice of $\lambda > 0$ such that $u^{\top}(\pphi^* + \lambda \pphi')u \ge 0$ for any arbitrary vector $u \in \text{span}\inner{V_{[r]} \cup \curlybracket{v}}$. Hence, we have achieved the conditions in \eqnref{eq: repsd}, which is the simplification of \eqnref{eq: psd}. This implies that $\pphi^* + \lambda \pphi'$ is positive semi-definite. 
    
    % This implies that there doesn't exist a $v \in \text{span} \inner{\nul{\pphi^*}}$ such that $vv^{\top} \notin \text{span} \inner{\cF}$ otherwise the assumption on $\cF$ to be an oblivious feedback set for $\pphi^*$ is violated. Thus, the statement \lemref{lem: unique} has to hold. 
    
    
    Since the dimension of $\nul{\pphi^*}$ is at least $(d-r)$ thus there are at least $(d-r)$ directions or linearly independent matrices (in $\symm$) that need to be spanned by $\cF$.

    Thus, \lemref{lem: inclusion} implies there are $\frac{r(r+1)}{2} -1$ linearly independent matrices (in $\mathcal{O}_{\pphi^*}$) that need to be spanned by $\cF$. Similarly, \lemref{lem: unique} implies there are $p-r$ linearly independent matrices (in $\mathcal{O}_{\pphi^*}$) that need to be spanned by $\cF$. Note that the column vectors of these matrices from the two statements are spanned by orthogonal set of vectors, i.e. one by $V_{[r]}$ and the other by $\nul{\pphi^*}$ respectively. Thus, these $\frac{r(r+1)}{2} -1 + (p-r)$ are linearly independent in $\symm$, but this forces a lower bound on the size of $\cF$ (a lower dimensional span can't contain a set of vectors spanning higher dimensional space). This completes the proof of the lower bound in \thmref{thm: constructgeneral}.

    %\end{proof}

\iffalse

\section{Proof of the Upper Bound in \thmref{thm:constructgeneral}}

We provide the proof of the upper bound stated in \thmref{thm:constructgeneral}.

\subsection{Eigendecomposition of target feature matrix}

Consider the eigendecomposition of the matrix $\pphi^*$. There exists a set of orthonormal vectors $\{v_1, v_2, \ldots, v_r\}$ with corresponding eigenvalues $\{\gamma_1, \gamma_2, \ldots, \gamma_r\}$ such that
\begin{align}
    \pphi^* = \sum_{i=1}^r \gamma_i v_i v_i^{\top} \label{eq:target}
\end{align}
Denote the set of orthogonal vectors $\{v_1, v_2, \ldots, v_r\}$ as $V_{(r)}$.

\subsection{Orthogonal Extension to a Basis}

Extend the set $V_{(r)}$ to an orthonormal basis for $\mathbb{R}^p$ by including vectors $\{v_{r+1}, \ldots, v_p\}$. Let $V_{(p-r)} = \{v_{r+1}, \ldots, v_p\}$, and define the complete basis as
\[
    V_{(d)} = \{v_1, v_2, \ldots, v_p\}.
\]
Note that $\{v_{r+1}, \ldots, v_p\}$ precisely defines the null space of $\pphi^*$:
\[
    \mathcal{N}(\pphi^*) = \text{span}\{v_{r+1}, \ldots, v_p\}.
\]

\subsection{Strategy for Teaching the Null Set and Eigenvectors}

The key idea is to manipulate the null space to satisfy the feedback set condition in \eqnref{eq:orthosat} for the target matrix $\pphi^*$. Since $\pphi^*$ has rank $r \leq d$, the number of degrees of freedom is $\frac{r(r+1)}{2}$. Alternatively, the null space of $\pphi^*$, which has dimension $d - r$, constrains the remaining entries in $\pphi^*$. Leveraging this intuition, the teacher can provide pairs $(y, z) \in \mathcal{X}^2$ to teach the null space and the eigenvectors $\{v_1, v_2, \ldots, v_r\}$ separately. We ensure the optimality of this strategy concerning sample efficiency in the following lemmas.

\subsection{Teaching the Null Space}

Consider the partial feedback set
\[
    \cF_{\text{null}} = \{(0, v_i)\}_{i=r+1}^p.
\]
\begin{lemma}\label{lem:nullset}
    If the teacher provides the set $\cF_{\text{null}}$, then the null space of any positive semidefinite (PSD) symmetric matrix $\pphi'$ that satisfies \eqnref{eq:orthosat} contains the span of $\{v_{r+1}, \ldots, v_p\}$, i.e.,
    \[
        \{v_{r+1}, \ldots, v_p\} \subseteq \mathcal{N}(\pphi').
    \]
\end{lemma}
\begin{proof}
    Let $\pphi' \in \symmp$ be a matrix satisfying \eqnref{eq:orthosat} (notably, $\pphi^*$ satisfies this condition). The equality constraints imposed by $\cF_{\text{null}}$ are
    \[
        \forall (0, v) \in \cF_{\text{null}}, \quad v^{\top} \pphi' v = 0.
    \]
    Since $\{v_{r+1}, \ldots, v_p\}$ are linearly independent vectors, it suffices to show that
    \begin{align}
        v^{\top} \pphi' v = 0 \quad \forall v \in V_{(p-r)} \implies \pphi' v = 0 \quad \forall v \in V_{(d-r)}. \label{eq:lemmain}
    \end{align}
    
    Consider the eigendecomposition of $\pphi'$:
    \[
        \pphi' = \sum_{i=1}^{s} \gamma_i' u_i u_i^{\top},
    \]
    where $\{u_i\}_{i=1}^{s}$ are the eigenvectors and $\{\gamma_i'\}_{i=1}^{s}$ are the corresponding eigenvalues of $\pphi'$. Assume $x \neq 0 \in \mathbb{R}^p$ satisfies
    \[
        x^{\top} \pphi' x = 0.
    \]
    Decompose $x$ as $x = \sum_{i=1}^s a_i u_i + v'$, where $a_i$ are scalars and $v' \perp \{u_i\}_{i=1}^s$. Expanding the quadratic form:
    \begin{align*}
        x^{\top} \pphi' x &= \left(\sum_{i=1}^s a_i u_i + v'\right)^{\top} \pphi' \left(\sum_{i=1}^s a_i u_i + v'\right) \\
        &= \left(\sum_{i=1}^s a_i u_i\right)^{\top} \pphi' \left(\sum_{i=1}^s a_i u_i\right) + v'^{\top} \pphi' \left(\sum_{i=1}^s a_i u_i\right) + \left(\sum_{i=1}^s a_i u_i\right)^{\top} \pphi' v' + v'^{\top} \pphi' v' \\
        &= \sum_{i=1}^s a_i^2 \gamma_i' + 2 v'^{\top} \pphi' \left(\sum_{i=1}^s a_i u_i\right) + v'^{\top} \pphi' v'.
    \end{align*}
    Since $v' \perp \{u_i\}_{i=1}^s$ and $\pphi'$ is symmetric, the cross terms vanish:
    \[
        2 v'^{\top} \pphi' \left(\sum_{i=1}^s a_i u_i\right) = 0 \quad \text{and} \quad v'^{\top} \pphi' v' = 0.
    \]
    Therefore,
    \[
        \sum_{i=1}^s a_i^2 \gamma_i' = 0.
    \]
    Since $\pphi'$ is PSD, $\gamma_i' \geq 0$ for all $i$, implying $a_i = 0$ for all $i$. Consequently,
    \[
        \pphi' x = \pphi' v' = 0.
    \]
    Thus, $x \in \mathcal{N}(\pphi')$, establishing \eqnref{eq:lemmain}.
    
    Therefore, if the teacher provides $\cF_{\text{null}}$, any solution $\pphi'$ to \eqnref{eq:orthosat} must satisfy
    \[
        \mathcal{N}(\pphi') \supseteq \text{span}\{v_{r+1}, \ldots, v_p\}.
    \]
\end{proof}

\subsection{Teaching the Eigenvectors $V_{(r)}$}

Next, we aim to teach the span of the eigenvectors $\mathcal{S}_{\pphi^*} = \text{span}\{v_1, v_2, \ldots, v_r\}$ by specifying the eigendecomposition of $\pphi^*$. We demonstrate that within the version space $\textsf{VS}(\maha, \cF_{\text{null}})$, a feedback set of size at most $\frac{r(r+1)}{2} - 1$ suffices to teach $\pphi^*$ up to a linear scaling relation $\sim_{R_l}$.

Consider reformulating the problem in \eqnref{eq:orthosat} as
\begin{align}
    \forall (y, z) \in \mathcal{F}(\mathcal{X}, \textsf{VS}(\maha, \cF_{\text{null}}), \pphi^*), \quad \pphi \cdot (yy^{\top} - zz^{\top}) = 0 \label{eq:redorthosat}
\end{align}
where the feedback set $\mathcal{F}(\mathcal{X}, \textsf{VS}(\maha, \cF_{\text{null}}), \pphi^*)$ is designed to solve within the restricted space $\textsf{VS}(\maha, \cF_{\text{null}})$.

\begin{lemma}\label{lem:orthoset}
    In the context of \eqnref{eq:redorthosat}, where the null space $\mathcal{N}(\pphi^*)$ of the target matrix $\pphi^*$ is known, the teacher can sufficiently and necessarily identify a feedback set $\mathcal{F}(\mathcal{X}, \textsf{VS}(\maha, \cF_{\text{null}}), \pphi^*)$ of size $\frac{r(r+1)}{2} - 1$ for oblivious teaching up to the linear scaling relation $\sim_{R_l}$.
\end{lemma}
\begin{proof}
    Any solution $\pphi'$ to \eqnref{eq:redorthosat} has its column space spanned exactly by $V_{(r)}$. Specifically, the eigenvectors of $\pphi'$ reside in $\text{span}\{V_{(r)}\}$. Given that $\pphi^*$ has rank $r$, there are $\frac{r(r+1)}{2}$ degrees of freedom corresponding to the entries of $\pphi^*$ that need to be determined.

    Consider the set of indices $\{j_1, j_2, \ldots, j_r\}$ corresponding to the linearly independent columns of $\pphi^*$. Define the set of matrices
    \[
        \mathcal{S} = \left\{\pphi^{(i,j)} \,\bigg|\, i \in [d], \, j \in \{j_1, j_2, \ldots, j_r\}, \, \pphi^{(i,j)}_{i'j'} = \mathds{1}[i' \in \{i,j\}, \, j' \in \{i,j\} \setminus \{i'\}]\right\}.
    \]
    This set $\mathcal{S}$ forms a basis for generating any matrix with independent columns along the specified indices.

    The span of $\mathcal{S}_{\pphi^*}$ induces a subspace of symmetric matrices of dimension $\frac{r(r+1)}{2}$ within the vector space $\textsf{symm}(\mathbb{R}^p)$. Therefore, selecting a feedback set of size $\frac{r(r+1)}{2} - 1$ in the orthogonal complement of $\pphi^*$, denoted as $\mathcal{O}_{\pphi^*}$, suffices to teach $\pphi^*$ up to a positive scaling factor. Specifically, this feedback set can be constructed using the basis established in \lemref{lem:orthobasis}.

    To demonstrate necessity, assume for contradiction that a smaller feedback set $\mathcal{F}_{\text{small}}$ exists. This would imply the existence of a matrix $\pphi'$ in $\textsf{VS}(\maha, \cF_{\text{null}})$ orthogonal to all elements in $\mathcal{F}_{\text{small}}$ but not a scalar multiple of $\pphi^*$. If $\pphi'$ is PSD, it satisfies \eqnref{eq:redorthosat} and contradicts the uniqueness of $\pphi^*$ up to scaling. If $\pphi'$ is not PSD, by \lemref{lem:sum}, there exists a scalar $\lambda > 0$ such that $\pphi^* + \lambda \pphi'$ is PSD, again contradicting the minimality of the feedback set.

    Hence, a feedback set of size exactly $\frac{r(r+1)}{2} - 1$ is both sufficient and necessary for oblivious teaching of $\pphi^*$ up to linear scaling.
\end{proof}

\subsection{Constructing the Feedback Set}

Up to this point, we have established the theoretical foundation for the required feedback set size. We now outline a constructive method to achieve this feedback set.

Consider the following union of rank-1 matrices:
\[
    \mathcal{B} = \left\{
        v_1 v_1^{\top},
        v_2 v_2^{\top}, \, (v_1 + v_2)(v_1 + v_2)^{\top},
        \ldots,
        v_r v_r^{\top}, \, (v_1 + v_r)(v_1 + v_r)^{\top}, \ldots, \, (v_{r-1} + v_r)(v_{r-1} + v_r)^{\top}
    \right\}.
\]
\begin{lemma}\label{lem:orthobasis}
    Let $\{v_i\}_{i=1}^r$ be a set of orthonormal vectors. Then, the set of rank-1 matrices
    \[
        \mathcal{B} = \left\{
            v_1 v_1^{\top},
            v_2 v_2^{\top}, \, (v_1 + v_2)(v_1 + v_2)^{\top},
            \ldots,
            v_r v_r^{\top}, \, (v_1 + v_r)(v_1 + v_r)^{\top}, \ldots, \, (v_{r-1} + v_r)(v_{r-1} + v_r)^{\top}
        \right\}
    \]
    is linearly independent in the vector space $\symm(\mathbb{R}^p)$.
\end{lemma}
\begin{proof}
    We prove the linear independence of $\mathcal{B}$ by contradiction, considering two cases:

    \textbf{Case 1:} Assume that for some $i \in [r]$, the matrix $v_i v_i^{\top}$ can be expressed as a linear combination of other matrices in $\mathcal{B} \setminus \{v_i v_i^{\top}\}$. Then, there exist scalars $\alpha_j$ and $\beta_k$ such that
    \[
        v_i v_i^{\top} = \sum_{j \neq i} \alpha_j v_j v_j^{\top} + \sum_{k \neq i} \beta_k (v_l + v_m)(v_l + v_m)^{\top},
    \]
    where $l$ and $m$ are indices distinct from $i$. Expanding the right-hand side and leveraging the orthonormality of $\{v_i\}$ leads to a contradiction, as the presence of $v_i$ in the basis cannot be accounted for by linear combinations of other orthogonal vectors.

    \textbf{Case 2:} Assume that for some distinct $i, j \in [r]$, the matrix $(v_i + v_j)(v_i + v_j)^{\top}$ can be expressed as a linear combination of other matrices in $\mathcal{B}$. Expanding and simplifying similarly leads to a contradiction due to the orthogonality of the vectors.

    Since neither $v_i v_i^{\top}$ nor $(v_i + v_j)(v_i + v_j)^{\top}$ can be written as linear combinations of other elements in $\mathcal{B}$, the set $\mathcal{B}$ is linearly independent.
\end{proof}

\begin{lemma}\label{lem:orthocons}
    Given the target matrix $\pphi^* = \sum_{i=1}^r \gamma_i v_i v_i^{\top}$ and the basis set of matrices $\mathcal{B}$ as defined in \lemref{lem:orthobasis}, the following set spans a subspace of dimension $\frac{r(r+1)}{2} - 1$ in $\symm(\mathbb{R}^p)$:
    \[
        \mathcal{O}_{\mathcal{B}} = \left\{
            v_1 v_1^{\top} - \lambda_{11} yy^{\top}, \,
            v_2 v_2^{\top} - \lambda_{22} yy^{\top}, \,
            (v_1 + v_2)(v_1 + v_2)^{\top} - \lambda_{12} yy^{\top}, \,
            \ldots, \,
            v_r v_r^{\top} - \lambda_{rr} yy^{\top}, \,
            (v_1 + v_r)(v_1 + v_r)^{\top} - \lambda_{1r} yy^{\top}, \,
            \ldots, \,
            (v_{r-1} + v_r)(v_{r-1} + v_r)^{\top} - \lambda_{(r-1)r} yy^{\top}
        \right\},
    \]
    where $y \pphi^* y^{\top} \neq 0$ and for all $i, j$,
    \[
        \lambda_{ii} = \frac{v_i \pphi^* v_i^{\top}}{y \pphi^* y^{\top}}, \quad \lambda_{ij} = \frac{(v_i + v_j) \pphi^* (v_i + v_j)^{\top}}{y \pphi^* y^{\top}} \quad \text{for } i \neq j.
    \]
\end{lemma}
\begin{proof}
    Since $\pphi^*$ has at least $r$ positive eigenvalues, there exists a vector $y \in \mathbb{R}^p$ such that $y \pphi^* y^{\top} \neq 0$. Observe that each element of $\mathcal{O}_{\mathcal{B}}$ is orthogonal to $\pphi^*$:
    \[
        \pphi^* \cdot (v_i v_i^{\top} - \lambda_{ii} yy^{\top}) = v_i^{\top} \pphi^* v_i - \lambda_{ii} y \pphi^* y^{\top} = 0,
    \]
    and similarly for $(v_i + v_j)(v_i + v_j)^{\top} - \lambda_{ij} yy^{\top}$.

    The set $\mathcal{O}_{\mathcal{B}}$ is a subset of $\text{span}\langle \mathcal{B} \rangle$. Since $\pphi^* \perp \mathcal{O}_{\mathcal{B}}$, the dimension of $\text{span}\langle \mathcal{O}_{\mathcal{B}} \rangle$ is $\frac{r(r+1)}{2} - 1$.
\end{proof}

\subsection{Finalizing the Feedback Set Construction}

Combining the results from \lemref{lem:nullset}, \lemref{lem:orthoset}, and \lemref{lem:orthocons}, we conclude that the feedback setup in \eqnref{eq:orthosat} can be effectively decomposed into teaching the null space and the span of the eigenvectors of $\pphi^*$. The constructed feedback sets ensure that $\pphi^*$ is uniquely identified up to a linear scaling factor with optimal sample efficiency.
\fi
\newpage

\newpage

\iffalse
\section{Feature learning with feedbacks: Constructing sparse activations}
\akash{I think this can be taken off}\akash{if adding them add appendix as well}
In this appendix, we provide the proof of \thmref{thm: constructsparse}.

Idea is there exist a sparse set of activations as shown in \cite{kumar2024learningsmoothdistancefunctions} which can be used here.

Let $U_{ext}$ consist of $p(p+1)/2$ unit vectors in $\reals^p$, as follows:
$$ U_{ext} = \{e_i: 1 \leq i \leq p\} \cup \{(e_i + e_j)/\sqrt{2}: 1 \leq i < j \leq p \} .$$
Here $e_i$ is the $i$th standard basis vector.
\fi
% \section{Learning in sampling case}
% \akash{How does sparsity play a role here? Intuitively, you want to pick points close to nodes of a unit cube. \\
% Question is what is the complexity and probabilistic guarantee you can give in this case? }

\section{Proof of \thmref{thm: samplegeneral}: General Activations Sampling}\label{app: samplegeneral}

%\begin{proof}[Proof of \thmref{thm: samplegeneral}]%\propref{prop: sampling}]
    
    %First, we provide the proof of the case when the agent receives representations sampled from a general Lebesgue distribution.
    \iffalse
    Fix a positive index $P = \frac{p(p+1)}{2}$; the agent receives $P$ representations: $\cV_n := \{v_1,v_2,\ldots, v_{P}\} \sim \cD_{\cV}$. Denote for each $i$, $V_i = v_iv_i^{\top}$.
    
    Now, consider the matrix $\mathbb{M} = \bracket{ V_1 \quad V_2 \quad \ldots V_P}$
       %\begin{align*}
    %\mathbb{M} = \begin{bmatrix}
  %\Big\lvert & \Big\lvert& \ldots & \Big\lvert \\
  %V_1 & V_2 & \vdots & V_D\\
  %\Big\lvert& \Big\lvert& \ldots & \Big\lvert
%\end{bmatrix}
%\end{align*}
where we treat each matrix $V_i$ as a column vector in $\reals^{p(p+1)/2}$. We use the following vectorization operation: for a matrix $A \in \symm$ and for all $1 \le i,j \le p$
\begin{align*}
    \text{vec}(A)_i &= A_{ii}, \\
    \text{vec}(A)_{ij} &= A_{ij} + A_{ji}.
\end{align*}
Note that the zero set of $\curlybracket{\det(\mathbb{M}) = 0}$ has measure zero in $\cV^P$, i.e.
\begin{align*}
    \cP_{\cV_n \sim \cD_{\cV}}(\det(\mathbb{M}) = 0) = 0,
\end{align*}
as $\det(\mathbb{M})$ is a (identically) non-zero polynomial over random vectors $v_1,v_2,\ldots, v_{D}$. This implies that
\begin{align}
    \cP_{\cV_n}\paren{\curlybracket{v_iv_i^{\top}} \textnormal{ is linearly independent in } \symm} = 1 \label{eq: fullprob}
\end{align}
Assume $\Sigma^* \neq 0$ be an arbitrary target feature matrix for learning with feedback in \algoref{alg: randmaha}. WLOG assume that $v := v_1 \neq 0$. Consider the following set $\cF$ of rescaled pairs
\begin{align*}
    \cF = \curlybracket{(v, \sqrt{\gamma_{i}}v_i)\,\, |\,\, \Sigma^* \idot (vv^{\top} - \gamma_{i}v_iv_i^{\top}) = 0, \sqrt{\gamma_{i}} > 0}
    %\forall i \in \curlybracket{2,\ldots, P} (v_1, \sqrt{\gamma_{1i}}v_i) \in \cF, \quad \Sigma^* \idot (v_1v_1^{\top} - \gamma_{1i}v_iv_i^{\top}) = 0, \sqrt{\gamma_{1i}} > 0.
\end{align*}
Note that $|\cF| = P - 1$. Now, we show that the elements of $\cF$ are linearly independent in $\symm$. Assuming the contrary, there exists scalars $\curlybracket{a_i}$ (not all zero) such that
\begin{align*}
    \sum_{i = 2}^P a_i (vv^{\top} - \gamma_{i}v_iv_i^{\top}) = 0 \,(\in \symm) 
    \implies \paren{\sum_{i = 2}^P a_i} vv^{\top} = \sum_{i = 2}^P a_i \gamma_{i}v_iv_i^{\top}
\end{align*}
But since $\curlybracket{v_iv_i^{\top}}$ are linearly independent $\sum_{i = 2}^P a_i$ and $\curlybracket{a_i\gamma_i}$ are necessarily zero which implies $a_i = 0$ (for all $i$) as $\gamma_{i} > 0$. This contradicts the assumption on the dependence of elements in $\cF$. 

This implies that $\cF$ induces a set of linearly independent matrices, i.e., $\curlybracket{vv^{\top} - \gamma_{i}v_iv_i^{\top}}$ in the orthogonal complement $\mathcal{O}_{\Sigma^*}$. But then $\Sigma^*$ only has $P$ many degrees of freedom. Thus, any matrix $\Sigma' \in \symm$ that satisfies the equations:
\begin{align*}
    \forall i \in \curlybracket{2,\ldots, D} (x_1, \sqrt{\gamma_{1i}}x_i) \in \cF,\quad \Sigma' \idot (v_1v_1^{\top} - \gamma_{i}v_iv_i^{\top}) = 0
\end{align*}
is at most a positive linear scaling of $\Sigma^*$. Now, using \eqnref{eq: fullprob} we know that
\begin{align*}
    \cP_{\cV_n}\paren{ \cF \textnormal{ is a valid feedback set}} = 1
\end{align*}
Since $\Sigma^*$ was picked arbitrarily the worst-case bound on feedback complexity is upper bounded by $ P - 1$ for $1$-accuracy.

Now, we show the proof for the lower bound.

In the proof of the lower bound of \thmref{thm: constructgeneral}, we prove \lemref{lem: inclusion} which stipulates a necessary property of a feedback set $\cF$ in \algoref{alg: main} with pairs (we assume equality constraint). We state it again for better clarity: given any target matrix $\Sigma^* \in \symm$, if $\Sigma \in \mathcal{O}_{\Sigma^*}$ such that $\text{span}\inner{\col{\Sigma}} \subset \text{span} \inner{Z_{\bracket{r}}}$ then $\Sigma \in span \inner{\cF}$ where $Z_{[r]}$ ($r \le d$) is defined as the set of eigenvectors in the eigendecompostion of $\Sigma^*$ (see \eqnref{eq: target}). 
    %If the rank of $M^*$, then the lower bound follows from the deterministic setting. If the rank is strictly less than $d$, then using \lemref{lem: uniquevec} 
    
This observation implies there exists a matrix $\Sigma' \in \symm$ that needs to be spanned by any feedback set $\cF(\cV_n, {\Sigma^*})$ in \algoref{alg: randmaha} for oblivious learning. We argue that it requires $\frac{p(p+1)}{2}$ sized random representations $\cV_n \sim \cD_{\cV}$, to construct a feedback set that spans any such matrix $\Sigma'$.

Fixing some postive index $\ell > 0$, the agent receives $\ell$ representations: $v_1,v_2,\ldots, v_{\ell} \sim \cD_{\cV}$. Denote for each $i$, $V_i = v_iv_i^{\top}$. Now, consider the matrix $\mathbb{M} = \bracket{ \Sigma' \quad V_1 \quad \ldots V_\ell}$
%        \begin{align*}
%    \mathbb{M} = \begin{bmatrix}
%  \Big\lvert & \Big\lvert& \ldots & \Big\lvert \\
%  \Sigma' & X_1 & \vdots & X_\ell\\
%  \Big\lvert& \Big\lvert& \ldots & \Big\lvert
%\end{bmatrix}
%\end{align*}
where we treat each matrix $V_i$ and $\Sigma'$ as column vectors in $\reals^{p^2}$. Now, consider the polynomial equation $\det(\mathbb{M}) = 0$. Since every entry of $\mathbb{M}$ is semantically different, the determinant $\det(\mathbb{M})$ is a non-zero polynomial. Note that there are $\frac{p(p+1)}{2}$ many degrees of freedom for the rows. Thus, it is clear that the zero set $\curlybracket{\det(\mathbb{M}) = 0}$ has Lebesgue measure zero if $\ell < \frac{p(p+1)}{2}$, i.e. $\mathbb{M}$ requires at least $\frac{p(p+1)}{2}$ columns for $\det(\mathbb{M})$ to be identically zero. But this implies that set $\curlybracket{v_iv_i^{\top}}_{i=1}^\ell$ can't span $\Sigma'$ (almost surely) if $\ell \le \frac{p(p+1)}{2} - 1$.
Hence, (almost surely) the agent can't devise a feedback set for oblivious learning in \algoref{alg: randmaha}.
%must have at least $\frac{d(d+1)}{2} - 1$ pairs. 
In other words, if $\ell \le \frac{p(p+1)}{2} - 1$,
\begin{align*}
    \cP_{\cV_\ell}\paren{ \textnormal{agent devises} \textnormal{ a feedback set } \cF \textnormal{ up to feature equivalence}} = 0
\end{align*}
But this shows that one cannot achieve non-zero accuracy over any random set of representations $\cV_{\ell}$ if $\ell = \Omega(\frac{p(p+1)}{2})$ which proves the claim of the lower bound.
\fi
We aim to establish both upper and lower bounds on the feedback complexity for oblivious learning in Algorithm~\ref{alg: randmaha}. The proof revolves around the linear independence of certain symmetric matrices derived from random representations and the dimensionality required to span a target feature matrix.

Let us define a positive index \( P = \frac{p(p+1)}{2} \). The agent receives \( P \) representations:
\[
\mathcal{V}_n := \{v_1, v_2, \ldots, v_P\} \sim \mathcal{D}_{\mathcal{V}}.
\]
For each \( i \), we define the symmetric matrix \( V_i = v_i v_i^{\top} \).

Consider the matrix \( \mathbb{M} \) formed by concatenating the vectorized \( V_i \):
\[
\mathbb{M} = \begin{bmatrix} \text{vec}(V_1) & \text{vec}(V_2) & \cdots & \text{vec}(V_P) \end{bmatrix},
\]
where each \( \text{vec}(V_i) \) is treated as a column vector in \( \mathbb{R}^{P} \). The vectorization operation for a symmetric matrix \( A \in \symm \) is defined as:
\[
\text{vec}(A)_k =
\begin{cases}
A_{ii} & \text{if } k \text{ corresponds to } (i,i), \\
A_{ij} + A_{ji} & \text{if } k \text{ corresponds to } (i,j),\ i < j.
\end{cases}
\]

The determinant \( \det(\mathbb{M}) \) is a non-zero polynomial in the entries of \( v_1, v_2, \ldots, v_P \). Since the vectors \( v_i \) are drawn from a continuous distribution \( \mathcal{D}_{\mathcal{V}} \), using Sard's theorem the probability that \( \det(\mathbb{M}) = 0 \) is zero, i.e.,
\[
\mathcal{P}_{\mathcal{V}_n}(\det(\mathbb{M}) = 0) = 0.
\]
This implies that, with probability 1, the set \( \{V_1, V_2, \ldots, V_P\} \) is linearly independent in \( \symm \):
\begin{align}
\mathcal{P}_{\mathcal{V}_n}\left(\{v_i v_i^{\top}\} \text{ is linearly independent in } \symm\right) = 1. \label{eq: independence}
\end{align}


Next, let \( \Sigma^* \neq 0 \) be an arbitrary target feature matrix for learning with feedback in Algorithm~\ref{alg: randmaha}. Without loss of generality, assume \( v := v_1 \neq 0 \). Define the set \( \mathcal{F} \) of rescaled pairs as:
\[
\mathcal{F} = \left\{ \left(v, \sqrt{\gamma_i} v_i\right) \,\bigg|\, \Sigma^* \cdot \left(v v^{\top} - \gamma_i v_i v_i^{\top}\right) = 0, \ \sqrt{\gamma_i} > 0 \right\},
\]
noting that \( |\mathcal{F}| = P - 1 \).

Assume, for contradiction, that the elements of \( \mathcal{F} \) are linearly dependent in \( \symm \). Then, there exist scalars \( \{a_i\} \) (not all zero) such that:
\[
\sum_{i=2}^P a_i \left(v v^{\top} - \gamma_i v_i v_i^{\top}\right) = 0 \quad \Rightarrow \quad \left(\sum_{i=2}^P a_i\right) v v^{\top} = \sum_{i=2}^P a_i \gamma_i v_i v_i^{\top}.
\]
However, since \( \{v_i v_i^{\top}\} \) are linearly independent with probability 1, it must be that:
\[
\sum_{i=2}^P a_i = 0 \quad \text{and} \quad a_i \gamma_i = 0 \quad \forall i.
\]
Given that \( \gamma_i > 0 \), this implies \( a_i = 0 \) for all \( i \), contradicting the assumption of linear dependence. Therefore, matrices induced by \( \mathcal{F} \) are linearly independent.

This implies that $\cF$ induces a set of linearly independent matrices, i.e., $\curlybracket{vv^{\top} - \gamma_{i}v_iv_i^{\top}}$ in the orthogonal complement $\mathcal{O}_{\Sigma^*}$, and since \( \Sigma^* \) has at most \( P \) degrees of freedom, any matrix \( \Sigma' \in \symm \) satisfying:
\[
\Sigma' \cdot \left(v v^{\top} - \gamma_i v_i v_i^{\top}\right) = 0 \quad \forall i
\]
must be a positive scalar multiple of \( \Sigma^* \).

Thus, using \eqnref{eq: independence},  with probability 1, the feedback set \( \mathcal{F} \) is valid:
\[
\mathcal{P}_{\mathcal{V}_n}\left(\mathcal{F} \text{ is a valid feedback set}\right) = 1.
\]
Since \( \Sigma^* \) was arbitrary, the worst-case feedback complexity is almost surely upper bounded by \( P - 1 \) for achieving feature equivalence.

For the lower bound, consider the proof of the lower bound in Theorem~\ref{thm: constructgeneral}, specifically Lemma~\ref{lem: inclusion}, which asserts that for any feedback set \( \mathcal{F} \) in Algorithm~\ref{alg: main}, given any target matrix $\Sigma^* \in \symm$, if $\Sigma \in \mathcal{O}_{\Sigma^*}$ such that $\text{span}\inner{\col{\Sigma}} \subset \text{span} \inner{Z_{\bracket{r}}}$ then $\Sigma \in \text{span} \inner{\cF}$ where $Z_{[r]}$ ($r \le d$) is defined as the set of eigenvectors in the eigendecompostion of $\Sigma^*$ (see \eqnref{eq: target}).%if \( \Sigma \in \mathcal{O}_{\Sigma^*} \) and \( \text{span}\{\Sigma\} \subset \text{span}\{Z_{[r]}\} \), then \( \Sigma \in \text{span}\{\mathcal{F}\} \), where \( Z_{[r]} \) are the eigenvectors from the eigendecomposition of \( \Sigma^* \).

This implies that any feedback set \( \mathcal{F}(\mathcal{V}_n, \Sigma^*) \) must span certain matrices \( \Sigma' \in \symm \). Suppose the agent receives \( \ell \) representations \( v_1, v_2, \ldots, v_{\ell} \sim \mathcal{D}_{\mathcal{V}} \) and constructs:
\[
\mathbb{M} = \begin{bmatrix} \text{vec}(\Sigma') & \text{vec}(V_1) & \cdots & \text{vec}(V_\ell) \end{bmatrix}.
\]
Now, consider the polynomial equation $\det(\mathbb{M}) = 0$. Since every entry of $\mathbb{M}$ is semantically different, the determinant $\det(\mathbb{M})$ is a non-zero polynomial. Note that there are $\frac{p(p+1)}{2}$ many degrees of freedom for the rows. Thus, it is clear that the zero set $\curlybracket{\det(\mathbb{M}) = 0}$ has Lebesgue measure zero if $\ell < \frac{p(p+1)}{2}$, i.e. $\mathbb{M}$ requires at least $\frac{p(p+1)}{2}$ columns for $\det(\mathbb{M})$ to be identically zero. But this implies that set $\curlybracket{v_iv_i^{\top}}_{i=1}^\ell$ can't span $\Sigma'$ (almost surely) if $\ell \le \frac{p(p+1)}{2} - 1$.
Hence, (almost surely) the agent can't devise a feedback set for oblivious learning in \algoref{alg: randmaha}.
%must have at least $\frac{d(d+1)}{2} - 1$ pairs. 
In other words, if $\ell \le \frac{p(p+1)}{2} - 1$,
\begin{align*}
    \cP_{\cV_\ell}\paren{ \textnormal{agent devises} \textnormal{ a feedback set } \cF \textnormal{ up to feature equivalence}} = 0
\end{align*}
%The determinant \( \det(\mathbb{M}) \) is a non-zero polynomial. Since each entry in \( \mathbb{M} \) is distinct, \( \det(\mathbb{M}) \) is not identically zero. The zero set \( \{\det(\mathbb{M}) = 0\} \) has Lebesgue measure zero provided \( \ell < \frac{p(p+1)}{2} \). Therefore, with probability 1, \( \mathbb{M} \) has full rank only if \( \ell \geq \frac{p(p+1)}{2} \).
Hence, to span \( \Sigma' \), it almost surely requires at least \( \frac{p(p+1)}{2} \) representations. Therefore, the feedback complexity cannot be lower than \( \Omega\left(\frac{p(p+1)}{2}\right) \).

Combining the upper and lower bounds, we conclude that the feedback complexity for oblivious learning in Algorithm~\ref{alg: randmaha} is tightly bounded by \( \Theta\left(\frac{p(p+1)}{2}\right) \).

%Now, using Sard's theorem it is clear that if $r < \frac{d(d+1)}{2} - 1$ then the measure of the samples for which $\det(\mathbb{M}) = 0$ is zero and the lower bound holds.
%\end{proof}

\section{Proof of \thmref{thm: samplingsparse}: Sparse Activations Sampling}\label{app: samplesparse}

%\begin{proof}[Proof of \thmref{thm: samplingsparse}]
    Here we consider the analysis for the case when the activations $\cV$ are sampled from the sparse distribution as stated in \defref{def: sparse}.

    In \thmref{thm: samplingsparse}, we assume that the activations are sampled from a Lebesgue distribution. This, sufficiently, ensures that (almost surely) any random sampling of $P$ activations induces a set of linearly independent rank-1 matrices. Since the distribution in \assref{ass: sparse} is not a Lebesgue distribution over the entire support $\bracket{0,1}$, requiring an understanding of certain events of the sampling of activations which could lead to linearly independent rank-1 matrices.

    In the proof of \thmref{thm: constructsparse}, we used a set of sparse activations using the standard basis of the vector space $\reals^p$. We note that the idea could be generalized to arbitrary choice of scalars as well, i.e.,
    \begin{align*}
         U_g = \{\lambda_i e_i: \lambda_i \neq 0, 1 \leq i \leq p\} \cup \{(\lambda_{iji} e_i + \lambda_{ijj}e_j): \lambda_{iji},\lambda_{ijj} \neq 0,  1 \leq i < j \leq p \}.
    \end{align*}
    Here $e_i$ is the $i$th standard basis vector. Note that the corresponding set of rank-1 matrices, denoted as $\widehat{U}_g$ 
    \begin{align*}
        \widehat{U}_g = \curlybracket{\lambda_i^2 e_ie_i^T: 1 \leq i \leq p} \cup \curlybracket{(\lambda_{iji} e_i + \lambda_{ijj}e_j)(\lambda_{iji} e_i + \lambda_{ijj}e_j)^T: 1 \leq i < j \leq p}
    \end{align*}
    is linearly independent in the space of symmetric matrices on $\reals^p$, i.e., $\symm$.

    Assume that activations are sampled $P$ times, denoted as $\cV_P$. Now, consider the design matrix $\mathbb{M} = \bracket{ V_1 \quad V_2 \quad \ldots V_P}$ as shown in the proof of \thmref{thm: samplegeneral}. We know that if $\det(\mathbb{M})$ is non-zero then $\curlybracket{V_i}'s$ are linearly independent in $\symm$. To show if a sampled set $\cV_P$ exhibits this property we need to show that $\det(\mathbb{M})$ is not identically zero, which could be possible for activations sampled from sparse distributions as stated in \assref{ass: sparse}, i.e. $\cP_{v \sim \cD_{\sf{sparse}}}( v_i \neq 0) > 0$.

    
    
    Note that $\det(\mathbb{M}) = \sum_{\sigma \in \sf{P}_P} \prod_i \mathbb{M}_{i \sigma(i)}$. Consider the diagonal of $\mathbb{M}$. Consider the situation where all the entries are non-zero. This corresponds to sampling a set of activations of the form $\widehat{U}_g$. Consider the following random design matrix $\mathbb{M}$.

    \[
\mathbb{M} = \begin{bmatrix}
\lambda_1^2 & \cdot & \cdot & \cdots & \cdot &\lambda_{121}^2 & \cdots & \cdot\\
\cdot & \lambda_2^2 & \cdot & \cdots & \cdot &\lambda_{122}^2 & \cdots & \vdots\\
\cdot & \cdot & \lambda_3^2 & \cdots & \cdot &\cdot & \cdots & \lambda_{(p-1)p(p-1)}^2\\
\vdots & \cdots & \cdots & \cdots & \lambda_p^2 & \cdots & \cdots & \lambda_{(p-1)pp}^2\\
\vdots & \cdots & \cdots & \cdots & \cdot & \lambda_{121}\lambda_{122} & \cdots & \cdots\\
\vdots & \cdots & \cdots & \cdots & \cdot & \cdot & \cdots & \cdots\\
\cdot & \cdots & \cdots & \cdots & \cdot & \cdot & \cdot & \lambda_{(p-1)p(p-1)}\lambda_{(p-1)pp}\\
\end{bmatrix}.
\]
Now, a random design matrix $\mathbb{M}$ is not identically zero for any set of $P$ randomly sampled activations that satisfy the following indexing property:
    \begin{align}
        \cR := \{v: v_i \neq 0, 1 \leq i \leq p\} \cup \{v: v_i,v_j \neq 0,  1 \leq i < j \leq p \}. \label{eq: random}
    \end{align}
    This is so because for identity permutation, we have $\prod_{i} \mathbb{M}_{ii} \neq 0$.
    Now, we will compute the probability that $\cR$ is sampled from $\cD_{\sf{sparse}}$. Using the independence of sampling of each index of an activation, the probabilities for the two subsets of $\cR$ can be computed as follows:
    \begin{itemize}
        \item $p$ activations $\curlybracket{\alpha_1, \alpha_2,\cdots, \alpha_p} \sim \cD_{\sf{sparse}}^p$ such that $\alpha_{ii} \neq 0$. Using independence, we have $$\cP_1 = \sum_{i = 0}^{s-1} \binom{p-1}{i} p_{nz}^{i+1} (1 - p_{nz})^{p-1-i},$$
        \item Rest of $p(p-1)/2$ activations of $\cR$ in \eqnref{eq: random} require at least two indices to be non-zero. This could be computed as $$\cP_2 = \sum_{i = 0}^{s-2} \binom{p-2}{i} p_{nz}^{i+2} (1 - p_{nz})^{p-2-i}.$$
    \end{itemize}
    Now, note that these $P$ activations can be permuted in $P!$ ways and thus
    \begin{align}
        \cP_{\cV_p}(\cV_P \equiv \cR) \ge P!\cdot\cP_1 \cdot \cP_2 = \underbrace{\textcolor{black}{P!\cdot \paren{\sum_{i = 0}^{s-1} \binom{p-1}{i} p_{nz}^{i+1} (1 - p_{nz})^{p-1-i}} \paren{\sum_{i = 0}^{s-2} \binom{p-2}{i} p_{nz}^{i+2} (1 - p_{nz})^{p-2-i}}}}_{p_{\sf{s}}}\label{eq: lb}
    \end{align}
    Now, we will complete the proof of the theorem using Hoeffding's inequality. Assume that the agent samples $N$ activations, we will compute the probability that $\cR \subset \cV_N$. Consider all possible $P$-subsets of $N$ items, enumerated as $\curlybracket{1,2,\ldots, \binom{N}{P}}$. Now, define random variables $X_i$ as
    \begin{align*}
        X_i = \begin{cases}
            1 \textnormal{ if $i$th subset equals } \cR,\\
            0 \textnormal{ o.w.}
        \end{cases}
    \end{align*}
    Now, define sum random variable $X = \sum_i^{\binom{N}{P}} X_i$. We want to understand the probability $\cP_{\cV_N}(X \ge 1)$. Now note that,
    \begin{align*}
        \expctover{\cV_N \sim \cD_{\sf{sparse}}}{X} = \sum_i \expct{X_i} = \binom{N}{P}\cdot \cP_{\cV_P}(\cV_P \equiv \cR)
    \end{align*}
    Now, using Hoeffding's inequality
    \begin{align*}
        \cP_{\cV_N}(X > 0) \ge 1 - 2\exp^{-2\expct{X}^2} \ge 1 - 2\exp^{-2\binom{N}{P}^2 p_{\sf{s}}^2}
    \end{align*}
    Now, for a given choice of of $\delta > 0$, we want $\delta \ge 2\exp^{-2\binom{N}{P}^2 p_{\sf{s}}^2}$. Using Sterling's approximation
    \begin{align*}
          \binom{N}{P} \ge \frac{1}{p_{\sf{s}}} \sqrt{\log \frac{4}{\delta^2}}
        \implies \paren{\frac{eN}{P}}^P \ge \frac{1}{p_{\sf{s}}} \sqrt{\log \frac{4}{\delta^2}} \implies N \ge \frac{P}{e} \paren{\frac{1}{p_{\sf{s}}^2} \log \frac{4}{\delta^2}}^{1/2P}
    \end{align*}

\newpage














    \iffalse
    ----------------------------\\
    We assume that the agent receives representations from a joint Beta distribution over every index of a random activation $\alpha \in \cV \subseteq \reals^p$:
    \begin{align*}
        \forall i,\quad    \textbf{Beta}(v_i; \mu_i, \nu_i) = v_i^{\mu_i -1}(1 - v_i)^{\nu_i-1}
    \end{align*}
    
\begin{enumerate}
    \item Count the number of directions such that only $ps$ many indices are large, i.e. close to 1 rest close to zero. There are $2^{ps}$ such directions. If we are sampling activations what is the probability that only $ps$ many indices are large but others close to zero. Now, how many times do we have to simply activations so that at least $\frac{p(p+1)}{2}$ many sparse activations are available?
\end{enumerate}
   \fi 
\section{Additional Experiments}\label{app: additional}
We evaluate our methods on large-scale models: Pythia-70M~\citep{pythia} and Board Games Models~\citep{karvonen2024measuring}. These models present significant computational challenges, as different feedback methods generate millions of constraints. This scale necessitates specialized approaches for both memory management and computational efficiency.

\paragraph{Memory-efficient constraint storage} The high dimensionality of model dictionaries makes storing complete activation indices for each feature prohibitively memory-intensive. We address this by enforcing constant sparsity constraints, limiting activations to a maximum sparsity of 3. This constraint enables efficient storage of large-dimensional arrays while preserving the essential characteristics of the features.
\paragraph{Computational optimization} To efficiently handle constraint satisfaction at scale, we reformulate the problem as a matrix regression task, as detailed in \figref{fig:gradient}. The learner maintains a low-rank decomposition of the feature matrix $\pphi$, assuming $\pphi = UU^\top$, where $U$ represents the learned dictionary. This formulation allows for efficient batch-wise optimization over the constraint set while maintaining feasible memory requirements.
% In this section, we provide experiments on large scale models: Pythia-70M~\cite{pythia} and Board Games Models~\cite{karvonen2024measuring}. For these models, the number of constraints generated for different feedback methods scale in millions, we can not explicitly store all the constraints or fit to them directly. We propose the following schemes to resolve these computational and memory issues.

% \paragraph{Memory bottleneck of storing constraints:} Since the dimension of the dictionary could be very high, strong all the indices of an activation used for feature is memory inefficient., which could be handled with constant sparsity constraints. In our experiments, we assume that sparsity is at most 3. THis allows to store numpy arrays of large dimensions. 

% \paragraph{Computational issues:} In order to solve the problem constraints satisfaction, we exploy a matrix version of regression with constraints as samples as shown in \figref{fig:gradient}. Learner maintains a decomposition $U$ of the feature matrix $\pphi$ assuming $\pphi = UU^\top$.


\paragraph{Dictionary features of Pythia-70M} We use the publicly available repository for dictionary learning via sparse autoencoders on neural network activations~\citep{marks2024dictionarylearning}. We consider the dictionaries trained for Pythia-70M~\citep{pythia} (a general-purpose LLM trained on publicly available datasets). We retrieve the corresponding autoencoders for attention output layers which have dimensions $32768 \times 512$. Note that $p(p+1)/2 \approx, 512M$.
For the experiments, we use $3$-sparsity on uniform sparse distributions. We present the plots for ChessGPT in two parts in \figref{fig: subsample} and \figref{fig: pythiasample} for different feedback methods.


\newpage

\begin{figure}[h!]
\centering
    \begin{minipage}{0.99
    \textwidth}  % Controls width of the content
    \begin{algorithm}[H]
    \small  % or \footnotesize for smaller text
    \caption{Optimization via Gradient Descent}
    \label{alg:gradient}
    \begin{enumerate}
        \item Given a dictionary $U \in \mathbb{R}^{p \times r}$, minimize the loss $\mathcal{L}(U)$:
        \begin{equation}
            \mathcal{L}(U) = \mathcal{L}_{\text{MSE}}(U) + \mathcal{L}_{\text{reg}}(U)
        \end{equation}
        where MSE loss is:
        \begin{equation}
            \mathcal{L}_{\text{MSE}}(U) = \frac{1}{|B|}\sum_{i \in B} (\|U^\top u_i\|^2 - c_i\|U^\top y\|^2)^2
        \end{equation}
        and regularization term is:
        \begin{equation}
            \mathcal{L}_{\text{reg}}(U) = \lambda\|U\|_F^2
        \end{equation}
        
        \item For each batch containing indices $i$, values $v$, and targets $c$:
            \begin{enumerate}
                \item Construct sparse vectors $u_i$ using $(i,v)$ pairs
                \item Compute projected values: $U^\top u_i$ and $U^\top y$ where $y = e_1$
                \item Calculate residuals: $r_i = \|U^\top u_i\|^2 - c_i\|U^\top y\|^2$
            \end{enumerate}
        \item Update $U$ using Adam optimizer with gradient clipping
        \item Enforce fixed entries in $U$ after each update ($U[0,0] = 1$ \text{is enforced to be 1}.)
    \end{enumerate}
    where $B$ represents the batch of samples, $\lambda=10^{-4}$ is the regularization coefficient, and $y = e_1$ is the fixed unit vector.
    \end{algorithm}
    \end{minipage}
    \caption{Gradient-based optimization procedure for learning a dictionary decomposition $U$ with fixed entries.}
    \label{fig:gradient}
\end{figure}
% First Page of the Figure



% Pythia Code starts here
%\iffalse
\begin{figure*}[!]
    \centering
    % First Row of Subfigures
    \begin{subfigure}[b]{0.4\textwidth}
        \centering
        \includegraphics[width=\textwidth]{target_sae_pythia.png}
        \label{fig:sub1}
    \end{subfigure}
    \qquad
    \begin{subfigure}[b]{0.4\textwidth}
        \centering
        \includegraphics[width=\textwidth]{learnt_PCC-0.9367,feedbacks-135316.png}
        \label{fig: pythiaeigen}
    \end{subfigure}
    \caption{Feature learning on a subsampled dictionary of dimension $4500 \times 512$ of SAE trained for Pythia-70M. \thmref{thm: constructgeneral} states that Eigendecompostion method requires 135316 constructive feedback. After few 100 iterations of gradient descent as shown in \figref{fig:gradient}, a PCC of 93\% is achieved on ground truth. For visualization, only the first 100 dimensions are used.}
    \label{fig: subsample}
\end{figure*}

\begin{figure*}[t]\ContinuedFloat
    \centering

    % First Row of Subfigures
    \begin{subfigure}[b]{0.4\textwidth}
        \centering
        \includegraphics[width=\textwidth]{target_sae_pythiafull.png}
        \label{fig:sub3}
    \end{subfigure}
    \qquad
    \begin{subfigure}[b]{0.4\textwidth}
        \centering
        \includegraphics[width=\textwidth]{learnt_PCC-0.0242,feedbacks-200000.png}
        \label{fig: chessconst}
    \end{subfigure}
    \par % Start a new row

    % Second Row of Subfigures
    \begin{subfigure}[b]{0.4\textwidth}
        \centering
        \includegraphics[width=\textwidth]{learnt_PCC-0.3815,feedbacks-2000000.png}
        \label{fig:sub5}
    \end{subfigure}
\qquad    
    \begin{subfigure}[b]{0.4\textwidth}
        \centering
        \includegraphics[width=\textwidth]{learnt_PCC-0.5759,feedbacks-5000000.png}
        \label{fig:sub6}
    \end{subfigure}
    \par % Start a new row
\begin{subfigure}[b]{0.4\textwidth}
        \centering
        \includegraphics[width=\textwidth]{learnt_PCC-0.6570,feedbacks-10000000.png}
        \label{fig:sub5}
    \end{subfigure}
\qquad  
    % Third Row of Subfigures
    \begin{subfigure}[b]{0.4\textwidth}
        \centering
        \includegraphics[width=\textwidth]{learnt_PCC-0.7716,feedbacks-20000000.png}
        \label{fig:sub7}
    \end{subfigure}
    % \qquad
    % \begin{subfigure}[b]{0.4\textwidth}
    %     \centering
    %     \includegraphics[width=\textwidth]{ICML'25/Images/learnt_PCC- 0.9741, feedbacks- 10000000.png}
    %     \label{fig: chessssample}
    % \end{subfigure}

    \caption{\textbf{Sparse sampling for Pythia-70M}: Dimension of feature matrix: $32768 \times 512$ and the rank is 215. Plots for varying feedback complexity sizes. Note that $p(p+1)/2 \approx$ 512M. We run experiments with 3-sparse activations for uniform sparse distributions. The Pearson Correlation Coefficient (PCC) to feedback size (PCC, Feedback size) improves as follows: $(200k, .0242), (2M, .38), (5M, .54),(10M, .65)$, and $(20M, .77)$.
    %Plots show the improvement in PCC with 3-sparse uniformly sampled activations with respect to the Ground Truth shown above.
    }
    \label{fig: pythiasample}
\end{figure*}
%\fi




%\input{app_mahalanobis}
%
\section{Relevant Results}


\begin{figure}[h!]
    \centering
    \begin{minipage}{0.5\textwidth}
        \centering
        \begin{tikzpicture}[scale=1.8]
            % Rotate and draw the bean shape
            \begin{scope}[local bounding box=C, shift={(0,0)}, rotate=90] 
                \draw[fill=gray!50] 
                    (0,-1.5) .. controls +(170:.5) and +(180:2) ..
                    (.5,2) .. controls +(0:1.2) and +(0:.5) .. cycle;
            \end{scope}

            % Draw the inner circle
            \draw[thick,fill=white!50,opacity=.5] (0,0) circle (.5);

            % Draw the center of the circle
            \fill (0,0) circle (1.2pt);

            % Draw two lines intersecting the outer shape from the center of the circle
            \draw[thick] (0,0) -- (1.5/2.1,1.2/2.1);
            \draw[thick] (0,0) -- (-1.9,0);

            % Draw the points on the longer segment between the circle and the outer shape
            \fill (-0.5,0) circle (1pt);
            \fill (-1.9,0) circle (1pt);
            \fill (1.5/2.1,1.2/2.1) circle (1pt);

            % Label the inner circle
            \node[below] at (0.7,0) {\small{$\|\cdot\|_d = r$}};

            % Label the outer shape
            \node at (-2.5/2.5,2.8/2.5) {\small{$\|\cdot\|_{d'} = r$}};

            % Label the center
            \node at (-0.2/1.7,-0.2/1.7) {$x$};

            % Label the intersections of the line segments with the outer shape
            \node[above] at (1.5/2,1.2/2) {$y$};
            \node[left] at (-1.9,0) {$z$};

            % Label the point on the longer segment
            \node[above] at (-.5,0) {$z'$};
        \end{tikzpicture}
        
        %\label{fig:bean_shape}
    \end{minipage}\hfill
    \begin{minipage}{0.5\textwidth}
        \centering
        \begin{tikzpicture}[scale=1.8]
            % Rotate and draw the bean shape
            \begin{scope}[local bounding box=C, shift={(0,0)}, rotate=90] 
                \draw[fill=gray!50] 
                    (0,-1.5) .. controls +(170:.5) and +(180:2) ..
                    (.5,2) .. controls +(0:1.2) and +(0:.5) .. cycle;
            \end{scope}

            % Draw the bigger inner circle
            \draw[thick,fill=white!50,opacity=.5] (0,0) circle (1);

            % Draw the center of the circle
            \fill (0,0) circle (1.2pt);

            % Draw two lines intersecting the outer shape from the center of the circle
            \draw[thick] (0,0) -- (-1.5/3.79,3.5/3.79);
            \draw[thick] (0,0) -- (-1.9,0);

            % Draw the points on the longer segment between the circle and the outer shape
            \fill (-1.35,0) circle (1pt);
            \fill (-1.9,0) circle (1pt);
            \fill (-1.5/3.79,3.5/3.79) circle (1pt);

            % Label the inner circle
            \node[below] at (1.2,0) {\small{$\|\cdot\|_d = r$}};

            % Label the outer shape
            \node at (-2.5/2.5,2.8/2.5) {\small{$\|\cdot\|_{d'} = r$}};

            % Label the center
            \node at (-0.2/1.7,-0.2/1.7) {$x$};

            % Label the intersections of the line segments with the outer shape
            \node[above] at (-1.5/3.72,3.5/3.72) {$y$};
            \node[left] at (-1.9,0) {$z$};

            % Label the point on the longer segment
            \node[below] at (-1.35,0) {$z'$};
        \end{tikzpicture}
        %\caption{A bean-shaped diagram with a larger circle intersecting the outer shape}
        %\label{fig:bean_shape_bigger_circle}
    \end{minipage}
    \caption{Relative scaling of $r$-balls of norms $||\cdot||_d$ and $||\cdot||_{d'}$.}
    \label{fig:norm}
\end{figure}

\begin{theorem}
    Let $\cB$ be a complete normable vector space. Let $$\mathfrak{B}:= \curlybracket{d: \cB \times \cB \to \reals_{+}\,|\, \forall x,x' \in \cB, d(x, x') = ||x-x'|| \textnormal{ for a fixed norm on } \cB. }$$ 
    Using triplet constraints, a teacher can specify a target metric $d \in \mathfrak{B}$ up to linear scaling relation $\sim_R$. 
    %be a real normed vector space with the norm $||\cdot||$. Then, we can obliviously teach the norm up to 
\end{theorem}
\begin{proof}
Let $d$ and $d'$ be two norm induced metrics not related up to linear scaling but they satisfy the same triplet relation on any $x,y,z \in \cB$. Now, for a positive scalar $r > 0$ the closed balls $\overline{B_{d}(x,r)}$ and $\overline{B_{d'}(x,r)}$. Since $d$ and $d'$ are not related up to linear scaling, there exists a scalar $r > 0$ and a point $x$ such that the closed balls of radius $r$ centered at $x$ are scaled different, i.e there exists $y,z$ such that $\frac{d(x,y)}{d(x,z)} \neq \frac{d'(x,y)}{d'(x,z)}$.

Now, there are two possibilities: i) $\partial{B_{d}(x,r)} \cap \partial{B_{d'}(x,r)} \neq \emptyset$ or ii) $\partial{B_{d}(x,r)} \cap \partial{B_{d'}(x,r)} = \emptyset$.

We will show a contradiction for $i)$, whereas the contradiction for $ii)$ follows from a simple manipulation using $i)$. 

(consider right diagram in \figref{fig:norm})
If $\partial{B_{d}(x,r)} \cap \partial{B_{d'}(x,r)} \neq \emptyset$ there exists points (wlog) $y,z$ such that $y \in \partial{B_{d}(x,r)} \cap \partial{B_{d'}(x,r)}$ but $z \in \overline{B_{d'}(x,r)} \setminus \overline{B_{d}(x,r)}$. Thus, note that $d(x,z) > r$ but $d(x,y) \le r$. Now, there exists scalar $1> t > 0$ such that $z' := t x + (1-t) z \in \overline{B_{d'}(x,r)} \setminus \partial{B_{d}(x,r)}$, but then $d(x,z') > r$. Furthermore, $d'(x,z') > r$. Thus, for the points $x,y,z'$ we have
\begin{align*}
    d(x,y) = r < d(x,z') \textnormal{ but } d'(x,y) = r > d'(x,z')
\end{align*}
But this is contrary to the assumption that $d$ and $d'$ respect all triplet comparisons.

(consider left diagram in \figref{fig:norm}) If $\partial{B_{d}(x,r)} \cap \partial{B_{d'}(x,r)} = \emptyset$ there exists points (wlog) $y,z \in \partial{B_{d'}(x,r)} \setminus \overline{B_{d}(x,r)}$. Thus, note that $d'(x,z) = d'(x,y) = r$. Now, there exists scalar $1> t > 0$ such that $z' := t x + (1-t) z \in \partial{B_{d}(x,r)}$, but then $d(x,z') = r$. Furthermore, $d'(x,z') < r$. Thus, for the points $x,y,z'$ we have
\begin{align*}
    d(x,y) > r = d'(x,y) \textnormal{ but } d(x,z') = r < d'(x,z')
\end{align*}
Thus, even in this case, there exists a triplet comparison on which $d$ and $d'$ differ.

Hence, if $d$ and $d'$ are not related up to linear scaling, triplet comparisons can specify the same.


\end{proof}

\begin{theorem}
    For $c$-approximte teaching, there exists a set of size $2d$ for oblivious teaching.
\end{theorem}

\sanjoy{Some questions:}
\begin{enumerate}
\item \sanjoy{Triplets specify a Mahalanobis distance function upto a constant multiple. Is this true of any norm?} \akash{True} \sanjoy{Can you put in the proof?}
\item \sanjoy{Let's consider a $c$-approximate version of the learning problem, for $c \geq 1$. Let $d_M$ be the Mahalanobis distance induced by matrix $M$. If the target is $M^*$ then the $c$-approximate problem is to find a matrix $M \succeq 0$ such that $d_M(x,x') \geq d_M(x,x'')$ whenever $d_{M^*}(x,x') \geq c \cdot d_{M^*}(x,x'')$. What is the sample complexity of this relaxed learning objective? Here's one possible notion: we say $M, M'$ are $c$-close if there exists $\lambda > 0$ such that
$$ \frac{1}{c} \leq \lambda \cdot \frac{u^T M' u}{u^T M u} \leq c$$
for all $u \neq 0$.}
\item \sanjoy{In the setting where triplets need to be chosen from some pre-defined pool of points, it may not be possible to use multiples of the points (they may not correspond to real examples). In such cases, what is a good strategy for choosing triplets?}
\item \sanjoy{One setting of interest is the high- or infinite-dimensional regime, where the learner would need to choose a distance function based on just a few features. What are reasonable learning strategies in such cases?}
\end{enumerate}

\section{Oblivious teaching of Mahalanobis distance metric}

In this section, we will discuss oblivious teaching of the Mahalanobis distance metric. We assume that the metric model, denoted as $\cM_{\mathsf{maha}}$, contains Mahalanobis distance metrics as described below:
\begin{align*}
    \cM_{\mathsf{maha}} = \curlybracket{d: \cX \times \cX \to \reals_{+}\,|\, \exists M \in \textsf{symm}_{+}(\reals^d), s.t.\, d = d_M \textit{ as shown in }\eqnref{eq: maha} }
\end{align*}
The interaction protocol is shown in \algoref{alg: maha}. 

\begin{algorithm}[H]
\caption{Teaching a mahalanobis distance metric}
\label{alg: maha}
\textbf{Given}: Input space $\cX \subset \reals^d$, metric model $\cM_{\mathsf{maha}}$\\
\vspace{1mm}
\textit{In batch setting:}\vspace{1mm}
\begin{enumerate}
    \item teacher picks triplets $\cT(\cX,d_{M^*}) =$ $\curlybracket{(x,y,z) \in \cX^3 \,|\, (x-y)^{\top}M^*(x-y) \ge (x-z)^{\top}M^*(x-z)}$
    \item learner receives $\cT$; and obliviously picks a metric $d \in \cM_{\mathsf{maha}}$ as per \eqnref{eq: sol}
\end{enumerate}
\end{algorithm}

First, we note that using triplet comparisons alone learner can't find a fixed target matrix $M^*$ because for any $\lambda > 0$ and $(x,y,z) \in \cX^3$
\begin{align*}
    (x-y)^{\top}M^*(x-y) \ge (x-z)^{\top}M^*(x-z) \implies (x-y)^{\top}(\lambda M^*)(x-y) \ge (x-z)^{\top}(\lambda M^*)(x-z)
\end{align*}

Thus, we can only hope to find $d_{M^*}$, alternatively $M^*$ up to a positive linear scaling. Thus, we define a linear scaling relation $\sim_{R_l}$ on the space $\cM_{\mathsf{maha}}$ as follows:
\begin{align*}
   \textnormal{for any } d_M, d_{M'} \in \cM_{\mathsf{maha}}, d_M \sim_{R_l} d_{M'} \textnormal{ iff }  M = \gamma \cdot M' \textnormal{ for } \gamma > 0
\end{align*}
In the rest of the discussion, we study the oblivious teaching complexity of metric learning of mahalanobis distance metric wrt to the linear scaling relation $\sim_{R_l}$. 

In the rest of the section, we follow the notations as mentioned below:
%\sanjoy{I didn't get far in sections 2-3 because of high-level issues (see my comments). Can you fix these things and then I will go through these two sections more carefully.}
%\sanjoy{What is the formal model -- what problem is being solved?}
\paragraph{Notations:} For a given matrix $M \in \reals^{d\times d}$ and indices $i,j \in \bracket{d}$ $M_{ij}$ denotes the entry of $M$ at $ith$ row and $jth$ column. Matrices are denoted as $M,M',N$. Unless stated otherwise, a target matrix (for teaching a mahalanobis metric) is denoted as $M^*$. We denote that null set of a matrix $M$, i.e. $\curlybracket{x \in \reals^d\,|\, Mx = 0}$, as $\nul{M}$.
For a matrix, we denote its eigenvalues as $\gamma,\lambda, \gamma_i$ or $\lambda_i$ and its eigenvectors (orthogonal vectors) as $\mu_i ,u_i$ or $v_i$. We define the element-wise product of two matrices $M,M'$ via an inner product $\inner{M', M} := \sum_{i,j} M'_{ij}M_{ij}$. 
We denote vectors in $\reals^d$ as $x,y$ or $z$.
Note, for any $x \in \reals^d$ $\inner{xx^{\top}, M} = x^{\top}Mx$. For ease of notation, we also write the inner product as $M \idot M'$.

We denote the space of symmetric matrices in $\reals^{d \times d}$ as $\symm$, and similarly the space of symmetric, positive semi-definite matrices as $\symmp$.
Since the space of matrices on $\reals^{d\times d}$ is isomorphic to the Euclidean vector space $\reals^{d^2}$ for any matrix $M$ we also call it a vector identified as an element of $\reals^{d^2}$. We say two matrices $M,M'$ are \tt{orthogonal}, denoted as $M \bot M'$, if $M \idot M' = 0$. For a set of vectors/matrices $\cC \subset \reals^{d\times d}$, the subspace induced by the elements in $\cC$ is denoted as $span \inner{\cC} := \curlybracket{\alpha M + \beta M' \,|\, M,M' \in \cC,\, \alpha, \beta \in \reals}$. Similarly, the set of columns of a matrix $M$ is denoted as $\col{M}$, and its span as $span \inner{\col{M}}$.
%For a given matrix $M \in \reals^{d\times d}$ and a vector $x \in \reals^d$, define an inner product $\inner{xx^{\top}, M} := x^{\top} Mx$. We denote the set of symmetric matrices as $\sf symm(\reals^{d \times d})$.

%\sanjoy{For a Mahalanobis metric it is essential for the matrix to be positive semidefinite. This should be made clear in the definitions and will be part of the proofs.}

%\begin{lemma}
%    Any symmetric matrix of dimension $d \times d$ of rank-1 such that $M_{1,1}$ is positive can be written as
%    \begin{align}
%        M = xx^{\top}
%    \end{align}
%    for $x \in \reals^n$.
%\end{lemma}
%\begin{proof}
%    Wlog assume that $M_{1,1}$ is positive otherwise
%    Since $M$ is rank-1 every column vector $M_i$ is a linear scaling of any other column $M_j$ for $i \neq j$, i.e. there exists $\lambda_{j} \in \reals$ such that $M_1 = \lambda_{j}M_j$. But $M$ is symmetric and thus $R_1 = \lambda_{j}R_j$ for each $R_i = C_i$.
%    \begin{align*}
%    M = \begin{bmatrix}
%  M_{1,1} & \lambda_2 M_{1,1}& \ldots & \lambda_n M_{1,1} \\
%  \lambda_2 M_{1,1} & \lambda_2^2 M_{1,1}& \ldots & \lambda_2 \lambda_n M_{1,1}\\
%  \vdots & \vdots & \vdots & \vdots\\
%  \lambda_n M_{1,1}& \lambda_n\lambda_2M_{1,1}& \ldots & \lambda_n^2 M_{1,1}
%\end{bmatrix}
%\end{align*}
%Now, $M_{1,1}$ is positive thus we can write $M_{1,1} = (\sqrt{M_{1,1}})^2$. Thus, we  can write
%\begin{align}
%    M = \paren{\sqrt{M_{1,1}}, \lambda_2\sqrt{M_{1,1}},\ldots, \lambda_n\sqrt{M_{1,1}}}^{\top} \paren{\sqrt{M_{1,1}}, \lambda_2\sqrt{M_{1,1}},\ldots, \lambda_n\sqrt{M_{1,1}}} =: xx^{\top}
%\end{align}
%\end{proof}

\subsection{Reduction to pairs}
Now, we would argue that the general triplet comparisons provided by the teacher can be simplified to pair comparisons of the form $(0, y,z) \in \cX^3$, denoted as $(y,z)$ for simplification, for strict equality constraint in \eqnref{eq: sol}. 


\begin{lemma}\label{lem: reduction}
    Consider an input space $\cX \subset \reals^d$ and the corresponding Mahalanobis metric space $\cM_{\textsf{maha}}$. Fix a target matrix $M^*$ for oblivious teaching of Mahalanobis metric learning. If there exists a teaching set $\cT(\cX, \cM_{\textsf{maha}}) = \curlybracket{(x,y,z) \in \cX^3 \,|\, (x-y)^{\top}M^*(x-y) \ge (x-z)^{\top}M^*(x-z)} $, such that for any $d' \in \textsf{VS}(\cT, \cM_{\textsf{maha}})$, $d' \sim_{R_l} d_{M^*}$ then there exists a teaching set $\cT' = \curlybracket{(0,y',z') \in\cX^3\,|\, d_{M^*}(0, y') = d_{M^*}(0, z')}$ such that $d' \in \textsf{VS}(\cT',\cM_{\textsf{maha}})$.
\end{lemma}

\begin{proof}
    WLOG assume that for all $(x,y,z) \in \cT(\cX, \cM_{\textsf{maha}})$ $x \neq z$. Now, consider the following 
   \begin{enumerate}
       \item if for $(x,y,z) \in \cT(\cX, \cM_{\textsf{maha}})$ $d_{M^*}(x,y) = d_{M^*}(x,z)$ then $(0, x-y, x-z)$ satisfy the same equality wrt $M^*$
       \item if for $(x,y,z) \in \cT(\cX, \cM_{\textsf{maha}})$ $d_{M^*}(x,y) > d_{M^*}(x,z)$ then there exists $\gamma > 0$ such that
       \begin{align*}
           &(x-y)^{\top}M^*(x-y) = \gamma + (x-z)^{\top}M^*(x-z)
       \end{align*}
       Since $x \neq z$ we can rewrite $\gamma$ as a scalar multiple of $(x-z)^{\top}M^*(x-z)$ and 
       \begin{align*}
           (x-y)^{\top}M^*(x-y) = \lambda\cdot (x-z)^{\top}M^*(x-z) + (x-z)^{\top}M^*(x-z) = (1 + \lambda) (x-z)^{\top}M^*(x-z)
       \end{align*}
       Hence, $(0, x - y, (1 + \lambda)(x-z))$ satisfy the equality constraint wrt $M^*$.
   \end{enumerate}
   So, we have provided a procedure to translate any triplet comparison in $\cT(\cX, \cM_{\textsf{maha}})$ to a form as specified in $\cT' = \curlybracket{(0,y',z') \in\cX^3\,|\, d_{M^*}(0, y') = d_{M^*}(0, z')}$. As the equations in $1.$ and $2.$ are agnostic to a positive linear scaling thus we have: if $d' \in \cT(\cX, \cM_{\textsf{maha}})$ such that $d' \sim_{R_l} d_{M^*}$ then $d' \in \textsf{VS}(\cT', \cM_{\textsf{maha}})$.
\end{proof}

\lemref{lem: reduction} implies that if there are triplet comparisons to \eqnref{eq: sol} then there are also pair comparisons that satisfy the following equation:
\begin{align}
    d' \sim_{R_l} d_{M^*}, d'  \in \curlybracket{d_M \in \cM_{\textsf{maha}}\,|\, \forall (y,z) \in \cT(\cX,\cM_{\textsf{maha}},M^*),\, y^{\top}My = z^{\top}Mz} \label{eq: redsol}
\end{align}
where $\cT(\cX,\cM_{\textsf{maha}},M^*)$ is  the teaching set that satisfy equality constraints with $M^*$.

In the rest of this section, we consider reformulation of oblivious teaching of Mahalanobis metric learning with pair comparisons.

For a given pair $(y,z)$ and a matrix $M$, if we have equality constraint $y^{\top}My = z^{\top}Mz$ then with some rewriting we observe that
\begin{align*}
    y^{\top}My = z^{\top}Mz \iff \inner{yy^{\top} - zz^{\top}, M} = 0 \iff (yy^{\top} - zz^{\top}) \idot M = 0
\end{align*}
Thus, $(yy^{\top} - zz^{\top})$ is orthogonal to $M$. So, we can pose the teaching problem of \eqnref{eq: redsol} for a set of pairs of examples $\cT(\cX,\cM_{\textsf{maha}},M^*) := \curlybracket{(y_i,z_i)}_{i=1}^k $ corresponding to a target matrix $M^*$ as follows:
\begin{align}
  \forall (y,z) \in \cT(\cX,\cM_{\textsf{maha}},M^*), \quad M \idot (yy^{\top} - zz^{\top})  = 0 \label{eq: orthosat}
\end{align}
Essentially, the learner obliviously picks a matrix $M'$ that satisfy \eqnref{eq: orthosat}. Note that this has a nice geometric picture to this as shown in \figref{fig: geom} where $\mathcal{O}_{M^*} := \curlybracket{S \in \reals^{d \times d} \,|\, S\idot M^* = 0}$.
\begin{figure}
\begin{center}
\tdplotsetmaincoords{60}{120}

\begin{tikzpicture}[tdplot_main_coords]


% Draw the plane
\fill[fill=blue!10] (0,0,0) -- (3,-1,0) -- (4,1,0) -- (1,2,0) -- cycle;

% Define the center point on the plane
\coordinate (center) at (2,0.5,0);

% Draw the arrows
\draw[->, line width=.5mm] (center) -- ++(0,0,2.5) node[anchor=south] {$M^*$};
%\draw[->, thick] (0,0) -- (3,-1) node[anchor=north west] {$\mathbb{R}^{d \times d}$};
% \draw[->, thick] (0,0) -- (4,1);

% Draw the dashed arrows from the center
\draw[dashed, ->] (center) -- ++(-1,-1,0) node[above, anchor=south] {\tiny{$yy^{\top} - zz^{\top}$}};
\draw[dashed, ->] (center) -- ++(.7,-.8,0) node[below, anchor=east] {\tiny{$y'y'^{\top} - z'z'^{\top}$}};

% Add the label on the plane
\node at (4.2,0.5,0) {$\mathcal{O}_{M^*}$};
\node at (3,3,3) {$\mathbb{R}^{d \times d}$};
\end{tikzpicture}
\end{center}
\caption{Geometry of solving \eqnref{eq: redsol}}
    \label{fig: geom}
\end{figure}

%\sanjoy{How is this related to the problem we are trying to solve? Should give a reduction.}
%Thus, if we think of $M$ as a vector then $y_iy_i^{\top} - z_iz_i^{\top}$ lies in the orthogonal complement as vectors.

\subsection{Oblivious teaching complexity of metric learning: Pessimistic lower bound}

In this subsection, we study the \tt{the minimal size of $\cT(\cX,\cM_{\textsf{maha}})$ that exactly fixes a target matrix $M^*$ (upto linear scaling relation $\sim_{R_l}$) in \eqnref{eq: redsol}}.

First, we start with a strong lower bound on the worst-case $M^*$ that turns out to be of order $\Omega(d^2)$. Note that the $\dim(\symm) = \frac{d(d+1)}{2}$.

%\sanjoy{These results should incorporate the PSD constraint. They might need to be reworked slightly for that.}

\begin{proposition}\label{prop: worstcase} The worst-case oblivious teaching complexity of a teaching set $\cT(\cX, \maha)$ up to linear scaling relation $\sim_{R_l}$ for \eqnref{eq: redsol} is at the least $\paren{\frac{d(d+1)}{2} - 1}$.
\end{proposition}

Before we proof this lower bound, we state a useful property of the sum of a symmetric, positive semi-definite matrix and a general symmetric matrix.
\begin{lemma}\label{lem: sum}
    Consider a PSD symmetric matrix $M \in \symmp$ of rank $d$. Assume that $M' \in \symm$, then there exists a scalar $\lambda > 0$ such that $M + \lambda M'$ is
    PSD.
\end{lemma}
\begin{proof}
    Consider the eigendecomposition of the matrix $M$ as follows:
    \begin{align*}
        M = \sum_{i=1}^d \lambda_i u_iu_i^{\top}%, M' = \sum_{j=1}^r \delta_j v_jv_j^\top and $\curlybracket{\delta_j,v_j}_{j=1}^r$ are
    \end{align*}
    where $\curlybracket{\lambda_i,u_i}_{i=1}^d$ is eigenvalue and eigenvector pairs of $M$. Note that since $M$ is full rank, for any $i$ $\lambda_i > 0$. Also, $\curlybracket{u_i}_{i=1}^d$ is an orthonormal basis of $\reals^d$.
    Consider the constant $\gamma$ as follows:
    \begin{align*}
        \gamma := \max_i |u_i^{\top}M'u_i| 
    \end{align*}
    Now, consider the constant $\lambda = \frac{\min \lambda_i}{\gamma}$. It is straight-forward that $u_i^{\top}(M + \lambda M')u_i$ is non-negative on the set $\curlybracket{u_i}$; but since it is an orthonormal basis of the space $\reals^d$, $x^{\top}(M + \lambda M')x \ge 0$ for any $x \in \reals^d$. This implies for the choice of $\lambda > 0$ shown above, $(M + \lambda M')$ is PSD.
\end{proof}

Now, we provide the proof of \propref{prop: worstcase} in the following:
\begin{proof}[Proof of \propref{prop: worstcase}] 
    Consider a symmetric, positive semi-definite matrix $M^* \neq {\bf 0} \in \symmp$ of full rank. For the sake of contradiction, let $\cT(\cX, \maha, M^*)$ be a teaching set for \eqnref{eq: redsol} up to linear scaling relation with size strictly less than $\paren{\frac{d(d+1)}{2} - 1}$.
    
    Now, for any pair $(y,z) \in \cT$, $M^*$ is orthogonal to $(yy^{\top} - zz^{\top})$. Thus, if we define $\mathcal{O}_{M^*}$ as the orthogonal complement of $M^*$ then $(yy^{\top} - zz^{\top}) \in \mathcal{O}_{M^*}$ for any $(y,z) \in \cT$. Thus,
    \begin{align*}
        span\inner{\{yy^{\top} - zz^{\top}\}_{(y,z) \in \cT}} \subset \mathcal{O}_{M^*}
    \end{align*}
    Hence,
    \begin{align*}
        M^* \perp span\inner{\{yy^{\top} - zz^{\top}\}_{(y,z) \in \cT}}
    \end{align*}
    Since the teaching set $|\cT| < \paren{\frac{d(d+1)}{2} - 1}$ we note that $\dim (span\inner{(yy^{\top} - zz^{\top})}) < \paren{\frac{d(d+1)}{2} - 1}$. 
    
    Since $M^*$ is a singleton vector in $\reals^{d \times d}$ the union  $\curlybracket{M^*} \cup \{yy^{\top} - zz^{\top}\}_{(y,z) \in \cT}$ will only add an extra dimension in the space $\reals^{d \times d}$. This implies that
    \begin{align*}
        \dim(span\inner{M^*\cup \{yy^{\top} - zz^{\top}\}_{(y,z) \in \cT}} \le \paren{\frac{d(d+1)}{2} - 1}
    \end{align*}
    Since $\symm$ is a vector space over $\reals$ and $\dim(\symm) = \frac{d(d+1)}{2}$ there is a symmetric matrix $M'$ such that the following holds
    \begin{gather*}    
        M' \in \mathcal{O}_{M^*},\\
        \forall (y,z) \in \cT,\,  M' \perp (yy^{\top} - zz^{\top})
    \end{gather*}
    But \lemref{lem: sum} implies there exists $\lambda > 0$ such that $M^* + \lambda M'$ is PSD and symmetric (sum of symmetric matrices is symmetric). Since $M' \in \mathcal{O}_{M^*}$, $M'$ is not identical to $M^*$ up to a linear scaling. This implies that there exists a matrix in the form $M^* + \lambda M'$ ($\,\not \sim_{R_l} M^*$) that is orthogonal to all the matrices $(yy^{\top} - zz^{\top})$ for any pair $(y,z) \in \cT$.
    
    Thus, if the teaching set is smaller than $\frac{d(d+1)}{2} - 1$, we can find symmetric, PSD matrices not related up to linear scaling that satisfy \eqnref{eq: redsol}. This contradicts the assumption on $\cT$ This establishes the stated lower bound on the oblivious teaching complexity of the teaching set.
\end{proof}

Later in the section we will show that worst-case upper bound on oblivious teaching complexity is $\frac{d(d+1)}{2} - 1$, which is expected as there are $\frac{d(d+1)}{2}$ degree of freedom for any symmetric, postive semi-definite matrix. Since the worst-case bounds are pessimistic for oblivious teaching of \eqnref{eq: redsol} a general question of interest is how does oblivious teaching complexity vary over the metric model $\maha$. In the following subsection, we study the teaching complexity for Mahalanobis distance metric based on the rank of the underlying matrix. 

\subsection{Oblivious teaching for low-rank matrices}
In general, the input space $\cX \subset \reals^d$ exists in a small dimensional manifold. So, if there is a distance function which linearly transforms the space to capture the distance well, e.g. in the form of a matrix in Mahalanobis metric, we expect the rank of the matrix to be small as well.
So, it is interesting to understand if the oblivious teaching can capture this small dimension with an optimistic teaching complexity. 

%\begin{lemma}
%    Let $M$ be a positive definite matrix. Then, the lower bound on the teaching complexity is $\frac{d(d+1)}{2} - 1$.
%\end{lemma}

\begin{theorem}\label{thm: obv}
    Consider an input space $\cX \subset \reals^d$, and a metric model $\maha$.
    Consider a symmetric, positive semi-definite matrix $M^*$ of rank $r \le d$. Assume that $d_{M^*}$ be the target metric for oblivious teaching in \eqnref{eq: sol}. Then, the teaching complexity $TC_O(\cX, \maha, d_{M^*})$ has the tight bound of $\paren{\frac{r(r+1)}{2} + (d - r) - 1}$.
\end{theorem}

Below we provide the proof of the \thmref{thm: obv}. 

Consider the eigendecomposition of $M^*$. So, there exists a set of orthonormal vectors $\curlybracket{v_1,v_2,\ldots,v_r}$ with eigenvalues $\curlybracket{\gamma_1,\gamma_2,\ldots, \gamma_r}$ such that 
\begin{align}
    M^* = \sum_{i=1^r} \gamma_i v_iv_i^{\top} \label{eq: target}
\end{align}
Denote the set of orthogonal vectors $\curlybracket{v_1,v_2,\ldots,v_r}$ as $V_{\bracket{r}}$.
Now, let $\curlybracket{v_{r+1},\dots,v_d}$, denoted as $V_{\bracket{d - r}}$, be an orthogonal extension to the orthogonal vectors $v_i$, $i = 1,2,\ldots, r$ so that $\curlybracket{v_1,v_2,\ldots,v_r} \cup \curlybracket{v_{r+1},v_{r+2},\ldots,v_d}$ forms a basis for $\reals^d$. Denote the basis $\curlybracket{v_{1},v_{2},\ldots,v_d}$ as $V_{\bracket{d}}$.
Note that $\curlybracket{v_{r+1},\dots,v_d}$ precisely defines the null set of $M^*$, i.e.
\begin{align*}
    \nul{M^*} = span\inner{\curlybracket{v_{r+1},\ldots,v_d}}
\end{align*}
The key idea of the proof is to manipulate this null set to show teaching set in \eqnref{eq: orthosat} for the target matrix $M^*$. Since $M^*$ is of rank $r \le d$ the number of degrees of freedom is exactly $\frac{r(r+1)}{2}$. Alternatively, the span of the null set of $M^*$, which is of dimension exactly $d-r$, fixes rest of the entries in $M^*$. Using this intuition, the teacher can provide pairs $(y,z) \in \cX^2$ to teach the null set and the eigenvectors $\curlybracket{v_1,v_2,\ldots,v_r}$ separately. But we need to ensure this strategy is optimal up to sample efficiency. We confirm the optimality of this strategy in the next two lemmas.

Our first result is on nullifying the null set of $M^*$ in the \eqnref{eq: orthosat}. Consider a partial teaching set 
\begin{align*}
    \mathcal{T}_{\sf {null}} = \curlybracket{(0, v_{i})}_{i = r+1}^d
\end{align*}
\begin{lemma}\label{lem: nullset}
    If the teacher provides the set $\mathcal{T}_{\sf{null}}$, then the null set of any psd symmetric matrix $M'$ that satisfies \eqnref{eq: orthosat} contains the span of $\{v_{r+1},\ldots, v_d\}$, i.e.
    \begin{align*}
       \{v_{r+1},\ldots, v_d\} \subset \nul{M'}
    \end{align*}
\end{lemma}
\begin{proof}
    Let $M' \in \symmp$ be a matrix that satisfy \eqnref{eq: orthosat} (at least $M^*$ satisfy the \eqnref{eq: orthosat}). Thus, we have the following equality constraints
    \begin{align*}
        \forall (0,v) \in \cT_{\sf {null}}, v^{\top}M'v = 0
    \end{align*}
    Since $\{v_{r+1},\ldots, v_n\}$ a set of linearly independent vectors, in order to complete the proof, it is sufficient to show that
    \begin{align}
       \forall (0,v) \in \cT_{\sf {null}}, v^{\top}M'v = 0 \implies M'v = 0 \label{eq: lemmain}
    \end{align}
    To prove that we use some general properties of the eigendecompositon of a symmetric, positive semi-definite matrix. We write $M$ in the form of its eigendecompositon 
    \begin{align*}
        M' = \sum_{i = 1}^{s} \gamma_i'u_iu_i^{\top}
    \end{align*}
    where $\curlybracket{u_i}_{i=1}^{s}$ and $\curlybracket{\gamma'_i}_{i=1}^s$ are the eigenvectors and the eigenvalues respectively. Assume $x\neq 0 \in \reals^d$ be such that
    \begin{align*}
        x^{\top}M'x = 0
    \end{align*}
    Consider the decomposition $x = \sum_{i=1}^s a_iu_i + v'$ for scalars $a_i$ and $v' \bot \{u_i\}_{i=1}^s$ . Now, expanding the equation above, we get
    \begin{align*}
       x^{\top}M'x &= \paren{\sum_{i=1}^s a_iu_i + v'}^{\top}M'\paren{\sum_{i=1}^s a_iu_i + v'}  \\
       & = \paren{\sum_{i=1}^s a_iu_i}^{\top}M'\paren{\sum_{i=1}^s a_iu_i } + v'^{\top}M'\paren{\sum_{i=1}^s a_iu_i} + \paren{\sum_{i=1}^s a_iu_i}M'v' + v'^{\top}M'v'\\
       & = \paren{\sum_{i=1}^s a_iu_i}^{\top}\paren{\sum_{i = 1}^{s} \gamma_i'u_iu_i^{\top}}\paren{\sum_{i=1}^s a_iu_i } + \underbrace{2v'^{\top}\paren{\sum_{i = 1}^{s} \gamma_i'u_iu_i^{\top}}\paren{\sum_{i=1}^s a_iu_i} + v'^{\top}\paren{\sum_{i = 1}^{s} \gamma_i'u_iu_i^{\top}}v'}_{ =\, 0 \textnormal{ as } v' \bot \curlybracket{u_i}} \\
       & = \sum_{i,j,k} a_i u_i^{\top} (\gamma_j'u_ju_j^{\top}) a_k u_k\\
       & =\sum_{i=1}^s a_i^2\gamma_i' = 0
    \end{align*}
    Since $\gamma_i'$, $i = 1,\ldots,r$ are strictly positive ($M$ is psd) it implies that each $a_i$ is zero. Thus, we have
    \begin{align*}
        M'x = M'v' = 0
    \end{align*}
    This implies that $x$ is in $\nul{M'}$. This proves the statement in \eqnref{eq: lemmain}.
    
    Thus, if the teacher provides $\mathcal{T}_{\sf{null}}$, 
    it implies, any solution $M'$ to \eqnref{eq: orthosat} has the null set that contains at the least $\{v_i\}_{i=r+1}^d$, and hence the $span \inner{\curlybracket{v_{r+1},\ldots,v_d}}$. 
\end{proof}

With this we will argue that the teaching setup in \eqnref{eq: orthosat} can be decomposed in two parts: first is teaching the null set $ \nul{M^*}:= span \inner{\{v_i\}_{i=r+1}^n}$, and second is teaching $\mathcal{S}_{M^*} = span \inner{\{v_i\}_{i=1}^r}$ in the form of $M^* = \sum_{i=1}^r \gamma_i v_iv_i^{\top}$. 

\lemref{lem: nullset} implies that using a teaching set of the form $\mathcal{T}_{\sf {null}}$ any solution $M' \in \symmp$ to \eqnref{eq: orthosat} satisfies the property $V_{\bracket{d - r}} \subset \nul{M'}$. Furthermore, $|\mathcal{T}_{\sf {null}}| = d - r$. Now, we discuss how to teach $V_{\bracket{r}}$, i.e. $V_{\bracket{r}}$ span the rows of any solution $M' \in \symmp$ to \eqnref{eq: orthosat} with the corresponding eigenvalues $\curlybracket{\gamma_i}_{i=1}^r$. We show that if the search space of metrics in \eqnref{eq: orthosat} is the version space $\textsf{VS}(\maha,\mathcal{T}_{\sf {null}})$  which is a restriction of the space $\maha$ to teaching set $\mathcal{T}_{\sf {null}}$, then a teaching set of size at most $\frac{r(r+1)}{2} -1$ is sufficiently required to teach $M^*$ up to linear scaling relation $\sim_{R_l}$. Thus, we consider the reformation of the problem in \eqnref{eq: orthosat} as 
\begin{align}
  \forall (y,z) \in \cT(\cX,\textsf{VS}(\maha,\mathcal{T}_{\sf {null}}),M^*), \quad M \idot (yy^{\top} - zz^{\top})  = 0  \label{eq: redorthosat}
\end{align}
where the teaching set $\cT(\cX,\textsf{VS}(\maha,\mathcal{T}_{\sf {null}}),M^*)$ is devised to solve a smaller space $\textsf{VS}(\maha,\mathcal{T}_{\sf {null}})$.

%It is straight-forward that, since $\dim(\mathcal{N}_{M^*}) = d - r$ one needs at least $(d-r)$ pairs to nullify $\mathcal{N}_{M^*}$ for any psd symmetric matrix. On the other hand, using \lemref{lemma: nullset} we note that one can sufficiently nullify it with just $(d-r)$ pairs of the form $\curlybracket{(0, v_i)}_{i=r+1}^d$. But still one question remains if this set of pairs can be used in a different form. For that, we consider the following result.

\begin{lemma}\label{lemma: orthoset}
    Consider the problem as formulated in \eqnref{eq: redorthosat} in which the null set $\nul{M^*}$ of the target matrix $M^*$ is known. Then, the teacher sufficiently and necessarily finds a set $\cT(\cX,\textsf{VS}(\cT_{\sf{null}}),M^*)$ of size $\frac{r(r+1)}{2} - 1$ for oblivious teaching up to linear scaling relation $\sim_{R_l}$.
\end{lemma}
\begin{proof}

    Note that any solution $M'$ of \eqnref{eq: redorthosat} has its columns spanned exactly by $V_{\bracket{r}}$. Alternatively, if we consider the eigendecompostion of $M'$ then the corresponding eigenvectors exists in $span \inner{V_{\bracket{r}}}$. Furthermore, note that $M^*$ is of rank $r$ which implies there are only $\frac{r(r+1)}{2}$ degrees of freedom, i.e. entries in the matrix $M^*$, that need to be fixed.

    Thus, there are exactly $r$ linearly independent columns of $M^*$, indexed as $\{j_1,j_2,\ldots, j_r\}$. Now, consider the set of matrices
    \begin{align*}
        \curlybracket{M^{(i,j)}\,|\, i \in \bracket{d}, j \in \{j_1,j_2,\ldots, j_r\}, M^{(i,j)}_{i'j'} = \mathds{1}[i'\in \{i,j\}, j' \in \{i,j\}\setminus \{i'\}]}
    \end{align*}
    This forms a basis to generate any matrix with independent columns along the indexed set. Hence, the span of $\mathcal{S}_{M^*}$ induces a subspace of symmetric matrices of dimension $\frac{r(r+1)}{2}$ in the vector space $\sf{symm}(\reals^d)$, i.e. the column vectors along the indexed set is spanned by elements of $\mathcal{S}_{M^*}$. Thus, it is clear that picking a teaching set of size $\frac{r(r+1)}{2} -1$ in the orthogonal complement of $M^*$, i.e. $\mathcal{O}_{M^*}$ restricted by this span sufficiently teaches $M^*$ if $\nul{M^*}$ is known. One exact form of this set is proven in \lemref{lemma: basis}. Since any solution $M'$ is agnostic to the scaling of the target matrix $M'$, we have shown that the sufficiency on the oblivious teaching complexity for $M^*$ up to linear scaling relation $\sim_{R_l}$.

   Now, we show that the stated teaching set size is necessary. The argument is similar to the proof of \lemref{lem: sum}.
   
   For the sake of contradiction assume that there is a smaller sized teaching set $\cT_{small}$. This implies that there is some matrix in $\textsf{VS}(\maha,\mathcal{T}_{\sf {null}})$, a subspace induced by span $\mathcal{S}_{M^*}$, orthogonal to $(M^*)$ is not in the span of $\cT_{small}$, denoted as $M'$. If $M'$ is PSD then it is a solution to \eqnref{eq: redorthosat} and $M'$ is not a scalar multiple of $M^*$. Now, if $M'$ is not PSD we show that there exists scalar $\lambda > 0$ such that
    \begin{align*}
        M^* + \lambda M' \in \symmp,
    \end{align*}
     i.e. the sum is PSD. Consider the eigendecompostion of $M'$ (assume $\rank{M'} = r'$)
     \begin{align*}
         M' = \sum_{i = 1}^{r'} \delta_i\mu_i\mu_i^{\top}
     \end{align*}
     for orthogonal eigenvectors $\curlybracket{\mu_i}_{i=1}^{r'}$ and the corresponding eigenvalues $\curlybracket{\delta_i}_{i=1}^{r'}$. Since (assume) $r_0 \le r'$ of the eigenvalues are negative we can rewrite $M'$ as
     \begin{align*}
         M' = \sum_{i=1}^{r_0} \delta_i \mu_i\mu_i^{\top} + \sum_{j=r_0 + 1}^{r'} \delta_j \mu_j\mu_j^{\top} 
     \end{align*}
     Thus, if we can regulate the values of $\mu^{\top}_iM^*\mu_i$, $i = 1,2,\ldots,r_0$, noting they are positive, then we can find an appropriate $\lambda$. Let $m^* := \min_{i \in [r_0]} \mu_i^{\top}M^*\mu_i$ and $\ell^* := \max_{i \in [r_0]} |\delta_i|$. Now, setting $\lambda \le \frac{m^*}{\ell^*}$ achieves the desired property of $M^* + \lambda M'$ as shown in the proof of \lemref{lem: sum}. 

     But now note that both $M'$ and $M^*$ are orthogonal to any element in the teaching set $\cT_{small}$ implying $M^*$ is not a unique solution to the \eqnref{eq: redorthosat} up to a positive scaling factor.

     Hence, we have shown that the null set $\nul{M^*}$ of the target matrix $M^*$ is known then a teaching set of size exactly $\frac{r(r+1)}{2} -1$ is both sufficient and necessary.
\end{proof}

Up until this point we haven's shown how to construct this $\frac{r(r+1)}{2}-1$ sized teaching set. 
Consider the following union:
\begin{align*}
    \curlybracket{v_1v_1^{\top}} \cup \curlybracket{v_2v_2^{\top}, (v_2 + v_1)(v_2 + v_1)^{\top}} \cup \ldots \cup \curlybracket{v_rv_r^{\top}, (v_1 + v_r)(v_1 + v_r)^{\top},\ldots, (v_{r-1} + v_r)(v_{r-1} + v_r)^{\top}}
\end{align*}
We can show that this union is a set of linearly independent matrices of rank 1.
%First, note that if this set is not linearly independent then 
\begin{lemma}\label{lemma: basis}
    Let $\curlybracket{v_i}_{i=1}^r$ be a set of orthogonal vectors, then the following set of rank-1 matrices
    \begin{align*}
        \mathcal{B} := \curlybracket{v_1v_1^{\top},v_2v_2^{\top}, (v_1 + v_2)(v_1 + v_2)^{\top},\ldots,v_rv_r^{\top}, (v_1 + v_r)(v_1 + v_r)^{\top},\ldots, (v_{r-1} + v_r)(v_{r-1} + v_r)^{\top}}
    \end{align*}
    is linearly independent in the vector space $\symm$.
\end{lemma}
\begin{proof}
    We prove the claim in two parts based on separate cases. For the sake of contradiction, let us assume that $\cB$ is linearly dependent. This implies that there exists either some $v_iv_i^{\top}$ or some $(v_i+ v_j)(v_i+ v_j)^{\top}$ that can be written as a linear combination of other matrices in $\cB$. Now, we consider the two cases.
    
    First, we assume that for some $i \in [r]$, $v_iv_i^{\top}$ can be written as a linear combination. Thus, there exists scalars that satisfy the following property
    \begin{gather}
        v_iv_i^{\top} = \sum_{j = 1}^{r'} \alpha_{j}v_{i_j}v_{i_j}^{\top} + \sum_{k = 1}^{r''} \beta_{k}(v_{l_k} + v_{m_k})(v_{l_k} + v_{m_k})^{\top}\\
        \forall j,k,\quad \alpha_j, \beta_k > 0, i_j \neq i, l_k < m_k
    \end{gather}
    Now, note that we can write
    \begin{align*}
       \sum_{k = 1}^{r''} \beta_{k}(v_{l_k} + v_{m_k})(v_{l_k} + v_{m_k})^{\top} =  \sum_{k = 1, l_k = i}^{r''} \beta_{k}(v_{l_k} + v_{m_k})v_{l_k}^{\top} + \sum_{k = 1, l_k \neq i}^{r''} \beta_{k}(v_{l_k} + v_{m_k})v_{l_k}^{\top} + \sum_{k = 1}^{r''} \beta_{k}(v_{l_k} + v_{m_k})v_{m_k}^{\top}
    \end{align*}
    Now, we note that the sum 
    \begin{align*}
        \sum_{j = 1}^{r'} \alpha_{j}v_{i_j}v_{i_j}^{\top} + \sum_{k = 1, l_k \neq i}^{r''} \beta_{k}(v_{l_k} + v_{m_k})v_{l_k}^{\top} + \sum_{k = 1}^{r''} \beta_{k}(v_{l_k} + v_{m_k})v_{m_k}^{\top}
    \end{align*}
    doesn't span (as column vectors) a space that contains the column vector $v_i$ because $\curlybracket{v_i}_{i=1}^r$ is a set of orthogonal vectors. Thus, 
    \begin{align}
        v_iv_i^{\top} = \sum_{k = 1, l_k = i}^{r''} \beta_{k}(v_{l_k} + v_{m_k})v_{l_k}^{\top} = \paren{\sum_{k = 1, l_k = i}^{r''} \beta_k v_{l_k} + \sum_{k = 1, l_k = i}^{r''} \beta_k v_{m_k}}v_i^{\top} \label{eq: v1}
    \end{align}
    This implies that 
    \begin{align}
        \sum_{k = 1, l_k = i}^{r''} \beta_k v_{m_k} = 0 \implies \textnormal{ if } l_k = i, \beta_k = 0 \label{eq: v2}
    \end{align}
    Since not all $\beta_k = 0$ corresponding to $l_k = i$ (otherwise $\sum_{k = 1, l_k = i}^{r''} \beta_k v_{l_k} = 0$ ) we have shown that $v_iv_i^{\top}$ can not be written as a linear combination of elements in $\cB \setminus \curlybracket{v_i}$.

    Now, we consider the second case where there exists some indices $i,j$ such that $(v_i + v_j)(v_i+v_j)^{\top}$ is a sum of linear combination of elements in $\cB$. Note that this linear combination can't have an element of type $v_kv_k^{\top}$ as it contradicts the first case. So, there are scalars such that
    \begin{gather}
        (v_i + v_j)(v_i+v_j)^{\top} = \sum_{k = 1}^{r''} \beta_{k}(v_{l_k} + v_{m_k})(v_{l_k} + v_{m_k})^{\top}\\
        \forall k,\quad l_k < m_k
    \end{gather}
    But we rewrite this as 
    \begin{align*}
        (v_i + v_j)v_i^{\top} + (v_i + v_j)v_j^{\top} = \sum_{k = 1, l_k = i}^{r''} \beta_{k}(v_{i} + v_{m_k})v_{i}^{\top} + \sum_{k = 1, m_k = j}^{r''} \beta_{k}(v_{l_k} + v_{j})v_{j}^{\top} + \sum_{\substack{k = 1, l_k \neq i,\\ m_k \neq j}}^{r''} \beta_{k}(v_{l_k} + v_{m_k})(v_{l_k} + v_{m_k})^{\top}
    \end{align*}
    Note that if $l_k = i$ then the corresponding $m_k \neq j$ and vice versa. Since $\curlybracket{v_i}_{i=1}^r$ are orthogonal, the decomposition above implies
    \begin{gather}
        (v_i + v_j)v_i^{\top} = \sum_{k = 1, l_k = i}^{r''} \beta_{k}(v_{i} + v_{m_k})v_{i}^{\top} \label{eq: vplusv1}\\
        (v_i + v_j)v_j^{\top} =  \sum_{k = 1, m_k = j}^{r''} \beta_{k}(v_{l_k} + v_{j})v_{j}^{\top}\label{eq: vplusv2}\\
        \sum_{\substack{k = 1, l_k \neq i,\\ m_k \neq j}}^{r''} \beta_{k}(v_{l_k} + v_{m_k})(v_{l_k} + v_{m_k})^{\top} = 0
    \end{gather}
    But using the arguments in \eqnref{eq: v1} and \eqnref{eq: v2}, we can achieve \eqnref{eq: vplusv1} or \eqnref{eq: vplusv2}.

    Thus, we have shown that the set of rank-1 matrices as described in $\cB$ are linearly independent.
\end{proof}
In \lemref{lemma: orthoset}, we discussed that in order to teach $M^*$ sufficiently one needs a teaching set of size $\frac{r(r+1)}{2} -1$ if the null set of $M^*$ is known. We can establish this teaching set using the basis shown in \lemref{lemma: basis}. We state this result in the following lemma.
\begin{lemma}\label{lemma: orthocons}
    For a  given target matrix $M^* = \sum_{i=1}^r \gamma_iv_iv_i^{\top}$ and basis set of matrices $\cB$ as shown in \lemref{lemma: basis}, the following set spans a subspace of dimension $\frac{r(r+1)}{2} -1$ in $\sf{symm}(\reals^{d \times d})$. 
\begin{equation*}
\mathcal{O}_{\cB} := \left\{
\begin{aligned}
&v_1v_1^{\top} - \lambda_{11}yy^{\top}, v_2v_2^{\top} - \lambda_{22}yy^{\top}, (v_1 + v_2)(v_1 + v_2)^{\top} - \lambda_{12}yy^{\top}, \ldots,\\
&v_rv_r^{\top} - \lambda_{rr}yy^{\top}, (v_1 + v_r)(v_1 + v_r)^{\top} - \lambda_{1r}yy^{\top}, \ldots, \\
&(v_{r-1} + v_r)(v_{r-1} + v_r)^{\top} - \lambda_{(r-1)r}yy^{\top}
\end{aligned}
\right\}
\end{equation*}

\begin{equation*}
yM^*y^{\top} \neq 0
\end{equation*}

\begin{equation*}
\forall i,j,\quad \lambda_{ii} = \frac{v_iM^*v_i^{\top}}{yM^*y^{\top}}, \quad \lambda_{ij} = \frac{(v_i + v_j)M^*(v_i+ v_j)^{\top}}{yM^*y^{\top}} \quad (i \neq j)
\end{equation*}


\end{lemma}
\begin{proof}
    Since $M^*$ has at least $r$ positive eigenvalues there exists a vector $y \in \reals^d$ such that $yM^*y^{\top} \neq 0$. It is straight forward to note that $\mathcal{O}_{\cB}$ is orthogonal to $M^*$. As $\mathcal{O}_{\cB} \subset span\langle \cB \rangle$ and $M^* \bot \mathcal{O}_{\cB}$, $\dim(span\langle \mathcal{O}_{\cB} \rangle) = \frac{r(r+1)}{2} -1$. 
\end{proof}

Now, it is clear that a teaching set of size $\frac{r(r+1)}{2} - 1 + (d -r)$ is sufficient to teach $M^*$. In the following, we show the teaching set size of $\frac{r(r+1)}{2} - 1 + (d -r)$ is also necessary for oblivious teaching.

\begin{lemma}\label{lemma: lowerbound}
    Consider an input space $\cX \subset \reals^d$, and a metric model $\maha$.
    Consider a symmetric, positive semi-definite matrix $M^*$ of rank $r \le d$. Assume that $d_{M^*}$ be the target metric for oblivious teaching in \eqnref{eq: sol}. Then, the teaching complexity $TC_O(\cX, \maha, d_{M^*})$ has a lower bound of $\paren{\frac{r(r+1)}{2} + (d - r) - 1}$.
\end{lemma}
\begin{proof}
    The key idea of the proof is that any teaching set, say $\cT$ for the oblivious teaching in \eqnref{eq: sol} must have matrices that satisfy the following properties:
    \begin{enumerate}
        \item[\textcolor{blue}{S(1)}] if $M \in \mathcal{O}_{M^*}$ such that $span\inner{\col{M}} \subset span\inner{V_{\bracket{r}}}$ then $M \in span \inner{\cT}$.
        \item[\textcolor{blue}{S(2)}] there exists vectors $U_{\bracket{d-r}} \subset \nul{M^*}$ (of size $d - r $) such that $span \inner{U_{\bracket{d-r}} } = \nul{M^*}$ and 
        for any vector $v \in U_{\bracket{d-r}}$, $vv^{\top} \in span \inner{\cT}$.
    \end{enumerate}
    If any of the two statements are violated, we can show there exists $\lambda > 0$ and $M$ such that $M^* + \lambda M \in \sf{VS}(\cT,\maha)$. In  order to ensure these statements, the teaching set should have $\paren{\frac{r(r+1)}{2} + (d - r) - 1}$ many elements which proves the lower bound on $TC_O(\cX, \maha, d_{M^*})$. Now, we argue the necessity of these statements.

    Consider the first statement. Consider an $M \in \mathcal{O}_{M^*}$ such that $span\inner{\col{M}} \subset span\inner{V_{\bracket{r}}}$. Note that the eigendecompostion of $M$ (assume $\rank{M} = r' < r$)
     \begin{align*}
         M = \sum_{i = 1}^{r'} \delta_i\mu_i\mu_i^{\top}
     \end{align*}
     for orthogonal eigenvectors $\curlybracket{\mu_i}_{i=1}^{r'}$ and the corresponding eigenvalues $\curlybracket{\delta_i}_{i=1}^{r'}$ has the property that $span \inner{\curlybracket{\mu_i}_{i=1}^{r'}} \subset span \inner{V_{\bracket{r}}}$. Using the arguments exactly as shown in the second half of the proof of \lemref{lemma: orthoset} we can show there exists $\lambda > 0$ such that $M^* + \lambda M \in \sf{VS}(\cT, \maha)$. But then $M \not\sim_{R_l} M^*$. But this contradicts the assumption on $\cT$ being a valid oblivious teaching set for $\eqnref{eq: sol}$ up to linear scaling relation $\sim_{R_l}$. But using \lemref{lemma: orthoset} and \lemref{lemma: orthocons} we know that the dimension of the span of matrices that satisfy the condition in $S(1)$ is at the least $\frac{r(r+1)}{2} -1$. We can use \lemref{lemma: orthocons} where $y = \sum_{i = 1}^r v_r$ (note $M^*v \neq 0$). Thus, any basis matrix in $\mathcal{O}_{\cB}$ satisfy the conditions in $S(1)$.

    Now, consider the second statement. Assuming the contrary, there exists $v \in span \inner{\nul{M^*}}$ such that $vv^{\top} \notin span \inner{\cT}$.

    Now if $vv^{\top}\, \bot\, \cT$, then for any scalar $\lambda > 0$, $M^* + \lambda vv^{\top}$ is both symmetric and positive semi-definite and satisfies all the conditions in \eqnref{eq: redsol} wrt $\cT$ a contradiction as $M^* + \lambda vv^{\top} \not\sim_{R_l} M^*$. 
    
    So, consider the case when $vv^{\top}\, \not\perp\, \cT$. Let $\curlybracket{v_{r+1},\ldots,v_{d-1}}$ be an orthogonal extension of $v$ such that $\curlybracket{v_{r+1},\ldots,v_{d-1}, v}$ forms a basis of $\nul{M^*}$, i.e., in other words 
    \begin{align*}
    v \bot \curlybracket{v_{r+1},\ldots,v_{d-1}}\quad \&\quad span \inner{\curlybracket{v_{r+1},\ldots,v_{d-1}, v}} = \nul{M^*}.
    \end{align*}
    We will first show that there exists some $M'$ $(\not \sim_{R_l} M^*)$ $\in \symm$ orthogonal to $\cT$ and furthermore $\curlybracket{v_{r+1},\ldots,v_{d-1}} \subset \nul{M'}$ . 
    
    
    Consider the intersection (in the space $\symm$) of the orthogonal complement of the matrices $\curlybracket{v_{r+1}v_{r+1}^{\top},\ldots,v_{d-1}v_{d-1}^{\top}}$, denote it as $\mathcal{O}_{rest}$, i.e.,
    \begin{align*}
        \mathcal{O}_{rest} := \bigcap_{i = r+1}^{d-1} \mathcal{O}_{v_iv_i^{\top}} 
    \end{align*}
    Note that
    \begin{align*}
        \dim(\mathcal{O}_{rest}) = D - d + r
    \end{align*}
    Since $vv^{\top}$ is in $\mathcal{O}_{rest}$ and $\dim(\mathcal{O}_{rest}) > 1$ there exists some $M'$ such that $M' \perp M^*$, and also orthogonal to elements in the teaching set $\cT$. Thus, $M'$ has a null set which includes the subset $\curlybracket{v_{r+1},\ldots,v_{d-1}}$. 
    
    Now, the rest of the proof involves showing existence of some scalar $\lambda > 0$ such that $M^* + \lambda M'$ satisfies the conditions of \eqnref{eq: redsol} for the teaching set $\cT$. Note that if $vM'v^{\top} = 0$ then the proof is straightforward as $ span \inner{\curlybracket{v_{r+1},\ldots,v_{d-1}, v}} \subset \nul{M'}$, which implies $span \inner{\col{M'}} \subset span \inner{V_{[r]}}$. But this is precisely the condition for $S(1)$ to hold. 
    
     
     Without loss of generality assume that $vM'v^{\top} > 0$. First note that the eigendecomposition of $M'$ has eigenvectors that are contained in $V_{[r]} \cup \curlybracket{v}$. Consider some arbitrary choice of $\lambda > 0$, we will fix a value later. It is straightforward that $M^* + \lambda M'$ is symmetric for $M^*$ and $M'$ are symmetric. In order to show it is positive semi-definite, it suffices to show that
     \begin{align}
         \forall u \in \reals^d, u^{\top}(M^* + \lambda M') u \ge 0 \label{eq: psd}
     \end{align}
    Since  $\curlybracket{v_{r+1},\ldots, v_{d-1}} \subset \paren{\nul{M^*} \cap \nul{M'}}$ we can simplify \eqnref{eq: psd} to
    \begin{align}
        \forall u \in span\inner{V_{[r]} \cup \curlybracket{v}}, u^{\top}(M^* + \lambda M') u \ge 0 \label{eq: repsd}
    \end{align}
    Consider the decomposition of any arbitrary vector $u \in span\inner{V_{[r]} \cup \curlybracket{v}}$ as follows:
    \begin{gather}
        u = u_{[r]} + v', \textnormal{ such that } u_{[r]} \in span\inner{V_{[r]}}, v' \in span \inner{\{v\}} \label{eq: decom1}\\
        u_{[r]} := \sum_{i =1}^r \alpha_i v_i,\;\; \forall i\; \alpha_i \in \reals \label{eq: decom2}
    \end{gather}
    From here on we assume that $u_{[r]} \neq 0$. The alternate case is trivial as $v'^{\top}M'v' > 0$.
    
    Now, we write the vectors as scalar multiples of their corresponding unit vectors
    \begin{gather}
        u_{[r]} = \delta_r \cdot \hat{u}_r,\;\; \hat{u}_r := \frac{u_{[r]}}{||u_{[r]}||^2_{V_{[r]}}}, ||u_{[r]}||^2_{V_{[r]}} := \sum_{i =1}^r \alpha_i^2 \label{eq: scale1}\\
        v' = \delta_{v'}\cdot \hat{v},\;\; \hat{v} := \frac{v}{||v||_2^2} \label{eq: scale2}
    \end{gather}
    \tt{Remark}: Although we have computed the norm of $ u_{[r]}$  as $||u_{[r]}||^2_{V_{[r]}}$ in the orthonormal basis $V_{[r]}$ it is straightforward to note that the norm remains unchanged (same as the $\ell_2$). It is done for ease of analysis later on.
    
    Using the decomposition in \eqnref{eq: decom1}-(\ref{eq: decom2}), we can write \eqnref{eq: repsd} as
    \begin{align}
        u^{\top}(M^* + \lambda M')u &= (u_{[r]} + v')^{\top}(M^* + \lambda M')(u_{[r]} + v') \nonumber\\
        &= u_{[r]}^{\top} M^*u_{[r]} + \lambda (u_{[r]} + v')^{\top}M'(u_{[r]} + v')\nonumber\\
        & = \delta_r^2 \cdot\hat{u}_r^{\top}M^*\hat{u}_r + \lambda\big( \delta_r^2 \cdot\hat{u}_r^{\top}M'\hat{u}_r + 2 \delta_r \delta_{v'}\cdot \hat{u}_r^{\top} M' \hat{v} + \delta^2_{v'}\cdot \hat{v}^{\top}M'\hat{v} \big) \label{eq: eq1}
    \end{align}
    Since we want $u^{\top}(M^* + \lambda M') \ge 0$ we can further simplify \eqnref{eq: eq1} as 
    \begin{align}
        \hat{u}_r^{\top}M^*\hat{u}_r + \lambda\paren{ \hat{u}_r^{\top}M'\hat{u}_r + 2 \textcolor{gray}{\frac{\delta_r\delta_{v'}}{\delta_r^2 }} \cdot \hat{u}_r^{\top} M' \hat{v} + \textcolor{gray}{\frac{\delta^2_{v'}}{\delta^2_r}}\cdot \hat{v}^{\top}M'\hat{v} } \underset{?}{\ge} 0 \label{eq: equiv1}\\
        \Longleftrightarrow \underbrace{\hat{u}_r^{\top}M^*\hat{u}_r}_{\textcolor{red}{(1)}} + \lambda\paren{ \underbrace{\hat{u}_r^{\top}M'\hat{u}_r}_{\textcolor{violet}{(3)}} + \underbrace{2 \textcolor{gray}{\xi}\cdot \hat{u}_r^{\top} M' \hat{v} + \textcolor{gray}{\xi^2}\cdot \hat{v}^{\top}M'\hat{v} }_{\textcolor{blue}{(2)}}} \underset{?}{\ge} 0 \label{eq: equiv2}
    \end{align}
    where we have used $\xi = \frac{\delta_{v'}}{\delta_r}$. The next part of the proof we show that $\textcolor{red}{(1)}$ is lower bounded by a positive constant whereas $\textcolor{blue}{(2)}$ is upper bounded by a positive constant and there is a choice of $\lambda$ so that $\textcolor{blue}{(3)}$ is always smaller than $\textcolor{red}{(1)}$.
    
    Considering $\textcolor{red}{(1)}$ we note that $\hat{u}_r$ is a unit vector wrt the orthonormal set of basis $V_{[r]}$. Expanding using the eigendecomposition of \eqnref{eq: target}
    \begin{align*}
        \hat{u}_r^{\top}M^*\hat{u}_r = \sum_{i=1}^r \frac{\alpha^2_i}{\sum_{i=1}^r \alpha_i^2}\cdot \gamma_i \ge \min_i \gamma_i > 0
    \end{align*}
    The last inequality follows as all the eigenvalues in the eigendecompostion are (strictly) positive. Denote this minimum eigenvalue as $\gamma_{\min} := \min_i \gamma_i$.
    
    Considering $\textcolor{blue}{(2)}$ note that only terms that are variable (i.e. could change value) is $\xi$ as $\hat{u}_r^{\top} M' \hat{v}$ is 

    Note that $\hat{v}$ is a fixed vector and $\hat{u}_r$ has a fixed norm (using \eqnref{eq: scale1}-(\ref{eq: scale2})), so $|\hat{u}_r^{\top} M' \hat{v}| \le C$ for some bounded constant $C > 0$ whereas $\hat{v}^{\top}M'\hat{v}$ is already a constant. Now, $|2 \textcolor{gray}{\xi}\cdot \hat{u}_r^{\top} M' \hat{v}|$ exceeds $\textcolor{gray}{\xi^2}\cdot \hat{v}^{\top}M'\hat{v}$ only if
    \begin{align*}
        |2 \textcolor{gray}{\xi}\cdot \hat{u}_r^{\top} M' \hat{v}| \ge |\textcolor{gray}{\xi^2}\cdot \hat{v}^{\top}M'\hat{v}| %\ge |2 \textcolor{gray}{\xi}\cdot \hat{u}_r^{\top} M' \hat{v} + \textcolor{gray}{\xi^2}\cdot \hat{v}^{\top}M'\hat{v}|
        \Longleftrightarrow \frac{|\hat{u}_r^{\top} M' \hat{v}|}{\hat{v}^{\top}M'\hat{v}} \ge \textcolor{gray}{\xi} \implies \frac{C}{\hat{v}^{\top}M'\hat{v}} \ge \textcolor{gray}{\xi}
    \end{align*}
    Rightmost inequality implies that $2 \textcolor{gray}{\xi}\cdot \hat{u}_r^{\top} M' \hat{v} + \textcolor{gray}{\xi^2}\cdot \hat{v}^{\top}M'\hat{v}$ is negative only for an $\textcolor{gray}{\xi}$ bounded from above by a positive constant. But since $\xi$ is non-negative 
    \begin{align*}
        |2 \textcolor{gray}{\xi}\cdot \hat{u}_r^{\top} M' \hat{v} + \textcolor{gray}{\xi^2}\cdot \hat{v}^{\top}M'\hat{v}| \le C' (\textnormal{bounded constant})
    \end{align*}
    Now using an argument similar to the second half of the proof of \lemref{lemma: orthoset}, it is straight forward to show that there is a choice of $\lambda' > 0$ so that $\textcolor{violet}{(3)}$ is always smaller than $\textcolor{red}{(1)}$.

    Now, for $\lambda = \frac{\lambda'}{2\lceil C' \rceil \lambda''}$ where $\lambda''$ is chosen so that $\lambda_{\min} \ge \frac{\lambda'}{\lambda''}$, we note that
    \begin{align*}
        \hat{u}_r^{\top}M^*\hat{u}_r + \lambda\paren{ \hat{u}_r^{\top}M'\hat{u}_r + 2 \textcolor{gray}{\xi}\cdot \hat{u}_r^{\top} M' \hat{v} + \textcolor{gray}{\xi^2}\cdot \hat{v}^{\top}M'\hat{v} } \ge \lambda_{\min} + \frac{\lambda'}{2\lceil C' \rceil \lambda''} \hat{u}_r^{\top}M'\hat{u}_r -\frac{\lambda'}{2\lambda''} >  0.
    \end{align*}
    Using the equivalence in \eqnref{eq: eq1}, \eqnref{eq: equiv1} and \eqnref{eq: equiv2}, we have a choice of $\lambda > 0$ such that $u^{\top}(M^* + \lambda M')u \ge 0$ for any arbitrary vector $u \in span\inner{V_{[r]} \cup \curlybracket{v}}$. Hence, we have achieved the conditions in \eqnref{eq: repsd}, which is the simplification of \eqnref{eq: psd}. This implies that $M^* + \lambda M'$ is positive semi-definite. 
    
    This implies that there doesn't exist a $v \in span \inner{\nul{M^*}}$ such that $vv^{\top} \notin span \inner{\cT}$ otherwise the assumption on $\cT$ to be an oblivious teaching set for $M^*$ is violated. Thus, the statement $S(2)$ has to hold. Since the dimension of $\nul{M^*}$ is at least $(d-r)$ thus there are at least $(d-r)$ directions or linearly independent matrices (in $\symm$) that need to be spanned by $\cT$.

    Thus, $S(1)$ implies there are $\frac{r(r+1)}{2} -1$ linearly independent matrices (in $\mathcal{O}_{M^*}$) that need to be spanned by $\cT$. Similarly, $S(2)$ implies there are $d-r$ linearly independent matrices (in $\mathcal{O}_{M^*}$) that need to be spanned by $\cT$. Note that the column vectors of these matrices from the two statements are spanned by orthogonal set of vectors, i.e. one by $V_{[r]}$ and the other by $\nul{M^*}$ respectively. Thus, these $\frac{r(r+1)}{2} -1 + (d-r)$ are linearly independent in $\symm$, but this forces a lower bound on the size of $\cT$ (a lower dimensional span can't contain a set of vectors spanning higher dimensional space). This completes the proof of the lemma.
\end{proof}


%\subsection{Reduction to zero signal}

\section{Oblivious teaching with sampled data points}
In the previous section, we consider the oblivious teaching setup for Mahalanobis distance metric where the teacher could \tt{constructively} pick triplets (or pairs) from the input space $\cX$. In the statistical learning framework, we expect the data points to be identically and independently sampled from the input space. In this section, we assume that the teacher receives a set of datapoints $\cX_n := \curlybracket{x_1,x_2,\ldots,x_n} \sim \cD_{\cX}$, where $\cD_{\cX}$ is an unknown Lebesgue measure over a continuous space $\cX \subset \reals^d$. Given this set teacher designs pairs of the form $\cT(\cX_n, \maha, M^*) := \curlybracket{(x,y,z) \in \cX_n \,|\, (x-y)^{\top}M^*(x-y) \ge (x-z)^{\top}M^*(x-z)}$ for a target matrix $M^*$.

In \lemref{lem: reduction}, we showed the reduction from inequality constraints with triplet comparisons to equality constraints with pairs in the constructive setting. But given that the teacher can no longer select samples $x,y \in \cX$ to construct pairs, but rather selects from a restricted set $\cX_n$, the reduction no longer holds. This is clear from this observation: let $x,y \sim_{iid} \cD_{\cX}$ and $M \neq 0$ a non-degenerate matrix then
\begin{gather*}
    x^{\top}Mx = y^{\top}My \implies \sum_{ij} (x_ix_j - y_iy_j)M_{ij} = 0
\end{gather*}
This is a polynomial equation (non-zero). But the zero set $\{x^{\top}Mx = y^{\top}My\}$ of a non-zero polynomial equation has Lebesgue measure zero, i.e
\begin{align*}
    \cP_{(x,y)}( \{x^{\top}Mx = y^{\top}My\}) = 0
\end{align*}


%\begin{theorem}[Sard's Theorem]
%Let \( f: X \to Y \) be a smooth map of manifolds, and let \( C \) be the set of critical points of \( f \) in \( X \). Then \( f(C) \) has measure zero in \( Y \).
%\end{theorem}

%Now, there are some questions to study:
%\begin{enumerate}
%    \item In the oblivious teaching scenario, what is the teaching complexity?
%    \item If the answer to 1. leads to poor lower bound on the teaching complexity, what are some ways we can improve the complexity.
%\end{enumerate}
\paragraph{Rescaled pairs}: For a given matrix $M \neq 0$ a sampled input $x \sim \cD_{\cX}$ is (almost) never orthogonal, i.e. $Mx \neq 0$. This can be used to rescale an input to construct pairs to satisfy equality constraints. In other words, there exists $\gamma, \lambda >0$ such that (assume wlog $x^{\top}Mx > y^{\top}My$)
\begin{align*}
    x^{\top}Mx = \gamma + y^{\top}My = \lambda\cdot y^{\top}My + y^{\top}My = (\sqrt{1 + \lambda})y^{\top}M(\sqrt{1 + \lambda})y 
\end{align*}
Thus, $(x, (\sqrt{1 + \lambda})y)$ satisfy the equality constraints. With this understanding we provide a reformulation of \algoref{alg: maha} into \algoref{alg: randmaha}. In the rest of the section, we study the expected oblivious teaching complexity for  \algoref{alg: randmaha} under Lebesgue distribution $\cD_{\cX}$ over the input space $\cX$. We show tight bounds on the teaching complexity highlighting a gap between the constructive and sampled teaching scenarios. 
%In the constructive setting, we discussed concrete strategies for devising optimal teaching set, the number of triplet or pairs for oblivious teaching. In the sampled case, we want to study the size of samples $\cX_n$, both necessary and sufficient, so that teacher can devise rescaled pairs. Thus, the oblivious teaching complexity is defined as the size $|\cX_n|$ such that the learner finds $M^*$ up to linear scaling relation $\sim_{R_l}$ in \algoref{alg: randmaha}, $\expctover{\cX_n}{}$

In the sampling case, we define the teaching complexity in terms of the size of the sampled set $\cX_n$ so that the teacher can devise a teaching set $\cT(\cX, \maha,M^*)$ for a target matrix $M^*$.
\begin{definition}\label{defn: lebsample}
    Consider an input space $\cX \subset \reals^d$. Let $\cD_{\cX}$ be a Lebesgue distribution over $\cX$ from which iid samples $\cX_n$ are selected. Fix a constant $\epsilon > 0$ and a target matrix $M^*$. We define the oblivious teaching complexity for $\epsilon$-accuracy for random samples is $n$ if the teacher provides a teaching set $\cT(\cX_n, \maha,M^*)$ for \algoref{alg: randmaha} using the samples $\cX_n$ such that
    \begin{align*}
        \cP_{\cX_n}(d_{M'} \in \textsf{VS}(\cT(\cX_n, \maha,M^*), \maha) \textnormal{ such that } d_{M'} \sim_{R_l} d_{M^*}) \ge \epsilon
    \end{align*}
\end{definition}

First, we start with an upper bound on the worst-case (across the space $\maha$) oblivious teaching complexity that achieves $1$-accuracy.

\begin{proposition}
    Consider an input space $\cX \subset \reals^d$. Let $\cD_{\cX}$ be a Lebesgue distribution over $\cX$ from which iid samples $\cX_n$ are selected. Then, the worst-case oblivious teaching complexity for \algoref{alg: randmaha} has an upper bound of $O\paren{\frac{d(d+1)}{2}}$ that achieves 1-accuracy for teaching up to linear scaling relation $\sim_{R_l}$.
\end{proposition}
\begin{proof}
    
    Fix a positive index $D = \frac{d(d+1)}{2}$; the teacher samples $D$ inputs: $x_1,x_2,\ldots, x_{D} \sim \cD_{\cX}$. Denote for each $i$, $X_i = x_ix_i^{\top}$. Now, consider the matrix

        \begin{align*}
    \mathbb{M} = \begin{bmatrix}
  \Big\lvert & \Big\lvert& \ldots & \Big\lvert \\
  X_1 & X_2 & \vdots & X_D\\
  \Big\lvert& \Big\lvert& \ldots & \Big\lvert
\end{bmatrix}
\end{align*}
where we treat each matrix $X_i$ as a column vector in $\reals^{d^2}$. Note that the zero set of $\curlybracket{\det(\mathbb{M}) = 0}$ has measure zero in $\cX^D$ as $\det(\mathbb{M})$ is a non-zero polynomial equation over random vectors $x_1,x_2,\ldots, x_{D}$. This implies that
\begin{align}
    \cP_{\cX_n}\paren{\curlybracket{x_ix_i^{\top}} \textnormal{ is linearly independent in } \symm} = 1 \label{eq: fullprob}
\end{align}

Assume $M^* \neq 0$ be an arbitrary target matrix for oblivious teaching in \algoref{alg: randmaha}. WLOG assume that $x_1 \neq 0$. Consider the following set $\cT$ of rescaled pairs
\begin{align*}
    \forall i \in \curlybracket{2,\ldots, D} (x_1, \sqrt{\gamma_{1i}}x_i) \in \cT,\quad (x_1x_1^{\top} - \gamma_{1i}x_ix_i^{\top}) \idot M^* = 0, \sqrt{\gamma_{1i}} > 0
\end{align*}
Note that $|\cT| = D - 1$. Now, we show that the elements of $\cT$ are linearly independent in $\symm$. Assuming the contrary, there exists scalars $\curlybracket{\alpha_i}$ (not all zero) such that
\begin{align*}
    &\sum_{i = 2}^D \alpha_i (x_1x_1^{\top} - \gamma_{1i}x_ix_i^{\top}) = 0 \,(\in \symm)\\
    \implies &\paren{\sum_{i = 2}^D \alpha_i} x_1x_1^{\top} = \sum_{i = 2}^D \alpha_i \gamma_{1i}x_ix_i^{\top}
\end{align*}
But since $\curlybracket{x_ix_i^{\top}}$ are linearly independent $\sum_{i = 2}^D \alpha_i$ and $\curlybracket{\alpha_i\gamma_i}$ are necessarily zero which implies $\alpha_i = 0$ (for all $i$) as $\gamma_{1i} > 0$. This is in contradiction to the assumption on the dependence of elements in $\cT$. 

This implies that $\cT$ is a set of linearly independent matrices in the orthogonal complement $\mathcal{O}_{M^*}$. But then $M^*$ only has $D$ many degree of freedoms. Thus, any matrix $M'$ which satisfies the equations:
\begin{align*}
    \forall i \in \curlybracket{2,\ldots, D} (x_1, \sqrt{\gamma_{1i}}x_i) \in \cT,\quad (x_1x_1^{\top} - \gamma_{1i}x_ix_i^{\top}) \idot M' = 0
\end{align*}
is at most a positive linear scaling of $M^*$. Now, using \eqnref{eq: fullprob} we know that
\begin{align*}
    \cP_{\cX_n}\paren{ \cT \textnormal{ is an oblivious teaching set up to linear scaling relation } \sim_{R_l}} = 1
\end{align*}
Since $M^*$ was picked arbitrarily the worst-case oblivious expected teaching complexity is upper bounded by $ D - 1 = \frac{d(d+1)}{2} -1$ for $1$-accuracy.
\end{proof}

\begin{algorithm}[t]
\caption{Teaching a mahalanobis distance metric with sampled data points}
\label{alg: randmaha}
\textbf{Given}: Input space $\cX_n \sim \cD_{\cX}$, metric model $\cM_{\mathsf{maha}}$\\
\vspace{1mm}
\textit{In batch setting:}\vspace{1mm}
\begin{enumerate}
    \item teacher picks pairs $\cT(\cX_n,d_{M^*}) =$ $\curlybracket{(x,\lambda_{x}y)\,|\, (x,y) \in \cX_n^2,\, x^{\top}M^*x = \lambda_{x}y^{\top}M^*(\lambda_{x}y)}$
    \item learner receives $\cT$; and obliviously picks a metric $d \in \cM_{\mathsf{maha}}$ as per \eqnref{eq: sol}
\end{enumerate}
\end{algorithm}

Now, we show a lower bound on the worst-case oblivious teaching complexity in the sampling case even to achieve any non-zero accuracy.



\begin{lemma}
       Consider an input space $\cX \subset \reals^d$. Let $\cD_{\cX}$ be a Lebesgue distribution over $\cX$ from which iid samples $\cX_n$ are selected. For any $\epsilon \in (0,1]$, the worst-case oblivious teaching complexity for \algoref{alg: randmaha} has a lower bound of $\Omega\paren{\frac{d(d+1)}{2}}$ that achieves $\epsilon$-accuracy for teaching up to linear scaling relation $\sim_{R_l}$.
    %For any target matrix $M^*$, the teacher needs at least $\Omega\paren{\frac{d(d+1)}{2}}$ (almost surely) samples to teach an oblivious learner.
\end{lemma}
\begin{proof}
    In \lemref{lemma: lowerbound}, we stated the statement dented as $S(1)$ that provided a necessary property of a set $\cT$ for oblivious teaching in \algoref{alg: maha} with pairs (we assume equality constraint). We state it again for better clarity: given any target matrix $M^*$ for oblivious teaching, if $M \in \mathcal{O}_{M^*}$ such that $span\inner{\col{M}} \subset span\inner{V_{\bracket{r}}}$ then $M \in span \inner{\cT}$ where $V_{[r]}$ ($r \le d$) is defined as the set of eigenvectors in the eigendecompostion of $M^*$ (see \eqnref{eq: target}). 
    %If the rank of $M^*$, then the lower bound follows from the deterministic setting. If the rank is strictly less than $d$, then using \lemref{lem: uniquevec} 
    
    With this observation, we know that there exists a matrix $M'$ for some matrix $M' \in \symm$ that needs to be spanned by any teaching set $\cT(\cX_n, d_{M^*})$ in \algoref{alg: randmaha} for oblivious teaching up to linear scaling relation $\sim_{R_l}$. But, we argue that it requires $\frac{d(d+1)}{2}$ sized samples $\cX_n$, i.e. $n = \frac{d(d+1)}{2}$ to construct a teaching set that spans M'.

    Fixing some postive index $\ell > 0$, the teacher samples $\ell$ samples: $x_1,x_2,\ldots, x_{\ell}$. Now, denote for each $i$, $X_i = x_ix_i^{\top}$. Now, consider the matrix

        \begin{align*}
    \mathbb{M} = \begin{bmatrix}
  \Big\lvert & \Big\lvert& \ldots & \Big\lvert \\
  M & X_1 & \vdots & X_\ell\\
  \Big\lvert& \Big\lvert& \ldots & \Big\lvert
\end{bmatrix}
\end{align*}
where we treat each matrix $X_i$ and M as column vectors in $\reals^{d^2}$. Now, consider the $\det(\mathbb{M}) = 0$. Since every entry of $\mathbb{M}$ is semantically different, the determinant is a non-zero polynomial. Note that there are $\frac{d(d+1)}{2}$ many degrees of freedom for the rows. Thus, it is clear that the zero set $\curlybracket{\det(\mathbb{M}) = 0}$ has Lebesgue measure zero if $\ell < \frac{d(d+1)}{2}$, i.e. $\mathbb{M}$ requires at least $\frac{d(d+1)}{2}$ columns for $\det(\mathbb{M})$ to be identically zero. But this implies that set $\curlybracket{x_ix_i^{\top}}_{i=1}^\ell$ can't span $M$ (almost surely) if $\ell \le \frac{d(d+1)}{2} - 1$.
Hence, almost surely the teacher can't devise a teaching set $\cT(\cX_n,\maha, M^*)$ for oblivious teaching in \algoref{alg: randmaha}.
%must have at least $\frac{d(d+1)}{2} - 1$ pairs. 
In other words, if $\ell \le \frac{d(d+1)}{2} - 1$,
\begin{align*}
    \cP_{\cX_\ell}\paren{ \textnormal{teacher devises} \textnormal{ an oblivious teaching set } \cT \textnormal{ up to linear scaling relation } \sim_{R_l}} = 0
\end{align*}
But this shows that one can't achieve non-zero accuracy over any random sample $\cX_{\ell}$ if $\ell = \Omega(\frac{d(d+1)}{2})$ which proves the claim of the lemma.

%Now, using Sard's theorem it is clear that if $r < \frac{d(d+1)}{2} - 1$ then the measure of the samples for which $\det(\mathbb{M}) = 0$ is zero and the lower bound holds.
\end{proof}

\subsection{Approximation of a smooth distance function via local Mahalanobis distance functions}
\begin{theorem} Consider a compact separable metric space $\cX \subset \reals^n$ with a metric $d$ that is a $C^2$-map in one argument. Consider a distribution $\mu$ over $\cX$ which is $L$-lipschitz. %for some constant $\frac{1}{\lambda_{\textsf{diam}}}\ge L > 0$ where $\lambda_{\textsf{diam}} = \textsf{diam}(\cX)$. 
Then, there exists a metric $d'$ which can be taught in $\cN(\cX, \frac{\epsilon}{4})\times (n^2 + \cN(\cX, \frac{\epsilon}{4}))$ triplet comparisons achieving an $\epsilon$-error, i.e. there exists a constant $C > 0$ such that
\begin{align*}
    \expctover{(x,y,z) \sim \cX^3}{\cR_{d'}(x,y,z)} \le C\cdot\epsilon
\end{align*}
    
\end{theorem}
%\begin{algorithm}[t]
%\caption{Teaching a general distance metric with triplet comparisons}
%\label{alg: tree}
%\textbf{Given}: Input space $\cX_n \sim \cD_{\cX}$, metric model $\cM_{\mathsf{maha}}$\\
%\vspace{1mm}
%\textit{In batch setting:}\vspace{1mm}
%\begin{enumerate}
%    \item teacher picks pairs $\cT(\cX,d_{w^*}, \tree) =$ $\curlybracket{(t,u,v) \in V \,|\, \sum_{e' \in P_G^*(t, u)} w^*(e') \ge \sum_{e' \in P_G^*(u,v)} w^*(e')}$
%    \item learner receives $\cT$; and obliviously picks a metric $d \in \tree$ as per \eqnref{eq: sol}
%\end{enumerate}
%\end{algorithm}

\begin{proof}
    The proof works in two parts which are described as follows:
    \begin{enumerate}
        \item[i)] \tt{Global approximation:} Consider an $\frac{\epsilon}{4}$-cover $\cC(\cX,\epsilon,d)$, denoted as $\cC_{\cX}$ (in short) with centers $\cD$. The teacher provides triplet comparisons to teach a graph metric on $(\cD, C_{n})$, with the metric denoted as $d_G$, where $C_{n}$ is a complete graph on $\cD$ such that
        \begin{align*}
            x,y,z \in \cD,\quad (x,y,z)_{d_G} \textit{ iff } (x,y,z)_{d}
        \end{align*}
        For oblivious teaching,  $\cN(\cX,\frac{\epsilon}{4})^2$ (square of covering number) many triplet comparisons are sufficient for $d_G$.
        \item[ii)] \tt{Local approximation:} For each center $x \in \cD$, consider local linear approximations $\sf{H}_x d$ with error bounded by $\xi$ where $\xi \le \frac{\epsilon}{6}$ (see \eqnref{eq: 1}). For each hessian operator $\sf{H}_x$, teacher provides triplets based on its eigendecompositon as discussed in previous sections. Each linear transformations requires at most $n^2$ (where $n$ is the ambient space dimension) many triplets. We use the notation $\hat{\sf{H}}_x$ for a taught matrix against the correct hessian matrix $\sf{H}_x$ for any $x \in \cX$. 
    \end{enumerate}
    Now, we discuss how a learner assigns comparisons for $d'$ for a given triplet $(x_1,x_2,x_3) \in \cX^3$. Denote by $\sf{NN}: \cX \to \cX$, a nearest neighbor map based on local linear approximations (for the centers $\cD$ as shown in ii)). Thus, 
    \begin{align*}
        \sf{NN}(x) = \argmin_{x' \in \cD} \paren{(x-x')\hat{\sf{H}}_{x'}(x-x')^{\top}}
    \end{align*}
    We break the ties arbitrarily. This could potentially create some error in assigning the right comparison, which would at the worst only add a constant multiple of $\epsilon$ on the error. Denote by $\hat{x}$, the nearest neighbor of $x \in \cX$, i.e.  $\hat{x} \in \sf{NN}(x)$.
    $d'$ assigns the comparison on the triplet $\hat{x}_1, \hat{x}_2,\hat{x}_3$ as follows:
    %Now, the learner constructs a metric $d'$ to assign relation on any arbitrary triplet $(x_1,x_2,x_3) \in \cX^3$ as follows: %(this could be extended to quadruples similarly):
    %\begin{align}
    %    d'(x_1,x_2) \ge d'(x_1,x_3) = (\sf{NN}(x_1), \sf{NN}(x_2), \sf{NN}(x_3))_{d_G} \label{eq: assign}
    %\end{align}
    %Here, $\sf{NN}$ denotes the nearest neighbor based on local linear approximations. In the \eqnref{eq: assign} above, we assume that $\sf{NN}(x_2)$ is closer to $\sf{NN}(x_1)$ than $\sf{NN}(x_3)$, otherwise the relation could be written accordingly.
    %There are two possibilities for the 
    \begin{align}\label{eq: approxmetric}
        (x_1,x_2,x_3)_{d'} = \begin{cases}
            (\hat{x}_1,\hat{x}_2,\hat{x}_3)_{d_G} & \textit{ if } 
            \hat{x}_1 \neq \hat{x}_2 \textit{ or } \hat{x}_1 \neq \hat{x}_3\\
            (\hat{x}_1,\hat{x}_2,\hat{x}_3)_{d_{\sf{local}}}  & \textit{ if }   \hat{x}_1 = \hat{x}_2 = \hat{x}_3       \textit{ and } ({x}_1 -  {x}_2) \hat{\sf{H}}_{x_1}({x}_1 -  {x}_2)^{\top} \ge ({x}_1 -  {x}_3) \hat{\sf{H}}_{x_1}({x}_1 -  {x}_3)^{\top}\\
            (\hat{x}_1,\Hat{x}_3,\hat{x}_2)_{d_{\sf{local}}}  & \textit{ if }   \hat{x}_1 = \hat{x}_2 = \hat{x}_3       \textit{ and } ({x}_1 -  {x}_3) \hat{\sf{H}}_{x_1}({x}_1 -  {x}_3)^{\top}  \ge   ({x}_1 -  {x}_2) \hat{\sf{H}}_{x_1}({x}_1 -  {x}_2)^{\top}
        \end{cases}
    \end{align}
    where $d_{\sf{local}}$ is written as an abuse of notation for $d_{\hat{\sf{H}}_x}$ for any $x \in \cX$. 
    
     So, the learner answers: if ``$x_2$ is closer to $x_1$ than $x_3$'' by first finding the nearest neighbors of these points and then checking the corresponding relation to these in the graph metric $(\cD, d_G)$ and local Mahalanobis metrics induced by $\{\hat{\sf{H}}_x: x \in \cD\}$.

    Note, that the specific choice of the cover $\cC_{\cX}$ in i), and of the approximation constant $\xi$ for local approximation in ii), ensures that if any points $y,z$ are far apart wrt a given $x$ then the distances are correctly captured, i.e. for any $x,y \in \cX$:
    \begin{align}
        d(x,y) - \epsilon \le d(\hat{x}, \hat{y}) \le  d(x,y) + \epsilon \label{eq: apdist}
    \end{align}
    This in turn gives
    \begin{align*}
      \forall x,y,z \in \cX,\quad  d(x,y) > d(x,z) + 2\epsilon \implies d(\hat{x},\hat{y}) > d(\hat{x},\hat{z}) \implies d'(\hat{x},\hat{y}) \ge d'(\hat{x},\hat{z}).
    \end{align*}
     So, $d'$ could make error in assigning a correct comparison on a triplet $x,y,z$ but only if $|d(x,y) - d(x,z)| \le 2\epsilon$. In order to show the bound on the expected error of $\cR_{d'}$, we consider this case.
    %First, we note that for a given triplet $(x,y,z) \in \cX^3$, if  then \eqnref{eq: approxmetric} gives the correct relation as the comparison reduces to the graph metric. 
    
    
    Now, we will discuss the approximation of the error. First, note that by definition of conditional expectation
\begin{align*}
    \expctover{(x,y,z) \sim \cX^3}{\cR_{d'}(x,y,z)} = \expctover{x}{\expctover{(y,z\,|\,x)}{\cR_{d'}(x,y,z)}} 
\end{align*}
%\le \sum_{x \in \cD:  B(x,\epsilon)} \expctover{(y,z\,|\,x)}{\cR_{d'}(x,y,z)}\cdot \cP(B(x,\epsilon))

Now, we will analyze $\expctover{(y,z\,|\,x)}{\cR_{d'}(x,y,z)}$ for the cases where $|d(x,y) - d(x,z)| \le 2\epsilon$. But this condition leads to two possibility: either $y,z \in B(x,2\epsilon)$, or $y,z \in B(x,r + 2\epsilon)\setminus B(x,r)$ for some $r > 0$.


Thus, we can write:
\begin{align*}
    &\expctover{(y,z\,|\,x)}{\cR_{d'}(x,y,z)}\\
    &\le \underbrace{\expctover{(y,z\,|\,x)}{\mathds{1}[y,z \in B(x,2\epsilon)]\cdot \cR_{d'}(x,y,z)}}_{(\star)} + \underbrace{\expctover{(y,z\,|\,x)}{\mathds{1}[\exists r>0, y,z \in B(x,r + 2\epsilon)\setminus B(x,r)]\cdot\cR_{d'}(x,y,z)}}_{(\star \star)}
\end{align*}
But by definition, the first term can be bounded as follows:
\begin{align*}
    (\star) \le \cP_{(y,z|x)} (y,z \in B(x,2\epsilon)) = \cP_{(y|x,z)} (y \in B(x,2\epsilon)) \cP_{(z|x)} (z \in B(x,2\epsilon)) \le L^2\cdot (2\epsilon)^{2\alpha},
\end{align*}
where we have used the observation from \eqnref{eq: rbound} and independence of the variables $y$ and $z$. We can bound the second term as
\begin{center}
\begin{align*}
    (\star \star) &\le   \int_{2\epsilon}^{\lambda_{\textsf{diam}}} \cP_{(y,z\,|\, x)}( y,z \in B(x,r+2\epsilon))\setminus B(x,r)) d r\\
    %&=  \int_{2\epsilon}^{\lambda_{\textsf{diam}}} \cP_{(y\,|\, x,z)}( y \in B(x,r+2\epsilon))\setminus B(x,r))\cdot \cP_{(z\,|\, x)}( z \in B(x,r+2\epsilon))\setminus B(x,r)) dr\\
    & \le \int_{2\epsilon}^{\lambda_{\textsf{diam}}} L^2\cdot (r + 2\epsilon  - r )^{2\alpha} dr\\
    &= (\lambda_{\textsf{diam}} - 2\epsilon)\cdot L^2(2\epsilon)^{2\alpha}\\ 
    %&\le 4(1 + \lambda_{\textsf{diam}}L^2)\cdot \epsilon^2\\
    %&\le 5L\cdot \epsilon^2
\end{align*}
\end{center}
Adding the bounds for $(\star)$ and $(\star\star)$
\begin{align}
    (\star) + (\star \star) \le L^2\cdot (2\epsilon)^{2\alpha} + (\lambda_{\textsf{diam}} - 2\epsilon)\cdot L^2(2\epsilon)^{2\alpha} = (\lambda_{\textsf{diam}} - 2\epsilon + 1)\cdot L^2(2\epsilon)^{2\alpha} < (\lambda_{\textsf{diam}} + 1)L^2(2\epsilon)^{2\alpha}
\end{align}
Thus,
\begin{align*}
    \expctover{(x,y,z) \sim \cX^3}{\cR_{d'}(x,y,z)} = \expctover{x}{\expctover{(y,z\,|\,x)}{\cR_{d'}(x,y,z)}} \le \expctover{x}{ (\lambda_{\textsf{diam}} + 1)L^2(2\epsilon)^{2\alpha}} =  (\lambda_{\textsf{diam}} + 1)L^2(2\epsilon)^{2\alpha}
\end{align*}
It takes $\cN(\cX, \frac{\epsilon}{4})\times (n^2 + \cN(\cX, \frac{\epsilon}{4}))$ triplet comparisons to teach $d'$, as each Mahalanobis distance metric induced by $\{\hat{\sf{H}}_x: x \in \cD\}$ requires $n^2$ triplets and $|\cD| = \cN(\cX, \frac{\epsilon}{4})$.

This completes the proof of the theorem.
\end{proof}
%\usetikzlibrary{shapes.geometric, calc}



\newpage




\end{document}
