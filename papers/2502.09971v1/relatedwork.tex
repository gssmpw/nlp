\section{Related Work and Unique Contributions}
\label{sec:related_work}
Deep learning-based image compression has achieved remarked progress in recent years. Ballé \textit{et al.} \cite{balle2017end} pioneered an end-to-end optimizable architecture, later enhancing it with a hyperprior model \cite{balle2018variational} to improve entropy estimation. Transformer architectures have been proposed by Qian \textit{et al.} \cite{qian2022entroformer} to improve probability distribution estimation. Similarly, Cheng \textit{et al.} \cite{cheng2020learned} parameterizes the distributions of latent codes with discretized Gaussian Mixture models. Liu \textit{et al.} \cite{liu2023learned} combined CNNs and Transformers in the TCM block to explore the local and non-local source correlation. Yang \textit{et al.} \cite{yang2023tinc} proposed a Tree-structured Implicit Neural Compression (TINC) to maintain the continuity among regions and remove the local and non-local redundancy. To enhance the entropy coding performance, the conditional probability model and joint autoregressive and hierarchical priors model have been developed in \cite{mentzer2018conditional, minnen2018joint}. Jia \textit{et al.} \cite{jia2024generative} introduced a Generative Latent Coding (GLC) architecture to achieve high-realism and high-fidelity compression by transform coding in the latent space. 

This work is related to reference-based deep image compression, where reference information is used to improve coding efficiency. For example, Li \textit{et al.} \cite{li2021deep} pioneered this approach in video compression, while Ayzik \textit{et al.} \cite{ayzik2020deep} applied it at the decoder level. Sheng \textit{et al.} \cite{sheng2022temporal} proposed a temporal context mining module to propagate features and learn multi-scale temporal contexts. Huang \textit{\textit{et al.}} \cite{huang2023learned} extended the concept to multi-view image compression with advanced feature extraction and fusion. Li \textit{et al.} \cite{li2023neural} introduced the group-based offset diversity to explore the image context for better prediction. Zhao \textit{et al.} \cite{zhao2021universal} optimized the reference information using a universal rate-distortion optimization framework. \cite{zhao2023universal} integrated side information optimization with latent optimization to further enhance the compression ratio. In \cite{li2023rfd}, within the context of underwater image compression, a multi-scale feature dictionary was manually created to provide a reference for deep image compression based on feature matching. A content-aware reference frame selection method was developed in \cite{wu2022content} for deep video compression. 

\textbf{Unique contributions.} 
In comparison to existing methods, our work has the following unique contributions. (1) We develop a new approach, called conditional latent coding (CLC), which learns to synthesize a dynamic reference for each input image to achieve highly efficient conditional coding in the latent domain. 
(2) We develop a fast and efficient feature matching scheme based on ball tree search and an effective feature alignment strategy that dynamically balances compression bit-rate and reconstruction quality. (3) We developed a theoretical analysis to show that the proposed CLC method is robust to perturbations in the external dictionary samples and the selected conditioning latent, with an error bound that scales logarithmically with the dictionary size, ensuring stability even with large and diverse dictionaries.

\begin{figure}[tb]
    \centering
    \includegraphics[width=\linewidth]{figure/figure_main.pdf}
    \caption{Overview of the proposed Conditional Latent Coding (CLC) framework.}
    %  \vspace{-0.2cm}
    \label{fig:main0810}
\end{figure}