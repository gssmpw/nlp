\def\year{2022}\relax
%File: formatting-instructions-latex-2022.tex
%release 2022.1
\documentclass[letterpaper]{article} % DO NOT CHANGE THIS
% \usepackage[draft]{aaai25}  % DO NOT CHANGE THIS
\usepackage{aaai25}
\usepackage{mathrsfs}
\usepackage{times}  % DO NOT CHANGE THIS
\usepackage{helvet}  % DO NOT CHANGE THIS
\usepackage{courier}  % DO NOT CHANGE THIS
\usepackage[hyphens]{url}  % DO NOT CHANGE THIS
\usepackage{graphicx} % DO NOT CHANGE THIS
\usepackage{booktabs}
\urlstyle{rm} % DO NOT CHANGE THIS
\def\UrlFont{\rm}  % DO NOT CHANGE THIS
\usepackage{natbib}  % DO NOT CHANGE THIS AND DO NOT ADD ANY OPTIONS TO 
\usepackage{caption} % DO NOT CHANGE THIS AND DO NOT ADD ANY OPTIONS TO IT

\DeclareCaptionStyle{ruled}{labelfont=normalfont,labelsep=colon,strut=off} % DO NOT CHANGE THIS
\frenchspacing  % DO NOT CHANGE THIS
\setlength{\pdfpagewidth}{8.5in}  % DO NOT CHANGE THIS
\setlength{\pdfpageheight}{11in}  % DO NOT CHANGE THIS
%
\usepackage{multirow}
% \usepackage{graphicx}
% \usepackage{booktabs}
% \usepackage{hyperref}
% \usepackage{cleveref}
% These are recommended to typeset algorithms but not required. See the subsubsection on algorithms. Remove them if you don't have algorithms in your paper.
% \usepackage{algorithm}
% \usepackage{algorithmic}
\usepackage[linesnumbered, ruled, vlined]{algorithm2e}
\usepackage{amsmath,amssymb,amsthm}
\usepackage{bm}
\newtheorem{assumption}{Assumption}
\newtheorem{lemma}{Lemma}
\newtheorem{theorem}{Theorem}
\newtheorem{remark}{Remark}
\newcommand{\step}[1]{\noindent\textbf{Step #1.} }
% \newtheorem{proof}{Proof}
%
\usepackage{xcolor}
\newcommand{\choiceYes}{\textcolor{blue}{Yes}}
\newcommand{\choiceNo}{\textcolor{red}{No}}
\newcommand{\choicePartial}{\textcolor{orange}{Partial}}
\newcommand{\choiceNA}{\textcolor{gray}{NA}}
\usepackage{amsmath}
% These are are recommended to typeset listings but not required. See the subsubsection on listing. Remove this block if you don't have listings in your paper.
\usepackage{newfloat}
\usepackage{listings}
\usepackage{float}
\lstset{%
	basicstyle={\footnotesize\ttfamily},% footnotesize acceptable for monospace
	numbers=left,numberstyle=\footnotesize,xleftmargin=2em,% show line numbers, remove this entire line if you don't want the numbers.
	aboveskip=0pt,belowskip=0pt,%
	showstringspaces=false,tabsize=2,breaklines=true}
\floatstyle{ruled}
\newfloat{listing}{tb}{lst}{}
\floatname{listing}{Listing}
%
%\nocopyright
%
% PDF Info Is REQUIRED.
% For /Title, write your title in Mixed Case.
% Don't use accents or commands. Retain the parentheses.
% For /Author, add all authors within the parentheses,
% separated by commas. No accents, special characters
% or commands are allowed.
% Keep the /TemplateVersion tag as is
\pdfinfo{
/TemplateVersion (2025.1)
}

% DISALLOWED PACKAGES
% \usepackage{authblk} -- This package is specifically forbidden
% \usepackage{balance} -- This package is specifically forbidden
% \usepackage{color (if used in text)
% \usepackage{CJK} -- This package is specifically forbidden
% \usepackage{float} -- This package is specifically forbidden
% \usepackage{flushend} -- This package is specifically forbidden
% \usepackage{fontenc} -- This package is specifically forbidden
% \usepackage{fullpage} -- This package is specifically forbidden
% \usepackage{geometry} -- This package is specifically forbidden
% \usepackage{grffile} -- This package is specifically forbidden
% \usepackage{hyperref} -- This package is specifically forbidden
% \usepackage{navigator} -- This package is specifically forbidden
% (or any other package that embeds links such as navigator or hyperref)
% \indentfirst} -- This package is specifically forbidden
% \layout} -- This package is specifically forbidden
% \multicol} -- This package is specifically forbidden
% \nameref} -- This package is specifically forbidden
% \usepackage{savetrees} -- This package is specifically forbidden
% \usepackage{setspace} -- This package is specifically forbidden
% \usepackage{stfloats} -- This package is specifically forbidden
% \usepackage{tabu} -- This package is specifically forbidden
% \usepackage{titlesec} -- This package is specifically forbidden
% \usepackage{tocbibind} -- This package is specifically forbidden
% \usepackage{ulem} -- This package is specifically forbidden
% \usepackage{wrapfig} -- This package is specifically forbidden
% DISALLOWED COMMANDS
% \nocopyright -- Your paper will not be published if you use this command
% \addtolength -- This command may not be used
% \balance -- This command may not be used
% \baselinestretch -- Your paper will not be published if you use this command
% \clearpage -- No page breaks of any kind may be used for the final version of your paper
% \columnsep -- This command may not be used
% \newpage -- No page breaks of any kind may be used for the final version of your paper
% \pagebreak -- No page breaks of any kind may be used for the final version of your paperr
% \pagestyle -- This command may not be used
% \tiny -- This is not an acceptable font size.
% \vspace{- -- No negative value may be used in proximity of a caption, figure, table, section, subsection, subsubsection, or reference
% \vskip{- -- No negative value may be used to alter spacing above or below a caption, figure, table, section, subsection, subsubsection, or reference

\setcounter{secnumdepth}{2} %May be changed to 1 or 2 if section numbers are desired.

% The file aaai22.sty is the style file for AAAI Press
% proceedings, working notes, and technical reports.
%

% Title

% Your title must be in mixed case, not sentence case.
% That means all verbs (including short verbs like be, is, using,and go),
% nouns, adverbs, adjectives should be capitalized, including both words in hyphenated terms, while
% articles, conjunctions, and prepositions are lower case unless they
% directly follow a colon or long dash
\title{Conditional Latent Coding with Learnable Synthesized Reference for Deep Image Compression}
\author{
% Authors
% All authors must be in the same font size and format.
Siqi Wu\textsuperscript{\rm 1}\thanks{This work was completed during Siqi Wu's visiting research period at Southern University of Science and Technology.},
Yinda Chen\textsuperscript{\rm 2}\thanks{Co-first author.},
Dong Liu\textsuperscript{\rm 2},
Zhihai He\textsuperscript{\rm 3}\thanks{Corresponding author. Email: hezh@sustech.edu.cn.}
}
\affiliations{
% Affiliations
\textsuperscript{\rm 1}University of Missouri, Columbia, MO, USA\\
\textsuperscript{\rm 2}University of Science and Technology of China, Hefei, China\\
\textsuperscript{\rm 3}Southern University of Science and Technology, Shenzhen, China
}

% REMOVE THIS: bibentry
% This is only needed to show inline citations in the guidelines document. You should not need it and can safely delete it.
\usepackage{bibentry}
% END REMOVE bibentry

\begin{document}

\maketitle

\begin{abstract}
In this paper, we study how to synthesize a dynamic reference from an external dictionary to perform conditional coding of the input image in the latent domain and how to learn the conditional latent synthesis and coding modules in an end-to-end manner.
Our approach begins by constructing a universal image feature dictionary using a multi-stage approach involving modified spatial pyramid pooling, dimension reduction, and multi-scale feature clustering. For each input image, we learn to synthesize a conditioning latent by selecting and synthesizing relevant features from the dictionary, which significantly enhances the model's capability in capturing and exploring image source correlation. This conditional latent synthesis involves a correlation-based feature matching and alignment strategy, comprising a Conditional Latent Matching (CLM) module and a Conditional Latent Synthesis (CLS) module. The synthesized latent is then used to guide the encoding process, allowing for more efficient compression by exploiting the correlation between the input image and the reference dictionary. According to our theoretical analysis, the proposed conditional latent coding (CLC) method is robust to perturbations in the external dictionary samples and the selected conditioning latent, with an error bound that scales logarithmically with the dictionary size, ensuring stability even with large and diverse dictionaries. Experimental results on benchmark datasets show that our new method improves the coding performance by a large margin (up to 1.2 dB) with a very small overhead of approximately 0.5\%  bits per pixel. 
% Our code is publicly available at \url{https://github.com/ydchen0806/CLC}.
\end{abstract}
\begin{links}
    \link{Code}{https://github.com/ydchen0806/CLC}.
\end{links}
\section{Introduction}

With the rapid development of the Internet and mobile devices, billions of images are available in the world. For a given image, it is easy to find many correlated images on the Internet. It will be very interesting to explore how to utilize this vast amount of data to establish a highly efficient representation of the input image to improve the performance of deep image compression.
Continuous efforts have been made in the past two decades. The early attempt is to extract low-level feature patches from external images as a dictionary for image super-resolution \cite{sun2003resolution} and quality enhancement \cite{xiong2010robust}. Yue \textit{\textit{et al.}} \cite{yue2013cloud} proposed a cloud-based image coding scheme that utilizes a large-scale image database for reconstruction, achieving high compression ratios while maintaining visual quality. As data compression has shifted to the deep image/video compression paradigm in recent years, we would like to explore how to utilize the external dictionary of images to generate a dynamic reference representation to perform conditional coding of the input image within the deep image compression framework. 


Deep neural network-based image compression methods \cite{balle2017end, toderici2017full, lee2019context} have made significant progress in recent years, surpassing traditional transform coding methods like JPEG in compression efficiency. However, current deep learning compression still faces challenges in efficiently exploring the source correlation of the image and maintaining high reconstruction quality at low bit rates. To further improve compression efficiency, researchers have begun to explore the use of external images as side information in distributed deep compression. For example, Ayzik \textit{et al.} \cite{ayzik2020deep} used auxiliary image information to perform block matching in the image domain, while Huang \textit{et al.} \cite{huang2023learned} extended this concept by introducing a multi-scale patch matching approach. However, this approach relies on specific auxiliary images, limiting its applicability and improvement.

% \begin{figure}[t]
%     \centering
%     \includegraphics[width=\linewidth]{figure/teasor2.pdf}
%     \caption{Overview of the proposed CLC: Key contributions (highlighted in red boxes) include conditional latent encoding/decoding and generating conditions from an external dictionary.}
%     \label{fig:teasor}
% \end{figure}

To overcome these limitations, we propose a novel framework called Conditional Latent Coding (CLC), which uses auxiliary information as a conditional probability at both the encoder and decoder. Our approach constructs a universal image feature dictionary using a multi-stage process involving modified spatial pyramid pooling (SPP), dimensionality reduction, and multi-scale feature clustering. For each input image, we generate a conditioning latent by adaptively selecting and learning to combine relevant features from the dictionary to generate a highly efficient reference representation, called \textit{conditioning latent}, for the input image. 
We then apply an advanced feature matching and alignment strategy, comprising a Conditional Latent Matching (CLM) module and a Conditional Latent Synthesis (CLS) module. This process leverages the conditioning latent to guide the encoding process, allowing for more efficient compression by exploiting similarities between the input image and the reference features. As demonstrated in our experimental results on benchmark datasets, our new method improves the coding performance by a large margin (up to 1.2 dB) at low bit-rates.


\section{Related Work and Unique Contributions} \label{sec:related_work}
Deep learning-based image compression has achieved remarked progress in recent years. Ballé \textit{et al.} \cite{balle2017end} pioneered an end-to-end optimizable architecture, later enhancing it with a hyperprior model \cite{balle2018variational} to improve entropy estimation. Transformer architectures have been proposed by Qian \textit{et al.} \cite{qian2022entroformer} to improve probability distribution estimation. Similarly, Cheng \textit{et al.} \cite{cheng2020learned} parameterizes the distributions of latent codes with discretized Gaussian Mixture models. Liu \textit{et al.} \cite{liu2023learned} combined CNNs and Transformers in the TCM block to explore the local and non-local source correlation. Yang \textit{et al.} \cite{yang2023tinc} proposed a Tree-structured Implicit Neural Compression (TINC) to maintain the continuity among regions and remove the local and non-local redundancy. To enhance the entropy coding performance, the conditional probability model and joint autoregressive and hierarchical priors model have been developed in \cite{mentzer2018conditional, minnen2018joint}. Jia \textit{et al.} \cite{jia2024generative} introduced a Generative Latent Coding (GLC) architecture to achieve high-realism and high-fidelity compression by transform coding in the latent space. 

This work is related to reference-based deep image compression, where reference information is used to improve coding efficiency. For example, Li \textit{et al.} \cite{li2021deep} pioneered this approach in video compression, while Ayzik \textit{et al.} \cite{ayzik2020deep} applied it at the decoder level. Sheng \textit{et al.} \cite{sheng2022temporal} proposed a temporal context mining module to propagate features and learn multi-scale temporal contexts. Huang \textit{\textit{et al.}} \cite{huang2023learned} extended the concept to multi-view image compression with advanced feature extraction and fusion. Li \textit{et al.} \cite{li2023neural} introduced the group-based offset diversity to explore the image context for better prediction. Zhao \textit{et al.} \cite{zhao2021universal} optimized the reference information using a universal rate-distortion optimization framework. \cite{zhao2023universal} integrated side information optimization with latent optimization to further enhance the compression ratio. In \cite{li2023rfd}, within the context of underwater image compression, a multi-scale feature dictionary was manually created to provide a reference for deep image compression based on feature matching. A content-aware reference frame selection method was developed in \cite{wu2022content} for deep video compression. 

\textbf{Unique contributions.} 
In comparison to existing methods, our work has the following unique contributions. (1) We develop a new approach, called conditional latent coding (CLC), which learns to synthesize a dynamic reference for each input image to achieve highly efficient conditional coding in the latent domain. 
(2) We develop a fast and efficient feature matching scheme based on ball tree search and an effective feature alignment strategy that dynamically balances compression bit-rate and reconstruction quality. (3) We developed a theoretical analysis to show that the proposed CLC method is robust to perturbations in the external dictionary samples and the selected conditioning latent, with an error bound that scales logarithmically with the dictionary size, ensuring stability even with large and diverse dictionaries.

\begin{figure}[tb]
    \centering
    \includegraphics[width=\linewidth]{figure/figure_main.pdf}
    \caption{Overview of the proposed Conditional Latent Coding (CLC) framework.}
    %  \vspace{-0.2cm}
    \label{fig:main0810}
\end{figure}

\section{The Proposed CLC Method} \label{sec:method} 

\subsection{Method Overview}

% Given an input image $X \in \mathbb{R}^{H \times W \times 3}$ and a pre-trained feature reference dictionary $\mathcal{D} = \{\mathbf{d}_1, \mathbf{d}_2, ..., \mathbf{d}_K\}$, our method enhances image compression by leveraging dictionary information. We first extract features from $x$ to query $\mathcal{D}$, reconstructing a reference image $x_r$. Both $X$ and $x_r$ are processed through identical network paths, using an analysis transform $g_a$ to obtain latent representations $\mathbf{y}$ and $\mathbf{y}_r$, respectively. 

The overall architecture of our proposed CLC framework is illustrated in Figure~\ref{fig:main0810}. Given an input image $x$, we first construct a pre-trained feature reference dictionary $D$ from a large reference image dataset using a multi-stage approach involving feature extraction with modified spatial pyramid pooling (SPP), dimensionality reduction, and multi-scale feature clustering. Then, given an input image $x$, we extract its feature using an encoder $F_\theta$ which is used to query the dictionary $\mathcal{D}$ and find the top $M$ best-matching reference images $X_r^M=\{x_r^1, x_r^2, \cdots, x_r^M\}$. In this work, the default value of $M$ is 3. Both $x$ and the queried reference $X_r^M$ are passed through the encoder transform network $g_a$ to obtain their latent representations $y$ and $Y_r^M$, respectively. Using $Y_r^M$ as reference, we obtain $y_f$ through adaptive feature matching and multi-scale alignment and then learn a network to perform conditional latent coding of $y_f$. Simultaneously, a hyperprior network $h_a$ estimates a hyperprior $z$ from $y_f$ to provide additional context for entropy estimation.
A slice-based autoregressive context model is used for entropy coding, dividing $y_f$ into slices and using both $z$ and previously coded elements to estimate probabilities. During decoding, we first reconstruct $z$ and $y_f$ from the bitstream, then use the dictionary indices passed from the encoder to apply the same reference processing and alignment procedure to reconstruct $y$ from $y_f$, and finally reconstruct the image $\hat{x}$ using the synthesis transform $g_s$.
In the following section, we will explain the proposed CLC method in more detail.

% A hyperprior network $h_a$ further compresses these representations to generate $\mathbf{z}$ and $\mathbf{z}_r$. A patch matching module aligns $\mathbf{y}$ with $\mathbf{y}_r$, followed by an alignment module that fuses the matched features. We employ a slice-based autoregressive context model with attention mechanisms, dividing $\mathbf{y}$ into $K$ slices and using both previously processed slices and the aligned reference features to estimate the probability distribution of each slice. An improved KV-cache compression technique is applied to the attention mechanism to reduce memory usage. The estimated probabilities guide the entropy coding of $\mathbf{y}$ and $\mathbf{z}$, generating the final bitstream $b$. During decoding, we reconstruct $\hat{\mathbf{z}}$ and $\hat{\mathbf{y}}$ from $b$, and use the same reference image processing and alignment procedure to enhance the reconstruction. A synthesis transform $g_s$ then produces the final reconstructed image $\hat{x}$. The network is optimized end-to-end to minimize the rate-distortion function $L = D(X, \hat{x}) + \lambda(\mathbf{p}) R(b)$, where $\lambda(\mathbf{p})$ is an adaptive Lagrange multiplier. Our approach integrates aligned reference information at both encoder and decoder, leveraging rich prior knowledge to enhance compression efficiency and reconstruction quality.

\subsection{Constructing the Support Dictionary}
As stated in the above section, our main idea is to construct a universal feature dictionary from which a reference latent can be dynamically generated to perform conditional latent coding of each image.  Here, a critical challenge is constructing a universal feature dictionary that effectively represents diverse image content and enables efficient feature utilization throughout the compression pipeline. We address this challenge using a multi-stage approach that combines advanced feature extraction, dimensionality reduction, feature clustering, and fast and efficient dictionary access by the deep image compression system, as illustrated in Figure~\ref{fig:method1_framework}.

\begin{figure}[tb]
    \centering
    \includegraphics[width=\linewidth]{figure/method1.pdf}
    \caption{Universal Feature Dictionary Construction. (a) Dictionary Generation using diverse images $\mathcal{R}$ to create initial $\mathcal{D}$. (b) Reference Retrieval for querying and updating dictionary with inputs $x$ and $x'$. (c) Examples of reference candidates $X_r^M$ retrieved from the dictionary.}
    \label{fig:method1_framework}
\end{figure}

\paragraph{(1) Constructing the reference feature dictionary.} Our method begins with a large  reference dataset $\mathcal{R} = \{x_1, x_2, ..., x_N\}$. 
In this work, we randomly download 3000 images from the web. 
We use a modified pre-trained ResNet-50 model with Spatial Pyramid Pooling (SPP) as our feature extractor. For each image $x_i$, we extract its feature $\mathbf{v}_i = \text{SPP}(f_\theta(x_i))$, where $f_\theta(\cdot)$ represents the ResNet-50 backbone, and SPP aggregates features at scales $\{1\times1, 2\times2, 4\times4\}$. This multi-scale approach captures both global and local image characteristics.

To manage the high dimensionality of these features, we apply Principal Component Analysis (PCA), reducing each vector to 256 dimensions $\hat{\mathbf{v}}_i$. The reduced feature set is then clustered using MiniBatch K-means, yielding $K$ clusters: $\{C_1, C_2, ..., C_K\}$. From each cluster $C_j$, we select the feature vector closest to the centroid as its representative: $\mathbf{d}_j = \arg\min_{\hat{\mathbf{v}} \in C_j} \|\hat{\mathbf{v}} - \boldsymbol{\mu}_j\|_2$, where $\boldsymbol{\mu}_j$ is the centroid of $C_j$. These representatives form our feature dictionary $\mathcal{D} = \{\mathbf{d}_1, \mathbf{d}_2, ..., \mathbf{d}_K\}$.

\paragraph{(2) Fast and efficient dictionary matching.}
Our proposed CLC method deep image compression needs to access this dictionary during training and inference. 
One central challenge here is the dictionary search and matching efficiency. 
For efficient feature dictionary management and access, we introduce a KV-cache mechanism that is employed in both the initial feature retrieval and the subsequent encoding-decoding process. Specifically, we define our KV-cache as a tuple $(\mathbf{K}, \mathbf{V})$, where $\mathbf{K} \in \mathbb{R}^{N \times d_k}$ represents the keys and $\mathbf{V} \in \mathbb{R}^{N \times d_v}$ represents the values. Here, $N$ is the number of entries in the cache, $d_k$ is the dimension of the keys, and $d_v$ is the dimension of the values.

In the feature retrieval phase, we construct a ball tree over $\mathcal{D}$ for the initial coarse search, while maintaining the KV-cache. During compression, given an input image $x$, we extract its feature $f_\theta(x)$ and use it to query both the Ball Tree and the KV-cache. The retrieval process is formulated as a scaled dot-product attention mechanism:
\begin{equation}
    A(\mathbf{Q}, \mathbf{K}, \mathbf{V}) = \text{softmax}\left(\frac{\mathbf{Q}\mathbf{K}^T}{\sqrt{d_k}}\right)\mathbf{V},
\end{equation}
where $\mathbf{Q} = f_\theta(x)$, and $\mathbf{K}$ and $\mathbf{V}$ are the keys and values in the KV-cache, respectively.
To manage the size of the KV-cache and improve the matching efficiency, we implement a compression technique. Let $C: \mathbb{R}^{d} \rightarrow \mathbb{R}^{d'}$ be our compression function, where $d' < d$. We apply this to both keys and values:
\begin{equation}
    \mathbf{K}_c = C(\mathbf{K}), \quad \mathbf{V}_c = C(\mathbf{V}).
\end{equation}
The compression function $C$ is designed to preserve the most important information while reducing the dimensionality. In practice, we implement $C$ as a learnable neural network layer, optimized jointly with the rest of the system.
Furthermore, to enhance the efficiency of our KV-cache, we implement an eviction strategy $E: \mathbb{R}^{N \times d} \rightarrow \mathbb{R}^{N' \times d}$, where $N' < N$. This strategy removes less useful entries from the cache based on a relevance metric $\rho: \mathbb{R}^d \rightarrow \mathbb{R}$:
\begin{equation}
    (\mathbf{K}_e, \mathbf{V}_e) = E(\mathbf{K}, \mathbf{V}) = \text{TopK}(\rho(\mathbf{K}_i), \mathbf{K}, \mathbf{V}),
\end{equation}
where $\text{TopK}$ selects the top $K$ entries based on the relevance scores.
To further enhance robustness, we implement a multi-query strategy. For an input image $x$, we generate an augmented version $x'$ (e.g., by rotation) and perform separate queries for both. The final set of reference features is obtained by merging and de-duplicating the results.

% This comprehensive approach to feature dictionary construction and utilization enables our compression system to leverage rich reference information efficiently throughout the entire compression pipeline. By integrating the KV-cache mechanism with our feature dictionary and Ball Tree structure, we create a hybrid system that maintains high performance while dealing with the computational challenges of processing and utilizing large amounts of feature information. This approach potentially leads to improved compression performance across a diverse range of input images.

\subsection{Conditional Latent Synthesis and Coding}
As the unique contribution of this work, instead of simply finding the best match in existing methods \cite{jia2024generative}, the reference or side information for each image is dynamically generated in the latent domain by a learned network to best represent the input image.
Our method is motivated by the following observation: the central challenge in reference-based image compression is the large deviation between the arbitrary input image and the fixed and limited set of reference images. Our method finds multiple closest reference images and dynamically fuses them to form a best approximation of the input image in the latent domain. Specifically, the proposed conditional latent synthesis and coding method has the following major components:

\begin{figure}[t]
    \centering
    \includegraphics[width=\linewidth]{figure/method1_2.pdf}
    \caption{The detail of our proposed CLM and CLS module.}
    \label{fig:enter-label}
\end{figure}

\paragraph{(1) Feature Matching and Alignment.} 
We first propose an advanced feature matching and alignment scheme that aligns reference features from the dictionary with the input image. Our approach begins with a Conditional Latent Matching (CLM) module. Given an input image $x \in \mathbb{R}^{H \times W \times 3}$ and a pre-built feature reference dictionary $\mathcal{D} = \{\mathbf{d}_1, \mathbf{d}_2, ..., \mathbf{d}_K\}$, we first extract features from $x$ to query $\mathcal{D}$, retrieving the top $M$ feature and their corresponding reference images $X_r^M=\{x_r^1, x_r^2, \cdots, x_r^M\}$. Both $x$ and $X_r^M$ are then processed through the same analysis transform network. In this work, we use the Transformer-CNN Mixture (TCM) block \cite{liu2023learned}, which efficiently combines the strengths of CNNs for local feature extraction and transformers for capturing long-range dependencies. TCM blocks are used at both encoder $g_a$, decoder $g_s$, and hyperprior network $h_a$, enabling effective feature processing at various stages of the compression pipeline.

The analysis transform $g_a$ converts $x$ and $x_r^M$ into latent representations $y$ and $Y_r^M$, respectively. The CLM then establishes correspondences between $Y_r^M$ and $y$, addressing the issues of spatial inconsistencies. It computes $y_m = \mathcal{F}_m(y, Y_r^M; \theta_m)$, where $\mathcal{F}_m$ is a learnable function parameterized by $\theta_m$. This function computes a similarity matrix $\mathbf{S}$ between features of $y$ and $y_r$:
\begin{equation}
    S_{ij} = \frac{\exp(\langle \phi(y_i), \phi(y_{r,j}) \rangle / \tau)}{\sum_k \exp(\langle \phi(y_i), \phi(y_{r,k}) \rangle / \tau)},
\end{equation}
where $\phi(\cdot)$ is a learnable feature transformation that maps input features to a higher-dimensional space, $\langle \cdot, \cdot \rangle$ denotes inner product, and $\tau$ is a temperature parameter. 
We also introduce a learnable alignment module within the CLM to refine the alignment between reference and target features: $y_a = \mathcal{F}_a(y, y_m; \theta_a)$, where $\mathcal{F}_a$ is implemented as a series of deformable convolution layers operating at multiple scales.

\paragraph{(2) Conditional Latent Synthesis.} In the final stage of our feature matching and alignment strategy, we develop a Conditional Latent Synthesis (CLS) module to fuse the aligned reference features with the target image feature. We model this fusion process as a conditional probability with learnable weights:
\begin{equation}
    p(y_f | y, y_a) = \mathcal{N}(\mu(y, y_a), \sigma^2(y, y_a)),
\end{equation}
where $y_f$ is the final latent representation, and $\mu(\cdot)$ and $\sigma^2(\cdot)$ are learnable functions implemented as neural networks. These functions estimate the mean and variance of the Gaussian distribution for $y_f$ conditioned on both $y$ and $y_a$. The mean function $\mu(\cdot)$ is designed to incorporate adaptive weighting:
\begin{equation}
    \mu(y, y_a) = \alpha \odot y + (1 - \alpha) \odot y_a,
\end{equation}
where $\alpha$ are dynamically computed weights based on content: $\alpha = \sigma(\mathcal{F}_w([y, y_a]; \theta_f))$. Here, $\sigma$ is the sigmoid function, and $\mathcal{F}_w$ is a small neural network predicting optimal fusion weights. This conditional generation approach with adaptive weights allows our model to capture complex dependencies between the input image and the reference image from the dictionary in the latent space, resulting in more flexible and powerful conditional coding.
During training, we sample from this distribution to obtain $y_f$, while during inference, we use the mean $\mu(y, y_a)$ as the final latent representation. This probabilistic formulation enables our model to handle uncertainties in the feature integration process and potentially generate diverse latent representations during training, which can improve the robustness and generalization capability of our deep compression system.

\paragraph{(3) Entropy Coding and Hyperprior.} To further improve compression efficiency, we introduce a hyperprior network $h_a$ that estimates a hyperprior $z$ from the conditional latent $y_f = h_a(y_f)$. This hyperprior $z$ provides additional context for more accurate probability estimation of $y_f$, enhancing the entropy model. The hyperprior is quantized and encoded separately, $\hat{z} = Q(z)$, where $Q(\cdot)$ denotes the quantization operation.

For entropy coding, we adopt a slice-based auto-regressive context model \cite{}. The conditional representation $y_f$ is divided into $K$ slices: $y_f = [y_f^1, y_f^2, ..., y_f^K]$. The probability distribution of each slice is estimated using both previously processed slices and the hyperprior information. For the $i$-th slice, the probability model is expressed as:
\begin{equation}
    p(y_f^i|y_f^{<i}, \hat{z}) = f_\theta(y_f^{<i}, \hat{z}),
\end{equation}
where $f_\theta$ is a neural network parameterized by $\theta$, and $y_f^{<i} = [y_f^1, ..., y_f^{i-1}]$ represents all previously encoded slices. 
The output of $f_\theta$ is used to parametrize a probability distribution. Specifically, we model each element of $y_f^i$ as a Gaussian distribution with mean $\mu_i$ and scale $\sigma_i$:
\begin{equation}
    p(y_f^i|y_f^{<i}, \hat{z}) \sim \mathcal{N}(\mu_i, \sigma_i^2),
\end{equation}
where $\Phi_i = (\mu_i, \sigma_i) = f_\theta(y_f^{<i}, \hat{z})$. Here, $\Phi_i$ represents the distribution parameters for the $i$-th slice.
This approach captures complex dependencies within the latent representation, leading to more efficient compression.
During the entropy coding process, we compute a residual $r_i$ for each slice: $r_i = y_f^i - \hat{y}_f^i$, where $\hat{y}_f^i$ is the quantized version of $y_f^i$. This residual helps to reduce quantization errors and improve reconstruction quality.
The actual encoding process involves quantizing $y_f^i - \mu_i$ and entropy encoding the result using the estimated distribution $\mathcal{N}(0, \sigma_i^2)$. During decoding, we reconstruct $\hat{y}_f^i$ as $\hat{y}_f^i = Q(y_f^i - \mu_i) + \mu_i$, where $Q(\cdot)$ denotes the quantization operation.

\paragraph{(4) Decoding and Optimization.} During decoding, we first reconstruct $\hat{z}$ and $\hat{y}_f$ from the bitstream. Then, using the dictionary indices passed from the encoder, we apply the same reference processing and alignment procedure to reconstruct $y$ from $\hat{y}_f$. Next, $y$ is fed into the synthesis transform $g_s$ to produce the final reconstructed image $\hat{x}$. It is important that we employ the same conditional latent synthesis pipeline on the decoder side to ensure consistency.
The combination of the hyperprior $z$ and the slice-based autoregressive model enables our system to achieve a fine balance between capturing global image statistics and local, contextual information, resulting in improved compression performance.
To optimize our network end-to-end, we minimize the rate-distortion function:
\begin{equation}
L = D(x, \hat{x}) + \lambda R(b),
\end{equation}
where $D(x, \hat{x})$ is the distortion between the original and reconstructed images, $R(b)$ is the bitrate of the encoded stream, and $\lambda$ is an adaptive coefficient used to balance the rate-distortion trade-off. This optimization balances compression efficiency and reconstruction quality, allowing our approach to effectively leverage the aligned reference information at both the encoder and decoder stages.


% \subsection{Network Architecture for Conditional Latent Coding}

% Our network architecture is built upon Transformer-CNN Mixture (TCM) blocks \cite{liu2023learned}, which efficiently combine the strengths of both CNNs for local feature extraction and transformers for capturing long-range dependencies. The TCM block is defined as:
% \begin{multline}
%     \mathbf{F}_o = \text{Conv1x1}(\text{Concat}[\text{SwinTransformer}(\mathbf{F}_i), \\
%     \text{ResidualCNN}(\mathbf{F}_i)])
% \end{multline}
% This design allows our model to process features at multiple scales, capturing both fine-grained textures and global semantic information. The TCM blocks are used throughout our encoder $g_a$, decoder $g_s$, and hyperprior network $h_a$, enabling effective feature processing at various stages of the compression pipeline.

% The overall architecture consists of an analysis transform $g_a$ that converts input images into latent representations, a synthesis transform $g_s$ that reconstructs images from these representations, and a hyperprior network $h_a$ that further compresses the latent representations to generate $\mathbf{z}$. 

% At the encoder side, after obtaining the latent representations $\mathbf{y}$ and $\mathbf{y}_r$ from $g_a$, we introduce the reference information through our patch matching, alignment, and fusion modules. The fused representation $\mathbf{y}_f$ is then processed by the slice-based autoregressive context model. The estimated probabilities from this model guide the entropy coding of $\mathbf{y}$ and $\mathbf{z}$, generating the final bitstream $b$.

% During decoding, we first reconstruct $\hat{\mathbf{z}}$ and $\hat{\mathbf{y}}$ from $b$. Crucially, we employ the same reference image processing pipeline at the decoder side. We query the reference dictionary to obtain $x_r$, process it through $g_a$ to get $\mathbf{y}_r$, and then apply the patch matching, alignment, and fusion modules to enhance $\hat{\mathbf{y}}$. This approach ensures that the decoder can leverage the same reference information used during encoding to improve reconstruction quality.

% The enhanced latent representation is then fed into the synthesis transform $g_s$ to produce the final reconstructed image $\hat{x}$. By integrating reference information at both the encoder and decoder, our method effectively utilizes rich prior knowledge to improve both compression efficiency and reconstruction quality.

% To optimize our network end-to-end, we minimize the rate-distortion function:
% \begin{equation}
%     L = D(X, \hat{x}) + \lambda(\mathbf{p}) R(b)
% \end{equation}
% where $D(X, \hat{x})$ is the distortion between the original and reconstructed images, $R(b)$ is the bitrate of the encoded stream, and $\lambda(\mathbf{p})$ is an adaptive Lagrange multiplier. This optimization balances compression efficiency and reconstruction quality, allowing our approach to effectively leverage the aligned reference information at both encoder and decoder stages.
% \begin{figure*}[t]
% \centering
% \begin{minipage}{\linewidth}
%     \centering
%     \includegraphics[width=\linewidth]{figure/visual.pdf}
%     %  \vspace{-0.2cm}
%     \caption{Image reconstruction results. From left to right: Raw inputs, reference images, reconstructed images. Red and blue boxes highlight specific areas of improvement.}
%     \label{fig:main_visaul}
% \end{minipage}


% \begin{minipage}{\linewidth}
%     \centering
%     \includegraphics[width=\linewidth]{figure/figure_main_results2.pdf}
%     \caption{The rate-distortion performance comparison of different methods.}
%     \label{fig:rate_distortion}
% \end{minipage}
% \end{figure*}

\subsection{Theoretical Perturbation Analysis} \label{sec:analysis}
In image compression with auxiliary information, some degree of error in feature retrieval is inevitable due to the inherent complexity of the problem and the presence of noise. Understanding the bounds of this error is crucial for assessing and improving compression algorithms. We present a theoretical framework that quantifies these errors and provides insights into the factors affecting compression performance.

We formulate the problem as a rate-distortion optimization:
\begin{equation}
\min_{G_1, G_2, D} \mathbb{E}\left[R(G_1(x), G_2(\tilde{x})) + \lambda \mathscr{D}\left(x, D(G_1(x), G_2(\tilde{x}))\right)\right]\nonumber
\end{equation}
where $x \in \mathbb{R}^d$ is the original image, $\tilde{x} \in \mathbb{R}^d$ the auxiliary image, $G_1$ and $G_2$ are encoders, $D$ is a decoder, $R$ is the rate loss, and $\mathscr{D}$ is the distortion loss.

Our analysis is based on several key assumptions. We model the original image using a spiked covariance model: $x = U^s + \xi$, and the auxiliary image similarly: $\tilde{x} = U^*\tilde{s} + \tilde{\xi}$. The rate loss is entropy-based: $R(z, \tilde{z}) = \mathbb{E}[-\log_2 p_\theta(z|\tilde{z})]$, while the distortion loss is mean squared error: $\mathscr{D}(x, \hat{x}) = \|x - \hat{x}\|^2$. We assume sub-Gaussian noise with parameter $\sigma^2$, and allow for possible irrelevant information in the auxiliary image, with proportion $p \in [0, 1)$.

Our theoretical analysis aims to quantify the error in feature retrieval when using auxiliary information for image compression, specifically establishing an upper bound on the error in estimating the feature subspace of the original image, with a focus on the impact of irrelevant information in the auxiliary image. This analysis provides a rigorous foundation for understanding our Conditional Latent Coding (CLC) method, quantifies trade-offs between factors affecting compression performance, and offers insights into the method's robustness to imperfect auxiliary data. By emphasizing the importance of minimizing irrelevant information, it guides the design and optimization of our dictionary construction process. By deriving this error bound, we bridge the gap between theoretical understanding and practical implementation, providing a solid basis for the development and refinement of our compression algorithm.

Our main result quantifies the unavoidable error in feature retrieval:

\begin{theorem}
For any $\delta > 0$, with probability at least $1-\delta$:
\begin{equation}
\resizebox{.9\hsize}{!}{$
\|\sin \Theta(\mathrm{Pr}(\hat{G}_1), U^*)\|_F \leq C\left(\sqrt{r} \wedge \sqrt{\frac{r}{1-p}} \sqrt{\frac{(r + r(\Sigma_\xi))\log(d/\delta)}{n}}\right)
$}\nonumber
\end{equation}

where $C > 0$ is a constant, $p$ is the proportion of irrelevant parts in the auxiliary image, $n$ is the number of training samples, $r(\Sigma_\xi)$ is the effective rank of the noise covariance matrix, and $\hat{G}_1$ is the estimated encoder for the original image.
\end{theorem}

This bound provides key insights: it reveals a trade-off between problem dimensionality ($r$), sample size ($n$), noise structure ($r(\Sigma_\xi)$), and auxiliary image quality ($p$). The system's tolerance to irrelevant information is quantified by $\frac{1}{1-p}$, while noise complexity is captured by the effective rank $r(\Sigma_\xi)$. The result also suggests potential for mitigation through increased sample size or improved auxiliary image quality.

% This analysis not only acknowledges the inherent limitations in feature retrieval but also provides a foundation for understanding and potentially improving compression performance in practical applications, especially in scenarios with imperfect or noisy auxiliary data.

% A complete problem description, detailed proofs, and robustness experiments are provided in the \textbf{Supplementary Materials}.

\section{Experimental Results} \label{sec:experiments}
\begin{figure*}[t]
\centering
\includegraphics[width=0.9\linewidth]{figure/main0214.pdf}
 \vspace{-0.2cm}
\caption{The rate-distortion performance comparison of different methods.}
 \vspace{-0.2cm}
\label{fig:rate_distortion}
\end{figure*}
In this section, we provide extensive experimental results to evaluate the proposed CLC method and ablation studies to understand its performance.

\subsection{Experimental Settings}

\subsubsection{(1) Datasets.}
In our experiments, we use two benchmark datasets: Flickr2W \cite{liu2020unified} and Flickr2K \cite{timofte2017ntire}. The Flickr2W dataset, containing 20,745 high-quality images, was used for training our model. To construct the image feature dictionary, we employed the Flickr2K dataset, which comprises 2,650 images. These Flickr2K images were randomly cropped into 256×256 patches to build the feature reference dictionary. We evaluated our algorithm on the Kodak \cite{kodak1993suite} and CLIC \cite{toderici2020clic} datasets to evaluate its performance.

%For performance evaluation, we adopted two widely used metrics in image coding: Peak Signal-to-Noise Ratio (PSNR) and Multi-Scale Structural Similarity Index (MS-SSIM). PSNR provides a quantitative measure of the reconstructed image quality, while MS-SSIM offers a perceptual quality assessment that correlates well with human visual perception. These metrics were chosen for their complementary nature in assessing both objective and perceptual image quality.

\subsubsection{(2) Implementation Details.}
Our model was implemented using PyTorch and trained on 8 NVIDIA RTX 3090 GPUs. We trained the network for 30 epochs using the Adam optimizer with an initial learning rate of $1\times10^{-4}$, which was reduced by a factor of 0.5 every 10 epochs. The batch size was set to 16 for each GPU.
For the patch matching module, we used a patch size of 16$\times$16 pixels. The initial value of the adaptive fusion weight $\alpha$ was set to 0.5. The number of slices K in the slice-based autoregressive context model was set to 8.
In the KV-cache, we set the dimension of keys $d_k$ and values $d_v$ to 256. The cache size $N$ was initially set to 300 and dynamically adjusted based on the GPU memory availability. The number of clusters $K$ in Mini-Batch K-means was set to 3,000.

%We compared our method with standard codecs including VVC (VTM 12.0), HEVC (HM 16.20), and JPEG (libjpeg 9d), using their default configurations. For learned compression methods, we used the official implementations with their default parameters.

\subsection{Performance Results}

% We report the rate-distortion results in Figure~\ref{fig:rate_distortion}, which shows that our proposed CLC method outperforms existing methods across different bit-rates. The compared methods include traditional codecs such as BPG~\cite{bellard2014bpg}, VTM (Versatile Video Coding Test Model)~\cite{bross2021overview}, HM (HEVC Test Model)~\cite{sullivan2012overview}, and JPEG~\cite{wallace1992jpeg}. We also compare with recent learning-based methods, including the hyperprior model~\cite{balle2018variational}, Cheng et al.'s approach~\cite{cheng2021learned}, ELIC~\cite{zou2022elic}, and TCM~\cite{chen2023transformer}. Additionally, we include results from the AV1 codec~\cite{chen2018overview} to provide a comprehensive comparison against both traditional and modern compression techniques. The improvement in compression efficiency is significant. For example, on the Kodak dataset, when the MS-SSIM is 0.95, the bit rate of our CLC method is about 0.1 bpp, while the bit rates for the TCM, VTM, BPG, and JPEG methods are 0.15, 0.18, 0.22, and 0.38 bpp, respectively. Compared with TCM, VTM, BPG, and JPEG, the CLC method has increased the compression ratio by 1.5 times, 1.8 times, 2.2 times, and 3.8 times, respectively. Figure \ref{fig:enter-label} shows samples of visual comparison results. 
% On the CLIC dataset, the performance improvement is even more significant. When the PSNR is 34 dB, the bit-rate of CLC is about 0.2 bpp, while the bit rates for the TCM, VTM, Hyperprior, and JPEG methods are 0.25, 0.28, 0.35, and 0.45 bpp, respectively. This indicates that our CLC methods achieve an even larger compression efficiency gain on this dataset compared to other methods. 
% We have visualized our experimental results in Figure \ref{fig:main_visaul}. Our analysis reveals that our method demonstrates significant performance improvements in preserving detailed texture structures, particularly at low bit rates. This enhancement is especially noticeable in horizontal and vertical textures, as exemplified by the railing sections depicted in the figure.
We report the rate-distortion results in Figure~\ref{fig:rate_distortion}, showing our proposed CLC method outperforms existing methods across different bit-rates. The compared methods include traditional codecs like BPG~\cite{bellard2014bpg}, VTM~\cite{bross2021overview}, HM~\cite{sullivan2012overview}, and JPEG~\cite{wallace1992jpeg}, as well as recent learning-based methods: the hyperprior model~\cite{balle2018variational}, Cheng et al.'s approach~\cite{cheng2021learned}, ELIC~\cite{zou2022elic}, and TCM~\cite{chen2023transformer}. We also include results from AV1~\cite{chen2018overview} for comparison. The improvement in compression efficiency is significant. On Kodak at MS-SSIM 0.95, CLC achieves 0.1 bpp, while TCM, VTM, BPG, and JPEG require 0.15, 0.18, 0.22, and 0.38 bpp, respectively, representing a 1.5 to 3.8 times increase in compression ratio. On CLIC at 34 dB PSNR, CLC achieves 0.2 bpp, compared to 0.25, 0.28, 0.35, and 0.45 bpp for TCM, VTM, Hyperprior, and JPEG, indicating larger efficiency gains. Figure~\ref{fig:main_visaul} demonstrates our method's superior performance in preserving detailed textures, particularly horizontal and vertical structures at low bit rates, as seen in railings and architectural features.

\subsection{Ablation Studies}

We conducted ablation studies to evaluate components of our CLC method, focusing on reference images, dictionary cluster size, and component contributions. We report results on both Kodak and CLIC datasets to demonstrate the performance across different image types.



\paragraph{(1) Ablation Studies on the Number of Reference Images.}
% We changed the number of reference images from 1 to 5 to examine its impact on the compression performance. Table \ref{tab:ref_images} shows the BD-rate savings compared to the VTM method with different numbers of reference images. Here, BD-Rate$_\text{P}$ represents the BD-rate savings in terms of PSNR, while BD-Rate$_\text{M}$ represents the BD-rate savings in terms of MS-SSIM.
% We can see that using three reference images achieves the best performance on both datasets. Compared to the VTM methods, it saves the BD-rate by 14.5\% and 13.9\% on the Kodak and CLIC datasets, respectively. When the number of reference images becomes larger than 3, the performance degrades. This is because too much redundancy has been introduced into the conditional coding process.

We changed the number of reference images from 1 to 5 to examine the impact on compression performance. Table \ref{tab:ref_images} shows BD-rate savings compared to the VTM method with different numbers of reference images. BD-Rate$_\text{P}$ represents savings in PSNR, while BD-Rate$_\text{M}$ represents savings in MS-SSIM. Using three reference images achieves the best performance on both datasets, saving 14.5\% and 13.9\% BD-rate on Kodak and CLIC, respectively. More than three images introduce redundancy, degrading performance.

\begin{table}[t]
%  \vspace{-0.2cm}

\centering
\small
\setlength{\tabcolsep}{4pt}
\renewcommand{\arraystretch}{0.85} % 调整行间距
\begin{tabular}{@{}lcccc@{}}
\toprule[1.5pt]
\multirow{2}{*}{\begin{tabular}[c]{@{}l@{}}Num of\\Ref. Images\end{tabular}} & \multicolumn{2}{c}{Kodak} & \multicolumn{2}{c}{CLIC} \\
\cmidrule(lr){2-3} \cmidrule(lr){4-5}
& BD-Rate$_\text{P}$ & BD-Rate$_\text{M}$ & BD-Rate$_\text{P}$ & BD-Rate$_\text{M}$ \\
\midrule
1 & -10.2 & -11.5 & -9.8 & -10.9 \\
2 & -12.8 & -13.7 & -12.1 & -13.2 \\
3 & \textbf{-14.5} & \textbf{-15.2} & \textbf{-13.9} & \textbf{-14.7} \\
4 & -14.3 & -15.0 & -13.7 & -14.5 \\
5 & -14.2 & -14.9 & -13.6 & -14.4 \\
\bottomrule[1.5pt]
\end{tabular}
\caption{BD-rate savings (\%) vs. VTM for different numbers of reference images.}
\label{tab:ref_images}
%  \vspace{-0.2cm}
\end{table}

\begin{figure*}[t]
    \centering
    \includegraphics[width=0.85\linewidth]{figure/visual.pdf}
     \vspace{-0.2cm}
    \caption{Image reconstruction results at around 0.1 bpp. From left to right: Raw inputs, reference images, reconstructed images. Red and blue boxes highlight specific areas of improvement.}
    \label{fig:main_visaul}
\end{figure*}
\paragraph{(2) Ablation Studies on the Dictionary Cluster Size.}
We conducted experiments with different dictionary cluster sizes to find the balance between compression efficiency and computational complexity. Table \ref{tab:dict_size} shows the BD-rate savings and encoding time for different cluster sizes. A cluster size of 3000 provides the best trade-off between performance and complexity for both datasets, achieving significant BD-rate savings with reasonable encoding times. The sharp increase in the encoding time for cluster sizes beyond 3000 highlights the importance of carefully selecting this parameter to balance compression efficiency and computational cost.

\begin{table}[t]

%  \vspace{-0.2cm}

\centering
\small
\setlength{\tabcolsep}{2pt} % 调整列间距
\renewcommand{\arraystretch}{0.85} % 调整行间距
\begin{tabular}{@{}lccccc@{}}
\toprule[1.5pt]
\multirow{2}{*}{\begin{tabular}[c]{@{}l@{}}Cluster\\Size\end{tabular}} & \multicolumn{2}{c}{Kodak} & \multicolumn{2}{c}{CLIC} & \multirow{2}{*}{\begin{tabular}[c]{@{}c@{}}Encoding\\Time (s)\end{tabular}} \\
\cmidrule(lr){2-3} \cmidrule(lr){4-5}
& BD-Rate$_\text{P}$ & BD-Rate$_\text{M}$ & BD-Rate$_\text{P}$ & BD-Rate$_\text{M}$ & \\
\midrule
1000 & -11.8 & -12.5 & -11.2 & -11.9 & 0.52 \\
2000 & -13.7 & -14.3 & -13.1 & -13.8 & 0.78 \\
3000 & \textbf{-14.5} & \textbf{-15.2} & \textbf{-13.9} & \textbf{-14.7} & 1.05 \\
4000 & -14.6 & -15.3 & -14.0 & -14.8 & 2.31 \\
5000 & -14.7 & -15.4 & -14.1 & -14.9 & 5.67 \\
\bottomrule[1.5pt]
\end{tabular}
%  \vspace{-0.2cm}
\caption{BD-rate savings (\%) vs. VTM and encoding time for different dictionary cluster sizes}
\label{tab:dict_size}
\end{table}

\paragraph{(3) Ablation Studies on Major Algorithm Components.}
We conducted ablation experiments to evaluate the contribution of major components. Table \ref{tab:components} shows each component's impact on the Kodak dataset performance. All components contribute significantly, with CLS having the most substantial impact (4.7\% BD-rate savings), highlighting the importance of adaptive feature modulation. The KV-cache, while minimally impacting compression performance, significantly reduces encoding time (from 1.87s to 1.05s). Multi-sample query in dictionary construction improves BD-rate savings by 0.7\% (BD-Rate$_\text{P}$) and 0.6\% (BD-Rate$_\text{M}$), enhancing overall compression capability through more diverse representations.

\begin{table}[t]

%  \vspace{-0.2cm}

\centering
\small
\setlength{\tabcolsep}{3pt}
\renewcommand{\arraystretch}{0.85} % 调整行间距
\begin{tabular}{@{}lccc@{}}
\toprule[1.5pt]
\multirow{2}{*}{\begin{tabular}[c]{@{}l@{}}Model\\Configuration\end{tabular}} & \multicolumn{2}{c}{BD-Rate Savings} & \multirow{2}{*}{\begin{tabular}[c]{@{}c@{}}Encoding\\Time (s)\end{tabular}} \\
\cmidrule(lr){2-3}
& BD-Rate$_\text{P}$ & BD-Rate$_\text{M}$ & \\
\midrule
Full Model & \textbf{-14.5} & \textbf{-15.2} & 1.05 \\
w/o CLM & -12.3 & -13.1 & 0.98 \\
w/o CLS & -9.8 & -10.5 & 0.92 \\
w/o KV-cache & -14.4 & -15.1 & 1.87 \\
w/o Multi-example Query & -13.8 & -14.6 & 0.97 \\
\bottomrule[1.5pt]
\end{tabular}
%  \vspace{-0.2cm}
\caption{BD-rate savings (\%) vs. VTM for different model configurations on Kodak dataset}
\label{tab:components}
\end{table}

 


\section{Conclusion} \label{sec:conclusion}
This study proposes Conditional Latent Coding (CLC), a novel deep learning-based image compression method that dynamically generates latent reference representations through a universal image feature dictionary. We develop innovative techniques for dictionary construction, efficient search/matching, alignment, and fusion, with theoretical analysis of robustness to dictionary and latent perturbations. While focused on compression, CLC's adaptive feature utilization principles may inspire broader vision tasks. Future work includes balancing compression efficiency and visual information utilization to address growing data transmission demands.

\clearpage
\section*{Acknowledgements}
This research was supported by the National Natural Science Foundation of China (No. 62331014) and Grant 2021JC02X103.
% \bibliographystyle{aaai25.bst}
% \bibliography{aaai25}
%File: anonymous-submission-latex-2025.tex
\documentclass[letterpaper]{article} % DO NOT CHANGE THIS
\usepackage{aaai25}  % DO NOT CHANGE THIS
\usepackage{times}  % DO NOT CHANGE THIS
\usepackage{helvet}  % DO NOT CHANGE THIS
\usepackage{courier}  % DO NOT CHANGE THIS
\usepackage[hyphens]{url}  % DO NOT CHANGE THIS
\usepackage{graphicx} % DO NOT CHANGE THIS
\urlstyle{rm} % DO NOT CHANGE THIS
\def\UrlFont{\rm}  % DO NOT CHANGE THIS
\usepackage{natbib}  % DO NOT CHANGE THIS AND DO NOT ADD ANY OPTIONS TO IT
\usepackage{caption} % DO NOT CHANGE THIS AND DO NOT ADD ANY OPTIONS TO IT
\usepackage{subcaption}
\frenchspacing  % DO NOT CHANGE THIS
\setlength{\pdfpagewidth}{8.5in} % DO NOT CHANGE THIS
\setlength{\pdfpageheight}{11in} % DO NOT CHANGE THIS
%
% These are recommended to typeset algorithms but not required. See the subsubsection on algorithms. Remove them if you don't have algorithms in your paper.
\usepackage{algorithm}
\usepackage{algorithmic}
\usepackage{comment}

%
% These are are recommended to typeset listings but not required. See the subsubsection on listing. Remove this block if you don't have listings in your paper.
\usepackage{newfloat}
\usepackage{listings}
\DeclareCaptionStyle{ruled}{labelfont=normalfont,labelsep=colon,strut=off} % DO NOT CHANGE THIS
\lstset{%
	basicstyle={\footnotesize\ttfamily},% footnotesize acceptable for monospace
	numbers=left,numberstyle=\footnotesize,xleftmargin=2em,% show line numbers, remove this entire line if you don't want the numbers.
	aboveskip=0pt,belowskip=0pt,%
	showstringspaces=false,tabsize=2,breaklines=true}
\floatstyle{ruled}
\newfloat{listing}{tb}{lst}{}
\floatname{listing}{Listing}
%
% Keep the \pdfinfo as shown here. There's no need
% for you to add the /Title and /Author tags.
\pdfinfo{
/TemplateVersion (2025.1)
}

% DISALLOWED PACKAGES
% \usepackage{authblk} -- This package is specifically forbidden
% \usepackage{balance} -- This package is specifically forbidden
% \usepackage{color (if used in text)
% \usepackage{CJK} -- This package is specifically forbidden
% \usepackage{float} -- This package is specifically forbidden
% \usepackage{flushend} -- This package is specifically forbidden
% \usepackage{fontenc} -- This package is specifically forbidden
% \usepackage{fullpage} -- This package is specifically forbidden
% \usepackage{geometry} -- This package is specifically forbidden
% \usepackage{grffile} -- This package is specifically forbidden
% \usepackage{hyperref} -- This package is specifically forbidden
% \usepackage{navigator} -- This package is specifically forbidden
% (or any other package that embeds links such as navigator or hyperref)
% \indentfirst} -- This package is specifically forbidden
% \layout} -- This package is specifically forbidden
% \multicol} -- This package is specifically forbidden
% \nameref} -- This package is specifically forbidden
% \usepackage{savetrees} -- This package is specifically forbidden
% \usepackage{setspace} -- This package is specifically forbidden
% \usepackage{stfloats} -- This package is specifically forbidden
% \usepackage{tabu} -- This package is specifically forbidden
% \usepackage{titlesec} -- This package is specifically forbidden
% \usepackage{tocbibind} -- This package is specifically forbidden
% \usepackage{ulem} -- This package is specifically forbidden
% \usepackage{wrapfig} -- This package is specifically forbidden
% DISALLOWED COMMANDS
% \nocopyright -- Your paper will not be published if you use this command
% \addtolength -- This command may not be used
% \balance -- This command may not be used
% \baselinestretch -- Your paper will not be published if you use this command
% \clearpage -- No page breaks of any kind may be used for the final version of your paper
% \columnsep -- This command may not be used
% \newpage -- No page breaks of any kind may be used for the final version of your paper
% \pagebreak -- No page breaks of any kind may be used for the final version of your paperr
% \pagestyle -- This command may not be used
% \tiny -- This is not an acceptable font size.
% \vspace{- -- No negative value may be used in proximity of a caption, figure, table, section, subsection, subsubsection, or reference
% \vskip{- -- No negative value may be used to alter spacing above or below a caption, figure, table, section, subsection, subsubsection, or reference

\setcounter{secnumdepth}{0} %May be changed to 1 or 2 if section numbers are desired.

% The file aaai25.sty is the style file for AAAI Press
% proceedings, working notes, and technical reports.
%

% Title

% Your title must be in mixed case, not sentence case.
% That means all verbs (including short verbs like be, is, using,and go),
% nouns, adverbs, adjectives should be capitalized, including both words in hyphenated terms, while
% articles, conjunctions, and prepositions are lower case unless they
% directly follow a colon or long dash
\title{Norm Growth and Stability Challenges in Localized \\Sequential Knowledge Editing}
%\title{Can we make Localized Updates to Large Language Models? \\A Case Study in Knowledge Editing}
%\title{The Curious Case of Increasing Norm during Post-Training Interventions}
\author{
    %Authors
    % All authors must be in the same font size and format.
    Akshat Gupta\textsuperscript{\rm 1}\footnote{Correspondence to: akshat.gupta@berkeley.edu}, Christine Fang\textsuperscript{\rm 1}, Atahan Ozdemir\textsuperscript{\rm 1}, Maochuan Lu\textsuperscript{\rm 1}, \\Ahmed Alaa\textsuperscript{\rm 1}, Thomas Hartvigsen\textsuperscript{\rm 2}, Gopala Anumanchipalli\textsuperscript{\rm 1}
}
\affiliations{
    %Afiliations
    \textsuperscript{\rm 1}University of California Berkeley, \textsuperscript{\rm 2}University of Virginia\\
%
% See more examples next
}

%Example, Single Author, ->> remove \iffalse,\fi and place them surrounding AAAI title to use it
\iffalse
\title{My Publication Title --- Single Author}
\author {
    Author Name
}
\affiliations{
    Affiliation\\
    Affiliation Line 2\\
    name@example.com
}
\fi

\iffalse
%Example, Multiple Authors, ->> remove \iffalse,\fi and place them surrounding AAAI title to use it
\title{My Publication Title --- Multiple Authors}
\author {
    % Authors
    First Author Name\textsuperscript{\rm 1},
    Second Author Name\textsuperscript{\rm 2},
    Third Author Name\textsuperscript{\rm 1}
}
\affiliations {
    % Affiliations
    \textsuperscript{\rm 1}Affiliation 1\\
    \textsuperscript{\rm 2}Affiliation 2\\
    firstAuthor@affiliation1.com, secondAuthor@affilation2.com, thirdAuthor@affiliation1.com
}
\fi


% REMOVE THIS: bibentry
% This is only needed to show inline citations in the guidelines document. You should not need it and can safely delete it.
\usepackage{bibentry}
% END REMOVE bibentry

\begin{document}

\maketitle

\begin{abstract}
This study investigates the impact of localized updates to large language models (LLMs), specifically in the context of knowledge editing - a task aimed at incorporating or modifying specific facts without altering broader model capabilities. We first show that across different post-training interventions like continuous pre-training, full fine-tuning and LORA-based fine-tuning, the Frobenius norm of the updated matrices always increases. This increasing norm is especially detrimental for localized knowledge editing, where only a subset of matrices are updated in a model . We reveal a consistent phenomenon across various editing techniques, including fine-tuning, hypernetwork-based approaches, and locate-and-edit methods: the norm of the updated matrix invariably increases with successive updates. Such growth disrupts model balance, particularly when isolated matrices are updated while the rest of the model remains static, leading to potential instability and degradation of downstream performance. Upon deeper investigations of the intermediate activation vectors, we find that the norm of internal activations decreases and is accompanied by shifts in the subspaces occupied by these activations, which shows that these activation vectors now occupy completely different regions in the representation space compared to the unedited model. With our paper, we highlight the technical challenges with continuous and localized sequential knowledge editing and their implications for maintaining model stability and utility.

%Our findings underscore the challenges of continuous, localized updates and their implications for maintaining model stability and utility, contributing to the broader understanding of post-training interventions in LLMs.
%All types of post-training interventions to a model lead to increase in the norm of the matrix being updated. This has severe implications when performing localized updates - if the norm of one part of the model grows disproportionately while the remaining model remains frozen, the balance and stability of the entire system may be compromised. This is exactly what is observed in sequential knowledge editing studies. We show that this phoenomon is also universal across all localized knowledge editing methods - including fine-tuning, hypernetwork based, and locate and edit knowledge editing methods. We also study the internal activations of edited models as they are sequentially edited and find that the norm of the activation vectors contiously decreases, and they start occupying significanlty different regions in the representation space. We suggest methods to mitigate such issues by constraining weights and find that this doesnt help much. The conclusion of the study is that post-pretraining updates to a model always leads to increase in norm, and this is detrimental for dreams of performing contious localized to a model. 
\end{abstract}

% Uncomment the following to link to your code, datasets, an extended version or similar.
%
% \begin{links}
%     \link{Code}{https://aaai.org/example/code}
%     \link{Datasets}{https://aaai.org/example/datasets}
%     \link{Extended version}{https://aaai.org/example/extended-version}
% \end{links}

\section{Introduction}
Recent advances in model interpretability have led to methods that perform localized updates to large language models by intervening at very specific locations \cite{ROME, MEMIT, hernandez2023inspecting}. A popular domain in which such methods are employed is knowledge editing \cite{editing-survey} - a task where singular facts are added or updated inside of a model in a data and compute efficient manner. This paper starts with a simple yet powerful observation that sequential knowledge updates made to a model always leads to an increase in the norm of the matrix being updated. We first ask the question - is the increase in norm with continous updates specific to localized knowledge editing methods or is this a general phenomenon? 

Our experiments with numerous post-training interventions, including continual pretraining, full fine-tuning and LORA-based fine-tuning, present a very surprising result - \textbf{the norm of the weight matrices being updated always increases for all these post-training interventions}. While there has been dispersed work showing norm growth \cite{norm-growth}, to the best of our knowledge, our study is the first to emprically evaluate this comprehensively for large number of important post-training interventions.

We next study the presence and implications of this phenomenon when performing localized updates. To do so we study the task of knowledge editing where localized updates are very common. Knowledge editing methods usually update specific parts of the model, for example the MLP sub-modules of certain layers, to add or update new information. This allows data and compute efficient updates to be made to a model. We perform continuous sequential knowledge edits to a model using various parameter modifying knowledge editing methods along with localized fine-tuning, and find that for all these scenarios, the norm of the edited matrix always increases with the number of updates. While the increasing norm may not be concering in general, it is especially detrimental for performing localized updates. This is because disproportionate and contionuous growth in the norm of one or few layers of a model, while the rest of the model remains frozen, will compromise the balance and stability of the entire system, eventually leading to a breaking point as observed in prior work \cite{akshat-catastrophic, akshat-rebuilding}. This disprortionate growth is shown in Figure \ref{fig:norm-growth}. 

We further analyze the effects of performing continous localized knowledge updates to a model by studying how the hidden activations of the model changes. We find that contrary to the increasing norm of edited matrix, the norm of the activation vectors generated after the edited layers continuously decreases. We also show that these activation begin to occupy different regions in space when compared to the original. A follow-up to our work by \citet{encore} resolve the problems of disproportionate norm growth by using regularization methods and propose a more robust knowledge editing method called ENCORE. 

%To counter this continuous growth of matrix norm, we modify popular knowledge editing methods like MEMIT and localized fine-tuning with an additional term in the loss function that penalizes growth in the norm of the edited matrix. We find that such a regularization still leads to increase in norm, although at a much slower rate. This leads to more stable localized updates for a much longer time, but eventually leads to the same issues. 



To summarize, we make the following contributions in this paper:
\begin{enumerate}
    \item We show that the frobenius norm of the updated weight matrices always increases during post training interventions.
    \item The norm of edited matrix increases disproportionately for localized knowledge updates, leading to model collapse. 
    \item This collapse is accompanied by a change in the norm and orientations of the resultant hidden activations, showing that the activations of the edited models now lie in a different region of representation space.
\end{enumerate}


% 
\section{Norm Growth During Post-Training Interventions}
We focus on the following common interventions that are applied on a model after the pretraining step - continual pretraining, full fine-tuning, LORA based fine-tuning \cite{hu2021lora}. We discuss our experiment settings for each of the following below:

\begin{enumerate}
    \item \textbf{Continued pretraining (CPT)} - We consider continued pretraining as as separate case from full fine-tuning although in both cases all the weights of the model are updated using a next-token prediction loss. We define continued pretraining as a process where the foundational knowledge of a model is extended by training on a large domain-specific corpora. In our experiments, we present the results for performing CPT on 20 billion tokens for Python programming \cite{li2023starcoder} on Llama-2 (7B) \cite{llama2}.
    
    \item \textbf{Full Fine-Tuning (FFT)} - We define full fine-tuning as task specific next-token prediction training of a model to optimize the model's parameters on a particular task. We present the results for fine-tuning Llama-2 (7B) model on 110k question answer pairs for programming \cite{wei2024magicoder}. 

    \item \textbf{LORA based full fine-tuning (LFFT)} - Here we use LORA \cite{hu2021lora} to fine-tune all the model weights in the same setting as FFT. 

    %\item \textbf{Reinforcement learning with human feedback (RLHF)} trains language models to align with human values and preferences by optimizing their responses based on human-provided rewards or feedback. We use both PPO \cite{rlhf-ouyang2022training} and DPO \cite{dpo-rafailov2024direct} algorithms for RLHF. 

    %\item \textbf{Unlearning} in LLMs is the task of removing undesired data and associated capabilities form a model \cite{unlearningsurvey-hase}. Various methods are employed for unlearning. 
\end{enumerate}


\begin{figure*}[h]
    \centering
    % First row
    \begin{subfigure}{0.22\textwidth}
        \includegraphics[width=\linewidth]{figures/figure1,4/starcoder_full_up_projs.png}
        \caption{CPT MLP-Up}
    \end{subfigure}
    \begin{subfigure}{0.22\textwidth}
        \includegraphics[width=\linewidth]{figures/figure1,4/starcoder_full_down_projs.png}
        \caption{CPT MLP-Down}
    \end{subfigure}
    \begin{subfigure}{0.22\textwidth}
        \includegraphics[width=\linewidth]{figures/figure1,4/starcoder_full_W_k.png}
        \caption{CPT Attention-Key}
    \end{subfigure}
        \begin{subfigure}{0.22\textwidth}
        \includegraphics[width=\linewidth]{figures/figure1,4/starcoder_full_W_v.png}
        \caption{CPT Attention-Value}
    \end{subfigure}

    \vspace{0.5cm} % Space between rows

    % Second row
    \begin{subfigure}{0.22\textwidth}
        \includegraphics[width=\linewidth]{figures/figure1,4/magicoder_full_up_projs.png}
        \caption{FFT MLP-Up}
    \end{subfigure}
    \begin{subfigure}{0.22\textwidth}
        \includegraphics[width=\linewidth]{figures/figure1,4/magicoder_full_down_projs.png}
        \caption{FFT MLP-Down}
    \end{subfigure}
    \begin{subfigure}{0.22\textwidth}
        \includegraphics[width=\linewidth]{figures/figure1,4/magicoder_full_W_k.png}
        \caption{FFT Attention-Key}
    \end{subfigure}
    \begin{subfigure}{0.22\textwidth}
        \includegraphics[width=\linewidth]{figures/figure1,4/magicoder_full_W_v.png}
        \caption{FFT Attention-Value}
    \end{subfigure}

    \vspace{0.5cm} % Space between rows

    % Third row
    \begin{subfigure}{0.22\textwidth}
        \includegraphics[width=\linewidth]{figures/figure1,4/magicoder_lora_up_projs.png}
        \caption{LFFT MLP-Up}
    \end{subfigure}
    \begin{subfigure}{0.22\textwidth}
        \includegraphics[width=\linewidth]{figures/figure1,4/magicoder_lora_down_projs.png}
        \caption{LFFT MLP-Down}
    \end{subfigure}
    \begin{subfigure}{0.22\textwidth}
        \includegraphics[width=\linewidth]{figures/figure1,4/magicoder_lora_W_k.png}
        \caption{LFFT Attention-Key}
    \end{subfigure}
    \begin{subfigure}{0.22\textwidth}
        \includegraphics[width=\linewidth]{figures/figure1,4/magicoder_lora_W_v.png}
        \caption{LFFT Attention-Value}
    \end{subfigure}

    \caption{Norm growth during post-training interventions }
    \label{fig:post-training}
\end{figure*}










For each of the above, we present post-training intervention results using the checkpoints provided in the study by \citet{lora-learns-less}. In each of the above cases, the model update equation can be written as:

\begin{equation}
    W_{new} = W_{old} + \Delta W
\end{equation}



Thus, the norm of the new weight matrix does not neccessarily have to decrease and can lie in the range as shown below using the triangle inequality:

\begin{equation}
   | |W_{old}|_F - |\Delta W|_F | \leq  |W_{new}|_F \leq |W_{old}|_F + |\Delta W|_F
\end{equation}

which means that after each update, the norm of the matrix being updated can both decrease or increase. Yet we find that the norm of the updated matrix in each of these interventions always increases. This can be seen in Figure \ref{fig:post-training}. The frobenius norm of all three MLP\footnote{Note that the Llama architecture has three MLP matrices instead of the common practice of two.} and attention matrices in Llama-2 (7B) can be seen to increase during CPT, FFT and LFFT. While this increase in norm is not detrimental to model performance, we believe this is because the norm of all the weight matrices involved increases in conjunction with each other. 

While our study is not exhaustive across different types of models or datasets used, we use the above findings to set the stage for the coming sections and to motivate studying norms of edited matrices when performing localized updates to a model. As only a few weight matrices are updated during parameter modifying knowledge editing methods, this phenomenon can have adverse consequences when the norm of some intermediate matrices grow while the remaining model remains frozen.

%TO DO: study other spectral properties of the weight matrices.
% A proper study where we do all of these on our own.

%Potential explanations:
%While our study is not exhaustive across different types of models, data types and other factors, we use the above findings as motivation to look specifically at performing localized updates to a model. 
%Learning rate issue, issue with optimizer and momentum etc. 

\begin{figure*}[h]
    \centering
    % First row
    \begin{subfigure}{0.22\textwidth}
        \includegraphics[width=\linewidth]{figures/figure2/ROME_gpt2xl_norm.png}
        \caption{ROME-norm}
    \end{subfigure}
    \begin{subfigure}{0.22\textwidth}
        \includegraphics[width=\linewidth]{figures/figure2/MEMIT_gpt2xl_norm.png}
        \caption{MEMIT-norm}
    \end{subfigure}
    \begin{subfigure}{0.22\textwidth}
        \includegraphics[width=\linewidth]{figures/figure2/MEND_gpt2xl_norms.png}
        \caption{MEND-norm}
    \end{subfigure}
        \begin{subfigure}{0.22\textwidth}
        \includegraphics[width=\linewidth]{figures/figure2/PMET_gpt2xl_norms.png}
        \caption{PMET-norm}
    \end{subfigure}

    \vspace{0.5cm} % Space between rows

    % Second row
    \begin{subfigure}{0.22\textwidth}
        \includegraphics[width=\linewidth]{figures/figure2/ROME_gpt2xl_downstream.png}
        \caption{ROME-downstream}
    \end{subfigure}
    \begin{subfigure}{0.22\textwidth}
        \includegraphics[width=\linewidth]{figures/figure2/MEMIT_gpt2xl_downstream.png}
        \caption{MEMIT-downstream}
    \end{subfigure}
    \begin{subfigure}{0.22\textwidth}
        \includegraphics[width=\linewidth]{figures/figure2/MEND_gpt2xl_downstream.png}
        \caption{MEND-downstream}
    \end{subfigure}
        \begin{subfigure}{0.22\textwidth}
        \includegraphics[width=\linewidth]{figures/figure2/PMET_gpt2xl_downstream.png}
        \caption{PMET-downstream}
    \end{subfigure}

    %\vspace{0.5cm} % Space between rows

    \caption{Norm growth and downstream performance during knowledge editing on GPT2-XL for different methods.}
    \label{fig:editing}
\end{figure*}



\begin{figure*}[h]
    \centering
    % First row
    \begin{subfigure}{0.22\textwidth}
        \includegraphics[width=\linewidth]{figures/figure3/fig3_gpt2xl_rome.png}
        \caption{ROME}
    \end{subfigure}
    \begin{subfigure}{0.22\textwidth}
        \includegraphics[width=\linewidth]{figures/figure3/fig3_gpt2xl_memit.png}
        \caption{MEMIT}
    \end{subfigure}
    \begin{subfigure}{0.22\textwidth}
        \includegraphics[width=\linewidth]{figures/figure3/fig3_gpt2xl_mend.png}
        \caption{MEND}
    \end{subfigure}
        \begin{subfigure}{0.22\textwidth}
        \includegraphics[width=\linewidth]{figures/figure3/fig3_gpt2xl_pmet.png}
        \caption{PMET}
    \end{subfigure}

    % \vspace{0.5cm} % Space between rows

    \caption{Norm growth for edits 100, 500, 2000 for GPT2-XL. }
    \label{fig:norm-growth}
\end{figure*}


\section{Localized Updates during Knowledge Editing}
Knowledge editing is defined as the task of making data and compute efficient knowledge updates to large language model without compromising their general ability \cite{editing-survey, composable-inteventions}. In this paper, we focus on parameter-modifying knowledge editing methods, where knowledge is updated by changing the weights of the model \cite{ROME, MEMIT, akshat-unified}. The compute efficient component of knowledge editing comes from the fact that usually only one or a few layers of a model are updated when incorporating new knowledge. In this paper we focus on four popular knowledge editing methods - \textbf{ROME} \cite{ROME}, \textbf{MEMIT} \cite{MEMIT}, \textbf{MEND} \cite{MEND} and \textbf{PMET} \cite{PMET}. We perform sequential edits using these methods on GPT2-XL (1.5B) \cite{gpt-2} and GPT-J (6B) \cite{gpt-j} for 2000 edits and analyze the different properties of the updated matrices. The list of layers updated for the different model editing algorithms can be seen in Table \ref{tab:layers-edited}.

\begin{table}[h]
    \centering
    \begin{tabular}{c|c|c}
        \textbf{Algorithm} & \textbf{GPT2-XL} & \textbf{GPT-J} \\ \hline
        ROME & 17 & 5 \\ \hline
        MEMIT & 13-17 & 3-8 \\ \hline
        MEND & 45-47 & 25-27 \\ \hline
        PMET & 13-17 & 3-8 \\ 
    \end{tabular}
    \caption{List of layers edited when using each of the above algorithms for GPT2-XL and GPT-J.}
    \label{tab:layers-edited}
\end{table}


As can be seen in Figure \ref{fig:editing}, the norm of the edited matrices always increases for all four types of model editing methods used. Figure \ref{fig:editing} also shows evidences of model degradation as a function of sequential model editing evaluated on 8 downstream tasks including MMLU \cite{MMLU} and a few tasks from the GLUE benchmark \cite{glue}. A more detailed account of downstream measurement is provided in appendix \ref{sec:app:downstream}. To put the norm growth in perspective of the rest of the model, we plot the norm of the edited matrix after 100, 500 and 2000 edits along with the norm of other matrices present inside the model in Figure \ref{fig:norm-growth}. We clearly see the anomolous growth in norm for the edited layers, while the norm of the remaining layers remains constant. 

\paragraph{Effect on Norm of Internal Activations.} To understand further effects of the increase norm of the edited matrix, we look at its effect on the corresponding activations that are output from that layer. To do so, we send 1 million tokens of wikipedia articles through both GPT2-XL and GPT-J, and study the norm and orientations of the internal activations of the model before and after editing. Specifically, we are looking at the residual stream vectors after each layers inside an LLM. The norm of the activation before and after editing can be seen in Figure \ref{fig:activation-norm-growth}. In this figure, we present the norm of the activation vectors for GPT2-XL when editing the model using ROME, comparing the unedited model with the model after 100, 500 and 2000 edits. We see that the activation norms remains the same until layer 17 for all cases, but slowly starts to decrease as we edit the model more. After 2000 edits, the norm of the activation vectors at layer 40 is much almost have that of the original enedited model, showing that the decrease in norm compounds as the activation vectors pass through the model. In contrast to the increasing norm of the edited weight matrices, the average norm of the activations decreases post editing. While in this paper we do not analyze the implications of this, it is studied in more detail in \citet{encore}. They show that the norms of activation vectors generated from the edited matrices increase while the norms of activations from the subsequent layers decreases, resulting in an increased importance of vectors generated from those layers. 


\begin{figure*}[h]
    \centering
    % First row
    \begin{subfigure}{0.22\textwidth}
        \includegraphics[width=\linewidth]{figures/activation_plots/gpt2-xl_unedited_norm_plot.png}
        \caption{Unedited Model}
    \end{subfigure}
    \begin{subfigure}{0.22\textwidth}
        \includegraphics[width=\linewidth]{figures/activation_plots/gpt2-xl_ROME_edits_100_norm_plot.png}
        \caption{After 100 edits}
    \end{subfigure}
    \begin{subfigure}{0.22\textwidth}
        \includegraphics[width=\linewidth]{figures/activation_plots/gpt2-xl_ROME_edits_500_norm_plot.png}
        \caption{After 500 edits}
    \end{subfigure}
    \begin{subfigure}{0.22\textwidth}
        \includegraphics[width=\linewidth]{figures/activation_plots/gpt2-xl_ROME_edits_2000_norm_plot.png}
        \caption{Adter 2000 edits}
    \end{subfigure}
       

    \vspace{0.5cm} % Space between rows

    \caption{Activation Norms at different layers for unedited, edits 100, 500, 2000 for GPT2-XL/ROME.}
    \label{fig:activation-norm-growth}
\end{figure*}




\begin{figure*}[h]
    \centering
    \begin{subfigure}{0.24\textwidth}
        \includegraphics[width=\linewidth]{figures/boundary_plots/boundary_gpt2xl_unedited.png}
        \caption{GPT2-XL unedited}
    \end{subfigure}
    \begin{subfigure}{0.24\textwidth}
        \includegraphics[width=\linewidth]{figures/boundary_plots/boundary_gpt2-xl_ROME_100.png}
        \caption{100 edits}
    \end{subfigure}
    \begin{subfigure}{0.24\textwidth}
        \includegraphics[width=\linewidth]{figures/boundary_plots/boundary_gpt2-xl_ROME_500.png}
        \caption{500 edits}
    \end{subfigure}
    \begin{subfigure}{0.24\textwidth}
        \includegraphics[width=\linewidth]{figures/boundary_plots/boundary_gpt2-xl_ROME_2000.png}
        \caption{2000 edits}
    \end{subfigure}
       
    \vspace{0.5cm} % Space between rows

    \caption{Classificatioin plots comparing GPT2-XL unedited model with model after 100, 500, 2000 edits using ROME.}
    \label{fig:classification}
\end{figure*}





\paragraph{Effect on Orientation of Internal Activations.}  We further analyze the orientations of the activation vectors. To do so, we create classifiers between the residual stream vectors of layers “$i$” versus “$j$”. Each classifier is a simple binary logistic regression classifier. A classifier $C_{i,j}$ is a binary classifier between two groups of vectors - residual stream vectors between layer $i$ and residual stream vectors between layer $j$. The classifier is trained with 800k training vectors and tested on 200k test vectors, equally divided into the two classes. This classifiers are trained only on the activations of the un-edited model. The accuracy of the classifier can be seen in Figure \ref{fig:classification} (a). We see that the simple binary classifiers reach almost a 100\% accuracy when classifying the activation vectors from different layers. \textbf{This shows that the activation vectors of different layers are linearly separable from each other}. When a classifier is trained with activation vectors from the same layers, which forms the diagonal line in Figure \ref{fig:classification} (a), we see that the classification accuracy is 50\% or at random. This shows that the activation vectors between the same layers are not linearly separable, stengthening the linear seprability claim. 

We save the classifiers trained on the un-edited model and use them to clasify the activation vectors of the edited model. Figure \ref{fig:classification} shows this for ROME, where we analyze the output for 200k activation vectors when passed through a model sequentially edited for 100, 500 and 2000 edits. The cell $i-j$ corresponds to the case where classifier $C_{i,j}$, trained on the activations coming from the $i^{th}$ and $j^{th}$ layer of the un-edited model, are used to classifiy the activations of the $j^{th}$ layer of the edited model. We want to re-iterate to the reader that the graphs in Figure \ref{fig:classification} are not supposed to be symmetrical by definition. We will give an example below. For cell 3-5, we use the classifier $C_{3,5}$ trained between activations of layer 3 and layer 5 of the un-edited model, but use it to classify the activations of layer 5 at test time. For cell 5-3, we use the same classifier $C_{3,5}$, but this time we use it to classify the activations of layer 3. 

For cell $i-j$, if the orientations in space of the activation vectors of the edited models remains the same as that of the unedited model, the classifier $C_{i,j}$ would assign it the class $j$. But we see in Figure \ref{fig:classification} that this is not the case. The test time accuracy of the classifier begins to get worse for activation vectors right after the edited layer. This shows that the activation vectors are not only changing in norms but also the regions they occupy in space. This is so much so that the activations coming out of layer $j$ of the edited model after 2000 edits now lie in a region for layer $i$, leading to an incorrect classification accuracy for classifier $C_{i,j}$.  

With this study, we show that the increasing norm not only changes the norm of the internal activations of the model, but also the orientations in space in which these vectors lie. We observe that post-editing, the internal activations now lie in a very different region in space. The region is so different that a linear classifier trained to identify the activation vectors of a specific layer is now unable to identify the output activations from the specific layer. 


\section{Conclusion}
This study highlights a critical challenge in the domain of localized updates for large language models: the persistent increase in the norm of updated matrices during sequential knowledge editing. While this phenomenon appears universal across post-training intervention methods, its implications are particularly pronounced in localized knowledge editing, where only specific parts of the model are modified. The resulting imbalance leads to downstream performance degradation, as evidenced by changes in both the norm and orientation of internal activations. Our findings emphasize the need for innovative strategies to address these challenges, paving the way for more robust and sustainable approaches to localized knowledge editing in LLMs. This work serves as a foundational step towards understanding and mitigating the inherent limitations of current techniques, with the ultimate goal of enabling dynamic and scalable updates to pre-trained models. A follow-up work by \citet{encore} overcomes these highlighted limitations by using appropriate regularizations with knowledge editing, allowing long-term sequential knowledge editing. 



%We argue that the increase in the matrix norm is detrimental to model performance. This is because of the sharp increase in the norm of the edited layers as the model is edited continuously, as can be seen in Figure \ref{fig:norm-growth}. For ROME, we can see that the norm of the edited matrix after 100, 500 and 2000 edits grows to approximately four times the original norm, while the norm of the previous and the following layers remain the same as we are performing localized updates. 







\bibliography{aaai25}

\appendix

\subsection{A.1 Downstream Performance Measurement}\label{sec:app:downstream}

In this paper, we assess model degradation by measuring downstream performance at regular intervals of edits. Our evaluation suite is wide-ranging and consists of the following 8 tasks – sentiment analysis (SST2) \cite{sst2}, paraphrase detection (MRPC) \cite{mrpc}, natural language inference (NLI, RTE) \cite{nli1, nli2, nli3, nli4}, commonsense natural language inference (HellaSwag) \cite{hellaswag}, linguistic acceptability classification (CoLA) \cite{cola}, multi-turn dialogue reasoning (MuTual) \cite{mutual} and massive multitask language understanding (MMLU) \cite{MMLU}.

For each task, we created a subset of 100 examples balanced across all multiple-choice options. The models were evaluated on the tasks above, and the accuracy score was measured at intervals of 5 edits for MEND, and 20 edits for PMET, ROME, and MEMIT. MEND leads to a rapid degradation within 100 edits, hence why a smaller granularity of 5 edit interval was used; a 20 edit interval was used for other methods to cut down on computation time. 

In order to improve models' initial performance and achieve meaningful signals, we provided few-shot examples. The few-shot prompt templates used for each task are shown in Figures \ref{fig:sst-prompt}-\ref{fig:hellaswag-prompt}.

\begin{figure*}
    \centering
    \fbox{
        \parbox{0.8\textwidth}{
            Review : an exhilarating futuristic thriller-noir , minority report twists the best of technology around a gripping story , delivering a riveting , pulse intensifying escapist adventure of the first order \newline
            Sentiment : positive \newline \newline
            Review : try as i may , i ca n't think of a single good reason to see this movie , even though everyone in my group extemporaneously shouted , ` thank you ! ' \newline
            Sentiment : negative \newline \newline
            Review : the film 's performances are thrilling .  \newline
            Sentiment : positive \newline \newline
            Review : vera 's technical prowess ends up selling his film short ; he smoothes over hard truths even as he uncovers them . \newline
            Sentiment : negative \newline \newline
            Review : [input] \newline
            Sentiment :
        }
    }
    \caption{Few shot prompt template used for SST-2 }
    \label{fig:sst-prompt}
\end{figure*}

\begin{figure*}
    \centering
    \fbox{
        \parbox{0.8\textwidth}{
            Question: Which expression is equivalent to 4 x 9? \newline
            (A) (4x 4) + (4x5) \newline
            (B) (4+4) x (4+5) \newline
            (C) (4+4)+(4+5) \newline
            (D) (4x 4) x (4x5) \newline
            Answer: A\newline\newline
            Question: A marketing researcher is conducting a survey in a large selling area by contacting a small group of people that is representative of all people in that area. The small, representative group is known as the \newline
            (A) population\newline
            (B) sample\newline
            (C) stratification\newline
            (D) universe\newline
            Answer: B\newline\newline
            Question: A research participant eats half a bowl of M\&M candies, and then stops eating. How would a motivation researcher using drive reduction theory explain this participant's behavior?\newline
            (A) Humans are instinctively driven to eat sugar and fat when presented to them.\newline
            (B) The Yerkes-Dodson law explains that people will eat food when presented to them, but usually in moderate amounts in order to avoid being perceived as selfish.\newline
            (C) The primary drive of hunger motivated the person to eat, and then stop when she/he regained homeostasis.\newline
            (D) The research participant was satisfying the second step on the hierarchy of needs: Food needs.\newline
            Answer: C\newline\newline
            Question: In a deductively valid argument\newline
            (A) If all the premises are true, the conclusion must be true\newline
            (B) The conclusion has already been stated in its premises\newline
            (C) If all the premises are true, the conclusion may still be false\newline
            (D) Both A and B\newline
            Answer: D\newline\newline
            Question: [input] \newline
            Answer:
        }
    }
    \caption{Few shot prompt template used for MMLU}
    \label{fig:mmlu-prompt}
\end{figure*}


\begin{figure*}
    \centering
    \fbox{
        \parbox{0.8\textwidth}{
            Are the sentences paraphrases of each other. \newline
            Sentence 1: Federal regulators have turned from sour to sweet on a proposed \$2.8 billion merger of ice cream giants Nestle Holdings Inc. and Dreyer 's Grand Ice Cream Inc .\newline
            Sentence 2: Federal regulators have changed their minds on a proposed \$2.8 billion merger of ice cream giants Nestle Holdings and Dreyer 's Grand Ice Cream .\newline
            Answer: Yes\newline\newline
            Are the sentences paraphrases of each other.\newline
            Sentence 1: In the year-ago quarter , the steelmaker recorded a profit of \$16.2 million , or 15 cents per share , on sales of \$1.14 billion .\newline
            Sentence 2: In the second quarter last year , AK Steel reported a profit of \$16.2 million , or 15 cents a share .\newline
            Answer: No\newline\newline
            Are the sentences paraphrases of each other.\newline
            Sentence 1: He added : ``I 've never heard of more reprehensiblebehaviour by a doctor .\newline
            Sentence 2: The Harrisons ’ lawyer Paul LiCalsi said : “ I ’ ve never heard of more reprehensible behaviour by a doctor .\newline
            Answer: Yes\newline\newline
            Are the sentences paraphrases of each other.\newline
            Sentence 1: While dioxin levels in the environment were up last year , they have dropped by 75 percent since the 1970s , said Caswell .\newline
            Sentence 2: The Institute said dioxin levels in the environment have fallen by as much as 76 percent since the 1970s .\newline
            Answer: No\newline\newline
            Are the sentences paraphrases of each other.\newline
            Sentence 1: [input 1]\newline
            Sentence 2: [input 2]\newline
            Answer:
        }
    }
    \caption{Few shot prompt template used for MRPC}
    \label{fig:mrpc-prompt}
\end{figure*}

\begin{figure*}
    \centering
    \fbox{
        \parbox{0.8\textwidth}{
            Is this sentence linguistically acceptable?\newline
            Sentence: The evidence assembled by the prosecution convinced the jury.\newline
            Answer: Yes\newline\newline
            Is this sentence linguistically acceptable?\newline
            Sentence: I live at the place where Route 150 crosses the Hudson River and my dad lives at it too.\newline
            Answer: No\newline\newline
            Is this sentence linguistically acceptable?\newline
            Sentence: The government's imposition of a fine.\newline
            Answer: Yes\newline\newline
            Is this sentence linguistically acceptable?\newline
            Sentence: Sam gave the ball out of the basket.\newline
            Answer: No\newline\newline
            Is this sentence linguistically acceptable?\newline
            Sentence: [input] \newline
            Answer: 
        }
    }
    \caption{Few shot prompt template used for RTE}
    \label{fig:rte-prompt}
\end{figure*}

\begin{figure*}
    \centering
    \fbox{
        \parbox{0.8\textwidth}{
            The town is also home to the Dalai Lama and to more than 10,000 Tibetans living in exile. \newline
            Question: The Dalai Lama has been living in exile since 10,000. True or False? \newline
            Answer: True \newline\newline
            P. Prayong, who like Kevala belongs to the Theravada sect of Buddhism, chose India over other Buddhist majority nations as it is the birthplace of Gautama Buddha. \newline
            Question: P. Prayong is a member of Theravada. True or False? \newline
            Answer: False \newline\newline
            The medical student accused of murdering an erotic masseuse he met on Craigslist is drowning in more than \$100,000 in student loan debt and is so broke he can't afford to pay an attorney, according to court papers. Philip Markoff, a 23-year-old suspended Boston University medical school student, owes \$130,000 in student loans and does not get money from his parents, leaving him to lean on a taxpayer-funded attorney for his defense, according to a court document in Boston Municipal Court that labels him indigent. Markoff graduated from the State University of New York-Albany and was a second-year medical student at BU.\newline
            Question: The medical student Philip Markoff was engaged. True or False?\newline
            Answer: True\newline\newline
            Traditionally, the Brahui of the Raisani tribe are in charge of the law and order situation through the Pass area. This tribe is still living in present day Balochistan in Pakistan. \newline
            Question: The Raisani tribe resides in Pakistan. True or False? \newline
            Answer: False \newline\newline
            The latest attacks targeted the U-S embassy and a top prosecutor's office in the Uzbek capital.\newline
            Question: [input]. True or False?\newline
            Answer: 
        }
    }
    \caption{Few shot prompt template used for CoLA}
    \label{fig:cola-prompt}
\end{figure*}

\begin{figure*}
    \centering
    \fbox{
        \parbox{0.8\textwidth}{
            Turkey is unlikely to become involved in, or allow U.S. forces to use Turkish territory in a Middle East war that does not threaten her territory directly. entails the U.S. to use Turkish military bases. \newline True or False? \newline Answer: False \newline \newline
            Brooklyn Borough Hall featured a Who's Who in New York's literary community during the second annual Brooklyn Book Festival. According to Brooklyn Borough President Marty Markowitz, the borough's zip code 11215 boasts more authors than anywhere else in the country. It appeared to be the case on Sunday. More than 100 authors were featured at the day-long event, including The Basketball Diaries writer Jim Carroll, former M*A*S*H star Mike Farrell, author and illustrator Mo Willems, Jack Kerouac's sometime lover and National Book Critics Circle Award recipient Joyce Johnson and PEN American Center President Francine Prose. entails the The Brooklyn Book Festival is held in Brooklyn Borough every year. \newline True or False? \newline Answer: True \newline\newline
            NASA's Saturn exploration spacecraft, Cassini , has discovered an atmosphere about the moon Enceladus . This is the first such discovery by Cassini, other than Titan , of the presence of an atmosphere around a Saturn moon. entails the Titan is the fifteenth of Saturn's known satellites.\newline True or False? \newline Answer: False \newline\newline
            Dr. Eric Goosby, a pioneer in the fight against AIDS, is President Obama's choice to run the American effort to combat the disease globally, the White House announced Monday. The President's Emergency Plan For AIDS Relief, known as Pepfar, was championed by President George W. Bush. It is expected to spend \$48 billion over the next five years and is credited with markedly reducing the disease's death rate. Its prevention policy has been controversial because of its emphasis on socially conservative methods. With a new administration and a Democratic majority in the House, organizations seeking prevention choices beyond abstinence and fidelity — including a renewed commitment to distributing condoms — are eager to try to rewrite the guidelines. entails the Pepfar is committed to fighting AIDS. \newline True or False? \newline Answer: True\newline\newline
            [input] \newline True or False? \newline Answer:
        }
    }
    \caption{Few shot prompt template used for NLI}
    \label{fig:nli-prompt}
\end{figure*}

\begin{figure*}
    \centering
    \fbox{
        \parbox{0.8\textwidth}{
            Given the following: \newline f : hi how are you doing ? \newline m : i 've been good . i 'm in school right now . \newline f : what school do you go to ? \newline m : i go to a cooking school . i will spend one year there . \newline f : really ? i know you love drawing and designing most . how do you like cooking so far ? \newline m : i like it so far , my classes are pretty good , and i plan to have my own restaurant in the future . \newline 
            Which choice is correct?\newline 
            (A)f : really ? you mean you do n't like cooking but you plan to start a restaurant in the future ? \newline 
            (B)f : really ? you mean your classes are pretty good and you plan to start a restaurant in the future ? \newline 
            (C)f : so , although your classes are not pretty good , you plan to become a teacher in the future ? \newline 
            (D)f : so , although you do n't love drawing or designing , you want to design a building in the future ? \newline 
            Answer: B \newline\newline
            
            Given the following: \newline f : dad , can i go out tonight ? \newline m : no , i 'm sorry . you ca n't . \newline f : can i ask nancy for dinner ?\newline  m : ok , but you ca n't let your brother alone .\newline 
            Which choice is correct?\newline 
            (A)f : ok. then i will ask nancy for dinner tonight .\newline 
            (B)f : i will stay at home alone because i do n't want ask nancy for dinner .\newline 
            (C)f : ok. so i can ask nancy for dinner tonight if i do n't have to have my brother companied .\newline 
            (D)f : i have to stay home with me brother because i will not ask nancy to have dinner .\newline 
            Answer: A\newline\newline
            
            Given the following:\newline m : that was such an interesting english program , i wish you enjoyed it as much as i did .\newline f : i must tell you the truth that i fell asleep after the first few minutes , as i could n't understand many of the words in the program .\newline
            Which choice is correct?\newline
            (A)m : seems that you found the program boring .\newline
            (B)m : the english program is indeed difficult . i feel it , too !\newline
            (C)m : so you think the english program is difficult . do n't worry , let me help you .\newline
            (D)m : good to know that you also found it interesting .\newline
            Answer: C \newline\newline
            
            Given the following: \newline m : should n't we invite kathy to the party tonight ? \newline f : invite kathy ? she is the one who 's planning the whole thing .\newline
            Which choice is correct?\newline
            (A)m : i have invited cathy .\newline
            (B)m : cathy planed the party . but she wo n't attend it cause she has no time .\newline
            (C)m : what a pity ! cathy is too busy to come .\newline
            (D)m : cathy planed the party . of course she will attend it .\newline
            Answer: D\newline\newline
            
            Q: Given the following: [input] \newline
            Which choice is correct? \newline
            (A) \newline
            (B) \newline
            (C) \newline
            (D) \newline
            Answer:
        }
    }
    \caption{Few shot prompt template used for MuTual}
    \label{fig:mutual-prompt}
\end{figure*}


\begin{figure*}
    \centering
    \fbox{
        \parbox{0.8\textwidth}{
            Activity: Hitting a pinata \newline
            Finish this sentence: A small girl runs away from the pinata while holding the stick. The child returns and hits the pinata one time. the people \newline
            Options: \newline
            (A) continue to remove key ingredients from the bags. \newline
            (B) cheer for the child getting hit. \newline
            (C) watch while the girl throws the stick twice. \newline
            (D) all clap and the girl smiles and turns toward the camera. \newline
            Answer: D \newline\newline
            Activity: Personal Care and Style \newline
            Finish this sentence: [header] How to apply base makeup [title] Use concealer. [step] You will wreck your look if you've got dark circles under your eyes or have blemishes throughout your face. Don't feel badly about such circles or blemishes, though. \newline
            Options: \newline
            (A) This is a natural part of makeup, and you can use that to your advantage. If you must draw your brows in, that will be where most of the concealer and powder will come in.\newline
            (B) You can always cover them up if need be. You can also apply some false lashes-it will look better, even if you only have the eyelashes.\newline
            (C) You can cover them up with concealer. Apply the concealer in an upside-down pyramid shape.\newline
            (D) [substeps] Use some concealer up off of your eyelids so you can wear concealer successfully without using too much. Apply concealer a little higher up on your eyelid than the same way.\newline
            Answer: C \newline\newline
            Activity: Vacuuming floor \newline
            Finish this sentence: The video shows a demonstration of dyson vacuum cleaner and how well it can clean particles from the floor. There's rice grain, flour and other food particles scattered on the floor. the demonstrator\newline
            Options:\newline
            (A) then kneels down next to the tested vacuum cleaner and then goes on the floor to clean the floor using a towel.\newline
            (B) turns on the vacuum cleaner and begins vacuuming the floor with the yellow dyson vacuum.\newline
            (C) wipes down the floor with towels.\newline
            (D) talks about a very hard floor that is soft and easy to clean.\newline
            Answer: B\newline\newline
            Activity: Personal Care and Style\newline
            Finish this sentence: [header] How to grow an afro with african american hair [title] Use the right comb. [step] Traditional combs and brushes will damage your curly hair, making it frizzier. Instead of these, use a wide-tooth comb, afro pick, or even your fingers to comb your hair.\newline
            Options:\newline
            (A) [substeps] Don't comb more than necessary! Just use your comb, pick, or fingers to get rid of any tangles. If you have a lot of tangles, try buying a detangling spray specifically designed for african american hair. \newline
            (B) Most african american hair is fairly straight, but with wavy hair that isn't straight, you have a naturally naturally curly afro. [substeps] If you have short hair, use a comb with some thinner bristles to help tame your hair and prevent frizz. \newline
            (C) Using a large comb will really work as well to take care of your hair without damaging its natural oils. [substeps] Small combs are best because more tangles can be too rough for curly hair. \newline
            (D) If you have very curly hair, you may prefer using a cap or a hair net instead, as these tend to do little damage instead of much. [title] Trim your hair every six to eight months. \newline
            Answer: A \newline\newline
            Activity: [input]\newline
            Finish this sentence: [input] \newline
            Options: \newline
            (A)  \newline
            (B)  \newline
            (C) \newline
            (D) \newline
            Answer:
        }
    }
    \caption{Few shot prompt template used for HellaSwag}
    \label{fig:hellaswag-prompt}
\end{figure*}

\begin{figure*}[h]
    \centering
    % First row
    \begin{subfigure}{0.22\textwidth}
        \includegraphics[width=\linewidth]{figures/figure1,4/starcoder_full_gate_projs.png}
        \caption{CPT MLP-Gate}
    \end{subfigure}
    \begin{subfigure}{0.22\textwidth}
        \includegraphics[width=\linewidth]{figures/figure1,4/starcoder_full_W_q.png}
        \caption{CPT Attention-Query}
    \end{subfigure}
    \begin{subfigure}{0.22\textwidth}
        \includegraphics[width=\linewidth]{figures/figure1,4/starcoder_full_W_o.png}
        \caption{CPT Attention-Combo}
    \end{subfigure}

    \vspace{0.5cm} % Space between rows

    % Second row
    \begin{subfigure}{0.22\textwidth}
        \includegraphics[width=\linewidth]{figures/figure1,4/magicoder_full_gate_projs.png}
        \caption{FFT MLP-Gate}
    \end{subfigure}
    \begin{subfigure}{0.22\textwidth}
        \includegraphics[width=\linewidth]{figures/figure1,4/magicoder_full_W_q.png}
        \caption{FFT Attention-Query}
    \end{subfigure}
    \begin{subfigure}{0.22\textwidth}
        \includegraphics[width=\linewidth]{figures/figure1,4/magicoder_full_W_o.png}
        \caption{FFT Attention-Combo}
    \end{subfigure}

    \vspace{0.5cm} % Space between rows

    % Third row
    \begin{subfigure}{0.22\textwidth}
        \includegraphics[width=\linewidth]{figures/figure1,4/magicoder_lora_gate_projs.png}
        \caption{LFFT MLP-Gate}
    \end{subfigure}
    \begin{subfigure}{0.22\textwidth}
        \includegraphics[width=\linewidth]{figures/figure1,4/magicoder_lora_W_q.png}
        \caption{LFFT Attention-Query}
    \end{subfigure}
    \begin{subfigure}{0.22\textwidth}
        \includegraphics[width=\linewidth]{figures/figure1,4/magicoder_lora_W_o.png}
        \caption{LFFT Attention-Combo}
    \end{subfigure}

    \caption{Additional figures for post-training interventions.}
    \label{fig:post-training-appendix}
\end{figure*}



\begin{figure*}[h]
    \centering
    % First row
    \begin{subfigure}{0.22\textwidth}
        \includegraphics[width=\linewidth]{figures/figure5/ROME_gptj_norm.png}
        \caption{ROME-norm}
    \end{subfigure}
    \begin{subfigure}{0.22\textwidth}
        \includegraphics[width=\linewidth]{figures/figure5/MEMIT_gptj_norm.png}
        \caption{MEMIT-norm}
    \end{subfigure}
    \begin{subfigure}{0.22\textwidth}
        \includegraphics[width=\linewidth]{figures/figure5/MEND_gptj_norms.png}
        \caption{MEND-norm}
    \end{subfigure}
        \begin{subfigure}{0.22\textwidth}
        \includegraphics[width=\linewidth]{figures/figure5/PMET_gptj_norms.png}
        \caption{PMET-norm}
    \end{subfigure}

    \vspace{0.5cm} % Space between rows

    % Second row
    \begin{subfigure}{0.22\textwidth}
        \includegraphics[width=\linewidth]{figures/figure5/ROME_gptj_downstream.png}
        \caption{ROME-downstream}
    \end{subfigure}
    \begin{subfigure}{0.22\textwidth}
        \includegraphics[width=\linewidth]{figures/figure5/MEMIT_gptj_downstream.png}
        \caption{MEMIT-downstream}
    \end{subfigure}
    \begin{subfigure}{0.22\textwidth}
        \includegraphics[width=\linewidth]{figures/figure5/MEND_gptj_downstream.png}
        \caption{MEND-downstream}
    \end{subfigure}
        \begin{subfigure}{0.22\textwidth}
        \includegraphics[width=\linewidth]{figures/figure5/PMET_gptj_downstream.png}
        \caption{PMET-downstream}
    \end{subfigure}

    \vspace{0.5cm} % Space between rows

    % Third row
    \begin{subfigure}{0.22\textwidth}
        \includegraphics[width=\linewidth]{figures/figure5/ROME_editing_score_gptj.png}
        \caption{ROME-editing}
    \end{subfigure}
    \begin{subfigure}{0.22\textwidth}
        \includegraphics[width=\linewidth]{figures/figure5/MEMIT_editing_score_gptj.png}
        \caption{MEMIT-editing}
    \end{subfigure}
    \begin{subfigure}{0.22\textwidth}
        \includegraphics[width=\linewidth]{figures/figure5/MEND_gptj_editing_score.png}
        \caption{MEND-editing}
    \end{subfigure}
        \begin{subfigure}{0.22\textwidth}
        \includegraphics[width=\linewidth]{figures/figure5/PMET_gptj_editing_score.png}
        \caption{PMET-editing}
    \end{subfigure}

    \caption{Norm growth during knowledge editing on GPT-J.}
    \label{fig:editing-appendix}
\end{figure*}


\begin{figure*}[h]
    \centering
    % First row
    \begin{subfigure}{0.22\textwidth}
        \includegraphics[width=\linewidth]{figures/figure3/fig3_gptj_rome.png}
        \caption{ROME}
    \end{subfigure}
    \begin{subfigure}{0.22\textwidth}
        \includegraphics[width=\linewidth]{figures/figure3/fig3_gptj_memit.png}
        \caption{MEMIT}
    \end{subfigure}
    \begin{subfigure}{0.22\textwidth}
        \includegraphics[width=\linewidth]{figures/figure3/fig3_gptj_mend.png}
        \caption{MEND}
    \end{subfigure}
        \begin{subfigure}{0.22\textwidth}
        \includegraphics[width=\linewidth]{figures/figure3/fig3_gptj_pmet.png}
        \caption{PMET}
    \end{subfigure}

    \vspace{0.5cm} % Space between rows

    \caption{Norm growth for edits 100, 500, 2000 for GPT-J.}
    \label{fig:norm-growth-gptj}
\end{figure*}

\begin{figure*}[h]
    \centering
    % First row
    \begin{subfigure}{0.22\textwidth}
        \includegraphics[width=\linewidth]{figures/figure3/fig3_gpt2xl_ft.png}
        \caption{GPT2-XL/FT}
    \end{subfigure}
    \begin{subfigure}{0.22\textwidth}
        \includegraphics[width=\linewidth]{figures/figure3/fig3_gptj_ft.png}
        \caption{GPT-J/FT}
    \end{subfigure}
    \vspace{0.5cm} % Space between rows

    \caption{Norm growth for edits 100, 500, 2000 for FT editing.}
    \label{fig:norm-growth-ft}
\end{figure*}


\begin{figure*}[h]
    \centering
    % First row
    \begin{subfigure}{0.32\textwidth}
        \includegraphics[width=\linewidth]{figures/boundary_plots/boundary_gpt2xl_unedited.png}
        \caption{GPT2-XL}
    \end{subfigure}\
    \begin{subfigure}{0.32\textwidth}
        \includegraphics[width=\linewidth]{figures/boundary_plots/boundary_gptj_unedited.png}
        \caption{GPT-J}
    \end{subfigure}
    
       
    \vspace{0.5cm} % Space between rows

    \caption{Boundary plots for unedited GPT2-XL and GPT-J.}
    \label{fig:boundary-unedited}
\end{figure*}

\begin{figure*}[h]
    \centering
    % First row
    \begin{subfigure}{0.32\textwidth}
        \includegraphics[width=\linewidth]{figures/boundary_plots/boundary_gpt2-xl_ROME_100.png}
        \caption{100}
    \end{subfigure}\
    \begin{subfigure}{0.32\textwidth}
        \includegraphics[width=\linewidth]{figures/boundary_plots/boundary_gpt2-xl_ROME_500.png}
        \caption{500}
    \end{subfigure}
    \begin{subfigure}{0.32\textwidth}
        \includegraphics[width=\linewidth]{figures/boundary_plots/boundary_gpt2-xl_ROME_2000.png}
        \caption{2000}
    \end{subfigure}
       
    \vspace{0.5cm} % Space between rows

    \caption{Boundary plots for edits 100, 500, 2000 for GPT2-XL/ROME.}
    \label{fig:boundary-GPT2XL-ROME}
\end{figure*}

\begin{figure*}[h]
    \centering
    % First row
    \begin{subfigure}{0.32\textwidth}
        \includegraphics[width=\linewidth]{figures/boundary_plots/boundary_gpt2-xl_PMET_100.png}
        \caption{100}
    \end{subfigure}
    \begin{subfigure}{0.32\textwidth}
        \includegraphics[width=\linewidth]{figures/boundary_plots/boundary_gpt2-xl_PMET_500.png}
        \caption{500}
    \end{subfigure}
    \begin{subfigure}{0.32\textwidth}
        \includegraphics[width=\linewidth]{figures/boundary_plots/boundary_gpt2-xl_PMET_2000.png}
        \caption{2000}
    \end{subfigure}
       
    \vspace{0.5cm} % Space between rows

    \caption{Boundary plots for edits 100, 500, 2000 for GPT2-XL/PMET.}
    \label{fig:boundary-GPT2XL-PMET}
\end{figure*}

\begin{figure*}[h]
    \centering
    % First row
    \begin{subfigure}{0.32\textwidth}
        \includegraphics[width=\linewidth]{figures/boundary_plots/boundary_gpt2-xl_MEND_100.png}
        \caption{100}
    \end{subfigure}
    \begin{subfigure}{0.32\textwidth}
        \includegraphics[width=\linewidth]{figures/boundary_plots/boundary_gpt2-xl_MEND_500.png}
        \caption{500}
    \end{subfigure}
    \begin{subfigure}{0.32\textwidth}
        \includegraphics[width=\linewidth]{figures/boundary_plots/boundary_gpt2-xl_MEND_2000.png}
        \caption{2000}
    \end{subfigure}
       
    \vspace{0.5cm} % Space between rows

    \caption{Boundary plots for edits 100, 500, 2000 for GPT2-XL/MEND.}
    \label{fig:boundary-GPT2XL-MEND}
\end{figure*}

\begin{figure*}[h]
    \centering
    % First row
    \begin{subfigure}{0.32\textwidth}
        \includegraphics[width=\linewidth]{figures/boundary_plots/boundary_gpt2-xl_MEMIT_100.png}
        \caption{100}
    \end{subfigure}
    \begin{subfigure}{0.32\textwidth}
        \includegraphics[width=\linewidth]{figures/boundary_plots/boundary_gpt2-xl_MEMIT_500.png}
        \caption{500}
    \end{subfigure}
    \begin{subfigure}{0.32\textwidth}
        \includegraphics[width=\linewidth]{figures/boundary_plots/boundary_gpt2-xl_MEMIT_2000.png}
        \caption{2000}
    \end{subfigure}
       
    \vspace{0.5cm} % Space between rows

    \caption{Boundary plots for edits 100, 500, 2000 for GPT2-XL/MEMIT.}
    \label{fig:boundary-GPT2XL-MEMIT}
\end{figure*}

\begin{figure*}[h]
    \centering
    % First row
    \begin{subfigure}{0.32\textwidth}
        \includegraphics[width=\linewidth]{figures/boundary_plots/boundary_gpt2-xl_FT_100.png}
        \caption{100}
    \end{subfigure}
    \begin{subfigure}{0.32\textwidth}
        \includegraphics[width=\linewidth]{figures/boundary_plots/boundary_gpt2-xl_FT_500.png}
        \caption{500}
    \end{subfigure}
    \begin{subfigure}{0.32\textwidth}
        \includegraphics[width=\linewidth]{figures/boundary_plots/boundary_gpt2-xl_FT_2000.png}
        \caption{2000}
    \end{subfigure}
       
    \vspace{0.5cm} % Space between rows

    \caption{Boundary plots for edits 100, 500, 2000 for GPT2-XL/FT.}
    \label{fig:boundary-GPT2XL-FT}
\end{figure*}




\begin{figure*}[h]
    \centering
    % First row
    \begin{subfigure}{0.32\textwidth}
        \includegraphics[width=\linewidth]{figures/boundary_plots/boundary_gptj_ROME_100.png}
        \caption{100}
    \end{subfigure}
    \begin{subfigure}{0.32\textwidth}
        \includegraphics[width=\linewidth]{figures/boundary_plots/boundary_gptj_ROME_500.png}
        \caption{500}
    \end{subfigure}
    \begin{subfigure}{0.32\textwidth}
        \includegraphics[width=\linewidth]{figures/boundary_plots/boundary_gptj_ROME_2000.png}
        \caption{2000}
    \end{subfigure}
       
    \vspace{0.5cm} % Space between rows

    \caption{Boundary plots for edits 100, 500, 2000 for GPT-J/ROME.}
    \label{fig:boundary-GPTJ-ROME}
\end{figure*}

\begin{figure*}[h]
    \centering
    % First row
    \begin{subfigure}{0.32\textwidth}
        \includegraphics[width=\linewidth]{figures/boundary_plots/boundary_gptj_PMET_100.png}
        \caption{100}
    \end{subfigure}
    \begin{subfigure}{0.32\textwidth}
        \includegraphics[width=\linewidth]{figures/boundary_plots/boundary_gptj_PMET_500.png}
        \caption{500}
    \end{subfigure}
    \begin{subfigure}{0.32\textwidth}
        \includegraphics[width=\linewidth]{figures/boundary_plots/boundary_gptj_PMET_2000.png}
        \caption{2000}
    \end{subfigure}
       
    \vspace{0.5cm} % Space between rows

    \caption{Boundary plots for edits 100, 500, 2000 for GPT-J/PMET.}
    \label{fig:boundary-GPTJ-PMET}
\end{figure*}

\begin{figure*}[h]
    \centering
    % First row
    \begin{subfigure}{0.32\textwidth}
        \includegraphics[width=\linewidth]{figures/boundary_plots/boundary_gptj_MEND_100.png}
        \caption{100}
    \end{subfigure}
    \begin{subfigure}{0.32\textwidth}
        \includegraphics[width=\linewidth]{figures/boundary_plots/boundary_gptj_MEND_500.png}
        \caption{500}
    \end{subfigure}
    \begin{subfigure}{0.32\textwidth}
        \includegraphics[width=\linewidth]{figures/boundary_plots/boundary_gptj_MEND_2000.png}
        \caption{2000}
    \end{subfigure}
       
    \vspace{0.5cm} % Space between rows

    \caption{Boundary plots for edits 100, 500, 2000 for GPT-J/MEND.}
    \label{fig:boundary-GPTJ-MEND}
\end{figure*}

\begin{figure*}[h]
    \centering
    % First row
    \begin{subfigure}{0.32\textwidth}
        \includegraphics[width=\linewidth]{figures/boundary_plots/boundary_gptj_MEMIT_100.png}
        \caption{100}
    \end{subfigure}
    \begin{subfigure}{0.32\textwidth}
        \includegraphics[width=\linewidth]{figures/boundary_plots/boundary_gptj_MEMIT_500.png}
        \caption{500}
    \end{subfigure}
    \begin{subfigure}{0.32\textwidth}
        \includegraphics[width=\linewidth]{figures/boundary_plots/boundary_gptj_MEMIT_2000.png}
        \caption{2000}
    \end{subfigure}
       
    \vspace{0.5cm} % Space between rows

    \caption{Boundary plots for edits 100, 500, 2000 for GPT-J/MEMIT.}
    \label{fig:boundary-GPTJ-MEMIT}
\end{figure*}

\begin{figure*}[h]
    \centering
    % First row
    \begin{subfigure}{0.32\textwidth}
        \includegraphics[width=\linewidth]{figures/boundary_plots/boundary_gptj_FT_100.png}
        \caption{100}
    \end{subfigure}
    \begin{subfigure}{0.32\textwidth}
        \includegraphics[width=\linewidth]{figures/boundary_plots/boundary_gptj_FT_500.png}
        \caption{500}
    \end{subfigure}
    \begin{subfigure}{0.32\textwidth}
        \includegraphics[width=\linewidth]{figures/boundary_plots/boundary_gptj_FT_2000.png}
        \caption{2000}
    \end{subfigure}
       
    \vspace{0.5cm} % Space between rows

    \caption{Boundary plots for edits 100, 500, 2000 for GPT-J/FT.}
    \label{fig:boundary-GPTJ-FT}
\end{figure*}

\begin{figure*}[h]
    \centering
    % First row
    \begin{subfigure}{0.22\textwidth}
        \includegraphics[width=\linewidth]{figures/activation_plots/gpt2-xl_MEMIT_edits_100_norm_plot.png}
        \caption{MEMIT/100}
    \end{subfigure}
    \begin{subfigure}{0.22\textwidth}
        \includegraphics[width=\linewidth]{figures/activation_plots/gpt2-xl_MEMIT_edits_500_norm_plot.png}
        \caption{MEMIT/500}
    \end{subfigure}
    \begin{subfigure}{0.22\textwidth}
        \includegraphics[width=\linewidth]{figures/activation_plots/gpt2-xl_MEMIT_edits_2000_norm_plot.png}
        \caption{MEMIT/2000}
    \end{subfigure}

    \vspace{0.5cm}

    \begin{subfigure}{0.22\textwidth}
        \includegraphics[width=\linewidth]{figures/activation_plots/gpt2-xl_MEND_edits_100_norm_plot.png}
        \caption{MEND/100}
    \end{subfigure}
    \begin{subfigure}{0.22\textwidth}
        \includegraphics[width=\linewidth]{figures/activation_plots/gpt2-xl_MEND_edits_500_norm_plot.png}
        \caption{MEND/500}
    \end{subfigure}
    \begin{subfigure}{0.22\textwidth}
        \includegraphics[width=\linewidth]{figures/activation_plots/gpt2-xl_MEND_edits_2000_norm_plot.png}
        \caption{MEND/2000}
    \end{subfigure}

    \vspace{0.5cm}

    \begin{subfigure}{0.22\textwidth}
        \includegraphics[width=\linewidth]{figures/activation_plots/gpt2-xl_PMET_edits_100_norm_plot.png}
        \caption{PMET/100}
    \end{subfigure}
    \begin{subfigure}{0.22\textwidth}
        \includegraphics[width=\linewidth]{figures/activation_plots/gpt2-xl_PMET_edits_500_norm_plot.png}
        \caption{PMET/500}
    \end{subfigure}
    \begin{subfigure}{0.22\textwidth}
        \includegraphics[width=\linewidth]{figures/activation_plots/gpt2-xl_PMET_edits_2000_norm_plot.png}
        \caption{PMET/2000}
    \end{subfigure}

    \vspace{0.5cm}

    \begin{subfigure}{0.22\textwidth}
        \includegraphics[width=\linewidth]{figures/activation_plots/gpt2-xl_FT_edits_100_norm_plot.png}
        \caption{FT/100}
    \end{subfigure}
    \begin{subfigure}{0.22\textwidth}
        \includegraphics[width=\linewidth]{figures/activation_plots/gpt2-xl_FT_edits_500_norm_plot.png}
        \caption{FT/500}
    \end{subfigure}
    \begin{subfigure}{0.22\textwidth}
        \includegraphics[width=\linewidth]{figures/activation_plots/gpt2-xl_FT_edits_2000_norm_plot.png}
        \caption{FT/2000}
    \end{subfigure}
    
    \caption{Activation Norms at different layers for edits 100, 500, 2000 for all editing methods for GPT2-XL.}
    \label{fig:activation-norm-growth-APPENDIX}
\end{figure*}




\end{document}


\appendix
% \def\year{2022}\relax
% %File: formatting-instructions-latex-2022.tex
% %release 2022.1
% \documentclass[letterpaper]{article} % DO NOT CHANGE THIS
% \usepackage[submission]{aaai25}  % DO NOT CHANGE THIS
% % \usepackage{hyperref}       % hyperlinks
% % \usepackage[hidelinks]{hyperref}
% \usepackage{url}            % simple URL typesetting
% \usepackage{booktabs}       % professional-quality tables
% \usepackage{amsfonts}       % blackboard math symbols
% \usepackage{nicefrac}       % compact symbols for 1/2, etc.
% \usepackage{microtype}      % microtypography
% \usepackage{xcolor}         % colors
% \usepackage{multirow}
% % \usepackage{wrapfig}
% \usepackage{rotating}
% % \usepackage{colortbl}
% \usepackage{graphicx}
% \usepackage{subcaption}
% \usepackage{amsmath,amsfonts,amssymb,amsthm}
% \usepackage{enumitem}
% % \usepackage[numbers]{natbib}
% % \usepackage{amsmath}
% \usepackage[linesnumbered, ruled, vlined]{algorithm2e}
% \usepackage{caption}
% \usepackage{colortbl}      % Additional color options for tables
% % \usepackage{authblk}
% \definecolor{lightgray}{gray}{0.9}
% \definecolor{red}{rgb}{1,0,0}

% % Define a command to display the gain in red
% \newcommand{\gain}[1]{\textcolor{red}{(+#1)}}
% % \newtheorem{assumption}{Assumption}
% % \newtheorem{theorem}{Theorem}
% % \newtheorem{lemma}{Lemma}
% \newcommand{\norm}[1]{\left\lVert#1\right\rVert}
% \newcommand{\abs}[1]{\left\lvert#1\right\rvert}
% % \newcommand{\mathbb{E}}{\mathbb{E}}
% % \newcommand{\mathbb{R}}{\mathbb{R}}
% \newcommand{\supp}{\mathrm{supp}}
% \newcommand{\argmin}{\mathop{\mathrm{arg,min}}}
% \newcommand{\Proj}{\mathcal{P}}
% \newtheorem{assumption}{Assumption}
% \newtheorem{lemma}{Lemma}
% \newtheorem{theorem}{Theorem}
% \newtheorem{remark}{Remark}
% \setcounter{secnumdepth}{2} %May be changed to 1 or 2 if section numbers are desired.
% \newcommand{\step}[1]{\noindent\textbf{Step #1.} }
% % \usepackage{amsmath}
% % These are are recommended to typeset listings but not required. See the subsubsection on listing. Remove this block if you don't have listings in your paper.
% \usepackage{newfloat}
% \usepackage{listings}
% \lstset{%
% 	basicstyle={\footnotesize\ttfamily},% footnotesize acceptable for monospace
% 	numbers=left,numberstyle=\footnotesize,xleftmargin=2em,% show line numbers, remove this entire line if you don't want the numbers.
% 	aboveskip=0pt,belowskip=0pt,%
% 	showstringspaces=false,tabsize=2,breaklines=true}
% % \floatstyle{ruled}
% % \newfloat{listing}{tb}{lst}{}
% % \floatname{listing}{Listing}
% %
% %\nocopyright
% %
% % PDF Info Is REQUIRED.
% % For /Title, write your title in Mixed Case.
% % Don't use accents or commands. Retain the parentheses.
% % For /Author, add all authors within the parentheses,
% % separated by commas. No accents, special characters
% % or commands are allowed.
% % Keep the /TemplateVersion tag as is
% \pdfinfo{
% /Title (AAAI Press Formatting Instructions for Authors Using LaTeX -- A Guide)
% /Author (AAAI Press Staff, Pater Patel Schneider, Sunil Issar, J. Scott Penberthy, George Ferguson, Hans Guesgen, Francisco Cruz, Marc Pujol-Gonzalez)
% /TemplateVersion (2022.1)
% }

% % DISALLOWED PACKAGES
% % \usepackage{authblk} -- This package is specifically forbidden
% % \usepackage{balance} -- This package is specifically forbidden
% % \usepackage{color (if used in text)
% % \usepackage{CJK} -- This package is specifically forbidden
% % \usepackage{float} -- This package is specifically forbidden
% % \usepackage{flushend} -- This package is specifically forbidden
% % \usepackage{fontenc} -- This package is specifically forbidden
% % \usepackage{fullpage} -- This package is specifically forbidden
% % \usepackage{geometry} -- This package is specifically forbidden
% % \usepackage{grffile} -- This package is specifically forbidden
% % \usepackage{hyperref} -- This package is specifically forbidden
% % \usepackage{navigator} -- This package is specifically forbidden
% % (or any other package that embeds links such as navigator or hyperref)
% % \indentfirst} -- This package is specifically forbidden
% % \layout} -- This package is specifically forbidden
% % \multicol} -- This package is specifically forbidden
% % \nameref} -- This package is specifically forbidden
% % \usepackage{savetrees} -- This package is specifically forbidden
% % \usepackage{setspace} -- This package is specifically forbidden
% % \usepackage{stfloats} -- This package is specifically forbidden
% % \usepackage{tabu} -- This package is specifically forbidden
% % \usepackage{titlesec} -- This package is specifically forbidden
% % \usepackage{tocbibind} -- This package is specifically forbidden
% % \usepackage{ulem} -- This package is specifically forbidden
% % \usepackage{wrapfig} -- This package is specifically forbidden
% % DISALLOWED COMMANDS
% % \nocopyright -- Your paper will not be published if you use this command
% % \addtolength -- This command may not be used
% % \balance -- This command may not be used
% % \baselinestretch -- Your paper will not be published if you use this command
% % \clearpage -- No page breaks of any kind may be used for the final version of your paper
% % \columnsep -- This command may not be used
% % \newpage -- No page breaks of any kind may be used for the final version of your paper
% % \pagebreak -- No page breaks of any kind may be used for the final version of your paperr
% % \pagestyle -- This command may not be used
% % \tiny -- This is not an acceptable font size.
% % \vspace{- -- No negative value may be used in proximity of a caption, figure, table, section, subsection, subsubsection, or reference
% % \vskip{- -- No negative value may be used to alter spacing above or below a caption, figure, table, section, subsection, subsubsection, or reference

% \setcounter{secnumdepth}{0} %May be changed to 1 or 2 if section numbers are desired.

% % The file aaai22.sty is the style file for AAAI Press
% % proceedings, working notes, and technical reports.
% %

% % Title

% % Your title must be in mixed case, not sentence case.
% % That means all verbs (including short verbs like be, is, using,and go),
% % nouns, adverbs, adjectives should be capitalized, including both words in hyphenated terms, while
% % articles, conjunctions, and prepositions are lower case unless they
% % directly follow a colon or long dash
% \title{Conditional Latent Coding with Learnable Synthesized Reference for Deep Image Compression Supplemental Material}
% % \author{
% %     %Authors
% %     % All authors must be in the same font size and format.
% %     Written by AAAI Press Staff\textsuperscript{\rm 1}\thanks{With help from the AAAI Publications Committee.}\\
% %     AAAI Style Contributions by Pater Patel Schneider,
% %     Sunil Issar,\\
% %     J. Scott Penberthy,
% %     George Ferguson,
% %     Hans Guesgen,
% %     Francisco Cruz\equalcontrib,
% %     Marc Pujol-Gonzalez\equalcontrib
% % }
% % \affiliations{
% %     %Afiliations
% %     \textsuperscript{\rm 1}Association for the Advancement of Artificial Intelligence\\
% %     % If you have multiple authors and multiple affiliations
% %     % use superscripts in text and roman font to identify them.
% %     % For example,

% %     % Sunil Issar, \textsuperscript{\rm 2}
% %     % J. Scott Penberthy, \textsuperscript{\rm 3}
% %     % George Ferguson,\textsuperscript{\rm 4}
% %     % Hans Guesgen, \textsuperscript{\rm 5}.
% %     % Note that the comma should be placed BEFORE the superscript for optimum readability

% %     2275 East Bayshore Road, Suite 160\\
% %     Palo Alto, California 94303\\
% %     % email address must be in roman text type, not monospace or sans serif
% %     publications22@aaai.org
% % %
% % % See more examples next
% % }

% %Example, Single Author, ->> remove \iffalse,\fi and place them surrounding AAAI title to use it



% % REMOVE THIS: bibentry
% % This is only needed to show inline citations in the guidelines document. You should not need it and can safely delete it.
% \usepackage{bibentry}
% % END REMOVE bibentry

% \begin{document}

% \maketitle


% \appendix
\section{Theoretical Proof}

\subsection{Problem Formulation}

Let $x \in \mathbb{R}^d$ be the original image, and $\tilde{x} \in \mathbb{R}^d$ be the reference image used for side information. We define encoders $G_1: \mathbb{R}^d \to \mathbb{R}^r$ and $G_2: \mathbb{R}^d \to \mathbb{R}^r$ for the original and reference images respectively, and a decoder $D: \mathbb{R}^r \times \mathbb{R}^r \to \mathbb{R}^d$.

The rate-distortion optimization problem is formulated as:
\begin{equation}\label{eq:RD_optimization}
\min_{G_1, G_2, D} \ \mathbb{E}_{x, \tilde{x}} \left[ R\left( G_1(x), G_2(\tilde{x}) \right) + \lambda \cdot D\left( x, D\left( G_1(x), G_2(\tilde{x}) \right) \right) \right],
\end{equation}
where $R(\cdot, \cdot)$ is the rate (compression) loss, $D(\cdot, \cdot)$ is the distortion loss (e.g., reconstruction error), and $\lambda > 0$ is a weighting parameter balancing rate and distortion.

\subsection{Assumptions}

\begin{assumption}\label{assump:spiked_covariance}
\textbf{Spiked Covariance Model for Images:}

The original image $x$ follows a spiked covariance model:
\begin{equation}\label{eq:spiked_model_original}
x = U^* s + \xi,
\end{equation}
where:
\begin{itemize}
    \item $U^* \in \mathbb{R}^{d \times r}$ is the true low-rank feature matrix with orthonormal columns ($U^{*T} U^* = I_r$).
    \item $s \in \mathbb{R}^r$ is the latent representation, with $\mathbb{E}[s] = 0$ and $\mathbb{E}[s s^T] = \Sigma_s$.
    \item $\xi \in \mathbb{R}^d$ is additive noise, independent of $s$, with zero mean and covariance $\Sigma_{\xi} = \sigma_{\xi}^2 I_d$.
\end{itemize}
\end{assumption}

\begin{assumption}\label{assump:reference_image}
\textbf{Reference Image with Irrelevant Parts:}

The reference image $\tilde{x}$ is given by:
\begin{equation}\label{eq:reference_image_model}
\tilde{x} = U^* (\rho s + \sqrt{1 - \rho^2} s_\perp) + \tilde{\xi},
\end{equation}
where:
\begin{itemize}
    \item $\rho \in [0,1]$ represents the correlation between $x$ and $\tilde{x}$.
    \item $s_\perp \in \mathbb{R}^r$ is independent of $s$, with $\mathbb{E}[s_\perp] = 0$ and $\mathbb{E}[s_\perp s_\perp^T] = \Sigma_s$.
    \item $\tilde{\xi} \in \mathbb{R}^d$ is additive noise, independent of $s$ and $s_\perp$, with zero mean and covariance $\Sigma_{\tilde{\xi}} = \sigma_{\tilde{\xi}}^2 I_d$.
    \item The total irrelevant proportion in $\tilde{x}$ is characterized by $p = 1 - \rho^2$.
\end{itemize}
\end{assumption}

\begin{assumption}\label{assump:entropy_model}
\textbf{Entropy Model for Rate Loss:}

The rate loss is based on a Gaussian entropy model:
\begin{equation}\label{eq:rate_loss}
R(z, \tilde{z}) = \mathbb{E}_{z, \tilde{z}} \left[ - \log_2 p_\theta( z \mid \tilde{z} ) \right],
\end{equation}
where $p_\theta( z \mid \tilde{z} )$ is a conditional Gaussian distribution:
\begin{equation}\label{eq:conditional_gaussian}
p_\theta( z \mid \tilde{z} ) = \mathcal{N}\left( z; \mu( \tilde{z} ), \Sigma_z \right),
\end{equation}
with $\mu( \tilde{z} )$ and $\Sigma_z$ being the mean and covariance conditioned on $\tilde{z}$.
\end{assumption}

\begin{assumption}\label{assump:distortion_loss}
\textbf{Distortion Loss:}

The distortion loss is defined as the mean squared error between the original image and the reconstructed image:
\begin{equation}\label{eq:distortion_loss}
D( x, \hat{x} ) = \| x - \hat{x} \|_2^2,
\end{equation}
where $\hat{x} = D( G_1(x), G_2(\tilde{x}) )$.
\end{assumption}

\begin{assumption}\label{assump:subgaussian_noise}
\textbf{Sub-Gaussian Noise:}

The noise vectors $\xi$ and $\tilde{\xi}$ are sub-Gaussian with parameter $\sigma^2$, i.e., for any $u \in \mathbb{R}^d$ with $\| u \|_2 = 1$,
\begin{equation}
\mathbb{P}\left( | u^T \xi | \geq t \right) \leq 2 \exp\left( - \frac{ t^2 }{ 2 \sigma^2 } \right), \quad \forall t > 0.
\end{equation}
\end{assumption}

\subsection{Main Results}

\begin{lemma}\label{lem:conditional_entropy}
Under Assumptions \ref{assump:spiked_covariance}--\ref{assump:entropy_model}, the rate loss $R(z, \tilde{z})$ can be expressed as:
\begin{equation}\label{eq:rate_loss_expression}
R(z, \tilde{z}) = \frac{1}{2 \ln 2} \left( r \ln (2 \pi e) + \ln \det( \Sigma_z ) \right).
\end{equation}
\end{lemma}

\begin{proof}
Since $p_\theta( z \mid \tilde{z} )$ is a Gaussian distribution, the differential entropy is:
\begin{equation}
h( z \mid \tilde{z} ) = \frac{1}{2} \ln \left( (2 \pi e)^r \det( \Sigma_z ) \right).
\end{equation}
Thus, the rate loss is:
\begin{equation}
R(z, \tilde{z}) = - \mathbb{E}_{z, \tilde{z}} \left[ \log_2 p_\theta( z \mid \tilde{z} ) \right] = \frac{1}{\ln 2} h( z \mid \tilde{z} ),
\end{equation}
which leads to Equation (\ref{eq:rate_loss_expression}).
\end{proof}

\begin{theorem}\label{thm:recovery_error_bound}
Under Assumptions \ref{assump:spiked_covariance}--\ref{assump:subgaussian_noise}, let $\hat{G}_1$ be the estimated encoder for the original image obtained from solving the optimization problem (\ref{eq:RD_optimization}). Then, for any $\delta \in (0,1)$, with probability at least $1 - \delta$, the following holds:
\begin{equation}\label{eq:sin_theta_bound}
\left\| \sin \Theta\left( \operatorname{span}( \hat{G}_1 ), \operatorname{span}( U^* ) \right) \right\|_F \leq C \cdot \frac{ \sqrt{ r ( \sigma_{\xi}^2 + \sigma_{\tilde{\xi}}^2 ) \log( d / \delta ) } }{ (1 - \rho) \lambda_{\min}( \Sigma_s ) \sqrt{ n } },
\end{equation}
where:
\begin{itemize}
    \item $C > 0$ is an absolute constant.
    \item $\rho$ is defined in Assumption \ref{assump:reference_image}, representing the correlation between $x$ and $\tilde{x}$.
    \item $\lambda_{\min}( \Sigma_s )$ is the minimum eigenvalue of $\Sigma_s$.
    \item $n$ is the number of training samples.
\end{itemize}
\end{theorem}

\begin{proof}
\textbf{Step 1: Formulate the Empirical Covariance Matrix}

Let $\{ x_i, \tilde{x}_i \}_{i=1}^n$ be $n$ independent samples drawn according to the model in Assumptions \ref{assump:spiked_covariance} and \ref{assump:reference_image}. Define the empirical covariance matrix:
\begin{equation}\label{eq:empirical_covariance}
S = \frac{1}{n} \sum_{i=1}^n x_i x_i^T = U^* \Sigma_s U^{*T} + \Sigma_{\xi} + \Delta,
\end{equation}
where $\Delta$ represents the sampling error.

\textbf{Step 2: Bound the Sampling Error}

Using the Matrix Bernstein Inequality for sub-Gaussian variables (see Tropp, 2012), we have:
\begin{equation}\label{eq:bernstein_bound}
\left\| \Delta \right\|_2 \leq \sigma_{\xi}^2 \sqrt{ \frac{ 2 \log( d / \delta ) }{ n } } + \sigma_{\xi}^2 \frac{ 2 \log( d / \delta ) }{ 3 n },
\end{equation}
with probability at least $1 - \delta$.

\textbf{Step 3: Analyze the Eigenvalue Gap}

The population covariance matrix is:
\begin{equation}
\Sigma_x = \mathbb{E}[ x x^T ] = U^* \Sigma_s U^{*T} + \Sigma_{\xi}.
\end{equation}
The eigenvalues of $\Sigma_x$ consist of $r$ large eigenvalues corresponding to the signal components and $d - r$ smaller eigenvalues corresponding to the noise.

The eigenvalue gap between the $r$-th and $(r+1)$-th eigenvalue is at least:
\begin{equation}\label{eq:eigenvalue_gap}
\delta_{\text{gap}} = \lambda_{\min}( U^* \Sigma_s U^{*T} ) - \lambda_{\max}( \Sigma_{\xi} ) = \lambda_{\min}( \Sigma_s ) - \sigma_{\xi}^2.
\end{equation}

\textbf{Step 4: Apply Davis-Kahan Sin Theta Theorem}

Let $\hat{U}$ be the matrix of leading $r$ eigenvectors of $S$. By the Davis-Kahan theorem, the subspace distance is bounded as:
\begin{equation}\label{eq:dk_bound}
\left\| \sin \Theta( \operatorname{span}( \hat{U} ), \operatorname{span}( U^* ) ) \right\|_F \leq \frac{ \sqrt{ 2 } \left\| \Delta \right\|_2 }{ \delta_{\text{gap}} }.
\end{equation}

\textbf{Step 5: Incorporate the Reference Image}

The presence of $\tilde{x}$ introduces additional noise due to the irrelevant components. From Assumption \ref{assump:reference_image}, the irrelevant proportion is $p = 1 - \rho^2$. This affects the effective eigenvalue gap, reducing it to:
\begin{equation}\label{eq:effective_gap}
\delta_{\text{eff}} = \lambda_{\min}( \Sigma_s ) (1 - \rho) - \sigma_{\xi}^2 - \sigma_{\tilde{\xi}}^2.
\end{equation}

\textbf{Step 6: Final Bound}

Combining Equations (\ref{eq:bernstein_bound}), (\ref{eq:dk_bound}), and (\ref{eq:effective_gap}), we have:
\begin{equation}\label{eq:final_bound}
\left\| \sin \Theta( \operatorname{span}( \hat{U} ), \operatorname{span}( U^* ) ) \right\|_F \leq \frac{ C \cdot ( \sigma_{\xi}^2 + \sigma_{\tilde{\xi}}^2 ) \sqrt{ \frac{ \log( d / \delta ) }{ n } } }{ \lambda_{\min}( \Sigma_s ) (1 - \rho) - \sigma_{\xi}^2 - \sigma_{\tilde{\xi}}^2 }.
\end{equation}

For sufficiently large $n$ and small noise levels such that $\lambda_{\min}( \Sigma_s ) (1 - \rho ) > \sigma_{\xi}^2 + \sigma_{\tilde{\xi}}^2$, the denominator is positive.

\textbf{Step 7: Simplify and Conclude}

Assuming $\sigma_{\xi}^2 + \sigma_{\tilde{\xi}}^2$ is small compared to $\lambda_{\min}( \Sigma_s ) (1 - \rho )$, we can approximate:
\begin{equation}
\left\| \sin \Theta( \operatorname{span}( \hat{U} ), \operatorname{span}( U^* ) ) \right\|_F \leq C' \cdot \frac{ \sqrt{ r ( \sigma_{\xi}^2 + \sigma_{\tilde{\xi}}^2 ) \log( d / \delta ) } }{ (1 - \rho ) \lambda_{\min}( \Sigma_s ) \sqrt{ n } }.
\end{equation}

This completes the proof.

\end{proof}

\begin{remark}
The factor $\frac{1}{1 - \rho}$ reflects the system's sensitivity to the correlation between the original and reference images. As $\rho \to 1$, indicating highly correlated images, the denominator approaches zero, and the bound grows large, showing that the system becomes more sensitive to irrelevant parts in $\tilde{x}$.
\end{remark}

\begin{remark}
This result shows a trade-off between the sample size $n$, the dimensionality $d$, the signal-to-noise ratio (through $\lambda_{\min}( \Sigma_s )$, $\sigma_{\xi}^2$, $\sigma_{\tilde{\xi}}^2$), and the correlation $\rho$ between $x$ and $\tilde{x}$. Increasing $n$ or the eigenvalue gap improves the bound, while higher noise levels or higher correlation (leading to larger $p = 1 - \rho^2$) degrade the performance.
\end{remark}

\begin{remark}
If we let $\tau = p = 1 - \rho^2$ represent the proportion of irrelevant information, as $\tau \to 1$, the bound grows as $O\left( \frac{1}{1 - \sqrt{1 - \tau}} \right)$, which can be approximated as $O\left( \frac{1}{1 - \rho} \right)$ for small $\tau$. This indicates a nonlinear degradation in feature learning efficiency, and the system maintains stability only when $\tau < \tau_c$ for some critical tolerance rate $\tau_c$.
\end{remark}

\begin{remark}
The above analysis assumes that the noise levels $\sigma_{\xi}^2$ and $\sigma_{\tilde{\xi}}^2$ are small compared to the signal strength $\lambda_{\min}( \Sigma_s )$. In practice, this means that the data should have a sufficiently strong signal component relative to noise for effective learning.
\end{remark}

\begin{remark}
The use of the Matrix Bernstein Inequality allows for tight probabilistic bounds on the sampling error, leveraging the sub-Gaussian nature of the noise. This is crucial for high-dimensional settings where $d$ is large.
\end{remark}

% \subsection{Discussion}

% This theorem provides a bound on how close the learned features (represented by $\text{Pr}(\hat{G}_1)$) are to the true features (represented by $U^*$). The bound depends on several factors:

% 1. The number of training samples $n$: As $n$ increases, the bound decreases at a rate of $O(1/\sqrt{n})$.

% 2. The proportion of irrelevant parts in the reference image $p$: As $p$ increases, the bound increases, but not catastrophically.

% 3. The intrinsic dimension $r$ and the effective rank of the noise $r(\Sigma_\xi)$: These determine the complexity of the learning problem.

% 4. The dimension of the data $d$: The bound has a mild logarithmic dependence on $d$.

% The inclusion of the entropy model in the rate loss allows us to more accurately capture the true compression performance. This result suggests that even when the reference image contains irrelevant parts or errors, the method can still learn features close to the true features, provided that the proportion of irrelevant parts is not too large and there are enough training samples.

\section{Robustness Experiments}

To validate our theoretical analysis and assess the robustness of the proposed CLC method, we conducted experiments simulating perturbations in the conditional latent. Controlled errors were introduced during both training and inference stages to evaluate the method's resilience to imperfect feature matching.

Specifically, we define a perturbation level $\epsilon \in [0, 0.5]$, which represents the probability of random feature matching. For each feature in the conditional latent, the correct match is used with probability $1-\epsilon$, and a random match from the dictionary is used with probability $\epsilon$. This perturbation is applied consistently during both training and inference, allowing the model to adapt to the noise during training while simultaneously testing its robustness during inference.

To quantify the impact of these perturbations, we adopt the Performance Reduction (PR) metric as defined in \cite{huang2023learned}:

\begin{equation}
    \text{PR} = 1 - \frac{\text{performance improvement w/ perturbation}}{\text{performance improvement w/o perturbation}},
\end{equation}

where performance improvement is measured in terms of PSNR and MS-SSIM gains over the baseline model without conditional latent coding.

Figure \ref{fig:robustness} illustrates the PR of CLC under varying levels of perturbation for both PSNR and MS-SSIM metrics. The results indicate that CLC exhibits significant robustness at lower perturbation levels. For instance, at $\epsilon = 0.1$, the PR values are 3.7\% for PSNR and 4.5\% for MS-SSIM, demonstrating a minimal impact on performance. However, as $\epsilon$ increases, the PR values rise more sharply, with PSNR and MS-SSIM reaching 43.5\% and 47.8\%, respectively, at $\epsilon = 0.5$. This trend aligns with our theoretical predictions, where the performance degradation accelerates as perturbations exceed certain thresholds.
\begin{figure}[t]
    \centering
    \includegraphics[width=0.9\linewidth]{figure/robustness_plot.pdf}
    \caption{Performance Reduction (PR) of CLC under varying perturbation levels. Lower PR indicates higher robustness.}
    \label{fig:robustness}
\end{figure}
These results confirm that while CLC can tolerate moderate levels of feature mismatch, higher levels of perturbation lead to a substantial increase in performance reduction, highlighting the importance of accurate feature matching.

\section{Additional Visualization Results}

To provide a more comprehensive understanding of the performance of our proposed method, we present additional visualization results in this section. Figures \ref{fig:visual2} and \ref{fig:visual3} showcase the reconstructed images generated by our method under typical conditions.

In these figures, the regions highlighted within the red and blue boxes represent magnified areas of the images. The red boxes focus on key details such as texture and edge sharpness, while the blue boxes highlight other regions of interest. These zoomed-in areas allow for a closer inspection of the image quality, demonstrating how our method effectively preserves fine details and maintains high visual fidelity across different scenarios.

Overall, these visual results further confirm the effectiveness of our approach in producing high-quality reconstructions with detailed preservation of critical image features.

\begin{figure*}[t]
    \centering
    \begin{minipage}{\textwidth}
        \centering
        \includegraphics[width=\textwidth]{figure/visual2.pdf}
        \caption{Visualization of reconstructed images using our method. The red and blue boxes highlight magnified areas for detailed inspection.}
        \label{fig:visual2}
    \end{minipage}
    
    \vspace{10pt} % 调整上下图像之间的间距
    
    \begin{minipage}{\textwidth}
        \centering
        \includegraphics[width=\textwidth]{figure/visual3.pdf}
        \caption{Visualization of reconstructed images using our method. The red and blue boxes highlight magnified areas for detailed inspection.}
        \label{fig:visual3}
    \end{minipage}
\end{figure*}

\section{Future Work}
While our current work has demonstrated the effectiveness of conditional latent coding (CLC) in deep image compression, its potential extends to broader vision tasks. The dynamic reference synthesis mechanism could be adapted for pose estimation \cite{shen2024imagpose,shenadvancing,li2024deviation,li2024translating}, where conditional feature alignment might enhance keypoint localization through self-supervised learning paradigms like \cite{chen2024learning}. The multi-scale dictionary construction and adaptive fusion strategies could further benefit ultra-high-resolution image segmentation \cite{sun2024ultrahighresolutionsegmentationboundaryenhanced,sunprogram,yin2024class} and remote sensing image enhancement \cite{ma2024logcanadaptivelocalglobalclassaware,10095835,tao2023dudb}, particularly when combined with token-based representation learning \cite{chen2024tokenunify}.

For medical imaging applications, our framework could integrate with 3D vision-language pretraining \cite{chen2023generative,liu2023t3d} and cross-dimension distillation \cite{liu2024cross} to handle multimodal data synthesis. The topological constraints in \cite{RAMMVC,scMFC} may synergize with our latent space modeling to improve structural coherence in electron microscopy segmentation \cite{chen2024learning} and CT text-image retrieval \cite{chen2024bimcv}. The error-bound analysis (Theorem 1) could also enhance medical image compression through knowledge distillation \cite{yang2024unicompress} while maintaining diagnostic fidelity.

In autonomous driving systems \cite{Zhang2023EHSSAE,zhang2024mapexpertonlinehdmap}, our method's robustness could be strengthened by unsupervised domain adaptation techniques \cite{deng2024unsupervised} to handle sensor noise. For multi-view estimation \cite{Yuan2024h,Yuan2024i,Yuan2024j}, the dictionary-based conditioning might unify cross-view correlations through reinforcement learning frameworks \cite{chen2023self}. However, as generative components may introduce noise \cite{qian2024maskfactory}, future work should explore quality evaluation metrics for synthesized latents and develop noise-robust training strategies like selective feature pruning \cite{chen2024tokenunify} or adversarial validation \cite{deng2024unsupervised}, ensuring reliability in downstream tasks while maintaining computational efficiency \cite{yang2024unicompress}.

\section{Pseudo-code for Encoding and Decoding}
To clearly illustrate the implementation of our proposed Conditional Latent Coding (CLC) method, we provide detailed pseudo-code. The pseudo-code covers the main steps for both encoding and decoding, including feature extraction, reference retrieval, conditional latent synthesis, and finally entropy coding and image reconstruction. The details of the encoding pseudo-code can be found in Algorithm \ref{clc_algorithm}, and the decoding pseudo-code is provided in Algorithm \ref{decodeclc}.


\begin{algorithm}[h]
\caption{Conditional Latent Coding (CLC)}
\label{clc_algorithm}
\SetKwInOut{Input}{Input}
\SetKwInOut{Output}{Output}

\Input{Image $x$, Feature dictionary $D$}
\Output{Compressed bitstream}

\BlankLine
\textbf{Function ConstructDictionary($R$):} \\
\For{each image $x_i$ in reference dataset $R$}{
    $v_i \gets \text{spp}(f_\theta(x_i))$ \\
    $\hat{v}_i \gets \text{PCA}(v_i)$
}
\texttt{Clusters:} $\{C_1, C_2, ..., C_K\} \gets \text{MiniBatchKMeans}(\{\hat{v}_i\})$ \\
Dictionary: $D \gets \{d_j = \text{argmin}_{\hat{v} \in C_j} \|\hat{v} - \mu_j\|_2\}_{j=1}^K$ \\
\Return $D$

\BlankLine
\textbf{Function ConditionalLatentCoding($x, D$):} \\
$y \gets g_a(x)$ \\
$X_r^M \gets \text{QueryDictionary}(D, f_\theta(x))$ \\
$Y_r^M \gets g_a(X_r^M)$

\BlankLine
\textbf{Conditional Latent Matching (CLM):} \\
$S_{ij} \gets \frac{\exp(\langle\phi(y_i), \phi(y_{r,j})\rangle/\tau)}{\sum_k \exp(\langle\phi(y_i), \phi(y_{r,k})\rangle/\tau)}$ \\
$y_m \gets F_m(y, Y_r^M; \theta_m)$ \\
$y_a \gets F_a(y, y_m; \theta_a)$

\BlankLine
\textbf{Conditional Latent Synthesis (CLS):} \\
$\alpha \gets \sigma(F_w([y, y_a]; \theta_f))$ \\
$\mu(y, y_a) \gets \alpha \odot y + (1 - \alpha) \odot y_a$ \\
$y_f \sim \mathcal{N}(\mu(y, y_a), \sigma^2(y, y_a))$

\BlankLine
\textbf{Entropy Coding:} \\
$z \gets h_a(y_f)$ \\
$\hat{z} \gets Q(z)$ \\
\For{$i \gets 1$ \KwTo $K$}{
    $p(y_f^i | y_f^{<i}, \hat{z}) \sim \mathcal{N}(\mu_i, \sigma_i^2)$ \\
    $r_i \gets y_f^i - \hat{y}_f^i$
}

\Return EncodedBitstream
\end{algorithm}

\begin{algorithm}[h]
\caption{Decoding with Conditional Latent Coding (CLC)}
\label{decodeclc}
\SetKwInOut{Input}{Input}
\SetKwInOut{Output}{Output}

\Input{Encoded bitstream $b$}
\Input{Feature reference dictionary $D = \{d_1, d_2, \dots, d_K\}$}
\Output{Reconstructed image $\hat{x}$}

\BlankLine
\textbf{Step 1: Extract and Decode Hyperprior} \\
Decode hyperprior $z$ from $b$: $z \gets \text{Decode}(b)$ \\
Use $z$ to estimate the initial latent representation $\hat{y}_f$: $\hat{y}_f \gets h_a^{-1}(z)$

\BlankLine
\textbf{Step 2: Retrieve Reference Features} \\
Extract features from $\hat{y}_f$ to query the dictionary $D$ \\
Retrieve top $M$ matching features $Y_r^M = \{\hat{y}_r^1, \hat{y}_r^2, \dots, \hat{y}_r^M\}$

\BlankLine
\textbf{Step 3: Conditional Latent Synthesis} \\
\For{$m \gets 1$ \KwTo $M$}{
    Perform feature matching and alignment: \\
    $\hat{y}_a^m \gets \text{Align}(\hat{y}_f, \hat{y}_r^m)$
}
Fuse aligned features to obtain final latent $\hat{y}$: \\
$\hat{y} \gets \sum_{m=1}^M \alpha_m \cdot \hat{y}_a^m$ \\
where $\alpha_m$ are dynamically computed fusion weights

\BlankLine
\textbf{Step 4: Entropy Decoding and Reconstruction} \\
Entropy decode each slice of $\hat{y}$ using $z$: \\
\For{$i \gets 1$ \KwTo $K$}{
    Decode slice $\hat{y}_i$ from $b$ using context: \\
    $\hat{y}_i \gets \text{EntropyDecode}(b, \hat{y}_{<i}, z)$
}

\BlankLine
\textbf{Step 5: Image Reconstruction} \\
Reconstruct the final image $\hat{x}$ from $\hat{y}$ using synthesis transform $g_s$: \\
$\hat{x} \gets g_s(\hat{y})$

\Return $\hat{x}$
\end{algorithm}

\section{Social Impact}

The proposed Conditional Latent Coding (CLC) framework presents significant implications for the field of deep image compression, particularly in terms of its potential for broader societal applications. By leveraging a fixed, pre-constructed feature dictionary, the CLC method enables end-to-end efficient compression without the need for complex or resource-intensive processing during runtime. This approach not only improves compression efficiency but also reduces the computational load, making it highly suitable for deployment in resource-constrained environments such as mobile devices, IoT systems, and edge computing. The ability to achieve high-quality compression with minimal overhead could lead to more widespread adoption of advanced image compression techniques, improving the accessibility and efficiency of digital communications and storage across diverse sectors.




% \bibliographystyle{plain}
% \bibliography{aaai25}
% \end{document}


\end{document}
