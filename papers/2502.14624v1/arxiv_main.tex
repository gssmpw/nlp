\documentclass[11pt]{article}

\usepackage{natbib}
\usepackage{booktabs} % For formal tables
\usepackage[ruled,linesnumbered]{algorithm2e} % For algorithms
\renewcommand{\algorithmcfname}{ALGORITHM}
\SetAlFnt{\small}
\SetAlCapFnt{\small}
\SetAlCapNameFnt{\small}
\SetAlCapHSkip{0pt}
\IncMargin{-\parindent}

\renewcommand{\cite}{\citep}

\usepackage{fullpage}


% ------------ Packages -------------
\usepackage{amsmath,amsfonts,mathtools,amsthm}
\usepackage{cleveref}
\usepackage{multirow}
\usepackage{tikz}
\usepackage{tikz-dependency}
\usepackage{enumitem}


\usepackage{tabularx}
\usepackage{tikz}
\renewcommand\tabularxcolumn[1]{m{#1}}
\newcolumntype{M}{>{\centering\arraybackslash}m{1cm}}
\newcommand\tikzmark[2]{%
\tikz[remember picture,baseline] \node[inner sep=0.1pt,outer sep=0] (#1){#2};%
}


\newcommand\link[2]{%
\begin{tikzpicture}[remember picture, overlay, >=stealth, shift={(0,0)}]
  \draw[-implies,double equal sign distance] (#1) to (#2);
\end{tikzpicture}%
}
\usepackage{capt-of}
\usepackage{makecell}
\newcommand{\edit}[1]{{#1}}

\newtheorem{theorem}{Theorem}
\newtheorem{definition}{Definition}
\newtheorem{corollary}{Corollary}
\newtheorem{lemma}{Lemma}
\newtheorem{observation}{Observation}
\newtheorem{proposition}{Proposition}
\newtheorem{claim}{Claim}


% ----------- Macros ---------------
\DeclareMathOperator*{\E}{\mathbb{E}}
\DeclareMathOperator{\aut}{aut}
\DeclareMathOperator{\stab}{Stab}
\DeclareMathOperator{\orb}{Orb}
\newcommand{\set}[1]{\left\{#1\right\}}
\newcommand{\norm}[1]{\left\lVert#1\right\rVert}

\newcommand{\Prs}[1]{\Pr\left[#1\right]}



\newcommand{\suchthat}{\,\middle\vert\,}

% macros for comments
\usepackage{color}
\newcommand{\daniel}[1]{\textcolor{blue}{Daniel: #1}}
\newcommand{\alex}[1]{\textcolor{red}{Alex: #1}}
\newcommand{\paritosh}[1]{\textcolor{teal}{Paritosh: #1}}


% macros for fair division related notations
\newcommand{\ENVY}{\mathsf{Envy}}
\newcommand{\env}{\ENVY}
\newcommand{\envy}{\ENVY}
\newcommand{\sw}{\mathsf{sw}}
\newcommand{\OPT}{\mathsf{OPT}}
\newcommand{\defeq}{\vcentcolon=}

\newcommand{\dist}{\mathcal{D}}
\newcommand{\items}{\mathcal{M}}
\newcommand{\agents}{\mathcal{N}}
\newcommand{\alloc}{\mathcal{A}}
\newcommand{\allocB}{\mathcal{B}}
\newcommand{\sdpref}{\succeq^{\mathrm{sd}}}
\newcommand{\notsdpref}{\not\succeq^{\mathrm{sd}}}
\newcommand{\sdprefneq}{\succ^{\mathrm{sd}}}

\newcommand{\ef}{EF}
\newcommand{\EF}{\mathrm{EF}}
\newcommand{\EFX}{\mathrm{EFX}}
\newcommand{\SDEF}{\mathrm{SD}\text{-}\mathrm{EF}}
\newcommand{\SDPO}{\mathrm{SD}\text{-}\mathrm{PO}}
\newcommand{\SDEFX}{\mathrm{SD}\text{-}\mathrm{EFX}}

% miscellaneous macros
\newcommand{\hatsig}{\widehat{\sigma}}
\newcommand{\hatA}{\widehat{A}}
\newcommand{\cE}{\mathcal{E}}

\newcommand{\Sph}{\mathcal{S}}
\newcommand{\Ball}{\mathcal{B}}

\newcommand{\calC}{\mathcal{C}}
\newcommand{\calT}{\mathcal{T}}
\newcommand{\calD}{\mathcal{D}}
\newcommand{\calW}{\mathcal{W}}
\newcommand{\calS}{\mathcal{S}}
\newcommand{\calA}{\mathcal{A}}
\newcommand{\conv}{\mathrm{Conv}}
\DeclareMathOperator*{\argmin}{arg\,min}
\DeclareMathOperator*{\argmax}{arg\,max}

\DeclarePairedDelimiter{\ceil}{\lceil}{\rceil}
\DeclarePairedDelimiter{\floor}{\lfloor}{\rfloor}
\allowdisplaybreaks

\newcommand\numberthis{\addtocounter{equation}{1}\tag{\theequation}}
% ----------------------------------

% Choose a citation style by commenting/uncommenting the appropriate line:
%\setcitestyle{acmnumeric}
%\setcitestyle{authoryear}

% Title. Note the optional short title for running heads. In the interest of anonymization, please do not include any acknowledgements.
\title{Online Envy Minimization and Multicolor Discrepancy: Equivalences and Separations}

\author{
Daniel Halpern\thanks{Harvard University. Email: dhalpern@g.harvard.edu} \and 
Alexandros Psomas\thanks{Purdue University. Email: apsomas@purdue.edu} \and 
Paritosh Verma\thanks{Purdue University. Email: verma136@purdue.edu} \and 
Daniel Xie\thanks{GEICO. Email: danielyxie2002@gmail.com. Work conducted while the author was at Purdue University.}
}
\date{}

% Abstract. Note that this must come before \maketitle.
\begin{document}

 
\maketitle
\thispagestyle{empty}

\begin{abstract}
We consider the fundamental problem of allocating $T$ indivisible items that arrive over time to $n$ agents with additive preferences, with the goal of minimizing \emph{envy}. This problem is tightly connected to the problem of \emph{online multicolor discrepancy}: vectors $v_1, \dots, v_T \in \mathbb{R}^d$ with $\norm{v_i}_2 \leq 1$ arrive one at a time and must be, immediately and irrevocably, assigned to one of $n$ colors to minimize $\max_{i,j \in [n]} \norm{ \sum_{v \in S_i} v - \sum_{v \in S_j} v }_{\infty}$ at each step, where $S_\ell$ is the set of vectors that are assigned color $\ell$. The special case of $n = 2$ is called \emph{online vector balancing}, introduced by Spencer nearly half a century ago ~\cite{spencer1977balancing}. It is known that multicolor discrepancy is at least as hard as envy minimization: any bound for the former implies the same bound for the latter. Against an adaptive adversary, both problems have the same optimal bound: $\Theta(\sqrt{T})$; it is not known, however, whether the optimal bounds match against weaker adversaries.

Against an oblivious adversary, \citet{alweiss2021discrepancy} give an elegant upper bound of $O(\log T)$, with high probability, for the online multicolor discrepancy problem. In a recent breakthrough, \citet{kulkarni2024optimal} improve this to $O(\sqrt{\log T})$ for the case of online vector balancing and give a matching lower bound. However, it has remained an open problem whether a $O(\sqrt{\log T})$ bound is possible for multicolor discrepancy. Furthermore, these results give, as corollaries, the state-of-the-art upper bounds for online envy minimization (against an oblivious adversary) for $n$ and two agents, respectively; it is an open problem whether better bounds are possible.

In this paper, we resolve all aforementioned open problems. We establish that online envy minimization is, in fact, equivalent to online multicolor discrepancy against the oblivious adversary: we give an upper bound of $O(\sqrt{\log T})$, with high probability, for multicolor discrepancy, and a lower bound of $\Omega(\sqrt{\log T})$ for envy minimization, resolving both problems. We proceed to study weaker adversaries, where we prove that the two problems are no longer equivalent. Against an i.i.d. adversary, we establish a separation: For online vector balancing, we give a lower bound of $\Omega\left(\sqrt{\frac{\log T}{\log \log T}}\right)$, while for envy minimization, we give an algorithm that guarantees a constant upper bound.
\end{abstract}


\section{Introduction}

Deep Reinforcement Learning (DRL) has emerged as a transformative paradigm for solving complex sequential decision-making problems. By enabling autonomous agents to interact with an environment, receive feedback in the form of rewards, and iteratively refine their policies, DRL has demonstrated remarkable success across a diverse range of domains including games (\eg Atari~\citep{mnih2013playing,kaiser2020model}, Go~\citep{silver2018general,silver2017mastering}, and StarCraft II~\citep{vinyals2019grandmaster,vinyals2017starcraft}), robotics~\citep{kalashnikov2018scalable}, communication networks~\citep{feriani2021single}, and finance~\citep{liu2024dynamic}. These successes underscore DRL's capability to surpass traditional rule-based systems, particularly in high-dimensional and dynamically evolving environments.

Despite these advances, a fundamental challenge remains: DRL agents typically rely on deep neural networks, which operate as black-box models, obscuring the rationale behind their decision-making processes. This opacity poses significant barriers to adoption in safety-critical and high-stakes applications, where interpretability is crucial for trust, compliance, and debugging. The lack of transparency in DRL can lead to unreliable decision-making, rendering it unsuitable for domains where explainability is a prerequisite, such as healthcare, autonomous driving, and financial risk assessment.

To address these concerns, the field of Explainable Deep Reinforcement Learning (XRL) has emerged, aiming to develop techniques that enhance the interpretability of DRL policies. XRL seeks to provide insights into an agent’s decision-making process, enabling researchers, practitioners, and end-users to understand, validate, and refine learned policies. By facilitating greater transparency, XRL contributes to the development of safer, more robust, and ethically aligned AI systems.

Furthermore, the increasing integration of Reinforcement Learning (RL) with Large Language Models (LLMs) has placed RL at the forefront of natural language processing (NLP) advancements. Methods such as Reinforcement Learning from Human Feedback (RLHF)~\citep{bai2022training,ouyang2022training} have become essential for aligning LLM outputs with human preferences and ethical guidelines. By treating language generation as a sequential decision-making process, RL-based fine-tuning enables LLMs to optimize for attributes such as factual accuracy, coherence, and user satisfaction, surpassing conventional supervised learning techniques. However, the application of RL in LLM alignment further amplifies the explainability challenge, as the complex interactions between RL updates and neural representations remain poorly understood.

This survey provides a systematic review of explainability methods in DRL, with a particular focus on their integration with LLMs and human-in-the-loop systems. We first introduce fundamental RL concepts and highlight key advances in DRL. We then categorize and analyze existing explanation techniques, encompassing feature-level, state-level, dataset-level, and model-level approaches. Additionally, we discuss methods for evaluating XRL techniques, considering both qualitative and quantitative assessment criteria. Finally, we explore real-world applications of XRL, including policy refinement, adversarial attack mitigation, and emerging challenges in ensuring interpretability in modern AI systems. Through this survey, we aim to provide a comprehensive perspective on the current state of XRL and outline future research directions to advance the development of interpretable and trustworthy DRL models.

%!TEX root = Article.tex

% Begin of file 2-Preliminaries.tex

\section{Foundations}
\label{sec:prl}

In this section we present some material that we will need in the subsequent
sections, and define a data model that consists of common aspects of RDF and
Property Graphs.


\subsection{A Common Data Model}

When developing a common framework for SHACL, ShEx, and PG-Schema, the first
challenge is establishing  a \emph{common data model}, since SHACL and ShEx work
on RDF, whereas PG-Schema works on Property Graphs.
Rather than using a model that generalises  both RDF and Property Graphs, we
propose a simple model, called \emph{common graphs}, which we obtained by asking
what, fundamentally, are the \emph{common aspects} of RDF and Property Graphs
(Appendix~\ref{sec:appendix-foundations} gives more details on the distilling of
common graphs).

Let us assume disjoint countable sets of nodes $\Nodes$, values $\Values$,
predicates $\Predicates$, and keys $\Keys$ (sometimes called properties).

% We sometimes say \emph{element} for a node or a value, and \emph{label} for a predicate or key. \todo{Drop if not used.}

\begin{definition}
  A \emph{common graph} is a pair $\graph = (E, \rho)$ where
  \begin{itemize}[\textbullet]
  \item
    $E \subseteq_{\mathit{fin}} \Nodes \times \Predicates \times \Nodes$ is its
    set of edges (which carry predicates), and
  \item
    $\rho \colon \Nodes \times \Keys \pto \Values$ is a finite-domain partial
    function mapping node-key pairs to values.
  \end{itemize}
  The set of nodes of a common graph $\graph$, written $\nodes(\graph)$,
  consists of all elements of $\Nodes$ that occur in $E$ or in the domain of
  $\rho$.
  Similarly, $\keys(\graph)$ is the subset of $\Keys$ that is used in $\rho$,
  and $\values(\graph)$ is the subset of $\Values$ that is used in $\rho$ (that
  is, the range of $\rho$).
\end{definition}

% \begin{example}[Media Service Common Graph] \label{ex:sharedScenario}
% To illustrate the common graphs, we introduce the following scenario. We assume a data model that has users, who can access and own accounts and invite other users to their accounts. Users have keys, such as email and credit-card. An example for this can be seen in~\Cref{fig}.
% % The nodes correspond to  conceptual classes, which will be identified by their available properties and keys. Properties are depicted as directed arrows, and keys are shown inside the conceptual classes.
% % The boxes inform about the available categories of nodes, with the keys they may have available (such as the key $\Exkey{plan}$ for nodes of category ``Account''), and properties connect nodes via directed arrows (such as $\Exprop{buyer}$, which connects nodes of category ``Sale'' and ``Account'').
% \end{example}

\begin{example}
  \label{ex:common-graph}
  Consider Figure~\ref{fig:common-graph}, containing a graph to store
  information about \emph{users} who may have access to (possibly multiple)
  \emph{accounts} in, \eg, a media streaming service.
  In this example, we have six nodes describing four persons ($u_1,...,u_4$) and
  two accounts ($a_1$, $a_2$).
  As a common graph $\graph = (E, \rho)$, the nodes are $a_1$, $u_1$, etc.
  Examples of edges in $E$ are $(u_2, \exaccess,a_1)$ and $(u_3, \exinvited,
  u_2)$.
  Furthermore, we have $\rho(u_2, \exemail) =$ d@d.d and $\rho(a_1,card) =
  1234$.
  So, $E$ captures the arrows in the figure (labelled with predicates) and
  $\rho$ captures the key/value information for each node.
  %
% Moreover, 3 predicates are used, appearing in Figure~\ref{fig:common-graph} as labels on links between nodes, representing the relation~$E$. Nodes are further associated with some key-value pairs, representing the function $\rho$.
  %
  Notice that a person may be the owner of an account, and may potentially have
  access to other accounts.
  This is captured using the predicates $\exowns$ and $\exaccess$, respectively.
  In addition, the system implements an invitation functionality, where users
  may invite other people to join the platform.
  The previous invitations are recorded using the predicate $\exinvited$.
  Both accounts and users may be privileged, which is stored via a Boolean value
  of the key~$\exprivileged$.
  We note that the presence of the key $\exemail$ (\resp, of the key (credit)
  $\excard$) is associated with, and indeed identifies users (\resp, accounts).
\end{example}

% \todo[inline]{In the example, worth noting that the graph node names are names, and not identities. Maybe it would be better to name them A, B, C, D to avoid misunderstanding?}

\begin{figure}[t]
\resizebox{1\linewidth}{!}{
  \includegraphics{example-common.pdf}
}
\Description{A diagram of the user common graph.}
\caption{The media service common graph. }
\label{fig:common-graph}
\end{figure}

It is easy to see that every common graph is a property graph (as per the formal
definition of property graphs~\cite{ABDF23}).
A common graph can also be seen as a set of triples, as in RDF.
Let
\[
  \Triples
=
  \left( \Nodes \times \Predicates \times \Nodes \right)
\;\cup\;
  \left( \Nodes \times \Keys \times \Values \right)\,.
\]
Then, a common graph can be seen as a finite set $\graph \subseteq \Triples$
such that for each $u \in \Nodes$ and $k \in \Keys$ there is at most one
$v \in \Values$ such that $(u, k, v) \in \graph$.
Indeed, a common graph $(E, \rho)$ corresponds to
\[
  E \;\cup\; \{ (u, k, v) \mid \rho(u,k) = v\}\;.
\]
When we write $\rho(u, k) = v$ we assume that $\rho$ is defined on $(u, k)$.

\medskip

\noindent\emph{Throughout the paper we see property graph $\graph$
simultaneously as a pair $(E, \rho)$ and as a set of triples from $\Triples$,
switching between these perspectives depending on what is most convenient at a
given moment.}


\subsection{Node Contents and  Neighbourhoods}

Let $\Records$ be the set of all \emph{records}, \ie, finite-domain partial
functions $r \colon \Keys \pto \Values$.
We write records as sets of pairs $\left\{ (k_1, w_1), \dots (k_n, w_n)
\right\}$ where $k_1, \dots, k_n$ are all different, meaning that $k_i$ is
mapped to $w_i$.

For a common graph $\graph = (E,\rho)$ and node $v$ in $\graph$, by a slight
abuse of notation we write $\rho(v)$ for the record $\left\{ (k, w) \mid
\rho(v,k) = w \right\}$ that collects all key-value pairs associated with node
$v$ in $\graph$.
We call $\rho(v)$ the \emph{content} of node $v$ in $\graph$.
This is how PG-Schema interprets common graphs: it views key-value pairs in
$\rho(v)$ as \emph{properties} of the node $v$, rather than independent,
navigable objects in the graph.

SHACL and ShEx, on the other hand, view common graphs as sets of triples and
make little distinction between keys and predicates.
The following notion---when applied to a node---uniformly captures the local
context of this node from that perspective: the content of the node and all
edges incident with the node.

%\begin{definition}[Neighbourhood]
%Given a common graph $\graph = (E,\rho)$ and a node $v\in\Nodes$, we write $\neigh_\graph(v)$ for the common graph $(E',\rho')$ where $E' = \left \{ (u_1,p,u_2) \in  E \mid u_1 = v \text{ or } u_2 = v\right\}$ and $\rho'$ is obtained by restricting $\rho$ so that $\rho'(v) = \rho(v)$ and $\rho'(u)$ is empty for all $u\neq v$. Similarly, for $w\in\Values$, we let $\neigh_\graph(w)$ be the common graph $(\emptyset,\rho')$ where $\rho'(u) = \left\{(k,w')\in\rho(u)\mid w'=w\right\}$ for all $u\in\Nodes$.
%Given a common graph $\graph$ and a node or value $v\in\Nodes\cup\Values$, the \emph{neighbourhood of $v$ in $\graph$}, written $\neigh_\graph(v)$, is the common graph consisting of triples $(u_1, p, u_2)$ from $\graph$ such that $p\in\Predicates\cup\Keys$ and either $u_1=v$ or $u_2=v$.
%\end{definition}

%That is, for $v\in\Nodes$,  $\neigh_\graph(v)$ is a star-shaped graph where only the central node has non-empty content.  For $w\in\Values$, $\neigh_\graph(w)$ is a graph with no edges and only a single value occurring in the contents of nodes.

%If we view common graphs as sets of triples, $\neigh_\graph(v)$ for $v\in\Nodes\cup\Values$ is simply the set of all triples from $\graph$ that mention $v$.

%We will also use the notion of \emph{partial neighbourhoods}, where only specified subsets of keys and predicates are taken into account.

%It is easiest to define it seeing common graphs as sets of triples.

\begin{definition}[Neighbourhood]
  Given a common graph $\graph$ and a node or value $v \in \Nodes \cup \Values$,
  the \emph{neighbourhood} of $v$ in $\graph$ is $\neigh_\graph(v) = \left\{
  (u_1, p, u_2) \in \graph \mid u_1 = v \text{ or } u_2 = v \right\}$.
  %
% \todo[inline]{Wim: This is ill-defined. We do say before that a common graph can be viewed as a set of triples if we want to think about it as RDF. But this definition should also apply to the PG view. We should be clearer about what we mean with the key/value pairs and only use ingredients from Def 1. In fact, if we take the RDF view, the definition is inconsistent with text below that says that, if $v$ is a value, then the neighborhood has no edges.}
% \todo[inline]{Suggestion to rephrase: introduce $\graph = (E,\rho)$ and say $\neigh_\graph(v) = \{(u_1,p,u_2) \in E \mid ... \} \cup \{???\}$ (Actually I don't understand yet what we want wrt $\rho$.)}
% \todo[inline]{Filip: In many places in the paper we treat $\graph$ as a pair $(E,\rho)$ or as a subset of $\Triples$, whatever is more convenient. It should suffice to warn the reader that we do this. We could write the definition in terms of $(E,\rho)$, but it would be clumsy. I really think it is fine as written.  On the other hand, if this is not helping, we can probably just skip this definition entirely and introduce only the $\pm$ variant of neighbourhoods in the section on ShEx.}
% \todo[inline]{Wim: OK, I understand better now what's intended and clarified below.}
\end{definition}

\todo{JH: Is this actually used anywhere?}

When $v \in \Nodes$, then $\neigh_\graph(v)$ is a star-shaped graph
where only the central node has non-empty content.
When $v \in \Values$, then $\neigh_\graph(v)$ consists of all the nodes in
$\graph$ that have some key with value $v$, which is a common graph with no
edges and a restricted function $\rho$.

%\todo[inline]{Maybe move to respective sections. Could also save space.}


\subsection{Value Types}

We assume an enumerable set of \emph{value types} $\ValueTypes$.
The reader should think of value types as \texttt{integer}, \texttt{boolean},
\texttt{date}, \etc
Formally, for each value type $\vtype \in \ValueTypes$, we assume that there is
a set $\sem{\vtype} \subseteq \Values$ of all values of that type and that each
value $v \in \Values$ belongs to some type, \ie, there is at least one $\vtype
\in \ValueTypes$ such that $v \in \sem{\vtype}$.
Finally, we assume that there is a type $\any \in \ValueTypes$ such that
$\sem{\any} = \Values$.


\subsection{Shapes and Schemas}
\label{ssec:shapes}

We formulate all three schema languages using \emph{shapes}, which are unary
formulas describing the graph's structure around a \emph{focus} node or a value.
Shapes will be expressed in different formalisms, specific to the schema
language; for each of these formalisms we will define when a focus node or value
$v \in \Nodes \cup \Values$ \emph{satisfies} shape $\varphi$ in a common graph
$\graph$, written $\graph, v \models \varphi$.

Inspired by ShEx \emph{shape maps}, we abstract a schema $\schema$ as a set of
pairs $(\sel,\varphi)$, where $\varphi$ is a shape and $\sel$ is a
\emph{selector}.
A selector is also a shape, but usually a very simple one, typically checking
the presence of an incident edge with a given predicate, or a property with a
given key.
A graph $\graph$ is \emph{valid} \wrt $\schema$, in symbols $\graph \models
\schema$, if
\[
  \graph, v \models \sel
\quad \text{implies} \quad
  \graph, v \models \varphi,
\]
for all $v \in \Nodes \cup \Values$ and $(\mathit{sel}, \varphi) \in \schema$.
That is, for each focus node or value satisfying the selector, the graph around
it looks as specified by the shape.
We call schemas $\schema$ and $\schema'$ \emph{equivalent} if $\graph \models
\schema$ \iff $\graph \models \schema'$, for all $\graph$.
In what follows, we may use $\mathit{sel} \Rightarrow \varphi$ to indicate a
pair $(\mathit{sel}, \varphi)$ from a schema $\SHACLSchema$.

% \begin{example}[Schemas over Media Service Common Graph]
%     \label{ex:ShapeExample}

% We stay in the same scenario introduced in \Cref{ex:sharedScenario}. We list here illustrative examples for requirements on common graphs that can be imposed via schemas.  To give an intuitive idea about the selector and the shape, we indicate this informally by splitting the sentences into an initial part that selects nodes or values, and the second part which must hold for these elements:\\
% \noindent
% \emph{For every account}, there must exist a primary credit card ; \\
% \noindent \emph{For every account}, there are  five users of it or less;\\
% \emph{Every owner of an account}, has a unique email address.
% \end{example}

\begin{example}
  \label{ex:constraint-desc}
  We next describe some constraints one may want to express in the domain of
  Example~\ref{ex:common-graph}.
  \begin{enumerate}[(C1)]
  \item
    We may want the values associated to certain keys to belong to concrete
    datatypes, like strings or Boolean values.
    In our example, we want to state that the value of the key $\excard$ is
    always an integer.
  \item
    We may expect the existence of a value associated to a key, an outgoing
    edge, or even a complex path for a given source node.
    For our example, we require that all owners of an account have an email
    address defined.
  \item
    We may want to express database-like uniqueness constraints.
    For instance, we may wish to ensure that the email address of an account
    owner uniquely identifies them.
  \item
    We may want to ensure that all paths of a certain kind end in nodes with
    some desired properties. For example, if an account is privileged, then all
    users that have access to it should also be privileged.
  \item
    We may want to put an upper bound on the number of nodes reached from a
    given node by certain paths. For instance, every user may have access to at
    most 5 accounts.
\end{enumerate}

% \todo[inline]{Wim: Reminder to self. I'd like to illustrate some open/closed things here. (There's no time anymore for this.)}
% \todo[inline]{Wim: More urgently though, we should explain better about how we model things. Let's say that ``users'' are those nodes that have an email key and ``accounts'' are those that have a card key?}
% \todo[inline]{Iovka: I support the need to make this precise. Then, should we use these two selectors in all examples?\\
% Also, we might say that we need this trick because we do not have rdf:type nor labels on nodes.}
% \todo[inline]{Cem: After discussion with Filip, I fixed the setting such that it is keys that identify users and accounts. Problem: this makes C2 awkward. }

\end{example}

% End of file 2-Preliminaries.tex


\section{Performance against an oblivious adversary}\label{sec: oblivious}
 
In this section, we prove that there exists an algorithm for the online multicolor discrepancy problem that guarantees, with high probability, a maximum discrepancy of $O(\sqrt{\log T})$ against an oblivious adversary (\Cref{subsec: optimal multicolor discrepancy}); there is a matching lower bound for the online vector balancing problem~\cite{kulkarni2024optimal}, therefore our algorithm for the online multicolor discrepancy problem is optimal. As a corollary, we get an algorithm that guarantees, with high probability, an envy of $O(\sqrt{\log T})$, against an oblivious adversary, for the online envy minimization problem; we give a matching lower bound in~\Cref{subsec: envy for oblivious}. Missing proofs are deferred to~\Cref{app: missing from section 3}.

\subsection{Balancing sets of vectors}\label{subsec: balance sets of vectors}

% \alex{Define $\gamma_d(K)$, $\conv(.)$, $\mathbf{0}$. }


Similar to the result of \citet{kulkarni2024optimal}, we think of a discretized adversary whose choices correspond to the edges of a (massive) rooted tree. \citeauthor{kulkarni2024optimal} prove the following:


\begin{theorem}[\cite{kulkarni2024optimal}]\label{theorem:tree-kulkarni} Let $\calT = (V,E)$ be a rooted tree, where every edge $e \in E$ has a corresponding vector $v_e \in \Ball^d_2$. Let $K \subseteq \mathbb{R}^d$ be a convex body with $\gamma_d(K) \geq 1 - \frac{1}{2|E|}$. Then there exists $z \in \{-1,1\}^{|E|}$ such that, for all $u \in V$, the vector sum $\sum_{e \in P_u} z_e v_e \in 5K$, where $P_u$ is the set of edges of the path from the root to the node $u$.
\end{theorem}

For our result, we think of the adversary as assigning a set of vectors $S_e$ to each edge $e$, and prove that we can choose a vector from each set $S_e$, so that all paths from the root are balanced.


\begin{theorem}\label{theorem:tree-reduction}
 Let $\calT = (V,E)$, $|E| \geq 2$, be a rooted tree such that: (1) every $e \in E$ has an associated set of vectors $S_e \subseteq \Ball^d_2$ satisfying $\mathbf{0} \in \conv(S_e)$, and (2) there exists an $\ell \in \mathbb{N}$, $\ell \geq 2$, such that, for all $e \in E$, $\mathbf{0}$ is a convex combination of at most $\ell$ vectors in $S_e$. Let $K \subseteq \mathbb{R}^d$ be a symmetric convex body with $\gamma_d(K) \geq 1 - \frac{1}{\ell\, |E|}$. Then, for every edge $e \in E$, there exists a vector $v_e \in S_e$, such that for all $u \in V$, $\sum_{e \in P_u} v_e \in 11 \, K$, where $P_u$ is the set of edges of the path from the root to the node $u$.
\end{theorem}
\begin{proof}
    Our goal is to select a vector $v_e \in S_e$ from each set $S_e$ to satisfy the desired property. At a high level, we start with a \emph{fractional} selection of vectors from each set $S_e$ such that, for every node $u \in V$, the fractional vector sum of the edges in the path from the root to $u$ is $\mathbf{0}$ (so, the desired property of being contained in $11 K$ is clearly satisfied). We iteratively round this fractional selection to get a single vector from each set $S_e$, in a way that every rounding step does not increase the vector sums we are interested in by too much.


    For all $e \in E$, by definition, $\mathbf{0} \in \conv(S_e)$ and $\mathbf{0}$ is a convex combination of at most $\ell$ vectors in $S_e$. %\footnote{Via Caratheodory's theorem, $\ell$ must be at most $n+1$. \alex{why is this being pointed out? where is it used?}} 
    Therefore there exists a fractional selection of vectors $X^e = \{(v^e_1,x^e_1), (v^e_2,x^e_2), \ldots, (v^e_{\ell},x^e_{\ell})\}$, with $(v^e_i,x^e_i) \in S_e \times [0,1]$, such that $\sum_{i=1}^{\ell} x^e_{i} \cdot v^e_i = \mathbf{0}$, and $X^e$ is \emph{feasible}, i.e., $\sum_{i=1}^{\ell} x^e_{i} = 1$ and $x^e_i \in [0,1]$, for all $i \in [\ell]$. In the subsequent proof, we show the following two claims:
    \begin{enumerate}
        \item [(a)] 
            We can round each $x^e_i$ to $\hat{x}^e_i$ such that (i) for every $e \in E$, $\sum_{i=1}^{\ell} \hat{x}^e_i = 1$, and $\hat{x}^e_i \in [0,1]$ for all $i\in [\ell]$, (ii) the fractional part of each $\hat{x}^e_i$ is at most $\log(\frac{2 \, \ell \, h}{\varepsilon})$ bits long, where $h$ is the height of the tree $\calT$ and $\varepsilon = \frac{1}{2}-\frac{1}{\ell |E|}  > 0$, and (iii) for all $u \in V$, we have $\sum_{e \in P_u} \sum_{i=1}^{\ell} \hat{x}^e_i \cdot v^e_i \in K$.
        
        %Note that $\ell \geq 2$, and without loss of generality $|E| \geq 2$ (otherwise the theorem is trivial), so $\frac{1}{2}-\frac{1}{\ell |E|} > 0$. \alex{fix} 
        \item [(b)] 
            Let $Y^e = \{(v^e_1,y^e_1), \ldots, (v^e_{\ell},y^e_{\ell})\}$ be a feasible ($\sum_{i=1}^{\ell} y^e_i = 1$ and $y^e_i \in [0,1]$ for all $i \in [\ell]$), fractional selection of vectors. If, for every $e \in E$ and $i \in [\ell]$, the fractional part of $y^e_i$ is $k \geq 1$ bits long, then we can round the $k^{th}$-bit of all $\{y^e_i\}_{i,e}$ to get $\{\hat{y}^e_i\}_{i,e}$, whose fractional parts are at most $k-1$ bits long, and, for every $u \in V$, $\left( \sum_{e \in P_u} \sum_{i=1}^{\ell} \hat{y}^e_i v^e_i - \sum_{e \in P_u} \sum_{i=1}^{\ell} y^e_i v^e_i  \right) \in 2^{-k} \cdot 10 K$.
    \end{enumerate}

Starting from the fractional selection $X^e$, applying the rounding process of (a) gives us a feasible fractional selection $\hat{X}^e$ such that $\sum_{e \in P_u} \sum_{i=1}^{\ell} \hat{x}^e_i \cdot v^e_i \in K$. Then, repeatedly applying the rounding process of (b), starting from the fractional selection $\hat{X}^e$, results in an integral selection, after at most $\log(\frac{2 \, \ell \, h}{\varepsilon})$ steps. Let $A^e = \{ a^e_i \}_{i \in [\ell]}$, where $a^e_i \in \{ 0 , 1 \}$ and $\sum_{i=1}^\ell a^e_i = 1$, be the integral selection obtained at the end of the process. We have that $\left( \sum_{e \in P_u} \sum_{i=1}^{\ell} a^e_i v^e_i - \sum_{e \in P_u} \sum_{i=1}^{\ell} \hat{x}^e_i v^e_i  \right) \in \left( 2^{-\log(\frac{2 \, \ell \, h}{\varepsilon})} + 2^{-\log(\frac{2 \, \ell \, h}{\varepsilon})+1} + \dots + 2^{-1} \right) \cdot 10 K \subseteq 10 K$. And, since $\sum_{e \in P_u} \sum_{i=1}^{\ell} \hat{x}^e_i \cdot v^e_i \in K$, we have that $\sum_{e \in P_u} \sum_{i=1}^{\ell} a^e_i v^e_i \in 11 K$.

\paragraph{Proving $(a)$.} 
Let $\varepsilon = \frac{1}{2}-\frac{1}{\ell |E|}  > 0$ be a constant, and let $b = \log(\frac{2 \, \ell \, h}{\varepsilon})$. Construct $z^e_i$, for every $e \in E$ and $i \in [\ell]$, by taking $x^e_i$ and setting to zero all the bits in the fractional part of $x^e_i$ after the $b^{th}$ bit. 
Then, $\hat{x}^e_1 = z^e_1 + \left( 1 - \sum_{i=1}^{\ell} z^e_i \right)$, and $\hat{x}^e_i = z^e_i$ for all $i = 2, \dots, \ell$.
Clearly, for all $i \geq 2$, $\hat{x}^e_i \in [0,1]$, and the fractional part of $\hat{x}^e_i$ is at most $b$ bits long. 
By definition, $\sum_{i=1}^\ell \hat{x}^e_i = \hat{x}^e_1 + \sum_{i=2}^{\ell} \hat{x}^e_i = z^e_1 + \left( 1 - \sum_{i=1}^{\ell} z^e_i \right) + \sum_{i=2}^{\ell} z^e_i = 1$.
Furthermore, since $\sum_{i=2}^{\ell} \hat{x}^e_i \leq 1$ and $\sum_{i=1}^\ell \hat{x}^e_i = 1$, we have that $\hat{x}^e_1 \in [0,1]$. By the construction of the $z^e_i$s, the fractional part of $1 - \sum_{i=1}^{\ell} z^e_i = \sum_{i=1}^{\ell} x^e_i - \sum_{i=1}^{\ell} z^e_i$ is at most $b$ bits long, and therefore, the fractional part of $\hat{x}^e_1$ is at most $b$ bits long. Finally, for all $u \in V$,
\begin{align*}
\norm{ \sum_{e \in P_u} \sum_{i=1}^\ell \hat{x}^e_i \cdot v^e_i}_2 &= \norm{ \sum_{e \in P_u} \sum_{i=1}^\ell \hat{x}^e_i \cdot v^e_i - \sum_{e \in P_u} \sum_{i=1}^\ell x^e_i \cdot v^e_i}_2 \\
&= \norm{ \sum_{e \in P_u} \sum_{i=1}^\ell (\hat{x}^e_i - x^e_i) \cdot v^e_i }_2 \\
&\leq h \cdot \left( \ell \cdot 2^{-b} + (\ell - 1) \cdot 2^{-b} \right) \\
&\leq \varepsilon,
\end{align*}
where in the first equality we used the fact that $\sum_{e \in P_u} \sum_{i=1}^\ell x^e_i \cdot v^e_i = \mathbf{0}$ and in the first inequality we used that $\hat{x}^e_1 - x^e_1 \leq \ell \cdot 2^{-b}$. Therefore, $\sum_{e \in P_u} \sum_{i=1}^\ell \hat{x}^e_i \cdot v^e_i \in \varepsilon \, \Ball^d_2$. Since $\gamma_d(K) \geq 1 - \frac{1}{\ell|E|} \geq \frac{1}{2} + \varepsilon$, we have $\varepsilon \, \Ball^d_2 \subseteq K$ (\Cref{proposition:large-body-ball}). So, overall, $\sum_{e \in P_u} \sum_{i=1}^\ell \hat{x}^e_i \cdot v^e_i \in K$.
    
\paragraph{Proving $(b)$.} Assume that for all $e \in E$ and $i\in [\ell]$, the fractional part of $y^e_i$ is at most $k$ bits long. To round the $k^{th}$-bits of $\{y^e_i\}_{i,e}$, we first construct a new tree $\calT' = (V',E')$ that has vectors (instead of sets) associated to each edge, and then invoke~\Cref{theorem:tree-kulkarni} with $\mathcal{T}'$ and $K$. Intuitively, the signs from the guarantee of~\Cref{theorem:tree-kulkarni} will tell us how to round (up or down) the $k^{th}$-bit of $\{y^e_i\}_{i,e}$, so that the resulting $\{\hat{y}^e_i\}_{i,e}$ (after rounding) have at most $k-1$ bits in the fractional part, and this rounding process doesn't incur too much cost. 

For each $e \in E$, let $I^e = \{i_1, i_2, \ldots, i_{2q}\} \subseteq [\ell]$ be the set of indices such that, for every $j \in I^e$, the $k^{th}$ bit of the fractional part of $y^e_{j}$ is $1$; the set $I^e$ may be empty. Since $\sum_{i=1}^{\ell} y^e_i = 1$, $|I^e|$ must be even. For every $e \in E$, we pair up consecutive indices in $I^e$, and corresponding to each pair we define a vector. Formally, if $I^e = \emptyset$, set $\widetilde{S}^e = \{\mathbf{0}\}$; otherwise, if $I^e \neq \emptyset$, define the set of vectors 
\[
\widetilde{S}^e = \left\{ \frac{1}{2}\left(v^e_{i_{2p}} - v^e_{i_{2p+1}}\right) : \text{ for every } i_{2p}, i_{2p+1} \in I^e \right\}.
\]
For each $e \in E$, we have $\widetilde{S}^e \subseteq \Ball^d_2$ since $\norm{\frac{1}{2}\left(v^e_{i_{2p}} - v^e_{i_{2p+1}}\right)}_2 \leq \frac{1}{2}\left( \norm{v^e_{i_{2p}}}_2 + \norm{v^e_{i_{2p+1}}}_2\right) \leq 1$.  Also, by definition, $|\widetilde{S}^e| \leq \frac{|I^e|}{2} \leq \lfloor \frac{\ell}{2} \rfloor$. 


To construct $\calT'$, we start from $\calT$ and replace every edge $e \in E$ by a path of $|\widetilde{S}^e|$ edges $e^{(1)}, e^{(2)}, \ldots, e^{(|\widetilde{S}^e|)}$. For every edge $e^{(i)} \in E'$, associate a unique vector $u^{(i)}_e \in \widetilde{S}^e$ (so, there's a bijection between $\widetilde{S}^e$ and $\{e^{(1)}, e^{(2)}, \ldots, e^{(|\widetilde{S}^e|)}\}$). Note that, since  $|\widetilde{S}^e| \leq \lfloor \frac{\ell}{2} \rfloor$, we have $|E'| \leq |E| \cdot \lfloor \frac{\ell}{2}\rfloor$. $\calT'$ satisfies the conditions of~\Cref{theorem:tree-kulkarni}: vectors associated with edges belong to the sets $\widetilde{S}^e$, where $\widetilde{S}^e \subseteq \Ball^d_2$. Applying~\Cref{theorem:tree-kulkarni} for $\calT'$ and $K$ (which also satisfies the conditions of~\Cref{theorem:tree-kulkarni}), there exists a sign $z_e^{(i)} \in \{-1,1\}$ for every $e \in E$ and $i \in [|\widetilde{S}^e|]$ (i.e., a sign for every edge in $E'$), such that, for every $u \in V$, $\sum_{e \in P_u} \sum_{i=1}^{|\widetilde{S}^e|} z_e^{(i)}u_e^{(i)} \in 5 K$ where $P_u$ is the set of edges on the path from the root node of $\calT$ to the node $u$ in $\calT$.

Consider an edge $e \in E$. We round every $y^e_i$ to $\hat{y}^e_i$ as follows: If $I^e = \emptyset$, then $\hat{y}^e_i = y^e_i$ for every $i \in [\ell]$. Otherwise, if $I^e \neq \emptyset$, then for every vector $u^{(i)}_e = \frac{1}{2}( v^e_{i_{2p}} - v^e_{i_{2p+1}} ) \in \widetilde{S}^e$, whose corresponding sign is $z^{(i)}_e$, we set $\hat{y}^e_{i_{2p}} =  y^e_{i_{2p}} + 2^{-k}\cdot z^{(i)}_e$ and $\hat{y}^e_{i_{2p+1}} = y^e_{i_{2p+1}} - 2^{-k}\cdot z^{(i)}_e$; all other $j \in [\ell] \setminus I^e$ are left unupdated, $\hat{y}^e_j = y^e_j$. This specifies a way to round each $y^e_{j}$ to $\hat{y}^e_{j}$.

It remains to show that our rounding procedure $(i)$ preserves feasibility, $(ii)$ sets the $k^{th}$-bit of the fractional part of $\hat{y}^e_i$ to $0$ for all $e$ and $i$, and $(iii)$ does not increase the vector sums of interest by too much: for all $u \in V$, 
$\left( \sum_{e \in P_u} \sum_{i=1}^{\ell} \hat{y}^e_i v^e_i - \sum_{e \in P_u} \sum_{i=1}^{\ell} y^e_i v^e_i  \right) \in 2^{-k} \cdot 10 K$.

Recall that $I^e$ is the set of all indices $i$ where the $k^{th}$ bit of the fractional part of $y^e_{i}$ is $1$. As per the aforementioned rounding process, for all $e \in E$ such that $I^e = \emptyset$, we have $\hat{y}^e_{i} = y^e_{i}$ for all $i \in [\ell]$, hence $(ii)$ holds.
Otherwise, if $I^e \neq \emptyset$, during the rounding we add or subtract $2^{-k}$ from every $y^e_{i_{2p}}$ and $y^e_{i_{2p+1}}$ respectively, where $i_{2p},i_{2p+1} \in I^e$. This addition and subtraction ensures that the $k^{th}$-bits of $y^e_{i_{2p}}$ and $y^e_{i_{2p+1}}$ are zero, additionally, for all $j \in [\ell] \setminus I^e$, $\hat{y}^e_j = y^e_j$, i.e., the $k^{th}$ bit remains zero; $(ii)$ follows. %(and thus, for both $y^e_{i_{2p}}$ $y^e_{i_{2p+1}}$ the $k^{th}$ bit of their fractional part is equal to $1$); therefore $(ii)$ is guaranteed for all such $\hat{y}^e_{j}$. 
Since $\hat{y}^e_{i_{2p}} + \hat{y}^e_{i_{2p+1}} = y^e_{i_{2p}} + y^e_{i_{2p+1}}$, the equality $\sum_{i=1}^{\ell} \hat{y}^e_i = \sum_{i=1}^{\ell} y^e_i = 1$ is maintained. Additionally, $\hat{y}^e_j \geq 0$ for all $e, j$, the feasibility condition $(i)$, also holds.
Finally, for $(iii)$, recall that for all $u \in V$, $\sum_{e \in P_u} \sum_{i=1}^{|\widetilde{S}^e|} z_e^{(i)} u_e^{(i)} = \sum_{e \in P_u} \sum_{i=1}^{|\widetilde{S}^e|} z_e^{(i)} \frac{1}{2} \left(v^e_{i_{2p}} - v^e_{i_{2p+1}}\right) \in 5 K$, by~\Cref{theorem:tree-kulkarni}. Therefore, by our rounding process: $\sum_{e \in P_u} \sum_{i=1}^{\ell} (\hat{y}^e_i -  y^e_i ) \cdot v^e_i \leq \sum_{e \in P_u} \sum_{i=1}^{|\widetilde{S}^e|} 2^{-k} z^{(i)}_e \left(v^e_{i_{2p}} - v^e_{i_{2p+1}}\right) \in 2^{-k} \cdot 10 K$.
\end{proof}








% ======


% We will show that each $x^e_i \in [0,1]$ that we start off with can be assumed to have at most $\log(\frac{\ell\cdot d}{\varepsilon})$ decimal digits where $d$ is the depth of the tree $\calT$ and $\varepsilon>0$ is any constant such that $\gamma_d(K) \geq 1 - \frac{1}{\ell|E|} \geq \frac{1}{2} + \varepsilon$; such an $\varepsilon$ must exist for a large enough tree with $\ell|E| > 2$. Starting from $x^e_i$ if we drop all the decimal digits of $x^e_i$ after the $\log(\frac{\ell \cdot d}{\varepsilon})^{th}$ decimal bit, then the vector sum $\norm{\sum_{e \in P_u} \sum_{i=1}^\ell x^e_i \cdot v^e_i}_2$ will increase by at most $\ell \cdot d 2^{-\log(\frac{\ell\cdot d}{\varepsilon})} = \varepsilon$. Additionally, since $\gamma_d(\widetilde{K}) \geq \frac{1}{2} + \varepsilon$, we know that $\varepsilon \cdot \Ball^d_2 \subseteq \widetilde{K}$. Therefore assuming that $\{x^e_i\}_{i,e}$ has at most $\log(\frac{\ell \cdot d}{\varepsilon})$ decimal digits increases the concerned vectors sums by a vector that is contained in $\widetilde{K}$. 

Given~\Cref{theorem:tree-reduction}, our next task is to show that, given a rooted tree $\calT$ as above, there exists a distribution $\calD$ over vectors (one from each edge set) such that for $x \sim \calD$, $\sum_{e \in P_u} x_e$ is subgaussian, for every node $u \in V$. 


\begin{theorem}\label{theorem:tree-subgaussianity}
    Let $\calT = (V,E)$ be a rooted tree, where every $e \in E$ has an associated set of vectors $S_e \subseteq \Ball^d_2$ satisfying $\mathbf{0} \in \conv(S_e)$. Then there exists a distribution $\calD$ supported on $\bigtimes_{e \in E} S_e$ such that for $x \sim \calD$, $\sum_{e \in P_u} x_e$ is $22.11$-subgaussian for every $u \in V$, where $P_u$ is the set of edges of the path from the root to the node $u$.
\end{theorem}





    
    
    
    %However, this further implies that the following distribution $\calD$, supported over $\bigtimes_{e \in E} S_e$, is $22.11$-subgaussian: first select an element $j \in \{1,2,\ldots, N\}$ uniformly at random, and then get $\{\check{s}_e^{(i)}\}_{e \in E} \in \bigtimes_{e \in E} S_e$, where each $\check{s}_e^{(i)}$ is obtained by projecting ${s}_e^{(i)} = (\mathbf{0}, \mathbf{0}, \ldots, v, \ldots, \mathbf{0}) \in \mathbb{R}^{Nd}$ to $\check{s}_e^{(i)} \coloneqq v \in S_e$. \alex{Not sure what's happening here. Define ``project.'' Also, $j$ is never used. Is $i$ supposed to be $j$??} This concludes the proof.
    %defined as the uniform distribution over the set $\{ \{s^{(1)}_e\}_{e \in E}, \{s^{(2)}_e\}_{e \in E}, \ldots, \{s^{(N)}_e\}_{e \in E}\}$ \alex{I know what you want to say here, but the notation is wrong. $s^{(i)}_e \in \mathbb{R}^{Nd}$. $\calD$ needs to be supported on $\bigtimes_{e \in E} S_e$} is such that, for $x \sim \calD$, $\sum_{e \in P_u} x_e$ is $22.11$-subgaussian, for any node $u \in V$. 


Finally, we prove that there exists an algorithm that, given sets of vectors one at a time, selects a vector from each set such that the vector sum is $O(1)$ subgaussian.

\begin{theorem}\label{theorem:subgauss-algo}
    For every $T, k \in \mathbb{N}$, there exists an online algorithm that, given sets $S_1, S_2, \ldots, S_T \subseteq \Ball^d_2$ satisfying $1 \leq |S_i| \leq k$ and $\mathbf{0} \in \conv(S_i)$, chosen by an oblivious adversary and arriving one at a time, selects a vector $s_i \in S_i$ from each arriving set $S_i$ such that, for every $t \in [T]$, the $\sum_{i=1}^t s_i$ is $23$-subgaussian. %The time and space complexity of this algorithm is $O(T^{CdT})$ and $O(k^{\left(\frac{T}{\delta}\right)^{CdT}})$, respectively, for some constant $C > 0$.
\end{theorem}

\begin{proof}
    Let $\beta = 22.11 = 23-\delta$ be the subgaussianity parameter in the guarantee of~\Cref{theorem:tree-subgaussianity}. Additionally, let $\calW$ be the smallest $\varepsilon$-net of the set $\calS = \cup_{i=1}^k \left( \bigtimes_{j=1}^i \Ball^d_2 \right)$ for $\varepsilon = \frac{\delta}{2T}$; here, $\calS$ represents the set of all subsets $A \subseteq \Ball^d_2$ satisfying $1 \leq |A| \leq k$. From~\Cref{proposition:size-of-net}, we know that $\calW$ has size at most $\sum_{i=1}^k \left(\frac{3}{\varepsilon}\right)^{di} \leq \left(\frac{3}{\varepsilon}\right)^{d(k+1)}$. We consider a complete and full $|\calW|$-ary tree $\calT = (V,E)$ of height $T$, where every internal node $u \in V$ of $\calT$ has $|\calW|$ children, where each edge to a child-node is associated with an element of (or, set in) $\calW$. Let $A_e$ be the set that corresponds to edge $e \in E$ in our construction, where, by the definition of $\calW$, $A_e \subseteq \Ball^d_2$ and $1 \leq |A_e| \leq k$. ~\Cref{theorem:tree-subgaussianity} implies the existence of distribution $\calD$ over $\bigtimes_{e \in E} A_e$  such that for any node $u \in V$, $\sum_{e \in P_u} y_e$ is $\beta$-subgaussian for $y \sim \mathcal{D}$.
    At time $t=0$, we sample an $y \sim \calD$ and start at the root node of $\calT$. We will keep track of a location $p_t \in V$, which at the beginning of time $t$ will be a node at depth $t-1$. A time $t$, when the set $S_t$ arrives, we map it to a set $Y_t \in \argmin_{Z \in \calW \cap \Ball_2^{|S_t|}} \norm{Z - S_t}_2$, i.e., $Y_t$ is an element of $\calW$, with the same number of elements as $S_t$, closest to $S_t$. $Y_t$ corresponds to some edge $e_t$ incident to the current node $p_t$ (that is $A_{e_{t}} = Y_t$). Let $y_{t} \in Y_t$ be the vector corresponding to edge $e_t$ in the sample $y$ from $\calD$ (from time $0$). Given $y_{e_t}$, our algorithm selects vector $x_{t} \in \argmin_{v \in S_t}\norm{v - y_{e_t}}_2$. Overall, at time $t$ we: (i) map set $S_t$ to a set $Y_t$ (or, equivalently, edge $e_t$), (ii) use the sample $y$ (the same across all times) to identify a vector $y_{e_t} \in Y_t$, and finally (iii) map $y_{e_t}$ to a vector in $x_{e_t} \in S_t$; $x_{t}$ is our output in time $t$.
    Finally, we update $p_{t+1}$ to be the child of $p_t$ along edge $e_t$.

    Next, we prove that for all $t \in [T]$, $\sum_{i=1}^t x_t$ is $23$-subgaussian. Since $\calW$ is an $\varepsilon$-net of $\calS = \cup_{i=1}^k \left( \bigtimes_{j=1}^i \Ball^d_2 \right)$, $x_t \in S_t$ (and therefore, $x_t \in \calS$) and $y_{e_t} \in Y_t$ (and therefore $y_{e_t} \in \calW$), we have that that $\norm{x_t - y_{e_t}}_2 \leq \varepsilon$. Furthermore, $y \sim \calD$, and therefore $\sum_{e \in P_u} y_e$ is $\beta$-subgaussian for all $u \in V$. Noticing that $e_{1}, e_{2}, \dots, e_t$ form a path from the root of $\calT$ to some node $u$, we have that $\sum_{i=1}^t y_{e_i}$ is $\beta$-subgaussian, or, equivalently, $\norm{\sum_{i=1}^t y_{e_i}}_{\psi_2, \infty} \leq \beta = 23 - \delta$.
    
    Towards proving subgaussianity for $\sum_{i=1}^t x_i$ we have
    \begin{align*}
        &\norm{\sum_{i=1}^t x_i}_{\psi_2, \infty} \leq \norm{\sum_{i=1}^t y_{e_i}}_{\psi_2, \infty} + \norm{\sum_{i=1}^t x_i - y_{e_i}}_{\psi_2, \infty} \tag{triangle inequality}\\
        & \leq (23 - \delta) + \sum_{i=1}^t \norm{x_i - y_{e_i}}_{\psi_2, \infty} \tag{subgaussianity of $\sum_{i=1}^t y_{e_i}$ and triangle inequality}\\
        & \leq 23 - \delta + T \cdot \sup_{d \in \Sph^{d-1}} \norm{\langle x_i - y_{e_i}, d \rangle}_{\psi_2} \tag{definition of $\norm{.}_{\psi_2, \infty}$}\\
        &= 23 - \delta + T \cdot \sup_{d \in \Sph^{d-1}} \norm{ \, \norm{x_i - y_{e_i}}_2 \cdot \norm{d}_2 \cdot \cos(\theta) }_{\psi_2},
    \end{align*}
    where $\theta$ is the random angle between the vectors $x_i - y_{e_i}$ and $d$.  $\cos(\theta)$ is random variable supported on $[-1,1]$, $\norm{x_i - y_{e_i}}_2 \in [0,\varepsilon]$ and $\norm{d}_2 = 1$. Therefore, $\norm{x_i - y_{e_i}}_2 \cdot \norm{d}_2 \cdot \cos(\theta) \in [-\varepsilon,\varepsilon]$. From the definition of $\norm{.}_{\psi_2}$ norm we therefore have that $\norm{ \, \norm{x_i - y_{e_i}}_2 \cdot \norm{d}_2 \cdot \cos(\theta) }_{\psi_2} \leq \inf\{ t > 0: \E[exp(\varepsilon^2/t^2)] \leq 2 \} \leq 2 \varepsilon$. Our upper bound on $\norm{\sum_{i=1}^t x_i}_{\psi_2, \infty}$ becomes $23 - \delta + T \cdot 2 \varepsilon = 23$. %\alex{@Paritosh: double check the argument. Yours was different}\paritosh{which part in particular should I review? } \alex{Everything after ``Towards proving subgaussianity for...''}\paritosh{This looks good; what I had in mind was pretty much the same argument.}
    %The size of $\calT$ is the number of its edges, which is $|\calW| + |\calW|^2 + \ldots + |\calW|^{T} \leq (2|\calW|)^{T} \leq ( 2 (\frac{6 T}{\delta})^{d(k+1)})^{T} \in O((\frac{2T}{\delta})^{dT})$\alex{why?}, hence, the space complexity is at least this much. The time complexity of the algorithm is dominated by the time required to find the distribution $\calD$ that is guaranteed to exist by~\Cref{theorem:tree-subgaussianity}. Through~\Cref{theorem:tree-subgaussianity}, we know that $\calD$ is supported on $N = \left\lfloor \left((d+1)c_\delta^d|E|\right)^{1/\delta}\right\rfloor$ elements, where each element belongs to $\varprod_{e \in E} S_e$. Therefere, the number of such distributions are at most $k^{N |E|} = O(k^{\left(\frac{T}{\delta}\right)^{dT}})$.
\end{proof}



\subsection{Optimal online multicolor discrepancy}\label{subsec: optimal multicolor discrepancy}

Here, we prove our main result for this section, an optimal algorithm for online multicolor discrepancy. We start by giving an algorithm for \emph{weighted} online vector balancing.

\begin{lemma}\label{lemma:weighted-vector-balancing}
     For every $\alpha \in \left[\frac{1}{2}, \frac{2}{3}\right]$ and $T \in \mathbb{N}$, there exists an online algorithm that, given vectors $v_1, v_2, \ldots, v_T \in \Ball^d_2$ chosen by an oblivious adversary and arriving one at a time, assigns to each vector $v_i$ a weight $w_i \in \{1-\alpha, -\alpha\}$, such that, with probability at least $1 - \delta$, for any $\delta \in (0,1/2]$, for all $t \in [T]$, $\norm{\sum_{i=1}^t w_iv_i}_\infty \leq \sqrt{\log(T)} + \sqrt{\log(2/ \delta)}$.
\end{lemma}

Given an algorithm for weighted online vector balancing, we give an algorithm for online multicolor discrepancy: we construct a binary tree, where the leaves correspond to colors, and the internal nodes execute the weighted online vector balancing algorithm. We note that this trick has been used in the same context in previous work~\cite{alweiss2021discrepancy,bansal2021online}.

\begin{theorem}\label{thm: multi color main upper bound}
    For every $T \in \mathbb{N}$, there exists an online algorithm that, given vectors $v_1, v_2, \ldots, v_T \in \Ball^d_2$ chosen by an oblivious adversary and arriving one at a time, assigns each arriving vector $v_i$ to one of $n$ colors such that, with probability at least $1-\delta$, for any $\delta \in (0,1/2]$, for all $t \in [T]$,
    \[\max_{i,j \in [n]} \norm{\sum_{v \in \calC^t_i} v - \sum_{v \in \calC^t_j} v}_\infty \leq 6 \left( \sqrt{\log(T)} + \sqrt{\log(2/\delta)} \right) \]
 where $\calC^t_i$ is the set of all vectors that got assigned color $i \in [n]$ up to time $t \in [T]$.
\end{theorem}
\begin{proof}
    Let $\calA_\alpha$ be the algorithm of~\Cref{lemma:weighted-vector-balancing}, for an $\alpha \in [1/2, 2/3]$. We recursively construct a binary tree with $n$ leaves, corresponding to the $n$ colors. For a tree with $k > 1$ leaves we add a root node, as its left subtree recursively construct a tree with $\lceil k/2\rceil$ leaves, and as its right subtree recursively construct a tree with $\lfloor k/2\rfloor$ leaves; for $k=1$, we simply have a leaf. 
    
    Given vectors, one at a time, our algorithm for the online multicolor discrepancy problem decides which set/color a vector gets by repeatedly running $\calA_\alpha$ for the online vector balancing problem at each internal node of the tree. Specifically, at an internal node with $k$ descendent leaves, we will run a copy of $\calA_\alpha$ by setting $\alpha = \lceil k/2\rceil/k$, and by recursively passing the vectors that are assigned $1-\alpha$ (resp. $-\alpha$) to the left (resp. right) subtree (until they reach the leaves). Note that for any $k \geq 1$, we have $\alpha = \lceil k/2\rceil/k \in [1/2, 2/3]$. Vectors are assigned the color of the leaf they reach.

    Let $p_e$ be a weight for each edge $e$: the edge between the left (resp. right) child of an internal node with $k$ children has a weight $p_e = \alpha = \lceil k/2\rceil/k$ (resp. $p_e = 1 - \alpha$). The weights on the edges of an internal node is the ``opposite'' with respect to the weight of its children in the execution of $\calA_\alpha$. Intuitively, $p_e$ for an edge $(u,v)$ is the expected fraction of vectors that go to node $v$, out of the vectors that arrive at the parent node $u$.

    There are $n-1$ internal nodes in our tree. Let $\mathcal{E}$ be the event that all $n-1$ executions of $\calA_\alpha$ have maintained the discrepancy at most $\sqrt{\log(T)} + \sqrt{\log(2/ \delta)}$ between the corresponding two children nodes; $\mathcal{E}$ occurs with probability at least $1 - (n-1)\delta$. Let $S^{sum} = \sum_{i=1}^t v_i$ be the sum of all vectors until time $t$, and let $S_u$ be the sum of all vectors that have passed through node $u$ until time $t$ (so, $S_r = S^{sum}$ for the root node $r$). Also, let $\pi_u = \Pi_{e \in P_u} p_e$, for a node $u$; intuitively, $\pi_u$ is the (expected) fraction of vectors (out of $\{ v_1, \dots, v_t \}$) that arrive at node $u$.    

    We will prove, via induction on $\ell$, that conditioned on $\mathcal{E}$, for all nodes $u$ on level $\ell \leq 0$ we have
\[
\norm{ S_u - \pi_u S^{sum}}_{\infty} \leq 3 \left( \sqrt{\log(T)} + \sqrt{\log(2/ \delta)} \right).
\]

For $\ell = 0$ the statement trivially holds: for the root $r$ at level zero we have $\norm{ S_r - \pi_r S^{sum}}_{\infty} = 0 \leq 3 \left( \sqrt{\log(T)} + \sqrt{\log(2/ \delta)} \right)$. Suppose that the statement holds for level $\ell$, and let $u$ be a node in level $\ell + 1$, with parent node $p$ (on level $\ell$) and sibling node $v$ (on level $\ell + 1$). Assume that $u$ is the left child of $p$ (the other case is identical). We have that $S_p = S_u + S_v$, and $\pi_u = \pi_p \cdot \alpha$. Also, conditioned on $\mathcal{E}$ we have $\norm{ (1-\alpha)S_u - \alpha S_v}_{\infty} \leq \sqrt{\log(T)} + \sqrt{\log(2/ \delta)}$. So, overall:
\begin{align*}
    &\norm{ S_u - \pi_u S^{sum}}_{\infty} = \norm{ (1-\alpha)S_u + \alpha S_u - \alpha \pi_p S^{sum}}_{\infty} \\
    &\leq \norm{ (1-\alpha)S_u - \alpha S_v }_{\infty} + \norm{ \alpha S_v + \alpha S_u - \alpha \pi_p S^{sum}}_{\infty} \tag{triangle inequality}\\
    &= \norm{ (1-\alpha)S_u - \alpha S_v }_{\infty} + \alpha \norm{ S_p - \pi_p S^{sum}}_{\infty} \\
    &\leq \left( \sqrt{\log(T)} + \sqrt{\log(2/ \delta)} \right) + \frac{2}{3} \, 3 \left( \sqrt{\log(T)} + \sqrt{\log(2/ \delta)} \right) \\
    &\leq 3 \left( \sqrt{\log(T)} + \sqrt{\log(2/ \delta)} \right).
\end{align*}

We will also prove, via induction on $k$, that for a node $u$ that is the root of a subtree with $n-k+1$ leaves, $\pi_u = (n-k+1) \cdot \frac{1}{n}$. For the root $r$ (whose subtree has $n = n-1+1$ leaves) we have $\pi_r = 1 = n \cdot \frac{1}{n}$. Consider a node $u$ that is the left child of a node $p$, such that $p$ is the root of a subtree with $k$ leaves. Then $u$ is the root of a subtree with $\lceil k/2\rceil$ leaves, and $\pi_u = \pi_p \cdot \alpha = k \, \frac{1}{n} \cdot \lceil k/2\rceil/k = \lceil k/2\rceil \cdot \frac{1}{n}$; the case that $u$ is the right child of $p$
 is identical.

Finally, consider two arbitrary leaves $v_1$ and $v_2$. From the previous arguments we have that $\pi_{v_1} = \pi_{v_2} = \frac{1}{n}$, and $\norm{ S_{v_i} - \frac{1}{n} \, S^{sum}}_{\infty} \leq 3 \left( \sqrt{\log(T)} + \sqrt{\log(2/ \delta)} \right) $. Therefore, $\norm{ S_{v_1} - S_{v_2}}_{\infty} \leq 6 \left( \sqrt{\log(T)} + \sqrt{\log(2/ \delta)} \right)$.
\end{proof}

The lower bound for the online envy minimization problem in the next section implies that the bound of~\Cref{thm: multi color main upper bound} is optimal.

% \begin{theorem}[\cite{kulkarni2024optimal}]
%     For any $d \geq 2$, there is a strategy for an oblivious adversary that yields a sequence of vectors $v_1, \dots, v_T \in \mathbb{R}^d$ so that for any online algorithm that picks $x_t \in \{ -1 , 1 \}$ at every step $t$, with probability at least $1 - 2^{-T^{\Omega(1)}}$, one has $\max_{t \in [T]} \norm{\sum_{i=1}^t x_i v_i}_{\infty} \in \Omega(\sqrt{ \log T})$.
% \end{theorem}
    
    
    % ======


    % Define for any internal node $u$, $\pi_u = \Pi_{e \in P_u} p_e$ and for the root node $r$, $\pi_r = 1$. For the subsequent analysis, we 
    % will condition on the ``good event," i.e., the event happening with probability $1 - (n-1) \delta$ where $\calA$ running at each internal node given the guarantee mentioned in~\Cref{corollary:weighted-vector-balancing}.
    % Conditioned on this event, we consider a time $t \in [T]$ during the execution of the algorithm. Let $S^* = \sum_{i=1}^t v_i$ denote the sum of all vectors till time $t$ and for any node $u \in V$, denote by $S_u$ the sum of all vectors that pass through that node during the execution of the algorithm till time $t$; indeed, we have $S^* = S_r$ where $r$ is the root node. To prove~\Cref{equation:multicolor-reduction-guarantee}, we will prove the following two claims.
    % \begin{enumerate}
    %     \item [(a)] With probability at least $1-(n-1)\delta$, for any node $u$ in the tree, we have $\norm{S_u - \pi_u \cdot S^*}_2 \leq 3 f(T,\delta)$.
    %     \item [(b)] For all leaves $\ell$, we have $\pi_\ell = \frac{1}{n}$.
    % \end{enumerate}
    % \Cref{equation:multicolor-reduction-guarantee} follows from the above two claims as with probability at least $1-(n-1)\delta$ (and for any $t \in [T]$) for any two leaves $\ell_1$ and $\ell_2$, we have $\norm{S_{\ell_1} - S_{\ell_2}}_2 = \norm{(S_{\ell_1} - 1/n \cdot S^*) - (S_{\ell_2} - 1/n \cdot S^*)}_2 \leq^{(b)} \norm{S_{\ell_1} - p_{\ell_1} S^*}_2  + \norm{S_{\ell_2} - p_{\ell_2} S^*}_2 \leq^{(a)} 6 \cdot f(T,\delta)$. The obtained inequality $\norm{S_{\ell_1} - S_{\ell_2}}_2 \leq 6 \cdot f(T,\delta)$ is precisely same as~\Cref{equation:multicolor-reduction-guarantee} and thus plugging in $f(T,\delta) = \sqrt{\log(T)} + \sqrt{\log(1/ \delta)}$ from~\Cref{corollary:weighted-vector-balancing} will complete the proof. Henceforth, we will prove $(1)$ and $(2)$. The claim $(2)$ directly follows from the definition of $\pi_u$s and the structure of the tree: the edge $e$ going from an internal node having $k$ leaves to its left subtree having $\lceil k/2 \rceil$ leaves has weight equal to $p_e = \lceil k/2 \rceil / k$ and thus for any internal node $u$, $\pi_u$ equals to the number of its descendent leaves (including itself) times $1/n$, i.e., for a leaf $\ell$, $\pi_\ell = 1/n$. 
    
    % To prove $(2)$, we will consider an inductive argument. As the base case, for the root $r$, we trivially have $\norm{S_r - \pi_r \cdot S^*}_2 =  \norm{S^* - 1 \cdot S^*}_2  = 0 \leq 3 \cdot d(t)$. For the induction step, assume that for an internal node $u$, we have $\norm{S_u - \pi_u \cdot S^*}_2 \leq 3 \cdot f(T,\delta)$ and the left (right) child of $u$ is node $l$ ($r$). Since we are operating a copy of $\calA$ with parameter $\alpha$ (equal to the proportion of leaves in the left subtree), we have the guarantee $\norm{(1-\alpha) S_l - \alpha S_r}_2 \leq f(T,\delta)$. Along with the fact that $S_u = S_l + S_r$, we get that $\norm{S_l - \alpha S_u}_2 = \norm{S_l - \alpha (S_r + S_l)}_2 = \norm{(1-\alpha) S_l - \alpha S_r}_2 \leq f(T,\delta)$. Adding this inequality with the inequality $\norm{S_u - \pi_u \cdot S^*}_2 \leq 3 \cdot f(T,\delta)$ with $\alpha$ multiplied by $\alpha$ gives us,

    % \begin{align*}
    %     3 \cdot f(T, \delta) & \geq \alpha \cdot 3f(T, \delta) + f(T,\delta) \tag{$2/3 \geq \alpha$}\\
    %     & \geq \alpha \norm{S_u - p_u \cdot S^*}_2 + \norm{S_l - \alpha \cdot S_u}_2\\
    %     & \geq \norm{S_l - \alpha p_u \cdot S^*}_2 \\
    %     & = \norm{S_l - p_l \cdot S^*}_2,
    % \end{align*}
    % where the last equality follows from the fact that $\pi_l = \alpha p_u$. This proves the claim for node $l$ and a similar argument establishes the same for $r$, i.e., $3 \cdot f(T, \delta) \geq \norm{S_r - p_r \cdot S^*}_2$. This concludes the induction step and hence the proof





% \subsection{Alternative Reduction}

% \alex{what is this?? a proof for the corollary above?}\paritosh{From Theorem 4 we get our main results via Lemma 1 and Theorem 5; this is an alternative that we can use to get Theorem 5 from Theorem 4.}\alex{You mean our main result from Thm 4? If this is a proof of Thm 5 I don't see why it's phrased in the fair division language.}
% Each edge corresponds to an item that is valued as $(v_1, \ldots, v_n)$. We must choose one of $n$ ``vectors'' $\{\mathbf{u}^1, \ldots, \mathbf{u}^n\}$ where $\mathbf{u}^i$ that corresponds to placing the item in bundle $i$ (although think of these as anonymous). Our goal is to create an almost perfect partition where all agents value all bundles about the same. To track this, each $\mathbf{u}^i$ vector has $n \cdot (n - 1)$ dimensions, i.e., there is a value $u^i_{jk}$ for all $k \ne i$ and $i, j, k \in [n]$. The coordinate $u^i_{jk}$ tracks $V_j(A_k) - V_j(A_j)$. Therefore,
% \[
% u^i_{jk} = 
%     \begin{cases}
%         v_j &\text{ if } k = i, j\ne i\\
%         -v_j & \text{ if } j = i, k \ne i\\
%         0 & \text{ otherwise}.
%     \end{cases}
% \]
% Note that $\mathbf{0}$ is in the convex hull because $\sum_{i = 1}^n \mathbf{u}^i/n  = 0$, positive for exactly one vector and negative for exactly one.

% If we've chosen a sequence of vectors such that all coordinates are small, this means that we have found an almost perfect partition. 





\subsection{Optimal online envy minimization}\label{subsec: envy for oblivious}


\Cref{thm: multi color main upper bound} immediately implies, for the online envy minimization problem, a $O_n(\sqrt{\log{T}})$ upper bound against an oblivious adversary.

\begin{corollary}\label{cor: main result for oblivious and fair division}
    For any $n \geq 2, T \geq 1$ and $\delta \in (0,1/2]$, there exists an online algorithm that, given a sequence of $T$ items with $v_{i,t} \in [0,1]$ for all $i \in [n]$ and $t \in [T]$ selected by an oblivious adversary and arriving one at a time, allocates each item to an agent such that the envy between any pair of agents $i,j \in [n]$ satisfies, $\envy^t_{i,j} \in O_n(\sqrt{\log{T}})$ with probability at least $1 - \frac{1}{T^c}$, for any constant $c$.
\end{corollary}


Here, we prove a lower bound of $\Omega_n((\log(T))^{r/2})$, for all $r < 1$, for the online envy minimization problem, against an oblivious adversary. Our proof crucially uses the construction in the lower bound of~ \cite{benade2024fair} for the online envy minimization problem, against an adaptive adversary; for completeness, we include a proof of this result in~\Cref{app:proof from OR paper}.


\begin{theorem}[Theorem 2 of \cite{benade2024fair} restated]\label{theorem:adaptive-lb}
    For any $n \geq 2$, $r < 1$ and $T \geq 1$, there exists a set $S_T$ of instances with $|S_T|  \leq 2^T$ such that for any online algorithm $\mathcal{A}$, there exists an instance $I \in [0,1]^{n\cdot m}$ in $S_T$ such that running algorithm $\mathcal{A}$ on the sequence of items $1,2 \ldots, T$ described by $I$ results in a maximum envy of at least $\envy^T \in \Omega_n(T^{r/2})$ at time $T$.
\end{theorem}

Our result is stated as follows.



\begin{theorem}\label{thm: envy lower bound for oblivious}
    Fix any $n \geq 2$, $T \geq 1$, and $r \in (0,1)$. Let $\mathcal{A}$ be a (possibly randomized) online algorithm. There exists an oblivious adversary that can select a sequence of $T$ items such that the allocation $A^T$ constructed by $\mathcal{A}$ has $\envy^T \in \Omega_n((\log(T))^{r/2})$ with probability at least $1/T$.
\end{theorem}
\begin{proof}
    By Yao's minimax principle, we can, without loss of generality, focus on deterministic algorithms $\mathcal{A}$ and an adversary that selects distributions over instances. We will construct a distribution $\mathcal{D}$ over instances, such that any deterministic algorithm $\mathcal{A}$ has $\envy^T \in \Omega_n((\log(T))^{r/2})$ with probability at least $1/T$, where the randomness is over instances drawn from $\mathcal{D}$. $\mathcal{D}$ is defined as follows: for a fixed $T$, consider the set of instances $S_{\log{T}}$ described in~\Cref{theorem:adaptive-lb}, and select an instance uniformly at random from this set. This gives us a sequence of $\log{T}$ items; to get to $T$ items, include $T - \log{T}$ items zero value for all the agents. Note that, by definition, $S_{\log{T}}$ contains an instance $I^*$ for which algorithm $\mathcal{A}$ incurs an maximum envy of $\Omega_n((\log(T))^{r/2})$ at time $\log(T)$, and therefore at time $T$ as well, since all items after step $\log{T}$ have zero value. $\mathcal{D}$ samples $I^*$ with probability exactly $1/|S_{\log{T}}|$, which is at least $1/2^{\log{T}} = 1/T$, by~\Cref{theorem:adaptive-lb}.
\end{proof}


\section{Performance Against an i.i.d.\@ Adversary}\label{sec: iid}

In this section, we study an i.i.d.\@ adversary. In this model, we show that online envy minimization is easier than online multicolor discrepancy.
We first prove a super-constant lower bound for the online vector balancing problem (\Cref{thm: lower bound for iid vector balancing}), which, naturally, implies a super-constant lower bound for the online multicolor discrepancy problem. In~\Cref{subsec: iid envy n agents} we give a simple algorithm for online envy minimization and $n$ agents. All missing proofs can be found in~\Cref{app:missing proofs from iid}.

\subsection{Lower bounds for online vector balancing}

In the following lower bound, we show that if for all $t \in [T]$, each coordinate of all the vectors $v_t$ are i.i.d.\@ drawn from the distribution $\mathcal{U}([-1,1])$, then the discrepancy at time $T$ of any online algorithm must be $\Omega\left(\sqrt{\frac{\log{T}}{\log{\log{T}}}}\right)$. Note that a drawn vector might not be a member of $\mathcal{B}_2^d$. However, the same lower bound will hold up to a factor of $\sqrt{d}$ if each coordinate is drawn from $\mathcal{U}([-1/\sqrt{d},1/\sqrt{d}])$ which ensures $v_t \in \mathcal{B}_2^d$; we use $\mathcal{U}([-1,1])$ for the ease of exposition.

% \alex{Can we make it $v_t \in \Ball^d_2$ like the other results?}\paritosh{ We have had this conversation before. The reason of \emph{not} going for $\mathcal{B}_2^d$ is that it introduces correlation between the various coordinates, whereas, our iid positive result that follows is for iid wrt items and iid wrt agents, and hence, to present a valid comparison (aka separation) we need to have independence between the coordinates. This leads to the choice of the distribution $\mathcal{U}([-1,1])^d$, as defined below.} \daniel{Could we just scale it down by $\sqrt{d}$ so they all are in the ball?} \paritosh{We can, but why is this scaling needed? Using $\mathcal{U}([1, 1])$ or $v_{i,j} \sim \mathcal{U}([1/\sqrt{d}, 1/\sqrt{d}])$ will not affect the final conclusion, however, $\mathcal{U}([1, 1])$ looks cleaner.}
% \alex{$v_t \in \Ball^d_2$ and $v_{i,j} \sim \mathcal{U}([-1/\sqrt{d}, 1/\sqrt{d}])$ would raise fewer eyebrows. I'm scared that with the $\mathcal{U}([-1, 1])$ reviewer 2 will think that we're comparing apples to oranges}\paritosh{My point is the following: this negative result is to complement the iid result that follows, and also is for the same setting as the iid positive result. Just like the choice of $\mathcal{D}$ being supported on $[0,1]$ or $[0,B]$ doesn't matter in the iid positive result, similarly, here it doesn't matter. Added a line to explain this point above. Additionally, the choice of $\mathcal{U}([-1/\sqrt{d}, 1/\sqrt{d}])$ will make the argument messy, because we need to consider an inscribed ball within the hypercube we define.} \daniel{Why don't we just have a comment after the theorem that, although this isn't in the ball, we could easiliy scale down by $\sqrt{d}$ and then they would all be in the ball and the bound is only affected by $\sqrt{d}$ so the lower bound still holds.}\paritosh{I added this clarification at the top (before this comment thread). Please take a read and make any necessary edits.}

\begin{theorem}\label{thm: lower bound for iid vector balancing}
    Even for $n=2$ colors, for any $T \in \mathbb{N}$, any online algorithm $\mathcal{A}$, and any $d > 2$, when $\mathcal{A}$ is presented with a sequence of vectors $v_1, \dots, v_T \in \mathbb{R}^d$, where $v_{t,i} \sim \mathcal{U}([-1,1])$ in an i.i.d.\@ fashion, the discrepancy of $\mathcal{A}$ is $\Omega\left(\sqrt{\frac{\log{T}}{\log{\log{T}}}}\right)$, with probability at least $1 - 1/T^{\Theta(1)}$. %\alex{dfn whp}
\end{theorem}
\begin{proof}
    Let $\dist = \mathcal{U}([-1,1])$. We use the notation $v \sim \dist^d$ to denote a random vector $v \in \mathbb{R}^d$ each of whose coordinates are drawn independently from $\dist$. 
    The key observation is that, with sufficiently high probability, there is a long enough sequence of input vectors that are orthogonal to the current discrepancy vector; this leads to a large discrepancy at the end of this sequence. The following claim will be used to formalize this idea. 

    \begin{claim}\label{claim:hypercube_prob}
        There exists a constant $c>0$ such that, for all $\delta > 0$, and $u \in \mathbb{R}^d$, we have $\Pr_{v \sim \dist^d}[|\langle v, u\rangle| \leq \delta \norm{u}_2 \text{ and } \norm{v}_2 \in [1/2,1]] \geq c \delta$, where the constant $c$ depends on $d$. 
    \end{claim}
    \begin{proof} We can rewrite the probability of interest as
    \begin{align*}
        & \phantom{{}={}} \Pr_{v \sim \dist^d}[|\langle v, u\rangle| \leq \delta \norm{u}_2 \text{ and } \norm{v}_2 \in [1/2,1]]\\
        & = \Pr_v[\norm{v}_2 \in [1/2,1]] \cdot \Pr_v[|\langle v, u\rangle| \leq \delta \norm{u}_2 \mid \norm{v}_2 \in [1/2,1]] \numberthis \label{equation:obv-iid-lower-bound}
    \end{align*}

    We will show that $\Pr_v[\norm{v}_2 \in [1/2,1]] \geq c_1$ and $\Pr_v[|\langle v, u\rangle| \leq \delta \norm{u}_2 \mid \norm{v}_2 \in [1/2,1]] \geq c_2 \delta$ where $c_1$ and $c_2$ are constants that depends on $d$. These two inequalities, along with \Cref{equation:obv-iid-lower-bound}, imply that $\Pr_{v \sim \dist^d}[|\langle v, u\rangle| \leq \delta \norm{u}_2 \text{ and } \norm{v}_2 \in [1/2,1]] \geq c_1c_2 \delta = c \delta$, where $c = c_1 c_2$.

    To prove that $\Pr_v[\norm{v}_2 \in [1/2,1]] \geq c_1$ we use the fact that the volume of the unit Euclidean ball is given by $\textrm{vol}(\mathcal{B}^d_2) = \frac{\pi^{d/2}}{\Gamma(d/2+1)}$ where $\Gamma$ represents the gamma function~\cite{smith1989small}:
    $\Pr[\norm{v}_2 \in [1/2,1]] = \frac{\textrm{vol}(\mathcal{B}^d_2) - \textrm{vol}(\mathcal{B}^d_2)/2^d}{2^d} \geq \frac{\textrm{vol}(\mathcal{B}^d_2)}{2^{d+1}} = c_1,$
    where $c_1$ only depends on $d$. 
    
    It remains to prove that $\Pr_v[|\langle v, u\rangle| \leq \delta \norm{u}_2 \mid \norm{v}_2 \in [1/2,1]] \geq c_2 \delta$ for a constant $c_2$ that depends only on $d$. Conditioning on the event $\norm{u}_2 \in [1/2,1]$, the distribution of the random vector $v \sim \mathcal{D}^d$ is centrally symmetric, i.e., the probability density of $v$ only depends on $\norm{v}_2$ and not the direction of $v$. Define $\theta$ to be the random angle between $u$ and $v$. All possible angles $\theta \in [0, 2\pi]$ that $u$ can make with $v \sim \dist^d$ are equally likely. Using this fact, we get
    %For a vector $v \in [0,1]^d$ whose coordinates are drawn independently from $\mathcal{U}([0,1])$, we have that 
        \begin{align*}
 &\phantom{{}={}}\Pr_{v \sim \dist^d}[|\langle v, u\rangle| \leq \delta \norm{u}_2 \mid \norm{v}_2 \in [1/2,1]] \\
 & =  \Pr_{v \sim \dist^d}[ | \sqrt{d} \cos{\theta} | \leq \delta  \mid \norm{v}_2 \in [1/2,1]]\\
  & =  \Pr_{v \sim \dist^d}[ | \cos{\theta} | \leq \frac{\delta}{\sqrt{d}}  \mid \norm{v}_2 \in [1/2,1]]\\
  & =  \frac{(\pi/2 - \arccos(\delta/\sqrt{d}))}{\pi/2} \tag{$\theta \in [0, 2\pi]$ is uniformly distributed} \\
  & \geq 1 - \frac{\arccos(\delta/\sqrt{d})}{\pi/2} = \frac{2}{\pi\sqrt{d}} \cdot \delta,
        \end{align*}
        the penultimate inequality here follows from the Taylor expansion of $\arccos$, which implies that $\arccos(x) \leq \pi/2 - x$ for $x \geq 0$. Setting $c_2 = \frac{2}{\pi\sqrt{d}}$ completes the proof of the claim.
    \end{proof}
    Denote by $d_t \coloneqq \sum_{i=1}^t \chi_i v_i$, where $\chi_i \in \{ -1, 1 \}$ is the sign the algorithm picks, the discrepancy at time $t$. We know that, $\norm{d_t}_2^2 \geq \norm{d_{t-1}}_2^2 + \norm{v_t}_2^2 - 2|\langle d_{t-1}, v_t \rangle|$. For the case when $\norm{d_{t-1}}_2 \leq \frac{1}{8 \delta}$, from \Cref{claim:hypercube_prob}, we have that, with probability at least $c \delta$, $|\langle d_{t-1}, v_t\rangle| \leq 1/8$ and $\norm{v_t}_2 \in [1/2,1]$. Both these events imply that $\norm{d_t}_2^2 \geq \norm{d_{t-1}}_2^2 + 1/2 - 2\cdot 1/8 = \norm{d_{t-1}}_2^2 + 1/4.$
    
    We now divide the time horizon from $1,\ldots, T$ into $T/\tau$ contiguous chunks having $\tau$ timesteps each. Consider a contiguous chunk spanning timesteps $t_s, \ldots, t_e$ where $t_e - t_s = \tau$. Note that with probability at least $(c\delta)^\tau$ all the incoming vectors in this chunk will satisfy the condition in \Cref{claim:hypercube_prob}, thereby implying that $\norm{d_{t_e}}_2^2 - \norm{d_{t_s}}_2^2 \geq \tau/4$, which in turn will imply that $\norm{d_{t_e}}_2 \geq \sqrt{\tau/4}$.
    
    We now set $\delta = 1/(c \log{T})$ and $\tau = \log{T}/(2\log{\log{T}})$. Either at some point we have $\norm{d_{t-1}}_2 > \frac{1}{8 \delta} = (c\log{T})/8$, in which case the lower bound holds. Otherwise $\norm{d_{t-1}}_2 < \frac{1}{8 \delta}$ for all the timesteps, and with probability at least $1- \left(1 - (c\delta)^\tau\right)^{T/\tau} = 1- \left(1 - (1/\log{T})^{\log{T}/(2\log{\log{T}})}\right)^{(2T\log{\log{T}})/\log{T}} = 1- 1/T^{\Theta(1)}$ at least one of the chunks will have all its vectors almost orthogonal to the current discrepancy vector (i.e., all vectors will satisfy the condition in \Cref{claim:hypercube_prob}), leading to a discrepancy of at least $\sqrt{\tau/4} = O\left(\sqrt{\frac{\log{T}}{\log{\log{T}}}}\right)$. This concludes the proof of~\Cref{thm: lower bound for iid vector balancing}.
\end{proof}

% \subsection{Online Envy Minimization Warm-Up: The Case of Two Agents}\label{subsec: iid envy two agents}

% In this section, we give an algorithm for online envy minimization, against an i.i.d.\@ adversary, for the case of $n=2$ agents. We extend this algorithm to the $n$ agent case in the next section.

% \begin{algorithm}[t]
% \caption{Two-Phase Envy Minimization Algorithm for Two Agents}\label{algo:online envy for two agents}
% \SetAlgoLined
% \DontPrintSemicolon

% Set $T^{(1)} \gets T - \lceil \log T \sqrt{T} \rceil$, and $T^{(2)} \gets \lceil \log T \sqrt{T} \rceil$\;
% Run welfare maximization (i.e., allocate item $t$ to $\argmax_{i \in [n]} v_{i,t}$ breaking ties randomly)  for $T^{(1)}$ steps \;
% \For{$t \gets T^{(1)} + 1$ \KwTo $T$}{
%     Allocate item $t$ to the most envious agent, i.e., $\argmax_{i \in [n]} \max_{j \in [n]} \envy^t_{i,j}$. \;
% }
% \end{algorithm}

% \begin{theorem}\label{thm:two agent upper bound iid fair division}
% Algorithm~\ref{algo:online envy for two agents} has envy at most $c + 1$ with probability at least $1 - O(1 / T^{c/2})$, for all integers $c \ge 1$. 
% \end{theorem}

% \begin{proof}
% Fix a value distribution $\dist$. For two independent random variables $X, Y \sim \dist$, let $\dist^{gap}$ and $\dist^{|gap|}$ be the distribution of $X - Y$ and $|X - Y|$, respectively (i.e., $\dist^{gap}$ the difference of two independent draws, and  $\dist^{|gap|}$ is the absolute difference of two independent draws).  The following lemma implies that, with high probability, Algorithm~\ref{algo:online envy for two agents} cannot have heavy envy cycles.
% \begin{lemma}\label{lem:envy-cycle}
%     Fix a positive integer $c$. For $T \geq C_0$, for a universal constant $C_0$ (independent of $\dist$), with probability at least $1 - 2 \left(\frac{2eT^{(2)}}{T} \right)^c$, it holds that $\envy^t_{1,2} + \envy^t_{2,1} \le 2c$ for all $t$.
% \end{lemma}

% \begin{proof}[Proof of \Cref{lem:envy-cycle}]
% Fix $i \ne j$. We will first prove that $v_i(A_i^t) + c \ge v_j(A_i^t)$ for all $t$, with probability at least $1 - 2 \left(\frac{2eT^{(2)}}{T} \right)^c$. Without loss of generality, consider agents $i = 1$ and $j = 2$.  Let $S_t := v_1(A_1^t) - v_2(A_1^t)$ be the difference in values at time $t$. Our goal is to show that $S_t \ge -c$ with high probability at every step. Let $X_t$ be contribution of good $g_t$ to $S_t$, i.e., \[
% X_t = \begin{cases}
%     v_{1,t} - v_{2,t} & \text{ if } g_t \in A_1^t\\
%     0 & \text{ otherwise.}
% \end{cases}
% \]

% It will be convenient to think of $X_t$ as a draw from $W_t \cdot I_t$, where $W_t \sim D^{gap}$, and $I_t$ is an indicator for the event that agent $1$ gets the item. Importantly, $W_t$ and $I_t$ are independent. 

% We can write $S_t = \sum_{\ell = 1}^t X_{\ell}$. Let $\dist^{gap0}$ be the distribution of $\max(X, 0)$ for $X \sim \dist^{gap}$, i.e., the distribution which takes a value of zero half the time, and samples from $|\dist^{gap}|$ half the time ($\dist^{gap}$ is a symmetric distribution).
% For $t \le T^{(1)}$, $X_t \sim \dist^{gap0}$, independent of all previous timesteps, and therefore $X_t \ge 0$ with probability $1$. Thus, for all $t \le T^{(1)}$, $S_t \ge 0$ with probability $1$.

% For $t > T^{(1)}$, $X_t$ depends on the history. There is some probability that $g_t$ is given to agent $i=1$, in which case $X_t \sim \dist^{gap}$, and otherwise it is not, in which case $X_t = 0$. We will couple this with an alternate draw $X'_t$ sampled independently from $-\dist^{gap0} = \min(W_t, 0)$. Clearly, $X_t$ (drawn from $W_t \cdot I_t$) first-order stochastically dominates $X'_t$ (drawn from $\min(W_t, 0)$). It also holds that $S_t = \sum_{\ell = 1}^t X_{\ell}$ first-order stochastically dominates $S'_t = S_{T^{(1)}} + \sum_{t' = T^{(1)}}^{t} X'_{t'}$. Therefore, it suffices to show that $S'_t \ge -c$ for all $t > T^{(1)}$, with probability at least $1 - 2 \left(\frac{2eT^{(2)}}{T} \right)^c$.
% Next, note that $X'_t \le 0$ for all $t > T^{(1)}$. Thus, $S'_t$ is nonincreasing for such $t$, and it is sufficient to show that $S'_T \geq -c$ with probability at least $1 - 2 \left(\frac{2eT^{(2)}}{T} \right)^c$. 
% $S'_T$ is the sum of $T^{(1)}$ independent draws from $\dist^{gap0}$ and $T^{(2)}$ independent draws from $-\dist^{gap0}$. From \Cref{lem:concentration}, plugging in $K = T$ and $L = T^{(2)}$ (as long as $T^{(2)} = \lceil \log T \sqrt{T} \rceil < T/8e$), we have that $v_i(A_i^t) + c \ge v_j(A_i^t)$ for all $t$.
% Conditioned on this event, we have that, $\envy^t_{1,2} + \envy^t_{2,1} = (v_1(A_2^t) - v_1(A_1^t)) + (v_2(A_1^t) - v_2(A_2^t)) = (v_2(A_1^t) - v_1(A_1^t)) + (v_1(A_2^t) - v_2(A_2^t)) \le 2c$, as needed.
% \end{proof}


% The next lemma shows that, with high probability, a variant of Algorithm~\ref{algo:online envy for two agents} that allocates the last $T^{(2)}$ items to a single agent, guarantees small envy to that agent.

% \begin{lemma}\label{lem:high-value}For all $c > 1$ and $T^{(2)} \ge \log T \sqrt{T}$, the algorithm that runs welfare maximization for $T^{(1)} = T - T^{(2)}$ and then allocates all remaining $T^{(2)}$ items to a single agent must leave that agent with envy as most $c$ with probability at least $1 - O(1/T^{c})$. 
% \end{lemma}

% Finally, the next lemma shows that when there are no heavy envy cycles and the aforementioned variant of Algorithm~\ref{algo:online envy for two agents} succeeds (guarantees small envy to the agent that got the last $T^{(2)}$ items), then Algorithm~\ref{algo:online envy for two agents} guarantees small envy to both agents.

% \begin{lemma}\label{lem:small envy for two agents}
%     Conditioned on the events of \Cref{lem:envy-cycle,lem:high-value}, 
%     $\envy^T_{ij} \le c + 1$ for all $i, j$.
% \end{lemma}
% \begin{proof}
% First, we will prove that there exists a time $t^* \geq T^{(1)}$ such that $\envy^t_{ij} \le c + 1$ for all $i, j$. If not, $\envy^t_{ij} > c$ for all $t$. We've conditioned on $\envy^t_{12} + \envy^t_{21} \le 2c$ for all $t$ (\Cref{lem:envy-cycle}), therefore $\envy^t_{ji} \leq c$, and therefore $i$ must have been the most envious agent in all final $T^{(2)}$ steps. Therefore, $i$ received the last $T^{(2)}$ items. By~\Cref{lem:high-value}, $\envy^t_{ij} \leq c$; a contradiction.

% Second, we will prove that $\envy^{t'}_{ij} \le c + 1$ for all $i,j$ and all $t' \ge  t^*$, which implies the lemma. As argued earlier, in the last $T^{(2)}$ steps of the algorithm the envy of an agent can increase only if they are \emph{not} the most envious agent. We've conditioned on $\envy^t_{12} + \envy^t_{21} \le 2c$ for all $t$ (\Cref{lem:envy-cycle}), therefore, the envy of such an agent is at most $c$, and it can increase to at most $c+1$. Therefore, starting at step $t^*$, no agent can have envy strictly greater than $c+1$.
% \end{proof}

% The events of~\Cref{lem:envy-cycle,lem:high-value} occur with probability at least $1 - O(1/T^{c})$; therefore,~\Cref{lem:small envy for two agents} implies the theorem.
% \end{proof}

\subsection{Online envy minimization}\label{subsec: iid envy n agents}

In this section, we give an algorithm,    ~\Cref{algo:online envy for n agents}, for online envy minimization, against an i.i.d.\@ adversary. \Cref{algo:online envy for n agents} works in two phases. In phase 1, which lasts $T^{(1)}$ steps, it makes allocations using the welfare maximization algorithm (``item $j$ is allocated to the agent with the largest value''). In Phase 2, at every step $t$ the algorithm singles out the set of agents who have not received a large number of items (within phase 2, up until $t$); among this set, it allocates item $t$ to the agent who is envied the least by agents in this set.


\begin{theorem}\label{thm:n agent upper bound iid fair division}
For all positive integers $c$, \Cref{algo:online envy for n agents} has envy at most $c + 1$ with probability at least $1 - O(T^{-c/2})$.
\end{theorem}

Note that the $O(\cdot)$ hides constants that depend on the number of agents, but is independent of the value distribution.

\begin{proof}
    Fix a positive integer $c$, an arbitrary distribution $\calD$ supported on $[0, 1]$, and a time horizon $T$. Throughout, we assume that $T$ is sufficiently large, i.e., larger than some number $T_0$ that depends only on $n$ and $c$, not on the distribution $\calD$. Let $F$ denote the CDF of $\calD$. 

    

    A key observation is that we can analyze the algorithm using an equivalent, but more structured method of sampling item values. Normally, at each time step $t$, the item values revealed to the algorithm are sampled i.i.d.\@ from $\mathcal{D}$, independent of every decision made so far. Instead, we define an equivalent experiment as follows. Let $G^{welf} = \{g^{welf}_1, \ldots, g^{welf}_{T^{(1)}}\}$ be a set of $T^{(1)}$ goods and, for each agent $i \in [n]$,  let $G^{i} = \{g^i_1, \ldots, g^i_{T^{(2)}}\}$ be a set of $T^{(2)}$ goods.
    \begin{enumerate}[leftmargin=*]
        \item Before the algorithm begins, nature samples values $(V^g_1, \ldots, V^g_n)$ for each $g \in G^{welf} \cup \bigcup_i G^i$, where each $V^g_i \stackrel{i.i.d.}{\sim} \calD$. 
        \item During \emph{Phase 1} of the algorithm (welfare maximization), when the $t^{th}$ item arrives, it is revealed to be  item $g^{welf}_t$, with pre-sampled values $(V^{g^{welf}_t}_1, \ldots, V^{g^{welf}_t}_n)$.
        \item During \emph{Phase 2} (lines 3-7 in~\Cref{algo:online envy for n agents}), suppose item $t$ will be assigned to agent $i$ who, at this point, has received $k$ items during phase $2$ ($|A^t_i \setminus G^{welf}| = k$). Then, item $t$ is revealed to be $g^i_{k + 1}$ with pre-sampled values $(V^{g^i_{k + 1}}_1, \ldots, V^{g^i_{k + 1}}_n)$.
    \end{enumerate}
    % \alex{why is this something important to note right here?}\daniel{happy to delete} Note that in the final allocation, $A^T_i \subseteq G^{welf} \cup G^i$, and furthermore, $A^T_i \cap G^i = \set{g^i_1, \ldots, g^i_k}$ for some $k$.

    \begin{algorithm}[t]
\caption{Two-Phase Envy Minimization Algorithm}\label{algo:online envy for n agents}
\SetAlgoLined
\DontPrintSemicolon

Set $T^{(1)} \gets T - \frac{n(n - 1)}{2}\lceil \log T \sqrt{T} \rceil$, and $T^{(2)} \gets \frac{n(n - 1)}{2} \lceil \log T \sqrt{T} \rceil$\;
Run welfare maximization (i.e., allocate item $t$ to $\argmax_{i \in [n]} v_{i,t}$ breaking ties randomly)  for $T^{(1)}$ steps \;
\For{$t \gets T^{(1)} + 1$ \KwTo $T$}{
    Let $w^t_i$ be the number of items agent $i$ has received in steps $t' >T^{(1)}$.\;
    Let $S$ be the smallest (in terms of cardinality) subset of agents, such that $\forall i \in S, j \notin S$ $w^t_i \leq w^t_j - \lceil \log T \sqrt{T} \rceil$. \;
    Allocate item $t$ to an agent $i \in S$ who is envied the least, i.e., $\argmin_{i \in S} \max_{j \in S} \envy^t_{j,i}$. \;
}
\end{algorithm}

    Importantly, the allocation decision for item $t$ does not depend on agents' values for this item. This ensures that the value vector $(V^{g^i_{k + 1}}_1, \ldots, V^{g^i_{k + 1}}_n)$ is independent of all decisions made by the algorithm. Consequently, this modified experiment is statistically identical to the original setup in terms of the envy of the final allocation.

    A second useful modification is to work with item \emph{quantiles} instead of item values. More formally, instead of directly sampling $V^g_i$, we will first sample a quantile $Q^g_i \sim \mathcal{U}[0, 1]$ and then set $V^g_i = F^{-1}(Q^g_i)$ where $F^{-1}$ is the generalized inverse of $F$. Throughout the remainder of this proof, we condition on the probability $1$ event that all $Q^g_i$s are distinct.
    Note that for $g \in G^{welf}$, allocating item $g$ to an agent with the highest quantile, $i \in \argmax_{j} Q^g_j$, is equivalent to welfare maximization with random tie-breaking.\footnote{We make this point, since unequal quantiles does not imply unequal values.} Thus, we will assume these are coupled. Since all quantiles are distinct by assumption, ties never occur, and this allocation is always well-defined. 

    \textbf{No heavy envy-cycles.} Our first high-level step will be to show that, with high probability, no envy cycles with large weight exist during the execution of the algorithm.

    \begin{lemma}\label{lem:no-cycle}
        With probability $1 - O ( T^{-c/2}  )$, at every time $t \ge T^{(1)}$, there does not exist a cycle of agents $i_1, \ldots, i_k, i_{k+1}=i_1$ such that $\envy_{i_j, i_{j + 1}} > c$ for all $j = 1, \ldots, k$. 
    \end{lemma}

    The proof of~\Cref{lem:no-cycle} crucially relies on the following concentration inequality (which, to the best of our knowledge, is not known), that might be of independent interest.

    \begin{lemma}\label{lem:concentration}
    Fix positive integers $L, K$, and $c$, with $L < \frac{K}{4e}$. Let $Y_1, \ldots, Y_K$ be i.i.d.\@ draws from a distribution supported on $[0, 1]$. Then, $\Pr\left[\sum_{i \le K - L} Y_i - \sum_{i > K - L} Y_i < -c \right] \leq 4 \cdot \left(\frac{2 e L}{K} \right)^{c + 1}$.
    \end{lemma}

    The proof of~\Cref{lem:no-cycle}  also relies on two (relatively more straightforward) facts,~\Cref{lem:phase-2-items,lem:halls}. 
    The first lemma shows that the items allocated in phase 2 are relatively balanced among the agents, up to additive $\ceil{\log T \sqrt{T}}$ factors.
    
    
    %the way the algorithm is defined, there is a lot of structure in the number of items agents receive during phase $2$. 
    \begin{lemma}\label{lem:phase-2-items}
        Fix a time $t$. Let $w^t_i|$ be the number of items agent $i$ has received in phase 2, i.e., $w^t_i = |A^T_i \setminus G^{welf}|$. Let $(w^t_{i_1}, \ldots, w^t_{i_n})$ be these numbers sorted from smallest to largest; so, $w^t_{i_j} \le w^t_{i_{j + 1}}$. Then, for all $j \le n - 1$, $w^t_{i_{j + 1}} \le w^t_{i_j} + \ceil{\log T \sqrt{T}}$. Furthermore, $w^t_{i_n} \le (n - 1) \ceil{\log T \sqrt{T}}$, i.e., no agent ever receives more than $(n - 1) \ceil{\log T \sqrt{T}}$ phase 2 items. 
    \end{lemma}
    The second lemma is a sufficient condition for bounding the envy between two sets of values and is reminiscent of approximate \emph{stochastic-dominance envy-freeness} (SD-EF). The proof is based on a generalization of Hall's theorem.
    \begin{lemma}\label{lem:halls}
        Given two sequences of values $a_1, \ldots, a_k$ and $b_1, \ldots, b_{\ell}$ where each $a_i, b_i \in [0, 1]$. Suppose that, for each $a_i$, $|\set{i' | a_{i'} \ge a_i}| \le |\set{i' | b_{i'} \ge a_i}| + c$. Then, $\sum_i a_i \le \sum_i b_i + c$. 
    \end{lemma}

    \textbf{Long phase 2 eliminates envy.} Our second high-level step will be to show that if phase 2 is sufficiently long, then with high probability, envy can be eliminated.

    % \begin{lemma}\label{lem:high-value-n} Fix any two agents $i, j \in \agents$ and an integer $L \le (n - 2) \ceil{\log T \sqrt{T}}$. Let $A_j = A^{T^{(1)}}_j \cup \{g^j_1, \ldots, g^j_L\}$ and  $A_i = A^{T^{(1)}}_i \cup \{g^i_1, \ldots,  g^i_{L + \ceil{\log T \sqrt{T}}}\}$. That is, suppose we run phase 1 as usual, then in phase 2, give $L$ phase 2 items to agent $j$, and $L + \ceil{\log T \sqrt{T}}$ phase 2 items to agent $i$. Then, $\envy_{ij} \le c$ with probability \alex{TODO}.   
    % \end{lemma}

    \begin{lemma}\label{lem:high-value-n} With probability $1 - O(T^{-c/2})$ it is the case that for all agents $i, j \in \agents$ and all time steps $t \ge T^{(1)}$, if $|A^t_i \setminus G^{welf}| \ge |A^t_j \setminus G^{welf}| + \ceil{\log T \sqrt{T}}$, then $\envy^t_{i, j} \le c$. 
    \end{lemma}


    \textbf{Putting it all together.}
    With~\Cref{lem:no-cycle,lem:high-value-n} in hand, we are ready to prove the theorem.
    
    Let $H^t$ be a graph with nodes $[n]$ where there is an edge $(i, j)$ if $\envy^t_{i, j} > c$. Condition on the events in \Cref{lem:no-cycle} and \Cref{lem:high-value-n}. %\daniel{for all L maybe if we don't have this in the lemma statement}. 
    These happen with probability $1 - O(T^{-c/2})$. Then,~\Cref{lem:no-cycle} ensures that $H^t$ is acyclic for all $t$, while~\Cref{lem:high-value-n} ensures that if $|A^t_i \setminus G^{welf}| \le  |A^t_j \setminus G^{welf}| + \ceil{\log T \sqrt{T}}$, then $(j, i) \notin H^t$.%\alex{need to confirm this part} \daniel{I think any explanation of this should go inside of the lemma, so we can change the statement to match what is written here. High level though, the argument  is that if agent $i$ has received $L^i$ items and $j$ has received $L^j$ with $L^j \ge L^i + \ceil{\log T \sqrt{T}}$, then the envy at most as much had $j$ received only $ L^i + \ceil{\log T \sqrt{T}}$ exactly (i.e., we remove their final $L^j - (L^i + \ceil{\log T \sqrt{T}})$). The lemma guarantees that $j$ doesn't envy $i$ by more than $c$ in this case. } \alex{so, shouldn't it be $(j,i) \notin H^t$?}


    First, we prove that if an agent $i$ received an item at some point during phase 2, then $\envy^T_{j, i} \le c + 1$, for all $j \in \agents$. To this end, suppose that $i$ does indeed receive an item at some point during phase 2, and let $t$ be the \emph{last} time step for which $i$ received an item. 

    We first claim that $\envy^{t - 1}_{j, i} \le c$ for all $j \ne i$, i.e., $i$ is a source node in $H^{t - 1}$. Let $S$ be the set defined in~\Cref{algo:online envy for n agents} for time step $t$, i.e., for all $j \in S$ and $j' \notin S$, $j$ has received at least $\ceil{\log T \sqrt{T}}$ fewer items in phase 2, or  $|A^{t - 1}_j \setminus G^{welf}| \le |A^{t - 1}_{j'} \setminus G^{welf}| - \ceil{\log T \sqrt{T}}$. Note that $i \in S$ as they received item $t$, and for all $j' \notin S$, $(j', i) \notin H^{t - 1}$, by~\Cref{lem:high-value-n}. Then, consider $H^{t - 1}[S]$, the subgraph of $H^{t - 1}$ that only containing the nodes $S$. Note that $H^{t - 1}[S]$ is acyclic, since $H^{t - 1}$ is acyclic. Furthermore, by definition, source nodes in $H^{t - 1}[S]$ are envied by $\le c$ by all agents in $S$, while non-source nodes are envied by $>c$ by at least one agent in $S$. Hence, $i$ must be a source node in $H^{t - 1}[S]$. Together with the fact that there are no $(j, i)$ edges for $j \notin S$, we have that $i$ is a source node in $H^{t - 1}$.

    Now, since $\envy^{t - 1}_{j, i} \le c$, giving an item to $i$ can increase envy by at most $1$. Hence, $\envy^{t}_{j, i} \le c + 1$. Furthermore, as $i$ never received any more items after this time ($t$ was defined as the last item $i$ received), envy toward $i$ cannot increase. Hence, $\envy^T_{j, i} \le c + 1$, as needed. 

    Finally, let $w^t_i = |A^T_i \setminus G^{welf}|$ be the number of items allocated to agent $i$ in phase $2$. Note that if $w^t_i > 0$, by our previous argument, $\envy_{j, i} \le c + 1$ for all $j \ne i$. So, if $w^t_i > 0$ for all $i \in \agents$, we are done. Suppose this is not the case. So, there exists an agent $i$ such that  $w^t_i = 0$. Let $i_1, \ldots, i_n$ be an ordering of the agents sorted by $w^t_i$, i.e., $w^t_{i_1} \le \cdots \le w^t_{i_n}$. The assumption that $w^t_i = 0$ implies that $w^t_{i_1} = 0$. We claim that $w^t_{i_2} \ge \ceil{\log T \sqrt{T}}$. Indeed, by induction,~\Cref{lem:phase-2-items} ensures that for all $j \ge 2$, $w^t_{i_j} \le w^t_{i_2} + (j - 2) \ceil{\log T \sqrt{T}}$. Hence, 
    $\sum_{j=1}^n w^t_{i_j} \le (n - 1) \cdot w^t_{i_2} + \frac{(n - 2)(n - 1)}{2} \, \ceil{\log T \sqrt{T}}$. 
    However, since this is time $T$,  $\sum_j w^t_{i_j} = T^{(2)} = \frac{n(n - 1)}{2} \cdot \ceil{\log T \sqrt{T}}$. Together, these imply that $w^t_{i_2} \geq \ceil{\log T \sqrt{T}}$. Therefore, all agents other than $i_1$ received at least one item during phase 2, and hence are not envied by more than $c + 1$. On the other hand, all agents $j \ne i_1$ received at least $\ceil{\log T \sqrt{T}}$ items more than $i_1$ in phase $2$, and therefore, $\envy_{j, {i_1}} \le c \le c + 1$ as needed.
\end{proof}


% Bibliography
\bibliographystyle{plainnat}
\bibliography{abb,references}


\newpage
\appendix
\subsection{Lloyd-Max Algorithm}
\label{subsec:Lloyd-Max}
For a given quantization bitwidth $B$ and an operand $\bm{X}$, the Lloyd-Max algorithm finds $2^B$ quantization levels $\{\hat{x}_i\}_{i=1}^{2^B}$ such that quantizing $\bm{X}$ by rounding each scalar in $\bm{X}$ to the nearest quantization level minimizes the quantization MSE. 

The algorithm starts with an initial guess of quantization levels and then iteratively computes quantization thresholds $\{\tau_i\}_{i=1}^{2^B-1}$ and updates quantization levels $\{\hat{x}_i\}_{i=1}^{2^B}$. Specifically, at iteration $n$, thresholds are set to the midpoints of the previous iteration's levels:
\begin{align*}
    \tau_i^{(n)}=\frac{\hat{x}_i^{(n-1)}+\hat{x}_{i+1}^{(n-1)}}2 \text{ for } i=1\ldots 2^B-1
\end{align*}
Subsequently, the quantization levels are re-computed as conditional means of the data regions defined by the new thresholds:
\begin{align*}
    \hat{x}_i^{(n)}=\mathbb{E}\left[ \bm{X} \big| \bm{X}\in [\tau_{i-1}^{(n)},\tau_i^{(n)}] \right] \text{ for } i=1\ldots 2^B
\end{align*}
where to satisfy boundary conditions we have $\tau_0=-\infty$ and $\tau_{2^B}=\infty$. The algorithm iterates the above steps until convergence.

Figure \ref{fig:lm_quant} compares the quantization levels of a $7$-bit floating point (E3M3) quantizer (left) to a $7$-bit Lloyd-Max quantizer (right) when quantizing a layer of weights from the GPT3-126M model at a per-tensor granularity. As shown, the Lloyd-Max quantizer achieves substantially lower quantization MSE. Further, Table \ref{tab:FP7_vs_LM7} shows the superior perplexity achieved by Lloyd-Max quantizers for bitwidths of $7$, $6$ and $5$. The difference between the quantizers is clear at 5 bits, where per-tensor FP quantization incurs a drastic and unacceptable increase in perplexity, while Lloyd-Max quantization incurs a much smaller increase. Nevertheless, we note that even the optimal Lloyd-Max quantizer incurs a notable ($\sim 1.5$) increase in perplexity due to the coarse granularity of quantization. 

\begin{figure}[h]
  \centering
  \includegraphics[width=0.7\linewidth]{sections/figures/LM7_FP7.pdf}
  \caption{\small Quantization levels and the corresponding quantization MSE of Floating Point (left) vs Lloyd-Max (right) Quantizers for a layer of weights in the GPT3-126M model.}
  \label{fig:lm_quant}
\end{figure}

\begin{table}[h]\scriptsize
\begin{center}
\caption{\label{tab:FP7_vs_LM7} \small Comparing perplexity (lower is better) achieved by floating point quantizers and Lloyd-Max quantizers on a GPT3-126M model for the Wikitext-103 dataset.}
\begin{tabular}{c|cc|c}
\hline
 \multirow{2}{*}{\textbf{Bitwidth}} & \multicolumn{2}{|c|}{\textbf{Floating-Point Quantizer}} & \textbf{Lloyd-Max Quantizer} \\
 & Best Format & Wikitext-103 Perplexity & Wikitext-103 Perplexity \\
\hline
7 & E3M3 & 18.32 & 18.27 \\
6 & E3M2 & 19.07 & 18.51 \\
5 & E4M0 & 43.89 & 19.71 \\
\hline
\end{tabular}
\end{center}
\end{table}

\subsection{Proof of Local Optimality of LO-BCQ}
\label{subsec:lobcq_opt_proof}
For a given block $\bm{b}_j$, the quantization MSE during LO-BCQ can be empirically evaluated as $\frac{1}{L_b}\lVert \bm{b}_j- \bm{\hat{b}}_j\rVert^2_2$ where $\bm{\hat{b}}_j$ is computed from equation (\ref{eq:clustered_quantization_definition}) as $C_{f(\bm{b}_j)}(\bm{b}_j)$. Further, for a given block cluster $\mathcal{B}_i$, we compute the quantization MSE as $\frac{1}{|\mathcal{B}_{i}|}\sum_{\bm{b} \in \mathcal{B}_{i}} \frac{1}{L_b}\lVert \bm{b}- C_i^{(n)}(\bm{b})\rVert^2_2$. Therefore, at the end of iteration $n$, we evaluate the overall quantization MSE $J^{(n)}$ for a given operand $\bm{X}$ composed of $N_c$ block clusters as:
\begin{align*}
    \label{eq:mse_iter_n}
    J^{(n)} = \frac{1}{N_c} \sum_{i=1}^{N_c} \frac{1}{|\mathcal{B}_{i}^{(n)}|}\sum_{\bm{v} \in \mathcal{B}_{i}^{(n)}} \frac{1}{L_b}\lVert \bm{b}- B_i^{(n)}(\bm{b})\rVert^2_2
\end{align*}

At the end of iteration $n$, the codebooks are updated from $\mathcal{C}^{(n-1)}$ to $\mathcal{C}^{(n)}$. However, the mapping of a given vector $\bm{b}_j$ to quantizers $\mathcal{C}^{(n)}$ remains as  $f^{(n)}(\bm{b}_j)$. At the next iteration, during the vector clustering step, $f^{(n+1)}(\bm{b}_j)$ finds new mapping of $\bm{b}_j$ to updated codebooks $\mathcal{C}^{(n)}$ such that the quantization MSE over the candidate codebooks is minimized. Therefore, we obtain the following result for $\bm{b}_j$:
\begin{align*}
\frac{1}{L_b}\lVert \bm{b}_j - C_{f^{(n+1)}(\bm{b}_j)}^{(n)}(\bm{b}_j)\rVert^2_2 \le \frac{1}{L_b}\lVert \bm{b}_j - C_{f^{(n)}(\bm{b}_j)}^{(n)}(\bm{b}_j)\rVert^2_2
\end{align*}

That is, quantizing $\bm{b}_j$ at the end of the block clustering step of iteration $n+1$ results in lower quantization MSE compared to quantizing at the end of iteration $n$. Since this is true for all $\bm{b} \in \bm{X}$, we assert the following:
\begin{equation}
\begin{split}
\label{eq:mse_ineq_1}
    \tilde{J}^{(n+1)} &= \frac{1}{N_c} \sum_{i=1}^{N_c} \frac{1}{|\mathcal{B}_{i}^{(n+1)}|}\sum_{\bm{b} \in \mathcal{B}_{i}^{(n+1)}} \frac{1}{L_b}\lVert \bm{b} - C_i^{(n)}(b)\rVert^2_2 \le J^{(n)}
\end{split}
\end{equation}
where $\tilde{J}^{(n+1)}$ is the the quantization MSE after the vector clustering step at iteration $n+1$.

Next, during the codebook update step (\ref{eq:quantizers_update}) at iteration $n+1$, the per-cluster codebooks $\mathcal{C}^{(n)}$ are updated to $\mathcal{C}^{(n+1)}$ by invoking the Lloyd-Max algorithm \citep{Lloyd}. We know that for any given value distribution, the Lloyd-Max algorithm minimizes the quantization MSE. Therefore, for a given vector cluster $\mathcal{B}_i$ we obtain the following result:

\begin{equation}
    \frac{1}{|\mathcal{B}_{i}^{(n+1)}|}\sum_{\bm{b} \in \mathcal{B}_{i}^{(n+1)}} \frac{1}{L_b}\lVert \bm{b}- C_i^{(n+1)}(\bm{b})\rVert^2_2 \le \frac{1}{|\mathcal{B}_{i}^{(n+1)}|}\sum_{\bm{b} \in \mathcal{B}_{i}^{(n+1)}} \frac{1}{L_b}\lVert \bm{b}- C_i^{(n)}(\bm{b})\rVert^2_2
\end{equation}

The above equation states that quantizing the given block cluster $\mathcal{B}_i$ after updating the associated codebook from $C_i^{(n)}$ to $C_i^{(n+1)}$ results in lower quantization MSE. Since this is true for all the block clusters, we derive the following result: 
\begin{equation}
\begin{split}
\label{eq:mse_ineq_2}
     J^{(n+1)} &= \frac{1}{N_c} \sum_{i=1}^{N_c} \frac{1}{|\mathcal{B}_{i}^{(n+1)}|}\sum_{\bm{b} \in \mathcal{B}_{i}^{(n+1)}} \frac{1}{L_b}\lVert \bm{b}- C_i^{(n+1)}(\bm{b})\rVert^2_2  \le \tilde{J}^{(n+1)}   
\end{split}
\end{equation}

Following (\ref{eq:mse_ineq_1}) and (\ref{eq:mse_ineq_2}), we find that the quantization MSE is non-increasing for each iteration, that is, $J^{(1)} \ge J^{(2)} \ge J^{(3)} \ge \ldots \ge J^{(M)}$ where $M$ is the maximum number of iterations. 
%Therefore, we can say that if the algorithm converges, then it must be that it has converged to a local minimum. 
\hfill $\blacksquare$


\begin{figure}
    \begin{center}
    \includegraphics[width=0.5\textwidth]{sections//figures/mse_vs_iter.pdf}
    \end{center}
    \caption{\small NMSE vs iterations during LO-BCQ compared to other block quantization proposals}
    \label{fig:nmse_vs_iter}
\end{figure}

Figure \ref{fig:nmse_vs_iter} shows the empirical convergence of LO-BCQ across several block lengths and number of codebooks. Also, the MSE achieved by LO-BCQ is compared to baselines such as MXFP and VSQ. As shown, LO-BCQ converges to a lower MSE than the baselines. Further, we achieve better convergence for larger number of codebooks ($N_c$) and for a smaller block length ($L_b$), both of which increase the bitwidth of BCQ (see Eq \ref{eq:bitwidth_bcq}).


\subsection{Additional Accuracy Results}
%Table \ref{tab:lobcq_config} lists the various LOBCQ configurations and their corresponding bitwidths.
\begin{table}
\setlength{\tabcolsep}{4.75pt}
\begin{center}
\caption{\label{tab:lobcq_config} Various LO-BCQ configurations and their bitwidths.}
\begin{tabular}{|c||c|c|c|c||c|c||c|} 
\hline
 & \multicolumn{4}{|c||}{$L_b=8$} & \multicolumn{2}{|c||}{$L_b=4$} & $L_b=2$ \\
 \hline
 \backslashbox{$L_A$\kern-1em}{\kern-1em$N_c$} & 2 & 4 & 8 & 16 & 2 & 4 & 2 \\
 \hline
 64 & 4.25 & 4.375 & 4.5 & 4.625 & 4.375 & 4.625 & 4.625\\
 \hline
 32 & 4.375 & 4.5 & 4.625& 4.75 & 4.5 & 4.75 & 4.75 \\
 \hline
 16 & 4.625 & 4.75& 4.875 & 5 & 4.75 & 5 & 5 \\
 \hline
\end{tabular}
\end{center}
\end{table}

%\subsection{Perplexity achieved by various LO-BCQ configurations on Wikitext-103 dataset}

\begin{table} \centering
\begin{tabular}{|c||c|c|c|c||c|c||c|} 
\hline
 $L_b \rightarrow$& \multicolumn{4}{c||}{8} & \multicolumn{2}{c||}{4} & 2\\
 \hline
 \backslashbox{$L_A$\kern-1em}{\kern-1em$N_c$} & 2 & 4 & 8 & 16 & 2 & 4 & 2  \\
 %$N_c \rightarrow$ & 2 & 4 & 8 & 16 & 2 & 4 & 2 \\
 \hline
 \hline
 \multicolumn{8}{c}{GPT3-1.3B (FP32 PPL = 9.98)} \\ 
 \hline
 \hline
 64 & 10.40 & 10.23 & 10.17 & 10.15 &  10.28 & 10.18 & 10.19 \\
 \hline
 32 & 10.25 & 10.20 & 10.15 & 10.12 &  10.23 & 10.17 & 10.17 \\
 \hline
 16 & 10.22 & 10.16 & 10.10 & 10.09 &  10.21 & 10.14 & 10.16 \\
 \hline
  \hline
 \multicolumn{8}{c}{GPT3-8B (FP32 PPL = 7.38)} \\ 
 \hline
 \hline
 64 & 7.61 & 7.52 & 7.48 &  7.47 &  7.55 &  7.49 & 7.50 \\
 \hline
 32 & 7.52 & 7.50 & 7.46 &  7.45 &  7.52 &  7.48 & 7.48  \\
 \hline
 16 & 7.51 & 7.48 & 7.44 &  7.44 &  7.51 &  7.49 & 7.47  \\
 \hline
\end{tabular}
\caption{\label{tab:ppl_gpt3_abalation} Wikitext-103 perplexity across GPT3-1.3B and 8B models.}
\end{table}

\begin{table} \centering
\begin{tabular}{|c||c|c|c|c||} 
\hline
 $L_b \rightarrow$& \multicolumn{4}{c||}{8}\\
 \hline
 \backslashbox{$L_A$\kern-1em}{\kern-1em$N_c$} & 2 & 4 & 8 & 16 \\
 %$N_c \rightarrow$ & 2 & 4 & 8 & 16 & 2 & 4 & 2 \\
 \hline
 \hline
 \multicolumn{5}{|c|}{Llama2-7B (FP32 PPL = 5.06)} \\ 
 \hline
 \hline
 64 & 5.31 & 5.26 & 5.19 & 5.18  \\
 \hline
 32 & 5.23 & 5.25 & 5.18 & 5.15  \\
 \hline
 16 & 5.23 & 5.19 & 5.16 & 5.14  \\
 \hline
 \multicolumn{5}{|c|}{Nemotron4-15B (FP32 PPL = 5.87)} \\ 
 \hline
 \hline
 64  & 6.3 & 6.20 & 6.13 & 6.08  \\
 \hline
 32  & 6.24 & 6.12 & 6.07 & 6.03  \\
 \hline
 16  & 6.12 & 6.14 & 6.04 & 6.02  \\
 \hline
 \multicolumn{5}{|c|}{Nemotron4-340B (FP32 PPL = 3.48)} \\ 
 \hline
 \hline
 64 & 3.67 & 3.62 & 3.60 & 3.59 \\
 \hline
 32 & 3.63 & 3.61 & 3.59 & 3.56 \\
 \hline
 16 & 3.61 & 3.58 & 3.57 & 3.55 \\
 \hline
\end{tabular}
\caption{\label{tab:ppl_llama7B_nemo15B} Wikitext-103 perplexity compared to FP32 baseline in Llama2-7B and Nemotron4-15B, 340B models}
\end{table}

%\subsection{Perplexity achieved by various LO-BCQ configurations on MMLU dataset}


\begin{table} \centering
\begin{tabular}{|c||c|c|c|c||c|c|c|c|} 
\hline
 $L_b \rightarrow$& \multicolumn{4}{c||}{8} & \multicolumn{4}{c||}{8}\\
 \hline
 \backslashbox{$L_A$\kern-1em}{\kern-1em$N_c$} & 2 & 4 & 8 & 16 & 2 & 4 & 8 & 16  \\
 %$N_c \rightarrow$ & 2 & 4 & 8 & 16 & 2 & 4 & 2 \\
 \hline
 \hline
 \multicolumn{5}{|c|}{Llama2-7B (FP32 Accuracy = 45.8\%)} & \multicolumn{4}{|c|}{Llama2-70B (FP32 Accuracy = 69.12\%)} \\ 
 \hline
 \hline
 64 & 43.9 & 43.4 & 43.9 & 44.9 & 68.07 & 68.27 & 68.17 & 68.75 \\
 \hline
 32 & 44.5 & 43.8 & 44.9 & 44.5 & 68.37 & 68.51 & 68.35 & 68.27  \\
 \hline
 16 & 43.9 & 42.7 & 44.9 & 45 & 68.12 & 68.77 & 68.31 & 68.59  \\
 \hline
 \hline
 \multicolumn{5}{|c|}{GPT3-22B (FP32 Accuracy = 38.75\%)} & \multicolumn{4}{|c|}{Nemotron4-15B (FP32 Accuracy = 64.3\%)} \\ 
 \hline
 \hline
 64 & 36.71 & 38.85 & 38.13 & 38.92 & 63.17 & 62.36 & 63.72 & 64.09 \\
 \hline
 32 & 37.95 & 38.69 & 39.45 & 38.34 & 64.05 & 62.30 & 63.8 & 64.33  \\
 \hline
 16 & 38.88 & 38.80 & 38.31 & 38.92 & 63.22 & 63.51 & 63.93 & 64.43  \\
 \hline
\end{tabular}
\caption{\label{tab:mmlu_abalation} Accuracy on MMLU dataset across GPT3-22B, Llama2-7B, 70B and Nemotron4-15B models.}
\end{table}


%\subsection{Perplexity achieved by various LO-BCQ configurations on LM evaluation harness}

\begin{table} \centering
\begin{tabular}{|c||c|c|c|c||c|c|c|c|} 
\hline
 $L_b \rightarrow$& \multicolumn{4}{c||}{8} & \multicolumn{4}{c||}{8}\\
 \hline
 \backslashbox{$L_A$\kern-1em}{\kern-1em$N_c$} & 2 & 4 & 8 & 16 & 2 & 4 & 8 & 16  \\
 %$N_c \rightarrow$ & 2 & 4 & 8 & 16 & 2 & 4 & 2 \\
 \hline
 \hline
 \multicolumn{5}{|c|}{Race (FP32 Accuracy = 37.51\%)} & \multicolumn{4}{|c|}{Boolq (FP32 Accuracy = 64.62\%)} \\ 
 \hline
 \hline
 64 & 36.94 & 37.13 & 36.27 & 37.13 & 63.73 & 62.26 & 63.49 & 63.36 \\
 \hline
 32 & 37.03 & 36.36 & 36.08 & 37.03 & 62.54 & 63.51 & 63.49 & 63.55  \\
 \hline
 16 & 37.03 & 37.03 & 36.46 & 37.03 & 61.1 & 63.79 & 63.58 & 63.33  \\
 \hline
 \hline
 \multicolumn{5}{|c|}{Winogrande (FP32 Accuracy = 58.01\%)} & \multicolumn{4}{|c|}{Piqa (FP32 Accuracy = 74.21\%)} \\ 
 \hline
 \hline
 64 & 58.17 & 57.22 & 57.85 & 58.33 & 73.01 & 73.07 & 73.07 & 72.80 \\
 \hline
 32 & 59.12 & 58.09 & 57.85 & 58.41 & 73.01 & 73.94 & 72.74 & 73.18  \\
 \hline
 16 & 57.93 & 58.88 & 57.93 & 58.56 & 73.94 & 72.80 & 73.01 & 73.94  \\
 \hline
\end{tabular}
\caption{\label{tab:mmlu_abalation} Accuracy on LM evaluation harness tasks on GPT3-1.3B model.}
\end{table}

\begin{table} \centering
\begin{tabular}{|c||c|c|c|c||c|c|c|c|} 
\hline
 $L_b \rightarrow$& \multicolumn{4}{c||}{8} & \multicolumn{4}{c||}{8}\\
 \hline
 \backslashbox{$L_A$\kern-1em}{\kern-1em$N_c$} & 2 & 4 & 8 & 16 & 2 & 4 & 8 & 16  \\
 %$N_c \rightarrow$ & 2 & 4 & 8 & 16 & 2 & 4 & 2 \\
 \hline
 \hline
 \multicolumn{5}{|c|}{Race (FP32 Accuracy = 41.34\%)} & \multicolumn{4}{|c|}{Boolq (FP32 Accuracy = 68.32\%)} \\ 
 \hline
 \hline
 64 & 40.48 & 40.10 & 39.43 & 39.90 & 69.20 & 68.41 & 69.45 & 68.56 \\
 \hline
 32 & 39.52 & 39.52 & 40.77 & 39.62 & 68.32 & 67.43 & 68.17 & 69.30  \\
 \hline
 16 & 39.81 & 39.71 & 39.90 & 40.38 & 68.10 & 66.33 & 69.51 & 69.42  \\
 \hline
 \hline
 \multicolumn{5}{|c|}{Winogrande (FP32 Accuracy = 67.88\%)} & \multicolumn{4}{|c|}{Piqa (FP32 Accuracy = 78.78\%)} \\ 
 \hline
 \hline
 64 & 66.85 & 66.61 & 67.72 & 67.88 & 77.31 & 77.42 & 77.75 & 77.64 \\
 \hline
 32 & 67.25 & 67.72 & 67.72 & 67.00 & 77.31 & 77.04 & 77.80 & 77.37  \\
 \hline
 16 & 68.11 & 68.90 & 67.88 & 67.48 & 77.37 & 78.13 & 78.13 & 77.69  \\
 \hline
\end{tabular}
\caption{\label{tab:mmlu_abalation} Accuracy on LM evaluation harness tasks on GPT3-8B model.}
\end{table}

\begin{table} \centering
\begin{tabular}{|c||c|c|c|c||c|c|c|c|} 
\hline
 $L_b \rightarrow$& \multicolumn{4}{c||}{8} & \multicolumn{4}{c||}{8}\\
 \hline
 \backslashbox{$L_A$\kern-1em}{\kern-1em$N_c$} & 2 & 4 & 8 & 16 & 2 & 4 & 8 & 16  \\
 %$N_c \rightarrow$ & 2 & 4 & 8 & 16 & 2 & 4 & 2 \\
 \hline
 \hline
 \multicolumn{5}{|c|}{Race (FP32 Accuracy = 40.67\%)} & \multicolumn{4}{|c|}{Boolq (FP32 Accuracy = 76.54\%)} \\ 
 \hline
 \hline
 64 & 40.48 & 40.10 & 39.43 & 39.90 & 75.41 & 75.11 & 77.09 & 75.66 \\
 \hline
 32 & 39.52 & 39.52 & 40.77 & 39.62 & 76.02 & 76.02 & 75.96 & 75.35  \\
 \hline
 16 & 39.81 & 39.71 & 39.90 & 40.38 & 75.05 & 73.82 & 75.72 & 76.09  \\
 \hline
 \hline
 \multicolumn{5}{|c|}{Winogrande (FP32 Accuracy = 70.64\%)} & \multicolumn{4}{|c|}{Piqa (FP32 Accuracy = 79.16\%)} \\ 
 \hline
 \hline
 64 & 69.14 & 70.17 & 70.17 & 70.56 & 78.24 & 79.00 & 78.62 & 78.73 \\
 \hline
 32 & 70.96 & 69.69 & 71.27 & 69.30 & 78.56 & 79.49 & 79.16 & 78.89  \\
 \hline
 16 & 71.03 & 69.53 & 69.69 & 70.40 & 78.13 & 79.16 & 79.00 & 79.00  \\
 \hline
\end{tabular}
\caption{\label{tab:mmlu_abalation} Accuracy on LM evaluation harness tasks on GPT3-22B model.}
\end{table}

\begin{table} \centering
\begin{tabular}{|c||c|c|c|c||c|c|c|c|} 
\hline
 $L_b \rightarrow$& \multicolumn{4}{c||}{8} & \multicolumn{4}{c||}{8}\\
 \hline
 \backslashbox{$L_A$\kern-1em}{\kern-1em$N_c$} & 2 & 4 & 8 & 16 & 2 & 4 & 8 & 16  \\
 %$N_c \rightarrow$ & 2 & 4 & 8 & 16 & 2 & 4 & 2 \\
 \hline
 \hline
 \multicolumn{5}{|c|}{Race (FP32 Accuracy = 44.4\%)} & \multicolumn{4}{|c|}{Boolq (FP32 Accuracy = 79.29\%)} \\ 
 \hline
 \hline
 64 & 42.49 & 42.51 & 42.58 & 43.45 & 77.58 & 77.37 & 77.43 & 78.1 \\
 \hline
 32 & 43.35 & 42.49 & 43.64 & 43.73 & 77.86 & 75.32 & 77.28 & 77.86  \\
 \hline
 16 & 44.21 & 44.21 & 43.64 & 42.97 & 78.65 & 77 & 76.94 & 77.98  \\
 \hline
 \hline
 \multicolumn{5}{|c|}{Winogrande (FP32 Accuracy = 69.38\%)} & \multicolumn{4}{|c|}{Piqa (FP32 Accuracy = 78.07\%)} \\ 
 \hline
 \hline
 64 & 68.9 & 68.43 & 69.77 & 68.19 & 77.09 & 76.82 & 77.09 & 77.86 \\
 \hline
 32 & 69.38 & 68.51 & 68.82 & 68.90 & 78.07 & 76.71 & 78.07 & 77.86  \\
 \hline
 16 & 69.53 & 67.09 & 69.38 & 68.90 & 77.37 & 77.8 & 77.91 & 77.69  \\
 \hline
\end{tabular}
\caption{\label{tab:mmlu_abalation} Accuracy on LM evaluation harness tasks on Llama2-7B model.}
\end{table}

\begin{table} \centering
\begin{tabular}{|c||c|c|c|c||c|c|c|c|} 
\hline
 $L_b \rightarrow$& \multicolumn{4}{c||}{8} & \multicolumn{4}{c||}{8}\\
 \hline
 \backslashbox{$L_A$\kern-1em}{\kern-1em$N_c$} & 2 & 4 & 8 & 16 & 2 & 4 & 8 & 16  \\
 %$N_c \rightarrow$ & 2 & 4 & 8 & 16 & 2 & 4 & 2 \\
 \hline
 \hline
 \multicolumn{5}{|c|}{Race (FP32 Accuracy = 48.8\%)} & \multicolumn{4}{|c|}{Boolq (FP32 Accuracy = 85.23\%)} \\ 
 \hline
 \hline
 64 & 49.00 & 49.00 & 49.28 & 48.71 & 82.82 & 84.28 & 84.03 & 84.25 \\
 \hline
 32 & 49.57 & 48.52 & 48.33 & 49.28 & 83.85 & 84.46 & 84.31 & 84.93  \\
 \hline
 16 & 49.85 & 49.09 & 49.28 & 48.99 & 85.11 & 84.46 & 84.61 & 83.94  \\
 \hline
 \hline
 \multicolumn{5}{|c|}{Winogrande (FP32 Accuracy = 79.95\%)} & \multicolumn{4}{|c|}{Piqa (FP32 Accuracy = 81.56\%)} \\ 
 \hline
 \hline
 64 & 78.77 & 78.45 & 78.37 & 79.16 & 81.45 & 80.69 & 81.45 & 81.5 \\
 \hline
 32 & 78.45 & 79.01 & 78.69 & 80.66 & 81.56 & 80.58 & 81.18 & 81.34  \\
 \hline
 16 & 79.95 & 79.56 & 79.79 & 79.72 & 81.28 & 81.66 & 81.28 & 80.96  \\
 \hline
\end{tabular}
\caption{\label{tab:mmlu_abalation} Accuracy on LM evaluation harness tasks on Llama2-70B model.}
\end{table}

%\section{MSE Studies}
%\textcolor{red}{TODO}


\subsection{Number Formats and Quantization Method}
\label{subsec:numFormats_quantMethod}
\subsubsection{Integer Format}
An $n$-bit signed integer (INT) is typically represented with a 2s-complement format \citep{yao2022zeroquant,xiao2023smoothquant,dai2021vsq}, where the most significant bit denotes the sign.

\subsubsection{Floating Point Format}
An $n$-bit signed floating point (FP) number $x$ comprises of a 1-bit sign ($x_{\mathrm{sign}}$), $B_m$-bit mantissa ($x_{\mathrm{mant}}$) and $B_e$-bit exponent ($x_{\mathrm{exp}}$) such that $B_m+B_e=n-1$. The associated constant exponent bias ($E_{\mathrm{bias}}$) is computed as $(2^{{B_e}-1}-1)$. We denote this format as $E_{B_e}M_{B_m}$.  

\subsubsection{Quantization Scheme}
\label{subsec:quant_method}
A quantization scheme dictates how a given unquantized tensor is converted to its quantized representation. We consider FP formats for the purpose of illustration. Given an unquantized tensor $\bm{X}$ and an FP format $E_{B_e}M_{B_m}$, we first, we compute the quantization scale factor $s_X$ that maps the maximum absolute value of $\bm{X}$ to the maximum quantization level of the $E_{B_e}M_{B_m}$ format as follows:
\begin{align}
\label{eq:sf}
    s_X = \frac{\mathrm{max}(|\bm{X}|)}{\mathrm{max}(E_{B_e}M_{B_m})}
\end{align}
In the above equation, $|\cdot|$ denotes the absolute value function.

Next, we scale $\bm{X}$ by $s_X$ and quantize it to $\hat{\bm{X}}$ by rounding it to the nearest quantization level of $E_{B_e}M_{B_m}$ as:

\begin{align}
\label{eq:tensor_quant}
    \hat{\bm{X}} = \text{round-to-nearest}\left(\frac{\bm{X}}{s_X}, E_{B_e}M_{B_m}\right)
\end{align}

We perform dynamic max-scaled quantization \citep{wu2020integer}, where the scale factor $s$ for activations is dynamically computed during runtime.

\subsection{Vector Scaled Quantization}
\begin{wrapfigure}{r}{0.35\linewidth}
  \centering
  \includegraphics[width=\linewidth]{sections/figures/vsquant.jpg}
  \caption{\small Vectorwise decomposition for per-vector scaled quantization (VSQ \citep{dai2021vsq}).}
  \label{fig:vsquant}
\end{wrapfigure}
During VSQ \citep{dai2021vsq}, the operand tensors are decomposed into 1D vectors in a hardware friendly manner as shown in Figure \ref{fig:vsquant}. Since the decomposed tensors are used as operands in matrix multiplications during inference, it is beneficial to perform this decomposition along the reduction dimension of the multiplication. The vectorwise quantization is performed similar to tensorwise quantization described in Equations \ref{eq:sf} and \ref{eq:tensor_quant}, where a scale factor $s_v$ is required for each vector $\bm{v}$ that maps the maximum absolute value of that vector to the maximum quantization level. While smaller vector lengths can lead to larger accuracy gains, the associated memory and computational overheads due to the per-vector scale factors increases. To alleviate these overheads, VSQ \citep{dai2021vsq} proposed a second level quantization of the per-vector scale factors to unsigned integers, while MX \citep{rouhani2023shared} quantizes them to integer powers of 2 (denoted as $2^{INT}$).

\subsubsection{MX Format}
The MX format proposed in \citep{rouhani2023microscaling} introduces the concept of sub-block shifting. For every two scalar elements of $b$-bits each, there is a shared exponent bit. The value of this exponent bit is determined through an empirical analysis that targets minimizing quantization MSE. We note that the FP format $E_{1}M_{b}$ is strictly better than MX from an accuracy perspective since it allocates a dedicated exponent bit to each scalar as opposed to sharing it across two scalars. Therefore, we conservatively bound the accuracy of a $b+2$-bit signed MX format with that of a $E_{1}M_{b}$ format in our comparisons. For instance, we use E1M2 format as a proxy for MX4.

\begin{figure}
    \centering
    \includegraphics[width=1\linewidth]{sections//figures/BlockFormats.pdf}
    \caption{\small Comparing LO-BCQ to MX format.}
    \label{fig:block_formats}
\end{figure}

Figure \ref{fig:block_formats} compares our $4$-bit LO-BCQ block format to MX \citep{rouhani2023microscaling}. As shown, both LO-BCQ and MX decompose a given operand tensor into block arrays and each block array into blocks. Similar to MX, we find that per-block quantization ($L_b < L_A$) leads to better accuracy due to increased flexibility. While MX achieves this through per-block $1$-bit micro-scales, we associate a dedicated codebook to each block through a per-block codebook selector. Further, MX quantizes the per-block array scale-factor to E8M0 format without per-tensor scaling. In contrast during LO-BCQ, we find that per-tensor scaling combined with quantization of per-block array scale-factor to E4M3 format results in superior inference accuracy across models. 



\end{document}
