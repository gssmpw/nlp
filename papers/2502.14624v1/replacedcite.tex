\section{Related work}
\paragraph{Online vector balancing.}
    When the incoming vectors satisfy $\norm{v_i}_2 \leq 1$, then randomly coloring the vectors $\{-1,+1\}$ achieves a discrepancy of $\sqrt{T \log{d}}$, even for the case of an adaptive adversary. A  matching lower bound of $\Omega(\sqrt{T})$ was proved by ____. The stochastic setting for online vector balancing was first studied in____. They considered a setting where the incoming vectors are chosen i.i.d.\@ from the uniform distribution over all vectors in $\{-1,1\}^n$ and gave an algorithm for achieving a $O(\sqrt{d}\log{T})$ bound on discrepancy at all time steps till $T$. This was later improved by ____ who showed a $O(d^2 \log{nT})$ upper bound for any distribution supported on $[-1,1]^n$. The work of ____ improved the dependence on $d$ by showing that $O(\sqrt{d}\log^4{dT})$ discrepancy can be achieved. The dependence on $T$ was further improved to $O_d(\sqrt{\log(T)})$ by ____, where the implicit dependence on $d$ was super-exponential. 
    
    The setting of an oblivious adversary remained relatively under-explored until the recent work of ____. They design an extremely simple and elegant algorithm that achieves a bound of $O(\log{nT})$ for both online vector balancing and online multicolor discrepancy. Their algorithm is based on a self-balancing random walk, which for vector balancing, ensures that the discrepancy prefix vector is $O(\sqrt{\log{nT}})$-subgaussian. A tight bound of $\Theta(\sqrt{\log{T}})$ for online vector balancing was achieved by ____, who proved the existence of an algorithm that maintains $O(1)$-subgaussian prefix vectors.


\paragraph{Stochastic fair division.}

Stochastic fair division, introduced by ____, studies the existence of fair allocations when valuations are drawn from a distribution. ____ show that maximizing utilitarian welfare produces an envy-free allocation with high probability when the number of items $T \in \Omega(n \log n)$ and items values are drawn i.i.d.\@ from a fixed ``constant distribution'' (i.e., the distribution does not depend on the number of items). ____ establish tight bounds for the existence of envy-free allocations in the ``constant distribution'' i.i.d.\@ model: $T \in \Omega (n \log n / \log \log n )$ is both a necessary and sufficient condition. ____ extend the result to the case of independent but non-identical additive agents.

Beyond envy-freeness, weaker fairness notions such as maximin share fair____ and proportional____ allocations exist with high probability. Finally, the existence of fair allocations for agents with non-additive stochastic valuations are studied in____.

%\alex{this paper has discrepancy in the title: ____; need to check}

\paragraph{Online fair division.}
A rich literature studies online or dynamic fair division. Numerous works study the problem where divisible or indivisible items arrive over time, with the goal of optimizing utilitarian welfare____, egalitarian welfare____, Nash welfare____, the generalized mean of the agents' utilities____, and approximation of the maximin share guarantee____.

Closer to our work, the aforementioned works____ prove that maximizing welfare and weighted welfare---algorithms that can be implemented online---achieve envy-freeness with high probability, in addition to Pareto efficiency, when valuations are drawn i.i.d.\@ from ``constant'' distributions for identical and non-identical agents. ____ prove that even when correlation is allowed between agents (but items are independent), Pareto efficiency and fairness are compatible in the online setting---the specific fairness guarantee for a pair of agents $i,j$ is ``either $i$ envies $j$ by at most 1 item, or $i$ does not envy $j$ with high probability''; again, distributions are treated as constants. To the best of our knowledge, we are the first to study the stochastic setting (online or offline) without the ``constant distribution'' assumption.



There are several variations of the standard models, e.g., revocable allocations____, only having access to pairwise comparisons____, two-sided matching____, items arriving in batches____, and repeated allocations____. Further afield, many works study the ``inverse'' problem of allocating static resources to agents that arrive/depart over time____.