%%%%%%%%%%%%%%%%%%%%%%%%%%%%%%%%%%%%%%%%%%%%%%%%%%%%%%%%%%%%%%%%%%%%%%%%%%%%%%%%
%2345678901234567890123456789012345678901234567890123456789012345678901234567890
%        1         2         3         4         5         6         7         8

\documentclass[letterpaper, 10 pt, conference]{ieeeconf}  % Comment this line out if you need a4paper

%\documentclass[a4paper, 10pt, conference]{ieeeconf}      % Use this line for a4 paper


\usepackage{graphicx}

\IEEEoverridecommandlockouts                              % This command is only needed if 
                                                          % you want to use the \thanks command

\overrideIEEEmargins                                      % Needed to meet printer requirements.

%In case you encounter the following error:
%Error 1010 The PDF file may be corrupt (unable to open PDF file) OR
%Error 1000 An error occurred while parsing a contents stream. Unable to analyze the PDF file.
%This is a known problem with pdfLaTeX conversion filter. The file cannot be opened with acrobat reader
%Please use one of the alternatives below to circumvent this error by uncommenting one or the other
%\pdfobjcompresslevel=0
%\pdfminorversion=4

% See the \addtolength command later in the file to balance the column lengths
% on the last page of the document

% The following packages can be found on http:\\www.ctan.org
%\usepackage{graphics} % for pdf, bitmapped graphics files
%\usepackage{epsfig} % for postscript graphics files
%\usepackage{mathptmx} % assumes new font selection scheme installed
%\usepackage{times} % assumes new font selection scheme installed
%\usepackage{amsmath} % assumes amsmath package installed
%\usepackage{amssymb}  % assumes amsmath package installed

\title{\LARGE \bf
Evolution 6.0: Evolving Robotic Capabilities Through Generative Design
}


\author{Muhammad Haris Khan, Artyom Myshlyaev, Artem Lykov,  Miguel Altamirano Cabrera, \\ and Dzmitry Tsetserukou% <-this % stops a space
% \thanks{*This work was not supported by any organization}% <-this % stops a space
\thanks{The authors are with the Intelligent Space Robotics Laboratory, Center for Digital Engineering, Skolkovo Institute of Science and Technology. 
 {\tt\small\{haris.khan, Artyom.Myshlyaev, Artem.Lykov, m.altamirano, d.tsetserukou\}@skoltech.ru}
}}


\begin{document}



\maketitle
\thispagestyle{empty}
\pagestyle{empty}


%%%%%%%%%%%%%%%%%%%%%%%%%%%%%%%%%%%%%%%%%%%%%%%%%%%%%%%%%%%%%%%%%%%%%%%%%%%%%%%%

\begin{abstract}

We propose a new concept, Evolution 6.0, which represents the evolution of robotics driven by Generative AI. When a robot lacks the necessary tools to accomplish a task requested by a human, it autonomously designs the required instruments and learns how to use them to achieve the goal. Evolution 6.0 is an autonomous robotic system powered by Vision-Language Models (VLMs), Vision-Language Action (VLA) models, and Text-to-3D generative models for tool design and task execution. The system comprises two key modules: the Tool Generation Module, which fabricates task-specific tools from visual and textual data, and the Action Generation Module, which converts natural language instructions into robotic actions. It integrates QwenVLM for environmental understanding, OpenVLA for task execution, and Llama-Mesh for 3D tool generation. Evaluation results demonstrate a 90\% success rate for tool generation with a 10-second inference time and action generation achieving 83.5\% in physical and visual generalization, 70\% in motion generalization, and 37\% in semantic generalization. Future improvements will focus on bimanual manipulation, expanded task capabilities, and enhanced environmental interpretation to improve real-world adaptability.

\end{abstract}

\begin{figure}[t!]
  \centering
  \includegraphics[width=0.48\textwidth]{figures/Evolution.png}
  \caption{The Evolution 6.0 system workflow, which autonomously analyzes a scene, generates a task-specific tool, and executes actions. The process includes scene interpretation, tool generation via an autoregressive model, rendering, 3D printing, and action execution using a 7D action vector derived from task instructions via VLA.}
  \label{fig:overview}
\end{figure}
%%%%%%%%%%%%%%%%%%%%%%%%%%%%%%%%%%%%%%%%%%%%%%%%%%%%%%%%%%%%%%%%%%%%%%%%%%%%%%%%
\section{Introduction}

The rapid development of generative AI and its associated tools has unlocked extraordinary opportunities for automation in robotics ~\cite{xu2024surveyroboticsfoundationmodels}. Visual Language Action (VLA) models enable cognitive robots to perform complex manipulation tasks in dynamic environments by leveraging task instructions and visual information to generate control signals for manipulator movements, bridging perception and action. Simultaneously, Industry 6.0 ~\cite{lykov2024industry} envisions a heterogeneous swarm of robots autonomously designing, manufacturing, delivering, and assembling products based on user specifications. This paradigm eliminates human involvement across all stages of production, empowering cognitive robots to utilize and fabricate tools autonomously. However, Industry 6.0 lacks the ability to assess its environment and autonomously determine the required tools, relying instead on direct user instructions. This limits its adaptability in environments with unforeseen challenges.

Unlike Industry 6.0, which operates in predefined industrial settings, we propose Evolution 6.0 illustrated in Fig. \ref{fig:overview}, a framework that extends these capabilities to unpredictable environments, such as Mars, where tool scarcity and environmental variability are major challenges. Evolution 6.0 empowers autonomous robots to analyze unfamiliar surroundings, interpret contextual cues via VLM, and generate actionable solutions with VLA models. This enables robots to autonomously determine, design, and fabricate tools as needed, eliminating dependency on predefined instructions and enhancing operational flexibility.
\begin{figure*}[!t]
  \centering
  \includegraphics[width=\textwidth]{figures/system_arch_evolution.png}
  \caption{Evolution 6.0 system architecture.}
  \label{fig:arch}
\end{figure*}

Inspired by the evolutionary leap of early humans who crafted tools to overcome challenges, Evolution 6.0 allows robots to autonomously perceive their surroundings, conceptualize solutions, and fabricate tools in real time. By integrating VLA models with Text-to-3D generative models, this framework enables robots to analyze novel challenges, dynamically design task-specific tools, and apply them seamlessly. In environments like Mars, where carrying a predefined toolset is impractical, this approach offers transformative potential, allowing robots to respond to unforeseen challenges with high autonomy. The Evolution 6.0 system is inherently designed to generalize across an open-ended range of tasks by dynamically interpreting new scenarios and autonomously generating the required tools. The presented evaluation with specific tasks serves as a benchmark to illustrate the system’s effectiveness and does not imply a limitation in its capabilities. Evolution 6.0 represents a paradigm shift by fostering adaptability in environments where traditional manufacturing infrastructures are unavailable, paving the way for a future of self-sufficient, adaptable robotic systems.
\begin{figure*}[!t]
  \centering
  \includegraphics[width=\textwidth]{figures/CAD_GenAi.png}
  \caption{CAD model design by Generative AI.}
  \label{fig:cad}
\end{figure*}
\section{Related Work}


The development of autonomous systems for tool design, fabrication, and utilization hinges on advancements in three areas: Vision-Language Models (VLMs) for environmental interpretation, Vision-Language Action (VLA) models for robotic manipulation, and Text-to-3D models for tool generation. VLMs enable robots to process multimodal inputs—images, videos, and natural language—to make informed decisions. Recent improvements in VLMs, such as BLIP-2~\cite{li2023blip}, Flamingo~\cite{alayrac2022flamingo}, Kosmos-2~\cite{peng2023kosmos}, Molmo and pixmo~\cite{deitke2024molmo}, have significantly enhanced visual and textual data interpretation by leveraging transformer-based architectures. Notably, Qwen2-VL~\cite{wang2024qwen2} excels in generalization and efficient multimodal learning, making it ideal for real-time robotic reasoning.

Concurrently, VLA models have enabled robots to interpret natural language instructions alongside visual data to perform complex manipulation tasks~\cite{gbagbe2024bi},~\cite{khan2025shake}. These models generate precise 7D action outputs—translational and rotational displacements and gripper actuation—using transformer architectures. Pioneering work like PaLM-E~\cite{driess2023palm} integrated sensory data with natural language for long-horizon robotic tasks, while the RT-series models (RT-1~\cite{brohan2022rt}, RT-2~\cite{brohan2023rt}, RT-X~\cite{o2023open}) leveraged large-scale datasets to learn versatile, task-conditioned policies. RT-2, in particular, builds on RT-1 by incorporating vision-language representations to enable cross-domain generalization without retraining. Efforts like OpenVLA~\cite{kim2024openvla} and MiniVLA~\cite{belkhale2024minivla} have further enhanced model accessibility and real-time performance, though challenges remain in real-time adaptation and robustness.

Text-to-3D models ~\cite{liu2022towards}, ~\cite{jain2022zero}, ~\cite{yu2023text}, ~\cite{liu2024sherpa3d} are key for converting language descriptions into tangible 3D geometries, enabling robots to autonomously create task-specific tools. Early methods like Text2Shape~\cite{chen2019text2shape} used deep generative models for this purpose, while subsequent approaches such as ShapeCrafter~\cite{fu2022shapecrafter} improved controllability and fidelity. Recent innovations include the Industry 6.0 framework, which employed GPT-4o~\cite{openai2024gpt4technicalreport} to generate Signed Distance Functions (SDFs)~\cite{osher2004level} for additive manufacturing. However, high computational demands have led to more efficient models like LLaMa-Mesh~\cite{wang2024llama} and MeshGPT~\cite{siddiqui2024meshgpt}, with our work adopting LLaMa-Mesh for its efficiency and adaptability.

The convergence of VLMs, VLA models, and Text-to-3D technologies offers a transformative opportunity for creating autonomous systems capable of tool design and utilization. Integrating VLMs' environmental reasoning with VLA's action generation and Text-to-3D's tool creation enables robots to dynamically generate and employ task-specific tools. This approach enhances flexibility in diverse environments and facilitates real-time execution, reducing reliance on predefined tools and human intervention. However, seamless integration requires addressing challenges in data consistency, real-time processing, and structural integrity. Our work builds upon these advancements to establish a comprehensive framework for autonomous tool design and usage, positioning Evolution 6.0 as a novel application that merges these capabilities into versatile robotic systems capable of dynamic tool generation and deployment.

% The development of autonomous systems for tool design, fabrication, and utilization hinges on advancements in three main areas: Vision-Language Models (VLMs) for environmental interpretation, Vision-Language Action (VLA) models for robotic manipulation, and Text-to-3D models for tool generation. VLMs are crucial for enabling robots to process multimodal inputs—images, videos, and natural language descriptions—to make informed decisions. Recent improvements in VLMs, such as BLIP-2\cite{li2023blip}, Flamingo\cite{alayrac2022flamingo}, and Kosmos-2\cite{peng2023kosmos}, Molmo and pixmo \cite{deitke2024molmo}, have significantly enhanced the interpretation of visual and textual data. These models utilize transformer-based architectures to merge visual and textual information, allowing context-aware decision-making in complex scenarios. Notably, Qwen2-VL\cite{wang2024qwen2} stands out as an advanced open-source model that excels in generalization and efficient multimodal learning, making it well-suited for robotic applications requiring real-time reasoning.
% Concurrently, VLA models have emerged as effective tools that enable robots to interpret natural language instructions alongside visual data to perform complex manipulation tasks \cite{gbagbe2024bi}. These models also leverage transformer architectures to produce precise 7D action outputs, including translational and rotational displacements and gripper actuation. Pioneering work like PaLM-E~\cite{driess2023palm} introduced a multimodal language model capable of reasoning over long-horizon robotic tasks by integrating sensory data with natural language input. The RT-series models—RT-1~\cite{brohan2022rt}, RT-2~\cite{brohan2023rt}, and RT-X~\cite{o2023open}—further advanced this field by utilizing large-scale robotic datasets to learn task-conditioned policies applicable across diverse environments. RT-2 notably enhances RT-1 by incorporating vision-language representations that facilitate cross-domain generalization without explicit retraining. Recent efforts to improve the accessibility of VLA models have resulted in frameworks like OpenVLA~\cite{kim2024openvla}, which allows fine-tuning of pre-trained models across various robotic platforms using consumer-grade hardware. Additionally, MiniVLA~\cite{belkhale2024minivla} proposed a lightweight architecture optimized for real-time inference, making VLA models more suitable for mobile applications. Despite these advancements, challenges persist regarding real-time adaptation to novel tasks and robustness in unstructured environments.
% Text-to-3D models \cite{liu2022towards}, \cite{jain2022zero}, \cite{yu2023text}, \cite{liu2024sherpa3d}, \cite{bensadoun2024meta} are essential for bridging abstract language descriptions with tangible 3D geometries, enabling robots to autonomously create task-specific tools based on user instructions or environmental needs. Early methods like Text2Shape~\cite{chen2019text2shape} translated textual descriptions into corresponding 3D representations using deep generative models. Subsequent developments such as ShapeCrafter~\cite{fu2022shapecrafter} improved the controllability and fidelity of generated structures. Recent innovations have demonstrated enhanced accuracy and generalization capabilities; for instance, the Industry 6.0 framework utilized GPT-4o~\cite{openai2024gpt4technicalreport} to generate Signed Distance Functions (SDFs) \cite{osher2004level} representing complex geometries for additive manufacturing processes. However, high computational demands have spurred the development of more efficient alternatives like LLaMa-Mesh~\cite{wang2024llama} and MeshGPT~\cite{siddiqui2024meshgpt}, which feature specialized architectures for 3D object generation. We adopt LLaMa-Mesh due to its computational efficiency and adaptability across various robotic applications.
% The convergence of VLMs, VLA models, and Text-to-3D technologies presents a transformative opportunity for creating autonomous systems capable of tool design and utilization. Integrating VLMs' environmental reasoning with VLA's action generation and Text-to-3D's tool creation allows robots to dynamically generate and employ task-specific tools. This integration fosters adaptive tool creation based on high-level task requirements while enhancing flexibility in diverse environments. It also facilitates real-time task execution, minimizing reliance on predefined tools and human intervention. However, achieving seamless integration requires overcoming challenges related to data consistency, real-time processing, and structural integrity of tools. Our work builds upon advancements in these areas to establish a comprehensive framework for autonomous tool design and usage, empowering robots to perform complex tasks with minimal human oversight. Evolution 6.0 introduces a novel application by merging these capabilities into versatile robotic systems that can autonomously generate and utilize tools in dynamic settings.
\begin{figure*}[!t]
  \centering
  \includegraphics[width=\linewidth]{figures/picture.png}
    (a)\ \ \ \ \ \ \ \ \ \ \ \ \ \ \ \ \ \ \ \ \ \ \ \ \ \ \ \ \ \ \ \ \ \ (b)\ \ \ \ \ \ \ \ \ \ \ \ \ \ \ \ \ \ \ \ \ \ \ \ \ \ \ \ \ \ \ \ \ \ (c)\ \ \ \ \ \ \ \ \ \ \ \ \ \ \ \ \ \ \ \ \ \ \ \ \ \ \ \ \ \ \ \ \ \ (d)\ \ \ \ \ \ \ \ \ \ \ \ \ \ \ \ \ \ \ \ \ \ \ \ \ \ \ \ \ \ \ \ \ \ (e)
    \caption{Evaluation tasks. (a) Visual generalization: presence of several unseen distractor objects. (b) Visual generalization: unseen background. (c) Motion generalization: the plate is elevated from its original position. (d) Physical generalization: the plate has a different size than the one used in the dataset. (e) All types of generalization: the cake is initialized in unseen position and orientation, the background is changed, distractor object presents in the scene.}
  
  
  \label{fig:scenarios}
\end{figure*}
\section{System Architecture of Evolution 6.0}

\subsection{Tool generation module}
The Tool Generation Module is responsible for analyzing the robot’s environment and autonomously generating the necessary tools to accomplish a given task. It begins with the Robot Scene Input, where sensors such as cameras capture the robot’s environment to identify objects requiring interaction. The captured data is processed through a Qwen2-VL-2B-Instruct, where a Vision Encoder extracts relevant visual features, which are then interpreted by the QwenLM decoder to generate contextual textual descriptions. Based on the scene analysis, the system formulates a prompt, such as “Create a 3D model of a knife,” which is passed to an Autoregressive Language Model (LlamMa-Mesh). This model synthesizes a 3D representation of the required tool in mesh format, containing vertex (v) and face (f) data that define the tool’s geometry.  The generated mesh proceeds to the Mesh Rendering phase, where it is visualized and validated to ensure it meets the task requirements. Once validated, the finalized mesh is converted into G-Code, enabling the tool to be fabricated using a 3D printer. It is important to note that the system scales the tool by a factor, depending on the size of the item that needs to be manipulated in the scene. Fig. \ref{fig:cad} shows several tool examples produced by the LlamMa-Mesh. 

The Evolution 6.0 is illustrated in Fig. \ref{fig:arch}. Our system integrates two connected modules that work together to enable flexible robotic manipulation. Tool generation module focuses on creating situation-specific tools, while the Action generation module determines how to perform each task. This design adapts its actions based on observations of the environment and instructions expressed in everyday language. The approach uses QwenVLM for interpreting surroundings, OpenVLA for determining task steps, and LlamMa-Mesh for shaping three-dimensional tools. Through this arrangement, robots can operate effectively in environments that require tailored solutions and adaptive behaviors.


Overall, the Tool Generation Module provides a dynamic and adaptive mechanism for equipping the robot with the right instruments needed to effectively perform tasks in ever-changing environments.

\subsection{Action generation module}
The Action Generation Module plays a crucial role in converting high-level natural language instructions into precise robotic actions using a multimodal approach. This module is built on OpenVLA, a model developed by the Stanford AI Lab, and has been fine-tuned on a dataset of cake-cutting and cake-picking tasks. Each episode in the dataset corresponds to a specific task performed by a remotely controlled robotic manipulator, helping the model adapt effectively to dynamic environments. The architecture of the Vision-Language-Action (VLA) model remains consistent with OpenVLA, retaining its multimodal input-output structure. The system takes third-person camera frames and natural language task descriptions as input. It outputs a 7D action vector that directly controls the robotic manipulator. This action vector consists of spatial displacements \( \Delta X \), angular movements \( \Delta \theta \), and gripping actions \( \Delta Grip \).

The system operates iteratively, continuously processing environmental input and adjusting its actions in real time. Once it receives a task description, the module continuously analyzes third-person visual data and textual instructions to compute and execute actions. This continuous processing approach ensures smooth and uninterrupted task execution by eliminating delays between movements. As a result, the manipulator can respond in real time to changes in the environment, improving overall task efficiency and adaptability. 

To facilitate real-time processing, the QwenVLM and LlamMa-Mesh models were optimized by reducing numerical precision to int8, while the fine-tuned Vision-Language Action (VLA) model was kept in its original precision. This optimization, in conjunction with the utilization of a high-performance NVIDIA RTX 4090 24Gb GPU and streamlined communication protocols, enables the system to operate at a frequency of 5 Hz. This approach strikes an effective balance between computational efficiency and responsiveness, allowing Evolution 6.0 to meet the operational requirements of complex manipulation tasks.





\begin{table*}[t]
\centering
\vspace{3mm} % Adjust this value to your preference
\caption{Phase one results for tool generation, showing success rates and average inference times.}
\label{table:tool_generation_results}
\begin{tabular}{|c|c|c|}
\hline
\textbf{Task} & \textbf{Success Rate} & \textbf{Average Inference Time (seconds)} \\
\hline
Environmental interpretation via VLM & 85\% & 4 \\
3D tool generation & 90\% & 10 \\
\hline
\end{tabular}
\end{table*}


\begin{figure*}[!t]
  \centering
  \includegraphics[width=\textwidth]{figures/download.png}
  \caption{Phase two results: success rates across generalization categories.}
  \label{fig:res}
\end{figure*}


\section{Dataset Collection and Training Pipeline}

The dataset for training was collected using a UR10 robotic arm equipped with a Logitech C920e HD 1080p camera to capture data for robotic manipulation tasks. The dataset comprises 20 episodes, evenly divided into two tasks: cutting a piece of cake and picking and placing the cake. Throughout the data collection process, the robot’s gripper position, orientation, and state were recorded, alongside synchronized high-definition visual feedback. Each episode includes language instructions such as “Cut one piece of cake” or “Pick one piece of cake and place on plate”. The collected data was formatted into the RLDS (Reinforcement learning dataset) structure, ensuring compatibility with OpenVLA for efficient organization of actions, images, and task instructions, making it suitable for imitation learning and task-specific action prediction. The collected dataset was used to fine-tune an OpenVLA-7b model using a parameter-efficient LoRA approach, applying rank-32 adapters to optimize memory usage while adjusting a minimal set of trainable weights. The training setup included a batch size of 16, a learning rate of  $5 \times 10^{-4}$, and 4000 gradient steps, with image augmentation disabled. Training was conducted on a single A100 GPU, with periodic checkpointing. Key metrics such as training loss, action accuracy, and L1 loss were monitored to ensure model performance. Additionally, smoothed metrics were employed for stable progress tracking. 

\section{Experiments}
\subsection*{Evaluation Methodology}
The evaluation was conducted in two phases. In the first phase, the tool generation module was assessed using success rate and inference time. In the second phase, the action generation module was evaluated on a UR-10 robot from a third-person view, with success rate as the primary metric. This phase encompassed five scenarios: the first scenario provided the model with previously encountered instructions and scenes from training, while the remaining four scenarios tested its generalization abilities. Specifically, physical generalization addressed variations in the target object's size or color, motion generalization involved changes in the object's position, semantic generalization required adapting to new instructions (such as replacing “cut the cake” with “cut the banana” or “cut the tomato” and “pick up the cake” with “pick up the cube”), and visual generalization evaluated performance when the scene was altered or distractors were introduced. The selection of Task 1 and Task 2 served as a validation step to demonstrate the system’s feasibility rather than its limitations, while Evolution 6.0's architecture supports seamless adaptation to new tasks without retraining due to robust multimodal integration.
% The evaluation was conducted in two phases. The first phase assessed the performance of the tool generation module, focusing on two primary metrics: success rate and inference time. In the second phase, the action generation module was evaluated using a UR-10 robot, with a third-person view to monitor the robot's actions. In this phase, the success rate was the main evaluation metric. The action generation module was tested in five distinct scenarios. The first scenario involved providing the model with previously encountered instructions and scenes during training. The remaining four scenarios were designed to evaluate the model’s generalization ability in various contexts: physical, motion, visual, and semantic generalization. Physical generalization in this context refers to variations in the size or color of the target object. Motion generalization involved changing the position of the target object, while semantic generalization tested the model's adaptability to new instructions (for instance, changing “cut the cake” to “cut the banana” or “cut the tomato” and replacing “pick up the cake” with “pick up the cube”). Lastly, visual generalization evaluated the model’s performance when the scene was altered or when distractors were introduced. The selection of Task 1 and Task 2 is a validation step to demonstrate the feasibility of the system, rather than an indication of its limitations. Evolution 6.0's architecture allows for seamless adaptation to new tasks without requiring retraining, thanks to its robust multimodal integration.


\subsubsection{Phase One: Tool Generation}
The first phase focused on the tool generation module, which was evaluated based on success rate and inference time. In this phase, the model was tasked with generating the required tools for different actions. The Qwen2-VL-2B-Instruct model serves as a sophisticated tool for interpreting environmental contexts and generating instructions for an autoregressive language model (LM) to create tools. This model is tested with 10 distinct robotic scenarios, successfully producing relevant prompts for the Text-to-3D’s model to facilitate tool generation. Out of these 10 scenarios, the visual language model (VLM) accurately described 8, yielding appropriate prompts. 





The average inference time for these interpretations was approximately 4 seconds, contingent upon the specific scene. However, the model encountered challenges when presented with scenes containing multiple identical target objects or objects belonging to the same class as the target. For instance, when tasked with distinguishing between screws and bolts, the model consistently misinterpreted the prompt as "Create the 3D model of the screwdriver," which is incompatible with the bolt. Despite this limitation, the autoregressive LM demonstrated a high success rate in generating tools based on these prompts, achieving successful outputs in 9 out of 10 instances, with an average generation time of 10 seconds, varying according to the complexity of the tool. Moreover, the model struggled with generating complex curves or fillets in its outputs. As illustrated in accompanying Fig. \ref{fig:cad}, all generated tools predominantly exhibited either sharp edges or simple circular cross-sections. This highlights a significant area for improvement in the model's ability to handle more intricate geometrical features in tool design. The results are summarized in Table 1.



\subsubsection{Phase Two: Action Generation}
In the second phase of the study, the action generation module was evaluated using a UR-10 robotic arm, with success rate serving as the primary metric. This phase involved testing the module across 10 distinct scenarios, including variations in physical, motion, visual, and semantic generalization. The evaluation aimed to assess the module's adaptability and robustness under different conditions.

\textbf{For Task 1}, the module demonstrated strong performance, achieving success rates of 87\% in physical generalization, 83\% in visual generalization, 75\% in motion generalization, and 41\% in semantic generalization. Notably, when provided with scenarios closely aligned with training settings, the module achieved a perfect success rate of 100\%.

\textbf{In Task 2}, performance slightly declined across most categories, with success rates of 80\% in physical generalization, 84\% in visual generalization, 65\% in motion generalization, and 33\% in semantic generalization. However, similar to Task 1, the module maintained a perfect success rate of 100\% when tested on scenarios identical to its training settings.

Experimental results (summarized in Fig. \ref{fig:res}) highlight the module's proficiency in physical and visual generalization and its resilience under varying conditions. However, there is significant room for improvement in motion and semantic generalization. For instance, when presented with instructions such as "Cut the banana" or "Place the tomato on the plate," the module exhibited inconsistent behavior. Examples of these scenarios are illustrated in Fig. ~\ref{fig:scenarios}. 

Overall, while the action generation module shows promise in handling familiar settings and certain types of generalizations, further refinement is needed to enhance its reliability in more complex or novel scenarios.


\section{Conclusion and Future Work}
% Evolution 6.0 is a novel framework that integrates Vision-Language Models (VLMs), Vision-Language Action (VLA) models, and Text-to-3D generative models to enable autonomous robotic tool design and utilization in dynamic environments. Our experimental results demonstrated the system’s effectiveness, achieving a tool generation success rate of 90\% with an average inference time of 10 seconds, while the action generation module achieved average success rates of 83.5\% in physical generalization, 83.5\% in visual generalization, 70\% in motion generalization, and 37\% in semantic generalization. These results highlight the system's adaptability and potential for addressing real-world challenges, but also indicate areas for improvement, particularly in semantic generalization and complex geometric tool generation. Future work will focus on enhancing the system’s capabilities by extending it to bimanual manipulation, allowing for coordinated control of dual robotic arms to handle more complex tasks such as cooperative object handling and assembly. Additionally, an extended set of tasks will be incorporated to improve the system’s versatility in diverse environments, including industrial applications requiring higher precision and dexterity. Another key direction is the optimization of environmental interpretation by integrating more advanced models and imitation learning techniques to improve accuracy in complex, cluttered, or ambiguous scenarios. By pursuing these enhancements, Evolution 6.0 aims to become a more robust, flexible, and autonomous robotic framework capable of addressing a broader range of operational challenges with improved efficiency and adaptability. 
Evolution 6.0 is a novel framework that integrates Vision-Language Models (VLMs), Vision-Language Action (VLA) models, and Text-to-3D generative models to enable autonomous robotic tool design and utilization in dynamic environments. Our experimental results demonstrate the system’s effectiveness, with the tool generation module achieving a 90\% success rate and an average inference time of 10 seconds, while the action generation module reached average success rates of 83.5\% in physical and visual generalization, 70\% in motion generalization, and 37\% in semantic generalization. These outcomes highlight the system's adaptability and potential for real-world applications, while also indicating areas for improvement in semantic generalization and complex geometric tool generation. Future work will extend the framework to bimanual manipulation for coordinated dual-arm control in complex tasks like cooperative object handling and assembly, incorporate an expanded set of tasks to enhance versatility in industrial environments requiring higher precision and dexterity, and optimize environmental interpretation by integrating more advanced models and imitation learning techniques to improve accuracy in complex, cluttered, or ambiguous scenarios.

% \bibliographystyle{IEEEtran}
% \bibliography{ref}
% \balance


%%%%%%%%%%%%%%%%%%%%%%%%%%%%%%%%%%%%%%%%%%%%%%%%%%%%%%%%%%%%%%%%%%%%%%%%%%%%%%%%
%2345678901234567890123456789012345678901234567890123456789012345678901234567890
%        1         2         3         4         5         6         7         8

\documentclass[letterpaper, 10 pt, conference]{ieeeconf}  % Comment this line out if you need a4paper

%\documentclass[a4paper, 10pt, conference]{ieeeconf}      % Use this line for a4 paper

\IEEEoverridecommandlockouts                              % This command is only needed if 
                                                          % you want to use the \thanks command

\overrideIEEEmargins                                      % Needed to meet printer requirements.

%In case you encounter the following error:
%Error 1010 The PDF file may be corrupt (unable to open PDF file) OR
%Error 1000 An error occurred while parsing a contents stream. Unable to analyze the PDF file.
%This is a known problem with pdfLaTeX conversion filter. The file cannot be opened with acrobat reader
%Please use one of the alternatives below to circumvent this error by uncommenting one or the other
%\pdfobjcompresslevel=0
%\pdfminorversion=4

% See the \addtolength command later in the file to balance the column lengths
% on the last page of the document

% The following packages can be found on http:\\www.ctan.org
%\usepackage{graphics} % for pdf, bitmapped graphics files
%\usepackage{epsfig} % for postscript graphics files
%\usepackage{mathptmx} % assumes new font selection scheme installed
%\usepackage{times} % assumes new font selection scheme installed
%\usepackage{amsmath} % assumes amsmath package installed
%\usepackage{amssymb}  % assumes amsmath package installed
\usepackage{amsmath}
\usepackage{amssymb}
\usepackage{hyperref}
\usepackage{multirow}
\usepackage{graphicx}


\title{\LARGE \bf
4DR P2T: 4D Radar Tensor Synthesis with Point Clouds
}


\author{Woo-Jin Jung, Dong-Hee Paek, and Seung-Hyun Kong% <-this % stops a space
\thanks{This work was supported by the National Research Foundation of Korea(NRF) grant funded by the Korea government(MSIT) (No. 2021R1A2C3008370).}% <-this % stops a space
\thanks{Woo-Jin Jung, Dong-Hee Paek, and Seung-Hyun Kong are with the CCS Graduate School of Mobility, Korea Advanced Institute of Science and Technology, Daejeon, Korea, 34051 
        {\tt\small \{woo-jin.jung, donghee.paek, skong\}@kaist.ac.kr}}}%



\begin{document}



\maketitle
\thispagestyle{empty}
\pagestyle{empty}


%%%%%%%%%%%%%%%%%%%%%%%%%%%%%%%%%%%%%%%%%%%%%%%%%%%%%%%%%%%%%%%%%%%%%%%%%%%%%%%%
\begin{abstract}

In four-dimensional (4D) Radar-based point cloud generation, clutter removal is commonly performed using the constant false alarm rate (CFAR) algorithm. However, CFAR may not fully capture the spatial characteristics of objects. To address limitation, this paper proposes the 4D Radar Point-to-Tensor (4DR P2T) model, which generates tensor data suitable for deep learning applications while minimizing measurement loss. Our method employs a conditional generative adversarial network (cGAN), modified to effectively process 4D Radar point cloud data and generate tensor data. Experimental results on the K-Radar dataset validate the effectiveness of the 4DR P2T model, achieving an average PSNR of 30.39dB and SSIM of 0.96. Additionally, our analysis of different point cloud generation methods highlights that the 5\% percentile method provides the best overall performance, while the 1\% percentile method optimally balances data volume reduction and performance, making it well-suited for deep learning applications.

\end{abstract}


%%%%%%%%%%%%%%%%%%%%%%%%%%%%%%%%%%%%%%%%%%%%%%%%%%%%%%%%%%%%%%%%%%%%%%%%%%%%%%%%
\section{INTRODUCTION}

In recent autonomous driving research, 4D Radar has gained increasing attention as an advanced sensing technology. Traditional Radar sensors, often employed as auxiliary sensors, measure range, azimuth, and Doppler information. In contrast, 4D Radar incorporates elevation into these measurements, enabling more precise spatial perception. Consequently, 4D Radar provides more robust measurements than cameras and LiDAR under adverse weather conditions such as snow or rain. Furthermore, it surpasses traditional Radar in detecting object contours, demonstrating superior object detection capabilities. Owing to these advantages, 4D Radar has emerged as a key sensing modality in autonomous driving systems, offering enhanced object detection across diverse operational environments.

Most 4D Radar data are provided as point clouds, which are typically generated by traditional handcrafted methods such as CFAR to remove clutter \cite{clutter} from the tensor data. However, CFAR processes each cell independently, disregarding spatial continuity across adjacent cells. As a result, CFAR-generated point clouds often fail to preserve essential spatial characteristics—such as object size, shape, and continuous contours—thereby limiting their ability to accurately represent complex objects \cite{4D_radar_survey}. In autonomous driving scenarios where objects vary in size and shape, this limitation constrains environmental perception. Moreover, CFAR-based point clouds typically exhibit much lower point density than LiDAR, reducing the fidelity of captured object features \cite{rpfa_net} and complicating subsequent sensor fusion processes \cite{dpft}.

\begin{figure}[t!]
  \centering
  \includegraphics[width=1.0\columnwidth]{fig/fig0.png}
  \caption{
   4DR P2T overview. The 4DR P2T model generates tensor data from 4D Radar point clouds, which are represented in bird’s-eye view (BEV) as a 2D projection. Traditional point cloud generation methods often suffer from measurement loss, which may affect their suitability for deep learning training. To mitigate this limitation, the model generates tensor data to prevent measurement loss, ensuring that crucial information is retained for deep learning tasks.}
  \label{fig0.overview}
\end{figure}

\begin{figure*}[!th]
 \centering
\vspace{1mm} 
 \includegraphics[width=1.0\textwidth]{fig/fig1.png}
    \caption{4D Radar signal processing and data representation \cite{4D_radar_survey, 4dradar_tutorial, 4dradar_data_representation}. The Radar power values are normalized and represented using colors. The Radar point cloud is shown as black points, and the bounding box for the objects is indicated with a red box.}
  \label{fig1.4d_radar_data}

\end{figure*}

To mitigate these limitations, previous studies have proposed methods for reconstructing points representing objects \cite{3DRIMR} or generating tensor data prior to CFAR \cite{radarpointgenerator1} \cite{4D_radar_survey}. One notable method \cite{radarpointgenerator1} utilizes a conditional generative adversarial network (cGAN) with a UNet \cite{unet} architecture, leveraging LiDAR data to supervise the generation of denser Radar point clouds. However, fundamental differences between LiDAR (near-infrared) and Radar (electromagnetic waves) result in heterogeneous data characteristics, leading to distortions in power values and contour representations, which may degrade the reliability of the generated Radar data.

As shown in Fig. \ref{fig0.overview}, a method is required to directly generate tensor data using the original 4D Radar tensor as supervision, thereby avoiding cross-sensor inconsistencies. In this study, we leverage the K-Radar dataset \cite{KRadar}, currently the only publicly available dataset that provides 4D tensor data. Prior to its release in 2023, no dataset included 4D tensor data, making direct data-driven methods infeasible. With this new dataset, it is now possible to train models that generate tensor representations from 4D Radar point clouds collected by the same sensor.

Accordingly, we propose the 4D Radar Point cloud-to-Tensor (4DR P2T) model, which utilizes a cGAN-based architecture to generate tensor data from 4D Radar point clouds. This study conducts two primary investigations. First, we identify the point cloud generation method that achieves the best tensor generation performance—measured by peak signal-to-noise ratio (PSNR) and structural similarity index measure (SSIM)—among CFAR \cite{rtnh+} and percentile-based methods \cite{KRadar, enhancekradar} with different densities. Second, we determine the optimal point cloud generation method for deep learning applications, specifically the one that minimizes data volume while preserving sufficiently high tensor generation performance. To enable these investigations, we interpret point cloud data as the encoded version of tensor data, with our 4DR P2T model serving as a decoder that generates the original tensor. Consequently, the tensor generation performance of the 4DR P2T model serves as a proxy for assessing how well a given point cloud preserves environmental information, which in turn facilitates the selection of the most suitable point cloud generation method for deep learning model training and interpretation. Through our experiments, the proposed 4DR P2T model achieves an average PSNR of 30.39dB and SSIM of 0.96, demonstrating its effectiveness and stability. Our findings reveal that the percentile 5\% method yields the best tensor generation performance, while the percentile 1\% method offers an optimal balance between data volume reduction and performance, making it well-suited for deep learning training.

The key contributions of this study are summarized as follows:
\begin{itemize}
    \item Development of the 4DR P2T model, which generates tensor data from 4D Radar point cloud data.
    \item Experimental validation showing that the percentile 5\% data provides the best tensor generation performance.
    \item Confirmation that the percentile 1\% method effectively reduces data volume while maintaining high tensor generation performance.
\end{itemize}

This paper is organized as follows. Section \ref{sec:related_works} discusses 4D Radar signal processing and data generation processes, and reviews related models that convert point cloud data into tensors. Section \ref{sec:method} describes the proposed model architecture. Section \ref{sec:experiments} presents and analyzes the quantitative and qualitative experimental results. Finally, Section \ref{sec:conclusion} concludes the paper and discusses future research directions.

\section{Related work} \label{sec:related_works}
In this section, we provide an overview of related works, focusing specifically on 4D Radar signal processing and data generation, as well as previous studies on data translation methods using cGANs, which form the basis for developing models that generate tensor data from point cloud data.

\begin{figure*}[!th]
    \centering
    \vspace{1mm} % 위쪽 여백 추가
    \includegraphics[width=1\textwidth]{fig/fig2.png} 
    \caption{Overall structure of 4DR P2T. The encoder utilizes 3D sparse convolution to process 4D Radar point cloud data, while the decoder employs 3D dense convolution to generate tensor data.}
    \label{fig2.model}
\end{figure*}

\subsection{4D Radar signal processing and data generation}
The 4D Radar signal processing and data generation process, as applied in autonomous driving, is illustrated in Figure \ref{fig1.4d_radar_data}. The core analog components of a 4D Radar system consist of a synthesizer, transmission (TX) antennas, reception (RX) antennas, and a mixer. The TX antennas emit electromagnetic waves, which reflect off objects in the environment and are received by the RX antennas. The transmitted signal is generated by the synthesizer and radiated through the TX antennas. This signal is a frequency-modulated continuous wave (FMCW), composed of a sequence of frequency-modulated signals, commonly referred to as chirps.

The signal emitted by the TX antennas and the signal received by the RX antennas are combined using a mixer, producing an intermediate frequency (IF) signal. This IF signal represents the frequency difference between the transmitted and received signals, which is used to extract the distance and velocity of the reflected objects. The generated IF signal is then converted into a digital form through an analog-to-digital converter (ADC), creating ADC sample data. This data is separated into a fast time axis, which calculates range information through chirp sampling, and a slow time axis, which calculates Doppler information through frame sampling.

The ADC sample data is processed through a 2D Fast Fourier Transform (FFT), which is applied to perform range FFT and Doppler FFT. The range FFT estimates the distance to objects, while the Doppler FFT estimates their relative velocity, resulting in the generation of an RD heatmap. Although the RD heatmap contains information about range and velocity, it does not include azimuth or elevation information, making it less intuitive to interpret.

To extract azimuth and elevation information, an additional angle FFT is applied to the RD heatmap. The angle FFT utilizes the positional information of the TX and RX antennas arrays in a multiple-input and multiple-output (MIMO) antennas design to analyze the phase differences in the reflected signals. This process generates a 4D tensor that includes range, azimuth, elevation, and Doppler information, with each tensor cell representing the corresponding signal strength. The 4D tensor is represented in a polar coordinate system, but for better interpretability, the visualization shown in the Fig. \ref{fig1.4d_radar_data} is converted into a Cartesian coordinate system.

The generated 4D tensor data is filtered using the CFAR method. CFAR dynamically adjusts the threshold by comparing the signal strength of each cell to its surrounding cells, effectively removing noise and identifying actual targets. This filtering process is applied across all dimensions of the tensor, ultimately producing point cloud data that contains information about actual targets.

The resulting point cloud data includes the position (range, azimuth, elevation) and Doppler of the detected objects and is utilized in various autonomous driving applications, such as object detection and tracking \cite{4dradar_tutorial, 4D_radar_survey, FFT-RadNet}.

\subsection{Image translation}

Image translation focuses on style translation while preserving key information. Notable methods include pix2pix \cite{pix2pix} and pix2pixHD \cite{pix2pixhd}. These methods utilize cGANs to translate input images into output images. These methods have been successfully applied to various image synthesis and transformation tasks \cite{radsimreal, l2r, radarpointgenerator1}. Pix2pix employs a U-Net-based generator and a patch-based discriminator, enabling applications such as image synthesis and color translation. Pix2pixHD extends this framework to handle high-resolution images by incorporating boundary maps, multi-scale generators, and multi-scale discriminators, achieving improved quality. These methods excel at 2D image-to-image translation while maintaining structural information.

However, this study deals with generating tensor data from 4D Radar point cloud data, making it challenging to directly apply conventional image translation models. Existing methods are primarily optimized for 2D image data, necessitating structural modifications to handle higher-dimensional data such as point clouds and tensors. To address this, this study extends the fundamental method of pix2pixHD by modifying the model architecture to effectively process 3D or higher-dimensional data.

\section{Method} \label{sec:method}

This section outlines the data dimensions used for training, the model architecture, and the objective function.

\subsection{Data preparation}
In this study, the 4D Radar tensor data is reduced to 3D spatial information by excluding Doppler information for the training process. As a result, the input data for training consists of four channels, including \(x\), \(y\), \(z\) coordinates, and power values. Including Doppler data would require processing additional values beyond the existing spatial information \((x, y, z)\) and power, necessitating the use of convolution layers with at least four dimensions. This would significantly increase the complexity of the model and computational costs, making it challenging to achieve the primary goal of verifying implementation feasibility in the initial stage of the research. Moreover, according to RTNH \cite{KRadar}, an early model utilizing 4D Radar tensor data, excluding Doppler information still achieves sufficient object detection performance. Therefore, this study focuses on minimizing model complexity while verifying the feasibility of generating tensor data from 4D Radar point cloud data. This method also lays the foundation for future studies incorporating Doppler information.


\subsection{Model structure}
The proposed model is inspired by image translation methods, such as pix2pix and pix2pixHD, and referenced recent studies like l2r \cite{l2r} and RadSimReal \cite{radsimreal} to balance generative feasibility and structural simplicity, while optimizing for input and output data dimensions. To capture the spatial characteristics of Radar point cloud data, the model employs an encoding method based on Voxelnet \cite{voxelnet}, drawing from Lee’s method \cite{l2rtranslation_voxel}. The 4DR P2T model extends the U-Net structure \cite{unet}, commonly used in image translation tasks, with modifications to process 3D data.

As illustrated in Figure \ref{fig2.model}, the encoder uses 3D sparse convolution layers to account for the sparsity of 4D Radar point cloud data. Sparse convolution layers \cite{sparse_conv} are employed in stages where spatial resolution is reduced, while submanifold sparse convolution layers \cite{submaninfold} are utilized for operations where spatial resolution remains unchanged, thereby enhancing feature representation. In the decoder, 3D dense convolution layers are used to generate a dense 3D tensor. This method performs computations across all regions, making it suitable for producing complete tensors.

The generated tensor data is evaluated using a multi-scale discriminator  \cite{pix2pixhd}, which determines the authenticity of the data. To handle dense data, the discriminator also incorporates 3D dense convolution layers.

\subsection{Objective functions}
4DR P2T adopts a training framework using a Generator \(G\) and a Discriminator \(D\), inspired by traditional image translation methods. While image translation typically aims for a one-to-many mapping to generate diverse outputs, this study focuses on a one-to-one mapping, necessitating the design of appropriate loss functions. Following the method by Wang \cite{l2r}, the final loss function is defined as follows:
\begin{align}
\mathcal{L}(G, D) &= \mathcal{L}_{cGAN}(G, D) + \lambda_{L1} \mathcal{L}_{L1}(G) \nonumber \\
&\quad + \lambda_{perc} \mathcal{L}_{perc}(G) \label{eq:loss}
\end{align}
where \(\lambda_{L1}\) and \(\lambda_{perc}\) are weights that control the importance of each loss component, ensuring balanced training.

% \subsubsubsection{Conditional Adversarial Loss ({L}_{cGAN})}
First, a conditional adversarial loss is used, where the \(G\) synthesizes data, and the \(D\) learns to distinguish between real and synthesized data. This is the core loss of cGANs, defined as:
\begin{align}
\mathcal{L}_{cGAN}(G, D) &= \mathbb{E}_{x,y}[\log D(x, y)] \nonumber \\
&\quad + \mathbb{E}_{x}[\log (1 - D(x, G(x)))] \label{eq:cgan_loss}
\end{align}
Here, \(x\) represents input data, \(y\) is the GT, and \(G(x)\) is the output of the \(G\). The \(D\) learns to differentiate \(G(x)\) from \(y\), while the \(G(x)\) is trained to deceive \(D\) by making \(G(x)\) resemble \(y\).

% \subsubsubsection{L1 Loss ({L}_{L1})}
Second, L1 loss minimizes the absolute error between synthesized data \(G(x)\) and GT \(y\). This simple and stable loss function ensures that the synthesized data closely resembles real data:
\begin{equation}
\mathcal{L}_{L1}(G) = \mathbb{E}_{x,y}\left[|G(x) - y|_1\right]
\label{eq:l1_loss}
\end{equation}
% \subsubsubsection{Perceptual Loss ({L}_{perc})}
Third, perceptual loss introduced in pix2pixHD is used to compare high-level feature distributions between synthesized and real data. By leveraging intermediate layer outputs from a pre-trained neural network, perceptual loss measures semantic differences, guiding the synthesized data to have similar high-level features to real data.

\begin{figure*}[!th]
{
  \centering
\vspace{1mm} 
 \includegraphics[width=1.0\textwidth]{fig/fig3.png}
    \caption{Qualitative experimental results of 4DR P2T. The top part shows the front camera image and LiDAR point cloud as reference data to understand the scene of the 4D Radar GT tensor data, while the bottom part presents the tensor data results generated by 4DR P2T under different point cloud generation methods and density conditions.}
  \label{fig3.result}
}
\end{figure*}

\section{Experiments} \label{sec:experiments}

This section describes the datasets and implementation details for training the 4DR P2T model, the evaluation metrics used, and the results and analysis of tensor generation performance.

\subsection{K-Radar dataset}
The K-Radar dataset \cite{KRadar}, which was used to train the 4DR P2T model, is the only dataset that provides 4D Radar tensor data (4DRT) consisting of the four dimensions: range, azimuth, elevation, and Doppler. This makes it of significant value. Additionally, K-Radar includes data from various weather conditions (clear, cloudy, fog, rain, sleet, light snow, and heavy snow), which distinguishes it from other autonomous driving 4D Radar datasets. Furthermore, the dataset includes 4D Radar data, high-resolution LiDAR data, and camera data, with 93.3K object labels for 35K frames, distributed across 58 different driving scenes.

K-Radar includes not only 4DRT tensor data but also point cloud data with CFAR applied, as used in the experiments of RTNH+ \cite{rtnh+}, and point cloud data with the percentile method applied, as used in the RTNH model \cite{KRadar}. The percentile method is effective in reducing memory and computational complexity while preserving the structure of tensor data, and thus was used as input data for training the RTNH model.

\subsection{Implementation details}
The experiments in this study were conducted using the K-Radar dataset, with data generated using various point cloud generation methods and density conditions for comparison. Specifically, point cloud data generated using the percentile method (top 0.1\%, 1\%, 5\%, 10\%) from enhanced K-Radar \cite{enhancekradar}, and point cloud data with hyper-parameter ($K_{1}$) of 2.5\% ($N_{2.5,a}$) and 10\% ($N_{10,a}$) using constant average CFAR from RTNH+ \cite{rtnh+} were used. These datasets were selected due to the significant differences in the point cloud distribution, making them suitable for comparison analysis.

The point cloud data used for training was extracted from 4D tensors in polar coordinates using CFAR or the percentile method and then converted into Cartesian coordinates. In this process, it can be observed that points become increasingly sparse as the range (distance) increases (Fig. \ref{fig0.overview}). The tensor data used for training was reconstructed into a dense cube shape through interpolation after converting from polar to Cartesian coordinates \cite{KRadar}. This data preparation process was set up to verify whether sparse point cloud data could be transformed into dense Cartesian tensor data and to expand its range of applicability.

The Region of Interest (ROI) was set as $x$-axis [0, 76.8], $y$-axis [-16, 16], and $z$-axis [-2, 10.8]. This range was chosen considering the scope of the RTNH\_WIDE \cite{kradargithub} object detection model trained with the widest range. All sequence data were used in the experiment, with the 4DR P2T model trained using the train set and performance evaluated using the test set. Model training was performed on an NVIDIA 3090 GPU, with a batch size of 8, a learning rate of 0.001, and Adam optimizer \cite{adam}, running for 20 epochs.

\subsection{Metrics}
For evaluation metrics, PSNR and SSIM were used, referencing \cite{l2r}. PSNR measures the signal-to-noise ratio between the synthetic data and the ground truth data, while SSIM measures the structural similarity between the two datasets. Both metrics indicate better performance with higher values. Although the generated data is a 3D tensor, the evaluation was performed by converting it to a 2D image through mean pooling along the height axis, and then calculating the metrics.

The deep-learning efficiency score (DES) metric, defined in Eq. \ref{eq:DES}, was used to identify efficient point cloud generation methods for deep learning. This metric aims to reduce data volume, which is related to point cloud density (PCD), while maintaining high tensor generation performance. First, the PSNR and SSIM values are normalized using min-max scaling, as shown in Eq. \ref{eq:psnr_norm} and Eq. \ref{eq:ssim_norm}, to ensure a fair comparison.
\begin{equation}
\text{PSNR}_{\text{norm}}^{(i)} = \frac{\text{PSNR}^{(i)} - \text{PSNR}_{\min}}{\text{PSNR}_{\max} - \text{PSNR}_{\min}}
\label{eq:psnr_norm}
\end{equation}
\begin{equation}
\text{SSIM}_{\text{norm}}^{(i)} = \frac{\text{SSIM}^{(i)} - \text{SSIM}_{\min}}{\text{SSIM}_{\max} - \text{SSIM}_{\min}}
\label{eq:ssim_norm}
\end{equation}
Where \( \text{PSNR}^{(i)} \) and \( \text{SSIM}^{(i)} \) represent the PSNR and SSIM for the \textit{\( i \)-th method}, respectively. \( \text{PSNR}_{\min} \) and \( \text{PSNR}_{\max} \) denote the minimum and maximum PSNR values across all evaluated methods, and similarly \( \text{SSIM}_{\min} \) and \( \text{SSIM}_{\max} \) represent the minimum and maximum SSIM values. Using these normalized values, the DES metric is computed as shown in Eq. \ref{eq:DES}.
\begin{equation}
M = \alpha \times \frac{\text{PSNR}_{\text{norm}}^{(i)}}{D^{(i)}} 
+ \beta \times \frac{\text{SSIM}_{\text{norm}}^{(i)}}{D^{(i)}}
\label{eq:DES}
\end{equation}
Where \( D^{(i)} \) is the PCD, defined as the ratio of detected points to the total possible points within the ROI for method \( i \). The weighting factors \( \alpha \) and \( \beta \), which control the relative importance of PSNR and SSIM, satisfy the constraint \( \alpha + \beta = 1 \). In this study, equal weights of 0.5 were assigned to both PSNR and SSIM.

\subsection{Results}

\begin{table}[ht]
\caption{Quantitative experimental results of 4DR P2T. 'Method' refers to the main categories of point cloud generation methods, while 'Hyper.' denotes the subcategories of point cloud generation methods, representing the hyper-parameters used in each point generation method. }
\label{tab:result}
\centering
\renewcommand{\arraystretch}{1.4} 
\begin{tabular}{c|c|c|c|c|c}
\hline \hline
\begin{tabular}[c]{@{}c@{}}Method\end{tabular} &
  Hyper. &
  \begin{tabular}[c]{@{}c@{}} PCD (\%)\end{tabular} &
  PSNR (dB) ↑ &
  SSIM ↑ &
  \begin{tabular}[c]{@{}c@{}}DES ↑\end{tabular} \\ \hline
\multirow{2}{*}{CFAR}       & 2.5 & 1.22   & 30.00 & 0.96 & 0.33 \\
                            & 10  & 2.42   & 28.14 & 0.96  & 0.11 \\ \hline
\multirow{4}{*}{Percentile} & 0.1 & 0.12   & 28.08 & 0.94 & 0.00 \\
                            & 1   & 1.11  & 31.66 & 0.96 & \textbf{0.48} \\
                            & 5   & 4.46 & \textbf{34.43} & \textbf{0.98} & 0.22 \\
                            & 10  & 8.17 & 30.00 & 0.96 & 0.05 \\ \hline \hline
\end{tabular}

\end{table}

Tab. \ref{tab:result} summarizes the tensor generation performance of the 4DR P2T model on 4D Radar point cloud data generated by various methods, evaluated using PSNR, SSIM, and DES. The average PSNR across all methods is 30.39dB—exceeding the 20–25 dB threshold commonly considered acceptable in wireless communication quality \cite{PSNR_1, PSNR_2}—indicating that the generated tensor data is of sufficiently high performance. As shown in Fig. \ref{fig3.result}, the percentile 5\% method achieves the best tensor generation performance, with a PSNR of 34.43dB and SSIM of 0.98. Meanwhile, the percentile 1\% method attains the highest DES value of 0.48, while also demonstrating superior point generation ability while reducing data volume, making it well-suited for deep learning model training.

\section{CONCLUSIONS} \label{sec:conclusion}

This study introduces the 4DR P2T model, which generates tensor data from 4D Radar point cloud data to address the limitation of inadequate spatial characteristic capture when CFAR is applied to 4D Radar data. By leveraging a cGAN-based architecture, our model effectively generates tensor data, as demonstrated by an average PSNR of 30.39dB and SSIM of 0.96. In addition, our comparative experiments show that the percentile 5\% method yields the best tensor generation performance,  while the percentile 1\% method offers an optimal balance between data volume reduction and performance, making it well-suited for deep learning training.

Future research will extend the 4DR P2T model to accommodate unpaired data, enabling tensor generation even for datasets lacking original tensor data.  Additionally, Doppler information will be incorporated to further enhance object representation. These advancements aim to improve the preservation of critical object features, enhance sensor fusion, and ultimately strengthen perception capabilities in autonomous driving systems.

\addtolength{\textheight}{-12cm}   % This command serves to balance the column lengths
                                  % on the last page of the document manually. It shortens
                                  % the textheight of the last page by a suitable amount.
                                  % This command does not take effect until the next page
                                  % so it should come on the page before the last. Make
                                  % sure that you do not shorten the textheight too much.

%%%%%%%%%%%%%%%%%%%%%%%%%%%%%%%%%%%%%%%%%%%%%%%%%%%%%%%%%%%%%%%%%%%%%%%%%%%%%%%%



%%%%%%%%%%%%%%%%%%%%%%%%%%%%%%%%%%%%%%%%%%%%%%%%%%%%%%%%%%%%%%%%%%%%%%%%%%%%%%%%



%%%%%%%%%%%%%%%%%%%%%%%%%%%%%%%%%%%%%%%%%%%%%%%%%%%%%%%%%%%%%%%%%%%%%%%%%%%%%%%%
\section*{ACKNOWLEDGMENT}

This work was supported by the National Research Foundation of Korea (NRF) grant funded by the Korea government (MSIT) (No. 2021R1A2C3008370).



%%%%%%%%%%%%%%%%%%%%%%%%%%%%%%%%%%%%%%%%%%%%%%%%%%%%%%%%%%%%%%%%%%%%%%%%%%%%%%%%

\bibliographystyle{unsrt}
\bibliography{ref}






\end{document}



\end{document}
