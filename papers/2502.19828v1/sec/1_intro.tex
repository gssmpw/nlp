% \section{Introduction}
% \label{sec:intro}


% Contrastive Language-Image Pre-training (CLIP), introduced by OpenAI in 2021, has emerged as a pivotal development in the field of Vision-Language Models (VLMs). CLIP's innovative approach involves training neural networks on a diverse dataset of image-text pairs sourced from the internet, enabling the model to learn robust visual concepts aligned with natural language descriptions. This pre-training strategy allows CLIP to perform zero-shot learning tasks, demonstrating remarkable versatility across various visual recognition challenges without the need for task-specific fine-tuning. The success of CLIP has sparked a new wave of research in multimodal learning, influencing the design of subsequent VLMs and pushing the boundaries of what's possible in image understanding and cross-modal reasoning. Its ability to bridge the gap between visual and textual modalities has made it a cornerstone in developing more advanced and flexible AI systems capable of understanding and generating content across multiple domains.
% % %-------------------------------------------------------------------------
% \section{Introduction}
% \label{sec:intro}
% % Contrastive Language-Image Pre-training (CLIP), introduced by OpenAI in 2021, has emerged as a pivotal development in the field of Vision-Language Models (VLMs). CLIP's innovative approach involves training neural networks on a diverse dataset of image-text pairs sourced from the internet, enabling the model to learn robust visual concepts aligned with natural language descriptions. This pre-training strategy allows CLIP to perform zero-shot learning tasks, demonstrating remarkable versatility across various visual recognition challenges without the need for task-specific fine-tuning. The success of CLIP has sparked a new wave of research in multimodal learning, influencing the design of subsequent VLMs and pushing the boundaries of what's possible in image understanding and cross-modal reasoning. Its ability to bridge the gap between visual and textual modalities has made it a cornerstone in developing more advanced and flexible AI systems capable of understanding and generating content across multiple domains.
% Contrastive Language-Image Pre-training (CLIP) models have demonstrated impressive capabilities in bridging visual and textual modalities. However, their performance in complex multi-object scenarios remains an area of active research. While CLIP excels at zero-shot classification tasks, recent studies have highlighted limitations when dealing with images containing multiple objects of varying sizes and positions (Smith et al., 2023; Jones and Lee, 2024).


% However, despite its groundbreaking capabilities, CLIP's performance in multi-object scenarios remains an area ripe for investigation. As real-world visual scenes often contain multiple objects with varying sizes, positions, and complexities, understanding CLIP's behavior in these contexts is crucial for assessing its practical applicability and limitations.

% This paper aims to conduct a comprehensive analysis of CLIP's performance in multi-object scenarios through a series of controlled high-resolution studies. Our research focuses on several key aspects:

% \begin{enumerate}
%     \item Evaluating CLIP's image encoder performance in scenarios with multiple objects of varying sizes and positions.
%     \item Analyzing CLIP's text encoder behavior when processing descriptions of multi-object scenes.
%     \item Investigating potential biases in both the image and text encoders that may affect overall model performance.
%     \item Examining the origins of these biases and their implications for CLIP's training process and real-world applications.
% \end{enumerate}

% By delving into these areas, we seek to provide valuable insights into the strengths and limitations of CLIP in complex visual environments. Our findings aim to contribute to the ongoing development of more robust and versatile VLMs, ultimately advancing the field towards AI systems that can more accurately interpret and reason about the multi-faceted nature of real-world visual and textual information.

% % The rest of this paper is organized as follows: Section \ref{sec:methodology} describes our experimental methodology, including dataset design and experimental setup. Section \ref{sec:results} presents the results of our analysis on both the text and image encoders. In Section \ref{sec:bias_origin}, we explore our hypothesis regarding the origin of observed biases. Section \ref{sec:impact} discusses the impact of these encoder biases on CLIP's overall performance. Finally, Section \ref{sec:conclusion} concludes the paper and suggests directions for future research.

% \label{sec:impact}

\section{Introduction}
\label{sec:intro}
Contrastive Language-Image Pre-training (CLIP), introduced by OpenAI in 2021 \cite{radford2021learningtransferablevisualmodels}, represents a significant advancement in Vision-Language Models (VLMs). This innovative approach has demonstrated exceptional performance in zero-shot classification tasks, setting a new benchmark for the integration of visual and linguistic data \cite{Cherti_2023,gadre2023datacompsearchgenerationmultimodal,schuhmann2021laion400mopendatasetclipfiltered,thrush2022winoground}. However, its efficacy in processing complex multi-object scenarios remains an active area of research and development \cite{chen2024multi,ye2024beaf,kil2024compbench,castro2024clove,zarei2024understanding,tong2024eyes}.

% While CLIP's capabilities are impressive, recent studies have identified notable limitations in its handling of images containing multiple objects 
While CLIP's capabilities are impressive, recent studies have identified notable limitations in its handling of scenarios involving images with multiple objects and their corresponding captions
\cite{Cherti_2023,yuksekgonul2023visionlanguagemodelsbehavelike,ma2023crepe}. These investigations have consistently highlighted multi-object scenarios as a particular challenge for the model. However, efforts to address these shortcomings have often proceeded without comprehensive evaluation and detailed analysis of the underlying causes, thereby necessitating a more thorough investigation to uncover the root causes and potential avenues for improving CLIP's performance in this critical domain.

\begin{figure} [t]
    \centering
    \includegraphics[width=\columnwidth]{sec/images/first_figv2.pdf}
    \caption{CLIP performance on multi-object image-caption matching. Left: Correct vs. incorrect captions with large object first. Right: Incorrect vs. reordered correct captions (large object last). Results show CLIP's bias for captions starting with larger objects, reducing accuracy when this order is altered.}
    \label{fig:first_fig}
\end{figure}

In our work, we tried to fill this gap by conducting a comprehensive analysis of CLIP's performance in multi-object scenarios through a series of controlled fine-grained studies. Our research focuses on several key aspects:

\begin{enumerate}
    \item Evaluating CLIP's performance using two synthetic datasets, SimCO and CompCO, designed specifically for controlled multi-object scenarios.
    \item Analyzing both CLIP's image encoder and text encoder biases when processing multi-object scenes and descriptions.
    \item Examining CLIP's biases in multi-object processing, their origins, and implications for image-caption matching and text-to-image generation tasks.
\end{enumerate}


% This research aims to offer insights into CLIP's strengths and limitations in complex visual environments, contributing to the development of more robust and versatile VLMs capable of accurately interpreting real-world visual and textual information.