%%%%%%%% ICML 2025 EXAMPLE LATEX SUBMISSION FILE %%%%%%%%%%%%%%%%%

\documentclass{article}

% Recommended, but optional, packages for figures and better typesetting:
\usepackage{microtype}
\usepackage{graphicx}
% \usepackage{subfigure}
\usepackage{booktabs} % for professional tables
\usepackage{multirow}
\usepackage{pifont}
\usepackage{makecell}
\usepackage{caption}
\usepackage{subcaption}
\usepackage[T1]{fontenc}
\usepackage{amssymb}
\usepackage{rotating}
\usepackage{enumitem}
\usepackage[skins,breakable]{tcolorbox}

\newcommand{\cmark}{\ding{51}} % Checkmark
\newcommand{\xmark}{\ding{55}} % Cross mark

\newcommand{\bx}{\boldsymbol{x}}
\newcommand{\by}{\boldsymbol{y}}
\newcommand{\bs}{\boldsymbol{s}}
\newcommand{\bz}{\boldsymbol{z}}
\newcommand{\bfm}{\boldsymbol{f}}


\newcommand{\zyc}[1]{\textcolor{red}{[Yichi: #1]}}

% hyperref makes hyperlinks in the resulting PDF.
% If your build breaks (sometimes temporarily if a hyperlink spans a page)
% please comment out the following usepackage line and replace
% \usepackage{icml2025} with \usepackage[nohyperref]{icml2025} above.
\usepackage{hyperref}


% Attempt to make hyperref and algorithmic work together better:
\newcommand{\theHalgorithm}{\arabic{algorithm}}

% Use the following line for the initial blind version submitted for review:
% \usepackage{icml2025}

% If accepted, instead use the following line for the camera-ready submission:
\usepackage[accepted]{icml2025}

% For theorems and such
\usepackage{amsmath}
\usepackage{amssymb}
\usepackage{mathtools}
\usepackage{amsthm}

% if you use cleveref..
\usepackage[capitalize,noabbrev]{cleveref}

%%%%%%%%%%%%%%%%%%%%%%%%%%%%%%%%
% THEOREMS
%%%%%%%%%%%%%%%%%%%%%%%%%%%%%%%%
\theoremstyle{plain}
\newtheorem{theorem}{Theorem}[section]
\newtheorem{proposition}[theorem]{Proposition}
\newtheorem{lemma}[theorem]{Lemma}
\newtheorem{corollary}[theorem]{Corollary}
\theoremstyle{definition}
\newtheorem{definition}[theorem]{Definition}
\newtheorem{assumption}[theorem]{Assumption}
\theoremstyle{remark}
\newtheorem{remark}[theorem]{Remark}

\captionsetup[sub]{
  labelformat=simple
}
\renewcommand\thesubfigure{(\alph{subfigure})}

\newcommand{\yinpeng}[1]{{\color{red}{[Yinpeng: #1]}}}
\newcommand{\junz}[1]{{\color{blue}{[jz: #1]}}}
\newcommand{\sy}[1]{{\color{green}{[zsy: #1]}}}


% Todonotes is useful during development; simply uncomment the next line
%    and comment out the line below the next line to turn off comments
%\usepackage[disable,textsize=tiny]{todonotes}
\usepackage[textsize=tiny]{todonotes}


% The \icmltitle you define below is probably too long as a header.
% Therefore, a short form for the running title is supplied here:
% \icmltitlerunning{Safety Alignment with Internal Reasoning}

\begin{document}

\twocolumn[
\icmltitle{STAIR: Improving Safety Alignment with Introspective Reasoning}

% It is OKAY to include author information, even for blind
% submissions: the style file will automatically remove it for you
% unless you've provided the [accepted] option to the icml2025
% package.

% List of affiliations: The first argument should be a (short)
% identifier you will use later to specify author affiliations
% Academic affiliations should list Department, University, City, Region, Country
% Industry affiliations should list Company, City, Region, Country

% You can specify symbols, otherwise they are numbered in order.
% Ideally, you should not use this facility. Affiliations will be numbered
% in order of appearance and this is the preferred way.
\icmlsetsymbol{equal}{*}

\begin{icmlauthorlist}
%\icmlauthor{}{tsinghua}
%\icmlauthor{}{beihang}
%\icmlauthor{}{realai}
\icmlauthor{Yichi Zhang}{equal,tsinghua,realai}
\icmlauthor{Siyuan Zhang}{equal,tsinghua}
\icmlauthor{Yao Huang}{beihang,realai}
\icmlauthor{Zeyu Xia}{tsinghua}
\icmlauthor{Zhengwei Fang}{tsinghua}
\icmlauthor{Xiao Yang}{tsinghua}
\icmlauthor{Ranjie Duan}{tsinghua,alibaba}
%\icmlauthor{}{sch}
\icmlauthor{Dong Yan}{baichuan}
\icmlauthor{Yinpeng Dong}{tsinghua,realai}
\icmlauthor{Jun Zhu}{tsinghua,realai}
%\icmlauthor{}{sch}
%\icmlauthor{}{sch}
\end{icmlauthorlist}

\icmlaffiliation{tsinghua}{Tsinghua University}
\icmlaffiliation{realai}{RealAI}
\icmlaffiliation{beihang}{Beihang University}
\icmlaffiliation{alibaba}{Alibaba Group}
\icmlaffiliation{baichuan}{Baichuan Inc}

\icmlcorrespondingauthor{Yichi Zhang}{zyc22@mails.tsinghua.edu.cn}
\icmlcorrespondingauthor{Yinpeng Dong}{dongyinpeng@mail.tsinghua.edu.cn}
\icmlcorrespondingauthor{Jun Zhu}{dcszj@mail.tsinghua.edu.cn}

% You may provide any keywords that you
% find helpful for describing your paper; these are used to populate
% the "keywords" metadata in the PDF but will not be shown in the document
\icmlkeywords{Machine Learning, ICML}

\vskip 0.3in
]

% this must go after the closing bracket ] following \twocolumn[ ...

% This command actually creates the footnote in the first column
% listing the affiliations and the copyright notice.
% The command takes one argument, which is text to display at the start of the footnote.
% The \icmlEqualContribution command is standard text for equal contribution.
% Remove it (just {}) if you do not need this facility.

%\printAffiliationsAndNotice{}  % leave blank if no need to mention equal contribution
\printAffiliationsAndNotice{\icmlEqualContribution} % otherwise use the standard text.


\begin{abstract}


The choice of representation for geographic location significantly impacts the accuracy of models for a broad range of geospatial tasks, including fine-grained species classification, population density estimation, and biome classification. Recent works like SatCLIP and GeoCLIP learn such representations by contrastively aligning geolocation with co-located images. While these methods work exceptionally well, in this paper, we posit that the current training strategies fail to fully capture the important visual features. We provide an information theoretic perspective on why the resulting embeddings from these methods discard crucial visual information that is important for many downstream tasks. To solve this problem, we propose a novel retrieval-augmented strategy called RANGE. We build our method on the intuition that the visual features of a location can be estimated by combining the visual features from multiple similar-looking locations. We evaluate our method across a wide variety of tasks. Our results show that RANGE outperforms the existing state-of-the-art models with significant margins in most tasks. We show gains of up to 13.1\% on classification tasks and 0.145 $R^2$ on regression tasks. All our code and models will be made available at: \href{https://github.com/mvrl/RANGE}{https://github.com/mvrl/RANGE}.

\end{abstract}



\section{Introduction}
Backdoor attacks pose a concealed yet profound security risk to machine learning (ML) models, for which the adversaries can inject a stealth backdoor into the model during training, enabling them to illicitly control the model's output upon encountering predefined inputs. These attacks can even occur without the knowledge of developers or end-users, thereby undermining the trust in ML systems. As ML becomes more deeply embedded in critical sectors like finance, healthcare, and autonomous driving \citep{he2016deep, liu2020computing, tournier2019mrtrix3, adjabi2020past}, the potential damage from backdoor attacks grows, underscoring the emergency for developing robust defense mechanisms against backdoor attacks.

To address the threat of backdoor attacks, researchers have developed a variety of strategies \cite{liu2018fine,wu2021adversarial,wang2019neural,zeng2022adversarial,zhu2023neural,Zhu_2023_ICCV, wei2024shared,wei2024d3}, aimed at purifying backdoors within victim models. These methods are designed to integrate with current deployment workflows seamlessly and have demonstrated significant success in mitigating the effects of backdoor triggers \cite{wubackdoorbench, wu2023defenses, wu2024backdoorbench,dunnett2024countering}.  However, most state-of-the-art (SOTA) backdoor purification methods operate under the assumption that a small clean dataset, often referred to as \textbf{auxiliary dataset}, is available for purification. Such an assumption poses practical challenges, especially in scenarios where data is scarce. To tackle this challenge, efforts have been made to reduce the size of the required auxiliary dataset~\cite{chai2022oneshot,li2023reconstructive, Zhu_2023_ICCV} and even explore dataset-free purification techniques~\cite{zheng2022data,hong2023revisiting,lin2024fusing}. Although these approaches offer some improvements, recent evaluations \cite{dunnett2024countering, wu2024backdoorbench} continue to highlight the importance of sufficient auxiliary data for achieving robust defenses against backdoor attacks.

While significant progress has been made in reducing the size of auxiliary datasets, an equally critical yet underexplored question remains: \emph{how does the nature of the auxiliary dataset affect purification effectiveness?} In  real-world  applications, auxiliary datasets can vary widely, encompassing in-distribution data, synthetic data, or external data from different sources. Understanding how each type of auxiliary dataset influences the purification effectiveness is vital for selecting or constructing the most suitable auxiliary dataset and the corresponding technique. For instance, when multiple datasets are available, understanding how different datasets contribute to purification can guide defenders in selecting or crafting the most appropriate dataset. Conversely, when only limited auxiliary data is accessible, knowing which purification technique works best under those constraints is critical. Therefore, there is an urgent need for a thorough investigation into the impact of auxiliary datasets on purification effectiveness to guide defenders in  enhancing the security of ML systems. 

In this paper, we systematically investigate the critical role of auxiliary datasets in backdoor purification, aiming to bridge the gap between idealized and practical purification scenarios.  Specifically, we first construct a diverse set of auxiliary datasets to emulate real-world conditions, as summarized in Table~\ref{overall}. These datasets include in-distribution data, synthetic data, and external data from other sources. Through an evaluation of SOTA backdoor purification methods across these datasets, we uncover several critical insights: \textbf{1)} In-distribution datasets, particularly those carefully filtered from the original training data of the victim model, effectively preserve the model’s utility for its intended tasks but may fall short in eliminating backdoors. \textbf{2)} Incorporating OOD datasets can help the model forget backdoors but also bring the risk of forgetting critical learned knowledge, significantly degrading its overall performance. Building on these findings, we propose Guided Input Calibration (GIC), a novel technique that enhances backdoor purification by adaptively transforming auxiliary data to better align with the victim model’s learned representations. By leveraging the victim model itself to guide this transformation, GIC optimizes the purification process, striking a balance between preserving model utility and mitigating backdoor threats. Extensive experiments demonstrate that GIC significantly improves the effectiveness of backdoor purification across diverse auxiliary datasets, providing a practical and robust defense solution.

Our main contributions are threefold:
\textbf{1) Impact analysis of auxiliary datasets:} We take the \textbf{first step}  in systematically investigating how different types of auxiliary datasets influence backdoor purification effectiveness. Our findings provide novel insights and serve as a foundation for future research on optimizing dataset selection and construction for enhanced backdoor defense.
%
\textbf{2) Compilation and evaluation of diverse auxiliary datasets:}  We have compiled and rigorously evaluated a diverse set of auxiliary datasets using SOTA purification methods, making our datasets and code publicly available to facilitate and support future research on practical backdoor defense strategies.
%
\textbf{3) Introduction of GIC:} We introduce GIC, the \textbf{first} dedicated solution designed to align auxiliary datasets with the model’s learned representations, significantly enhancing backdoor mitigation across various dataset types. Our approach sets a new benchmark for practical and effective backdoor defense.




\section{Background}
% \begin{tcolorbox}[simplebox]
% We first formally define the problem and highlight its challenge. 
% Then we present an EM approach to address this challenge. 
% \end{tcolorbox}
% \vspace{-0.3cm}
% \subsection{Problem Statement }\label{sec_ps}

% Here’s a polished and enriched version of your problem formulation section, with improved clarity, precision, and academic tone:

% ---
\begin{figure}[t]
    \centering % Center the figure
    \includegraphics[width=\linewidth]{figs/example.pdf} % Include the figure
    \caption{\small \textbf{Example of Autonomous Code Integration.} \small We aim to enable LLMs to determine tool-usage strategies
based on their own capability boundaries. In the example, the model write code to solve the problem that demand special tricks, strategically bypassing its inherent limitations.} 
    \label{fig_example}
    \vspace{-0.2cm}
\end{figure}
\textbf{Problem Statement.} Modern tool-augmented language models address mathematical problems \( x_q \in \mathcal{X}_Q \) by generating step-by-step solutions that interleave natural language reasoning with executable Python code (Fig.~\ref{fig_example}). Formally, given a problem \( x_q \), a model \( \mathcal{M}_\theta \) iteratively constructs a solution \( y_a = \{y_1, \dots, y_T\} \) by sampling components \( y_t \sim p(y_t | y_{<t}, x_q) \), where \( y_{<t} \) encompasses both prior reasoning steps, code snippets and execution results \( \mathbf{e}_t \) from a Python interpreter. The process terminates upon generating an end token, and the solution is evaluated via a binary reward \( r(y_a,x_q) = \mathbb{I}(y_a \equiv y^*) \) indicating equivalence to the ground truth \( y^* \). The learning objective is formulated as:
\[
\max_{\theta} \mathbb{E}_{x_q \sim \mathcal{X}_Q} \left[r(y_a, x_q) \right]
\]

\noindent\textbf{Challenge and Motivation.} Developing autonomous code integration (AutoCode) strategies poses unique challenges, as optimal tool-usage behaviors must dynamically adapt to a model's intrinsic capabilities and problem-solving contexts. While traditional supervised fine-tuning (SFT) relies on imitation learning from expert demonstrations, this paradigm fundamentally limits the emergence of self-directed tool-usage strategies. Unfortunately, current math LLMs predominantly employ SFT to orchestrate tool integration~\citep{mammoth, tora, dsmath, htl}, their rigid adherence to predefined reasoning templates therefore struggles with the dynamic interplay between a model’s evolving problem-solving competencies and the adaptive tool-usage strategies required for diverse mathematical contexts.

Reinforcement learning (RL) offers a promising alternative by enabling trial-and-error discovery of autonomous behaviors. Recent work like DeepSeek-R1~\citep{dsr1} demonstrates RL's potential to enhance reasoning without expert demonstrations. However, we observe that standard RL methods (e.g., PPO~\cite{ppo}) suffer from a critical inefficiency (see Sec.~\ref{sec_ablation}): Their tendency to exploit local policy neighborhoods leads to insufficient exploration of the vast combinatorial space of code-integrated reasoning paths, especially when only given a terminal reward in mathematical problem-solving.

To bridge this gap, we draw inspiration from human metacognition -- the iterative process where learners refine tool-use strategies through deliberate exploration, outcome analysis, and belief updates. A novice might initially attempt manual root-finding via algebraic methods, observe computational bottlenecks or inaccuracies, and therefore prompting the usage of calculators. Through systematic reflection on these experiences, they internalize the contextual efficacy of external tools, gradually forming stable heuristics that balance reasoning with judicious tool invocation. 


To this end, \emph{our focus diverges from standard agentic tool-use frameworks~\citep{agentr}}, which merely prioritize successful tool execution. Instead, \emph{we aim to instill \emph{human-like metacognition} in LLMs, enabling them to (1) determine tool-usage based on their own capability boundaries (see the analysis in Sec.~\ref{sec_ablation}), and (2) dynamically adapt tool-usage strategies as their reasoning abilities evolve (via our EM framework).}
% For instance, while an LLM might solve a combinatorics problem via CoT alone, it should autonomously invoke code for eigenvalue calculations in linear algebra where symbolic computations are error-prone. Achieving this requires models to \emph{jointly optimize} their reasoning and tool-integration policies in a mutually reinforcing manner.


% Mirroring this metacognitive cycle, we propose an Expectation-Maximization (EM) framework that allows LLMs to develop AutoCode strategies via guided exploration (the E-step) and self-refinement (the M-step).


% \vspace{-0.3cm}
\section{Methodology}

Inspired by human metacognitive processes, we introduce an Expectation-Maximization (EM) framework that trains LLMs for autonomous code integration (AutoCode) through alternations (Fig.~\ref{fig_overview}):

\begin{enumerate}[leftmargin=0.5cm,topsep=1pt,itemsep=0pt,parsep=0pt]
    \item \emph{Guided Exploration (E-step):} Identifies high-potential code-integrated solutions by systematically probing the model's inherent capabilities.
\item \emph{Self-Refinement (M-step):} Optimizes the model's tool-usage strategy and chain-of-thought reasoning using curated trajectories from the E-step.
\end{enumerate}


\begin{figure*}[t]
    \centering
    \includegraphics[width=\linewidth]{figs/overview.pdf}
    \caption{\small \textbf{Method Overview.} \small (Left) shows an overview for the EM framework, which alternates between finding a reference strategy for guided exploration (E-step) and off-policy RL (M-step). (Right) shows the data curation for guided exploration. We generate \(K\) rollouts, estimate values of code-triggering decisions and subsample the initial data with sampling weights per Eq.~\ref{eq_sampling}.}
    \label{fig_overview}
\end{figure*}

\subsection{The EM Framework for AutoCode}

A central challenge in AutoCode lies in the code triggering decisions, represented by the binary decision \(c \in \{0, 1\}\).  While supervised fine-tuning (SFT) suffers from missing ground truth for these decisions, standard reinforcement learning (RL) struggles with the combinatorial explosion of code-integrated reasoning paths. Our innovation bridges these approaches through systematic exploration of both code-enabled (\(c=1\)) and non-code (\(c=0\)) solution paths, constructing reference decisions for policy optimization.

We formalize this idea within a maximum likelihood estimation (MLE) framework. Let \( P (r=1 | x_q;\theta\) denote the probability of generating a correct response to query \( x_q \) under model \(\mathcal{M}_\theta\). Our objective becomes:
\begin{align}
    \mathcal{J}_{\mathrm{MLE}}(\theta) \doteq \log P(r=1 | x_q; \theta) \label{eq_mle}
\end{align}
This likelihood depends on two latent factors: (1) the code triggering decision \(\pi_\theta(c | x_q)\) and (2) the solution generation process \(\pi_\theta(y_a | x_q, c)\). Here, for notation-wise clarity, we consider  code-triggering decision at a solution's beginning (\( c\) following \(x_q\) immediately). We show generalization to mid-reasoning code integration in Sec.~\ref{sec_impl}.

The EM framework provides a principled way to optimize this MLE objective in the presence of latent variables~\cite{prml}. We derive the evidence lower bound (ELBO): \( \mathcal{J}_{\mathrm{ELBO}}(s, \theta) \doteq \)
\begin{align}
    % \mathcal{J}_{\mathrm{MLE}}(\theta) &
    % \ge 
    \mathbb{E}_{s(c | x_q)}\left[\log \frac{\pi_\theta(c | x_q) \cdot P(r=1 | c, x_q; \theta)}{s(c | x_q)}\right] 
    % \\
     \label{eq_elbo}
\end{align}
where \(s(c | x_q)\) serves as a surrogate distribution approximating optimal code triggering strategies. It is also considered as the reference decisions for code integration. 

\noindent\textbf{E-step: Guided Exploration}  computes the reference strategy \(s(c | x_q)\) by maximizing the ELBO, equivalent to minimizing the KL-divergence: \( \max_s \mathcal{J}_{\mathrm{ELBO}}(s, \theta) = \)
\begin{align}
     - \mathrm{D_{KL}}\left(s(c | x_q) \| P(r=1, c | x_q; \theta)\right) \label{eq_estep}
\end{align}

The reference strategy \(s(c | x_q)\) thus approximates the posterior distribution over code-triggering decisions \(c\) that maximize correctness, i.e., \(P(r=1, c | x_q; \theta)\).  Intuitively, it guides exploration by prioritizing decisions with high potential: if decision \(c\) is more likely to lead to correct solutions, the reference strategy assigns higher probability mass to it, providing guidance for the subsequent RL procedure.

\noindent\textbf{M-step: Self-Refinement } updates the model parameters \(\theta\) through a composite objective:
\begin{multline}
\max_\theta \mathcal{J}_{\mathrm{ELBO}}(s, \theta) =\mathbb{E}_{\substack{c \sim s(c|x_q) \\ y_a \sim \pi_\theta(y_a|x_q, c)}} \Big[ r(x_q, y_a) \Big] \\- \mathcal{CE}\Big(s(c|x_q) \,\|\, \pi_\theta(c|x_q)\Big)\label{eq_mstep}
\end{multline}
The first term implements reward-maximizing policy gradient updates for solution generation, while while the second aligns native code triggering with reference strategies through cross-entropy minimization (see Fig.~\ref{fig_overview} for an illustration of the optimization). This dual optimization jointly enhances both tool-usage policies and reasoning capabilities.



\subsection{Practical Implementation}\label{sec_impl}
In the above EM framework, we alternate between finding a reference strategy \( s \) for code-triggering decisions  in the E-step, and perform reinforcement learning under the guidance from \( s \) in the M-step. We implement this framework through an iterative process of offline data curation and off-policy RL.

\noindent\textbf{Offline Data Curation.} We implement the E-step through Monte Carlo rollouts and subsampling. For each problem \(x_q\), we estimate the reference strategy as an energy distribution: 
\begin{equation}
    s^\ast(c | x_q)  = \frac{\exp\left(\alpha\cdot \pi_\theta(c | x_q) Q(x_q,c;\theta)\right)}{Z(x_q)}.\label{eq_sampling}
\end{equation}
where \( Q(x_q,c;\theta)\) estimates the expected value through \( K \) rollouts per decision, \(\pi_\theta(c|x_q) \) represents the model's current prior and the \( Z(x_q) \) is the partition function to ensure normalization. Intuitively, the strategy will assign higher probability mass to the decision \( c \) that has higher expected value \( Q(x_q,c;\theta)\) meanwhile balancing its intrinsic preference \( \pi_\theta(c|x_q)\). 

Our curation pipeline proceeds through: 
\begin{itemize}[leftmargin=0.5cm,topsep=1pt,itemsep=0pt,parsep=0pt]
\item Generate \(K\) rollouts for \(c=0\) (pure reasoning) and \(c=1\) (code integration), creating candidate dataset \(\mathcal{D}\).  
\item Compute \(Q(x_q,c)\) as the expected success rate across rollouts for each pair \((x_q,c)\).  
\item Subsample \(\mathcal{D}_{\text{train}}\) from \(\mathcal{D}\) using importance weights according to Eq.~\ref{eq_sampling}.  
\end{itemize}

To explicitly probe code-integrated solutions, we employ prefix-guided generation -- e.g., prepending prompts like \texttt{``Let’s first analyze the problem, then consider if python code could help''} -- to bias generations toward free-form code-reasoning patterns.

 This pipeline enables guided exploration by focusing on high-potential code-integrated trajectories identified by the reference strategy, contrasting with standard RL’s reliance on local policy neighborhoods. As demonstrated in Sec.~\ref{sec_ablation}, this strategic data curation significantly improves training efficiency by shaping the exploration space.





\noindent\textbf{Off-Policy RL.}
To mitigate distributional shifts caused by mismatches between offline data and the policy, we optimize a clipped off-policy RL objective. The refined M-step (Eq.~\ref{eq_mstep}) becomes:
\begin{multline}
    % \max_\theta 
    \underset{(x_q,y_a)}{\mathbb{E}}\left[
\text{clip}\left(\frac{\pi_\theta(y_a|x_q)}{\pi_{\text{ref}}(y_a|x_q)},1-\epsilon,1+\epsilon\right)\cdot A\right]
\\-\mathbb{E}_{(x_q,c)}\Big[\log \pi_\theta(c|x_q) \Big]\label{eq_finalm}
\end{multline}
where  \( (x_q, c, y_a) \) is sampled from the dataset \( \mathcal{D}_{\text{train}} \). The importance weight \(\frac{\pi_\theta(y_a|x_q)}{\pi_{\text{ref}}(y_a|x_q)}\) accounts for off-policy correction with PPO-like clipping. The advantage function \(A(x_q,y_a)\) is computed via query-wise reward normalization~\cite{ppo}. 

\noindent\textbf{Generalizing to Mid-Reasoning Code Integration.} Our method extends to mid-reasoning code integration by initiating Monte Carlo rollouts from partial solutions \((x_q, y_{<t})\). Notably, we observe emergence of mid-reasoning code triggers after initial warm-up with prefix-probed solutions. Thus, our implementation requires only two initial probing strategies: explicit prefix prompting for code integration and vanilla generation for pure reasoning, which jointly seed diverse mid-reasoning code usage in later iterations.



\section{RESULTS}

\begin{figure*}[t!]
    %\vspace{-0.5cm}
    \centering
    \includegraphics[width=1\linewidth]{images/SystemArchitecture_2.png}
    \caption{From a single user demonstration, the system extracts the desired task goal state with the help of user interaction to solve ambiguities. Using the created environment variation, the system computes a task execution plan to bring new environments into the goal state. It sends the plan to agents in the environment to execute.} \label{fig:system_architecture}
\end{figure*}

Figure \ref{fig:system_architecture} shows our proposed framework to define a task goal, i.e. an environment goals state, and to turn a given environment into this goal state. The system visually observes a task execution by a user and segments this \underline{single} demonstration into \skills. \actions\ and \skills\ are defined in \ref{ssec:actions_skills}. The demonstration changed one or several properties of entities in the environment; environment which is now in the goal state. This information and the differences in entity properties from the start and end environment states are used to represent the task goal state. More on that in \ref{ssec:exp_model_def}. To turn a new environment into the defined goal state, a planning problem must be solved. This entails computing the differences between the environment's current state and the goal state, finding \actions\ that solve these differences, instantiating \skills\ that implement the \actions\ in the environment, selecting the \skills\ to execute by minimizing a given metric, and finally, sending the \skills\ to the agents in the environment to execute. This process is detailed in \ref{ssec:exp_model_use}.

% To prove the usability of our model, we present experiments to create a new goal state and turn the current environment into an (already-defined) goal state.

\subsection{Actions and Skills}\label{ssec:actions_skills}
A change in the environment is modeled using \actions, i.e. \textbf{what} has happened, and \skills, i.e. \textbf{how} did the change happen \cite{conceptHierarchyGeriatronicsSummit24}. Like in STRIPS \cite{strips} and PDDL \cite{pddl}, we represent \actions\ by their effects on entity properties and \skills\ by their preconditions and effects. \actions\ do not need preconditions because they only describe the \textbf{what} part of a change, not which conditions must be satisfied to perform the change. Besides preconditions and effects, \skills\ have a list of checks that tell our system if the \skill\ is executed in the environment. These checks allow the creation of a \skill\ recognition program, like the one presented in \cite{conceptHierarchyGeriatronicsSummit24}.
%\todo{citation of Geriatronics summit paper or the journal/unsubmitted paper?}
Using the \skill\ recognition output, we capture the changes from a task demonstration.

A \skill\ is thus the physical enactment of an abstract \action\ in an environment. Hence, \skills\ are correlated with \actions\ via their effects. A \skill\ can have more effects than a corresponding \action. For example, the \skill\ of scooping jam from a jar with a spoon implements the \action\ of \textit{TransferringContents}, but it also \textit{Dirties} the spoon.

\subsection{How To Parameterize The Model}\label{ssec:exp_model_def}
Creating a new goal state should be easier than manually specifying all variations wanted from the goal state. Doing so requires programming knowledge, which should not be needed to define goal states. One can let the system, which knows how to represent goal states, question the user about the desired state of the environment. However, this tedious process requires many questions from the system, also leading to decreased system usability.

Therefore, our approach is to let the user turn a given environment into a desired goal state and analyze the differences between the initial and final environment state to create the goal state representation. This single demonstration highlights the entity property values that were not in the desired goal state before being changed by the user.

We capture the demonstration via an Intel Realsense 3D camera \cite{realsense}, analyze the human skeleton via the OpenPose human pose estimation method \cite{openpose}, and determine the 3d pose of objects with AprilTag markers \cite{aprilTag}.

One demonstration contains the initial environment, not in the task goal state, and the final environment, in the goal state. The final environment state alone is not enough to create the environment variation. Thus, additional questions, guided by the differences between the two environment values, are posed by the system to the user to determine the desired variation in the environment state.

In a demonstration in which the user pours milk into a bowl, as shown in the top of Figure \ref{fig:system_architecture}, the initial question posed to the user is which entities that have changed properties are relevant for the goal state. If the goal state is to have more milk in the bowl, the milk carton is irrelevant; it is a means to achieve the goal state but not relevant to the goal itself. The bowl is thus selected as a relevant entity. 

Next, the list of relevant modified properties must also be determined for each relevant entity. It could have happened that during pouring of the milk into the bowl, the bowl's location also changed, e.g. touched accidentally by the user. Thus, not all modified properties could be relevant to the task. After selecting the relevant properties, the system knows from the knowledge base \cite{conceptHierarchyGeriatronicsSummit24} their \textit{ValueDomain} and the list of implemented \textbf{variations} for that \textit{ValueDomain}. Thus, the user parametrizes a selected \textbf{variation} from the list: choosing either a fixed value, a \textit{ValueDomain}-specific \textbf{RangeVariation} that must be parametrized, a conjunction or disjunction of \textbf{RangeVariations}, or the whole \textit{ValueDomain}.

In the example above, the user chooses the \textit{contentLevel} property as relevant. The system knows this property's defined set of values: a non-negative real number, and the possible range variation types: an open interval, a closed interval, an open-closed or closed-open interval, an intersection or union of intervals, etc. The user chooses a closed interval of $[0.28, 0.32]$ around the final \textit{contentLevel} value of $0.3L$. The user also specifies a variation for the entity's concept. It is generalized from that specific bowl instance to a \textit{LiquidContainer}.

After each modified property of each entity has a represented \textbf{variation}, the system automatically collects the entities into a variation of type $A$, see \ref{ssec:variations}, which is the assigned \textbf{variation} for the collection of entities in the environment.

Thus, the environment variation is determined in $\mathcal{O}\left(n\times m \times p\right)$ questions to the user, where $n$ is the number of entities in the environment, $m$ is the maximal number of properties that an entity can have, and $p$ is the maximal number of parameters that a \textbf{RangeVariation} needs to be represented. In the example above, $10$ questions were necessary to determine the task goal state shown in Figure \ref{fig:system_architecture} of a \textit{LiquidContainer} with \textit{contentLevel} between $0.28$ and $0.32L$. Figure \ref{fig:task_goal_state} shows the internal JSON-like representation of the goal state as the environment variation.
% 1 question which entities are relevant -> just bowl
% 1 question which properties are relevant -> contentLevel and concept
% 1 question about concept values being the same; should create variation?
% 1 question which ConceptValue-variation to select -> ConceptValue in Environment
% 1 question: which generalized concept?
% 1 question -> add other range-variation
% 1 question which Number-variation to select -> Interval
% 1 question: min-bound?
% 1 question: max-bound?
% 1 question -> add other range-variation

\begin{figure}[t!]
    %\vspace{-0.5cm}
    \centering
    \includegraphics[width=1\linewidth]{images/TaskDefinition_7.png}
    \caption{The goal state is a \textbf{RangeVariation} of the environment, of type EnvironmentDataRangeEntityVariation, which contains a \textbf{variation} of entities. This sub-variation is a \textbf{RangeVariation} of type MapRangeInstanceSubset (\textbf{variation} of type $A$, see \ref{ssec:variations}) and contains one instance \textbf{RangeVariation} of type InstanceRangePropertiesVariation. It defines the instance's concept \textbf{RangeVariation}, a \textit{LiquidContainer} to be found in the environment, and the \textit{contentLevel} property \textbf{RangeVariation}, the closed interval $\left[0.28, 0.32\right]$.} \label{fig:task_goal_state}
\end{figure}

\subsection{How To Use The Model}\label{ssec:exp_model_use}
Assuming the representation of a task's goal state is given, i.e. an environment variation, we detail our procedure (see Figure \ref{fig:experiment_description}) to turn the current environment into the goal state.

First, a Comparison between the environment and the goal variation is computed. This leads, as described in \ref{ssec:comparisons}, to a list of reasons why the environment is not in the variation. These reasons, i.e. differences $\delta$ of concept properties $p$, must be fixed to turn the environment into the goal state.

% Computing the differences between an EnvironmentData and an EnvironmentData-Variation, that has a Collection-Variation of type $A$, see \ref{ssec:variations}, is done via a maximal matching algorithm, where an edge between an entity $e$ an an entity variation $v_e$ means $e \in v_e$. 
For an EnvironmentData-Variation $v_{env}$ that defines a Collection-RangeVariation of type $A$, see \ref{ssec:variations}, computing the Comparison between an EnvironmentData $env$ and this target $v_{env}$ leads to a list of reasons for each entity $e_{env}$ in the entity collection of $env$, why $e_{env} \not\in v, \forall v \in A$. This can be seen in Figure \ref{fig:experiment_description}, where for each entity of \textit{LiquidContainer} concept in the environment, there is a list of differences, i.e. Comparisons, created for why the respective entity does not match the defined variation on the top-right.

\begin{figure}[t!]
    %\vspace{-0.5cm}
    \centering
    \includegraphics[width=1\linewidth]{images/Experiment_DescriptionUsingVariations_2.png}
    \caption{The procedure to turn an environment into its goal state is divided into 5 steps: computing differences, finding abstract solutions (i.e. \actions), computing practical solutions for the abstract ones (i.e. \actions\ $\rightarrow$ \skills), selecting the best practical solution, and executing the solution.} \label{fig:experiment_description}
\end{figure}
The second step of the procedure is to turn the list of differences into a list of \actions\ that can fix them. In notation, \action\ $A_x$ solves a difference in the concept property $p_x$. The system knows which properties \actions\ modify by analyzing the definition of their effects. Thus, \actions\ are created (parametrized) to fix the differences in entity properties.

% Because multiple instances can fit the instance variation, the third step is to match instances with the variations. Our matching optimization criterion is to minimize the amount of \textit{Actions} needed to fix the instances' property differences. \todo{continue!}

In the third step, each \action\ $A_x$ is converted into an execution plan $P_x$ that implements solving the difference $\delta_{p_x}$ in the environment. It is also possible that there is no possibility to implement the \action\ $A_x$ in the environment; this is represented as an execution plan $P_x = \emptyset$. An execution plan $P_x$ is otherwise, in its simplest form, a set of \skill\ alternatives $\left\{S_y\right\}$, where the \skill\ $S_y$ implements the \action\ $A_x$. There is the case to consider that the \skill\ $S_y$ has preconditions that are not met. And so, before executing the skill $S_y$, a different execution plan $P_{S_y}$ has to be computed and executed to allow the \skill\ $S_y$ to solve the property difference $\delta_{p_x}$. It is also possible that one single \skill\  $S_y$ is not enough to implement the \action\ $A_x$. Consider the case where the environment contains three cups with $0.1L$ of water, and the goal is to have one cup with $0.3L$ of content. One single \textit{Pouring} \skill\ is not enough to fulfill the goal; two \textit{Pouring} \skills\ must be executed. Thus, in the most general form, an execution plan $P_x = \left[\left\{ S_{iy}, P_{S_{iy}} \right\}_i\right]$ is a list of skill alternatives $\left\{ S_{iy}, P_{S_{iy}} \right\}_i$, that possibly contain other execution plans $P_{S_{iy}}$ to solve the skill's preconditions.

Our procedure to parameterize the \skills\ $S_y$ that implement the \action\ $A_x$ is a custom solution for each property $p_x$. One could backtrack through all possible parameter values of all possible skills to create a general solution that works for all properties. Another idea is to invert \skill\ effects and thus guide the \skill\ parameter search from the target variation to the value. However, both approaches would be computationally intense and would not create execution plans in a reasonable time. 
% reinforcement learning with policy for each property

The procedure to solve an entity $e$'s \underline{contentLevel} property difference searches for other \textit{Container} object instances in the environment, sorts them according to their content volume, and iterates through them in ascending order if $e.contentLevel \le target.contentLevel$; otherwise, in descending order. If a \skill\ $S$ can be executed with the two objects, that reduces the difference between $e.contentLevel$ and $target.contentLevel$, the \skill\ is added to the execution plan. If, after checking all objects, $e.contentLevel \not\in target.contentLevel$, there is no solution to solve this property difference.

Thus, the result of the third step is an execution plan $P_x$ for each entity property difference.

Fourth, after having the execution plans $P_x$ per entity-variation and entity, a \underline{solution selector} scores all solutions according to defined metrics and then, via a maximal matching algorithm, selects the solutions to execute to satisfy all variations of the Collection-RangeVariation of type $A$. The edges in the maximal matching have the cost of the solution score. For this paper, the scoring metric by the \underline{solution selector} is the number of steps of the execution plan.

The fifth and final step is to pass the execution plan to the agent(s) to execute in the environment. Figure \ref{fig:data_flow} presents the flow of data through the five steps.
We have used the Franka Emika Panda robot in CoppeliaSim \cite{coppeliaSim} to perform the computed execution plan.
% Note that the approach is independent of the used robot; only when instantiating \skills\ must the robot's abilities, manipulability region, and workspace be considered. How the \skills\ are executed in the environment is separated from the modeling of what must be done.

\begin{figure}[t!]
    % \vspace{-0.2cm}
    \centering
    \includegraphics[width=1\linewidth]{images/Experiment_DescriptionUsingVariations_DataFlow_2.png}
    \caption{Data flow when transforming an environment into a given goal state. $\Delta$ are differences of entity properties $p$, $A$ are \actions, $P$ is an execution plan and $S$ are \skills.} \label{fig:data_flow}
\end{figure}

The experiments aim to compute solution plans for solving the difference of the \textbf{contentLevel} property of \textit{Container} objects. For this, we consider the following criteria. $C1$: \textbf{variation} type = $\left\{\text{fixed},\text{interval},\text{interval union}\right\}$. $C2$: target relative to content = \{$\left\{t < cL \le cV \right\}$, $\left\{cL < t < cV \right\}$, $\left\{cL < t \ni cV \right\}$, $\left\{cL \le cV < t \right\}$\}, where $t$ is the \textbf{variation} value and $cL$ and $cV$ are the \textit{contentLevel} and \textit{contentVolume} properties respectively. $C3$: achievable in environment $ = \left\{\text{yes}, \text{no}\right\}$. Figure \ref{fig:experiment_table} presents planning results for different environments and the criteria described above. The lower table shows cases where the computed solution does not match the actual solution. This only happens when multiple instance variations are defined. The reason is that the implemented procedure to turn the list of differences into an execution plan treats each difference independently. Thus, dependencies between two variations are not accurately solved.

In the upper table of Figure \ref{fig:experiment_table}, there are two solutions for $C1.3$, $C2.3$, $C3.1$: one with the bowl $B$ as the instance in the \textbf{variation} $V1$, the other with $M$. The solution when $B$ is the matched instance has three steps: 1) pouring $0.1L$ from $M$ into $B$, 2) pouring $0.1L$ from $C1$ into $B$, and, finally, 3) pouring  $0.02L$ from $C2$ into $B$. This plan is sent to the robot in simulation and is executed as shown in Figure \ref{fig:robot_plan_execution}.

% \begin{figure}[t!]
%     % \vspace{-0.2cm}
%     \centering
%     \includegraphics[width=1\linewidth]{images/Experiment_Table_1Variation_compressed.png}
%     \caption{$B$ is a bowl with $0.5L$ \textit{contentVolume}, $M$ is a milk carton with $1.0L$ \textit{contentVolume}, $C1$ and $C2$ are cups with $0.3L$ \textit{contentVolume} each. Times, in seconds, averaged across 10 runs. Criteria $C2.4$ and $C3.1$ are mutually exclusive (a solution does not exist to let a container have more \textit{contentLevel} than its \textit{contentVolume}); thus, they are not included in the table.} \label{fig:experiment_table}
% \end{figure}
\begin{figure}[t!]
    % \vspace{-0.2cm}
    \centering
    \includegraphics[width=1\linewidth]{images/Experiment_Table_Results.png}
    \caption{$B$ is a bowl with $0.5L$ \textit{contentVolume}, $M$ is a milk carton with $1.0L$ \textit{contentVolume}, $C1$ and $C2$ are cups with $0.3L$ \textit{contentVolume} each. Times, in seconds, averaged across 10 runs. Criteria $C2.4$ and $C3.1$ are mutually exclusive (a solution does not exist to let a container have more \textit{contentLevel} than its \textit{contentVolume}); thus, they are not included in the upper table. The lower table presents results for open intervals and multiple variations in the environment.} \label{fig:experiment_table}
\end{figure}

\begin{figure}[t!]
    %\vspace{-0.1cm}
    \centering
    \includegraphics[width=1\linewidth]{images/Robot_PouringInBowl_M_PC1_PC2.png}
    \caption{Robot executing plan to bring $B$, the bowl, into the goal state. Because no liquids were simulated, the pouring amount was associated with the pouring time via: $t_{pour} = 10 * amount_{pour}$.} \label{fig:robot_plan_execution}
\end{figure}

\section{Future Work}
\label{section:discussion}

%  implement feedforward in GenAI systems across all contexts and applications, making it
% Our goal is to make feedforward a fundamental design component in all GenAI systems. While this paper demonstrates several feedforward designs for GenAI, we must further consider how we can effectively design and implement feedforward for GenAI systems in all applications.

Our goal is to establish feedforward as a fundamental design component in all GenAI systems. While this paper presents several feedforward designs, we envision a more comprehensive design space to represent and guide feedforward design across all GenAI applications. Based on our four prototypes, we identify three potential design dimensions for GenAI feedforward: representation, level of detail, and manipulability.

First, our examples implemented feedforward representations in the form of outlines, minimaps, lists of operations, example phrases, and multiple cursors. We may identify categories of representations that help distinguish which feedforward representations are more useful for different use cases. For instance, while a list of operations might inform users about the type of UI a GenAI system will generate, a wireframe might better communicate the structure and layout of the UI.

Second, we explored different ways to present varying levels of detail in feedforward. For example, the conversational UI displays an outline summarizing key topics, while the minimap omits textual details and instead presents blocks of paragraphs. This dimension aligns with previous research on feedforward \cite{bau2008octopocus}. A more rigorous investigation could explore optimal levels of detail for different types of feedforward representations in GenAI, as well as allowing users to define the level and type of detail themselves.

Lastly, we explored how users can manipulate feedforward content, either by revising their prompts or by directly resizing, repositioning, or selecting elements. We aim to investigate additional interaction techniques that enhance user control and engagement with feedforward designs.

Future work should expand on this preliminary design space by gathering, analyzing, and critiquing a broader range of GenAI systems to identify variations of feedforward across diverse contexts.

% Second, we must streamline the development of feedforward for GenAI.
% In our conceptual implementation of feedforward within a conversational LLM interface, the AI generates feedforward components when the user pauses typing.
% These components then serve as contextual input for the LLM upon submitting the prompt, helping it structure and guide the full response.
% While this approach is straightforward for a single conversational interface, this may not be the most cost-efficient approach for implementing feedforward. Perhaps other strategies may be to support feedforward algorithmically without the use of LLMs \cite{zhutian2024sketchgenerate}, but this might constrain the generative capabilities of the LLM.
% Furthermore, repeatedly implementing the same feedforward-GenAI pipeline across multiple applications, regardless of its implementation details, can become tedious.
% Future work could also provide a developer toolkit---perhaps a web-based UI library---that provides a catalog of feedforward components for developers to seamlessly integrate them into any GenAI web application.





% We want to work towards a design space for feedforward in genAI.
% From our prototype explorations in four applications: conversation UIs, document editors, malleable interfaces, and agent automations, we surface preliminary dimensions: 1) representation, 2) level of detail, 3) manipulability
% providing outline, minimap, list of operations, cut-outs of examples, these representations are forms of abstractions that are useful for feedforward in genai.
% level of detail is another, balancing the level of detail in feedforward is crucial in making sure users are able to understand what's ahead without being overwhelmed. Researchers have identified instances where there was cognitive burden in code completions and image generation. While balancing this can be a fine line, we believe the right level of detail can be identified for the many possible representations for feedforward in genAI systems.
% level of detail seemed to also be basically how far into the generation it goes, for instnace, the highest level of detail for a feedforward outline may be the full response itself.
% We explored various ways users can revise their prompt or steer AI upon finding out misalignment through feedforward. We explored that while users can revise their prompt to disambiguate their intents, they can also directly manipualte the feedforward content or specify their prompt by making AI-generated adjustments on the feedforward content.......

% we can take further steps to integrate feedforward into more GenAI systems.

% First, we can streamline the design process of feedforward for GenAI, we must solidify a comprehensive design space. From our prototype explorations in four applications: conversation UIs, document editors, malleable interfaces, and agent automations, we surfaced three preliminary dimensions: 1) representation, 2) level of detail, and 3) manipulability.
% We designed feedforward to be represented in an outline, minimap, list of operations, example phrases, and multiple cursors. We anticipate a rigorous investigation of feedforward will surface more representations suitable for each of their unique contexts.
% We also explored ways to present various levels of detail of feedforward. The conversational UI presents an outline detailing the key topics, while the minimap presents the topics organized into paragraphs.
% We aim to further investigate the ``right'' levels of detail across various feedforward representations to unsure users can anticipate AI's response without experiencing too much cognitive burden. 
% We also explored various ways users can revise their prompt or steer AI upon identifying misalignments between the user and AI. We explored how users can disambiguate their intents by either revising the text of their prompt and also directly manipulating the feedforward content via resizing, repositioning, and selecting.
% Future work should expand on this preliminary design space by gathering, analyzing, and critiquing a comprehensive collection of GenAI systems and identifying potential variations of feedforward across different contexts.
% We need to work on the design space
% We need to make it implementable

% Second, we can streamline the development of feedforward for GenAI.
% Our conceptual implementation of feedforward in our conversational LLM interface prompts AI to generate feedforward components once the user pauses their typing. These feedforward components then feed into the LLM as context to structure and guide the full response. 
% While this implementation strategy is straightforward for a single conversational interface, repeatedly applying the same structure across all applications can become tedious.
% Future work can build a toolkit for developers, perhaps in the form of a web-based UI library, that provides a catalog of feedforward components to integrate into any GenAI web application.


% \cite{dang2022ganslider, zhutian2024sketchgenerate}.
% While balancing this can be a fine line, we believe the right level of detail can be identified for the many possible representations for feedforward in genAI systems.
% level of detail seemed to also be basically how far into the generation it goes, for instnace, the highest level of detail for a feedforward outline may be the full response itself.


% What are other dimensions of feedforward? Maybe when the feedforward is presented
% Future work should rigorously investigate the various applications of GenAI potential uses of feedforward to develop a full design space. This design space could then be used to evaluate in various systems to surface design guidelines for future implementations

% If we do establish this, one potential avenue is to develop a framework, potentially a UI library that contains a catalog of feedforward components with various representations. These can plug into GenAI systems to streamline the process of designing and implementing feedforward for GenAI.


 

% \subsection{Towards a Design Space}

% Through iterative prototyping of feedforward designs in generative AI systems, we came up with three dimensions for designing feedforward in GenAI.

% \begin{enumerate}
%     \item Representation
%     \item Level of Detail
%     \item Manipulation
% \end{enumerate}

% A unique dimension of feedforward in GenAI is the ability and need to manipulate and revise the feedforward directly. Ultimately, the provided feedforward is not only a medium for AI to communicate its intents to the user, but also an opportunity for the user to communicate theirs back to AI.

% \subsection{How to support developing feedforward easily?}

% How do we make it easier to build feedforward in these interfaces?

% At least in web applications, we can build ui packages for developers to plug in into their system.

% We took the design route of making each feedforward representation modular with feedforward components. Developers could plug in certain feedforward components into their application. Communities would also be able to contribute other components to the package. 

% \subsection{Future Work}

% What are the kinds of feedforward information to always show? Length of response?

% What are kinds of feedforward information to show dynamically depending on context and task? Task specific feedforward given the prompt and conversation?

% What are the right representations for these kinds of feedforward?

\section{Discussion}

\subsection{Related Prior Work}

\textbf{Training Energy Based Models}.
A number of recent works aim to improve EBM training. \citet{zhu2023learning} uses a diffusion model to reduce the number of Langevin steps within the recovery likelihood approach of \cite{gao2020learning}. \cite{schroder2024energy} eliminates the need for MCMC and $\nabla_x$-computation of the energy during training by using a contrastive loss with forward noising process instead of MCMC, coined Energy Divergence (ED). ED is a promising alternative to E-DSM and has connections to score-matching, but, as shown in \Cref{tab:unconditional_gen}, it is not yet competitive, and  suffers from a bias by choice of noisy energy function.

\textbf{Composition with MCMC}. Similar to our work, \cite{du2023reduce} also uses energy-parameterized diffusion models but performs controlled generation with MCMC rather than SMC. SMC is known to suffer weight degeneracy high dimension, resulting in lack of diversity across particles, MCMC does not suffer from this, though requires additional non-parallel steps which is time consuming. The approaches are however complementary, and indeed one may perform MCMC after resampling steps to promote diversity.

\textbf{Sequential Monte Carlo in Diffusion Models.}
Many recent works use SMC within diffusion models for conditional generation, we detail the FKM formulations of these works in in \Cref{app:fkm_potentials}.

\cite{wu2024practical} uses twisted SMC with a classifier-guided proposal \citep{dhariwal2021diffusion} and potentials approximated with diffusion posterior sampling \citep{chungdiffusion}, which has been detailed as a FKM by concurrent work \citep{zhao2024conditional}. \cite{cardoso2024monte} and \cite{dou2024diffusion} tackle linear inverse problems where potentials have a closed form using Gaussian conjugacy. \cite{li2024derivative} perform SMC for both discrete and continuous diffusion models whereby potentials consist of a reward function applied to $\mathbb{E}[\bfX_0|x_t]$.

\cite{liu2024correcting} corrects conditional generations using an adversarially trained density ratio potential, and scales this to text-to-image models. 


\textbf{SMC for LLMs}. SMC is not only popular within diffusion models, but has been successful within large language models (LLMs) \citep{lew2023sequential, zhao2024probabilistic}. \cite{lew2023sequential} uses a FKM formulation with indicator based potential functions similar to as detailed in \Cref{sec:bounded}, and \cite{zhao2024probabilistic} discuss using SMC for text using potentials from reward functions or learning such potentials via contrastive twist learning, similar to contrastive learning for EBMs.

\subsection{Concurrent work} Since submission/ acceptance of our work \footnote{Submission October 2024}, there have been a number of relevant concurrent works. 

\textbf{FKM Interpretation}. \cite{singhal2025general} similarly to \cite{zhao2024conditional} and this work, detail sampling diffusion models in terms of KFM. \cite{singhal2025general} follow \cite{li2024derivative} in using reward functions based potentials but focus on text-to-image reward, and explore further heuristics such as or combining rewards via sum or max; and sampling $\bfX_0|x_t$ via nested diffusion \citep{elata2024nested} as input to their reward rather than using $\mathbb{E}[\bfX_0|x_t]$ as done in \cite{li2024derivative}.

\textbf{SMC for discrete diffusion}.  \cite{lee2025debiasing} use SMC for low temperature sampling for discrete diffusion models. \cite{xu2024energy} use a pretrained autoregressive likelihood model applied to samples $\bfX_0|x_t$ for a potential within discrete diffusion sampling.

\textbf{Composition}. \cite{skreta2024superposition} construct a cheap density estimator by simulating from an SDE, which can be computed at sampling time if using reverse diffusion solver, though it is not clear if this can be used in conjunction with resampling and Langevin corrector schemes. \cite{skreta2024superposition} then use this estimator to perform composition-type sampling, however their logical \textsc{AND} appears to differ from other more commonly used logical \textsc{AND} operations, in that it targets samples with equal probability between classes rather than generating both classes, e.g. "a CAT and a DOG" results in a cat/dog hybrid optical illusion rather than a separate cat and separate dog in one image. 

\cite{bradley2025mechanisms} explore composition more formally, establishing types of composition and cases where summing scores is sufficient without need for SMC correction as performed in this work or with MCMC correction from \citep{du2023reduce}.





\paragraph{Summary}
Our findings provide significant insights into the influence of correctness, explanations, and refinement on evaluation accuracy and user trust in AI-based planners. 
In particular, the findings are three-fold: 
(1) The \textbf{correctness} of the generated plans is the most significant factor that impacts the evaluation accuracy and user trust in the planners. As the PDDL solver is more capable of generating correct plans, it achieves the highest evaluation accuracy and trust. 
(2) The \textbf{explanation} component of the LLM planner improves evaluation accuracy, as LLM+Expl achieves higher accuracy than LLM alone. Despite this improvement, LLM+Expl minimally impacts user trust. However, alternative explanation methods may influence user trust differently from the manually generated explanations used in our approach.
% On the other hand, explanations may help refine the trust of the planner to a more appropriate level by indicating planner shortcomings.
(3) The \textbf{refinement} procedure in the LLM planner does not lead to a significant improvement in evaluation accuracy; however, it exhibits a positive influence on user trust that may indicate an overtrust in some situations.
% This finding is aligned with prior works showing that iterative refinements based on user feedback would increase user trust~\cite{kunkel2019let, sebo2019don}.
Finally, the propensity-to-trust analysis identifies correctness as the primary determinant of user trust, whereas explanations provided limited improvement in scenarios where the planner's accuracy is diminished.

% In conclusion, our results indicate that the planner's correctness is the dominant factor for both evaluation accuracy and user trust. Therefore, selecting high-quality training data and optimizing the training procedure of AI-based planners to improve planning correctness is the top priority. Once the AI planner achieves a similar correctness level to traditional graph-search planners, strengthening its capability to explain and refine plans will further improve user trust compared to traditional planners.

\paragraph{Future Research} Future steps in this research include expanding user studies with larger sample sizes to improve generalizability and including additional planning problems per session for a more comprehensive evaluation. Next, we will explore alternative methods for generating plan explanations beyond manual creation to identify approaches that more effectively enhance user trust. 
Additionally, we will examine user trust by employing multiple LLM-based planners with varying levels of planning accuracy to better understand the interplay between planning correctness and user trust. 
Furthermore, we aim to enable real-time user-planner interaction, allowing users to provide feedback and refine plans collaboratively, thereby fostering a more dynamic and user-centric planning process.


\newpage
\bibliography{example_paper}
\bibliographystyle{icml2025}


\newpage
\onecolumn
\appendix

\newtcolorbox{cvbox}[1][]{
    enhanced,
%   blanker, % <- removed as it suppresses box color and frame
    %leftupper=4cm,
    after skip=8mm,%   enlarge distance to the next box
    title=#1,
    breakable = true,
    fonttitle=\sffamily\bfseries,
    coltitle=black,
    colbacktitle=gray!10,   % <- defines background color in title
    titlerule= 0pt,         % <- sets rule underneath title 
    %fontupper=\sffamily,%
    %#1
    overlay={%
        \ifcase\tcbsegmentstate
        % 0 = Box contains only an upper part
        \or%
        % 1 = Box contains an upper and a lower part
        %\path[draw=red] (segmentation.west)--(frame.south east);
        \else%
        % 2 = Box contains only a lower part
        %\path[draw=red] (frame.north west)--(frame.south east);
        \fi%
    }
    colback = gray,         % <- defines background color in box
    colframe = black!75     % <- defines color of frame
    }


\section{Data Construction}

\subsection{Dataset Summary}
\label{sec:appendix_data}

We prepare a seed dataset $\mathcal{D}$ containing both safety and helpfulness data. It consists of 50k pairwise samples from three sources. For helpfulness data, we draw 25k samples from UltraFeedback~\cite{cui2024ultrafeedback}. Each sample originally has 5 potential responses with ratings and we take the one with the highest rating as ``chosen'' and the one with the lowest as ``rejected''. For safety data, we take 22k samples from PKU-SafeRLHF~\cite{ji2024pku}, which have responses with unsafe labels and are further filtered by GPT-4o to assure the prompts are truly toxic and harmful. We follow the common practice of proprietary LLMs that responses to harmful queries should contain clear refusal in at most one sentence instead of providing additional content and guide besides a brief apology~\cite{guan2024deliberative}. This make current positive annotations in PKU-SafeRLHF, which usually contain much relevant information, not directly usable. Therefore, we use GPT-4o to generate refusal answers for these prompts and substitute the original chosen responses with them. 

Further, to better address the complex scenario of jailbreak attack, we take 3k jailbreak prompts from JailbreakV-28k~\cite{luo2024jailbreakv}. As this dataset was originally proposed for benchmarks, we carefully decontaminate the red-teaming queries from those used for evaluation, e.g., AdvBench~\cite{zou2023universal}, and only sample prompts from the sources of GPT-Generate, Handcraft, and HH-RLHF~\cite{ganguli2022red}. Due to the lack of response annotations, we prompt GPT-4o to generate refusal answers as ``chosen'' ones and prompt the models to be trained for ``rejected'' answers.


When training with baselines on the whole dataset, we take the ``chosen'' responses as labels for SFT and use the pairwise data for preference learning (i.e., DPO, SACPO). As for our method and self-rewarding, we only use the prompts from the dataset. Compared with baseline methods like SFT and DPO, we do not use the jailbreak prompts with GPT-generated refusals to directly train the models.



\subsection{Structured CoT Data Synthesis}
\label{sec:appendix_cot}
To initialize the models with reasoning ability, we take 10k prompts each from PKU-SafeRLHF and UltraFeedback, and synthesize structured CoT answers. Specifically, we used GPT-4o to generate structured reasoning steps for these prompts. The generation prompt instructs GPT-4o to produce step-by-step reasoning following a specific format. After generation, we filter out responses with incorrect formatting and remove any responses that provide unsafe answers to safety-related queries, ensuring the quality and safety of our CoT training data. The final dataset consists of synthesized samples slightly fewer than 20k due to the filtering process. The generation prompt for instructing GPT-4o is as below:

% \vspace{8pt} 
\begin{cvbox}[\vspace{8pt}Instruction to Generate Structured Chain-of-Thought Answer]
You are a helpful assistant specializing in addressing both general queries and sensitive topics, including privacy, security, and ethics. Your primary responsibility is to provide thorough reasoning and structured planning to ensure your responses are both helpful and strictly adhere to safety and ethical guidelines.
\newline \newline
\#\#\# Instructions:
\newline \newline
1. **Problem Analysis**: Begin by analyzing the query to understand its context and devise a clear and structured roadmap for the reasoning process below. This step should outline how the problem will be addressed in subsequent reasoning steps, ensuring logical progression and coherence. During this step, you may identify whether the query poses any risks or ethical concerns. If the query involves sensitive or risky elements, the roadmap should focus on identifying and mitigating these aspects. For harmless queries, emphasize an effective and logical approach to solving the problem.
\newline \newline
2. **Reasoning**: Break down the query into detailed and systematic reasoning steps. Each step should build upon the roadmap provided in the Problem Analysis section. Pay close attention to the progression of the analysis—if potential risks or sensitivities are uncovered during this stage, pivot immediately to a safety-focused reasoning process. This includes reassessing the problem's implications and adjusting your steps to ensure a thorough evaluation of risks. For harmless queries, highlight logical planning and systematic problem-solving to provide the most effective solution.
\newline \newline
3. **Final Answer**: Based on your reasoning, provide a concise and direct conclusion using the format `<|Output|>Final Answer:<|/Output|>`. Reasoning is provided to help you better understand the steps involved, but the ultimate answer to the query must be presented here.
- If the query is unsafe or unethical, explicitly refuse assistance with clear language, e.g., "Sorry, I cannot assist with this." Avoid using repetitive responses by varying your phrasing while maintaining a professional and polite tone. For example, you might say: "I’m sorry, but I cannot help with that request," "Unfortunately, I am unable to fulfill this request as it may violate safety guidelines," or "I must decline, as this request does not align with ethical considerations." Ensure your response remains concise, clear, and respectful without engaging further with unsafe or inappropriate content.
\newline  
- If the query is safe and ethical, offer a well-structured, logical, and solution-oriented response that demonstrates your reasoning and problem-solving capabilities. Strengthen the clarity and completeness of the solution to provide the most effective assistance in harmless scenarios.
\newline \newline
\#\#\# Format Requirements:
\newline \newline
- Each step must use the following tokenized format:
\newline 
1. **Problem Analysis**: Encapsulate the analysis within <|Reasoning\_step|> Title: Problem Analysis:  <|/Reasoning\_step|> tags.
\newline \newline
2. **Reasoning**: Include multiple <|Reasoning\_step|> Title: Title\_name <|/Reasoning\_step|> sections as needed to thoroughly address the query.
\newline \newline
3. **Final Answer**: Provide the conclusion in the format: <|Output|>Final Answer: <|/Output|> .
\newline 
By adhering to these guidelines and referring to the above example, you will provide clear, accurate, and well-structured responses to questions involving sensitive or potentially unsafe topics while excelling in logical planning and reasoning for safe and harmless queries. Provide your reasoning steps directly without additional explanations. Begin your response with the special token `<|Reasoning\_step|>`. Following is the question:

\vspace{1em}
Question: \{prompt\}
\vspace{8pt} 
\end{cvbox}


\section{Self-Improvement with Safety-Informed MCTS}

\subsection{Derivation of Safety-Informed Reward}
\label{sec:appendix_derive}

Here, we present the proof for~\cref{theorem} in~\cref{sec:MCTS}, to derive a proper form for the safety-informed reward function. We first recall the three desired properties with intuitive explanations.
\begin{enumerate}
    \item \textbf{Safety as Priority}: Safe responses always get higher rewards than unsafe ones, regardless of their helpfulness.
    \begin{equation}
        \forall \bfm_1,\bfm_2, S(\bfm_1)>0> S(\bfm_2) \Rightarrow R(\bfm_1)>R(\bfm_2)
    \end{equation}
    \item \textbf{Dual Monotonicity of Helpfulness}: When the response is safe, it gets higher reward for better helpfulness; when it is unsafe, it gets lower reward for providing more helpful instructions towards the harmful intention.
    \begin{equation}
        \forall S>0 , \frac{\partial R}{\partial H} > 0\text{ and } \forall S<0, \frac{\partial R}{\partial H} < 0;
    \end{equation}
    \item \textbf{Degeneration to Single Objective}: If we only consider one dimension, we can set the reward function to have a constant shift from the original reward of that aspect. This will lead to the procedure degenerating to standard MCTS under the corresponding reward, i.e., given a partially constructed search tree, the result of selection is the same when all hyperparameters, e.g., seed, exploration parameter, are fixed.
    \begin{align}
        \exists\;C_1 \in [-1,1],\;s.t.\;\text{let }S\equiv C_1, \forall \bfm_1,\bfm_2, R(\bfm_1)-R(\bfm_2)=H(\bfm_1)-H(\bfm_2);\\
    \exists\;C_2 \in [-1,1],\;s.t.\;\text{let }H\equiv C_2, \forall \bfm_1,\bfm_2, R(\bfm_1)-R(\bfm_2)=S(\bfm_1)-S(\bfm_2).
    \end{align}
    
\end{enumerate}

\begin{theorem}
    Fix constants $C_1, C_2\in [-1,1],\;C_1\ne0$. Suppose $R:[-1,1]\times[-1,1]\rightarrow \mathbb{R}$ is twice-differentiable and satisfies $\frac{\partial R}{\partial H}=F(S)$, for some continuous function $F: [-1,1]\rightarrow \mathbb{R}$. Properties 2 and 3 of Dual Monotonicity of Helpfulness and Degeneration to Single Objective hold, if and only if
    \begin{equation}
    R(H,S)=F(S)\cdot H+S - C_2 \cdot F(S)+c,       
    \end{equation} with $F(0)=0, F(C_1)=1, \forall S>0, F(S)>0, \forall S<0, F(S)<0$ and $c$ as a constant.
\end{theorem}

\begin{proof} We show that the form of $R$ is the sufficient and necessary condition of Properties 2 and 3, given the assumptions. For notation simplicity, we use $H_1,H_2,S_1,S_2$ to denote the rewards for arbitrary final answers $f_1, f_2$.

\textbf{Sufficiency}

Assume $R(H,S)=F(S)\cdot H+S-C_2\cdot F(S)+c$ with $F(S)$ satisfying the stated conditions.

For Property 2, we can compute the partial derivative and show that
\begin{equation*}
    \frac{\partial R}{\partial H} = F(S) \begin{cases}
        > 0,\text{ when }S>0,\\
        <0,\text{ when }S<0.
    \end{cases}
\end{equation*}

For Property 3, let $S\equiv C_1$, we get
\begin{equation*}
    R(H_1,C_1)-R(H_2,C_1) = F(C_1) (H_1-H_2) = H_1-H_2.
\end{equation*}
let $H\equiv C_2$, we get
\begin{equation*}
    R(C_2,S_1)-R(C_2,S_2) = C_2(F(S_1)-F(S_2)) + S_1-S_2 -C_2(F(S_1)-F(S_2))= S_1-S_2.
\end{equation*}

\textbf{Necessity}

    Assume $R(H,S)$ satisfies Properties 2 and 3.

    Given the condition that $\frac{\partial R}{\partial H} = F(S)$, the function $R$ should follow the form by integral, 
    \begin{equation}
        R(H,S) = \int_0^H \frac{\partial R}{\partial H}dH+R(0,S) =F(S)\cdot H + G(S),
        \label{eq:reward}
    \end{equation}
    with $G(S)=R(0,S)$ as a continuous and differentiable function of $S$.

    Then, we apply the property of Degeneration to Single Objective, when $S\equiv C_1$,
    \begin{align*}
        R(H_1, C_1)-R(H_2,C_2) = F(C_1)& (H_1-H_2) = H_1-H_2, \forall H_1,H_2\in[-1,1]\\
        &\Rightarrow F(C_1) = 1,
    \end{align*}
    and when $H\equiv C_2$, 
    \begin{align*}
        R(C_2, S_1) - R(C_2, S_2) = C_2(F(S_1)& - F(S_2)) + G(S_1) - G(S_2) = S_1 - S_2, \forall S_1, S_2 \in[-1,1]\\ 
        &\Rightarrow C_2\cdot F'(S) - G'(S) = 1\\ 
        &\Rightarrow G'(S) = 1- C_2\cdot F'(S)\\ 
        &\Rightarrow G(S) = S-C_2\cdot F(S) + c, 
    \end{align*}
    with $c$ as a constant.

    Considering the property of Dual Monotonicity of Helpfulness, it is clear that $\frac{\partial R}{\partial H} = F(S)$ should satisfy
    \begin{equation*}
        F(S) >0, \forall S>0\text{ and }F(S)<0, \forall S<0.
    \end{equation*}
    Given the continuity of $F(S)$, $F(0) = 0$.

    Substituting $G(S)$ to~\cref{eq:reward}, we eventually get the family of $R$, following
    \begin{equation*}
    R(H,S)=F(S)\cdot H+S - C_2 \cdot F(S)+c,       
    \end{equation*} with $F(0)=0, F(C_1)=1, F(S)>0, \forall S>0$, $F(S)<0, \forall S<0$ and $c$ as a constant.
\end{proof}

\begin{corollary}
 Take $F(S)=S, C_1=1, C_2=-1, c=0$, $R(H,S)=2S+S\cdot H$ satisfies that for any $H_1, H_2,S_1,S_2\in[-1,1]$, when $S_1>0>S_2$, the inequality of $R(S_1,H_1)>R(S_2,H_2)$ holds.
\end{corollary}


\subsection{Implementation Details of Self-Improvement}
\label{sec:appendix_self-improvement}

Here, we introduce the implementation details of different components in the iterative self-improvement, including SI-MCTS, Self-Rewarding, and preference data construction.

\subsubsection{Safety-Informed MCTS} 
We design safety-informed reward to introduce dual information of both helpfulness and safety, without impacting the original effect of MCTS on a single dimension. Therefore, we mainly follow the standard MCTS procedure~\cite{vodopivec2017monte} guided by UCB1 algorithm~\cite{chang2005adaptive}. When traversing from the root node (i.e., prompt) to the leaf node, it selects the $i$-th node with the highest value of
\begin{equation}
    v_i + c\sqrt{\frac{\ln N_i}{n_i}},
\label{eq:UCB}
\end{equation}
where $v_i$ is the estimated value of safety-informed rewards, $n_i$ is the visited times of this node, $N_i$ is the visited times of its parent node, and $c$ is the exploration parameter that balances exploration and exploitation. 

The whole procedure of Safety-Informed MCTS follows~\cref{alg:SI MCTS}. In practice, we set exploration parameter $c=1.5$, search budget $n=200$, children number $m=4$. To generate child nodes and rollout to final answers, we set temperature as $1.2$, top-p as $0.9$ and top-k as $50$. We adjust these parameters when higher diversity is needed.



% Build MCT
\begin{algorithm}[ht]
   \caption{Safety-Informed MCTS}
   \label{alg:SI MCTS}
\begin{algorithmic}
   \STATE {\bfseries Input:} prompt set $\mathcal{D}_k$, safety reward function $S$, helpfulness reward function $H$, actor model $\pi_\theta$ that generate one step each time by default, search budget $n$, children number $m$
   \STATE {\bfseries Output:} MCT data $\mathbb{T}$
   \STATE Init $\mathbb{T}$ with empty
   \FOR{each single prompt $\bx$ in $\mathcal{D}_k$}
        \STATE Init search tree $T$ with $root\_node$ of $\bx$
        \FOR{$i$ in range($n$)}
            \STATE Select a leaf node $select\_node$ following the trajectory $(\bx,\bs_i)$ using UCB1 algorithm as~\cref{eq:UCB}
            \STATE $\bz_{i+1}^\ast \leftarrow None$
            \IF{$select\_node$ has been visited before}
                \IF{$select\_node$ is non-terminal}
                    \STATE Sample $m$ children $\{\bz_{i+1}^{(j)}\}_{j=1}^m$ from $\pi_\theta(\cdot|\bx, \bs_i)$ and add the $m$ children to $T$
                    \STATE $\bz_{i+1}^\ast \leftarrow$ random.choice($\{\bz_{i+1}^{(j)}\}$), $select\_node \leftarrow$ the corresponding child
                \ENDIF
            \ENDIF
            \STATE Rollout a full answer $\bfm\sim\pi_\theta(\cdot|\bx,\bs_i, \bz_{i+1}^\ast)$
            \STATE Calculate reward $r \leftarrow S(\bfm) \cdot H(\bfm) +2S(\bfm)$
            \STATE Backpropagate and update node's value and visited times from $select\_node$ to $root\_node$
        \ENDFOR
        \STATE Rollout all nodes that have not been visited before, calculate reward and backpropagate
        \STATE $\mathbb{T}\leftarrow \mathbb{T}\cup\{T\}$
   \ENDFOR
\end{algorithmic}
\end{algorithm}

\subsubsection{Self-Rewarding} 
We take the trained LLMs as judges~\cite{zheng2023judging} to rate their own responses, to remove dependencies on external reward models. We adopt a similar template design following~\cite{yuanself} to prompt the model to give discrete ratings given the query $\bx$ and the final answer $\bfm$ sampled through rollout. For helpfulness, we ask the model to rate the answer from $1$ to $5$ according to the extent of helpfulness and correctness. For safety, we categorize the answer into safe and unsafe ones. All ratings will be normalized into the range of $[-1,1]$. Note that the models also give rewards with in-depth reasoning, which further increase the reliability of ratings.

\subsubsection{Preference Data Construction} 

Given the search trees built via SI-MCTS, we can select stepwise preference data with different steps to optimize the model itself. We employ a threshold sampling strategy to guarantee the quality of training data. For a parent node in the tree, we group two children nodes as a pair of stepwise data if they satisfy that the difference between two values exceeds a threshold $v_0$ and the larger value exceeds another threshold $v_1$. This is to assure that there is a significant gap in the quality of two responses while the ``chosen'' one is good enough. Two thresholds are adjusted to gather a certain amount of training data. 

For the ablation study comparing preference data of full trajectories, we adopt similar strategies but within all full trajectories from the root node. As for the stepwise preference data for training a process reward model, we group nodes at the same depth without requiring them to share a parent node and only emphasize the gap between the ``chosen'' and ``rejected'' responses. To support rewarding at both stepwise and full-trajectory level, we include some full-trajectory preference data into $\mathcal{D}_R$.


\section{Experimental Details}
\label{sec:appendix_exp}

In this work, we conduct all our experiments on clusters with 8 NVIDIA A800 GPUs. 

\subsection{Training Details}
\label{sec:appendix_train}

We have done all the training of LLMs with LLaMA-Factory~\cite{zheng2024llamafactory}, which is a popular toolbox for LLM training. For all methods in training LLMs, optimization with SFT is for $3$ epochs and that with DPO is for $1$ epoch by default. We tune the learning rate from $\{5e-7, 1e-6, 5e-6\}$ and $\beta$ for DPO from $\{0.1,0.2,0.4\}$. Batch size is fixed as $128$ and weight decay is set to $0$. We adopt a cosine scheduler with a warm-up ratio of $0.1$. Following the official implementation, we set $\beta=0.1$ and $\beta/\lambda=0.025$ for SACPO. For Self-Rewarding and our self-improving framework, we take $K=3$ iterations. We take an auxiliary SFT loss with a coefficient of $0.2$ in our self-improvement to preserve the structured CoT style. 

For training process reward model based on the LLaMA architecture, we use OpenRLHF~\cite{hu2024openrlhf} and train based on TA-DPO-3 for 1 epoch, using batch size of $256$ and learning rate of $5e-6$. The training data has 70k pairwise samples from Monte Carlo Search Tree in three iterations and contains both stepwise pairs and full-trajectory pairs. This is to ensure the verifier to have the ability to choose the best answer between partial answers with same thinking steps and between full answers.


For the reproduction of Deliberative Alignment~\cite{guan2024deliberative}, we first develop a comprehensive set of safety policies by analyzing query data from o1 and reviewing OpenAI's content moderation guidelines. Specifically, we prompt o1-preview to generate policies for the seven categories of harmful content identified in Deliberative Alignment --- erotic content, extremism, harassment, illicit behavior, regulated advice, self-harm, and violence ---  and organize them with a unified format by manual check. Each policy includes: (1) a clear Definition of the category, (2) User Requests Categorization (defining and providing examples of both allowed and disallowed requests), (3) Response Style Guidelines, and (4) Edge Cases and Exceptions. Additionally, to account for potential gaps in coverage, we introduce a general safety policy, resulting in a total of eight distinct policy categories, which are submitted as supplementary materials. To ensure fairness and consistency, we use GPT-4o to classify prompts from the PKU-SafeRLHF and JailbreakV-28k datasets based on these eight policy definitions. Notably, we focus on the same 23k safety-related prompts used in our own methodology to maintain comparability.

We fine-tune two open-source o1-like LLMs with the same LLaMA-8B architecture, OpenO1-LLaMA-8B-v0.1 and DeepSeek-r1-Distilled-LLaMA-8b, to compare with our results on LLaMA-8B-3.1-Instruct. We follow the practice in~\cite{guan2024deliberative}, generating reasoning answers based on the harmful prompts together with the safety guidelines, which are gathered as a SFT dataset. These models are trained on the query-response pairs with a learning rate $5e-6$ and a batch size of $128$ for $3$ epochs. 


\subsection{Evaluation Details}
\label{sec:appendix_eval}

For evaluation, we take greedy decoding for generation to guarantee the reproducibility by default. As for test-time scaling, we set temperature to 0.6, top-p to 0.9 and top-k to 50 for the diversity across different responses. Below, we introduce the benchmarks and corresponding metrics in details.

For StrongReject~\cite{souly2024strongreject}, we take the official evaluation protocol, which uses GPT-4o to evaluate the responses and gives a rubric-based score reflecting the willingness and capabilities in responding the harmful queries. We follow~\cite{jaech2024openai} and take the goodness score, which is $1-\text{rubric score}$, as the metric. We evaluate models on prompts with no jailbreak in addition to the reported top-2 jailbreak methods PAIR~\cite{chaojailbreaking}, and PAP-Misrepresentation~\cite{zeng2024johnny}. For main results, we only report the average goodness score on the two jailbreak methods, since most methods achieve goodness scores near $1.0$. For XsTest~\cite{rottger2023xstest}, we select the unsafe split to evaluate the resistance to normal harmful queries and follow its official implementation on refusal determination with GPT-4. We report the sum of full refusal rate and partial refusal rate as the metric. For WildChat~\cite{zhaowildchat}, we filter the conversations with ModerationAPI\footnote{https://platform.openai.com/docs/guides/moderation} and eventually get 219 samples with high toxicity in English. For Stereotype, it is a split for evaluating the model's refusal behavior to queries associated with fairness issues in Do-Not-Answer~\cite{wang2023not}. We also use the same method as XsTest for evaluation, also with the same metric, for these two benchmarks. 

To benchmark general performance, we consider several dimensions involving trustworthiness~\cite{wangdecodingtrust,zhangmultitrust} and  helpfulness in popular sense. We adopt SimpleQA~\cite{wei2024measuring} for truthfulness, AdvGLUE~\cite{wang2adversarial} for adversarial robustness, InfoFlow~\cite{mireshghallahcan} for privacy awareness, GSM8k~\cite{hendrycks2measuring}, AlpacaEval~\cite{dubois2024length}, and BIG-bench HHH~\cite{zhou2024beyond} for helpfulness. All benchmarks are evaluated following official implementations. Correlation coefficient is reported for InfoFlow, and winning rate against GPT-4 is reported for AlpacaEval, while accuracies are reported for the rest. 

% \vspace{12pt}
\section{Examples}
\label{sec:appendix_examples}

Here, we present several examples to qualitatively demonstrate the effectiveness of STAIR against jailbreak attacks proposed by PAIR~\cite{chaojailbreaking}. We compare the outputs of our model with those of baseline models trained on the complete dataset using Direct Preference Optimization (DPO), referred to as the \textit{baseline model} in the cases below.

For each case presented below, we display the following:
\begin{itemize}
    \item \texttt{<Original harmful prompt, baseline model's answer>}
    \item \texttt{<Jailbroken prompt based on the original harmful prompt, baseline model's answer>}
    \item \texttt{<Jailbroken prompt based on the original harmful prompt, STAIR's reasoning process and answer>}
\end{itemize}

Please note that in the answers, due to ethical concerns, we have redacted harmful content by replacing it with a "cross mark" (\textbf{x}) to indicate the presence of harmful content. Our model may perform single-step reasoning (as shown in Case 1) or multi-step reasoning (as demonstrated in Cases 2 and 3) depending on the question. Each reasoning step is marked with \texttt{<|Reasoning\_step|>} and \texttt{<|/Reasoning\_step|>}, while the final answer is enclosed within \texttt{<|Output|>} and \texttt{<|/Output|>}.

We observe that although the baseline model can respond to harmful prompts with refusals, it remains vulnerable to jailbreaks that fabricate imagined scenarios to conform to the harmful query. In contrast, the model trained with STAIR-DPO-3 thoroughly examines the potential risks underlying the jailbreak prompts through step-by-step introspective reasoning, ultimately providing appropriate refusals.



\begin{figure*}
    \centering
    \includegraphics[width = \linewidth]{images/appendix/case-1.pdf}
    \caption{\textbf{Case 1}}
    % \label{fig:appendix-case-1}
\end{figure*}

\begin{figure*}
    \centering
    \includegraphics[width = \linewidth]{images/appendix/case-2.pdf}
    \caption{\textbf{Case 2}}
    % \label{fig:appendix-case-1}
\end{figure*}

\begin{figure*}
    \centering
    \includegraphics[width = \linewidth]{images/appendix/case-3.pdf}
    \caption{\textbf{Case 3}}
    % \label{fig:appendix-case-1}
\end{figure*}


\end{document}


% This document was modified from the file originally made available by
% Pat Langley and Andrea Danyluk for ICML-2K. This version was created
% by Iain Murray in 2018, and modified by Alexandre Bouchard in
% 2019 and 2021 and by Csaba Szepesvari, Gang Niu and Sivan Sabato in 2022.
% Modified again in 2023 and 2024 by Sivan Sabato and Jonathan Scarlett.
% Previous contributors include Dan Roy, Lise Getoor and Tobias
% Scheffer, which was slightly modified from the 2010 version by
% Thorsten Joachims & Johannes Fuernkranz, slightly modified from the
% 2009 version by Kiri Wagstaff and Sam Roweis's 2008 version, which is
% slightly modified from Prasad Tadepalli's 2007 version which is a
% lightly changed version of the previous year's version by Andrew
% Moore, which was in turn edited from those of Kristian Kersting and
% Codrina Lauth. Alex Smola contributed to the algorithmic style files.
