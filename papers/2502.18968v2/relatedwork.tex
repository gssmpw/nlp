\section{Related Works}
\subsection{General User Simulator}

Early user simulators including agenda-based methods~\cite{schatzmann2007agenda,schatzmann2009hidden} and model-based methods ~\cite{asri2016sequence,kreyssig2018neural}. These simulators were initially designed with a narrow scope due to limited natural language generation capabilities, such as generating synthetic binary preference responses~\cite{christakopoulou2016towards} in conversational recommendation systems.  

Recent advancements in LLMs enabled more sophisticated simulations of realistic conversations, offering significantly enhanced natural language flexibility. These advances include the use of LLMs for self-chat~\cite{XuGDM23} and dual LLM architectures, where separate models role-play user and assistant based on seed conversations~\cite{DingCXQHL0Z23}. Following these innovations, other trained user simulators, such as PlatoLM~\cite{KongFWJW24} and Parrot~\cite{SunLZHSZZZG24}, learn human discourse patterns directly from human-LLM interactions in conversations.

% Early user simulators were constrained by the limited natural language generation capabilities of their time, primarily relying on rule-based behavior modeling approaches. These included agenda-based methods~\cite{schatzmann2007agenda,schatzmann2009hidden} with handcrafted rules for mimicking user behaviors, and model-based approaches utilizing neural networks~\cite{asri2016sequence,kreyssig2018neural}. In the context of conversational recommendation systems, simulators were initially designed with narrow scope, focusing on generating synthetic binary preference responses~\cite{christakopoulou2016towards}. 

% Recent advances have enabled more sophisticated simulation of realistic conversations with enhanced natural language flexibility. These advances include employing response models for self-chat~\cite{XuGDM23} or dual LLM architectures where separate models assume user and assistant roles based on seed conversations~\cite{DingCXQHL0Z23}. Alternative approaches, such as PlatoLM~\cite{KongFWJW24} and Parrot~\cite{SunLZHSZZZG24}, have emerged to learn human discourse patterns directly from authentic human-LLM interactions.



\subsection{Persona-based User Simulator}

General user simulators often struggle to capture the full spectrum of diverse user needs, leading to a growing interest in persona-based personalization to improve both controllability and diversity in simulations~\cite{takanobu2020multi}. Some researchers attempt to leverage goal generators~\cite{takanobu2020multi} to create diverse user goals or retrieval-based personas derived from historical data~\cite{shi2019build} to guide user simulators in task-oriented dialogue (ToD) systems.

With the rise of LLMs and their impressive zero-shot role-playing abilities~\cite{njifenjou2024role}, prompt-driven user simulation has become the dominant approach. For example, LLMs have been used with carefully designed predefined profiles to align with human beliefs~\cite{ChuangNSGFYSHR24}, simulate consultation scenarios with users exhibiting varying personalities and needs in ToD systems~\cite{ZhangH0LLWLC24}, and model user preferences in conversational recommendation systems~\cite{yoon2024evaluating}.

% General simulation approaches often fall short in simulating diverse user needs, leading to increased interest in persona-based personalization to enhance both controllability and diversity in simulation~\cite{takanobu2020multi}. In task-oriented dialogue (ToD) systems, researchers have explored various approaches: some utilize Goal Generators~\cite{takanobu2020multi} to create diverse user goals, while others leverage retrieval-based personas for natural language generation, incorporating either retrieval agendas or historical data~\cite{shi2019build}. With the advent of LLMs, and their remarkable zero-shot and role-playing capabilities~\cite{njifenjou2024role}, prompt-driven user simulation has emerged as the dominant paradigm. For instance, LLMs have been employed with carefully crafted predefined profiles to align with human beliefs~\cite{ChuangNSGFYSHR24}, simulate consultation scenarios with users of varying personalities and needs in ToD~\cite{ZhangH0LLWLC24}, and model user preferences in conversational recommendation systems~\cite{yoon2024evaluating}.

% Generating responses based on persona conditions to enhance engagement and personalization is a current trend in conversational area~\cite{abs-2404-18231}. Research has progressed from exploring coarse-grained demographic roles~\cite{sociodemographic} to developing specific user simulators~\cite{0003SM24}. Since the release of the Persona-Chat dataset~\cite{KielaWZDUS18}, numerous studies have focused on improving conditional generation based on personas. These approaches include incorporating memory mechanisms~\cite{abs-2212-10276}, disentangled representations~\cite{00020ZZ020}, and CVAE-based methods~\cite{TangWFZHHH23}. In the era of large language models (LLMs), research directions have become more diversified. For instance, LLMs are employed to role-play and explore personality consistency~\cite{investigating} or to align with human beliefs~\cite{ChuangNSGFYSHR24}.  Additionally, user simulators leverage LLMs to act as general users, simulating user behaviors to provide replies and feedback for response-side models, which significantly enhances model performance~\cite{KongFWJW24,SunLZHSZZZG24,0002ZLNC24}. Moreover, studies~\cite{ZhangH0LLWLC24,Liu0YWM0WW23} highlight that the diversity of user simulators helps alleviate stagnation in model responses, thereby improving adaptability and robustness.




% \paragraph{Personalized Agents.} Differ from persona-based agents by focusing on tailoring responses to meet the specific needs of individual users~\cite{BakO19}. For example, \cite{LiZC23} proposed an efficient method for incorporating users’ personal information to enhance personalized recommendations, while \cite{abs-2307-03744} explored LLM-based search systems to improve user performance in specific scenarios. Recently, another emerging trend involves continuously sensing and updating user profiles to enhance the quality of personalized responses~\cite{abs-2406-13960}.






%