% First Table: Method + Cite and Method + Not Cited
\begin{table*}[ht]
\small
\centering
\renewcommand{\arraystretch}{1.1}
\caption{We showcase \textsc{Tree-of-Debate}'s strong performance across all comparison settings. \textbf{Bolded} values indicate the top score; \underline{underlined} indicates second-highest. We include two ablations which remove the tree structure (\textit{No Tree}) and self-deliberation (\textit{No SD}), respectively.}
\label{tab:method_results}
\resizebox{\textwidth}{!}{%
\begin{tabular}{|c|ccc|ccc|ccc|}
\hline
\multirow{2}{*}{\textbf{Baseline}} 
& \multicolumn{3}{c|}{\textbf{Method + Cite}} 
& \multicolumn{3}{c|}{\textbf{Method + Not Cited}} 
& \multicolumn{3}{c|}{\textbf{Overall (Method)}} \\ \cline{2-10}
& \textit{Breadth} & \textit{Context} & \textit{Factuality} 
& \textit{Breadth} & \textit{Context} & \textit{Factuality} 
& \textit{Breadth} & \textit{Context} & \textit{Factuality} \\ \hline

% Single Stage
Single Stage   
& \underline{93.33} & \underline{80.00} & \textbf{93.48} 
& \underline{89.81} & 73.15 & \underline{94.90} 
& \underline{91.07} & 75.59 & \underline{94.39} \\

% Two Stage
Two Stage      
& 85.00 & 71.67 & 91.42 
& 87.03 & 72.22 & 94.74 
& 86.31 & 72.02 & 93.69 \\ \hline

% \textbf{Tree-of-Debate} 
\textbf{Tree-of-Debate}       
& \textbf{96.67} & \textbf{93.33} & \underline{92.80} 
& \textbf{93.52} & \underline{93.52} & \textbf{96.80} 
& \textbf{94.64} & \textbf{93.45} & \textbf{95.37} \\

% ToD (No Tree)
\textit{ToD (No Tree)}  
& 81.67 & 63.33 & 91.25 
& 79.63 & 70.37 & 89.46 
& 80.36 & 67.86 & 89.82 \\

% ToD (No SD)
\textit{ToD (No SD) }
& 91.67 & 76.67 & 82.33 
& 86.11 & \textbf{94.44} & 76.33 
& 88.10 & \underline{88.10} & 78.21 \\ \hline

\end{tabular}%
}
\newline\newline\newline
\resizebox{\textwidth}{!}{%
\begin{tabular}{|c|ccc|ccc|ccc|}
\hline
\multirow{2}{*}{\textbf{Baseline}} 
& \multicolumn{3}{c|}{\textbf{Task + Cite}} 
& \multicolumn{3}{c|}{\textbf{Task + Not Cited}} 
& \multicolumn{3}{c|}{\textbf{Overall (Task)}} \\ \cline{2-10}
& \textit{Breadth} & \textit{Context} & \textit{Factuality} 
& \textit{Breadth} & \textit{Context} & \textit{Factuality} 
& \textit{Breadth} & \textit{Context} & \textit{Factuality} \\ \hline

% Single Stage
Single Stage   
& 81.82 & 65.91 & \underline{94.44} 
& \underline{88.46} & 71.15 & 92.31 
& 87.00 & 70.00 & \underline{92.78} \\

% Two Stage
Two Stage      
& 89.29 & 82.14 & \textbf{97.14} 
& 86.84 & 78.29 & \textbf{94.18} 
& \underline{87.22} & 78.89 & \textbf{94.54} \\ \hline

% \textbf{Tree-of-Debate} 
\textbf{Tree-of-Debate}       
& \underline{92.86} & \textbf{100} & 87.89 
& \textbf{96.05} & \textbf{96.05} & \underline{94.02} 
& \textbf{95.19} & \textbf{97.11} & 92.36 \\

% ToD (No Tree)
\textit{ToD (No Tree)}  
& 67.86 & 67.86 & 61.11 
& 83.55 & 73.03 & 91.43 
& 81.11 & 72.22 & 89.74 \\

% ToD (No SD)
\textit{ToD (No SD)} 
& \textbf{96.43} & \underline{96.43} & 75.00 
& 81.58 & \underline{94.08} & 82.27 
& 83.89 & \underline{94.44} & 81.79 \\ \hline

\end{tabular}%
}
\end{table*}


\begin{table*}[ht!]
    \small
    \centering
    \renewcommand{\arraystretch}{1.5}
    \caption{Debate Round for Tree of Thoughts (ToT) vs. Chain-of-Thought (CoT); turns are summarized.}
    \begin{tabular}{|p{1.1cm}|p{14cm}|}
        \hline
        \textbf{Stage} & \textbf{Dialogue for Subtopic: 	``Generalizability and Flexibility of Prompting Methods''} \\ 
        \hline
        \textbf{Present} & 
        \begin{tabular}[t]{@{}p{14cm}@{}}
            \textbf{ToT}: ToT \textit{enhances flexibility} via deliberate reasoning by exploring multiple paths, self-evaluation of choices, and lookahead/backtracking capabilities-- shown on a diverse set of tasks like Game of 24 and Creative Writing. This is in contrast to CoT's lack of deliberate reasoning and adaptability. \\ \hline
            \textbf{CoT}: CoT excels in \textit{generalizability through its simplicity}, requiring minimal examples to achieve strong results across tasks. Unlike ToT, it \textit{avoids complex setups and can generalize to unseen tasks}, making it more adaptable with fewer computational demands.
        \end{tabular} \\ 
        \hline
        \textbf{Respond} & 
        \begin{tabular}[t]{@{}p{14cm}@{}}
            \textbf{ToT}: While CoT is simpler, ToT’s added complexity is a necessary \textit{trade-off for deeper reasoning tasks}, requiring non-trivial planning and search. CoT’s ability to handle deeply nested or conditional reasoning remains unproven. \\ \hline
            \textbf{CoT}: ToT may help in specific tasks, but CoT's simplicity deliberately ensures \textit{broader applicability} without overfitting to specific task requirements. Though direct evidence for nested reasoning is lacking, CoT’s state-of-the-art GSM8K performance suggests complex reasoning robustness. A future study can compare both methods on nested reasoning to settle this question empirically.
        \end{tabular} \\ 
        \hline
        \textbf{Revise} & 
        \begin{tabular}[t]{@{}p{14cm}@{}}
            \textbf{ToT}: ToT has \textit{enhanced flexibility via adaptive reasoning path exploration}-- exploring multiple reasoning paths and leveraging self-evaluation and lookahead/backtracking to tackle complex tasks-- as shown on our diverse problem domains. Our deliberate decision-making is advantageous for tasks with \textit{deeply nested or highly conditional reasoning paths}, where CoT may falter. \\ \hline
            \textbf{CoT}: CoT \textit{maintains strong generalizability by balancing simplicity and effectiveness}-- including those with complex reasoning requirements-- \textit{without overfitting} to specific task needs. This is demonstrated through its SoTA performance on GSM8K.
        \end{tabular} \\ 
        \hline
    \end{tabular}
    \label{tab:evolution}
\end{table*}