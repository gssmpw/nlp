\section{Experimental Settings}
\label{appendix: settings}
\par We choose \texttt{Llama-3.1-Nemotron-70B-Instruct-\\HF}, an \textit{open-source model}, as the base model for all experiments. We sample from the top 1\% of the tokens and set the temperature between $(0,0.5)$ to trade-off between deterministic and creative generation based on the nature of the given task (same setting across all samples):
\begin{itemize}
    \item \underline{Paper} generates arguments: $0.3$
    \item \underline{Paper} whether evidence is relevant: $0$
    \item \underline{Paper} presents its argument: $0.1$
    \item \underline{Paper} responds to the opposition's argument: $0.4$
    \item \underline{Paper} revises its argument: $0.4$
    \item \underline{Moderator} generating subtopics: $0.3$
    \item \underline{Moderator} determines whether to expand the debate note: $0.1$
    \item \underline{Moderator} summarizes a debate path: $0.4$
\end{itemize}

\par We set the number of retrieved segments $\delta = 5$ so that we can gather a sufficient amount of evidence while not overwhelming the debate with long-context. We set the number of generated subtopics $k = 3$, for covering a reasonable breadth of topics while minimizing redundancy. Finally, we set the maximum debate tree depth $l=3$ for adequate exploration. We use vLLM \cite{kwon2023efficient} for distributed and constrained generation on four \texttt{NVIDIA A100} GPUs.

\section{Baselines}
\label{appendix:baselines}
\par We compare Tree-of-Debate (ToD) with the following prompting-based baseline methods. 
We use the same base language model for all comparisons. 
\begin{itemize}[leftmargin=*]
    \item \textbf{Single-stage}: We prompt an LLM with the title, abstract and introduction sections of both focus and opposition papers. We prompt the model to directly generate a contrastive summary of the two papers \cite{MartinBoyle2024ShallowSO}. 
    \item \textbf{Two-stage}: We first instruct an LLM to individually summarize each paper based on the title, abstract and introductions. We then use the generated summaries to prompt the model to generate a contrastive summary \cite{Zhang2024FromCT}. 
\end{itemize}

To contextualize improvements from each component in Tree-of-Debate we construct the following ablative methods:
\begin{itemize}
    \item \textbf{ToD (No Tree)}: We remove the tree structure from Tree-of-Debate by merging child arguments into one. We do so by concatenating the topics and descriptions of the child subtopics and tag them to distinguish the topics. In each debate round, the model is prompted with the combined subtopic and its corresponding description. 
    \item \textbf{ToD (No SD)}: We remove self-deliberation (SD) to test the impact of iterative retrieval based on debate progression. We do so by prompting the model with title, abstract, and introduction of each paper instead of retrieving based on the subtopic. 
\end{itemize}


\section{Evaluation Metrics}
\label{appendix:eval_metrics}
\par The same domain-experts from Section \ref{sec:dataset} manually evaluate each sample in-depth, assessing various qualities of the summaries:

\begin{itemize}
    \item \textbf{Factuality}: \textit{How factual is the summary?} Each sentence is given a 1/0 binary score for factual or not, and the scores are averaged across the summary.
    \item \textbf{Completeness}: \textit{Is the summary comprehensive and complete?} This is evaluated using the following Likert scale:
    \begin{enumerate}
        \setcounter{enumi}{-1}
        \item Not at all, the summary misses (MULTIPLE) major points.
        \item No, the summary misses a (SINGULAR) major point.
        \item Somewhat, the summary misses minor points.
        \item Yes, the summary covers the major points, but still is not what I would expect.
        \item Very comprehensive, the summary covers the major points.
    \end{enumerate}
    \item \textbf{Contextualization}: 
    \textit{Does the summary explain and/or justify the posed similarities/differences between the papers, as opposed to just mentioning them?}
    \begin{enumerate}
        \setcounter{enumi}{-1}
        \item Not at all, the summary is simply extractive-- it just seems to take different subtopics from each paper and doesn’t synthesize them– no justification behind similarities and differences.
        \item No, the summary attempts at some level of synthesis, but it is not meaningful.
        \item Somewhat, the summary attempts at synthesizing at most one point.
        \item Yes, the summary contains meaningful synthesis but only for a minority of points.
        \item  Strongly contextualized, the summary contains meaningful synthesis across all major points.
    \end{enumerate}
\end{itemize}

\section{Domain-Expert Profiles}
\label{appendix: annotators}
\par Given that novelty comparison between papers is a highly specialized and expensive task, we gather five domain experts to both collect and annotate our dataset, as well as evaluate Tree-of-Debate's generated summaries based on their respective samples. Each domain expert is a graduate student with $3+$ years of research experience in a specialized area:
\begin{enumerate}
    \item \textbf{Domain Expert \#1:} A third-year PhD in computer science with ten publications; research expertise is text mining and data mining.
    \item \textbf{Domain Expert \#2:} A third-year PhD in aerospace engineering with two publications; research expertise is in electric propulsion.
    \item \textbf{Domain Expert \#3:} A second-year PhD (with two years of a research-track Masters) in electrical engineering with four publications; research expertise is in in-memory computing and wireless communications.
    \item \textbf{Domain Expert \#4:} A first-year PhD (with two years of a research-track Masters) in computer science with six publications; research expertise is data-efficient natural language processing.
    \item \textbf{Domain Expert \#5:} A first-year PhD (with two years of a research-track Masters) in computer science with twenty-five publications; research expertise is large language model training and efficiency.

\section{Dataset}
\label{sec:dataset}

\subsection{Data Collection}

To analyze political discussions on Discord, we followed the methodology in \cite{singh2024Cross-Platform}, collecting messages from politically-oriented public servers in compliance with Discord's platform policies.

Using Discord's Discovery feature, we employed a web scraper to extract server invitation links, names, and descriptions, focusing on public servers accessible without participation. Invitation links were used to access data via the Discord API. To ensure relevance, we filtered servers using keywords related to the 2024 U.S. elections (e.g., Trump, Kamala, MAGA), as outlined in \cite{balasubramanian2024publicdatasettrackingsocial}. This resulted in 302 server links, further narrowed to 81 English-speaking, politics-focused servers based on their names and descriptions.

Public messages were retrieved from these servers using the Discord API, collecting metadata such as \textit{content}, \textit{user ID}, \textit{username}, \textit{timestamp}, \textit{bot flag}, \textit{mentions}, and \textit{interactions}. Through this process, we gathered \textbf{33,373,229 messages} from \textbf{82,109 users} across \textbf{81 servers}, including \textbf{1,912,750 messages} from \textbf{633 bots}. Data collection occurred between November 13th and 15th, covering messages sent from January 1st to November 12th, just after the 2024 U.S. election.

\subsection{Characterizing the Political Spectrum}
\label{sec:timeline}

A key aspect of our research is distinguishing between Republican- and Democratic-aligned Discord servers. To categorize their political alignment, we relied on server names and self-descriptions, which often include rules, community guidelines, and references to key ideologies or figures. Each server's name and description were manually reviewed based on predefined, objective criteria, focusing on explicit political themes or mentions of prominent figures. This process allowed us to classify servers into three categories, ensuring a systematic and unbiased alignment determination.

\begin{itemize}
    \item \textbf{Republican-aligned}: Servers referencing Republican and right-wing and ideologies, movements, or figures (e.g., MAGA, Conservative, Traditional, Trump).  
    \item \textbf{Democratic-aligned}: Servers mentioning Democratic and left-wing ideologies, movements, or figures (e.g., Progressive, Liberal, Socialist, Biden, Kamala).  
    \item \textbf{Unaligned}: Servers with no defined spectrum and ideologies or opened to general political debate from all orientations.
\end{itemize}

To ensure the reliability and consistency of our classification, three independent reviewers assessed the classification following the specified set of criteria. The inter-rater agreement of their classifications was evaluated using Fleiss' Kappa \cite{fleiss1971measuring}, with a resulting Kappa value of \( 0.8191 \), indicating an almost perfect agreement among the reviewers. Disagreements were resolved by adopting the majority classification, as there were no instances where a server received different classifications from all three reviewers. This process guaranteed the consistency and accuracy of the final categorization.

Through this process, we identified \textbf{7 Republican-aligned servers}, \textbf{9 Democratic-aligned servers}, and \textbf{65 unaligned servers}.

Table \ref{tab:statistics} shows the statistics of the collected data. Notably, while Democratic- and Republican-aligned servers had a comparable number of user messages, users in the latter servers were significantly more active, posting more than double the number of messages per user compared to their Democratic counterparts. 
This suggests that, in our sample, Democratic-aligned servers attract more users, but these users were less engaged in text-based discussions. Additionally, around 10\% of the messages across all server categories were posted by bots. 

\subsection{Temporal Data} 

Throughout this paper, we refer to the election candidates using the names adopted by their respective campaigns: \textit{Kamala}, \textit{Biden}, and \textit{Trump}. To examine how the content of text messages evolves based on the political alignment of servers, we divided the 2024 election year into three periods: \textbf{Biden vs Trump} (January 1 to July 21), \textbf{Kamala vs Trump} (July 21 to September 20), and the \textbf{Voting Period} (after September 20). These periods reflect key phases of the election: the early campaign dominated by Biden and Trump, the shift in dynamics with Kamala Harris replacing Joe Biden as the Democratic candidate, and the final voting stage focused on electoral outcomes and their implications. This segmentation enables an analysis of how discourse responds to pivotal electoral moments.

Figure \ref{fig:line-plot} illustrates the distribution of messages over time, highlighting trends in total messages volume and mentions of each candidate. Prior to Biden's withdrawal on July 21, mentions of Biden and Trump were relatively balanced. However, following Kamala's entry into the race, mentions of Trump surged significantly, a trend further amplified by an assassination attempt on him, solidifying his dominance in the discourse. The only instance where Trump’s mentions were exceeded occurred during the first debate, as concerns about Biden’s age and cognitive abilities temporarily shifted the focus. In the final stages of the election, mentions of all three candidates rose, with Trump’s mentions peaking as he emerged as the victor.
\section{Qualitative Case Study: TreeInstruct vs. BRIDGE}
\label{appendix:summary}
\begin{table*}[]
    \centering
    \caption{Summaries from two methods: ToD (No Tree) -- the \textsc{Tree-of-Debate} ablation -- and the Two-Stage baseline. The papers being compared are TreeInstruct and BRIDGE: they develop LLM-based instructors that guide students towards the answers to their questions rather than give away the answers. The rest of the table follows in \ref{table: qualitative_study_appendix_part2}.}
    \begin{tabular}{|p{8cm}|p{8cm}|}
        \toprule
        \textbf{Two-Stage} & \textbf{ToD (No Tree)} \\
        \midrule
        Both papers focus on enhancing educational experiences through innovative technological approaches. Both utilize novel methodologies to address specific learning challenges (code debugging for Paper 0 and math remediation for Paper 1). Both papers emphasize the importance of real-world validation and case studies to assess the effectiveness of their proposed solutions. Topic Focus: Paper 0 focuses on code debugging with TreeInstruct, while Paper 1 addresses math mistake remediation with the Bridge method. \textcolor{red}{Methodological Approach: Paper 0 employs state space estimation and dynamic tree-based questioning, whereas Paper 1 utilizes cognitive task analysis and decision-making models.} Evaluation Metrics: Paper 0 measures performance on debugging datasets and student engagement, whereas Paper 1 assesses the preference for LLM responses with expert decisions and conducts lexical analysis. Target Audience: Paper 0's work is geared towards students of varying skill levels in code debugging, whereas Paper 1 focuses on 1st-5th grade students in Title I schools for math tutoring. While both papers contribute to the educational technology landscape, Paper 0's TreeInstruct offers a novel approach to code debugging through adaptive Socratic questioning, and Paper 1's Bridge method innovatively embeds expert thought processes into LLMs for math remediation. \textcolor{red}{Despite overlapping in their pursuit of enhancing educational experiences, the papers diverge significantly in their topics, methodologies, and target audiences, indicating a broad and diverse landscape in educational technology research.} & Papers on adaptive guidance systems debate similarities and differences for effective Socratic questioning. While both papers aim to enhance adaptive guidance, they differ in approach: \textcolor{red}{TreeInstruct (Paper 0) emphasizes adaptive, expert-agnostic planning, leveraging dynamic question trees and state space-based planning to address interdependent errors, whereas Bridge (Paper 1) focuses on expert-informed decision-making, incorporating domain-specific expert knowledge for more effective Socratic questioning.} A key difference lies in their handling of unforeseen errors, with TreeInstruct relying on dynamic adaptation and \textcolor{red}{Bridge on continuous expert annotation and model fine-tuning. }Another distinction is their applicability: \textcolor{red}{TreeInstruct is designed for diverse educational domains}, while Bridge initially focused on math education, later expanding to other domains. Both authors eventually converge on a hybrid approach, combining expert guidance with adaptive planning for a more comprehensive solution. Despite this, their initial differences in methodology and error handling strategies remain a focal point of the debate. The debate highlights the importance of both adaptive planning and expert guidance in effective adaptive guidance systems. Ultimately, the synthesis of both approaches offers a more robust solution for personalizing student learning experiences. \\
\bottomrule
    \end{tabular}
    \label{table: qualitative_study_appendix_part1}
\end{table*}

\begin{table*}[]
    \centering
    \caption{Summaries from two methods: ToD (No SD) -- the \textsc{Tree-of-Debate} ablation -- and \textsc{Tree-of-Debate}. The papers being compared are TreeInstruct and BRIDGE: they develop LLM-based instructors that guide students towards the answers to their questions rather than give away the answers. The rest of the table follows in \ref{table: qualitative_study_appendix_part1}.}
    \begin{tabular}{|p{8cm}|p{8cm}|}
        \toprule
        \textbf{ToD (No SD)} & \textbf{ToD (Final)} \\
        \midrule
        The papers share similarities in their focus on adaptive, student-centered approaches for mistake correction, leveraging AI-driven tools to enhance personalized learning experiences. Both papers emphasize the importance of context-aware feedback and expert guidance in promoting deeper understanding. However, the papers diverge in their novelties. Paper 0's paper, 'Instruct, Not Assist,' uniquely integrates adaptive, multi-turn Socratic questioning with state space estimation, \textcolor{red}{providing a more holistic and adaptable learning experience}. In contrast, Paper 1's paper, 'Bridging the Novice-Expert Gap,' focuses on modular, expert-guided decision-making, offering a more explicit and scalable approach to addressing complex, interdependent mistakes. \textcolor{ForestGreen}{While Paper 0's approach excels in handling ambiguous student responses, Paper 1's method provides more nuanced, human-centric guidance.} The papers also differ in their evaluation metrics, with Paper 0 emphasizing robust state space estimation and Paper 1 highlighting the importance of contextual relevance. Ultimately, \textcolor{red}{the papers present complementary approaches, with Paper 0's work prioritizing adaptability and Paper 1's work emphasizing expert-driven contextualization}. & The papers by Paper 0 and Paper 1 share similarities in their focus on adaptive error resolution strategies, emphasizing the importance of context-sensitive decision-making and expert knowledge in enhancing student learning outcomes. \textcolor{ForestGreen}{Both papers also highlight the limitations of current LLMs in educational settings} and propose novel methodologies to address these challenges. However, the papers diverge in their approaches to achieving these goals. \textcolor{ForestGreen}{Paper 0's paper, 'Instruct, Not Assist,' uniquely integrates dynamic planning and hierarchical questioning to provide a more nuanced understanding of student thought processes}, scaling to accommodate diverse student populations. In contrast, \textcolor{ForestGreen}{Paper 1's paper, 'Bridging the Novice-Expert Gap,' leverages expert decision-making models to inform adaptive error resolution strategies, offering a more domain-agnostic framework}. The papers also differ in their evaluation frameworks, with Paper 0's approach utilizing a state space-based planning algorithm and Paper 1's method incorporating transfer learning and domain-specific expert annotations. Furthermore, \textcolor{ForestGreen}{Paper 0 emphasizes the importance of adaptive Socratic questioning, while Paper 1 highlights the value of expert-guided decision-making in enhancing educational support.} Ultimately, the papers present distinct novelties in addressing the novice-expert gap, with \textcolor{ForestGreen}{Paper 0 focusing on adaptive structured planning and Paper 1 on context-aware expert decision embedding}. \\
\bottomrule
    \end{tabular}
    \label{table: qualitative_study_appendix_part2}
\end{table*}

Tables \ref{table: qualitative_study_appendix_part1} and \ref{table: qualitative_study_appendix_part2} contain comparative summaries from the baseline, ablations, and our final method on the papers, TreeInstruct \cite{kargupta2024instructassistllmbasedmultiturn} and BRIDGE \cite{bridge}--- Papers 0 and 1, respectively. Below, we qualitatively compare each summary, pointing out the weaknesses and strengths, and show how our method is able to address all the issues brought up in the baseline summaries.

The top left contains the Two-Stage baseline. The Two-Stage baseline tends to contain near-copy phrases from the paper, resulting in \textbf{an overly specific, extractive and unnatural summary} (an example is the first line highlighted in red: ``Methodological Approach: Paper 0 employs...''). As a result, the differences that are extracted are not explained very well, requiring more work to understand the terminology-heavy summary. It also makes vague claims near the end of summaries (example is the second line highlighted in red: ``Despite overlapping in their pursuit of enhancing educational experiences...''). \textbf{The overall structure results in a suboptimal summary}.

Next, the top right box contains the summary for ToD (No Tree). The use of the debate format improves the quality of the generated claims. Unlike in the Two-Stage summary, it does not contain many extractive phrases, however \textbf{the structure of the debate is still fine-grained to coarse-grained}. Intuitively, the summaries should develop coarse-grained claims into fine-grained arguments. Moreover, there are \textbf{slight hallucinations} (examples are in the second and third lines highlighted in red: ``Bridge on continuous expert...'' and ``TreeInstruct is designed...''). Still, the conclusion (last sentence) of the summary is not as vague as the conclusion from Two-Stage, but it still does not capture the intricacies of the two methods well enough.

Subsequently, the summary for ToD (No SD) is on the bottom left. The benefits of the tree are drastic, as the summary starts by discussing the high-level summaries, and breaks down the individual fine-grained differences. This is much less extractive and more abstractive. \textbf{Using the tree structure along with the debate allows each argument to be explored further}-- this is evident as after each claim, an explanation of why it matters follows (example is the line highlighted in green: ``While Paper 0's approach excels in...''). Still, a few of these explanations are vague and \textbf{do not reveal the true underlying motivation of the claims} (highlighted in red).

Finally, the summary for ToD (our final method) is in the bottom right box. With the self-deliberation, it was able to extract a short phrase of the motivation behind both works (the ``limitations of current LLMs in educational settings''). The arguments are developed from \textbf{high-level claims to low-level}, technical concepts. The \textbf{facts are correctly identified} and do not contain any hallucinations. Moreover, the explanations preceding the claims also \textbf{reveal the underlying motivation} behind the specific novelty. Finally, the concluding sentence explains the exact difference between the two works.

\end{enumerate}