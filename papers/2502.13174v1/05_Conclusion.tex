\section{Conclusion}



This paper presents Generative Topology Optimization (GenTO), a novel approach that addresses a limitation in traditional TO methods.
By leveraging neural networks to parameterize shapes and enforcing solution diversity as an explicit constraint, GenTO enables the exploration of multiple near-optimal designs.
This is crucial for industrial applications, where manufacturing or aesthetic constraints often necessitate the selection of alternative geometries.
Our empirical results demonstrate that GenTO is both effective and scalable, generating more diverse solutions than prior methods while adhering to mechanical requirements.
Our main contributions include the introduction of GenTO as the first data-free, solver-in-the-loop neural network training method and the empirical demonstration of its effectiveness across various problems.
This work opens new avenues for generative design approaches that do not rely on large datasets, addressing current limitations in engineering and design.


\paragraph{Limitations and future work.}
While GenTO shows notable progress, several areas warrant further investigation to fully realize its potential.
Qualitatively, we observe, e.g. in Figure \ref{fig:results_2d_mbb_beam}, that shapes produced by GenTO can contain floaters (small disconnected components) and have surface undulations.
This warrants further investigation and can be preliminarily corrected by post-processing steps \citep{subedi2020review}.
\\
More research is needed to explore different dissimilarity metrics for the diversity loss, as this could further enhance the variety of generated designs.
Improving sample efficiency is also crucial for practical applications, as the current method may require significant computational resources.
Promising directions include using second-order optimizers and using the mutual information of concurrent solver outputs at early training stages. 
Lastly, future work could train GenTO on coarse designs and subsequently refine them with classical TO methods, combining the strengths of both approaches to achieve even better results.
