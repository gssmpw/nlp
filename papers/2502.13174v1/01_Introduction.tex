
\isformat{icml}

    \begin{figure}[ht!]
        \centering
        \begin{subfigure}{\columnwidth}
            \centering
            \includegraphics[width=\linewidth]{figures/results/cantilever_equidistant_9.pdf}
            \caption{GenTO (ours, 5 min).}
            \label{fig:cantilever_GenTO}
        \end{subfigure}
        \hfill %
        \begin{subfigure}{\columnwidth}
            \centering
                \includegraphics[width=0.66\linewidth]{figures/results/DAB_cantilever_comparison.pdf}
            \caption{Deflated Barrier (baseline, 123 min).}
            \label{fig:cantilever_db}
        \end{subfigure}
        
        \caption{Generative Topology Optimization (GenTO) produces a diverse set of near-optimal solutions on the cantilever problem defined in Figure \ref{fig:cantilever_def}.
        Compared to the baseline method, GenTO generates more shapes with \emph{greater diversity} in a significantly \emph{lower time} due to the inherent parallelizability.
        The structural \emph{compliance} $C$ describes the stiffness of the shape measured as the total displacement under the load.
        Both methods try to minimize $C$ and achieve comparable values with similar best designs.
        GenTO uses a \emph{conditional neural field} to parameterize the shapes forgoing adaptive mesh refinement used in the deflated barrier method.}
        \label{fig:results_cantilever}
    \end{figure}

\else

    \begin{figure}[ht!]
        \centering
        \begin{subfigure}{0.48\textwidth}
            \centering
            \raisebox{1.45cm}{
                \includegraphics[width=0.6\linewidth]{figures/results/DAB_cantilever_comparison.pdf}
            }
            \caption{Deflated Barrier (baseline, 123 min).}
            \label{fig:cantilever_db}
        \end{subfigure}
        \hfill %
        \begin{subfigure}{0.48\textwidth}
            \centering
            \includegraphics[width=\linewidth]{figures/results/cantilever_equidistant_9.pdf}
            \caption{GenTO (ours, 5 min).}
            \label{fig:cantilever_GenTO}
        \end{subfigure}
        
        \caption{Generative Topology Optimization (GenTO) produces a diverse set of near-optimal solutions on the cantilever problem defined in Figure \ref{fig:cantilever_def}.
        Compared to the baseline method, GenTO generates more shapes with \emph{greater diversity} in a significantly \emph{lower time} due to the inherent parallelizability.
        The structural \emph{compliance} $C$ describes the stiffness of the shape measured as the total displacement under the load.
        Both methods try to minimize $C$ and achieve comparable values with similar best designs.
        GenTO uses a \emph{conditional neural field} to parameterize the shapes forgoing adaptive mesh refinement used in the deflated barrier method.}
        \label{fig:results_cantilever}
    \end{figure}

\fi



\section{Introduction}

Topology optimization (TO) is a computational design technique to determine the optimal material distribution within a given design space under prescribed boundary conditions. 
A common objective in TO is the minimization of structural compliance, also called strain energy, which measures the displacement under load and is inverse to the stiffness of the generated design.

Due to the non-convex nature of TO problems, these methods generally provide near-optimal solutions, with no guarantees of converging to global optima \citep{allaire2021shape}.
Despite this, TO has been a critical tool in engineering design since the late 1980s and continues to evolve with advances in computational techniques, including the application of machine learning.

Traditional TO methods also produce a single design, which limits its utility. 
Engineering problems often require designs that balance performance with other considerations, such as manufacturability, cost, and aesthetics. These are associated with uncertainty, implicit knowledge, and subjectivity and are hard to optimize in a formal framework.
Generating multiple diverse solutions allows for exploring these trade-offs and provides more flexibility in choosing designs that meet both technical and real-world constraints. 
However, conventional TO methods cannot efficiently explore and generate multiple high-quality designs in a single optimization run, making it difficult to address this need.

We propose Generative Topology Optimization (GenTO), an approach to incorporate neural networks into the TO process to generate a set of diverse designs: A neural network parametrizes the shape representations and is trained to generate near-optimal solutions. The model is optimized using a solver-in-the-loop approach \citep{2020solver_in_the_loop}, where the neural network iteratively adjusts the design based on feedback from a physics-based solver. To enhance the diversity of solutions, we introduce an explicit diversity constraint during training, ensuring the network produces a range of solutions that still adhere to mechanical compliance objectives. GenTO not only enables the generation of multiple, diverse designs but also leverages machine learning to explore the design space more efficiently than traditional TO methods. 
















Empirically, we validate our method on TO for linear elasticity problems in 2D and 3D. 
Our results demonstrate that GenTO obtains more diverse solutions than prior work while being substantially faster, remaining near-optimal, and circumventing adaptive mesh refinement by using continuous neural fields. 
This addresses a current limitation in TO and opens new avenues for automated engineering design. 




\begin{figure*}[ht!]
    \centering
    \isformat{icml}
        \includegraphics[height=0.21\textwidth]{figures/method/Method-wider_bottom2.png}
    \else
        \includegraphics[width=\textwidth]{figures/method/Method-wider_bottom2.png}
    \fi
    \caption{A single iteration of GenTO trained with 3 shapes in parallel.
    In each iteration, the input to the network consists of mesh locations $\mathbf{x}_i, i \in \{1, ...,  N\}$ and modulation vectors $\mathbf{z}_j, j \in \{1, ..., M\}$ with $M=3$ for this example.
    The network outputs densities $\rho_j$ at mesh points $\mathbf{x}_i $ for each shape.
    The densities are passed into a FEM solver to compute compliances $C_j$ and volumes $V_j$, as well as their gradients $\nabla_\rho C_j$ and $\nabla_\rho V_j$.
    The diversity loss is based on the chamfer discrepancy between the surface points of the shapes.
    These pairwise discrepancies are used to compute the diversity loss $\delta(\rho)$ and its gradient $\nabla_\rho \delta$.
    }
    \label{fig:placeholder}
\end{figure*}

\isformat{arxiv}
    \newpage
\fi

Our main contributions are summarized as follows:
    
\begin{enumerate}
    \item We introduce GenTO, the first method for data-free, solver-in-the-loop neural network training, which generates diverse solutions adhering to structural requirements.
    \item We introduce a novel diversity constraint variant for neural density fields based on the chamfer discrepancy that ensures the generation of distinct and meaningful shapes and enhances the exploration of the designs.
    \item Empirically, we demonstrate the efficacy and scalability of GenTO on 2D and 3D problems, showcasing its ability to generate a variety of near-optimal designs.
    Our approach significantly outperforms existing methods in terms of speed and solution diversity.

\end{enumerate}


