\section{Method}


\subsection{Generative Topology Optimization}


\paragraph{Definitions.}
Let $\mathcal{X} \subset \mathbb{R}^{d_x}$
be the domain of interest in which there is a shape $\Omega \subset \mathcal{X}$. 
Let $\mathcal{Z} \subset \mathbb{R}^{d_z}$ be a discrete or continuous modulation space, where each $\mathbf{z} \in \mathcal{Z}$ parametrizes a shape $\Omega_\mathbf{z}$. The possibly infinite set of all shapes is denoted by $\Omega_\mathcal{Z} = \{ \Omega_\mathbf{z} | \mathbf{z} \in \mathcal{Z} \}$.
The modulation vectors $\{\mathbf{z}_{i} \}$ are either elements in a fixed, finite set $\mathcal{Z}$ or sampled according to a continuous probability distribution $p(\mathcal{Z})$ on an interval $[a, b]^n \subset \mathbb{R}^n$. For brevity, we will use $\mathbf{z} \sim p(\mathcal{Z})$ for both of these cases. \newline
For density representations a shape $\Omega_\mathbf{z}$ is defined as the set of points with a density greater than the level $\tau \in [0,1]$, formally $\Omega_\mathbf{z} = \{\mathbf{x} \in \mathcal{X} \mid \rho_\mathbf{z}(\mathbf{x}) > \tau \}$.
While some priors work on occupancy networks \citep{Occupancy_Networks} treat $\tau$ as a tunable hyperparameter, we follow \citet{fenitop} and fix the level at $\tau = 0.5$. 
We model the density $\rho_\mathbf{z}(\mathbf{x}) = f_\theta(\mathbf{x}, \mathbf{z})$, corresponding to the modulation vector $\mathbf{z}$ at a point $\mathbf{x}$ using a neural network $f_\theta$, with $\theta$ being the learnable parameters. 

\begin{algorithm}[tb]
   \caption{GenTO}
   \label{alg:GenTO}
\begin{algorithmic}
   \STATE {\bfseries Input:} 
   parameterized density $f_\theta$, 
   vector of mesh points $\mathbf{x_i} \in \mathbb{R}^{d_x}$,
   $\beta$ and annealing factor $\Delta_\beta$,
   number of shapes per iteration $k$,
   learning rate scheduler $\gamma(t)$,
   iterations $T$
   \FOR{$t=1$ {\bfseries to} $T$}
   \STATE $\mathbf{z_j} \sim p(\mathcal{Z})$ \hfill\COMMENT{sample modulation vectors}
   \STATE $\Tilde{\rho}_{z_j} \leftarrow f_\theta(\mathbf{x_i}, \mathbf{z_j})$  \hfill\COMMENT{net forward all shapes $j$}
   \STATE $\rho_j \leftarrow H(\Tilde{\rho}_{z_j}, \beta)$ \hfill\COMMENT{Heaviside contrast filter}
   \STATE $C, V, \frac{\partial C}{\partial \rho}, \frac{\partial V}{\partial \rho} \leftarrow \text{FEM}(\rho_j)$ \hfill\COMMENT{solver step}
   \STATE $\frac{\partial \rho}{\partial \theta} \leftarrow \text{backward}(f_\theta, \rho_j)$ \hfill\COMMENT{autodiff backward}
   \STATE $\nabla_\theta C \leftarrow \frac{\partial C}{\partial \rho} \frac{\partial \rho}{\partial \theta} $ \hfill\COMMENT{compliance gradient}
    \STATE $\nabla_\theta V \leftarrow \lambda_V \frac{\partial C}{\partial \rho} \frac{\partial \rho}{\partial \theta} $ \hfill\COMMENT{volume gradient}
   \STATE $L_\text{G} \leftarrow \lambda_\text{interface} L_\text{interface} + ...$ \ \hfill\COMMENT{GINN constraints}
   \STATE $\Delta \theta \leftarrow  \text{ALM} \left( \nabla_\theta C, \nabla_\theta V, \nabla_\theta L_\text{G} \right) $ \hfill\COMMENT{ALM}
   \STATE $\theta \leftarrow \theta - \gamma(t) \Delta \theta $ \hfill\COMMENT{parameter update}
   \STATE $\beta \leftarrow \beta \Delta_\beta$ \hfill\COMMENT{annealing}
   \ENDFOR
\end{algorithmic}
\end{algorithm}

\paragraph{GenTO}
aims to solve a constrained optimization problem with the objective of minimizing the expected compliance of multiple shapes subject to a volume constraint on each.
At the core of our method is the diversity constraint $\delta(\Omega_\mathcal{Z})$, defined over multiple shapes to make them less similar. This leads to the following constrained optimization problem:

\begin{equation}
\begin{aligned}
\label{eq:GenTo}
\min : & \quad \mathbb{E}_{\mathbf{z} \sim p(\mathcal{Z})} \left[ C(\rho_{\mathbf{z}}) \right]
\\
\text{s.t.} : & \quad V_{\mathbf{z}} = \int_\mathcal{X} \mathbf{\rho}_{\mathbf{z}}(\mathbf{x}) d\mathbf{x} \leq V^*\\
& \quad 0 \leq \mathbf{\rho}_{\mathbf{z}}(\mathbf{x}) \leq 1 \\
& \quad \delta^* \leq \delta(\Omega_\mathcal{Z})
\end{aligned}
\end{equation}

where $V^*$ and $\delta^*$ are target volumes and diversities, respectively.
GenTO updates the density fields of multiple shapes iteratively. In each iteration, the density distribution of each shape is computed and the resulting densities are passed to a finite element method (FEM) solver. 
The FEM solver calculates the compliance loss $C$ and gradients $\nabla C$ using the adjoint equation instead of differentiating through the solver.
To accelerate the computation, parallelization of the FEM solver is employed across multiple CPU cores, with a separate process dedicated to each shape.
\\ 
The diversity loss (see Section \ref{sec:diversity}) is computed on the GPU, along with any optional geometric losses similar to GINNs (see Section \ref{sec:geom_constraints}).
Gradients for these losses are obtained via automatic differentiation, based on which GenTO updates the network parameters $\theta$.
\\
We use the adaptive augmented Lagrangian method (ALM) \citep{basir2023adaptive} to automatically balance multiple loss terms. 
Additionally, we gradually increase the sharpness parameter $\beta$ of the Heaviside filter (Equation \ref{eq:heaviside}) which acts as annealing. This outline of the GenTO method is summarized in Algorithm \ref{alg:GenTO}.











\subsection{Diversity}
\label{sec:diversity}

The diversity loss, defined in Equation \ref{eq:div}, requires the definition of a dissimilarity measure between a pair of shapes. 
Generally, the dissimilarity between two shapes can be based on either the volume of a shape $\Omega$ or its boundary $\partial \Omega$.
GINNs \citep{GINNs} use boundary dissimilarity, which is easily optimized for SDFs since the value at a point is the distance to the zero level set. However, this dissimilarity measure is not applicable to our density field shape representation.
Hence we propose a boundary dissimilarity based on the chamfer discrepancy in the 
We also developed a volume-based dissimilarity (detailed in Appendix \ref{subsec:volume_dissimilarity}), however we focus on boundary diversity due to its promising results in early experiments.



\paragraph{Diversity on the boundary via differentiable chamfer discrepancy.}
To define the dissimilarity on the boundaries of a pair of shapes $(\partial \Omega_1, \partial \Omega_2)$, we use the one-sided chamfer discrepancy (CD):
\begin{align}
\label{eq:1_sided_chamfer}
    \text{CD}(\partial \Omega_1, \partial \Omega_2) = \frac{1}{\left| \partial \Omega_1 \right|} \sum_{x \in \partial \Omega_1} \min_{\Tilde{x} \in \partial \Omega_2} ||x-\Tilde{x}||_2
\end{align}
where $x$ and $\Tilde{x}$ are sampled points on the boundaries.
To use the CD as a loss, it must be differentiable w.r.t. the network parameters $\theta$.
However, the chamfer discrepancy $\text{CD}(\partial\Omega_1, \partial\Omega_2)$ depends only on the boundary points $x_i \in \partial \Omega$ which only depend on $f_\theta (x_i)$ implicitly.
Akin to prior work \citep{chen_converting_2010, Berzins2023bs, mehta_level_2022}, we apply the chain rule and use the level-set equation to derive
\isformat{icml}

    \begin{align}
        \frac{\partial \text{CD}}{\partial \theta} &= \frac{\partial \text{CD}}{\partial x} \frac{\partial x}{\partial y} \frac{\partial y}{\partial \theta} \nonumber \\ 
        &= \frac{\partial \text{CD}}{\partial x} \frac{\nabla_x f_\theta}{|\nabla_x f_\theta|^2} \frac{\partial y}{\partial \theta} \label{eq:level_set}
    \end{align}
\else
    \begin{align}
    \label{eq:level_set}
        \frac{\partial \text{CD}}{\partial \theta} = \frac{\partial \text{CD}}{\partial x} \frac{\partial x}{\partial y} \frac{\partial y}{\partial \theta} = \frac{\partial \text{CD}}{\partial x} \frac{\nabla_x f_\theta}{|\nabla_x f_\theta|^2} \frac{\partial y}{\partial \theta} 
    \end{align}
\fi
where $y = \rho$ is the density field in our case.
We detail this derivation in Appendix \ref{subsec:chamfer_diversity}.


\paragraph{Finding surface points on density fields.}
Finding surface points for densities is substantially harder than for SDFs.
For an SDF $f$, surface points can be obtained by flowing randomly initialized points along the gradient $\nabla f$ to the boundary at $f=0$. 
However, density fields $g$ do not satisfy the eikonal equation $|\nabla g| \neq 1$, causing gradient flows to easily get stuck in local minima. 
To overcome this challenge, we employ a robust algorithm that relies on dense sampling and binary search, as detailed in Algorithm \ref{alg:boundary_points}.



\subsection{Formalizing geometric constraints}
\label{sec:geom_constraints}

The compliance and volume losses are computed by the FEM solver on a discrete grid.
In addition, we use geometric constraints similar to GINNs.
This leverages the continuous field representation and allows learning finer details where necessary, e.g. at the interfaces.
In principle, these constraints could also be imposed via the TO framework, by defining solid regions ($\rho=1$ values) to specific mesh cells.
However, this comes at the cost of requiring increased mesh resolution, which raises the computational cost for the CPU-based FEM solver.
Instead, we adapt the geometric constraints of GINNs to operate with a general level set $\tau$ and reflect the change from an SDF to a density representation.
The applied constraints and further details are in Table \ref{tab:constraints} and Appendix \ref{app:constraints}. Crucially, these are computed in parallel on the GPU, accelerating the optimization process.
