\documentclass[11pt]{amsart}

%\usepackage[foot]{amsaddr}

\author{Matthew D. Kvalheim}
\address{Department of Mathematics and Statistics, University of Maryland, Baltimore County, MD, USA} 
\email{kvalheim@umbc.edu}

\author{Eduardo D. Sontag}
\address{Departments of Electrical and Computer Engineering and Bioengineering, and affiliate of Departments of Mathematics and
	Chemical Engineering, Northeastern University, Boston, MA, USA}
% commented out Harvard Med School (and MIT LIDS) to keep it simple
%. Laboratory of Systems Pharmacology, Harvardedical School, Boston, MA, USA}	
\email{sontag@sontaglab.org, e.sontag@northestern.edu, eduardo.sontag@gmail.com}

\usepackage{amsthm}
\usepackage{amsmath}
\usepackage{xcolor}
\usepackage{mathtools}


\newcommand{\MK}[1]{[{\color{magenta}(M.D.K.) \color{cyan}\textsc{#1}}]}
\newcommand{\EDS}[1]{[{\color{blue}(E.D.S.) \textsc{#1}}]}

\newcommand{\CW}[1]{[{\color{red}(CITE) \color{blue}\textsc{#1}}]}

\newcommand{\N}{\mathbb{N}}
\newcommand{\R}{\mathbb{R}}

\newcommand{\sph}{S}
\newcommand{\st}{M}
\newcommand{\fl}{\Phi}
\newcommand{\eq}{x}
\newcommand{\bas}{B}
\newcommand{\mat}{A}
\newcommand{\conj}{h}
\newcommand{\lyap}{V}
\newcommand{\level}{L}
\newcommand{\wil}{F}
\newcommand{\imp}{\tau}
\newcommand{\vfo}{f}
\newcommand{\vft}{g}

\newcommand{\F}{\mathcal{F}}

\theoremstyle{definition}
\newtheorem{Lem}{Lemma}
\newtheorem{Th}{Theorem}
\newtheorem{Co}{Corollary}
\newtheorem{Cl}{Claim}
\newtheorem{Prop}{Proposition}
\newtheorem{Quest}{Question}
\newtheorem*{Quest-non}{Question}
\newtheorem{Conj}{Conjecture}
\newtheorem{Rem}{Remark}

\newcommand{\concept}[1]{\emph{#1}}

%\iffalse
\usepackage{marvosym}
\usepackage[%hyperref
hypertexnames,%
citecolor=blue,%
colorlinks=true,%
linkcolor=red%
]{hyperref}
\usepackage[all]{hypcap}

\topmargin -0.5in
\textheight 9.0in
\oddsidemargin 0.25in
\evensidemargin 0.25in
\textwidth 6.25in
\parskip=5pt plus 1pt minus 1pt
\parindent0pt
%\fi



\title{Global linearization without hyperbolicity}
%\title{Hartman-Grobman linearization without hyperbolicity}
%\author{}

\begin{document}
	
	\begin{abstract}
We give a proof of an extension of the Hartman-Grobman theorem to nonhyperbolic but asymptotically stable equilibria of vector fields.
Moreover, the linearizing topological conjugacy is (i) defined on the entire basin of attraction if the vector field is complete, and (ii) a $C^{k\geq 1}$ diffeomorphism on the complement of the equilibrium if the vector field is $C^k$ and the underlying space is not $5$-dimensional.
We also show that  the $C^k$ statement in the $5$-dimensional case is equivalent to the $4$-dimensional smooth Poincar\'{e} conjecture.
	\end{abstract}
	
	\maketitle


\section{Introduction}\label{sec:intro}

Consider a nonlinear system of ordinary differential equations
\begin{equation}\label{eq:ode}
\dot{x}(t)=f(x(t)),	
\end{equation}
where $f$ is a vector field on an $n$-dimensional manifold $M$.
If $f\in C^1$ and $x_*\in M$ is a hyperbolic equilibrium for $f$, the Hartman-Grobman theorem guarantees existence of continuous local coordinates on a neighborhood of $x_*$ in which the \emph{nonlinear} dynamics \eqref{eq:ode} become \emph{linear} \cite{hartman1960lemma,grobman1959homeomorphism}.


In this paper, we provide a proof of an extension of the Hartman-Grobman theorem to nonhyperbolic but asymptotically stable equilibria.
We also assume only that $f$ is continuous and uniquely integrable (e.g. locally Lipschitz).
But if additionally $f\in C^k$ and $n\neq 5$, we construct continuous linearizing coordinates that are $C^k$ away from $x_*$.
Finally, if $f$ is complete, we prove the existence of globally linearizing coordinates on the entire basin of attraction of $x_*$.
In fact, the linear dynamics can be taken to be $\dot{y}=-y$, in which case the coordinates transform the nonlinear dynamics into the negative gradient flow of the convex function $y\mapsto \|y\|^2/2$.

To our knowledge, the first trace of these extensions appeared in work of Coleman, who proved a theorem equivalent to our local result for $k=0$ assuming $x_*$ has a Lyapunov function with level sets homeomorphic to spheres \cite{coleman1965local,coleman1966addendum} (see \cite[p.~247]{wilson1978reformulation}).
Our proof of the global result fills in details of one previously sketched by Gr\"une, Wirth, and the second author for $k=0$ and $n\neq 4$, and for general $k$ and $n\neq 4, 5$ \cite[remark in p.~133]{grune1999asymptotic}.
Using the same techniques and Perelman's solution to the $3$-dimensional Poincar\'{e} conjecture \cite{perelman2002entropy,perelman2003finite,perelman2003ricci}, we extend this global result to all $n$ for $k=0$, and to $n\neq 5$ for general $k$.
In fact, we prove that whether the $C^{k\geq 1}$ linearization results hold for $n=5$ is \emph{equivalent} to the $4$-dimensional smooth Poincar\'{e} conjecture (still open).
Jongeneel recently proved other interesting stability results using Perelman's solution \cite{jongeneel2024asymptotic}.
We also note connections to classical dynamical systems results showing that systems with globally asymptotically stable equilibria admit $C^0$ Lyapunov functions $V$ with exponential decrease $V(x(t)) = e^{-t}V(x(0))$ 
% above adapted from grune1999asymptotic, which also gives an earlier paper refernce to Szego
\cite[Chapter V.2]{BhatSzeg70}.
Furthermore, in the special case of linear systems, it has long been known that each linear system is equivalent, under a continuous coordinate change, to the system $\dot y = -y$ of the same dimension \cite{Arnold92}.

The local linearization result is Theorem~\ref{th:local-lin}, and the global result is Theorem~\ref{th:global-lin}.
While some techniques  in the hyperbolic case establish global linearizations by extending local ones \cite{lan2013linearization, eldering2018global,kvalheim2021existence}, in this paper we prove the global result directly (and under no hyperbolicity assumptions) and the local result as a consequence.
 
 It seems worth noting that Theorems~\ref{th:local-lin}, \ref{th:global-lin} give new existence results for targets of algorithms like extended Dynamic Mode Decomposition \cite{williams2015data} studied by applied Koopman operator theorists \cite{budisic2012appliedkoopmanism,mezic2020spectrum,brunton2022modern}.
 Such algorithms seek to compute $N\geq \dim M$ linearizing ``observables'' through which nonlinear systems appear linear.
 Theorems~\ref{th:local-lin}, \ref{th:global-lin} (see also Remark~\ref{rem:eigenfunctions-independent}) give existence results in the case $N=\dim M$, a case of arguably practical importance \cite{haller2024data} complementing recent (non)existence results for the case $N>\dim M$ \cite{liu2023non,kvalheim2024linearizability,arathoon2023koopman,liu2024properties}.

\section{Results}\label{sec:results}

We begin with the local linearization result, whose statement refers to the initial value problem
\begin{equation}\label{eq:ivp}
	\dot{x}(t)=f(x(t)),\quad x(0)=x_0.	
\end{equation}


\begin{Th}\label{th:local-lin}
	Let $x_*$ be an asymptotically stable equilibrium for a uniquely integrable continuous vector field $f$ on an $n$-dimensional $C^\infty$ manifold $M$.
	There is an open neighborhood $U\subset M$ of $x_*$ such that, for any Hurwitz matrix $A\in \R^{n\times n}$, there is a topological embedding $h\colon U\to\R^n$ such that, for all $x_0\in U$, the maximal solution of \eqref{eq:ivp} satisfies $x(t)\in U$ and
	\begin{equation}\label{eq:th:local-lin}
		\textnormal{$x(t)=h^{-1}(e^{At}h(x_0))$ \quad for all $t\geq 0$.}	
	\end{equation}
	Moreover, if $n\neq 5$ and $f\in C^k$ with $k\in \N_{\geq 1}\cup \{\infty\}$,  then there exists such an $h$ that additionally restricts to a $C^k$ embedding $U\setminus \{x_*\}\to \R^n\setminus \{0\}$.
\end{Th}

An advantage of Theorem~\ref{th:local-lin} is that it does not assume completeness of $f$.
On the other hand, the global linearization result below is formulated more optimally for smoothness, since the vector field generating a $C^k$ flow need only be $C^{k-1}$ in general.


\begin{Th}\label{th:global-lin}
	Let $x_*$ be an asymptotically stable equilibrium with basin of attraction $B$ for the flow $\Phi$ of a complete uniquely integrable continuous vector field on an $n$-dimensional $C^\infty$  manifold $M$.
	For any Hurwitz matrix $A\in \R^{n\times n}$ there is a homeomorphism $h\colon B\to \R^n$ satisfying
	\begin{equation}\label{eq:th:global-lin}
		%\Phi^t|_B = h^{-1} \circ e^{At}\circ h.
		\textnormal{$\Phi^t|_B = h^{-1} \circ e^{At}\circ h$ \quad for all \quad $t\in \R$.}
	\end{equation}
	Moreover, if $n\neq 5$ and $\Phi\in C^k$ with $k\in \N_{\geq 1}\cup \{\infty\}$,  then there exists such an $h$ that additionally restricts to a $C^k$ diffeomorphism $B\setminus \{x_*\}\to \R^n\setminus \{0\}$.
\end{Th}

\begin{Rem}\label{rem:gradient}
In particular, consider $A=-I_{n\times n}$ in the above theorems.
Then as noted in \S \ref{sec:intro}, $h$ transforms the nonlinear dynamics into the negative gradient flow of the dynamics $\dot{y}=-y$ of the convex function $y\mapsto \|y\|^2/2$.
This is distinct from the fact that a Riemannian metric always exists making the vector field a gradient on the complement of the equilibrium within the basin of attraction \cite[Thm~1]{barta2012lyapunov} in all dimensions (but for $n\neq 5$ this directly follows from the above theorems).
The same explanation shows that $h$ transforms the nonlinear dynamics into a ``contractive system'' \cite{sontag10yamamoto}.
\end{Rem}

\begin{Rem}\label{rem:eigenfunctions-independent}
By choosing $A$ to be diagonalizable over $\R$, the conclusion of Theorem~\ref{th:global-lin} furnishes $n$-tuples $(\psi_1,\ldots,\psi_n)$ of continuous real eigenfunctions of the Koopman operator such that $(\psi_1,\ldots,\psi_n)\colon B\to \R^n$ is a homeomorphism.
\end{Rem}

The final result establishes the mentioned relationship to the Poincar\'{e} conjecture.

\begin{Prop}\label{prop:poincare}
	Fix $k\in \N_{\geq 1}\cup \{\infty\}$.
	The $4$-dimensional $C^\infty$ Poincar\'{e} conjecture is true if and only if the $C^k$ statement of Theorem~\ref{th:local-lin} (or Theorem~\ref{th:global-lin}) is true for $n=5$.
\end{Prop}



\section{Proofs}\label{sec:proofs}

We first assume Theorem~\ref{th:global-lin} to give a short proof of Theorem~\ref{th:local-lin}.

\begin{proof}[Proof of Theorem~\ref{th:local-lin}]
	Fix any Hurwitz matrix $A\in \R^{n\times n}$.
	Let $\psi\colon M \to [0,\infty)$ be a $C^\infty$ function equal to $1$ on a neighborhood $U_0$ of $x_*$ and equal to zero outside of a compact subset of $M$.
	Then $\psi f$ is a complete uniquely integrable continuous vector field generating a flow $\Phi$,  $x_*$ is asymptotically stable for $\Phi$, and $\Phi\in C^k$ if $f\in C^k$.
	According to Wilson, there is a proper strict $C^\infty$ Lyapunov function $V\colon B\to [0,\infty)$ for $x_*$ and $\Phi$ \cite[Thm~3.2]{wilson1969smooth} (see also \cite[Sec.~6]{fathi2019smoothing}).
	Theorem~\ref{th:global-lin} furnishes a homeomorphism $h_0\colon B\to \R^n$ that satisfies \eqref{eq:th:global-lin} and, if $n\neq 5$, restricts to a $C^k$ diffeomorphism $B\setminus \{x_*\}\to \R^n\setminus \{0\}$.
	Let $c>0$ be sufficiently small that $U\coloneqq V^{-1}([0,c))$ is contained in $U_0\cap B$.
	Then $h\coloneqq h_0|_U\colon U\to \R^n$ is the desired embedding.
\end{proof}

We now prove Theorem~\ref{th:global-lin} (without assuming Theorem~\ref{th:local-lin}).

\begin{proof}[Proof of Theorem~\ref{th:global-lin}]
It suffices to find $h\colon B\to \R^n$ satisfying \eqref{eq:th:global-lin} for $A=-I$.
Indeed, repeating the proof with $B$ and $f$ replaced by $\R^n$ and $\tilde{f}(x)=Ax$ then yields a corresponding transformation $\tilde{h}\colon \R^n\to \R^n$, so the composition $\tilde{h}^{-1}\circ h\colon B\to \R^n$  satisfies \eqref{eq:th:global-lin} for general $A$.

Let  $V\colon B\to [0,\infty)$ be a proper strict $C^\infty$ Lyapunov function for $x_*$ and $\Phi$ (\cite[Thm~3.2]{wilson1969smooth}, \cite[Sec.~6]{fathi2019smoothing}).
Fix any $c>0$ and note that $V^{-1}([0,c])$ is a contractible compact $C^\infty$ embedded submanifold with codimension-$1$ boundary $L\coloneqq V^{-1}(c)$ homotopy equivalent to the $(n-1)$-sphere $S^{n-1}\coloneqq \{y\in \R^n\colon \|y\|=1\}$ \cite[pp.~326--327]{wilson1967structure}.
Thus, if $n\neq 5$, there is a $C^\infty$ diffeomorphism $P\colon L\to S^{n-1}$ according to classical facts for $n=2$ \cite[Appendix]{milnor1997topology} and $n=3$  \cite[Thm~9.3.11]{hirsch1994differential}, Perelman for $n=4$ \cite{perelman2002entropy,perelman2003ricci,perelman2003finite} (see \cite[Cor.~0.2]{morgan2007ricci}), and Smale for $n\geq 6$ \cite[Thm~5.1]{smale1962structure}.
If $n=5$ there is still a homeomorphism $P\colon L\to S^{n-1}$ according to Freedman \cite[Thm~1.6]{freedman1982topology}.

Since for each $x\in B\setminus \{x_*\}$ the trajectory $t\mapsto \Phi^t(x)$  converges to $x_*$ and crosses $L$ exactly once and transversely, the map $\R\times L\to B \setminus \{x_*\}$ defined by $(t,x)\mapsto \Phi^t(x)$ is a homeomorphism and $C^k$ diffeomorphism if $\Phi\in C^k$, with inverse $g=(\tau,\rho)\colon B\setminus \{x_*\}\to \R\times L$ satisfying $\tau(x)\to -\infty$ as $x\to x_*$ (cf. \cite[p.~327]{wilson1967structure}).
% defined h as follows: composition B -> RxL -> RxS^{n-1} -> R^n (extended to 0) given by maps (tau,rho), (IxP), and (r,v) |-> e^r v with the first one having the inverse (r,v) |-> phi_r(v)
Thus $h\colon B\to \R^n$ defined by $h(x_*)=0$ and
\begin{equation*}
	h(x)=e^{\tau(x)}P( \rho(x))
\end{equation*}
is a homeomorphism that, if $n\neq 5$ and $\Phi\in C^k$, restricts to a $C^k$ diffeomorphism $B\setminus \{x_*\}\to \R^n\setminus \{0\}$. 
Finally, $h$ satisfies \eqref{eq:th:global-lin} with $A=-I_{n\times n}$ since $\rho\circ \Phi^t|_{B\setminus \{x_*\}}=\rho$ and $\tau \circ \Phi^t|_{B\setminus \{x_*\}}=\tau - t$.
\end{proof}

Finally, we prove Proposition~\ref{prop:poincare}.

\begin{proof}[Proof of Proposition~\ref{prop:poincare}]
Assume that the $4$-dimensional $C^\infty$ Poincar\'{e} conjecture is true.
Then $P\colon L\to S^{4}$ in the proof of Theorem~\ref{th:global-lin} can be taken to be a diffeomorphism when $n=5$.
With this fact, repeating the proofs of Theorem~\ref{th:global-lin} and Theorem~\ref{th:local-lin} verbatim show that their $C^k$ statements are true for $n=5$.

Conversely, assume that the $C^k$ statement of Theorem~\ref{th:local-lin} (or Theorem~\ref{th:global-lin}) is true for $n=5$ and $A=-I_{n\times n}$.
Let $L$ be any $4$-dimensional $C^\infty$ homotopy sphere.
According to Hirsch \cite[Thm~2]{hirsch1965homotopy}, there is a $C^\infty$ function $V\colon S^5\to [0,1]$ having only two critical points, a maximum $N=V^{-1}(1)$ and minimum $S=V^{-1}(0)$, such that there is a diffeomorphism $P_c\colon L\to V^{-1}(c)$ for any $c\in (0,1)$.

Equip $S^5$ with a Riemannian metric and let $\Phi$ be the flow of the complete $C^\infty$ vector field $f=-\nabla V$ on $S^5$.
Observe that $S\in S^5$ is asymptotically stable with basin of attraction $B\coloneqq S^5\setminus \{N\}$ and that $f$ points inward at the boundary $V^{-1}(c)$ of $V^{-1}([0,c])$ for any $c\in (0,1)$.
Thus, Theorem~\ref{th:local-lin} (or Theorem~\ref{th:global-lin})  furnishes an open neighborhood $U\subset S^5$ of $S$ and a $C^k$ embedding $h\colon U\to \R^5$ such that, for any $c>0$ small enough that $V^{-1}(c)\subset U$, $h$ diffeomorphically maps $V^{-1}(c)$ onto a $C^k$ embedded submanifold $L_c\subset \R^5$ intersecting every line through the origin in a single point and transversely.
Fixing such a small $c>0$ and letting $\rho\colon \R^5\setminus \{0\}\to S^{4}$ be the straight line retraction $\rho(y)=\frac{y}{\|y\|}$, it follows that
\begin{equation*}
	L\xrightarrow{P_c} V^{-1}(c) \xrightarrow{h}L_c  \xrightarrow{\rho} S^{4}
\end{equation*}
is a well-defined $C^k$ diffeomorphism. % for any $c\in (0,1)$ small enough that $V^{-1}(c)\subset U$.
Since $L$ is a $C^\infty$ manifold $C^k$ diffeomorphic to $S^4$, $L$ is $C^\infty$ diffeomorphic to $S^4$ \cite[Thm~2.2.7]{hirsch1994differential}. 
\end{proof}

\section*{Additional Comments}

The paper \cite{grune1999asymptotic} dealt, more generally, with systems $\dot x = f(x,u)$, where $u$ denotes an input or disturbance. That paper showed that the input to state stability property \cite{mct} is equivalent, under similar global changes of variables, to finiteness of the $L^2$ norm (``$H_\infty$ gain'') of the input/state operator $(x(0),u(\cdot))\mapsto x(\cdot)$. As remarked in \cite{grune1999asymptotic}, however, the generalization of linearization constructions to systems with disturbances is not immediate. We thus leave open the study of such extensions.

We also note the relationship between the work reported here and areas of current machine learning research. In \cite[Chapter 6]{bramburger} one finds a discussion of ``autoencoder'' deep neural networks for the  numerical approximation of Grobman-like conjugacies for linearizations and Koopman eigenvalues. Our results, especially in combination with new theoretical results about the existence of such autoencoders \cite{kvalheim_sontag_2023autoencoders}, contributes to the theoretical foundation for such studies.

Finally, the proof of Theorem~\ref{th:global-lin} is closely related to the proof of \cite[Prop.~1]{grune1999asymptotic}, which can be viewed as a (global) extension of the Morse lemma \cite[Lem.~2.2]{milnor1963morse} to local minima of non-Morse functions.
Incorporating Perelman's result as in the proof of Theorem~\ref{th:global-lin}, the proof of \cite[Prop.~1]{grune1999asymptotic} can otherwise be repeated verbatim to prove Proposition~\ref{prop:gen-morse} below.
It extends the two-part statement \cite[Prop.~1]{grune1999asymptotic} by removing the hypothesis ``$n\neq 4$'' from the first part and relaxing the hypothesis ``$n\neq 4,5$'' from the second part to ``$n\neq 5$'', which reflects Perelman's result.
(It also contains the superficial extension of replacing the domain $\R^n$ of $V$ with a $C^\infty$ manifold $M$, but the hypotheses imply that $M$ is diffeomorphic to $\R^n$.)
Proposition~\ref{prop:gen-morse} is a global statement, but it readily implies a local statement in a manner similar to the implication of Theorem~\ref{th:local-lin} by Theorem~\ref{th:global-lin}.

For the following statement, a class $\mathcal{K}_\infty$ function is a strictly increasing and continuous function $\gamma\colon [0,\infty)\to [0,\infty)$ satisfying $\lim_{s\to\infty}\gamma(s)=\infty$.

\begin{Prop}\label{prop:gen-morse}
Let $x_*$ be the unique critical point of a proper $C^1$ function $V\colon M\to \R$ on a connected $n$-dimensional $C^\infty$ manifold $M$. 
Assume furthermore that $V$ is $C^\infty$ on $M\setminus \{x_*\}$.
Then for each class $\mathcal{K}_\infty$ function $\gamma$ that is $C^\infty$ on $(0,\infty)$, there exists a homeomorphism $T\colon M\to \R^n$ with $T(x_*)=0$ such that
\begin{equation*}
    V\circ T^{-1}(y)=\gamma(\|y\|).
\end{equation*}
In particular this holds for $\gamma(\|y\|)=\|y\|^2$.

If $n\neq 5$ then $T$ can be chosen to restrict to a $C^\infty$ diffeomorphism  $M\setminus \{x_*\} \to \R^n\setminus \{0\}$.
Furthermore, in this case there exists a class $\mathcal{K}_\infty$ function $\gamma$ which is $C^\infty$ on $(0,\infty)$ and satisfies $\gamma(s)/\gamma'(s)\geq s$ such that $T$ is $C^1$ with $DT(0)=0$.
\end{Prop}
%MK Note: I believe this proposition can be extended to allow V to be C^k on the complement of x_* rather than C^inf, but since grad V is then only C^{k-1} it seems to me that the proof of [GSW99, Prop 1] only yields a transformation T that is C^{k-1} on the complement of x_*.

\section*{Acknowledgments} This material is based upon work supported by the Air Force Office of Scientific Research under award number FA9550-24-1-0299 to Kvalheim and award number AFOSR FA9550-21-1-0289 to Sontag. 

	
\bibliographystyle{amsalpha}
\bibliography{global_linearization_kvalheim_sontag}
	
	
\end{document}
