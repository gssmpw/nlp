\appendix


\begin{figure*}[t]
    \centering
    \includegraphics[width=\textwidth]{figs/additional_example.pdf}
    \caption{
    An illustration example where the summary $\pi_{2}$ is removed due to the unsatisfiable summary condition $\phi_{\pi_{2}}$.
    Bottom-up analysis for the code shown in (a).
    The (b) shows the corresponding program dependence graph (PDG). 
    %In the PDG, solid black arrows ($\rightarrow$) are data dependence edges, while dashed curved arrows ($\dashrightarrow$) depict control dependence edges. 
    %The guards for data dependence are labeled.
    (c) shows the function summaries collected during the bottom-up analysis.}
    \label{fig:additional_ex}
    \vspace{-0.4cm}
\end{figure*}

The appendix includes an example that is removed because of the unsatisfiable summary condition in Section~\ref{app:unsat-summary},
soundness proof of contribution identification algorithm~\ref{alg:main}
in Section~\ref{app:sound} and a comparison of results of Fusion, Light-Fusion, and \textit{PhASAR} in Section~\ref{app:comparison}.

\subsection{Pruning Unsatisfied Summary}
\label{app:unsat-summary}
In traditional path-sensitive value flow analysis, if a summary condition is unsatisfiable(\textit{unsat}), it is removed to avoid maintaining an unfeasible summary.
Figure~\ref{fig:additional_ex} provides an example of this.
The summary $\pi_{2}$ that is collected from function ~\textit{bar} will be discarded because its summary condition $\phi_{\pi_{2}}$ is \textit{unsat}.
The summary condition contains clauses indicating that $a_3$ could be a NULL value and not be a NULL value at the same time, as highlighted in the dashed box.
As a result, the function \textit{foo} clones no summary from the function \textit{bar}, and no feasible source-sink path is detected in the case.
% However, our contribution identification algorithm focuses on removing summaries with satisfied summary conditions that do not contribute to source-sink paths.


\subsection{Soundness Proof of Algorithm~\ref{alg:main}}
\label{app:sound}
\begin{theorem}[Soundness]
\label{thm:sound}

Given $V^{\texttt{N}}$ identified, for any function $f \in P$, if a summary $s=(\pi, \phi)$ is being collected and neither $\pi[0]$ nor $\pi[-1]$ appear in $V^{\texttt{N}}$, it must be a non-contributing summary for function $f$. Canceling the corresponding operations does not affect $S(V_{\text{src}}, V_{\text{sink}})$.
\begin{proof}

Proof by contradiction.

In Algorithm 1, suppose the summary $s$ that is collected between $V_{h}^{f}$ and $V_{t}^{f}$ is a contributing summary, even though $\pi[0] \notin V^{\texttt{N}}$ and $\pi[-1] \notin V^{\texttt{N}}$. 
Thus, the source-sink pair $(src, sink)$ or a guard vertex $g$ labeled on the source-sink path that reaches the head $\pi[0]$ or tail $\pi[-1]$ are not in $V^{\texttt{N}}$. 
However, if such a head or tail vertex exists, Algorithm 2 would include it in $V^{\texttt{N}}$: either on Line 12 in the procedure \texttt{identifyPathContrib}, or on Line 16 in the procedure \texttt{identifyCondContrib}.
Therefore, we arrive at a contradiction, as this implies that $\pi[0] \in V^{\texttt{N}}$ or $\pi[-1] \in V^{\texttt{N}}$. 

In Algorithm 1, when collecting the summary $s$ by concatenating it with the summaries that are cloned from callee functions, following the same judgment conditions as described in the previous case, if it is determined to be a non-contributing summary, the operation of cloning $s^{c}$ should be canceled.
\end{proof}
\end{theorem}


\subsection{Comparison Results of Fusion, Light-Fusion, and \textit{PhASAR}}
\label{app:comparison}

Table~\ref{table:report} shows the classification results of unsafe sinks and sinks reported by Fusion, Light-Fusion, and \textit{PhASAR},
by examining unsafe sinks and safe sinks reported by both tools (LF-US$\cap$P-US and LF-S$\cap$P-S columns), and unsafe sinks reported by one tool but not the other (LF-US$\cap$P-S and LF-S$\cap$P-US columns).

First, the unsafe sinks reported by Fusion (F-US) and Light-Fusion (LF-US) are identical, demonstrating the precision-preserving nature of our approach.


\begin{table}[t] \small
\renewcommand\thetable{A.1}
\caption{Classification of unsafe sinks (US) and safe sinks (S) reported by Fusion (F), Light-Fusion (LF), and \textit{PhASAR} (P).
}
\label{table:report}
\resizebox{\columnwidth}{!} {
\begin{tabular}{r|rccr|crcc}
\hline
ID & \multicolumn{1}{c}{\#Sinks} & \multicolumn{1}{c}{F-US} & LF-US & \multicolumn{1}{c|}{P-US} & \multicolumn{1}{c}{LF-US$\cap$P-US} & \multicolumn{1}{c}{LF-S$\cap$P-S} & \multicolumn{1}{c}{LF-S$\cap$P-US} & \multicolumn{1}{c}{LF-US$\cap$P-S} \\ \hline
1  & 3184                        & 2                        & 2     & 11                        & 2                                    & 3173                               & 9                                   & 0                                   \\
2  & 3581                        & 9                        & 9     & 86                        & 9                                    & 3,495                              & 77                                  & 0                                   \\
3  & 7657                        & 0                        & 0     & 5                         & 0                                    & 7652                               & 5                                   & 0                                   \\
4  & 4273                        & 1                        & 1     & 261                       & 0                                    & 4012                               & 261                                 & 1                                   \\
5  & 27489                       & 28                       & 28    & 207                       & 21                                   & 27282                              & 186                                 & 7                                   \\
6  & 14490                       & 31                       & 31    & 93                        & 24                                   & 14397                              & 69                                  & 7                                   \\
7  & 49769                       & 26                       & 26    & 378                       & 24                                   & 49391                              & 354                                 & 2                                   \\
8  & 12156                       & 12                       & 12    & 111                       & 10                                   & 12045                              & 101                                 & 2                                   \\
9  & 40888                       & 19                       & 19    & 669                       & 16                                   & 40219                              & 653                                 & 3                                   \\
10 & 3418                        & 8                        & 8     & 217                       & 4                                    & 3201                               & 213                                 & 4                                   \\
11 & 215747                      & 35                       & 35    & 103                       & 35                                   & 215644                             & 68                                  & 0                                   \\
12 & 77705                       & 28                       & 28    & 281                       & 24                                   & 77424                              & 257                                 & 4                                   \\
13 & 146208                      & 79                       & 79    & 3592                      & 76                                   & 142616                             & 3516                                & 3                                   \\
14 & 182621                      & 65                       & 65    & 3902                      & 63                                   & 178719                             & 3839                                & 2                                   \\ \hline
15 & 76727                       & 103                      & 103   & 2382                      & 93                                  & 74345                              & 2289                                & 10                                  \\
16 & 146284                      & 92                       & 92    & 1102                      & 84                                   & 145182                             & 1018                                & 8                                   \\
17 & 215431                      & 168                      & 168   & 2007                      & 162                                  & 213424                             & 1845                                & 6                                   \\ \hline
\end{tabular}
}
\vspace{-0.4cm}
\end{table}

Second, as shown in the LF-S$\cap$P-US column, \textit{PhASAR} mistakenly reports a large number of sinks as unsafe, resulting in high false positive rates due to not collecting and verifying the path conditions.
The majority of these sinks actually cannot be successfully reached by sources due to unsatisfiable path conditions.
Different from the Phasar, Light-Fusion considers path conditions, thereby excluding the mistakenly reported sinks from the top-down approaches. 

Additionally, \textit{PhASAR} fails to report some unsafe sinks that are reported by Light-Fusion as shown in the LF-US$\cap$P-S column.