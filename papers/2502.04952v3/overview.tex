\section{Overview}
\label{sec:overview}
In this section, we illustrate the problem using the motivating example and briefly describe our key idea.
% \vspace{-0.5em}
% \subsection{The Explosive Summary Problem}

% We reuse the example shown in Fig.~\ref{fig:motivation_ex}(a) to show the previous limitation.
% Specifically, the code snippet consists of four functions: \textit{foo}, \textit{bar}, \textit{baz}, and \textit{qux}, where \textit{bar}, \textit{baz}, and \textit{qux} are callees of \textit{foo}.
% % Specifically, the function \textit{qux} is invoked twice by \textit{foo}, and variables $a$ and $b$ are assigned to receive their return values.
% % In function \textit{qux}, \textit{malloc} may create a $NULL$ value, which is the source, for variable $m$ and return it to callers.
% % After that, variable $a$ is passed to \textit{bar} as an actual parameter in Line 6 and finally dereferenced by the \textit{printf} statement in Line 9.
% % In function \textit{bar}, the formal parameter $p$ is dereferenced by the \textit{printf} statement under the branch conditions $\varphi_2$.
% % Variable $b$ is passed to \textit{baz} as an actual parameter in Line 6.
% % The return value from \textit{baz} is assigned to variable $e$, which is used in branch condition $ e \neq NULL_8 $ in Line 8 to guard against the assignment between variables $d$ and $c$.
% % In function \textit{baz}, the formal parameter $f$ is returned to callers.
% The program contains two source-sink paths denoted as the summary paths $\pi_4$ and $\pi_5$, shown in Fig.~\ref{fig:motivation_ex}(c),
% while only the $\pi_5$ will be reported as the NPD bug after verifying their summary conditions.

% Following the existing efforts on value flow analysis~\cite{shi2018pinpoint, shi2021path}, the resulting PDG is depicted in Fig.~\ref{fig:motivation_ex}(b),
% % Particularly, an edge $(u_i, v_j)$ labeled a guard $\phi$ in the PDG indicates that the value $u_i$ can flow to $v_j$ when $\phi$ holds, where $i$ and $j$ denote the positions in the control flow graph.
% To detect the NPD bug for the given program, existing path-sensitive value flow approaches~\cite{babic2008calysto, xie2005scalable, shi2018pinpoint} analyze the program in a compositional manner.
% Specifically, they first analyze three callee functions (\textit{bar}, \textit{baz}, and \textit{qux}), collect the intra-procedural value flow path between parameters and returns, and instantiate the path conditions as function summaries. 
% With the callee function summaries in hand, the function \textit{foo} is then analyzed.
% % The summary $\pi_1$ is generated for the function \textit{qux}, summarizing the creation of a null pointer through \textit{malloc} in Line 19.
% % When analyzing function \textit{foo}, this summary is cloned twice to complete the origin path of source value $NULL$ along with the propagation of different variables $a$ and $b$. 
% % The counterparts of these clones are underlined in the summary: $\pi_4$, $\pi_5$, and $\pi_6$.
% % Summary $\pi_3$ summarizes the propagation path of the formal parameter $p$ for the function \textit{bar}.
% % The guard $\varphi_2$ that is labeled on the data dependence edge is instantiated by traversing the corresponding vertices on the data dependence graph, resulting in $\varphi_2 \equiv p_{11} \neq NULL_{11}$.
% % They are cloned into function \textit{foo} when passing $a$ as an actual parameter to function \textit{bar}, completing the summaries $\pi_4$ 
% % and its summary condition $\phi_{\pi_4}$, where the constraints represented by guard $\varphi_2$ are conjuncted.
% % The propagation path of variable $f$ within function \textit{baz} is summarized as $\pi_2$. This summary is cloned to complete summary $\pi_6$ summarizing the propagation path of variable $b$.
% The summaries are collected by traversing the edge and instantiating the guard.
% As shown in Fig.~\ref{fig:motivation_ex}(c), a total of seven summaries are collected.
% The summary $\pi_{1}$ is collected from the function \textit{qux} from the source site to return.
% The summary $\pi_{2}$ is collected from the function \textit{baz} from the parameter to return.
% The summary $\pi_{3}$ is collected from the function \textit{bar} from the parameter to a sink site.
% Notice that the summary path, $\pi_7$, does not involve propagation paths across callees. 
% However, its summary condition $\varphi_{1} \equiv e_8 \neq NULL_8$ relies on the origin path of variable $e$, which is summarized by $\pi_6$. 
% Thus, the summary condition $\phi_{\pi_7}$ is formed by conjuncting the summary condition $\phi_{\pi_6}$ with the constraints represented by the guard $\varphi_1$.
% By the end of analyzing function \textit{foo}, the two source-sink propagation paths of NPD are obtained and denoted as $\pi_4$ and $\pi_5$.

\subsection{Explosive Summary Problem}
To detect the bug, the bottom-up collection of function summaries outlined in Algorithm~\ref{alg:ps-vf-analysis} can collect and maintain a superset of the function summaries that are actually required.
As highlighted in red in Fig.~\ref{fig:motivation_ex}(c), the summaries $s_2$, $s_6$, and $s_7$ do not contribute to the two source-sink paths, $s_{4}$ and $s_{5}$, for detecting the NPD bug, i.e., either as components of these paths or their associated path conditions.
Specifically, non-contributing summaries can arise in two scenarios in Algorithm~\ref{alg:ps-vf-analysis}.
First, when collecting three types of summaries, over-summarization can occur.
For instance,
when analyzing the function $baz$, there is no prior knowledge of which the specific summary contributes, resulting in the conservative collection of all summaries within it.
Consequently, the non-contributing summary $s_2$ is collected as a transfer summary.
Second, non-contributing summaries can be induced through the cloning of callee summaries. 
For example, the summary $s_6$ is a non-contributing summary generated by cloning and concatenating with the callee summary $s_2$. 
Additionally, given that there are no source-sink paths within the function $boo$ that rely on $s_2$, and considering that $s_6$ is deemed non-contributory, the cloning of $s_2$ from function $baz$ should be avoided during the analysis of \textit{foo}.
As summaries are maintained and cloned into higher-level functions, the size of $S^f$ can exponentially increase due to the explosion of paths.
More importantly, collecting non-contributing summaries introduces expensive constraint-solving.
To sum up, collecting and cloning only the contributing summaries can significantly improve scalability.
The challenge lies in identifying redundant summaries precisely and efficiently without hurting the precision of the analysis.

% Specifically, the analysis of the function \textit{baz} that produces the summary $\pi_2$, cloning of $\pi_{1}^{\prime}$, collecting of summaries $\pi_6$ and $\pi_7$, and solving their summary condition can be unnecessary.
% Consequently, the computation for collecting and maintaining the redundant summaries can be avoided.

% \subsection{Challenges}
% Existing efforts to enhance performance primarily focus on parallelizing the collection of function summaries~\cite{shi2020pipelining, tang2023Scaling} and using graph representations of summary conditions~\cite{shi2021path}. 
% However, these optimizations cannot prevent the generation of redundant summaries. 
% In our evaluation, even after adopting these optimizations in our baseline implementation, about 20\% of redundant summaries remain. 
% Currently, there is NO solution to eliminate these redundant summaries and reduce the size of the summaries efficiently.
% However, efficiently identifying redundant summaries during analysis without sacrificing precision is challenging. 

% \textbf{Challenge 1.}
% First, determining non-contributing summaries necessitates the prior instantiation of source-sink paths.
% As demonstrated in Algorithm~\ref{alg:ps-vf-analysis}, summaries are computed using a bottom-up approach, analyzing the source-sink paths from shorter to longer summaries. 
% This indicates that non-contributing summaries are always computed before the construction of source-sink paths. 
% Hence, it is difficult to identify non-contributing summaries ahead of constructing source-sink paths in the conventional value flow analysis. 

% \textbf{Challenge 2.}
% Second, a summary contributing to a source-sink path could be implicit.
% As mentioned, a summary $s_i$ is applied to the instantiation of summary $s_{j}$'s condition without being concatenated to $s_{j}$'s path aspect. This means that $s_i$ has an implicit contribution to the source-sink path if $s_j$ is a component of a source-sink path. 
% As shown in Example~\ref{example:path_condition}, $\pi_6$ is not concatenated to $\pi_7$, but $\varphi_{\pi_{6}}$ is used to concatenate $\varphi_{\pi_{7}}$.

% \textbf{Challenge 3.}
% Additionally, the identification algorithm should be carefully designed to recognize as many non-contributing summaries as possible, ensuring that its overhead does not outweigh the redundant computation it aims to eliminate. 
% At the same time, it must maintain precision, meaning that contributing summaries should not be mistakenly identified as non-contributing summaries.


% To prevent non-contributing summaries, it is necessary to take a preliminary step during the collection of $S^{f}$ and the cloning of callee's summary $S^{c}$ in Algorithm~\ref{alg:ps-vf-analysis}. 
% Each summary is evaluated thoroughly to ensure that only the contributing summary is cloned and collected.
% However, this assessment introduces a contradiction, as it relies on the assumption that $S(V_{\text{src}}, V_{\text{sink}})$ is obtained as defined in Definition~\ref{def:contributing_summary}.


\subsection{Removing Redundant Summaries}

We first propose the key idea of assessing whether a summary is contributing or not by solving two graph reachability problems between heads and tails of the summary $s$ with distinct reaching targets (source-sink pairs or guards) without computing any summaries.
Based on the graph reachability abstraction, we design the contribution identification algorithm, which identifies a set of necessary head and tail vertices for identifying the contributing summaries. 
Summaries that are collected outside these necessary head and tail vertices are identified as non-contributing automatically.
Next, we explain how our graph reachability abstraction and algorithmic design overcome the above challenges.

\textbf{Contributing Summmary.}
Our key observation is that whether a summary is contributing hinges on two aspects: its path contribution, where it must be a component of source-sink paths, and its condition contribution, where it must be involved in the path conditions associated with a source-sink path.
Thus, we establish the definition of a contributing summary generated from a function based on its contribution to the source-sink paths $S(V_{\text{src}}, V_{\text{sink}})$.

\begin{definition}[Contributing Summary] \label{def:contributing_summary}
A summary $s \in S^f$ is considered a contributing summary for function $f$ if it satisfies at least one of the following criteria:
(1) Path Contribution: The summary path $\pi$ is used to connect at least one source-sink path, i.e., $\pi \in \Pi(V_{\text{src}}, V_{\text{sink}})$.
(2) Condition Contribution: The summary path $\pi$ is used to instantiate at least one guard vertex labeled along the source-sink paths, i.e., $\phi_{\pi} \in \Phi(V_{\text{src}}, V_{\text{sink}}, V_{\text{g}})$.
\end{definition}

We use Fig.~\ref{fig:motivation_ex} as an example.
The path $\pi_{1}$ is reachable from both source-sink pairs $(NULL_{19}, printf(*p_{13}))$ and $(NULL_{19}, printf(*a_9))$.
In addition, the path $\pi_3$ is reachable from the source-sink pair $(NULL_{19}, printf(*p_{13}))$.
Therefore, they are identified as contributing summaries; indeed, they are components of two source-sink paths, $\pi_4$ and $\pi_5$.
Comparatively, the paths $\pi_{2}$, $\pi_{6}$, and $\pi_{7}$ are neither reachable by any pair of sources and sinks nor reachable by the guard vertex $\varphi_2$ labeled on the source-sink path $\pi_4$. 
Thus, these paths $\pi_{2}$, $\pi_{6}$, and $\pi_{7}$ are identified as non-contributing summaries.
Additionally, despite the reachability of the summary $\pi_{6}$ from the guard vertex $\varphi_1$ labeled on $\pi_7$, more specifically, the tail of $\pi_{6}$, denoted as $(e_7)$, being reachable from $\varphi_1$, the summary $\pi_{7}$ does not contribute to any source-sink path or conditions. 
Consequently, it becomes redundant for NPD detection.

In summary, the summary contribution is identified by assessing two reachabilities between specific heads and tails of the summary path with various targets:
\begin{enumerate}
    \item Path contribution: If the summary $s=(\pi, \phi_{\pi})$ has the path contribution, a source-sink pair ($(src, sink) \in V_{\text{src}} \times V_{\text{sink}}$) exists, where $src$ can reach both $\pi[0]$ and $\pi[-1]$, and $sink$ is reached by both $\pi[0]$ and $\pi[-1]$.
   \item Condition contribution: If the summary $s=(\pi, \phi_{\pi})$ has the condition contribution criteria, there exists a guard vertex $\text{g} \in V_{\text{g}}$ from $\Phi(V_{\text{src}}, V_{\text{sink}}, V_{\text{g}})$ that are reached by $\pi[0]$ and $\pi[-1]$.
\end{enumerate}

With the above abstraction, the assessment of contributing summaries is reduced to two graph reachability problems. 
In Section~\ref{sec:contribution-indentify}, we give a sound, efficient, and effective contribution identification algorithm by applying the abstraction.
% \textit{How to preserve the precision of the analysis?}
%     To ensure this, instead of directly utilizing the abstractions to identify non-contributing summaries,
%     \tool\ uses the abstractions to soundly identify all necessary head vertices $V^{*}_{h}$ and tail vertices $V^{*}_{t}$ that are reached by source-sink pairs (path contribution) as well as guard vertices $V^{*}_{g}$ that are labeled on source-sink paths (condition contribution).
%     This means that contributing summaries can only be collected within necessary vertices, $V^{*}_{h} \cup V^{*}_{t} \cup V^{*}_{g}$, which we denote as $V^{N}$.
%     Therefore, summaries contributing to source-sink paths can be collected within $V^{N}$ as in traditional methods.
%     In constrast, summaries collected outside $V^{N}$ are considered non-contributing summaries.

    
    % \textit{How to achieve the identification both efficiently and effectively?}
    % The identification process relies on graph reachability. 
    % More precise graph reachability results in fewer necessary vertices $V^{N}$ being identified, allowing for recognizing more non-contributing summaries starting outside of $V^{N}$. 
    % However, using advanced reachability algorithms increases the identification overhead. 
    % The complexity of more advanced reachability algorithms often outweighs the precision gains they can provide. 
    % To strike a balance, i.e., spending minimal overhead while significantly boosting the efficiency of path-sensitive analysis, we select the classic breadth-first search (BFS) algorithm for implementing our abstractions. We give more discussion about this in Section~\ref{sec:discussion}.





% \textbf{Problem Statement:} Our research goal is to identify non-contributing summaries without the need to collect $S(V_{\text{src}}, V_{\text{sink}})$. 
% It is crucial for the identification process to be efficient, ensuring that the cost of identification is lower than the potential redundancy it aims to prevent.






% our key insight is that a useful summary should \emph{be a component of} paths or path conditions associated with an interesting source-sink path.
% Specifically, there are two implications from the perspective of PDG reachability:

% \begin{enumerate}
%     \item First, any summary should be reachable from at least one pair of sources and sinks if it is used to concatenate a source-sink path. That is to say, we could determine the usefulness of a summary by pre-computing the reachability of the summary to any source-sink pairs.
%    \item Second, for any summary to be derived from the path condition of source-sink paths, the summary should be reachable from the guards annotated on those paths.
%    Note that the guards are also vertices in PDG.
% \end{enumerate}


% These two observations suggest the chance for abstracting the identification of the summary contribution by assessing the reachability between specific vertices on the summary path with varying targets.
% We use Fig.~\ref{fig:motivation_ex} as an example.
% The path $\pi_{1}$ is reachable from the both source-sink pairs $(NULL_{19}, printf(*p_{13}))$ and $(NULL_{19}, printf(*a_9))$.
% In addition, the path $\pi_3$ is reachable from the source-sink pair $(NULL_{19}, printf(*p_{13}))$.
% Therefore, they are identified as useful summaries and, indeed, they are components of two source-sink paths, $\pi_4$ and $\pi_5$.
% Comparatively, the paths $\pi_{2}$, $\pi_{6}$, and $\pi_{7}$ are neither reachable by any pair of sources and sinks nor reachable by the guard vertex $\varphi_2$ labeled on the source-sink path $\pi_4$. 
% Thus, these paths $\pi_{2}$, $\pi_{6}$, and $\pi_{7}$ are identified as "useless" summaries.
% Additionally, despite the reachability of summary $\pi_{6}$ from the guard vertex $\varphi_1$ labeled on $\pi_7$, more specifically, the tail of $\pi_{6}$, denoted as $(e_8)$, being reachable from $\varphi_1$, it is worth noting that the summary $\pi_{7}$ does not contribute to any source-sink path or conditions. 
% Consequently, it becomes redundant and serves no purpose in NPD detection.




% Furthermore, even though the summary $\pi_{6}$ is reachable from the guard vertex $\varphi_1$ labeled on $\pi_7$, specifically the tail of $\pi_{6}$, $(e_8)$, is reachable from $\varphi_1$, the summary $\pi_{7}$ does not actually contribute to any source-sink path or conditions, rendering it useless for NPD detection.


% ~\cai{Rewrite the below; you should give two different examples to illustrate the two observations using Fig.1 }.

% Furthermore, we can cut down on even more computations that are not needed by not cloning $\pi_1$ and collecting $\pi_2$ to finish the "useless" $\pi_6$ and $\pi_7$.

% For instance, in the case of summaries $\pi_1$ and $\pi_2$ implicitly contributing to $\pi_7$, both summaries can be reached from the guard $\varphi_1$ annotated on the path edge of $\pi_7$ in the program dependence graph.



