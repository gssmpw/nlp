\section{Related Work}
% \vspace{-0.2em}
\textbf{Compositional Analysis.}
Many compositional analyses aim to improve efficiency by reusing information within a procedure as summaries____, which include top-down and bottom-up summaries.
Most program analyses prefer a bottom-up approach____. 
In these approaches, a function's effect is represented using bottom-up summaries, and the summary of a callee is inlined into the summary of its caller. 
This avoids redundant analysis of individual functions.
However, when collecting the summary of a callee, it is challenging to determine whether the summary will be useful to callers since the callers are typically analyzed afterward. 
Our work employs a lightweight analysis to pre-compute a set of vertices, which allows us to determine whether a summary should be computed and reduces unnecessary computations.


\textbf{Value Flow Analysis.}
Cherem et al.____ utilized value flow analysis to detect software bugs such as memory leaks. 
Subsequently, several works have aimed to refine the recall and precision of this analysis____. 
However, most of these analyses are not inter-procedurally path-sensitive, with the exceptions of Pinpoint____ and Fusion____.
Fusion is an optimization of the performance issues identified in Pinpoint, which were caused by the explosion of summaries and paths in inter-procedural analysis. Fusion addresses this problem by eliminating the storage of path conditions. 
However, it is worth noting that even with Fusion, there are still redundant summaries and paths being computed in function summaries, which cannot be fully optimized.
While this work helps Fusion overcome redundancy issues related to function summaries.
% it is acknowledged that there are still remaining redundancies that need to be identified. 
% Therefore, designing a higher-precision and more efficient approach would be a promising research direction.

% some work about garbage  collection