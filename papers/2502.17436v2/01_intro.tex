\section{Introduction}
\label{sec:intro}
Diffusion models~\citep{ho2020denoising,song2021denoising,SongICLR2021} and particularly also flow matching~\citep{liu2023flow,LipmanICLR2023,albergo2023building,albergo2023stochastic} have gained significant attention recently. This is partly due to impressive results that have been reported across domains from computer vision~\citep{ho2020denoising} and medical imaging~\citep{song2022solving} to robotics~\citep{kapelyukh2023dall} and computational biology~\citep{guo2024diffusion}. Beyond impressive results, flow matching was also reported to faithfully model multimodal data distributions. In addition, sampling is reasonably straightforward: it requires to solve an ordinary differential equation (ODE) via forward integration of a set of source distribution points along an estimated velocity field from time zero to time one. The source distribution points are sampled from a simple and known source distribution, e.g., a standard Gaussian.


The velocity field is obtained by matching velocities from a constructed ``ground-truth'' integration path with a parametric deep net using a mean squared error (MSE) objective.
See \cref{fig:teaser}(a) for the ``ground-truth'' integration paths of classic rectified flow. 
Studying the ``ground-truth'' velocity distribution at a distinct location and time for %
rectified flow  reveals a multimodal distribution. We derive an analytic expression for the multimodal velocity distribution in case of a mixture-of-Gaussian data distribution in \cref{sec:method:casestudy}. It is known that the  MSE objective used in classic rectified flow does not permit to capture this multimodal distribution. Instead, classic rectified flow leads to a model that aims to capture the mean of the velocity distribution. This is illustrated in \cref{fig:teaser}(b). 

We do want to emphasize that capturing the mean of the velocity distribution is sufficient for characterizing a multimodal data distribution~\citep{liu2023flow}. However, only capturing the mean velocity also leads to unnecessarily curved forward integration paths. This is due to the fact that integration paths cannot intersect when using an MSE objective, as can be observed in \cref{fig:teaser}(b).


In this paper, we hence wonder whether it is possible
to capture the velocity distribution in its entirety. This  enables  integration paths to intersect during data generation, as illustrated in \cref{fig:teaser}(c). 
Intuitively, and as detailed in \cref{sec:method:approach}, we can capture the velocity distribution by formulating a rectified flow objective 
in the velocity space rather than the location space. Hence, instead of training a deep net to estimate the velocity for integration in location space, as done in classic rectified flow, we train a deep net that estimates the acceleration for integration in velocity space. Sampling can then be done by forward integrating two hierarchically coupled processes: first, forward integrate in velocity space to obtain a sample from the velocity distribution; then use the velocity sample to perform a step in location space. While this nested integration of two processes seems computationally more demanding at first, it turns out that fewer integration steps are needed, particularly in the latter process. This is due to the fact that the integration path is indeed less curved, as shown in \cref{fig:teaser}(c). We  also show in \cref{sec:method:theory} that  capturing the velocity distribution in its entirety permits to capture a multimodal data distribution. The data generation process is governed by a random differential equation (RDE)~\citep{xiaoying2018random} with learned random velocity field. 

Going forward, instead of using `just' two hierarchically coupled processes we can extend the formulation to an arbitrary depth, which is detailed in \cref{sec:method:extension}. Using a depth of one defaults to classic rectified flow (deep net captures the  expected velocity field), while a depth of two leads to a deep net that captures the acceleration, etc. We refer to this construction of hierarchically coupled processes as a `hierarchical rectified flow.'


Empirically, we find that the studied hierarchical rectified flow  leads to samples that better fit the data distribution. Specifically, we find that this hierarchical rectified flow leads to slightly better results than the vanilla rectified flow.

\begin{figure}[t]
    \centering
    \begin{tabular}{ccc}
    \includegraphics[width=0.3\linewidth]{fig/fig1_traj/true_traj.pdf}&
    \includegraphics[width=0.3\linewidth]{fig/fig1_traj/2to2_traj_400.pdf}&
    \includegraphics[width=0.3\linewidth]{fig/fig1_traj/2to2_traj_20_20.pdf}\\
    (a) Linear Interpolation & (b) Rectified Flow & (c) Ours
    \end{tabular}
    \caption{Particles flow from starting points (grey) to endpoints (blue) as time increases from $0$ to $1$. Ideally, the trajectories (green) are straight lines connecting two ends as shown in (a). Rectified Flow captures the expected velocity field while our Hierarchical Rectified Flow can model the true velocity field thus generating intersecting and more straight paths. }
    \label{fig:teaser}
\end{figure}

