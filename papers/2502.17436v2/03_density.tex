\section{Density estimation}
\label{sec:density}
In the following, we describe two approaches for density estimation. The resulting procedures are summarized in \cref{alg:density1} and \cref{alg:density2}. To empirically verify the correctness of the density estimation procedures, 
we train an RF baseline and an HRF2 model using a bimodal Gaussian target distribution and a standard Gaussian source distribution (see \cref{sec:additional1d} for more details). In \cref{fig:density} we compare 1) the ground truth density, 2) the density estimated for the RF baseline model, and 3) the densities estimated for the HRF2 model with both procedures. 
We also report bits per dimension (bpd) for experiments on the 1D $1\mathcal{N}\to2\mathcal{N}$, 2D $8\mathcal{N}\to$ moon, CIFAR-10, and ImageNet-32 data.
The results are shown in \cref{tab:bpd2}. 
We observe that HRF2 consistently outperforms the RF baseline. 

To estimate the density, according to~\cref{eq:vgxt} in~\cref{the:pvgivenxt}, we have
\begin{equation}
\label{eq:log_den}
    \log \rho_1(z_1) = \log \pi_1(u; z_t, t) + \log \rho_t(z_t) - \log \rho_0(z_t - tu),\, \text{with } u = \frac{z_1 - z_t}{1-t}. 
\end{equation}
This implies that for any given $t \in [0, 1]$, we can use~\cref{eq:log_den} to estimate the density for a generated sample $z_1$. We can choose $z_t$ using the linear interpolation in~\cref{eq:lin_int} with $z_0 \sim \rho_0$. 

For $t = 0$, we observe that $\rho_1(z_1) = \pi_1(z_1 - z_0; z_0, 0)$, where $z_0 \sim \rho_0$. In this case, we can directly evaluate the likelihood of the generated sample via the velocity distribution. We discuss evaluation of the likelihood %
below. The procedure to compute the density is summarized in \cref{alg:density1}.

For $t = 1$, the right-hand side of~\cref{eq:log_den} becomes $\log \rho_1(z_1)$ because $\log \pi_1(u; z_1, 1) = \log \rho_0(z_1 - u)$, which cancels out with the last term in~\cref{eq:log_den}. Hence, $t=1$ can't be used to estimate the density.

For $ t\in (0, 1)$, we need to evaluate $\rho_t(z_t)$ to estimate the likelihood of $z_1$. Considering a one step linear flow from $z_0$ at time $0$ to $z_t$ at $t$, we have $z_t = z_0 + v t$ and $\rho_t(z_t | z_0) = \frac{1}{t} \pi_1(v; z_0, 0)$. Using it, the density at time $t$ can be computed according to 
\begin{equation}
\label{eq:rho_t_est}
\rho_t(z_t) = \int \frac{1}{t} \pi_1\left( \frac{z_t - z_0}{t} ; z_0, 0\right) \rho_0(z_0)\, d z_0 \approx \frac{1}{N} \sum_{i = 1}^N \frac{1}{t}\pi_1\left( \frac{z_t - z^{(i)}_0}{t}; z^{(i)}_0, 0\right), 
\end{equation}
where $z^{(i)}_0 \sim \rho_0$.
\cref{alg:density2} outlines the procedure for the likelihood computation with a randomly drawn $t \in (0, 1)$. Optionally, we can average across randomly drawn $t\in(0,1)$. 

To evaluate the (log-)likelihood of a velocity $u$ at location $z_t$ and time $t$, %
which is needed in both cases ($t=0$ and $t\in(0,1)$), we follow the approach introduced by \citet{ChenARXIV2018,SongICLR2021} and numerically evaluate 
\begin{equation}
\label{eq:v_loglikelihood}
\log \pi_1(u; z_t, t) = \log \pi_0(u_0; z_t, t) - \int_1^0 \nabla_{u_\tau} \cdot a_\theta(z_t, t, u_\tau, \tau)\, d \tau.
\end{equation}
Here, the random variable $u_\tau$ as a function of $\tau$ can be obtained by solving the ODE in~\cref{eq:udiffeq} backward with a fixed $u$ at $\tau = 1$. The term $\nabla_{u_\tau} \cdot a_\theta(z_t, t, u_\tau, \tau)$ is computed by using the Skilling-Hutchinson trace estimator $\mathbb{E}_{p(\epsilon)} \left[ \epsilon^T \nabla_{u_\tau} a(z_t, t, u_\tau, \tau) \epsilon\right]$~\citep{Skilling1989,hutchinson1989stochastic,GrathwohlICLR2018}.
The vector-Jacobian product $ \epsilon^T \nabla_{v_\tau} a(z_t, t, u_\tau, \tau)$ can be efficiently computed by using reverse mode automatic differentiation, at approximately the same cost as evaluating $a(z_t, t, u_\tau, \tau)$. 

In our experiments, we use the RK45 ODE solver~\citep{dormand1980family} provided by the  \texttt{scipy.integrate.solve{\_}ivp} package. We use atol $=1\mathrm{e}{-5}$ and rtol $=1\mathrm{e}{-5}$. When implementing~\cref{alg:density2}, we use $N = 1000$ to evaluate $\rho_t(x_t)$. 

As mentioned above, to empirically verify the correctness of the density estimation procedures, 
we train an RF baseline and an HRF2 model using a bimodal Gaussian target distribution and a standard Gaussian source distribution. We compare the density estimated for the RF baseline model and the densities estimated for the HRF2 model with both \cref{alg:density1} and \cref{alg:density2}. 
\cref{fig:density}(a) compares the results obtained with \cref{alg:density1} to the RF baseline and the ground truth. 
\cref{fig:density}(b) compares the density estimated for different times $t$ with \cref{alg:density2} to the RF baseline and the ground truth.
Regardless of the choice of algorithm and time, we observe that the HRF2 model obtains a better estimation of the likelihood. Importantly, both procedures provide a compelling way to estimate densities. 



\begin{algorithm}[t]
\SetKwComment{Comment}{//}{}
\SetKwInOut{input}{Input}
\SetKwInOut{output}{Output}
\caption{Density Estimation 1 ($t=0$)}\label{alg:density1}
\input{Generated sample $z_1$ and the source distribution $\rho_0$.}
Sample  $z_0\sim \rho_0$ \;
Compute $u = z_1 - z_0 $ \;
Compute $\hat{\rho}_1(z_1) = \pi_1(u; z_0, 0)$ according to~\cref{eq:v_loglikelihood} \;
(Optional) Compute $\hat{\rho}_1(z_1) = \frac{1}{N} \sum_{i = 1}^N \pi_1(u^{(i)}; z^{(i)}_0, 0)$, with $u^{(i)} = z_1 - z^{(i)}_0$  and $z^{(i)}_0\sim \rho_0$ \;
\output{$\hat{\rho}_1(z_1)$}
\end{algorithm}

\begin{algorithm}[t]
\SetKwComment{Comment}{//}{}
\SetKwInOut{input}{Input}
\SetKwInOut{output}{Output}
\caption{Density Estimation 2 ($t\in(0,1)$)}\label{alg:density2}
\input{Generated sample $z_1$ and the source distributions $\rho_0$ and $\pi_0$.}
Draw random $t \sim \mathrm{Unif}(0, 1)$ \;
Sample  $z_0\sim \rho_0$ \;
Compute $z_t = t z_1 + (1-t) z_0$ and $u = \frac{z_1 - z_t}{1-t}$ \;
Evaluate $\rho_0(z_t - tu)$, $\rho_t(z_t)$ according to~\cref{eq:rho_t_est}, and $\pi_1(u; z_t, t)$ according to~\cref{eq:v_loglikelihood} \;
Compute the log likelihood according to~\cref{eq:log_den} \;
\output{$\hat{\rho}_1(z_1)$}
\end{algorithm}

\begin{figure}[t]
    \centering
    \setlength{\tabcolsep}{0pt}
    \begin{tabular}{cc}
    \includegraphics[width=0.45\linewidth]{fig/fig8_density/log_likelihood_algo3.pdf}&
    \includegraphics[width=0.45\linewidth]{fig/fig8_density/log_likelihood_alt.pdf}\\
    (a) \cref{alg:density1} & (b) \cref{alg:density2} 
    \end{tabular}
    \caption{Density estimation results and comparison to ground truth. Irrespective of the choise of algorithm and the choice of time, we observe compelling density estimation results. We also note that the HRF2 model improves upon the RF baseline.}
    \label{fig:density}
\end{figure}


In \cref{tab:bpd2}, we report bits per dimension (bpd) for experiments on the 1D $1\mathcal{N}\to2\mathcal{N}$, 2D $8\mathcal{N}\to$ moon, CIFAR-10, and ImageNet-32 data.  
For 1D data, $z_0=0$ suffices for compelling results. For higher dimensional data, we use $20$ samples of $z_0$ as shown in the optional line 4 of \cref{alg:density1} to compute the bits per dimension. 
We observe that HRF2 consistently outperforms the RF baseline. 


\begin{table}[t]
\centering
\resizebox{0.85\columnwidth}{!}{
\begin{tabular}{ccccc}
\toprule
NLL (BPD$\downarrow$) & $1\mathcal{N}\to2\mathcal{N}$ & $8\mathcal{N}\to$ moon & CIFAR-10 & ImageNet-32 \\
\midrule
Baseline (RF) & 0.275 & 2.119 & 2.980 & 3.416 \\
Ours (HRF2) & 0.261 & 2.113 & 2.975 & 3.397 \\
\bottomrule
\end{tabular}
}
\caption{Density estimation on 1D $1\mathcal{N}\to2\mathcal{N}$, 2D $8\mathcal{N}\to$ moon, CIFAR-10, and ImageNet-32 data using bits per dimension (bpd). We observe a consistently better density estimation with the HRF2 model. }
\label{tab:bpd2}
\end{table}
