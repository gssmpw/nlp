The appendix is organized as follows. We first provide a proof of \cref{the:pvgivenxt} (the velocity distribution given $x_t$) in \cref{sec:proofpvgivenxt}. We then provide a proof of \cref{clm:velocitydistribution} (velocity distribution for the special case of a mixture of Gaussians target distribution) in \cref{sec:proof_claim_1}. Afterwards we provide the proof of \cref{the:1} (correctness of the marginals) in \cref{sec:proof_thm_1}. 
Then we discuss density estimation for HRF models in \cref{sec:density}.
Next we provide more details regarding the  hierarchical rectified flow formulation in \cref{sec:HRFformulationdetails}. 
Subsequently, we discuss experimental and implementation details in \cref{sec:exp_details}. Finally, we provide additional ablation studies in \cref{sec:ablation} and additional experimental results in \cref{sec:app:additional}.

\section{Proof of \cref{the:pvgivenxt}}
\label{sec:proofpvgivenxt}
\textbf{Proof of~\cref{the:pvgivenxt}:}
The velocity at location $x_t$ and time $t$ is $v = x_1 - x_0 = \frac{x_1 - x_t}{1-t}$. The last equality holds because $(1-t) x_0 + tx_1 = x_t$. Recall that for a random variable $Y = \alpha X + \beta$ with $\alpha, \beta \in \mathbb{R}$ and $\alpha \neq 0$, we have $p_Y(y) = \frac{1}{\alpha} p_X\left( \frac{y - \beta}{\alpha}\right)$. Since the random variable $V$ is a linear transform of the random variable $X_1$, we get
\begin{equation}
\label{eq:pvgivenxt}
\pi_1(v; x_t, t) = p_{V | X_t} (v | x_t) = (1-t) p_{X_1 | X_t}\left( (1-t)v + x_t | x_t \right).
\end{equation}
Therefore, we need to evaluate $p_{X_1 | X_t}$. Using Bayes' formula, 
\begin{equation}
\label{eq:bayes}
p_{X_1 | X_t} (x_1 | x_t) = \frac{p_{X_t | X_1}(x_t | x_1) p_{X_1}(x_1)}{p_{X_t}(x_t)},
\end{equation}
assuming that $p_{X_t}(x_t) \neq 0$. It is undefined if $p_{X_t}(x_t) \neq 0$. 
Now it remains to find $p_{X_t | X_1}$ and we have
\begin{align}
\label{eq:pxtgivenx1}
p_{X_t | X_1} (x_t | x_1) & = p_{(1-t)X_0 + t x_1}(x_t) =\frac{1}{1-t}p_{X_0}\left( \frac{x_t -t x_1}{1-t}\right). 
\end{align}
Plugging~\cref{eq:bayes} and~\cref{eq:pxtgivenx1} into~\cref{eq:pvgivenxt} and using $x_1 = x_t + (1-t) v$, we have 
\begin{align}
\pi_1(v; x_t, t)  = p_{V|X_t} (v | x_t) 
& = \frac{p_{X_0}(x_t - tv) p_{X_1}(x_t + (1-t)v)}{p_{X_t} (x_t)} \nonumber \\
& = \frac{\rho_0(x_t - tv) \rho_1(x_t + (1-t)v)}{\rho_t (x_t)}.
\end{align}

Since the random variable $X_t$ is a linear combination of two independent random variables $X_0$ and $X_1$ as defined in~\cref{eq:lin_int}, we have 
\begin{align}
\label{eq:pxt}
 \rho_t (x_t) & = p_{(1-t)X_0} (x_t) * p_{tX_1} (x_t)   = \int p_{(1-t)X_0} (z) p_{tX_1} (x_t-z)  dz \nonumber \\
 & = \int \frac{1}{1-t} p_{X_0} \left( \frac{z}{1-t} \right) \frac{1}{t}p_{X_1} \left(\frac{x_t-z}{t} \right) d z \nonumber \\
 & = \frac{1}{t(1-t)} \rho_0 \left( \frac{x_t}{1-t} \right) * \rho_1 \left( \frac{x_t}{t}\right), \quad \text{for }t \in (0, 1).
\end{align}
At $t = 0$, $\rho_t = \rho_0$ since $x_t = x_0$. At $t = 1$, $\rho_t = \rho_1$, since $x_t = x_1$. $ \pi_1(v; x_t, t) $ is undefined if $\rho_t(x_t) = 0$.
This completes the proof.
\hfill$\blacksquare$


\section{Proof of \cref{clm:velocitydistribution}}
\label{sec:proof_claim_1}
\citet{bromiley2003products} summarizes a few useful properties for the product and convolution of Gaussian distributions. We state the relevant results here for our proof of \cref{clm:velocitydistribution}.  
\begin{lemma}
\label{lem:GLT}
For the linear transform of a Gaussian random variable, we have $$\mathcal{N}(ax + b; \mu, \sigma^2) = \frac{1}{a} \mathcal{N}\left(x; \frac{\mu - b}{a} , \frac{\sigma^2}{a^2} \right).$$
\end{lemma}

\begin{lemma}
\label{lem:Gconv}
 For the convolution of two Gaussian distributions, we have $$\mathcal{N}(x; \mu_1, \sigma_1^2) * \mathcal{N}(x; \mu_2, \sigma_2^2) = \mathcal{N}(x; \mu_1 + \mu_2, \sigma_1^2 + \sigma_2^2).$$
\end{lemma}

\begin{lemma}
\label{lem:Gprod}
 For the product of two Gaussian distributions, we have $$ \mathcal{N}(x; \mu_1, \sigma_1^2) \cdot \mathcal{N}(x; \mu_2, \sigma_2^2) = \frac{1}{\sqrt{2\pi(\sigma_1^2 + \sigma_2^2)}} \exp\left[- \frac{(\mu_1 - \mu_2)^2}{\sigma_1^2 + \sigma_2^2} \right] \mathcal{N}\left (x; \frac{\mu_1 \sigma_2^2 + \mu_2 \sigma_1^2}{\sigma_1^2 + \sigma_2^2} , \frac{\sigma_1^2 \sigma_2^2}{\sigma_1^2 + \sigma_2^2} \right). $$
\end{lemma}
The proofs of the Lemmas are detailed by \citet{bromiley2003products}.


\textbf{Proof of \cref{clm:velocitydistribution}:}
We first compute the density of $X_t$ using~\cref{the:pvgivenxt} with the specific $\rho_0$ and $\rho_1$:  %
\begin{align}
\label{eq:pxtgaussian}
 \rho_t(x_t) %
 & = \frac{1}{t(1-t)} \rho_0 \left( \frac{x_t}{1-t} \right)* \rho_1\left( \frac{x_t}{t}\right)   \nonumber \\
 & =  \frac{1}{t(1-t)} \mathcal{N}\left(\frac{x_t}{1-t}; 0, 1  \right) *\left(\sum_{k = 1}^K w_k \mathcal{N}\left(\frac{x_t}{t}; \mu_k, \sigma_k^2  \right) \right). 
\end{align}
By applying \cref{lem:GLT} and \cref{lem:Gconv} to \cref{eq:pxtgaussian}, we get 
\begin{align}
\label{eq:pxt2}
 \rho_t (x_t)  & = \mathcal{N}\left(x_t; 0, (1-t)^2  \right) * \left(\sum_{k = 1}^K w_k \mathcal{N}\left(x_t; t\mu_k, t^2\sigma_k^2  \right) \right)  \nonumber \\ 
 &= \sum_{k = 1}^K w_k  \left(   \mathcal{N}\left(x_t; 0, (1-t)^2  \right)  * \mathcal{N}\left(x_t; t\mu_k, t^2\sigma_k^2  \right) \right) \nonumber \\
 & = \sum_{k = 1}^K w_k \mathcal{N} \left(x_t; t\mu_k, \tilde{\sigma}_{k, t}^2 \right).
\end{align}

Using \cref{the:pvgivenxt} and~\cref{eq:pxt2}, %
we have
\begin{align}
    \label{eq:pvgivenxt2}
    p_{V | X_t} (v | x_t) %
    & = \frac{\mathcal{N}\left( x_t - tv; 0, 1 \right) \left( \sum_{k=1}^K w_k \mathcal{N}\left(x_t + (1-t)v; \mu_k, \sigma_k^2 \right) \right) }{ \sum_{k' = 1}^K w_{k'} \mathcal{N} \left(x_t; t\mu_{k'}, \tilde{\sigma}_{k', t}^2 \right)} \nonumber \\
    & \stackrel{a}{=} \frac{\mathcal{N}\left( v; \frac{x_t}{t}, \frac{1}{t^2} \right) \left( \sum_{k=1}^K w_k \mathcal{N}\left(v; \frac{\mu_k - x_t}{1-t}, \frac{\sigma_k^2}{(1-t)^2} \right) \right) }{ \sum_{k' = 1}^K w_{k'} \mathcal{N} \left(x_t; t\mu_{k'}, \tilde{\sigma}_{k', t}^2 \right)} \nonumber \\
    & =  \frac{\sum_{k=1}^K w_k\mathcal{N}\left( v; \frac{x_t}{t}, \frac{1}{t^2} \right) \mathcal{N}\left(v; \frac{\mu_k - x_t}{1-t}, \frac{\sigma_k^2}{(1-t)^2} \right) }{ t(1-t)\sum_{k' = 1}^K w_{k'} \mathcal{N} \left(x_t; t\mu_{k'}, \tilde{\sigma}_{k', t}^2 \right)} \nonumber \\
    & \stackrel{b}{=} \frac{\sum_{k = 1}^K w_k \frac{t(1-t)}{\sqrt{2\pi((1-t)^2 + t^2 \sigma_k^2)}} \exp\left( -\frac{(x_t - t\mu_k)^2}{(1-t)^2 + t^2 \sigma_k^2}\right) \mathcal{N}\left(v; \frac{(1-t)(\mu_k - x_t) + t\sigma_k^2 x_t }{\tilde{\sigma}_{k, t}^2}, \frac{\sigma_k^2}{\tilde{\sigma}_{k, t}^2} \right)}{t(1-t)\sum_{k' = 1}^K w_{k'} \mathcal{N} \left(x_t; t\mu_{k'}, \tilde{\sigma}_{k', t}^2 \right)} \nonumber \\
    & \stackrel{c}{=} \frac{\sum_{k = 1}^K w_k \mathcal{N}\left(x_t; t\mu_{k}, \tilde{\sigma}_{k, t}^2 \right) \mathcal{N}\left(v; \frac{(1-t)(\mu_k - x_t) + t\sigma_k^2 x_t }{\tilde{\sigma}_{k, t}^2}, \frac{\sigma_k^2}{\tilde{\sigma}_{k, t}^2} \right)}{\sum_{k' = 1}^K w_{k'} \mathcal{N} \left(x_t; t\mu_{k'}, \tilde{\sigma}_{k', t}^2 \right)} \nonumber \\
    & = \sum_{k = 1}^K \tilde{w}_{k, t} \mathcal{N}\left(v; \frac{(1-t)(\mu_k - x_t) + t\sigma_k^2 x_t }{\tilde{\sigma}_{k, t}^2}, \frac{\sigma_k^2}{\tilde{\sigma}_{k, t}^2} \right).
\end{align}
The equality $a$ holds by applying \cref{lem:GLT}. %
The equality $b$ is derived by applying \cref{lem:Gprod} to the product of two Gaussian distributions. Simplifying the expressions, we get equality $c$ and the final expression of $p_{V|X_t}(v|x_t)$. This completes the proof. 
\hfill$\blacksquare$



\section{Proof of \cref{the:1}}
\label{sec:proof_thm_1}
According to Theorem 3.3 of~\citet{liu2023flow}, the ODE in~\cref{eq:udiffeq} generates the samples from the ground-truth velocity distributions at space time location $(x_t, t)$. In other words, the random variable $V \sim \pi_1$.

Now we consider the characteristic function of $Z_{t + \Delta t} = Z_t + V \Delta t$ for $t \in [0, 1]$ and $\Delta t \in [0, 1-t]$, assuming that $Z_t$ has the same distribution as $X_t$. If the characteristic functions of $Z_{t + \Delta t}$ and $X_{t + \Delta t}$ agree, then $Z_{t + \Delta t}$ and $X_{t + \Delta t}$ have the same distribution.
 
To show this, we evaluate the characteristic function of $Z_{t + \Delta t}$,
\begin{align}
 \E \left[e^{\imath \langle k, Z_{t + \Delta t} \rangle} \right] &= \E_{\rho_t, \pi_1 } \left[ e^{\imath \langle k, X_t + V \Delta t \rangle}  \right] \nonumber \\
& = \int \int e^{\imath \langle k, x_t + v \Delta t \rangle } \pi_1(v; x_t, t) \rho_t(x_t)  dv dx_t \nonumber \\
& \stackrel{a}{=} \int \int e^{\imath \langle k, x_t + v \Delta t \rangle }  \frac{\rho_0(x_t-t v) \rho_1 (x_t+(1-t)v)}{\rho_t(x_t)} {\rho_t (x_t)}  dv dx_t \nonumber \\
& = \int \int e^{\imath \langle k, (x_t + v \Delta t) \rangle } \rho_0 (x_t-t v) \rho_1 (x_t+(1-t)v)  dv dx_t \nonumber \\
& \stackrel{b}{=} \int \int e^{\imath \langle k, (1-t-\Delta t)x_0 + (t + \Delta t) x_1 \rangle }  \rho_0(x_0) \rho_1(x_1)  dx_0 dx_1 \nonumber \\
& = \E_{\rho_{t + \Delta t}} \left[e^{\imath \langle k,  X_{t + \Delta t} \rangle }  \right]. 
\end{align}
We use the notation $\langle \cdot, \cdot \rangle$ to denote the inner product. The equality $a$ is valid due to \cref{the:pvgivenxt}. 
The equality $b$ holds because $x_0 = x_t - tv$ and $x_1 = x_t + (1-t) v$ with the linear interpolation. Therefore, we find that $Z_{t + \Delta t}$ and $X_{t + \Delta t}$ follow the same distribution.  In addition, since $Z_0$ and $X_0$ follow the same distribution $\rho_0$, we can conclude that $Z_t$ and $X_t$ follow the same marginal distribution at $t$ for $t \in [0, 1]$. This completes the proof. 



\section{Density estimation}
\label{sec:density}
In the following, we describe two approaches for density estimation. The resulting procedures are summarized in \cref{alg:density1} and \cref{alg:density2}. To empirically verify the correctness of the density estimation procedures, 
we train an RF baseline and an HRF2 model using a bimodal Gaussian target distribution and a standard Gaussian source distribution (see \cref{sec:additional1d} for more details). In \cref{fig:density} we compare 1) the ground truth density, 2) the density estimated for the RF baseline model, and 3) the densities estimated for the HRF2 model with both procedures. 
We also report bits per dimension (bpd) for experiments on the 1D $1\mathcal{N}\to2\mathcal{N}$, 2D $8\mathcal{N}\to$ moon, CIFAR-10, and ImageNet-32 data.
The results are shown in \cref{tab:bpd2}. 
We observe that HRF2 consistently outperforms the RF baseline. 

To estimate the density, according to~\cref{eq:vgxt} in~\cref{the:pvgivenxt}, we have
\begin{equation}
\label{eq:log_den}
    \log \rho_1(z_1) = \log \pi_1(u; z_t, t) + \log \rho_t(z_t) - \log \rho_0(z_t - tu),\, \text{with } u = \frac{z_1 - z_t}{1-t}. 
\end{equation}
This implies that for any given $t \in [0, 1]$, we can use~\cref{eq:log_den} to estimate the density for a generated sample $z_1$. We can choose $z_t$ using the linear interpolation in~\cref{eq:lin_int} with $z_0 \sim \rho_0$. 

For $t = 0$, we observe that $\rho_1(z_1) = \pi_1(z_1 - z_0; z_0, 0)$, where $z_0 \sim \rho_0$. In this case, we can directly evaluate the likelihood of the generated sample via the velocity distribution. We discuss evaluation of the likelihood %
below. The procedure to compute the density is summarized in \cref{alg:density1}.

For $t = 1$, the right-hand side of~\cref{eq:log_den} becomes $\log \rho_1(z_1)$ because $\log \pi_1(u; z_1, 1) = \log \rho_0(z_1 - u)$, which cancels out with the last term in~\cref{eq:log_den}. Hence, $t=1$ can't be used to estimate the density.

For $ t\in (0, 1)$, we need to evaluate $\rho_t(z_t)$ to estimate the likelihood of $z_1$. Considering a one step linear flow from $z_0$ at time $0$ to $z_t$ at $t$, we have $z_t = z_0 + v t$ and $\rho_t(z_t | z_0) = \frac{1}{t} \pi_1(v; z_0, 0)$. Using it, the density at time $t$ can be computed according to 
\begin{equation}
\label{eq:rho_t_est}
\rho_t(z_t) = \int \frac{1}{t} \pi_1\left( \frac{z_t - z_0}{t} ; z_0, 0\right) \rho_0(z_0)\, d z_0 \approx \frac{1}{N} \sum_{i = 1}^N \frac{1}{t}\pi_1\left( \frac{z_t - z^{(i)}_0}{t}; z^{(i)}_0, 0\right), 
\end{equation}
where $z^{(i)}_0 \sim \rho_0$.
\cref{alg:density2} outlines the procedure for the likelihood computation with a randomly drawn $t \in (0, 1)$. Optionally, we can average across randomly drawn $t\in(0,1)$. 

To evaluate the (log-)likelihood of a velocity $u$ at location $z_t$ and time $t$, %
which is needed in both cases ($t=0$ and $t\in(0,1)$), we follow the approach introduced by \citet{ChenARXIV2018,SongICLR2021} and numerically evaluate 
\begin{equation}
\label{eq:v_loglikelihood}
\log \pi_1(u; z_t, t) = \log \pi_0(u_0; z_t, t) - \int_1^0 \nabla_{u_\tau} \cdot a_\theta(z_t, t, u_\tau, \tau)\, d \tau.
\end{equation}
Here, the random variable $u_\tau$ as a function of $\tau$ can be obtained by solving the ODE in~\cref{eq:udiffeq} backward with a fixed $u$ at $\tau = 1$. The term $\nabla_{u_\tau} \cdot a_\theta(z_t, t, u_\tau, \tau)$ is computed by using the Skilling-Hutchinson trace estimator $\mathbb{E}_{p(\epsilon)} \left[ \epsilon^T \nabla_{u_\tau} a(z_t, t, u_\tau, \tau) \epsilon\right]$~\citep{Skilling1989,hutchinson1989stochastic,GrathwohlICLR2018}.
The vector-Jacobian product $ \epsilon^T \nabla_{v_\tau} a(z_t, t, u_\tau, \tau)$ can be efficiently computed by using reverse mode automatic differentiation, at approximately the same cost as evaluating $a(z_t, t, u_\tau, \tau)$. 

In our experiments, we use the RK45 ODE solver~\citep{dormand1980family} provided by the  \texttt{scipy.integrate.solve{\_}ivp} package. We use atol $=1\mathrm{e}{-5}$ and rtol $=1\mathrm{e}{-5}$. When implementing~\cref{alg:density2}, we use $N = 1000$ to evaluate $\rho_t(x_t)$. 

As mentioned above, to empirically verify the correctness of the density estimation procedures, 
we train an RF baseline and an HRF2 model using a bimodal Gaussian target distribution and a standard Gaussian source distribution. We compare the density estimated for the RF baseline model and the densities estimated for the HRF2 model with both \cref{alg:density1} and \cref{alg:density2}. 
\cref{fig:density}(a) compares the results obtained with \cref{alg:density1} to the RF baseline and the ground truth. 
\cref{fig:density}(b) compares the density estimated for different times $t$ with \cref{alg:density2} to the RF baseline and the ground truth.
Regardless of the choice of algorithm and time, we observe that the HRF2 model obtains a better estimation of the likelihood. Importantly, both procedures provide a compelling way to estimate densities. 



\begin{algorithm}[t]
\SetKwComment{Comment}{//}{}
\SetKwInOut{input}{Input}
\SetKwInOut{output}{Output}
\caption{Density Estimation 1 ($t=0$)}\label{alg:density1}
\input{Generated sample $z_1$ and the source distribution $\rho_0$.}
Sample  $z_0\sim \rho_0$ \;
Compute $u = z_1 - z_0 $ \;
Compute $\hat{\rho}_1(z_1) = \pi_1(u; z_0, 0)$ according to~\cref{eq:v_loglikelihood} \;
(Optional) Compute $\hat{\rho}_1(z_1) = \frac{1}{N} \sum_{i = 1}^N \pi_1(u^{(i)}; z^{(i)}_0, 0)$, with $u^{(i)} = z_1 - z^{(i)}_0$  and $z^{(i)}_0\sim \rho_0$ \;
\output{$\hat{\rho}_1(z_1)$}
\end{algorithm}

\begin{algorithm}[t]
\SetKwComment{Comment}{//}{}
\SetKwInOut{input}{Input}
\SetKwInOut{output}{Output}
\caption{Density Estimation 2 ($t\in(0,1)$)}\label{alg:density2}
\input{Generated sample $z_1$ and the source distributions $\rho_0$ and $\pi_0$.}
Draw random $t \sim \mathrm{Unif}(0, 1)$ \;
Sample  $z_0\sim \rho_0$ \;
Compute $z_t = t z_1 + (1-t) z_0$ and $u = \frac{z_1 - z_t}{1-t}$ \;
Evaluate $\rho_0(z_t - tu)$, $\rho_t(z_t)$ according to~\cref{eq:rho_t_est}, and $\pi_1(u; z_t, t)$ according to~\cref{eq:v_loglikelihood} \;
Compute the log likelihood according to~\cref{eq:log_den} \;
\output{$\hat{\rho}_1(z_1)$}
\end{algorithm}

\begin{figure}[t]
    \centering
    \setlength{\tabcolsep}{0pt}
    \begin{tabular}{cc}
    \includegraphics[width=0.45\linewidth]{fig/fig8_density/log_likelihood_algo3.pdf}&
    \includegraphics[width=0.45\linewidth]{fig/fig8_density/log_likelihood_alt.pdf}\\
    (a) \cref{alg:density1} & (b) \cref{alg:density2} 
    \end{tabular}
    \caption{Density estimation results and comparison to ground truth. Irrespective of the choise of algorithm and the choice of time, we observe compelling density estimation results. We also note that the HRF2 model improves upon the RF baseline.}
    \label{fig:density}
\end{figure}


In \cref{tab:bpd2}, we report bits per dimension (bpd) for experiments on the 1D $1\mathcal{N}\to2\mathcal{N}$, 2D $8\mathcal{N}\to$ moon, CIFAR-10, and ImageNet-32 data.  
For 1D data, $z_0=0$ suffices for compelling results. For higher dimensional data, we use $20$ samples of $z_0$ as shown in the optional line 4 of \cref{alg:density1} to compute the bits per dimension. 
We observe that HRF2 consistently outperforms the RF baseline. 


\begin{table}[t]
\centering
\resizebox{0.85\columnwidth}{!}{
\begin{tabular}{ccccc}
\toprule
NLL (BPD$\downarrow$) & $1\mathcal{N}\to2\mathcal{N}$ & $8\mathcal{N}\to$ moon & CIFAR-10 & ImageNet-32 \\
\midrule
Baseline (RF) & 0.275 & 2.119 & 2.980 & 3.416 \\
Ours (HRF2) & 0.261 & 2.113 & 2.975 & 3.397 \\
\bottomrule
\end{tabular}
}
\caption{Density estimation on 1D $1\mathcal{N}\to2\mathcal{N}$, 2D $8\mathcal{N}\to$ moon, CIFAR-10, and ImageNet-32 data using bits per dimension (bpd). We observe a consistently better density estimation with the HRF2 model. }
\label{tab:bpd2}
\end{table}


\section{Hierarchical Rectified Flow Formulation Details}
\label{sec:HRFformulationdetails}
In this section, we show how \cref{eq:opt} can be derived from \cref{eq:opt:hierarchy}.
For convenience we re-state \cref{eq:opt:hierarchy}:
\begin{equation}
\label{eq:opt:hierarchy_copy}
\inf_f \mathbb{E}_{{\bm x}_0\sim{\bm \rho}_0,x_1\sim\rho_1,{\bm t}\sim U[0,1]^D}\left[\left\|\left(x_1 - {\bm 1}_D^T{\bm x}_0\right) - f\left({\bm x}_{\bm t},{\bm t}\right)\right\|^2_2\right].
\end{equation}

For $D=2$, we note that $x_1 - {\bm 1}_D^T{\bm x}_0$ is equivalent to $x_1-x_0^{(1)}-x_0^{(2)}$. Letting $x_0 = x_0^{(1)}$ and $v_0 = x_0^{(2)}$, we obtain  $x_1 - {\bm 1}_D^T{\bm x}_0 = x_1-x_0-v_0$. 

Further note that we obtain the time variables $\bm t=[t^{(1)}, t^{(2)}]=[t, \tau]\sim U[0,1]^2$, since $t$ and $\tau$ are drawn independently  from a uniform distribution $U[0,1]$. Also, ${\bm x}_0=[x_0^{(1)},x_0^{(2)}] = [x_0, v_0]\sim {\bm \rho}_0$, where $x_0$ and $v_0$ are drawn independently from standard Gaussian source distributions $\rho_0$ and $\pi_0$ because ${\bm \rho}_0$ is a $D$-dimensional standard Gaussian. 

Based on the general expression $x^{(d)}_{\bm t} = (1-t^{(d)})x_0^{(d)} + t^{(d)}(x_1 - \sum_{k=1}^{d-1} x_0^{(k)})$ and the previous results, we have 
$x_t = x^{(1)}_{\bm t} = (1-t^{(1)})x_0^{(1)} + t^{(1)}x_1=(1-t)x_0 + tx_1$ 
and $v_{\tau}=x^{(2)}_{\bm t} = (1-t^{(2)})x_0^{(2)} + t^{(2)}(x_1-x_0^{(1)}) = (1-\tau)v_0 + \tau v_1$. This is identical to the computation of $x_t$ and $v_{\tau}$. Combining all of these results while renaming the function from $f$ to $a$, we arrive at
\begin{equation}
\label{eq:opt_copy}
\inf_a \mathbb{E}_{x_0\sim\rho_0,x_1\sim \mathcal{D},t\sim U[0,1],v_0\sim\pi_0,\tau\sim U[0,1]}\left[\|(x_1 - x_0 - v_0) - a(x_t,t,v_\tau, \tau)\|^2_2\right].
\end{equation}
This program is identical to the one stated in \cref{eq:opt}. 




\section{Experimental and Implementation Details}
\label{sec:exp_details}
\subsection{Low Dimensional Experiments}
For the 1D and 2D experiments, we use the same neural network. It consists of two parts. The first part processes the space and time information separately using sinusoidal positional embedding and linear layers. In the second part, the processed information is concatenated and passed through a series of linear layers to produce the final output. Compared to the baseline, our HRF model with depth $D$ takes $D$ times more space and time information as input. Therefore, the first part of the network has $D$ times more embedding and linear layers to handle spatial and temporal information from different depths. However, by adjusting the dimensions of the hidden layers, we reduced the network size to just one-fourth of the baseline, while achieving superior performance. For each dataset in the low-dimensional experiments, we use 100,000 data points for training and another 100,000 data points for evaluation. For each set of experiments, we train five different models using five random seeds. During the evaluation, we performed a total of 125 experiments and averaged the results to ensure the fairness and validity of our findings. 

\subsection{High Dimensional Experiments}
In the high-dimensional image experiments, we used the U-Net architecture described by \citet{LipmanICLR2023} for the baseline model. %
To handle extra inputs $v_\tau$ and $\tau$, we designed new U-Net-based network architectures for MNIST, CIFAR-10, and ImageNet-32 data. 

\noindent\textbf{MNIST.} For MNIST, we use a single U-Net and modify the ResNet block. Similar to the neural network used in our low-dimensional experiments, each ResNet block has two parts. In the first part, we handle two sets of space-time information, i.e., $(x_t,t)$ and $(v_\tau,\tau)$, separately with 2 distinct pathways: convolutional layers for spatial data and linear layers for time embeddings. In the second part, all the spatial data and time embeddings are added together and passed through a series of linear layers to capture the space-time dependencies. For a fair evaluation, we adjusted the number of channels such that the model sizes approximately match (ours: 1.07M parameters vs.\ baseline: 1.08M parameters). We note that the HRF formulation significantly outperforms the baseline. The results were shown in \cref{fig:img_data}. More results are provided in \cref{sec:add_results}.

\noindent \textbf{CIFAR-10.} For CIFAR-10, we use two U-Nets with the same number of layers but different channel sizes. We use a larger U-Net with channel size 128 to process the velocity $v_{\tau}$ and time $\tau$. We use another smaller U-Net with channel size 32 to process the location $x_t$ and time $t$. We merge the output of each ResNet block of the smaller U-Net with the corresponding ResNet block of the bigger U-Net. %
The size of this new U-Net structure is $1.25\times$ larger than the baseline (44.81M parameters in our model and 35.75M parameters in the baseline). Our model achieves a slightly better generation quality (see~\cref{fig:img_data} in~\cref{sec:exp} and~\cref{tab:performance} in~\cref{sec:add_results}).

\noindent \textbf{ImageNet-32.} For ImageNet-32, we adopt the same architectural setup as for CIFAR-10 but modify the attention resolution to ``16,8'' instead of just ``16'' to better capture the increased multimodality of the ImageNet-32 dataset. Our U-Net model has a parameter size of 46.21M, compared to 37.06M for the baseline. It demonstrates slightly improved generation quality (see \cref{fig:img_data} in \cref{sec:exp} and \cref{tab:performance} in \cref{sec:add_results}). 

For training, we adopt the procedure and parameter settings from \citet{tong2023improving} and \citet{LipmanICLR2023}. We use the Adam optimizer with $\beta_1=0.9$, $\beta_2=0.999$, and $\epsilon=10^{-8}$, with no weight decay. For MNIST, the U-Net has channel multipliers $ [1,2,2]$, while for CIFAR-10 and ImageNet-32, the channel multipliers are $ [1,2,2,2]$. The learning rate is set to $1\times 10^{-4}$ with a batch size 128 for MNIST and CIFAR-10. For ImageNet-32, we increase the batch size to 512 and adjust the learning rate to $2\times 10^{-4}$. We train all models on a single NVIDIA RTX A6000 GPU. For MNIST, we train both the baseline and our model for 150,000 steps while we use 400,000 steps for CIFAR-10. 



\begin{table}[t]
\centering
\resizebox{1.0\columnwidth}{!}{
\setlength{\tabcolsep}{3pt}
\begin{tabular}{ccccccc}
\toprule
\textbf{Total NFEs} & 
\textbf{Sampling Steps} & 
$\mathcal{N}\to2\mathcal{N}$ & 
$\mathcal{N}\to5\mathcal{N}$ & 
$2\mathcal{N}\to2\mathcal{N}$ &
$\mathcal{N}\to6\mathcal{N} (2D)$ &
$8\mathcal{N}\to$ moon \\
& & 1-WD & 1-WD & 1-WD & 2-SWD & 2-SWD \\
\midrule
100 & $(1,100)$ & \textbf{0.020} & 0.031 & 0.045 & 0.070 & 0.172 \\
100 & $(2,50)$ & 0.025 & \underline{0.019} & \underline{0.011} & \textbf{0.037} & \textbf{0.107} \\
100 & $(5,20)$ & \underline{0.022} & 0.020 & \textbf{0.010} & \underline{0.045} & \underline{0.119} \\
100 & $(10,10)$ & 0.025 & \underline{0.019} & 0.017 & 0.053 & 0.163 \\
100 & $(20,5)$ & 0.026 & \textbf{0.017} & 0.030 & 0.062 & 0.201 \\
100 & $(50,2)$ & 0.047 & 0.030 & 0.075 & 0.081 & 0.222 \\
100 & $(100,1)$ & 0.032 & 0.030 & 0.050 & 0.085 & 0.177 \\
\bottomrule
\end{tabular}
}
\caption{HRF2 performance for low dimensional experiments under the same $\text{NFE}=100$ budget with different choices of sampling steps. Sampling steps $(J,L)$ indicates that we use $J$ steps to integrate $x$ and $L$ steps to integrate $v$. 1-WD refers to the 1-Wasserstein distance and 2-SWD refers to the Sliced 2-Wasserstein distance. \textbf{Bold} for the best. \underline{Underline} for the runner-up. }
\label{tab:ablation_nfe}
\end{table}

\begin{figure}[t]
    \centering
    \includegraphics[width=0.6\linewidth]{fig/fig9_ablation_D/loss_convergence_v1.pdf}
    \caption{Training losses of HRF with different depths on 1D data, a standard Gaussian source distribution to a mixture of 2 Gaussians target distribution. We observe training to remain stable.}
    \label{fig:loss_convergence}
\end{figure}


\section{Ablation Studies}
\label{sec:ablation}
\subsection{Ablation Study for NFE}
The sampling process of HRF with depth $D$ involves integrating $D$ ODEs using Euler's method. The total number of neural function evaluations (NFE) is defined as $\text{NFE}=\prod_d N^{(d)}$ where $N^{(d)}$ is the number of integration steps at depth $d$. Note, for a constant NFE budget, varying the $N^{(d)}$ values can lead to different results. Therefore, we conduct an ablation study to understand suitable choices for $N^{(d)}$. 

As shown in \cref{fig:emp_v_dist}, increasing the number of integration steps improves the sampling of the velocity distribution. %
However, beyond a certain threshold, the benefit of additional steps does not justify the increased computational cost. \cref{tab:ablation_nfe} further illustrates that, for a fixed NFE budget, a compelling strategy is to allocate a sufficient number of steps to accurately sample $v$ for a precise velocity distribution while using fewer steps to integrate over $x$. 

\subsection{Ablation Study for Depth}
\begin{table}[t]
\centering
\resizebox{1.0\columnwidth}{!}{
\begin{tabular}{ccccccc}
\toprule
\textbf{Training} 
& \multicolumn{3}{c}{\textbf{1D data}} 
& \multicolumn{3}{c}{\textbf{2D data}} \\
\cmidrule(r){2-4} \cmidrule(r){5-7}
& \textbf{RF (0.30M)} & \textbf{HRF2 (0.07M)} & \textbf{HRF3 (0.67M)} & \textbf{RF (0.33M)} & \textbf{HRF2 (0.08M)} & \textbf{HRF3 (0.71M)} \\
\midrule
Time ($\times 10^{-2}$ s/iter) & 1.292 & 0.736 & 2.202 & 1.503 & 0.737 & 2.252 \\
Memory (MB) & 2011 & 1763 & 2417 & 2091 & 1803 & 2605 \\
Param.\ Counts & 297,089 & 74,497 & 673,793 & 329,986 & 76,674 & 711,042\\
\bottomrule
\end{tabular}
}
\caption{Computational requirements for training on synthetic datasets. All models in this table are trained for 15000 iterations with a batch size of 51200. }
\label{tab:training_syn}
\end{table}

\begin{table}[t]
\centering
\resizebox{1.0\columnwidth}{!}{
\begin{tabular}{ccccccc}
\toprule
\textbf{Inference Time (s)} 
& \multicolumn{3}{c}{\textbf{1D data}} 
& \multicolumn{3}{c}{\textbf{2D data}} \\
\cmidrule(r){2-4} \cmidrule(r){5-7}
\textbf{Total NFEs} & \textbf{RF (0.30M)} & \textbf{HRF2 (0.07M)} & \textbf{HRF3 (0.67M)} & \textbf{RF (0.33M)} & \textbf{HRF2 (0.08M)} & \textbf{HRF3 (0.71M)} \\
\midrule
5 & 0.030 ± 0.014 & 0.014 ± 0.005 & 0.037 ± 0.030 & 0.035 ± 0.017 & 0.017 ± 0.006 & 0.041 ± 0.034 \\
10 & 0.069 ± 0.020 & 0.033 ± 0.000 & 0.128 ± 0.001 & 0.078 ± 0.025 & 0.039 ± 0.000 & 0.145 ± 0.001 \\
50 & 0.372 ± 0.024 & 0.164 ± 0.000 & 0.642 ± 0.001 & 0.440 ± 0.001 & 0.193 ± 0.000 & 0.727 ± 0.001 \\
100 & 0.755 ± 0.001 & 0.327 ± 0.000 & 1.291 ± 0.002 & 0.884 ± 0.001 & 0.385 ± 0.000 & 1.455 ± 0.003 \\
\bottomrule
\end{tabular}
}
\caption{Inference time comparison for synthetic data using a varying NFE budget. For HRF2, we used sampling step combinations: $(1,5), (2,5), (5,10), (10,10)$. For HRF3, we used sampling step combinations: $(1,1,5), (1,2,5), (1,5,10), (2,5,10)$. For all experiments, we set our batch size to 100,000. } 
\label{tab:infer_time_syn}
\end{table}

Our HRF framework can be extended to an arbitrary depth $D$. Here, we compare the training loss convergence of HRF with depths ranging from 1 to 5, where HRF1 corresponds to the baseline RF. As illustrated by the training losses shown in \cref{fig:loss_convergence}, training stability remains consistent across different depths, with higher-depth HRFs demonstrating comparable stability to lower-depth models. Importantly, note that \cref{fig:loss_convergence} mainly serves to compare convergence behavior and not loss magnitudes as those magnitudes reflect different objects, i.e., velocity for a depth of 1, acceleration for a depth of 2, etc. Moreover, the deep net structure for the functional field of directions $f$ depends on the depth, which makes a comparison more challenging. 
\cref{tab:training_syn} and \cref{tab:infer_time_syn} indicate that increasing the depth results in manageable model size, training time, and inference time. These trade-offs are justified by the significant performance improvements observed in \cref{fig:1d_data} and \cref{fig:2d_data}. See \cref{sec:additional1d} for details regarding the training data. 


\section{Additional Experimental Results}
\label{sec:app:additional}
\subsection{Additional 1D Results}
\label{sec:additional1d}
The results for experiments used in \cref{fig:teaser} and \cref{fig:emp_v_dist} are shown in \cref{fig:more_1d_data}. 
\label{sec:exp_more_results}
\begin{figure}[t]
    \centering
    \setlength{\tabcolsep}{0pt}
    \begin{tabular}{cccc}
    \includegraphics[width=0.25\linewidth]{fig/fig7_more_1d/1to2_dist.pdf}&
    \includegraphics[width=0.25\linewidth]{fig/fig7_more_1d/new_1to2_WD_NFE.pdf}&
    \includegraphics[width=0.25\linewidth]{fig/fig7_more_1d/1to2_traj_400.pdf}&
    \includegraphics[width=0.25\linewidth]{fig/fig7_more_1d/1to2_traj_20_20.pdf}\\
    \includegraphics[width=0.25\linewidth]{fig/fig7_more_1d/2to2_dist.pdf}&
    \includegraphics[width=0.25\linewidth]{fig/fig7_more_1d/new_2to2_WD_NFE.pdf}&
    \includegraphics[width=0.25\linewidth]{fig/fig1_traj/2to2_traj_400.pdf}&
    \includegraphics[width=0.25\linewidth]{fig/fig1_traj/2to2_traj_20_20.pdf}\\
    (a) Data distribution & (b) Metrics & (c) RF trajectories & (d) HRF trajectories
    \end{tabular}
    \caption{More experiments on 1D data: top row shows results for a standard Gaussian source distribution and a mixture of 2 Gaussians target distribution; bottom row shows results for a mixture of 2 Gaussians source distribution and the same mixture of 2 Gaussians target distribution. }
    \label{fig:more_1d_data}
\end{figure}

\subsection{Hierarchical Rectified Flow with OTCFM}
\label{sec:batchOT}
As mentioned in~\cref{sec:rel}, various approaches for straightening the paths in flow matching models exist. These approaches are orthogonal to our work and can be easily incorporated in the HRF formulation. To demonstrate this, we incorporate the minibatch optimal transport conditional flow matching (OTCFM)~\citep{tong2023improving} into the two layered hierarchical rectified flow (HRF2). In OTCFM, for each batch of data $(\{x_0^{(i)} \}_{i = 1}^B, \{x^{(i)}_1\}_{i=1}^B)$ seen during training, we sample pairs of points
from the joint distribution $\gamma_{\text{batch}}(x_0, x_1)$ given by the optimal transport plan between the source and target points in the batch. We follow the same procedure to couple noise with the data points and use the batch-wise coupled $x_0$ and $x_1$ to learn the parameters in $a_\theta$. We refer to this approach as HOTCFM2. We test its performance on two synthetic examples: 1) a 1D example with a standard Gaussian source distribution and a mixture of two Gaussians as the target distribution; and 2) a 2D example with a mixture of eight Gaussians as the source distribution and the moons dataset as the target distribution.  \cref{fig:1d_batchot} and \cref{fig:2d_batchot} show that hierarchical rectified flow improves the performance of OTCFM.   

\begin{figure}[t]
    \centering
    \setlength{\tabcolsep}{0pt}
    \begin{tabular}{cccc}
    \includegraphics[width=0.25\linewidth]{fig/fig10_batchot/1to2_dist.pdf}&
    \includegraphics[width=0.25\linewidth]{fig/fig10_batchot/1to2_WD_NFE.pdf}&
    \includegraphics[width=0.25\linewidth]{fig/fig10_batchot/1to2_traj_400.pdf}&
    \includegraphics[width=0.25\linewidth]{fig/fig10_batchot/1to2_traj_20_20.pdf} \\
    (a) Data distribution & (b) Metrics & (c) OTCFM trajectories & (d) HOTCFM2 trajectories
    \end{tabular}
    \caption{Results for 1D data, with $\rho_0$ being a standard Gaussian and $\rho_1$ being a mixture of 2 Gaussians. (a) Histograms of generated samples and $\rho_1$. (b) The 1-Wasserstein distance vs.\ total NFEs. (c,d) The trajectories of particles flowing from source distribution (grey) to target distribution (blue). }
    \label{fig:1d_batchot}
\end{figure}

\begin{figure}[t]
    \centering
    \setlength{\tabcolsep}{0pt}
    \begin{tabular}{cccc}
    \includegraphics[width=0.3\linewidth]{fig/fig10_batchot/moon_WD_NFE.pdf}&
    \includegraphics[width=0.3\linewidth]{fig/fig10_batchot/moon_traj_400.pdf}&
    \includegraphics[width=0.3\linewidth]{fig/fig10_batchot/moon_traj_20_20.pdf} \\
    (a) Metrics & (b) OTCFM trajectories & (c) HOTCFM2 trajectories
    \end{tabular}
    \caption{Results for 2D data, with $\rho_0$ being a mixture of 8 Gaussians and $\rho_1$ being represented by the moons data. (a) Sliced 2-Wasserstein distance vs.\ total NFEs. (b) and (c) show the trajectories (green) of sample particles flowing from source distribution (grey) to target distribution (blue). }
    \label{fig:2d_batchot}
\end{figure}




\subsection{Additional results on MNIST, CIFAR-10, and ImageNet-32}
\label{sec:add_results}
Here we show additional results for experiments with MNIST, CIFAR-10, and ImageNet-32 data. From \cref{tab:training_image,tab:infer_time_image,tab:performance}, we can observe the following: For MNIST, our model is comparable in size, comparable in training times, and comparable in inference times, while  outperforming the baseline. For CIFAR-10 and ImageNet-32, our model is $1.25\times$ larger and has a slower inference time. However, as shown in \cref{tab:performance}, it still outperforms the baseline. We believe that the modest trade-off in model size and inference time is acceptable given the performance gains.

\begin{table}[t]
\centering
\resizebox{1.0\columnwidth}{!}{
\begin{tabular}{ccccccc}
\toprule
\textbf{Training} 
& \multicolumn{2}{c}{\textbf{MNIST}} 
& \multicolumn{2}{c}{\textbf{CIFAR-10}}
& \multicolumn{2}{c}{\textbf{ImageNet-32}}\\
\cmidrule(r){2-3} \cmidrule(r){4-5} \cmidrule(r){6-7}
& \textbf{RF (1.08M)} & \textbf{HRF2 (1.07M)} & \textbf{RF (35.75M)} & \textbf{HRF2 (44.81M)} &
\textbf{RF (37.06M)} & \textbf{HRF2 (46.21M)}\\
\midrule
Time (s/iter) & 0.1 & 0.1 & 0.3 & 0.4 & 0.7 & 0.8 \\
Memory (MB) & 3935 & 3931 & 8743 & 10639 & 27234 & 33838 \\
Param.\ Counts & 1,075,361 & 1,065,698 & 35,746,307 & 44,807,843 & 37,064,707 & 46,210,083\\
\bottomrule
\end{tabular}
}
\caption{Computational requirements during training on image datasets. }
\label{tab:training_image}
\end{table}

\begin{table}[t]
\centering
\resizebox{1.0\columnwidth}{!}{
\begin{tabular}{ccccccc}
\toprule
\textbf{Inference time (s)} 
& \multicolumn{2}{c}{\textbf{MNIST}} 
& \multicolumn{2}{c}{\textbf{CIFAR-10}}
& \multicolumn{2}{c}{\textbf{ImageNet-32}}\\
\cmidrule(r){2-3} \cmidrule(r){4-5} \cmidrule(r){6-7}
\textbf{Total NFEs} & 
\textbf{RF (1.08M)} & \textbf{HRF2 (1.07M)} & 
\textbf{RF (35.75M)} & \textbf{HRF2 (44.81M)} &
\textbf{RF (37.06M)} & \textbf{HRF2 (46.21M)} \\
\midrule
5 & 0.084 ± 0.001 & 0.085 ± 0.001 & 0.221 ± 0.000 & 0.295 ± 0.000 & 0.229 ± 0.000 & 0.301 ± 0.000 \\
10 & 0.168 ± 0.000 & 0.169 ± 0.000 & 0.441 ± 0.001 & 0.589 ± 0.001 & 0.458 ± 0.000 & 0.601 ± 0.000 \\
20 & 0.336 ± 0.000 & 0.339 ± 0.000 & 0.889 ± 0.001 & 1.176 ± 0.001 & 0.918 ± 0.001 & 1.207 ± 0.001 \\
50 & 0.843 ± 0.001 & 0.851 ± 0.002 & 2.229 ± 0.001 & 2.953 ± 0.004 & 2.302 ± 0.002 & 3.029 ± 0.004 \\
100 & 1.693 ± 0.002 & 1.706 ± 0.003 & 4.471 ± 0.004 & 5.921 ± 0.003 & 4.618 ± 0.003 & 6.100 ± 0.014 \\
500 & 8.538 ± 0.030 & 8.598 ± 0.010 & 22.375 ± 0.011 & 29.701 ± 0.011 & 23.110 ± 0.005 & 30.863 ± 0.083 \\
\bottomrule
\end{tabular}
}
\caption{Inference time comparison for MNIST, CIFAR-10, and ImageNet-32 datasets using a varying total NFEs budget. For HRF2 on MNIST we used sampling step combinations: $(1,5),(2,5),(5,4),(5,10),(5,20),(5,100)$. For HRF2 on CIFAR-10 and ImageNet-32 we used sampling step combinations: $(1,5),(1,10),(1,20),(1,50),(2,50),(2,250)$. All experiments are conducted with a batch size of 128. }
\label{tab:infer_time_image}
\end{table}

\begin{table}[t]
\centering
\resizebox{1.0\columnwidth}{!}{
\begin{tabular}{ccccccc}
\toprule
\textbf{Performance (FID)} 
& \multicolumn{2}{c}{\textbf{MNIST}} 
& \multicolumn{2}{c}{\textbf{CIFAR-10}}
& \multicolumn{2}{c}{\textbf{ImageNet-32}}\\
\cmidrule(r){2-3} \cmidrule(r){4-5} \cmidrule(r){6-7}
\textbf{Total NFEs} & 
\textbf{RF (1.08M)} & \textbf{HRF2 (1.07M)} & 
\textbf{RF (35.75M)} & \textbf{HRF2 (44.81M)} & 
\textbf{RF (37.06M)} & \textbf{HRF2 (46.21M)} \\
\midrule

5 & 19.187 ± 0.188 & \textbf{15.798 ± 0.151} & 36.209 ± 0.142 & \textbf{30.884 ± 0.104} & 69.233 ± 0.166 & \textbf{48.933 ± 0.177} \\
10 & 7.974 ± 0.119 & \textbf{6.644 ± 0.076} & 14.113 ± 0.092 & \textbf{12.065 ± 0.024} & 21.744 ± 0.045 & \textbf{20.286 ± 0.022} \\
20 & 6.151 ± 0.090 & \textbf{3.408 ± 0.076} & 8.355 ± 0.065 & \textbf{7.129 ± 0.027} & \textbf{12.411 ± 0.002} & 12.492 ± 0.100 \\
50 & 5.605 ± 0.057 & \textbf{2.664 ± 0.058} & 5.514 ± 0.034 & \textbf{4.847 ± 0.028} & \textbf{8.910 ± 0.137} & 9.024 ± 0.112 \\
100 & 5.563 ± 0.049 & \textbf{2.588 ± 0.075} & 4.588 ± 0.013 & \textbf{4.334 ± 0.054} & 7.873 ± 0.110 & \textbf{7.679 ± 0.022} \\
500 & 5.453 ± 0.047 & \textbf{2.574 ± 0.121} & 3.887 ± 0.035 & \textbf{3.706 ± 0.043} & 6.962 ± 0.087 & \textbf{6.503 ± 0.035} \\
\bottomrule
\end{tabular}
}
\caption{Performance comparison for MNIST, CIFAR-10, and ImageNet-32 datasets using a varying total NFEs budget. For HRF2 on MNIST we used sampling step combinations: $(5,1),(10,1),(5,4),(10,5),(10,10),(100,5)$. For HRF2 on CIFAR-10 and ImageNet-32 we used sampling step combinations: $(1,5),(1,10),(1,20),(1,50),(2,50),(2,250)$. \textbf{Bold} for lower mean. }
\label{tab:performance}
\end{table}

