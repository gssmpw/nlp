\section{Experiments}
\label{sec:exp}
The studied hierarchical rectified flow (HRF) formulation couples multiple ODEs to accurately model the multimodal velocity distribution. To assess  efficacy of this formulation, we first validate the approach in low-dimensional settings, where the analytical form of the velocity distribution is straightforward to compute. This allows us to verify that the model can indeed capture the velocity distribution accurately. We then investigate whether fitting the velocity distribution enhances the model's ability to fit the data distribution in generative tasks. We perform experiments on 1D data (\cref{sec:exp:1D}), 2D data (\cref{sec:exp:2D}), and high-dimensional image data (\cref{sec:exp:img}) with depth two HRF (HRF2) models: the models not only fit the velocity distribution but also enhance the quality of the generative process. We also include results for depth three HRF (HRF3) models on low dimensional data to show the potential for exploring deeper hierarchical structures. Importantly, for all experiments we report \emph{total number of function evaluations (NFEs)}, i.e., the product of the number of integration steps at all HRF levels. %

\begin{figure}[t]
    \centering
    \setlength{\tabcolsep}{0pt}
    \begin{tabular}{cccc}
    \includegraphics[width=0.25\linewidth]{fig/fig3_v_dist/vdist_hrf_-1_0.0.pdf}&
    \includegraphics[width=0.25\linewidth]{fig/fig3_v_dist/vdist_hrf_0_0.4.pdf}&
    \includegraphics[width=0.25\linewidth]{fig/fig3_v_dist/vdist_hrf_0.5_0.6.pdf}&
    \includegraphics[width=0.25\linewidth]{fig/fig3_v_dist/vdist_hrf_1_1.0.pdf}\\
    (a) $(x_t,t)=(-1.0, 0.0)$ & (b) $(x_t,t)=(0.0, 0.4)$ & (c) $(x_t,t)=(0.5, 0.6)$ & (d) $(x_t,t)=(1.0, 1.0)$
    \end{tabular}
    \caption{Numerical estimation of $\pi_1(x_t, t)$ in HRF2 with different number of $v$ integration steps. The blue line shows the ground-truth $\pi_1$, where $\rho_0$ is a standard Gaussian and $\rho_1$ is a mixture of two Gaussians. The 1-Wasserstein distances (WD) for the estimates w.r.t. $\pi_1$ are shown in the legend. 
    }
    \label{fig:emp_v_dist}
\end{figure}



\subsection{Synthetic 1D Data}
\label{sec:exp:1D}
For the 1D experiments, we first consider a standard Gaussian source distribution and a target distribution represented by a mixture of two Gaussians. Using \cref{eq:true_v_pdf}, we can compute the analytical form of the velocity distribution. As shown in \cref{fig:emp_v_dist}, our model captures the analytic velocity distribution with high accuracy. As expected, an increasing number of velocity integration steps increases the accuracy of the estimated velocity. The only exception occurs when $t$ approaches $1$. The model performance deteriorates, and excessive steps accumulate errors, leading to less accurate results. %

Next, we examine the data generation quality. We use a mixture of five Gaussians as illustrated in \cref{fig:1d_data}(a) and compare HRF to a baseline rectified flow (RF). We use the $1$-Wasserstein distance (WD) as a metric to assess the quality of the generated data. As shown in \cref{fig:1d_data}(b), for the same neural function evaluations (NFEs), the HRF models outperform the baseline, producing data distributions with a lower WD, indicating superior quality. %
In~\cref{fig:1d_data}'s legend, the term ``v steps'' refers to the number of velocity integration steps. In this 1D experiment, HRF3 demonstrates better performance compared to HRF2. More 1D  results are provided in \cref{sec:exp_more_results}. 

Additionally, we observe a fundamental difference in the generated trajectories. %
Since rectified flow estimates only the mean of the velocity distribution, it tends to move towards the center of the target distributions initially. %
In contrast, the HRF model determines the next direction at each space-time location based on the current velocity distribution. As shown in \cref{fig:1d_data}(d), the HRF2 trajectories are nearly linear and can intersect, which permits to use fewer data sampling steps during generation. 

For the deep net, we use simple embedding layers and linear layers to first process the space and time information separately. Afterward, we concatenate these representations. %
This combined input is then passed through a series of fully connected layers, allowing the model to capture complex interactions and extract high-level features essential for accurate velocity prediction. We use the same architecture for the baseline model but increase the dimension of the hidden layers to optimize its performance. In contrast, the HRF2 model %
contains only 74,497 parameters compared to 297,089 parameters for the baseline model. This demonstrates the potential efficiency of HRF in handling higher-dimensional data while maintaining a more compact architecture. More details of the experiments are provided in \cref{sec:exp_details}. 

\begin{figure}[t]
    \centering
    \setlength{\tabcolsep}{0pt}
    \begin{tabular}{cccc}
    \includegraphics[width=0.25\linewidth]{fig/fig4_1d/1to5_dist.pdf}&
    \includegraphics[width=0.25\linewidth]{fig/fig4_1d/new_1to5_WD_NFE.pdf}&
    \includegraphics[width=0.25\linewidth]{fig/fig4_1d/1to5_traj_400.pdf}&
    \includegraphics[width=0.25\linewidth]{fig/fig4_1d/1to5_traj_20_20.pdf}\\
    (a) Data distribution & (b) Metrics & (c) RF trajectories & (d) HRF2 trajectories
    \end{tabular}
    \vspace{-3mm}
    \caption{Results on 1D example, where $\rho_0$ is a standard Gaussian and $\rho_1$ is a mixture of 5 Gaussians. (a) Histograms of generated samples and $\rho_1$. (b) The 1-Wasserstein distance vs.\ NFE. (c) and (d) The trajectories of particles flowing from source distribution (grey) to target distribution (blue). }
    \label{fig:1d_data}
    \vspace{-3mm}
\end{figure}

\begin{figure}[t]
    \centering
    \setlength{\tabcolsep}{0pt}
    \begin{tabular}{cccc}
    \includegraphics[width=0.25\linewidth]{fig/fig5_2d/new_2D1to6_WD_NFE.pdf}&
    \includegraphics[width=0.25\linewidth]{fig/fig5_2d/2D1to6_traj_400.pdf}&
    \includegraphics[width=0.25\linewidth]{fig/fig5_2d/2D1to6_traj_20_20.pdf}&
    \includegraphics[width=0.25\linewidth]{fig/fig5_2d/2D1to6_traj_hrf3.pdf}\\
    \includegraphics[width=0.25\linewidth]{fig/fig5_2d/new_moon_WD_NFE.pdf}&
    \includegraphics[width=0.25\linewidth]{fig/fig5_2d/moon_traj_400.pdf}&
    \includegraphics[width=0.25\linewidth]{fig/fig5_2d/moon_traj_20_20.pdf}&
    \includegraphics[width=0.25\linewidth]{fig/fig5_2d/moon_traj_hrf3.pdf}\\
    (a) Metrics & (b) RF trajectories & (c) HRF2 trajectories & (d) HRF3 trajectories
    \end{tabular}
    \vspace{-3mm}
    \caption{Results on 2D data. Top row: $\rho_0$ is a standard Gaussian and $\rho_1$ is a mixture of 6 Gaussians. Bottom row: $\rho_0$ is a mixture 8 Gaussians and $\rho_1$ is represented by the moons data. (a) Sliced 2-Wasserstein distance with respect to NFE. (b) and (c) show the trajectories (green) of sample particles flowing from source distribution (grey) to target distribution (blue).
    }
    \label{fig:2d_data}
    \vspace{-1mm}
\end{figure}

\subsection{Synthetic 2D Data}
\label{sec:exp:2D}
For the 2D experiments, we consider two settings: 1) a standard Gaussian source distribution and a target distribution consisting of a mixture of six Gaussians; and 2) a mixture of eight Gaussians as the source distribution and the moons dataset as the target distribution. We employ the same network architecture as used in the 1D experiments. Due to the 2D data, we now have 76,674 parameters for the HRF2 model and 329,986 parameters for the baseline RF model. We measure the quality of data generation using the sliced 2-Wasserstein distance (SWD). \cref{fig:2d_data} shows the results. It is evident that on these more complex datasets, the performance gap between an HRF model and the rectified flow baseline is more pronounced. The trajectories demonstrate similar patterns to those observed in the 1D experiments: HRF2 produces significantly straighter paths, while the rectified flow baseline often exhibits large directional changes. Additionally, the HRF models consistently achieve higher quality in data generation compared to the baseline. %
HRF3 outperforms HRF2 for generating the moon data from a mixture of 8 Gaussians. However, HRF2 works better for the simpler mixture of Gaussian target. There is  room to improve the training and scheduling of the integration steps among different layers for deeper HRF models.  %


\subsection{Image Data}
\label{sec:exp:img}
In addition to low-dimensional data, we also conduct experiments on high-dimensional image datasets including MNIST~\citep{lecun1998gradient}, CIFAR-10~\citep{krizhevsky2009learning}, and ImageNet-32~\citep{deng2009imagenet}. We employ the Fr\'{e}chet Inception Distance (FID) as the metric for evaluating image generation quality. 
For the baseline, we use the same U-Net architecture as \citet{LipmanICLR2023} and successfully reproduced the state-of-the-art results across all datasets. The HRF model builds upon this U-Net structure. 
We follow the parameter settings and training procedures from \citet{tong2023improving} and \citet{LipmanICLR2023}. Further details on the architecture and training setup are provided in \cref{sec:exp_details}. 

As shown in \cref{fig:img_data}, for the same total NFEs, the HRF2 model demonstrates better performance on MNIST and CIFAR-10, and on-par performance on ImageNet-32 when compared to the baseline. 

\begin{figure}[t]
    \centering
    \setlength{\tabcolsep}{0pt}
    \begin{tabular}{ccc}
    \hspace{10pt}
    \includegraphics[width=0.3\linewidth]{fig/fig6_img/mnist_10_50_48.pdf}&
    \hspace{10pt}
    \includegraphics[width=0.3\linewidth]{fig/fig6_img/cifar10_2_250_48.pdf}&
    \hspace{10pt}
    \includegraphics[width=0.3\linewidth]{fig/fig6_img/imagenet_2_250_48.pdf}\\
    \includegraphics[width=0.3\linewidth]{fig/fig6_img/mnist_fid_v2.pdf}&
    \includegraphics[width=0.3\linewidth]{fig/fig6_img/cifar10_fid_v2.pdf}&
    \includegraphics[width=0.3\linewidth]{fig/fig6_img/imagenet_fid.pdf}\\
    (a) MNIST & (b) CIFAR-10 & (c) ImageNet-32
    \end{tabular}
    \vspace{-3mm}
    \caption{ Experimental results on (a) MNIST, (b) CIFAR-10, and (c) ImageNet-32 datasets. Top row: samples of generated images, bottom row: FID scores with respect to total NFEs. }
    \label{fig:img_data}
    \vspace{-1mm}
\end{figure}

