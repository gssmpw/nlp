\subsection{Hierarchical Rectified Flow with OTCFM}
\label{sec:batchOT}
As mentioned in~\cref{sec:rel}, various approaches for straightening the paths in flow matching models exist. These approaches are orthogonal to our work and can be easily incorporated in the HRF formulation. To demonstrate this, we incorporate the minibatch optimal transport conditional flow matching (OTCFM)~\citep{tong2023improving} into the two layered hierarchical rectified flow (HRF2). In OTCFM, for each batch of data $(\{x_0^{(i)} \}_{i = 1}^B, \{x^{(i)}_1\}_{i=1}^B)$ seen during training, we sample pairs of points
from the joint distribution $\gamma_{\text{batch}}(x_0, x_1)$ given by the optimal transport plan between the source and target points in the batch. We follow the same procedure to couple noise with the data points and use the batch-wise coupled $x_0$ and $x_1$ to learn the parameters in $a_\theta$. We refer to this approach as HOTCFM2. We test its performance on two synthetic examples: 1) a 1D example with a standard Gaussian source distribution and a mixture of two Gaussians as the target distribution; and 2) a 2D example with a mixture of eight Gaussians as the source distribution and the moons dataset as the target distribution.  \cref{fig:1d_batchot} and \cref{fig:2d_batchot} show that hierarchical rectified flow improves the performance of OTCFM.   

\begin{figure}[t]
    \centering
    \setlength{\tabcolsep}{0pt}
    \begin{tabular}{cccc}
    \includegraphics[width=0.25\linewidth]{fig/fig10_batchot/1to2_dist.pdf}&
    \includegraphics[width=0.25\linewidth]{fig/fig10_batchot/1to2_WD_NFE.pdf}&
    \includegraphics[width=0.25\linewidth]{fig/fig10_batchot/1to2_traj_400.pdf}&
    \includegraphics[width=0.25\linewidth]{fig/fig10_batchot/1to2_traj_20_20.pdf} \\
    (a) Data distribution & (b) Metrics & (c) OTCFM trajectories & (d) HOTCFM2 trajectories
    \end{tabular}
    \caption{Results for 1D data, with $\rho_0$ being a standard Gaussian and $\rho_1$ being a mixture of 2 Gaussians. (a) Histograms of generated samples and $\rho_1$. (b) The 1-Wasserstein distance vs.\ total NFEs. (c,d) The trajectories of particles flowing from source distribution (grey) to target distribution (blue). }
    \label{fig:1d_batchot}
\end{figure}

\begin{figure}[t]
    \centering
    \setlength{\tabcolsep}{0pt}
    \begin{tabular}{cccc}
    \includegraphics[width=0.3\linewidth]{fig/fig10_batchot/moon_WD_NFE.pdf}&
    \includegraphics[width=0.3\linewidth]{fig/fig10_batchot/moon_traj_400.pdf}&
    \includegraphics[width=0.3\linewidth]{fig/fig10_batchot/moon_traj_20_20.pdf} \\
    (a) Metrics & (b) OTCFM trajectories & (c) HOTCFM2 trajectories
    \end{tabular}
    \caption{Results for 2D data, with $\rho_0$ being a mixture of 8 Gaussians and $\rho_1$ being represented by the moons data. (a) Sliced 2-Wasserstein distance vs.\ total NFEs. (b) and (c) show the trajectories (green) of sample particles flowing from source distribution (grey) to target distribution (blue). }
    \label{fig:2d_batchot}
\end{figure}
