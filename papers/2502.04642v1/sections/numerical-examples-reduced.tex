\section{Numerical Example: Dynamic Price Control for EV Charging}
\label{sec:numerical-examples}

In this section, we consider dynamic price control for electric vehicle (EV) charging (see Sec.~\ref{subsec:motivating-example}).
We mostly use the dynamical systems and costs in \cite{ma2013decentralized,zou2017efficient}, but specify when we deviate.
The set of MPC problems solved by the EVs comprises the set of LoMPCs.
The optimization problem solved by the ISO comprises the BiMPC problem.
Together, the BiMPC and the LoMPC problems form an incentive hierarchical MPC problem, where the incentive provided by the ISO is the unit price of electricity.
The Python code for the example can be found in the repository\footnote{
\label{code}
The implementation code can be found at \url{https://github.com/AkshayThiru/incentive-design-mpc}.
}.
Next, we describe the LoMPC and BiMPC problems in detail.


\subsection{LoMPC Formulation}
\label{subsec:ev-lompc-formulation}

The $i$-th EV has a state variable $y^i_k \in \real$, which denotes its battery state of charge (SoC) as a fraction of its total capacity $\Theta^i$ at time step $k$.
$y^i_k$ is called the normalized SoC of EV $i$.
The fraction of the EV battery capacity charged during time interval $k$ is given by $w^i_k \in \real$.
$w^i_k$ is called the normalized charging rate at time step $k$.
In addition to the system considered in \cite{zou2017efficient}, we assume that $w^i_k$ is bounded by $w^i_\text{max}$.
While a more realistic model of battery charging might consider $w^i_\text{max}$ as a function of the SoC, we ignore this complication.
The normalized SoC of EV $i$ evolves according to the dynamics:
\begin{equation}
\label{eq:ev-lompc-dynamics}
\begin{gathered}
    y^i_{k+1} = y^i_k + w^i_k, \\
    0 \leq w^i_k \leq w^i_\text{max} \leq 1, \quad 0 \leq y^i_k \leq y^i_\text{max} \leq 1.
\end{gathered}
\end{equation}
% where $y^i_\text{max}$ is the maximum normalized SoC of EV $i$.
We assume that the EV population comprises $500$ small and large EVs for a total population size of $M = 1000$.
Additionally, all the small EVs have the same values for $\Theta, w_\text{max}$, and $y_\text{max}$, given by $\Theta^s, w^s_\text{max}$, and $y^s_\text{max}$, respectively.
Likewise, the large EVs share the values $\Theta^l, w^l_\text{max}$, and $y^l_\text{max}$.
While this is a simplifying assumption compared to \cite{zou2017efficient}, we note that \cite{zou2017efficient} sets different prices for each EV.
On the other hand, we set identical electricity prices for large parts of the EV population.
In the rest of the subsection, we discuss the small EV LoMPC formulation; a similar formulation is used for the large EVs.

Each small EV $i$ solves a LoMPC problem with horizon $\horizon$ to determine the amount of electricity it draws during a time interval.
Following the notation in Tab.~\ref{tab:hierarchical-mpc-variable-definitions}, let $y^i = (y^i_0, ..., y^i_{\horizon})$ and $w^i = (w^i_0, ..., w^i_{\horizon-1})$.
The dynamics and constraints of small EV $i$ are given by \eqref{eq:ev-lompc-dynamics} with $w^i_\text{max} = w^s_\text{max}$ and $y^i_\text{max} = y^s_\text{max}$.
Without loss of generality, we ignore the state constraints $\mathbb{0} \leq y^i \! \leq y^s_\text{max} \mathbb{1}$ by assuming that EVs stop charging once their maximum normalized SoC $y^s_\text{max}$ is attained.
We also assume, for simplicity, that once an EV is fully charged, it is replaced by another EV with a random normalized SoC.
We also define the total battery capacity of the EV population as follows:
\begin{equation}
\label{eq:ev-lompc-b}
    B = (M/2) \Theta^s + (M/2) \Theta^l.
\end{equation}

The LoMPC of each EV has a cost function with three components, which we describe next.

\subsubsection{Battery Degradation Cost}

The charging behavior over time affects an EV battery's health.
We associate a battery degradation cost function $g^i_\text{bat}$, a function of the normalized charging rate $w^i_k$, to the rate of decline of the battery health \cite{zou2017efficient}.
The total battery degradation cost is given by
\begin{equation}
\label{eq:ev-lompc-battery-degradation-cost}
    (\Theta^i)^2 \, \textstyle{\sum}_{k=0}^{\horizon-1} \, g^i_\text{bat}(w^i_k),
\end{equation}
where the $(\Theta^i)^2$ factor normalizes the battery degradation cost term with respect to the other cost terms.
We assume that $g^i_\text{bat}$ can be any closed convex function (possibly nonsmooth) and that $g^i_\text{bat}$ is unknown to the ISO.
Additionally, all small and large EVs share the battery degradation cost functions $g^s_\text{bat}$ and $g^l_\text{bat}$, respectively (the exact forms can be found in the code\footref{code}).

\subsubsection{Charging Cost}

Each EV $i$ has a quadratic tracking cost at time $k$ given by $\delta (\Theta^i)^2(y^i_\text{max} - y^i_k)^2$, where $\delta > 0$ is a fixed parameter.
This cost function imposes a penalty for not being at the desired SoC.
The total charging cost is given by
\begin{equation}
\label{eq:ev-lompc-tracking-cost}
    \delta (\Theta^i)^2 \textstyle{\sum}_{k=1}^N \, (y^i_\text{max} - y^i_k)^2.
\end{equation}

\subsubsection{Electricity Cost}

While \cite{zou2017efficient} considers an auction mechanism, we assume that the ISO sets identical electricity prices for parts of the EV population.
% We also associate a cost with the electricity consumed by EV $i$ at time interval $k$.
Since electricity prices are provided by the ISO as incentives to the EVs, the electricity cost is the incentive cost term, as defined in \eqref{eq:linear-convex-incentivized-lompc}.
The total electricity cost for EV $i$ is given by
\begin{equation}
\label{eq:ev-lompc-electricity-price-cost}
    \Theta^i \bigl(\lambda_{l_1}^\top w^i + \lambda_{l_2}^\top (w^i_\text{max} \mathbb{1} - w^i) + (w^i)^\top \text{diag}(\lambda_q) w^i \bigr),
\end{equation}
where $\lambda_{l_1}, \lambda_{l_2}, \lambda_q \in \real^\horizon_+$
% , $\lambda_r \geq 0$,
and $\lambda = (\lambda_{l_1}, \lambda_{l_2}, \lambda_q) \in \real^{3\horizon}_+$ is the incentive.
The first term in \eqref{eq:ev-lompc-electricity-price-cost} penalizes higher electricity consumption, while the third term adds a nonlinear electricity cost.
The second term reduces prices for higher consumption.
This can be interpreted as the ISO providing compensation to the EVs for charging at a higher rate at the expense of EV battery degradation.
The electricity cost for small EV $i$ can be expressed as $\langle \lambda, \phi^s(w^i)\rangle$, where % $\phi^s: \real^\horizon \rightarrow \real^{3\horizon}$ is given by
\begin{equation}
\label{eq:ev-phi-function}
    \phi^s(w^i) = \Theta^s \bigl(w^i, w^s_\text{max} \mathbb{1} - w^i, w^i \odot w^i\bigr),
\end{equation}
where $\odot$ represents the Hadamard product.
From \eqref{eq:ev-phi-function} and Rem.~\ref{rem:linear-convex-incentive-componentwise}, $\phi^s$ is $\mathset{K}$-convex with $\mathset{K} = \real^{3\horizon}_+$, and $\lambda \in \mathset{K}^* = \real^{3\horizon}_+$.
Thus, $\langle \lambda, \phi^s(w^i)\rangle$ is a linear-convex incentive (see Def.~\ref{def:linear-convex-incentive}).

The LoMPC problem for small EV $i$ is given by the dynamics \eqref{eq:ev-lompc-dynamics} (without the state constraints) and the three cost terms discussed above.
We can verify that the small EV LoMPC matches the form in \eqref{eq:input-constrained-lqr}.
Thus, by Rem.~\ref{rem:multiple-lompcs-lompc-lqr}, the small EV LoMPC satisfies Assumption~\ref{assum:multiple-lompcs-properties-of-gi} and can be written in the parametric form \eqref{eq:parametric-form-gi} with
\begin{equation}
\begin{gathered}
    \theta^i(\bm{y}_0) = 2\delta (\Theta^s)^2 \bigl( y^s_{0, c} - y^i_0 \bigr) \mathbb{1}, \\
    y^s_{0, c} = \frac{1}{2} \biggl(\max_{i \in \text{small EVs}} \{y^i_0\} + \min_{i \in \text{small EVs}} \{y^i_0\}\biggr).
\end{gathered}
\end{equation}
Similarly, we can define the LoMPC problem for a large EV.


\subsection{LoMPC Solution Error Bounds}
\label{subsec:ev-lompc-solution-error-bounds}

\begin{figure}%[!htp]
    \centering
    \includegraphics[width=0.99\columnwidth]{figures/robustness_bounds.png}
    \caption{Verification of the LoMPC solution error bound in \eqref{eq:ev-w-error-bound} for large EVs with $M = 20$.
    For a given $\Delta y^l_0$, we randomly generate an incentive $\lambda^l$, compute the team-optimal solution $\hat{w}^l$ as the optimal solution of the EV LoMPC with $\theta^i = 0$ (see \eqref{eq:bounded-incentive-controllability-w-hat}), and compute $w^l$ using \eqref{eq:ev-w-error-bound}.
    The plot shows that for any desired average electricity consumption $\hat{w}^l$ (that is feasible), we can set the unit price of electricity $\lambda^l$ such that the actual average electricity consumption $w^l$ has a bounded error compared to $\hat{w}^l$.
    The error bound is given by $\sqrt{\horizon} \Delta y^l_0$, where $\Delta y^l_0$ is the range of normalized SoC of the large EV population.
    }
    \label{fig:ev-robustness-bound}
\end{figure}

In this subsection, we show that Assumptions~\ref{assum:properties-of-gi}, \ref{assum:linear-convex-incentive-controllability}, and \ref{assum:multiple-lompcs-properties-of-gi} hold for the EV charging example, and thus we can apply all the results derived in the paper.
In particular, we compute the error bound provided by Lem.~\ref{lem:bounded-incentive-controllability}.

Since the EV LoMPC costs are convex, Assumption~\ref{assum:properties-of-gi} is satisfied.
Moreover, the small EV LoMPC cost functions have a strong-convexity modulus of $\constsc^s = 2 \delta (\Theta^s)^2$, while those of large EVs have a strong-convexity modulus of $\constsc^l = 2 \delta (\Theta^l)^2$.

Next, we show that Assumption~\ref{assum:linear-convex-incentive-controllability} holds for both small EV and large EV incentives.
If $\phi^s(w^1) \preceq_{\mathset{K}} \phi^s(w^2)$, then from \eqref{eq:ev-phi-function}, $w^1 \leq w^2$ and $w^2 \leq w^1$ $\Rightarrow \ w^1 = w^2$.
For any $w^* \in \real^\horizon$,
\begin{equation*}
    D\phi^s(w^*)^\top = \Theta^s [I, \ -I, \ 2 \, \text{diag}(w^*)].
\end{equation*}
Then, for any $v \in \real^\horizon$, $D\phi^s(w^*)^\top\lambda = v$, where
\begin{equation*}
\lambda = (\max\{v, \mathbb{0}\}, -\min\{v, \mathbb{0}\}, \mathbb{0}) / \Theta^s \succeq_{\mathset{K}^*} 0.
\end{equation*}
Thus $\phi^s$, and similarly $\phi^l$, satisfies Assumption~\ref{assum:linear-convex-incentive-controllability}.

Finally, as discussed in Sec.~\ref{subsec:ev-lompc-formulation}, Assumption~\ref{assum:multiple-lompcs-properties-of-gi} holds for small EVs (and likewise for large EVs), with
\begin{equation}
\label{eq:ev-theta-bar}
\begin{gathered}
    \lVert \theta^i(\bm{y}_0)\rVert \leq \bar{\theta}^s = 2 \delta (\Theta^s)^2 \Delta y_0^s \sqrt{\horizon}, \\
    \Delta y_0^s = \frac{1}{2} \biggl(\max_{i \in \text{small EVs}} \{y^i_0\} - \min_{i \in \text{small EVs}} \{y^i_0\}\biggr).
\end{gathered}
\end{equation}

Since Assumptions~\ref{assum:linear-convex-incentive-controllability} and \ref{assum:multiple-lompcs-properties-of-gi} hold, we can use Lem.~\ref{lem:bounded-incentive-controllability} to provide an error bound on the small EV LoMPC solution.
Let $\hat{w}^s$ be such that $\mathbb{0} \leq \hat{w}^s \leq w^s_\text{max} \mathbb{1}$.
Then, we can find an incentive $\lambda^s \succeq_{\mathset{K}^*} 0$ such that
\begin{equation}
\label{eq:ev-w-error-bound}
\begin{gathered}
    \lVert \hat{w}^s - w^s \rVert \leq \bar{\theta}^s/\constsc^s = \sqrt{N} \Delta y_0^s, \\
    w^s = \frac{1}{M} \sum_{i \in \text{small EVs}} w^{i*}(\xi_0, \lambda^s),
\end{gathered}
\end{equation}
where $w^{i*}(\xi_0, \lambda)$ is the optimal solution of the EV $i$ LoMPC.
In particular, we note that the error bound in \eqref{eq:ev-w-error-bound} depends on the horizon $N$ and the range of initial normalized SoCs $\Delta y^s_0$, but not on the population size $M$.
We verify the dependence of the error bound on $\Delta y^s_0$ in Fig.~\ref{fig:ev-robustness-bound}.


For a given $\hat{w}^s$ such that $\mathbb{0} \leq \hat{w}^s \leq w^s_\text{max} \mathbb{1}$, an optimal incentive $\lambda^{s*}$ is computed using the incentive solver given by Alg.~\ref{alg:iterative-method}.
In Fig.~\ref{fig:ev-dual-cost-decrease}, we verify the incentive solver, derived from Thm.~\ref{thm:linear-convex-incentives-iterative-method}, by checking the dual cost decrease condition \eqref{eq:linear-convex-incentives-iterative-method-dual-cost-decrease} for each solver iteration.


\begin{figure}%[!htp]
    \centering
    \includegraphics[width=0.99\columnwidth]{figures/dual_cost_decrease.png}
    \caption{Verification of the incentive solver in Alg.~\ref{alg:iterative-method}.
    The incentive solver uses the majorization-minimization method to compute an optimal incentive $\lambda^*$ iteratively.
    For each iterate $\lambda^{(k)}$, the dual cost $-\tilde{g}^*$ is majorized by $-\tilde{g}^*(\cdot \,; \lambda^{(k)})$, which is then minimized (see Thm.~\ref{thm:linear-convex-incentives-iterative-method}).
    The solid line shows the actual decrease in the dual cost function $-\tilde{g}^*$ at each iteration, which is guaranteed to be greater than the decrease in $-\tilde{g}^*(\cdot \,; \lambda^{(k)})$ (the dashed line).
    The plot is shown for large EVs with $M = 200$.
    }
    \label{fig:ev-dual-cost-decrease}
\end{figure}

For the robust BiMPC \eqref{eq:robust-bimpc}, we use a robustness horizon $\horizon_r = 1$, i.e. we only account for the error in LoMPC solution at time step $0$ (see Rem.~\ref{rem:robustness-horizon}).
In Sec.~\ref{subsec:ev-bimpc-formulation}, we find conditions under which the robust BiMPC \eqref{eq:robust-bimpc} with $\horizon_r = 1$ is always feasible.
The LoMPC solution error bound at time step $0$ can be found from \eqref{eq:ev-w-error-bound} as follows:
\begin{equation}
\label{eq:ev-w0-error-bound}
    | \hat{w}^s_0 - w^s_0 | \leq \sqrt{N} \Delta y_0^s, \quad | \hat{w}^l_0 - w^l_0 | \leq \sqrt{N} \Delta y_0^l.
\end{equation}


\subsection{BiMPC Formulation}
\label{subsec:ev-bimpc-formulation}

The ISO decides the amount of electricity to be generated in each time interval, depending on the SoCs of the EV population and a forecast of the external electricity demand~\cite{zou2017efficient}.
In addition to the model in \cite{zou2017efficient}, we assume that the ISO controls a storage battery that can be charged or depleted depending on the total electricity demand.
The ISO has state variable $x_k \in \real$, which denotes the state of charge of its storage battery at the time step $k$.
The amount of electricity generated in time interval $k$ is denoted by $u^g_k \in \real$, and the electricity discharged from the storage battery in time interval $k$ is denoted by $u^b_k \in \real$.
The unit prices of electricity for small and large EVs are given by $\lambda^s, \lambda^l \in \mathset{K}^*$, respectively.
The state dynamics are as follows:
\begin{equation}
\label{eq:ev-bimpc-dynamics}
\begin{gathered}
    x_{k+1} = x_k + u^b_k, \\
    u^g_k = (M/2) (\Theta^s w^s_k + \Theta^l w^l_k) + D_k + u^b_k, \\
    0 \leq x_k \leq x_\text{max}, \\
    0 \leq u^g_k \leq u^g_\text{max}, \quad -u^b_\text{max} \leq u^b_k \leq u^b_\text{max},
\end{gathered}
\end{equation}
where $w^s$ (and likewise $w^l$), defined in \eqref{eq:ev-w-error-bound}, is the average electricity consumption for small EVs corresponding to unit price $\lambda^s$.
$D_k$ is the forecast of external electricity demand in time interval $k$~\cite{zou2017efficient}, and $x_\text{max}$ is the capacity of the storage battery.
$u^g_\text{max}$ is the maximum electricity generation in a time interval, and $u^b_\text{max}$ is the maximum charge/discharge amount for the storage battery in a time interval.

The BiMPC cost comprises electricity generation cost and aggregate charging cost~\cite{zou2017efficient}.
The electricity generation cost penalizes high electricity generation and is given by
\begin{equation}
\label{eq:ev-bimpc-generation-cost}
    \textstyle{\sum}_{k=0}^{\horizon-1} \, c^g (u^g_k)^{1.7},
\end{equation}
where $c^g > 0$ is a parameter.
The aggregate charging cost is similar to the LoMPC charging cost and is given by
\begin{align}
\label{eq:ev-bimpc-tracking-cost}
    \smash{\sum_{k=1}^N} \, \gamma^{k-N} \bigl[ \, & ( \textstyle{\sum}_{k'=1}^k w^s_{k'} - (y^s_\text{max} - y^s_{0, m}))^2 \\
    & + ( \textstyle{\sum}_{k'=1}^k w^l_{k'} - (y^l_\text{max} - y^l_{0, m}))^2 \,\bigr], \nonumber
\end{align}
where $y^s_{0, m}$ and $y^l_{0, m}$ are mean initial normalized SoCs and $\gamma > 1$ is a parameter.
% The aggregate charging cost imposes a penalty for uncharged EVs that increases exponentially over the horizon.
For the robust BiMPC formulation, we define $\hat{w}^s$ (and likewise $\hat{w}^l$) as the desired average electricity consumption for small EVs.
Then, for any $\hat{w}^s$ satisfying $\mathbb{0} \leq \hat{w}^s \leq w^s_\text{max} \mathbb{1}$, we can find an incentive $\lambda^s$ such that the error bounds in \eqref{eq:ev-w-error-bound} and \eqref{eq:ev-w0-error-bound} hold.
Using the LoMPC solution error bound \eqref{eq:ev-w0-error-bound} for robustness horizon $\horizon_r = 1$, we can write the constraints of the robust BiMPC \eqref{eq:robust-bimpc} as (see Rem.~\ref{rem:explicit-robust-bimpc-formulation}) follows:
\begin{equation}
\label{eq:ev-robust-bimpc-constraints}
\begin{gathered}
    \hat{u}^b = u^g - D - (M/2)(\Theta^s \hat{w}^s + \Theta^l \hat{w}^l), \\
    x_{k+1} = x_k + \hat{u}^b_k, \quad k \in \langle N\rangle, \\
    \mathbb{0} \leq u^g \leq u^g_\text{max} \mathbb{1}, \\
    -u^b_\text{max}\mathbb{1} + \Delta e_1 \leq \hat{u}^b \leq u^b_\text{max}\mathbb{1} - \Delta e_1, \\
    \Delta \mathbb{1} \leq \hat{x}_{1:\horizon} \leq x_\text{max} - \Delta \mathbb{1},
\end{gathered}
\end{equation}
where $e_1 = (1, 0, ..., 0)$ is a unit vector, and
\begin{equation}
\label{eq:ev-robust-bimpc-delta}
    \Delta = (M/2) \sqrt{N} (\Theta^s \Delta y^s_0 + \Theta^l \Delta y^l_0).
\end{equation}
Note that the constraint bounds in \eqref{eq:ev-robust-bimpc-constraints}, as compared to \eqref{eq:ev-bimpc-dynamics}, are reduced by an amount proportional to $\Delta$.
Moreover, if $u^g_\text{max}$ is large enough, the set of constraints \eqref{eq:ev-robust-bimpc-constraints} is feasible if, $\Delta \leq x_0 \leq x_\text{max} - \Delta$ and
\begin{equation}
\label{eq:ev-robust-constraints-feasibility}
    \Delta \leq u^b_\text{max}, \quad 2\Delta \leq x_\text{max}.
\end{equation}
We can find a feasible vector by setting $u^g$ such that $\hat{u}^b = \mathbb{0}$.

The Robust BiMPC problem \eqref{eq:robust-bimpc} is as follows:
% \begin{equation}
% \begin{split}
% \label{eq:ev-bimpc}
%     (u^{g*}, \hat{w}^{s*}, \hat{w}^{l*}) & (\xi_0) = \\
%     \argmin_{u^g, \hat{w}^s, \hat{w}^l} \ & f(u, \hat{w}^s, \hat{w}^l; \xi_0), \\
%     \text{s.t} \quad &
%     \hat{u}^b = u^g - D - (M/2)(\Theta^s \hat{w}^s + \Theta^l \hat{w}^l), \\
%     & \hat{x}_{1:\horizon} = A \hat{u}^b + x_0 \mathbb{1}, \\
%     & \mathbb{0} \leq u^g \leq u^g_\text{max} \mathbb{1}, \\
%     & -u^b_\text{max}\mathbb{1} + \Delta e_1 \leq \hat{u}^b \leq u^b_\text{max}\mathbb{1} - \Delta e_1, \\
%     & \Delta \mathbb{1} \leq \hat{x}_{1:\horizon} \leq x_\text{max} - \Delta \mathbb{1},
% \end{split}
% \end{equation}
\begin{align}
\label{eq:ev-bimpc}
    (u^{g*}, \hat{w}^{s*}, \hat{w}^{l*}) (\xi_0) = \argmin_{u^g, \hat{w}^s, \hat{w}^l} \ & f(u, \hat{w}^s, \hat{w}^l; \xi_0), \\
    \text{s.t} \quad &
    (u^g, \hat{w}^s, \hat{w}^l) \text{ satisfies \eqref{eq:ev-robust-bimpc-constraints}}, \nonumber
\end{align}
where the cost function $f$ is the sum of \eqref{eq:ev-bimpc-generation-cost} and \eqref{eq:ev-bimpc-tracking-cost} with $(w^s, w^l) = (\hat{w}^s, \hat{w}^l)$.
In the hierarchical MPC formulation (see Sec.~\ref{subsec:hierarchical-formulation}), we don't impose convexity assumptions on the BiMPC cost $f$.
Thus, more complex costs can be incorporated into the BiMPC formulation without affecting the solution.

Lastly, we describe the quantities known to the ISO.
The ISO knows the values of $N$, $\Theta^s$, $\Theta^l$, $w^s_\text{max}$, and $w^l_\text{max}$.
Notably, the battery degradation costs are private to the EVs.
Additionally, at each time step, the ISO gets the current normalized SoCs of all EVs and thus knows $\xi_0 = (x_0, \bm{y}_0)$.
These are the same as the assumptions in \cite{zou2017efficient}.
We assume that EVs are truthful when reporting their SoCs. 


\begin{remark} (Differential pricing based on initial SoC)
\label{rem:ev-differential-pricing}
For smaller values of $\Delta$, the feasible set of \eqref{eq:ev-bimpc} is larger.
To reduce the value of $\Delta$, we can reduce $\Delta y^s_0$ and $\Delta y^l_0$ by differentially pricing EVs depending on their initial normalized SoC $y^i_0$.
As an example, the unit price of electricity $\lambda^s$ for a small EV $i$ with initial SoC $y^i_0 \in [0.3, 0.4)$ might be different from that of a small EV $j$ with $y^j_0 \in [0.4, 0.5)$.
Differential pricing allows for tighter control on $\Delta y^s_0, \Delta y^l_0$ for EVs in each interval, and thus the error bound $\Delta$.
In the extreme case, we set different prices for each EV, allowing us to get an error bound of $0$.
In general, we can partition the set of small and large EVs into $P$ sets depending on $y^i_0$ and set electricity prices accordingly (see Rem.~\ref{rem:partition-of-lompcs}).
For clarity of presentation, we don't consider differential pricing in the BiMPC formulation \eqref{eq:ev-bimpc}, but we do so in the implementation\footref{code}.
\end{remark}


\subsection{Simulation Setup}
\label{subsec:ev-simulation-setup}

\begin{figure}%[!htp]
    \centering
    \includegraphics[width=0.99\columnwidth]{figures/aggregate_ev_charging_rate.png}
    \caption{The total electricity consumption by the EVs over $48$ hrs, normalized by $B$ (see \eqref{eq:ev-lompc-b}).
    The solid line shows the actual electricity consumption $w$, while the dotted line shows the predicted electricity consumption $\hat{w}$ (the team-optimal solution from the BiMPC).
    The colored region shows the error bound; the actual consumption $w$ is guaranteed to be within the error bound range around the predicted consumption $\hat{w}$.
    The actual consumption reduces during peak external demand and rises when external demand is low.
    }
    \label{fig:ev-aggregate-charging-rate}
\end{figure}

% \begin{table}[tbp]
% \footnotesize
% \setlength\extrarowheight{1pt}
% \caption{Parameters for EV Charging Example}\label{tab:ev-parameters}
% \begin{tabularx}{\columnwidth}{|c|c|L|}
% \hline
% Parameter & Value & Definition \\
% \hline
% $M$ & $1000$ & EV population size \\
% $(\Theta^s, \Theta^l)$ & $(10, 50)$ & EV battery capacity (in \SI{}{kWh}), Sec.~\ref{subsec:ev-lompc-formulation} \\
% $(w^s_\text{max}, \ $ & $(0.25, \ $ & \multirow{2}{\linewidth}{Maximum normalized charging rate, \eqref{eq:ev-lompc-dynamics}} \\
% $\ w^l_\text{max})$ & $\ 0.15)$ & \\
% $(y^s_\text{max}, y^l_\text{max})$ & $(0.9, 0.9)$ & Maximum normalized SoC, \eqref{eq:ev-lompc-dynamics} \\
% $\horizon$ & $12$ & LoMPC horizon length, \eqref{eq:ev-lompc-dynamics} \\
% $\delta$ & $0.05$ & LoMPC cost parameter, \eqref{eq:ev-lompc-tracking-cost} \\
% $B$ & $30$ & Total EV battery capacity (in \SI{}{MWh}), \eqref{eq:ev-lompc-b} \\
% \hline
% $x_\text{max}/B$ & 0.3 & Normalized storage battery capacity, \eqref{eq:ev-bimpc-dynamics} \\
% $u^g_\text{max}/B$ & 1 & Normalized maximum electricity generation, \eqref{eq:ev-bimpc-dynamics} \\
% $u^b_\text{max}/B$ & 0.3 & Normalized maximum charge/discharge amount, \eqref{eq:ev-bimpc-dynamics} \\
% $c^gB^{1.7}$ & $10^{-3}$ & Normalized BiMPC cost parameter, \eqref{eq:ev-bimpc-generation-cost} \\
% $\gamma$ & $5$ & BiMPC cost parameter, \eqref{eq:ev-bimpc-tracking-cost} \\
% \hline
% $\epsilon_\text{tol}$ & $0.01$ & Incentive solver tolerance, Alg.~\ref{alg:iterative-method} \\
% $\epsilon_k$ & $0.01$ & Regularization parameter, \eqref{subeq:linear-convex-incentives-iterative-method-const-step-flow} \\
% \hline
% $\horizon_r$ & $1$ & Robustness horizon, Rem.~\ref{rem:robustness-horizon} \\
% $P$ & $12$ & Number of partitions for differential pricing, Rem.~\ref{rem:ev-differential-pricing} \\
% \hline
% \end{tabularx}
% \end{table}

% All the parameter values for the EV charging example are tabulated in Tab.~\ref{tab:ev-parameters}.
The external electricity demand vector $D$, shown in Fig.~\ref{fig:ev-demand-electricity-generation}, is obtained for a day in August 2024 in the Midcontinent ISO region of North America and scaled down by a factor of $4000$.
The external demand is at its peak during the afternoon and at its lowest during the early morning.
The simulations are shown for a duration of $48$ hrs, with MPC horizon length $\horizon = 12$ hrs and an EV population size of $1000$.
All the results are normalized with respect to the total EV battery capacity $B$, as defined in \eqref{eq:ev-lompc-b}.

All the convex optimization problems (the robust BiMPC in \eqref{eq:ev-bimpc}, the incentive solver $\lambda$ update step in line \ref{alg-line:lambda-update} Alg.~\ref{alg:iterative-method}, and the EV LoMPCs) are formulated using CVXPY \cite{diamond2016cvxpy} in parametric form and solved using the CLARABEL solver \cite{goulart2024clarabel}.
The simulation parameters and the Python code for all the results can be found in the repository\footref{code}.


\subsection{Results}
\label{subsec:ev-results}

\begin{figure}%[!htp]
    \centering
    \includegraphics[width=0.99\columnwidth]{figures/demand_energy_generation.png}
    \caption{Electricity generation by the ISO over $48$ hrs, normalized by $B$ (see \eqref{eq:ev-lompc-b}).
    The solid line shows the amount of electricity generated $u^g$, while the dotted line shows the external demand.
    The total electricity supply from the ISO always fulfills the total demand.
    The unit price of electricity is set by the ISO so that the total EV electricity consumption fills the overnight demand valley~\cite{ma2013decentralized,zou2017efficient}.
    }
    \label{fig:ev-demand-electricity-generation}
\end{figure}

\begin{figure}%[!htp]
    \centering
    \includegraphics[width = 0.99\linewidth]{figures/storage_battery_state.png}
    \caption{The SoC of the storage battery over $48$ hrs, normalized by $B$ (see \eqref{eq:ev-lompc-b}).
    The solid line shows the actual SoC, while the dotted line shows the predicted SoC (from the BiMPC).
    The colored region shows the error bound; the actual SoC is guaranteed to lie within the error bound range of the predicted SoC.
    We note that the plot only shows the usable SoC range of the storage battery; most batteries have a minimum SoC, which is used as a baseline for the plot.
    The ISO stores the excess electricity generated in a storage battery.
    The storage battery is charged during low external demand and depleted during peak external demand to reduce electricity generation costs.
    Since the total EV electricity consumption is uncertain, the ISO always maintains a nonzero storage battery SoC.
    }
    \label{fig:ev-storage-battery-state}
\end{figure}


The solution trajectories obtained from the simulation are shown in Figs.~\ref{fig:ev-aggregate-charging-rate}, \ref{fig:ev-demand-electricity-generation}, and \ref{fig:ev-storage-battery-state}.
Fig.~\ref{fig:ev-demand-electricity-generation} shows that the ISO sets the unit price of electricity to ensure that electricity generation is evenly spread throughout the day.
Fig.~\ref{fig:ev-aggregate-charging-rate} shows the total electricity consumption by the EV population.
We can observe that EV electricity consumption is close to zero during the afternoon and high during the early morning.
Fig.~\ref{fig:ev-storage-battery-state} shows the state of charge of the storage battery.
The ISO charges the storage battery during low external demand and depletes it during peak external demand.
Moreover, the ISO always maintains a nonzero storage battery SoC to ensure that unexpected EV electricity demands can be satisfied.

During the simulation, a total of $3793$ small EVs and $1715$ large EVs are fully charged, which is $38.7\%$ and $29.2\%$ of the maximum possible value (at the highest charging rate), respectively.
On average, the incentive solver requires $14.7$ iterations and $22.7$ iterations for convergence for small and large EVs, respectively.

Ideally, the total electricity consumption by the EVs is such that there are no spikes in electricity generation.
This type of electricity consumption is called valley-filling~\cite{ma2013decentralized,zou2017efficient}.
From Figs.~\ref{fig:ev-aggregate-charging-rate} and \ref{fig:ev-demand-electricity-generation}, we can see that spikes in electricity generation are avoided by shifting EV consumption loads towards periods of low external demand.
% However, valley-filling consumption can result in high battery degradation costs for the EV population.
In our formulation, we allow EVs to set their optimal charging rates while also ensuring that the goals of the ISO are also met.
% From Fig.~\ref{fig:ev-demand-electricity-generation}, we can see that valley-filling is achieved.
