

The MF-LNO framework demonstrates significant advances in multi-fidelity operator learning through its integration of LNOs with reSGLD. By combining frequency-domain analysis through pole-residue decomposition with adaptive stochastic optimization, the method achieves robust surrogate modeling across diverse dynamical systems while providing calibrated uncertainty estimates. Experimental results on chaotic systems and high-dimensional PDEs show testing error reductions of 40-80\% compared to single-fidelity baselines (see Table \ref{tab:test_loss}), with particularly strong performance in extrapolation tasks (Figs. \ref{fig:lorenz_linear}-\ref{fig:brusselator}).

The three-LNO architecture effectively balances computational efficiency and predictive accuracy through its phased training approach. The initial LF model captures global system behavior with abundant LF data, while subsequent linear and nonlinear correctors adaptively refine predictions using sparse HF data. This structure proves particularly advantageous in scenarios with strong inter-fidelity correlations, as demonstrated by the significant error reduction compared to baselines (see Table \ref{tab:test_loss}). The replica exchange mechanism enhances the exploration of parameter space, with ablation studies indicating optimal performance using 4-6 parallel chains for most benchmark problems.

Key limitations center on three aspects: (1) The current implementation assumes fixed fidelity hierarchies rather than dynamically adjusting fidelity levels during training. Incorporating active learning strategies into this framework could address this limitation to some extent by adaptively selecting HF data that maximizes model improvement. Such an approach would enhance data efficiency by focusing computational resources on the most informative regions of the input space. (2) Replica exchange mechanisms introduce non-negligible overhead for high-dimensional tasks. Exploring hybrid optimization techniques that combine stochastic sampling with deterministic methods could mitigate this issue. (3) Performance shows sensitivity to temperature ladder configurations in reSGLD (Fig. \ref{fig:duffing_lr}). Automated meta-learning approaches could streamline hyper-parameter selection and broaden the framework's applicability. These challenges suggest promising directions for future research, including hybrid optimization strategies combining deterministic and stochastic methods, as well as automated hyper-parameter tuning through meta-learning.

In summary, MF-LNO offers a flexible and efficient solution for MF operator learning with strong potential for real-world applications. Its ability to integrate MF data while quantifying uncertainties makes it a promising tool for scientific digital twins, multi-physics modeling, and experimental design. Future work should focus on enhancing scalability and adaptability through active learning and hybrid optimization techniques to further extend its utility in applied mathematics and computational science.

