\clearpage
\setcounter{page}{1}
\maketitlesupplementary


\section{Additional More Experimental Details}
\label{sec:add_detail}


\noindent {\bf More Implementation Details.} \quad
MSR-VTT~\cite{DBLP:conf/cvpr/XuMYR16} is a large-scale dataset designed for open-domain video captioning, comprising 10,000 video clips across 20 categories. 
Each video clip is annotated with 20 English sentences by Amazon Mechanical Turks. 
The DiDeMo~\cite{DBLP:conf/iccv/HendricksWSSDR17} dataset includes training, validation, and test sets containing 8,395, 1,065, and 1,004 videos, respectively.
During the training state, the loss weight $\alpha$ is set to $0.5$. 

For Visual Question Answering, we first train LLaMA3.2-1B model using the LLaVA training methodology to obtain the LLaVA-LLaMA3.2-1B model. We have restructured the Mamba model by integrating it with the LLaMA3.2-1B architecture, replacing the attention mechanism with a combination of Mamba and CrossMamba modules. In the subsequent fine-tuning stage, we utilized the projector obtained from the pretraining phase of LLaMA3.2-1B. 

\noindent {\bf Details of Comparison Methods.} \quad
We primarily compare with recent state-of-the-art MLLMs, such as BLIP-2 \citep{Li2023BLIP2BL}, InstructBLIP \citep{Dai2023InstructBLIPTG}, Qwen-VL-Chat \citep{bai2023qwen}, LLaVA-1.5-7B \citep{Liu2023ImprovedBW}, LLaVA-Next \citep{li2024llava}.


% \noindent\textbf{Metrics:} 


\section{Additional More Results}
\label{sec:add_results}


% Please add the following required packages to your document preamble:
% \usepackage{multirow}
\begin{table}[]
\centering
\scalebox{0.9}
{
\begin{tabular}{c|ll}
\toprule
\multirow{2}{*}{Model} & \multicolumn{2}{c}{ImageNet-Subset} \\
                       & \multicolumn{1}{c|}{Acc1}   & Acc5  \\ \midrule

{\color[HTML]{808080}RWKV-T}                 &   {\color[HTML]{808080}83.14}                          &  {\color[HTML]{808080}95.32}     \\
{\color[HTML]{808080}RWKV-S}                 &     {\color[HTML]{808080}84.54}                        &   {\color[HTML]{808080}95.70}    \\
TransRWKV-T            &      84.24 {\color[HTML]{FE0000} ($\uparrow$1.10)}                      &   96.30 {\color[HTML]{FE0000} ($\uparrow$0.98)}   \\
TransRWKV-S            &      85.18 {\color[HTML]{FE0000} ($\uparrow$0.64)}                      &   96.54 {\color[HTML]{FE0000} ($\uparrow$0.84)}   \\ \midrule
\end{tabular}}
\caption{
Ablation studies with different architectures.
}
\label{tab:aba_archi}
\end{table}

% Please add the following required packages to your document preamble:
% \usepackage{multirow}
\begin{table}[]
\centering
\scalebox{0.9}
{
\begin{tabular}{c|cc}
\midrule
\multirow{2}{*}{Model} & \multicolumn{2}{c}{ImageNet-Subset} \\
                       & \multicolumn{1}{c|}{Acc1}   & Acc5  \\ \midrule
teacher-layer4          & 84.26                            &     96.90  \\
teacher-layer8                 &   83.78                          &   96.36    \\
teacher-lastlayer                 &   87.10                          &  97.22     \\
\midrule
\end{tabular}}
\caption{
Ablation studies with different layer of the teacher model as supervisory signal.
}
\label{tab:aba_one}
\end{table} 

% Please add the following required packages to your document preamble:
% \usepackage{multirow}
\begin{table}[]
\centering
\scalebox{0.9}
{
\begin{tabular}{l|c}
\toprule
\multicolumn{1}{c|}{Configuration} & Hyperparameter                         \\ \midrule
LLM init                           & Mamba-0.6B              \\
VL Adaptor init                    & MLP                     \\
ViT init                           & CLIP-ViT-L/14           \\
LLM sequence length                & 2048                    \\
Optimizer                          & AdamW                   \\
Optimizer hyperparameter           & $\beta_1$=0.9, $\beta_2$=0.98, $eps$=$1e^{-6}$ \\
Learning rate                      & $2e^{-5}$                    \\
Learning rate schedule             & Cosine decay            \\
Weight decay                       & 0.0                     \\
Training epoch                     & 1                       \\
Warm-up ratio                      & 0.03                    \\
Global batch size                  & 128                     \\
Numerical precision                & FP16                    \\
Model parallelism                  & Zero2 offload           \\ \midrule
\end{tabular}}
\caption{
Training hyperparameters of visual question answering task.
}
\label{tab:aba_hyper}
\end{table}
\noindent\textbf{Generalization:} 
We also applied adaptive distillation to RWKV (without the bidirectional distillation). From the Table \ref{tab:aba_archi}, it can be seen that adaptive distillation is effective in the new RWKV framework. In the Tiny and Small scales, TransRWKV outperforms RWKV, achieving state-of-the-art (SOTA) results.

\begin{figure}[t]
  \centering
  % \fbox{\rule{0pt}{2in} \rule{0.9\linewidth}{0pt}}
   \includegraphics[width=0.9\linewidth]{figs/results_acc_imnt_tiny_ws.pdf}

   \caption{TransPMamba-T/accuracy.}
   \label{fig:aba_ws}
\end{figure}


%%%% 感觉放不开了,其余的放到附录里面
\noindent\textbf{The Influence of Different Teacher Layer:}
In Table~\ref{tab:aba_one}, different layers of the teacher model have a significant impact on the distillation outcomes. Distilling information from shallower layers can even result in performance that is inferior to that of the pure model. This highlights the importance of selecting appropriate layers for knowledge transfer in the distillation process.




\noindent\textbf{The Influence of Weight Subcloning:}
In Table~\ref{fig:aba_ws}, weight subcloning accelerates the convergence speed of the model significantly during the early stages of training. This observation underscores the effectiveness of weight subcloning in expediting the training process, particularly in the initial phases.

% 
% Please add the following required packages to your document preamble:
% \usepackage{multirow}
\begin{table}[]
\centering
\scalebox{0.9}
{
\begin{tabular}{c|cc}
\midrule
\multirow{2}{*}{Model} & \multicolumn{2}{c}{ImageNet-Subset} \\
                       & \multicolumn{1}{c|}{Acc1}   & Acc5  \\ \midrule
KD          &                             &       \\
OFA-KD                 &                             &       \\
TransMamba                 &                             &       \\
\midrule
\end{tabular}}
\caption{
Ablation studies with different methods.
}
\label{tab:aba_teacher}
\end{table} displays different teacher model.

% % 
% Please add the following required packages to your document preamble:
% \usepackage{multirow}
\begin{table}[]
\centering
\scalebox{0.9}
{
\begin{tabular}{c|cc}
\midrule
\multirow{2}{*}{Data Size} & \multicolumn{2}{c}{ImageNet-Subset} \\
                       & \multicolumn{1}{c|}{Acc1}   & Acc5  \\ \midrule
 25\%         &    83.20                         & 96.38      \\
 50\%                &  87.02                           &   97.74    \\
75\%                 &  87.10                           &  97.22     \\
100\%     &  86.96                           &  97.26     \\
\midrule
\end{tabular}}
\caption{
Ablation studies with different datasize (TransPMamba).
}
\label{tab:aba_datasize}
\end{table} displays the influences of different data size.



% 
% Please add the following required packages to your document preamble:
% \usepackage{multirow}
\begin{table}[]
\centering
\scalebox{0.9}
{
\begin{tabular}{c|cc}
\midrule
\multirow{2}{*}{Model} & \multicolumn{2}{c}{ImageNet-Subset} \\
                       & \multicolumn{1}{c|}{Acc1}   & Acc5  \\ \midrule
teacher-layer4          & 84.26                            &     96.90  \\
teacher-layer8                 &   83.78                          &   96.36    \\
teacher-lastlayer                 &   87.10                          &  97.22     \\
\midrule
\end{tabular}}
\caption{
Ablation studies with different layer of the teacher model as supervisory signal.
}
\label{tab:aba_one}
\end{table} displays the influences of different teacher's feature position.








\noindent \textbf{Limitations:}
% limitation rehearsal-free is not verified.
The proposed method is built on the rehearsal strategy. However, the rehearsal-free setting on the model performance is not evaluated.


% \section{Rationale}
% \label{sec:rationale}
% % 
% Having the supplementary compiled together with the main paper means that:
% % 
% \begin{itemize}
% \item The supplementary can back-reference sections of the main paper, for example, we can refer to \cref{sec:intro};
% \item The main paper can forward reference sub-sections within the supplementary explicitly (e.g. referring to a particular experiment); 
% \item When submitted to arXiv, the supplementary will already included at the end of the paper.
% \end{itemize}
% % 
% To split the supplementary pages from the main paper, you can use \href{https://support.apple.com/en-ca/guide/preview/prvw11793/mac#:~:text=Delete%20a%20page%20from%20a,or%20choose%20Edit%20%3E%20Delete).}{Preview (on macOS)}, \href{https://www.adobe.com/acrobat/how-to/delete-pages-from-pdf.html#:~:text=Choose%20%E2%80%9CTools%E2%80%9D%20%3E%20%E2%80%9COrganize,or%20pages%20from%20the%20file.}{Adobe Acrobat} (on all OSs), as well as \href{https://superuser.com/questions/517986/is-it-possible-to-delete-some-pages-of-a-pdf-document}{command line tools}.