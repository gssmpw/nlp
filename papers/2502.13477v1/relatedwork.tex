\section{Literature Review}
Some studies have been done that explored the different initialization methods for metaheuristic algorithms, including quasi-random initialization, chaotic systems, anti-symmetric learning methods, and Latin hypercube sampling \cite{li2020influence}. Despite the methods being able to enhance the performance of metaheuristic algorithms such as Particle Swarm Optimization (PSO) and Genetic Algorithms (GA), they have significant drawbacks. Quasi-random initialization struggles with the curse of dimensionality \cite{maaranen2004quasi}. Chaos-based approaches produce random sequences using a few chaotic maps and parameters, but they are highly sensitive to initial conditions in certain situations \cite{dos2008use}. The anti-symmetric learning method demands twice the population size to select solutions for the next generation, thereby doubling the computational cost. Latin hypercube sampling is effective in low-dimensional spaces but can perform poorly in higher-dimensional ones.

Meng et al. \cite{meng2019multi} proposed a population strategy based on constraint transformation and the individual constraints and group constraints technique (ICGC) to improve the population initialization for increased search efficiency. This was proposed to overcome the shortcoming of the multi-objective cuckoo search (MOCS), which was developed by Yang \& Deb \cite{yang2013multiobjective} to the multi-objective optimization problem. The improved algorithm was applied to the multi-objective hydropower station optimal operation (MOHSOO) model of Xialongdi and Xixiayuan cascade hydropower stations in the Yellow River. The results show that the population strategy can enhance search efficiency by constraining the initial solution within a defined range, thereby reducing the search space and enhancing the quality of the initial feasible solution. However, the study mentioned that population initialization strategy using ICGC may be more suitable for short-term hydropower station operation, as monthly water level variations are significant and may not fully utilize ICGC's potential. Thus, exploring more effective improvement strategies and applying them to more complex models is crucial for better performance.

According to Valian et al. \cite{valian2011improved} from their study, “Improved Cuckoo Search Algorithm for Global Optimization”, a key drawback of using fixed values is that it affects the number of iterations required to find an optimal solution, which may lead to decreasing the efficiency of the algorithm. To enhance the performance and convergence rate of the CS algorithm and address the limitations associated with fixed parameter values, an adaptive method for the parameters was proposed. The enhanced CS algorithm was called Improved Cuckoo Search (ICS) algorithm in the study. The primary distinction between the ICS and CS algorithms lies in the way $P_a$ and $a$ are adjusted. The values of $P_a$ and $a$ are dynamically changed with the numbers of generations, wherein the values of $P_a$ and $a$ are large to enhance the diversity of solution vectors and are gradually decreased in the final generations for better fine-tuning of solution vectors. According to the simulation results, the ICS algorithm demonstrated its superiority over the standard CS algorithm in terms of both accuracy and convergence rate in several benchmark problems. The study showed that for varying population sizes (N = 10, 30, and 50) and iteration counts (1000, 3000, and 5000), dynamically adjusting $P_a$ and $a$ had a significant impact on the algorithm's performance. The results indicated that reducing the minimum value of $P_a$ without altering $a$ led to better results in most cases, especially for test functions with high decision variables, while increasing the maximum value of $P_a$ also enhanced performance. Although reducing the minimum value of $\alpha$ had little impact, increasing $\alpha$ excessively could degrade performance. Based on the tests, optimal values for the parameters were identified as approximately  
$P_{a\text{min}} = 0.005, \quad P_{a\text{max}} = 1, \quad a_{\text{min}} = 0.05, \quad a_{\text{max}} = 0.5.$


A study entitled “Improved cuckoo search algorithm and its application to permutation flow shop scheduling problem” by Zhang et al. \cite{zhang2020improved}, developed a self-adaptive step length CS algorithm based on a dynamic balance factor, referred to as the dynamic cuckoo search (DCS) algorithm. According to Zhang et al. \cite{zhang2020improved}, in the standard CS algorithm, while a smaller step size improves local search accuracy, it slows down convergence and increases the risk of getting trapped in local optima. Conversely, a larger step size enhances global search and accelerates convergence but may cause the algorithm to skip the optimal value, resulting in oscillation around it or even population loss. The randomness of the Levy flight does not guarantee that the search process converges quickly and steadily towards the optimal solution. Zhang et al. \cite{zhang2020improved}, mentioned that a better approach is to integrate both global and local search capabilities, striking a balance between fast global exploration and precise local optimization to improve overall algorithm performance. The DCS algorithm introduces two key parameters, the iteration number ratio parameter and the adaptability ratio parameter, which are controlled by a dynamic balance factor ($\vartheta$). The dynamic balance factor ($\vartheta$) is introduced and used to adjust the weight number of iteration number ratio and adaptability ratio. The DCS algorithm adjusts these parameters throughout the optimization process to achieve a more flexible and adaptive search, closely mimicking natural evolution. Additionally, the self-adaptive step is realized through the dynamic adjustment of the parameter $\beta$ in the standard CS algorithm, enhancing the algorithm’s ability to explore and exploit the search space. The effectiveness of the DCS algorithm was verified using six standard test functions (Ackley, Griewank, Rastrigin, Rosenbrock, Schwefel, and Sphere) and demonstrated superior performance in solving the permutation flow shop scheduling problem (PFSP), particularly when tested against eight operators Carl-Car8 of Car benchmark class.

A multi-strategy adaptive cuckoo search algorithm was proposed to further enhance a self-adaptive variant of the cuckoo search algorithm which aims to balance the exploration (global search) and exploitation (local search) capability of the algorithm as well as address the limitations of the variant. In this study, 5 different search strategies were proposed which are fairly controlled by a certain probability. The different search strategies used by the algorithm comprises of the following: (1) Maintaining the original global and local search of the cs algorithm, (2 \& 3) Selection of 2 or 4 individuals for location information sharing, (4) A search strategy with a learning matrix to record solutions, and (5) Increased randomness when searching and made the whole population jump out of the local optimal region by the nonlinear inertia coefficient. The multi-strategy adaptive cuckoo search algorithm yielded positive results by achieving the best result on 17 out of 28 optimization benchmark functions \cite{gao2021adaptive}.

An experimental study was conducted by Salgotra et al. \cite{salgotra2018new} wherein multiple variants of the cuckoo search algorithm that enhances both global and local search were proposed. The three versions of the algorithm are as follows: (1) in version 1, the exploration enhancement is based on the search equations of the Grey Wolf Optimizer while the exploitation enhancement utilized two different searching strategies to evaluate the final solution, (2) in version 2, the exploration and exploitation enhancement focused on two-fold division in population using a pair of equation on the divided populations to generate new solutions, (3) lastly in version 3, a Cauchy based mutation operator was added to the explorative capabilities of CS and a four-fold division of population in the global search space was also added to balance exploration and exploitation using different search equations per part. Out of the three proposed versions, the Cuckoo search version 1 outperformed state-of-the-art algorithms in the experiments which used Cauchy based exploration, and better search of the search space.