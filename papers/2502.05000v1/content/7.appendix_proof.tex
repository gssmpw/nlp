\renewcommand{\theequation}{\thesection.\arabic{equation}}
\renewcommand{\thefigure}{\thesection.\arabic{figure}}
\renewcommand{\thetable}{\thesection.\arabic{table}}
\setcounter{equation}{0}
\setcounter{figure}{0}
\setcounter{table}{0}

\section{Proof and Derivation}
\setcounter{table}{0}
\setcounter{figure}{0}
\setcounter{equation}{0}
% % \setcounter{proposition}{1}
% \renewcommand{\thetable}{\ref*{app:proof}.\arabic{table}}
% \renewcommand{\thefigure}{\ref*{app:proof}.\arabic{figure}}
% \renewcommand{\theequation}{\ref*{app:proof}.\arabic{equation}}
\subsection{Proof of Proposition \ref{equivalence}}
\label{appendix:proof}
We first restate Propostition~\ref{equivalence}.

\begin{prop}
    For each edge at time $t$, the adjacency matrix is updated as $\mathbf{A}^{(t)}_{ij} =\mathbf{A}_{ij}\big(\bar{\mathbf{Q}}^{\prime(t)}_{\mathbf{A}}\big)_{ij}$, where the non-isotropic transition matrix is  $\big(\bar{\mathbf{Q}}_{\mathbf{A}}^{\prime(t)}\big)_{ij} = \bar{\alpha}^{(t)} \mathbf{I} + (\boldsymbol{\Lambda}_{\mathbf{A}})_{ij}(1-\bar{\alpha})\mathbf{1} \mathbf{m}_{\mathbf{A}}^{T}$. There exists a unique time $\hat{t}\big(\mathbf{A}_{ij}\big)\in [0, T]$ such that $\big(\bar{\mathbf{Q}^\prime}_{\mathbf{A}}^{(t)}\big)_{ij}\Leftrightarrow \big(\bar{\mathbf{Q}}_{\mathbf{A}}^{\hat{t}(\mathbf{A}_{ij})}\big)_{ij}$, where:
    \begin{equation}
        \hat{t}\big(\mathbf{A}_{ij}\big)\!=\! T\!\left(\frac{2(1\!+\!s)}{\pi} \cos^{-1}\! \left(\sqrt{\frac{\bar{\alpha}^{(t)}}{\big[\boldsymbol{\Lambda}(\mathbf{A})_{ij} (1-\bar{\alpha}^{(t)}) + \bar{\alpha}^{(t)}\big]}}\right)\!-\!s\right).\notag
    \end{equation}
\end{prop}
% Begin proof
% \textit{Proof.}
\begin{proof}
$\bar{\mathbf{Q}}^{(t)}_{\mathbf{A}} = \bar{\alpha}^{(t)}\mathbf{I} + \big(1-\bar{\alpha}^{(t)}\big)\mathbf{1} \mathbf{m}_{\mathbf{A}}^{\top}$ indicates the degree of noise added to the adjacency matrix $\mathbf{A}$ at time step $t$. Let $\text{SNR}_{\bar{\mathbf{Q}}_{\mathbf{A}}}(t)$ denotes the signal-to-noise of $\bar{\mathbf{Q}}_{\mathbf{A}}^{(t)}$ at time step $t$, we have:
\begin{equation}
    \text{SNR}_{\bar{\mathbf{Q}}_{\mathbf{A}}}(t) = \frac{1-\bar{\alpha}^{(t)}}{\bar{\alpha}^{(t)}}.
\end{equation}
Such that:
\begin{align}
    &\big(\bar{\mathbf{Q}}_{\mathbf{A}}^{\prime(t)}\big)_{ij}\Leftrightarrow \big(\bar{\mathbf{Q}}_{\mathbf{A}}^{\hat{t}(\mathbf{A}_{ij})}\big)_{ij} \\
    \Rightarrow & \text{SNR}_{\bar{\mathbf{Q}}^{\prime}_{\mathbf{A}}}(t) = \text{SNR}_{\bar{\mathbf{Q}}_{\mathbf{A}}}\big(\hat{t}(\mathbf{A}_{ij})\big) \\
    \Rightarrow & \frac{(\mathbf{\Lambda}_{\mathbf{A}})_{ij}\big(1-\bar{\alpha}^{(t)}\big)}{\bar{\alpha}^{(t)}} = \frac{1-\bar{\alpha}^{(t^{\prime})}}{\bar{\alpha}^{(t^{\prime})}}.
\end{align}

We first prove that for each time step $t$, there exists and only exists one $t^{\prime}$ that satisfies $\big(\bar{\mathbf{Q}}_{\mathbf{A}}^{\prime(t)}\big)_{ij}\Leftrightarrow \big(\bar{\mathbf{Q}}_{\mathbf{A}}^{\hat{t}(\mathbf{A}_{ij})}\big)_{ij}$. 
Left $g(t^{\prime})=\frac{(\mathbf{\Lambda}_{\mathbf{A}})_{ij}(1-\bar{\alpha}^{(t)})}{\bar{\alpha}^{(t)}} -\frac{1-\bar{\alpha}^{(t^{\prime})}}{\bar{\alpha}^{(t^{\prime})}}$ represents the function of $t^{\prime}\in[0,T]$. $\bar{\alpha}^{(t)}=\cos^2\big(\frac{t/T+s}{1+s}\cdot \frac{\pi}{2}\big)$ is the scheduler with a small constant $s$. We have $\alpha^{(0)}=\cos^2\big(\frac{0+s}{1+s}\cdot\frac{\pi}{2}\big) \approx \cos^{2}(0)=0$, and $\alpha^{(T)}=\cos^2(\frac{1 + s}{1+s}\cdot\frac{\pi}{2}) = \cos^2(\frac{\pi}{2})=1$. It is known that $(1-\bar{\alpha})$ monotonically decreasing over the domain, while $\bar{\alpha}$ monotonically increasing, with $1-\bar{\alpha}>0$ and $\bar{\alpha}>0$. Therefore, $g(t^{\prime})$ is a monotonic function over the domain.
So we achieve:
\begin{equation}
    g(0)=\frac{(\mathbf{\Lambda}_{\mathbf{A}})_{ij}(1-\bar{\alpha}^{(t)})}{\bar{\alpha}^{(t)}} - 0 > 0 .
\end{equation}
Having $\mathbf{\Lambda}(\mathbf{A})_{ij} \in [0, 1]$ indicates the node adversarial score, we can then derive the following:
\begin{align}
     g(T)&=\frac{(\mathbf{\Lambda}_{\mathbf{A}})_{ij}(1-\bar{\alpha}^{(t)})}{\bar{\alpha}^{(t)}} - 1 \\ 
    &< \frac{(1-\bar{\alpha}^{(t)})}{\bar{\alpha}^{(t)}} -1 \\ 
    &< 0.
\end{align}
Thus, we have $g(0)g(T)<0$, and since $g(t^{\prime})$ is a monotonically decreasing function, the intermediate value theorem guarantees that there exists exactly one $t^{\prime}_{0}\in[0, T]$ satisfies $g(t^{\prime}_{0})=0$.
By setting $g(t^{\prime})=0$, we obtain:
\begin{align}
    & \mathbf{\Lambda}(\mathbf{A})_{ij}\bar{\alpha}^{(t^{\prime})}\big(1-\bar{\alpha}^{(t)}\big)=\bar{\alpha}^{(t)}\big(1-\bar{\alpha}^{(t^{\prime})}\big) \\
     \Rightarrow &
    \bar{\alpha}^{(t^{\prime})} \big[\mathbf{\Lambda}(\mathbf{A})_{ij} (1-\bar{\alpha}^{(t)}) + \bar{\alpha}^{(t)}\big] = \bar{\alpha}^{(t)} \\
    \Rightarrow &
    \bar{\alpha}^{(t^{\prime})} = \frac{\bar{\alpha}^{(t)}}{\big[\mathbf{\Lambda}(\mathbf{A})_{ij} (1-\bar{\alpha}^{(t)}) + \bar{\alpha}^{(t)}\big]} \\
    \Rightarrow &
    t^{\prime} = T\left(\frac{2(1+s)}{\pi} \cos^{-1} \left(\sqrt{\frac{\bar{\alpha}^{(t)}}{\big[\mathbf{\Lambda}(\mathbf{A})_{ij} (1-\bar{\alpha}^{(t)}) + \bar{\alpha}^{(t)}\big]}}\right) -s\right).
\end{align}

This concludes the proof of the proposition.
\end{proof}



