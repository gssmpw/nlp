\documentclass[a4paper]{article} % 一般的なスタイルの書き方
\usepackage[margin=1in]{geometry}
\renewcommand{\baselinestretch}{1.3}


\usepackage{microtype}
\usepackage{graphicx}
% \usepackage{subfigure}
\usepackage{subcaption}
\usepackage{booktabs} % for professional tables
\usepackage{here}
% \usepackage{authblk}

% \usepackage{CJKutf8}
% \usepackage[whole]{bxcjkjatype}

% Recommended, but optional, packages for figures and better typesetting:

\usepackage{hyperref}

% Attempt to make hyperref and algorithmic work together better:
\newcommand{\theHalgorithm}{\arabic{algorithm}}

%%%%%%%%%%%%%%%%%%%%%%%%%%%%%%%%%%%%%%%%%%%%%%%%%%%%%%%%%%%%%%%%%%%%%%%%%%%%%%%%%%%%%%%%%%%%%%
\usepackage{amsmath,amsthm,amsfonts,amssymb}
\usepackage{mathtools}
\usepackage{xcolor}
\usepackage{natbib}
\usepackage{algorithm} % for algorithm
\usepackage{algpseudocode} % for algorithm
\usepackage{url}
\usepackage{multirow}
% \usepackage{hyperref} 

% \usepackage{subcaption}
% \usepackage{booktabs} % for professional tables

\usepackage[T1]{fontenc}
\usepackage[utf8]{inputenc}
\usepackage{bbm}
% \usepackage[whole]{bxcjkjatype}

\newcommand{\red}[1]{\textcolor{red}{#1}}
\newcommand{\blue}[1]{\textcolor{blue}{#1}}
\newcommand{\magenta}[1]{\textcolor{magenta}{#1}}

\newcommand{\mybecause}[1]{&& \bigl(\because #1 \bigr)}
\newcommand{\indep}{\mathop{\perp\!\!\!\!\perp}}
\newcommand{\disteq}{\overset{\mathrm{d}}{=}}

\newcommand{\thought}[1]{{\color[rgb]{0.2,0.39,0.66}(#1)}}
\newcommand{\todo}[1]{{\color[rgb]{1.0,0.0,0.0}(#1)}}
\newcommand{\hsh}[1]{{\color{green!50!black} Henrik: #1}}
\newcommand{\st}[1]{{\color{red!50!black} Sebastian: #1}}

\newcommand{\ulm}[1]{_{\scaleto{\mathrm{#1}}{3pt}}}
\newcommand\at[2]{\left.#1\right|_{#2}}











\newtheorem{assumption}{Assumption}

\DeclareMathOperator*{\argmax}{arg\,max}
\DeclareMathOperator*{\argmin}{arg\,min}

\newcommand{\swname}[1]{\texttt{#1}}
\newcommand{\ie}{i\/.\/e\/.,\/~}
\newcommand{\eg}{e\/.\/g\/.,\/~}
\newcommand{\cf}{cf\/.\/~}

\newcommand{\fig}{Fig\/.\/~}
\newcommand{\defn}{Def\/.\/~}
\newcommand{\sect}{Sec\/.\/~}
\newcommand{\tabl}{Tab\/.\/~}
\newcommand{\algo}{Algorithm~}
\newcommand{\theo}{Theorem~}

\newcommand{\bnnl}{3 hidden layers}
\newcommand{\bnnn}{50 neurons}
\newcommand{\bnna}{tanh activations}

\newcommand{\capt}[1]{\mdseries{\emph{#1}}}

\newcommand{\videolink}{at \url{https://youtu.be/_d7AqTRjz6g}}
\newcommand{\codelink}{\url{https://github.com/wheelbot/mini-wheelbot}}

\newcommand{\fakepar}[1]{\vspace{0mm}\noindent\textbf{#1.}}

\newcommand{\needref}{\textcolor{red}{[REF]}}

\newcommand{\plotfontsize}{9pt}


% show equation number only if referenced
\mathtoolsset{showonlyrefs}
% \renewcommand{\arraystretch}{1.2}
%%%%%%%%%%%%%%%%%%%%%%%%%%%%%%%%%%%%%%%%%%%%%%%%%%%%%%%%%%%%%%%%%%%%%%%%%%%%%%%%%%%%%%%%%%%%%%
\usepackage{authblk}

\title{Distributionally Robust Active Learning \\for Gaussian Process Regression}
\date{}

\author[1,2]{Shion Takeno}

\author[1]{Yoshito Okura}
\author[3]{Yu Inatsu}
\author[1]{Aoyama Tatsuya}
\author[1]{Tomonari Tanaka}
\author[1]{Akahane Satoshi}
\author[2]{Hiroyuki Hanada}
\author[2]{Noriaki Hashimoto}
\author[4]{Taro Murayama}
\author[4]{Hanju Lee}
\author[4]{Shinya Kojima}
\author[1,2]{Ichiro Takeuchi}

\affil[1]{Nagoya University}
\affil[2]{RIKEN AIP}
\affil[3]{Nagoya Institute of Technology}
\affil[4]{DENSO CORPORATION}

\affil[ ]{\texttt{{takeno.s.mllab.nit@gmail.com}}}
\affil[ ]{\texttt{{takeuchi.ichiro.n6@f.mail.nagoya-u.ac.jp}}}

\begin{document}
\maketitle


\begin{abstract}
     Gaussian process regression (GPR) or kernel ridge regression is a widely used and powerful tool for nonlinear prediction.
     %
     Therefore, active learning (AL) for GPR, which actively collects data labels to achieve an accurate prediction with fewer data labels, is an important problem.
     %
     However, existing AL methods do not theoretically guarantee prediction accuracy for target distribution.
     %
     Furthermore, as discussed in the distributionally robust learning literature, specifying the target distribution is often difficult.
     %
     Thus, this paper proposes two AL methods that effectively reduce the worst-case expected error for GPR, which is the worst-case expectation in target distribution candidates.
     %
     We show an upper bound of the worst-case expected squared error, which suggests that the error will be arbitrarily small by a finite number of data labels under mild conditions.
     %
     Finally, we demonstrate the effectiveness of the proposed methods through synthetic and real-world datasets.
\end{abstract}

% Gaussian process regression (GPR) or kernel ridge regression is a widely used and powerful tool for nonlinear prediction. Therefore, active learning (AL) for GPR, which actively collects data labels to achieve an accurate prediction with fewer data labels, is an important problem. However, existing AL methods do not theoretically guarantee prediction accuracy for target distribution. Furthermore, as discussed in the distributionally robust learning literature, specifying the target distribution is often difficult. Thus, this paper proposes two AL methods that effectively reduce the worst-case expected error for GPR, which is the worst-case expectation in target distribution candidates. We show an upper bound of the worst-case expected squared error, which suggests that the error will be arbitrarily small by a finite number of data labels under mild conditions. Finally, we demonstrate the effectiveness of the proposed methods through synthetic and real-world datasets.


% Gaussian process regression (GPR) or kernel ridge regression is a widely used and powerful tool for nonlinear prediction. Therefore, active learning for GPR, which actively collects data labels to achieve an accurate prediction with fewer data labels, is an important problem. However, existing approaches do not theoretically guarantee prediction accuracy for target distribution. Furthermore, as discussed in the distributionally robust learning literature, specifying the target distribution is often difficult. Thus, this paper proposes two active learning methods that effectively reduce the worst-case target prediction error for GPR, where the worst-case is taken with the target distribution candidates. We show an information-theoretic upper bound of the worst-case expected squared error, which suggests that under mild conditions, the error will be arbitrarily small within a finite number of data labels. Finally, we demonstrate the effectiveness of the proposed methods through synthetic and real-world datasets.

% The first version
% Active learning is a framework for actively collecting data labels to achieve an accurate prediction with fewer data labels under some statistical model. However, many existing approaches do not theoretically guarantee prediction accuracy for target distribution. In addition, as discussed in the literature on distributionally robust learning, specifying the target distribution is often difficult. This paper proposes two active learning methods that effectively reduce the worst-case target prediction error when we use the Gaussian process regression, a widely used kernel regression model. We show an information-theoretic upper bound for the worst-case expected squared error, where the worst-case is taken with the target distribution candidates. This upper bound shows that the worst-case prediction error will be arbitrarily small with a finite number of data labels under mild conditions. Finally, we demonstrate the effectiveness of the proposed methods through synthetic and real-world datasets.

%%%%%%%%%%%%%%%%%%%%%%%%%%%%%%%%%%%%%%%%%%%%%%%%%%%%%%%%%%%%%%%%%%%%%%%%%%%%%%%%%%%%%%%%

\section{Introduction}
\label{sec:intro}
% Image editing methods in diffusion models depend on user-defined control directions - users can unlock their creativity using these methods by specifying the desired manipulation through prompts~\cite{gandikota2023concept}, reference images~\cite{ruiz2022dreambooth, kumari2022customdiffusion, gal2022image, chen2024trainingfreeregionalpromptingdiffusion}, or attribute vectors~\cite{parmar2023zero,hertz2022prompt}. In this work, we ask a fundamentally different question: \emph{Can we automatically discover the underlying visual structure of a concept within diffusion model's knowledge?} %Rather than requiring user-specified controls, we aim to decompose the model's internal knowledge into meaningful directions.

% This question touches on a fundamental limitation in how we interact with diffusion models. Current control methods ~\cite{zhang2023addingconditionalcontroltexttoimage, gandikota2023concept, ye2023ipadaptertextcompatibleimage,ye2023ipadaptertextcompatibleimage, hertz2024stylealignedimagegeneration, li2023photomaker, shi2024instantbooth, chen2024trainingfreeregionalpromptingdiffusion} require users to specify their desired manipulations in advance, limiting interactive creativity. This contrasts with natural human artistic workflows, where creators dynamically explore creative ideas while jointly refining them toward meaningful artistic outcomes~\cite{hoffmann2016modeling}. This synergy between specification and exploration is not new to generative models. Early GAN architectures naturally developed disentangled latent spaces that enabled continuous\cite{harkonen2020ganspace,radford2015unsupervised, wu2021stylespace, shen2020interfacegan}, compositional control over generated images. Users could explore these spaces to discover interesting variations that would be difficult to describe in words~\cite{wu2021stylespace}, then combine them to achieve their creative goals~\cite{grabe2022towards}. 


% While diffusion models have largely superseded GANs in conditional image synthesis~\cite{dhariwal2021diffusion},  their underlying structure remains less understood. Diffusion models achieve remarkable diversity through high-dimensional latents, unlike GANs' compact latent spaces.  With a single prompt, diffusion models can generate radically different variations through different random initializations of input noise. We ask - Is it possible to discover interpretable structure within this vast space of variations?

Text-to-image diffusion models are capable of generating remarkable visual variations from a single prompt through different random initializations. However, this vast creative potential remains largely opaque to users---while we can generate diverse images, we lack understanding of the underlying structure of these variations. This presents a fundamental challenge: how can we discover and expose the latent visual capabilities encoded within these models?

\let\thefootnote\relax \footnote{$^{*}$Correspondence to \texttt{gandikota.ro@northeastern.edu}}

The challenge touches on a key limitation in how we interact with diffusion models today. Current control methods require users to explicitly specify their desired edits in advance through prompts~\cite{gandikota2023concept}, reference images~\cite{zhang2023addingconditionalcontroltexttoimage, chen2024trainingfreeregionalpromptingdiffusion, ruiz2022dreambooth,kumari2022customdiffusion, Ryu_lora, hu2021lora}, or attribute vectors~\cite{ye2023ipadaptertextcompatibleimage, hertz2024stylealignedimagegeneration, li2023photomaker, shi2024instantbooth,parmar2023zero,hertz2022prompt}. That contrasts sharply with natural human creative workflows, where artists dynamically explore creative ideas and jointly refine them toward meaningful artistic outcomes~\cite{hoffmann2016modeling}. The need for pre-specified controls creates a barrier between users and the full creative potential of these models.

Interestingly, earlier generative models like GANs~\cite{gans,karras2019style,brock2018large} naturally developed more interpretable internal structures. Their compact latent spaces often exhibited emergent disentanglement~\cite{harkonen2020ganspace,radford2015unsupervised, wu2021stylespace, shen2020interfacegan}, enabling continuous and compositional control over generated images. Users could explore these spaces to discover interesting variations that would be difficult to describe in words~\cite{wu2021stylespace}, then combine them to achieve their creative goals~\cite{grabe2022towards}.

Diffusion models have largely superseded GANs in conditional image synthesis~\cite{dhariwal2021diffusion}, achieving greater diversity through much higher-dimensional latents. And yet an understanding of the underlying structure of these larger latent spaces has remained elusive. In this work, we ask a fundamental question: \emph{Can we automatically discover the visual structure within a diffusion model's knowledge of a concept?} Rather than requiring user-specified controls, we aim to decompose the model's internal representations into expressive directions that users can explore and combine.

To address these needs, we present \textbf{SliderSpace}, a framework that brings systematic explorability to diffusion models. Given just a text prompt, SliderSpace discovers a canonical set of meaningful, diverse, and controllable directions within the model's knowledge of that concept. Each direction is implemented as a low-rank adapter~\cite{hu2021lora} that can be scaled and composed with others, allowing users to explore and smoothly combine different aspects of variation, as shown in Figure~\ref{fig:intro}.

We ground SliderSpace discovery in three key requirements for meaningful decomposition of a diffusion model's visual manifold: 
\begin{enumerate}
    \item \textbf{Unsupervised Discovery:} The decomposition process should emerge from the intrinsic structure of the model's learned representation, rather than being guided by predefined attributes. This ensures we capture the true topology of the model's knowledge space rather than projecting our assumptions onto it.
    
    \item \textbf{Semantic Orthogonality:} Each discovered control must represent a distinct semantic direction. This is enforced in a semantic feature space, like CLIP, where every slider has an orthogonal effect in embeddings. This prevents discovering multiple controls that create similar semantic effects, making the system more efficient and easier.
    
    \item \textbf{Distribution Consistency:} Directions must induce consistent transformations across both random seeds and prompt variations. 
\end{enumerate}

These requirements naturally lead to our proposed framework, which we formalize in Section~\ref{sec:method}. As we show in our experiments, SliderSpace is architecture-agnostic, working with both conventional U-Net based models like Stable Diffusion~\cite{rombach2022high, rombach2022sd20, podell2023sdxl, turbo, dmd} and recent transformer-based architectures like Flux~\cite{flux}.

We demonstrate the expressiveness of SliderSpace through three applications: First, we show how SliderSpace can decompose high-level concepts into diverse and expressive components, revealing the natural axes of variation in the model's understanding. Second, we explore artistic style variation, where SliderSpace discovers directions that match or exceed the diversity of manually curated artist lists while being judged more useful by human evaluators. Finally, we show how SliderSpace can help reverse the mode collapse commonly observed in distilled diffusion models, restoring diversity while maintaining generation speed.

Beyond providing practical creative control, SliderSpace opens new avenues for understanding and utilizing the latent capabilities of diffusion models. By mapping these models' visual potential into intuitive, composable directions, we take a step toward making their creative possibilities more accessible and interpretable to users.

% Image editing methods in diffusion models unlock the creativity of users. In this work we ask an alternate question: \emph{Can we organize and expose what of the diffusion model is already capable of?}.
% Existing methods for controlling image generation typically require users to manually specify edit directions for desired changes. This process is time-consuming, requires technical expertise, and limits the spontaneity of the creative process. For instance, if a user wants to adjust the smile of a generated person, they must explicitly request this edit, often through imprecise prompt engineering or model fine-tuning. This approach of predefined controls or manual specifications restricts users from fully exploring the latent capabilities of the model. There may be interesting stylistic variations or attributes that the model can generate, but users have no easy way to discover or utilize these.

% Natural visual disentanglement was an emergent property in the latent space of Generative Adversarial Models (GANs) \cite{harkonen2020ganspace,radford2015unsupervised, wu2021stylespace, shen2020interfacegan}. In particular, it has been observed that StyleGAN~\cite{karras2019style} stylespace neurons offer detailed control over many meaningful aspects of images that would be difficult to describe in words~\cite{wu2021stylespace}. However, diffusion models do not share such a compact latent space~\cite{park2023unsupervised}; and efforts to uncover such a space in the semantic embeddings of the text conditioning have met with limited success \nik{Nick - is there a specific citation you were thinking about?}.

% In this work we introduce \textbf{SliderSpace}, which takes a step towards uncovering an analogous low dimensional representation of diffusion models' visual breadth; in essence treating the diffusion model as many generators sharing parameters, where a particular generator is defined by a specific prompt. For a given prompt we sample many random seeds (and optionally prompt expansions using an LLM), generate the corresponding images, and apply an off the shelf feature extractor (in this work CLIP, but our method can be applied to any differentiable feature extractor). We use PCA to analyze these features, and for each of the leading $k$ principal components we train a LoRA \cite{} which causes the diffusion model to produces images which increase the feature magnitude along that component when passed back through the same feature extractor. This leads to a 'Slider' for each principal component, because each LoRA can be scaled and applied to the original diffusion model, continuously varying those visual features in the generated results (as measured, in our case, by CLIP).

% There are many other works that enhance the controllability of diffusion models. One common approach is enabling users to add spatial constraints to a generation either manually, or via a reference image \cite{zhang2023addingconditionalcontroltexttoimage, chen2024trainingfreeregionalpromptingdiffusion}, a second is leveraging more abstract embeddings (e.g. identity, style) extracted from a reference image \cite{ye2023ipadaptertextcompatibleimage, hertz2024stylealignedimagegeneration, li2023photomaker, shi2024instantbooth}, a third is finetuning a foundation model to better generate a concept important to the user \cite{ruiz2022dreambooth, kumari2022customdiffusion, Ryu_lora, hu2021lora}, and a fourth (most relevant to this work) is finding low-rank adaptors of the model based on a prompt or small training set which can be scaled to provide continous control over one aspect of generated image (e.g. night vs day, basic vs luxury, etc.) \cite{gandikota2023concept}. SliderSpace is complementary to all of these methods and offers something distinct. All of the other methods we are aware require the user (and / or model designer) to know in advance what type of control they want. In contrast SliderSpace assists users in discovering and controlling hidden capabilities present in the diffusion model's distribution of possible generations.

%We propose that truly intuitive creative control in a text-to-image model should meet three key criteria: \emph{discoverability}, \emph{intuitiveness}, and \emph{specificity}. The model should reveal controllable attributes that may not be immediately obvious, offer controls that are easy to understand and manipulate, and ensure each control affects a distinct attribute of the generated image.

% We demonstrate the utility and power of SliderSpace using three applications built on top of SDXL-DMD \cite{dmd}, because its fast generation speed lends itself well to the continuous control offered by SliderSpace.

% First, we study concept decomposition (Section \ref{sec:concept_exp}), where we learn sliders for a specific concept (e.g. 'monster', 'waterfall', 'car'). Through quantitative metrics of diversity and text alignment we demonstrate that the learned sliders dramatically boost the diversity of generations when randomly applied without harming text alignment; we also ask humans to qualitatively judge these results in a user study where they find the SliderSpace results to be more 'Diverse', 'Useful', and 'Creative' than our baselines.

% Second, we attempt to compare the automatic discoveries of SliderSpace to a large scale manual study of artistic styles (Section \ref{sec:art_exp}), open-sourced by ParrotZone \cite{parrotzone}. In this study SDXL was prompted with over 4300 artist names,  and based on visual inspection the cases of successful stylistic mimicry recorded. Quantitatively SliderSpace more closely matches the distribution of artistic variation discovered by ParrotZone than other baselines, and in our user studies was judged to be significantly more 'Diverse' and 'Useful' than the baselines. To our surprise humans even judged SliderSpace results to be slightly more 'Diverse' than the results generated by the manually discovered artist names of \cite{parrotzone}.

% Third, we attempt to use SliderSpace to reverse the mode collapse commonly observed in distilled few-step diffusion models relative to the original teacher model (Section \ref{sec:diverse_exp}). We quantitatively demonstrate that applying SliderSpace to SDXL-DMD leads to more closely matching the distribution of images by the original teacher, SDXL.

%Through extensive experiments on various state-of-the-art text-to-image models, we demonstrate that SliderSpace significantly enhances user control and creative expression in AI-assisted image generation tasks. Our method enables a range of applications, including concept decomposition and control, diversity improvement in generated images, customization dissection and edits, and the exploration of artistic styles inherent in the model.

% SliderSpace goes beyond providing a practical tool for enhanced creative control. By mapping the visual potential of diffusion models it can open new avenues for generative creativity and deepens our understanding of each model's hidden potential.
\begin{figure*}[t]
\vskip 0.2in
\begin{center}
\centerline{\includegraphics[width=\textwidth]{Figures/pipeline-vlm-v4.pdf}}
\caption{Overview of our data-aware preference optimization. For each preference instance: (1) We first break the preferred and rejected response into sub-sentences by prompting a large language model (LLM); 
(2) Next, we estimate the similarity scores between each sub-sentence and the given image using the CLIP classifier, and then calculate the differences between the preferred and rejected response as the hardness of the data; 
(3) Finally, we incorporate the estimated hardness into the preference optimization process by modifying $\beta$ in Equ~\eqref{equ:dpo}, allowing the model to adjust based on the data hardness.}
\label{fig:pipleine-vlm}
\end{center}
\vskip -0.2in
\end{figure*}


\section{Preliminary}
\label{sec:preliminary}
In this section, we briefly review the MLLM preference learning procedure, which starts by sampling pairwise preference data with a supervised fine-turned (SFT) model, and then optimizes on such preference data. Specifically, we categorize this process into the following aspects:

\noindent \textbf{Supervised Fine-Tuning (SFT).}
Preference learning of an MLLM $\bm{\pi}$ begins with an SFT model $\bm{\pi}_{\text{SFT}}$. Concretely, the SFT process fine-tunes the pre-trained MLLM model with millions of multi-modal question-answer pairs to align LLM with multi-modal tasks. 
After this process, we construct preference data by sampling pair-wise preference responses from $\bm{\pi}_{\mathrm{SFT}}$, formalized as $(y_w, y_l) \sim \bm{\pi}_{\mathrm{SFT}}(y|x,\mathcal{I})$, where $(\mathcal{I}$ denotes the image and $x$ is the prompt question. 
Meanwhile, $(y_w, y_l)$ are labeled as preferred and less preferred responses by humans, formalized as $(y_w \succ  y_l | \mathcal{I}, x)$.

\noindent \textbf{RLHF with Reward Models.}
Given pair-wise preference data $(y_w, y_l) \sim \bm{\pi}_{\mathrm{SFT}}(y|x,\mathcal{I})$, the preference learning process can be described in 2 stages: reward modeling and preference optimization. 
Specifically, the reward model $r_{\bm{\theta}}(y|\mathcal{I}, x)$ is defined to rank the model responses by learning to distinguish $y_w$ from $y_l$, and the preference optimization aims to distill the preference knowledge into MLLM. 
To learn a reward model, pioneering work \cite{rlhf} employs the Bradley-Terry model \cite{BT_model} to model the pair-wise preference distribution as:
\begin{equation}
\resizebox{.9\hsize}{!}{
\begin{math}
\begin{aligned}
    \mathrm{P}(y_w \succ  y_l|\mathcal{I}, x) & =  \sigma(r^{*}(y_w|\mathcal{I}, x)- (r^{*}(y_l|\mathcal{I}, x)) \\
     & = \frac{\mathrm{exp}(r^{*}(y_w|\mathcal{I}, x))}{\mathrm{exp}(r^{*}(y_w|\mathcal{I}, x))+\mathrm{exp}(r^{*}(y_l|\mathcal{I}, x))}.
\end{aligned}
\end{math}
}
\end{equation}

Thus, the learning process can be achieved by minimizing the negative log-likelihood $-\mathrm{logP}(y_w \succ y_l|\mathcal{I}, x)$ over the preference data with the parametrized reward model $r_{\bm{\phi}}(y_w|\mathcal{I}, x)$ initialized as $\bm{\pi}_{\mathrm{SFT}}$ with a simple linear layer to produce reward prediction. 
With the well-optimized reward model $r_{\phi}^{*}(y|\mathcal{I}, x)$, prior work \cite{rlhf} proposes to employ policy optimization algorithms in RL such as PPO \cite{PPO} to maximize the learned reward with KL-penalty, which can be formalized as:
\begin{equation}
\label{equ:ppo}
\begin{aligned}
    \underset{\bm{\pi}_{\theta}}{\text{max}} & \  \mathbf{E}_{(\mathcal{I},x) \sim \mathcal{D}, y \sim \bm{\pi}_{\theta}(\cdot|\mathcal{I}, x)} [r_{\phi}^{*}(y|\mathcal{I}, x)] \\
    & -\beta \mathbb{D}_{\mathbf{KL}}[\bm{\pi}_{\theta}(y|\mathcal{I},x)||\bm{\pi}_{\text{ref}}(y|\mathcal{I},x)], 
\end{aligned}
\end{equation}
where the fixed reference model $\bm{\pi}_{\text{ref}}$ is parameterized as $\bm{\pi}_{\text{SFT}}$, and the hyper-parameter $\beta$ controls the deviation of $\bm{\pi}_{\theta}$ from $\bm{\pi}_{\text{ref}}$ during the optimization process.

\noindent \textbf{Direct Preference Optimization (DPO).}
To relieve the high computational complexity of reward training in RLHF, DPO \cite{DPO} is proposed, which provides a simple way to directly optimize $\bm{\pi}_{\theta}$ with the pair-wise preference data, without parametrized reward model. Specifically, the DPO loss can be described as:
\begin{equation}
\label{equ:dpo}
\begin{aligned}
    \mathcal{L}_{\mathrm{dpo}} = - \bm{\mathrm{E}}_{(\mathcal{I},x, y_{w}, y_{l})} [ {\log \sigma}( & \beta \log \frac{{\pi}_{\bm{\theta}}(y_{w}|\mathcal{I},x)}{{\pi}_{\mathrm{ref}}(y_{w}|\mathcal{I},x)} \\
    - & \beta \log \frac{{\pi}_{\bm{\theta}}(y_{l}|\mathcal{I},x)}{{\pi}_{\mathrm{ref}}(y_{l}|\mathcal{I},x)}) ].
\end{aligned}
\end{equation}
This paper addresses a human-robot cooperative navigation task under incomplete information. The remote human operator possesses an outdated map of the environment, while the robot can acquire accurate local observations. The human provides navigation guidance, and the robot communicates environmental updates. Together, they aim to reach a set of goal locations as efficiently as possible.

We design a simulated maze environment, \emph{CoNav-Maze}, adapted from MemoryMaze~\cite{pasukonis2022evaluating} to study this setting. In CoNav-Maze, the robot has perfect knowledge of its position and uses motion primitives to navigate between adjacent grid cells. This setup abstracts away low-level control and estimation errors, focusing on high-level human-robot coordination.

Formally, the environment is modeled as a Markov Decision Process (MDP) defined by the tuple $(\Scal, \Acal, T, R_\mathrm{env}, \gamma)$. \( \mathcal{S} \) is a product space comprising the robot’s discrete finite state and the set of remaining goal locations, capturing both its position and task progress. $\Acal$ is a finite set of actions, including movement to adjacent grids and transmitting a first-person image from one of eight evenly spaced camera angles. $T: \mathcal{S} \times \mathcal{A} \to \mathcal{S}$ is a deterministic transition function. $R_\mathrm{env}: \mathcal{S} \to \mathbb{R}$: is a real-valued reward function. $\gamma \in [0, 1)$ is a discount factor.

At each step $t$, the robot collects a local observation of nearby traversable and blocked cells within a radius $r$. It may also receive a human-provided trajectory $\zeta_t$. The robot then selects an action $a_t$ to either move or transmit an image.

The human operator starts with an inaccurate global map $x \in \mathcal{X}$, representing traversable and blocked cells. By analyzing the robot’s trajectory and image transmissions, the human refines their map to provide more accurate guidance.
\section{Proposed Methods and Analysis}
\label{sec:proposed}

We aim to design algorithms that enjoy both a similar convergence guarantee as the US and RS and practical effectiveness incorporating the information of $\cP$.
%
In particular, we consider two algorithms inspired by the greedy algorithm and the RS and show theoretical guarantees.
%
Algorithm~\ref{alg:proposed} shows the pseudo-code of proposed algorithms.


\begin{algorithm}[!t]
    \caption{Proposed DRAL methods}\label{alg:proposed}
    \begin{algorithmic}[1]
        \Require Domain $\cX$, GP prior $\mu$ and $k$, ambiguity set $\cP$
        \State $\cD_{0} \gets \emptyset$
        \For{$t = 1, \dots, T$}
            \State Update $\sigma_{t-1}^2 (\cdot)$ according to Eq.~\eqref{eq:GP}
            \State Compute $\*x_t$ according to Eq.~\eqref{eq:RS} or Eq.~\eqref{eq:greedy}
        \EndFor
        \State Observe $y_1, \dots, y_T$ 
        \State Update $\mu_{T} (\cdot)$ and $\sigma_{T}^2 (\cdot)$ according to Eq.~\eqref{eq:GP}
        \State \Return $\mu_{T} (\cdot)$ and $\sigma_{T}^2 (\cdot)$
    \end{algorithmic}
\end{algorithm}



\subsection{Algorithms}

First, we consider the RS-based algorithm.
%
The algorithm is straightforward as follows:
\begin{align}
    \*x_t \sim p_t(\*x),
    \label{eq:RS}
\end{align}
where $p_t(\*x) = \argmax_{p \in \cP} \EE_{p(\*x^{*})}[\sigma_{t-1}^2 (\*x^{*})]$ and we assume that we can generate the sample from $p_t$.
%
By using the worst-case distribution $p_t$ for each iteration, this algorithm incorporates the information of $\cP$.


% \begin{itemize}
%     \item 貪欲法は一般に強いためそれに基づく方法を考える.
%     \item しかし, $\max_{p \in \cP}$を次ステップの全候補に対し計算するのに多大な計算量が必要なため, 貪欲法すら計算不可能
%     \item そこで, $p_t(\*x) = \argmax_{p \in \cP} \EE_{p(\*x)}[\sigma_t^2 (\*x)]$を固定した貪欲法を考える.
%     \item しかし, もはや貪欲法ですらないこの方法の近似保証は我々には難しかった.
%     \item そこで, 保守的なUSに理論保証があることから, 少し保守的になる (uncertainな候補を選択する) ように候補を制限することを考えた.
%     \item 最終的なアルゴリズムをXXXに示す.
% \end{itemize}

Second, we consider the greedy algorithm since its practical efficiency has often been reported~\citep[e.g., ][]{bian2017guarantees}.
%
However, in our setup, the greedy algorithm should be
\begin{align*}
    \argmin_{\*x \in \cX} \max_{p \in \cP} \EE_{p(\*x^{*})}[\sigma_{t}^2 (\*x^{*} \mid \*x)],
\end{align*}
which requires huge computational time in general due to min-max optimization.
%
Thus, we consider an approximately greedy algorithm as follows:
\begin{align*}
    \argmin_{\*x \in \cX} \EE_{p_t(\*x^{*})}[\sigma_{t}^2 (\*x^{*} \mid \*x)],
\end{align*}
where $p_t(\*x) = \argmax_{p \in \cP} \EE_{p(\*x^{*})}[\sigma_{t-1}^2 (\*x^{*})]$ is the worst-case distribution defined by $(\*x_i)_{i \in [t-1]}$.
%
On the other hand, the theoretical guarantee for this algorithm is challenging for us.
%
Hence, inspired by the fact that the US has a theoretical guarantee, we set the constraint so that the chosen input is uncertain than $\EE_{p_t(\*x^{*})}[\sigma_{t-1}^2 (\*x^{*})]$:
\begin{align}
    \*x_t = \argmin_{\*x \in \cX_t} \EE_{p_t(\*x^{*})}[\sigma_t^2 (\*x^{*} \mid \*x)],
    \label{eq:greedy}
\end{align}
where $\cX_t \coloneqq \{ \*x \in \cX \mid \sigma^2_{t-1}(\*x) \geq \EE_{p_t(\*x^{*})}[\sigma_{t-1}^2 (\*x^{*})] \}$.
%
Note that $|\cX_t| \geq 1$ holds due to the definition.


%%%%%%%%%%%%%%%%%%%%%%%%%%%%%%%%%%%%%%%%%%%%%%%%%%%%%%%%%%%%%%%%%%%%%%%%%%%%%%%%%%%%%%%%%%%%%%%%%%
\subsection{Analysis}
\label{sec:analysis}

Here, we show the error convergence by Eqs.~\eqref{eq:RS} and \eqref{eq:greedy}:
\begin{theorem}
    Fix $\delta \in (0, 1)$.
    %
    Assume that $\cX \subset \RR^d$ is a compact subset.
    %
    If we run Algorithm~\ref{alg:proposed} with Eq.~\eqref{eq:RS}, the following holds with probability at least $1 - \delta$:
    \begin{align*}
        \max_{p \in \cP} \EE_{p(\*x^{*})}[\sigma_{T}^2 (\*x^{*})] 
        &\leq \frac{2 C_1 \gamma_T}{T} + \cO \left( \frac{\log (1 / \delta)}{T} \right),
    \end{align*}
    where $C_1 = 1 / \log(1 + \sigma^{-2})$.
    \label{theo:error_convergence_RS}
\end{theorem}
%
\begin{theorem}
    Assume that $\cX \subset \RR^d$ is a compact subset.
    %
    If we run Algorithm~\ref{alg:proposed} with Eq.~\eqref{eq:greedy}, the following holds:
    \begin{align*}
        \max_{p \in \cP} \EE_{p(\*x^{*})}[\sigma_{T}^2 (\*x^{*})] 
        &\leq \frac{C_1 \gamma_T}{T},
    \end{align*}
    where $C_1 = 1 / \log(1 + \sigma^{-2})$.
    \label{theo:error_convergence_greedy}
\end{theorem}
See Appendix~\ref{sec:proposed_proof} for the proof, in which Lemma~3 in \citet{kirschner2018-information} is used to show Theorem~\ref{theo:error_convergence_RS}.



Consequently, our proposed methods achieve almost the same convergence as those of the US and RS shown in Proposition~\ref{prop:us_rs}.
%
Furthermore, by combining Lemmas~\ref{lem:UB_error_discrete}, \ref{lem:UB_error_frequentist_continuous}, and \ref{lem:UB_error_bayesian_continuous}, we can see that the upper bound of $E_T$:
\begin{corollary}
    Fix $\delta \in (0, 1)$ and $T \in \NN$.
    %
    Then, if we run Algorithm~\ref{alg:proposed}, the following hold with probability at least $1 - \delta$:
    \begin{enumerate}
        \item When Assumption~\ref{assump:Bayesian} or Assumptions~\ref{assump:frequentist} holds, 
        \begin{align*}
            E_T = \cO\left( \frac{\log (|\cX| / \delta) \gamma_T}{T} \right);
        \end{align*}
        \item When Assumptions~\ref{assump:Bayesian} and \ref{assump:Bayesian_continuous} or Assumptions~\ref{assump:frequentist} and \ref{assump:frequentist_continuous} hold, 
        \begin{align*}
            E_T = \cO\left( \frac{\log (T / \delta) \gamma_T}{T} \right),
        \end{align*}
    \end{enumerate}
\end{corollary}
\begin{proof}
    We can obtain the result by combining Lemmas~\ref{lem:UB_error_discrete}, \ref{lem:UB_error_frequentist_continuous}, and \ref{lem:UB_error_bayesian_continuous}, Theorems~\ref{theo:error_convergence_RS} and \ref{theo:error_convergence_greedy}, and the union bound.
    %
    Note that we assume $|\cX| > T$.
\end{proof}
%
Thus, the error incurred by the proposed algorithms converges to $0$ with high probability for discrete and continuous input domains, at least with linear, SE, and Mat\'ern kernels.


\section{Related work}
\label{sec:related}



Learning to execute programs as  a benchmarking task for code reasoning capabilities has been long studied in the machine learning community \citep{learning_to_execute2014}, sometimes with niche architectures
\citep{GravesWD14, GauntBKT16, learning_to_execute2020}, typically on toy or restricted programs. \citet{bieber2022staticpredictionruntimeerrors} proposed learning to predict runtime errors as a practical application of neural program evaluation. More recently, \citet{gu2024cruxeval} introduced  a program output (and input) benchmark for LLMs to measure code understanding capabilities, which we used for evaluation in this work.
Most closely to ours, \citet{scratchpad} propose the use of \textit{scratchpads} to let LLMs write down the results of intermediate computations rather than directly aiming at predicting the final output. With Python output prediction being one of their use cases, they represent traces of intermediate states as JSON dictionaries. 
\citet{ni2024nextteachinglargelanguage} introduce \textit{Naturalized} EXecution Tuning (NExT) and propose the compact representation of Python traces that we followed. Unlike \citet{scratchpad} and this work, NeXT simplifies loops and uses traces in the \textit{input}, improving program repair.  \citet{ding2024semcodertrainingcodelanguage} propose natural language explanations based on executions, leading to further improvements. 
Finally, recent work uses execution \textit{feedback}, rather than traces, in SFT or reinforcement learning settings \citep{dong2024selfplayexecutionfeedbackimproving,gehring2024rlefgroundingcodellms}. 

\section{Experiments}
\label{sec:experiments}

\subsection{Experiment Setting}
\label{sec:exp_setting}

\textbf{Hyperparameters.}
We described details in~\cref{sec:appendix_hyperparameter}.

\textbf{Baselines.}
We compare the performance of \ours against the following baselines, mostly chosen for their long-context capabilities.
(1) \textbf{Truncated FA2}: The input context is truncated in the middle to fit in each model's pre-trained limit, and we perform dense attention with FlashAttention2 (FA2)~\citep{dao_flashattention_2022}.
(2) \textbf{DynamicNTK}~\citep{bloc97_ntk-aware_2023} and (3) \textbf{Self-Extend}~\citep{jin_llm_2024} adjust the RoPE for OOL generalization. We perform dense attention with FA2 without truncating the input context for these baselines.
Both (4) \textbf{LM-Infinite}~\citep{han_lm-infinite_2024} and (5) \textbf{StreamingLLM}~\citep{xiao_efficient_2024} use a combination of sink and streaming tokens while also adjusting the RoPE for OOL generalization.
(6) \textbf{H2O}~\citep{zhang_h_2o_2023} is a KV cache eviction strategy which retains the top-$k$ KV tokens at each decoding step. 
(7) \textbf{InfLLM}~\citep{xiao_infllm_2024} selects a set of representative tokens for each chunk of the context, and uses them for top-$k$ context selection.
%(8) \textbf{Double Sparse Attention}~\citep{yang_post-training_2024} estimates the top-$k$ tokens by sampling few channels of the key vectors.
(8) \textbf{HiP Attention}~\citep{lee_training-free_2024} uses a hierarchical top-$k$ token selection algorithm based on attention locality.

\textbf{Benchmarks.}
We evaluate the performance of \ours on mainstream long-context benchmarks. 
(1) LongBench~\citep{bai_longbench_2023}, whose sequence length averages at around 32K tokens, 
and (2) $\infty$Bench~\citep{zhang_inftybench_2024} with a sequence length of over 100K tokens. 
Both benchmarks feature a diverse range of tasks, such as long document QA, summarization, multi-shot learning, and information retrieval.
We apply our method to the instruction-tuned Llama 3 8B model~\citep{grattafiori_llama_2024} and the instruction-tuned Mistral 0.2 7B model~\citep{jiang_mistral_2023}. As our framework is training-free, applying our method to these models incurs zero extra cost.

\begin{figure}
\centering
\vspace{0.5em}
\includegraphics[width=0.485\linewidth]{figures/images/plot_sglang_decoding_RTX4090.pdf}
\includegraphics[width=0.485\linewidth]{figures/images/plot_sglang_decoding_L40s.pdf}
\vspace{0.5em}
\includegraphics[width=\linewidth]{figures/images/plot_sglang_decoding_legend.pdf}
\vspace{-2em}
\caption{\textbf{SGlang Decoding Throughput Benchmark.} Dashed lines are estimated values. RTX4090 has 24GB and L40s has 48GB of VRAM. We used is AWQ Llama3.1 with FP8 KV cache.}
\label{fig:sglang_decoding}
\vspace{-1.8em}
\end{figure}
\subsection{Results}
\textbf{LongBench.}
In \Cref{tab:longbench}, our method achieves about 7.17\%p better relative score using Llama 3 and 3.19\%p better using Mistral 0.2 compared to the best-performing baseline InfLLM.
What makes this significant is that our method processes 4$\times$ fewer key tokens through sparse attention in both models compared to InfLLM, leading to better decoding latency as shown in \cref{tab:latency}.

\textbf{$\infty$Bench.}
We show our results on $\infty$Bench in \Cref{tab:infbench}. The \textit{3K-fast and 3K-flash} window option of ours uses the same setting as \textit{3K} except using a longer mask refreshing interval as detailed in \Cref{sec:exp_setting}.
Our method achieves 9.99\%p better relative score using Llama 3 and 4.32\%p better using Mistral 0.2 compared to InfLLM. The performance gain is larger than in LongBench, which has a fourfold shorter context. This suggests that our method is able to better utilize longer contexts than the baselines.

To further demonstrate our method's superior OOL generalization ability, we compare $\infty$Bench's En.MC score in various context lengths with Llama 3.1 8B in \cref{fig:infbench_llama3.1}.
While \ours keeps gaining performance as the context length gets longer, baselines with no OOL generalization capability degrade significantly beyond the pretrained context length (128K).
In \cref{fig:infbench_gemma_exaone}, we experiment with other short-context LLMs: Exaone 3 (4K)~\citep{research_exaone3_2024}, Exaone 3.5 (32K)~\citep{research_exaone_2024} and Gemma2 (8K)~\citep{team_gemma_2024}.
We observe the most performance gain in an extended context with these short-context models. For instance, with Gemma2, we gain an impressive +24.45\%p in En.MC and +22.03\%p in En.QA compared to FA2.

\subsection{Analysis}
In this section, we analyze the latency and the effect of each of the components of our method.


\textbf{Latency.}
We analyze the latency of our method on a 1-million-token context and compare it against baselines with settings that yield similar benchmark scores. In \cref{tab:latency}, we measure the latencies of attention methods.
%InfLLM uses a 12K context window, HiP uses a 1K window, and ours uses the 3K window setting.
During a 1M token prefill, our method is 20.29$\times$ faster than FlashAttention2 (FA2), 6\% faster than InfLLM, and achieves similar latency with the baseline HiP.
During decoding with a 1M token context, our method significantly outperforms FA2 by 19.85$\times$, InfLLM by 4.98$\times$, and HiP by 92\%.
With context extension (dynamic RoPE) enabled, our method slows down about 1.6$\times$ in prefill and 5\% in decoding due to overheads incurred by additional memory reads of precomputed cos and sin vectors.
Therefore, our method is 50\% slower than InfLLM in context extension-enabled prefill, but it is significantly faster in decoding because decoding is memory-bound:
Our method with a 3K token context window reads fewer context tokens than InfLLM with a 12K token context window.

\textbf{Latency with KV Offloading.} In \cref{tab:latency_offload}, we measure the decoding latency with KV cache offloading enabled on a Passkey retrieval task sample.
We keep FA2 in the table for reference, even though FA2 with UVM offloading is 472$\times$ slower than the baseline HiP.
Among the baseline methods, only InfLLM achieves KV cache offloading in a practical way.
In 256K context decoding, we outperform InfLLM by 3.64$\times$.
With KV cache offloading, the attention mechanism is extremely memory-bound, because accessing the CPU memory over PCIe is 31.5$\times$ more expensive in terms of latency than accessing VRAM.
InfLLM chooses not to access the CPU memory while executing its attention kernel, so it has to sacrifice the precision of its top-k estimation algorithm. This makes larger block and context window sizes necessary to maintain the model's performance on downstream tasks.
In contrast, we choose to access the CPU memory during attention kernel execution like baseline HiP.
This allows more flexibility for the algorithm design, performing better in downstream NLU tasks.
Moreover, our UVM implementation makes the KV cache offloaded attention mechanism a graph-capturable operation, which allows us to avoid CPU overheads, unlike InfLLM.
In contrast to the offloading framework proposed by \citet{lee_training-free_2024}, we cache the sparse attention mask separately for each pruning stage. 
This enables us to reduce the frequency of calling the costly initial pruning stage, which scales linearly.

\textbf{Throughput.} In \cref{fig:sglang_decoding}, we present the decoding throughput of our method using RTX 4090 (24GB) and L40S (48GB) GPUs. On the 4090, our method achieves a throughput of 3.20$\times$ higher at a 1M context length compared to the estimated decoding throughput of SRT (SGlang Runtime with FlashInfer). Similarly, on the L40S, our method surpasses SRT by 7.25$\times$ at a 3M context length.
Due to hardware limitations, we estimated the decoding performance since a 1M and 3M context requires approximately 64GB and 192GB of KV cache, respectively, which exceeds the memory capacities of 24GB and 48GB GPUs.
We further demonstrate that adjusting the mask refreshing interval significantly enhances decoding throughput without substantially affecting performance. The \textit{Flash} configuration improves decoding throughput by approximately 3.14$\times$ in a 3M context compared to the \textit{Fast} configuration.

%\begin{wrapfigure}[9]{l}{0.35\linewidth}
\begin{figure}[t]
\vspace{0.5em}
\centering
\begin{subfigure}[b]{0.50\linewidth}
\includegraphics[width=\linewidth]{figures/images/plot_topk_recall.pdf}
\vspace{-1.5em}
\caption{\textbf{Recall.}}
\label{fig:topk_recall}
\end{subfigure}
% \hfill
% \hspace{0.2em}
\raisebox{0.2in}{
\begin{subtable}[b]{0.46\linewidth}
\vspace{0.5em}
\centering
\resizebox{0.8\linewidth}{!}{%
\begin{minipage}{\linewidth}
\centering
\begin{tabular}{lr}\toprule%
&En-MC \\\midrule%
FA2 (128K) &67.25 \\%
Ours ($N=2$) &70.31 \\%
Ours ($N=3$) &\textbf{74.24} \\%
\bottomrule%
\end{tabular}%
\end{minipage}%
}
\caption{\textbf{Pruning Stage Ablation Study in $\infty$Bench En.MC.}}
\label{tab:stage_ablation}%
\end{subtable}
}
\vspace{-2.2em}
\caption{\textbf{Analysis}}
\vspace{-1.8em}
\end{figure}
%\end{wrapfigure}
\textbf{Accuracy of top-$k$ estimation.}
In \cref{fig:topk_recall}, we demonstrate our method has better coverage of important tokens, which means higher recall of attention probabilities of selected key tokens. 
Our method performs 1.57\%p better than InfLLM and 4.72\%p better than baseline HiP.
The better recall indicates our method follows pretrained attention patterns more closely than the baselines. 

%\begin{wraptable}[8]{r}{3.2cm}
\vspace{-4.5ex}
\caption{\textbf{Pruning Stage Ablation Study in InfiniteBench En.MC.}}
\centering
\small
\vspace{1.0em}
\label{tab:stage_ablation}
\begin{tabular}{l@{\hskip-2pt}r}\toprule
&En-MC \\\midrule
FA2 (128K) &67.25 \\
Ours ($N=2$) &70.31 \\
Ours ($N=3$) &\textbf{74.24} \\
\bottomrule
\end{tabular}
\end{wraptable}
\textbf{Ablation on Depth of Stage Modules.}
In \Cref{tab:stage_ablation}, we perform an ablation study on a number of stages ($N$) that are used in ours. The latency-performance optimal pruning module combination for each setting is found empirically.

\begin{table}[t]
\caption{\textbf{RoPE Ablation Study in Context Pruning and Sparse Attention.} We measure the accuracy of $\infty$Bench En.MC subset truncated with $T$=128K with various combinations of RoPE extends style in context pruning and sparse attention kernels. Each row represents a single RoPE extend style in the context pruning procedure, and each column represents the RoPE extend style in block sparse attention. \textit{SA} stands for sparse attention, \textit{DE} stands for dynamic RoPE extend (SelfExtend variant), \textit{IL} stands for InfLLM style RoPE, \textit{ST} stands for StreamingLLM style RoPE, \textit{RT} stands for relative RoPE in hierarchical representative token selection.}
\label{tab:rope_ablation}
\vspace{1.0em}
\centering
% \resizebox{8\linewidth}{!}{
\small
\begin{tabular}{lrrrrr}\toprule
\makecell[r]{RoPE Style in\\Pruning \textbackslash\ SA} &DE &IL &ST &AVG. \\\midrule
DE (Dynamic) 
    &\cellcolor[HTML]{ff8c6c}52.40 
    &\cellcolor[HTML]{ff9c64}54.59 
    &\cellcolor[HTML]{ff8370}51.09 
    &\cellcolor[HTML]{ff8370}52.69 \\
IL (InfLLM)
    &\cellcolor[HTML]{3ab799}68.12
    &\cellcolor[HTML]{5dba8b}66.81 
    &\cellcolor[HTML]{00b1b0}\textbf{70.31} 
    &\cellcolor[HTML]{05b2ae}68.41 \\
CI (Chunk-Indexed)
    &\cellcolor[HTML]{46b895}67.69 
    &\cellcolor[HTML]{5dba8b}66.81 
    &\cellcolor[HTML]{46b894}67.69 
    &\cellcolor[HTML]{26b5a1}67.39 \\
RT (Relative)
    &\cellcolor[HTML]{5dba8b}66.81 
    &\cellcolor[HTML]{2fb69d}68.56 
    &\cellcolor[HTML]{01b2af}\textbf{70.31} 
    &\cellcolor[HTML]{00b1b0}\textbf{68.56} \\
AVG.
    &\cellcolor[HTML]{ff8370}63.76 
    &\cellcolor[HTML]{ffba54}64.19 
    &\cellcolor[HTML]{00b1b0}\textbf{64.85} 
    &- \\
\bottomrule
\end{tabular}
% }
\vspace{-1em}
\end{table}
\textbf{Ablation on RoPE interpolation strategies.}
In \cref{tab:rope_ablation}, we perform an ablation study on the dynamic RoPE extrapolation strategy in masking and sparse attention.
We choose the best-performing RT/ST combination for our method.
\section{Conclusion and future directions} \label{sec:conclusion}

In this paper we proposed a nested MLMC framework that offers important computational savings by performing most calculations in low precision and exploiting approximate random normal variables for the low precision path calculations. The low precision calculations could be performed in fixed precision on an FPGA for greater efficiency, and we suggested a procedure to optimise the bit-widths of every variable at each Monte Carlo level. This is an important improvement over previous mixed precision MLMC frameworks which held the lower precision fixed \cite{Rounding_error_oliver} or defined uniform bit-width at every level heuristically \cite{brugger2014mixed}. Our numerical results suggest that for the first levels our procedure reduces the cost at these levels by a factor 5 or 7. Hence the overall savings are significant since most paths are calculated on the first levels. Our approach would be even more efficient for the Milstein scheme because its higher order strong convergence leads to a greater proportion of the computational costs being on the coarsest levels.

The next stage of the research project will be to implement the RNG methods and the nested framework on FPGAs to determine the hardware requirements and confirm the extent of the computational savings. It would also be good to compare the performance benefits to using half-precision floating point arithmetic on GPUs or CPUs for the low-accuracy computations.




\section*{Acknowkedgements}
This work was partially supported by 
JST ACT-X Grant Number (JPMJAX23CD and JPMJAX24C3), 
JST PRESTO Grant Number JPMJPR24J6, 
JST CREST Grant Numbers (JPMJCR21D3 including AIP challenge program and JPMJCR22N2),
JST Moonshot R\&D Grant Number JPMJMS2033-05, 
%%%%%%%
JSPS KAKENHI Grant Number (JP20H00601, JP23K16943, JP23K19967, JP24K15080, and JP24K20847), 
%%%%%%%
NEDO (JPNP20006),
and RIKEN Center for Advanced Intelligence Project.


% MEXT Program: Data Creation and Utilization-Type Material Research and Development Project Grant Number JPMXP1122712807,
% JSPS KAKENHI Grant Numbers JP20H00601,JP21H03498,JP22H00300,JP23K16943,JP23K19967,
% JST CREST Grant Numbers JPMJCR21D3,JPMJCR22N2, 

% JST ACT-X Grant Number JPMJAX23CD,
% NEDO Grant Numbers JPNP18002,JPNP20006, 
% and RIKEN Center for Advanced Intelligence Project.

%%%%%%%%%%%%%%%%%%%%%%%%%%%%%%%%%%%%%%%%%%%%%%%%%%%%%%%%%%%%%%%%%%%%%%%%%%%%%%%%%%%%%%%%%%%%%%
% \clearpage

% \section*{Impact Statements}

% This paper focuses on the theoretical and algorithmic aspects of the machine learning method.
% %
% We consider that some potential societal consequences of this paper need not necessarily be highlighted here.

%%%%%%%%%%%%%%%%%%%%%%%%%%%%%%%%%%%%%%%%%%%%%%%%%%%%%%%%%%%%%%%%%%%%%%%%%%%%%%%%%%%%%%%%%%%%%%
% % \subsubsection*{References}
\bibliography{ref}
\bibliographystyle{plainnat}
% %%%%%%%%%%%%%%%%%%%%%%%%%%%%%%%%%%%%%%%%%%%%%%%%%%%%%%%%%%%%%%%%%%%%%%%%%%%%%%%%%%%%%%%%%%%%%%

%%%%%%%%%%%%%%%%%%%%%%%%%%%%%%%%%%%%%%%%%%%%%%%%%%%%%%%%%%%%%%%%%%%%%%%%%%%%%%%%%%%%%%%%%%%%%%
\appendix
\begin{table}[t!]
  \centering
  


% \renewcommand{\arraystretch}{1.2} % 调整行高
% \setlength{\tabcolsep}{10pt}  % 调整列间距
\resizebox{0.48\textwidth}{!}{%
\begin{tabular}{lcrr}
        \toprule
        \textbf{Dataset} & \textbf{Full Size*} & \textbf{Consistency}  & \textbf{\dataset{}} \\
        \midrule
        HotpotQA  & 5,901 & 2,973 {\footnotesize \textcolor{gray}{(50\%)}}  & 1,476 {\footnotesize \textcolor{gray}{(25\%)}}  \\
        NewsQA    & 4,212 & 1,260 {\footnotesize \textcolor{gray}{(30\%)}} & 934  {\footnotesize \textcolor{gray}{(22\%)}}  \\
        NQ        & 7,314 & 4,419 {\footnotesize \textcolor{gray}{(60\%)}}  & 1,479 {\footnotesize \textcolor{gray}{(20\%)}}  \\
        SearchQA  & 16,980 & 12,133 {\footnotesize \textcolor{gray}{(71\%)}} & 1,497 {\footnotesize \textcolor{gray}{(9\%)}}  \\
        SQuAD     & 10,490 & 5,024 {\footnotesize \textcolor{gray}{(48\%)}}  & 2,351 {\footnotesize \textcolor{gray}{(22\%)}}  \\
        TriviaQA  & 7,785 & 6654 {\footnotesize \textcolor{gray}{(85\%)}}  & 792  {\footnotesize \textcolor{gray}{(10\%)}}  \\
        \bottomrule
    \end{tabular}
}




 \caption{Number of instances at each stage in the \dataset{} construction pipeline.}
 \label{tab:our_bench_stats_each_step}
\end{table}
\section{Appendix}
\subsection{License}
We present the licenses of the datasets used in this study: Natural Questions (CC BY-SA 3.0 license), NewsQA (MIT License), SearchQA and TriviaQA (Apache License 2.0), HotpotQA and SQuAD (CC BY-SA 4.0 license).

All these licenses and agreements permit the use of their data for academic purposes.

\subsection{Details of Data Constructing}
\label{append:prompts}
In this section, we detail the two main steps in constructing \dataset{}. The dataset sizes at each stage of the pipeline are shown in Table~\ref{tab:our_bench_stats_each_step}.


\textbf{Parametric Knowledge Elicitation.} First, we elicit the LLM's parametric knowledge by prompting it in a closed-book setting (i.e., without any context). To ensure the reliability of the elicited knowledge, we apply a consistency-based filtering method. Specifically, for each query, the LLM is prompted five times, and the frequency of each response is recorded. The response with the highest frequency is identified as the majority answer. Queries where the majority answer appears fewer than three times are discarded, in order to filter out inconsistent responses and enhance data quality. The following prompt is used to instruct the LLM:
\begin{tcolorbox}
[title=Prompt for eliciting parametric knowledge,colback=blue!10,colframe=blue!50!black,arc=1mm,boxrule=1pt,left=1mm,right=1mm,top=1mm,bottom=1mm]
Answer the question \textcolor{blue}{\{\textit{brevity\_instruction}\}} and provide supporting evidence.

Question: \textcolor{blue}{\{\textit{question}\}}
\end{tcolorbox}
\noindent The ``\textit{brevity\_instruction}'' is used to guide the LLM to generate responses in a more concise form.

\textbf{Conflict Data Selection.} Next, we filter the data to retain only instances where the LLM's parametric knowledge directly conflicts with the contextual answer. Specifically, we categorize the data obtained from the previous step into two groups, conflicting and non-conflicting instances, based on the detailed results of conflict detection. All non-conflicting instances are discarded. GPT-4o-mini is then used to detect the presence of a conflict, using the following prompt:

\begin{tcolorbox}
[title=Prompt for identifying conflict knowledge,colback=blue!10,colframe=blue!50!black,arc=1mm,boxrule=1pt,left=1mm,right=1mm,top=1mm,bottom=1mm]
\small
You are tasked with evaluating the correctness of a model-generated answer based on the given information. 

\small
Context: \textcolor{blue}{\{\textit{context}\}}

Question: \textcolor{blue}{\{\textit{question}\}}

Contextual Answer: \textcolor{blue}{\{\textit{contextual\_answer}\}}

Model-Generated Answer: \textcolor{blue}{\{\textit{Model-Generated\_answer}\}}

\textcolor{blue}{[\textit{Detailed task description...}]}

Output Format:

Evaluate result: (Correct / Partially Correct / Incorrect) 
\end{tcolorbox}




\subsection{Assessing the Reliability of GPT-4o-mini in Knowledge Conflict Identification}
\label{append:human_eval}
In this subsection, we conduct the human evaluation to assess the reliability of GPT-4o-mini in identifying knowledge conflicts, which is a critical task in our data construction process to guarantee the data quality.

We randomly sampled 100 examples from each of the six subsets of \dataset{}, yielding a total of 600 samples. Six senior computational linguistics researchers were then asked to evaluate whether a knowledge conflict was present in each example. For each instance, the evaluators were provided with the question, the contextual answer, the model-generated response, and the corresponding supporting evidence. The results were classified into three categories: No Conflict, Somewhat Conflict, and High Conflict. The detailed annotation instructions are as follows:

\begin{tcolorbox}
[title=Annotation Instruction,colback=blue!10,colframe=blue!50!black,arc=1mm,boxrule=1pt,left=1mm,right=1mm,top=1mm,bottom=1mm]
\small
You are tasked with determining whether the parametric knowledge of LLMs conflicts with the given context to facilitate the study of knowledge conflicts in large language models.

Each data instance contains the following fields: 

Question: \textcolor{blue}{\{\textit{question}\}}


Answers: \textcolor{blue}{\{\textit{answers}\}}


Context: \textcolor{blue}{\{\textit{context}\}}

Parametric\_knowledge: \textcolor{blue}{\{\textit{LLMs' parametric\_knowledge }\}} 

The annotation process consists of two steps. 

\textbf{Step 1}: Compare the model-generated answer with the ground truth answers, based on the given question and context, to determine whether the model’s parametric knowledge conflicts with the context.

\textbf{Step 2}: Classify the results into one of three categories: 

\textcolor{blue}{\{\textit{No Conflict}\}} if the model-generated answer is consistent with the ground truth answers and context, 

\textcolor{blue}{\{\textit{Somewhat Conflict}\}}  if it is partially inconsistent

\textcolor{blue}{\{\textit{High Conflict}\}} if it significantly contradicts the ground truth answers or context.
\end{tcolorbox}


The evaluation results, shown in Table~\ref{tab:append_human_eval}, reveal a high level of agreement between the human annotators and GPT-4o-mini. Over 85\% of the examples reach consensus among the annotators, with an average agreement rate of 85.6\% across all subsets. These findings underscore the reliability of GPT-4o-mini as an effective tool for identifying knowledge conflicts.




\begin{table}[t]
  \centering
  
\centering
\begin{tabular}{l c}
\toprule
\textbf{Subset} & \textbf{Agreement (\%)} \\ \midrule
HotpotQA        & 81.4                        \\
NewsQA          & 72.7                        \\
NQ              & 88.7                        \\
SearchQA        & 95.3                        \\
SQuAD           & 86.1                        \\
TriviaQA        & 90.7                        \\ \midrule
\textbf{Average} & \textbf{85.6}            \\ \bottomrule
\end{tabular}

 \caption{Agreement between human annotators and GPT-4o-mini across different subsets of our \dataset{} benchmark.}
 \label{tab:append_human_eval}
\end{table}



\subsection{Evaluating the Effectiveness of Our Consistency-Based Filtering Method}
\label{append:data_freq}

In this subsection, we evaluate the effectiveness of our consistency-based knowledge conflict filtering method. As described in Appendix~\ref{append:prompts}, for each query, we prompt the model five times and record the most frequently generated answer along with its occurrence frequency. Based on this frequency, we divide the data into sub-datasets, where all queries within each sub-dataset share the same answer frequency. We then apply ``Conflict Data Selection'' to each sub-dataset, retaining only instances where knowledge conflicts occur. Finally, we evaluate ConR and MemR on these sub-datasets.

As shown in Figure~\ref{fig:diff_freq}, a clear trend emerges: as answer frequency increases, ConR consistently decreases, while MemR increases. This pattern indicates that as answer frequency rises, the model becomes increasingly reliant on its internal knowledge. Notably, for data with an answer frequency of 1, MemR is only 3\%, indicating minimal dependence on internal knowledge. Retaining only high-answer-frequency data improves the quality of \dataset{}. This data construction approach distinguishes our methodology from previous studies~\cite{longpre2021entity,xie2023adaptive}.

\begin{figure}[t!]
  \centering
  \includegraphics[width=0.4\textwidth]{figs/diff_freq.pdf}
  \caption{Performance comparison of ConR and MemR across sub-datasets grouped by the answer frequency of LLMs.}
  \label{fig:diff_freq}
\end{figure}





\subsection{Additional Implementation Details of Our Experiments}
\label{append:implementation}
This subsection outlines the training prompt, describes more details of the training data, and provides details of the experimental setup used in our experiments.

\textbf{Training Prompts.}
We adopt a simple QA-format training prompt following~\citet{zhou2023context} for all methods except \attrprompt{} and \oiprompt{}.
\begin{tcolorbox}
[title=Base Prompt ,colback=blue!10,colframe=blue!50!black,arc=1mm,boxrule=1pt,left=1mm,right=1mm,top=1mm,bottom=1mm]
% \small
\textcolor{blue}{\{\textit{context}\}} 
Q: \textcolor{blue}{\{\textit{question}\}} ? 
A: \textcolor{blue}{\{\textit{answer}\}}.
\end{tcolorbox}


\textbf{Training Datasets.} During \method{}, we randomly sample 32,580 instances from the training set of the MRQA 2019 benchmark~\cite{fisch2019mrqa} to construct our training data.



\textbf{Experimental Setup.} In this work, all models are trained for 2,100 steps with a total batch size of 32 and a learning rate of 1e-4. To enhance training efficiency, we implemented \method{} with LoRA~\cite{hu2021lora}, setting both the rank $\text{r}$ and scaling factor $\text{alpha}$ to 64. For \method{}, we set $\alpha$ to 0.1 (Eq.~\ref{eq:selct_layers}), which determines the minimum activation ratio difference required for a layer to be pruned. Additionally, we adopt a dynamic $\gamma$ in $\mathcal{L}_{\text{KC}}$ (Eq.~\ref{eq:kc_loss}), which linearly transitions from an initial margin ($\gamma_{0}=1$) to a final margin ($\gamma^*=5$) as training progresses. This adaptive strategy gradually reduces the model's reliance on internal parametric knowledge, encouraging it to rely more on external knowledge provided by the KAG system.


\subsection{Implementation Details of Baselines}
\label{append:baseline}
This subsection describes the implementation details of all baseline methods.

We adopt two prompt-based baselines: the attributed prompt ($\text{Attr}_{\text{prompt}}$) and a combination of opinion-based and instruction-based prompts ($\text{O\&I}_{\text{prompt}}$). The corresponding prompt templates are as follows:

\begin{tcolorbox}
[title=Attr based prompt ,colback=blue!10,colframe=blue!50!black,arc=1mm,boxrule=1pt,left=1mm,right=1mm,top=1mm,bottom=1mm]
% \small
\textcolor{blue}{\{\textit{context}\}} Q: \textcolor{blue}{\{\textit{question}\}} based on the given text? A: \textcolor{blue}{\{\textit{answer}\}}.
\end{tcolorbox}

\begin{tcolorbox}
[title=O\&I based prompt ,colback=blue!10,colframe=blue!50!black,arc=1mm,boxrule=1pt,left=1mm,right=1mm,top=1mm,bottom=1mm]

Bob said ``\textcolor{blue}{\{\textit{context}\}}'' Q: \textcolor{blue}{\{\textit{question}\}} in Bob's opinion? A: \textcolor{blue}{\{\textit{answer}\}}.
\end{tcolorbox}
For the SFT baseline, we incorporate context during training, similar to \method{}, while keeping the remaining experimental settings identical. To construct preference pairs for DPO training, we use contextually aligned answers from the dataset as ``preferred responses'' to ensure the consistency with the provided context. The ``rejected responses'' are generated by identifying parametric knowledge conflicts through our data construction methodology (Sec.~\ref{sec:benchmark}).

For KAFT, we employ a hybrid dataset containing both counterfactual and factual data. Specifically, we integrate the counterfactual data developed by \citet{xie2023adaptive}, leveraging their advanced data construction framework.

By maintaining equivalent dataset sizes and ensuring comparable data quality across all baselines, we provide a rigorous and fair comparison with our proposed \method{}.




\subsection{Extending \method{} to More LLMs}
\label{append:diff_model_performance}


\begin{figure}[t!]
  \centering
  
\subfigure[ConR Results]{
        \label{fig:diff_model:llama_conr}
        \includegraphics[width=0.462\linewidth]{append_fig/llama_conr.pdf}
    }
    \hspace{0.0005\linewidth} 
    \subfigure[MemR Results]{
        \label{fig:diff_model:llama_memr}
        \includegraphics[width=0.462\linewidth]{append_fig/llama_memr.pdf}
    }


  % \includegraphics[width=0.48\textwidth]{figs/diff_model_double.pdf}
 \caption{Average ConR and MemR across different models implemented by LLMs of LLaMA series, before and after applying \method{}.
 }
 \label{fig:diff_model_double_llama}
\end{figure}

\begin{figure}[t]
  \centering
  \subfigure[ConR Results]{
        \label{fig:diff_model:qwen_conr}
        \includegraphics[width=0.462\linewidth]{append_fig/qwen_conr.pdf}
    }
    \hspace{0.0005\linewidth} 
    \subfigure[MemR Results]{
        \label{fig:diff_model:qwen_memr}
        \includegraphics[width=0.462\linewidth]{append_fig/qwen_memr.pdf}
    }
  % \includegraphics[width=0.48\textwidth]{figs/diff_model_double.pdf}
 \caption{Average ConR and MemR across different models implemented by LLMs of Qwen series, before and after applying \method{}.
 }
 \label{fig:diff_model_double_qwen}
\end{figure}






We extend \method{} to a diverse range of LLMs, encompassing multiple model families and sizes. 

Specifically, our evaluation includes LLaMA3-8B-Instruct, LLaMA3.2-1B-Instruct, LLaMA3.2-3B-Instruct, Qwen2.5-0.5B-Instruct, Qwen2.5-1.5B-Instruct, Qwen2.5-3B-Instruct, Qwen2.5-7B-Instruct, and Qwen2.5-14B-Instruct. The results on ConR and MemR are summarized in Figures~\ref{fig:diff_model_double_llama} and \ref{fig:diff_model_double_qwen}, while Table~\ref{tab:append:all_model_res} presents the average performance of all models on \dataset{} and ConFiQA. Additionally, Table~\ref{tab:diff_model_param} provides detailed parameter information and specifies the layers selected for pruning for each model. This comprehensive evaluation demonstrates the versatility and scalability of \method{} across a wide spectrum of model architectures and sizes.

\begin{table}[!t]
  
    \resizebox{0.48\textwidth}{!}{%
\begin{tabular}{l|c|c|c}
\toprule
\textbf{Models}     & \textbf{Param.} & \textbf{\method{} Param.} & \textbf{Selected Layers} \\
\midrule
\rowcolor{gray!10}
LLaMA3.2-1B        & 1.24B  & 1.08B \small\textcolor{gray}{(87\%)}   & [12, 14]                 \\
LLaMA3.2-3B        & 3.21B  & 2.60B \small\textcolor{gray}{(81\%)}   &  [18, 25]   \\
\rowcolor{gray!10}
LLaMA3-8B          & 8.03B  & 6.97B \small\textcolor{gray}{(87\%)}   & [24, 29]      \\
LLaMA3.1-8B          & 8.03B  & 6.27B \small\textcolor{gray}{(78\%)}   & [20, 29]      \\
\rowcolor{gray!10}
Qwen2.5-0.5B         & 0.49B  & 0.44B \small\textcolor{gray}{(90\%)}   &  [19, 22]       \\
Qwen2.5-1.5B         & 1.54B  & 1.34B \small\textcolor{gray}{(87\%)}   & [21, 25]        \\
\rowcolor{gray!10}
Qwen2.5-3B         & 3.09B  & 2.68B \small\textcolor{gray}{(87\%)}   & [29, 34]        \\
Qwen2.5-7B         & 7.61B  & 7.21B \small\textcolor{gray}{(95\%)}   &   [25, 26 ]     \\
\rowcolor{gray!10}
Qwen2.5-14B        & 14.70B & 12.43B \small\textcolor{gray}{(85\%)}  &  [35, 45]   \\
\bottomrule
\end{tabular}
}

% \end{sidewaystable}

% \end{document}

  \caption{The total number of parameters for various models before and after applying \method{}. \textcolor{gray}{\small$(\cdot)\%$} represents the proportion relative to the original model, and the last column lists the layers selected for pruning.}
   \label{tab:diff_model_param}
\end{table}

These experimental results illustrate several key insights: 1) Larger models tend to rely more on parametric memory. As model size increases in both the LLaMA and Qwen families, MemR also grows, indicating a tendency to overlook external knowledge in favor of internal parameters. \method{} counteracts this behavior, decreasing larger models' MemR score to even below that of smaller models. 2) \method{} consistently benefits all evaluated models. Across both LLaMA and Qwen model families, \method{} outperforms Vanilla-KAG by boosting accuracy and context faithfulness, underscoring its broad applicability and effectiveness. 3) Not all parameters in KAG models are essential. Pruning parametric knowledge not only reduces computation costs but also fosters better generalization without sacrificing accuracy, highlighting the potential of building a parameter-efficient LLM within the KAG framework.




\begin{table*}[!t]
  
\centering
\resizebox{0.96\textwidth}{!}{%
\begin{tabular}{l|c|cccc|cccc}
\toprule
\multirow{2}{*}{\textbf{Models}} & \multirow{2}{*}{\textbf{Param.}} & \multicolumn{4}{c|}{\textbf{\dataset{}}} & \multicolumn{4}{c}{\textbf{ConFiQA}} \\ 
\cmidrule(lr){3-6}  \cmidrule(lr){7-10}
 &  & ConR $\uparrow$ & MemR $\downarrow$ & MR $\downarrow$ & EM $\uparrow$ & ConR $\uparrow$ & MemR $\downarrow$ & MR $\downarrow$ & EM $\uparrow$ \\ 
\midrule
LLaMA3-8B   & 8.03B  & 66.99  & 11.75  & 14.99  & 13.83  & 22.52  & 31.15  & 59.77  & 2.47 \\
\rowcolor{gray!10}
+\method{}    & 6.97B  & 71.50  & 6.48   & 8.41   & 66.19  & 70.43  & 8.82   & 11.32  & 67.29 \\
LLaMA3.1-8B & 8.03B  & 63.15  & 11.69  & 15.93  & 21.85  & 15.38  & 29.97  & 68.98  & 6.69 \\
\rowcolor{gray!10}
+\method{}   & 6.27B  & 70.41  & 6.95   & 9.17   & 63.58  & 71.12  & 9.01   & 11.44  & 66.61 \\
LLaMA3.2-1B & 1.24B  & 39.06  & 10.49  & 21.83  & 5.13   & 32.09  & 18.32  & 36.28  & 7.15 \\
\rowcolor{gray!10}
+\method{}   & 1.08B  & 51.75  & 6.51   & 11.34  & 47.60  & 62.70  & 7.63   & 11.38  & 61.85 \\
LLaMA3.2-3B & 3.21B  & 56.75  & 11.53  & 17.11  & 12.69  & 26.16  & 23.47  & 49.05  & 9.84 \\
\rowcolor{gray!10}
+\method{}   & 2.60B  & 67.00  & 6.80   & 9.35   & 61.59  & 69.61  & 8.39   & 11.09  & 66.53 \\
Qwen2.5-0.5B & 0.49B  & 47.17  & 11.36  & 19.48  & 2.06   & 50.72  & 17.15  & 26.20  & 3.78 \\
\rowcolor{gray!10}
+\method{}   & 0.44B  & 58.13  & 6.63   & 10.41  & 52.56  & 67.54  & 8.04   & 11.03  & 66.33 \\
Qwen2.5-1.5B & 1.54B  & 58.08  & 11.28  & 16.48  & 10.30  & 51.69  & 19.87  & 28.23  & 10.78 \\
\rowcolor{gray!10}
+\method{}   & 1.34B  & 63.78  & 6.74   & 9.76   & 57.67  & 69.61   & 8.35   & 11.05   & 66.04 \\
Qwen2.5-3B   & 3.09B  & 62.22  & 14.45  & 18.88  & 0.10   & 25.47  & 29.34  & 55.70  & 0.01 \\
\rowcolor{gray!10}
+\method{}     & 2.68B  & 66.31  & 6.75   & 9.38   & 59.42  & 66.30   & 8.62  & 11.94   & 63.03 \\
Qwen2.5-7B    & 7.61B  & 65.46  & 14.93  & 18.57  & 0.80   & 24.75  & 33.09  & 59.04  & 0.10 \\
\rowcolor{gray!10}
+\method{}      & 6.60B  & 67.75  & 6.60   & 9.01   & 61.77  & 69.54  & 8.85   & 11.58  & 66.68 \\
Qwen2.5-14B   & 14.70B & 65.75  & 16.13  & 19.75  & 0.00   & 7.86   & 32.88  & 83.71  & 0.01 \\
\rowcolor{gray!10}
+\method{}     & 12.43B & 70.01  & 6.43   & 8.55   & 64.43  & 71.70  & 8.90   & 11.29  & 68.40 \\
\bottomrule
\end{tabular}%
}


  \caption{Average performance of LLMs on \dataset{} and ConFiQA before and after applying \method{}.}
   \label{tab:append:all_model_res}
\end{table*}

\subsection{Neuron Activations in Different LLMs}\label{app:activation}
We present the neuron activations for the LLaMA family models, including LLaMA-3.2-1B-Instruct, LLaMA-3.2-3B-Instruct, LLaMA-3-8B-Instruct, and LLaMA-3.1-8B-Instruct, as well as the Qwen family models, including Qwen-2.5-0.5B-Instruct, Qwen-2.5-1.5B-Instruct, Qwen-2.5-3B-Instruct, Qwen-2.5-7B-Instruct, and Qwen-2.5-14B-Instruct, in Figures~\ref{fig:act_llama} and \ref{fig:act_qwen}, respectively. 
% 我们发现qwen系列模型


\begin{figure*}[t]
  \centering
  \subfigure[Neuron activations of LLaMA-3.2-1B-Instruct]{
        \label{fig:act_llama:3.2-1b}
        \includegraphics[width=0.9\linewidth]{append_fig/act_llama32_1b_all.pdf}
    }
\subfigure[Neuron activations of LLaMA-3.2-3B-Instruct]{
        \label{fig:act_llama:3.2-3b}
        \includegraphics[width=0.9\linewidth]{append_fig/act_llama32_3b_all.pdf}
    }
 \subfigure[Neuron activations of LLaMA-3-8B-Instruct]{
        \label{fig:act_llama:3-8b}
        \includegraphics[width=0.9\linewidth]{append_fig/act_llama_3_8b.pdf}
    }
 \subfigure[Neuron activations of LLaMA-3.1-8B-Instruct]{
        \label{fig:act_llama:3.1-8b}
        \includegraphics[width=0.9\linewidth]{append_fig/act_llama_31_8b.pdf}
    }
 

 \caption{Neuron activations across different layers of the LLaMA series models. We present the inhibition ratio $\Delta R$ under two conditions: with contextual knowledge input (w/ context) and without it (w/o context).}
 \label{fig:act_llama}
\end{figure*}

\begin{figure*}[t]
  \centering
  \subfigure[Neuron activations of Qwen-2.5-0.5B-Instruct]{
        \label{fig:act_qwen:2.5-0.5b}
        \includegraphics[width=0.75\linewidth]{append_fig/act_qwen25_0_5b_all.pdf}
    }
\subfigure[Neuron activations of Qwen-2.5-1.5B-Instruct]{
        \label{fig:act_qwen:2.5-1.5b}
        \includegraphics[width=0.75\linewidth]{append_fig/act_qwen25_1_5b_all.pdf}
    }
\subfigure[Neuron activations of Qwen-2.5-3B-Instruct]{
        \label{fig:act_qwen:2.5-3b}
        \includegraphics[width=0.75\linewidth]{append_fig/act_qwen25_3b_all.pdf}
    }
\subfigure[Neuron activations of Qwen-2.5-7B-Instruct]{
        \label{fig:act_qwen:2.5-7b}
        \includegraphics[width=0.75\linewidth]{append_fig/act_qwen25_7b_all.pdf}
    }
\subfigure[Neuron activations of Qwen-2.5-14B-Instruct]{
        \label{fig:act_qwen:2.5-14b}
        \includegraphics[width=0.75\linewidth]{append_fig/act_qwen25_14b_all.pdf}
    }


 \caption{Neuron activations across different layers of the Qwen series models. We present the inhibition ratio $\Delta R$ under two conditions: with contextual knowledge input (w/ context) and without it (w/o context). }
 \label{fig:act_qwen}
\end{figure*}

%%%%%%%%%%%%%%%%%%%%%%%%%%%%%%%%%%%%%%%%%%%%%%%%%%%%%%%%%%%%%%%%%%%%%%%%%%%%%%%%%%%%%%%%%%%%%%
\end{document}
