\documentclass[a4paper]{article} % 一般的なスタイルの書き方
\usepackage[margin=1in]{geometry}
\renewcommand{\baselinestretch}{1.3}


\usepackage{microtype}
\usepackage{graphicx}
% \usepackage{subfigure}
\usepackage{subcaption}
\usepackage{booktabs} % for professional tables
\usepackage{here}
% \usepackage{authblk}

% \usepackage{CJKutf8}
% \usepackage[whole]{bxcjkjatype}

% Recommended, but optional, packages for figures and better typesetting:

\usepackage{hyperref}

% Attempt to make hyperref and algorithmic work together better:
\newcommand{\theHalgorithm}{\arabic{algorithm}}

%%%%%%%%%%%%%%%%%%%%%%%%%%%%%%%%%%%%%%%%%%%%%%%%%%%%%%%%%%%%%%%%%%%%%%%%%%%%%%%%%%%%%%%%%%%%%%
\usepackage{amsmath,amsthm,amsfonts,amssymb}
\usepackage{mathtools}
\usepackage{xcolor}
\usepackage{natbib}
\usepackage{algorithm} % for algorithm
\usepackage{algpseudocode} % for algorithm
\usepackage{url}
\usepackage{multirow}
% \usepackage{hyperref} 

% \usepackage{subcaption}
% \usepackage{booktabs} % for professional tables

\usepackage[T1]{fontenc}
\usepackage[utf8]{inputenc}
\usepackage{bbm}
% \usepackage[whole]{bxcjkjatype}

\newcommand{\red}[1]{\textcolor{red}{#1}}
\newcommand{\blue}[1]{\textcolor{blue}{#1}}
\newcommand{\magenta}[1]{\textcolor{magenta}{#1}}

\newcommand{\mybecause}[1]{&& \bigl(\because #1 \bigr)}
\newcommand{\indep}{\mathop{\perp\!\!\!\!\perp}}
\newcommand{\disteq}{\overset{\mathrm{d}}{=}}


%
\setlength\unitlength{1mm}
\newcommand{\twodots}{\mathinner {\ldotp \ldotp}}
% bb font symbols
\newcommand{\Rho}{\mathrm{P}}
\newcommand{\Tau}{\mathrm{T}}

\newfont{\bbb}{msbm10 scaled 700}
\newcommand{\CCC}{\mbox{\bbb C}}

\newfont{\bb}{msbm10 scaled 1100}
\newcommand{\CC}{\mbox{\bb C}}
\newcommand{\PP}{\mbox{\bb P}}
\newcommand{\RR}{\mbox{\bb R}}
\newcommand{\QQ}{\mbox{\bb Q}}
\newcommand{\ZZ}{\mbox{\bb Z}}
\newcommand{\FF}{\mbox{\bb F}}
\newcommand{\GG}{\mbox{\bb G}}
\newcommand{\EE}{\mbox{\bb E}}
\newcommand{\NN}{\mbox{\bb N}}
\newcommand{\KK}{\mbox{\bb K}}
\newcommand{\HH}{\mbox{\bb H}}
\newcommand{\SSS}{\mbox{\bb S}}
\newcommand{\UU}{\mbox{\bb U}}
\newcommand{\VV}{\mbox{\bb V}}


\newcommand{\yy}{\mathbbm{y}}
\newcommand{\xx}{\mathbbm{x}}
\newcommand{\zz}{\mathbbm{z}}
\newcommand{\sss}{\mathbbm{s}}
\newcommand{\rr}{\mathbbm{r}}
\newcommand{\pp}{\mathbbm{p}}
\newcommand{\qq}{\mathbbm{q}}
\newcommand{\ww}{\mathbbm{w}}
\newcommand{\hh}{\mathbbm{h}}
\newcommand{\vvv}{\mathbbm{v}}

% Vectors

\newcommand{\av}{{\bf a}}
\newcommand{\bv}{{\bf b}}
\newcommand{\cv}{{\bf c}}
\newcommand{\dv}{{\bf d}}
\newcommand{\ev}{{\bf e}}
\newcommand{\fv}{{\bf f}}
\newcommand{\gv}{{\bf g}}
\newcommand{\hv}{{\bf h}}
\newcommand{\iv}{{\bf i}}
\newcommand{\jv}{{\bf j}}
\newcommand{\kv}{{\bf k}}
\newcommand{\lv}{{\bf l}}
\newcommand{\mv}{{\bf m}}
\newcommand{\nv}{{\bf n}}
\newcommand{\ov}{{\bf o}}
\newcommand{\pv}{{\bf p}}
\newcommand{\qv}{{\bf q}}
\newcommand{\rv}{{\bf r}}
\newcommand{\sv}{{\bf s}}
\newcommand{\tv}{{\bf t}}
\newcommand{\uv}{{\bf u}}
\newcommand{\wv}{{\bf w}}
\newcommand{\vv}{{\bf v}}
\newcommand{\xv}{{\bf x}}
\newcommand{\yv}{{\bf y}}
\newcommand{\zv}{{\bf z}}
\newcommand{\zerov}{{\bf 0}}
\newcommand{\onev}{{\bf 1}}

% Matrices

\newcommand{\Am}{{\bf A}}
\newcommand{\Bm}{{\bf B}}
\newcommand{\Cm}{{\bf C}}
\newcommand{\Dm}{{\bf D}}
\newcommand{\Em}{{\bf E}}
\newcommand{\Fm}{{\bf F}}
\newcommand{\Gm}{{\bf G}}
\newcommand{\Hm}{{\bf H}}
\newcommand{\Id}{{\bf I}}
\newcommand{\Jm}{{\bf J}}
\newcommand{\Km}{{\bf K}}
\newcommand{\Lm}{{\bf L}}
\newcommand{\Mm}{{\bf M}}
\newcommand{\Nm}{{\bf N}}
\newcommand{\Om}{{\bf O}}
\newcommand{\Pm}{{\bf P}}
\newcommand{\Qm}{{\bf Q}}
\newcommand{\Rm}{{\bf R}}
\newcommand{\Sm}{{\bf S}}
\newcommand{\Tm}{{\bf T}}
\newcommand{\Um}{{\bf U}}
\newcommand{\Wm}{{\bf W}}
\newcommand{\Vm}{{\bf V}}
\newcommand{\Xm}{{\bf X}}
\newcommand{\Ym}{{\bf Y}}
\newcommand{\Zm}{{\bf Z}}

% Calligraphic

\newcommand{\Ac}{{\cal A}}
\newcommand{\Bc}{{\cal B}}
\newcommand{\Cc}{{\cal C}}
\newcommand{\Dc}{{\cal D}}
\newcommand{\Ec}{{\cal E}}
\newcommand{\Fc}{{\cal F}}
\newcommand{\Gc}{{\cal G}}
\newcommand{\Hc}{{\cal H}}
\newcommand{\Ic}{{\cal I}}
\newcommand{\Jc}{{\cal J}}
\newcommand{\Kc}{{\cal K}}
\newcommand{\Lc}{{\cal L}}
\newcommand{\Mc}{{\cal M}}
\newcommand{\Nc}{{\cal N}}
\newcommand{\nc}{{\cal n}}
\newcommand{\Oc}{{\cal O}}
\newcommand{\Pc}{{\cal P}}
\newcommand{\Qc}{{\cal Q}}
\newcommand{\Rc}{{\cal R}}
\newcommand{\Sc}{{\cal S}}
\newcommand{\Tc}{{\cal T}}
\newcommand{\Uc}{{\cal U}}
\newcommand{\Wc}{{\cal W}}
\newcommand{\Vc}{{\cal V}}
\newcommand{\Xc}{{\cal X}}
\newcommand{\Yc}{{\cal Y}}
\newcommand{\Zc}{{\cal Z}}

% Bold greek letters

\newcommand{\alphav}{\hbox{\boldmath$\alpha$}}
\newcommand{\betav}{\hbox{\boldmath$\beta$}}
\newcommand{\gammav}{\hbox{\boldmath$\gamma$}}
\newcommand{\deltav}{\hbox{\boldmath$\delta$}}
\newcommand{\etav}{\hbox{\boldmath$\eta$}}
\newcommand{\lambdav}{\hbox{\boldmath$\lambda$}}
\newcommand{\epsilonv}{\hbox{\boldmath$\epsilon$}}
\newcommand{\nuv}{\hbox{\boldmath$\nu$}}
\newcommand{\muv}{\hbox{\boldmath$\mu$}}
\newcommand{\zetav}{\hbox{\boldmath$\zeta$}}
\newcommand{\phiv}{\hbox{\boldmath$\phi$}}
\newcommand{\psiv}{\hbox{\boldmath$\psi$}}
\newcommand{\thetav}{\hbox{\boldmath$\theta$}}
\newcommand{\tauv}{\hbox{\boldmath$\tau$}}
\newcommand{\omegav}{\hbox{\boldmath$\omega$}}
\newcommand{\xiv}{\hbox{\boldmath$\xi$}}
\newcommand{\sigmav}{\hbox{\boldmath$\sigma$}}
\newcommand{\piv}{\hbox{\boldmath$\pi$}}
\newcommand{\rhov}{\hbox{\boldmath$\rho$}}
\newcommand{\upsilonv}{\hbox{\boldmath$\upsilon$}}

\newcommand{\Gammam}{\hbox{\boldmath$\Gamma$}}
\newcommand{\Lambdam}{\hbox{\boldmath$\Lambda$}}
\newcommand{\Deltam}{\hbox{\boldmath$\Delta$}}
\newcommand{\Sigmam}{\hbox{\boldmath$\Sigma$}}
\newcommand{\Phim}{\hbox{\boldmath$\Phi$}}
\newcommand{\Pim}{\hbox{\boldmath$\Pi$}}
\newcommand{\Psim}{\hbox{\boldmath$\Psi$}}
\newcommand{\Thetam}{\hbox{\boldmath$\Theta$}}
\newcommand{\Omegam}{\hbox{\boldmath$\Omega$}}
\newcommand{\Xim}{\hbox{\boldmath$\Xi$}}


% Sans Serif small case

\newcommand{\Gsf}{{\sf G}}

\newcommand{\asf}{{\sf a}}
\newcommand{\bsf}{{\sf b}}
\newcommand{\csf}{{\sf c}}
\newcommand{\dsf}{{\sf d}}
\newcommand{\esf}{{\sf e}}
\newcommand{\fsf}{{\sf f}}
\newcommand{\gsf}{{\sf g}}
\newcommand{\hsf}{{\sf h}}
\newcommand{\isf}{{\sf i}}
\newcommand{\jsf}{{\sf j}}
\newcommand{\ksf}{{\sf k}}
\newcommand{\lsf}{{\sf l}}
\newcommand{\msf}{{\sf m}}
\newcommand{\nsf}{{\sf n}}
\newcommand{\osf}{{\sf o}}
\newcommand{\psf}{{\sf p}}
\newcommand{\qsf}{{\sf q}}
\newcommand{\rsf}{{\sf r}}
\newcommand{\ssf}{{\sf s}}
\newcommand{\tsf}{{\sf t}}
\newcommand{\usf}{{\sf u}}
\newcommand{\wsf}{{\sf w}}
\newcommand{\vsf}{{\sf v}}
\newcommand{\xsf}{{\sf x}}
\newcommand{\ysf}{{\sf y}}
\newcommand{\zsf}{{\sf z}}


% mixed symbols

\newcommand{\sinc}{{\hbox{sinc}}}
\newcommand{\diag}{{\hbox{diag}}}
\renewcommand{\det}{{\hbox{det}}}
\newcommand{\trace}{{\hbox{tr}}}
\newcommand{\sign}{{\hbox{sign}}}
\renewcommand{\arg}{{\hbox{arg}}}
\newcommand{\var}{{\hbox{var}}}
\newcommand{\cov}{{\hbox{cov}}}
\newcommand{\Ei}{{\rm E}_{\rm i}}
\renewcommand{\Re}{{\rm Re}}
\renewcommand{\Im}{{\rm Im}}
\newcommand{\eqdef}{\stackrel{\Delta}{=}}
\newcommand{\defines}{{\,\,\stackrel{\scriptscriptstyle \bigtriangleup}{=}\,\,}}
\newcommand{\<}{\left\langle}
\renewcommand{\>}{\right\rangle}
\newcommand{\herm}{{\sf H}}
\newcommand{\trasp}{{\sf T}}
\newcommand{\transp}{{\sf T}}
\renewcommand{\vec}{{\rm vec}}
\newcommand{\Psf}{{\sf P}}
\newcommand{\SINR}{{\sf SINR}}
\newcommand{\SNR}{{\sf SNR}}
\newcommand{\MMSE}{{\sf MMSE}}
\newcommand{\REF}{{\RED [REF]}}

% Markov chain
\usepackage{stmaryrd} % for \mkv 
\newcommand{\mkv}{-\!\!\!\!\minuso\!\!\!\!-}

% Colors

\newcommand{\RED}{\color[rgb]{1.00,0.10,0.10}}
\newcommand{\BLUE}{\color[rgb]{0,0,0.90}}
\newcommand{\GREEN}{\color[rgb]{0,0.80,0.20}}

%%%%%%%%%%%%%%%%%%%%%%%%%%%%%%%%%%%%%%%%%%
\usepackage{hyperref}
\hypersetup{
    bookmarks=true,         % show bookmarks bar?
    unicode=false,          % non-Latin characters in AcrobatÕs bookmarks
    pdftoolbar=true,        % show AcrobatÕs toolbar?
    pdfmenubar=true,        % show AcrobatÕs menu?
    pdffitwindow=false,     % window fit to page when opened
    pdfstartview={FitH},    % fits the width of the page to the window
%    pdftitle={My title},    % title
%    pdfauthor={Author},     % author
%    pdfsubject={Subject},   % subject of the document
%    pdfcreator={Creator},   % creator of the document
%    pdfproducer={Producer}, % producer of the document
%    pdfkeywords={keyword1} {key2} {key3}, % list of keywords
    pdfnewwindow=true,      % links in new window
    colorlinks=true,       % false: boxed links; true: colored links
    linkcolor=red,          % color of internal links (change box color with linkbordercolor)
    citecolor=green,        % color of links to bibliography
    filecolor=blue,      % color of file links
    urlcolor=blue           % color of external links
}
%%%%%%%%%%%%%%%%%%%%%%%%%%%%%%%%%%%%%%%%%%%



% show equation number only if referenced
\mathtoolsset{showonlyrefs}
% \renewcommand{\arraystretch}{1.2}
%%%%%%%%%%%%%%%%%%%%%%%%%%%%%%%%%%%%%%%%%%%%%%%%%%%%%%%%%%%%%%%%%%%%%%%%%%%%%%%%%%%%%%%%%%%%%%
\usepackage{authblk}

\title{Distributionally Robust Active Learning \\for Gaussian Process Regression}
\date{}

\author[1,2]{Shion Takeno}

\author[1]{Yoshito Okura}
\author[3]{Yu Inatsu}
\author[1]{Aoyama Tatsuya}
\author[1]{Tomonari Tanaka}
\author[1]{Akahane Satoshi}
\author[2]{Hiroyuki Hanada}
\author[2]{Noriaki Hashimoto}
\author[4]{Taro Murayama}
\author[4]{Hanju Lee}
\author[4]{Shinya Kojima}
\author[1,2]{Ichiro Takeuchi}

\affil[1]{Nagoya University}
\affil[2]{RIKEN AIP}
\affil[3]{Nagoya Institute of Technology}
\affil[4]{DENSO CORPORATION}

\affil[ ]{\texttt{{takeno.s.mllab.nit@gmail.com}}}
\affil[ ]{\texttt{{takeuchi.ichiro.n6@f.mail.nagoya-u.ac.jp}}}

\begin{document}
\maketitle


\begin{abstract}
     Gaussian process regression (GPR) or kernel ridge regression is a widely used and powerful tool for nonlinear prediction.
     %
     Therefore, active learning (AL) for GPR, which actively collects data labels to achieve an accurate prediction with fewer data labels, is an important problem.
     %
     However, existing AL methods do not theoretically guarantee prediction accuracy for target distribution.
     %
     Furthermore, as discussed in the distributionally robust learning literature, specifying the target distribution is often difficult.
     %
     Thus, this paper proposes two AL methods that effectively reduce the worst-case expected error for GPR, which is the worst-case expectation in target distribution candidates.
     %
     We show an upper bound of the worst-case expected squared error, which suggests that the error will be arbitrarily small by a finite number of data labels under mild conditions.
     %
     Finally, we demonstrate the effectiveness of the proposed methods through synthetic and real-world datasets.
\end{abstract}

% Gaussian process regression (GPR) or kernel ridge regression is a widely used and powerful tool for nonlinear prediction. Therefore, active learning (AL) for GPR, which actively collects data labels to achieve an accurate prediction with fewer data labels, is an important problem. However, existing AL methods do not theoretically guarantee prediction accuracy for target distribution. Furthermore, as discussed in the distributionally robust learning literature, specifying the target distribution is often difficult. Thus, this paper proposes two AL methods that effectively reduce the worst-case expected error for GPR, which is the worst-case expectation in target distribution candidates. We show an upper bound of the worst-case expected squared error, which suggests that the error will be arbitrarily small by a finite number of data labels under mild conditions. Finally, we demonstrate the effectiveness of the proposed methods through synthetic and real-world datasets.


% Gaussian process regression (GPR) or kernel ridge regression is a widely used and powerful tool for nonlinear prediction. Therefore, active learning for GPR, which actively collects data labels to achieve an accurate prediction with fewer data labels, is an important problem. However, existing approaches do not theoretically guarantee prediction accuracy for target distribution. Furthermore, as discussed in the distributionally robust learning literature, specifying the target distribution is often difficult. Thus, this paper proposes two active learning methods that effectively reduce the worst-case target prediction error for GPR, where the worst-case is taken with the target distribution candidates. We show an information-theoretic upper bound of the worst-case expected squared error, which suggests that under mild conditions, the error will be arbitrarily small within a finite number of data labels. Finally, we demonstrate the effectiveness of the proposed methods through synthetic and real-world datasets.

% The first version
% Active learning is a framework for actively collecting data labels to achieve an accurate prediction with fewer data labels under some statistical model. However, many existing approaches do not theoretically guarantee prediction accuracy for target distribution. In addition, as discussed in the literature on distributionally robust learning, specifying the target distribution is often difficult. This paper proposes two active learning methods that effectively reduce the worst-case target prediction error when we use the Gaussian process regression, a widely used kernel regression model. We show an information-theoretic upper bound for the worst-case expected squared error, where the worst-case is taken with the target distribution candidates. This upper bound shows that the worst-case prediction error will be arbitrarily small with a finite number of data labels under mild conditions. Finally, we demonstrate the effectiveness of the proposed methods through synthetic and real-world datasets.

%%%%%%%%%%%%%%%%%%%%%%%%%%%%%%%%%%%%%%%%%%%%%%%%%%%%%%%%%%%%%%%%%%%%%%%%%%%%%%%%%%%%%%%%

\section{Introduction}
Backdoor attacks pose a concealed yet profound security risk to machine learning (ML) models, for which the adversaries can inject a stealth backdoor into the model during training, enabling them to illicitly control the model's output upon encountering predefined inputs. These attacks can even occur without the knowledge of developers or end-users, thereby undermining the trust in ML systems. As ML becomes more deeply embedded in critical sectors like finance, healthcare, and autonomous driving \citep{he2016deep, liu2020computing, tournier2019mrtrix3, adjabi2020past}, the potential damage from backdoor attacks grows, underscoring the emergency for developing robust defense mechanisms against backdoor attacks.

To address the threat of backdoor attacks, researchers have developed a variety of strategies \cite{liu2018fine,wu2021adversarial,wang2019neural,zeng2022adversarial,zhu2023neural,Zhu_2023_ICCV, wei2024shared,wei2024d3}, aimed at purifying backdoors within victim models. These methods are designed to integrate with current deployment workflows seamlessly and have demonstrated significant success in mitigating the effects of backdoor triggers \cite{wubackdoorbench, wu2023defenses, wu2024backdoorbench,dunnett2024countering}.  However, most state-of-the-art (SOTA) backdoor purification methods operate under the assumption that a small clean dataset, often referred to as \textbf{auxiliary dataset}, is available for purification. Such an assumption poses practical challenges, especially in scenarios where data is scarce. To tackle this challenge, efforts have been made to reduce the size of the required auxiliary dataset~\cite{chai2022oneshot,li2023reconstructive, Zhu_2023_ICCV} and even explore dataset-free purification techniques~\cite{zheng2022data,hong2023revisiting,lin2024fusing}. Although these approaches offer some improvements, recent evaluations \cite{dunnett2024countering, wu2024backdoorbench} continue to highlight the importance of sufficient auxiliary data for achieving robust defenses against backdoor attacks.

While significant progress has been made in reducing the size of auxiliary datasets, an equally critical yet underexplored question remains: \emph{how does the nature of the auxiliary dataset affect purification effectiveness?} In  real-world  applications, auxiliary datasets can vary widely, encompassing in-distribution data, synthetic data, or external data from different sources. Understanding how each type of auxiliary dataset influences the purification effectiveness is vital for selecting or constructing the most suitable auxiliary dataset and the corresponding technique. For instance, when multiple datasets are available, understanding how different datasets contribute to purification can guide defenders in selecting or crafting the most appropriate dataset. Conversely, when only limited auxiliary data is accessible, knowing which purification technique works best under those constraints is critical. Therefore, there is an urgent need for a thorough investigation into the impact of auxiliary datasets on purification effectiveness to guide defenders in  enhancing the security of ML systems. 

In this paper, we systematically investigate the critical role of auxiliary datasets in backdoor purification, aiming to bridge the gap between idealized and practical purification scenarios.  Specifically, we first construct a diverse set of auxiliary datasets to emulate real-world conditions, as summarized in Table~\ref{overall}. These datasets include in-distribution data, synthetic data, and external data from other sources. Through an evaluation of SOTA backdoor purification methods across these datasets, we uncover several critical insights: \textbf{1)} In-distribution datasets, particularly those carefully filtered from the original training data of the victim model, effectively preserve the model’s utility for its intended tasks but may fall short in eliminating backdoors. \textbf{2)} Incorporating OOD datasets can help the model forget backdoors but also bring the risk of forgetting critical learned knowledge, significantly degrading its overall performance. Building on these findings, we propose Guided Input Calibration (GIC), a novel technique that enhances backdoor purification by adaptively transforming auxiliary data to better align with the victim model’s learned representations. By leveraging the victim model itself to guide this transformation, GIC optimizes the purification process, striking a balance between preserving model utility and mitigating backdoor threats. Extensive experiments demonstrate that GIC significantly improves the effectiveness of backdoor purification across diverse auxiliary datasets, providing a practical and robust defense solution.

Our main contributions are threefold:
\textbf{1) Impact analysis of auxiliary datasets:} We take the \textbf{first step}  in systematically investigating how different types of auxiliary datasets influence backdoor purification effectiveness. Our findings provide novel insights and serve as a foundation for future research on optimizing dataset selection and construction for enhanced backdoor defense.
%
\textbf{2) Compilation and evaluation of diverse auxiliary datasets:}  We have compiled and rigorously evaluated a diverse set of auxiliary datasets using SOTA purification methods, making our datasets and code publicly available to facilitate and support future research on practical backdoor defense strategies.
%
\textbf{3) Introduction of GIC:} We introduce GIC, the \textbf{first} dedicated solution designed to align auxiliary datasets with the model’s learned representations, significantly enhancing backdoor mitigation across various dataset types. Our approach sets a new benchmark for practical and effective backdoor defense.



\section{Preliminary} \label{sec:preliminary}
\paragraph{Random variable and distribution.} Let $\mathcal{X} = \mathcal{X}_v \times \mathcal{X}_t$ denote the input space, where $\mathcal{X}_v$ and $\mathcal{X}_t$ correspond to the visual and textual feature spaces, respectively. Similarly, let $\mathcal{Y}$ denote the response space. We define the random variables $\mathbf{X} = (X_v, X_t) \in \mathcal{X}$ and $Y \in \mathcal{Y}$, where $\mathbf{X}$ is the sequence of tokens that combine visual and text input queries, and $Y$ represents the associated response tokens. The joint population is denoted by $P_{\mathbf{X}Y}$, with marginals $P_{\mathbf{X}}$, $P_{Y}$, and the conditional distribution $P_{Y|\mathbf{X}}$. For subsequent sections, $P_{\mathbf{X}Y}$ refers to the instruction tuning data distribution which we consider as in-distribution (ID). 

\paragraph{MLLM and visual instruction tuning.} MLLM usually consists of three components: (1) a visual encoder, (2) a vision-to-language projector, and (3) an LLM that processes a multimodal input sequence to generate a valid textual output $y$ in response to an input query $\mathbf{x}$. An MLLM can be regarded as modeling a conditional distribution $P_{\theta}(y|\mathbf{x})$, where $\theta$ is the model parameters. To attain the multimodal conversation capability, MLLMs commonly undergo a phase so-called \textit{visual instruction tuning} \cite{liu2023visual, dai2023instructblip} with an autoregressive objective as follows:
{
\begin{align} \label{eq::1}
    % & \min_{\theta\in\Theta} \mathbb{E}_{\mathbf{x},y\sim P_{\mathbf{X}Y}} [-\log P_{\theta}(y|\mathbf{x})] \nonumber \\
     \min_{\theta\in\Theta} \mathbb{E}_{\mathbf{x},y\sim P_{\mathbf{X}Y}} [\sum_{l=0}^{L}-\log P_{\theta}(y_{l}|\mathbf{x},y_{<l})],
\end{align}}
where $L$ is a sequence length and $y=(y_{0},...,y_{L})$. After being trained by Eq. \eqref{eq::1}, MLLM produces a response given a query of any possible tasks represented by text.

\paragraph{Evaluation of open-ended generations.} 

(M)LLM-as-a-judge method \cite{zheng2023judging, kim2023prometheus} is commonly adopted to evaluate open-ended generation. In this paradigm, a judge model produces preference scores or rankings for the responses given a query, model responses, and a scoring rubric. Among the evaluation metrics, the \emph{win rate} (Eq. \eqref{eq:win_rate}) is one of the most widely used and representative.


\begin{definition}[\textbf{Win Rate}] Given a parametric reward function $r:\mathcal{X}\times \mathcal{Y}\rightarrow \mathbb{R}$, the 
win rate (WR) of model $P_{\theta}$ w.r.t. $P_{\mathbf{X}Y}$ are defined as follows: 
\begin{equation} \label{eq:win_rate}
\begin{split}
    &\text{WR}(P_{\mathbf{X}Y};\theta):=\mathbb{E}_{\begin{subarray}{l} \mathbf{x},y \sim P_{\mathbf{X}Y} \\ \hat{y} \sim P_{\theta}(\cdot|\mathbf{x}) \end{subarray}}[\mathbb{I}(r(\mathbf{x},\hat{y}) > r(\mathbf{x},y))],
\end{split}
\end{equation}
where $\mathbb{I}(\cdot)$ is  the indicator function.
\end{definition}
Here, the reward function $r(\cdot,\cdot)$, can be any possible (multimodal) LLMs such as GPT-4o \cite{hurst2024gpt}.


\section{Problem Statements and Its Property}
\label{sec:problem}

This section provides our problem setup and its properties.

%%%%%%%%%%%%%%%%%%%%%%%%%%%%%%%%%%%%%%%%%%%%%%%%%%%%%%%%%%%%%%%%%%%%%%%%%%%%
\subsection{Problem Statement}

We aim to minimize the worst-case expected errors regarding the GP prediction $\mu_T (\*x)$ after $T$-th function evaluations:
\begin{align}
    % {\rm DRAE}_T &\coloneqq \max_{p \in \cP} \EE_{p(\*x)} \left[ | f(\*x) - \mu_T(\*x) | \right] \\
    E_T &\coloneqq \max_{p \in \cP} \EE_{p(\*x^{*})} \left[ ( f(\*x^{*}) - \mu_T(\*x^{*}) )^2 \right],
    \label{eq:target_error}
\end{align}
where $\cP$ is a set of target distributions over the input space $\cX$ called ambiguity set~\citep{chen2020distributionally}.
%
We assume that $\max_{p \in \cP} \EE_{p(\*x^*)} \left[ g(\*x^*) \right]$ exists for any continuous function $g: \cX \rightarrow \RR$.
%
This paper concentrates on the setting where the training input space from which we can obtain labels includes the test input space.



Our problem setup can be seen as the generalization of the target distribution-aware AL and the AL for the worst-case error $\max_{\*x \in \cX} ( f(\*x) - \mu_T(\*x) )^2$.
%
This is because our problem is equivalent to the target distribution-aware AL if we set $|\cP| = 1$ and to the worst-case error minimization if $\cP$ includes $\{p \in \cP_{\cX} \mid \exists \*x \in \cX, p(\*x) = 1 \}$, where $\cP_{\rm \cX}$ is the set of the distributions over $\cX$.


%%%%%%%%%%%%%%%%%%%%%%%%%%%%%%%%%%%%%%%%%%%%%%%%%%%%%%%%%%%%%%%%%%%%%%%%%%%%
\subsection{High Probability Bound of Error}

% First, we provide the upper bound by the posterior variance.
%
If the input space $\cX$ is finite, we can obtain the upper bound of Eq.~\eqref{eq:target_error} as the direct consequence of Lemmas~\ref{lem:bound_srinivas} and \ref{lem:bound_vakili}:
\begin{lemma}
    Fix $\delta \in (0, 1)$ and $T \in \NN$.
    %
    Suppose that Assumption~\ref{assump:Bayesian} holds and $\beta_\delta$ is set as in Lemma~\ref{lem:bound_srinivas}, or Assumption~\ref{assump:frequentist} holds and $\beta_\delta$ is set as in Lemma~\ref{lem:bound_vakili}.
    %
    Then, the following holds with probability at least $1 - \delta$:
    \begin{align*}
        E_T &\leq \beta_{\delta} \max_{p \in \cP} \EE_{p(\*x^{*})}\left[ \sigma^2_{T}(\*x^{*}) \right].
    \end{align*}
    \label{lem:UB_error_discrete}
\end{lemma}


% Next, let us consider the case that $\cX = [0, r]^d$.
%
For continuous $\cX$, the confidence parameter $\beta_\delta \propto \log |\cX|$ diverges if we apply Lemmas~\ref{lem:bound_srinivas} and \ref{lem:bound_vakili} directly.
%
Therefore, in this case, the Lipschitz property is often leveraged~\citep{Chowdhury2017-on,vakili2021-optimal}.
%
The Lipschitz constant of $f$ can be directly derived from the Assumption~\ref{assump:Bayesian_continuous}, or Assumption~\ref{assump:frequentist_continuous} and Lemma~\ref{lem:RKHS_lipschitz}~\citep{Srinivas2010-Gaussian,freitas2012exponential}.


Furthermore, we need the Lipschitz constant of $\mu_T$.
%
In the frequentist setting, the Lipschitz constant for $\mu_T$ can be derived as $\cO(L_k \sqrt{t \log t})$ by Lemma~4 in \citet{vakili2021-optimal} and Lemma~\ref{lem:RKHS_lipschitz}.
%
To obtain a slightly tighter upper bound, we show the following lemma:
% \begin{lemma}[Modified from Lemma~F.1 of \citet{vakili2022improved}]
%     Fix $\delta \in (0, 1)$ and $t \in [T]$.
%     %
%     Suppose that Assumptions~\ref{assump:frequentist} and ~\ref{assump:frequentist_continuous} hold.
%     %
%     Then, the RKHS norm of $\mu_t(\cdot)$ satisfies the following with probability at least $1 - \delta$:
%     \begin{align*}
%         \| \mu_t \|_{\cH_k} \leq B + \frac{R}{\sigma} \sqrt{ 2t \log \left( \frac{2t}{\delta} \right)}.
%     \end{align*}
%     %
%     Thus, $\mu_T$ is $L_k \bigl( B + \frac{R}{\sigma} \sqrt{ 2t \log \left( 2t / \delta \right)} \bigr)$ Lipschitz continuous.
%     \label{lem:RKHS_norm_posterior_mean}
% \end{lemma}
\begin{lemma}
    Fix $\delta \in (0, 1)$ and $t \in [T]$.
    %
    Suppose that Assumptions~\ref{assump:frequentist} and ~\ref{assump:frequentist_continuous} hold.
    %
    Then, $\mu_t(\cdot)$ is Lipschitz continuous with the Lipschitz constant,
    \begin{align*}
        L_k \left( B + \frac{R}{\sigma} \sqrt{ 2 \gamma_t + 2 \log \left( \frac{d}{\delta} \right)} \right)
    \end{align*}
    with probability at least $1 - \delta$.
    \label{lem:lipschitz_posterior_mean}
\end{lemma}
We show the proof in Appendix~\ref{sec:proof_lipschitz_posterior_mean}.
%
Since the MIG $\gamma_T$ is sublinear for the kernels on which we mainly focus, the upper bound $\cO(L_k \sqrt{\gamma_t})$ is tighter than $\cO(L_k \sqrt{t \log t})$.



In the Bayesian setting, the upper bound of the Lipschitz constant for $\mu_T$ has not been shown to our knowledge.
%
Therefore, we show the following lemma:
\begin{lemma}
    Fix $\delta \in (0, 1)$ and $t \in [T]$.
    %
    Suppose that Assumptions~\ref{assump:Bayesian} and \ref{assump:Bayesian_continuous} hold and the kernel has mixed partial derivative $\frac{\partial^2 k(\*x, \*z)}{ \partial x_j \partial z_j}$ for all $j \in [d]$.
    %
    Set $a$ and $b$ as in Lemma~\ref{assump:Bayesian_continuous}.
    %
    Assume that $(\*x_i)_{i \in [t]}$ is independent of $(\epsilon_i)_{i \in [t]}$ and $f$.
    %
    Then, $\mu_t$ and $r_t(\*x) \coloneqq f(\*x) - \mu_t(\*x)$ satisfies the following:
    \begin{align*}
        \Pr \left( \sup_{\*x \in \cX} \left| \frac{\partial \mu_t(\*u)}{\partial u_j} \Big|_{\*u = \*x} \right| > L \right) \leq 2a \exp \left( - \frac{L^2}{b^2} \right), \\
        \Pr \left( \sup_{\*x \in \cX} \left| \frac{\partial r_t(\*u)}{\partial u_j} \Big|_{\*u = \*x} \right| > L \right) \leq 2a \exp \left( - \frac{L^2}{b^2} \right), 
    \end{align*}
    for all $j \in [d]$.
    \label{lem:bayesian_lipschitz_posterior_mean}
\end{lemma}
See Appendix~\ref{sec:proof_bayesian_lipschitz_posterior_mean} for the proof, in which we leverage Slepian's inequality~\citep[Proposition~A.2.6 in][]{van1996weak} and the fact that the derivative of the sample path follows GP jointly when the kernel is differentiable.



By leveraging the above results, even if $\cX$ is continuous, we can obtain the following upper bound of Eq.~\eqref{eq:target_error}:
\begin{lemma}
    Suppose that Assumptions~\ref{assump:frequentist} and ~\ref{assump:frequentist_continuous} hold.
    %
    Fix $\delta \in (0, 1)$ and $T \in \NN$.
    %
    Then, the following holds with probability at least $1 - \delta$:
    \begin{align*}
        E_T 
        &\leq 2 \beta_{\delta, T} \max_{p \in \cP} \EE_{p(\*x^*)} \left[  \sigma_T^2(\*x^*) \right] 
        + \cO \left( \frac{\max\{\gamma_T, \log(\frac{T}{\delta})\}}{T^2} \right).
    \end{align*}
    where $\beta_{\delta, T} = \left( B + \frac{R}{\sigma} \sqrt{ 2 d \log \left( T d r + 1 \right) + 2 \log \left( \frac{4}{\delta} \right)} \right)^2$.
    \label{lem:UB_error_frequentist_continuous}
\end{lemma}
\begin{lemma}
    Suppose that Assumptions~\ref{assump:Bayesian} and \ref{assump:Bayesian_continuous} hold.
    %
    Fix $\delta \in (0, 1)$ and $T \in \NN$.
    %
    Then, the following holds with probability at least $1 - \delta$:
    \begin{align*}
        E_T 
        &\leq 2 \beta_{\delta, T} \max_{p \in \cP} \EE_{p(\*x^*)} \left[  \sigma_T^2(\*x^*) \right] 
        + \cO\left( \frac{\log(\frac{T}{\delta})}{T^2} \right),
    \end{align*}
    where $\beta_{\delta, T} = 2d \log (T d r + 1) + 2 \log (2 / \delta)$.
    \label{lem:UB_error_bayesian_continuous}
\end{lemma}
See Appendices~\ref{sec:proof_UB_error_frequentist_continuous} and ~\ref{sec:proof_UB_error_bayesian_continuous} for the proof.


Consequently, we can  minimize Eq.~\eqref{eq:target_error} by minimizing $\max_{p \in \cP} \EE_{p(\*x^{*})}\left[ \sigma^2_{T}(\*x^{*}) \right]$.
%
In this perspective, the US and RS are theoretically guaranteed because of $\max_{p \in \cP} \EE_{p(\*x^{*})}\left[ \sigma^2_{T}(\*x^{*}) \right] \leq \max_{\*x \in \cX} \sigma^2_T (\*x)$ and Proposition~\ref{prop:us_rs}.
%
However, the US and RS do not incorporate the information of $\cP$.
%
Therefore, the practical effectiveness of the US and RS is limited.
%
% Hence, we design algorithms that enjoy both a similar convergence guarantee and practical effectiveness incorporating the information of $\cP$.


\subsection{Other Performance Mesuares}

Although we mainly discuss the squared error, other measures can also be bounded from above:
\begin{lemma}
    The worst-case expected absolute error for any $T \in \NN$ is bounded from above as follows:
    \begin{align*}
        \max_{p \in \cP} \EE_{p(\*x^{*})} \left[ |f(\*x^{*}) - \mu_T(\*x^{*})| \right]
        \leq \sqrt{E_T},
        % &\coloneqq \max_{p \in \cP} \EE_{p(\*x^{*})} \left[ ( f(\*x^{*}) - \mu_T(\*x^{*}) )^2 \right]
    \end{align*}
    where $E_T$ is defined as in Eq.~\eqref{eq:target_error}.
    \label{lem:UB_absolute_error}
\end{lemma}
%
\begin{lemma}
    The worst-case expectation of entropy for any $T \in \NN$ is bounded from above as follows:
    \begin{align*}
        \max_{p \in \cP} \EE_{p(\*x^{*})} \left[ H\left[ f(\*x^*) \mid \cD_T \right] \right]
        % &= \max_{p \in \cP} \EE_{p(\*x^{*})} \left[ \frac{1}{2} \log \left(2 \pi e \sigma_T^2(\*x^*) \right) \right] \\
        &\leq \frac{1}{2} \log \left(2 \pi e \tilde{E}_T \right),
        % &\leq \frac{1}{2} \log \left(2 \pi e \max_{p \in \cP} \EE_{p(\*x^{*})} \left[ \sigma_T^2(\*x^*) \right] \right),
        % &= \cO\left( \log \left( \max_{p \in \cP} \EE_{p(\*x^{*})} \left[ \sigma_T^2(\*x^*) \right] \right)\right)
    \end{align*}
    where $\tilde{E}_T = \max_{p \in \cP} \EE_{p(\*x^{*})}\left[ \sigma^2_{T}(\*x^{*}) \right]$ and $H[f(\*x) \mid \cD_T] = \log \left(\sqrt{2 \pi e} \sigma_T(\*x) \right)$ is Shannon entropy.
    \label{lem:UB_entropy}
\end{lemma}
%
See Appendices~\ref{sec:UB_absolute_error_proof} and \ref{sec:UB_entropy_proof} for the proof.
%
Therefore, minimizing $\max_{p \in \cP} \EE_{p(\*x^{*})}\left[ \sigma^2_{T}(\*x^{*}) \right]$ also provides the convergence of the absolute error and the entropy\footnote{For the absolute error, we can design algorithms that directly reduce $\sigma_t$, not $\sigma_t^2$, and achieves the similar theoretical guarantee.}.


% \subsection{Discussion}

% Our problem setup can be seen as the generalization of the target distribution-aware AL and the AL for the worst-case error, i.e., $\max_{\*x \in \cX} ( f(\*x) - \mu_T(\*x) )^2$.
% %
% This is because our problem is equivalent to the target distribution-aware AL if we set $|\cP| = 1$ and to the worst-case error minimization if $\cP$ includes $\{p \in \cP_{\cX} \mid \exist \*x \in \cX, p(\*x) = 1 \}$, where $\cP_{\rm \cX}$ is the set of the distributions over $\cX$.
% %
% Clearly, for the worst-case analysis for $\max_{\*x \in \cX} ( f(\*x) - \mu_T(\*x) )^2$, we must use the method that reduce the largest variance $\max_{\*x \in \cX} \sigma_t^2(\*x)$.
% %
% This is satisfied by the US and RS, as shown in Proposition~\ref{prop:us_rs}.
\section{Proposed Methods and Analysis}
\label{sec:proposed}

We aim to design algorithms that enjoy both a similar convergence guarantee as the US and RS and practical effectiveness incorporating the information of $\cP$.
%
In particular, we consider two algorithms inspired by the greedy algorithm and the RS and show theoretical guarantees.
%
Algorithm~\ref{alg:proposed} shows the pseudo-code of proposed algorithms.


\begin{algorithm}[!t]
    \caption{Proposed DRAL methods}\label{alg:proposed}
    \begin{algorithmic}[1]
        \Require Domain $\cX$, GP prior $\mu$ and $k$, ambiguity set $\cP$
        \State $\cD_{0} \gets \emptyset$
        \For{$t = 1, \dots, T$}
            \State Update $\sigma_{t-1}^2 (\cdot)$ according to Eq.~\eqref{eq:GP}
            \State Compute $\*x_t$ according to Eq.~\eqref{eq:RS} or Eq.~\eqref{eq:greedy}
        \EndFor
        \State Observe $y_1, \dots, y_T$ 
        \State Update $\mu_{T} (\cdot)$ and $\sigma_{T}^2 (\cdot)$ according to Eq.~\eqref{eq:GP}
        \State \Return $\mu_{T} (\cdot)$ and $\sigma_{T}^2 (\cdot)$
    \end{algorithmic}
\end{algorithm}



\subsection{Algorithms}

First, we consider the RS-based algorithm.
%
The algorithm is straightforward as follows:
\begin{align}
    \*x_t \sim p_t(\*x),
    \label{eq:RS}
\end{align}
where $p_t(\*x) = \argmax_{p \in \cP} \EE_{p(\*x^{*})}[\sigma_{t-1}^2 (\*x^{*})]$ and we assume that we can generate the sample from $p_t$.
%
By using the worst-case distribution $p_t$ for each iteration, this algorithm incorporates the information of $\cP$.


% \begin{itemize}
%     \item 貪欲法は一般に強いためそれに基づく方法を考える.
%     \item しかし, $\max_{p \in \cP}$を次ステップの全候補に対し計算するのに多大な計算量が必要なため, 貪欲法すら計算不可能
%     \item そこで, $p_t(\*x) = \argmax_{p \in \cP} \EE_{p(\*x)}[\sigma_t^2 (\*x)]$を固定した貪欲法を考える.
%     \item しかし, もはや貪欲法ですらないこの方法の近似保証は我々には難しかった.
%     \item そこで, 保守的なUSに理論保証があることから, 少し保守的になる (uncertainな候補を選択する) ように候補を制限することを考えた.
%     \item 最終的なアルゴリズムをXXXに示す.
% \end{itemize}

Second, we consider the greedy algorithm since its practical efficiency has often been reported~\citep[e.g., ][]{bian2017guarantees}.
%
However, in our setup, the greedy algorithm should be
\begin{align*}
    \argmin_{\*x \in \cX} \max_{p \in \cP} \EE_{p(\*x^{*})}[\sigma_{t}^2 (\*x^{*} \mid \*x)],
\end{align*}
which requires huge computational time in general due to min-max optimization.
%
Thus, we consider an approximately greedy algorithm as follows:
\begin{align*}
    \argmin_{\*x \in \cX} \EE_{p_t(\*x^{*})}[\sigma_{t}^2 (\*x^{*} \mid \*x)],
\end{align*}
where $p_t(\*x) = \argmax_{p \in \cP} \EE_{p(\*x^{*})}[\sigma_{t-1}^2 (\*x^{*})]$ is the worst-case distribution defined by $(\*x_i)_{i \in [t-1]}$.
%
On the other hand, the theoretical guarantee for this algorithm is challenging for us.
%
Hence, inspired by the fact that the US has a theoretical guarantee, we set the constraint so that the chosen input is uncertain than $\EE_{p_t(\*x^{*})}[\sigma_{t-1}^2 (\*x^{*})]$:
\begin{align}
    \*x_t = \argmin_{\*x \in \cX_t} \EE_{p_t(\*x^{*})}[\sigma_t^2 (\*x^{*} \mid \*x)],
    \label{eq:greedy}
\end{align}
where $\cX_t \coloneqq \{ \*x \in \cX \mid \sigma^2_{t-1}(\*x) \geq \EE_{p_t(\*x^{*})}[\sigma_{t-1}^2 (\*x^{*})] \}$.
%
Note that $|\cX_t| \geq 1$ holds due to the definition.


%%%%%%%%%%%%%%%%%%%%%%%%%%%%%%%%%%%%%%%%%%%%%%%%%%%%%%%%%%%%%%%%%%%%%%%%%%%%%%%%%%%%%%%%%%%%%%%%%%
\subsection{Analysis}
\label{sec:analysis}

Here, we show the error convergence by Eqs.~\eqref{eq:RS} and \eqref{eq:greedy}:
\begin{theorem}
    Fix $\delta \in (0, 1)$.
    %
    Assume that $\cX \subset \RR^d$ is a compact subset.
    %
    If we run Algorithm~\ref{alg:proposed} with Eq.~\eqref{eq:RS}, the following holds with probability at least $1 - \delta$:
    \begin{align*}
        \max_{p \in \cP} \EE_{p(\*x^{*})}[\sigma_{T}^2 (\*x^{*})] 
        &\leq \frac{2 C_1 \gamma_T}{T} + \cO \left( \frac{\log (1 / \delta)}{T} \right),
    \end{align*}
    where $C_1 = 1 / \log(1 + \sigma^{-2})$.
    \label{theo:error_convergence_RS}
\end{theorem}
%
\begin{theorem}
    Assume that $\cX \subset \RR^d$ is a compact subset.
    %
    If we run Algorithm~\ref{alg:proposed} with Eq.~\eqref{eq:greedy}, the following holds:
    \begin{align*}
        \max_{p \in \cP} \EE_{p(\*x^{*})}[\sigma_{T}^2 (\*x^{*})] 
        &\leq \frac{C_1 \gamma_T}{T},
    \end{align*}
    where $C_1 = 1 / \log(1 + \sigma^{-2})$.
    \label{theo:error_convergence_greedy}
\end{theorem}
See Appendix~\ref{sec:proposed_proof} for the proof, in which Lemma~3 in \citet{kirschner2018-information} is used to show Theorem~\ref{theo:error_convergence_RS}.



Consequently, our proposed methods achieve almost the same convergence as those of the US and RS shown in Proposition~\ref{prop:us_rs}.
%
Furthermore, by combining Lemmas~\ref{lem:UB_error_discrete}, \ref{lem:UB_error_frequentist_continuous}, and \ref{lem:UB_error_bayesian_continuous}, we can see that the upper bound of $E_T$:
\begin{corollary}
    Fix $\delta \in (0, 1)$ and $T \in \NN$.
    %
    Then, if we run Algorithm~\ref{alg:proposed}, the following hold with probability at least $1 - \delta$:
    \begin{enumerate}
        \item When Assumption~\ref{assump:Bayesian} or Assumptions~\ref{assump:frequentist} holds, 
        \begin{align*}
            E_T = \cO\left( \frac{\log (|\cX| / \delta) \gamma_T}{T} \right);
        \end{align*}
        \item When Assumptions~\ref{assump:Bayesian} and \ref{assump:Bayesian_continuous} or Assumptions~\ref{assump:frequentist} and \ref{assump:frequentist_continuous} hold, 
        \begin{align*}
            E_T = \cO\left( \frac{\log (T / \delta) \gamma_T}{T} \right),
        \end{align*}
    \end{enumerate}
\end{corollary}
\begin{proof}
    We can obtain the result by combining Lemmas~\ref{lem:UB_error_discrete}, \ref{lem:UB_error_frequentist_continuous}, and \ref{lem:UB_error_bayesian_continuous}, Theorems~\ref{theo:error_convergence_RS} and \ref{theo:error_convergence_greedy}, and the union bound.
    %
    Note that we assume $|\cX| > T$.
\end{proof}
%
Thus, the error incurred by the proposed algorithms converges to $0$ with high probability for discrete and continuous input domains, at least with linear, SE, and Mat\'ern kernels.


\section{Discussion}

\subsection{Related Prior Work}

\textbf{Training Energy Based Models}.
A number of recent works aim to improve EBM training. \citet{zhu2023learning} uses a diffusion model to reduce the number of Langevin steps within the recovery likelihood approach of \cite{gao2020learning}. \cite{schroder2024energy} eliminates the need for MCMC and $\nabla_x$-computation of the energy during training by using a contrastive loss with forward noising process instead of MCMC, coined Energy Divergence (ED). ED is a promising alternative to E-DSM and has connections to score-matching, but, as shown in \Cref{tab:unconditional_gen}, it is not yet competitive, and  suffers from a bias by choice of noisy energy function.

\textbf{Composition with MCMC}. Similar to our work, \cite{du2023reduce} also uses energy-parameterized diffusion models but performs controlled generation with MCMC rather than SMC. SMC is known to suffer weight degeneracy high dimension, resulting in lack of diversity across particles, MCMC does not suffer from this, though requires additional non-parallel steps which is time consuming. The approaches are however complementary, and indeed one may perform MCMC after resampling steps to promote diversity.

\textbf{Sequential Monte Carlo in Diffusion Models.}
Many recent works use SMC within diffusion models for conditional generation, we detail the FKM formulations of these works in in \Cref{app:fkm_potentials}.

\cite{wu2024practical} uses twisted SMC with a classifier-guided proposal \citep{dhariwal2021diffusion} and potentials approximated with diffusion posterior sampling \citep{chungdiffusion}, which has been detailed as a FKM by concurrent work \citep{zhao2024conditional}. \cite{cardoso2024monte} and \cite{dou2024diffusion} tackle linear inverse problems where potentials have a closed form using Gaussian conjugacy. \cite{li2024derivative} perform SMC for both discrete and continuous diffusion models whereby potentials consist of a reward function applied to $\mathbb{E}[\bfX_0|x_t]$.

\cite{liu2024correcting} corrects conditional generations using an adversarially trained density ratio potential, and scales this to text-to-image models. 


\textbf{SMC for LLMs}. SMC is not only popular within diffusion models, but has been successful within large language models (LLMs) \citep{lew2023sequential, zhao2024probabilistic}. \cite{lew2023sequential} uses a FKM formulation with indicator based potential functions similar to as detailed in \Cref{sec:bounded}, and \cite{zhao2024probabilistic} discuss using SMC for text using potentials from reward functions or learning such potentials via contrastive twist learning, similar to contrastive learning for EBMs.

\subsection{Concurrent work} Since submission/ acceptance of our work \footnote{Submission October 2024}, there have been a number of relevant concurrent works. 

\textbf{FKM Interpretation}. \cite{singhal2025general} similarly to \cite{zhao2024conditional} and this work, detail sampling diffusion models in terms of KFM. \cite{singhal2025general} follow \cite{li2024derivative} in using reward functions based potentials but focus on text-to-image reward, and explore further heuristics such as or combining rewards via sum or max; and sampling $\bfX_0|x_t$ via nested diffusion \citep{elata2024nested} as input to their reward rather than using $\mathbb{E}[\bfX_0|x_t]$ as done in \cite{li2024derivative}.

\textbf{SMC for discrete diffusion}.  \cite{lee2025debiasing} use SMC for low temperature sampling for discrete diffusion models. \cite{xu2024energy} use a pretrained autoregressive likelihood model applied to samples $\bfX_0|x_t$ for a potential within discrete diffusion sampling.

\textbf{Composition}. \cite{skreta2024superposition} construct a cheap density estimator by simulating from an SDE, which can be computed at sampling time if using reverse diffusion solver, though it is not clear if this can be used in conjunction with resampling and Langevin corrector schemes. \cite{skreta2024superposition} then use this estimator to perform composition-type sampling, however their logical \textsc{AND} appears to differ from other more commonly used logical \textsc{AND} operations, in that it targets samples with equal probability between classes rather than generating both classes, e.g. "a CAT and a DOG" results in a cat/dog hybrid optical illusion rather than a separate cat and separate dog in one image. 

\cite{bradley2025mechanisms} explore composition more formally, establishing types of composition and cases where summing scores is sufficient without need for SMC correction as performed in this work or with MCMC correction from \citep{du2023reduce}.




\section{Evaluations}
\label{sec:experiment}

In this section, we demonstrate that \sassha can indeed improve upon existing second-order methods available for standard deep learning tasks.
We also show that \sassha performs competitively to the first-order baseline methods.
Specifically, \sassha is compared to AdaHessian \citep{adahessian}, Sophia-H \citep{sophia}, Shampoo \cite{gupta2018shampoo}, SGD, AdamW \citep{loshchilov2018decoupled}, and SAM \citep{sam} on a diverse set of both vision and language tasks.
We emphasize that we perform an \emph{extensive} hyperparameter search to rigorously tune all optimizers and ensure fair comparisons.
We provide the details of experiment settings to reproduce our results in \cref{app:hypersearch}.
The code to reproduce all results reported in this work is made available for download at \url{https://github.com/LOG-postech/Sassha}.

\subsection{Image Classification}
\begin{table*}[t!]
    \vspace{-0.5em}
    \centering
    \caption{Image classification results of various optimization methods in terms of final validation accuracy (mean$\pm$std).
    \sassha consistently outperforms the other methods for all workloads.
    * means \emph{omitted} due to excessive computational requirements.}
    
    \vskip 0.1in
    \resizebox{0.8\linewidth}{!}{
        \begin{tabular}{clcccccc}
        \toprule
         & 
         & \multicolumn{2}{c}{CIFAR-10} 
         & \multicolumn{2}{c}{CIFAR-100} 
         & \multicolumn{2}{c}{ImageNet} \\
         \cmidrule(l{3pt}r{3pt}){3-4} \cmidrule(l{3pt}r{3pt}){5-6} \cmidrule(l{3pt}r{3pt}){7-8}
         \multicolumn{1}{c}{ Category }
         & \multicolumn{1}{c}{ Method }
         & \multicolumn{1}{c}{ ResNet-20 } 
         & \multicolumn{1}{c}{ ResNet-32 } 
         & \multicolumn{1}{c}{ ResNet-32 }  
         & \multicolumn{1}{c}{ WRN-28-10} 
         & \multicolumn{1}{c}{ ResNet-50 } 
         & \multicolumn{1}{c}{ ViT-s-32} \\ \midrule

        
       \multirow{4}{*}{First-order}  
       & SGD       & 
         $ 92.03 _{ \textcolor{black!60}{\pm 0.32} } $    &
         $ 92.69 _{\textcolor{black!60}{\pm 0.06} }  $    &
         $ 69.32 _{\textcolor{black!60}{\pm 0.19} }  $    &
         $ 80.06 _{\textcolor{black!60}{\pm 0.15} }  $    &
         $ 75.58 _{\textcolor{black!60}{\pm 0.05} }  $    &
         $ 62.90 _{\textcolor{black!60}{\pm 0.36} }  $   \\

        & AdamW      & 
        $ 92.04 _{\textcolor{black!60}{\pm 0.11} }  $     &
        $ 92.42 _{\textcolor{black!60}{\pm 0.13} }  $     &
        $ 68.78 _{\textcolor{black!60}{\pm 0.22} }  $     &
        $ 79.09 _{\textcolor{black!60}{\pm 0.35} }  $     &
        $ 75.38 _{\textcolor{black!60}{\pm 0.08} }  $     &
        $ 66.46 _{\textcolor{black!60}{\pm 0.15} }  $    \\
        
        & SAM $_{\text{SGD}}$  &
        $ 92.85 _{\textcolor{black!60}{\pm 0.07} }  $    &
        $ 93.89 _{\textcolor{black!60}{\pm 0.13} }  $    &
        $ 71.99 _{\textcolor{black!60}{\pm 0.20} }  $    &
        $ 83.14 _{\textcolor{black!60}{\pm 0.13} }  $    &
        $ 76.36 _{\textcolor{black!60}{\pm 0.16} }  $    &
        $ 64.54 _{\textcolor{black!60}{\pm 0.63} }  $    \\
        
        & SAM $_{\text{AdamW}}$  &
        $ 92.77 _{\textcolor{black!60}{\pm 0.29} }  $    &
        $ 93.45 _{\textcolor{black!60}{\pm 0.24} }  $    &
        $ 71.15 _{\textcolor{black!60}{\pm 0.37} }  $    &
        $ 82.88 _{\textcolor{black!60}{\pm 0.31} }  $    &
        $ 76.35 _{\textcolor{black!60}{\pm 0.16} }  $    &
        $ 68.31 _{\textcolor{black!60}{\pm 0.17} }  $    \\

        \midrule
        
        \multirow{4}{*}{Second-order} &
        AdaHessian &
        $ 92.00 _{\textcolor{black!60}{\pm 0.17} } $  &
        $ 92.48 _{\textcolor{black!60}{\pm 0.15} } $  &
        $ 68.06 _{\textcolor{black!60}{\pm 0.22} } $  &
        $ 76.92 _{\textcolor{black!60}{\pm 0.26} } $  &
        $ 73.64 _{\textcolor{black!60}{\pm 0.16} } $  &
        $ 66.42 _{\textcolor{black!60}{\pm 0.23} } $  \\
        
        & Sophia-H   & 
        $ 91.81 _{\textcolor{black!60}{\pm 0.27} } $  &
        $ 91.99 _{\textcolor{black!60}{\pm 0.08} } $  &
        $ 67.76 _{\textcolor{black!60}{\pm 0.37} } $  & 
        $ 79.35 _{\textcolor{black!60}{\pm 0.24} } $  & 
        $ 72.06 _{\textcolor{black!60}{\pm 0.49} } $  &
        $ 62.44 _{\textcolor{black!60}{\pm 0.36} } $  \\
        
        & Shampoo    & 
        $ 88.55 _ {\textcolor{black!60}{\pm 0.83}}$  &
        $ 90.23 _{\textcolor{black!60}{\pm 0.24}} $  &
        $ 64.08 _{\textcolor{black!60}{\pm 0.46}} $  &
        $ 74.06 _{\textcolor{black!60}{\pm 1.28}} $  &
        $*$                                          &
        $*$  \\
        
        \cmidrule(l{3pt}r{3pt}){2-8}
        
        \rowcolor{green!20} &
        \sassha    &
        $ \textbf{92.98} _{\textcolor{black!60}{\pm 0.05} }  $ &
        $ \textbf{94.09} _{\textcolor{black!60}{\pm 0.24} }  $ &
        $ \textbf{72.14} _{\textcolor{black!60}{\pm 0.16} }  $ & 
        $ \textbf{83.54} _{\textcolor{black!60}{\pm 0.08} }  $ &
        $ \textbf{76.43} _{\textcolor{black!60}{\pm 0.18} }  $ &
        $ \textbf{69.20} _{\textcolor{black!60}{\pm 0.30} }  $ \\
        
        \bottomrule
        \end{tabular}
    }
    \vskip 0.1in
    \label{tab:im_cls_results}
\end{table*}

\begin{figure*}[t!]
    \vspace{-0.5em}
    \centering
    \resizebox{0.8\linewidth}{!}{
    \includegraphics[width=0.325\linewidth]{figures/validation/Res32-CIFAR10-Acc.pdf}
    \includegraphics[width=0.325\linewidth]{figures/validation/WRN28-CIFAR100-Acc.pdf}
    \includegraphics[width=0.325\linewidth]{figures/validation/Res50-ImageNet-Acc.pdf}
    }
    \vspace{-0.5em}
    \caption{
    Validation accuracy curves along the training trajectory.
    We also provide loss curves in \cref{app:valloss}.
    }
    \label{fig:im_cls_results}
    \vspace{-0.7em}
\end{figure*}

\begin{table*}[ht!]
    \centering
    \caption{
    Language finetuning and pertraining results for various optimizers. For finetuning, \sassha achieves better results than AdamW and AdaHessian and compares competitively with Sophia-H. For pretraining, \sassha achieves the lowest perplexity among all optimizers.
    }
    \vskip 0.1in
    \resizebox{\linewidth}{!}{
        \begin{tabular}{lc}
            \toprule
             & \multicolumn{1}{c}{$\textbf{Pretrain} / $ GPT1-mini} \\
             \cmidrule(l{3pt}r{3pt}){2-2}
             & Wikitext-2 \\
             & \texttt{Perplexity}\\
            \midrule
            
            AdamW & $ 175.06 $ \\
            SAM $_{\text{AdamW}}$ & $ 158.06 $ \\
            AdaHessian & $ 407.69 $ \\
            Sophia-H & $ 157.60 $ \\
            
            \midrule 
            
            \rowcolor{green!20}
            \sassha &
            $ \textbf{122.40} $ \\
            
            \bottomrule
        \end{tabular}
        
        \begin{tabular}{|ccccccc}
            \toprule
                         \multicolumn{7}{|c}{ \textbf{Finetune} /  SqeezeBERT } \\
                         \cmidrule(l{3pt}r{3pt}){1-7}
                         SST-2 &  MRPC & STS-B & QQP & MNLI & QNLI & RTE \\
             \texttt{Acc} &  \texttt{Acc / F1}  & \texttt{S/P corr.} & \texttt{F1 / Acc} & \texttt{mat/m.mat} &  \texttt{Acc} &  \texttt{Acc} \\
            \midrule
            
            %AdamW         & 
            $ 90.29 _{\textcolor{black!60}{\pm 0.52}} $ 
            & $ 84.56 _{ \textcolor{black!60}{\pm 0.25} } $ / $ 88.99 _{\textcolor{black!60}{\pm 0.11}} $ 
            & $ 88.34 _{\textcolor{black!60}{\pm 0.15}} $ / $ 88.48 _{\textcolor{black!60}{\pm 0.20}} $ 
            & $ 89.92 _{\textcolor{black!60}{\pm 0.05}} $ / $ 86.58 _{\textcolor{black!60}{\pm 0.11}} $ 
            & $ 81.22 _{\textcolor{black!60}{\pm 0.07}} $ / $ 82.26 _{\textcolor{black!60}{\pm 0.05}} $ 
            & $ 89.93 _{\textcolor{black!60}{\pm 0.14}} $ 
            & $ 68.95 _{\textcolor{black!60}{\pm 0.72}} $  \\
    
            %SAM _{\text{AdamW}}   &
            $ \textbf{90.52} _{\textcolor{black!60}{\pm 0.27}} $ 
            & $ 83.25 _{\textcolor{black!60}{\pm 2.79}} $ / $ 87.90 _{\textcolor{black!60}{\pm 2.21}} $ 
            & $ 88.38 _{\textcolor{black!60}{\pm 0.01}} $ / $ 88.79 _{\textcolor{black!60}{\pm 0.99}} $ 
            & $ 90.26 _{\textcolor{black!60}{\pm 0.28}} $ / $ 86.99 _{\textcolor{black!60}{\pm 0.31}} $ 
            & $ 81.56 _{\textcolor{black!60}{\pm 0.18}} $ / $ \textbf{82.46} _{\textcolor{black!60}{\pm 0.19}} $ 
            & $ \textbf{90.38} _{\textcolor{black!60}{\pm 0.05}} $ 
            & $ 68.83 _{\textcolor{black!60}{\pm 1.46}} $  \\
    
            %AdaHessian    & 
            $ 89.64 _{\textcolor{black!60}{\pm 0.13}} $ 
            & $ 79.74 _{\textcolor{black!60}{\pm 4.00}} $ / $ 85.26 _{\textcolor{black!60}{\pm 3.50}} $ 
            & $ 86.08 _{\textcolor{black!60}{\pm 4.04}} $ / $ 86.46 _{\textcolor{black!60}{\pm 4.06}} $ 
            & $ 90.37 _{\textcolor{black!60}{\pm 0.05}} $ / $ 87.07 _{\textcolor{black!60}{\pm 0.05}} $ 
            & $ 81.33 _{\textcolor{black!60}{\pm 0.17}} $ / $ 82.08 _{\textcolor{black!60}{\pm 0.02}} $ 
            & $ 89.94 _{\textcolor{black!60}{\pm 0.12}} $ 
            & $ 71.00 _{\textcolor{black!60}{\pm 1.04}} $ \\
            
            % Sophia-H  &
            $ 90.44 _{\textcolor{black!60}{\pm 0.46}} $ 
            & $ 85.78 _{\textcolor{black!60}{\pm 1.07}} $ / $ 89.90 _{\textcolor{black!60}{\pm 0.82}} $ 
            & $ 88.17 _{\textcolor{black!60}{\pm 1.07}} $ / $ 88.53 _{\textcolor{black!60}{\pm 1.13}} $ 
            & $ 90.70 _{\textcolor{black!60}{\pm 0.04}} $ / $ 87.60 _{\textcolor{black!60}{\pm 0.06}} $ 
            & $ \textbf{81.77} _{\textcolor{black!60}{\pm 0.18}} $ / $ 82.36 _{\textcolor{black!60}{\pm 0.22}} $ 
            & $ 90.12_{\textcolor{black!60}{\pm 0.14}} $ 
            & $ 70.76 _{\textcolor{black!60}{\pm 1.44}} $  \\
            
            \midrule
            
            \rowcolor{green!20} 
            $ 90.44 _{\textcolor{black!60}{\pm 0.98}} $    &
            $ \textbf{86.28} _{\textcolor{black!60}{\pm 0.28}} $ / $ \textbf{90.13} _{\textcolor{black!60}{\pm 0.161}} $     &
            $ \textbf{88.72} _{\textcolor{black!60}{\pm 0.75}} $ / $ \textbf{89.10} _{\textcolor{black!60}{\pm 0.70}}  $     &
            $ \textbf{90.91} _{\textcolor{black!60}{\pm 0.06}} $ / $ \textbf{87.85}  _{\textcolor{black!60}{\pm 0.09}} $     &
            $ 81.61 _{\textcolor{black!60}{\pm 0.25}} $ / $ 81.71 _{\textcolor{black!60}{\pm 0.11}} $     &
            $ 89.85_{\textcolor{black!60}{\pm 0.20}} $    &
            $ \textbf{72.08} _{\textcolor{black!60}{\pm 0.55}} $  \\
            
            \bottomrule
        \end{tabular}
    }
    \vspace{-0.5em}
    \label{tab:language}
\end{table*}

We first evaluate \sassha for image classification on CIFAR-10, CIFAR-100, and ImageNet.
We train various models of the ResNet family \citep{he2016deep,zagoruyko2016wide} and an efficient variant of Vision Transformer \citep{beyer2022better}.
We adhere to standard inception-style data augmentations during training instead of making use of advanced data augmentation techniques \citep{devries2017improved} or regularization methods \citep{gastaldi2017shake}.
Results are presented in \cref{tab:im_cls_results} and \cref{fig:im_cls_results}.

We begin by comparing the generalization performance of adaptive second-order methods to that of first-order methods.
Across all settings, adaptive second-order methods consistently exhibit lower accuracy than their first-order counterparts.
This observation aligns with previous studies indicating that second-order optimization often result in poorer generalization compared to first-order approaches.
In contrast, \sassha, benefiting from sharpness minimization, consistently demonstrates superior generalization performance, outperforming both first-order and second-order methods in every setting.
Particularly, \sassha is up to 4\% more effective than the best-performing adaptive or second-order methods (\eg, WRN-28-10, ViT-s-32).
Moreover, \sassha continually surpasses SGD and AdamW, even when they are trained for twice as many epochs, achieving a performance margin of about 0.3\% to 3\%. 
Further details are provided in \cref{app:comp_fo_fair}.

Interestingly, \sassha also outperforms SAM.
Since first-order methods typically exhibit superior generalization performance compared to second-order methods, it might be intuitive to expect SAM to surpass \sassha if the two are viewed merely as the outcomes of applying sharpness minimization to first-order and second-order methods, respectively.
However, the results conflict with this intuition.
We attribute this to the careful design choices made in \sassha, stabilizing Hessian approximation under sharpness minimization, so as to unleash the potential of the second-order method, leading to its outstanding performance.
As a support, we show that naively incorporating SAM into other second-order methods does not yield these favorable results in \cref{app:samsophia}.
We also make more comparisons with SAM in \cref{sec:sassha_vs_sam}.

\subsection{Language Modeling}

Recent studies have shown the potential of second-order methods for pretraining language models.
Here, we first evaluate how \sassha performs on this task.
Specifically, we train GPT1-mini, a scaled-down variant of GPT1 \citep{radford2019language}, on Wikitext-2 dataset \citep{merity2022pointer} using various methods including \sassha and compare their results (see the left of \cref{tab:language}).
Our results show that \sassha achieves the lowest perplexity among all methods including Sophia-H \citep{sophia}, a recent method that is designed specifically for language modeling tasks and sets state of the art, which highlights generality in addition to the numerical advantage of \sassha.

We also extend our evaluation to finetuning tasks.
Specifically, we finetune SqueezeBERT \citep{iandola2020squeezebert} for diverse tasks in the GLUE benchmark \citep{wang2018glue}.
The results are on the right side of \cref{tab:language}.
It shows that \sassha compares competitively to other second-order methods.
Notably, it also outperforms AdamW---often the method of choice for training language models---on nearly all tasks.

\subsection{Comparison to SAM}\label{sec:sassha_vs_sam}

So far, we have seen that \sassha outperforms second-order methods quite consistently on both vision and language tasks.
Interestingly, we also find that \sassha often improves upon SAM.
In particular, it appears that the gain is larger for the Transformer-based architectures, \ie, ViT results in \cref{tab:im_cls_results} or GPT/BERT results in \cref{tab:language}.

We posit that this is potentially due to the robustness of \sassha to the block heterogeneity inherent in Transformer architectures, where the Hessian spectrum varies significantly across different blocks.
This characteristic is known to make SGD perform worse than adaptive methods like Adam on Transformer-based models \citep{zhang2024why}.
Since \sassha leverages second-order information via preconditioning gradients, it has the potential to address the ill-conditioned nature of Transformers more effectively than SAM with first-order methods.

To push further, we conducted additional experiments.
First, we allocate more training budgets to SAM to see whether it compares to \sassha.
% additionally compare \sassha to SAM with more training budgets.
The results are presented in \cref{tab:sam}.
We find that SAM still underperforms \sassha, even though it is given more budgets of training iterations over data or wall-clock time.
Furthermore, we also compare \sassha to more advanced variants of SAM including ASAM \citep{asam} and GSAM \citep{gsam}, showing that \sassha performs competitively even to these methods (\cref{app:samvariants_vs_sassha}).
Notably, however, these variants of SAM require a lot more hyperparameter tuning to be compared.


\section{Discussion}\label{sec:discussion}



\subsection{From Interactive Prompting to Interactive Multi-modal Prompting}
The rapid advancements of large pre-trained generative models including large language models and text-to-image generation models, have inspired many HCI researchers to develop interactive tools to support users in crafting appropriate prompts.
% Studies on this topic in last two years' HCI conferences are predominantly focused on helping users refine single-modality textual prompts.
Many previous studies are focused on helping users refine single-modality textual prompts.
However, for many real-world applications concerning data beyond text modality, such as multi-modal AI and embodied intelligence, information from other modalities is essential in constructing sophisticated multi-modal prompts that fully convey users' instruction.
This demand inspires some researchers to develop multimodal prompting interactions to facilitate generation tasks ranging from visual modality image generation~\cite{wang2024promptcharm, promptpaint} to textual modality story generation~\cite{chung2022tale}.
% Some previous studies contributed relevant findings on this topic. 
Specifically, for the image generation task, recent studies have contributed some relevant findings on multi-modal prompting.
For example, PromptCharm~\cite{wang2024promptcharm} discovers the importance of multimodal feedback in refining initial text-based prompting in diffusion models.
However, the multi-modal interactions in PromptCharm are mainly focused on the feedback empowered the inpainting function, instead of supporting initial multimodal sketch-prompt control. 

\begin{figure*}[t]
    \centering
    \includegraphics[width=0.9\textwidth]{src/img/novice_expert.pdf}
    \vspace{-2mm}
    \caption{The comparison between novice and expert participants in painting reveals that experts produce more accurate and fine-grained sketches, resulting in closer alignment with reference images in close-ended tasks. Conversely, in open-ended tasks, expert fine-grained strokes fail to generate precise results due to \tool's lack of control at the thin stroke level.}
    \Description{The comparison between novice and expert participants in painting reveals that experts produce more accurate and fine-grained sketches, resulting in closer alignment with reference images in close-ended tasks. Novice users create rougher sketches with less accuracy in shape. Conversely, in open-ended tasks, expert fine-grained strokes fail to generate precise results due to \tool's lack of control at the thin stroke level, while novice users' broader strokes yield results more aligned with their sketches.}
    \label{fig:novice_expert}
    % \vspace{-3mm}
\end{figure*}


% In particular, in the initial control input, users are unable to explicitly specify multi-modal generation intents.
In another example, PromptPaint~\cite{promptpaint} stresses the importance of paint-medium-like interactions and introduces Prompt stencil functions that allow users to perform fine-grained controls with localized image generation. 
However, insufficient spatial control (\eg, PromptPaint only allows for single-object prompt stencil at a time) and unstable models can still leave some users feeling the uncertainty of AI and a varying degree of ownership of the generated artwork~\cite{promptpaint}.
% As a result, the gap between intuitive multi-modal or paint-medium-like control and the current prompting interface still exists, which requires further research on multi-modal prompting interactions.
From this perspective, our work seeks to further enhance multi-object spatial-semantic prompting control by users' natural sketching.
However, there are still some challenges to be resolved, such as consistent multi-object generation in multiple rounds to increase stability and improved understanding of user sketches.   


% \new{
% From this perspective, our work is a step forward in this direction by allowing multi-object spatial-semantic prompting control by users' natural sketching, which considers the interplay between multiple sketch regions.
% % To further advance the multi-modal prompting experience, there are some aspects we identify to be important.
% % One of the important aspects is enhancing the consistency and stability of multiple rounds of generation to reduce the uncertainty and loss of control on users' part.
% % For this purpose, we need to develop techniques to incorporate consistent generation~\cite{tewel2024training} into multi-modal prompting framework.}
% % Another important aspect is improving generative models' understanding of the implicit user intents \new{implied by the paint-medium-like or sketch-based input (\eg, sketch of two people with their hands slightly overlapping indicates holding hand without needing explicit prompt).
% % This can facilitate more natural control and alleviate users' effort in tuning the textual prompt.
% % In addition, it can increase users' sense of ownership as the generated results can be more aligned with their sketching intents.
% }
% For example, when users draw sketches of two people with their hands slightly overlapping, current region-based models cannot automatically infer users' implicit intention that the two people are holding hands.
% Instead, they still require users to explicitly specify in the prompt such relationship.
% \tool addresses this through sketch-aware prompt recommendation to fill in the necessary semantic information, alleviating users' workload.
% However, some users want the generative AI in the future to be able to directly infer this natural implicit intentions from the sketches without additional prompting since prompt recommendation can still be unstable sometimes.


% \new{
% Besides visual generation, 
% }
% For example, one of the important aspect is referring~\cite{he2024multi}, linking specific text semantics with specific spatial object, which is partly what we do in our sketch-aware prompt recommendation.
% Analogously, in natural communication between humans, text or audio alone often cannot suffice in expressing the speakers' intentions, and speakers often need to refer to an existing spatial object or draw out an illustration of her ideas for better explanation.
% Philosophically, we HCI researchers are mostly concerned about the human-end experience in human-AI communications.
% However, studies on prompting is unique in that we should not just care about the human-end interaction, but also make sure that AI can really get what the human means and produce intention-aligned output.
% Such consideration can drastically impact the design of prompting interactions in human-AI collaboration applications.
% On this note, although studies on multi-modal interactions is a well-established topic in HCI community, it remains a challenging problem what kind of multi-modal information is really effective in helping humans convey their ideas to current and next generation large AI models.




\subsection{Novice Performance vs. Expert Performance}\label{sec:nVe}
In this section we discuss the performance difference between novice and expert regarding experience in painting and prompting.
First, regarding painting skills, some participants with experience (4/12) preferred to draw accurate and fine-grained shapes at the beginning. 
All novice users (5/12) draw rough and less accurate shapes, while some participants with basic painting skills (3/12) also favored sketching rough areas of objects, as exemplified in Figure~\ref{fig:novice_expert}.
The experienced participants using fine-grained strokes (4/12, none of whom were experienced in prompting) achieved higher IoU scores (0.557) in the close-ended task (0.535) when using \tool. 
This is because their sketches were closer in shape and location to the reference, making the single object decomposition result more accurate.
Also, experienced participants are better at arranging spatial location and size of objects than novice participants.
However, some experienced participants (3/12) have mentioned that the fine-grained stroke sometimes makes them frustrated.
As P1's comment for his result in open-ended task: "\emph{It seems it cannot understand thin strokes; even if the shape is accurate, it can only generate content roughly around the area, especially when there is overlapping.}" 
This suggests that while \tool\ provides rough control to produce reasonably fine results from less accurate sketches for novice users, it may disappoint experienced users seeking more precise control through finer strokes. 
As shown in the last column in Figure~\ref{fig:novice_expert}, the dragon hovering in the sky was wrongly turned into a standing large dragon by \tool.

Second, regarding prompting skills, 3 out of 12 participants had one or more years of experience in T2I prompting. These participants used more modifiers than others during both T2I and R2I tasks.
Their performance in the T2I (0.335) and R2I (0.469) tasks showed higher scores than the average T2I (0.314) and R2I (0.418), but there was no performance improvement with \tool\ between their results (0.508) and the overall average score (0.528). 
This indicates that \tool\ can assist novice users in prompting, enabling them to produce satisfactory images similar to those created by users with prompting expertise.



\subsection{Applicability of \tool}
The feedback from user study highlighted several potential applications for our system. 
Three participants (P2, P6, P8) mentioned its possible use in commercial advertising design, emphasizing the importance of controllability for such work. 
They noted that the system's flexibility allows designers to quickly experiment with different settings.
Some participants (N = 3) also mentioned its potential for digital asset creation, particularly for game asset design. 
P7, a game mod developer, found the system highly useful for mod development. 
He explained: "\emph{Mods often require a series of images with a consistent theme and specific spatial requirements. 
For example, in a sacrifice scene, how the objects are arranged is closely tied to the mod's background. It would be difficult for a developer without professional skills, but with this system, it is possible to quickly construct such images}."
A few participants expressed similar thoughts regarding its use in scene construction, such as in film production. 
An interesting suggestion came from participant P4, who proposed its application in crime scene description. 
She pointed out that witnesses are often not skilled artists, and typically describe crime scenes verbally while someone else illustrates their account. 
With this system, witnesses could more easily express what they saw themselves, potentially producing depictions closer to the real events. "\emph{Details like object locations and distances from buildings can be easily conveyed using the system}," she added.

% \subsection{Model Understanding of Users' Implicit Intents}
% In region-sketch-based control of generative models, a significant gap between interaction design and actual implementation is the model's failure in understanding users' naturally expressed intentions.
% For example, when users draw sketches of two people with their hands slightly overlapping, current region-based models cannot automatically infer users' implicit intention that the two people are holding hands.
% Instead, they still require users to explicitly specify in the prompt such relationship.
% \tool addresses this through sketch-aware prompt recommendation to fill in the necessary semantic information, alleviating users' workload.
% However, some users want the generative AI in the future to be able to directly infer this natural implicit intentions from the sketches without additional prompting since prompt recommendation can still be unstable sometimes.
% This problem reflects a more general dilemma, which ubiquitously exists in all forms of conditioned control for generative models such as canny or scribble control.
% This is because all the control models are trained on pairs of explicit control signal and target image, which is lacking further interpretation or customization of the user intentions behind the seemingly straightforward input.
% For another example, the generative models cannot understand what abstraction level the user has in mind for her personal scribbles.
% Such problems leave more challenges to be addressed by future human-AI co-creation research.
% One possible direction is fine-tuning the conditioned models on individual user's conditioned control data to provide more customized interpretation. 

% \subsection{Balance between recommendation and autonomy}
% AIGC tools are a typical example of 
\subsection{Progressive Sketching}
Currently \tool is mainly aimed at novice users who are only capable of creating very rough sketches by themselves.
However, more accomplished painters or even professional artists typically have a coarse-to-fine creative process. 
Such a process is most evident in painting styles like traditional oil painting or digital impasto painting, where artists first quickly lay down large color patches to outline the most primitive proportion and structure of visual elements.
After that, the artists will progressively add layers of finer color strokes to the canvas to gradually refine the painting to an exquisite piece of artwork.
One participant in our user study (P1) , as a professional painter, has mentioned a similar point "\emph{
I think it is useful for laying out the big picture, give some inspirations for the initial drawing stage}."
Therefore, rough sketch also plays a part in the professional artists' creation process, yet it is more challenging to integrate AI into this more complex coarse-to-fine procedure.
Particularly, artists would like to preserve some of their finer strokes in later progression, not just the shape of the initial sketch.
In addition, instead of requiring the tool to generate a finished piece of artwork, some artists may prefer a model that can generate another more accurate sketch based on the initial one, and leave the final coloring and refining to the artists themselves.
To accommodate these diverse progressive sketching requirements, a more advanced sketch-based AI-assisted creation tool should be developed that can seamlessly enable artist intervention at any stage of the sketch and maximally preserve their creative intents to the finest level. 

\subsection{Ethical Issues}
Intellectual property and unethical misuse are two potential ethical concerns of AI-assisted creative tools, particularly those targeting novice users.
In terms of intellectual property, \tool hands over to novice users more control, giving them a higher sense of ownership of the creation.
However, the question still remains: how much contribution from the user's part constitutes full authorship of the artwork?
As \tool still relies on backbone generative models which may be trained on uncopyrighted data largely responsible for turning the sketch into finished artwork, we should design some mechanisms to circumvent this risk.
For example, we can allow artists to upload backbone models trained on their own artworks to integrate with our sketch control.
Regarding unethical misuse, \tool makes fine-grained spatial control more accessible to novice users, who may maliciously generate inappropriate content such as more realistic deepfake with specific postures they want or other explicit content.
To address this issue, we plan to incorporate a more sophisticated filtering mechanism that can detect and screen unethical content with more complex spatial-semantic conditions. 
% In the future, we plan to enable artists to upload their own style model

% \subsection{From interactive prompting to interactive spatial prompting}


\subsection{Limitations and Future work}

    \textbf{User Study Design}. Our open-ended task assesses the usability of \tool's system features in general use cases. To further examine aspects such as creativity and controllability across different methods, the open-ended task could be improved by incorporating baselines to provide more insightful comparative analysis. 
    Besides, in close-ended tasks, while the fixing order of tool usage prevents prior knowledge leakage, it might introduce learning effects. In our study, we include practice sessions for the three systems before the formal task to mitigate these effects. In the future, utilizing parallel tests (\textit{e.g.} different content with the same difficulty) or adding a control group could further reduce the learning effects.

    \textbf{Failure Cases}. There are certain failure cases with \tool that can limit its usability. 
    Firstly, when there are three or more objects with similar semantics, objects may still be missing despite prompt recommendations. 
    Secondly, if an object's stroke is thin, \tool may incorrectly interpret it as a full area, as demonstrated in the expert results of the open-ended task in Figure~\ref{fig:novice_expert}. 
    Finally, sometimes inclusion relationships (\textit{e.g.} inside) between objects cannot be generated correctly, partially due to biases in the base model that lack training samples with such relationship. 

    \textbf{More support for single object adjustment}.
    Participants (N=4) suggested that additional control features should be introduced, beyond just adjusting size and location. They noted that when objects overlap, they cannot freely control which object appears on top or which should be covered, and overlapping areas are currently not allowed.
    They proposed adding features such as layer control and depth control within the single-object mask manipulation. Currently, the system assigns layers based on color order, but future versions should allow users to adjust the layer of each object freely, while considering weighted prompts for overlapping areas.

    \textbf{More customized generation ability}.
    Our current system is built around a single model $ColorfulXL-Lightning$, which limits its ability to fully support the diverse creative needs of users. Feedback from participants has indicated a strong desire for more flexibility in style and personalization, such as integrating fine-tuned models that cater to specific artistic styles or individual preferences. 
    This limitation restricts the ability to adapt to varied creative intents across different users and contexts.
    In future iterations, we plan to address this by embedding a model selection feature, allowing users to choose from a variety of pre-trained or custom fine-tuned models that better align with their stylistic preferences. 
    
    \textbf{Integrate other model functions}.
    Our current system is compatible with many existing tools, such as Promptist~\cite{hao2024optimizing} and Magic Prompt, allowing users to iteratively generate prompts for single objects. However, the integration of these functions is somewhat limited in scope, and users may benefit from a broader range of interactive options, especially for more complex generation tasks. Additionally, for multimodal large models, users can currently explore using affordable or open-source models like Qwen2-VL~\cite{qwen} and InternVL2-Llama3~\cite{llama}, which have demonstrated solid inference performance in our tests. While GPT-4o remains a leading choice, alternative models also offer competitive results.
    Moving forward, we aim to integrate more multimodal large models into the system, giving users the flexibility to choose the models that best fit their needs. 
    


\section{Conclusion}\label{sec:conclusion}
In this paper, we present \tool, an interactive system designed to help novice users create high-quality, fine-grained images that align with their intentions based on rough sketches. 
The system first refines the user's initial prompt into a complete and coherent one that matches the rough sketch, ensuring the generated results are both stable, coherent and high quality.
To further support users in achieving fine-grained alignment between the generated image and their creative intent without requiring professional skills, we introduce a decompose-and-recompose strategy. 
This allows users to select desired, refined object shapes for individual decomposed objects and then recombine them, providing flexible mask manipulation for precise spatial control.
The framework operates through a coarse-to-fine process, enabling iterative and fine-grained control that is not possible with traditional end-to-end generation methods. 
Our user study demonstrates that \tool offers novice users enhanced flexibility in control and fine-grained alignment between their intentions and the generated images.


\section*{Acknowkedgements}
This work was partially supported by 
JST ACT-X Grant Number (JPMJAX23CD and JPMJAX24C3), 
JST PRESTO Grant Number JPMJPR24J6, 
JST CREST Grant Numbers (JPMJCR21D3 including AIP challenge program and JPMJCR22N2),
JST Moonshot R\&D Grant Number JPMJMS2033-05, 
%%%%%%%
JSPS KAKENHI Grant Number (JP20H00601, JP23K16943, JP23K19967, JP24K15080, and JP24K20847), 
%%%%%%%
NEDO (JPNP20006),
and RIKEN Center for Advanced Intelligence Project.


% MEXT Program: Data Creation and Utilization-Type Material Research and Development Project Grant Number JPMXP1122712807,
% JSPS KAKENHI Grant Numbers JP20H00601,JP21H03498,JP22H00300,JP23K16943,JP23K19967,
% JST CREST Grant Numbers JPMJCR21D3,JPMJCR22N2, 

% JST ACT-X Grant Number JPMJAX23CD,
% NEDO Grant Numbers JPNP18002,JPNP20006, 
% and RIKEN Center for Advanced Intelligence Project.

%%%%%%%%%%%%%%%%%%%%%%%%%%%%%%%%%%%%%%%%%%%%%%%%%%%%%%%%%%%%%%%%%%%%%%%%%%%%%%%%%%%%%%%%%%%%%%
% \clearpage

% \section*{Impact Statements}

% This paper focuses on the theoretical and algorithmic aspects of the machine learning method.
% %
% We consider that some potential societal consequences of this paper need not necessarily be highlighted here.

%%%%%%%%%%%%%%%%%%%%%%%%%%%%%%%%%%%%%%%%%%%%%%%%%%%%%%%%%%%%%%%%%%%%%%%%%%%%%%%%%%%%%%%%%%%%%%
% % \subsubsection*{References}
\bibliography{ref}
\bibliographystyle{plainnat}
% %%%%%%%%%%%%%%%%%%%%%%%%%%%%%%%%%%%%%%%%%%%%%%%%%%%%%%%%%%%%%%%%%%%%%%%%%%%%%%%%%%%%%%%%%%%%%%

%%%%%%%%%%%%%%%%%%%%%%%%%%%%%%%%%%%%%%%%%%%%%%%%%%%%%%%%%%%%%%%%%%%%%%%%%%%%%%%%%%%%%%%%%%%%%%


\newpage
\appendix
\section{Applicability of SparseTransX for dense graphs} 
\label{A:density}
Even for fully dense graphs, our KGE computations remain highly sparse. This is because our SpMM leverages the incidence matrix for triplets, rather than the graph's adjacency matrix. In the paper, the sparse matrix $A \in \{-1,0,1\}^{M \times (N+R)}$ represents the triplets, where $N$ is the number of entities, $R$ is the number of relations, and $M$ is the number of triplets. This representation remains extremely sparse, as each row contains exactly three non-zero values (or two in the case of the "ht" representation). Hence, the sparsity of this formulation is independent of the graph's structure, ensuring computational efficiency even for dense graphs.

\section{Computational Complexity}
\label{A:complexity}
 For a sparse matrix $A$ with $m \times k$ having $nnz(A)=$ number of non zeros and dense matrix $X$ with $k \times n$ dimension, the computational complexity of the SpMM is $O(nnz(A) \cdot n)$ since there are a total of $nnz(A)$ number of dot products each involving $n$ components. Since our sparse matrix contains exactly three non-zeros in each row, $nnz(A) = 3m$. Therefore, the complexity of SpMM is $O(3m \cdot n)$ or $O(m \cdot n)$, meaning the complexity increases when triplet counts or embedding dimension is increased. Memory access pattern will change when the number of entities is increased and it will affect the runtime, but the algorithmic complexity will not be affected by the number of entities/relations.

\section{Applicability to Non-translational Models}
\label{A:non_trans}
Our paper focused on translational models using sparse operations, but the concept extends broadly to various other knowledge graph embedding (KGE) methods. Neural network-based models, which are inherently matrix-multiplication-based, can be seamlessly integrated into this framework. Additionally, models such as DistMult, ComplEx, and RotatE can be implemented with simple modifications to the SpMM operations. Implementing these KGE models requires modifying the addition and multiplication operators in SpMM, effectively changing the semiring that governs the multiplication.   

In the paper, the sparse matrix $A \in \{-1,0,1\}^{M \times (N+R)}$ represents the triplets, and the dense matrix $E \in \mathbb{R}^{(N+R) \times d}$ represents the embedding matrix, where $N$ is the number of entities, $R$ is the number of relations, and $M$ is the number of triplets. TransE’s score function, defined as $h + r - t$, is computed by multiplying $A$ and $E$ using an SpMM followed by the L2 norm. This operation can be generalized using a semiring-based SpMM model: $Z_{ij} = \bigoplus_{k=1}^{n} (A_{ik} \otimes E_{kj})$

Here, $\oplus$ represents the semiring addition operator, and $\otimes$ represents the semiring multiplication operator. For TransE, these operators correspond to standard arithmetic addition and multiplication, respectively.

\subsection*{DistMult} 
DistMult’s score function has the expression $h \odot r \odot t$. To adapt SpMM for this model, two key adjustments are required: The sparse matrix $A$ stores $+1$ at the positions corresponding to $h_{\text{idx}}$, $t_{\text{idx}}$, and $r_{\text{idx}}$. Both the semiring addition and multiplication operators are set to arithmetic multiplication. These changes enable the use of SpMM for the DistMult score function.

\subsection*{ComplEx} 
ComplEx’s score function has $h \odot r \odot \bar{t}$, where embeddings are stored as complex numbers (e.g., using PyTorch). In this case, the semiring operations are similar to DistMult, but with complex number multiplication replacing real number multiplication.

\subsection*{RotatE} 
RotatE’s score function has $h \odot r - t$. For this model, the semiring requires both arithmetic multiplication and subtraction for $\oplus$. With minor modifications to our SpMM implementation, the semiring addition operator can be adapted to compute $h \odot r - t$.

\subsection*{Support from other libraries}
Many existing libraries, such as GraphBLAS (Kimmerer, Raye, et al., 2024), Ginkgo (Anzt, Hartwig, et al., 2022), and Gunrock (Wang, Yangzihao, et al., 2017), already support custom semirings in SpMM. We can leverage C++ templates to extend support for KGE models with minimal effort.


\begin{figure*}[t]
\centering     %%% not \center
\includegraphics[width=\textwidth]{figures/all-eval.pdf}
\caption{Loss curve for sparse and non-sparse approach. Sparse approach eventually reaches the same loss value with similar Hits@10 test accuracy.}
\label{fig:loss_curve}
\end{figure*}

\section{Model Performance Evaluation and Convergence}
\label{A:eval}
SpTransX follows a slightly different loss curve (see Figure \ref{fig:loss_curve}) and eventually converges with the same loss as other non-sparse implementations such as TorchKGE. We test SpTransX with the WN18 dataset having embedding size 512 (128 for TransR and TransH due to memory limitation) and run 200-1000 epochs. We compute average Hits@10 of 9 runs with different initial seeds and a learning rate scheduler. The results are shown below. We find that Hits@10 is generally comparable to or better than the Hits@10 achieved by TorchKGE.

\begin{table}[h]
\centering
\caption{Average of 9 Hits@10 Accuracy for WN18 dataset}
\begin{tabular}{|c|c|c|}
\hline
\textbf{Model} & \textbf{TorchKGE} & \textbf{SpTransX} \\ \hline
TransE         & 0.79 ± 0.001700   & 0.79 ± 0.002667   \\ \hline
TransR         & 0.29 ± 0.005735   & 0.33 ± 0.006154   \\ \hline
TransH         & 0.76 ± 0.012285   & 0.79 ± 0.001832   \\ \hline
TorusE         & 0.73 ± 0.003258   & 0.73 ± 0.002780   \\ \hline
\end{tabular}
\label{table:perf_eval}
\end{table}

% We also plot the loss curve for different models in Figure \ref{fig:loss_curve}. We observe that the sparse approach follows a similar loss curve and eventually converges to the same final loss.

\section{Distributed SpTransX and Its Applicability to Large KGs}
\label{A:dist}
SpTransX framework includes several features to support distributed KGE training across multi-CPU, multi-GPU, and multi-node setups. Additionally, it incorporates modules for model and dataset streaming to handle massive datasets efficiently. 

Distributed SpTransX relies on PyTorch Distributed Data Parallel (DDP) and Fully Sharded Data Parallel (FSDP) support to distribute sparse computations across multiple GPUs. 

\begin{table}[h]
\centering
\caption{Average Time of 15 Epochs (seconds). Training time of TransE model with Freebase dataset (250M triplets, 77M entities. 74K relations, batch size 393K)  on 32 NVIDIA A100 GPUs. FSDP enables model training with larger embedding when DDP fails.}
\begin{tabular}{|p{2cm}|p{2.5cm}|p{2.5cm}|}
\hline
\textbf{Embedding Size} & \textbf{DDP (Distributed Data Parallel)} & \textbf{FSDP (Fully Sharded Data Parallel)} \\ \hline
16                      & 65.07 ± 1.641                            & 63.35 ± 1.258                               \\ \hline
20                      & Out of Memory                            & 96.44 ± 1.490                               \\ \hline
\end{tabular}
\end{table}

We run an experiment with a large-scale KG to showcase the performance of distributed SpTransX. Freebase (250M triplets, 77M entities. 74K relations, batch size 393K) dataset is trained using the TransE model on 32 NVIDIA A100 GPUs of NERSC using various distributed settings. SpTransX’s Streaming dataset module allows fetching only the necessary batch from the dataset and enables memory-efficient training. FSDP enables model training with larger embedding when DDP fails.

\section{Scaling and Communication Bottlenecks for Large KG Training}
\label{A:scaling}
Communication can be a significant bottleneck in distributed KGE training when using SpMM. However, by leveraging Distributed Data-Parallel (DDP) in PyTorch, we successfully scale distributed SpTransX to 64 NVIDIA A100 GPUs with reasonable efficiency. The training time for the COVID-19 dataset with 60,820 entities, 62 relations, and 1,032,939 triplets is in Table \ref{table:scaling}. 
% \vspace{-.3cm}
\begin{table}[h]
\centering
\caption{Scaling TransE model on COVID-19 dataset}
\begin{tabular}{|c|c|}
\hline
\textbf{Number of GPUs} & \textbf{500 epoch time (seconds)} \\ \hline
4                       & 706.38                            \\ \hline
8                       & 586.03                            \\ \hline
16                      & 340.00                               \\ \hline
32                      & 246.02                            \\ \hline
64                      & 179.95                            \\ \hline
\end{tabular}
\label{table:scaling}
\end{table}
% \vspace{-.2cm}
It indicates that communication is not a bottleneck up to 64 GPUs. If communication becomes a performance bottleneck at larger scales, we plan to explore alternative communication-reducing algorithms, including 2D and 3D matrix distribution techniques, which are known to minimize communication overhead at extreme scales. Additionally, we will incorporate model parallelism alongside data parallelism for large-scale knowledge graphs.

\section{Backpropagation of SpMM}
\label{A:backprop}
 Our main computational kernel is the sparse-dense matrix multiplication (SpMM). The computation of backpropagation of an SpMM w.r.t. the dense matrix is also another SpMM. To see how, let's consider the sparse-dense matrix multiplication $AX = C$ which is part of the training process. As long as the computational graph reduces to a single scaler loss $\mathfrak{L}$, it can be shown that $\frac{\partial C}{\partial X} = A^T$. Here, $X$ is the learnable parameter (embeddings), and $A$ is the sparse matrix. Since $A^T$ is also a sparse matrix and $\frac{\partial \mathfrak{L}}{\partial C}$ is a dense matrix, the computation $\frac{\partial \mathfrak{L}}{\partial X} = \frac{\partial C}{\partial X} \times \frac{\partial \mathfrak{L}}{\partial C} = A^T \times \frac{\partial \mathfrak{L}}{\partial C} $ is an SpMM. This means that both forward and backward propagation of our approach benefit from the efficiency of a high-performance SpMM.

\subsection*{Proof that $\frac{\partial C}{\partial X} = A^T$}
 To see why $\frac{\partial C}{\partial X} = A^T$ is used in the gradient calculation, we can consider the following small matrix multiplication without loss of generality.
\begin{align*}
A &= \begin{bmatrix}
a_1 & a_2 \\
a_3 & a_4
\end{bmatrix} \\ 
 X &= \begin{bmatrix}
x_1 & x_2 \\
x_3 & x_4
\end{bmatrix} \\
 C &=  \begin{bmatrix}
c_1 & c_2 \\
c_3 & c_4
\end{bmatrix}
\end{align*}
Where $C=AX$, thus-
\begin{align*}
c_1&=f(x_1, x_3) \\
c_2&=f(x_2, x_4) \\
c_3&=f(x_1, x_3) \\
c_4&=f(x_2, x_4) \\
\end{align*}
Therefore-
\begin{align*}
\frac{\partial \mathfrak{L}}{\partial x_1} &= \frac{\partial \mathfrak{L}}{\partial c_1} \times \frac{\partial c_1}{\partial x_1} + \frac{\partial \mathfrak{L}}{\partial c_2} \times \frac{\partial c_2}{\partial x_1} + \frac{\partial \mathfrak{L}}{\partial c_3} \times \frac{\partial c_3}{\partial x_1} + \frac{\partial \mathfrak{L}}{\partial c_4} \times \frac{\partial c_4}{\partial x_1}\\
&= \frac{\partial \mathfrak{L}}{\partial c_1} \times \frac{\partial \mathfrak{c_1}}{\partial x_1} + 0 + \frac{\partial \mathfrak{L}}{\partial c_3} \times \frac{\partial \mathfrak{c_3}}{\partial x_1} + 0\\
&= a_1 \times \frac{\partial \mathfrak{L}}{\partial c_1} + a_3 \times \frac{\partial \mathfrak{L}}{\partial c_3}\\
\end{align*}

Similarly-
\begin{align*}
\frac{\partial \mathfrak{L}}{\partial x_2}
&= a_1 \times \frac{\partial \mathfrak{L}}{\partial c_2} + a_3 \times \frac{\partial \mathfrak{L}}{\partial c_4}\\
\frac{\partial \mathfrak{L}}{\partial x_3}
&= a_2 \times \frac{\partial \mathfrak{L}}{\partial c_1} + a_4 \times \frac{\partial \mathfrak{L}}{\partial c_3}\\
\frac{\partial \mathfrak{L}}{\partial x_4}
&= a_2 \times \frac{\partial \mathfrak{L}}{\partial c_2} + a_4 \times \frac{\partial \mathfrak{L}}{\partial c_4}\\
\end{align*}
This can be expressed as a matrix equation in the following manner-
\begin{align*}
\frac{\partial \mathfrak{L}}{\partial X} &= \frac{\partial C}{\partial X} \times \frac{\partial \mathfrak{L}}{\partial C}\\
\implies \begin{bmatrix}
\frac{\partial \mathfrak{L}}{\partial x_1} & \frac{\partial \mathfrak{L}}{\partial x_2} \\
\frac{\partial \mathfrak{L}}{\partial x_3} & \frac{\partial \mathfrak{L}}{\partial x_4}
\end{bmatrix} &= \frac{\partial C}{\partial X} \times \begin{bmatrix}
\frac{\partial \mathfrak{L}}{\partial c_1} & \frac{\partial \mathfrak{L}}{\partial c_2} \\
\frac{\partial \mathfrak{L}}{\partial c_3} & \frac{\partial \mathfrak{L}}{\partial c_4}
\end{bmatrix}
\end{align*}
By comparing the individual partial derivatives computed earlier, we can say-

\begin{align*}
\begin{bmatrix}
\frac{\partial \mathfrak{L}}{\partial x_1} & \frac{\partial \mathfrak{L}}{\partial x_2} \\
\frac{\partial \mathfrak{L}}{\partial x_3} & \frac{\partial \mathfrak{L}}{\partial x_4}
\end{bmatrix} &= \begin{bmatrix}
a_1 & a_3 \\
a_2 & a_4
\end{bmatrix} \times \begin{bmatrix}
\frac{\partial \mathfrak{L}}{\partial c_1} & \frac{\partial \mathfrak{L}}{\partial c_2} \\
\frac{\partial \mathfrak{L}}{\partial c_3} & \frac{\partial \mathfrak{L}}{\partial c_4}
\end{bmatrix}\\
\implies \begin{bmatrix}
\frac{\partial \mathfrak{L}}{\partial x_1} & \frac{\partial \mathfrak{L}}{\partial x_2} \\
\frac{\partial \mathfrak{L}}{\partial x_3} & \frac{\partial \mathfrak{L}}{\partial x_4}
\end{bmatrix} &= A^T \times \begin{bmatrix}
\frac{\partial \mathfrak{L}}{\partial c_1} & \frac{\partial \mathfrak{L}}{\partial c_2} \\
\frac{\partial \mathfrak{L}}{\partial c_3} & \frac{\partial \mathfrak{L}}{\partial c_4}
\end{bmatrix}\\
\implies \frac{\partial \mathfrak{L}}{\partial X} &= A^T \times \frac{\partial \mathfrak{L}}{\partial C}\\
\therefore \frac{\partial C}{\partial X} &= A^T \qed
\end{align*}

%%%%%%%%%%%%%%%%%%%%%%%%%%%%%%%%%%%%%%%%%%%%%%%%%%%%%%%%%%%%%%%%%%%%%%%%%%%%%%%%%%%%%%%%%%%%%%
\end{document}
