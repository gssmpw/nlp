\typeout{IJCAI--25 Instructions for Authors}


\documentclass{article}
\pdfpagewidth=8.5in
\pdfpageheight=11in

\usepackage{ijcai25}

\usepackage{times}
\usepackage{soul}
\usepackage{url}
\usepackage[hidelinks]{hyperref}
\usepackage[utf8]{inputenc}
\usepackage[small]{caption}
\usepackage{graphicx}
\usepackage{amsmath}
\usepackage{amsthm}
\usepackage{booktabs}
\usepackage{algorithm}
\usepackage{algorithmic}
\usepackage[switch]{lineno}
\usepackage{xcolor}
\usepackage{float}
\usepackage{multirow}
\usepackage{makecell}
\usepackage{tikz}
\usepackage{tabularx}
\usepackage[edges]{forest}
\usepackage{pifont}
\usepackage{natbib}
\usepackage{mydef}
\newcommand{\cmark}{\ding{51}}
\newcommand{\xmark}{\ding{55}}

% \linenumbers

\urlstyle{same}


\newtheorem{example}{Example}
\newtheorem{theorem}{Theorem}


\pdfinfo{
/TemplateVersion (IJCAI.2025.0)
}

\title{Grounding Creativity in Physics: A Brief Survey of Physical Priors in AIGC}

\author{
Siwei Meng$^1$\and
Yawei Luo$^{2*}$\and
Ping Liu$^1$\thanks{Corresponding authors}\\
\affiliations
$^1$Department of Computer Science, University of Nevada, Reno\\
$^2$School of Software Technology, Zhejiang University
\emails
siweim@unr.edu,
yaweiluo@zju.edu.cn,
pino.pingliu@gmail.com
}

\begin{document}

\maketitle

\begin{abstract}
    Recent advancements in AI-generated content have significantly improved the realism of 3D and 4D generation. 
    However, most existing methods prioritize appearance consistency while neglecting underlying physical principles, leading to artifacts such as unrealistic deformations, unstable dynamics, and implausible objects interactions. 
    Incorporating physics priors into generative models has become a crucial research direction to enhance structural integrity and motion realism.
    This survey provides a review of physics-aware generative methods, systematically analyzing how physical constraints are integrated into 3D and 4D generation. 
    First, we examine recent works in incorporating physical priors into static and dynamic 3D generation, categorizing methods based on representation types, including vision-based, NeRF-based, and Gaussian Splatting-based approaches. 
    Second, we explore emerging techniques in 4D generation, focusing on methods that model temporal dynamics with physical simulations. 
    Finally, we conduct a comparative analysis of major methods, highlighting their strengths, limitations, and suitability for different materials and motion dynamics.
    By presenting an in-depth analysis of physics-grounded AIGC, this survey aims to bridge the gap between generative models and physical realism, providing insights that inspire future research in physically consistent content generation. 
\end{abstract}

\begin{table*}[htbp]
    \caption{Comparison between this work and previous surveys.}
    \label{tab:survey_comparison}
    \setlength{\tabcolsep}{10pt}
    \renewcommand{\arraystretch}{1.2}
    \centering
    \scalebox{0.8}{
    \begin{tabular}{c|c|c|c|c}
    \toprule
         \textbf{Works}& \textbf{Focused Tasks} & \textbf{Physics Priors}& \textbf{4D Generation}&\textbf{Released Date} \\
    \midrule
         \citep{cao2023comprehensive_arxiv2023}      & Generative Models in AIGC & \xmark & \xmark & Mar 2023     \\
         \citep{li2024advances_arxiv2024}        & 3D Generation             & \xmark & \xmark & Jan 2024     \\
         \citep{liu2024comprehensive_arxiv2024}     & 3D Generation             & \xmark & \xmark & Feb 2024     \\
         \citep{banerjee2024physics_acmcomput2024} & Physics-informed Computer Vision Models       & \cmark & \xmark & Oct 2024     \\
         \citep{liu2025generative_arxiv2025}      & Physical Simulation with Generative Models  & \cmark & \xmark & Jan 2025     \\
        \textbf{Ours} & \textbf{Physics-aware AIGC}  & \textbf{\cmark} & \textbf{\cmark}  &\textbf{Feb 2025} \\
    \bottomrule
    \end{tabular}}
\end{table*}

\section{Introduction}
Recent advancements in AI-generated content (AIGC) have significantly enhanced 3D and 4D content generation, with applications spanning gaming, simulation, animation, and robotics. 
Traditional AI-driven 3D generation methods primarily focus on improving geometric fidelity and rendering efficiency, leveraging representations such as Neural Radiance Fields (NeRF) \citep{mildenhall2021nerf_eccv2020} and 3D Gaussian Splatting (GS) \citep{kerbl20233d_tog2023}.
More recent models, including DreamFusion \citep{poole2022dreamfusion_iclr2023} and Magic3D \citep{lin2023magic3d_cvpr2023}, integrate Diffusion Models (DMs) to improve synthesis realism. 
However, these models primarily optimize for visual quality, often neglecting physical plausibility, leading to artifacts such as implausible deformations, unstable motion, and inconsistent object interactions.

The integration of physics priors into generative models is an emerging yet underdeveloped direction in 3D/4D generation. Most generative models are trained on datasets lacking explicit physical constraints, failing to capture material properties, object dynamics, and force interactions. 
Consequently, generated content often deviates from real-world physical laws, limiting its applicability in simulation-driven applications. 
To address this gap, recent research has explored differentiable physics-based models such as Material Point Method (MPM) \citep{jiang2016material_siggraph2016,hu2018moving_tog2018}, Finite Element Method (FEM) \citep{zienkiewicz2000finite_2000}, and differentiable physics engines into the generative pipeline. 
These approaches enable physics-informed content generation, ensuring structural integrity, dynamic realism, and physically consistent interactions.
However, a systematic review of these physics-based advances in generative models remains absent. 

Existing surveys on 3D and 4D generation primarily focus on three aspects: (1) scene representations and rendering \citep{li2024advances_arxiv2024}, which discuss different 3D representations and rendering optimizations; (2) generative models \citep{cao2023comprehensive_arxiv2023,liu2024comprehensive_arxiv2024}, which analyze synthesis techniques from text and images; and (3) applications and scalability \citep{li2024advances_arxiv2024}, covering gaming, animation, and robotics. 
However, none of these works systematically explore the role of physics priors in generative models. 
A recent survey \citep{liu2025generative_arxiv2025} provides the closest discussion to our work, categorizing physics-aware generation into explicit physics simulation-based (PAG-E) and implicit physics-informed (PAG-I) methods.
However, its focus remains on physics-aware video and 3D content generation, with limited discussion on 4D generation and dynamic scene modeling. 
In contrast, our survey systematically explores physics integration across static 3D, dynamic 3D, and 4D generation, offering a broader perspective on incorporating physical principles.

To bridge this gap, this survey provides an overview of physics-grounded generative models, categorizing recent advances into static 3D generation, dynamic 3D generation, and 4D generation. 
We introduce a taxonomy that organizes existing methods based on representation types and generation paradigms, covering vision-based, NeRF-based, and GS-based approaches (Figure \ref{fig:taxonomy}). 
By synthesizing insights from physics-based simulation and generative models, we aim to establish a systematic perspective on integrating physics priors with AI-driven content generation, offering new directions for research in this evolving field.

% \section{Taxonomy}

% As illustrated by Fig. \ref{}, the typical process of vision models based time series analysis has five components: (1) normalization/scaling; (2) time series to image transformation; (3) image modeling; (4) image to time series recovery; and (5) task processing. In the rest of this paper, we will discuss the typical methods for each of these components. The detailed taxonomy of the methods are summarized in Table \ref{tab.taxonomy}.

%Typical step: normalization/scaling, transformation, vision modeling, task-specific head, inverse transformation (for tasks that output time series, e.g., forecasting, generation, imputation, anomaly detection). Normalization is to fit the arbitrary range of time series values to RGB representation.

\begin{figure*}[!t]
\centering
\includegraphics[width=1.0\textwidth]{fig/fig_3.pdf}
% \vspace{-1em}
\caption{An illustration of different methods for imaging time series with a sample (length=336) from the \textit{Electricity} benchmark dataset \protect\cite{nie2023time}. (a)(c)(d)(e)(f) %are univariate methods.
visualize the same variate. (b) visualizes all 321 variates. Filterbank is omitted due to its %high
similarity to STFT.}\label{fig.tsimage}
\vspace{-0.2cm}
\end{figure*}

\begin{table*}[t]
\centering
\scriptsize
\setlength{\tabcolsep}{2.7pt}{
% \begin{tabular}{llllllllllll}
\begin{tabular}{llcccccccccl}
\toprule[1pt]
\multirow{2}{*}{Method} & \multirow{2}{*}{TS-Type} & \multirow{2}{*}{Imaging} & \multicolumn{5}{c}{Imaged Time Series Modeling} & \multirow{2}{*}{TS-Recover} & \multirow{2}{*}{Task} & \multirow{2}{*}{Domain} & \multirow{2}{*}{Code}\\ \cmidrule{4-8}
 & & & Multi-modal & Model & Pre-trained & Fine-tune & Prompt & & & & \\ \midrule
\cite{silva2013time} & UTS & RP & \xmark & \texttt{K-NN} & \xmark & \xmark & \xmark & \xmark & Classification & General & \xmark\\
\cite{wang2015encoding} & UTS & GAF & \xmark & \texttt{CNN} & \xmark & \cmark$^{\flat}$ & \xmark & \cmark & Classification & General & \xmark\\
\cite{wang2015imaging} & UTS & GAF & \xmark & \texttt{CNN} & \xmark & \cmark$^{\flat}$ & \xmark & \cmark & Multiple & General & \xmark\\
% \multirow{2}{*}{\cite{wang2015imaging}} & \multirow{2}{*}{UTS} & \multirow{2}{*}{GAF} & \multirow{2}{*}{\xmark} & \multirow{2}{*}{\texttt{CNN}} & \multirow{2}{*}{\xmark} & \multirow{2}{*}{\cmark$^{\flat}$} & \multirow{2}{*}{\xmark} & \multirow{2}{*}{\cmark} & Classification & \multirow{2}{*}{General} & \multirow{2}{*}{\xmark}\\
% & & & & & & & & & \& Imputation & & \\
\cite{ma2017learning} & MTS & Heatmap & \xmark & \texttt{CNN} & \xmark & \cmark$^{\flat}$ & \xmark & \cmark & Forecasting & Traffic & \xmark\\
\cite{hatami2018classification} & UTS & RP & \xmark & \texttt{CNN} & \xmark & \cmark$^{\flat}$ & \xmark & \xmark & Classification & General & \xmark\\
\cite{yazdanbakhsh2019multivariate} & MTS & Heatmap & \xmark & \texttt{CNN} & \xmark & \cmark$^{\flat}$ & \xmark & \xmark & Classification & General & \cmark\textsuperscript{\href{https://github.com/SonbolYb/multivariate_timeseries_dilated_conv}{[1]}}\\
MSCRED \cite{zhang2019deep} & MTS & Other ($\S$\ref{sec.othermethod}) & \xmark & \texttt{ConvLSTM} & \xmark & \cmark$^{\flat}$ & \xmark & \xmark & Anomaly & General & \cmark\textsuperscript{\href{https://github.com/7fantasysz/MSCRED}{[2]}}\\
\cite{li2020forecasting} & UTS & RP & \xmark & \texttt{CNN} & \cmark & \cmark & \xmark & \xmark & Forecasting & General & \cmark\textsuperscript{\href{https://github.com/lixixibj/forecasting-with-time-series-imaging}{[3]}}\\
\cite{cohen2020trading} & UTS & LinePlot & \xmark & \texttt{Ensemble} & \xmark & \cmark$^{\flat}$ & \xmark & \xmark & Classification & Finance & \xmark\\
% \cite{du2020image} & UTS & Spectrogram & \xmark & \texttt{CNN} & \xmark & \cmark$^{\flat}$ & \xmark & \xmark & Classification & Finance & \xmark\\
\cite{barra2020deep} & UTS & GAF & \xmark & \texttt{CNN} & \xmark & \cmark$^{\flat}$ & \xmark & \xmark & Classification & Finance & \xmark\\
% \cite{barra2020deep} & UTS & GAF & \xmark & \texttt{VGG-16} & \xmark & \cmark$^{\flat}$ & \xmark & \xmark & Classification & Finance & \xmark\\
% \cite{cao2021image} & UTS & RP & \xmark & \texttt{CNN} & \xmark & \cmark$^{\flat}$ & \xmark & \xmark & Classification & General & \xmark\\
VisualAE \cite{sood2021visual} & UTS & LinePlot & \xmark & \texttt{CNN} & \xmark & \cmark$^{\flat}$ & \xmark & \cmark & Forecasting & Finance & \xmark\\
% VisualAE \cite{sood2021visual} & UTS & LinePlot & \xmark & \texttt{CNN} & \xmark & \cmark$^{\flat}$ & \xmark & \xmark & Img-Generation & Finance & \xmark\\
\cite{zeng2021deep} & MTS & Heatmap & \xmark & \texttt{CNN,LSTM} & \xmark & \cmark$^{\flat}$ & \xmark & \cmark & Forecasting & Finance & \xmark\\
% \cite{zeng2021deep} & MTS & Heatmap & \xmark & \texttt{SRVP} & \xmark & \cmark$^{\flat}$ & \xmark & \cmark & Forecasting & Finance & \xmark\\
AST \cite{gong2021ast} & UTS & Spectrogram & \xmark & \texttt{DeiT} & \cmark & \cmark & \xmark & \xmark & Classification & Audio & \cmark\textsuperscript{\href{https://github.com/YuanGongND/ast}{[4]}}\\
TTS-GAN \cite{li2022tts} & MTS & Heatmap & \xmark & \texttt{ViT} & \xmark & \cmark$^{\flat}$ & \xmark & \cmark & Ts-Generation & Health & \cmark\textsuperscript{\href{https://github.com/imics-lab/tts-gan}{[5]}}\\
SSAST \cite{gong2022ssast} & UTS & Spectrogram & \xmark & \texttt{ViT} & \cmark$^{\natural}$ & \cmark & \xmark & \xmark & Classification & Audio & \cmark\textsuperscript{\href{https://github.com/YuanGongND/ssast}{[6]}}\\
MAE-AST \cite{baade2022mae} & UTS & Spectrogram & \xmark & \texttt{MAE} & \cmark$^{\natural}$ & \cmark & \xmark & \xmark & Classification & Audio & \cmark\textsuperscript{\href{https://github.com/AlanBaade/MAE-AST-Public}{[7]}}\\
AST-SED \cite{li2023ast} & UTS & Spectrogram & \xmark & \texttt{SSAST,GRU} & \cmark & \cmark & \xmark & \xmark & EventDetection & Audio & \xmark\\
\cite{jin2023classification} & UTS & %Multiple
LinePlot & \xmark & \texttt{CNN} & \cmark & \cmark & \xmark & \xmark & Classification & Physics & \xmark\\
ForCNN \cite{semenoglou2023image} & UTS & LinePlot & \xmark & \texttt{CNN} & \xmark & \cmark$^{\flat}$ & \xmark & \xmark & Forecasting & General & \xmark\\
Vit-num-spec \cite{zeng2023pixels} & UTS & Spectrogram & \xmark & \texttt{ViT} & \xmark & \cmark$^{\flat}$ & \xmark & \xmark & Forecasting & Finance & \xmark\\
% \cite{wimmer2023leveraging} & MTS & LinePlot & \xmark & \texttt{CLIP,LSTM} & \cmark & \cmark & \xmark & \xmark & Classification & Finance & \xmark\\
ViTST \cite{li2023time} & MTS & LinePlot & \xmark & \texttt{Swin} & \cmark & \cmark & \xmark & \xmark & Classification & General & \cmark\textsuperscript{\href{https://github.com/Leezekun/ViTST}{[8]}}\\
MV-DTSA \cite{yang2023your} & UTS\textsuperscript{*} & LinePlot & \xmark & \texttt{CNN} & \xmark & \cmark$^{\flat}$ & \xmark & \cmark & Forecasting & General & \cmark\textsuperscript{\href{https://github.com/IkeYang/machine-vision-assisted-deep-time-series-analysis-MV-DTSA-}{[9]}}\\
TimesNet \cite{wu2023timesnet} & MTS & Heatmap & \xmark & \texttt{CNN} & \xmark & \cmark$^{\flat}$ & \xmark & \cmark & Multiple & General & \cmark\textsuperscript{\href{https://github.com/thuml/TimesNet}{[10]}}\\
ITF-TAD \cite{namura2024training} & UTS & Spectrogram & \xmark & \texttt{CNN} & \cmark & \xmark & \xmark & \xmark & Anomaly & General & \xmark\\
\cite{kaewrakmuk2024multi} & UTS & GAF & \xmark & \texttt{CNN} & \cmark & \cmark & \xmark & \xmark & Classification & Sensing & \xmark\\
HCR-AdaAD \cite{lin2024hierarchical} & MTS & RP & \xmark & \texttt{CNN,GNN} & \xmark & \cmark$^{\flat}$ & \xmark & \xmark & Anomaly & General & \xmark\\
FIRTS \cite{costa2024fusion} & UTS & Other ($\S$\ref{sec.othermethod}) & \xmark & \texttt{CNN} & \xmark & \cmark$^{\flat}$ & \xmark & \xmark & Classification & General & \cmark\textsuperscript{\href{https://sites.google.com/view/firts-paper}{[11]}}\\
% \multirow{2}{*}{FIRTS \cite{costa2024fusion}} & \multirow{2}{*}{UTS} & Spectrogram & \multirow{2}{*}{\xmark} & \multirow{2}{*}{\texttt{CNN}} & \multirow{2}{*}{\xmark} & \multirow{2}{*}{\cmark$^{\flat}$} & \multirow{2}{*}{\xmark} & \multirow{2}{*}{\xmark} & \multirow{2}{*}{Classification} & \multirow{2}{*}{General} & \multirow{2}{*}{\cmark\textsuperscript{\href{https://sites.google.com/view/firts-paper}{[2]}}}\\
%  & & \& GAF,RP,MTF & & & & & & & & & \\
% \cite{homenda2024time} & UTS\textsuperscript{*} & Multiple & \xmark & \texttt{CNN} & \xmark & \cmark$^{\flat}$ & \xmark & \xmark & Classification & General & \xmark\\
CAFO \cite{kim2024cafo} & MTS & RP & \xmark & \texttt{CNN,ViT} & \xmark & \cmark$^{\flat}$ & \xmark & \xmark & Explanation & General & \cmark\textsuperscript{\href{https://github.com/eai-lab/CAFO}{[12]}}\\
% \multirow{2}{*}{CAFO \cite{kim2024cafo}} & \multirow{2}{*}{MTS} & \multirow{2}{*}{RP} & \multirow{2}{*}{\xmark} & \texttt{ShuffleNet,ResNet} & \multirow{2}{*}{\cmark} & \multirow{2}{*}{\cmark} & \multirow{2}{*}{\xmark} & \multirow{2}{*}{\xmark} & Classification & \multirow{2}{*}{General} & \multirow{2}{*}{\cmark}\\
%  & & & & \texttt{MLP-Mixer,ViT} & & & & & \& Explanation & & \\
ViTime \cite{yang2024vitime} & UTS\textsuperscript{*} & LinePlot & \xmark & \texttt{ViT} & \cmark$^{\natural}$ & \cmark & \xmark & \cmark & Forecasting & General & \cmark\textsuperscript{\href{https://github.com/IkeYang/ViTime}{[13]}}\\
ImagenTime \cite{naiman2024utilizing} & MTS & Other ($\S$\ref{sec.othermethod}) & \xmark & %\texttt{Diffusion}
\texttt{CNN} & \xmark & \cmark$^{\flat}$ & \xmark & \cmark & Ts-Generation & General & \cmark\textsuperscript{\href{https://github.com/azencot-group/ImagenTime}{[14]}}\\
TimEHR \cite{karami2024timehr} & MTS & Heatmap & \xmark & \texttt{CNN} & \xmark & \cmark$^{\flat}$ & \xmark & \cmark & Ts-Generation & Health & \cmark\textsuperscript{\href{https://github.com/esl-epfl/TimEHR}{[15]}}\\
VisionTS \cite{chen2024visionts} & UTS\textsuperscript{*} & Heatmap & \xmark & \texttt{MAE} & \cmark & \cmark & \xmark & \cmark & Forecasting & General & \cmark\textsuperscript{\href{https://github.com/Keytoyze/VisionTS}{[16]}}\\ \midrule
InsightMiner \cite{zhang2023insight} & UTS & LinePlot & \cmark & \texttt{LLaVA} & \cmark & \cmark & \cmark & \xmark & Txt-Generation & General & \xmark\\
\cite{wimmer2023leveraging} & MTS & LinePlot & \cmark & \texttt{CLIP,LSTM} & \cmark & \cmark & \xmark & \xmark & Classification & Finance & \xmark\\
% \cite{dixit2024vision} & UTS & Spectrogram & \cmark & \texttt{GPT4o,Gemini} & \cmark & \xmark & \cmark & \xmark & Classification & Audio & \xmark\\
\multirow{2}{*}{\cite{dixit2024vision}} & \multirow{2}{*}{UTS} & \multirow{2}{*}{Spectrogram} & \multirow{2}{*}{\cmark} & \texttt{GPT4o,Gemini} & \multirow{2}{*}{\cmark} & \multirow{2}{*}{\xmark} & \multirow{2}{*}{\cmark} & \multirow{2}{*}{\xmark} & \multirow{2}{*}{Classification} & \multirow{2}{*}{Audio} & \multirow{2}{*}{\xmark}\\
 & & & & \& \texttt{Claude3} & & & & & & & \\
\cite{daswani2024plots} & MTS & LinePlot & \cmark & \texttt{GPT4o,Gemini} & \cmark & \xmark & \cmark & \xmark & Multiple & General & \xmark\\
TAMA \cite{zhuang2024see} & UTS & LinePlot & \cmark & \texttt{GPT4o} & \cmark & \xmark & \cmark & \xmark & Anomaly & General & \xmark\\
\cite{prithyani2024feasibility} & MTS & LinePlot & \cmark & \texttt{LLaVA} & \cmark & \cmark & \cmark & \xmark & Classification & General & \cmark\textsuperscript{\href{https://github.com/vinayp17/VLM_TSC}{[17]}}\\
\bottomrule[1pt]
\end{tabular}}
\vspace{-0.25cm}
\caption{Taxonomy of vision models on time series. The top panel includes single-modal models. The bottom panel includes multi-modal models. {\bf TS-Type} denotes type of time series. {\bf TS-Recover} denotes %whether time series recovery ($\S$\ref{sec.processing}) has been performed.
recovering time series from predicted images ($\S$\ref{sec.processing}). \textsuperscript{*}: %the model has been %applied on MTSs by %processing %modeling the individual UTSs of each MTS.
the method has been used to model the individual UTSs of an MTS. $^{\natural}$: a new pre-trained model was proposed in the work. $^{\flat}$: %without using a pre-trained model, fine-tune means training from scratch.
when pre-trained models were unused, ``Fine-tune'' refers to train a task-specific model from scratch. %In the
{\bf Model} column: \texttt{CNN} could be regular CNN, ResNet, VGG-Net, %U-Net,
{\em etc.}}\label{tab.taxonomy}
% The code only include verified official code from the authors.
\vspace{-0.3cm}
\end{table*}

\begin{table*}[t]
\centering
\small
\setlength{\tabcolsep}{2.9pt}{
\begin{tabular}{l|l|l|l}\hline
% \toprule[1pt]
\rowcolor{gray!20}
{\bf Method} & {\bf TS-Type} & {\bf Advantages} & {\bf Limitations}\\ \hline
Line Plot ($\S$\ref{sec.lineplot}) & UTS, MTS & matches human perception of time series & limited to MTSs with a small number of variates\\ \hline
Heatmap ($\S$\ref{sec.heatmap}) & UTS, MTS & straightforward for both UTSs and MTSs & the order of variates may affect their correlation learning\\ \hline
Spectrogram ($\S$\ref{sec.spectrogram}) & UTS & encodes the time-frequency space & limited to UTSs; needs a proper choice of window/wavelet\\ \hline
GAF ($\S$\ref{sec.gaf}) & UTS & encodes the temporal correlations in a UTS & limited to UTSs; $O(T^{2})$ time and space complexity\\ \hline% for long time series\\ \hline
% RP ($\S$\ref{sec.rp}) & UTS & flexibility in image size by tuning $m$ and $\tau$ & limited to UTSs; the pattern has a threshold-dependency\\ \hline
RP ($\S$\ref{sec.rp}) & UTS & flexibility in image size by tuning $m$ and $\tau$ & limited to UTSs; information loss after thresholding\\ \hline
% \bottomrule[1pt]
\end{tabular}}
\vspace{-0.2cm}
\caption{Summary of the five primary methods for transforming time series to images. {\bf TS-Type} denotes type of time series.}\label{tab.tsimage}
\vspace{-0.2cm}
\end{table*}

\section{Time Series To Image Transformation}\label{sec.tsimage}

% This section summarizes 5 major methods for imaging time series ($\S$\ref{sec.lineplot}-$\S$\ref{sec.rp}). We also discuss some other methods ($\S$\ref{sec.othermethod}) and how to model MTS with these methods ($\S$\ref{sec.modelmts}).
This section summarizes the methods for imaging time series ($\S$\ref{sec.lineplot}-$\S$\ref{sec.othermethod}) and their extensions to encode MTSs ($\S$\ref{sec.modelmts}).

% This section summarizes 5 major methods for transforming time series to images, including Line Plot, Heatmap, Spetrogram, GAF and RP, and several minor methods. We discuss their pros and cons and how to deal with MTS.

% This section discusses the advantages and limitations of different methods for time series to image transformation (invertible, efficiency, information preservation, MTS, long-range time series, parametric, etc.).

%\subsection{Methods}

\vspace{-0.08cm}

\subsection{Line Plot}\label{sec.lineplot}

Line Plot is a straightforward way for visualizing UTSs for human analysis ({\em e.g.}, stocks, power consumption, {\em etc.}). As illustrated by Fig. \ref{fig.tsimage}(a), the simplest approach is to draw a 2D image with x-axis representing %the time horizon
time steps and y-axis representing %the magnitude of the normalized time series.
time-wise values, %A line is used to connect all values of the series over time.
with a line connecting all values of the series over time. This image can be %represented by either three-channel pixels or single-channel pixels
either three-channel ({\em i.e.}, RGB) or single-channel as the colors may not %provide additional information
be informative %\cite{cohen2020trading,sood2021visual,jin2023classification,zhang2023insight,zhuang2024see}.
\cite{cohen2020trading,sood2021visual,jin2023classification,zhang2023insight}. ForCNN \cite{semenoglou2023image} even uses a single 8-bit integer to represent each pixel for black-white images. So far, there is no consensus on whether other graphical components, such as legend, grids and tick labels, could provide extra benefits in any task. For example, ViTST \cite{li2023time} finds these components are superfluous in a classification task, while TAMA \cite{zhuang2024see} finds grid-like auxiliary lines help enhance anomaly detection.

In addition to the regular Line Plot, MV-DTSA \cite{yang2023your} and ViTime \cite{yang2024vitime} divide an image into $h\times L$ grids, %where $h$ is the number of rows and $L$ is the number of columns,
and %introduced
define a function to map each time step of a UTS to a grid, producing a grid-like Line Plot. Also, we include methods that use Scatter Plot \cite{daswani2024plots,prithyani2024feasibility} in this category because %the only difference between a Scatter Plot and a Line Plot is whether the time-wise values are connected by lines.
a Scatter Plot resembles a Line Plot but doesn't connect %time-wise values
data points with a line. By comparing them, \cite{prithyani2024feasibility} finds a Line Plot could induce better time series classification.

For MTSs, we defer the discussion on Line Plot to $\S$\ref{sec.modelmts}.

% For MTS, some methods use the channel-independence assumption proposed in \cite{nie2023time} and represent each variate in MTS with an individual Line Plot \cite{yang2023your,yang2024vitime}. ViTST \cite{li2023time} also uses an individual Line Plot per variate, but colors different lines and assembles all plots to form a bigger image. The method in \cite{wimmer2023leveraging} plots %the time series of
% all variates in a single Line Plot and distinguish them by %use different
% types of lines ({\em e.g.}, solid, dashed, dotted, {\em etc.}). %to distinguish them.
% However, these methods only work for a small number of variates. For example, in \cite{wimmer2023leveraging}, there are only 4 variates in its financial MTSs.

%\cite{li2023time} space-costly because of blank pixels. scatter plot.

%Invertible with a numeric prediction head \cite{sood2021visual}. It fits tasks such as forecasting, imputation, etc.

\vspace{-0.08cm}

\subsection{Heatmap}\label{sec.heatmap}

As shown in Fig. \ref{fig.tsimage}(b), Heatmap is a 2D visualization of the magnitude of the values in a matrix using color. %The variation of color represents the intensity of each value. %Therefore,
It has been used to %directly
represent the matrix of an MTS, {\em i.e.}, $\mat{X} \in \mathbb{R}^{d\times T}$, as a one-channel $d\times T$ image \cite{li2022tts,yazdanbakhsh2019multivariate}. Similarly, TimEHR \cite{karami2024timehr} represents an {\em irregular} MTS, where the intervals between time steps are uneven, as a $d\times H$ Heatmap image by grouping the uneven time steps into $H$ even time bins. In \cite{zeng2021deep}, a different method is used for visualizing a 9-variate financial %time series.
MTS. It reshapes the 9 variates at each time step to a $3\times 3$ Heatmap image, and uses the sequence of images to forecast future %image
frames, achieving %time series
%MTS
time series forecasting. In contrast, VisionTS \cite{chen2024visionts} uses Heatmap to visualize UTSs. %instead.
Similar to TimesNet \cite{wu2023timesnet}, it first segments a length-$T$ UTS into $\lfloor T/P\rfloor$ length-$P$ subsequences, where $P$ is a parameter representing a periodicity of the UTS. Then the subsequences are stacked into a $P\times \lfloor T/P\rfloor$ matrix, %and duplicated 3 times to produce a 3-channel
with 3 duplicated channels, to produce a grayscale image %which serves as an
input to %a vision foundation model.
an LVM. To encode MTSs, VisionTS adopts the channel independence assumption \cite{nie2023time} and individually models each variate in an MTS.

\vspace{0.2cm}

\noindent{\bf Remark.} Heatmap can be used to visualize matrices of various forms. It is also used for matrices generated by the subsequent methods ({\em e.g.}, Spectrogram, GAF, RP) in this section. In this paper, the name Heatmap refers specifically to images that use color to visualize the (normalized) values in UTS $\mat{x}$ or MTS $\mat{X}$ without performing other transformations.

%\cite{chen2024visionts,karami2024timehr} bin version of TSH \cite{karami2024timehr}, DE and STFT \cite{naiman2024utilizing} (DE can be used for constructing RP), rearrange variates for video version of TSH \cite{zeng2021deep}.

%\vspace{0.2cm}

\subsection{Spectrogram}\label{sec.spectrogram}

A {\em spectrogram} is a visual representation of the spectrum of frequencies of a signal as it varies with time, which are extensively used for analyzing audio signals \cite{gong2021ast}. Since audio signals are a type of UTS, spectrogram can be considered as a method for imaging a UTS. As shown in Fig. \ref{fig.tsimage}(c), a common format is a 2D heatmap image with x-axis representing time steps and y-axis representing frequency, {\em a.k.a.} a time-frequency space. %The color at each point
Each pixel in the image represents the (logarithmic) amplitude of a specific frequency at a specific time point. Typical methods for %transforming a UTS to
producing a spectrogram include {\bf Short-Time Fourier Transform (STFT)} \cite{griffin1984signal}, {\bf Wavelet Transform} \cite{daubechies1990wavelet}, and {\bf Filterbank} \cite{vetterli1992wavelets}.

\vspace{0.2cm}

\noindent{\bf STFT.} %Discrete Fourier transform (DFT) can be used to represent a UTS signal %$\mat{x}=[x_{1}, ..., x_{T}]$
%$\mat{x}\in\mathbb{R}^{1\times T}$ as a sum of sinusoidal components. The output of the transform is a function of frequency $f(w)$, describing the intensity of each constituent frequency $w$ of the entire UTS. 
Discrete Fourier transform (DFT) can be used to describe the intensity $f(w)$ of each constituent frequency $w$ of a UTS signal $\mat{x}\in\mathbb{R}^{1\times T}$. However, $f(w)$ has no time dependency. It cannot provide dynamic information such as when a specific frequency appear in the UTS. STFT addresses this deficiency by sliding a window function $g(t)$ over the time steps in %the UTS,
$\mat{x}$, and computing the DFT within each window by
\begin{equation}\label{eq.stft}
\small
\begin{aligned}
f(w,\tau) = \sum_{t=1}^{T}x_{t}g(t - \tau)e^{-iwt}
\end{aligned}
\end{equation}
where $w$ is frequency, $\tau$ is the position of the window, $f(w,\tau)$ describes the intensity of frequency $w$ at time step $\tau$.

%With a proper selection of the
By selecting a proper window function $g(\cdot)$ ({\em e.g.}, Gaussian/Hamming/Bartlett window), %({\em e.g.}, Gaussian window, Hamming window, Bartlett window), %{\em etc.}),
a 2D spectrogram ({\em e.g.}, Fig. \ref{fig.tsimage}(c)) can be drawn via a heatmap on the squared values $|f(w,\tau)|^{2}$, with $w$ as the y-axis, and $\tau$ as the x-axis. For example, \cite{dixit2024vision} uses STFT based spectrogram as an input to LMMs %\hh{do you mean LVMs? check}
for time series classification.

%Fourier transform is a powerful data analysis tool that represents any complex signal as a sum of sines and cosines and transforms the signal from the time domain to the frequency domain. However, Fourier transform can only show which frequencies are present in the signal, but not when these frequencies appear. The STFT divides original signal into several parts using a sliding window to fix this problem. STFT involves a sliding window for extracting frequency components within the window.

\vspace{0.2cm}

\noindent{\bf Wavelet Transform.} %Like Fourier transform, %\hh{this paragraph needs a citation}
Continuous Wavelet Transform (CWT) uses the inner product to measure the similarity between a signal function $x(t)$ and an analyzing function. %In STFT (Eq.~\eqref{eq.stft}), the analyzing function is a windowed exponential $g(t - \tau)e^{-iwt}$.
%In CWT,
The analyzing function is a {\em wavelet} $\psi(t)$, where the typical choices include Morse wavelet, Morlet wavelet, %Daubechies wavelet, %Beylkin wavelet, 
{\em etc.} %The
CWT compares $x(t)$ to the shifted and scaled ({\em i.e.}, stretched or shrunk) versions of the wavelet, and output a CWT coefficient by
\begin{equation}\label{eq.cwt}
\small
\begin{aligned}
c(s,\tau) = \int_{-\infty}^{\infty}x(t)\frac{1}{s}\psi^{*}(\frac{t - \tau}{s})dt
\end{aligned}
\end{equation}
where $*$ denotes complex conjugate, $\tau$ is the time step to shift, and $s$ represents the scale. In practice, a discretized version of CWT in Eq.~\eqref{eq.cwt} is implemented for UTS $[x_{1}, ..., x_{T}]$.

It is noteworthy that the scale $s$ controls the frequency encoded in a wavelet -- a larger $s$ leads to a stretched wavelet with a lower frequency, and vice versa. As such, by varying $s$ and $\tau$, a 2D spectrogram ({\em e.g.}, Fig. \ref{fig.tsimage}(d)) can be drawn %, often with a heatmap
on $|c(s,\tau)|$, where $s$ is the y-axis and $\tau$ is the x-axis. Compared to STFT, which uses a fixed window size, Wavelet Transform allows variable wavelet sizes -- a larger size %region
for more precise low frequency information. 
%Usually, $s$ and $\tau$ vary dependently -- a larger $s$ leads to a stretched wavelet that shifts slowly, {\em i.e.}, a smaller $\tau$. This property %of CWT
%yields a spectrogram that balances the resolutions of frequency %$s$
%and time, %$\tau$,
%which is an advantage over the fixed time resolution in STFT.
% Thus, both of the methods in %\cite{du2020image}
% \cite{namura2024training} and \cite{zeng2023pixels} choose CWT (with Morlet wavelet) to generate the spectrogram.
Thus, the methods in \cite{du2020image,namura2024training,zeng2023pixels} choose CWT (with Morlet wavelet) to generate the spectrogram.

%A wavelet is a wave-like oscillation that has zero mean and is localized in both time and frequency space.

\vspace{0.2cm}

\noindent{\bf Filterbank.} This method %is relevant to
resembles STFT and is often used in processing audio signals. Given an audio signal, it firstly goes through a {\em pre-emphasis filter} to boost high frequencies, which helps improve the clarity of the signal. Then, STFT is applied on the signal. %with a sliding window $g(t)$ of size $k$ that shifts in a fixed stride $\tau$. %where the adjacent windows may overlap in $k$ time length.
%Finally, filterbank features are computed by applying multiple ``triangle-shaped'' filters spaced on the Mel-scale to the STFT output $f(w, \tau)$. %where Mel-scale is a method to make the filters more discriminative on lower frequencies, %than higher frequencies,
%imitating the non-linear human ear perception of sound.
Finally, multiple ``triangle-shaped'' filters spaced on a Mel-scale are applied to the STFT power spectrum $|f(w, \tau)|^{2}$ to extract frequency bands. The outcome filterbank features $\hat{f}(w, \tau)$ can be used to yield a spectrogram with $w$ as the y-axis, and $\tau$ as the x-axis.

%Filterbank was introduced in AST \cite{gong2021ast} with %$k$=25ms
Filterbank was adopted in AST \cite{gong2021ast} with 
a 25ms Hamming window that shifts every 10ms for classifying audio signals using Vision Transformer (ViT). It then becomes widely used in the follow-up works such as SSAST \cite{gong2022ssast}, MAE-AST \cite{baade2022mae}, and AST-SED \cite{li2023ast}, as summarized in Table \ref{tab.taxonomy}.



%Use MLP to predict TS directly \cite{zeng2023pixels}.

%\vspace{0.2cm}

% \vspace{0.2cm}

\subsection{Gramian Angular Field (GAF)}\label{sec.gaf}

GAF was introduced for classifying UTSs using CNNs %using %image based CNNs
by \cite{wang2015encoding}. It was then extended %with an extension
to an imputation task in \cite{wang2015imaging}. Similarly, \cite{barra2020deep} applied GAF for financial time series forecasting.

Given a UTS $\mat{x}\in\mathbb{R}^{1\times T}$, %$[x_{1}, ..., x_{T}]$,
the first step %before GAF
is to rescale each $x_{t}$ to a value $\tilde{x}_{t}$ %in the interval of
within $[0, 1]$ (or $[-1, 1]$). %by min-max normalization.
This range enables mapping $\tilde{x}_{t}$ to polar coordinates by $\phi_{t}=\text{arccos}(\tilde{x}_{i})$, with a radius $r=t/N$ encoding the time stamp, where $N$ is a constant factor to regularize the span of the polar coordinates. %system. Then,
Two types of GAF, Gramian Sum Angular Field (GASF) and Gramian Difference Angular Field (GADF) are defined as
\begin{equation}\label{eq.gaf}
\small
\begin{aligned}
&\text{GASF:}~~\text{cos}(\phi_{t} + \phi_{t'})=x_{t}x_{t'} - \sqrt{1 - x_{t}^{2}}\sqrt{1 - x_{t'}^{2}}\\
&\text{GADF:}~~\text{sin}(\phi_{t} - \phi_{t'})=x_{t'}\sqrt{1 - x_{t}^{2}} - x_{t}\sqrt{1 - x_{t'}^{2}}
\end{aligned}
\end{equation}
which exploits the pairwise temporal correlations in the UTS. Thus, the outcome is a $T\times T$ matrix $\mat{G}$ with $\mat{G}_{t,t'}$ specified by either type in Eq.~\eqref{eq.gaf}. A GAF image is a heatmap on $\mat{G}$ with both axes representing time, as illustrated by Fig. \ref{fig.tsimage}(e).

% Invertible.

% \vspace{0.2cm}

\subsection{Recurrence Plot (RP)}\label{sec.rp}

%RP \cite{eckmann1987recurrence} is a method to encode a UTS into an image that aims to capture the periodic patterns in the UTS by using its reconstructed {\em phase space}. The phase space of a UTS $[x_{1}, ..., x_{T}]$ can be reconstructed by {\em time delay embedding}, which is a set of new vectors $\mat{v}_{1}$, ..., $\mat{v}_{l}$ with

RP \cite{eckmann1987recurrence} encodes a UTS into an image that captures its periodic patterns by using its reconstructed {\em phase space}. The phase space of %a UTS %$[x_{1}, ..., x_{T}]$
$\mat{x}\in\mathbb{R}^{1\times T}$ can be reconstructed by {\em time delay embedding} -- a set of new vectors $\mat{v}_{1}$, ..., $\mat{v}_{l}$ with
\begin{equation}\label{eq.de}
\small
\begin{aligned}
\mat{v}_{t}=[x_{t}, x_{t+\tau}, x_{t+2\tau}, ..., x_{t+(m-1)\tau}]\in\mathbb{R}^{m\tau},~~~1\le t \le l
\end{aligned}
\end{equation}
where $\tau$ is the time delay, $m$ is the dimension of the phase space, both %of which
are hyperparameters. Hence, $l=T-(m-1)\tau$. With vectors $\mat{v}_{1}$, ..., $\mat{v}_{l}$, an RP image %is constructed by measuring
measures their pairwise distances, results in an $l\times l$ image whose element
\begin{equation}\label{eq.rp}
\small
\begin{aligned}
\text{RP}_{i,j}=\Theta(\varepsilon - \|\mat{v}_{i} - \mat{v}_{j}\|),~~~1\le i,j\le l
\end{aligned}
\end{equation}
where $\Theta(\cdot)$ is the Heaviside step function, $\varepsilon$ is a threshold, and $\|\cdot\|$ is a norm function such as $\ell_{2}$ norm. Eq.~\eqref{eq.rp} %states RP produces a heatmap image on a binary matrix with $\text{RP}_{i,j}=1$ if $\mat{v}_{i}$ and $\mat{v}_{j}$ are sufficiently similar.
generates a binary matrix with $\text{RP}_{i,j}=1$ if $\mat{v}_{i}$ and $\mat{v}_{j}$ are sufficiently similar, producing a black-white image ({\em e.g.}, Fig. \ref{fig.tsimage}(f)).% ({\em e.g.}, a periodic pattern).

An advantage of RP is its flexibility in image size by tuning $m$ and $\tau$. Thus it has been used for time series classification %\cite{cao2021image},
\cite{silva2013time,hatami2018classification}, forecasting \cite{li2020forecasting}, anomaly detection \cite{lin2024hierarchical} and %feature-wise
explanation \cite{kim2024cafo}. Moreover, the method in \cite{hatami2018classification}, and similarly in HCR-AdaAD \cite{lin2024hierarchical}, omit the thresholding in Eq.~\eqref{eq.rp} and uses $\|\mat{v}_{i} - \mat{v}_{j}\|$ to produce continuously valued images %in a classification task
to avoid information loss.


% \vspace{0.2cm}

\subsection{Other Methods}\label{sec.othermethod}

%There are some less commonly used methods. For example, in
Additionally, %there are some peripheral methods. %In addition to GAF,
\cite{wang2015encoding} introduces Markov Transition Field (MTF) for imaging a UTS. %$\mat{x}\in\mathbb{R}^{1\times T}$. 
%MTF first assigns each $x_{t}$ to one of $Q$ quantile bins, then builds a $Q\times Q$ Markov transition matrix $\mat{M}$ {\em s.t.} $\mat{M}_{i,j}$ represents the frequency %with which
%of the case when a point $x_{t}$ in the $i$-th bin is followed by a point $x_{t'}$ in the $j$-th bin, {\em i.e.}, $t=t'+1$. Matrix $\mat{M}$ serves as the input of a heatmap image.
MTF is a matrix $\mat{M}\in\mathbb{R}^{Q\times Q}$ encoding the transition probabilities over time segments, where $Q$ is the number of segments. %Moreover,
ImagenTime \cite{naiman2024utilizing} stacks the delay embeddings $\mat{v}_{1}$, ..., $\mat{v}_{l}$ in Eq.~\eqref{eq.de} to an $l\times m\tau$ matrix for visualizing UTSs. %It also uses a variant of STFT.
% The method in \cite{homenda2024time} introduces five different 2D images by counting, rearranging, replicating the values in a UTS. 
MSCRED \cite{zhang2019deep} uses heatmaps on the $d\times d$ correlation matrices of MTSs with $d$ variates for anomaly detection. 
Furthermore, some methods use a mixture of imaging methods by stacking different transformations. \cite{wang2015imaging} stacks GASF, GADF, MTF to a 3-channel image. %Similarly,
FIRTS \cite{costa2024fusion} builds a 3-channel image by stacking GASF, MTF and RP. %the GASF, MTF, RP representations of each UTS.
%\cite{jin2023classification} combines Line Plot with Constant-Q Transform (CQT) \cite{brown1991calculation}, a method related to wavelet transform ($\S$\ref{sec.spectrogram}), to generate 2-channel images.
The mixture methods encode a UTS with multiple views and were found more robust than single-view images in these works for %time series
classification tasks.

\subsection{How to Model MTS}\label{sec.modelmts}

In the above methods, Heatmap ($\S$\ref{sec.heatmap}) can be %directly
used to visualize the %2D
variate-time matrices, $\mat{X}$, of MTSs ({\em e.g.}, Fig. \ref{fig.structure}(b)), where correlated variates %are better to
should be spatially close to each other. Line Plot ($\S$\ref{sec.lineplot}) can be used to visualize MTSs by plotting all variates in the same image \cite{wimmer2023leveraging,daswani2024plots} or combining all univariate images to compose a bigger %1-channel
image \cite {li2023time}, but these methods only work for a small number of variates. Spectrogram ($\S$\ref{sec.spectrogram}), GAF ($\S$\ref{sec.gaf}), and RP ($\S$\ref{sec.rp}) were designed specifically for UTSs. For these methods and Line Plot, which are not straightforward %for MTS transformation,
in imaging MTSs, the general approaches %to use them %for MTS
include using channel independence assumption to model each variate individually \cite{nie2023time}, %like VisionTS \cite{chen2024visionts},
or stacking the images of $d$ variates to form a $d$-channel image %as did by
\cite{naiman2024utilizing,kim2024cafo}. %\cite{prithyani2024feasibility,naiman2024utilizing,kim2024cafo}.
However, the latter does not fit some vision models pre-trained on RGB images which requires 3-channel inputs (more discussions are deferred to $\S$\ref{sec.processing}).

\vspace{0.2cm}

\noindent{\bf Remark.} As a summary, Table \ref{tab.tsimage} recaps the salient advantages and limitations of the five primary imaging methods that are introduced in this section.

% \hh{can we have a table (e.g., rows are different imaging methods and columns are a few desirable propoerties) or a short paragraph to discuss/summarize/compare the strenths and weakness of different imaging methods for ts? This might bring some structure/comprehension to this section (as opposed to, e.g., some reviewer might complain that what we do here is a laundry list)}

\section{Imaged Time Series Modeling}\label{sec.model}

With image representations, time series analysis can be readily performed with vision models. This section discusses such solutions from %traditional vision models %($\S$\ref{sec.cnns})
%to the recent large vision models %($\S$\ref{sec.lvms})
%and large multimodal models.% ($\S$\ref{sec.lmms}).
the traditional models to the SOTA models.

\begin{figure*}[!t]
\centering
\includegraphics[width=0.9\textwidth]{fig/fig_2.pdf}
% \vspace{-1em}
\caption{An illustration of different modeling strategies on imaged time series in (a)(b)(c) and task-specific heads in (d).}\label{fig.models}
\vspace{-0.2cm}
\end{figure*}

\subsection{Conventional Vision Models}\label{sec.cnns}

%Similar to
Following traditional %methods on
image classification, \cite{silva2013time} applies a K-NN classifier on the RPs of time series, \cite{cohen2020trading} applies an ensemble of fundamental classifiers such as %linear regression, SVM, Ada Boost, {\em etc.}
SVM and AdaBoost on the Line Plots %images
for time series classification. As an image encoder, %a typical encoder, %of images,
CNNs have been %extensively
widely used for learning image representations. %\cite{he2016deep}.
Different from using 1D CNNs on sequences %UTS or MTS
\cite{bai2018empirical}, %regular
2D or 3D CNNs can be applied on imaged time series as shown in Fig. \ref{fig.models}(a). %to learn time series representations by encoding their image transformations.
For example, %standard
regular CNNs have been used on Spectrograms \cite{du2020image}, tiled CNNs have been used on GAF images \cite{wang2015encoding,wang2015imaging}, dilated CNNs have been used on Heatmap images \cite{yazdanbakhsh2019multivariate}. More frequently, ResNet \cite{he2016deep}, Inception-v1 \cite{szegedy2015going}, and VGG-Net \cite{simonyan2014very} have been used on Line Plots \cite{jin2023classification,semenoglou2023image}, Heatmap images \cite{zeng2021deep}, RP images \cite{li2020forecasting,kim2024cafo}, GAF images \cite{barra2020deep,kaewrakmuk2024multi}, 
% Heatmaps \cite{zeng2021deep}, RPs \cite{li2020forecasting,kim2024cafo}, GAFs \cite{barra2020deep,kaewrakmuk2024multi},
and even a mixture of GAF, MTF and RP images \cite{costa2024fusion}. In particular, for time series generation tasks, %a diffusion model with U-Nets \cite{naiman2024utilizing} and GAN frameworks of CNNs \cite{li2022tts,karami2024timehr} have also been explored.%investigated.
GAN frameworks of CNNs \cite{li2022tts,karami2024timehr} and a diffusion model with U-Nets \cite{naiman2024utilizing} have also been explored.

Due to their small to medium sizes, these models are often trained from scratch using task-specific training data. %per task using the task's training set. %of time series images.
Meanwhile, fine-tuning {\em pre-trained vision models}  %such as those pre-trained on ImageNet, %\cite{deng2009imagenet}, 
have already been found promising in cross-modality knowledge transfer for time series anomaly detection \cite{namura2024training}, forecasting \cite{li2020forecasting} and classification \cite{jin2023classification}.

% \cite{li2020forecasting} uses ImageNet pretrained CNNs.

\subsection{Large Vision Models (LVMs)}\label{sec.lvms}

Vision Transformer (ViT) \cite{dosovitskiy2021image} has %given birth to
inspired the development of %some
modern LVMs %large vision models (LVMs)
such as %DeiT \cite{touvron2021training}, 
Swin \cite{liu2021swin}, BEiT \cite{bao2022beit}, and MAE \cite{he2022masked}. %Given an input image, ViT splits it
As Fig. \ref{fig.models}(b) shows, ViT splits an %input
image into {\em patches} of fixed size, then embeds each patch and augments it with a positional embedding. The %resulting
vectors of patches are processed by a Transformer %encoder
as if they were token embeddings. Compared to CNNs, ViTs are less data-efficient, but have higher capacity. %Consequently,
Thus, %the
{\em pre-trained} ViTs have been explored for modeling %the images of time series.
imaged time series. For example, AST \cite{gong2021ast} fine-tunes DeiT \cite{touvron2021training} on the filterbank spetrogram of audios %signals
for classification tasks and finds %using
ImageNet-pretrained DeiT is remarkably effective in knowledge transfer. The fine-tuning paradigm has also been %similarly
adopted in \cite{zeng2023pixels,li2023time} but with different pre-trained models %initializations
such as Swin by \cite{li2023time}. 
VisionTS \cite{chen2024visionts} %explains
attributes %the superiority of LVMs
LVMs' superiority over LLMs in knowledge transfer %over LLMs %as an outcome of
to the small gap between the pre-trained images and imaged time series. %the patterns learned from the large-scale pre-trained images and the patterns in the images of time series.
It %also
finds that with one-epoch fine-tuning, MAE becomes the SOTA time series forecasters on %many
some benchmark datasets.

Similar to %build
time series foundation models %\cite{das2024decoder,goswami2024moment,ansari2024chronos,shi2024time}, %such as TimesFM \cite{das2024decoder}, MOMENT \cite{goswami2024moment}, Chronos \cite{ansari2024chronos} and Time-MoE \cite{shi2024time},
such as TimesFM \cite{das2024decoder}, %and MOMENT \cite{goswami2024moment}, 
there are some initial efforts in pre-training ViT architectures with imaged time series. Following AST, SSAST \cite{gong2022ssast} introduced a %joint discriminative and generative
%masked spectrogram patch prediction self-supervised learning framework
masked spectrogram patch prediction framework for pre-training ViT on a large dataset -- AudioSet-2M. Then it becomes a backbone of some follow-up works such as AST-SED \cite{li2023ast} for sound event detection. %To be effective for UTSs,
For UTSs, ViTime \cite{yang2024vitime} generates a large set of Line Plots of synthetic UTSs for pre-training ViT, which was found superior over TimesFM in zero-shot forecasting tasks on benchmark datasets.

\subsection{Large Multimodal Models (LMMs)}\label{sec.lmms}

%As Large Multimodal Models (LMMs)
As LMMs %are getting
get growing attentions, some %of the
notable LMMs, such as LLaVA \cite{liu2023visual}, Gemini \cite{team2023gemini}, GPT-4o \cite{achiam2023gpt} and Claude-3 \cite{anthropic2024claude}, have been explored to consolidate the power of LLMs %on time series
and LVMs in time series analysis. 
Since LMMs support multimodal input via prompts, methods in this thread typically prompt LMMs with the textual and imaged representations of time series, %textual representation of time series and their %image transformations, transformed images,
%then instruct LMMs
and instructions on what tasks to perform ({\em e.g.}, Fig. \ref{fig.models}(c)).

InsightMiner \cite{zhang2023insight} is a pioneer work that uses the LLaVA architecture to generate %textual descriptions about
texts describing the trend of each input UTS. It extracts the trend of a UTS by Seasonal-Trend decomposition, encodes the Line Plot of the trend, and concatenates the embedding of the Line Plot with the embeddings of a textual instruction, which includes a sequence of numbers representing the UTS, {\em e.g.}, ``[1.1, 1.7, ..., 0.3]''. The concatenated embeddings are taken by a language model for generating trend descriptions. %It also fine-tunes a few layers with the generated texts to align LLaVA checkpoints with time series domain.
Similarly, \cite{prithyani2024feasibility} adopts the LLaVA architecture, but for MTS classification. An MTS is encoded by %a sequence of
the visual %token
embeddings of the stacked Line Plots of all variates. %meanwhile
%The method also stacks
%The time series of all variate are also stacked in a prompt % of all variates in a prompt
The matrix of the MTS is also verbalized in a prompt 
as the textual modality. %By manipulating token embeddings,
By integrating token embeddings, both %of these %works propose to
methods fine-tune some layers of the LMMs with some synthetic data.

Moreover, zero-shot and in-context learning performance of several commercial LMMs have been evaluated for audio classification \cite{dixit2024vision}, anomaly detection \cite{zhuang2024see}, and some synthetic tasks \cite{daswani2024plots}, where the image %({\em e.g.}, spectrograms, Line Plots)
and textual representations of a query %UTS or MTS
time series are integrated into a prompt. For in-context learning, these methods inject the images of a few example time series and their labels ({\em e.g.}, classes) %({\em e.g.}, classes, normal status)
into an instruction to prompt LMMs for assisting the prediction of the query time series.

\subsection{Task-Specific Heads}\label{sec.task}

%With the image embedding of a time series, the next step is to produce its prediction.
For classification tasks, most of the methods in Table \ref{tab.taxonomy} adopt a fully connected (FC) layer or multilayer perceptron (MLP) to transform an embedding into a probability distribution over all classes. For forecasting tasks, there are two approaches: (1) using a $d_{e}\times W$ MLP/FC layer to directly predict (from the $d_{e}$-dimensional embedding) the time series values in a future time window of size $W$ \cite{li2020forecasting,semenoglou2023image}; (2) predicting the pixel values that represent the future part of the time series and then recovering the time series from the predicted image \cite{yang2023your,chen2024visionts,yang2024vitime} ($\S$\ref{sec.processing} discusses the recovery methods). Imputation and generation tasks resemble forecasting %in the sense of predicting
as they also predict time series values. Thus approach (2) has been used for imputation \cite{wang2015imaging} and generation \cite{naiman2024utilizing,karami2024timehr}. %LMMs have been used for classification, text generation, and anomaly detection. For these tasks,
When using LMMs for classification, text generation, and anomaly detection, most of the methods prompt LMMs to produce the desired outputs in textual answers, circumventing task-specific heads \cite{zhang2023insight,dixit2024vision,zhuang2024see}.

%Forecasting: MLP, FC to predict numerical values using embeddings. Imputation of images (TSH). Classification: MLP, FC using embeddings.

\section{Pre-Processing and Post-Processing}\label{sec.processing}

To be successful in using vision models, some subtle design desiderata %to be considered
include {\bf time series normalization}, {\bf image alignment} and {\bf time series recovery}.

\vspace{0.2cm}

\noindent{\bf Time Series Normalization.} Vision models are usually trained on %images after Gaussian normalization (GN).
standardized images. To be aligned, the images introduced in $\S$\ref{sec.tsimage} should be normalized with a controlled mean and standard deviation, as did by \cite{gong2021ast} on spectrograms. In particular, as Heatmap is built on raw time series values, the commonly used Instance Normalization (IN) \cite{kim2022reversible} can be applied on the time series as suggested by VisionTS \cite{chen2024visionts} since IN share similar merits as Standardization. %although min-max normalization was used by \cite{karami2024timehr,zeng2021deep}.
Using Line Plot requires a proper range of y-axis. In addition to rescaling time series %by min-max or GN
\cite{zhuang2024see}, ViTST \cite{li2023time} introduced several methods to remove extreme values from the plot. GAF requires min-max normalization on its input, as it transforms time series values withtin $[0, 1]$ to polar coordinates ({\em i.e.}, arccos). In contrast, input to RP is usually normalization-free as an $\ell_{2}$ norm is involved in Eq.~\eqref{eq.rp} before thresholding.%for a comparison with a threshold.

\vspace{0.2cm}

\noindent{\bf Image Alignment.} When using pre-trained models, it is imperative to fit the image size to the input requirement of the models. This is especially true for Transformer based models as they use a fixed number of positional embeddings to encode the spacial information of image patches. For 3-channel RGB images such as Line Plot, it is straightforward to meet a pre-defined size by adjusting the resolution when producing the image. For images built upon matrices such as Heatmap, Spectrogram, GAF, RP, the number of channels and matrix size need adjustment. For the channels, one method is to duplicate a matrix to 3 channels \cite{chen2024visionts}, another way is to average the weights of the 3-channel patch embedding layer into a 1-channel layer \cite{gong2021ast}. For the image size, bilinear interpolation is a common method to resize input images \cite{chen2024visionts}. Alternatively, AST \cite{gong2021ast} %use cut and bilinear interpolation on
resizes the positional embeddings instead of the images to fit the model to a desired input size. However, the interpolation in these methods may either alter the time series or the spacial information in positional embeddings.

% single-channel (UTS), RGB channel (UTS), duplicate channels (UTS), multi-channel (MTS).

%Bilinear interpolation.

%Correlated variates are better to be spatially close to each other.

%\subsection{Pre-training}

\vspace{0.2cm}

\noindent{\bf Time Series Recovery.} As stated in $\S$\ref{sec.task}, tasks such as forecasting, imputation and generation requires predicting time series values. For models that predict pixel values of images, post-processing involves recovering time series from the predicted images. Recovery from Line Plots is tricky, it requires locating pixels that %correspond to
represent time series and mapping them back to the original values. This can be done by manipulating a grid-like Line Plot as introduced in \cite{yang2023your,yang2024vitime}, which has a recovery function. In contrast, recovery from Heatmap is straightforward as it directly stores the predicted time series values \cite{zeng2021deep,chen2024visionts}. Spectrogram is underexplored in these tasks and it remains open on how to recover time series from it. The existing work \cite{zeng2023pixels} uses Spectrogram for forecasting only with an MLP head that directly predicts time series. %predicts time series values.
GAF supports accurate recovery by an inverse mapping from polar coordinates to normalized time series \cite{wang2015imaging}. However, RP lost time series information during thresholding (Eq.~\ref{eq.rp}), thus may not fit recovery-demanded tasks without using an {\em ad-hoc} prediction head.


% Line Plot was regarded as matrices with rows and columns for mapping in \cite{sood2021visual}.


%\section{Tasks and Time Series Recovery}

%\subsection{Task-Specific Head}

% \subsection{Time Series Recovery}




\subsubsection{Scope of This Survey}
In this survey, we first explain the foundation of generative models for 3D content (Section 2.1), and introduce common 3D representations (Section 2.2) and 4D representations methods (Section 2.3). We also provide the introduction of main physical simulation techniques, including MPM, FEM, and DiffTaichi (Section 2.4). Then, we present a taxonomy of recent research that grounding physics in AIGC, categorizing it into three key areas: static 3D generation (Section 3.1), dynamic 3D generation (Section 3.2), and 4D generation (Section 3.3). Next, we present datasets (Section 4.1) and evaluation metrics (Section 4.2) in this field, and provide a detailed comparison results of several approaches on synthetic PAC-NeRF dataset in Table \ref{table:methodComparison}. Finally, our survey discuss non-negligible challenges and possible future directions for this exciting new area of research (Section 5).   


\section{Background}

\subsection{Generative Models for 3D Content Generation}

Generative models have significantly advanced 3D content creation by using deep learning to synthesize realistic structures. 
Generative Adversarial Networks (GANs) \citep{goodfellow2014generative_NeurIPS2014} and Diffusion Models \citep{ho2020denoising_neurips2020} are two key approaches, each with distinct strengths. 
Originally designed for 2D generation, GANs have been adapted to 3D by incorporating voxels, point clouds, and meshes. 
The generator synthesizes 3D data from latent codes, while the discriminator distinguishes real from generated samples. 
Though capable of producing high-quality shapes, GANs suffer from mode collapse and training instability, issues mitigated by subsequent advances in WGAN \citep{gulrajani2017wgan_neurips2017}, and SN-GAN \citep{miyato2018spectral_arxiv2018}.
Meanwhile, DMs provide a more stable alternative with superior sample diversity. 
They follow a two-step denoising process: first adding Gaussian noise in forward diffusion, then iteratively removing it in reverse. 
In 3D generation, these models employ diverse representations to produce high-resolution, geometrically consistent outputs \citep{poole2022dreamfusion_iclr2023,xu2024precise-physics_arxiv2024,liu2024physics3d_arxiv2024}.


\subsection{3D Representations}

3D representations include explicit and implicit methods for modeling and rendering complex objects and scenes.

\noindent \textbf{Explicit Methods}, including point clouds, meshes, and 3D Gaussian Splatting, offer diverse approaches to modeling and rendering complex objects and scenes. 
Point clouds represent objects as collections of discrete points with attributes like color and surface normals, with advanced methods like SynSin \citep{wiles2020synsin_cvpr2020} and Neural Point-based Rendering \citep{dai2020neural_cvpr2020} leveraging differentiable pipelines for optimization. 
Meshes define objects through vertices, edges, and faces in polygon networks, enabling accurate shape description and property refinement through differentiable rendering. 
3D Gaussian Splatting \citep{kerbl20233d_tog2023} employs learnable 3D Gaussian kernels optimized via multi-view supervision, providing efficient real-time rendering capabilities. 
These methods, each with their unique attributes and techniques, serve different applications while contributing to the advancement of 3D modeling and rendering technology.

\noindent \textbf{Implicit Methods}, such as Signed Distance Fields (SDF) and Neural Radiance Fields (NeRF), define the shapes and boundaries of objects not through explicit geometric components, but via functions that describe spatial occupancy, enabling continuous and detailed descriptions of geometries for realistic visualizations and complex operations.
Both techniques offer unique advantages: SDFs enable efficient rendering and precise geometric manipulations like blending and smoothing of surfaces, making them valuable for computer-aided design systems and dynamic simulations, whereas NeRFs enable highly photorealistic rendering of 3D scenes from novel viewpoints by modeling scenes as continuous volumetric fields within a neural network. 
These complementary approaches provide developers and researchers with powerful tools for advancing the state of 3D rendering and geometric computation.

\subsection{4D Representations}
4D representations incorporate spatiotemporal information for dynamic scene synthesis and reconstruction. 
Three notable approaches have emerged in this field: K-Planes, D-NeRF, and 4D Gaussian Splatting (4DGS).
K-Planes method represents 4D scenes by factorizing the 4D space into six planes, three for spatial dimensions and three for spatiotemporal variations \citep{fridovich2023k_cvpr2023}. K-Plances encodes 4D information by separating static and dynamic components and achieves fast optimization without relying on MLP-based decoders.
D-NeRF \citep{pumarola2021d_cvpr2021} extends the NeRF framework by conditioning the radiance field on time as an additional dimension, modeling scene dynamics through time-dependent radiance fields and deformation fields to handle temporal changes. 
Splatting methods \citep{wu4DGS_cvpr2024,yang2023real_iclr2024,zhang2024mega_arxiv2024} propose two primary approaches: deformation-based transformation of 3D Gaussian kernels in image space, and direct integration of the time dimension into 3D Gaussian kernels for temporal-spatial coherence. 
These 4DGS methods offer lightweight representations of spatiotemporal information while maintaining high-quality dynamic scene reconstructions through the compactness of Gaussian kernels.

\subsection{Physical Simulation}

While 3D and 4D representations focus on modeling static and dynamic scenes respectively, physical simulation methods are essential for understanding and predicting how these representations evolve under physical laws and material properties. 
The Material Point Method (MPM), Finite Element Method (FEM), and DiffTaichi framework employ different approaches to model and compute such dynamic material behaviors across various applications.


Material Point Method (MPM) \citep{jiang2016material_siggraph2016} combines Lagrangian material points with Eulerian Cartesian grids, where material properties like density and velocity are initially stored in particles and then interpolated onto a stationary grid. 
This hybrid approach enables the computation of spatial derivatives and enforcement of physical laws on the grid, after which the updated quantities are transferred back to the particles for state updates, creating an effective cycle for simulating complex material behaviors with high precision and stability.
Finite Element Method (FEM) \citep{hoang2005high_mms2005}, a numerical approach widely used in engineering and computer graphics, tackles the simulation of deformable objects by subdividing larger systems into finite elements through space discretization. 
FEM operates by formulating boundary value problems and approximating solutions through the minimization of associated error functions using variational calculus.
To advance the implementation and optimization of these simulation methods, particularly MPM, DiffTaichi \citep{hudifftaichi_iclr2020} provides comprehensive differentiable programming capabilities, featuring a two-scale automatic differentiation system that supports megakernels, imperative programming, and flexible indexing. 
This framework simplifies the implementation of MPM and other simulation techniques, demonstrating high productivity and efficient convergence in gradient-based learning and optimization tasks.


\section{Categories of Physics Prior}
The core idea behind this category is based on the spatial and temporal dimension of the generated content and task type, providing researcher a clear introduction of related AIGC approaches.
In this section, we classify physics-grounded generation methods broadly into four categories (Figure \ref{fig:taxonomy}), static 3D generation, dynamic 3D generation, 4D generation, and other tasks such as segmentation, editing and recognition.

\subsection{Static 3D Generation with Physics }
Advancements in 3D generation have enhanced both geometric accuracy and physical plausibility. 
Early works \citep{tang2023dreamgaussian_iclr2024,yi2024gaussiandreamer_cvpr2024} focused on high-fidelity shapes, while recent approaches integrate physical realism. 
Phy3DGen \citep{xu2024precise-physics_arxiv2024} incorporates solid mechanics into a 3D diffusion model, using a differentiable physics network for refinement. 
By leveraging Finite Element Method (FEM) \citep{zienkiewicz2000finite_2000} and a co-optimization strategy, it ensures both visual accuracy and physical consistency.

Generating physically coherent multi-component structures from a single image remains challenging. 
PhysComp3D \citep{guo2024physically_arxiv2024} optimizes physical compatibility by decomposing mechanical properties, external forces, and rest-shape geometry. 
Using plastic deformation parametrization and implicit differentiation, it refines object structures but relies on predefined material properties and mesh representations. 
Extending this, PhyCAGE \citep{yan2024phycage_arxiv2024} employs Gaussian Splatting (GS) and physics-based SDS optimization to enhance efficiency. 
By generating multi-view images and refining Gaussian positions through simulation-enhanced SDS gradients, it improves the physical coherence of 3D assets.

Text-based 3D generation faces challenges in maintaining stability due to missing physical constraints. 
Atlas3D \citep{chen2024atlas3d_arxiv2024} addresses this by integrating differentiable physics simulation into an SDS-based framework, predicting rigid body dynamics and enforcing standability and stable equilibrium losses for structural consistency. 
While effective for individual objects, it lacks scene-level physical interactions. 
To address this, PhiP-G \citep{Li2025PhiP-G_arxiv2025} introduces a physics-grounded text-to-3D scene generation framework, integrating a multi-agent text processor and a relational database-based physics pool for object interactions. 
Additionally, its physical magnet module applies vector approximations to align assets in semantically and physically consistent 3D scenes. 
\textcolor{black}{Continuing this progress, LAYOUTDREAMER \citep{zhou2025layoutdreamer_arxiv2025} leverages 3D GS to create compositional scenes by converting text into a direct scene graph, which guides objects layout and physical interactions. LAYOUTDREAMER excels in generating complex multi-object scene, allowing users to conveniently edit and expand disentangled scenes.}


\subsection{Dynamic 3D Generation}
Dynamic 3D generation focuses on capturing 3D content motion over time. According to 3D representation categories, this subsection discuss three dominant paradigms: vision-based dynamic 3D generation, NeRF-based dynamic 3D generation, and GS-based dynamic 3D generation.

\subsubsection{Vision-based Dynamic 3D Generation}
Several works explore motion modeling through frequency analysis and optical flow. 
Li et al. \citep{li2024generative_cvpr2024} introduce a frequency-domain motion model, integrating spatial and frequency attention within LDM UNet for motion prediction. 
Their feature pyramid softmax-splatting strategy generates future frames from spectral volume predictions, but low-frequency constraints cause artifacts in complex scenes. 
MOTIONCRAFT \citep{savant2024motioncraft_arxiv2024} extends this by mapping optical flow fields between RGB and latent space in Stable Diffusion, using physics simulators to animate input images. 
This approach improves zero-shot video generation with realistic fluid dynamics, rigid motion, and multi-agent interactions, avoiding extensive data and computation costs.

To ensure physically consistent motion, differentiable physics-based approaches explicitly model object dynamics. 
DANO \citep{le2023differentiable_RAL2023} estimates mass, inertia, and friction forces from density fields, introducing a differentiable contact model for rigid-body simulation. 
However, it lacks support for articulated and soft-body physics. Physics 101 \citep{wu2016physics101_bmvc2016} instead infers physical properties from video, categorizing them into intrinsic (unobservable) and descriptive (visually detectable) attributes. 
By combining visual recognition, physics interpretation, and world simulation, it extracts properties from unlabeled video scenarios, such as sliding, falling, and floating.

Beyond explicit physics modeling, PhysGen \citep{liu2024physgen_arxiv2024} integrates model-based physics simulation into image-to-video generation. 
It first employs GPT-4V \citep{OpenAI_GPT-4V_2023} for scene perception, extracting materials, object composition, and physical attributes from single input image. 
Rigid-body physics then simulates object dynamics, followed by motion-guided rendering for realistic and controllable training-free video synthesis.
\textcolor{black}{However, models still struggle to understand the interaction and movement of multiple objects. 
To enforce physics-grounded video generation, PhyT2V \citep{xue2024phyt2v_arxiv2024} embeds physical rules into the text prompts by local chain-of-thought (CoT) and global step-back reasoning. 
PhyT2V expends the exsiting video generation models to out-of-distribution domains with sufficient and appropriate contexts via LLM, enabling significant improvement of video quality.} 


\subsubsection{NeRF-based Dynamic 3D Generation}
Dynamic NeRF methods \citep{fang2022fast_siggraph2022,gao2021dynamic_ICCV2021,park2021nerfies_iccv2021} require full image sequences for training, limiting adaptability to dynamic scenes. 
ParticleNeRF \citep{abou2024particlenerf_wacv2024} addresses this by introducing a particle-based encoding, updating particle positions via backpropagated photometric loss while managing motion with a lightweight physics system. 
Unlike static grid encodings, it adapts to rigid, articulated, and deformable objects with higher fidelity and efficiency.
Further, PAC-NeRF \citep{lipac-nerf_iclr2023} extends NeRF’s capabilities by estimating geometry and physical parameters from multi-view videos using a hybrid Eulerian-Lagrangian representation. 
It combines NeRF density fields with Material Point Method (MPM) physics simulation \citep{jiang2015affine_tog2015}, enabling differentiable rendering and simulation without predefined object structures.

Despite its advantages, PAC-NeRF relies on first-frame grid representations, restricting optimization across sequences. 
LPO \citep{kaneko2024improving_cvpr2024} overcomes this by introducing Lagrangian particle optimization, which refines particle positions and features across entire videos. 
While these methods enhance motion modeling, NeRF remains computationally demanding for elastic dynamics simulation, especially for complex deformable objects. 
PIE-NeRF \citep{feng2024pie-nerf_cvpr2024} addresses this by integrating physics-based simulation into NeRF, employing quadratic generalized least squares (Q-GMLS) \citep{martin2010unified_tog2010} for nonlinear dynamics and large deformations. 
Using spatial reduction and real-time neural graphics primitives (NGP) \citep{muller2022instant_tog2022}, it enables interactive manipulation and real-time rendering, achieving both physical accuracy and visual fidelity.


\subsubsection{GS-based Dynamic 3D Generation}

Traditional NeRF-based dynamic models assume known material properties, limiting abilities to simulate heterogeneous objects. 
Spring-Gaus \citep{zhong2024reconstruction_arxiv2024} introduces a 3D Spring-Mass model with learnable mass points and springs, enabling elastic reconstruction from multi-view videos. 
By integrating Gaussian kernels with the Spring-Mass model, it decouples appearance and geometry, efficiently capturing geometry, motion, and physical properties.

To model action-conditioned dynamics, PhysDreamer \citep{zhang2025physdreamer_eccv2024} learns dynamics priors from video generation models, distilling them via differentiable Material Point Method (MPM) simulation. 
It optimizes physical parameters such as Young’s modulus to infer object stiffness and introduces K-means-driven driving particles, accelerating simulations while preserving physical fidelity. 
Extending this, Physics3D \citep{liu2024physics3d_arxiv2024} integrates viscoelastic MPM with Score Distillation Sampling (SDS) to simulate a wide range of materials, from elastic to plastic behaviors. 
\textcolor{black}{PhysMotion \citep{tan2024physmotion_arxiv2024} introduces physics-based simulation MPM to guide intermediate 3D representations from a single image. 
Unlike PhysDreamer and Physics3D, which focus on learning physical properties from video diffusion models, PhysMotion combines 3D GS and refines the coarse simulation using a 2D image DM with cross-frame attention. }

For fluid-solid coupling, Gaussian Splashing (GSP) \citep{feng2024splashing_arxiv2024} utilizes 3D Gaussian kernels as particles, tracking fluid surfaces and interpolating deformations onto Gaussian kernels. 
By incorporating surface tension and specular effects, it achieves physically consistent rendering of solid-fluid interactions. 
In underwater dynamic modeling, Gaussian Splashing for underwater imagery \citep{mualem2024gaussian_arxiv2024} extends 3D Gaussian Splatting with learnable backscatter and attenuation effects, introducing a depth-aware rasterization pipeline for robust underwater reconstruction. Additionally, Gaussian Splashing also provides a novel underwater dataset TableDB, consisting 172 images with a resolution of 1384 $\times$ 918, which are unbounded in camera-to-scene distances.

To address discontinuities in video diffusion-based dynamics, DreamPhysics \citep{huang2024dreamphysics_arxiv2024} introduces Motion Distillation Sampling (MDS), improving motion realism over standard SDS. 
It employs KAN-based material fields and frame boosting, enabling text- and image-conditioned physical simulations without requiring ground truth videos. 
PhysGaussian \citep{xie2024physgaussian_cvpr2024} further bridges Newtonian dynamics and 3D Gaussian kernels, leveraging continuum mechanics-driven deformation models to align physical simulation and visual rendering, handling a variety of materials including metals and non-Newtonian fluids.
\textcolor{black}{However, PhysGaussian still require manual tuning physical properties, which is time-consuming and highly relies on expert knowledge.
To enhance the flexibility and generalize the constitutive model instead of manually setting, OMNIPhysics \citep{lin2025omniphysgs_iclr2025} introduces a general physics-based 3D dynamic synthesis framework by treating each 3D asset as a collection of learnable constitutive 3D Gaussians.
It then leverages a pre-trained video diffusion model to supervise the estimation of material weighting factors, enabling the synthesis of physically plausible dynamics across a broad spectrum of materials.}

For generalizable modeling, NeuMA \citep{cao2024neural_arxiv2024} integrates physical laws with learned corrections, introducing Neural Constitutive Laws (NCLaw) for adaptive physics simulation. 
It incorporates Particle-GS, which binds simulation particles to Gaussian kernels, improving visual-grounded dynamic modeling. 
GIC (Gaussian-Informed Continuum) \citep{cai2024gic_NeurIPS2024} further refines physical property estimation, using motion decomposition networks and a coarse-to-fine density field generation strategy, enhancing dynamic scene reconstruction and continuum mechanics-based physical optimization.


\subsection{4D Generation}
4D generation aims to reconstruct 3D presentations from input conditions such as text, image, and video sequences. Currently, most 4D generation works heavily rely on powerful 3D generation works, which have high computational costs and face challenges in understanding real-world dynamics. 

To improve 4D generation models modeling and understanding ability, Phy124 \citep{lin2024phy124_arxiv2024} eliminates diffusion models, enabling fast, physics-driven 4D content generation by converting an image into a 3D Gaussian representation and applying MPM to simulate Gaussian field dynamics. 
Phy124 allows real-time motion control via external forces. 
GASP \citep{borycki2024gasp_arxiv2024} extends Gaussian-based modeling for real-time 3D simulation, leveraging the GaMeS framework \citep{waczynska2024games_arxiv2024} to represent Gaussian components as triangle faces, enabling MPM-driven deformations for both static and dynamic 3D scenes. 
Phys4DGen \citep{lin2024phys4dgen_arxiv2024} further extends Phy124 by integrating a PPM for material-aware physics simulation, segmenting material groups from images and inferring properties via GPT-4o \citep{OpenAI_GPT-4V_2023}, enhancing recognition and simulation fidelity.

For text-to-4D synthesis, TRANS4D \citep{zeng2024trans4d_arxiv2024} introduces multi-modal large language model priors to generate physically plausible 4D scenes, computing 3D Gaussian transformations at each timestep with a geometry-aware Transition Network to ensure smooth, physics-consistent transitions.
Similarly, Liu et al. \citep{liu2024unleashing_arxiv2024} propose a comprehensive 4D simulation framework, integrating GPT-4 material inference with optical flow-based loss for optimizing physical properties, improving dynamic simulation across diverse material types.

Beyond generation, understanding and reasoning about 4D dynamic scenes remain critical for enabling physics-aware AIGC.
\textcolor{black}{NS-4Dynamics \citep{Wang2025Compositional_iclr2025} serves as the first neural-symbolic models for explicit 4D scene reconstruction. 
NS-4Dynamics incorporates physical priors into the scene parsing process, further enhancing the realism and accuracy of 4D dynamic generation.
Additionally, this work proposes the SuperCLEVR-Physics dataset, designed for video question answering tasks focused on object dynamics and interaction properties, further bridging the gap between 4D generative models and physics-aware reasoning.}


\section{Benchmarks}
In this section, we discuss the benchmarks, including datasets and evaluation metrics used to evaluate physics-aware generation models. Additionally, we present a quantitative comparison of the state-of-the-art physics-aware dynamic 3D generation models on synthetic dataset.

\subsection{Datasets}
There are two types of datasets, synthetic and real-world datasets, and each serves a distinct purpose in evaluating the robustness and generalization of physics-aware generative models. 
Synthetic datasets provide controlled scenes dynamics with ground-truth physical properties, such as object mass, density, and material, while real-world datasets capture the variability of natural scenes.
Both datasets

\subsubsection{Synthetic Dataset}
Synthetic PAC-NeRF consists of 9 instances with deformable objects, plastics, granular, metal, and Newtonian/Non-Newtonian fluids \citep{lipac-nerf_iclr2023}. 
Each scene depicts the process of objects falling freely, colliding, and bouncing back, captured from 11 viewpoints with ground truth data generated by the MLS-MPM framework \citep{hu2018moving_tog2018}.
Synthetic Spring-Gaus includes $14$ 3D models \citep{zhong2024reconstruction_arxiv2024}, all of which are generated from PAC-NeRF \citep{lipac-nerf_iclr2023} and OmniObject3D \citep{wu2023omniobject3d_cvpr2023} approaches. 
This dataset features elastic object sequences captured from 10 viewpoints across 30 frames at a resolution of $512\times 512$.
VIDEOPHY evaluates whether generated videos adhere to physical commonsense \citep{bansal2024videophy_arxiv2024}. 
It comprises 688 human-verified high-quality captions, with 344 prompts for the test set and 344 prompts for train set. 
This dataset maintains a balanced distribution of the state of matter and complexity across both sets.


\subsubsection{Real-world Dataset}
Real-world PAC-NeRF captures a deformation ball falling onto a table using a capture system comprising four synchronized Intel RealSense D455 cameras. 
The real-world data in PAC-NeRF are RGB images at a resolution of $640\times 480$ and at a rate of 60 frames per second. 
Real-world Spring-Gaus contains both static scenes and dynamic multi-view videos \citep{zhong2024reconstruction_arxiv2024}. 
Static scenes include 50-70 images from various viewpoints, while dynamic scenes are recorded from three viewpoints at a resolution of $1980\times 1080$. 
Physics 101 contains $17,408$ videos of 101 objects made of 15 different materials \citep{wu2016physics101_bmvc2016}. 
Each material category has 4 to 12 objects of different sizes and colors, with recorded physical properties such as mass, volume, and density. 

\subsubsection{Physical Properties}
From the definition in MPM simulator \citep{jiang2016material_siggraph2016}, seven material types include Newtonian and non-Newtonian fluids, elasticity, plasticine, metal, foam, and sand are characterized by specific physical parameters (Table \ref{tab:physical_properties}). For instance, Newtonian fluids are defined by viscosity and bulk modulus, while Non-Newtonian fluids also consider shear modulus and yield stress. For the synthetic dataset, each object’s material type is predefined, while the real-world dataset usually lack the information of object material.

\begin{table}
    \caption{A taxonomy of seven common material types simulated with various physical properties.}
    \label{tab:physical_properties}
    \setlength{\tabcolsep}{4pt} % Default value: 6pt
    \renewcommand{\arraystretch}{1.2} % Default value: 
    \centering
    \scalebox{0.63}{
    \begin{tabular}{c|c}
    \toprule
         \textbf{Material Types}& \textbf{Physical Properties}  \\
    \midrule
         Newtonian fluid      & Fluid viscosity $\mu$, Bulk modulus $\kappa$ \\
         Non-Newtonian fluid  & Shear modulus $\mu$, Bulk modulus $\kappa$, Yield stress $\tau_Y$, Plastic viscosity $\eta$ \\
         Elasticity           & Young's modulus $E$, Poisson's ratio $\nu$ \\
         Plasticine           & Young's modulus $E$, Poisson's ratio $\nu$, Yield stress $\tau_Y$  \\
         Metal                & Young's modulus $E$, Poisson's ratio $\nu$, Yield stress $\tau_Y$  \\
         Foam                 & Young's modulus $E$, Poisson's ratio $\nu$, Plastic viscosity $\eta$  \\
         Sand                 & Friction angle $\theta_{fric}$  \\
    \bottomrule
    \end{tabular}}
\end{table}


\begin{table*}[htbp]
\renewcommand{\arraystretch}{1.2}
\centering
\caption{Performance comparison of dynamic 3D generation methods on the synthetic PAC-NeRF dataset.}
\scalebox{0.69}{
\begin{tabular}{ccccccccc}
\toprule
\multirow{2}{*}{\textbf{Materials}} & \multirow{2}{*}{\textbf{Objects}} & \multirow{2}{*}{\textbf{GT}} & \multirow{2}{*}{\textbf{Parameters}} & \multicolumn{2}{c}{\textbf{NeRF-based Methods}} & \multicolumn{2}{c}{\textbf{GS-based Methods}} \\
\cmidrule(lr){5-6} \cmidrule(lr){7-8} 
& & & & \textbf{PAC-NeRF} \citep{lipac-nerf_iclr2023} & \textbf{LPO} \citep{kaneko2024improving_cvpr2024} & \textbf{GIC} \citep{cai2024gic_NeurIPS2024} & \textbf{Unleashing} \citep{liu2024unleashing_arxiv2024} \\
\midrule
\multirow{3.5}{*}{Newtonian fluid} & Droplet & \begin{tabular}[c]{@{}c@{}}$200$\\ $1.0\times10^5$\end{tabular} & \begin{tabular}[c]{@{}c@{}}$\Delta\mu$\\ $\Delta\kappa$\end{tabular} & \begin{tabular}[c]{@{}c@{}}$9$\\ $8.0\times10^3$\end{tabular} & \begin{tabular}[c]{@{}c@{}}$41$\\ $2.8\times10^4$\end{tabular} & \begin{tabular}[c]{@{}c@{}}$1$\\ $8.2\times10^4$\end{tabular} & \begin{tabular}[c]{@{}c@{}}$\pmb{0.89}$\\ $\pmb{3.0\times10^3}$\end{tabular} \\
\cmidrule{2-8}
& Letter & \begin{tabular}[c]{@{}c@{}}$100$\\ $1.0\times10^5$\end{tabular} & \begin{tabular}[c]{@{}c@{}}$\Delta\mu$\\ $\Delta\kappa$\end{tabular} & \begin{tabular}[c]{@{}c@{}}$16.15$\\ $3.5\times10^4$\end{tabular} & \begin{tabular}[c]{@{}c@{}}$\pmb{2}$\\ $1.3\times10^4$\end{tabular} & \begin{tabular}[c]{@{}c@{}}$4.95$\\ $\pmb{0}$\end{tabular} & \begin{tabular}[c]{@{}c@{}}$2.27$\\ $7.0\times10^3$\end{tabular} \\
\midrule
\multirow{5.5}{*}{\begin{tabular}[c]{@{}c@{}}Non-Newtonian \\fluid\end{tabular}} 
        & Cream     
            & \begin{tabular}[c]{@{}c@{}}$1.0\times10^4$\\ $1.0\times10^6$\\ $3.0\times10^3$\\ $10$\end{tabular}   
            & \begin{tabular}[c]{@{}c@{}}$\Delta\mu$\\ $\Delta\kappa$\\ $\Delta\tau_Y$\\ $\Delta\eta$\end{tabular} 
            & \begin{tabular}[c]{@{}c@{}}$1.11\times10^5$\\ $5.7\times10^5$\\ $160$\\ $8000$\end{tabular} 
            & \begin{tabular}[c]{@{}c@{}}$2.6\times10^3$\\ $\pmb{3.2\times10^5}$\\ $40$\\ $\pmb{0.8}$\end{tabular} 
            & \begin{tabular}[c]{@{}c@{}}$\pmb{3.0\times10^2}$\\ $4.8\times10^5$\\ $\pmb{20}$\\ $3.4$\end{tabular} 
            & \begin{tabular}[c]{@{}c@{}}$1.21\times10^4$\\ $5.3\times10^5$\\ $1730$\\ $7.5$\end{tabular} \\ 
\cmidrule{2-8} 
        & Toothpaste 
            & \begin{tabular}[c]{@{}c@{}}$5.0\times10^3$\\ $1.0\times10^5$\\ $200$\\ $10$\end{tabular}    
            & \begin{tabular}[c]{@{}c@{}}$\Delta\mu$\\ $\Delta\kappa$\\ $\Delta\tau_Y$\\ $\Delta\eta$\end{tabular} 
            & \begin{tabular}[c]{@{}c@{}}$1510$\\ $5.122\times10^4$\\ $28$\\ $0.23$\end{tabular}                    
            & \begin{tabular}[c]{@{}c@{}}$340$\\ $7.88\times10^4$\\ $38$\\ $\pmb{0.20}$\end{tabular}          
            & \begin{tabular}[c]{@{}c@{}}$810$\\ $7.6\times10^4$\\ $\pmb{26}$\\ $0.90$\end{tabular}                       
            & \begin{tabular}[c]{@{}c@{}}$\pmb{70}$\\ $\pmb{1.3\times10^4}$\\ $46$\\ $28.99$\end{tabular} \\ 
\midrule
\multirow{3}{*}{Elasticity}                                                     
        & Torus      
            & \begin{tabular}[c]{@{}c@{}}$1.0\times10^6$\\ $0.3$\end{tabular}                           
            & \begin{tabular}[c]{@{}c@{}}$\Delta E$\\ $\Delta\nu$\end{tabular}                                            
            & \begin{tabular}[c]{@{}c@{}}$4.0\times10^4$\\ $0.022$\end{tabular}                                                 
            & \begin{tabular}[c]{@{}c@{}}$9.0\times10^4$\\ $0.007$\end{tabular}                            
            & \begin{tabular}[c]{@{}c@{}}$\pmb{1.0\times10^4}$\\ $\pmb{0.005}$\end{tabular}                                        
            & \begin{tabular}[c]{@{}c@{}}$3.9\times10^4$\\ $0.297$\end{tabular} \\ 
\cmidrule{2-8} 
        & Bird       
            & \begin{tabular}[c]{@{}c@{}}$3.0\times10^5$\\ $0.3$\end{tabular}                    
            & \begin{tabular}[c]{@{}c@{}}$\Delta E$\\ $\Delta\nu$\end{tabular}                                            
            & \begin{tabular}[c]{@{}c@{}}$2.2\times10^5$\\ $0.027$\end{tabular}                                                 
            & \begin{tabular}[c]{@{}c@{}}$1.9\times10^4$\\ $0.047$\end{tabular}                           
            & \begin{tabular}[c]{@{}c@{}}$\pmb{8.0\times10^3}$\\ $\pmb{0.016}$\end{tabular}                                       
            & \begin{tabular}[c]{@{}c@{}}$1.2\times10^4$\\ $0.171$\end{tabular} \\ 
\midrule
\multirow{4.5}{*}{Plasticine}                                                     
        & Playdoh    
            & \begin{tabular}[c]{@{}c@{}}$2.0\times10^6$\\ $0.3$\\ $1.54\times10^4$\end{tabular} 
            & \begin{tabular}[c]{@{}c@{}}$\Delta E$\\ $\Delta\nu$\\ $\Delta\tau_Y$\end{tabular}                        
            & \begin{tabular}[c]{@{}c@{}}$1.84\times10^6$\\ $0.028$\\ $150$\end{tabular}                                       
            & \begin{tabular}[c]{@{}c@{}}$7.2\times10^5$\\ $0.063$\\ $7600$\end{tabular}                 
            & \begin{tabular}[c]{@{}c@{}}$4.2\times10^5$\\ $\pmb{0.022}$\\ $\pmb{20}$\end{tabular}                               
            & \begin{tabular}[c]{@{}c@{}}$\pmb{2.11\times10^5}$\\ $0.098$\\ $41$\end{tabular} \\ 
\cmidrule{2-8} 
       & Cat        
           & \begin{tabular}[c]{@{}c@{}}$1.0\times10^6$\\ $0.3$\\ $3.85\times10^3$\end{tabular}        
           & \begin{tabular}[c]{@{}c@{}}$\Delta E$\\ $\Delta\nu$\\ $\Delta\tau_Y$\end{tabular}                        
           & \begin{tabular}[c]{@{}c@{}}$8.39\times10^5$\\ $0.007$\\ $280$\end{tabular}                                       
           & \begin{tabular}[c]{@{}c@{}}$8.03\times10^5$\\ $\pmb{0.003}$\\ $650$\end{tabular}                  
           & \begin{tabular}[c]{@{}c@{}}$\pmb{2.0\times10^3}$\\ $0.004$\\ $\pmb{90}$\end{tabular}                               
           & \begin{tabular}[c]{@{}c@{}}$3.87\times10^5$\\ $0.124$\\ $850$\end{tabular} \\ 
\midrule
Sand & Trophy & $40$ & $\Delta\theta_{fric}$ & $3.9^\circ$ & $2.25^\circ$ & $2.0^\circ$ & $\pmb{0.5^\circ}$ \\
\bottomrule
\end{tabular}}
\label{table:methodComparison}
\end{table*}

Evaluating physics-prior guided generation methods requires assessing semantic coherency, physical consistency, and model performance to ensure physical principle alignments.

\subsubsection{Metrics} 
For semantic coherency, the CLIP score measures the semantic alignment between the generated content and the textual descriptions by computing CLIP cross-modal embeddings. 
Semantic Adherence (SA) also evaluates whether the generated video is semantically grounded in its corresponding text caption.
This metric assigns a binary score, with $\rm SA=1$ indicating that the caption is semantically grounded in the generated video. 

For physical consistency, the Mean Absolute Error (MAE) calculates the average absolute difference between the generated physical properties and the ground truth values. 
Besides, the Physical Commonsense (PC) metric records the ratio of generated contents align with intuitive physics that humans acquire with their experience in the real-world. PC denotes as a binary score $\{0,1\}$, with $\rm PC=1$ means that the generated contents adhere to the physical laws.

For model evaluation, the Peak Signal-to-Noise Ratio (PSNR) evaluates the quality of generated images or videos. 
Additionally, Structural Similarity Index Metric (SSIM) measures the perceptual similarity between two images, which is widely used in dynamic scenes generation.

\subsubsection{Comparison Results}

We select the state-of-the-art approaches in physics-aware dynamic 3D generation, and each utilizing different representations and architectures to integrate physical priors. Specifically, PAC-NeRF \citep{lipac-nerf_iclr2023} and LPO \citep{kaneko2024improving_cvpr2024} are NeRF-based approaches, while GIC \citep{cai2024gic_NeurIPS2024} and unleashing \citep{liu2024unleashing_arxiv2024} are GS-based dynamic 3D generation. 
Table \ref{table:methodComparison} provides a comparative analysis of four physics-aware 3D generation methods on the Synthetic PAC-NeRF dataset, which includes nine objects composed of Newtonian fluids, Non-Newtonian fluids, elasticity-based materials, plasticine, and sand. 
Besides, the dataset provides ground-truth physical simulation data, allowing for a quantitative evaluation of these methods based on key physical parameters such as viscosity, yield stress, sheer modulus, bulk modulus, elasticity modulus, Poisson’s ratio, and friction angle.

The comparison reveals that Unleashing consistently achieves the lowest AE for most fluid and plasticine materials, including Newtonian fluids (Droplet, Letter), Non-Newtonian fluids (Toothpaste), and plasticine (Playdoh, Cat). 
LPO demonstrates superior performance in yield stress modeling for Non-Newtonian fluids, achieving the best bulk modulus ($\Delta\kappa$) results for Cream. 
GIC is the strongest performer in elasticity-based materials, obtaining the lowest AE in Torus and Bird ($\Delta E$, $\Delta \nu$). 
Unleashing is competitive in estimating sand friction angle ($\Delta \theta_{\text{fric}}$) for the Trophy object, indicating its effectiveness.

Across different methods, PAC-NeRF exhibits higher errors, particularly in yield stress ($\Delta \tau_Y$) and elasticity ($\Delta E$), probably due to its fixed first-frame optimization. 
LPO refines particle positions, improving performance in Non-Newtonian fluids but struggling with elasticity-based materials. 
GIC excels in elasticity modeling, achieving the lowest $\Delta E$ values for Torus and Bird. 
Unleashing demonstrates the most robust overall performance, excelling in Newtonian and Non-Newtonian fluids as well as plasticine, showcasing its effectiveness in capturing complex material behaviors.

Overall, the results highlight the importance of integrating physics-aware priors in generative models, as different methods exhibit distinct strengths depending on material properties. 
Unleashing demonstrates the most robust performance across various material types, particularly in modeling fluid dynamics and plasticine deformations. 
GIC excels in elasticity-based materials due to its Gaussian Splatting-based scene reconstruction, effectively capturing non-rigid deformations. 
LPO refines PAC-NeRF by optimizing particle positions in time-varying sequences, improving its accuracy in Non-Newtonian fluids, though it struggles with elasticity modeling. 
PAC-NeRF, despite pioneering a hybrid Eulerian-Lagrangian representation, suffers from larger errors due to its reliance on first-frame initialization, limiting its adaptability to dynamic materials.


\section{Challenges and Future Directions} 

Despite significant progress in physics-grounded generative models, several challenges remain that hinder their broader applicability and effectiveness. 
Current approaches often struggle with accurately modeling physical interactions, maintaining long-term dynamic consistency, and generalizing to diverse materials and real-world scenarios. 
Addressing these limitations requires advancements in dataset construction, model design, and integration with physics-aware reasoning.

One major challenge is the lack of accurate physical parameter annotations in existing datasets, which limits the scalability of data-driven generative models. 
Most datasets lack expert-annotated physical properties and fail to cover a diverse range of material behaviors and dynamic interactions. 
While recent works have explored leveraging large language models (LLMs) for physical reasoning \citep{liu2024physgen_arxiv2024,lin2024phys4dgen_arxiv2024,liu2024unleashing_arxiv2024}, these approaches struggle with capturing intricate physical dependencies. 
The effectiveness of LLMs in physical reasoning heavily depends on well-constructed prompts, and current models \citep{OpenAI_GPT-4V_2023} exhibit inconsistency in physics-related tasks. 
A promising research direction is to enhance learning and reasoning capabilities within LLMs.

Another fundamental challenge lies in embedding physical constraints into generative models, particularly in understanding the relationship between dynamic movements and physical laws. 
While properties such as mass, velocity, and friction play a crucial role in realistic motion synthesis, effectively integrating these parameters into neural representations remains an open problem. 
3D Gaussian Splatting (3DGS)-based approaches \citep{xie2024physgaussian_cvpr2024,cai2024gic_NeurIPS2024,borycki2024gasp_arxiv2024,liu2024unleashing_arxiv2024} attempt to encode physical parameters by extending Gaussian kernels with additional dimensions and incorporating physics-aware optimization in differentiable networks. 
However, for 4D dynamic generation, existing modeling techniques are still underdeveloped, and further exploration is needed to improve their ability to capture complex object deformations and temporal consistency.

Future research should focus on enhancing the integration of physics-aware priors in generative models, improving their adaptability to diverse materials and dynamic environments. 
For example, advancing Sim2Real transfer and Embodied AI will further bridge the gap between simulated and real-world interactions, enabling more physically consistent and generalizable generative models.


\section{Conclusion}
This survey reviews physics-grounded generative models for 3D and 4D content generation, categorizing methods based on their representation and generation paradigms. 
We analyze their strengths and limitations, highlighting how physical priors improve visual realism and structural consistency. Quantitative comparisons reveal gaps in physical accuracy and generalization.
Despite progress, challenges remain in modeling multi-object complex physical interactions and enhancing dataset diversity in this area. 
Future research should focus on improving physics-aware learning, integrating differentiable physics, and advancing Sim2Real transfer to bridge the gap between simulation and real-world applications.


\small
\bibliographystyle{named}
\bibliography{ijcai25}

\end{document}

