\section{Related work}
\label{sec: related}

Learning in games has a long and complex history ____. In this paper we focus on fictitious play (FP) and the replicator dynamic. The study of FP began with the work of ____, who showed that in zero-sum games, the empirical distribution of strategies converges to the Nash equilibrium. Further results suggested that the long-run outcomes of FP would always be Nash equilibria; FP also converges to NEs in congestion games ____, ordinal potential games ____ and $2\times n$ games ____. However, Shapley ____ demonstrated that FP does not converge to NEs in general. Despite significant further work ____, the attractors of FP remain unknown in general. Recently, however, the behavior of FP was shown to have a close relationship to the preference graph, with many classical facts being a result of graph structure ____. We explore these ideas further in Section~\ref{sec: FP}.

The replicator dynamic arose from the work of Maynard Smith on evolutionary game theory ____, being named and formalized in ____. Since then, it has retained its central role in evolutionary game theory ____ as well as online learning, where it is the continuous-time analogue of the multiplicative weight method ____. Finding its attractors is a central goal of the study of the replicator, both in evolutionary game theory ____ and more recently in learning ____. Since the work of Papadimitriou and Piliouras ____, a line of research has developed relating the replicator and the sink equilibria. ____ used the sink equilibria as an approximation of the outcome of games for the purpose of ranking the strength of game-playing algorithms. Similarly, ____ used the preference graph as a tool for representing the space of games for the purposes of learning. Later, Biggar and Shames wrote a series of papers on the preference graph and its relationship to the attractors of the replicator dynamic ____. Another recent work ____ explored the problem of computing the limit distributions over sink equilibria, given some prior over strategies. Our work extends the frontier of this line of investigation.