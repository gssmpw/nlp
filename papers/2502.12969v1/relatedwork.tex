\section{Literature Review}
\subsection{Frameworks Review for Information Asymmetry: Adverse Selection and Moral Hazard}
The economic literature on information asymmetry has long focused on two central problems: adverse selection and moral hazard. In the case of adverse selection, as first discussed by \citet{akerlof1970market}, markets fail to achieve efficient outcomes because uninformed parties cannot distinguish between high- and low-quality (or high- and low-risk) agents. Moral hazard, on the other hand, arises when an agent's actions are unobservable by the principal, leading to distortions in incentive structures \citep{holmstrom1979moral, grossman1983analysis}. To address these issues, economists have developed various mechanism design solutions based on signaling, screening, monitoring, and reputation systems (e.g., \citealp{myerson1981optimal, maskin1984optimal,laffont2009theory}).

With the advent of big data and artificial intelligence, new approaches have emerged that enhance the efficiency of information collection and analysis. For example, \citet{athey2019machine} explores the application of machine learning in mechanism design, while \citet{einav2014economics} and \citet{agrawal2018prediction} demonstrate that big data can lead to more accurate market signal identification. Additional work by \citet{fuster2022predictably}, \citet{varian2010computer}, \citet{einav2014data} and \citet{gatteschi2018blockchain} further supports the view that these technologies are reshaping how adverse selection and moral hazard are addressed in financial and insurance markets.

Despite these advances, several challenges remain. Signal noise, high regulatory costs, and incomplete contracts \citep{hart1988incomplete} continue to limit the effectiveness of traditional mechanisms. Moreover, concerns over transparency, privacy protection \citep{acquisti2016economics}, data quality and the inherent "black box" nature of many AI models hinder the full realization of these technological benefits.

\subsection{Emerging Applications of Generative AI in Economic Research}
Recent advancements in generative AI, including models such as Generative Adversarial Networks \citep{goodfellow2014generative} and large language models \citep{brown2020language}, mark a shift from static prediction toward active information synthesis. Early studies have applied these models to simulate counterfactual scenarios in macroeconomic analysis and to generate synthetic populations that maintain key statistical properties while preserving privacy \citep{assefa2020generating}. In the realms of risk assessment and insurance, synthetic data generation has been used to test policy robustness and analyze agent heterogeneity \citep{xu2019modeling}.

These emerging applications demonstrate generative AI's potential to produce richer and more adaptable signals than those available through traditional machine learning techniques. However, the integration of AI-generated signals into mechanism design frameworks remains underdeveloped. Although these techniques can enhance data granularity, they are often constrained by static data dependencies and limited model transparency \citep{rudin2019stop, varian2014big, athey2019machine}. A key gap in the literature is the systematic embedding of generative AI signals into contracts and incentive schemes—a gap this paper aims to bridge.

Overall, the literature suggests that traditional methods and emerging AI-based techniques are complementary; yet, the transition from static, conventional approaches to dynamic, AI-enhanced mechanisms is not fully realized.Building on the comprehensive review above, our study addresses these gaps by proposing a dynamic mechanism design framework that incorporates generative AI signals. Our model not only captures the temporal evolution and feedback of information signals but also extends the analysis to multiple market structures. This work contributes both a novel theoretical framework and empirical evidence, laying the foundation for future research and policy interventions in AI-enhanced contract design.