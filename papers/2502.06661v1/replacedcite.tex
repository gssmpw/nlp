\section{Related Works}
There is an extensive body of literature on feature importance metrics ____, but research on feature interactions is comparatively sparse. Shapley interaction values extend the game-theoretic concept of Shapley values to measure pairwise feature interactions, decomposing model predictions into main and interaction effects ____. Similarly, the H-statistic, derived from partial dependence, quantifies the proportion of variance explained by feature interactions ____. While these methods are theoretically well-grounded and model-agnostic, their computational complexity increases exponentially with the number of features, limiting their practical applicability.


Some approaches have been developed to detect and quantify interactions within specific model classes ____, while other approaches focus on detecting the presence of interactions without providing a formal quantitative measure ____. While these approaches can be effective, their insights may require manual inspection and interpretation, or may not generalize to other types of models. 

There is a growing body of research on uncertainty quantification for feature importance ____. However, to the best of our knowledge, no one has developed a technique for uncertainty quantification, specifically for feature interactions. Our goal is to develop confidence intervals for our feature interaction metric that reflect the statistical significance and uncertainty of this metric. Our method builds upon the Leave-One-Covariate-Out (\loco) metric and inference framework originally proposed by ____. 
Further, as computational considerations are a major challenge for interactions, we leverage the fast inference approach incorporating minipatches, simultaneous subsampling of observations and features, to enhance the computational efficiency of our approach.