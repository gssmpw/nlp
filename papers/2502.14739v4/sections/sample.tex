\section{Difficulty-Stratified Samples}
% 定义 Easy 难度的盒子
\newtcolorbox[auto counter, number within=section]{easybox}[2][]{%
  colback=white, 
  colframe=orange!90!black, 
  width=\textwidth,
  arc=2mm, 
  boxrule=0.5mm, 
  title={\normalsize\faLeaf\hspace{0.5em}#2},
  breakable, 
  fonttitle=\bfseries\Large, 
  fontupper=\small
  #1
}

% 定义 Middle 难度的盒子
\newtcolorbox[auto counter, number within=section]{middlebox}[2][]{%
  colback=white, 
  colframe=magenta!40!red!80!black, 
  width=\textwidth,
  arc=2mm, 
  boxrule=0.5mm, 
  title={\normalsize\faSearchPlus\hspace{0.5em}#2},
  breakable, 
  fonttitle=\bfseries\Large, 
  fontupper=\small
  #1
}

% 定义 Hard 难度的盒子
\newtcolorbox[auto counter, number within=section]{hardbox}[2][]{%
  colback=white, 
  colframe=red!50!black, 
  width=\textwidth,
  arc=2mm, 
  boxrule=0.5mm, 
  title={\normalsize\faTrophy\hspace{0.5em}#2},
  breakable, 
  fonttitle=\bfseries\Large, 
  fontupper=\small
  #1
}

\begin{easybox}{Easy Sample}
\textbf{Question:}
\begin{enumerate}[leftmargin=0.5cm, label={}]  
\item Which of the following statements about dance studies are correct?\begin{enumerate}[label=\arabic*.]
    \item According to Danto's definition, context is an art world with modern aspects.
    \item ``La Bayadère'' is a ballet created during the French July Revolution.
    \item The ballet ``Sylvia'' is a dance drama created during the Paris Commune period in 1871.
    \item Korean court dance, when calculated according to temporal principles, does not belong to secondary civilization.
\end{enumerate}
\end{enumerate}

\textbf{Options:}
\begin{enumerate}[leftmargin=0.5cm, label={}]  
\item A) $1,3$
\item B) $1,4$
\item C) 1,2,4
\item D) 2,3 
\item E) 1,2,3
\item F) 3 
\item G) 4
\item H) 3,4 
\item I) 1,2,3,4 
\item J) 2,4 
\end{enumerate}

\textbf{Answer:} $3,4$

\textbf{Answer letter:} H

\textbf{Discipline:} Literature and Arts

\textbf{Field:} Art Studies

\textbf{Subfield:} Dance Studies

\textbf{Difficulty:} easy
\end{easybox}

\begin{middlebox}{Middle Sample}
\textbf{Question:}
\begin{enumerate}[leftmargin=0.5cm, label={}]  
\item A deck of playing cards has 52 cards, and each shuffle changes the order of the cards, which is a permutation. If each shuffle strictly follows the operation below: first divide the cards into two equal parts, then interlace the cards from each part alternately, so that the first and last cards remain in their original positions while all other cards are rearranged. Express this permutation as a product of disjoint cycles, write out the cyclic structure of this permutation, and determine the minimum number of shuffles needed to restore the deck to its original order.
\end{enumerate}


\textbf{Options:}
\begin{enumerate}[leftmargin=0.5cm, label={}]  
\item A) $(1^2, 2, 8^6), 8$
\item B) $(1^2, 5, 5^6), 6$
\item C) $(1^3, 1, 8^5), 15$
\item D) $(1^1, 4, 7^6), 11$
\item E) $(1^2, 4, 6^4), 12$
\item F) $(2^2, 1, 9^5), 9$
\item G) $(1^2, 3, 7^5), 10$
\item H) $(2^1, 2, 7^6), 7$
\item I) $(2^2, 3, 6^5), 5$
\item J) $(1^2, 6, 4^6), 4$
\end{enumerate}


\textbf{Answer:}$(1^2, 2, 8^6), 8$

\textbf{Answer letter:} A

\textbf{Discipline:} Science

\textbf{Field:} Mathematics

\textbf{Subfield:} Group Theory

\textbf{Difficulty:} middle

\end{middlebox}

\begin{hardbox}{Hard Sample}


\textbf{Question:}
\begin{enumerate}[leftmargin=0.5cm, label={}]  
    \item A certain transmitter has a transmission power of 10 W, a carrier frequency of 900 MHz, a transmission antenna gain \( G_{\mathrm{T}}=2 \), and a receiving antenna gain \( G_{\mathrm{R}}=3 \). Calculate the output power of the receiver and the path loss at a distance of 10 km from the transmitter in free space.
\end{enumerate}

\textbf{Options:}
\begin{enumerate}[leftmargin=1cm, label=\Alph*)]  
    \item \( P_{\mathrm{R}}=R_{\mathrm{T}} G_{\mathrm{T}} G_{\mathrm{R}} \left( \frac{\lambda} {3 \pi d} \right)^{2}=2. \, 5 5 \times1 0^{-1 2} \, \, \, \mathrm{W} \) \\
          \( L={\frac{P_{\mathrm{T}}} {P_{\mathrm{R}}}}=5. \, 1 9 \times1 0^{1 1} \, ( \mathrm{~ B I I ~ 1 4 2. ~ 4 3 ~ d B} ) \)
    \item \( P_{\mathrm{R}}=R_{\mathrm{T}} G_{\mathrm{T}} G_{\mathrm{R}} \left( \frac{\lambda} {4 \pi d} \right)^{2}=5. \, 0 7 \times1 0^{-1 1} \, \, \, \mathrm{W} \) \\
          \( L={\frac{P_{\mathrm{T}}} {P_{\mathrm{R}}}}=1. \, 9 7 \times1 0^{1 0} \, ( \mathrm{~ B I I ~ 1 0 1. ~ 9 5 ~ d B} ) \)
    \item \( P_{\mathrm{R}}=R_{\mathrm{T}} G_{\mathrm{T}} G_{\mathrm{R}} \left( \frac{\lambda} {4 \pi d} \right)^{2}=1. \, 7 4 \times1 0^{-1 0} \, \, \, \mathrm{W} \) \\
          \( L={\frac{P_{\mathrm{T}}} {P_{\mathrm{R}}}}=4. \, 8 1 \times1 0^{9} \, ( \mathrm{~ B I I ~ 9 8. ~ 4 8 ~ d B} ) \)
    \item \( P_{\mathrm{R}}=R_{\mathrm{T}} G_{\mathrm{T}} G_{\mathrm{R}} \left( \frac{\lambda} {5 \pi d} \right)^{2}=8. \, 4 0 \times1 0^{-1 0} \, \, \, \mathrm{W} \) \\
          \( L={\frac{P_{\mathrm{T}}} {P_{\mathrm{R}}}}=3. \, 9 8 \times1 0^{9} \, ( \mathrm{~ B I I ~ 9 6. ~ 0 7 ~ d B} ) \)
    \item \( P_{\mathrm{R}}=R_{\mathrm{T}} G_{\mathrm{T}} G_{\mathrm{R}} \left( \frac{\lambda} {4 \pi d} \right)^{2}=2. \, 9 3 \times1 0^{-1 0} \, \, \, \mathrm{W} \) \\
          \( L={\frac{P_{\mathrm{T}}} {P_{\mathrm{R}}}}=6. \, 2 9 \times1 0^{9} \, ( \mathrm{~ B I I ~ 9 9. ~ 9 0 ~ d B} ) \)
    \item \( P_{\mathrm{R}}=R_{\mathrm{T}} G_{\mathrm{T}} G_{\mathrm{R}} \left( \frac{\lambda} {3 \pi d} \right)^{2}=6. \, 8 3 \times1 0^{-1 0} \, \, \, \mathrm{W} \) \\
          \( L={\frac{P_{\mathrm{T}}} {P_{\mathrm{R}}}}=1. \, 4 6 \times1 0^{1 0} \, ( \mathrm{~ B I I ~ 8 9. ~  35 ~ d B} ) \)
    \item \( P_{\mathrm{R}}=R_{\mathrm{T}} G_{\mathrm{T}} G_{\mathrm{R}} \left( \frac{\lambda} {4 \pi d} \right)^{3}=9. \, 1 1 \times1 0^{-1 1} \, \, \, \mathrm{W} \) \\
          \( L={\frac{P_{\mathrm{T}}} {P_{\mathrm{R}}}}=3. \, 6 4 \times1 0^{1 0} \, ( \mathrm{~ B I I ~ 1 0 4. ~ 7 8 ~ d B} ) \)
    \item \( P_{\mathrm{R}}=R_{\mathrm{T}} G_{\mathrm{T}} G_{\mathrm{R}} \left( \frac{\lambda} {4 \pi d} \right)^{1}=3. \, 1 2 \times1 0^{-0 9} \, \, \, \mathrm{W} \) \\
          \( L={\frac{P_{\mathrm{T}}} {P_{\mathrm{R}}}}=9. \, 2 4 \times1 0^{9} \, ( \mathrm{~ B I I ~ 9 9. ~ 6 3 ~ d B} ) \)
    \item \( P_{\mathrm{R}}=R_{\mathrm{T}} G_{\mathrm{T}} G_{\mathrm{R}} \left( \frac{\lambda} {4 \pi d} \right)^{2}=4. \, 2 2 \times1 0^{-1 0} \, \, \, \mathrm{W} \) \\
          \( L={\frac{P_{\mathrm{T}}} {P_{\mathrm{R}}}}=2. \, 3 7 \times1 0^{1 0} \, ( \mathrm{~ B I I ~ 1 0 3. ~ 7 5 ~ d B} ) \)
    \item \( P_{\mathrm{R}}=R_{\mathrm{T}} G_{\mathrm{T}} G_{\mathrm{R}} \left( \frac{\lambda} {6 \pi d} \right)^{2}=7. \, 9 5 \times1 0^{-1 1} \, \, \, \mathrm{W} \) \\
          \( L={\frac{P_{\mathrm{T}}} {P_{\mathrm{R}}}}=2. \, 5 0 \times1 0^{1 0} \, ( \mathrm{~ B I I ~ 1 0 2. ~ 2 0 ~ d B} ) \)
\end{enumerate}

\textbf{Answer:}
\[
P_{\mathrm{R}}=R_{\mathrm{T}} G_{\mathrm{T}} G_{\mathrm{R}} \left( \frac{\lambda} {4 \pi d} \right)^{2}=4. \, 2 2 \times1 0^{-1 0} \, \, \, \mathrm{W}
\]
\[
L={\frac{P_{\mathrm{T}}} {P_{\mathrm{R}}}}=2. \, 3 7 \times1 0^{1 0} \, ( \mathrm{~ B I I ~ 1 0 3. ~ 7 5 ~ d B} )
\]

\textbf{Answer letter:} I

\textbf{Discipline:} Engineering

\textbf{Field:} Information and Communication Engineering

\textbf{Subfield:} Communication Principles

\textbf{Difficulty:} hard

\end{hardbox}