\newtcolorbox[auto counter, number within=section]{questionbox}[2][]{%
  colback=white, 
  colframe=blue!60!black,  % 深蓝色
  width=\textwidth,
  arc=2mm, 
  boxrule=0.5mm, 
  title={\normalsize\faLeaf\hspace{0.5em}#2},
  breakable, 
  fonttitle=\bfseries\Large, 
  fontupper=\small
  #1
}

\newtcolorbox[auto counter, number within=section]{errorbox}[2][]{%
    colback=white,
    colframe=red!50!black,  % 深红色
    width=\textwidth,
     arc=2mm, 
    boxrule=0.5mm, 
    title={\normalsize\faExclamationTriangle\hspace{0.5em}#2}, 
    breakable, 
    fonttitle=\bfseries\Large, 
    fontupper=\small
    #1
}

\subsection{Reasons for Failing Quality Inspection}


\begin{questionbox}{Question Block 1}
\noindent \textbf{Question:} Consider a resistor made from a hollow cylinder of carbon as shown below. The inner radius of the cylinder is $R_i=0.2$mm and the outer radius is $R_o=0.3$mm. The length of the resistor is $L=0.9$mm. The resistivity of the carbon is $\rho=3.5 \times 10^{-5} \ \Omega \cdot m$. What is the resistance?

\bigskip
\noindent \textbf{Options:}
\begin{itemize}
    \item 3.0
    \item 1.0
    \item 5.0
    \item 2.0
    \item 1.5
    \item 2.8
    \item 3.5
    \item 4.0
    \item 2.5
    \item 1.8
\end{itemize}

\noindent \textbf{Answer:} 2.5
\end{questionbox}

\begin{errorbox}{Error Analysis Block 1}
\noindent \textbf{Error Type:} Condition Setting Defect

\noindent \textbf{Error Analysis:} The solution to the correction problem requires a diagram as a condition; without the diagram, the problem cannot be solved. Therefore, this question lacks condition setting and cannot be answered.
\end{errorbox}


\begin{questionbox}{Question Block 2}
\noindent \textbf{Question:} Which of the following descriptions about the function of gastrointestinal motility is incorrect?

\bigskip
\noindent \textbf{Options:}
\begin{itemize}
    \item Controlling the release of bile from the gallbladder
    \item Facilitating the absorption of nutrients into the bloodstream
    \item Assisting in the immune response of the gut
    \item Regulating the pace of digestion
    \item Stimulating enzyme production in the pancreas
    \item Breaking down large food molecules into smaller molecules
    \item Mixing food with digestive juices
    \item Propelling food through the digestive tract
    \item Coordinating muscle contractions in the digestive system
    \item Maintaining the pH balance in the stomach
\end{itemize}

\noindent \textbf{Answer:} Breaking down large food molecules into smaller molecules
\end{questionbox}

\begin{errorbox}{Error Analysis Block 2}
\noindent \textbf{Error Type:} Incorrect option construction

\noindent \textbf{Error Analysis:} Options such as stimulating enzyme production in the pancreas, maintaining the pH balance in the stomach, and several other choices are correct answers to the question. Questions involving terms like "incorrect" or "most" require careful construction of distractors, but the reviewer fails to ensure the uniqueness of the correct answer.
\end{errorbox}


\begin{questionbox}{Question Block 3}
\noindent \textbf{Question:} Ross claims that we learn of our prima facie duties:

\bigskip
\noindent \textbf{Options:}
\begin{itemize}
    \item from the moral judgments we make in various situations.
    \item by seeing the prima facie rightness of particular acts, and then apprehending general principles.
    \item from the explicit moral instruction we receive as children.
    \item through societal norms and cultural values.
    \item from religious teachings or scriptures.
    \item by observing the consequences of our actions.
    \item by intuitively understanding moral obligations.
    \item by proving them philosophically.
    \item by apprehending general principles, and then inferring the prima facie rightness of particular acts.
    \item through legal regulations and laws.
\end{itemize}

\noindent \textbf{Answer:} by seeing the prima facie rightness of particular acts, and then apprehending general principles
\end{questionbox}

\begin{errorbox}{Error Analysis Block 3}
\noindent \textbf{Error Type:} Missing Contextual Information

\noindent \textbf{Error Analysis:} The problem does not provide sufficient contextual information, such as Ross's background or theories, to allow for proper comprehension and answer selection.
\end{errorbox}

\begin{questionbox}{Question Block 4}
\noindent \textbf{Question:} About the I-type system tracking the step signal, what's the steady-state error between them?

\bigskip
\noindent \textbf{Options:}
\begin{itemize}
    \item 10
    \item infinity
    \item 2
    \item -0.5
    \item 1
    \item 0.5
    \item undefined
    \item 0
    \item -1
    \item 100
\end{itemize}

\noindent \textbf{Answer:} 0
\end{questionbox}

\begin{errorbox}{Error Analysis Block 4}
\noindent \textbf{Error Type:} Distractor Quality Issue

\noindent \textbf{Error Analysis:} Some options, such as "infinity" and "undefined," may not be relevant or may be poorly constructed. These distractors may not effectively contribute to creating meaningful confusion.
\end{errorbox}

\begin{questionbox}{Question Block 5}
\noindent \textbf{Question:} A cube of iron was heated to 70°C and transferred to a beaker containing 100g of water at 20°C. The final temperature of the water and the iron was 23°C. What is the heat capacity?

\bigskip
\noindent \textbf{Options:}
\begin{itemize}
    \item $0.310 \ K^{-1} \ g^{-1}$
    \item $0.740 \ K^{-1} \ g^{-1}$
    \item $0.582 \ K^{-1} \ g^{-1}$
    \item $1.200 \ K^{-1} \ g^{-1}$
    \item $0.453 \ K^{-1} \ g^{-1}$
    \item $0.235 \ K^{-1} \ g^{-1}$
    \item $1.004 \ K^{-1} \ g^{-1}$
    \item $0.505 \ K^{-1} \ g^{-1}$
    \item $0.875 \ K^{-1} \ g^{-1}$
    \item $0.690 \ K^{-1} \ g^{-1}$
\end{itemize}

\noindent \textbf{Answer:} $0.453 \ K^{-1} \ g^{-1}$
\end{questionbox}

\begin{errorbox}{Error Analysis Block 5}
\noindent \textbf{Error Type:} Condition Setting Defect

\noindent \textbf{Error Analysis:} The problem lacks critical information about the mass of the iron cube, which makes it unsolvable.
\end{errorbox}


\begin{questionbox}{Question Block 6}
\noindent \textbf{Question:} The Central Committee of the Communist Party of China and the Central Government are striving to enhance our country’s military security functions. In order to achieve this, they should further promote the governance of the military according to law and strict discipline, focusing on the crucial ( ).

\bigskip
\noindent \textbf{Options:}
\begin{itemize}
    \item Checks and Balances on the Exercise of Power
    \item Procedures and Control of Power
    \item Maintenance and Supervision of Power
    \item Regulation and Oversight of Authority
    \item Coordination and Monitoring of Authority
    \item Systems and Regulation of Governance
    \item Security and Management of Authority
    \item Power Operation Management
    \item Power Operation Guarantee
    \item Power Operation Coordination
\end{itemize}

\noindent \textbf{Answer:} Checks and Balances on the Exercise of Power
\end{questionbox}

\begin{errorbox}{Error Analysis Block 6}
\noindent \textbf{Error Type:} Region-Specific Context Missing

\noindent \textbf{Error Analysis:} The question has a clear regional context (China) that should be accounted for in item selection or modification. Additionally, the item could be better adapted or refined to enhance clarity and relevance.
\end{errorbox}


\begin{questionbox}{Question Block 7}
\noindent \textbf{Question:} Use the example below to answer the question that follows.  
Identify the type of dissonance found above the asterisk in the given voice-leading example.

\bigskip
\noindent \textbf{Options:}
\begin{itemize}
    \item Neighbor tone
    \item Appoggiatura
    \item Suspension
    \item Passing tone
\end{itemize}

\noindent \textbf{Answer:} Passing tone
\end{questionbox}

\begin{errorbox}{Error Analysis Block 7}
\noindent \textbf{Error Type:} Missing Contextual Information

\noindent \textbf{Error Analysis:} The question lacks the necessary example (voice-leading notation) and cannot be answered.
\end{errorbox}

\begin{questionbox}{Question Block 8}
\noindent \textbf{Question:} Suppose an electric field $\mathbf{E}(x, y, z)$ has the form  
$$  
E_{x} = ax, \quad E_{y} = 0, \quad E_{z} = 0  
$$  
where $a$ is a constant. What is the charge density? How do you account for the fact that the field points in a particular direction, when the charge density is uniform?

\bigskip
\noindent \textbf{Options:}
\begin{itemize}
    \item $\epsilon_{0} x$
    \item $\epsilon_{0} y$
    \item $x a \epsilon$
    \item $\epsilon_{0} a$
    \item $x \epsilon_{0}$
    \item $\epsilon_{z} a$
    \item $\epsilon_{0} z$
    \item $a \epsilon_{y}$
    \item $\epsilon a_{y}$
    \item $\epsilon_{x} a$
\end{itemize}

\noindent \textbf{Answer:} $\epsilon_{0} a$
\end{questionbox}

\begin{errorbox}{Error Analysis Block 8}
\noindent \textbf{Error Type:} Incomplete Question and Answer Construction

\noindent \textbf{Error Analysis:} The question contains two sub-questions, but the answer and options only address one, causing ambiguity.
\end{errorbox}

