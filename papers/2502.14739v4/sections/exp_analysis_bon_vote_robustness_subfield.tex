\section{Further Experiment Analysis}

\subsection{Impact of Subfield Information}
\label{appendix:Impact of Subfield Information}

We conduct zero-shot evaluations under two conditions to investigate the impact of subfield information. 
1) zero-shot-with-subfield, where the prompt includes a description of the problem’s subfield, and (2)zero-shot-without-subfield, where no subfield information is provided.The prompts employed for the zero-shot-with-subfield evaluation are presented below.


\begin{promptbox}{Zero-shot-with-subfield}
Answer the following multiple choice question about 
\begin{verbatim}
{subfield}
\end{verbatim}
There is only one correct answer. The last line of your response should be in the format 'Answer: \$LETTER' (without quotes), where LETTER is one of A, B, C, D, E, F, G, H, I, or J.
\end{promptbox}
\vspace{0.5em}  
\begin{promptbox}{Question format example:}

The common-mode rejection ratio of the first stage amplification circuit in a three-op-amp differential circuit is determined by ( ).  

A) the absolute value of the difference in the common-mode rejection ratio of A1 and A2 themselves  
\\
B) all of the above  
\\
C) the average of A1 and A2's common-mode rejection ratios  
\\
D) the sum of A1 and A2's common-mode rejection ratios  
\\
E) the product of A1 and A2's common-mode rejection ratios  
\\
F) the square root of the product of A1 and A2's common-mode rejection ratios  
\\
G) the size of A2's common-mode rejection ratio  
\\
H) the size of A1's common-mode rejection ratio  
\\
I) The difference in the common-mode rejection ratio of A1 and A2 themselves  
\\
J) input resistance  
\end{promptbox}





\subsection{Robustness of the Evaluation}
\label{appendix:robustness}
We employ 4 types of initial prompts and 6 types of question formats, resulting in a
combination of 24 different prompt styles to verify the robustness of our bench. The 4 initial prompts and 6 types of question formats are listed below.

\begin{promptbox}{Initial Prompt 1}
Answer the following multiple choice question. There is only one correct answer. The last line of your response should be in the format 'Answer: \$LETTER' (without quotes), where LETTER is one of A, B, C, D, E, F, G, H, I, or J.
\end{promptbox}



\begin{promptbox}{Initial Prompt 2}
You are a helpful assistant. Answer the given multiple-choice question. Only one option is correct. The last line of your response should be in the format 'The correct answer is: \$LETTER', where LETTER is one of A, B, C, D, E, F, G, H, I, or J.

\end{promptbox}


\begin{promptbox}{Initial Prompt 3}
Select the correct answer for the following multiple-choice question. There is only one valid choice. The last line of your response should be in the format 'Answer: \$LETTER' (without quotes), where LETTER is one of A, B, C, D, E, F, G, H, I, or J.

\end{promptbox}


\begin{promptbox}{Initial Prompt 4}
Review the following multiple-choice question and choose the one correct answer. Ensure that your response concludes with a line exactly formatted as 'The correct answer is: \$LETTER', where LETTER represents one of A, B, C, D, E, F, G, H, I, or J.
\end{promptbox}


\begin{promptbox}{Question Format Example 1}
The common-mode rejection ratio of the first stage amplification circuit in a three-op-amp differential circuit is determined by ( ).  
\\
A) the absolute value of the difference in the common-mode rejection ratio of A1 and A2 themselves  
\\
B) all of the above  
\\
C) the average of A1 and A2's common-mode rejection ratios  
\\
D) the sum of A1 and A2's common-mode rejection ratios  
\\
E) the product of A1 and A2's common-mode rejection ratios  
\\
F) the square root of the product of A1 and A2's common-mode rejection ratios  
\\
G) the size of A2's common-mode rejection ratio  
\\
H) the size of A1's common-mode rejection ratio  
\\
I) The difference in the common-mode rejection ratio of A1 and A2 themselves  
\\
J) input resistance  
\end{promptbox}

\begin{promptbox}{Question Format Example 2}
The common-mode rejection ratio of the first stage amplification circuit in a three-op-amp differential circuit is determined by ( ).  
\\
A. the absolute value of the difference in the common-mode rejection ratio of A1 and A2 themselves  
\\
B. all of the above  
\\
C. the average of A1 and A2's common-mode rejection ratios  
\\
D. the sum of A1 and A2's common-mode rejection ratios  
\\
E. the product of A1 and A2's common-mode rejection ratios  
\\
F. the square root of the product of A1 and A2's common-mode rejection ratios  
\\
G. the size of A2's common-mode rejection ratio  
\\
H. the size of A1's common-mode rejection ratio  
\\
I. The difference in the common-mode rejection ratio of A1 and A2 themselves  
\\
J. input resistance  
Your response: 

\end{promptbox}


\begin{promptbox}{Question Format Example 3}
Question:  
\\
The common-mode rejection ratio of the first stage amplification circuit in a three-op-amp differential circuit is determined by ( ).  
\\
Options:  
\\
A) the absolute value of the difference in the common-mode rejection ratio of A1 and A2 themselves  
\\
B) all of the above  
\\
C) the average of A1 and A2's common-mode rejection ratios  
\\
D) the sum of A1 and A2's common-mode rejection ratios  
\\
E) the product of A1 and A2's common-mode rejection ratios  
\\
F) the square root of the product of A1 and A2's common-mode 
rejection ratios  
\\
G) the size of A2's common-mode rejection ratio  
\\
H) the size of A1's common-mode rejection ratio  
\\
I) The difference in the common-mode rejection ratio of A1 and A2 themselves  
\\
J) input resistance  
Please begin answering.  

\end{promptbox}







\begin{promptbox}{Question Format Example 4}
Q: The common-mode rejection ratio of the first stage amplification circuit in a three-op-amp differential circuit is determined by ( ).  
(A) the absolute value of the difference in the common-mode rejection ratio of A1 and A2 themselves (B) all of the above (C) the average of A1 and A2's common-mode rejection ratios (D) the sum of A1 and A2's common-mode rejection ratios (E) the product of A1 and A2's common-mode rejection ratios (F) the square root of the product of A1 and A2's common-mode rejection ratios (G) the size of A2's common-mode rejection ratio (H) the size of A1's common-mode rejection ratio (I) The difference in the common-mode rejection ratio of A1 and A2 themselves (J) input resistance

\end{promptbox}



\begin{promptbox}{Question Format Example 5}
**Question**:  
The common-mode rejection ratio of the first stage amplification circuit in a three-op-amp differential circuit is determined by ( ).  
\\
**Options**:  
\\
(A) the absolute value of the difference in the common-mode rejection ratio of A1 and A2 themselves  
\\
(B) all of the above  
\\
(C) the average of A1 and A2's common-mode rejection ratios  
\\
(D) the sum of A1 and A2's common-mode rejection ratios  
\\
(E) the product of A1 and A2's common-mode rejection ratios  
\\
(F) the square root of the product of A1 and A2's common-mode rejection ratios  
\\
(G) the size of A2's common-mode rejection ratio  
\\
(H) the size of A1's common-mode rejection ratio  
\\
(I) The difference in the common-mode rejection ratio of A1 and A2 themselves  
\\
(J) input resistance  

\end{promptbox}



\begin{promptbox}{Question Format Example 6}

Question: The common-mode rejection ratio of the first stage amplification circuit in a three-op-amp differential circuit is determined by ( ).  
\\
Options:  
\\
A: the absolute value of the difference in the common-mode rejection ratio of A1 and A2 themselves  
\\
B: all of the above  
\\
C: the average of A1 and A2's common-mode rejection ratios  
\\
D: the sum of A1 and A2's common-mode rejection ratios  
\\
E: the product of A1 and A2's common-mode rejection ratios  
\\
F: the square root of the product of A1 and A2's common-mode rejection ratios  
\\
G: the size of A2's common-mode rejection ratio  
\\
H: the size of A1's common-mode rejection ratio  
\\
I: The difference in the common-mode rejection ratio of A1 and A2 themselves  
\\
J: input resistance  

\end{promptbox}
