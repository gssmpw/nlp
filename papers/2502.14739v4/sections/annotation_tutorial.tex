\newtcolorbox[auto counter, number within=section]{methodbox}[2][]{%
  colback=white, 
  colframe=teal!80!green!80!black,  
  width=\textwidth,
  arc=2mm, 
  boxrule=0.5mm, 
  title={\normalsize\faWrench\hspace{0.5em}#2}, 
  breakable,
  fonttitle=\bfseries\Large, 
  fontupper=\small, 
  % drop shadow south east, 
  % drop shadow={south east, teal!50!black, shadow xshift=1.5mm, shadow yshift=-1.5mm},
  % drop shadow={south east, teal!50!black},
  % drop shadow={south east, teal!50!black, opacity=0.8, shadow xshift=2mm, shadow yshift=-2mm},
  #1
}

\section{Annotation Tutorial}
\label{appendix: annotation tutorial}
\subsection{Material Requirements}
The specific requirements for material selection are as follows:
\begin{itemize}
    \item Ensure the selected materials are accurate, with correct answers.
    \item Ensure the materials cover a variety of knowledge points and are free from regional bias.
    \item The selected materials should be correctly stated and involve a certain level of academic reasoning, with a difficulty appropriate for graduate-level study.
    \item Avoid using materials that rely on images as conditions.
    \item Verify that the materials comply with copyright requirements.
\end{itemize}
\subsection{Annotation Methods}
During the annotation process, the following four methods are primarily used:  \textbf{1) Original Transcription Method}, \textbf{2) Non-Choice Conversion Method}, and \textbf{3) Statement Combination Method}.
\subsubsection{Original Transcription Method}
\begin{methodbox}{Original Transcription}
\subsection*{Description:}
\begin{itemize}
    \item Directly transcribe the original multiple-choice questions, ensuring the question maintains its original meaning and correctness. Additional distractors are included to increase the difficulty of the question or enhance its discriminative power.
\end{itemize}
\subsection*{Operations:}
\begin{enumerate}
    \item \textbf{Material Review:}
    \begin{itemize}
        \item Ensure that the materials meet the aforementioned requirements, including accuracy, diversity, neutrality (free from regional bias), non-image-based content, and compliance with copyright regulations.
    \end{itemize}
    \item \textbf{Question Transcription:}
    \begin{itemize}
        \item Use OCR tools to recognize the text in the materials or directly paste the original content. Transcribe the question content word-for-word, ensuring no key information is omitted, with particular attention to the accurate transcription of formulas.
    \end{itemize}
    \item \textbf{Option Transcription:}
    \begin{itemize}
        \item Use OCR tools to recognize the text in the materials or directly paste the original content. Transcribe the options word-for-word, ensuring no key information is omitted, with particular attention to the accurate transcription of formulas.
    \end{itemize}
    \item \textbf{Answer Identification:}
    \begin{itemize}
        \item Clearly mark the correct answer within the list of options, ensuring its accuracy and uniqueness.
    \end{itemize}
    \item \textbf{Distractor Addition:}
    \begin{itemize}
        \item Add distractors while maintaining the quality of the options (avoiding meaningless or excess correct answers) until the annotator deems the level of confusion sufficient. The number of options should be between 4 and 10, with more options being preferred to increase the difficulty and discriminatory power of the question.
    \end{itemize}
    \item \textbf{Field Completion:}
    \begin{itemize}
        \item Complete the category fields ("discipline","field","subfield") and select the appropriate difficulty level to ensure the integrity and accuracy of the question information.
    \end{itemize}
\end{enumerate}
\end{methodbox}

\subsubsection{Non-Choice Conversion Method}
\begin{methodbox}{Non-Choice Conversion}
\subsection*{Description:}
\begin{itemize}
    \item Convert non-multiple-choice questions (such as calculation questions, fill-in-the-blank questions, etc.) into multiple-choice format, ensuring that the conditions of the question are complete and the final result is accurate while generating reasonable distractors based on the analysis steps or calculation process.
\end{itemize}
\subsection*{Operations:}
\begin{enumerate}
    \item \textbf{Material Review:}
    \begin{itemize}
        \item Ensure that the materials meet the aforementioned requirements, including accuracy, diversity, neutrality (free from regional bias), non-image-based content, and compliance with copyright regulations.
        \item Ensure that the conditions of the question are complete and do not cause ambiguity. Add supplementary explanations if necessary.
        \item Ensure that the answer does not contain any essential explanatory process that is unsuitable for adaptation or transcription.
    \end{itemize}
    \item \textbf{Question Transcription:}
    \begin{itemize}
        \item Use OCR tools to recognize the text in the material or directly paste the original text. Transcribe the content of the original question word by word, ensuring that all numbers, formulas, or condition information in the question are accurate and complete to avoid ambiguity or incorrect answers.
        \item Add supplementary information if the question requires prior conditions from other questions.
    \end{itemize}
    \item \textbf{Answer Transcription:}
    \begin{itemize}
        \item Identify and select the correct result for the appropriate question based on the answer analysis process.
        item Use OCR tools to recognize results in the material and confirm that all numerical and formula information in the answer is accurate.
    \end{itemize}
    \item \textbf{Distractor Addition:}
    \begin{itemize}
        \item On the basis of the correct answer, consider setting common calculation errors (such as rounding errors, sign mistakes, digit errors, etc.) as distractors.
        \item For numerical results, the setting of distractors should consider reasonable ranges of calculation errors to ensure differentiation between options.
    \end{itemize}
    \item \textbf{Answer Identification:}
    \begin{itemize}
        \item Clearly mark the correct answer within the list of options, ensuring its accuracy and uniqueness.
    \end{itemize}
    \item \textbf{Field Completion:}
    \begin{itemize}
        \item Complete the category fields ("discipline","field","subfield") and select the appropriate difficulty level to ensure the integrity and accuracy of the question information.
    \end{itemize}
\end{enumerate}
\end{methodbox}
\subsubsection{Statement Combination Method}
\begin{methodbox}{Statement Combination}
\subsection*{Description:}
\begin{itemize}
    \item By integrating stated expressions related to multiple concepts or knowledge points, a multi-level, multi-perspective multiple-choice question is constructed to increase the comprehensiveness and depth of examination of the topic.
\end{itemize}
\subsection*{Operations:}
\begin{enumerate}
    \item \textbf{Statement Extraction:}
    \begin{itemize}
        \item Extract core concepts, definitions, relationships between concepts, application cases, and common misconceptions from textbooks and related learning resources, ensuring that the statements are representative and comprehensive.
        \item Extract important statements from multiple-choice questions, covering multiple knowledge points to avoid being limited to a single definition or formula, thus enhancing the overall comprehensiveness of the questions.
        \item Ensure that the extracted statements are accurate and concise, avoiding redundancy or vague expressions. The content should cover both correct and incorrect situations to provide a foundation for subsequent adaptation.
    \end{itemize}
    \item \textbf{Statement Adaptation:}
    \begin{itemize}
        \item Adapt the extracted statements to include both correct and incorrect versions.
        \item Incorrect statements should be somewhat misleading, avoiding obvious or easily dismissible errors.
        \item Number the adapted statements (e.g., I, II, III, etc.). Arrange the statements in a reasonable order, avoiding bias towards any direction (e.g., always placing correct statements first).
    \end{itemize}
    \item \textbf{Question Design:}
    \begin{itemize}
        \item Depending on the need, questions may limit the scope or conditions being tested.
        \item "Which of the following statements about [knowledge point] is correct?"
        \item "Which of the following statements about [knowledge point] is incorrect?"
    \end{itemize}
    \item \textbf{Combination Design:}
    \begin{itemize}
        \item Combine the numbered statements to create options, such as I and II; II and III; III, VI, and VII.
        \item Ensure that there is exactly one correct answer among the options.
    \end{itemize}
    \item \textbf{Answer Identification:}
    \begin{itemize}
        \item Clearly mark the correct answer within the list of options, ensuring its accuracy and uniqueness.
    \end{itemize}
    \item \textbf{Field Completion:}
    \begin{itemize}
        \item Complete the category fields ("discipline","field","subfield") and select the appropriate difficulty level to ensure the integrity and accuracy of the question information.
    \end{itemize}
\end{enumerate}
\end{methodbox}
\subsubsection{Confusion-options Generation}
During the annotation process, we use large models to assist in generating distractors. We select Claude-3.5, GPT-4, Doubao, and Qwen2.5-72B-Instruct, and follow the prompt below to generate confusion options. In the annotation process, a model is randomly selected to generate a confusion option, which is then double-checked by the annotator. Finally, the confirmed confusion option is used as the final option.

\begin{methodbox}{Prompt}
You are an expert in creating multiple-choice questions. Your task is to generate plausible but incorrect distractors for a given question that only has one correct answer. You are skilled at introducing subtle yet distinct mathematical errors to the existing answer options, making the distractor look reasonable but still wrong.

\vspace{0.5em}  
\textbf{Input Description:}
\begin{itemize}
    \item You will be given:
    \begin{itemize}
        \item A question stem.
        \item A set of answer options (no numbering).
        \item The correct answer (no numbering).
    \end{itemize}
\end{itemize}

\vspace{0.5em}  
\textbf{Guidelines:}
\begin{enumerate}
    \item \textbf{Generate Distractor:}
    \begin{itemize}
        \item The distractor must be incorrect (not the correct answer).
        \item The distractor should introduce a subtle mathematical error while maintaining the formula structure.
        \item It must be \textbf{distinct} from all existing options, including the correct answer.
        \item Avoid any \textbf{repetition} or overlap with the existing answer options in terms of value or meaning.
        \item \textbf{Plausibility}: The distractor should look reasonable and appear to be a potential answer, but still be wrong.
        \item Ensure the question still has exactly \textbf{one correct answer} after adding the distractor.
    \end{itemize}
    
    \item \textbf{Uniqueness Check:}
    \begin{itemize}
        \item \textbf{Thoroughly check} all existing options (including the correct one) to ensure the distractor is unique.
        \item If the distractor matches any existing option, regenerate it.
        \item The distractor must \textbf{not} create ambiguity; the correct answer must remain the sole valid choice.
    \end{itemize}
    
    \item \textbf{Formatting:}
    \begin{itemize}
        \item Ensure the distractor matches the format of the existing options (e.g., fractions, exponents, etc.).
        \item Avoid formatting inconsistencies such as misplaced symbols or spaces.
    \end{itemize}
\end{enumerate}

\vspace{0.5em}  
\textbf{Output Format:}
\begin{verbatim}
<distractor> your generated distractor here </distractor>
\end{verbatim}
\noindent Do not include explanations or extra information.  
\noindent Do not include numbering.

\vspace{0.5em}  
\textbf{Input:}
\begin{verbatim}
    Question: "{}",
    Options: {},
    Correct Answer: "{}"
\end{verbatim}
\noindent \textbf{Output:}
\end{methodbox}