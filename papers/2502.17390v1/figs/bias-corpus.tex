\begin{figure*}[t]
    \textbf{\hspace{1.5cm}\genderData\hspace{5.7cm}\politicalData}\\
    \centering
    \subfloat[Full range\label{fig:corpus-bias-figure}]{\includegraphics[width=0.25\textwidth]{images/corpus-bias-figures-llama405-nq.png}}\hfill
    \subfloat[Limited range\label{fig:corpus-bias-figure-lim}]{\includegraphics[width=0.25\textwidth]{images/corpus-bias-figures-llama405-nq-lim.png}}\hfill
    \subfloat[Full range\label{fig:corpus-bias-political}]{\includegraphics[width=0.25\textwidth]{images/corpus-bias-political-llama405-Pol_NLI.png}} \hfill
    \subfloat[Limited range\label{fig:corpus-bias-political-lim}]{\includegraphics[width=0.25\textwidth]{images/corpus-bias-political-llama405-Pol_NLI-lim.png}}\\

    \caption{\textbf{Corpus Bias.} RAG bias ($R_b$) when the corpus bias ($C_b$) changes for three different embedders. The base embedder is \texttt{GTE-base}, the optimal embedder is the embedder that results in $R_b \approx 0$ with a neutral corpus ($C_b$), and the degenerate embedder is a heavily reverse biased embedder. The RAG bias scales linearly with the corpus bias for the \textcolor{darkgray}{base} and \textcolor{blue}{optimal} embedder while the linearity breaks as the embedder becomes more \textcolor{red}{degenerate}.}
    \label{fig:corpus-bias}
\end{figure*}