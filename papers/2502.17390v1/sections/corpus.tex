\section{Results: Corpus \& RAG}
\label{sec:corpus}

In the previous section, we revealed a linear relationship between the embedder bias and RAG bias while keeping the corpus consistent. Here we investigate how changing the corpus bias ($C_b$) affects the linear trend seen previously in \reffig{training}.

We create small toy corpus with pre-evaluated biases of each document to systematically study this. For \genderData, we collect a subset of NQ by first selecting the top-100 documents related to each query with the base embedder. Next, we keep the number of documents that are biased towards males and females equal. This results in a small corpus of 352 documents (male: 176 / female: 176). We note that this subset has a different distribution from NQ. We repeat the same for \politicalData with PolNLI and get a corpus of 2564 documents (liberal: 1282 / conservative: 1282).

\paragraph{Corpus \& RAG Bias Results}

In \reffig{corpus-bias}, we control the ratio of bias ($C_b$) of the subset corpus and plot the RAG bias ($R_b$) of three embedders when connected to Llama 405B. The base embedder is \texttt{GTE-base}, the optimal embedder is the one that achieves $R_b \approx 0$ on the subset corpus, and the degenerate embedder is a heavily fine-tuned embedder past optimal. In \reffigs{corpus-bias-figure}{corpus-bias-political}, a linear relationship holds between the RAG bias ($R_b$) and corpus bias ($C_b$) for the base embedder and optimal embedder (\textcolor{gray}{black} and \textcolor{blue}{blue} lines). However, linearity does not hold with a heavily biased embedder (\textcolor{red}{red} line). Furthermore, with small variations in the corpus bias around 0 (\reffigs{corpus-bias-figure-lim}{corpus-bias-political-lim}), the optimal embedder for the original corpus is still optimal for small shifts in the corpus bias.

