% This must be in the first 5 lines to tell arXiv to use pdfLaTeX, which is strongly recommended.
\pdfoutput=1
% In particular, the hyperref package requires pdfLaTeX in order to break URLs across lines.

\documentclass[11pt]{article}

% Change "review" to "final" to generate the final (sometimes called camera-ready) version.
% Change to "preprint" to generate a non-anonymous version with page numbers.
\usepackage[preprint]{acl}

% Standard package includes
\usepackage{times}
\usepackage{latexsym}

% For proper rendering and hyphenation of words containing Latin characters (including in bib files)
\usepackage[T1]{fontenc}
% For Vietnamese characters
% \usepackage[T5]{fontenc}
% See https://www.latex-project.org/help/documentation/encguide.pdf for other character sets

% This assumes your files are encoded as UTF8
\usepackage[utf8]{inputenc}

% This is not strictly necessary, and may be commented out,
% but it will improve the layout of the manuscript,
% and will typically save some space.
\usepackage{microtype}

% This is also not strictly necessary, and may be commented out.
% However, it will improve the aesthetics of text in
% the typewriter font.
\usepackage{inconsolata}

%Including images in your LaTeX document requires adding
%additional package(s)
\usepackage{graphicx}

% If the title and author information does not fit in the area allocated, uncomment the following
%
%\setlength\titlebox{<dim>}
%
% and set <dim> to something 5cm or larger.

\usepackage[utf8]{inputenc} % allow utf-8 input
\usepackage[T1]{fontenc}    % use 8-bit T1 fonts
\usepackage{hyperref}       % hyperlinks
\usepackage{url}            % simple URL typesetting
\usepackage{booktabs}       % professional-quality tables
\usepackage{amsfonts}       % blackboard math symbols
\usepackage{nicefrac}       % compact symbols for 1/2, etc.
\usepackage{microtype}      % microtypography
\usepackage{graphicx}
\usepackage{lipsum}
% \usepackage[caption=false]{subfig}
\usepackage{amsmath}
\usepackage{tabularx}
\usepackage{makecell}
\usepackage{listings}
\usepackage{multirow}
\usepackage{array}
\usepackage{subcaption}
\usepackage{xcolor}
\usepackage{soul}
\usepackage{colortbl}
\usepackage{tcolorbox}
\usepackage{placeins}
\usepackage{xspace}
\tcbuselibrary{breakable}
%%% THIS FILE IS AUTOMATICALLY GENERATED.  DON'T MODIFY, OR YOUR CHANGES MIGHT BE OVERWRITTEN!
\newcommand\poi{\ensuremath{\text{Poi}}}
\newcommand\est[1]{\bs{\hat \mu_{#1}}}
\newcommand\true{\ensuremath{\bs{\mu^\star}}}
\newcommand\sa{\ensuremath{\mathcal{a}}}
\newcommand\sd{\ensuremath{\mathcal{d}}}
\newcommand\se{\ensuremath{\mathcal{e}}}
\newcommand\sg{\ensuremath{\mathcal{g}}}
\newcommand\sh{\ensuremath{\mathcal{h}}}
\newcommand\si{\ensuremath{\mathcal{i}}}
\newcommand\sj{\ensuremath{\mathcal{j}}}
\newcommand\sk{\ensuremath{\mathcal{k}}}
\newcommand\sm{\ensuremath{\mathcal{m}}}
\newcommand\sn{\ensuremath{\mathcal{n}}}
\newcommand\so{\ensuremath{\mathcal{o}}}
\newcommand\sq{\ensuremath{\mathcal{q}}}
\newcommand\sr{\ensuremath{\mathcal{r}}}
\newcommand\st{\ensuremath{\mathcal{t}}}
\newcommand\su{\ensuremath{\mathcal{u}}}
\newcommand\sv{\ensuremath{\mathcal{v}}}
\newcommand\sw{\ensuremath{\mathcal{w}}}
\newcommand\sx{\ensuremath{\mathcal{x}}}
\newcommand\sy{\ensuremath{\mathcal{y}}}
\newcommand\sz{\ensuremath{\mathcal{z}}}
\newcommand\sA{\ensuremath{\mathcal{A}}}
\newcommand\sB{\ensuremath{\mathcal{B}}}
\newcommand\sC{\ensuremath{\mathcal{C}}}
\newcommand\sD{\ensuremath{\mathcal{D}}}
\newcommand\sE{\ensuremath{\mathcal{E}}}
\newcommand\sF{\ensuremath{\mathcal{F}}}
\newcommand\sG{\ensuremath{\mathcal{G}}}
\newcommand\sH{\ensuremath{\mathcal{H}}}
\newcommand\sI{\ensuremath{\mathcal{I}}}
\newcommand\sJ{\ensuremath{\mathcal{J}}}
\newcommand\sK{\ensuremath{\mathcal{K}}}
\newcommand\sL{\ensuremath{\mathcal{L}}}
\newcommand\sM{\ensuremath{\mathcal{M}}}
\newcommand\sN{\ensuremath{\mathcal{N}}}
\newcommand\sO{\ensuremath{\mathcal{O}}}
\newcommand\sP{\ensuremath{\mathcal{P}}}
\newcommand\sQ{\ensuremath{\mathcal{Q}}}
\newcommand\sR{\ensuremath{\mathcal{R}}}
\newcommand\sS{\ensuremath{\mathcal{S}}}
\newcommand\sT{\ensuremath{\mathcal{T}}}
\newcommand\sU{\ensuremath{\mathcal{U}}}
\newcommand\sV{\ensuremath{\mathcal{V}}}
\newcommand\sW{\ensuremath{\mathcal{W}}}
\newcommand\sX{\ensuremath{\mathcal{X}}}
\newcommand\sY{\ensuremath{\mathcal{Y}}}
\newcommand\sZ{\ensuremath{\mathcal{Z}}}
\newcommand\ba{\ensuremath{\mathbf{a}}}
\newcommand\bb{\ensuremath{\mathbf{b}}}
\newcommand\bc{\ensuremath{\mathbf{c}}}
%\newcommand\bd{\ensuremath{\mathbf{d}}}
\newcommand\be{\ensuremath{\mathbf{e}}}
\newcommand\bg{\ensuremath{\mathbf{g}}}
\newcommand\bh{\ensuremath{\mathbf{h}}}
\newcommand\bi{\ensuremath{\mathbf{i}}}
\newcommand\bj{\ensuremath{\mathbf{j}}}
\newcommand\bk{\ensuremath{\mathbf{k}}}
\newcommand\bl{\ensuremath{\mathbf{l}}}
\newcommand\bn{\ensuremath{\mathbf{n}}}
\newcommand\bo{\ensuremath{\mathbf{o}}}
\newcommand\bp{\ensuremath{\mathbf{p}}}
\newcommand\bq{\ensuremath{\mathbf{q}}}
\newcommand\br{\ensuremath{\mathbf{r}}}
\newcommand\bs{\ensuremath{\boldsymbol}}
\newcommand\bt{\ensuremath{\mathbf{t}}}
\newcommand\bu{\ensuremath{\mathbf{u}}}
\newcommand\bv{\ensuremath{\mathbf{v}}}
\newcommand\bw{\ensuremath{\mathbf{w}}}
\newcommand\bx{\ensuremath{\mathbf{x}}}
\newcommand\by{\ensuremath{\mathbf{y}}}
\newcommand\bz{\ensuremath{\mathbf{z}}}
\newcommand\bA{\ensuremath{\mathbf{A}}}
\newcommand\bB{\ensuremath{\mathbf{B}}}
\newcommand\bC{\ensuremath{\mathbf{C}}}
\newcommand\bD{\ensuremath{\mathbf{D}}}
\newcommand\bE{\ensuremath{\mathbf{E}}}
\newcommand\bF{\ensuremath{\mathbf{F}}}
\newcommand\bG{\ensuremath{\mathbf{G}}}
\newcommand\bH{\ensuremath{\mathbf{H}}}
\newcommand\bI{\ensuremath{\mathbf{I}}}
\newcommand\bJ{\ensuremath{\mathbf{J}}}
\newcommand\bK{\ensuremath{\mathbf{K}}}
\newcommand\bL{\ensuremath{\mathbf{L}}}
\newcommand\bM{\ensuremath{\mathbf{M}}}
\newcommand\bN{\ensuremath{\mathbf{N}}}
\newcommand\bO{\ensuremath{\mathbf{O}}}
\newcommand\bP{\ensuremath{\mathbf{P}}}
\newcommand\bQ{\ensuremath{\mathbf{Q}}}
\newcommand\bR{\ensuremath{\mathbf{R}}}
\newcommand\bS{\ensuremath{\mathbf{S}}}
\newcommand\bT{\ensuremath{\mathbf{T}}}
\newcommand\bU{\ensuremath{\mathbf{U}}}
\newcommand\bV{\ensuremath{\mathbf{V}}}
\newcommand\bW{\ensuremath{\mathbf{W}}}
\newcommand\bX{\ensuremath{\mathbf{X}}}
\newcommand\bY{\ensuremath{\mathbf{Y}}}
\newcommand\bZ{\ensuremath{\mathbf{Z}}}
\newcommand\Ba{\ensuremath{\mathbb{a}}}
\newcommand\Bb{\ensuremath{\mathbb{b}}}
\newcommand\Bc{\ensuremath{\mathbb{c}}}
\newcommand\Bd{\ensuremath{\mathbb{d}}}
\newcommand\Be{\ensuremath{\mathbb{e}}}
\newcommand\Bf{\ensuremath{\mathbb{f}}}
\newcommand\Bg{\ensuremath{\mathbb{g}}}
\newcommand\Bh{\ensuremath{\mathbb{h}}}
\newcommand\Bi{\ensuremath{\mathbb{i}}}
\newcommand\Bj{\ensuremath{\mathbb{j}}}
\newcommand\Bk{\ensuremath{\mathbb{k}}}
\newcommand\Bl{\ensuremath{\mathbb{l}}}
\newcommand\Bm{\ensuremath{\mathbb{m}}}
\newcommand\Bn{\ensuremath{\mathbb{n}}}
\newcommand\Bo{\ensuremath{\mathbb{o}}}
\newcommand\Bp{\ensuremath{\mathbb{p}}}
\newcommand\Bq{\ensuremath{\mathbb{q}}}
\newcommand\Br{\ensuremath{\mathbb{r}}}
\newcommand\Bs{\ensuremath{\mathbb{s}}}
\newcommand\Bt{\ensuremath{\mathbb{t}}}
\newcommand\Bu{\ensuremath{\mathbb{u}}}
\newcommand\Bv{\ensuremath{\mathbb{v}}}
\newcommand\Bw{\ensuremath{\mathbb{w}}}
\newcommand\Bx{\ensuremath{\mathbb{x}}}
\newcommand\By{\ensuremath{\mathbb{y}}}
\newcommand\Bz{\ensuremath{\mathbb{z}}}
\newcommand\BA{\ensuremath{\mathbb{A}}}
\newcommand\BB{\ensuremath{\mathbb{B}}}
\newcommand\BC{\ensuremath{\mathbb{C}}}
\newcommand\BD{\ensuremath{\mathbb{D}}}
\newcommand\BE{\ensuremath{\mathbb{E}}}
\newcommand\BF{\ensuremath{\mathbb{F}}}
\newcommand\BG{\ensuremath{\mathbb{G}}}
\newcommand\BH{\ensuremath{\mathbb{H}}}
\newcommand\BI{\ensuremath{\mathbb{I}}}
\newcommand\BJ{\ensuremath{\mathbb{J}}}
\newcommand\BK{\ensuremath{\mathbb{K}}}
\newcommand\BL{\ensuremath{\mathbb{L}}}
\newcommand\BM{\ensuremath{\mathbb{M}}}
\newcommand\BN{\ensuremath{\mathbb{N}}}
\newcommand\BO{\ensuremath{\mathbb{O}}}
\newcommand\BP{\ensuremath{\mathbb{P}}}
\newcommand\BQ{\ensuremath{\mathbb{Q}}}
\newcommand\BR{\ensuremath{\mathbb{R}}}
\newcommand\BS{\ensuremath{\mathbb{S}}}
\newcommand\BT{\ensuremath{\mathbb{T}}}
\newcommand\BU{\ensuremath{\mathbb{U}}}
\newcommand\BV{\ensuremath{\mathbb{V}}}
\newcommand\BW{\ensuremath{\mathbb{W}}}
\newcommand\BX{\ensuremath{\mathbb{X}}}
\newcommand\BY{\ensuremath{\mathbb{Y}}}
\newcommand\BZ{\ensuremath{\mathbb{Z}}}
\newcommand\balpha{\ensuremath{\mbox{\boldmath$\alpha$}}}
\newcommand\bbeta{\ensuremath{\mbox{\boldmath$\beta$}}}
\newcommand\btheta{\ensuremath{\mbox{\boldmath$\theta$}}}
\newcommand\bphi{\ensuremath{\mbox{\boldmath$\phi$}}}
\newcommand\bpi{\ensuremath{\mbox{\boldmath$\pi$}}}
\newcommand\bpsi{\ensuremath{\mbox{\boldmath$\psi$}}}
\newcommand\bmu{\ensuremath{\mbox{\boldmath$\mu$}}}

% Figures
\newcommand\fig[1]{\begin{center} \includegraphics{#1} \end{center}}
\newcommand\Fig[4]{\begin{figure}[ht] \begin{center} \includegraphics[scale=#2]{#1} \end{center} \vspace{-1.0em} \caption{\label{fig:#3} #4} \vspace{-.5em} \end{figure}}
\newcommand\FigTop[4]{\begin{figure}[t] \begin{center} \includegraphics[scale=#2]{#1} \end{center} \vspace{-1.0em} \caption{\label{fig:#3} #4} \vspace{-.5em} \end{figure}}
\newcommand\FigStar[4]{\begin{figure*}[ht] \begin{center} \includegraphics[scale=#2]{#1} \end{center} \caption{\label{fig:#3} #4} \end{figure*}}
\newcommand\FigRight[4]{\begin{wrapfigure}{r}{0.5\textwidth} \begin{center} \includegraphics[width=0.5\textwidth]{#1} \end{center} \caption{\label{fig:#3} #4} \end{wrapfigure}}
\newcommand\aside[1]{\quad\text{[#1]}}
\newcommand\interpret[1]{\llbracket #1 \rrbracket} % Denotation
% operators
\DeclareMathOperator*{\tr}{tr}
%\DeclareMathOperator*{\sign}{sign}
\newcommand{\var}{\text{Var}} % Variance
\DeclareMathOperator*{\cov}{Cov} % Covariance
\DeclareMathOperator*{\diag}{diag} % Diagonal matrix
\newcommand\p[1]{\ensuremath{\left( #1 \right)}} % Parenthesis ()
\newcommand\pa[1]{\ensuremath{\left\langle #1 \right\rangle}} % <>
\newcommand\pb[1]{\ensuremath{\left[ #1 \right]}} % []
\newcommand\pc[1]{\ensuremath{\left\{ #1 \right\}}} % {}
\newcommand\eval[2]{\ensuremath{\left. #1 \right|_{#2}}} % Integral evaluation
\newcommand\inv[1]{\ensuremath{\frac{1}{#1}}}
\newcommand\half{\ensuremath{\frac{1}{2}}}
\newcommand\R{\ensuremath{\mathbb{R}}} % Real numbers
\newcommand\Z{\ensuremath{\mathbb{Z}}} % Integers
\newcommand\inner[2]{\ensuremath{\left< #1, #2 \right>}} % Inner product
\newcommand\mat[2]{\ensuremath{\left(\begin{array}{#1}#2\end{array}\right)}} % Matrix
\newcommand\eqn[1]{\begin{align} #1 \end{align}} % Equation (array)
\newcommand\eqnl[2]{\begin{align} \label{eqn:#1} #2 \end{align}} % Equation (array) with label
\newcommand\eqdef{\ensuremath{\stackrel{\rm def}{=}}} % Equal by definition
\newcommand{\1}{\mathbb{I}} % Indicator (don't use \mathbbm{1} because bbm is not TrueType)
\newcommand{\bone}{\mathbf{1}} % for vector one
\newcommand{\bzero}{\mathbf{0}} % for vector zero
\newcommand\refeqn[1]{(\ref{eqn:#1})}
\newcommand\refeqns[2]{(\ref{eqn:#1}) and (\ref{eqn:#2})}
\newcommand\refchp[1]{Chapter~\ref{chp:#1}}
\newcommand\refsec[1]{Section~\ref{sec:#1}}
\newcommand\refsecs[2]{Sections~\ref{sec:#1} and~\ref{sec:#2}}
\newcommand\reffig[1]{Figure~\ref{fig:#1}}
\newcommand\reffigs[2]{Figures~\ref{fig:#1} and~\ref{fig:#2}}
\newcommand\reffigss[3]{Figures~\ref{fig:#1},~\ref{fig:#2}, and~\ref{fig:#3}}
\newcommand\reffigsss[4]{Figures~\ref{fig:#1},~\ref{fig:#2},~\ref{fig:#3}, and~\ref{fig:#4}}
\newcommand\reftab[1]{Table~\ref{tab:#1}}
\newcommand\refapp[1]{Appendix~\ref{sec:#1}}
\newcommand\refthm[1]{Theorem~\ref{thm:#1}}
\newcommand\refthms[2]{Theorems~\ref{thm:#1} and~\ref{thm:#2}}
\newcommand\reflem[1]{Lemma~\ref{lem:#1}}
\newcommand\reflems[2]{Lemmas~\ref{lem:#1} and~\ref{lem:#2}}
\newcommand\refalg[1]{Algorithm~\ref{alg:#1}}
\newcommand\refalgs[2]{Algorithms~\ref{alg:#1} and~\ref{alg:#2}}
\newcommand\refex[1]{Example~\ref{ex:#1}}
\newcommand\refexs[2]{Examples~\ref{ex:#1} and~\ref{ex:#2}}
\newcommand\refprop[1]{Proposition~\ref{prop:#1}}
\newcommand\refdef[1]{Definition~\ref{def:#1}}
\newcommand\refcor[1]{Corollary~\ref{cor:#1}}
\newcommand\Chapter[2]{\chapter{#2}\label{chp:#1}}
\newcommand\Section[2]{\section{#2}\label{sec:#1}}
\newcommand\Subsection[2]{\subsection{#2}\label{sec:#1}}
\newcommand\Subsubsection[2]{\subsubsection{#2}\label{sec:#1}}
\ifthenelse{\isundefined{\definition}}{\newtheorem{definition}{Definition}}{}
\ifthenelse{\isundefined{\assumption}}{\newtheorem{assumption}{Assumption}}{}
\ifthenelse{\isundefined{\hypothesis}}{\newtheorem{hypothesis}{Hypothesis}}{}
\ifthenelse{\isundefined{\proposition}}{\newtheorem{proposition}{Proposition}}{}
\ifthenelse{\isundefined{\theorem}}{\newtheorem{theorem}{Theorem}}{}
\ifthenelse{\isundefined{\lemma}}{\newtheorem{lemma}{Lemma}}{}
\ifthenelse{\isundefined{\corollary}}{\newtheorem{corollary}{Corollary}}{}
\ifthenelse{\isundefined{\alg}}{\newtheorem{alg}{Algorithm}}{}
\ifthenelse{\isundefined{\example}}{\newtheorem{example}{Example}}{}
\newcommand\cv{\ensuremath{\to}} % Convergence
\newcommand\cvL{\ensuremath{\xrightarrow{\mathcal{L}}}} % Convergence in law
\newcommand\cvd{\ensuremath{\xrightarrow{d}}} % Convergence in distribution
\newcommand\cvP{\ensuremath{\xrightarrow{P}}} % Convergence in probability
\newcommand\cvas{\ensuremath{\xrightarrow{a.s.}}} % Convergence almost surely
\newcommand\eqdistrib{\ensuremath{\stackrel{d}{=}}} % Equal in distribution
\newcommand{\E}{\ensuremath{\mathbb{E}}} % Expectation
\newcommand\KL[2]{\ensuremath{\text{KL}\left( #1 \| #2 \right)}} % KL-divergence


\renewcommand{\lstlistingname}{} % Removes the "Listing" label
\captionsetup[lstlisting]{labelformat=empty} % Removes the numbering
\lstset{
    basicstyle=\ttfamily\small,    % Use monospaced font and small size
    backgroundcolor=\color{gray!10}, % Light gray background
    frame=single,                  % Add a box around the code
    framerule=1pt,                 % Thickness of the box frame
    rulecolor=\color{black},       % Color of the box frame
    keywordstyle=\bfseries\color{blue}, % Style for keywords (optional)
    captionpos=b,                  % Caption position (b for bottom, t for top)
    breaklines=true,               % Automatically break long lines
    breakindent=0pt,
    numbers=none,                  % Line numbers on the left
    xleftmargin=1em,               % Left margin for the box
    xrightmargin=1em,              % Right margin for the box
}


\title{Mitigating Bias in RAG: Controlling the Embedder}


% \usepackage{draftwatermark}
% \SetWatermarkText{Under Submission at ARR}
% \SetWatermarkScale{.25} % scale of the watermark
% \SetWatermarkAngle{45} % angle of the watermark

\newcommand{\aspace}{\hspace{2em}}
\newcommand{\cmuMLD}{$^\heartsuit$}
\newcommand{\cmuLTI}{$^\clubsuit$}

\author{
Taeyoun Kim\cmuMLD \aspace
Jacob Mitchell Springer\cmuMLD \\ \textbf{Aditi Raghunathan}\cmuMLD \aspace \textbf{Maarten Sap}\cmuLTI\\
% \vspace{4pt}
\small{\cmuMLD Machine Learning Department, Carnegie Mellon University \; \cmuLTI Language Technologies Institute, Carnegie Mellon University}\\
\small{\texttt{\{taeyoun3, jspringe, raditi, msap2\}@cs.cmu.edu}}
}


\begin{document}
\maketitle
\begin{abstract}
In retrieval augmented generation (RAG) systems, each individual component---the LLM, embedder, and corpus---could introduce biases in the form of skews towards outputting certain perspectives or identities. In this work, we study the conflict between biases of each component and their relationship to the overall bias of the RAG system, which we call \emph{bias conflict}. Examining both gender and political biases as case studies, we show that bias conflict can be characterized through a linear relationship among components despite its complexity in 6 different LLMs. Through comprehensive fine-tuning experiments creating 120 differently biased embedders, we demonstrate how to control bias while maintaining utility and reveal the importance of \emph{reverse-biasing} the embedder to mitigate bias in the overall system. Additionally, we find that LLMs and tasks exhibit varying \emph{sensitivities} to the embedder bias, a crucial factor to consider for debiasing. Our results underscore that a fair RAG system can be better achieved by carefully controlling the bias of the embedder rather than increasing its fairness.
\end{abstract}

\section{Introduction}
\label{sec:introduction}
The business processes of organizations are experiencing ever-increasing complexity due to the large amount of data, high number of users, and high-tech devices involved \cite{martin2021pmopportunitieschallenges, beerepoot2023biggestbpmproblems}. This complexity may cause business processes to deviate from normal control flow due to unforeseen and disruptive anomalies \cite{adams2023proceddsriftdetection}. These control-flow anomalies manifest as unknown, skipped, and wrongly-ordered activities in the traces of event logs monitored from the execution of business processes \cite{ko2023adsystematicreview}. For the sake of clarity, let us consider an illustrative example of such anomalies. Figure \ref{FP_ANOMALIES} shows a so-called event log footprint, which captures the control flow relations of four activities of a hypothetical event log. In particular, this footprint captures the control-flow relations between activities \texttt{a}, \texttt{b}, \texttt{c} and \texttt{d}. These are the causal ($\rightarrow$) relation, concurrent ($\parallel$) relation, and other ($\#$) relations such as exclusivity or non-local dependency \cite{aalst2022pmhandbook}. In addition, on the right are six traces, of which five exhibit skipped, wrongly-ordered and unknown control-flow anomalies. For example, $\langle$\texttt{a b d}$\rangle$ has a skipped activity, which is \texttt{c}. Because of this skipped activity, the control-flow relation \texttt{b}$\,\#\,$\texttt{d} is violated, since \texttt{d} directly follows \texttt{b} in the anomalous trace.
\begin{figure}[!t]
\centering
\includegraphics[width=0.9\columnwidth]{images/FP_ANOMALIES.png}
\caption{An example event log footprint with six traces, of which five exhibit control-flow anomalies.}
\label{FP_ANOMALIES}
\end{figure}

\subsection{Control-flow anomaly detection}
Control-flow anomaly detection techniques aim to characterize the normal control flow from event logs and verify whether these deviations occur in new event logs \cite{ko2023adsystematicreview}. To develop control-flow anomaly detection techniques, \revision{process mining} has seen widespread adoption owing to process discovery and \revision{conformance checking}. On the one hand, process discovery is a set of algorithms that encode control-flow relations as a set of model elements and constraints according to a given modeling formalism \cite{aalst2022pmhandbook}; hereafter, we refer to the Petri net, a widespread modeling formalism. On the other hand, \revision{conformance checking} is an explainable set of algorithms that allows linking any deviations with the reference Petri net and providing the fitness measure, namely a measure of how much the Petri net fits the new event log \cite{aalst2022pmhandbook}. Many control-flow anomaly detection techniques based on \revision{conformance checking} (hereafter, \revision{conformance checking}-based techniques) use the fitness measure to determine whether an event log is anomalous \cite{bezerra2009pmad, bezerra2013adlogspais, myers2018icsadpm, pecchia2020applicationfailuresanalysispm}. 

The scientific literature also includes many \revision{conformance checking}-independent techniques for control-flow anomaly detection that combine specific types of trace encodings with machine/deep learning \cite{ko2023adsystematicreview, tavares2023pmtraceencoding}. Whereas these techniques are very effective, their explainability is challenging due to both the type of trace encoding employed and the machine/deep learning model used \cite{rawal2022trustworthyaiadvances,li2023explainablead}. Hence, in the following, we focus on the shortcomings of \revision{conformance checking}-based techniques to investigate whether it is possible to support the development of competitive control-flow anomaly detection techniques while maintaining the explainable nature of \revision{conformance checking}.
\begin{figure}[!t]
\centering
\includegraphics[width=\columnwidth]{images/HIGH_LEVEL_VIEW.png}
\caption{A high-level view of the proposed framework for combining \revision{process mining}-based feature extraction with dimensionality reduction for control-flow anomaly detection.}
\label{HIGH_LEVEL_VIEW}
\end{figure}

\subsection{Shortcomings of \revision{conformance checking}-based techniques}
Unfortunately, the detection effectiveness of \revision{conformance checking}-based techniques is affected by noisy data and low-quality Petri nets, which may be due to human errors in the modeling process or representational bias of process discovery algorithms \cite{bezerra2013adlogspais, pecchia2020applicationfailuresanalysispm, aalst2016pm}. Specifically, on the one hand, noisy data may introduce infrequent and deceptive control-flow relations that may result in inconsistent fitness measures, whereas, on the other hand, checking event logs against a low-quality Petri net could lead to an unreliable distribution of fitness measures. Nonetheless, such Petri nets can still be used as references to obtain insightful information for \revision{process mining}-based feature extraction, supporting the development of competitive and explainable \revision{conformance checking}-based techniques for control-flow anomaly detection despite the problems above. For example, a few works outline that token-based \revision{conformance checking} can be used for \revision{process mining}-based feature extraction to build tabular data and develop effective \revision{conformance checking}-based techniques for control-flow anomaly detection \cite{singh2022lapmsh, debenedictis2023dtadiiot}. However, to the best of our knowledge, the scientific literature lacks a structured proposal for \revision{process mining}-based feature extraction using the state-of-the-art \revision{conformance checking} variant, namely alignment-based \revision{conformance checking}.

\subsection{Contributions}
We propose a novel \revision{process mining}-based feature extraction approach with alignment-based \revision{conformance checking}. This variant aligns the deviating control flow with a reference Petri net; the resulting alignment can be inspected to extract additional statistics such as the number of times a given activity caused mismatches \cite{aalst2022pmhandbook}. We integrate this approach into a flexible and explainable framework for developing techniques for control-flow anomaly detection. The framework combines \revision{process mining}-based feature extraction and dimensionality reduction to handle high-dimensional feature sets, achieve detection effectiveness, and support explainability. Notably, in addition to our proposed \revision{process mining}-based feature extraction approach, the framework allows employing other approaches, enabling a fair comparison of multiple \revision{conformance checking}-based and \revision{conformance checking}-independent techniques for control-flow anomaly detection. Figure \ref{HIGH_LEVEL_VIEW} shows a high-level view of the framework. Business processes are monitored, and event logs obtained from the database of information systems. Subsequently, \revision{process mining}-based feature extraction is applied to these event logs and tabular data input to dimensionality reduction to identify control-flow anomalies. We apply several \revision{conformance checking}-based and \revision{conformance checking}-independent framework techniques to publicly available datasets, simulated data of a case study from railways, and real-world data of a case study from healthcare. We show that the framework techniques implementing our approach outperform the baseline \revision{conformance checking}-based techniques while maintaining the explainable nature of \revision{conformance checking}.

In summary, the contributions of this paper are as follows.
\begin{itemize}
    \item{
        A novel \revision{process mining}-based feature extraction approach to support the development of competitive and explainable \revision{conformance checking}-based techniques for control-flow anomaly detection.
    }
    \item{
        A flexible and explainable framework for developing techniques for control-flow anomaly detection using \revision{process mining}-based feature extraction and dimensionality reduction.
    }
    \item{
        Application to synthetic and real-world datasets of several \revision{conformance checking}-based and \revision{conformance checking}-independent framework techniques, evaluating their detection effectiveness and explainability.
    }
\end{itemize}

The rest of the paper is organized as follows.
\begin{itemize}
    \item Section \ref{sec:related_work} reviews the existing techniques for control-flow anomaly detection, categorizing them into \revision{conformance checking}-based and \revision{conformance checking}-independent techniques.
    \item Section \ref{sec:abccfe} provides the preliminaries of \revision{process mining} to establish the notation used throughout the paper, and delves into the details of the proposed \revision{process mining}-based feature extraction approach with alignment-based \revision{conformance checking}.
    \item Section \ref{sec:framework} describes the framework for developing \revision{conformance checking}-based and \revision{conformance checking}-independent techniques for control-flow anomaly detection that combine \revision{process mining}-based feature extraction and dimensionality reduction.
    \item Section \ref{sec:evaluation} presents the experiments conducted with multiple framework and baseline techniques using data from publicly available datasets and case studies.
    \item Section \ref{sec:conclusions} draws the conclusions and presents future work.
\end{itemize}
\ifpdf
\DeclareGraphicsExtensions{.eps,.pdf,.png,.jpg}
\else
\DeclareGraphicsExtensions{.eps}
\fi

% Prevent itemized lists from running into the left margin inside theorems and proofs
\setlist[enumerate]{leftmargin=.5in}
\setlist[itemize]{leftmargin=.5in}

% Add a serial/Oxford comma by default.
\newcommand{\creflastconjunction}{, and~}

% Setting for algorithms
\Crefname{ALC@unique}{Line}{Lines}

% Tikz libraries
\usetikzlibrary{positioning}
\usetikzlibrary{calc}
\usetikzlibrary{shapes.geometric}
\usetikzlibrary{angles}
\usetikzlibrary{decorations.pathreplacing}
\usetikzlibrary{calligraphy}

\begin{table*}[!t]
\centering
\scalebox{0.6}{
    \begin{tabular}{l cccc}
      \toprule
      & & \multicolumn{4}{c}{\textbf{Intellipro Dataset}} \\
       & \multicolumn{2}{c}{Rank Resume} & \multicolumn{2}{c}{Rank Job} 
    \\
      \cmidrule(lr){2-3} \cmidrule(lr){4-5} 
      \textbf{Model} 
     & nDCG@100 & nDCG@10   \\
      \midrule
      \multirow{1}{*}{V2(w/o gender info)}
    
       &48.17
       &64.91
       \\
      \cmidrule(lr){2-5}
      \multirow{1}{*}{V2(w/ gender info)}
    
       & 49.83
    &67.45

       \\
      \cmidrule(lr){2-5}
      \multirow{1}{*}{V2(w/o gender info)+RUM}
    
         & 50.57
      & 72.38

       \\
      \cmidrule(lr){2-5}
      \multirow{1}{*}{V2(w/ gender info)+RUM}
    
       & 49.76
    &70.16

       \\

   
       
      \bottomrule

    \end{tabular}
  }
\caption{CONFIT bias analysis.}
\label{tbl:bias}
\end{table*}
\section{Results: Debiasing RAG} 
\label{sec:debiasing}

We compare the performance of agents trained on data from the InSTA pipeline to agents trained on human demonstrations from WebLINX \citep{WebLINX} and Mind2Web \citep{Mind2Web}, two recent and popular benchmarks for web navigation. Recent works that mix synthetic data with real data control the real data sampling probability in the batch $p_{\text{real}}$ independently from data size \citep{DAFusion}. We employ $p_{\text{real}} = 0.5$ in few-shot experiments and $p_{\text{real}} = 0.8$ otherwise. Shown in Figure~\ref{fig:data-statistics}, our data have a wide spread in performance, so we apply several filtering rules to select high-quality training data. First, we require the evaluator to return \texttt{conf} = 1 that the task was successfully completed, and that the agent was on the right track (this selects data where the actions are reliable, and directly caused the task to be solved). Second, we filter data where the trajectory contains at least three actions. Third, we remove data where the agent encountered any type of server error, was presented with a captcha, or was blocked at any point. These steps produce $7,463$ high-quality demonstrations in which agents successfully completed tasks on diverse websites. We sample 500 demonstrations uniformly at random from this pool to create a diverse test set, and employ the remaining $6,963$ demonstrations to train agents on a mix of real and synthetic data.

\subsection{Improving Data-Efficiency}
\label{sec:few-shot}

\begin{wrapfigure}{r}{0.48\textwidth}
    \centering
    \vspace{-0.8cm}
    \includegraphics[width=\linewidth]{assets/few_shot_results_weblinx_mind2web.pdf}
    \vspace{-0.3cm}
    \caption{\small \textbf{Data from InSTA improves efficiency.} Language model agents trained on mixtures of our data and human demonstrations scale faster than agents trained on human data. In a setting with 32 human actions, adding our data improves \textit{Step Accuracy} by +89.5\% relative to human data for Mind2Web, and +122.1\% relative to human data for WebLINX.}
    \vspace{-0.2cm}
    \label{fig:few-shot-results}
\end{wrapfigure}

In a data-limited setting derived from WebLINX \citep{WebLINX} and Mind2Web \citep{Mind2Web}, agents trained on our data \textit{scale faster with increasing data size} than human data alone. Without requiring laborious human annotations, the data produced by our pipeline leads to improvements on Mind2Web that range from +89.5\% in \textit{Step Accuracy} (the rate at which the correct element is selected and the correct action is performed on that element) with 32 human actions, to +77.5\% with 64 human actions, +13.8\% with 128 human actions, and +12.1\% with 256 human actions. For WebLINX, our data improves by +122.1\% with 32 human actions, and +24.6\% with 64 human actions, and +6.2\% for 128 human actions. Adding our data is comparable in performance gained to doubling the amount of human data from 32 to 64 actions. Performance on the original test sets for Mind2Web and WebLINX appears to saturate as the amount of human data increases, but these benchmark only test agent capabilities for a limited set of 150 popular sites.

\subsection{Improving Generalization} 
\label{sec:generalization}

\begin{wrapfigure}{r}{0.48\textwidth}
    \centering
    \vspace{-1.0cm}
    \includegraphics[width=\linewidth]{assets/diverse_results_weblinx_mind2web.pdf}
    \vspace{-0.3cm}
    \caption{\small \textbf{Our data improves generalization.} We train agents with all human data from the WebLINX and Mind2Web training sets, and resulting agents struggle to generalize to more diverse test data. Adding our data improves generalization by +149.0\% for WebLINX, and +156.3\% for Mind2Web.}
    \vspace{-0.3cm}
    \label{fig:generalization-results}
\end{wrapfigure}

To understand how agents trained on data from our pipeline generalize to diverse real-world sites, we construct a more diverse test set than Mind2Web and WebLINX using 500 held-out demonstrations produced by our pipeline. Shown in Figure~\ref{fig:generalization-results}, we train agents using all human data in the training sets for WebLINX and Mind2Web, and compare the performance with agents trained on 80\% human data, and 20\% data from our pipeline. Agents trained with our data achieve comparable performance to agents trained purely on human data on the official test sets for the WebLINX and Mind2Web benchmarks, suggesting that when enough human data are available, synthetic data may not be necessary. However, when evaluated in a more diverse test set that includes 500 sites not considered by existing benchmarks, agents trained purely on existing human data struggle to generalize. Training with our data improves generalization to these sites by +149.0\% for WebLINX agents, and +156.3\% for Mind2Web agents, with the largest gains in generalization \textit{Step Accuracy} appearing for harder tasks.

Given the complexity of bias conflict in a RAG system, is it feasible to mitigate bias in the entire RAG system? In this section, we try to control the embedder to mitigate bias. In \refsec{fine-tune} we first fine-tune several embedders to span a wide bias range. Then in \refsec{emb-rag}, we construct a RAG system with these embedders while keeping the LLM and corpus fixed to understand the relationship between the embedder bias and RAG bias (\refeqn{bias}). 

\subsection{Controlling the Embedder}
\label{sec:fine-tune}
We increasingly fine-tune the base embedder to retrieve more documents related to females and conservative views to mitigate its bias towards males and liberal views. We train the embedder through a contrastive loss similar to SimCSE \cite{gao2021simcse}. On the train splits of \genderData and \politicalData, we collect the positive documents to be related to females and conservative views and negative documents to be about males and liberal views from the training corpora. Training details are in \refapp{training}. 

To prevent the embedder from losing its original performance after fine-tuning, we implement two different fine-tuning methods.

\begin{enumerate}
    \item \textbf{PEFT} We fine-tune only the last few linear layers of the embedder. This helps the embedder retain its original low-level features and prevents overfitting. We vary the number of layers for each training run among $\ell = \{1, 2, 3, 4\}$.
    \item \textbf{WiSE-FT} After full fine-tuning, we produce a merged model as a convex combination of each parameter of the fine-tuned and base embedder. \citet{wortsman2022robust} show that this increases robustness while maintaining original performance. We choose the interpolation coefficient among $\lambda=\{0.1, 0.3, 0.5, 0.7, 0.9\}$ to produce 
    \[
    \theta^{merge} = (1-\lambda)\cdot\theta^{base} + \lambda\cdot\theta^{fine-tune}
    \]
    where $\theta^{merge}, \theta^{base}, \theta^{fine-tune}$ are the parameters of the merged embedder, base embedder, and fine-tuned embedder.
\end{enumerate}

For both methods, we sweep over learning rates of $\{3\times10^{-5}, 1\times10^{-5}\}$ and training epochs of $\{5, 10, 15\}$. Including normal full fine-tuning, the combination of learning rate, epoch, and training method results in 60 trained embedders per task. We use AdamW \citep{loshchilov2019decoupledweightdecayregularization} with a weight decay of $0.01$ and fix a seed to make training deterministic.

\paragraph{Fine-tuning Results}
\reffig{frontier} shows the bias and validation-task accuracy of the fine-tuned embedders. The bias is measured on a validation corpus and the accuracy is measured on RAG Mini-Wikipedia \cite{smith2008question} which is a small RAG QA benchmark (please refer to the details of validation in \refapp{validation}).

First, we find that light fine-tuning with PEFT or WiSE-FT is sufficient to reverse the embedder bias. On \genderData, the embedder bias started from $-0.52$ and increased to $1.00$. Second, there is a regime where the embedder bias is reversed but the accuracy drop on RAG Mini-Wikipedia is minimal. This results in an outward-pointing Pareto frontier which makes it possible to control the bias of embedders across a wide range while minimizing degeneration or loss in utility. 

\subsection{Embedder \& RAG}
\label{sec:emb-rag}
With our family of embedders controlled to have varying levels of bias, we explore how the embedder bias ($E_b$) affects the RAG bias ($R_b$), and whether there exists an embedder that can mitigate RAG bias to 0 ($R_b=0$).

Among the fine-tuned embedders, we take 20 that are evenly spread out across the full bias range. We compose a RAG system by connecting the embedders with the 6 LLMs and test corpus (NQ for \genderData and PolNLI for \politicalData) and measure the bias of the RAG system for each embedder on the test queries. We define the \emph{optimal embedder} as the embedder that results in $R_b=0$ and call the bias of this embedder the \emph{optimal bias}.

\paragraph{Embedder \& RAG Bias Results}
We show the results for Llama 8/405B, Gemma 27B, and Mistral in \reffig{training} (the full set of 6 LLMs are in \refapp{six-llms}). We see that the linear relationship in \refeqn{bias} holds across all LLMs. As the embedder bias increases, the RAG bias scales linearly. 

\begin{table*}[h] 
\centering
\begin{small}
\begin{sc}
\begin{tabular}{c||cccccc}
\toprule
\rowcolor{lightblue}
& \textbf{L 8B}& \textbf{L 70B}& \textbf{L 405B}& \textbf{G 9B} & \textbf{G 27B} &\textbf{M}  \\
\midrule
\genderData&0.38&0.34& 0.32 & 0.35&0.38 &  0.38 \\
\politicalData&0.40&0.11& 0.04 &0.43 &-0.22 & 0.53  \\
\bottomrule
\end{tabular}
% \begin{tabular}{c||ccc}
% \toprule
% \multirow{4}{*}{\genderData} 
%     & \multicolumn{1}{>{\columncolor{lightblue}}c}{\textbf{L 8B}} & \multicolumn{1}{>{\columncolor{lightblue}}c}{\textbf{L 70B}} & \multicolumn{1}{>{\columncolor{lightblue}}c}{\textbf{L 405B}} \\ \cline{2-4}
%     & 0.38           & 0.34          & 0.32          \\ \cline{2-4}
%     & \multicolumn{1}{>{\columncolor{lightblue}}c}{\textbf{G 9B}} & \multicolumn{1}{>{\columncolor{lightblue}}c}{\textbf{G 27B}} & \multicolumn{1}{>{\columncolor{lightblue}}c}{\textbf{M}} \\ \cline{2-4}
%     & 0.35           & 0.38          & 0.38          \\
% \midrule
% \multirow{4}{*}{\politicalData} 
%     & \multicolumn{1}{>{\columncolor{lightblue}}c}{\textbf{L 8B}} & \multicolumn{1}{>{\columncolor{lightblue}}c}{\textbf{L 70B}} & \multicolumn{1}{>{\columncolor{lightblue}}c}{\textbf{L 405B}} \\ \cline{2-4}
%     & 0.40           & 0.11          & 0.04          \\ \cline{2-4}
%     & \multicolumn{1}{>{\columncolor{lightblue}}c}{\textbf{G 9B}} & \multicolumn{1}{>{\columncolor{lightblue}}c}{\textbf{G 27B}} & \multicolumn{1}{>{\columncolor{lightblue}}c}{\textbf{M}} \\ \cline{2-4}
%     & 0.43           & -0.22         & 0.53          \\
% \bottomrule
% \end{tabular}
\end{sc}
\end{small}
\caption{\textbf{Optimal Embedder Bias.} The optimal bias ($E_b$-intercept) of the embedder that results in a debiased RAG system ($R_b=0$). L 8B: Llama 8B, L 70B: Llama 70B, L 405B: Llama 405B, G 9B: Gemma 9B, G 27B: Gemma 27B, M: Mistral}
\label{tab:optimal-full}
\end{table*}


To ensure we audit the privacy of synthetic text data in a realistic setup, the synthetic data needs to bear high utility. We measure the synthetic data utility by comparing the downstream classification performance of RoBERTa-base~\citep{DBLP:journals/corr/abs-1907-11692} when fine-tuned exclusively on real or synthetic data. We fine-tune models for binary (SST-2) and multi-class classification (AG News) for 1 epoch on the same number of real or synthetic data records using a batch size of $16$ and learning rate $\eta = \num{1e-5}$. We report the macro-averaged AUC score and accuracy on a held-out test dataset of real records. 

Table~\ref{tab:utility_no_canaries} summarizes the results for synthetic data generated based on original data which does not contain any canaries. While we do see a slight drop in downstream performance when considering synthetic data instead of the original data, AUC and accuracy remain high for both tasks. 

\begin{table}[ht]
    \centering
    \begin{tabular}{ccrr}
    \toprule
        & \multirow{2}{*}{Fine-tuning data} & \multicolumn{2}{c}{Classification} \\
        \cmidrule(lr){3-4}
        Dataset &  & AUC & Accuracy \\
        \midrule 
        \multirow{2}{*}{\parbox{2cm}{\centering SST-2}} & Real & $0.984$ & \SI{92.3}{\percent} \\ 
         & Synthetic & $0.968$ & \SI{91.5}{\percent} \\
         \midrule
        \multirow{2}{*}{\parbox{2cm}{\centering AG News}} & Real & $0.992$ & \SI{94.4}{\percent} \\ 
         & Synthetic & $0.978$ & \SI{90.0}{\percent} \\ 
        \bottomrule
    \end{tabular}
    \caption{Utility of synthetic data generated from real data \emph{without} canaries. We compare the performance of text classifiers trained on real or synthetic data---both evaluated on real, held-out test data.}
    \label{tab:utility_no_canaries}
\end{table}

We further measure the synthetic data utility when the original data contains standard canaries (see Sec.~\ref{sec:baseline_results}). Specifically, we consider synthetic data generated from a target model trained on data containing \num{500} canaries repeated $n_\textrm{rep} = 12$ times, so \num{6000} data records. When inserting canaries with an artificial label, we remove all synthetic data associated with labels not present originally when fine-tuning the RoBERTa-base model. 

\begin{table}[h]
    \centering
    \begin{tabular}{ccc@{\hskip 15pt}rr}
    \toprule
        & \multicolumn{2}{c}{Canary injection} & \multicolumn{2}{c}{Classification}\\
        \cmidrule(lr){2-3} \cmidrule(lr){4-5}
        Dataset & Source & Label & AUC & Accuracy \\
        \midrule
        \multirow{3}{*}{\parbox{1cm}{\centering SST-2}} & \multicolumn{2}{l}{In-distribution} & $0.972$ & \SI{91.6}{\percent} \\ 
        \cmidrule{2-5}
         & \multirow{2}{*}{\parbox{1.8cm}{Synthetic}} & Natural & $0.959$ & \SI{89.3}{\percent} \\ 
         & & Artificial & $0.962$ & \SI{89.9}{\percent} \\ 
        \midrule
        \multirow{3}{*}{\parbox{2cm}{\centering AG News}} & \multicolumn{2}{l}{In-distribution} & $0.978$ & \SI{89.8}{\percent}\\ 
        \cmidrule{2-5} 
         & \multirow{2}{*}{\parbox{1.8cm}{Synthetic}} & Natural & $0.977$ & \SI{88.6}{\percent} \\ 
         & & Artificial & $0.980$ & \SI{90.1}{\percent} \\         
         \bottomrule
    \end{tabular}
    \caption{Utility of synthetic data generated from real data \emph{with} canaries ($n_\textrm{rep}=12$). We compare the performance of text classifiers trained on real or synthetic data---both evaluated on real, held-out test data.}
    \label{tab:utility_canaries}
\end{table}

Table~\ref{tab:utility_canaries} summarizes the results. Across all canary injection methods, we find limited impact of canaries on the downstream utility of synthetic data. While the difference is minor, the natural canary labels lead to the largest utility degradation. This makes sense, as the high perplexity synthetic sequences likely distort the distribution of synthetic text associated with a certain real label. In contrast, in-distribution canaries can be seen as up-sampling certain real data points during fine-tuning, while canaries with artificial labels merely reduce the capacity of the model to learn from real data and do not interfere with this process as much as canaries with natural labels do.

\begin{figure*}[t]
    \textbf{\hspace{1.5cm}\genderData\hspace{5.7cm}\politicalData}\\
    \centering
    \subfloat[Full range\label{fig:corpus-bias-figure}]{\includegraphics[width=0.25\textwidth]{images/corpus-bias-figures-llama405-nq.png}}\hfill
    \subfloat[Limited range\label{fig:corpus-bias-figure-lim}]{\includegraphics[width=0.25\textwidth]{images/corpus-bias-figures-llama405-nq-lim.png}}\hfill
    \subfloat[Full range\label{fig:corpus-bias-political}]{\includegraphics[width=0.25\textwidth]{images/corpus-bias-political-llama405-Pol_NLI.png}} \hfill
    \subfloat[Limited range\label{fig:corpus-bias-political-lim}]{\includegraphics[width=0.25\textwidth]{images/corpus-bias-political-llama405-Pol_NLI-lim.png}}\\

    \caption{\textbf{Corpus Bias.} RAG bias ($R_b$) when the corpus bias ($C_b$) changes for three different embedders. The base embedder is \texttt{GTE-base}, the optimal embedder is the embedder that results in $R_b \approx 0$ with a neutral corpus ($C_b$), and the degenerate embedder is a heavily reverse biased embedder. The RAG bias scales linearly with the corpus bias for the \textcolor{darkgray}{base} and \textcolor{blue}{optimal} embedder while the linearity breaks as the embedder becomes more \textcolor{red}{degenerate}.}
    \label{fig:corpus-bias}
\end{figure*}

We make four observations in \reffig{training}. First, the bias of the optimal embedder is not neutral but mostly reverse biased. \reftab{optimal-full} shows the optimal bias being positive, while it was initially negative in \reftab{base-comp-bias}. This means that reverse biasing a small embedder of 109M parameters can overcome the bias of a larger language model of 405B parameters given high sensitivity ($s\uparrow$). 

Second, all LLMs are highly sensitive to gender bias and less sensitive to political bias. While LLMs are already RLHF fine-tuned to prevent traditional notions of gender bias which count pronouns and occupational bias \citep{lu2020gender,zmigrod2019counterfactual}, we see high sensitivity to \genderData because they are not fine-tuned for figure names. 

Third, the sensitivity for \politicalData is low and noticeably differs per LLM, resulting in different optimal embedders. For example, Llama 405B is easier to debias than Llama 8B or Mistral ($0.04 < 0.40,0.53$) because of its high sensitivity. We posit this is because larger models are more compliant with following instructions, including contextual information. Gemma models are the least sensitive, being consistent with prior work showing that Gemma \citep{trhlik2024quantifyinggenerativemediabias} mainly maintains a centric-view while slightly left-leaning.

Fourth, an LLM that is strongly biased ($|L_b|\uparrow$) does not necessarily mean it has lower sensitivity ($s\downarrow$). It is intuitive to think that a strongly biased LLM creates stronger bias conflict, making it less sensitive to bias from the embedder. However, we observe that Mistral has a very strong political bias ($L_b=-0.81$) but higher sensitivity than Gemma. Thus, it is important to assess $s$ independently of $L_b$.

These findings suggest that while there is a universal linear trend, the sensitivity differs per LLM and bias. \refapp{more-models} even shows the case where debiasing is not possible due to extremely low sensitivity ($s\downarrow$) and strong LLM bias ($|L_b|\uparrow$). It is important to carefully consider the sensitivity when debiasing RAG through the embedder. We show qualitative examples of retrieved documents and LLM responses in \refapp{examples}.

\paragraph{Utility and Robustness}
Although we assessed the utility of the full RAG pipeline in \reffig{frontier}, we also measure the retrieval performance of each optimal embedder on the BEIR benchmark \citep{thakur2021beir} with details mentioned in \refapp{beir}. \reftab{utility} shows that the utility (NDCG@1) of the optimal embedders drops minimally compared to the base embedder. 

We also try controlling the embedder bias through projections and sampling in \refapp{proj-samp} but find that fine-tuning is the most effective at maintaining utility. Additionally, we evaluate on a different embedder \citep[\texttt{E5-base-v2};][]{wang2022text} in \refapp{e5} and change our test corpora to out-of-distribution corpora (HotpotQA \citep{yang2018hotpotqa} and NQ \citep{burnham2024politicaldebateefficientzeroshot}) in \refapp{ood} to find that the trends resemble, suggesting that linearity hold regardless of the retrieval method or corpus.


\section{Corpus\label{sec:corpus}}


\paragraph{Overview} Our corpus consists of 141 literature reviews written in English by 51 L2 graduate students, with an average word count of 1321 (930 excluding references). The reviews cover five broad topics from the humanities and social sciences, chosen to minimize the need for specialized disciplinary knowledge: (1) the social consequences of legalized cannabis, (2) the Canadian linguistic landscape, (3) online learning, (4) lessons from the COVID-19 pandemic, and (5) pacifism. Essays on topics 1, 3, and 5 were written individually, while those on topics 2 and 4 were completed collaboratively by 2-4 authors.


The corpus is a result of a large research project conducted at the University of Saskatchewan in 2021 with an aim to examine the developmental trajectory of literature review writing skills among L2 graduate students. The project involved three rounds of a 5-unit online tutorial series conducted over the course of 2021, with each round lasting 13 weeks (see Appendix~\ref{app:corpus} for details). Participation was voluntary, with 31 participants completing all five writing tasks across all rounds, and 20 further students completing at least one task before withdrawing. 


\paragraph{Our Previous Studies} The corpus was used in our previous studies \citep{li2023assessment, li2023developing, makarova2024can}, but has never been made public. These three studies all use only a subset of the corpus, namely essays written individually or those based on topics 1, 3, and 5.   

More concretely, \citet{li2023assessment, li2023developing} focus on the individual writing skill development, but without examining the feedback comments provided in the corpus. In other words, these two studies belong to the field of English for Academic Purposes, but has less relevance to AWE. 

While \citet{makarova2024can} investigate whether ChatGPT can assess L2 academic English writing, they do not compare human- and ChatGPT-generated scores and comments on the basis of each assessment criterion. Rather, they simply compare ChatGPT-generated scores and comments with average scores and concatenated comments produced by multiple human assessors for all criteria, which is not only less nuanced but also over-simplistic. Moreover, their analysis of feedback comments utilizes surface-level linguistic features such as word counts, type-token ratio, comment length, and the experiments do not consider possible user-LLM interaction modes nor prompt variations, which we do in Sections~\ref{sec:experiments} and ~\ref{sec:furtherAnalyses}, respectively. 

In short, this study not only presents a more comprehensive and thorough evaluation of LLMs using the full corpus, instead of its subset, but also employs a different set of evaluation methodologies. As a result, we identify no substantial overlap between this study and our previous studies.



\paragraph{Essay Authors} The corpus authors comprise a diverse group of L2 learners, representing a wide range of first languages and enrolled in graduate programs across various disciplines at multiple Canadian universities. Their English proficiency ranged from upper-intermediate to advanced, with an average score equivalent to IELTS Band 7 based on conversions from various standardized English language tests. Scores varied from IELTS 6.5 to 8.5, with a standard deviation of 0.55.


\paragraph{Human Assessments} Most essays in the corpus were assessed by three (94.3\%) or two (5.0\%) independent human experts. As illustrated in Fig.~\ref{fig:illustraion}, the assessments consist of scores on a 10-point scale and comments based on 9 analytic assessment criteria. While scores were required, comments were optional for the assessors. A total of six assessors with professional experience in English language teaching participated at different stages of the research project. Table~\ref{tab:feedbackRate} provides basic information about them.

The 9 assessment criteria (see Appendix~\ref{app:criteria} for details) include: (C1) material selection; (C2) material integration and citation; (C3) quality of key components; (C4) logic of structure; (C5) content and clarity of ideas; (C6) coherence (flow of ideas) ; (C7) cohesion (use of connectors); (C8) grammar and sentence structure; and (C9) academic vocabulary. 




\paragraph{Assessment Quality}  The 31 students who completed all writing tasks evaluated the quality of human assessments on a 4-point scale in an anonymous final project survey. Based on the 30 submitted survey responses, all participants agreed that the assessments were at least ``useful'' (rating = 3), with 24 participants (80\%) rating them as ``very useful'' (rating = 4). 

\paragraph{Data Contamination} Since the corpus was created prior to the release of ChatGPT and has never been made public, it contains no LLM-generated contents and is free from the risk of data contamination \citep{jacovi-etal-2023-stop, sainz-etal-2023-nlp}, making it an ideal resource for LLM evaluation.


\begin{table}[]
    \centering
    \small
    \begin{tabular}{lllllll}
    \toprule
    Code & Role & Rounds & Topics & \# Essays \\
    
    \midrule
    A & Graduate RA & 1 & 1-5  &  27 \\
    B & Graduate RA &  1-3 & 1-5 &   141  \\
    C & Faculty Member & 1-3 & 1, 2, 5 &  93 \\
    D & Faculty Member & 1 & 2 &  4 \\
    E & Faculty Member & 1-3 & 3, 4 &  43 \\
    F & Graduate RA & 2, 3 & 1-5 & 106  \\
    \bottomrule
    \end{tabular}
    
    \caption{Anonymized information for the six assessors (A–F). The columns ``Rounds'' and ``Topics'' indicate the specific rounds and writing topics they participated in. Assessors C and E never co-assessed together.}
    \label{tab:feedbackRate}
\end{table}




% \begin{table*}[]
%     \centering
%     \small
%     \begin{tabular}{lllllll}
%     \toprule
%     % Code & Role & Rounds & Topics & \#Samples & Avg Cmt Rate (\%) & Avg Cmt Len \\
%     & & & & & \multicolumn{2}{c}{Avg Comment} \\
%     Code & Role & Rounds & Topics & \#Samples & Rate (\%) & Length \\
    
%     \midrule
%     A & Graduate RA & 1 & 1-5  &  27 & 99.2 & 51{\tiny±62} \\
%     B & Graduate RA &  1-3 & 1-5 &   141 & 23.9 & 104{\tiny±85} \\
%     C & Faculty Member & 1-3 & 1, 2, 5 &  93 & 99.8 & 63{\tiny±86} \\
%     D & Faculty Member & 1 & 2 &  4 & 100.0 & 57{\tiny±38} \\
%     E & Faculty Member & 1-3 & 3, 4 &  43 & 63.8 & 53{\tiny±45} \\
%     F & Graduate RA & 2, 3 & 1-5 & 106 & 89.5 & 48{\tiny±59} \\
%     \bottomrule
%     \end{tabular}
    
%     \caption{Anonymized information for the six assessors (A–F). The columns ``Rounds'' and ``Topics'' indicate the specific rounds and writing topics each assessor participated in, with assessors C and E never overlapping. ``Avg Comment Rate'' represents the percentage of time a comment was provided for all assessment criteria. ``Avg Comment Length,'' calculated only for provided comments, is reported with the standard deviation following ``{\tiny ±}.''}
%     \label{tab:feedbackRate}
% \end{table*}



\section{Discussion of Assumptions}\label{sec:discussion}
In this paper, we have made several assumptions for the sake of clarity and simplicity. In this section, we discuss the rationale behind these assumptions, the extent to which these assumptions hold in practice, and the consequences for our protocol when these assumptions hold.

\subsection{Assumptions on the Demand}

There are two simplifying assumptions we make about the demand. First, we assume the demand at any time is relatively small compared to the channel capacities. Second, we take the demand to be constant over time. We elaborate upon both these points below.

\paragraph{Small demands} The assumption that demands are small relative to channel capacities is made precise in \eqref{eq:large_capacity_assumption}. This assumption simplifies two major aspects of our protocol. First, it largely removes congestion from consideration. In \eqref{eq:primal_problem}, there is no constraint ensuring that total flow in both directions stays below capacity--this is always met. Consequently, there is no Lagrange multiplier for congestion and no congestion pricing; only imbalance penalties apply. In contrast, protocols in \cite{sivaraman2020high, varma2021throughput, wang2024fence} include congestion fees due to explicit congestion constraints. Second, the bound \eqref{eq:large_capacity_assumption} ensures that as long as channels remain balanced, the network can always meet demand, no matter how the demand is routed. Since channels can rebalance when necessary, they never drop transactions. This allows prices and flows to adjust as per the equations in \eqref{eq:algorithm}, which makes it easier to prove the protocol's convergence guarantees. This also preserves the key property that a channel's price remains proportional to net money flow through it.

In practice, payment channel networks are used most often for micro-payments, for which on-chain transactions are prohibitively expensive; large transactions typically take place directly on the blockchain. For example, according to \cite{river2023lightning}, the average channel capacity is roughly $0.1$ BTC ($5,000$ BTC distributed over $50,000$ channels), while the average transaction amount is less than $0.0004$ BTC ($44.7k$ satoshis). Thus, the small demand assumption is not too unrealistic. Additionally, the occasional large transaction can be treated as a sequence of smaller transactions by breaking it into packets and executing each packet serially (as done by \cite{sivaraman2020high}).
Lastly, a good path discovery process that favors large capacity channels over small capacity ones can help ensure that the bound in \eqref{eq:large_capacity_assumption} holds.

\paragraph{Constant demands} 
In this work, we assume that any transacting pair of nodes have a steady transaction demand between them (see Section \ref{sec:transaction_requests}). Making this assumption is necessary to obtain the kind of guarantees that we have presented in this paper. Unless the demand is steady, it is unreasonable to expect that the flows converge to a steady value. Weaker assumptions on the demand lead to weaker guarantees. For example, with the more general setting of stochastic, but i.i.d. demand between any two nodes, \cite{varma2021throughput} shows that the channel queue lengths are bounded in expectation. If the demand can be arbitrary, then it is very hard to get any meaningful performance guarantees; \cite{wang2024fence} shows that even for a single bidirectional channel, the competitive ratio is infinite. Indeed, because a PCN is a decentralized system and decisions must be made based on local information alone, it is difficult for the network to find the optimal detailed balance flow at every time step with a time-varying demand.  With a steady demand, the network can discover the optimal flows in a reasonably short time, as our work shows.

We view the constant demand assumption as an approximation for a more general demand process that could be piece-wise constant, stochastic, or both (see simulations in Figure \ref{fig:five_nodes_variable_demand}).
We believe it should be possible to merge ideas from our work and \cite{varma2021throughput} to provide guarantees in a setting with random demands with arbitrary means. We leave this for future work. In addition, our work suggests that a reasonable method of handling stochastic demands is to queue the transaction requests \textit{at the source node} itself. This queuing action should be viewed in conjunction with flow-control. Indeed, a temporarily high unidirectional demand would raise prices for the sender, incentivizing the sender to stop sending the transactions. If the sender queues the transactions, they can send them later when prices drop. This form of queuing does not require any overhaul of the basic PCN infrastructure and is therefore simpler to implement than per-channel queues as suggested by \cite{sivaraman2020high} and \cite{varma2021throughput}.

\subsection{The Incentive of Channels}
The actions of the channels as prescribed by the DEBT control protocol can be summarized as follows. Channels adjust their prices in proportion to the net flow through them. They rebalance themselves whenever necessary and execute any transaction request that has been made of them. We discuss both these aspects below.

\paragraph{On Prices}
In this work, the exclusive role of channel prices is to ensure that the flows through each channel remains balanced. In practice, it would be important to include other components in a channel's price/fee as well: a congestion price  and an incentive price. The congestion price, as suggested by \cite{varma2021throughput}, would depend on the total flow of transactions through the channel, and would incentivize nodes to balance the load over different paths. The incentive price, which is commonly used in practice \cite{river2023lightning}, is necessary to provide channels with an incentive to serve as an intermediary for different channels. In practice, we expect both these components to be smaller than the imbalance price. Consequently, we expect the behavior of our protocol to be similar to our theoretical results even with these additional prices.

A key aspect of our protocol is that channel fees are allowed to be negative. Although the original Lightning network whitepaper \cite{poon2016bitcoin} suggests that negative channel prices may be a good solution to promote rebalancing, the idea of negative prices in not very popular in the literature. To our knowledge, the only prior work with this feature is \cite{varma2021throughput}. Indeed, in papers such as \cite{van2021merchant} and \cite{wang2024fence}, the price function is explicitly modified such that the channel price is never negative. The results of our paper show the benefits of negative prices. For one, in steady state, equal flows in both directions ensure that a channel doesn't loose any money (the other price components mentioned above ensure that the channel will only gain money). More importantly, negative prices are important to ensure that the protocol selectively stifles acyclic flows while allowing circulations to flow. Indeed, in the example of Section \ref{sec:flow_control_example}, the flows between nodes $A$ and $C$ are left on only because the large positive price over one channel is canceled by the corresponding negative price over the other channel, leading to a net zero price.

Lastly, observe that in the DEBT control protocol, the price charged by a channel does not depend on its capacity. This is a natural consequence of the price being the Lagrange multiplier for the net-zero flow constraint, which also does not depend on the channel capacity. In contrast, in many other works, the imbalance price is normalized by the channel capacity \cite{ren2018optimal, lin2020funds, wang2024fence}; this is shown to work well in practice. The rationale for such a price structure is explained well in \cite{wang2024fence}, where this fee is derived with the aim of always maintaining some balance (liquidity) at each end of every channel. This is a reasonable aim if a channel is to never rebalance itself; the experiments of the aforementioned papers are conducted in such a regime. In this work, however, we allow the channels to rebalance themselves a few times in order to settle on a detailed balance flow. This is because our focus is on the long-term steady state performance of the protocol. This difference in perspective also shows up in how the price depends on the channel imbalance. \cite{lin2020funds} and \cite{wang2024fence} advocate for strictly convex prices whereas this work and \cite{varma2021throughput} propose linear prices.

\paragraph{On Rebalancing} 
Recall that the DEBT control protocol ensures that the flows in the network converge to a detailed balance flow, which can be sustained perpetually without any rebalancing. However, during the transient phase (before convergence), channels may have to perform on-chain rebalancing a few times. Since rebalancing is an expensive operation, it is worthwhile discussing methods by which channels can reduce the extent of rebalancing. One option for the channels to reduce the extent of rebalancing is to increase their capacity; however, this comes at the cost of locking in more capital. Each channel can decide for itself the optimum amount of capital to lock in. Another option, which we discuss in Section \ref{sec:five_node}, is for channels to increase the rate $\gamma$ at which they adjust prices. 

Ultimately, whether or not it is beneficial for a channel to rebalance depends on the time-horizon under consideration. Our protocol is based on the assumption that the demand remains steady for a long period of time. If this is indeed the case, it would be worthwhile for a channel to rebalance itself as it can make up this cost through the incentive fees gained from the flow of transactions through it in steady state. If a channel chooses not to rebalance itself, however, there is a risk of being trapped in a deadlock, which is suboptimal for not only the nodes but also the channel.

\section{Conclusion}
This work presents DEBT control: a protocol for payment channel networks that uses source routing and flow control based on channel prices. The protocol is derived by posing a network utility maximization problem and analyzing its dual minimization. It is shown that under steady demands, the protocol guides the network to an optimal, sustainable point. Simulations show its robustness to demand variations. The work demonstrates that simple protocols with strong theoretical guarantees are possible for PCNs and we hope it inspires further theoretical research in this direction.
\section{Limitation}
The use of 3D-printed PLA for structural components improves improving ease of assembly and reduces weight and cost, yet it causes deformation under heavy load, which can diminish end-effector precision. Using metal, such as aluminum, would remedy this problem. Additionally, \robot relies on integrated joint relative encoders, requiring manual initialization in a fixed joint configuration each time the system is powered on. Using absolute joint encoders could significantly improve accuracy and ease of use, although it would increase the overall cost. 

%Reliance on commercially available actuators simplifies integration but imposes constraints on control frequency and customization, further limiting the potential for tailored performance improvements.

% The 6 DoF configuration provides sufficient mobility for most tasks; however, certain bimanual operations could benefit from an additional degree of freedom to handle complex joint constraints more effectively. Furthermore, the limited torque density of commercially available proprioceptive actuators restricts the payload and torque output, making the system less suitability for handling heavier loads or high-torque applications. 

The 6 DoF configuration of the arm provides sufficient mobility for single-arm manipulation tasks, yet it shows a limitation in certain bimanual manipulation problems. Specifically, when \robot holds onto a rigid object with both hands, each arm loses 1 DoF because the hands are fixed to the object during grasping. This leads to an underactuated kinematic chain which has a limited mobility in 3D space. We can achieve more mobility by letting the object slip inside the grippers, yet this renders the grasp less robust and simulation difficult. Therefore, we anticipate that designing a lightweight 3 DoF wrist in place of the current 2 DoF wrist allows a more diverse repertoire of manipulation in bimanual tasks.

Finally, the limited torque density of commercially available proprioceptive actuators restricts the performance. Currently, all of our actuators feature a 1:10 gear ratio, so \robot can handle up to 2.5 kg of payload. To handle a heavier object and manipulate it with higher torque, we expect the actuator to have 1:20$\sim$30 gear ratio, but it is difficult to find an off-the-shelf product that meets our requirements. Customizing the actuator to increase the torque density while minimizing the weight will enable \robot to move faster and handle more diverse objects.

%These constraints highlight opportunities for improvement in future iterations, including alternative materials for enhanced rigidity, custom actuator designs for higher control precision and torque density, the adoption of absolute joint encoders, and optimized configurations to balance dexterity and weight.


\section{Acknowledgements}


\bibliography{ref}

\clearpage
\appendix
\subsection{Lloyd-Max Algorithm}
\label{subsec:Lloyd-Max}
For a given quantization bitwidth $B$ and an operand $\bm{X}$, the Lloyd-Max algorithm finds $2^B$ quantization levels $\{\hat{x}_i\}_{i=1}^{2^B}$ such that quantizing $\bm{X}$ by rounding each scalar in $\bm{X}$ to the nearest quantization level minimizes the quantization MSE. 

The algorithm starts with an initial guess of quantization levels and then iteratively computes quantization thresholds $\{\tau_i\}_{i=1}^{2^B-1}$ and updates quantization levels $\{\hat{x}_i\}_{i=1}^{2^B}$. Specifically, at iteration $n$, thresholds are set to the midpoints of the previous iteration's levels:
\begin{align*}
    \tau_i^{(n)}=\frac{\hat{x}_i^{(n-1)}+\hat{x}_{i+1}^{(n-1)}}2 \text{ for } i=1\ldots 2^B-1
\end{align*}
Subsequently, the quantization levels are re-computed as conditional means of the data regions defined by the new thresholds:
\begin{align*}
    \hat{x}_i^{(n)}=\mathbb{E}\left[ \bm{X} \big| \bm{X}\in [\tau_{i-1}^{(n)},\tau_i^{(n)}] \right] \text{ for } i=1\ldots 2^B
\end{align*}
where to satisfy boundary conditions we have $\tau_0=-\infty$ and $\tau_{2^B}=\infty$. The algorithm iterates the above steps until convergence.

Figure \ref{fig:lm_quant} compares the quantization levels of a $7$-bit floating point (E3M3) quantizer (left) to a $7$-bit Lloyd-Max quantizer (right) when quantizing a layer of weights from the GPT3-126M model at a per-tensor granularity. As shown, the Lloyd-Max quantizer achieves substantially lower quantization MSE. Further, Table \ref{tab:FP7_vs_LM7} shows the superior perplexity achieved by Lloyd-Max quantizers for bitwidths of $7$, $6$ and $5$. The difference between the quantizers is clear at 5 bits, where per-tensor FP quantization incurs a drastic and unacceptable increase in perplexity, while Lloyd-Max quantization incurs a much smaller increase. Nevertheless, we note that even the optimal Lloyd-Max quantizer incurs a notable ($\sim 1.5$) increase in perplexity due to the coarse granularity of quantization. 

\begin{figure}[h]
  \centering
  \includegraphics[width=0.7\linewidth]{sections/figures/LM7_FP7.pdf}
  \caption{\small Quantization levels and the corresponding quantization MSE of Floating Point (left) vs Lloyd-Max (right) Quantizers for a layer of weights in the GPT3-126M model.}
  \label{fig:lm_quant}
\end{figure}

\begin{table}[h]\scriptsize
\begin{center}
\caption{\label{tab:FP7_vs_LM7} \small Comparing perplexity (lower is better) achieved by floating point quantizers and Lloyd-Max quantizers on a GPT3-126M model for the Wikitext-103 dataset.}
\begin{tabular}{c|cc|c}
\hline
 \multirow{2}{*}{\textbf{Bitwidth}} & \multicolumn{2}{|c|}{\textbf{Floating-Point Quantizer}} & \textbf{Lloyd-Max Quantizer} \\
 & Best Format & Wikitext-103 Perplexity & Wikitext-103 Perplexity \\
\hline
7 & E3M3 & 18.32 & 18.27 \\
6 & E3M2 & 19.07 & 18.51 \\
5 & E4M0 & 43.89 & 19.71 \\
\hline
\end{tabular}
\end{center}
\end{table}

\subsection{Proof of Local Optimality of LO-BCQ}
\label{subsec:lobcq_opt_proof}
For a given block $\bm{b}_j$, the quantization MSE during LO-BCQ can be empirically evaluated as $\frac{1}{L_b}\lVert \bm{b}_j- \bm{\hat{b}}_j\rVert^2_2$ where $\bm{\hat{b}}_j$ is computed from equation (\ref{eq:clustered_quantization_definition}) as $C_{f(\bm{b}_j)}(\bm{b}_j)$. Further, for a given block cluster $\mathcal{B}_i$, we compute the quantization MSE as $\frac{1}{|\mathcal{B}_{i}|}\sum_{\bm{b} \in \mathcal{B}_{i}} \frac{1}{L_b}\lVert \bm{b}- C_i^{(n)}(\bm{b})\rVert^2_2$. Therefore, at the end of iteration $n$, we evaluate the overall quantization MSE $J^{(n)}$ for a given operand $\bm{X}$ composed of $N_c$ block clusters as:
\begin{align*}
    \label{eq:mse_iter_n}
    J^{(n)} = \frac{1}{N_c} \sum_{i=1}^{N_c} \frac{1}{|\mathcal{B}_{i}^{(n)}|}\sum_{\bm{v} \in \mathcal{B}_{i}^{(n)}} \frac{1}{L_b}\lVert \bm{b}- B_i^{(n)}(\bm{b})\rVert^2_2
\end{align*}

At the end of iteration $n$, the codebooks are updated from $\mathcal{C}^{(n-1)}$ to $\mathcal{C}^{(n)}$. However, the mapping of a given vector $\bm{b}_j$ to quantizers $\mathcal{C}^{(n)}$ remains as  $f^{(n)}(\bm{b}_j)$. At the next iteration, during the vector clustering step, $f^{(n+1)}(\bm{b}_j)$ finds new mapping of $\bm{b}_j$ to updated codebooks $\mathcal{C}^{(n)}$ such that the quantization MSE over the candidate codebooks is minimized. Therefore, we obtain the following result for $\bm{b}_j$:
\begin{align*}
\frac{1}{L_b}\lVert \bm{b}_j - C_{f^{(n+1)}(\bm{b}_j)}^{(n)}(\bm{b}_j)\rVert^2_2 \le \frac{1}{L_b}\lVert \bm{b}_j - C_{f^{(n)}(\bm{b}_j)}^{(n)}(\bm{b}_j)\rVert^2_2
\end{align*}

That is, quantizing $\bm{b}_j$ at the end of the block clustering step of iteration $n+1$ results in lower quantization MSE compared to quantizing at the end of iteration $n$. Since this is true for all $\bm{b} \in \bm{X}$, we assert the following:
\begin{equation}
\begin{split}
\label{eq:mse_ineq_1}
    \tilde{J}^{(n+1)} &= \frac{1}{N_c} \sum_{i=1}^{N_c} \frac{1}{|\mathcal{B}_{i}^{(n+1)}|}\sum_{\bm{b} \in \mathcal{B}_{i}^{(n+1)}} \frac{1}{L_b}\lVert \bm{b} - C_i^{(n)}(b)\rVert^2_2 \le J^{(n)}
\end{split}
\end{equation}
where $\tilde{J}^{(n+1)}$ is the the quantization MSE after the vector clustering step at iteration $n+1$.

Next, during the codebook update step (\ref{eq:quantizers_update}) at iteration $n+1$, the per-cluster codebooks $\mathcal{C}^{(n)}$ are updated to $\mathcal{C}^{(n+1)}$ by invoking the Lloyd-Max algorithm \citep{Lloyd}. We know that for any given value distribution, the Lloyd-Max algorithm minimizes the quantization MSE. Therefore, for a given vector cluster $\mathcal{B}_i$ we obtain the following result:

\begin{equation}
    \frac{1}{|\mathcal{B}_{i}^{(n+1)}|}\sum_{\bm{b} \in \mathcal{B}_{i}^{(n+1)}} \frac{1}{L_b}\lVert \bm{b}- C_i^{(n+1)}(\bm{b})\rVert^2_2 \le \frac{1}{|\mathcal{B}_{i}^{(n+1)}|}\sum_{\bm{b} \in \mathcal{B}_{i}^{(n+1)}} \frac{1}{L_b}\lVert \bm{b}- C_i^{(n)}(\bm{b})\rVert^2_2
\end{equation}

The above equation states that quantizing the given block cluster $\mathcal{B}_i$ after updating the associated codebook from $C_i^{(n)}$ to $C_i^{(n+1)}$ results in lower quantization MSE. Since this is true for all the block clusters, we derive the following result: 
\begin{equation}
\begin{split}
\label{eq:mse_ineq_2}
     J^{(n+1)} &= \frac{1}{N_c} \sum_{i=1}^{N_c} \frac{1}{|\mathcal{B}_{i}^{(n+1)}|}\sum_{\bm{b} \in \mathcal{B}_{i}^{(n+1)}} \frac{1}{L_b}\lVert \bm{b}- C_i^{(n+1)}(\bm{b})\rVert^2_2  \le \tilde{J}^{(n+1)}   
\end{split}
\end{equation}

Following (\ref{eq:mse_ineq_1}) and (\ref{eq:mse_ineq_2}), we find that the quantization MSE is non-increasing for each iteration, that is, $J^{(1)} \ge J^{(2)} \ge J^{(3)} \ge \ldots \ge J^{(M)}$ where $M$ is the maximum number of iterations. 
%Therefore, we can say that if the algorithm converges, then it must be that it has converged to a local minimum. 
\hfill $\blacksquare$


\begin{figure}
    \begin{center}
    \includegraphics[width=0.5\textwidth]{sections//figures/mse_vs_iter.pdf}
    \end{center}
    \caption{\small NMSE vs iterations during LO-BCQ compared to other block quantization proposals}
    \label{fig:nmse_vs_iter}
\end{figure}

Figure \ref{fig:nmse_vs_iter} shows the empirical convergence of LO-BCQ across several block lengths and number of codebooks. Also, the MSE achieved by LO-BCQ is compared to baselines such as MXFP and VSQ. As shown, LO-BCQ converges to a lower MSE than the baselines. Further, we achieve better convergence for larger number of codebooks ($N_c$) and for a smaller block length ($L_b$), both of which increase the bitwidth of BCQ (see Eq \ref{eq:bitwidth_bcq}).


\subsection{Additional Accuracy Results}
%Table \ref{tab:lobcq_config} lists the various LOBCQ configurations and their corresponding bitwidths.
\begin{table}
\setlength{\tabcolsep}{4.75pt}
\begin{center}
\caption{\label{tab:lobcq_config} Various LO-BCQ configurations and their bitwidths.}
\begin{tabular}{|c||c|c|c|c||c|c||c|} 
\hline
 & \multicolumn{4}{|c||}{$L_b=8$} & \multicolumn{2}{|c||}{$L_b=4$} & $L_b=2$ \\
 \hline
 \backslashbox{$L_A$\kern-1em}{\kern-1em$N_c$} & 2 & 4 & 8 & 16 & 2 & 4 & 2 \\
 \hline
 64 & 4.25 & 4.375 & 4.5 & 4.625 & 4.375 & 4.625 & 4.625\\
 \hline
 32 & 4.375 & 4.5 & 4.625& 4.75 & 4.5 & 4.75 & 4.75 \\
 \hline
 16 & 4.625 & 4.75& 4.875 & 5 & 4.75 & 5 & 5 \\
 \hline
\end{tabular}
\end{center}
\end{table}

%\subsection{Perplexity achieved by various LO-BCQ configurations on Wikitext-103 dataset}

\begin{table} \centering
\begin{tabular}{|c||c|c|c|c||c|c||c|} 
\hline
 $L_b \rightarrow$& \multicolumn{4}{c||}{8} & \multicolumn{2}{c||}{4} & 2\\
 \hline
 \backslashbox{$L_A$\kern-1em}{\kern-1em$N_c$} & 2 & 4 & 8 & 16 & 2 & 4 & 2  \\
 %$N_c \rightarrow$ & 2 & 4 & 8 & 16 & 2 & 4 & 2 \\
 \hline
 \hline
 \multicolumn{8}{c}{GPT3-1.3B (FP32 PPL = 9.98)} \\ 
 \hline
 \hline
 64 & 10.40 & 10.23 & 10.17 & 10.15 &  10.28 & 10.18 & 10.19 \\
 \hline
 32 & 10.25 & 10.20 & 10.15 & 10.12 &  10.23 & 10.17 & 10.17 \\
 \hline
 16 & 10.22 & 10.16 & 10.10 & 10.09 &  10.21 & 10.14 & 10.16 \\
 \hline
  \hline
 \multicolumn{8}{c}{GPT3-8B (FP32 PPL = 7.38)} \\ 
 \hline
 \hline
 64 & 7.61 & 7.52 & 7.48 &  7.47 &  7.55 &  7.49 & 7.50 \\
 \hline
 32 & 7.52 & 7.50 & 7.46 &  7.45 &  7.52 &  7.48 & 7.48  \\
 \hline
 16 & 7.51 & 7.48 & 7.44 &  7.44 &  7.51 &  7.49 & 7.47  \\
 \hline
\end{tabular}
\caption{\label{tab:ppl_gpt3_abalation} Wikitext-103 perplexity across GPT3-1.3B and 8B models.}
\end{table}

\begin{table} \centering
\begin{tabular}{|c||c|c|c|c||} 
\hline
 $L_b \rightarrow$& \multicolumn{4}{c||}{8}\\
 \hline
 \backslashbox{$L_A$\kern-1em}{\kern-1em$N_c$} & 2 & 4 & 8 & 16 \\
 %$N_c \rightarrow$ & 2 & 4 & 8 & 16 & 2 & 4 & 2 \\
 \hline
 \hline
 \multicolumn{5}{|c|}{Llama2-7B (FP32 PPL = 5.06)} \\ 
 \hline
 \hline
 64 & 5.31 & 5.26 & 5.19 & 5.18  \\
 \hline
 32 & 5.23 & 5.25 & 5.18 & 5.15  \\
 \hline
 16 & 5.23 & 5.19 & 5.16 & 5.14  \\
 \hline
 \multicolumn{5}{|c|}{Nemotron4-15B (FP32 PPL = 5.87)} \\ 
 \hline
 \hline
 64  & 6.3 & 6.20 & 6.13 & 6.08  \\
 \hline
 32  & 6.24 & 6.12 & 6.07 & 6.03  \\
 \hline
 16  & 6.12 & 6.14 & 6.04 & 6.02  \\
 \hline
 \multicolumn{5}{|c|}{Nemotron4-340B (FP32 PPL = 3.48)} \\ 
 \hline
 \hline
 64 & 3.67 & 3.62 & 3.60 & 3.59 \\
 \hline
 32 & 3.63 & 3.61 & 3.59 & 3.56 \\
 \hline
 16 & 3.61 & 3.58 & 3.57 & 3.55 \\
 \hline
\end{tabular}
\caption{\label{tab:ppl_llama7B_nemo15B} Wikitext-103 perplexity compared to FP32 baseline in Llama2-7B and Nemotron4-15B, 340B models}
\end{table}

%\subsection{Perplexity achieved by various LO-BCQ configurations on MMLU dataset}


\begin{table} \centering
\begin{tabular}{|c||c|c|c|c||c|c|c|c|} 
\hline
 $L_b \rightarrow$& \multicolumn{4}{c||}{8} & \multicolumn{4}{c||}{8}\\
 \hline
 \backslashbox{$L_A$\kern-1em}{\kern-1em$N_c$} & 2 & 4 & 8 & 16 & 2 & 4 & 8 & 16  \\
 %$N_c \rightarrow$ & 2 & 4 & 8 & 16 & 2 & 4 & 2 \\
 \hline
 \hline
 \multicolumn{5}{|c|}{Llama2-7B (FP32 Accuracy = 45.8\%)} & \multicolumn{4}{|c|}{Llama2-70B (FP32 Accuracy = 69.12\%)} \\ 
 \hline
 \hline
 64 & 43.9 & 43.4 & 43.9 & 44.9 & 68.07 & 68.27 & 68.17 & 68.75 \\
 \hline
 32 & 44.5 & 43.8 & 44.9 & 44.5 & 68.37 & 68.51 & 68.35 & 68.27  \\
 \hline
 16 & 43.9 & 42.7 & 44.9 & 45 & 68.12 & 68.77 & 68.31 & 68.59  \\
 \hline
 \hline
 \multicolumn{5}{|c|}{GPT3-22B (FP32 Accuracy = 38.75\%)} & \multicolumn{4}{|c|}{Nemotron4-15B (FP32 Accuracy = 64.3\%)} \\ 
 \hline
 \hline
 64 & 36.71 & 38.85 & 38.13 & 38.92 & 63.17 & 62.36 & 63.72 & 64.09 \\
 \hline
 32 & 37.95 & 38.69 & 39.45 & 38.34 & 64.05 & 62.30 & 63.8 & 64.33  \\
 \hline
 16 & 38.88 & 38.80 & 38.31 & 38.92 & 63.22 & 63.51 & 63.93 & 64.43  \\
 \hline
\end{tabular}
\caption{\label{tab:mmlu_abalation} Accuracy on MMLU dataset across GPT3-22B, Llama2-7B, 70B and Nemotron4-15B models.}
\end{table}


%\subsection{Perplexity achieved by various LO-BCQ configurations on LM evaluation harness}

\begin{table} \centering
\begin{tabular}{|c||c|c|c|c||c|c|c|c|} 
\hline
 $L_b \rightarrow$& \multicolumn{4}{c||}{8} & \multicolumn{4}{c||}{8}\\
 \hline
 \backslashbox{$L_A$\kern-1em}{\kern-1em$N_c$} & 2 & 4 & 8 & 16 & 2 & 4 & 8 & 16  \\
 %$N_c \rightarrow$ & 2 & 4 & 8 & 16 & 2 & 4 & 2 \\
 \hline
 \hline
 \multicolumn{5}{|c|}{Race (FP32 Accuracy = 37.51\%)} & \multicolumn{4}{|c|}{Boolq (FP32 Accuracy = 64.62\%)} \\ 
 \hline
 \hline
 64 & 36.94 & 37.13 & 36.27 & 37.13 & 63.73 & 62.26 & 63.49 & 63.36 \\
 \hline
 32 & 37.03 & 36.36 & 36.08 & 37.03 & 62.54 & 63.51 & 63.49 & 63.55  \\
 \hline
 16 & 37.03 & 37.03 & 36.46 & 37.03 & 61.1 & 63.79 & 63.58 & 63.33  \\
 \hline
 \hline
 \multicolumn{5}{|c|}{Winogrande (FP32 Accuracy = 58.01\%)} & \multicolumn{4}{|c|}{Piqa (FP32 Accuracy = 74.21\%)} \\ 
 \hline
 \hline
 64 & 58.17 & 57.22 & 57.85 & 58.33 & 73.01 & 73.07 & 73.07 & 72.80 \\
 \hline
 32 & 59.12 & 58.09 & 57.85 & 58.41 & 73.01 & 73.94 & 72.74 & 73.18  \\
 \hline
 16 & 57.93 & 58.88 & 57.93 & 58.56 & 73.94 & 72.80 & 73.01 & 73.94  \\
 \hline
\end{tabular}
\caption{\label{tab:mmlu_abalation} Accuracy on LM evaluation harness tasks on GPT3-1.3B model.}
\end{table}

\begin{table} \centering
\begin{tabular}{|c||c|c|c|c||c|c|c|c|} 
\hline
 $L_b \rightarrow$& \multicolumn{4}{c||}{8} & \multicolumn{4}{c||}{8}\\
 \hline
 \backslashbox{$L_A$\kern-1em}{\kern-1em$N_c$} & 2 & 4 & 8 & 16 & 2 & 4 & 8 & 16  \\
 %$N_c \rightarrow$ & 2 & 4 & 8 & 16 & 2 & 4 & 2 \\
 \hline
 \hline
 \multicolumn{5}{|c|}{Race (FP32 Accuracy = 41.34\%)} & \multicolumn{4}{|c|}{Boolq (FP32 Accuracy = 68.32\%)} \\ 
 \hline
 \hline
 64 & 40.48 & 40.10 & 39.43 & 39.90 & 69.20 & 68.41 & 69.45 & 68.56 \\
 \hline
 32 & 39.52 & 39.52 & 40.77 & 39.62 & 68.32 & 67.43 & 68.17 & 69.30  \\
 \hline
 16 & 39.81 & 39.71 & 39.90 & 40.38 & 68.10 & 66.33 & 69.51 & 69.42  \\
 \hline
 \hline
 \multicolumn{5}{|c|}{Winogrande (FP32 Accuracy = 67.88\%)} & \multicolumn{4}{|c|}{Piqa (FP32 Accuracy = 78.78\%)} \\ 
 \hline
 \hline
 64 & 66.85 & 66.61 & 67.72 & 67.88 & 77.31 & 77.42 & 77.75 & 77.64 \\
 \hline
 32 & 67.25 & 67.72 & 67.72 & 67.00 & 77.31 & 77.04 & 77.80 & 77.37  \\
 \hline
 16 & 68.11 & 68.90 & 67.88 & 67.48 & 77.37 & 78.13 & 78.13 & 77.69  \\
 \hline
\end{tabular}
\caption{\label{tab:mmlu_abalation} Accuracy on LM evaluation harness tasks on GPT3-8B model.}
\end{table}

\begin{table} \centering
\begin{tabular}{|c||c|c|c|c||c|c|c|c|} 
\hline
 $L_b \rightarrow$& \multicolumn{4}{c||}{8} & \multicolumn{4}{c||}{8}\\
 \hline
 \backslashbox{$L_A$\kern-1em}{\kern-1em$N_c$} & 2 & 4 & 8 & 16 & 2 & 4 & 8 & 16  \\
 %$N_c \rightarrow$ & 2 & 4 & 8 & 16 & 2 & 4 & 2 \\
 \hline
 \hline
 \multicolumn{5}{|c|}{Race (FP32 Accuracy = 40.67\%)} & \multicolumn{4}{|c|}{Boolq (FP32 Accuracy = 76.54\%)} \\ 
 \hline
 \hline
 64 & 40.48 & 40.10 & 39.43 & 39.90 & 75.41 & 75.11 & 77.09 & 75.66 \\
 \hline
 32 & 39.52 & 39.52 & 40.77 & 39.62 & 76.02 & 76.02 & 75.96 & 75.35  \\
 \hline
 16 & 39.81 & 39.71 & 39.90 & 40.38 & 75.05 & 73.82 & 75.72 & 76.09  \\
 \hline
 \hline
 \multicolumn{5}{|c|}{Winogrande (FP32 Accuracy = 70.64\%)} & \multicolumn{4}{|c|}{Piqa (FP32 Accuracy = 79.16\%)} \\ 
 \hline
 \hline
 64 & 69.14 & 70.17 & 70.17 & 70.56 & 78.24 & 79.00 & 78.62 & 78.73 \\
 \hline
 32 & 70.96 & 69.69 & 71.27 & 69.30 & 78.56 & 79.49 & 79.16 & 78.89  \\
 \hline
 16 & 71.03 & 69.53 & 69.69 & 70.40 & 78.13 & 79.16 & 79.00 & 79.00  \\
 \hline
\end{tabular}
\caption{\label{tab:mmlu_abalation} Accuracy on LM evaluation harness tasks on GPT3-22B model.}
\end{table}

\begin{table} \centering
\begin{tabular}{|c||c|c|c|c||c|c|c|c|} 
\hline
 $L_b \rightarrow$& \multicolumn{4}{c||}{8} & \multicolumn{4}{c||}{8}\\
 \hline
 \backslashbox{$L_A$\kern-1em}{\kern-1em$N_c$} & 2 & 4 & 8 & 16 & 2 & 4 & 8 & 16  \\
 %$N_c \rightarrow$ & 2 & 4 & 8 & 16 & 2 & 4 & 2 \\
 \hline
 \hline
 \multicolumn{5}{|c|}{Race (FP32 Accuracy = 44.4\%)} & \multicolumn{4}{|c|}{Boolq (FP32 Accuracy = 79.29\%)} \\ 
 \hline
 \hline
 64 & 42.49 & 42.51 & 42.58 & 43.45 & 77.58 & 77.37 & 77.43 & 78.1 \\
 \hline
 32 & 43.35 & 42.49 & 43.64 & 43.73 & 77.86 & 75.32 & 77.28 & 77.86  \\
 \hline
 16 & 44.21 & 44.21 & 43.64 & 42.97 & 78.65 & 77 & 76.94 & 77.98  \\
 \hline
 \hline
 \multicolumn{5}{|c|}{Winogrande (FP32 Accuracy = 69.38\%)} & \multicolumn{4}{|c|}{Piqa (FP32 Accuracy = 78.07\%)} \\ 
 \hline
 \hline
 64 & 68.9 & 68.43 & 69.77 & 68.19 & 77.09 & 76.82 & 77.09 & 77.86 \\
 \hline
 32 & 69.38 & 68.51 & 68.82 & 68.90 & 78.07 & 76.71 & 78.07 & 77.86  \\
 \hline
 16 & 69.53 & 67.09 & 69.38 & 68.90 & 77.37 & 77.8 & 77.91 & 77.69  \\
 \hline
\end{tabular}
\caption{\label{tab:mmlu_abalation} Accuracy on LM evaluation harness tasks on Llama2-7B model.}
\end{table}

\begin{table} \centering
\begin{tabular}{|c||c|c|c|c||c|c|c|c|} 
\hline
 $L_b \rightarrow$& \multicolumn{4}{c||}{8} & \multicolumn{4}{c||}{8}\\
 \hline
 \backslashbox{$L_A$\kern-1em}{\kern-1em$N_c$} & 2 & 4 & 8 & 16 & 2 & 4 & 8 & 16  \\
 %$N_c \rightarrow$ & 2 & 4 & 8 & 16 & 2 & 4 & 2 \\
 \hline
 \hline
 \multicolumn{5}{|c|}{Race (FP32 Accuracy = 48.8\%)} & \multicolumn{4}{|c|}{Boolq (FP32 Accuracy = 85.23\%)} \\ 
 \hline
 \hline
 64 & 49.00 & 49.00 & 49.28 & 48.71 & 82.82 & 84.28 & 84.03 & 84.25 \\
 \hline
 32 & 49.57 & 48.52 & 48.33 & 49.28 & 83.85 & 84.46 & 84.31 & 84.93  \\
 \hline
 16 & 49.85 & 49.09 & 49.28 & 48.99 & 85.11 & 84.46 & 84.61 & 83.94  \\
 \hline
 \hline
 \multicolumn{5}{|c|}{Winogrande (FP32 Accuracy = 79.95\%)} & \multicolumn{4}{|c|}{Piqa (FP32 Accuracy = 81.56\%)} \\ 
 \hline
 \hline
 64 & 78.77 & 78.45 & 78.37 & 79.16 & 81.45 & 80.69 & 81.45 & 81.5 \\
 \hline
 32 & 78.45 & 79.01 & 78.69 & 80.66 & 81.56 & 80.58 & 81.18 & 81.34  \\
 \hline
 16 & 79.95 & 79.56 & 79.79 & 79.72 & 81.28 & 81.66 & 81.28 & 80.96  \\
 \hline
\end{tabular}
\caption{\label{tab:mmlu_abalation} Accuracy on LM evaluation harness tasks on Llama2-70B model.}
\end{table}

%\section{MSE Studies}
%\textcolor{red}{TODO}


\subsection{Number Formats and Quantization Method}
\label{subsec:numFormats_quantMethod}
\subsubsection{Integer Format}
An $n$-bit signed integer (INT) is typically represented with a 2s-complement format \citep{yao2022zeroquant,xiao2023smoothquant,dai2021vsq}, where the most significant bit denotes the sign.

\subsubsection{Floating Point Format}
An $n$-bit signed floating point (FP) number $x$ comprises of a 1-bit sign ($x_{\mathrm{sign}}$), $B_m$-bit mantissa ($x_{\mathrm{mant}}$) and $B_e$-bit exponent ($x_{\mathrm{exp}}$) such that $B_m+B_e=n-1$. The associated constant exponent bias ($E_{\mathrm{bias}}$) is computed as $(2^{{B_e}-1}-1)$. We denote this format as $E_{B_e}M_{B_m}$.  

\subsubsection{Quantization Scheme}
\label{subsec:quant_method}
A quantization scheme dictates how a given unquantized tensor is converted to its quantized representation. We consider FP formats for the purpose of illustration. Given an unquantized tensor $\bm{X}$ and an FP format $E_{B_e}M_{B_m}$, we first, we compute the quantization scale factor $s_X$ that maps the maximum absolute value of $\bm{X}$ to the maximum quantization level of the $E_{B_e}M_{B_m}$ format as follows:
\begin{align}
\label{eq:sf}
    s_X = \frac{\mathrm{max}(|\bm{X}|)}{\mathrm{max}(E_{B_e}M_{B_m})}
\end{align}
In the above equation, $|\cdot|$ denotes the absolute value function.

Next, we scale $\bm{X}$ by $s_X$ and quantize it to $\hat{\bm{X}}$ by rounding it to the nearest quantization level of $E_{B_e}M_{B_m}$ as:

\begin{align}
\label{eq:tensor_quant}
    \hat{\bm{X}} = \text{round-to-nearest}\left(\frac{\bm{X}}{s_X}, E_{B_e}M_{B_m}\right)
\end{align}

We perform dynamic max-scaled quantization \citep{wu2020integer}, where the scale factor $s$ for activations is dynamically computed during runtime.

\subsection{Vector Scaled Quantization}
\begin{wrapfigure}{r}{0.35\linewidth}
  \centering
  \includegraphics[width=\linewidth]{sections/figures/vsquant.jpg}
  \caption{\small Vectorwise decomposition for per-vector scaled quantization (VSQ \citep{dai2021vsq}).}
  \label{fig:vsquant}
\end{wrapfigure}
During VSQ \citep{dai2021vsq}, the operand tensors are decomposed into 1D vectors in a hardware friendly manner as shown in Figure \ref{fig:vsquant}. Since the decomposed tensors are used as operands in matrix multiplications during inference, it is beneficial to perform this decomposition along the reduction dimension of the multiplication. The vectorwise quantization is performed similar to tensorwise quantization described in Equations \ref{eq:sf} and \ref{eq:tensor_quant}, where a scale factor $s_v$ is required for each vector $\bm{v}$ that maps the maximum absolute value of that vector to the maximum quantization level. While smaller vector lengths can lead to larger accuracy gains, the associated memory and computational overheads due to the per-vector scale factors increases. To alleviate these overheads, VSQ \citep{dai2021vsq} proposed a second level quantization of the per-vector scale factors to unsigned integers, while MX \citep{rouhani2023shared} quantizes them to integer powers of 2 (denoted as $2^{INT}$).

\subsubsection{MX Format}
The MX format proposed in \citep{rouhani2023microscaling} introduces the concept of sub-block shifting. For every two scalar elements of $b$-bits each, there is a shared exponent bit. The value of this exponent bit is determined through an empirical analysis that targets minimizing quantization MSE. We note that the FP format $E_{1}M_{b}$ is strictly better than MX from an accuracy perspective since it allocates a dedicated exponent bit to each scalar as opposed to sharing it across two scalars. Therefore, we conservatively bound the accuracy of a $b+2$-bit signed MX format with that of a $E_{1}M_{b}$ format in our comparisons. For instance, we use E1M2 format as a proxy for MX4.

\begin{figure}
    \centering
    \includegraphics[width=1\linewidth]{sections//figures/BlockFormats.pdf}
    \caption{\small Comparing LO-BCQ to MX format.}
    \label{fig:block_formats}
\end{figure}

Figure \ref{fig:block_formats} compares our $4$-bit LO-BCQ block format to MX \citep{rouhani2023microscaling}. As shown, both LO-BCQ and MX decompose a given operand tensor into block arrays and each block array into blocks. Similar to MX, we find that per-block quantization ($L_b < L_A$) leads to better accuracy due to increased flexibility. While MX achieves this through per-block $1$-bit micro-scales, we associate a dedicated codebook to each block through a per-block codebook selector. Further, MX quantizes the per-block array scale-factor to E8M0 format without per-tensor scaling. In contrast during LO-BCQ, we find that per-tensor scaling combined with quantization of per-block array scale-factor to E4M3 format results in superior inference accuracy across models. 



\end{document}
