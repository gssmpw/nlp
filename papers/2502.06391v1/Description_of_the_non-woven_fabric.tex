\section{Description of the non-woven fabric}
\label{description_nonwoven_fabric}
A non-woven fabric is a fabric-like material produced by bonding together staple short or continuous long fibers through chemical or mechanical processes. Many treatments for producing non-woven fabrics exist. For example, one can apply heat and pressure for bonding at limited areas of a non-woven film by passing it through the nip between heated calendar rolls either or both of which may have patterns of lands and depressions on their surfaces. During such a bonding process, depending of the types of fibers making up the non-woven film, the bonded regions may be formed independently of external influence or aid, i.e. the fibers of the film are melt fused at least in the pattern areas or with the addition of an adhesive. The advantages of thermally bonded non-woven fabrics include low energy costs and speed of production. 
\par
Non-woven fabrics can also be made by other processes. For more details see the patents \cite{Patent1} and \cite {Patent2} and references (and other patents) therein. However, in all of the non-woven fabrics, the producing process usually realizes punched rigid spots on the pattern giving them a sort of frame. For simplicity, from now on such punched rigid spots will be called \textit{tags}.
\par
The non-woven fabric provided by the company is used for producing diapers linings and is formed by polypropylene. Generally, non-woven fabrics are also used for medical and sanitary stuff like hospital gowns, wipes and masks and may be also formed by other types of polymers, like polyolefin and polyester. In particular, the company exploits a further bonding process to thermally bond pieces of raw non-woven fabric (made in turn by a bonding process like the ones described above) in order to build pieces of a diaper lining. The technical equipment used for such a bonding process is patented by the company and very similar to the one in \cite{Patent2}: two fiber webs pass through a pair of high velocity steel-made rollers which thermally fuse them in the pattern areas by applying a suitable almost instantaneous pressure. In principle, no heating process is needed for the rollers (for more details, see the Section \ref{bondingprocess}). Therefore, the description of the production suggests the numerical modelization of the bonding process.

\subsection{First collection of pictures}
In this section, we report some pictures of pieces of the non-woven fabric provided by the company, made by Heerbrugg Wild M3Z optical microscope, without any focus on the thermally bonded pattern. The presence of tags and the fibers is shown anyway. 

%%%%
\begin{figure}[H]
  \begin{center}
    \includegraphics[scale=0.20,angle=0]{figure/1.jpg}
    \includegraphics[scale=0.20,angle=0]{figure/2.jpg}
    %\includegraphics[scale=0.20,angle=0]{figure/3.jpg}
    \includegraphics[scale=0.20,angle=0]{figure/4.jpg}
  \end{center}
  \caption{First collection of pictures: you can see the elliptic shape of the tags and randomness of fibers.}
  \label{fig:microscopio1}
\end{figure}
%%%
\subsection{Second collection of pictures}
%%%%
Here we report another collection of pictures with a detailed focus on the thermally bonded areas, that unequivocally shows that, around the elliptic tags, the almost instantaneous compression process induces a thermal bonding with a resulting local fusing. The microscope we used is Zeiss Supra 40 field emission scanning electron microscope (FESEM), operating at an accelerating voltage of $2.5\ \unit{kV}$. A thin platinum palladium conductive coating was deposited on the surface of the samples before the observations.
\begin{figure}[H]
  \begin{center}
    \includegraphics[scale=0.15]{figure/tessuto_01.png}
    %\includegraphics[scale=0.15]{figure/tessuto_02.png}
    \includegraphics[scale=0.15]{figure/tessuto_03.png}
    \includegraphics[scale=0.15]{figure/tessuto_04.png}
  \end{center}
  \caption{Second collection of pictures: you can see the fused pattern areas near elliptic tags induced by the thermally bonding process.}
  \label{fig:microscopio3}
\end{figure}
%\clearpage