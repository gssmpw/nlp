%%
% !TEX root = main.tex
%%

\documentclass[a4paper,leqno]{CAIM_Sciendo_open_article_BY4}

\urlstyle{same}

%%%%%%%%%%%%%%%%%%%%%%%%%%%%%%%%%%%%%%%%%%%%%%%%%%%%%%%%%%%%%%%%%%%%%%
% To be customized by the editor
%%%%%%%%%%%%%%%%%%%%%%%%%%%%%%%%%%%%%%%%%%%%%%%%%%%%%%%%%%%%%%%%%%%%%%
% Issue number
\newcommand{\issue}{1}
% Last page of previous article
\newcommand{\lastprvs}{0}
% DOI
\newcommand{\DOI}{\thecwyear-XXXX}
% Communicated by
\newcommand{\caimeditor}{Associate Editor}
% Date of submission
\newcommand{\yyyyrec}{XXXX}
\newcommand{\mmrec}{XXXX}
\newcommand{\ddrec}{XX}
% Date of acceptance
\newcommand{\yyyyacc}{XXXX}
\newcommand{\mmacc}{XXXX}
\newcommand{\ddacc}{XX}
%%%%%%%%%%%%%%%%%%%%%%%%%%%%%%%%%%%%%%%%%%%%%%%%%%%%%%%%%%%%%%%%%%%%%%
\newcommand{\Tsteel}{T_{\mathrm{steel}}}
\newcommand{\Tflux}{T_{\mathrm{flux}}}
\newcommand{\Twork}{T_{\mathrm{work}}}
\newcommand{\Ksteel}{K_{\mathrm{steel}}}
\newcommand{\Ktissue}{K_{\mathrm{fabric}}}
\newcommand{\Ytissue}{\kappa_{\mathrm{fabric}}}
\newcommand{\htissue}{h_{\mathrm{fabric}}}
\newcommand{\htissuemin}{\htissue^{\min}}
\newcommand{\htissuemax}{\htissue^{\max}}
\newcommand{\wtissue}{w_{\mathrm{fabric}}}
\newcommand{\TmaxQ}{{T_{\max}^Q}}
\newcommand{\TmaxL}{{T_{\max}^L}}
\newcommand{\Cptissue}{C_{p,\textrm{fabric}}}
\newcommand{\celsius}{\textrm{\textcelsius}}%^\circ C}

\usepackage{mathrsfs}
\usepackage[ansinew]{inputenc}
\usepackage{float}
\usepackage{afterpage}
%\usepackage{MAIN-structure}
\let\captionbox\relax
\usepackage{caption}
\DeclareCaptionLabelSeparator{dotquad}{.\quad} % period and quad
\captionsetup{labelsep=dotquad, skip=0.25\baselineskip}
\usepackage{epstopdf}
\usepackage{blindtext}
%\usepackage{hyperref}
\usepackage{fancyvrb}
%%
% !TEX root = main.tex
%%

\makeatletter
\renewcommand\@biblabel[1]{#1.}
\makeatother

% units
\newcommand{\unit}[1]{{\color{blue}\textrm{#1}}}

% COLORS
\newcommand{\red}[1]{{\color{red} #1}}
\newcommand{\blue}[1]{{\color{blue} #1}}
\newcommand{\magenta}[1]{{\color{magenta} #1}}
\newcommand{\green}[1]{{\color{green} #1}}

\allowdisplaybreaks

% % % % % % % % % % % % % % % % % % % % % % % % % % % % % % % % % % % % % % % % % % %
%\newcommand{\cal}{\mathcal}
%\renewcommand{\thesubfigure}{\Alph{subfigure}}
\newcommand{\openvirg}{‘‘}
\newcommand{\closevirg}{’’}
\newcommand\sizefigure{0.45} %0.225
\newcommand\szfigrhos{0.45} %0.25
\renewcommand{\div}{\operatorname{div}}
\newcommand{\grad}{\operatorname{grad}}
\newcommand{\Rr}{{\mathbb{R}}}
\newcommand{\R}{{\mathbb{R}}}
\newcommand{\Tt}{{\mathbb{T}}}
\newcommand{\argmax}{\operatorname{\rm argmax}}
\newcommand{\lra}{\longrightarrow}

\newcommand{\rad}{2 pt}
\newcommand{\colo}{black}

\newcommand{\ep}{\varepsilon}
\newcommand{\eps}[1]{{#1}_{\varepsilon}}

%%%%%%%%%%%%%% My Macros %%%%%%%%%%%%%%%%%%%%%
\DeclareMathOperator*{\esssup}{ess\,sup}
\DeclareMathOperator{\diver}{div}
\DeclareMathOperator{\tr}{tr}
\DeclareMathOperator{\sgn}{sgn}
\DeclareMathOperator{\atan2}{atan2}
\DeclareMathOperator{\PP}{\mathbb{P}}
\DeclareMathOperator{\Q}{\mathbb{Q}}
\DeclareMathOperator{\A}{\mathcal{A}}
\DeclareMathOperator{\T}{\mathcal{T}}
\DeclareMathOperator{\Z}{\mathbb{Z}}
\DeclareMathOperator{\N}{\mathbb{N}}
\DeclareMathOperator{\de}{\!\nabla\!}
\DeclareMathOperator{\io}{\int_\Omega}
\DeclareMathOperator{\perogni}{\;\;\;\:\, \forall}
\DeclareMathOperator{\im}{Im}
\DeclareMathOperator{\sen}{sen}
\DeclareMathOperator{\loc}{loc}
\DeclareMathOperator{\co}{co}
\providecommand{\norm}[1]{\lVert#1\rVert}
\DeclareMathOperator*{\argmin}{arg\,min}
\newcommand{\G}{\Gamma}
\newcommand{\dt}{\Delta t}
\newcommand{\hk}{{hk}}
\newcommand{\HH}{\mathbb{H}}
\newcommand{\V}{\mathbb{V}}
\newcommand{\E}{\mathbb{E}}
\newcommand{\I}{\mathcal{I}}
\newcommand{\RN}{\mathbb{R}^N}
%\newcommand{\RR}{\mathbb{R}}
%\newcommand{\C}{{\mathbb C}}
\DeclareMathOperator*{\medcup}{\mathbin{\scalebox{0.85}{\ensuremath{\bigcup}}}}
\newcommand{\zz}{\mathbb{z}}
\newcommand{\ra}{{\rightarrow}}

\newcommand{\bO}{\mathbb{O}}
\newcommand{\bM}{\mathbb{M}}
\newcommand{\bL}{\mathbb{L}}
\newcommand{\cF}{\mathcal{F}}
\newcommand{\cB}{\mathcal{B}}
\newcommand{\cG}{\mathcal{G}}
\newcommand{\cE}{\mathcal{E}}
\newcommand{\cI}{\mathcal{I}}
\newcommand{\cJ}{\mathcal{J}}
\newcommand{\cA}{\mathcal{A}}
\newcommand{\ccB}{\mathcal{B}}
\newcommand{\cN}{\mathcal{N}}
\newcommand{\cS}{\mathcal{S}}
\newcommand{\cL}{\mathcal{L}}
\newcommand{\cD}{\mathcal{D}}
\newcommand{\cC}{\mathcal{C}}
\newcommand{\cH}{\mathcal{H}}
\newcommand{\cM}{\mathcal{M}}
\newcommand{\cP}{\mathcal{P}}
\newcommand{\cU}{\mathcal{U}}
\newcommand{\cV}{\mathcal{V}}
\newcommand{\cT}{\mathcal{T}}
\newcommand{\cO}{\mathcal{O}}
\newcommand{\cR}{\mathcal{R}}
\newcommand{\cK}{\mathcal{K}}
\newcommand{\cW}{\mathcal{W}}
\newcommand{\cQ}{\mathcal{Q}}
\newcommand{\wtM}{\widetilde{M}}
\newcommand{\wtU}{\widetilde{U}}
\newcommand{\wtlambda}{\widetilde{\lambda}}
\newcommand{\whM}{\widehat{M}}

\newcommand{\bfZERO}{\bm{0}}
\newcommand{\bfONE}{\bm{1}}
\newcommand{\bfa}{\bm{a}}
\newcommand{\bfb}{\bm{b}}
\newcommand{\bfd}{\bm{d}}
\newcommand{\bfF}{\bm{F}}
\newcommand{\bfI}{\bm{I}}
\newcommand{\bfR}{\bm{R}}
\newcommand{\bfT}{\bm{T}}
\newcommand{\bfx}{\bm{x}}
\newcommand{\bfy}{\bm{y}}
\newcommand{\bfp}{\bm{p}}
\newcommand{\bfq}{\bm{q}}

\newcommand{\bftheta}{\bm{\theta}}
\newcommand{\bftau}{\bm{\tau}}

% grec
\renewcommand{\phi}{\varphi}
\renewcommand{\a}{{\alpha}}
\renewcommand{\b}{{\beta}}
\newcommand{\g}{{\gamma}}
\renewcommand{\d}{{\delta}}
\newcommand{\e}{\epsilon}
\renewcommand{\k}{\kappa}
\renewcommand{\l}{\lambda}
\newcommand{\om}{\omega}
\newcommand{\s}{{\sigma}}
\renewcommand{\th}{\theta}
\newcommand{\Th}{\Theta}

\newcommand{\Exp}{{^}}
\newcommand{\deter}{\hbox{det}}
\newcommand{\Ker}{\ker}
\newcommand{\conv}{\hbox{conv}}
\newcommand{\pd}{\partial}
\newcommand{\Dx}{{\Delta x}}

\newcommand{\tv}{{\tilde v}}

\newcommand{\Lh}{\mathcal{L}_h}
\newcommand{\ds}{\displaystyle}
\def\squareforqed{\hbox{\rlap{$\sqcap$}$\sqcup$}}
%\def\qed{\ifmmode\else\unskip\quad\fi\squareforqed}
%\def\smartqed{\def\qed{\ifmmode\squareforqed\else{\unskip\nobreak\hfil
%\penalty50\hskip1em\null\nobreak\hfil\squareforqed
%\parfillskip=0pt\finalhyphendemerits=0\endgraf}\fi}}
\newcommand{\supp}{\mathrm{supp}\;}
\newcommand{\db}{\mathbf{d}}


% % % % % % % % % % % % % % % % % % % % % % % % % % % % % % % % % % % % % % % % % % %
% % % % % % % % % % % % % % % % % % % % % % % % % % % % % % % % % % % % % % % % % % %
%\newtheorem{teo}{Theorem}
%\newtheorem{theorem}{Theorem}
%\numberwithin{theorem}{section}
%\numberwithin{equation}{section}
%\newtheorem{proposition}[theorem]{Proposition}
%\newtheorem{proposition}{Proposition}
%\newtheorem{example}{Example}

%\theoremstyle{remark}
%\newtheorem{proof}{Proof} % * makes write proof and do not number it. Conflicts with package AMSTHM.

%\theoremstyle{definition}
%\newtheorem{cor}{Corollary}
%\newtheorem{lemma}{Lemma}
%\newtheorem{pro}{Proposition}
%\newtheorem{definition}{Definition}
%\newtheorem{hypothesis}{Assumption}

%\theoremstyle{remark}
%\newtheorem{remark}{Remark}
% % % % % % % % % % % % % % % % % % % % % % % % % % % % % % % % % % % % % % % % % % %

\endinput


% Identify the kind of paper: Research or Review
\newcommand{\typeofpaper}{Research Article}

% Define a command for each author
\newcommand{\authone}{F. Bagagiolo}
\newcommand{\authtwo}{E. Bertolazzi}
\newcommand{\auththree}{L. Marzufero}
\newcommand{\authfour}{A. Pegoretti}
\newcommand{\authfive}{D. Rigotti}
% ...

% List authors, using comma-separated above-defined commands
\newcommand{\authors}{\authone,~\authtwo,~\auththree, ~\authfour, ~\authfive}

% Running headings: short forms of title and author names
%\shorttitle{CAIM submission template}
%\shortauthor{F. Bagagiolo, E. Bertolazzi, L. Marzufero, A. Pegoretti, D. Rigotti}

\title{Modelization of the bonding process for a non-woven fabric:\\ analysis and numerics}

\author{\authone\aff{1},
        \authtwo\aff{2},
        \auththree\aff{3},
        \authfour\aff{2},
        \authfive\aff{2}}

\affiliation{\aff{1} Department of Mathematics, University of Trento, Italy
             \aff{2} Department of Industrial Engineering, University of Trento, Italy
             \aff{3} Faculty of Economics and Management, Free University of Bozen-Bolzano, Italy}

% Email address for the correspondence
%\email{enrico.bertolazzi@unitn.it}


\begin{document}

\setcounter{page}{\lastprvs+1}

\def\vet#1{{{\boldsymbol{#1}}}}

\maketitle

\begin{abstract}
\phantomsection\addcontentsline{toc}{section}{\numberline{}Abstract}

We are concerned with the study and the research developed at the University of Trento about a problem posed by the company Fater S.p.A., in Italy. Such a problem consists in a possible analysis of the behavior of the bonding process of a non-woven fabric. In particular, the bonding process is not given by the use of some kind of glue, but just by the pressure of two fiber webs through two high-velocity steel-made rollers. The aim of such a process is to produce diaper linings. The research comprised the formulation and theoretical as well as numerical analysis of analytical, mechanical and thermal models for the stress-strain behavior of the non-woven fabric's fibers and for the bonding process with heating effects.
\end{abstract}

\keywords{non-woven fabric; bonding process modeling; heat equation; numerical simulation.}

\AMScode{65L05; 65M06; 80A19; 80M20.}

\section{Introduction}

Large language models (LLMs) have achieved remarkable success in automated math problem solving, particularly through code-generation capabilities integrated with proof assistants~\citep{lean,isabelle,POT,autoformalization,MATH}. Although LLMs excel at generating solution steps and correct answers in algebra and calculus~\citep{math_solving}, their unimodal nature limits performance in plane geometry, where solution depends on both diagram and text~\citep{math_solving}. 

Specialized vision-language models (VLMs) have accordingly been developed for plane geometry problem solving (PGPS)~\citep{geoqa,unigeo,intergps,pgps,GOLD,LANS,geox}. Yet, it remains unclear whether these models genuinely leverage diagrams or rely almost exclusively on textual features. This ambiguity arises because existing PGPS datasets typically embed sufficient geometric details within problem statements, potentially making the vision encoder unnecessary~\citep{GOLD}. \cref{fig:pgps_examples} illustrates example questions from GeoQA and PGPS9K, where solutions can be derived without referencing the diagrams.

\begin{figure}
    \centering
    \begin{subfigure}[t]{.49\linewidth}
        \centering
        \includegraphics[width=\linewidth]{latex/figures/images/geoqa_example.pdf}
        \caption{GeoQA}
        \label{fig:geoqa_example}
    \end{subfigure}
    \begin{subfigure}[t]{.48\linewidth}
        \centering
        \includegraphics[width=\linewidth]{latex/figures/images/pgps_example.pdf}
        \caption{PGPS9K}
        \label{fig:pgps9k_example}
    \end{subfigure}
    \caption{
    Examples of diagram-caption pairs and their solution steps written in formal languages from GeoQA and PGPS9k datasets. In the problem description, the visual geometric premises and numerical variables are highlighted in green and red, respectively. A significant difference in the style of the diagram and formal language can be observable. %, along with the differences in formal languages supported by the corresponding datasets.
    \label{fig:pgps_examples}
    }
\end{figure}



We propose a new benchmark created via a synthetic data engine, which systematically evaluates the ability of VLM vision encoders to recognize geometric premises. Our empirical findings reveal that previously suggested self-supervised learning (SSL) approaches, e.g., vector quantized variataional auto-encoder (VQ-VAE)~\citep{unimath} and masked auto-encoder (MAE)~\citep{scagps,geox}, and widely adopted encoders, e.g., OpenCLIP~\citep{clip} and DinoV2~\citep{dinov2}, struggle to detect geometric features such as perpendicularity and degrees. 

To this end, we propose \geoclip{}, a model pre-trained on a large corpus of synthetic diagram–caption pairs. By varying diagram styles (e.g., color, font size, resolution, line width), \geoclip{} learns robust geometric representations and outperforms prior SSL-based methods on our benchmark. Building on \geoclip{}, we introduce a few-shot domain adaptation technique that efficiently transfers the recognition ability to real-world diagrams. We further combine this domain-adapted GeoCLIP with an LLM, forming a domain-agnostic VLM for solving PGPS tasks in MathVerse~\citep{mathverse}. 
%To accommodate diverse diagram styles and solution formats, we unify the solution program languages across multiple PGPS datasets, ensuring comprehensive evaluation. 

In our experiments on MathVerse~\citep{mathverse}, which encompasses diverse plane geometry tasks and diagram styles, our VLM with a domain-adapted \geoclip{} consistently outperforms both task-specific PGPS models and generalist VLMs. 
% In particular, it achieves higher accuracy on tasks requiring geometric-feature recognition, even when critical numerical measurements are moved from text to diagrams. 
Ablation studies confirm the effectiveness of our domain adaptation strategy, showing improvements in optical character recognition (OCR)-based tasks and robust diagram embeddings across different styles. 
% By unifying the solution program languages of existing datasets and incorporating OCR capability, we enable a single VLM, named \geovlm{}, to handle a broad class of plane geometry problems.

% Contributions
We summarize the contributions as follows:
We propose a novel benchmark for systematically assessing how well vision encoders recognize geometric premises in plane geometry diagrams~(\cref{sec:visual_feature}); We introduce \geoclip{}, a vision encoder capable of accurately detecting visual geometric premises~(\cref{sec:geoclip}), and a few-shot domain adaptation technique that efficiently transfers this capability across different diagram styles (\cref{sec:domain_adaptation});
We show that our VLM, incorporating domain-adapted GeoCLIP, surpasses existing specialized PGPS VLMs and generalist VLMs on the MathVerse benchmark~(\cref{sec:experiments}) and effectively interprets diverse diagram styles~(\cref{sec:abl}).

\iffalse
\begin{itemize}
    \item We propose a novel benchmark for systematically assessing how well vision encoders recognize geometric premises, e.g., perpendicularity and angle measures, in plane geometry diagrams.
	\item We introduce \geoclip{}, a vision encoder capable of accurately detecting visual geometric premises, and a few-shot domain adaptation technique that efficiently transfers this capability across different diagram styles.
	\item We show that our final VLM, incorporating GeoCLIP-DA, effectively interprets diverse diagram styles and achieves state-of-the-art performance on the MathVerse benchmark, surpassing existing specialized PGPS models and generalist VLM models.
\end{itemize}
\fi

\iffalse

Large language models (LLMs) have made significant strides in automated math word problem solving. In particular, their code-generation capabilities combined with proof assistants~\citep{lean,isabelle} help minimize computational errors~\citep{POT}, improve solution precision~\citep{autoformalization}, and offer rigorous feedback and evaluation~\citep{MATH}. Although LLMs excel in generating solution steps and correct answers for algebra and calculus~\citep{math_solving}, their uni-modal nature limits performance in domains like plane geometry, where both diagrams and text are vital.

Plane geometry problem solving (PGPS) tasks typically include diagrams and textual descriptions, requiring solvers to interpret premises from both sources. To facilitate automated solutions for these problems, several studies have introduced formal languages tailored for plane geometry to represent solution steps as a program with training datasets composed of diagrams, textual descriptions, and solution programs~\citep{geoqa,unigeo,intergps,pgps}. Building on these datasets, a number of PGPS specialized vision-language models (VLMs) have been developed so far~\citep{GOLD, LANS, geox}.

Most existing VLMs, however, fail to use diagrams when solving geometry problems. Well-known PGPS datasets such as GeoQA~\citep{geoqa}, UniGeo~\citep{unigeo}, and PGPS9K~\citep{pgps}, can be solved without accessing diagrams, as their problem descriptions often contain all geometric information. \cref{fig:pgps_examples} shows an example from GeoQA and PGPS9K datasets, where one can deduce the solution steps without knowing the diagrams. 
As a result, models trained on these datasets rely almost exclusively on textual information, leaving the vision encoder under-utilized~\citep{GOLD}. 
Consequently, the VLMs trained on these datasets cannot solve the plane geometry problem when necessary geometric properties or relations are excluded from the problem statement.

Some studies seek to enhance the recognition of geometric premises from a diagram by directly predicting the premises from the diagram~\citep{GOLD, intergps} or as an auxiliary task for vision encoders~\citep{geoqa,geoqa-plus}. However, these approaches remain highly domain-specific because the labels for training are difficult to obtain, thus limiting generalization across different domains. While self-supervised learning (SSL) methods that depend exclusively on geometric diagrams, e.g., vector quantized variational auto-encoder (VQ-VAE)~\citep{unimath} and masked auto-encoder (MAE)~\citep{scagps,geox}, have also been explored, the effectiveness of the SSL approaches on recognizing geometric features has not been thoroughly investigated.

We introduce a benchmark constructed with a synthetic data engine to evaluate the effectiveness of SSL approaches in recognizing geometric premises from diagrams. Our empirical results with the proposed benchmark show that the vision encoders trained with SSL methods fail to capture visual \geofeat{}s such as perpendicularity between two lines and angle measure.
Furthermore, we find that the pre-trained vision encoders often used in general-purpose VLMs, e.g., OpenCLIP~\citep{clip} and DinoV2~\citep{dinov2}, fail to recognize geometric premises from diagrams.

To improve the vision encoder for PGPS, we propose \geoclip{}, a model trained with a massive amount of diagram-caption pairs.
Since the amount of diagram-caption pairs in existing benchmarks is often limited, we develop a plane diagram generator that can randomly sample plane geometry problems with the help of existing proof assistant~\citep{alphageometry}.
To make \geoclip{} robust against different styles, we vary the visual properties of diagrams, such as color, font size, resolution, and line width.
We show that \geoclip{} performs better than the other SSL approaches and commonly used vision encoders on the newly proposed benchmark.

Another major challenge in PGPS is developing a domain-agnostic VLM capable of handling multiple PGPS benchmarks. As shown in \cref{fig:pgps_examples}, the main difficulties arise from variations in diagram styles. 
To address the issue, we propose a few-shot domain adaptation technique for \geoclip{} which transfers its visual \geofeat{} perception from the synthetic diagrams to the real-world diagrams efficiently. 

We study the efficacy of the domain adapted \geoclip{} on PGPS when equipped with the language model. To be specific, we compare the VLM with the previous PGPS models on MathVerse~\citep{mathverse}, which is designed to evaluate both the PGPS and visual \geofeat{} perception performance on various domains.
While previous PGPS models are inapplicable to certain types of MathVerse problems, we modify the prediction target and unify the solution program languages of the existing PGPS training data to make our VLM applicable to all types of MathVerse problems.
Results on MathVerse demonstrate that our VLM more effectively integrates diagrammatic information and remains robust under conditions of various diagram styles.

\begin{itemize}
    \item We propose a benchmark to measure the visual \geofeat{} recognition performance of different vision encoders.
    % \item \sh{We introduce geometric CLIP (\geoclip{} and train the VLM equipped with \geoclip{} to predict both solution steps and the numerical measurements of the problem.}
    \item We introduce \geoclip{}, a vision encoder which can accurately recognize visual \geofeat{}s and a few-shot domain adaptation technique which can transfer such ability to different domains efficiently. 
    % \item \sh{We develop our final PGPS model, \geovlm{}, by adapting \geoclip{} to different domains and training with unified languages of solution program data.}
    % We develop a domain-agnostic VLM, namely \geovlm{}, by applying a simple yet effective domain adaptation method to \geoclip{} and training on the refined training data.
    \item We demonstrate our VLM equipped with GeoCLIP-DA effectively interprets diverse diagram styles, achieving superior performance on MathVerse compared to the existing PGPS models.
\end{itemize}

\fi 

%\newpage
%\input{Cose da fare}
\section{Description of the non-woven fabric}
\label{description_nonwoven_fabric}
A non-woven fabric is a fabric-like material produced by bonding together staple short or continuous long fibers through chemical or mechanical processes. Many treatments for producing non-woven fabrics exist. For example, one can apply heat and pressure for bonding at limited areas of a non-woven film by passing it through the nip between heated calendar rolls either or both of which may have patterns of lands and depressions on their surfaces. During such a bonding process, depending of the types of fibers making up the non-woven film, the bonded regions may be formed independently of external influence or aid, i.e. the fibers of the film are melt fused at least in the pattern areas or with the addition of an adhesive. The advantages of thermally bonded non-woven fabrics include low energy costs and speed of production. 
\par
Non-woven fabrics can also be made by other processes. For more details see the patents \cite{Patent1} and \cite {Patent2} and references (and other patents) therein. However, in all of the non-woven fabrics, the producing process usually realizes punched rigid spots on the pattern giving them a sort of frame. For simplicity, from now on such punched rigid spots will be called \textit{tags}.
\par
The non-woven fabric provided by the company is used for producing diapers linings and is formed by polypropylene. Generally, non-woven fabrics are also used for medical and sanitary stuff like hospital gowns, wipes and masks and may be also formed by other types of polymers, like polyolefin and polyester. In particular, the company exploits a further bonding process to thermally bond pieces of raw non-woven fabric (made in turn by a bonding process like the ones described above) in order to build pieces of a diaper lining. The technical equipment used for such a bonding process is patented by the company and very similar to the one in \cite{Patent2}: two fiber webs pass through a pair of high velocity steel-made rollers which thermally fuse them in the pattern areas by applying a suitable almost instantaneous pressure. In principle, no heating process is needed for the rollers (for more details, see the Section \ref{bondingprocess}). Therefore, the description of the production suggests the numerical modelization of the bonding process.

\subsection{First collection of pictures}
In this section, we report some pictures of pieces of the non-woven fabric provided by the company, made by Heerbrugg Wild M3Z optical microscope, without any focus on the thermally bonded pattern. The presence of tags and the fibers is shown anyway. 

%%%%
\begin{figure}[H]
  \begin{center}
    \includegraphics[scale=0.20,angle=0]{figure/1.jpg}
    \includegraphics[scale=0.20,angle=0]{figure/2.jpg}
    %\includegraphics[scale=0.20,angle=0]{figure/3.jpg}
    \includegraphics[scale=0.20,angle=0]{figure/4.jpg}
  \end{center}
  \caption{First collection of pictures: you can see the elliptic shape of the tags and randomness of fibers.}
  \label{fig:microscopio1}
\end{figure}
%%%
\subsection{Second collection of pictures}
%%%%
Here we report another collection of pictures with a detailed focus on the thermally bonded areas, that unequivocally shows that, around the elliptic tags, the almost instantaneous compression process induces a thermal bonding with a resulting local fusing. The microscope we used is Zeiss Supra 40 field emission scanning electron microscope (FESEM), operating at an accelerating voltage of $2.5\ \unit{kV}$. A thin platinum palladium conductive coating was deposited on the surface of the samples before the observations.
\begin{figure}[H]
  \begin{center}
    \includegraphics[scale=0.15]{figure/tessuto_01.png}
    %\includegraphics[scale=0.15]{figure/tessuto_02.png}
    \includegraphics[scale=0.15]{figure/tessuto_03.png}
    \includegraphics[scale=0.15]{figure/tessuto_04.png}
  \end{center}
  \caption{Second collection of pictures: you can see the fused pattern areas near elliptic tags induced by the thermally bonding process.}
  \label{fig:microscopio3}
\end{figure}
%\clearpage
%%
% !TEX root = main.tex
%%

\section{Polypropylene data and modelization}
\label{polypropylene}
The aim of this section is to compute the temperature increment and the temperature decay of the non-woven fabric through a suitable compression rate. We will provide two models, a linear and a quadratic one. To do that, we will need some data of the fabric, polypropylene and steel, which we will recover a little from the literature and a little from laboratories experiments.
\subsection{Tacticity and some data from literature}
In order to establish which is the polypropylene's type (atactic, isotactic, syndiotactic) of the non-woven fabric, we performed a spectrum analysis, made with a Fourier transformed infrared (FT-IR) spectrophotometer, using a Avatar 330 by the Thermo Fisher company. The spectrum, as in figure \eqref{fig:spettro}, is the result of $64$ scans with a resolution of $4\ \unit{cm}^{-1}$, and turned out that the polypropylene is atactic.

%Fourier transformed infrared (FT-IR)spectroscopy was conducted by
%using a (nome modello non lo so non ho fatto io la misura) and operating
%in a wavenumber range $650–4000\ \unit{cm}^-1$

%%%%
\begin{figure}[H]
  \begin{center}
    \includegraphics[scale=0.2]{figure/atattico-banca-dati.png}
  \end{center}
  \caption{Spectrum comparison of our non-woven fabric with a sample atactic polypropylene.}
  \label{fig:spettro}
\end{figure}

For the description and the numerical modelization of the bonding process, we need some polyprolylene data, that we collect in the following Table \ref{tab:polipropilene} (see \cite{Passaglia, Maier, Young,Gianotti1968}):
%%%
\begin{table}[ht]
\centering
  \caption{Physical parameters of polypropylene}
   \label{tab:polipropilene}
  \begin{tabular}{llll}
    Parameter     & Value & Unit \\
    \hline
    Melting point        & $130$--$171$   & $\unit{\celsius}$        & measured $160$ $\celsius$ \\
    Density              & $855$--$946$   & $\unit{kg}/\unit{m}^3$   & atactic: $866$ \\
    %molar Heat Capacity  & $88$--$90$     & $\unit{J}/\unit{mol K}$  & \\
    Thermal conductivity & $0.17$--$0.22$  & $\unit{W}/(\unit{m K})$  &  at $23$ $\celsius$ typ $0.17$ \\
    Specific heat        & $581.97$--$2884.71$ & $\unit{J}/(\unit{kg K})$ & temp. in K \\
    %Tensile modulus      & $1200$         & $\unit{Mpa}$               & \\
    %Flexural modulus     & $1150$         & $\unit{Mpa}$               & \\
    Young's module         & $1300$--$1800$ & $\unit{N}/\unit{mm}^2=\unit{MPa}$               & \\
    %Elongation           & $450$          & \textrm{\%}                  & \\
    %Vicat softening temperature & $90$--$150$    & $\celsius$             & \\
    \hline
  \end{tabular}
\end{table}
%%%
Due to the high variability of the fusion temperature $130$--$171 \ \unit{\celsius}$, an accurate measurement is performed at $160\ \unit{\celsius}$. In order to detect that, polypropylene is heated slowly with a constant temperature rate. The heat flux to the polypropylene is measured and, when a negative peak of heat is detected, it means that polypropylene is changing phase from solid to liquid.
%%%%
\begin{figure}[H]
  \begin{center}
    \includegraphics[scale=0.4]{figure/Fusion.pdf}
    %\includegraphics[scale=0.25]{figure/campione.png}
    %\includegraphics[scale=0.05]{figure/measure_fusion.png}
  \end{center}
  \caption{Temperature and heat absorption: the negative peak is the fusion point of the polypropylene, here about $160$ $\unit{\celsius}$.}
  \label{fig:fusion}
\end{figure}
%%%

Since the rollers which thermally bond the pieces of non-woven fabric are steel-made, similarly to the polypropylene, we need some physical parameters of the steel, that we collect in the following Table~\ref{tab:acciaio} (see \cite{steel3, steel4, steel2, steel1}):

\begin{table}[H]
  \caption{Physical parameters of steel}\label{tab:acciaio}
  \label{tab:steel}
  \begin{center}
  \begin{tabular}{lll}
    Parameter     & Value & Unit \\
    \hline
    Melting point               & $1400$--$1530$ & $\unit{\celsius}$                   \\
    Density                     & $7500$--$8000$ & $\unit{kg}/\unit{m}^3$   \\
    Thermal conductivity (stainless) & $15$--$18$     & $\unit{W}/(\unit{m K})$  \\
    Thermal conductivity        & $44$--$80$     & $\unit{W}/(\unit{m K})$  \\
    Specific heat               & $500$          & $\unit{J}/(\unit{kg K})$ \\
    %Tensile modulus             & $1200$         & $\unit{Mpa}$               \\
    %Flexural modulus            & $1150$         & $\unit{Mpa}$               \\
    Young's modulus              & $180000$       & $\unit{N}/\unit{mm}^2=\unit{MPa}$ \\
    %Elongation                 & $450$          & \textrm{\%}                  \\
    \hline
  \end{tabular}
  \end{center}
\end{table}

%Vicat softening temperature is the temperature at which the specimen is penetrated to a depth of $1 \unit{mm}$
%by a flat-ended needle with a $1 \textrm{mm}^2$ circular or square cross-section.
%For the Vicat A test, a load of $10\unit{N}$ is used. For the Vicat B test, the load is $50\unit{N}$.

%reference \url{https://material-properties.org/polypropylene-density-strength-%melting-point-thermal-conductivity/}
%\url{https://polymerdatabase.com/polymers/polypropylene.html}
%\url{https://www.m-ep.co.jp/en/pdf/product/iupi_nova/physicality_04.pdf}
%\url{https://www.professionalplastics.com/professionalplastics/%ThermalPropertiesofPlasticMaterials.pdf}
%\url{https://designerdata.nl/materials/plastics/thermo-plastics/polypropylene-%(cop.)}
%\url{https://scipoly.com/density-of-polymers-by-density/}
%\url{https://www.centroinox.it/sites/default/files/pubblicazioni/245A.pdf}

%%%%
%\begin{figure}[!htb]
%  \begin{center}
%    \includegraphics[scale=0.35]{forza-spostamento}
%  \end{center}
%  \caption{Displacement vs force}
%  \label{fig:forza-spostamento}
%\end{figure}
%%%%

\subsection{Determination of fabric weight by unit area}
\label{subsec:weight}
%%%
In this section, we report the density of our non-woven fabric, determined by a $15\ \unit{cm} \times 15\ \unit{cm}$ fabric sample. The next figure shows the fabric piece and its weight, determined by a precision scale Netzsch DSC204.
%%%
\begin{figure}[H]
  \begin{center}
    \includegraphics[height=4.7cm]{figure/peso-15x15-1.png}
    \includegraphics[height=4.7cm]{figure/peso-15x15-2.png}
    \includegraphics[height=4.7cm]{figure/peso-15x15-3.png}
  \end{center}
  \caption{On the left the precision scale used for the weight determination and on the right the non-woven fabric piece used for the experiment.}
  \label{fig:weight}
\end{figure}
%%%
It follows that the weight by unit area $\wtissue$ is:
%%%
\begin{equation}\label{eq:wT}
   \wtissue=\dfrac{[\textrm{mass}]}{[\textrm{area}]}=
   \dfrac{0.2835 \ \unit{g}}{150 \ \unit{mm}\cdot 150 \ \unit{mm}} = 0.0126 \ 
   \dfrac{\unit{kg}}{\unit{m}^2}.
\end{equation}
%%%
As long as the non-woven fabric is fully compressed, from the density of the polypropylene (see Table \ref{tab:polipropilene}) we can infer its thickness:
%%%
\[
   \htissuemin = \dfrac{[\textrm{weigth}]/[\textrm{area}]}{[\textrm{density}]}
     = \dfrac{0.0126 \ \dfrac{\unit{kg}}{\unit{m}^2}}
             {900 \ \dfrac{\unit{kg}}{\unit{m}^3}}
     = 0.000014\,\unit{m}
     = 0.014\,\unit{mm}
     = 14\,\unit{$\mu$m}.
\]
%%%
\subsection{Computation of the thickness: pressure and displacement}
\label{subsec:YvsF}
In this subsection, we want to establish the thickness of the non-woven fabric not compressed. To do that, we use an indirect approach based on experimental measurements made with the dynamometer Instron 5969, at a speed of $1\ \unit{mm}/\unit{min}$, of the laboratories in the Department of Materials Engineering at the University of Trento.
\par 
In particular, at first we characterize the displacement/pressure curve of the dynamometer with the press disk made by seven circle tags without the non-woven fabric. In this way, we determine the offset corresponding to zero thickness to be eliminated in the measurement of the curve. We did two experiments and the results are the following.
%%%
\begin{figure}[H]
  \begin{center}
    \includegraphics[width=10cm]{figure/pressione-spostamento-free.pdf}
  \end{center}
  \caption{Displacement/pressure graph without the non-woven fabric}
\end{figure}
%%%
The fitting of the pressure is the following
%%%
\[
  P_{\textrm{base}}(x) =
  \dfrac{5461.352911\,\max(0,x)^{2.92599}}
  {0.0038158166 + 6.4490865\,\max(0,x)^{1.624481}}.
\]
%%%
After this measurement, ten sheets of non-woven fabric are put under the press disk, getting the following pressure curve.
%%%
\begin{figure}[H]
  \begin{center}
    \includegraphics[width=10cm]{figure/pressione-spostamento-10tessuti.pdf}
  \end{center}
  \caption{Displacement/pressure graph with ten non-woven fabric sheets}
\end{figure}
%%%
Fitting again the data, we obtain
%%%
\[
  P_{\textrm{base}+\textrm{10 fabric}}(x) =
  \dfrac{716.33893 \max(0,x+0.9703)^{12.67189680}}
        {14.10752 + 0.92219399\max(0,x+0.9703)^{30.037944}}.
\]
%%%
This shows that the pressure starts growing at $x=-0.9703$ which can be assumed as the thickness of the ten non-woven fabric sheets, and then we can set it to
%%%
\[
   \htissuemax = \dfrac{0.97\,\unit{mm}}{10} = 97\,\unit{$\mu$m}.
\]
%%%
Due to the very low speed of the displacement movement, the pressures are in equilibrium as follows:
%%%
\[
  P_{\textrm{base}+\textrm{10 fabric}}(x) =
  P_{\textrm{base}}(z) = P_{\textrm{base}}( x- w) =
  P_{\textrm{10 fabric}}(w),
\]
%%%o
where the total displacement $x=z+w$ is the sum of the displacement of the ten non-woven fabric sheets ($w$) and the press displacement ($z$). To obtain $P_{\textrm{10 fabric}}(w)$, first of all we have to determine $x$ as a function of $w$ by solving
%%%
\begin{equation}\label{eq:xw:solve}
  P_{\textrm{base}+\textrm{10 fabric}}(x)-
  P_{\textrm{base}}(x-w)= 0.
\end{equation}
%%%
This equation can be solved numerically with respect to $x$ and assuming $x(w)$ as known. Then
%%%
\[
  P_{\textrm{10 fabric}}(w) = P_{\textrm{base}+\textrm{10 fabric}}(x(w)).
\]
%%%
The function $x(w)$ is very ill-conditioned and defined for about $w\leq-0.08$. Thus it is better to compute the inverse function $w(x)$ by solving~\eqref{eq:xw:solve} w.r.t. $w$. Plotting $w(x)/10$ and $P_{\textrm{base}+\textrm{10 fabric}}(x)$, we obtain
%%%
\begin{figure}[H]
  \begin{center}
    \includegraphics[height=6cm]{figure/displacement-pressure-pp_page-0001.jpg}
  \end{center}
  \caption{Displacement/pressure graph of a single sheet of non-woven fabric}
  \label{fig:pp:pressure}
\end{figure}
%%%
and this is an approximation of the displacement/pressure curve on a single sheet of non-woven fabric. When the displacement is $0$, the fabric has non physical thickness $0$ and graph stops at about $-6 \ \unit{$\mu$m}$. The graph pressure is negligible until displacement is $-30 \ \unit{$\mu$m}$, and thus we can correct the thickness of the fabric (under mild compression) to $30 \ \unit{$\mu$m}$.
\par
Hence the fabric under compression at about $5 \ \unit{MPa}$ is fully compressed (the fibers are compacted) and the thickness is about $14\,\unit{$\mu$m}$. When pressure grows until $80\,\unit{MPa}$, the thickness is reduced to about $7\,\unit{$\mu$m}$, and the non-woven fabric assumed as a single block is under a strain of $7/14=0.5$. This information will be used to compute the increment of the temperature of the fabric under fast compression.
\par
To make computations workable, we derive from Figure~\ref{fig:pp:pressure} an approximated law linking the strain of the non-woven fabric with the pressure. In Figure~\ref{fig:sp:fitting}, a plot of the strain calculated with respect to the compressed fabric and the pressure is fitted with a hyperbolic curve. The $x$-axis contains the strain:
%%%
\[
   [\textrm{strain}]
   = \dfrac{[\textrm{displacement}]+\htissuemin}{\htissuemin}
   = \dfrac{[\textrm{displacement}]+14 \ \unit{$\mu$m}}{14 \ \unit{$\mu$m}}.
\]
%%%
Zero strain means that the fabric is fully compressed and positive strain means that the fabric is compressing. Negative strain means that the fabric is not fully compressed and the pressure is relatively low.
%%%
\begin{figure}[H]
  \begin{center}
    \includegraphics[width=6cm]{figure/strain-pressure-pp-fitting.pdf}
  \end{center}
  \caption{Strain/pressure curve fitting}
  \label{fig:sp:fitting}
\end{figure}
%%%
As you can notice, the pressure in the negative strain part is overestimated, while the positive part is well captured up to a strain of $0.4$ (compression of $40\%$). However, we are interested in the estimation of the heating of the non-woven fabric, and this part of the curve produces a very low effect. The approximated/fitted Young's modulus becomes
%%%
\begin{equation}\label{eq:Y:module:tissue}
  \Ytissue(s)
  = \dfrac{16}{1-2s}\quad [\unit{MPa}]
  = \dfrac{16\cdot 10^6}{1-2s}\quad [\unit{Pa}].
\end{equation}

\subsection{Computation of the temperature increment}
%%%
From the following experimental figure obtained at the University of Trento, it follows that $\Ytissue$ (i.e., the Young or Storage modulus) of the non-woven fabric linearly decreases by temperature. Dynamic mechanical analysis (DMA) tests were carried out using a TA Instrument DMA Q800 device, in the temperature range from $25\ \unit{\celsius}$ to $120\ \unit{\celsius}$, with a heating rate of $3\ \unit{\celsius}/\unit{min}$, a strain amplitude of $0.05\%$ and a frequency of $1\ \unit{Hz}$. Through this analysis, it was possible to evaluate the Storage modulus, the loss modulus and the loss tangent as a function of the temperature. 
%%%
\begin{figure}[H]
  \begin{center}
    \includegraphics[scale=0.6]{figure/Curva_modulo_elastico_temperatura.png}
  \end{center}
  \caption{Storage modulus/temperatue plot of the non-woven fabric}
  \label{fig:stiffness:temperature}
\end{figure}
%%%
In particular, the Young modulus linearly decreases from $T_0=20$ $\unit{\celsius}$ and reaches $0$ at about $90$ $\unit{\celsius}$. The value of the modulus of the non-woven fabric is not constant and depends on the displacement, but we can assume that its value linearly decreases by temperature in any condition. Thus we set
%%%
\[
  \kappa(s,T)
  = \Ytissue(s) \max\left(0,\dfrac{\TmaxL-T}{\TmaxL-T_0}\right).
  %= \dfrac{16\cdot 10^6}{1-2s}\cdot \dfrac{\TmaxL-T}{\TmaxL-T_0}
\]
%%%
Following \cite{Young}, instead of a linear decreasing, a parabolic fitting with $\kappa(T)=0$ near the fusing temperature ($\approx 160$ $\unit{\celsius}$) can be used. Moreover, it suggests an exponential fitting for Storage modulus depending on temperature. However we used linear and parabolic fitting to maintain the model as simple as possible.
Therefore
%%%
\begin{equation}\label{eq:kappaT}
   \kappa(s,T)
   = \Ytissue(s)\max\left(0,\dfrac{\TmaxQ-T}{\TmaxQ-T_0}\right)^2.
   %= \dfrac{16\cdot 10^6}{1-2s} \left(\dfrac{\TmaxQ-T}{\TmaxQ-T_0}\right)^2
\end{equation}
%%%
Using $\kappa(s,T)$, the force and the pressure per unit area become
%%%
\begin{equation}\label{eq:pressure:tissue}
  P(s,T) = \dfrac{F(s,T)}{[\textrm{area}]} = \kappa(s,T)s.
\end{equation}
%%%
The infinitesimal work done on the non-woven fabric (at fixed temperature) is
%%%
\[
   \mathrm{d}W(s,T) = F(s,T) \mathrm{d}s,
\]
%%%
and the variation of temperature is done by the work on the fabric as follows:
%%%
\[
  \begin{split}
    \dfrac{\mathrm{d} T(s)}{\mathrm{d}s} & =
    \dfrac{\Delta[\textrm{work}]'}{\Cptissue[\textrm{mass}]}
    =
    \dfrac{\dfrac{\Delta[\textrm{work}]'}{[\textrm{area}]}}
          {C_{p,\textrm{fabric}}\dfrac{[\textrm{mass}]}{[\textrm{area}]}}
    =
    \dfrac{F(s,T)/[\textrm{area}]}{\Cptissue\cdot \wtissue}
    =
    \dfrac{P(s,T)}{\Cptissue\cdot \wtissue} \\
    & =
    \dfrac{\kappa(s,T)s}{\Cptissue\cdot \wtissue},
  \end{split}
\]
%%%
where $\Cptissue$ is the specific heat of the fabric. Thus, using~(\ref{eq:pressure:tissue}), assuming that the cooling temperature negligible, the temperature increment is obtained by solving the ODE
%%%
\begin{equation}\label{ode:lin}
   \dfrac{\mathrm{d} T(s)}{\mathrm{d}s} =
   \dfrac{s\,\Ytissue(s)}{\Cptissue\cdot \wtissue}\cdot \max\left(0,\dfrac{\TmaxL-T(s)}{\TmaxL-T_0}\right)
\end{equation}
%%%
for the simple linear model, and
%%%
\begin{equation}\label{ode:para}
   \dfrac{\mathrm{d} T(s)}{\mathrm{d}s}  =
   \dfrac{s\,\Ytissue(s)}{\Cptissue\cdot \wtissue}\cdot
   \max\left(0,\dfrac{\TmaxQ-T(s)}{\TmaxQ-T_0}\right)^2
\end{equation}
%%%
for the parabolic interpolation model. Using the value in the table
%%%
\begin{center}
  \begin{tabular}{|cc|}
    \hline
    Parameter & Value \\
    \hline
    $T_0$    & $20\;\unit{\celsius}$ \\
    $\TmaxL$ & $90\;\unit{\celsius}$ \\
    $\TmaxQ$ & $160\;\unit{\celsius}$ \\
    $\Cptissue$ & $1800\;[\unit{J}/(\unit{kg K})]$ \\
    $\wtissue$  & $0.0126\;[\unit{kg}/\unit{m}^2]$ \\
    \hline
  \end{tabular}
\end{center}
%%%
the heating (numerically) computed is plotted in the following figure
%%%
\begin{figure}[H]
  \begin{center}
    \includegraphics[width=6cm]{figure/heating1.pdf}
  \end{center}
  \caption{Heating by pressure (adiabatic case)}
  \label{fig:heating1}
\end{figure}
%%%
and a moderate strain of the compressed non-woven fabric is enough to heat up to the melting temperature.

\subsection{Evaluation of the temperature decay by flux}
%%%
\begin{figure}[H]
    \begin{center}
      \includegraphics[width=4cm]{figure/flusso-calore.png} \\
      %\includegraphics[width=12cm]{tessuto/temperature_decay} \\
      %\includegraphics[width=12cm]{tessuto/temperature_raise}
    \end{center}
    \caption{Heat flux}% and temperature decay}
    \label{fig:heat-flux}
\end{figure}
%%%
%%%
From Figure~\ref{fig:heat-flux}, assuming that the steel has fixed temperature $\Tsteel$, the thermal flux from the non-woven fabric to the steel using the Newton's Law of cooling is
%%%
\[
   \mathcal{Q} = 2\Ksteel
   \dfrac{\Tsteel-T(t)}{\htissuemin(1-s)/2} [\textrm{area}]
   \qquad [\unit{W}],
\]
%%%
where $\Ksteel$ is the thermal conductivity of the steel. The thermal conductivity and the heat capacity of the steel are very high compared to the polypropylene ones, so that they can be assumed as infinity. The temperature variation due to the heat flux using~\eqref{eq:wT} becomes
%%%
\[
  \Cptissue[\textrm{mass}]\dfrac{\mathrm{d}\Tflux(t)}{\mathrm{d}t}
  = \mathcal{Q} = \dfrac{4\Ksteel(\Tsteel-T(t))}{\htissuemin(1-s)}  [\textrm{area}]
\]
%%%
so that
%%%
\begin{equation}\label{eq:T:cool:by:steel}
  \dfrac{\mathrm{d}\Tflux(t)}{\mathrm{d}t}
  =
  \dfrac{4\Ksteel(\Tsteel-\Tflux(t))}{\htissuemin(1-s)\Cptissue[\textrm{mass}]/[\textrm{area}]}
  =
  \dfrac{4\Ksteel(\Tsteel-\Tflux(t))}{\htissuemin(1-s)\Cptissue\wtissue}
\end{equation}
%%%
assuming that the compression is done at a constant speed $v\,[\unit{m/s}]$. Then we can write
%%%
\[
   s = \dfrac{v\cdot t}{\htissuemin}
\]
%%%
and
%%%
\begin{equation}\label{eq:Tflux}
  \dfrac{\mathrm{d}\Tflux(s)}{\mathrm{d}s}\dfrac{v}{\htissuemin}=
  \dfrac{4\Ksteel(\Tsteel-\Tflux(s))}{\htissuemin(1-s)\Cptissue\wtissue}.
\end{equation}
%%%
Using~\eqref{eq:Tflux} with~\eqref{ode:lin} or~\eqref{ode:para}, we have the two complete models:
%%%
\begin{equation}\label{ode:lin:full}
   \dfrac{\mathrm{d}T(s)}{\mathrm{d}s} = \dfrac{s\,\Ytissue(s)\max\left(0, \dfrac{\TmaxL-T(s)}{\TmaxL-T_0}\right)
   +\dfrac{4\Ksteel}{v(1-s)}
  (\Tsteel-T(s))}{\Cptissue\cdot \wtissue}
\end{equation}
%%%
for the simple linear model, and
%%%
\begin{equation}\label{ode:para:full}
   \dfrac{\mathrm{d}T(s)}{\mathrm{d}s} = \dfrac{s\,\Ytissue(s)\cdot
   \max\left(0,\dfrac{\TmaxQ-T(s)}{\TmaxQ-T_0}\right)^2
   +\dfrac{4\Ksteel}{v(1-s)}
  (\Tsteel-T(s))}{\Cptissue\cdot \wtissue}
\end{equation}
%%%
for the quadratic one.
%%%
We can solve numerically~\eqref{ode:lin:full} and~\eqref{ode:para:full} with parameters in the next table and velocity
%%%
\[
   v = \dfrac{\Delta s}{\Delta t}=\dfrac{2\htissuemin (1-r)}{\Delta t},
\]
%%%
where $\Delta s$ is the size of the compression of the non-woven fabric. It is set to $\htissuemin (1-r)$, where $r$ is the ratio of the fabric after the compression and the presence of the term $2$ is due the the fact that the two fabric sheets are bounded. Finally $\Delta t$ is the time spent through the compression.
%%%%
\begin{center}
  \begin{tabular}{|cc|}
    \hline
    Parameter & Value \\
    \hline
    $T_0$         & $20\;\unit{\celsius}$ \\
    $\TmaxL$      & $90\;\unit{\celsius}$ \\
    $\TmaxQ$      & $160\;\unit{\celsius}$ \\
    $\Cptissue$   & $1800\;[\unit{J}/(\unit{kg K})]$ \\
    $\wtissue$    & $0.0126\;[\unit{kg}/\unit{m}^2]$ \\
    $\htissuemin$ & $14\,\unit{$\mu$m}$ \\
    $r$           & $0.6$ \\
    $\Delta t$    & $10\,\unit{ms}$, $1\,\unit{ms}$, $0.1\,\unit{ms}$ \\
    $\Ksteel$     & $50\,[\unit{W}/(\unit{m K})]$ \\
    \hline
  \end{tabular}
\end{center}
%%%
The heating (numerically) computed is plotted in the following figure.

%%%%
\begin{figure}[H]
  \begin{center}
    \includegraphics[scale=0.5]{figure/heating}
  \end{center}
  \caption{
  Heating with and without heat flux for a strain of $0.4$ and a compression time of $10 \ \unit{ms}$, $1 \ \unit{ms}$ and $0.1 \ \unit{ms}$, from left to right
  }
  \label{fig:heating}
\end{figure}

%%
% !TEX root = main.tex
%%

\clearpage
\section{The modelization of the bonding process}
\label{bondingprocess}
In this section, we model the bonding process exploited by the company to thermally bond pieces of non-woven fabric, fitting better the geometry of the steel-made rollers and their rotational velocity. 

\newcommand{\VT}{v_{\mathrm{fabric}}}
\newcommand{\Roller}{R}
\newcommand{\cratio}{r}
\newcommand{\VRot}{\omega}
\newcommand{\droller}{d_r}
\newcommand{\Tmodel}{T}
\newcommand{\Tamb}{T_{\textrm{ambient}}}

%\newcommand{\Tsteel}{T_{\mathrm{steel}}}
%\newcommand{\Tflux}{T_{\mathrm{flux}}}
%\newcommand{\Twork}{T_{\mathrm{work}}}
%\newcommand{\Ksteel}{K_{\mathrm{steel}}}
%\newcommand{\Ktissue}{K_{\mathrm{tissue}}}
%\newcommand{\Ytissue}{\kappa_{\mathrm{tissue}}}
%\newcommand{\htissue}{h_{\mathrm{tissue}}}
%\newcommand{\htissuemin}{\htissue^{\min}}
%\newcommand{\htissuemax}{\htissue^{\max}}
%\newcommand{\wtissue}{w_{\mathrm{tissue}}}
%\newcommand{\TmaxQ}{{T_{\max}^Q}}
%\newcommand{\TmaxL}{{T_{\max}^L}}
%\newcommand{\Cptissue}{C_{p,\textrm{tissue}}}
%\newcommand{\celsius}{\textrm{\textcelsius}}%^\circ C}

\begin{figure}[H]
  \begin{center}
    \includegraphics[scale=0.25]{figure/Compressione_rulli.pdf}
  \end{center}
  \caption{Compression model of two steel-made rollers}
  \label{fig:compression}
\end{figure}
%%%%

The aim of the fast compression is to reach the bounding temperature close to $150\celsius$
as noticed by other research~\cite{Hegde2008,Bhat2004} for Polypropylene.

\subsection{Determination of the rollers velocity}
%%%
Given the velocity of the non-woven fabric on the assembly line $\VT$ (about $360\,\unit{m}/\unit{min}$, that is $6\,\unit{m}/\unit{s}$) and the ray of the rollers $\Roller$ (about $0.2 \ \unit{m}$), we can obtain the angular velocity of the rolls:
%%%
\begin{equation}\label{eq:omega:def}
    \VRot = \dfrac{\VT}{\Roller} = \dfrac{6 \ [\unit{m}/\unit{s}]}{0.2 \ [\unit{m}]} = 30\left[\dfrac{\unit{rad}}{\unit{s}}\right].
\end{equation}

\subsection{Determination of the compression angle}
%%%
Looking at Figure~\ref{fig:compression} by convention, we assume that the compression starts as the non-woven fabric is compacted (that is $\htissuemin=14\,\unit{$\mu$m}$) neglecting the effects of the first compaction. Using $\droller$, the distance between the rollers, we define by
%%%
\[
  \cratio = \droller/\htissuemin
\]
%%%
the compression ratio of the fabric as it is passing through the rollers. We can compute the angle $\theta_0$ which the roller has to be subject to in order to take the non-woven fabric from thickness $\htissuemin$ to $\droller$ by solving:
%%%
\[
    \htissuemin = \droller + 2\Roller(1-\cos\theta_0) = \htissuemin \cratio + 2\Roller(1-\cos\theta_0),
\]
%%%
from which
%%%
\[
   \cos\theta_0 = 1- \dfrac{\htissuemin(1-\cratio)}{2\Roller}\qquad \textrm{or}
   \qquad
   \theta_0 = \arccos\left( 1- \dfrac{\htissuemin(1-\cratio)}{2\Roller}\right).
\]
%%%
Assuming that the angle is much small, we can Taylor approximate $\cos\theta\approx 1-\theta^2/2$:
%%%
\begin{equation}\label{eq:theta0}
   \color{blue}
   \theta_0 \approx \dfrac{\sqrt{\htissuemin}}{\sqrt{\Roller}}\sqrt{1-\cratio}.
\end{equation}
%%%
For instance, using a ratio $\cratio=0.7$ (that is the non-woven fabric is compressed at $70\%$ w.r.t. its starting size with $\Roller=0.2\,\unit{m}$ and $\htissuemin=14\,\unit{$\mu$m}$), we obtain
%%%
\[
   \theta_0 \approx \dfrac{\sqrt{14\cdot10^{-6}}}{\sqrt{0.2}}\sqrt{1-0.7}\approx 0.004583 \,\unit{rad} = 0.2625\,\unit{deg}.
\]
%%%

\subsection{Determination of the bonding time and the velocity profile}
%%%
As the compression starts with $\theta(0) = -\theta_0$, the angle is changing with the law
%%%%
\begin{equation}\label{eq:theta:t}
   \theta(t) = \VRot t - \theta_0
\end{equation}
%%%%
and the thickness of the non-woven fabric does the same as follows (by Taylor approximating $\cos \theta(t)$ for small angles)
%%%
\[
    w(t) = \htissuemin \cratio + 2\Roller(1-\cos \theta(t))\quad\Rightarrow\ \textrm{[Taylor]}\ \Rightarrow\quad
    w(t) := \htissuemin \cratio + \Roller\,\theta(t)^2,
\]
%%%
with compression velocity $-w'(t)$
%%%
\[
  -w'(t) = -2\Roller\theta(t)\theta'(t)
         = 2\Roller\left(\theta_0-\VRot t\right)\VRot > 0.
\]
%%%
The time to perform the bonding is given by $\VRot \Delta t = \theta_0$, which for $\theta_0=0.004583 \,\unit{rad}$ and $\VRot=30 \ \unit{rad}/\unit{s}$
gives the solution
%%%
\[
    \Delta t  = \dfrac{\theta_0}{\VRot} =  \dfrac{0.004583}{30} = 0.00015275\,\unit{s} =  0.15275\,\unit{ms}.
\]

\subsection{A more accurate model for heating the non-woven fabric}
%%%
By $w(t)$ and Taylor approximating the $\cos$ ($\cos\theta\approx 1-\theta^2/2$), we can obtain the strain of the fabric w.r.t. the time
$s(t)$
%%%
\begin{equation}\label{eq:s:by:time}
   s(t) = 1-\dfrac{w(t)}{\htissuemin} = 
   1-\dfrac{\htissuemin \cratio + \Roller\,\theta(t)^2}{\htissuemin}=
   \color{blue}
   1-\cratio - \dfrac{\Roller}{\htissuemin}(\VRot t - \theta_0)^2,
\end{equation}
%%%
from which 
%%%
\begin{equation}\label{eq:vel:ratio}
   v(t)=s'(t)=\color{blue}\dfrac{2\Roller\VRot}{\htissuemin}\left(\theta_0-\VRot t\right).
\end{equation}
%%%
From \eqref{ode:para}, we obtain the contribution for the heating due to the compression
%%%
\begin{equation}\label{eq:model:A}
   \begin{split}
   \dfrac{\mathrm{d} T(t)}{\mathrm{d}t}  & =  \dfrac{\mathrm{d} T(s)}{\mathrm{d}s} s'(t)  =
    \dfrac{\mathrm{d} T(s)}{\mathrm{d}s} v(t)  \\
    & =
   \dfrac{v(t)s(t)\Ytissue(s(t))}{\Cptissue\cdot \wtissue}\cdot 
   \max\left(0,\dfrac{\TmaxQ-T(t)}{\TmaxQ-\Tamb}\right)^2,
   \end{split}
\end{equation}
%%%
where $s(t)$ is given by~\eqref{eq:s:by:time} and $\Ytissue(s)$ by~\eqref{eq:Y:module:tissue}. By~\eqref{eq:T:cool:by:steel}, we have the cooling law due to the touch with the steel rolls:
%%%
\begin{equation}\label{eq:model:B}
  \dfrac{\mathrm{d}\Tmodel(t)}{\mathrm{d}t}=
  \dfrac{4\Ksteel(\Tsteel-\Tmodel(t))}{\htissuemin(1-s(t))\Cptissue\wtissue}.
\end{equation}
%%%
Combining the contributions of~\eqref{eq:model:A} and~\eqref{eq:model:B}, we have the final model
%%%
\begin{equation}\label{eq:model:t}
   \dfrac{\mathrm{d} T(t)}{\mathrm{d}t}=
   \dfrac{
    v(t)s(t)\Ytissue(s(t))\cdot 
   \max\left(0,\dfrac{\TmaxQ-T(t)}{\TmaxQ-\Tamb}\right)^2
   +
   \dfrac{4\Ksteel(\Tsteel-\Tmodel(t))}{\htissuemin(1-s(t))}
   }{\Cptissue\wtissue}.
\end{equation}
%%%
In order to better compare the solutions, it is suitable to rewrite the equations in terms of the scaled time $\tau$
%%%
\[
    t(\tau) = \tau \Delta t = \tau\dfrac{\theta_0}{\VRot}
\]
%%%
in such a way that
%%%
%%%
\begin{equation}\label{eq:time:scale:derivative}
   \dfrac{\mathrm{d} T(\tau)}{\mathrm{d}\tau}=
   \dfrac{\mathrm{d} T(t(\tau))}{\mathrm{d}t}\dfrac{\mathrm{d} t(\tau)}{\mathrm{d}\tau}=
   \dfrac{\mathrm{d} T(t)}{\mathrm{d}t}\dfrac{\theta_0}{\VRot}.
\end{equation}
%%%
Moreover, by~\eqref{eq:theta0} and~\eqref{eq:omega:def},
%%%
\begin{equation}\label{eq:time:scale}
  \Delta t = 
  \dfrac{\theta_0}{\VRot} %= \dfrac{\sqrt{\htissuemin}}{\sqrt{\Roller}}\sqrt{1-\cratio}\dfrac{\Roller}{\VT}
  =\dfrac{\sqrt{\Roller(1-r)\htissuemin}}{\VT}.
\end{equation}
%%%
By~\eqref{eq:theta:t}, \eqref{eq:s:by:time} and~\eqref{eq:vel:ratio} together with~\eqref{eq:theta0}, which gives $R/\htissuemin=(1-r)/\theta_0^2$, we have
%%%%
\begin{equation}\label{eq:st:tau}
  \begin{split}
   \theta(\tau) = & \VRot t - \theta_0 = (\tau-1)\theta_0,
   \\
   s(\tau)
   = &
   1-\cratio - \dfrac{\Roller}{\htissuemin}(\tau-1)^2\theta_0^2
   = 
   (1-\cratio)(1 - (\tau-1)^2)
   = 
   (1-\cratio)\tau(2-\tau),
   \\
   v(\tau) = &
   \dfrac{2\Roller\VRot}{\htissuemin}(1-\tau)\theta_0=
   \dfrac{2}{\theta_0}\VRot(1-r)(1-\tau),
   \end{split}
\end{equation}
%%%
from which $s(0)=0$, $s(1)=1-\cratio$ and $1-s(\tau) = 1-(1-r)\tau(2-\tau)$.
Moreover $v(0)=2\sqrt{\Roller(1-\cratio)/\htissuemin}$ and $v(1)=0$.
%%%
Using~\eqref{eq:time:scale} with~\eqref{eq:time:scale:derivative} and~\eqref{eq:model:t}, we obtain
%%%
\begin{equation}\label{eq:model:tau}
   \color{blue}
   \dfrac{\mathrm{d} T(\tau)}{\mathrm{d}\tau}=
   %\dfrac{\VT}{\sqrt{\Roller(1-r)\htissuemin}}
   \dfrac{\theta_0}{\VRot}
   \dfrac{
    v\,s\,\Ytissue(s)\cdot 
   \max\left(0,\dfrac{\TmaxQ-T(\tau)}{\TmaxQ-\Tamb}\right)^2
   +
   \dfrac{4\Ksteel(\Tsteel-\Tmodel(\tau))}{\htissuemin(1-s)}
   }{\Cptissue\wtissue}.
\end{equation}
%%%


%%%%
\begin{figure}[H]
  \begin{center}
    \includegraphics[scale=0.5]{figure/heating-model.pdf}
  \end{center}
  \caption{
  Heating for a strain of $0.4$}
  \label{fig:heating2}
\end{figure}

\subsection{Parabolic model}
\label{subsec:para}
%%%%
In order to enhance the compression of the fusing process, we consider a section (along $z$) of the two pieces of non-woven fabric which pass through the rollers (seen as a unique piece of non-woven fabric which is compressed). Neglecting the heat diffusion along the directions $x$ and $y$ (the horizontal ones), since we are assuming that the temperature gradient along such directions is small, we obtain the parabolic equation of the temperature variation (heat equation)
%%%
\begin{equation}\label{pde:heat}
 \wtissue\Cptissue \dfrac{\partial T(t,z)}{\partial t}
 = 
 \Ktissue
 \dfrac{\partial^2 T(t,z)}{\partial^2 z}+\mathcal{Q} (t,z),
\end{equation}
%%%
for $z\in (-h(t),h(t))$, where $h(t)=(1-s(t))\htissuemin$, and $s(t)$ is the strain as a function of the time given by \eqref{eq:vel:ratio}; the boundary conditions are given by
%%%
\begin{equation}\label{pde:bc}
  \begin{cases}
  T(0,z) = \Tamb, & z\in[-\htissuemin,\htissuemin] \\[1em]
  T(t,-h(t)) = T(t,h(t)) = \Tsteel, & t\in[0,\Delta t] \\[1em]
  \dfrac{\partial T(t,-h(t))}{\partial x} =
  \dfrac{\partial T(t,h(t))}{\partial x}= 0, & t\geq \Delta t
  \end{cases}
\end{equation}
%%%
where $\Delta t$ is given by \eqref{eq:time:scale}.
The production of the heat $\mathcal{Q} (t,z)$ is due to the pressure on the non-woven fabric and, as a function of the strain, is given by the r.h.s. of \eqref{eq:model:A} scaled on the thickness of the fabric $2h(t)=2(1-s(s))\htissuemin$ and supposed as homogeneous (is not depending on $z$) along the thickness:
%%%
\begin{equation}
  \mathcal{Q}(t,z)=
  \begin{cases}
  \dfrac{v(t)s(t)\Ytissue(s(t))}{2h(t)}\max\left(0,\dfrac{\TmaxQ-T(t,z)}{\TmaxQ-\Tamb}\right)^2,
  & t \leq \Delta t
  \\
  0, & t > \Delta t
  \end{cases}
\end{equation}
%%%
where, by \eqref{eq:s:by:time} and \eqref{eq:vel:ratio}, we have
%%%
\begin{equation}
  s(t) =
  \begin{cases}
  1-\cratio - \dfrac{\Roller}{\htissuemin}(\VRot t - \theta_0)^2, & t\in[0,\Delta t] \\[1em]
  1-\cratio, &  t\geq \Delta t
  \end{cases}
\end{equation}
%%%
%%%
\begin{equation}
  v(t) =
  \dfrac{\Roller}{\htissuemin}
  \begin{cases}
  \theta_0-\VRot t, & t\in[0,\Delta t] \\[1em]
  0, &  t\geq \Delta t.
  \end{cases}
\end{equation}
%%%
Changing the time in a scaled instant $t = \tau\Delta t$ and $z\in[-h(t),h(t)]$ in $[-1,1]$, i.e. $z=\zeta h(t)$, we obtain the re-scaled equation
%%%
\begin{equation}\label{pde:heat:scaled}
 h(\tau)\wtissue\Cptissue \dfrac{\partial T(\tau,\zeta)}{\partial\tau}
 = 
 \Delta t \left(\Ktissue\dfrac{\partial^2 T(\tau,\zeta)}{\partial^2 \zeta}+\overline{\mathcal{Q}}(\tau,\zeta)\right),
\end{equation}
%%%
where (using \eqref{eq:st:tau}), we have
%%%
\begin{equation}
  \begin{split}
  \overline{\mathcal{Q}}(\tau,\zeta)=&
  \begin{cases}
  \dfrac{v(\tau)s(\tau)\Ytissue(s(\tau))}{2}\max\left(0,\dfrac{\TmaxQ-T(\tau,\zeta)}{\TmaxQ-\Tamb}\right)^2,
  & \tau \leq 1 \\
  0, & \tau > 1
  \end{cases}
  \\
   s(\tau)
   = & (1-\cratio)\tau(2-\tau), \\
   v(\tau) = & \dfrac{2}{\theta_0}\VRot(1-r)(1-\tau), \\
   h(\tau) = &
   \htissuemin \left(1-(1-r)\tau(2-\tau)\right),
  \end{split}
\end{equation}
%%%
and the boundary conditions become
%%%
\begin{equation}\label{pde:bc:scaled}
  \begin{cases}
  T(0,\zeta) = \Tamb, & z\in[-1,1] \\[1em]
  T(\tau,-1) = T(\tau,1) = \Tsteel, & \tau\in[0,1] \\[1em]
  \dfrac{\partial T(\tau,-1)}{\partial \zeta} =
  \dfrac{\partial T(\tau,1)}{\partial \zeta}= 0, & \tau\geq 1.
  \end{cases}
\end{equation}
%%%
Using a spatial discretization  $\Delta\zeta=2/N$ and $\zeta_k=-1+k\Delta\zeta$, we sample the temperature along the points $\zeta_k$ with the functions $T_k(\tau)\approx T(\tau,\zeta_k)$ for which, after the discretization, we obtain an ODE system for $\tau \leq 1$
%%%
\begin{equation}
  \left\{
  \begin{aligned}
    T_0'(\tau)
    =\,& 0,
    \\
    T_k'(\tau)
    =\,& \Delta t\dfrac{
    \big(\Ktissue/(h(\tau)\Delta\zeta^2)\big) 
    \big(T_{k-1}(\tau)-2T_k(\tau)+T_{k+1}(\tau)\big)
    +\overline{\mathcal{Q}}_k(\tau)
    }{\wtissue\Cptissue},
    \\
    T_N'(\tau)
    =\,& 0,
    \\
    %%%
    \overline{\mathcal{Q}}_k(\tau)=&
    \dfrac{v(\tau)s(\tau)\Ytissue(s(\tau))}{2}
    \max\left(0,\dfrac{\TmaxQ-T_k(\tau)}{\TmaxQ-\Tamb}\right)^2,
  \end{aligned}
  \right.
\end{equation}
%%%
and one for $\tau\geq 0$
%%%
\begin{equation}
  \left\{
  \begin{aligned}
    T_0'(\tau)
    =& 
    \dfrac{\Delta t\Ktissue}{\Delta\zeta^2h(\tau)\wtissue\Cptissue}
    \big(T_1(\tau)-T_0(\tau)\big),
    \\
    T_k'(\tau)
    =& 
    \dfrac{\Delta t\Ktissue}{\Delta\zeta^2h(\tau)\wtissue\Cptissue}
    \big(T_{k-1}(\tau)-2T_k(\tau)+T_{k+1}(\tau)\big),
    \\
    T_N'(\tau)
    =&
    \dfrac{\Delta t\Ktissue}{\Delta\zeta^2h(\tau)\wtissue\Cptissue}
    \big(T_{N-1}(\tau)-T_N(\tau)\big).
  \end{aligned}
  \right.
\end{equation}
%%%

\begin{figure}[H]
  \begin{center}
    \includegraphics[scale=0.20]{figure/para1-1.pdf}
    \includegraphics[scale=0.20]{figure/para1-2.pdf}
    \includegraphics[scale=0.20]{figure/para1-3.pdf}
  \end{center}
  \caption{Solution with the parabolic model and standard parameters:
  $\Ksteel=17\,\unit{W}/(\unit{m K})$, $\VT=6\,\unit{m/s}$ and $r=0.8$.
  Estimated outgoing homogeneous temperature of $25-55 \ \unit{\celsius}$.}
  \label{fig:para1}
\end{figure}
%%%%

\begin{figure}[H]
  \begin{center}
    \includegraphics[scale=0.20]{figure/para2-1.pdf}
    \includegraphics[scale=0.20]{figure/para2-2.pdf}
    \includegraphics[scale=0.20]{figure/para2-3.pdf}
  \end{center}
  \caption{Solution with the parabolic model and slow scrolling:
  $\Ksteel=17\,\unit{W}/(\unit{m K})$, $\VT=0.6\,\unit{m/s}$ and $r=0.8$}
  \label{fig:para2}
\end{figure}
%%%%

\begin{figure}[H]
  \begin{center}
    \includegraphics[scale=0.20]{figure/para3-1.pdf}
    \includegraphics[scale=0.20]{figure/para3-2.pdf}
    \includegraphics[scale=0.20]{figure/para3-3.pdf}
  \end{center}
  \caption{Solution with the parabolic model with lower compression:
  $\Ksteel=17\,\unit{W}/(\unit{m K})$, $\VT=6\,\unit{m/s}$ and  $r=0.95$.}
  \label{fig:para3}
\end{figure}
%%%%

%%%
%\begin{equation}\label{eq:Y:module:tissue}
%  \Ytissue(s) 
%  = \dfrac{16}{1-2s}\quad [\unit{MPa}]
%  = \dfrac{16\cdot 10^6}{1-2s}\quad [\unit{Pa}]
%\end{equation}





%\input{Parameters of the model}

%Misure fatte a trento e dati da database
%\begin{itemize}
 %  \item Spettro di Riccardo -> atattico
  % \item Peso e calore specifico (Riccardo)
  %\item Curva pressione spostamento
  %\item Curva Young temperatura (misura fatta fino a 80 Gradi) estrapolazione linere o parabolica.
%\end{itemize}


%\section{Descrizione del processo di bonding}

%\begin{itemize}
 %  \item Modello di bonding con 2 rulli.
  % \item Processo modellato come compressione veloce a velocità variabile
   %\item Modello parabolico 1D della sezione che passa tra i rulli 
   %\item Si assume e si verifica a posteriori che il processo si puo modellare
    %     con equaizione del calore con temperatura costante al bordo
   %\item Parametri de modello del processo: raggi rulli, velocita, distanza, legge velocita e temperature  etc. 
%\end{itemize}


%\section{Risultati numerici}
%\begin{itemize}
 %  \item Figure curve temperature
  % \item Discussione che i parametri calcolati sono compatibili con i parametri di fabbrica
%\end{itemize}


%\section{Prova referenze}
%~\cite{zemk}~\cite{Higham,Hairer}~\cite{bunc}~\cite{zemk,Hairer,cgol,student}~\cite{govl}~\cite{Cotronei}


%\section*{Acknowledgements}


\baselineskip=0.9\normalbaselineskip
\phantomsection\addcontentsline{toc}{section}{\numberline{}References}

\bibliographystyle{CAIM_Sciendo_bibstyle}

\bibliography{main}

\end{document}
