%%
% !TEX root = main.tex
%%

\clearpage
\section{The modelization of the bonding process}
\label{bondingprocess}
In this section, we model the bonding process exploited by the company to thermally bond pieces of non-woven fabric, fitting better the geometry of the steel-made rollers and their rotational velocity. 

\newcommand{\VT}{v_{\mathrm{fabric}}}
\newcommand{\Roller}{R}
\newcommand{\cratio}{r}
\newcommand{\VRot}{\omega}
\newcommand{\droller}{d_r}
\newcommand{\Tmodel}{T}
\newcommand{\Tamb}{T_{\textrm{ambient}}}

%\newcommand{\Tsteel}{T_{\mathrm{steel}}}
%\newcommand{\Tflux}{T_{\mathrm{flux}}}
%\newcommand{\Twork}{T_{\mathrm{work}}}
%\newcommand{\Ksteel}{K_{\mathrm{steel}}}
%\newcommand{\Ktissue}{K_{\mathrm{tissue}}}
%\newcommand{\Ytissue}{\kappa_{\mathrm{tissue}}}
%\newcommand{\htissue}{h_{\mathrm{tissue}}}
%\newcommand{\htissuemin}{\htissue^{\min}}
%\newcommand{\htissuemax}{\htissue^{\max}}
%\newcommand{\wtissue}{w_{\mathrm{tissue}}}
%\newcommand{\TmaxQ}{{T_{\max}^Q}}
%\newcommand{\TmaxL}{{T_{\max}^L}}
%\newcommand{\Cptissue}{C_{p,\textrm{tissue}}}
%\newcommand{\celsius}{\textrm{\textcelsius}}%^\circ C}

\begin{figure}[H]
  \begin{center}
    \includegraphics[scale=0.25]{figure/Compressione_rulli.pdf}
  \end{center}
  \caption{Compression model of two steel-made rollers}
  \label{fig:compression}
\end{figure}
%%%%

The aim of the fast compression is to reach the bounding temperature close to $150\celsius$
as noticed by other research~\cite{Hegde2008,Bhat2004} for Polypropylene.

\subsection{Determination of the rollers velocity}
%%%
Given the velocity of the non-woven fabric on the assembly line $\VT$ (about $360\,\unit{m}/\unit{min}$, that is $6\,\unit{m}/\unit{s}$) and the ray of the rollers $\Roller$ (about $0.2 \ \unit{m}$), we can obtain the angular velocity of the rolls:
%%%
\begin{equation}\label{eq:omega:def}
    \VRot = \dfrac{\VT}{\Roller} = \dfrac{6 \ [\unit{m}/\unit{s}]}{0.2 \ [\unit{m}]} = 30\left[\dfrac{\unit{rad}}{\unit{s}}\right].
\end{equation}

\subsection{Determination of the compression angle}
%%%
Looking at Figure~\ref{fig:compression} by convention, we assume that the compression starts as the non-woven fabric is compacted (that is $\htissuemin=14\,\unit{$\mu$m}$) neglecting the effects of the first compaction. Using $\droller$, the distance between the rollers, we define by
%%%
\[
  \cratio = \droller/\htissuemin
\]
%%%
the compression ratio of the fabric as it is passing through the rollers. We can compute the angle $\theta_0$ which the roller has to be subject to in order to take the non-woven fabric from thickness $\htissuemin$ to $\droller$ by solving:
%%%
\[
    \htissuemin = \droller + 2\Roller(1-\cos\theta_0) = \htissuemin \cratio + 2\Roller(1-\cos\theta_0),
\]
%%%
from which
%%%
\[
   \cos\theta_0 = 1- \dfrac{\htissuemin(1-\cratio)}{2\Roller}\qquad \textrm{or}
   \qquad
   \theta_0 = \arccos\left( 1- \dfrac{\htissuemin(1-\cratio)}{2\Roller}\right).
\]
%%%
Assuming that the angle is much small, we can Taylor approximate $\cos\theta\approx 1-\theta^2/2$:
%%%
\begin{equation}\label{eq:theta0}
   \color{blue}
   \theta_0 \approx \dfrac{\sqrt{\htissuemin}}{\sqrt{\Roller}}\sqrt{1-\cratio}.
\end{equation}
%%%
For instance, using a ratio $\cratio=0.7$ (that is the non-woven fabric is compressed at $70\%$ w.r.t. its starting size with $\Roller=0.2\,\unit{m}$ and $\htissuemin=14\,\unit{$\mu$m}$), we obtain
%%%
\[
   \theta_0 \approx \dfrac{\sqrt{14\cdot10^{-6}}}{\sqrt{0.2}}\sqrt{1-0.7}\approx 0.004583 \,\unit{rad} = 0.2625\,\unit{deg}.
\]
%%%

\subsection{Determination of the bonding time and the velocity profile}
%%%
As the compression starts with $\theta(0) = -\theta_0$, the angle is changing with the law
%%%%
\begin{equation}\label{eq:theta:t}
   \theta(t) = \VRot t - \theta_0
\end{equation}
%%%%
and the thickness of the non-woven fabric does the same as follows (by Taylor approximating $\cos \theta(t)$ for small angles)
%%%
\[
    w(t) = \htissuemin \cratio + 2\Roller(1-\cos \theta(t))\quad\Rightarrow\ \textrm{[Taylor]}\ \Rightarrow\quad
    w(t) := \htissuemin \cratio + \Roller\,\theta(t)^2,
\]
%%%
with compression velocity $-w'(t)$
%%%
\[
  -w'(t) = -2\Roller\theta(t)\theta'(t)
         = 2\Roller\left(\theta_0-\VRot t\right)\VRot > 0.
\]
%%%
The time to perform the bonding is given by $\VRot \Delta t = \theta_0$, which for $\theta_0=0.004583 \,\unit{rad}$ and $\VRot=30 \ \unit{rad}/\unit{s}$
gives the solution
%%%
\[
    \Delta t  = \dfrac{\theta_0}{\VRot} =  \dfrac{0.004583}{30} = 0.00015275\,\unit{s} =  0.15275\,\unit{ms}.
\]

\subsection{A more accurate model for heating the non-woven fabric}
%%%
By $w(t)$ and Taylor approximating the $\cos$ ($\cos\theta\approx 1-\theta^2/2$), we can obtain the strain of the fabric w.r.t. the time
$s(t)$
%%%
\begin{equation}\label{eq:s:by:time}
   s(t) = 1-\dfrac{w(t)}{\htissuemin} = 
   1-\dfrac{\htissuemin \cratio + \Roller\,\theta(t)^2}{\htissuemin}=
   \color{blue}
   1-\cratio - \dfrac{\Roller}{\htissuemin}(\VRot t - \theta_0)^2,
\end{equation}
%%%
from which 
%%%
\begin{equation}\label{eq:vel:ratio}
   v(t)=s'(t)=\color{blue}\dfrac{2\Roller\VRot}{\htissuemin}\left(\theta_0-\VRot t\right).
\end{equation}
%%%
From \eqref{ode:para}, we obtain the contribution for the heating due to the compression
%%%
\begin{equation}\label{eq:model:A}
   \begin{split}
   \dfrac{\mathrm{d} T(t)}{\mathrm{d}t}  & =  \dfrac{\mathrm{d} T(s)}{\mathrm{d}s} s'(t)  =
    \dfrac{\mathrm{d} T(s)}{\mathrm{d}s} v(t)  \\
    & =
   \dfrac{v(t)s(t)\Ytissue(s(t))}{\Cptissue\cdot \wtissue}\cdot 
   \max\left(0,\dfrac{\TmaxQ-T(t)}{\TmaxQ-\Tamb}\right)^2,
   \end{split}
\end{equation}
%%%
where $s(t)$ is given by~\eqref{eq:s:by:time} and $\Ytissue(s)$ by~\eqref{eq:Y:module:tissue}. By~\eqref{eq:T:cool:by:steel}, we have the cooling law due to the touch with the steel rolls:
%%%
\begin{equation}\label{eq:model:B}
  \dfrac{\mathrm{d}\Tmodel(t)}{\mathrm{d}t}=
  \dfrac{4\Ksteel(\Tsteel-\Tmodel(t))}{\htissuemin(1-s(t))\Cptissue\wtissue}.
\end{equation}
%%%
Combining the contributions of~\eqref{eq:model:A} and~\eqref{eq:model:B}, we have the final model
%%%
\begin{equation}\label{eq:model:t}
   \dfrac{\mathrm{d} T(t)}{\mathrm{d}t}=
   \dfrac{
    v(t)s(t)\Ytissue(s(t))\cdot 
   \max\left(0,\dfrac{\TmaxQ-T(t)}{\TmaxQ-\Tamb}\right)^2
   +
   \dfrac{4\Ksteel(\Tsteel-\Tmodel(t))}{\htissuemin(1-s(t))}
   }{\Cptissue\wtissue}.
\end{equation}
%%%
In order to better compare the solutions, it is suitable to rewrite the equations in terms of the scaled time $\tau$
%%%
\[
    t(\tau) = \tau \Delta t = \tau\dfrac{\theta_0}{\VRot}
\]
%%%
in such a way that
%%%
%%%
\begin{equation}\label{eq:time:scale:derivative}
   \dfrac{\mathrm{d} T(\tau)}{\mathrm{d}\tau}=
   \dfrac{\mathrm{d} T(t(\tau))}{\mathrm{d}t}\dfrac{\mathrm{d} t(\tau)}{\mathrm{d}\tau}=
   \dfrac{\mathrm{d} T(t)}{\mathrm{d}t}\dfrac{\theta_0}{\VRot}.
\end{equation}
%%%
Moreover, by~\eqref{eq:theta0} and~\eqref{eq:omega:def},
%%%
\begin{equation}\label{eq:time:scale}
  \Delta t = 
  \dfrac{\theta_0}{\VRot} %= \dfrac{\sqrt{\htissuemin}}{\sqrt{\Roller}}\sqrt{1-\cratio}\dfrac{\Roller}{\VT}
  =\dfrac{\sqrt{\Roller(1-r)\htissuemin}}{\VT}.
\end{equation}
%%%
By~\eqref{eq:theta:t}, \eqref{eq:s:by:time} and~\eqref{eq:vel:ratio} together with~\eqref{eq:theta0}, which gives $R/\htissuemin=(1-r)/\theta_0^2$, we have
%%%%
\begin{equation}\label{eq:st:tau}
  \begin{split}
   \theta(\tau) = & \VRot t - \theta_0 = (\tau-1)\theta_0,
   \\
   s(\tau)
   = &
   1-\cratio - \dfrac{\Roller}{\htissuemin}(\tau-1)^2\theta_0^2
   = 
   (1-\cratio)(1 - (\tau-1)^2)
   = 
   (1-\cratio)\tau(2-\tau),
   \\
   v(\tau) = &
   \dfrac{2\Roller\VRot}{\htissuemin}(1-\tau)\theta_0=
   \dfrac{2}{\theta_0}\VRot(1-r)(1-\tau),
   \end{split}
\end{equation}
%%%
from which $s(0)=0$, $s(1)=1-\cratio$ and $1-s(\tau) = 1-(1-r)\tau(2-\tau)$.
Moreover $v(0)=2\sqrt{\Roller(1-\cratio)/\htissuemin}$ and $v(1)=0$.
%%%
Using~\eqref{eq:time:scale} with~\eqref{eq:time:scale:derivative} and~\eqref{eq:model:t}, we obtain
%%%
\begin{equation}\label{eq:model:tau}
   \color{blue}
   \dfrac{\mathrm{d} T(\tau)}{\mathrm{d}\tau}=
   %\dfrac{\VT}{\sqrt{\Roller(1-r)\htissuemin}}
   \dfrac{\theta_0}{\VRot}
   \dfrac{
    v\,s\,\Ytissue(s)\cdot 
   \max\left(0,\dfrac{\TmaxQ-T(\tau)}{\TmaxQ-\Tamb}\right)^2
   +
   \dfrac{4\Ksteel(\Tsteel-\Tmodel(\tau))}{\htissuemin(1-s)}
   }{\Cptissue\wtissue}.
\end{equation}
%%%


%%%%
\begin{figure}[H]
  \begin{center}
    \includegraphics[scale=0.5]{figure/heating-model.pdf}
  \end{center}
  \caption{
  Heating for a strain of $0.4$}
  \label{fig:heating2}
\end{figure}

\subsection{Parabolic model}
\label{subsec:para}
%%%%
In order to enhance the compression of the fusing process, we consider a section (along $z$) of the two pieces of non-woven fabric which pass through the rollers (seen as a unique piece of non-woven fabric which is compressed). Neglecting the heat diffusion along the directions $x$ and $y$ (the horizontal ones), since we are assuming that the temperature gradient along such directions is small, we obtain the parabolic equation of the temperature variation (heat equation)
%%%
\begin{equation}\label{pde:heat}
 \wtissue\Cptissue \dfrac{\partial T(t,z)}{\partial t}
 = 
 \Ktissue
 \dfrac{\partial^2 T(t,z)}{\partial^2 z}+\mathcal{Q} (t,z),
\end{equation}
%%%
for $z\in (-h(t),h(t))$, where $h(t)=(1-s(t))\htissuemin$, and $s(t)$ is the strain as a function of the time given by \eqref{eq:vel:ratio}; the boundary conditions are given by
%%%
\begin{equation}\label{pde:bc}
  \begin{cases}
  T(0,z) = \Tamb, & z\in[-\htissuemin,\htissuemin] \\[1em]
  T(t,-h(t)) = T(t,h(t)) = \Tsteel, & t\in[0,\Delta t] \\[1em]
  \dfrac{\partial T(t,-h(t))}{\partial x} =
  \dfrac{\partial T(t,h(t))}{\partial x}= 0, & t\geq \Delta t
  \end{cases}
\end{equation}
%%%
where $\Delta t$ is given by \eqref{eq:time:scale}.
The production of the heat $\mathcal{Q} (t,z)$ is due to the pressure on the non-woven fabric and, as a function of the strain, is given by the r.h.s. of \eqref{eq:model:A} scaled on the thickness of the fabric $2h(t)=2(1-s(s))\htissuemin$ and supposed as homogeneous (is not depending on $z$) along the thickness:
%%%
\begin{equation}
  \mathcal{Q}(t,z)=
  \begin{cases}
  \dfrac{v(t)s(t)\Ytissue(s(t))}{2h(t)}\max\left(0,\dfrac{\TmaxQ-T(t,z)}{\TmaxQ-\Tamb}\right)^2,
  & t \leq \Delta t
  \\
  0, & t > \Delta t
  \end{cases}
\end{equation}
%%%
where, by \eqref{eq:s:by:time} and \eqref{eq:vel:ratio}, we have
%%%
\begin{equation}
  s(t) =
  \begin{cases}
  1-\cratio - \dfrac{\Roller}{\htissuemin}(\VRot t - \theta_0)^2, & t\in[0,\Delta t] \\[1em]
  1-\cratio, &  t\geq \Delta t
  \end{cases}
\end{equation}
%%%
%%%
\begin{equation}
  v(t) =
  \dfrac{\Roller}{\htissuemin}
  \begin{cases}
  \theta_0-\VRot t, & t\in[0,\Delta t] \\[1em]
  0, &  t\geq \Delta t.
  \end{cases}
\end{equation}
%%%
Changing the time in a scaled instant $t = \tau\Delta t$ and $z\in[-h(t),h(t)]$ in $[-1,1]$, i.e. $z=\zeta h(t)$, we obtain the re-scaled equation
%%%
\begin{equation}\label{pde:heat:scaled}
 h(\tau)\wtissue\Cptissue \dfrac{\partial T(\tau,\zeta)}{\partial\tau}
 = 
 \Delta t \left(\Ktissue\dfrac{\partial^2 T(\tau,\zeta)}{\partial^2 \zeta}+\overline{\mathcal{Q}}(\tau,\zeta)\right),
\end{equation}
%%%
where (using \eqref{eq:st:tau}), we have
%%%
\begin{equation}
  \begin{split}
  \overline{\mathcal{Q}}(\tau,\zeta)=&
  \begin{cases}
  \dfrac{v(\tau)s(\tau)\Ytissue(s(\tau))}{2}\max\left(0,\dfrac{\TmaxQ-T(\tau,\zeta)}{\TmaxQ-\Tamb}\right)^2,
  & \tau \leq 1 \\
  0, & \tau > 1
  \end{cases}
  \\
   s(\tau)
   = & (1-\cratio)\tau(2-\tau), \\
   v(\tau) = & \dfrac{2}{\theta_0}\VRot(1-r)(1-\tau), \\
   h(\tau) = &
   \htissuemin \left(1-(1-r)\tau(2-\tau)\right),
  \end{split}
\end{equation}
%%%
and the boundary conditions become
%%%
\begin{equation}\label{pde:bc:scaled}
  \begin{cases}
  T(0,\zeta) = \Tamb, & z\in[-1,1] \\[1em]
  T(\tau,-1) = T(\tau,1) = \Tsteel, & \tau\in[0,1] \\[1em]
  \dfrac{\partial T(\tau,-1)}{\partial \zeta} =
  \dfrac{\partial T(\tau,1)}{\partial \zeta}= 0, & \tau\geq 1.
  \end{cases}
\end{equation}
%%%
Using a spatial discretization  $\Delta\zeta=2/N$ and $\zeta_k=-1+k\Delta\zeta$, we sample the temperature along the points $\zeta_k$ with the functions $T_k(\tau)\approx T(\tau,\zeta_k)$ for which, after the discretization, we obtain an ODE system for $\tau \leq 1$
%%%
\begin{equation}
  \left\{
  \begin{aligned}
    T_0'(\tau)
    =\,& 0,
    \\
    T_k'(\tau)
    =\,& \Delta t\dfrac{
    \big(\Ktissue/(h(\tau)\Delta\zeta^2)\big) 
    \big(T_{k-1}(\tau)-2T_k(\tau)+T_{k+1}(\tau)\big)
    +\overline{\mathcal{Q}}_k(\tau)
    }{\wtissue\Cptissue},
    \\
    T_N'(\tau)
    =\,& 0,
    \\
    %%%
    \overline{\mathcal{Q}}_k(\tau)=&
    \dfrac{v(\tau)s(\tau)\Ytissue(s(\tau))}{2}
    \max\left(0,\dfrac{\TmaxQ-T_k(\tau)}{\TmaxQ-\Tamb}\right)^2,
  \end{aligned}
  \right.
\end{equation}
%%%
and one for $\tau\geq 0$
%%%
\begin{equation}
  \left\{
  \begin{aligned}
    T_0'(\tau)
    =& 
    \dfrac{\Delta t\Ktissue}{\Delta\zeta^2h(\tau)\wtissue\Cptissue}
    \big(T_1(\tau)-T_0(\tau)\big),
    \\
    T_k'(\tau)
    =& 
    \dfrac{\Delta t\Ktissue}{\Delta\zeta^2h(\tau)\wtissue\Cptissue}
    \big(T_{k-1}(\tau)-2T_k(\tau)+T_{k+1}(\tau)\big),
    \\
    T_N'(\tau)
    =&
    \dfrac{\Delta t\Ktissue}{\Delta\zeta^2h(\tau)\wtissue\Cptissue}
    \big(T_{N-1}(\tau)-T_N(\tau)\big).
  \end{aligned}
  \right.
\end{equation}
%%%

\begin{figure}[H]
  \begin{center}
    \includegraphics[scale=0.20]{figure/para1-1.pdf}
    \includegraphics[scale=0.20]{figure/para1-2.pdf}
    \includegraphics[scale=0.20]{figure/para1-3.pdf}
  \end{center}
  \caption{Solution with the parabolic model and standard parameters:
  $\Ksteel=17\,\unit{W}/(\unit{m K})$, $\VT=6\,\unit{m/s}$ and $r=0.8$.
  Estimated outgoing homogeneous temperature of $25-55 \ \unit{\celsius}$.}
  \label{fig:para1}
\end{figure}
%%%%

\begin{figure}[H]
  \begin{center}
    \includegraphics[scale=0.20]{figure/para2-1.pdf}
    \includegraphics[scale=0.20]{figure/para2-2.pdf}
    \includegraphics[scale=0.20]{figure/para2-3.pdf}
  \end{center}
  \caption{Solution with the parabolic model and slow scrolling:
  $\Ksteel=17\,\unit{W}/(\unit{m K})$, $\VT=0.6\,\unit{m/s}$ and $r=0.8$}
  \label{fig:para2}
\end{figure}
%%%%

\begin{figure}[H]
  \begin{center}
    \includegraphics[scale=0.20]{figure/para3-1.pdf}
    \includegraphics[scale=0.20]{figure/para3-2.pdf}
    \includegraphics[scale=0.20]{figure/para3-3.pdf}
  \end{center}
  \caption{Solution with the parabolic model with lower compression:
  $\Ksteel=17\,\unit{W}/(\unit{m K})$, $\VT=6\,\unit{m/s}$ and  $r=0.95$.}
  \label{fig:para3}
\end{figure}
%%%%

%%%
%\begin{equation}\label{eq:Y:module:tissue}
%  \Ytissue(s) 
%  = \dfrac{16}{1-2s}\quad [\unit{MPa}]
%  = \dfrac{16\cdot 10^6}{1-2s}\quad [\unit{Pa}]
%\end{equation}




