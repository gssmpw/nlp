\section{Introduction}
A non-woven fabric is a manufactured material made from polymer fibers which are bonded together through industrial processes that usually involve pressure and heating effects. There are several purposes for the use of nonwoven fabrics, such as medical fabrics, filters, or geotextile ones. Among them, we can also find the use for the production of diaper linings. The paper is indeed concerned with the modelization of a bonding process for a non-woven fabric used for the manufacturing of diapers. In particular, the problem was posed by the Italian company Fater S.p.A, specialized in the market of absorbent products, with the aim of a better understanding of their industrial bonding process for producing diapers, in the event that the nonwoven fabric is provided with different physical and mechanical parameters. Such a bonding process performed by the company does not require the use of some kind of glue but involves just the pressure of two fabric webs passing through two high-velocity steel-made rollers that compress them. In this way, an instantaneous heat diffusion and local fusing effect take place and the two webs of non-woven fabric will get glued. The mechanics behind this kind of bonding process is also partly shared by the bonding process used to produce a non-woven fabric; see \cite{Patent1, Patent2} for more details. Our purpose is then to model the bonding process, both from an analytical and numerical point of view, as a function of the compression and the velocity of the rollers. To this aim, two different models for the heat diffusion were studied, a linear and a parabolic one. The first is less accurate and involves just ordinary linear differential equations for temperature (see \cite{Cengel}), whereas the second aims to enhance the compression of the fusing process by exploiting the heat equation (see \cite{Formaggia}).
\par
As regards the experimental part, the use of laboratories and equipments, we made use of the expertise of some of the authors who have already worked on similar models as in \cite{Pegoretti}. See also the references therein for the industrial importance of such materials.
\par
This allows us to study the behavior of the resulting diaper lining even in the case some parameters like thickness are different. Our investigation is detailed as in the following sections.
%\par
%In Section \ref{fibers}, five possible mechanical models for a single fiber are described. Inspired by the relative literature, such models are given by a combination of simple rheological elements as viscous pistons and springs. For each of them, the corresponding evolution is modelized and analyzed. In particular the fifth model seems to be more suitable. 
%\par
%In Section \ref{configfibers}, the construction of models representing single fibers put in-parallel as well as in-series is described and analyzed.
%\par
%In Section \ref{mainmodel} the process for the fabrication of the non-woven material, which is also similar to the process of bonding, is considered. In particular, such a process consists in the punching of a film of fibers in some points realizing some more rigid spots giving a sort of frame to the fabric. {\it In the sequel, such rigid punched spots will be called tags}. In this chapter, we construct and analytically study a model of combinations of single fibers and tags representing the feasible configurations in view of the balance of internal forces and the applied external forces.
%\par
%In Section \ref{commentslit}, some comments on the relative technical literature, in particular on the unit of measure of constants and parameters involved in the models, are reported.
%\par
%In Section \ref{fastgeometric}, the model described and analytically studied in Section \ref{mainmodel} is faced from a numerical and computational point of view. In particular the numerical approach to the crucial point of the polytopic tags cutting the fibers is deeply analyzed.
%\par
%In Section \ref{laboratoryexp}, we report and comment several laboratory experiments concerning the strain-stress curves up to the breaking point for different situations: single material, bonded material, various velocities and angles of traction. A description of how to perform such experiments in a possible rigorous way is also reported.
\par
In Section \ref{description_nonwoven_fabric}, we briefly describe the non-woven fabric, the way it is usually produced and the intended use. Some pictures (made by an electronic microscope) of the non-woven fabric used by the company are reported. In particular, such pictures clearly enlighten the fact that the bonding process exploited for the production of the fabric presents an evident thermal effect. The following chapters will then study from an analytical, mechanical and numerical point of view a new model taking into account those thermal effects.
\par
In Section \ref{polypropylene}, after our laboratory experiments, we get that the thermal effect, i.e. the temperature increment of the non-woven fabric, is due to the almost instantaneous compression process. We then collect other experimental results as well as data from literature in order to precisely fit the nature of the fabric and estimate the involved parameters, such as specific heat, Young's modulus, etc. Thereafter, some analytical models inspired by the so-called heat equation are formulated and studied.
\par
In Section \ref{bondingprocess}, we further specify the analytical model in Section \ref{polypropylene}, making it more suitable for the real process of compression and bonding given by the passage of the fabric webs through the rollers. Then a one-dimensional and a bi-dimensional numerical model and the relative numerical results are reported. In particular, such results clearly show the patterns of effective thermally bonding as function of the compression of the fabric and of the velocity of the assembly line.
\par
%In Section \ref{parametersmodel}, with a particular reference to results of sections \ref{polypropylene} and \ref{bondingprocess}, we show a useful and detailed protocol of operations to be performed on site in view of managing new non-woven fabrics with different thicknesses.