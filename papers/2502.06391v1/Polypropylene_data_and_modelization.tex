%%
% !TEX root = main.tex
%%

\section{Polypropylene data and modelization}
\label{polypropylene}
The aim of this section is to compute the temperature increment and the temperature decay of the non-woven fabric through a suitable compression rate. We will provide two models, a linear and a quadratic one. To do that, we will need some data of the fabric, polypropylene and steel, which we will recover a little from the literature and a little from laboratories experiments.
\subsection{Tacticity and some data from literature}
In order to establish which is the polypropylene's type (atactic, isotactic, syndiotactic) of the non-woven fabric, we performed a spectrum analysis, made with a Fourier transformed infrared (FT-IR) spectrophotometer, using a Avatar 330 by the Thermo Fisher company. The spectrum, as in figure \eqref{fig:spettro}, is the result of $64$ scans with a resolution of $4\ \unit{cm}^{-1}$, and turned out that the polypropylene is atactic.

%Fourier transformed infrared (FT-IR)spectroscopy was conducted by
%using a (nome modello non lo so non ho fatto io la misura) and operating
%in a wavenumber range $650–4000\ \unit{cm}^-1$

%%%%
\begin{figure}[H]
  \begin{center}
    \includegraphics[scale=0.2]{figure/atattico-banca-dati.png}
  \end{center}
  \caption{Spectrum comparison of our non-woven fabric with a sample atactic polypropylene.}
  \label{fig:spettro}
\end{figure}

For the description and the numerical modelization of the bonding process, we need some polyprolylene data, that we collect in the following Table \ref{tab:polipropilene} (see \cite{Passaglia, Maier, Young,Gianotti1968}):
%%%
\begin{table}[ht]
\centering
  \caption{Physical parameters of polypropylene}
   \label{tab:polipropilene}
  \begin{tabular}{llll}
    Parameter     & Value & Unit \\
    \hline
    Melting point        & $130$--$171$   & $\unit{\celsius}$        & measured $160$ $\celsius$ \\
    Density              & $855$--$946$   & $\unit{kg}/\unit{m}^3$   & atactic: $866$ \\
    %molar Heat Capacity  & $88$--$90$     & $\unit{J}/\unit{mol K}$  & \\
    Thermal conductivity & $0.17$--$0.22$  & $\unit{W}/(\unit{m K})$  &  at $23$ $\celsius$ typ $0.17$ \\
    Specific heat        & $581.97$--$2884.71$ & $\unit{J}/(\unit{kg K})$ & temp. in K \\
    %Tensile modulus      & $1200$         & $\unit{Mpa}$               & \\
    %Flexural modulus     & $1150$         & $\unit{Mpa}$               & \\
    Young's module         & $1300$--$1800$ & $\unit{N}/\unit{mm}^2=\unit{MPa}$               & \\
    %Elongation           & $450$          & \textrm{\%}                  & \\
    %Vicat softening temperature & $90$--$150$    & $\celsius$             & \\
    \hline
  \end{tabular}
\end{table}
%%%
Due to the high variability of the fusion temperature $130$--$171 \ \unit{\celsius}$, an accurate measurement is performed at $160\ \unit{\celsius}$. In order to detect that, polypropylene is heated slowly with a constant temperature rate. The heat flux to the polypropylene is measured and, when a negative peak of heat is detected, it means that polypropylene is changing phase from solid to liquid.
%%%%
\begin{figure}[H]
  \begin{center}
    \includegraphics[scale=0.4]{figure/Fusion.pdf}
    %\includegraphics[scale=0.25]{figure/campione.png}
    %\includegraphics[scale=0.05]{figure/measure_fusion.png}
  \end{center}
  \caption{Temperature and heat absorption: the negative peak is the fusion point of the polypropylene, here about $160$ $\unit{\celsius}$.}
  \label{fig:fusion}
\end{figure}
%%%

Since the rollers which thermally bond the pieces of non-woven fabric are steel-made, similarly to the polypropylene, we need some physical parameters of the steel, that we collect in the following Table~\ref{tab:acciaio} (see \cite{steel3, steel4, steel2, steel1}):

\begin{table}[H]
  \caption{Physical parameters of steel}\label{tab:acciaio}
  \label{tab:steel}
  \begin{center}
  \begin{tabular}{lll}
    Parameter     & Value & Unit \\
    \hline
    Melting point               & $1400$--$1530$ & $\unit{\celsius}$                   \\
    Density                     & $7500$--$8000$ & $\unit{kg}/\unit{m}^3$   \\
    Thermal conductivity (stainless) & $15$--$18$     & $\unit{W}/(\unit{m K})$  \\
    Thermal conductivity        & $44$--$80$     & $\unit{W}/(\unit{m K})$  \\
    Specific heat               & $500$          & $\unit{J}/(\unit{kg K})$ \\
    %Tensile modulus             & $1200$         & $\unit{Mpa}$               \\
    %Flexural modulus            & $1150$         & $\unit{Mpa}$               \\
    Young's modulus              & $180000$       & $\unit{N}/\unit{mm}^2=\unit{MPa}$ \\
    %Elongation                 & $450$          & \textrm{\%}                  \\
    \hline
  \end{tabular}
  \end{center}
\end{table}

%Vicat softening temperature is the temperature at which the specimen is penetrated to a depth of $1 \unit{mm}$
%by a flat-ended needle with a $1 \textrm{mm}^2$ circular or square cross-section.
%For the Vicat A test, a load of $10\unit{N}$ is used. For the Vicat B test, the load is $50\unit{N}$.

%reference \url{https://material-properties.org/polypropylene-density-strength-%melting-point-thermal-conductivity/}
%\url{https://polymerdatabase.com/polymers/polypropylene.html}
%\url{https://www.m-ep.co.jp/en/pdf/product/iupi_nova/physicality_04.pdf}
%\url{https://www.professionalplastics.com/professionalplastics/%ThermalPropertiesofPlasticMaterials.pdf}
%\url{https://designerdata.nl/materials/plastics/thermo-plastics/polypropylene-%(cop.)}
%\url{https://scipoly.com/density-of-polymers-by-density/}
%\url{https://www.centroinox.it/sites/default/files/pubblicazioni/245A.pdf}

%%%%
%\begin{figure}[!htb]
%  \begin{center}
%    \includegraphics[scale=0.35]{forza-spostamento}
%  \end{center}
%  \caption{Displacement vs force}
%  \label{fig:forza-spostamento}
%\end{figure}
%%%%

\subsection{Determination of fabric weight by unit area}
\label{subsec:weight}
%%%
In this section, we report the density of our non-woven fabric, determined by a $15\ \unit{cm} \times 15\ \unit{cm}$ fabric sample. The next figure shows the fabric piece and its weight, determined by a precision scale Netzsch DSC204.
%%%
\begin{figure}[H]
  \begin{center}
    \includegraphics[height=4.7cm]{figure/peso-15x15-1.png}
    \includegraphics[height=4.7cm]{figure/peso-15x15-2.png}
    \includegraphics[height=4.7cm]{figure/peso-15x15-3.png}
  \end{center}
  \caption{On the left the precision scale used for the weight determination and on the right the non-woven fabric piece used for the experiment.}
  \label{fig:weight}
\end{figure}
%%%
It follows that the weight by unit area $\wtissue$ is:
%%%
\begin{equation}\label{eq:wT}
   \wtissue=\dfrac{[\textrm{mass}]}{[\textrm{area}]}=
   \dfrac{0.2835 \ \unit{g}}{150 \ \unit{mm}\cdot 150 \ \unit{mm}} = 0.0126 \ 
   \dfrac{\unit{kg}}{\unit{m}^2}.
\end{equation}
%%%
As long as the non-woven fabric is fully compressed, from the density of the polypropylene (see Table \ref{tab:polipropilene}) we can infer its thickness:
%%%
\[
   \htissuemin = \dfrac{[\textrm{weigth}]/[\textrm{area}]}{[\textrm{density}]}
     = \dfrac{0.0126 \ \dfrac{\unit{kg}}{\unit{m}^2}}
             {900 \ \dfrac{\unit{kg}}{\unit{m}^3}}
     = 0.000014\,\unit{m}
     = 0.014\,\unit{mm}
     = 14\,\unit{$\mu$m}.
\]
%%%
\subsection{Computation of the thickness: pressure and displacement}
\label{subsec:YvsF}
In this subsection, we want to establish the thickness of the non-woven fabric not compressed. To do that, we use an indirect approach based on experimental measurements made with the dynamometer Instron 5969, at a speed of $1\ \unit{mm}/\unit{min}$, of the laboratories in the Department of Materials Engineering at the University of Trento.
\par 
In particular, at first we characterize the displacement/pressure curve of the dynamometer with the press disk made by seven circle tags without the non-woven fabric. In this way, we determine the offset corresponding to zero thickness to be eliminated in the measurement of the curve. We did two experiments and the results are the following.
%%%
\begin{figure}[H]
  \begin{center}
    \includegraphics[width=10cm]{figure/pressione-spostamento-free.pdf}
  \end{center}
  \caption{Displacement/pressure graph without the non-woven fabric}
\end{figure}
%%%
The fitting of the pressure is the following
%%%
\[
  P_{\textrm{base}}(x) =
  \dfrac{5461.352911\,\max(0,x)^{2.92599}}
  {0.0038158166 + 6.4490865\,\max(0,x)^{1.624481}}.
\]
%%%
After this measurement, ten sheets of non-woven fabric are put under the press disk, getting the following pressure curve.
%%%
\begin{figure}[H]
  \begin{center}
    \includegraphics[width=10cm]{figure/pressione-spostamento-10tessuti.pdf}
  \end{center}
  \caption{Displacement/pressure graph with ten non-woven fabric sheets}
\end{figure}
%%%
Fitting again the data, we obtain
%%%
\[
  P_{\textrm{base}+\textrm{10 fabric}}(x) =
  \dfrac{716.33893 \max(0,x+0.9703)^{12.67189680}}
        {14.10752 + 0.92219399\max(0,x+0.9703)^{30.037944}}.
\]
%%%
This shows that the pressure starts growing at $x=-0.9703$ which can be assumed as the thickness of the ten non-woven fabric sheets, and then we can set it to
%%%
\[
   \htissuemax = \dfrac{0.97\,\unit{mm}}{10} = 97\,\unit{$\mu$m}.
\]
%%%
Due to the very low speed of the displacement movement, the pressures are in equilibrium as follows:
%%%
\[
  P_{\textrm{base}+\textrm{10 fabric}}(x) =
  P_{\textrm{base}}(z) = P_{\textrm{base}}( x- w) =
  P_{\textrm{10 fabric}}(w),
\]
%%%o
where the total displacement $x=z+w$ is the sum of the displacement of the ten non-woven fabric sheets ($w$) and the press displacement ($z$). To obtain $P_{\textrm{10 fabric}}(w)$, first of all we have to determine $x$ as a function of $w$ by solving
%%%
\begin{equation}\label{eq:xw:solve}
  P_{\textrm{base}+\textrm{10 fabric}}(x)-
  P_{\textrm{base}}(x-w)= 0.
\end{equation}
%%%
This equation can be solved numerically with respect to $x$ and assuming $x(w)$ as known. Then
%%%
\[
  P_{\textrm{10 fabric}}(w) = P_{\textrm{base}+\textrm{10 fabric}}(x(w)).
\]
%%%
The function $x(w)$ is very ill-conditioned and defined for about $w\leq-0.08$. Thus it is better to compute the inverse function $w(x)$ by solving~\eqref{eq:xw:solve} w.r.t. $w$. Plotting $w(x)/10$ and $P_{\textrm{base}+\textrm{10 fabric}}(x)$, we obtain
%%%
\begin{figure}[H]
  \begin{center}
    \includegraphics[height=6cm]{figure/displacement-pressure-pp_page-0001.jpg}
  \end{center}
  \caption{Displacement/pressure graph of a single sheet of non-woven fabric}
  \label{fig:pp:pressure}
\end{figure}
%%%
and this is an approximation of the displacement/pressure curve on a single sheet of non-woven fabric. When the displacement is $0$, the fabric has non physical thickness $0$ and graph stops at about $-6 \ \unit{$\mu$m}$. The graph pressure is negligible until displacement is $-30 \ \unit{$\mu$m}$, and thus we can correct the thickness of the fabric (under mild compression) to $30 \ \unit{$\mu$m}$.
\par
Hence the fabric under compression at about $5 \ \unit{MPa}$ is fully compressed (the fibers are compacted) and the thickness is about $14\,\unit{$\mu$m}$. When pressure grows until $80\,\unit{MPa}$, the thickness is reduced to about $7\,\unit{$\mu$m}$, and the non-woven fabric assumed as a single block is under a strain of $7/14=0.5$. This information will be used to compute the increment of the temperature of the fabric under fast compression.
\par
To make computations workable, we derive from Figure~\ref{fig:pp:pressure} an approximated law linking the strain of the non-woven fabric with the pressure. In Figure~\ref{fig:sp:fitting}, a plot of the strain calculated with respect to the compressed fabric and the pressure is fitted with a hyperbolic curve. The $x$-axis contains the strain:
%%%
\[
   [\textrm{strain}]
   = \dfrac{[\textrm{displacement}]+\htissuemin}{\htissuemin}
   = \dfrac{[\textrm{displacement}]+14 \ \unit{$\mu$m}}{14 \ \unit{$\mu$m}}.
\]
%%%
Zero strain means that the fabric is fully compressed and positive strain means that the fabric is compressing. Negative strain means that the fabric is not fully compressed and the pressure is relatively low.
%%%
\begin{figure}[H]
  \begin{center}
    \includegraphics[width=6cm]{figure/strain-pressure-pp-fitting.pdf}
  \end{center}
  \caption{Strain/pressure curve fitting}
  \label{fig:sp:fitting}
\end{figure}
%%%
As you can notice, the pressure in the negative strain part is overestimated, while the positive part is well captured up to a strain of $0.4$ (compression of $40\%$). However, we are interested in the estimation of the heating of the non-woven fabric, and this part of the curve produces a very low effect. The approximated/fitted Young's modulus becomes
%%%
\begin{equation}\label{eq:Y:module:tissue}
  \Ytissue(s)
  = \dfrac{16}{1-2s}\quad [\unit{MPa}]
  = \dfrac{16\cdot 10^6}{1-2s}\quad [\unit{Pa}].
\end{equation}

\subsection{Computation of the temperature increment}
%%%
From the following experimental figure obtained at the University of Trento, it follows that $\Ytissue$ (i.e., the Young or Storage modulus) of the non-woven fabric linearly decreases by temperature. Dynamic mechanical analysis (DMA) tests were carried out using a TA Instrument DMA Q800 device, in the temperature range from $25\ \unit{\celsius}$ to $120\ \unit{\celsius}$, with a heating rate of $3\ \unit{\celsius}/\unit{min}$, a strain amplitude of $0.05\%$ and a frequency of $1\ \unit{Hz}$. Through this analysis, it was possible to evaluate the Storage modulus, the loss modulus and the loss tangent as a function of the temperature. 
%%%
\begin{figure}[H]
  \begin{center}
    \includegraphics[scale=0.6]{figure/Curva_modulo_elastico_temperatura.png}
  \end{center}
  \caption{Storage modulus/temperatue plot of the non-woven fabric}
  \label{fig:stiffness:temperature}
\end{figure}
%%%
In particular, the Young modulus linearly decreases from $T_0=20$ $\unit{\celsius}$ and reaches $0$ at about $90$ $\unit{\celsius}$. The value of the modulus of the non-woven fabric is not constant and depends on the displacement, but we can assume that its value linearly decreases by temperature in any condition. Thus we set
%%%
\[
  \kappa(s,T)
  = \Ytissue(s) \max\left(0,\dfrac{\TmaxL-T}{\TmaxL-T_0}\right).
  %= \dfrac{16\cdot 10^6}{1-2s}\cdot \dfrac{\TmaxL-T}{\TmaxL-T_0}
\]
%%%
Following \cite{Young}, instead of a linear decreasing, a parabolic fitting with $\kappa(T)=0$ near the fusing temperature ($\approx 160$ $\unit{\celsius}$) can be used. Moreover, it suggests an exponential fitting for Storage modulus depending on temperature. However we used linear and parabolic fitting to maintain the model as simple as possible.
Therefore
%%%
\begin{equation}\label{eq:kappaT}
   \kappa(s,T)
   = \Ytissue(s)\max\left(0,\dfrac{\TmaxQ-T}{\TmaxQ-T_0}\right)^2.
   %= \dfrac{16\cdot 10^6}{1-2s} \left(\dfrac{\TmaxQ-T}{\TmaxQ-T_0}\right)^2
\end{equation}
%%%
Using $\kappa(s,T)$, the force and the pressure per unit area become
%%%
\begin{equation}\label{eq:pressure:tissue}
  P(s,T) = \dfrac{F(s,T)}{[\textrm{area}]} = \kappa(s,T)s.
\end{equation}
%%%
The infinitesimal work done on the non-woven fabric (at fixed temperature) is
%%%
\[
   \mathrm{d}W(s,T) = F(s,T) \mathrm{d}s,
\]
%%%
and the variation of temperature is done by the work on the fabric as follows:
%%%
\[
  \begin{split}
    \dfrac{\mathrm{d} T(s)}{\mathrm{d}s} & =
    \dfrac{\Delta[\textrm{work}]'}{\Cptissue[\textrm{mass}]}
    =
    \dfrac{\dfrac{\Delta[\textrm{work}]'}{[\textrm{area}]}}
          {C_{p,\textrm{fabric}}\dfrac{[\textrm{mass}]}{[\textrm{area}]}}
    =
    \dfrac{F(s,T)/[\textrm{area}]}{\Cptissue\cdot \wtissue}
    =
    \dfrac{P(s,T)}{\Cptissue\cdot \wtissue} \\
    & =
    \dfrac{\kappa(s,T)s}{\Cptissue\cdot \wtissue},
  \end{split}
\]
%%%
where $\Cptissue$ is the specific heat of the fabric. Thus, using~(\ref{eq:pressure:tissue}), assuming that the cooling temperature negligible, the temperature increment is obtained by solving the ODE
%%%
\begin{equation}\label{ode:lin}
   \dfrac{\mathrm{d} T(s)}{\mathrm{d}s} =
   \dfrac{s\,\Ytissue(s)}{\Cptissue\cdot \wtissue}\cdot \max\left(0,\dfrac{\TmaxL-T(s)}{\TmaxL-T_0}\right)
\end{equation}
%%%
for the simple linear model, and
%%%
\begin{equation}\label{ode:para}
   \dfrac{\mathrm{d} T(s)}{\mathrm{d}s}  =
   \dfrac{s\,\Ytissue(s)}{\Cptissue\cdot \wtissue}\cdot
   \max\left(0,\dfrac{\TmaxQ-T(s)}{\TmaxQ-T_0}\right)^2
\end{equation}
%%%
for the parabolic interpolation model. Using the value in the table
%%%
\begin{center}
  \begin{tabular}{|cc|}
    \hline
    Parameter & Value \\
    \hline
    $T_0$    & $20\;\unit{\celsius}$ \\
    $\TmaxL$ & $90\;\unit{\celsius}$ \\
    $\TmaxQ$ & $160\;\unit{\celsius}$ \\
    $\Cptissue$ & $1800\;[\unit{J}/(\unit{kg K})]$ \\
    $\wtissue$  & $0.0126\;[\unit{kg}/\unit{m}^2]$ \\
    \hline
  \end{tabular}
\end{center}
%%%
the heating (numerically) computed is plotted in the following figure
%%%
\begin{figure}[H]
  \begin{center}
    \includegraphics[width=6cm]{figure/heating1.pdf}
  \end{center}
  \caption{Heating by pressure (adiabatic case)}
  \label{fig:heating1}
\end{figure}
%%%
and a moderate strain of the compressed non-woven fabric is enough to heat up to the melting temperature.

\subsection{Evaluation of the temperature decay by flux}
%%%
\begin{figure}[H]
    \begin{center}
      \includegraphics[width=4cm]{figure/flusso-calore.png} \\
      %\includegraphics[width=12cm]{tessuto/temperature_decay} \\
      %\includegraphics[width=12cm]{tessuto/temperature_raise}
    \end{center}
    \caption{Heat flux}% and temperature decay}
    \label{fig:heat-flux}
\end{figure}
%%%
%%%
From Figure~\ref{fig:heat-flux}, assuming that the steel has fixed temperature $\Tsteel$, the thermal flux from the non-woven fabric to the steel using the Newton's Law of cooling is
%%%
\[
   \mathcal{Q} = 2\Ksteel
   \dfrac{\Tsteel-T(t)}{\htissuemin(1-s)/2} [\textrm{area}]
   \qquad [\unit{W}],
\]
%%%
where $\Ksteel$ is the thermal conductivity of the steel. The thermal conductivity and the heat capacity of the steel are very high compared to the polypropylene ones, so that they can be assumed as infinity. The temperature variation due to the heat flux using~\eqref{eq:wT} becomes
%%%
\[
  \Cptissue[\textrm{mass}]\dfrac{\mathrm{d}\Tflux(t)}{\mathrm{d}t}
  = \mathcal{Q} = \dfrac{4\Ksteel(\Tsteel-T(t))}{\htissuemin(1-s)}  [\textrm{area}]
\]
%%%
so that
%%%
\begin{equation}\label{eq:T:cool:by:steel}
  \dfrac{\mathrm{d}\Tflux(t)}{\mathrm{d}t}
  =
  \dfrac{4\Ksteel(\Tsteel-\Tflux(t))}{\htissuemin(1-s)\Cptissue[\textrm{mass}]/[\textrm{area}]}
  =
  \dfrac{4\Ksteel(\Tsteel-\Tflux(t))}{\htissuemin(1-s)\Cptissue\wtissue}
\end{equation}
%%%
assuming that the compression is done at a constant speed $v\,[\unit{m/s}]$. Then we can write
%%%
\[
   s = \dfrac{v\cdot t}{\htissuemin}
\]
%%%
and
%%%
\begin{equation}\label{eq:Tflux}
  \dfrac{\mathrm{d}\Tflux(s)}{\mathrm{d}s}\dfrac{v}{\htissuemin}=
  \dfrac{4\Ksteel(\Tsteel-\Tflux(s))}{\htissuemin(1-s)\Cptissue\wtissue}.
\end{equation}
%%%
Using~\eqref{eq:Tflux} with~\eqref{ode:lin} or~\eqref{ode:para}, we have the two complete models:
%%%
\begin{equation}\label{ode:lin:full}
   \dfrac{\mathrm{d}T(s)}{\mathrm{d}s} = \dfrac{s\,\Ytissue(s)\max\left(0, \dfrac{\TmaxL-T(s)}{\TmaxL-T_0}\right)
   +\dfrac{4\Ksteel}{v(1-s)}
  (\Tsteel-T(s))}{\Cptissue\cdot \wtissue}
\end{equation}
%%%
for the simple linear model, and
%%%
\begin{equation}\label{ode:para:full}
   \dfrac{\mathrm{d}T(s)}{\mathrm{d}s} = \dfrac{s\,\Ytissue(s)\cdot
   \max\left(0,\dfrac{\TmaxQ-T(s)}{\TmaxQ-T_0}\right)^2
   +\dfrac{4\Ksteel}{v(1-s)}
  (\Tsteel-T(s))}{\Cptissue\cdot \wtissue}
\end{equation}
%%%
for the quadratic one.
%%%
We can solve numerically~\eqref{ode:lin:full} and~\eqref{ode:para:full} with parameters in the next table and velocity
%%%
\[
   v = \dfrac{\Delta s}{\Delta t}=\dfrac{2\htissuemin (1-r)}{\Delta t},
\]
%%%
where $\Delta s$ is the size of the compression of the non-woven fabric. It is set to $\htissuemin (1-r)$, where $r$ is the ratio of the fabric after the compression and the presence of the term $2$ is due the the fact that the two fabric sheets are bounded. Finally $\Delta t$ is the time spent through the compression.
%%%%
\begin{center}
  \begin{tabular}{|cc|}
    \hline
    Parameter & Value \\
    \hline
    $T_0$         & $20\;\unit{\celsius}$ \\
    $\TmaxL$      & $90\;\unit{\celsius}$ \\
    $\TmaxQ$      & $160\;\unit{\celsius}$ \\
    $\Cptissue$   & $1800\;[\unit{J}/(\unit{kg K})]$ \\
    $\wtissue$    & $0.0126\;[\unit{kg}/\unit{m}^2]$ \\
    $\htissuemin$ & $14\,\unit{$\mu$m}$ \\
    $r$           & $0.6$ \\
    $\Delta t$    & $10\,\unit{ms}$, $1\,\unit{ms}$, $0.1\,\unit{ms}$ \\
    $\Ksteel$     & $50\,[\unit{W}/(\unit{m K})]$ \\
    \hline
  \end{tabular}
\end{center}
%%%
The heating (numerically) computed is plotted in the following figure.

%%%%
\begin{figure}[H]
  \begin{center}
    \includegraphics[scale=0.5]{figure/heating}
  \end{center}
  \caption{
  Heating with and without heat flux for a strain of $0.4$ and a compression time of $10 \ \unit{ms}$, $1 \ \unit{ms}$ and $0.1 \ \unit{ms}$, from left to right
  }
  \label{fig:heating}
\end{figure}
