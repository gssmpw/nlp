%%
% !TEX root = main.tex
%%

\section{Parameters of the model}
\label{parametersmodel}
This section is devoted to the explanation of how to test our model. As already noticed in the Introduction, in case a raw non-woven fabric with a different thickness (and also other parameters) is provided, the company needs an optimal bonding so that the diaper lining produced has an endurance as close as possible to the usual one (the range is established by the fusing temperature). Therefore, some parameters must be calculated, which we report in the following table.

\begin{center}
\begin{tabular}{|l|p{12cm}|}
  \hline
  $\Roller$ & Ray of the rollers which compress the non-woven fabric (tags included).
  \\
  $\htissuemin$ &
  Thickness of the compacted fabric. We can compute it by dividing the density over the weight for square meter $\rho/\wtissue$.
  \\
  $\cratio$ &
  Ratio of compression of rollers starting from the thickness of the compacted fabric $\htissuemin$.
  The distance between the rollers then becomes $\droller=2\cratio\htissuemin$.
  \\
  $\Tsteel$ & Temperature of the rolls. It is reasonable to assume that it is equal to the room temperature.
  \\
  $\Ksteel$ & Thermal conductivity of the steel. In the model we assume that $\Ktissue\ll \Ksteel$.
  \\
  $\VRot$ &
  Rotational velocity of the rolls (radians per second). We can obtain it from the velocity of the fabric $\omega =v/(2\pi\Roller)$.
  \\
  $\droller$ &
  Distance between the rolls.
  \\
  $\TmaxQ$ &
  Temperature at which the Young's modulus goes to $0$.
  \\
  $\Cptissue$ &
  Specific heat of polypropylene.
  \\
  $\wtissue$ &
  Weight of the fabric for square meter.
  \\
  $\rho$ & Density of the fabric.\\
  \hline
\end{tabular}
\end{center}

These parameters have to be calculated using models and formulas we provided in the previous sections. We also observe that such parameters are compatible with the company's fabric ones.