As mentioned previously, obtaining high-quality advertisement data is particularly challenging (given its proprietary nature). %
In this section we conduct experiments on two datasets built from real anonymous advertisement data, publicly available.

\smallskip
\noindent
\emph{Data generation.} 
Details on how we build instances to our problem based on two real-world datasets  
(videos from YouTube\footnote{\url{https://www.kaggle.com/datasets/sidharth178/youtube-adview-dataset}} and ads from the Criteo AI Lab\footnote{\url{https://go.criteo.net/criteo-research-kaggle-display-advertising-challenge-dataset.tar.gz}}) are in \cref{app:nativedata}.
Our instances successfully preserve the sequential and categorical distribution of advertisement rewards in the data, when available.
A summary of the key data statistics is reported in \cref{tab:stats}.

\smallskip
\noindent
\emph{Discussion.} 
First we report in \cref{fig:exp:real} the results, in terms of expected reward for the two datasets. 
We start by noting that on the YouTube dataset, the best performing algorithms %
are \alggback, \alggbackproxy, \alggglobal, and \algflowg, with \alggglobal outperforming all the other algorithms by a small margin. 
Surprisingly, the \alggonline algorithm also performs well.
Results for the Criteo dataset confirm a similar trend for the best performers, but this time together with \alggforward,
\alggonline performs poorly compared to others, given its very high sensitivity to \Cthr.
Such results are in line with what is observed on synthetic data, confirming the high quality solutions in output to our techniques. 

To further investigate the difference in the allocation strategies  produced by the algorithms, 
we analyzed how the various ads are placed over the slots. 
To do this, we report a cumulative distribution over the slot indices in output to each algorithm,
More specifically, suppose that an algorithm matches $k$ slots 
with indices $J \subseteq [\nV]$, then the cumulative value at index $j$ is $|\{j' \in J : j' \le j\}|/k$. 
The results are reported in \cref{fig:exp:cumulativeCurves}.
On the YouTube dataset, we observe very different allocation strategies.
We first note that methods with different ad allocation strategies may yield similar expected rewards, for example \alggglobal allocates more slots with larger indices than \algmwm despite achieving similar result on the Criteo dataset (see \cref{fig:exp:real:criteo}).
Our backwards greedy methods are the only ones that allocate ads to slots with large indices.
This is due to the backwards design,
which may allocate ads in bottom positions as long as they are beneficial,
even though their utility may diminish later.
In other words, our backwards greedy algorithms achieve a high recall rate of good allocations.
Ads with a diminished reward can be pruned with almost no loss in the final expected reward, e.g., by the pruning strategy we introduce in \cref{sec:exp:ablation}.

As a summary of our experiments, we observe that our proposed methods report high quality solutions with provable approximation guarantees (as captured by our analysis) on both synthetic and real data, and solve the \streamads problem much more efficiently than existing techniques.
