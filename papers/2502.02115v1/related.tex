
For lack of space we only discuss the most related work.
For discussion on further research that may be of interest
we refer the reader to \cref{app:related}.

\smallskip
\noindent
\emph{Native streaming advertising.}
The study of sequential ad allocations originates from simple cascade models~\citep{kempe2008cascade,aggarwal2008sponsored},
for which a dynamic-programming algorithm was developed.
However, when the reward of an ad depends on the slot position, more sophisticated algorithms are needed~\cite{ieong2014advertising}.
After the work by~\citet{ieong2014advertising}, several approaches have been proposed, discussed below.

\citet{gamzu2019advertisement} study a variant of native stream advertising, 
taking into account the distance between consecutive ads to avoid ad fatigue.
\citet{yan2020ads} present a practical solution with an industrial application, by maximizing the revenue while requiring that the total user engagement from organic items exceeds a given threshold.
\citet{liao2022cross} adopt a RL-based model to combine a list of content items and a list of ads to produce a user feed.
However, none of these works consider dynamic decay in attention caused by ads.


\smallskip
\noindent
\emph{Positive externalities in advertising.}
On a high level, the \streamads problem 
is based on a
form of negative externalities,
that is, the presence of an ad has a negative effect on future ads.
There has been extensive research on the opposite, i.e., positive externalities, in advertising.
One notable example is word-of-mouth marketing~\citep{kempe2003maximizing,hartline2008optimal},
where it is beneficial to offer products, even for free, to a small group of influencers at the beginning of an ad campaign, to attract more customers.


\smallskip
\noindent
\emph{Online matching.}
There is a rich body of work if externalities are not considered. %
For example, a standard model of position auctions such as~\citep{varian2007position} is based on the separability assumption, i.e., 
the probability an ad receives a click if placed in a position is simply the product of the quality scores associated to the ad and the position,
independent therefore of other ads.
Under such assumptions, the allocation problem can be treated as a matching problem, 
for which various algorithms have been developed. 
We refer the readers to some excellent surveys about matching for more details~\cite{mehta2007adwords,devanur2022online,huang2024online}.
Our greedy algorithms are partly inspired by a related streaming algorithm~\citep{feldman2009online};
however, as already mentioned, more sophisticated techniques are needed to handle externalities.

