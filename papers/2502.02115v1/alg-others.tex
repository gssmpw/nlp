In this section, we introduce 
various algorithms for the \streamads problem, including 
enhanced variants of existing algorithms (from \citep{ieong2014advertising}), and multiple practical heuristics.
We list all algorithms below, and discuss their important design choices. 

\smallskip
\noindent
\emph{Flow- and matching-based algorithms.}
\citet{ieong2014advertising} devised a 4-approximation algorithm \algflow by finding a maximum weighted matching with fixed weights.
That is, the matching only considers the decaying effects from items but not ads.
The key idea is to reduce the dynamic decaying effect of ad placement %
by limiting the number of allocated ads (i.e., the matching size) via an additional cardinality constraint.
In our evaluation, we implement the \algflow algorithm by a minimum-cost flow, 
as from its original paper.~%
%

We enhance the \algflow algorithm with greedy assignments over the slots not matched by the flow-based procedure,
such an algorithm is denoted by \algflowg.

We also introduce a natural heuristic \algmwm, mentioned in \cref{sec:case}.
\algmwm does not enforce a cardinality constraint to the matching size,
and is
implemented via a standard maximum-weighted matching algorithm.


\smallskip
\noindent
\emph{Global greedy algorithm.}
We introduce another natural algorithm \alggglobal that repeatedly allocates an ad to a slot that maximizes the marginal reward over all allocations, 
provided the reward being positive. %
This requires computing the marginal reward of every candidate allocation, with time complexity $\bigO(|\E|^2 |\M|)$, which is expensive.
We improve such computation by noting 
that the marginal reward of any possible allocation is non-increasing over time.
This 
can be used to perform
\emph{lazy evaluation} of the marginal reward, i.e., maintaining upper bounds to the actual rewards.
That is, we sort all candidate allocations by their rewards in a decreasing order using a heap, and
we complete a greedy step if the reward of the top allocation is greater than the %
upper bounds of all other candidate allocations.
Typically, only a few edge weights (i.e., marginal gains) need to be updated at each  greedy~iteration. %


\smallskip
\noindent
\emph{Online greedy algorithms.}
In \cref{sec:case}, we mention an online algorithm \alggforward that allocates an ad in real-time as a user browses its session.
Such an algorithm greedily assigns the most rewarding ad to the slot being processed.

In addition we also consider \alggonline, an online algorithm introduced by \citet{ieong2014advertising}. %
The idea is to pre-determine a threshold \Cthr, and 
for each slot, allocate the most rewarding ad if its reward is greater than \Cthr.
In our experiments, we validate some heuristics to determine the value of \Cthr, which is often difficult to obtain.
