% This must be in the first 5 lines to tell arXiv to use pdfLaTeX, which is strongly recommended.
\pdfoutput=1
% In particular, the hyperref package requires pdfLaTeX in order to break URLs across lines.

\documentclass[11pt]{article}

% Change "review" to "final" to generate the final (sometimes called camera-ready) version.
% Change to "preprint" to generate a non-anonymous version with page numbers.
%\usepackage[review]{acl}
\usepackage{acl}
% Standard package includes
\usepackage{times}
\usepackage{latexsym}

% For proper rendering and hyphenation of words containing Latin characters (including in bib files)
\usepackage[T1]{fontenc}
% For Vietnamese characters
% \usepackage[T5]{fontenc}
% See https://www.latex-project.org/help/documentation/encguide.pdf for other character sets

% This assumes your files are encoded as UTF8
\usepackage[utf8]{inputenc}

% This is not strictly necessary, and may be commented out,
% but it will improve the layout of the manuscript,
% and will typically save some space.
\usepackage{microtype}

% This is also not strictly necessary, and may be commented out.
% However, it will improve the aesthetics of text in
% the typewriter font.
\usepackage{inconsolata}

%Including images in your LaTeX document requires adding
%additional package(s)
\usepackage{graphicx}

% If the title and author information does not fit in the area allocated, uncomment the following
%
%\setlength\titlebox{<dim>}
%
% and set <dim> to something 5cm or larger.

\newcommand{\etal}{\textit{et al.}}
\usepackage{comment}
\usepackage{amsthm}
\usepackage{float}
\usepackage{graphicx}
\usepackage{subfigure}
\usepackage{caption}
\usepackage{amsmath,amssymb}
\usepackage{mathtools}
\usepackage{booktabs}
\usepackage{multirow}
\usepackage{siunitx}
\usepackage{adjustbox}
\usepackage{pifont}
\usepackage{scalerel}
\usepackage{tabularray}
\usepackage{makecell}
\usepackage{wrapfig}

\definecolor{stepcolor}{HTML}{d79b00}
\definecolor{contentcolor}{HTML}{3439a2}
\newcommand{\textttbf}[1]{\texttt{\textbf{#1}}\xspace}

\usepackage{tikz}
\newcommand*\circled[1]{\tikz[baseline=(char.base)]{
            \node[shape=circle,draw,inner sep=2pt] (char) {#1};}}
\newcommand{\lhr}[1]{\textcolor{blue}{#1}}
\newcommand{\tbc}[1]{\textcolor{blue}{#1}}
\usepackage{xspace}
\newcommand{\name}{\textit{PrivaCI-Bench}\xspace}

\definecolor{stepcolor}{HTML}{d79b00}
\definecolor{contentcolor}{HTML}{6c8ebf}

\newtheorem{definition}{Definition}


\title{PrivaCI-Bench: Evaluating Privacy with Contextual Integrity and Legal Compliance}

\newcommand{\hwbc}[1]{\textcolor{red}{[hwb: #1]}}
\newcommand{\hwb}[1]{\textcolor{red}{#1}}

% Author information can be set in various styles:
% For several authors from the same institution:
% \author{Author 1 \and ... \and Author n \\
%         Address line \\ ... \\ Address line}
% if the names do not fit well on one line use
%         Author 1 \\ {\bf Author 2} \\ ... \\ {\bf Author n} \\
% For authors from different institutions:
% \author{Author 1 \\ Address line \\  ... \\ Address line
%         \And  ... \And
%         Author n \\ Address line \\ ... \\ Address line}
% To start a separate ``row'' of authors use \AND, as in
% \author{Author 1 \\ Address line \\  ... \\ Address line
%         \AND
%         Author 2 \\ Address line \\ ... \\ Address line \And
%         Author 3 \\ Address line \\ ... \\ Address line}

\author {
    % Authors
    {\bf Haoran Li}\textsuperscript{\rm 1}\thanks{Haoran, Wenbin and Huihao contributed equally.},
    {\bf Wenbin Hu}\textsuperscript{\rm 1}\footnotemark[1],
    {\bf Huihao Jing}\textsuperscript{\rm 1}\footnotemark[1],
    {\bf Yulin Chen}\textsuperscript{\rm 2},
    {\bf Qi Hu}\textsuperscript{\rm 1}\\
    {\bf  Sirui Han}\textsuperscript{\rm 1},
    {\bf Tianshu Chu}\textsuperscript{\rm 3},
    {\bf Peizhao Hu}\textsuperscript{\rm 3},
    {\bf Yangqiu Song}\textsuperscript{\rm 1}\\
    \textsuperscript{\rm 1}HKUST, 
    \textsuperscript{\rm 2}National University of Singapore, 
    \textsuperscript{\rm 3}Huawei Technologies\\
    \texttt{hlibt@connect.ust.hk}\\
    Project Page: \url{https://hkust-knowcomp.github.io/privacy/}\\
}
%\author{
%  \textbf{First Author\textsuperscript{1}},
%  \textbf{Second Author\textsuperscript{1,2}},
%  \textbf{Third T. Author\textsuperscript{1}},
%  \textbf{Fourth Author\textsuperscript{1}},
%\\
%  \textbf{Fifth Author\textsuperscript{1,2}},
%  \textbf{Sixth Author\textsuperscript{1}},
%  \textbf{Seventh Author\textsuperscript{1}},
%  \textbf{Eighth Author \textsuperscript{1,2,3,4}},
%\\
%  \textbf{Ninth Author\textsuperscript{1}},
%  \textbf{Tenth Author\textsuperscript{1}},
%  \textbf{Eleventh E. Author\textsuperscript{1,2,3,4,5}},
%  \textbf{Twelfth Author\textsuperscript{1}},
%\\
%  \textbf{Thirteenth Author\textsuperscript{3}},
%  \textbf{Fourteenth F. Author\textsuperscript{2,4}},
%  \textbf{Fifteenth Author\textsuperscript{1}},
%  \textbf{Sixteenth Author\textsuperscript{1}},
%\\
%  \textbf{Seventeenth S. Author\textsuperscript{4,5}},
%  \textbf{Eighteenth Author\textsuperscript{3,4}},
%  \textbf{Nineteenth N. Author\textsuperscript{2,5}},
%  \textbf{Twentieth Author\textsuperscript{1}}
%\\
%\\
%  \textsuperscript{1}Affiliation 1,
%  \textsuperscript{2}Affiliation 2,
%  \textsuperscript{3}Affiliation 3,
%  \textsuperscript{4}Affiliation 4,
%  \textsuperscript{5}Affiliation 5
%\\
%  \small{
%    \textbf{Correspondence:} \href{mailto:email@domain}{email@domain}
%  }
%}

\begin{document}
\maketitle
\begin{abstract}
Recent advancements in generative large language models (LLMs) have enabled wider applicability, accessibility, and flexibility.
However, their reliability and trustworthiness are still in doubt, especially for concerns regarding individuals' data privacy.
Great efforts have been made on privacy by building various evaluation benchmarks to study LLMs' privacy awareness and robustness from their generated outputs to their hidden representations.
Unfortunately, most of these works adopt a narrow formulation of privacy and only investigate personally identifiable information (PII). 
In this paper, we follow the merit of the Contextual Integrity (CI) theory, which posits that privacy evaluation should not only cover the transmitted attributes but also encompass the whole relevant social context through private information flows.
We present \name, a comprehensive contextual privacy evaluation benchmark targeted at legal compliance to cover well-annotated privacy and safety regulations, real court cases, privacy policies, and synthetic data built from the official toolkit to study LLMs' privacy and safety compliance.
We evaluate the latest LLMs, including the recent reasoner models QwQ-32B and Deepseek R1.
Our experimental results suggest that though LLMs can effectively capture key CI parameters inside a given context, they still require further advancements for privacy compliance.



\end{abstract}


\section{Introduction}
\label{sec:intro}

Foundational models (FMs)~\cite{zhang2024data, zhou2023comprehensive} have shown remarkable progress in the healthcare domain, enabling professional-like assessment of disease diagnosis, treatment decision-making, and monitoring~\cite{zhang2023text, wang2022medclip, lu2023mi-zero}. 
Examples include LLaVA-Med~\cite{li2023llava}, Med-PaLM Multimodal~\cite{tu2024towards}, and Med-Flamingo~\cite{moor2023med}, have demonstrated their capacity on question answering, medical image analysis, and report generation.
These studies follow a predominant top-down model development strategy that requires upstream developers to collect data and train models for downstream tasks. 
Consequently, the developed model capabilities are heavily dependent on the training data, limiting their generalization performance in diverse clinical scenarios. 
For instance, Med-Gemini~\cite{yang2024advancing} reveals promising general capabilities in report generation while it lags behind state-of-the-art (SoTA) models on classification tasks, especially for out-of-domain applications. 
This indicates that while the generalizability of the foundation model is promising, more solutions are expected to meet the various specialized clinical needs.

To address these challenges, multi-center data centralization becomes essential to enhance model capacity and robustness across varied clinical scenarios~\cite{rajpurkar2022ai}. 
Centralizing distributed data can significantly improve model training and inference performance.
However, the process of medical data storage, transfer, and aggregation among centers requires extra efforts to ensure data security and system interoperability~\cite{bradford2020international}.
Moreover, a growing concern for patient privacy makes large-scale multi-center data sharing particularly challenging. 
While efforts like federated learning~\cite{wen2023survey, li2020review} can achieve good model performance on local data, the need for synchronized system coordination presents significant challenges, as clients are unable to update asynchronously. This limitation greatly restricts the practical capability of such approaches.
As a result, without a flexible collaboration, medical community still struggles to fully utilize the isolated data and local computation resources for comprehensive medical AI model development. 
To address this dilemma, open-source platforms encourage public data sharing and knowledge integration~\cite{markiewicz2021openneuro, zenodo}.
However, these platforms focus solely on raw data sharing while seldom providing collaborative model training or cooperation between different institutions.
Recently, collaborative learning has emerged as a viable approach for enhancing multi-model robustness~\cite{boulemtafes2020review}. 
For instance, software-like model development~\cite{raffel2023building} mimics software engineering practices by introducing structured workflows, enabling merging, version control, and continuous model integration.
Under this design, model ability can be strengthened with incremental knowledge updates similar to the version updating in software development. 

Although collaborative learning provides a multi-model collaboration, two key challenges remain in the leakage of raw data during collaboration~\cite{huang2023lorahub} and the synchronization of multiple collaborators~\cite{mcmahan2017communication} in the medical AI community. It is still challenging to integrate decentralized, privacy-sensitive data across institutions, leading to under-utilized insights and fragmented knowledge sharing~\cite{kaissis2020secure, rajpurkar2022ai, abdullah2021ethics}.
 To address these challenges, inspired by the collaborative software development, we propose \textbf{Med}ical \textbf{Fo}undation Models Me\textbf{rg}ing (\textbf{MedForge}), a cooperative workflow enabling continuously community-driven foundation model (FM) development.
MedForge enables a lightweight manner for individual centers to share their knowledge among multiple centers, minimizing the burden of data transmission and integration while enhancing model robustness.
Meanwhile, MedForge facilitates asynchronous and flexible collaboration, allowing individual centers to continuously update and improve medical FMs without the need for real-time synchronization.
Similar to open-source software development, MedForge incrementally updates medical knowledge and follows a sustainable model development scheme. 
This key design emphasizes a bottom-up construction of a multi-task medical FM, allowing downstream users to collaboratively build, refine, and update the upstream model according to their local resources. Our major contributions of MedForge are as below: 
\begin{enumerate}
    \item[$\bullet$] We introduce a collaborative workflow to promote the merging scheme of open-source software development. Our proposed MedForge allows distributed clinical centers to asynchronously contribute to comprehensive medical model construction while reducing transmitting costs among centers and avoiding the leakage of raw data, thus enhancing the utilization of private resources in the healthcare system. 
    \item[$\bullet$] We propose two effective knowledge-merging strategies for the asynchronous branch contribution. The MedForge-Fusion strategy updates the plugin module parameters of the main model during the merging phase, whereas the MedForge-Mixture strategy integrates the output of the plugin module by memorizing each contributor's coefficient. These strategies make MedForge more flexible and versatile. MedForge-Fusion is friendly to implement, while the MedForge-Mixture offers better performance and robustness.
    \item[$\bullet$]  We comprehensively evaluate model merging strategies to accumulate medical knowledge among multiple branch plugin modules. MedForge yields superior performance on medical classification tasks compared to other collaborative baselines across multiple datasets. We demonstrate the robustness of MedForge by shuffling the task order and evaluating various configurations of plugin modules and dataset distillation methods.
\end{enumerate}



\section{Related Works}

\subsection{Parental Homework Involvement in Education}

Parental homework involvement is commonly defined as \textit{parents' monitoring, supervision, and participation in their children's schoolwork and academic performance} \cite{pomerantz2007whom}. Research has shown that such involvement plays a crucial role in children's academic success, motivation, and well-being \cite{patall2008parent, dettmers2019antecedents, cooper1989synthesis}. While much of the literature focuses on Western contexts, studies in China highlight unique dynamics. Chinese parents often engage more directly, employing motivational strategies such as reasoning, monitoring, and even criticism to ensure academic success \cite{kim2013parents}. These behaviours align with broader cultural expectations in China, where academic achievement is highly valued, and parents feel a strong responsibility to support their children's education.

Some studies have also examined how Chinese parents adapted their involvement during the COVID-19 pandemic, where concerns over online education led to increased parental engagement \cite{wang2021parental}. Research has identified various types of parental involvement in Chinese families, ranging from supportive to disengaged, with the former most closely linked to academic success \cite{gan2019parental}. %These findings underscore the importance of understanding homework involvement in non-Western contexts.
%Further research has explored how Chinese parents adapted their involvement to specific contexts. For instance, Wang et al. \cite{wang2021parental} examined parental homework involvement during the COVID-19 pandemic, finding that Chinese parents, particularly mothers, became even more engaged in their children's learning due to concerns over the efficacy of online education. Similarly, Gan et al. \cite{gan2019parental} identified different types of parental involvement in Chinese families, ranging from supportive to disengaged, with supportive involvement being the most closely linked to academic success. 
These findings highlight the importance of understanding homework involvement in non-Western contexts.% how Chinese parents navigate the complexities of homework involvement in a rapidly changing educational landscape.

%Additionally, Lau et al. \cite{lau2011parental} and Liu et al. \cite{liu2019parental} examined parental involvement in younger children and its long-term impacts on academic readiness and emotional regulation. Their findings emphasize that early parental support in homework-related activities significantly influences children's long-term academic self-efficacy and emotional well-being, laying the groundwork for continued parental engagement during later school years.

The study of parental homework involvement typically relied on self-report surveys  (e.g., EMBU \cite{arrindell1999development}, QPH \cite{dumont2014quality}), interviews, and observations. While these methods provide valuable insights, they often fail to capture the nuanced emotions and behaviours that occur during real-world homework involvement, where conflicts may arise that are not evident in the presence of a human observer (due to the \textit{Hawthorne Effect} \cite{adair1984hawthorne}). Additionally, reliance on self-reported data introduces bias and may not fully reflect the subtleties of everyday involvement. For example, high-level categorizations of involvement, such as Gan et al.'s four types of involvement \cite{gan2019parental}, may miss the nuanced ways these interactions occur in day-to-day life.


\subsection{Emotional Experiences and Parent-child Conflicts During Homework Involvement}


While parental homework involvement is generally associated with positive educational outcomes, it can also lead to emotional strain and conflicts within families. Nnamani et al. \cite{nnamani2020impact} found that although parental involvement positively impacts students' emotional adjustment and academic performance, the emotional burden on parents often goes unnoticed. This emotional toll is particularly evident in cultures where academic success is strongly emphasized, as is the case in China. Kim et al. \cite{kim2020dyadic} developing Dyadic Mirror, a wearable smart mirror that provides parents with a second-person live-view of their own expressions as seen by their child during face-to-face interactions. Studies have shown that parents experience tension when balancing the desire to foster autonomy with the need to control their children's learning \cite{cunha2015parents}. The emotional stress parents feel during homework sessions can negatively affect children, creating a feedback loop of stress and conflict \cite{moe2018brief}. This dynamic is particularly significant in cultures where academic achievement is heavily emphasized, as in China \cite{nnamani2020impact}.


Homework involvement can exacerbate family conflicts, especially during adolescence, as parents strive to ensure academic success. Solomon et al. \cite{solomon2002helping} explored how homework involvement can become a source of conflict and found that the pressure parents feel to ensure their children's academic success can exacerbate tensions, often turning homework sessions into battlegrounds where unresolved issues about control and expectations surface. This finding is echoed by Suarez et al. \cite{suarez2022parental}, who reported high levels of family conflict and stress during the COVID-19 pandemic, a time when parental involvement in homework increased dramatically due to school closures and remote learning. 

Above findings underscore the complexity of parental homework involvement, where well-intentioned efforts to support academic achievement can inadvertently result in emotional strain and conflict. Our research aims to further unpack these emotional dynamics and explore how they are intertwined with parental behaviours and conflicts during homework involvement in Chinese families.


\subsection{Technology Supported Parent-Child Interaction}

Technological interventions in HCI have demonstrated the potential to enhance parent-child interactions in educational settings. Liu et al. \cite{liu2024he} explored the use of image-based generative AI in family expressive arts therapy. Fan et al. \cite{fan2019character} developed a tangible system for improving literacy in children with dyslexia, while Zhang et al. \cite{zhang2022storybuddy} introduced an AI-driven storytelling tool to balance parental involvement in learning activities. Although such technologies support collaborative learning, they have largely overlooked the specific challenges of homework involvement. Kalanadhabhatta et al. \cite{kalanadhabhatta2024playlogue} developed a dataset for analyzing adult-child conversations during play, demonstrating the potential of systematic conversation analysis in understanding parent-child interactions. 


In one of the few studies addressing this gap, Kerawalla et al. \cite{kerawalla2007exploring} examined a tablet-based platform designed to enhance parental understanding of classroom methods. This study is one of the few that addresses the role of technology in supporting homework parental involvement, showing the potential for digital tools to improve educational outcomes. Similarly, recent innovations like EduChat \cite{dan2023educhat}, an LLM-based educational chatbot, highlight the potential of AI in offering personalized support for both parents and children. Yu et al. \cite{yu2021parental} proposed a framework for parental mediation in children's use of creation-oriented educational media, and outlined three dimensions of mediation—creative, preparative, and administrative—offering insights for designing media that fosters creative learning while involving parents in the process. 



While technology-supported applications have made progress in facilitating parent-child interactions through storytelling, literacy development, and specific learning activities, a significant gap remains in addressing homework involvement, a crucial but underexplored aspect of parent-child interaction. Most research, focused on Western contexts \cite{pomerantz2007whom, patall2008parent, dettmers2019antecedents}, prioritizes academic outcomes, often overlook the unique emotional and behavioural dynamics that arise during homework involvement in non-Western cultures, particularly in China \cite{kim2013parents, gan2019parental, wang2021parental}. Additionally, reliance on self-report methods \cite{gan2019parental, patall2008parent} introduces bias and fails to capture real-time interactions.




Our study distinguishes itself in several ways: (1) we focus on the Chinese cultural context, shaped by unique parental expectations and pressures \cite{kim2013parents, suarez2022parental}; (2) unlike prior work that largely depends on subjective data, we utilize audio recordings of real-world homework sessions for a richer and more objective analysis; (3) by exploring the interplay of parental behaviours, emotions, and conflicts, we aim to deepen understanding of the complexities in homework involvement, contributing valuable insights to the family education research and designing technologies to improving parenting practices in China.

%\subsection{Data}
%Some prior work also used Booking.com data~\cite{ALDERIGHI2022769}

\vspace{-.5pc}
\paragraph{Data.}
We constructed a dataset of hotel reviews sourced from \texttt{Booking.com},\footnote{Data processing code: \href{https://github.com/Crowd-AI-Lab/Contextually-Aligned-Online-Reviews}{https://github.com/Crowd-AI-Lab/Contextually-Aligned-Online-Reviews}} which has been used in prior research studies~\cite{ALDERIGHI2022769,barnes-etal-2018-multibooked}.
%This dataset included 4,447,853 reviews in both \twChinese and \cnChinese, written by users from Taiwan and Mainland China. 
%This dataset consists of 4,447,853 reviews labeled as written in Chinese by the platform, written by users who self-identified as being from Taiwan and Mainland China. 
This dataset consists of 4,447,853 reviews labeled by the platform as written in Chinese.
%, authored by users who self-identified as being from Taiwan and Mainland China.
%The reviews encompass 149,879 hotels across Japan, Mainland China, South Korea, Taiwan, Thailand, and Vietnam, and were gathered between August 2021 and August 2024.
The reviews cover 149,879 hotels located in Japan, Mainland China, South Korea, Taiwan, Thailand, and Vietnam, and were collected from August 2021 to August 2024. 
These locations were selected to ensure a substantial volume of data, as they are popular destinations for Mandarin-speaking travelers.
%\kenneth{TODO Zixin: Update the locations and time span} 
%The dataset comprises both positive and negative feedback, ratings, room types, and traveler labels. 
%We constructed a dataset that consists of the hotel reviews from \texttt{booking.com}, which is a data source for a series of prior studies~\cite{ALDERIGHI2022769,barnes-etal-2018-multibooked}. \kenneth{TODO Sam: Find one or two more papers from NLP conference that also used Booking.com data.}
%We deliberately selected locations outside Taiwan and Mainland China to avoid potential biases associated with users posting reviews from a particular language variety's region.
Each review comprises three main components: the review title, positive feedback, and negative feedback.
Additionally, it includes review ratings (ranging from 1 to 10 stars) and metadata such as hotel ID, posting time, and more (see \Cref{app:booking-data-sample} for an actual sample). 
Booking.com claims to invest significant effort in ensuring that reviews are posted by real users and in maintaining review quality. 
%Based on their assertion, we chose not to conduct extensive data cleaning. 
We included only non-empty reviews, meaning reviewers provided input in at least one of the following: 
the review title, positive feedback, or negative feedback. 
In total, we collected 1,513,056 reviews written in Chinese.
%\kenneth{TODO Zixin: Update the numbers}


%\kenneth{TODO GG and HH: Please describe how you scraped Booking.com. What region (and why)? What time span (and why)? How many raw reviews were collected? How did you select hotels (and why?}

%\paragraph{Data Content and Pre-Processing.}

% \kenneth{TODO Zixin and CY: Describe how you cleaned and pre-processed the data. Mostly (1) We believe Booking.com already made sure all the comments are real. (2) Each review has what information: title, pos, neg, and all the meta data. (3) we had three settings (and why): (i) original, (ii) plain, and (iii) shuffled. Say we use original to calculate the length.}

%\begin{itemize}
    %\item \textbf{Language Identification:}\kenneth{TODO: CY will figure out}
    %\item \textbf{Short Comments:}\kenneth{What is the criteria? (Chinese word seg?)---CY will figure out, maybe use LLMs to plot accuracy drop (will it drop?)--- Pay attention to bias}
    %\item \textbf{Bot-like/Dup Comments:}\kenneth{Zixin will look into is to figure out the model}
    %\item \textbf{User Accounts Acting Like Bots:} (1) Always post the same content\kenneth{Zixin--Find papers used Booking.com data and see how they cleaned it}; 
    %(2) Unreasonable stay schedule\kenneth{Can random people leave reviews? Or do you need to stay in the hotel? YES??????}
    %\item \textbf{Lack of user id.....}
%\end{itemize}
%\paragraph{Aligning Reviews Using Contexts.}

%\kenneth{TODO Zixin and CY: Describe how you paired the reviews.}

%+Zixin should just use nationality
%+Chieh-Yang will run lang detection tool

\vspace{-.6pc}
\subsection{Contextually Aligning Reviews}
%We only included reviews with textual input, where reviewers have input in either the review title, the positive feedback, or the negative feedback. 
%The users' self-specified ``nationality/region'' labels, which are required by the Booking.com platform, defined the users' language varieties in this study. 
%We then aligned and formed review pairs following the rules below:
We used users' self-specified ``nationality/region'' labels from Booking.com to determine the reviews' language varieties. %in this study. 
In total, we collected 1,403,669 reviews written in \twChinese and \cnChinese, where 95.591\% of them come from \twChinese users.
To ensure a balanced representation between \textbf{\twChinese (TW)} and \textbf{\cnChinese (CN)} reviews, we paired them based on the following criteria:
%we employed a sampling method that pairs reviews in two varieties using the following criteria:
%We then paired \twChinese reviews with \cnChinese reviews according to the following criteria:


\vspace{-.8pc}

\begin{itemize}[leftmargin=*]
%\item \textbf{Both reviews contribute to the same hotel:} Both reviews in the same pair are from the same hotel. Such a pairing rule ensures that reviewers comment on highly similar scenarios or objects, which is the target hotel in our study.
\item
\textbf{Same hotel for both reviews:} 
Both reviews in each pair are from the same hotel, ensuring that the reviewers are commenting on similar scenarios or objects---the hotel itself.



\vspace{-.5pc}

    
%\item \textbf{Both reviews share similar ratings:} To maximize the size of paired reviews while maintaining similar sentiment components in each pair of reviews, we selected pairing candidates with a 3-class label scheme ( 1-3 as negative reviews, 4-7 as neutral reviews, and 8-10 as positive reviews). This process helps align reviews with similar ratings to form comparable pairs.
\item
\textbf{Similar ratings for both reviews:} 
To form comparable pairs with similar sentiments, we used a 3-class rating scheme (1-3 as negative, 4-7 as neutral, and 8-10 as positive) and paired reviews based on this classification. 
This approach maximizes the number of review pairs while maintaining comparable sentiment.



\vspace{-.5pc}

%\item \textbf{Both reviews have a similar amount of text input:} To minimize the effect of input size, we classified all reviews into 10-token wide bins before the pairing process. This ensures the paired reviews will share a similar size of text input. We remove reviews longer than 500 tokens to enhance the overall quality of the data set (see \cref{app:length-exp} for more information regarding potential impacts of text length).\zixin{we need a further explanation on this.}
\item
\textbf{Similar text length for both reviews:} 
%To minimize the effect of input text length, 
To ensure paired reviews have similar text lengths, we grouped reviews into 10-token bins before pairing and required both reviews in each pair to fall within the same length bin.
%This ensures paired reviews have similar text lengths.
% 22 r
Reviews longer than 500 tokens were excluded (see \Cref{app:length-exp}.)
%\kenneth{TODO Zixin: Update numbers}


\end{itemize}



\vspace{-.5pc}


The final dataset contained 22,918 review pairs, each with one TW and one CN user review.








%Following the criteria, the final paired dataset contained 22,918 pairs of reviews, with each pair containing one review from a \twChinese user and one from a \cnChinese user.
%from \twChinese users and \cnChinese users. 
%Each pair 
%making 45,836 data entries in total.

\subsection{Data Quality Validation\label{sec:data-quality-validation}}
%To validate the data quality, we recruited 10 participants, evenly divided between native speakers of \twChinese (TW) and \cnChinese (CN), to rate 200 randomly selected reviews of each language variety.
%To validate the data quality, 
%We recruited five native speakers of \twChinese (TW) to review 200 randomly selected reviews written in \twChinese. 
%We followed the same process for reviews in \cnChinese (CN).
Five native speakers of \twChinese reviewed 200 random \twChinese reviews; the same process applied to \cnChinese.
The focus was on two key aspects: {\em (i)} \textbf{writing quality} and {\em (ii)} \textbf{content-rating agreement}, evaluated on a 5-point Likert scale (see Appendix~\ref{app:human-validation}.) 
%Evaluations were conducted using a 5-point Likert scale, with 1 indicating strong disagreement and 5 indicating strong agreement.
%Appendix~\ref{app:human-validation} shows the material used. 
Each participant was paid \$10.
%\kenneth{TODO Sam: Describe human validation process and results}
%To validate the data quality,
%To construct a contextually-aligned review dataset for language varieties,
%we recruited 10 participants, evenly divided between native speakers of Taiwan Mandarin (TW) and Mainland Mandarin (CN), to rate 200 hotel reviews. The focus was on two key aspects: writing quality and content-rating agreement. Evaluations were conducted using a 5-point Likert scale, with 1 indicating strong disagreement and 5 indicating strong agreement. 
%The 200 comments from Taiwan and China were paired based on the same hotel, similar length, and category ratings. \sam{seems to be mentioned in the previous paragraph}
%\kenneth{TODO Sam: Report the mean and SD for each group here.}
%\sam{done}
As a result, for the writing quality ratings, the TW group had a mean of 4.18 (SD=0.44), and the CN group had a mean of 3.94 (SD=0.49). 
Regarding the rating-content agreement, the TW group had a mean of 4.00 (SD=0.46), and the CN group had a mean of 3.56 (SD=0.55).
%\kenneth{Displaying two decimal (instead of 3) places is sufficient}
%The average rating correlation between participants for writing quality was 0.22 for the TW group and 0.14 for the CN group.
%For content-rating agreement, the average rating correlation between participants was 0.21 for the TW group and 0.17 for the CN group.
%\kenneth{TODO Sam: (1) Displaying two decimal places is sufficient; our sample size isn't large enough to justify showing four decimal places. (2) Calculate (a) the avg correlation between two TW raters, and (b) the avg correlation between two CN raters. This is to show the inner-annotator agreement (IAA) in each group.}
%To validate the dataset's quality, 
%The correlation between the CN and TW groups was 0.258 for writing quality and 0.194 for content-rating agreement. These low positive correlations suggest limited agreement between the groups, indicating potential differences in evaluation criteria or perspectives.


%We calculated the Mean Squared Error (MSE) between the evaluations of TW and CN groups for both  writing quality and content-rating agreement, resulting in values of 0.3766 and 0.5738, respectively.
%\kenneth{I'd report avg corelation between two participants in each group. MSE is not very straighforward.} \sam{done!}

% These MSE values indicate a moderate level of agreement between the two groups. The relatively low MSE for writing quality suggests consistent evaluations across locales, enhancing the dataset's reliability for further analysis. However, significant differences were found in the Two Paired Samples T-Test between the CN and TW groups for both writing quality and content-rating agreement.\sam{singificant found which means CN vs TW groups are not aligned, probably find some interesting examples to address --> may delete if not enough space}
% Regarding this, we tried to study the example with the highest MSE within top 10\% of the example. In content-rating agreement, it is interesting that CN groups are consistently rating lower compared to TW group. And nearly all the comments are from the "positive" category (only 1 from neutral, and no from negative).
% In writing quality, it is also the similar case that most of all comments deviated are from positive comment (only 2 from neutral, and also no from negative)

%\kenneth{I don't feel we have space for expert vs. non-expert...}
%Considering the evaluation of an expert with a Master's degree in Hospitality Management from the CN group, we conducted a Wilcoxon Signed-Rank Test for paired samples between expert and non-expert evaluations. Significant differences were observed in both writing quality $(p\text{-value} = 7.08 \times 10^{-10})$ and content-rating agreement $(p\text{-value} = 7.76 \times 10^{-21})$. Additionally, the expert's ratings showed higher variability (SD = 0.91, 0.86) compared to non-experts (SD = 0.47, 0.57), suggesting that the expert's background may lead to more nuanced evaluations, reflecting deeper insights or different criteria compared to non-experts. 
%\sam{BUT if I do SD for individual participants one by one for each aspect, P1 (expert) rating is not that greatly deviated compared}
%We acknowledge that the sample size for the expert is small, indicating that further evaluation from more experts could be valuable in future studies.

%\begin{itemize}
%   \item The writing quality is reasonable.
%  \item The rating reflects the content of the review.
%\end{itemize}

%\subsection{Data Quality Validation}

\section{Evaluation Setups}
\label{sec: eval}

In this section, we detailedly illustrate our setups for LLMs' contextual privacy evaluation.

\subsection{Implementation of Judgment Modules}
\label{sec: judge}
% DP, COT, RAG
To evaluate LLMs' legal compliance for given benchmark samples, we mainly consider the following three straightforward strategies:

\noindent$\bullet$ Direct prompt (\textbf{DP}).
We prompt LLMs with only the context and directly instruct them to determine if the given context is permitted, prohibited, or unrelated to specific regulations.

\noindent$\bullet$ Chain-of-Thought reasoning (\textbf{CoT}).
We prompt LLMs to automatically list step-by-step plans to analyze the given case and then execute the steps to determine privacy violations similar to DP. 

\noindent$\bullet$ Retrieval augmented generation (\textbf{RAG}).
%We apply the \textit{LLM explanation} to clarify the case context with legal terms to facilitate the retrieval process.
Given the context, we first resort to the LLMs to explain the context by using their knowledge of the corresponding legal terms.
Then, we implement BM25 to search for relevant sub-rules.
Lastly, we feed both the retrieved sub-rules and the context into the prompt to improve in-context reasoning.

In addition to these naive implementations, we also consider feeding the ground truth CI parameters and regulations to the LLMs to evaluate the effectiveness of our \name.


\noindent$\bullet$ Direct prompt with ground truth CI parameters (\textbf{DP+CI}).
We instruct LLMs using the direct prompt template with our annotated CI parameters to determine legal compliance.


\noindent$\bullet$ Direct prompt with ground truth CI parameters and regulation content (\textbf{DP+CI+LAW}).
We extend the direct prompt template by including annotated CI parameters and applicable regulations to evaluate LLMs' compliance.


%%% 
\subsection{Evaluated LLMs}
% Qwen2.5-7B-Instruct
% Qwen--QwQ-32B
% Mistral-7B-Instruct-v0.2
% Llama-3.1-8B-Instruct
% gpt-4o-mini
% o1--mini ???
We evaluate a wide range of open-source and closed-source LLMs.
For open-source LLMs, we download their official model weights and generate responses on two NVIDIA  H800 80GB graphic cards.
We evaluate DeepSeek-R1 (671B)~\cite{guo2025deepseek}, Llama-3.1-8B-Instruct~\cite{llama3modelcard}, Qwen2.5-7B-Instruct,  Qwen-QwQ-32B~\cite{Yang2024Qwen2TR}, and Mistral-7B-Instruct-v0.2~\cite{jiang2023mistral}.
For the closed-source LLM, we evaluate the GPT-4o-mini performance with API accesses.
Notably, Qwen-QwQ-32B and DeepSeek R1 are specifically optimized to enhance their reasoning abilities.
Since they are both tuned for multi-step reasoning via reinforcement learning, we omit their results on retrieval augmented generation.

\subsection{Tasks and Metrics}
% 3 way classification
% RAG R performance??
% CI probing
Our designed tasks include legal compliance evaluation and context understanding probing.

For the legal compliance evaluation, we ask LLMs to perform a three-way classification to determine if the given contest is \textit{permitted} by, \textit{prohibited} by, or \textit{not applicable} to a specific regulation.
We implement regular expression parsers to capture the generated predictions and regard parsing failures as incorrect.
We report the accuracy, precision, recall and F1 score with a single run.

For context understanding probing, we ask LLMs to answer multiple choice questions mentioned in Section~\ref{sec: data processing} and calculate their accuracies across the 3 difficulty levels.
\section{Experimental Results}


In this section, we systematically evaluate current LLMs' performance on our \name.



\begin{table*}[t]
    \centering
    \small
    \setlength{\tabcolsep}{3pt}
    \begin{tabular}{l|ccc|ccc|ccc | cc} 
        \toprule
        \multirow{2}{*}{} 
        & \multicolumn{3}{c|}{\textbf{EU AI Act}} & \multicolumn{3}{c|}{\textbf{GDPR}} & \multicolumn{3}{c|}{\textbf{HIPAA}} & \multicolumn{2}{c}{\textbf{ACLU}}\\ 
        %\cmidrule(lr){2-4} \cmidrule(lr){5-7} \cmidrule(lr){8-10}
        \textbf{Model}  & DP & CoT & RAG & DP & CoT & RAG & DP & CoT & RAG & DP & CoT \\ 
        \midrule
        Mistral-7B-Instruct & 49.83 & 43.50 & 45.56 & 72.29 & 68.02 & 43.38 & 45.79 & 60.74 & 64.95 & 44.92 & \textbf{72.46}\\
        Qwen-2.5-7B-Instruct & 49.90 & 65.30 & \textbf{55.83} & 89.00 & 88.81 & 82.43 & 68.69 & 72.43 & 71.49 & 50.72 & 52.17 \\
        Llama-3.1-8B-Instruct & 61.30 & 59.40 & 53.50 & 85.30 & \textbf{90.27} & \textbf{76.60} & 77.57 & 85.51 & \textbf{88.31}  & 66.17 & 66.67\\
        GPT-4o-mini & 73.76 & 66.60 & - & \textbf{92.03} & 65.69 & - & 80.84 & 67.75 & - & \textbf{69.56} & 31.88\\
        QwQ-32B & \textbf{78.22} & \textbf{75.30} & - & 80.45 & 90.08 & - & 70.09 & \textbf{88.31} & - &  55.07& 55.07 \\

        \multirow{1}{*}{Deepseek R1 (671B)} & 72.90 & 60.67 & - & 90.66 & 47.88 & - & \textbf{89.25} & 81.77 & -& 65.21 & 59.42 \\
         
    %     \midrule
    % \multirow{1}{*}{Average} & 64.31 & 61.79 & xx.xx & 84.95 & 75.12 & xx.xx & 72.03 & 76.08 & xx.xx & 59.33 & 56.76 \\
        \bottomrule
    \end{tabular}
    \vspace{-0.1in}
    \caption{Accuracy Evaluation results of the legal compliance task. All results are reported in \%.}
    \label{tab:privacy_result}
    \vspace{-0.1in}
\end{table*}








\begin{table*}[t]
    \centering
    \small
    \setlength{\tabcolsep}{5pt}
    \begin{tabular}{l|ccc|ccc|ccc}
        \toprule
        \multirow{2}{*}{} 
         & \multicolumn{3}{c|}{\textbf{Permit}} & \multicolumn{3}{c|}{\textbf{Prohibit}} & \multicolumn{3}{c}{\textbf{Not Applicable}} \\
        %\cmidrule(lr){2-4} \cmidrule(lr){5-7} \cmidrule(lr){8-10}
        \textbf{Model\&Method} & Precision & Recall & F1 & Precision & Recall & F1 & Precision & Recall & F1 \\
        \midrule
        Qwen2.5-7B-Instruct-DP & 36.17  &  55.30  &  43.74 & 68.83  &  87.54  &  77.06 & 40.62   & 7.80  &  13.09 \\
        Qwen2.5-7B-Instruct-CoT & 52.93    &51.80    &52.36 &68.06  &  85.58   & 75.82  & 77.37   & 59.50    &67.27  \\
        Qwen2.5-7B-Instruct-RAG & 49.63  &  51.99  &  50.78  &  70.45  &  54.99  &  61.77 & 73.69  &  60.50  &  66.45  \\
        Mistral-7B-Instruct-DP & 83.33  &  0.49  &  0.97 & 73.50  & 50.57  &  59.91 & 42.97  &  99.90  &  60.09  \\
        Mistral-7B-Instruct-CoT & 52.83  &  2.72  &  5.18  & 80.23   &  28.84   & 42.42  &  40.74   &  99.70   &  57.85  \\
        Mistral-7B-Instruct-RAG & 46.55  &  7.87  &  13.47 &  81.95  &  29.45  &  43.33  &  42.86   & 100.00  &  60.01  \\
        
        % \midrule
        % Average & 53.74 & 28.03 & 27.08 & 73.50 & 56.33 & 59.72 & 53.38 & 71.23 & 54.46 \\
        \bottomrule
    \end{tabular}
    \vspace{-0.1in}
    \caption{The detailed investigation of Qwen2.5-7B-Instruct and Mistral-7B-Instruct models performance over 3 classes on the AI Act cases. All results are reported in \%.}
    \label{tab:compliance_detail}
    \vspace{-0.2in}
\end{table*}


\subsection{Evaluation on Legal Compliance}
To study whether LLMs can comply with existing privacy regulations, we prompt these LLMs with our collected cases.
Table~\ref{tab:privacy_result} evaluates LLMs' legal compliance accuracies over the four domains.
The compliance results suggest the following findings.

%%% TO DO LIST
%%% 1. EU AI Act ANALYSIS, why it is so bad for LLMs
%%% 2. CoT and RAG not working on AI Act, GDPR?
%%% 3. Parsing Errors analysis or Not Relevant Analysis?
%%% 
1) \textit{The collected EU AI Act and ACLU subsets are the most challenging subsets for legal compliance. }
As outlined in Section~\ref{sec: ai act}, cases from the EU AI Act are synthesized according to its official compliance checker.
Therefore, these cases are not likely to be accessed by LLMs and LLMs can only use their reasoning abilities to determine compliance.
We further investigate the precision, recall and F1 scores for LLMs' predictions over each class on Table~\ref{tab:compliance_detail}.
Both LLMs underperform in the permitted cases.
%For example, Mistral-7B-Instruct has recall scores of no more than 8\% on permitted cases while nearly 100\% on the not-applicable cases, which implies that it classify most the permitted cases as not applicable cases.
For instance, Mistral-7B-Instruct has recall scores of no more than 8\% on permitted cases, while getting nearly 100\% on not-applicable cases.
The results suggest that LLMs cannot distinguish between permitted and not applicable cases.
Regarding the ACLU cases, they always connect with a wide range of legal regulations, including the Fourth Amendment to the United States Constitution and the Freedom of Information Act.
The ACLU data demand a more comprehensive understanding of their applicable regulations, and compliance is harder to determine.
Consequently, even the best-performing reasoner models (QwQ-32B and Deepseek R1) fail to attain satisfactory results on the two subsets.
%These results suggest that current LLMs


2) \textit{Chain-of-Thought reasoning and naive RAG implementation may not always help improve LLMs' safety and privacy compliance.}
For CoT prompting, its effectiveness is model-specific.
Our evaluation of instruction-tuned LLMs, including Mistral-7B, Qwen-2.5-7B and Llama-3.1-8B, reveals general accuracy improvements compared to direct prompting (DP).
However, this trend does not hold for all models.
Specifically, GPT-4o-mini and Deepseek R1 reasoner exhibit degraded performance when using CoT prompting.
On the other hand, the performance of our implemented naive retrieval augmented generation (RAG) method is domain-specific.
For the HIPAA domain, RAG generally leads to the best performance, which aligns with findings from prior research ~\cite{li-2024-privacychecklist}.
However, this improvement fails to extend to the EU AI Act and GDPR domains, where RAG results in notable drops in accuracy.
%\tbc{We further give a detailed analysis in xxx.}




% \begin{table*}[htbp]
%     \centering
%     \small
%     \setlength{\tabcolsep}{3pt} 
%     \begin{tabular}{llccc|ccc|ccc|ccc|ccc}
%         \toprule
%         \textbf{} & \textbf{} & \multicolumn{3}{c|}{\textbf{Mistral-7B-Instruct}} & \multicolumn{3}{c|}{\textbf{Qwen-2.5-7B-Instruct}} & \multicolumn{3}{c|}{\textbf{Llama-3.1-8B-Instruct}} & \multicolumn{3}{c|}{\textbf{GPT-4o-mini}} & \multicolumn{3}{c}{\textbf{QwQ-32B}} \\ 
%         \cmidrule(lr){3-5} \cmidrule(lr){6-8} \cmidrule(lr){9-11} \cmidrule(lr){12-14} \cmidrule(lr){15-17}
%         \textbf{Dataset} & \textbf{} & Easy & Medium & Hard & Easy & Medium & Hard & Easy & Medium & Hard & Easy & Medium & Hard & Easy & Medium & Hard \\ 
%         \midrule
%         EU AI Act & & 81.01 & 69.86 & 50.13 & 91.84 & 83.50 & 57.01 & 80.56 & 66.61 & 50.20 & 96.59 & 87.07 & 59.21 & 91.26 & 82.80 & 57.17 \\
%         GDPR & & 85.54 & 75.92 & 55.99 & 93.61 & 87.78 & 63.86 & 85.22 & 75.17 & 57.81 & 97.11 & 94.34 & 75.84 & 96.07 & 93.01 & 75.52 \\
%         HIPAA & & 85.81 & 76.26 & 56.35 & 93.72 & 87.95 & 64.22 & 85.53 & 75.59 & 58.27 & 97.17 & 94.46 & 76.11 & 98.28 & 94.68 & 78.80 \\
%         \bottomrule
%     \end{tabular}
%     \caption{Evaluation results for Easy, Medium, and Hard categories.}
%     \label{tab:mcq_results_split}
% \end{table*}


% \begin{table*}[htbp]
%     \centering
%     \small
%     \setlength{\tabcolsep}{3pt}
%     \begin{tabular}{l|cccc|cccc|cccc} 
%         \toprule
%         \multirow{2}{*}{} 
%         & \multicolumn{4}{c|}{\textbf{EU AI Act}} & \multicolumn{4}{c|}{\textbf{GDPR}} & \multicolumn{4}{c}{\textbf{HIPAA}} \\ 
%         %\cmidrule(lr){2-4} \cmidrule(lr){5-7} \cmidrule(lr){8-10}
%         \textbf{Model}  & Easy & Medium & Hard & Avg & Easy & Medium & Hard & Avg & Easy & Medium & Hard & Avg \\ 
%         \midrule
%         Mistral-7B-Instruct & 81.01 & 69.86 & 50.13 & 85.54 & 75.92 & 55.99 & 85.81 & 76.26 & 56.35 \\
%         Qwen-2.5-7B-Instruct & 91.84 & 83.50 & 57.01 & 93.61 & 87.78 & 63.86 & 93.72 & 87.95 & 64.22 \\
%         Llama-3.1-8B-Instruct & 80.56 & 66.61 & 50.20 & 85.22 & 75.17 & 57.81 & 85.53 & 75.59 & 58.27 \\
%         GPT-4o-mini & 96.59 & 87.07 & 59.21 & 97.11 & 94.34 & 75.84 & 97.17 & 94.46 & 76.11 \\
%         QwQ-32B & 91.26 & 82.80 & 57.17 & 96.07 & 93.01 & 75.52 & 98.28 & 94.68 & 78.80 \\
%         \midrule
%     \multirow{1}{*}{Average} & 88.25 & 77.97 & 54.74 & 91.51 & 85.24 & 65.80 & 92.10 & 85.79 & 66.75 \\
%         \bottomrule
%     \end{tabular}
%     \caption{Evaluation results for Easy, Medium, and Hard categories.}
%     \label{tab:mcq_results_split}
% \end{table*}


\begin{table*}[t]
    \centering
    \small
    \setlength{\tabcolsep}{3pt}
    \begin{tabular}{l|cccc|cccc|cccc} 
        \toprule
        \multirow{2}{*}{} 
        & \multicolumn{4}{c|}{\textbf{EU AI Act}} & \multicolumn{4}{c|}{\textbf{GDPR}} & \multicolumn{4}{c}{\textbf{HIPAA}} \\ 
        %\cmidrule(lr){2-4} \cmidrule(lr){5-7} \cmidrule(lr){8-10}
        \textbf{Model}  & Easy & Medium & Hard & Avg & Easy & Medium & Hard & Avg & Easy & Medium & Hard & Avg \\ 
        \midrule
        Mistral-7B-Instruct & 81.01 & 69.86 & 50.13 & 67.00 & 85.54 & 75.92 & 55.99 & 72.48 & 85.81 & 76.26 & 56.35 & 72.81 \\
        Qwen-2.5-7B-Instruct & 91.84 & 83.50 & 57.01 & 77.45 & 93.61 & 87.78 & 63.86 & 81.75 & 93.72 & 87.95 & 64.22 & 81.96 \\
        Llama-3.1-8B-Instruct & 80.56 & 66.61 & 50.20 & 65.79 & 85.22 & 75.17 & 57.81 & 72.73 & 85.53 & 75.59 & 58.27 & 73.13 \\
        GPT-4o-mini & 96.59 & 87.07 & 59.21 & 80.96 & 97.11 & 94.34 & 75.84 & 89.09 & 97.17 & 94.46 & 76.11 & 89.25 \\
        QwQ-32B & 91.26 & 82.80 & 57.17 & 77.08 & 96.07 & 93.01 & 75.52 & 88.20 & 98.28 & 94.68 & 78.80 & 90.59 \\
        \midrule
    \multirow{1}{*}{Average} & 88.25 & 77.97 & 54.74 & 73.65 & 91.51 & 85.24 & 65.80 & 80.85 & 92.10 & 85.79 & 66.75 & 81.55 \\
        \bottomrule
    \end{tabular}
    \vspace{-0.1in}
    \caption{Accuracy Evaluation results of the context understanding task. All results are reported in \%.}
    \label{tab:mcq_results_split}
    \vspace{-0.1in}
\end{table*}
\subsection{Evaluation on Context Understanding}

Besides evaluating the overall performance on the compliance task, we also convert the parsed structured cases into multiple-choice questions as stated in Section~\ref{sec: data processing} with 3 difficulty levels for the EU AI Act, GDPR, and HIPAA domain.
These questions enable us to probe how well LLMs are able to understand the context and identify the key CI parameters inside its information flows.
Table~\ref{tab:mcq_results_split} shows LLMs' performance over these multiple-choice questions.
The results of the context understanding task imply the following findings.


3) \textit{Existing LLMs can explicitly identify the CI parameters of the information flow inside the given context.}
For prompted multiple-choice questions, LLMs, on average, can reach accuracies of approximately \textasciitilde 90\% on the Easy subset, \textasciitilde 80\% on the Medium subset, and \textasciitilde 60\% on the Hard subset.
The high accuracy suggests that LLMs are well aware of the context and its key characteristics inside the context's information flow.


%% qwq vs qwen
4) \textit{LLMs' reasoning enhanced by reinforcement learning further improves the context understanding abilities.}
When comparing Qwen-2.5-7B-Instruct with Qwen's latest QwQ-32B reasoner model, Qwen's QwQ-32B has higher accuracy over most subsets, especially on the hard questions.
The result indicates that reinforcement learning helps LLMs to better understand and analyze the context.
Consequently, better context-understanding abilities further improve legal compliance, as indicated by the results of Table~\ref{tab:privacy_result}.

5) \textit{The context of EU AI Act subset is challenging for LLMs to understand.}
On average, all LLMs have comparable performance across the Easy, Medium, and Hard subsets of the GDPR and HIPAA domains.
However, their accuracies on the EU AI Act subset fall significantly behind the other two domains.
We manually examine samples within the EU AI Act and observe that their parsed roles of CI parameters are mostly abstract legal terms such as ``Law Enforcement Agencies,'' ``Importer,'' ``Operator'' and ``provider.'' 
These terms make it hard to correctly identify the stakeholders for LLMs.
In addition, compared with real cases, the AI Act's synthetic vignettes also lack narrative coherence for describing the information flows.
Hence, LLMs struggle to perform well on the multiple-choice questions of the AI Act domain.
As a result, LLMs' compliance also degrades.
%The context understanding results on the EU AI Act data suggest that it is the most challenging subset and partially explains why LLMs underperform on the EU AI Act's legal compliance task.
%This manual inspection partially explains why LLMs underperform on the legal compliance tasks associated with the EU AI Act.


%% \begin{table*}[t]
%     \centering
%     \caption{}
%     % 子表1:Location Data
%     \begin{subtable}[h]{0.46\linewidth}
%     \centering
%     \resizebox{\linewidth}{!}{
%         \begin{tabular}{c|c|cccc}
%         \toprule
%         \textbf{Location Method} & \textbf{Model} & \textbf{harm} & \textbf{sorry} & \textbf{gsm8k} & \textbf{math} \\ \midrule
%         random                &                &              &                &               &              \\ 
%         sparsegpt             & LM+MATH       &              &                &               &              \\ 
%         Importance score      &                &              &                &               &              \\ 
%         wandg                 &                & 16.00        & 24.22          & 50.34         & 14.20        \\ \bottomrule
%         \end{tabular}
%     }
%     \label{tab:location}
%     \caption{}
%     \end{subtable}
%     \hfill
%     % 子表2:Election Data
%     \begin{subtable}[h]{0.46\linewidth}
%     \centering
%     \resizebox{\linewidth}{!}{
%         \begin{tabular}{c|c|cccc}
%         \toprule
%         \textbf{Election type} & \textbf{Model} & \textbf{harm} & \textbf{sorry} & \textbf{gsm8k} & \textbf{math} \\ \midrule
%         00                    &                &              &                &               &              \\ 
%         01                    & LM+MATH       &              &                &               &              \\ 
%         10                    &                &              &                &               &              \\ 
%         11                    &                & 16.00        & 24.22          & 50.34         & 14.20        \\ \bottomrule
%         \end{tabular}
%     }
%     \label{tab:election}
%     \caption{}
%     \end{subtable}
% \end{table*}


% % % 子表3:Ablation of Disjoint Data
% \begin{table}[h]
% \centering
% \resizebox{\linewidth}{!}{
% \begin{tabular}{c|c|cccc}
% \toprule
% \textbf{Disjoint} & \textbf{Model} & \textbf{harm} & \textbf{sorry} & \textbf{gsm8k} & \textbf{math} \\ \midrule
% w/o Disjoint          & LM+MATH       &              &                &               &              \\
% w/ Disjoint           &                & 16.00        & 24.22          & 50.34         & 14.20        \\ \bottomrule
% \end{tabular}
% }
% \caption{Ablation of Disjoint Data}
% \label{tab:ablation_disjoint_data}
% \end{table}



\begin{table}[!ht]
\centering
\caption{Ablation Study. Experiments are conducted on Mistral-7B series models. $\ast$ represents LLM's instruction following ability is impaired.}
\resizebox{\linewidth}{!}{
\begin{tabular}{c|c|cc|cc}
\toprule
\multirow{2}{*}{\begin{tabular}[c]{@{}c@{}}\textbf{Ablation} \\ \textbf{Part}\end{tabular}} & \multirow{2}{*}{\begin{tabular}[c]{@{}c@{}}\textbf{Alternative} \\ \textbf{Methods}\end{tabular}} & \multicolumn{2}{c|}{\begin{tabular}[c]{@{}c@{}}\textbf{Safety}\end{tabular}} & 
\multicolumn{2}{c}{\begin{tabular}[c]{@{}c@{}}\textbf{Mathematical} \\ \textbf{Reasoning}\end{tabular}} \\ \cmidrule{3-6}
&                         &                             \textbf{HarmBench}$\downarrow$                                                     & \textbf{SORRY-Bench}$\downarrow$                                                       & \textbf{GSM8K}$\uparrow$                                          & \textbf{MATH}$\uparrow$                                   \\ \midrule

\multirow{3}{*}{Location} & Random                   & $\ast$             & $\ast$              & 25.58              & 8.66             \\ 
                         & Wanda       & $\ast$             & $\ast$               & 39.58              & 11.37             \\  
                        & SNIP               & 16.00        & 24.22          & 50.34         & 14.20        \\ \midrule

\multirow{3}{*}{Election} 
                   & 01       &  58.00            & 83.77               & 54.13              & 13.12             \\ 
                   & 10                & 35.25             & 47.33               & 50.64              & 13.30             \\ 
                    & 11               & 16.00        & 24.22          & 50.34         & 14.20        \\ \midrule

\multirow{2}{*}{Disjoint}          & \textcolor{gray}{\usym{2717}}      & 63.00             & 85.33               & 72.93              & 23.18             \\
           & \textcolor{gray}{\usym{2714}}               & 16.00        & 24.22          & 50.34         & 14.20        \\ \bottomrule
\end{tabular}
}
\label{tab:ablation}
%\vspace{-10pt}
\end{table}
\begin{figure*}[t]
\centering
\includegraphics[width=0.999\textwidth]{figs/ablations.pdf}
\vspace{-0.3in}
\caption{
Ablation studies for the legal compliance task. All results are evaluated in \%.
}
\label{fig:ablations}
\vspace{-0.15in}
\end{figure*}
%%% case study?
\subsection{Ablation Studies}

To study the effectiveness of our annotated CI parameters and applicable regulation content, we further perform ablation studies by feeding LLMs with ground truth CI parameters and regulations as stated in Section~\ref{sec: judge}.

Figure~\ref{fig:ablations} presents the accuracies of DP+CI and DP+CI+LAW across various LLMs for the legal compliance task.
By comparing DP+CI with CI, we observe that appending the contextual integrity parameters significantly improves LLMs' accuracies, particularly in the HIPAA and ACLU domains. 
Such results suggest that CI parameters indeed help LLMs better understand the context and improve legal compliance performance.
Furthermore,  for DP+CI+LAW, we augment the applicable regulations to DP+CI and obtain consistent performance gains.
Consequently, DP+CI+LAW has the best performance compared with our implemented DP, CoT, and RAG methods.
The results of DP+CI+LAW highlight the effectiveness of retrieval augmented generation methods, provided that the retrieved documents are both relevant and applicable.
%These results further suggest that naive RAG implementation may not help improve LLMs' compliance due to wrong retrieval results, implying that there are gaps between common context and legal terminologies.
Moreover, our ablation studies also imply that naive RAG implementations may degrade LLMs' compliance when the retrieval step yields irrelevant results. 
Such retrieval failures disclose a discrepancy between general context and domain-specific legal terminologies, which suggests that our \name requires a tailored retrieval module for improvement.

%suggesting that careful curation and relevance of retrieved documents are essential for improving model performance in legal applications.

%\subsection{Case Studies}


\subsection{Human Evaluations}
\label{subsec:human_eval}
%% CI para inspection
To assess whether our parsed CI parameters and judgments are reliable, three authors manually inspect the data quality.
This inspection calculates annotators' agreement with the parsed roles and associated attributes (Role), the transmission principle (TP), and the parsed judgment results (Label).
For Role agreement, we assign an integer from 0 to 3 by considering the sender, receiver and subject.
For TP and Label, we assign a binary agreement score (0 or 1).
To ensure a representative assessment, we randomly sample 30 parsed regulations and cases for each domain.
We then average and re-scale the results under 100\% for consistency, as shown in Table~\ref{tab:human_eval}.

\begin{table}[h]
\small
    \centering
    \begin{tabular}{l l|ccc}
        \toprule
        \textbf{Domain} & 
        \textbf{Type} &
        \textbf{Role} & \textbf{TP} & \textbf{Label} \\
        \midrule

        \multirow{2}{*}{HIPAA} &
        Case & 97.78 & 96.67
 & 100.00 \\
        & Law & 98.89 & 93.33
 & 96.67
 \\
        \midrule
        \multirow{2}{*}{GDPR} &
        Case & 96.67 & 96.67
 & 96.67\\
        & Law & 94.44 & 96.67
 & 93.33
 \\
        \midrule
        \multirow{2}{*}{AI Act} &
        Case & 90.00 & 93.33 & 96.67 \\
        & Law & 98.89 & 96.67
 & 96.67 \\
        
        \bottomrule
    \end{tabular}
    \vspace{-0.1in}
    \caption{Averaged Human agreement with our parsed data. Results are averaged and rescaled under \%.}  
    \vspace{-0.15in}
    \label{tab:human_eval}
\end{table}
%% Case agreement

The manual inspection results indicate that the HIPPA domain achieves the highest agreement scores among parsed cases and regulations.
This can be attributed to the fact that HIPAA is related to the medical domain, where roles and transmitted attributes are more clear and consistent.
For instance, it is frequent to observe a covered entity sharing the patient's protected health information (PHI).
Hence, it is easier to parse CI parameters.
For the EU AI Act, its cases' role has the worst performance, with an agreement score of 0.9.
We further inspect the EU AI Act synthetic cases and find that even though these cases strictly follow the question-answering chains of the compliance checker, they still suffer from narrative incoherence.
We leave the detailed case analyses in Appendix~\ref{app: case}.
\section{Conclusion}
We introduced \methodname, an effective training framework defending against MIAs for LLMs. The extensive experiments demonstrate its robustness in protecting privacy while maintaining strong language modeling performance across various datasets and architectures. Although our study focuses on fine-tuning due to computational constraints, \methodname can be seamlessly applied to large-scale pretraining, as done in prior selective pretraining work~\cite{lin2024not}. By categorizing tokens and treating them appropriately, \methodname opens a novel pathway for MIA defense. Future work can explore improved token selection strategies and multi-objective training approaches.
% Bibliography entries for the entire Anthology, followed by custom entries
%\bibliography{anthology,custom}
% Custom bibliography entries only
%%\clearpage
\bibliography{custom}

\clearpage
\appendix
\section{} 
\label{reduction-P}

To prove our argument, we apply the splitting property of the Poisson process. Let \( N(t) \) be a Poisson process with rate parameter \( \lambda \). If events are split into two groups with probabilities \( p \) and \( 1-p \), then the resulting processes \( N_1(t) \) and \( N_2(t) \) are independent Poisson processes with rate parameters \( p\lambda \) and \( (1-p)\lambda \) respectively \cite{splitting_poisson}.

From process \( j \)'s perspective, we can split arrivals from sensor \( i \) into two groups: informative and uninformative arrivals with probabilities \( \nc_{ij} \) and \( 1-\nc_{ij} \), respectively. The rate of arrivals from sensor \( i \) is \( \lambda_i \), so the rate of informative arrivals for process \( j \) from sensor \( i \) is \( \nc_{ij}\lambda_i \). Additionally, we can further split the informative arrivals based on whether they can preempt ongoing service. The rate of informative arrivals that can preempt ongoing service for process \( j \) from sensor \( i \) is \( \np_{i}\nc_{ij}\lambda_i \) and the rate of informative arrivals that can not preempt ongoing service for process \( j \) from sensor \( i \) is \( (1-\np_{i})\nc_{ij}\lambda_i \). Since all these arrivals are Poisson, we can merge them into a single process. The total arrival rate of informative packets that can preempt ongoing service for process \( j \) is given by

\begin{equation}
\Tilde{\lambda}_j = \sum_{i=1}^{N} \np_{i}\nc_{ij}\lambda_i
\end{equation}

Similarly, the total arrival rate of informative packets that can not preempt ongoing service for process \( j \) is

\begin{equation}
\Tilde{\lambda}_j = \sum_{i=1}^{N} (1-\np_{i})\nc_{ij}\lambda_i
\end{equation}

We can express these rates in vector form as follows:

\begin{equation}
\boldsymbol{\Tilde{\lambda}}^T = \begin{bmatrix}
\Tilde{\lambda}_{1} & \Tilde{\lambda}_{2} & \dots & \Tilde{\lambda}_{M}
\end{bmatrix} = (\boldsymbol{\lambda}^T \odot \bfp^T) \bfc,
\end{equation}
\begin{equation}
\boldsymbol{\dot{\lambda}}^T = \begin{bmatrix}
\dot{\lambda}_{1} & \dot{\lambda}_{2} & \dots & \dot{\lambda}_{M}
\end{bmatrix} = (\boldsymbol{\lambda}^T \odot (1-\bfp^T)) \bfc,
\end{equation}


The importance of the packet is whether it has information of process $j$ so  we can say that The system with $N$ sensors and arrival rates $\boldsymbol{\lambda}$ shown in Figure \ref{fig:system_model} equivalents to the system with two sources as shown in Figure \ref{fig:equiv_model} from process $j$'s perspective.


\section{}\label{spv-appendix}


We adopt the stochastic hybrid system (SHS) model as defined in \cite{yates2019}, with a key distinction: our model incorporates probabilistic preemption. The system dynamics are depicted in Figure \ref{fig:equiv_model} so we can analyze the AoI for any process $i$ and generalize it. First, the discrete state is denoted as $q(t) = q \in Q = \{0, 1, 2\}$, where $q = 0$ represents an idle server, and $q \in \{1, 2\}$ signifies that an update packet is currently being serviced. The continuous state is described as $x(t) = [x_0(t), x_1(t)]$, where $x_0(t)$ represents the current age of the process, and $x_1(t)$ captures the potential age if the packet in service is successfully delivered. Notably, $x_1(t)$ is irrelevant in state $0$ since no packet is in service. In state $1$, $x_1(t)$ corresponds to the age of the informative update being serviced. Conversely, in state $2$, where an uninformative update is in service, the completion of this update does not affect the process age, rendering $x_1(t)$ irrelevant in this state as well.

\begin{table}[h]
\centering
\caption{Table of Transitions for the Markov Chain in Figure \ref{fig:shs}.}
\begin{tabular}{c c c c c c}
\toprule
$l$ & $q_l \rightarrow q'_l$ & $\lambda^{(l)}$ & $\mathbf{xA}_l$ & $\mathbf{A}_l$ & $\mathbf{v}_{q_l}\mathbf{A}_l$ \\
\midrule
1 & $0 \rightarrow 1$ & $\Tilde{\lambda}_{1}+\dot{\lambda}_{1}$ & $\begin{bmatrix} x_0 & 0 \end{bmatrix}$ & \small $\begin{bmatrix} 1 & 0 \\ 0 & 0 \end{bmatrix}$ \normalsize & $\begin{bmatrix} v_{00} & 0 \end{bmatrix}$ \\
2 & $0 \rightarrow 2$ & $\lambda_{C}-\Tilde{\lambda}_{1}-\dot{\lambda}_{1}$ & $\begin{bmatrix} x_0 & 0 \end{bmatrix}$ & \small $\begin{bmatrix} 1 & 0 \\ 0 & 0 \end{bmatrix}$ \normalsize & $\begin{bmatrix} v_{00} & 0 \end{bmatrix}$ \\
3 & $1 \rightarrow 0$ & $\mu$        & $\begin{bmatrix} x_1 & 0 \end{bmatrix}$ & \small$\begin{bmatrix} 0 & 0 \\ 1 & 0 \end{bmatrix}$ \normalsize & $\begin{bmatrix} v_{11} & 0 \end{bmatrix}$ \\
4 & $1 \rightarrow 1$ & $\Tilde{\lambda}_{1}
$  & $\begin{bmatrix} x_0 & 0 \end{bmatrix}$ & \small$\begin{bmatrix} 1 & 0 \\ 0 & 0 \end{bmatrix}$\normalsize & $\begin{bmatrix} v_{10} & 0 \end{bmatrix}$ \\
5 & $1 \rightarrow 2$ & $\Tilde{\lambda}_{C}-\Tilde{\lambda}_{1}$  & $\begin{bmatrix} x_0 & 0 \end{bmatrix}$ & \small$\begin{bmatrix} 1 & 0 \\ 0 & 0 \end{bmatrix}$ \normalsize & $\begin{bmatrix} v_{10} & 0 \end{bmatrix}$ \\
6 & $2 \rightarrow 0$ & $\mu$        & $\begin{bmatrix} x_0 & 0 \end{bmatrix}$ & \small$\begin{bmatrix} 0 & 0 \\ 1 & 0 \end{bmatrix}$ \normalsize & $\begin{bmatrix} v_{20} & 0 \end{bmatrix}$ \\
7 & $2 \rightarrow 1$ & $\Tilde{\lambda}_{1}$  & $\begin{bmatrix} x_0 & 0 \end{bmatrix}$ & \small$\begin{bmatrix} 1 & 0 \\ 0 & 0 \end{bmatrix}$ \normalsize & $\begin{bmatrix} v_{20} & 0 \end{bmatrix}$ \\
\bottomrule
\end{tabular}
\label{shs_table}
\end{table}


\begin{figure}
    \centering
    \includegraphics[width=0.5\linewidth]{figures/shs.png}
    \caption{The Markov chain for updates.}
    \label{fig:shs}
\end{figure}

A Markov chain representing the discrete state $q(t)$ is depicted in Figure~\ref{fig:shs}. The corresponding transitions of the SHS at state $q_l$ are detailed in Table~\ref{shs_table}. In the figure, a directed edge $l$ from node $q$ to node $q'$ indicates that transitions from state $q$ to state $q'$ occur at an exponential rate $\lambda^{(l)}$, as specified in the table. 





%\suresh{Incomplete.}

We first show that the stationary probability vector $\pi$ satisfies $
\mathbf{\pi D} = \mathbf{\pi Q} \quad \text{with}$ 
\begin{align}
\quad
\mathbf{D} = \text{diag}[\lambda_{C}, \mu + \Tilde{\lambda}_{C}, \mu + \Tilde{\lambda}_{1}], \quad  \\ \mathbf{Q} = 
\begin{bmatrix}
0 & \Tilde{\lambda}_{1}+\dot{\lambda}_{1} & \lambda_{C}-\Tilde{\lambda}_{1}-\dot{\lambda}_{1} \\
\mu & \Tilde{\lambda}_{1} & \Tilde{\lambda}_{C}-\Tilde{\lambda}_{1} \\
\mu & \Tilde{\lambda}_{1} & 0
\end{bmatrix}.
\end{align}
Applying $\sum_{i=0}^{2} \pi_i = 1$, the stationary probabilities are 
\begin{equation}
\pi_0 = \frac{\mu}{(\lambda_C + \mu)}, \label{pi0}
\end{equation}
\begin{equation}
\pi_1 = \frac{\lambda_C\Tilde{\lambda}_{1} + \dot{\lambda}_{1}\mu + \Tilde{\lambda}_{1}\mu}{(\lambda_C + \mu)(\Tilde{\lambda}_{C} + \mu)}, \label{pi1}
\end{equation}
\begin{equation}
\pi_2 = \frac{\Tilde{\lambda}_{C}\lambda_C + \lambda_C\mu -\lambda_C\Tilde{\lambda}_{1}  - \dot{\lambda}_{1}\mu - \Tilde{\lambda}_{1}\mu}{(\lambda_C + \mu)(\Tilde{\lambda}_{C} + \mu)} . \label{pi2}
\end{equation}

\section{}\label{aoi-appendix}





Given the SHS model and $\pi$ in Appendix \ref{spv-appendix}, we can evaluate $\bar{v}$ to find the AoI. Let 
\begin{equation}
\mathbf{\bar{v}} = [\mathbf{\bar{v}_0} \ \mathbf{\bar{v}_1} \ \mathbf{\bar{v}_2}] = [\bar{v}_{00} \ \bar{v}_{01} \ \bar{v}_{10} \ \bar{v}_{11} \ \bar{v}_{20} \ \bar{v}_{21}].   
\end{equation}
It follows that
\begin{equation}
\mathbf{\bar{v}D} = \mathbf{\pi B} +  \mathbf{\bar{v}R},
\end{equation}
where 
\begin{equation}
\mathbf{D} = \text{diag}[\lambda_C, \lambda_C, \mu + \Tilde{\lambda}_{C}, \mu + \Tilde{\lambda}_{C}, \mu + \Tilde{\lambda}_{1}, \mu + \Tilde{\lambda}_{1}],
\end{equation}
\begin{equation}
\mathbf{B} =
\begin{bmatrix}
1 & 0 & 0 & 0 & 0 & 0 \\
0 & 0 & 1 & 1 & 0 & 0 \\
0 & 0 & 0 & 0 & 1 & 0
\end{bmatrix},
\end{equation}
and
\begin{equation}
\mathbf{R} = 
\begin{bmatrix}
0 & 0 & \Tilde{\lambda}_{1}+\dot{\lambda}_{1}  & 0 & \lambda_{C}-\Tilde{\lambda}_{1}-\dot{\lambda}_{1} & 0 \\
0 & 0 & 0 & 0 & 0 & 0 \\
0 & 0 & \Tilde{\lambda}_{1} & 0 & \Tilde{\lambda}_{C}-\Tilde{\lambda}_{1} & 0 \\
\mu & 0 & 0 & 0 & 0 & 0 \\
\mu & 0 & \Tilde{\lambda}_{1} & 0 & 0 & 0 \\
0 & 0 & 0 & 0 & 0 & 0
\end{bmatrix}.
\end{equation}

Then, we obtain $\bar{v}_{01}=\bar{v}_{21} = 0 $ and 
\begin{align}
&
\label{pi_v}
\begin{bmatrix}
\bar{\pi}_0 & \bar{\pi}_1 & \bar{\pi}_1 & \bar{\pi}_2
\end{bmatrix}
= \\ \nonumber \hat{\mathbf{v}}&
\begin{bmatrix}
\lambda_{C} & -\Tilde{\lambda}_{1}-\dot{\lambda}_{1} & 0 & \Tilde{\lambda}_{1}+\dot{\lambda}_{1}-\lambda_{C} \\
0 & \mu + \Tilde{\lambda}_{C}-\Tilde{\lambda}_{1} & 0 & \Tilde{\lambda}_{1} - \Tilde{\lambda}_{C} \\
-\mu & 0 & \mu + \Tilde{\lambda}_{C} & 0 \\
-\mu & -\Tilde{\lambda}_{1} & 0 & \mu + \Tilde{\lambda}_{1}
\end{bmatrix}, \\
\text{where } 
\hat{\mathbf{v}} &= 
\begin{bmatrix}
\bar{v}_{00} & \bar{v}_{10} & \bar{v}_{11} & \bar{v}_{20}
\end{bmatrix}. \nonumber
\end{align}

After solving eq. (\ref{pi_v}) using eqs. (\ref{pi0}), (\ref{pi1}), and (\ref{pi2}), we determine $\mathbf{\bar{v}}$. Later, we find the average age of information using the formula for a single process $j$ $\Delta_j = \sum_{q=0}^2 \bar{v}_{10}$ as follows:

%\suresh{Is this what you defined as $\Delta_{\rm sum}$ earlier?}

\footnotesize
\begin{align}
\Delta_j = \frac{\lambda_{C}^{2} \tilde{\lambda}_C + \lambda_{C}^{2} \mu + \lambda_{C} \dot{\lambda}_1 \mu + 2 \lambda_{C} \tilde{\lambda}_C \mu + 2 \lambda_{C} \mu^{2} + \tilde{\lambda}_C \mu^{2} + \mu^{3}}{\mu \left(\lambda_{C}^{2} \tilde{\lambda}_1 + \lambda_{C} \dot{\lambda}_1 \mu + 2 \lambda_{C} \tilde{\lambda}_1 \mu + \dot{\lambda}_1 \mu^{2} + \tilde{\lambda}_1 \mu^{2}\right)}
\end{align}
\normalsize

\section{}\label{iteration-appendix}


In this section, we discuss the upper bound on the number of iterations required by the outer space accelerating branch-and-bound algorithm to achieve a global $\epsilon_0$-optimal solution. According to Theorem 5 in \cite{JIAO2022112701}, for any given positive error $\epsilon_0 \in (0, 1)$, the algorithm converges to the desired solution in at most
\begin{equation}
p \cdot \left\lceil \log_2 \frac{p\tau \delta(\Omega)}{\epsilon_0} \right\rceil 
\end{equation}
iterations.

Here, the symbols used in the theorem are defined as follows:

\begin{itemize}
    \item $\Omega \subseteq \mathbf{R}^p$ is a compact hyper-subrectangle, and $\delta(\Omega)$ is defined as:
    \begin{equation}
    \delta(\Omega) = \max_{i=1,2,\dots,p} \{ \bar{U}_i - \bar{L}_i \},    
    \end{equation}
    where $\bar{U}_i$ and $\bar{L}_i$ represent the upper and lower bounds of the $i$-th dimension of the rectangle $\Omega$.

    \item $\tau$ is defined as:
    \begin{equation}\label{tau_eq}
    \tau = \max_{i=1,\dots,p} \frac{4 \max\{|\bar{l}_i|, |\bar{u}_i|\}}{\min\{\bar{L}_i, \bar{U}_i, \bar{L}_i^2, \bar{U}_i^2\}},
    \end{equation}
    where the terms are determined as follows:
    \begin{align}
        \bar{l}_i &= \min_{y \in \Theta} n_i(y), \quad \bar{u}_i = \max_{y \in \Theta} n_i(y), \nonumber \\
        \bar{L}_i &= \min_{y \in \Theta} d_i(y), \quad \bar{U}_i = \max_{y \in \Theta} d_i(y).
    \end{align}

    \item The terms $n_i(y)$ and $d_i(y)$ come from the problem defined as:
    \begin{align}
    \quad \min f(y) = \sum_{i=1}^p \frac{n_i(y)}{d_i(y)}, \quad \nonumber \\ \text{s.t.} \; y \in \Theta = \{y \in \mathbf{R}^n \mid Ay \leq b \}. 
    \end{align}
    \end{itemize}


We can reformulate our problem to determine the upper bound using these definitions. The variable in our problem is $\bfp$, and the objective is specified in (\ref{objective_func}). There are $M$ different linear fractions in the objective. The numerators of these fractions increase as any element of $\bfp$ increases. Consequently, we obtain $\bar{l}_i$ when $\bfp = 0$ and $\bar{u}_i$ when $\bfp = 1$ as follows:
\begin{align}
            \bar{l}_i &= \mu(\mu + \lambda_C)^2 + \sum_{i=1}^{N} \lambda_{i}\mu\lambda_C \nc_{ij},\quad \bar{u}_i = (\mu + \lambda_{C})^3
\end{align}

 Similarly, the denominators of these fractions decrease as any element of $\bfp$ increases, leading to $\bar{L}_i$ when $\bfp = 0$ and $\bar{U}_i$ when $\bfp = 1$.
 \begin{align}
            \bar{L}_i &=  (\mu + \lambda_C) \mu^2 \sum_{i=1}^{N} \nc_{ij} \lambda_{i}, \quad \bar{U}_i =  (\mu + \lambda_C)^2 \mu \sum_{i=1}^{N} \nc_{ij} \lambda_{i}.
\end{align}

After that, $\delta(\Omega)$ becomes: 

\begin{align}
        \delta(\Omega) = \max_{i=1,2,\dots,M} \{(\mu + \lambda_C)\lambda_C \mu \sum_{i=1}^{N} \nc_{ij} \lambda_{i}\} \leq (\mu + \lambda_C)\lambda_C^2 \mu , 
\end{align}

Last, we find $\tau$. In our problem, all parameters and variables are positive, so both the nominators and the denominators are positive, which can help us simplify eq. (\ref{tau_eq}) and obtain $\tau$ as follows:

    \begin{align}
    \tau = \max_{i=1,\dots,M} \frac{4 \bar{u}_i}{\bar{L}_i^2} = \frac{4 (\mu + \lambda_{C})}{\mu^4 \hat{\lambda}_{\min}^2},\\ \nonumber
    \text{where } \hat{\lambda}_{\min} = \min(\boldsymbol{\lambda}^T \bfc)
    \end{align}

Putting all together, for any given positive error $\epsilon_0 \in (0, 1)$, the outer space accelerating branch-and-bound algorithm can seek out a global $\epsilon_0$-optimum solution in at most 
\begin{equation}
M \cdot \left\lceil \log_2 \frac{4M (\mu+\lambda_C)^2\lambda_C^2}{\epsilon_0\mu^3\hat{\lambda}_{\min}^{2}} \right\rceil
\end{equation}
iterations as shown in Theorem \ref{Theo2}.

\end{document}