% This must be in the first 5 lines to tell arXiv to use pdfLaTeX, which is strongly recommended.
\pdfoutput=1
% In particular, the hyperref package requires pdfLaTeX in order to break URLs across lines.

\documentclass[11pt]{article}

% Change "review" to "final" to generate the final (sometimes called camera-ready) version.
% Change to "preprint" to generate a non-anonymous version with page numbers.
%\usepackage[review]{acl}
\usepackage{acl}
% Standard package includes
\usepackage{times}
\usepackage{latexsym}

% For proper rendering and hyphenation of words containing Latin characters (including in bib files)
\usepackage[T1]{fontenc}
% For Vietnamese characters
% \usepackage[T5]{fontenc}
% See https://www.latex-project.org/help/documentation/encguide.pdf for other character sets

% This assumes your files are encoded as UTF8
\usepackage[utf8]{inputenc}

% This is not strictly necessary, and may be commented out,
% but it will improve the layout of the manuscript,
% and will typically save some space.
\usepackage{microtype}

% This is also not strictly necessary, and may be commented out.
% However, it will improve the aesthetics of text in
% the typewriter font.
\usepackage{inconsolata}

%Including images in your LaTeX document requires adding
%additional package(s)
\usepackage{graphicx}

% If the title and author information does not fit in the area allocated, uncomment the following
%
%\setlength\titlebox{<dim>}
%
% and set <dim> to something 5cm or larger.

\newcommand{\etal}{\textit{et al.}}
\usepackage{comment}
\usepackage{amsthm}
\usepackage{float}
\usepackage{graphicx}
\usepackage{subfigure}
\usepackage{caption}
\usepackage{amsmath,amssymb}
\usepackage{mathtools}
\usepackage{booktabs}
\usepackage{multirow}
\usepackage{siunitx}
\usepackage{adjustbox}
\usepackage{pifont}
\usepackage{scalerel}
\usepackage{tabularray}
\usepackage{makecell}
\usepackage{wrapfig}

\definecolor{stepcolor}{HTML}{d79b00}
\definecolor{contentcolor}{HTML}{3439a2}
\newcommand{\textttbf}[1]{\texttt{\textbf{#1}}\xspace}

\usepackage{tikz}
\newcommand*\circled[1]{\tikz[baseline=(char.base)]{
            \node[shape=circle,draw,inner sep=2pt] (char) {#1};}}
\newcommand{\lhr}[1]{\textcolor{blue}{#1}}
\newcommand{\tbc}[1]{\textcolor{blue}{#1}}
\usepackage{xspace}
\newcommand{\name}{\textit{PrivaCI-Bench}\xspace}

\definecolor{stepcolor}{HTML}{d79b00}
\definecolor{contentcolor}{HTML}{6c8ebf}

\newtheorem{definition}{Definition}


\title{PrivaCI-Bench: Evaluating Privacy with Contextual Integrity and Legal Compliance}

\newcommand{\hwbc}[1]{\textcolor{red}{[hwb: #1]}}
\newcommand{\hwb}[1]{\textcolor{red}{#1}}

% Author information can be set in various styles:
% For several authors from the same institution:
% \author{Author 1 \and ... \and Author n \\
%         Address line \\ ... \\ Address line}
% if the names do not fit well on one line use
%         Author 1 \\ {\bf Author 2} \\ ... \\ {\bf Author n} \\
% For authors from different institutions:
% \author{Author 1 \\ Address line \\  ... \\ Address line
%         \And  ... \And
%         Author n \\ Address line \\ ... \\ Address line}
% To start a separate ``row'' of authors use \AND, as in
% \author{Author 1 \\ Address line \\  ... \\ Address line
%         \AND
%         Author 2 \\ Address line \\ ... \\ Address line \And
%         Author 3 \\ Address line \\ ... \\ Address line}

\author {
    % Authors
    {\bf Haoran Li}\textsuperscript{\rm 1}\thanks{Haoran, Wenbin and Huihao contributed equally.},
    {\bf Wenbin Hu}\textsuperscript{\rm 1}\footnotemark[1],
    {\bf Huihao Jing}\textsuperscript{\rm 1}\footnotemark[1],
    {\bf Yulin Chen}\textsuperscript{\rm 2},
    {\bf Qi Hu}\textsuperscript{\rm 1}\\
    {\bf  Sirui Han}\textsuperscript{\rm 1},
    {\bf Tianshu Chu}\textsuperscript{\rm 3},
    {\bf Peizhao Hu}\textsuperscript{\rm 3},
    {\bf Yangqiu Song}\textsuperscript{\rm 1}\\
    \textsuperscript{\rm 1}HKUST, 
    \textsuperscript{\rm 2}National University of Singapore, 
    \textsuperscript{\rm 3}Huawei Technologies\\
    \texttt{hlibt@connect.ust.hk}\\
    Project Page: \url{https://hkust-knowcomp.github.io/privacy/}\\
}
%\author{
%  \textbf{First Author\textsuperscript{1}},
%  \textbf{Second Author\textsuperscript{1,2}},
%  \textbf{Third T. Author\textsuperscript{1}},
%  \textbf{Fourth Author\textsuperscript{1}},
%\\
%  \textbf{Fifth Author\textsuperscript{1,2}},
%  \textbf{Sixth Author\textsuperscript{1}},
%  \textbf{Seventh Author\textsuperscript{1}},
%  \textbf{Eighth Author \textsuperscript{1,2,3,4}},
%\\
%  \textbf{Ninth Author\textsuperscript{1}},
%  \textbf{Tenth Author\textsuperscript{1}},
%  \textbf{Eleventh E. Author\textsuperscript{1,2,3,4,5}},
%  \textbf{Twelfth Author\textsuperscript{1}},
%\\
%  \textbf{Thirteenth Author\textsuperscript{3}},
%  \textbf{Fourteenth F. Author\textsuperscript{2,4}},
%  \textbf{Fifteenth Author\textsuperscript{1}},
%  \textbf{Sixteenth Author\textsuperscript{1}},
%\\
%  \textbf{Seventeenth S. Author\textsuperscript{4,5}},
%  \textbf{Eighteenth Author\textsuperscript{3,4}},
%  \textbf{Nineteenth N. Author\textsuperscript{2,5}},
%  \textbf{Twentieth Author\textsuperscript{1}}
%\\
%\\
%  \textsuperscript{1}Affiliation 1,
%  \textsuperscript{2}Affiliation 2,
%  \textsuperscript{3}Affiliation 3,
%  \textsuperscript{4}Affiliation 4,
%  \textsuperscript{5}Affiliation 5
%\\
%  \small{
%    \textbf{Correspondence:} \href{mailto:email@domain}{email@domain}
%  }
%}

\begin{document}
\maketitle
\begin{abstract}
Recent advancements in generative large language models (LLMs) have enabled wider applicability, accessibility, and flexibility.
However, their reliability and trustworthiness are still in doubt, especially for concerns regarding individuals' data privacy.
Great efforts have been made on privacy by building various evaluation benchmarks to study LLMs' privacy awareness and robustness from their generated outputs to their hidden representations.
Unfortunately, most of these works adopt a narrow formulation of privacy and only investigate personally identifiable information (PII). 
In this paper, we follow the merit of the Contextual Integrity (CI) theory, which posits that privacy evaluation should not only cover the transmitted attributes but also encompass the whole relevant social context through private information flows.
We present \name, a comprehensive contextual privacy evaluation benchmark targeted at legal compliance to cover well-annotated privacy and safety regulations, real court cases, privacy policies, and synthetic data built from the official toolkit to study LLMs' privacy and safety compliance.
We evaluate the latest LLMs, including the recent reasoner models QwQ-32B and Deepseek R1.
Our experimental results suggest that though LLMs can effectively capture key CI parameters inside a given context, they still require further advancements for privacy compliance.



\end{abstract}


%!TEX root = gcn.tex
\section{Introduction}
Graphs, representing structural data and topology, are widely used across various domains, such as social networks and merchandising transactions.
Graph convolutional networks (GCN)~\cite{iclr/KipfW17} have significantly enhanced model training on these interconnected nodes.
However, these graphs often contain sensitive information that should not be leaked to untrusted parties.
For example, companies may analyze sensitive demographic and behavioral data about users for applications ranging from targeted advertising to personalized medicine.
Given the data-centric nature and analytical power of GCN training, addressing these privacy concerns is imperative.

Secure multi-party computation (MPC)~\cite{crypto/ChaumDG87,crypto/ChenC06,eurocrypt/CiampiRSW22} is a critical tool for privacy-preserving machine learning, enabling mutually distrustful parties to collaboratively train models with privacy protection over inputs and (intermediate) computations.
While research advances (\eg,~\cite{ccs/RatheeRKCGRS20,uss/NgC21,sp21/TanKTW,uss/WatsonWP22,icml/Keller022,ccs/ABY318,folkerts2023redsec}) support secure training on convolutional neural networks (CNNs) efficiently, private GCN training with MPC over graphs remains challenging.

Graph convolutional layers in GCNs involve multiplications with a (normalized) adjacency matrix containing $\numedge$ non-zero values in a $\numnode \times \numnode$ matrix for a graph with $\numnode$ nodes and $\numedge$ edges.
The graphs are typically sparse but large.
One could use the standard Beaver-triple-based protocol to securely perform these sparse matrix multiplications by treating graph convolution as ordinary dense matrix multiplication.
However, this approach incurs $O(\numnode^2)$ communication and memory costs due to computations on irrelevant nodes.
%
Integrating existing cryptographic advances, the initial effort of SecGNN~\cite{tsc/WangZJ23,nips/RanXLWQW23} requires heavy communication or computational overhead.
Recently, CoGNN~\cite{ccs/ZouLSLXX24} optimizes the overhead in terms of  horizontal data partitioning, proposing a semi-honest secure framework.
Research for secure GCN over vertical data  remains nascent.

Current MPC studies, for GCN or not, have primarily targeted settings where participants own different data samples, \ie, horizontally partitioned data~\cite{ccs/ZouLSLXX24}.
MPC specialized for scenarios where parties hold different types of features~\cite{tkde/LiuKZPHYOZY24,icml/CastigliaZ0KBP23,nips/Wang0ZLWL23} is rare.
This paper studies $2$-party secure GCN training for these vertical partition cases, where one party holds private graph topology (\eg, edges) while the other owns private node features.
For instance, LinkedIn holds private social relationships between users, while banks own users' private bank statements.
Such real-world graph structures underpin the relevance of our focus.
To our knowledge, no prior work tackles secure GCN training in this context, which is crucial for cross-silo collaboration.


To realize secure GCN over vertically split data, we tailor MPC protocols for sparse graph convolution, which fundamentally involves sparse (adjacency) matrix multiplication.
Recent studies have begun exploring MPC protocols for sparse matrix multiplication (SMM).
ROOM~\cite{ccs/SchoppmannG0P19}, a seminal work on SMM, requires foreknowledge of sparsity types: whether the input matrices are row-sparse or column-sparse.
Unfortunately, GCN typically trains on graphs with arbitrary sparsity, where nodes have varying degrees and no specific sparsity constraints.
Moreover, the adjacency matrix in GCN often contains a self-loop operation represented by adding the identity matrix, which is neither row- nor column-sparse.
Araki~\etal~\cite{ccs/Araki0OPRT21} avoid this limitation in their scalable, secure graph analysis work, yet it does not cover vertical partition.

% and related primitives
To bridge this gap, we propose a secure sparse matrix multiplication protocol, \osmm, achieving \emph{accurate, efficient, and secure GCN training over vertical data} for the first time.

\subsection{New Techniques for Sparse Matrices}
The cost of evaluating a GCN layer is dominated by SMM in the form of $\adjmat\feamat$, where $\adjmat$ is a sparse adjacency matrix of a (directed) graph $\graph$ and $\feamat$ is a dense matrix of node features.
For unrelated nodes, which often constitute a substantial portion, the element-wise products $0\cdot x$ are always zero.
Our efficient MPC design 
avoids unnecessary secure computation over unrelated nodes by focusing on computing non-zero results while concealing the sparse topology.
We achieve this~by:
1) decomposing the sparse matrix $\adjmat$ into a product of matrices (\S\ref{sec::sgc}), including permutation and binary diagonal matrices, that can \emph{faithfully} represent the original graph topology;
2) devising specialized protocols (\S\ref{sec::smm_protocol}) for efficiently multiplying the structured matrices while hiding sparsity topology.


 
\subsubsection{Sparse Matrix Decomposition}
We decompose adjacency matrix $\adjmat$ of $\graph$ into two bipartite graphs: one represented by sparse matrix $\adjout$, linking the out-degree nodes to edges, the other 
by sparse matrix $\adjin$,
linking edges to in-degree nodes.

%\ie, we decompose $\adjmat$ into $\adjout \adjin$, where $\adjout$ and $\adjin$ are sparse matrices representing these connections.
%linking out-degree nodes to edges and edges to in-degree nodes of $\graph$, respectively.

We then permute the columns of $\adjout$ and the rows of $\adjin$ so that the permuted matrices $\adjout'$ and $\adjin'$ have non-zero positions with \emph{monotonically non-decreasing} row and column indices.
A permutation $\sigma$ is used to preserve the edge topology, leading to an initial decomposition of $\adjmat = \adjout'\sigma \adjin'$.
This is further refined into a sequence of \emph{linear transformations}, 
which can be efficiently computed by our MPC protocols for 
\emph{oblivious permutation}
%($\Pi_{\ssp}$) 
and \emph{oblivious selection-multiplication}.
% ($\Pi_\SM$)
\iffalse
Our approach leverages bipartite graph representation and the monotonicity of non-zero positions to decompose a general sparse matrix into linear transformations, enhancing the efficiency of our MPC protocols.
\fi
Our decomposition approach is not limited to GCNs but also general~SMM 
by 
%simply 
treating them 
as adjacency matrices.
%of a graph.
%Since any sparse matrix can be viewed 

%allowing the same technique to be applied.

 
\subsubsection{New Protocols for Linear Transformations}
\emph{Oblivious permutation} (OP) is a two-party protocol taking a private permutation $\sigma$ and a private vector $\xvec$ from the two parties, respectively, and generating a secret share $\l\sigma \xvec\r$ between them.
Our OP protocol employs correlated randomnesses generated in an input-independent offline phase to mask $\sigma$ and $\xvec$ for secure computations on intermediate results, requiring only $1$ round in the online phase (\cf, $\ge 2$ in previous works~\cite{ccs/AsharovHIKNPTT22, ccs/Araki0OPRT21}).

Another crucial two-party protocol in our work is \emph{oblivious selection-multiplication} (OSM).
It takes a private bit~$s$ from a party and secret share $\l x\r$ of an arithmetic number~$x$ owned by the two parties as input and generates secret share $\l sx\r$.
%between them.
%Like our OP protocol, o
Our $1$-round OSM protocol also uses pre-computed randomnesses to mask $s$ and $x$.
%for secure computations.
Compared to the Beaver-triple-based~\cite{crypto/Beaver91a} and oblivious-transfer (OT)-based approaches~\cite{pkc/Tzeng02}, our protocol saves ${\sim}50\%$ of online communication while having the same offline communication and round complexities.

By decomposing the sparse matrix into linear transformations and applying our specialized protocols, our \osmm protocol
%($\prosmm$) 
reduces the complexity of evaluating $\numnode \times \numnode$ sparse matrices with $\numedge$ non-zero values from $O(\numnode^2)$ to $O(\numedge)$.

%(\S\ref{sec::secgcn})
\subsection{\cgnn: Secure GCN made Efficient}
Supported by our new sparsity techniques, we build \cgnn, 
a two-party computation (2PC) framework for GCN inference and training over vertical
%ly split
data.
Our contributions include:

1) We are the first to explore sparsity over vertically split, secret-shared data in MPC, enabling decompositions of sparse matrices with arbitrary sparsity and isolating computations that can be performed in plaintext without sacrificing privacy.

2) We propose two efficient $2$PC primitives for OP and OSM, both optimally single-round.
Combined with our sparse matrix decomposition approach, our \osmm protocol ($\prosmm$) achieves constant-round communication costs of $O(\numedge)$, reducing memory requirements and avoiding out-of-memory errors for large matrices.
In practice, it saves $99\%+$ communication
%(Table~\ref{table:comm_smm}) 
and reduces ${\sim}72\%$ memory usage over large $(5000\times5000)$ matrices compared with using Beaver triples.
%(Table~\ref{table:mem_smm_sparse}) ${\sim}16\%$-

3) We build an end-to-end secure GCN framework for inference and training over vertically split data, maintaining accuracy on par with plaintext computations.
We will open-source our evaluation code for research and deployment.

To evaluate the performance of $\cgnn$, we conducted extensive experiments over three standard graph datasets (Cora~\cite{aim/SenNBGGE08}, Citeseer~\cite{dl/GilesBL98}, and Pubmed~\cite{ijcnlp/DernoncourtL17}),
reporting communication, memory usage, accuracy, and running time under varying network conditions, along with an ablation study with or without \osmm.
Below, we highlight our key achievements.

\textit{Communication (\S\ref{sec::comm_compare_gcn}).}
$\cgnn$ saves communication by $50$-$80\%$.
(\cf,~CoGNN~\cite{ccs/KotiKPG24}, OblivGNN~\cite{uss/XuL0AYY24}).

\textit{Memory usage (\S\ref{sec::smmmemory}).}
\cgnn alleviates out-of-memory problems of using %the standard 
Beaver-triples~\cite{crypto/Beaver91a} for large datasets.

\textit{Accuracy (\S\ref{sec::acc_compare_gcn}).}
$\cgnn$ achieves inference and training accuracy comparable to plaintext counterparts.
%training accuracy $\{76\%$, $65.1\%$, $75.2\%\}$ comparable to $\{75.7\%$, $65.4\%$, $74.5\%\}$ in plaintext.

{\textit{Computational efficiency (\S\ref{sec::time_net}).}} 
%If the network is worse in bandwidth and better in latency, $\cgnn$ shows more benefits.
$\cgnn$ is faster by $6$-$45\%$ in inference and $28$-$95\%$ in training across various networks and excels in narrow-bandwidth and low-latency~ones.

{\textit{Impact of \osmm (\S\ref{sec:ablation}).}}
Our \osmm protocol shows a $10$-$42\times$ speed-up for $5000\times 5000$ matrices and saves $10$-2$1\%$ memory for ``small'' datasets and up to $90\%$+ for larger ones.

\section{Related Work}
\label{related}
\textbf{Copyright infringement in text-to-image models.}
Recent research \cite{carlini2023extracting, somepalli2023diffusion, somepalli2023understanding, gu2023memorization, wang2024replication, wen2024detecting, chiba2024probabilistic} highlights the potential for copyright infringement in text-to-image models. These models are trained on vast datasets that often include copyrighted material, which could inadvertently be memorized by the model during training \cite{vyas2023provable, ren2024copyright}. Additionally, several studies have pointed out that synthetic images generated by these models might violate IP rights due to the inclusion of elements or styles that resemble copyrighted works \cite{poland2023generative, wang2024stronger}. Specifically, models like stable diffusion \cite{Rombach_2022_CVPR} may generate images that bear close resemblances to copyrighted artworks, thus raising concerns about IP infringement \cite{shi2024rlcp}. 

\textbf{Image infringement detection and mitigation.}
The current mainstream infringing image detection methods primarily measure the distance or invariance in pixel or embedding space \cite{carlini2023extracting, somepalli2023diffusion, shi2024rlcp, wang2021bag, wang2024image}. For example, \citeauthor{carlini2023extracting} uses the $L_2$ norm to retrieve memorized images. \citeauthor{somepalli2023diffusion} use SSCD \cite{pizzi2022self}, which learns the invariance of image transformations to identify memorized prompts by comparing the generated images with the original training ones. \citeauthor{zhang2018unreasonable} compare image similarity using the LPIPS distance, which aligns with human perception but has limitations in capturing certain nuances. \citeauthor{wang2024image} transform the replication level of each image replica pair into a probability density function. Studies \cite{wen2024detecting, wang2024evaluating} have shown that these methods have lower generalization capabilities and lack interpretability because they do not fully align with human judgment standards. For copyright infringement mitigation, the current approaches mainly involve machine unlearning to remove the model's memory of copyright information \cite{bourtoule2021machine, nguyen2022survey, kumari2023ablating, zhang2024forget} or deleting duplicated samples from the training data \cite{webster2023duplication, somepalli2023understanding}. These methods often require additional model training. On the other hand, \citeauthor{wen2024detecting} have revealed the overfitting of memorized samples to specific input texts and attempt to modify prompts to mitigate the generation of memorized data. Similarly, \citeauthor{wang2024evaluating} leverage LVLMs to detect copyright information and use this information as negative prompts to suppress the generation of infringing images.














Effective follow-up question generation (QG) requires models to infer and target gaps between the provided answer and the broader context of a conversation. We follow the task definition presented by the \fd{} \cite{meng-etal-2023-followupqg}: ``to generate follow-up questions that seek new information given the initial question and answer''. For simplicity, we denote the ``initial question'' as IQ, ``initial answer'' as IA, and the ``follow-up question'' as FQ. Critical limitations in the training dataset are identified, including quality issues, which are addressed through dataset cleaning (Sec~\ref{sec:data:cleaning}). The small scale (2,790 instances) and low diversity of the dataset are tackled by a novel data augmentation pipeline introduced in this paper (Sec~\ref{sec:data:augmentation}). Specifically, we augment the \fd{} training set with synthetic data generated by a pipeline that mimics human-like gap-driven questioning. This approach utilizes LLM-generated comprehensive answers to systematically identify missing information in initial answers and generate follow-up questions targeting those gaps. We demonstrate that the augmented data retains high quality (Sec~\ref{sec:data:quality}).

\subsection{Data Cleaning}
\label{sec:data:cleaning}

The \fd{} dataset is limited by its small scale, comprising 3,790 samples: 2,790 for training, 500 for validation, and 500 for testing. Within the 2,790 training instances, there are only 2,651 unique (IQ, IA, FQ) triplets, indicating duplication. Additionally, the number of 2,648 unique (IQ, IA) pairs suggest minimal follow-up question diversity, as 99.8\% of pairs have only one reference FQ. Further analysis also uncovered data quality issues, likely stemming from automated data collection (see Appendix~\ref{app:problematic_sample}). To improve the data quality, we did the following:

\begin{itemize}[leftmargin=10pt]
    \item \textbf{Deduplication.} We removed 139 duplicate (IQ, IA, FQ) triplet instances.
    \item \textbf{Reference quality check.} We manually filtered out 84 instances where the reference FQ diverged entirely from the initial question. 
    \item \textbf{Sensitive content removal.} We excluded 24 instances involving topics like self-harm or crime, which LLMs are likely to refuse to answer.
\end{itemize}

% \paragraph{Deduplication.} We removed 139 duplicate (IQ, IA, FQ) triplet instances. 

% \paragraph{Reference quality check.} We manually filtered out 84 instances where the reference FQ diverged entirely from the initial question. %This step ensures coherence in the augmented dataset.   

% \paragraph{Sensitive content removal.} We excluded 24 instances involving topics like self-harm or crime, which modern LLMs are likely to refuse to answer. %, avoiding pipeline failures during augmentation.

The cleaned dataset (2,543 instances) retained broad topic coverage (2,533 unique question-answer pairs).

\subsection{Augmentation Pipeline}
\label{sec:data:augmentation}

As discussed in Section~\ref{sec:data:cleaning}, the limited scale of the dataset and the lack of follow-up question diversity hinder the coverage of diverse questioning strategies, restricting model generalization. To address this, we design a GPT-4-based pipeline that augments the original dataset by generating additional follow-up questions. Our pipeline simulates human reasoning through three interconnected stages: comprehensive answer generation, information gap identification, and follow-up question generation\footnote{Please refer to Appendix~\ref{app:prompts} for the LLM prompts used for the following stages.}. %Each stage aims to enhance diversity, improve quality, and align with human strategies for effective information-seeking dialogues.

\paragraph{Comprehensive answer generation.} To identify gaps in the IA, we generate a comprehensive answer (CA) that represents a complete and thorough response to the IQ. As shown in Figure~\ref{fig:data_aug_pipeline}, we prompt GPT-4 iteratively to generate answers to IQ that target different perspectives, such as technical, ethical, and practical, and synthesize a unified CA.
% Prompts such as ``Generate a concise answer focused on a single perspective'' and ``Synthesize prior answers into a comprehensive explanation'' ensure the CA is broad yet cohesive. This process mirrors how humans consider multiple dimensions to intuit missing information. 

\paragraph{Information gap identification.} The next step is to identify key concepts or details discussed in the comprehensive answer (CA) but not covered in the initial answer (IA). This is done by prompting GPT-4. For example, in the example shown in Figure~\ref{fig:data_aug_pipeline}, the initial answer covers the topic of privacy issues but does not cover areas of cyber security (i.e. an information gap).
%The LLM uses the prompt: ``Identify specific explanations or concepts in the comprehensive answer absent from the initial answer.'' 

\paragraph{Follow-up question generation.} Using the identified information gaps, we prompt GPT-4 to generate follow-up questions that address those gaps while maintaining contextual relevance to the IQ and IA. The generated questions must meet three criteria: be (1) answerable by the CA, (2) unanswerable by the IA, and (3) grounded in terminology and context from the IQ. % This ensures that the follow-up questions are focused, relevant, and factually grounded, emulating human curiosity and reasoning.

% \paragraph{Dataset Reformation and Merging}
To match the format of the original follow-up questions in the \fd{} dataset, we automatically reformat the generated FQs to remove artifacts such as bullets or numbering. Each (IQ, IA) pair is enriched with multiple follow-up questions generated by our pipeline, averaging 10.95 questions per pair. The restructured dataset merges synthetic questions with cleaned human-generated examples, resulting in 27,874 samples---a 10$\times$ increase from the original dataset size. This expanded dataset captures the open-ended nature of human questioning, providing models with diverse and explicit signals to learn strategies for addressing information gaps effectively.

\subsection{Augmented Data Validation}
\label{sec:data:quality}

To assess the quality of the generated follow-up questions, we conducted a human evaluation study using Cloud Connect. To ensure high-quality annotations, we restricted participation to native English-speaking annotators with a minimum of 1,000 completed annotation tasks and an approval rating exceeding 90\%. A randomly sampled subset of 100 (IQ, IA, FQ) triplets was evaluated based on three key criteria: (1) whether the follow-up question was a valid question\footnote{A valid question must be in a question format and ask meaningful information, including Wh-questions (what/why/where/etc.), open-ended questions, probing questions and etc. \cite{meng-etal-2023-followupqg}}, (2) whether any component of the triplet contained sensitive information, and (3) the degree of relatedness between the follow-up question and the initial question-answer pair. The full survey format, including example annotations, is provided in Appendix~\ref{sec:augmented_data_annotation_guideline}. 
The results show that 94\% of the follow-up questions are labeled as valid, 92\% as not sensitive, and 91\% are related to the original (IQ, IA) pair. Inter-annotator agreement was moderate, with a Cohen's Kappa score of $\kappa = 0.73$ \cite{mchugh2012interrater}.
%, demonstrating strong inter-rater reliability in the evaluation process.

% \label{human_GPT}

\section{Models Architectures}
\label{sec:models}

\begin{figure*}[htbp]
  \centering
  \includegraphics[width=\linewidth,keepaspectratio]{images/review_model_arch.pdf}
  \caption{Outline of the architecture common across modeling approaches for multi-image-to-text tasks.}
  \label{fig:model_arch}
\end{figure*}

Modeling approaches to multi-image-to-text tasks have evolved over time from being recurrent neural network (RNN)-based \cite{lstm} to being transformer-based \cite{transformer}. More recent models directly leverage pre-trained large (vision)-language models (LLMs/VLMs), often in a zero-shot manner. In this section, we discuss this evolution and summarize the various state-of-the-art model architectures proposed for the five multi-image-to-text tasks. Architectures proposed for these tasks primarily comprise three modules---a vision encoder, a language decoder, and an intermediate module (typically referred as the projector/adapter) for adapting visual information into contextualized representations for text generation. We describe the functionality of these modules and review the design principles common across all the tasks in the proposed approaches. Furthermore, we also discuss how off-the-shelf  pre-trained VLMs are currently being used to handle various multi-image-to-text tasks. Table~\ref{tab:models} outlines a summary of the models and details related to the selection procedure are provided in Appendix~\ref{appendix:b}.

\subsection{Vision Encoder}

The primary purpose of a vision encoder in vision-to-language tasks is to extract information from the input visual sequence and to optimally encode it into a contextual representation that guides language generation. To achieve this, encoders in the proposed models follow multiple steps, some of which are common across the five multi-image-to-text tasks. First, a pre-trained vision model is utilized for extracting feature representations of the raw input sequences of images/video frames. Earlier approaches used convolutional neural network (CNN)-based vision models such as ResNet \cite{resnet} or R3D \cite{resnet3d} that are primarily pre-trained on the object detection task using large amounts of image/video data. Most of the recent models across the tasks use transformer-based vision models pre-trained for various image-only and image-text alignment objectives, e.g., CLIP-ViT-L \cite{clip}.

We note that besides the primary input sequence of images/video frames, models proposed for some of the tasks, e.g., \color{xkcdVividBlue}MAAD\color{black}, utilize additional input data such as close-ups of characters in the movie clips (\textit{exemplars}) \cite{maad2}. Furthermore, the TAPM \cite{vist_tapm} model proposed for the \color{xkcdVividBlue}VST \color{black} task utilizes FasterRCNN \cite{faster-rcnn} for extracting such local character/object-level features alongside the global image-level features from ResNet. Following the extraction of visual features using pre-trained vision models, most vision encoders comprise an internal sequence-encoder component for learning relationships and dependencies between the individual image/frame-level features at different temporal positions. Some models implement this step either using RNNs or a transformer network with multi-head self-attention for learning temporal relationships and position-encoding mechanism for tracking the order of entities/events in the sequence.

Beyond these steps that are common across the tasks, vision-encoders may also contain additional task-specific steps for capturing the visual information in a way that suits the task's objective better. For instance, the ViLA model \cite{videoqa_vila} for the \color{xkcdVividBlue}VIDQA \color{black} task utilizes a learnable Frame-Sampler component to efficiently select a small subset of frames that are most likely to contain the relevant information needed to answer the question. Another example with a task-specific step is the MSCM+BART  model \cite{vist_kg2} for \color{xkcdVividBlue}VST\color{black}, in which the initial set of image objects/`concepts' are expanded using an external knowledge graph for generating diverse and informative stories.
Despite these task-specific steps, we found that the vision encoder module in the architectures proposed for the various multi-image-to-text tasks share a common set of components that are broadly outlined in Figure~\ref{fig:model_arch}.

\subsection{The Vision-to-Language Bridge}

Some V2L model architectures utilize an intermediate module that bridges the input and output modalities for effectively conditioning the text generation on the extracted visual features. Different models operationalize this module with different degrees of complexity. Earlier approaches for several multi-image-to-text tasks condition the text generation process by directly fusing vision encoder outputs with the language decoder input \cite{vist_glacnet}. Some architectures employ cross-attention mechanisms to focus on the relevant parts of the visual features at various temporal positions during decoding \cite{vc_task}. However, approaches that adopt pre-trained models---e.g., CLIP-ViT-L \cite{clip} as the visual model---tend to employ learnable intermediate layers for aligning and converting outputs of the vision encoder into a format that the language decoder can understand.

In some of the proposed models, this intermediate module is a single linear layer that transforms the visual features into a common shared space, which can be used by the language decoder \cite{videoqa_llamavqa,llava}. In other models, advanced transformer-based projectors such as a Q-Former \cite{blip2} are used for their ability to leverage cross-modal interactions effectively \cite{maad3}. In essence, Q-Former uses dynamic query vectors that are pre-trained to attend to both visual and textual representations, enhancing its ability to generalize and perform well (relative to a single linear layer) across different tasks. Besides these popular methods for adapting multimodal information, some approaches make use of graph neural networks for capturing relationships between objects in the images at different temporal positions and words in the corresponding sentences of the text \cite{vc_gnn}. While there is no definitive way to design this intermediate module, recent work has compared the two approaches, i.e., using cross-attention between modalities or using a multimodal projector for transforming vision encoder features into the language space, and found that the latter leads to a stable/improved performance of models \cite{what_matters}.

\subsection{Language Decoder}

After encoding and adapting the visual information, models employ a language decoder component for text generation. The decoder can either be learned from scratch or consist of a pre-trained language model with additional trainable task-specific layers. Figure~\ref{fig:model_arch} summarizes the different ways in which this step is operationalized across tasks in the proposed architectures. Earlier models learn an RNN by initializing it with the visual context embedding from the previous steps \cite{vc_task,vist_glacnet}. The decoder then typically follows a `teacher forcing' strategy during training to generate one word at a time autoregressively.

Subsequent models have replaced RNNs with the transformer architecture owing to its computation scalability and efficiency in handling long context-windows. Besides the initial word embedding layer and the position encoding step (for maintaining information about the input sequence token order), a transformer decoder is typically made up of multiple identical blocks. Each block comprises a multi-head self-attention layer for modeling intra-sentence relations (between the words) and a multi-head cross-attention layer for learning relationships between representations of each word and the outputs of the visual encoder/projector. For instance, in the \color{xkcdVividBlue}CC \color{black} task, this refers to conditioning each word in the caption on vision encoder outputs (denoted as `difference-representations').

Instead of training the decoder from scratch, some approaches use language models such as GPT-2 \cite{gpt2} and \textsc{LLAMA 2} \cite{llama2}, which are pre-trained on several text-only tasks such as question-answering and text classification/completion. The pre-trained language models are either used directly for generation by freezing their parameters \cite{maad1,maad2,maad3}, or by inserting and fine-tuning additional adaptive layers on top of them for ensuring relevance of the generated text to the downstream task of interest \cite{vist_tapm}. We also note that some models incorporate information from external knowledge bases/graphs into the decoder module to improve coherence and factuality of the generated text, e.g., TextKG \cite{vc_textkg} for the \color{xkcdVividBlue}VC \color{black} task and KG Story \cite{vist_kg2} for the \color{xkcdVividBlue}VST \color{black} task.

\subsection{Off-the-shelf Pre-trained VLMs}
\label{sec:4_4}

The standard model architecture we have discussed so far is also present in more powerful general-purpose foundation VLMs (pre-trained on several tasks using large amounts of data), which can be used directly for multi-image-to-text tasks. Their pre-training process typically happens in two stages---self-supervised alignment training and visual instruction tuning. During the first stage, only the parameters of the intermediate module connecting both unimodal backbones are updated (commonly using paired image-text data) utilizing a contrastive training objective.

In the second stage, models are instruction-tuned using multi-turn conversations obtained for visual data either through crowd-sourcing or by leveraging tools such as GPT-4 \cite{gpt4}. Contrary to task-specific modeling approaches, these pre-trained VLMs are simply prompted (typically in a zero-shot manner) using visual tokens accompanied by task-specific instructions. Some of the pre-trained VLMs that are used off-the-shelf for the multi-image-to-text tasks include: ViLA for \color{xkcdVividBlue}VIDQA\color{black}, mPLUG-2 for \color{xkcdVividBlue}VC\color{black}, VideoLLAMA for \color{xkcdVividBlue}MAAD\color{black}, and LLaVA-NeXT for \color{xkcdVividBlue}VST \color{black} \cite{videoqa_vila,mplug2,maad4,nytws}.
\section{Experiments}
\label{sec:exp}
\noindent\textbf{Datasets.} {\YuiR Following existing studies~\cite{liu2023hybrid,zhoustrengthened,liu2020learning,mccreesh2017partitioning,solnon2015complexity,hoffmann2017between},} we use four benchmark graph collections, namely biochemicalReactions (\textsf{BI}), images-CVIU11 (\textsf{CV}), images-PR15 (\textsf{PR}) and LV (\textsf{LV}), in the experiments. All datasets are collected from http://liris.cnrs.fr/csolnon/SIP.html and come from real-world applications in various domains, {\Yui as shown in Table~\ref{tab:my_label}}. Specifically, \textsf{BI} contains 136 unlabeled bipartite graphs, each of which corresponds to a biochemical reaction network. \textsf{CV} contains 44 pattern graphs and 146 target graphs, which are generated from segmented images. \textsf{PR} contains 24 pattern graphs and 1 target graph, which are also from segmented images. \textsf{LV} contains 112 graphs generated from biological networks. 
%
{\YuiR All graphs have up to thousands of vertices. We note that (1) solving our problem on two graphs with beyond 10K vertices is challenging based on the worst-case time complexity of $O^*((|V_G|+1)^{|V_Q|})$, (2) the largest graph used in previous studies~\cite{liu2023hybrid,zhoustrengthened,liu2020learning,mccreesh2017partitioning} has 6,771 vertices, which is also covered (in LV) by our experiments, and (3) finding the largest common subgraph between two graphs with thousands of vertices has found many real applications~\cite{ehrlich2011maximum}.}
%
{\Yui Following existing studies~\cite{liu2023hybrid,zhoustrengthened,liu2020learning,mccreesh2017partitioning,solnon2015complexity,hoffmann2017between}}, for \textsf{BI} and \textsf{LV}, we generate and test the problem instances (i.e., $Q$ and $G$) by pairing any two distinct graphs; and for \textsf{CV} and \textsf{PR} {\revision which consist of two types of graphs, namely pattern graphs and target graphs}, we test all those problem instances with one graph $Q$ from pattern graphs and the other $G$ from target graphs.

\begin{table*}[]
    \centering
    \caption{\Yui Datasets used in the experiments (``\# of solved instances'' refers to the number of instances solved by algorithms within 1,800 seconds and ``Achieved speedups'' refers to the percentage of the solved instances that \texttt{RRSplit} runs at least 5$\times$/10$\times$/100$\times$ faster than \texttt{McSplitDAL})}
    \vspace{-0.15in}
    \begin{tabular}{|c|c|c|c|c|c|c|c|c|c|c|}
        \hline
        \multirow{2}{*}{Dataset} & \multirow{2}{*}{Domain} & \multirow{2}{*}{\# of graphs} & \multirow{2}{*}{\# of instances} & \multirow{2}{*}{\# of vertices} & \multicolumn{2}{c|}{\# of solved instances} & \multicolumn{3}{c|}{Achieved speedups} \\
        \cline{6-10}
        & & & & & \texttt{RRSplit} & \texttt{McSplitDAL} & 5$\times$ & 10$\times$ & 100$\times$\\
        \hline
        \textsf{BI} & Biochemical & 136 & 9,180 & 9$\sim$ 386 & 7,730 & 4,696 & 91.3\% & 84.4\% & 69.7\% \\
        \textsf{CV} & Segmented images & 190 & 6,424 & 22$\sim$ 5,972 & 1,351 & 1,291& 76.5\% & 48.6\% & 0.2\% \\
        \textsf{PR} & Segmented images & 25& 24& 4$\sim$ 4,838  & 24 & 24 & 91.7\% & 91.7\% & 58.3\% \\
        \textsf{LV} & Synthetic & 112 & 6,216& 10$\sim$ 6,671 & 1,059 & 883 & 68.0\% & 54.7\% & 38.3\%\\
        \hline
    \end{tabular}
    
    \label{tab:my_label}
\end{table*}

\begin{table*}[]
    \centering
    \caption{\YuiR Comparison of running time on all datasets (statistics of achieved speedups in Figure~\ref{fig:all_datasets_T})}
    \vspace{-0.15in}
    \begin{tabular}{|c|c|c|c|c|c|c|}
        \hline
        \multirow{2}{*}{Dataset} & \multicolumn{3}{c|}{\texttt{RRSplit} runs faster} & \multicolumn{3}{c|}{\texttt{McSplitDAL} runs faster} \\
        \cline{2-7}
        & \% of instances & Avg. speedup & Max. speedup & \% of instances & Avg. speedup & Max. speedup\\
        \hline
        BI& 99.43\% & 3.3$\times 10^4$ & $10^6$ & 0.5\% & 24.81 & 872.37 \\
        CV& 92.15\% & 10.92 & 161 & 7.84\% & 4.96 & 38.97 \\
        PR& 95.83\% & 139.39 & 234 & 4.17\% & 1.23 & 1.23 \\
        LV& 93.48\% & 1.2$\times 10^4$ & $10^6$ & 6.51\% & 24.23 & 652.13 \\
        \hline
    \end{tabular}
    
    \label{tab:results}
\end{table*}

\smallskip
\noindent\textbf{Algorithms.} We compare the newly proposed algorithm \texttt{RRSplit} with \texttt{McSplitDAL}~\cite{liu2023hybrid}. To be specific, \texttt{McSplitDAL} is one variant of \texttt{McSplit} as introduced in Section~\ref{sec:sota}, which follows the framework of \texttt{McSplit} (i.e., Algorithm~\ref{alg:mcsplit}) and introduces some learning-based techniques for optimizing the policies of selecting vertices at line 6, line 8 and line 10 of Algorithm~\ref{alg:mcsplit}. To our best knowledge, \texttt{McSplitDAL} is the state-of-the-art algorithm and runs significantly faster than previous solutions, including \texttt{McSplitLL}~\cite{zhoustrengthened} and \texttt{McSplitRL}~\cite{liu2023hybrid}. Besides these, in order to study the effectiveness of different reductions employed in our algorithm \texttt{RRSplit}, we evaluate three variants of  \texttt{RRSplit} --  
{\YuiR \texttt{RRSplit-VE}, \texttt{RRSplit-MB}, and \texttt{RRSplit-UB}, respectively obtained by turning off vertex-equivalence based reductions, maximality based reductions,  and  vertex-equivalence based upper bound}. 
%namely \texttt{RRSplit-MR} and \texttt{RRSplit-VER},


\smallskip
\noindent\textbf{Implementation and metrics.} All algorithms are implemented in C++ and compiled with -O3 optimization. All experiments run on a Linux machine with a 2.10GHz Intel CPU and 128GB memory. Note that, for the implementation of \texttt{McSplitDAL}, we directly use the source code from the authors of~\cite{liu2023hybrid}. We record and compare the total running times of the algorithms on different problem instances (note that the measured running time excludes the I/O time of reading graphs from the disk). We set the running time limit (INF) as 1,800 seconds by default. Our data and code are available at https://github.com/KaiqiangYu/SIGMOD25-MCSS. 

\subsection{Comparison among algorithms}

\begin{figure}[]
		\subfigure[\textsf{BI}]{
			\includegraphics[width=4.0cm]{figure/BI_TDS.pdf}
		}
		\subfigure[\textsf{CV}]{
			\includegraphics[width=4.0cm]{figure/ICVIU11_TDS.pdf}
		}
		\subfigure[\textsf{PR}]{
			\includegraphics[width=4.0cm]{figure/PR15_TDS.pdf}
		}	
		\subfigure[\textsf{LV}]{
			\includegraphics[width=4.0cm]{figure/LV_TDS.pdf}
		}
        \vspace{-0.15in}
	\caption{Running time on all datasets. {\Yui For those problem instances locating at the right side of dash line `- .' with orange color (resp. `- -' with green color),  \texttt{RRSplit} achieves at least 100$\times$ (resp. 10$\times$) speedup compared with \texttt{McSplitDAL}.}}
	\label{fig:all_datasets_T}
\end{figure}

\noindent\textbf{All datasets (running time)}. We compare our algorithm \texttt{RRSplit} with the baseline \texttt{McSplitDAL} on all graph collections. {\YuiR Following some existing works~\cite{mccreesh2016clique}}, we report the running times of the algorithms on various problem instances in Figure~\ref{fig:all_datasets_T}. 
%
%
Specifically, each dot in the scatter figures represents a problem instance, with the $x$-axis (resp. $y$-axis) corresponding to the running time of \texttt{RRSplit} (resp. \texttt{McSplitDAL}) {\chengC on the instance}. Hence, for those problem instances with small values on $x$-axis and large values on $y$-axis (which thus locate on the top left region of the figures), \texttt{RRSplit} performs better than \texttt{McSplitDAL}.
%In particular, for those problem instances locating at the right side of dash line `- .' {\Yui with orange color} (resp. `- -' with green color),  \texttt{RRSplit} achieves at least 100$\times$ (resp. 10$\times$) speedup compared with \texttt{McSplitDAL}. 
We mark the running time as INF if the problem instance cannot be solved within the default time limit.
%
{\YuiR Besides, we also provide some statistics in Table~\ref{tab:my_label} and Table~\ref{tab:results}.}
%
We observe that (1) \texttt{RRSplit} outperforms \texttt{McSplitDAL} by achieving around one to {\Yui four} orders of magnitude speedup {\YuiR (in average)} on the majority {\YuiR (above 92\%)} of the tested problem instances and (2) \texttt{McSplitDAL} cannot handle all problem instances within the time limit. 
% This fact demonstrates the efficiency of our algorithm \texttt{RRSplit}. 
We do note that \texttt{McSplitDAL} runs slightly faster on a few {\YuiR (below 8\%)} problem instances in \textsf{CV} and \textsf{LV}. {\YuiR Some possible reasons are as follows. 
First, our \texttt{RRSplit} introduces some extra time costs for conducting the proposed reductions as well as computing the upper bound. Second, the heuristic polices adopted in \texttt{RRSplit} and \texttt{McSplitDAL} for branching may have different behaviors. 
%
In specific, on these problem instances, the heuristic policies may help \texttt{McSplitDAL} to find a large common subgraph quickly so as to prune more unpromising branches {\revision via the upper-bound based reduction} (note that they are based on reinforcement learning {\cheng and the behaviors of the learned policy is} based on the explored branches during the running time). } 

%{\cheng One possible reason could be that} their learned heuristic policies can help to find a large common subgraph quickly so as to prune more unpromising branches {\cheng on these datasets}. 


\begin{figure}[]
		\subfigure[\textsf{BI}]{
			\includegraphics[width=4.0cm]{figure/BI_Branch_TDS.pdf}
		}
		\subfigure[\textsf{CV}]{
			\includegraphics[width=4.0cm]{figure/ICVIU11_Branch_TDS.pdf}
		}
		\subfigure[\textsf{PR}]{
			\includegraphics[width=4.0cm]{figure/PR15_Branch_TDS.pdf}
		}	
		\subfigure[\textsf{LV}]{
			\includegraphics[width=4.0cm]{figure/LV_Branch_TDS.pdf}
		}
        \vspace{-0.15in}
	\caption{Number of formed branches on all datasets}
	\label{fig:all_datasets_BT}
\end{figure}

\smallskip
\noindent\textbf{All datasets (number of formed branches)}. We report the number of branches formed by the algorithms on different problem instances in Figure~\ref{fig:all_datasets_BT}. Similarly, each dot in the scatter figures represents a problem instance, with the $x$-axis (resp. $y$-axis) corresponding to the number of branches formed by \texttt{RRSplit} (resp. \texttt{McSplitDAL}) {\chengC on the instance}. We have the following observations. First, the number of branches formed by \texttt{RRSplit} is significantly {\chengC smaller} than that formed by \texttt{McSplitDAL}, e.g., the former is around 10\% - 0.01\% of the latter on the most of problem instances. This shows the effectiveness of our proposed maximality-based reductions and vertex-equivalence-based reductions.
%, and is also compatible with the theoretical results.
Second, the distribution of the number of formed branches in Figure~\ref{fig:all_datasets_BT} is consistent with that of the running time in Figure~\ref{fig:all_datasets_T}. This indicates the achieved speedups on the running time \laks{can be traced} to our newly-designed reductions.

\begin{figure}[]
		\subfigure[\textsf{BI}]{
			\includegraphics[width=4.0cm]{figure/BI_CDF.pdf}
		}
		\subfigure[\textsf{CV}]{
			\includegraphics[width=4.0cm]{figure/ICVIU11_CDF.pdf}
		}
		\subfigure[\textsf{PR}]{
			\includegraphics[width=4.0cm]{figure/PR15_CDF.pdf}
		}	
		\subfigure[\textsf{LV}]{
			\includegraphics[width=4.0cm]{figure/LV_CDF.pdf}
		}
        \vspace{-0.2in}
	\caption{Comparison by varying time limits}
	\label{fig:all_vary_T}
\end{figure}

\begin{figure}[]
		\subfigure[\textsf{BI}]{
			\includegraphics[width=4.0cm]{figure/BI_Branch_CDF.pdf}
		}
		\subfigure[\textsf{CV}]{
			\includegraphics[width=4.0cm]{figure/ICVIU11_Branch_CDF.pdf}
		}
		\subfigure[\textsf{PR}]{
			\includegraphics[width=4.0cm]{figure/PR15_Branch_CDF.pdf}
		}	
		\subfigure[\textsf{LV}]{
			\includegraphics[width=4.0cm]{figure/LV_Branch_CDF.pdf}
		}
        \vspace{-0.2in}
	\caption{Comparison by varying the limit of number of formed branches}
	\label{fig:all_vary_B}
\end{figure}

\smallskip
\noindent\textbf{Varying time limits}. We report the number of solved problem instances in Figure~\ref{fig:all_vary_T} as the time limit is varied. Clearly, all algorithms solve more problem instances as the time limit increases. We observe that \texttt{RRSplit} solves more problem instances than \texttt{McSplitDAL} within the same time limit. In particular, \texttt{RRSplit} with a time limit of 1 second even solves more problem instances than \texttt{McSplitDAL} with a time limit of 10 seconds in all graph collections {\cheng except for} \textsf{CV}; and on \texttt{PR}, \texttt{RRSplit} solves all problem instances within the time limit of 10 seconds. This further demonstrates the superiority of our algorithm \texttt{RRSplit} over the baseline \texttt{McSplitDAL}. 

\smallskip
\noindent\textbf{Varying the limits of number of formed branches}. We report the number of solved problem instances in Figure~\ref{fig:all_vary_B} as the limit on  number of formed branches is varied. We note that the more branches are allowed to be formed, the more instances will be solved. We observe that (1) \texttt{RRSplit} solves more problem instances than \texttt{McSplitDAL} within the same limit of the number of formed branches and (2) the results in Figure~\ref{fig:all_vary_B} show  similar tendencies as those in Figure~\ref{fig:all_vary_T} in general. This further {\cheng explains} the practical superiority of the newly proposed reductions.


\begin{figure}[]
		\subfigure[\textsf{Running time (BI)}]{
			\includegraphics[width=4.0cm]{figure/BI_SIM.pdf}
		}	
		\subfigure[\textsf{Running time (LV)}]{
			\includegraphics[width=4.0cm]{figure/LV_SIM.pdf}
		}
        \subfigure[\textsf{\# of branches (BI)}]{
			\includegraphics[width=4.0cm]{figure/BI_SIMB.pdf}
		}	
		\subfigure[\textsf{\# of branches (LV)}]{
			\includegraphics[width=4.0cm]{figure/LV_SIMB.pdf}
		}
        \vspace{-0.2in}
	\caption{Comparison by varying similarities}
	\label{fig:all_vary_S}
\end{figure}

\smallskip
\noindent\textbf{Varying the similarities of  input graphs}. We define the similarity of  input graphs $Q$ and $G$, $Sim(Q,G)$, as follows.
\begin{equation}
\label{eq:sim}
    Sim(Q,G)=\frac{|S^*|}{\min\{|V_Q|,|V_G|\}},
\end{equation}
where $S^*$ is the maximum common subgraph between $Q$ and $G$. Clearly, $Sim(Q,G)$ varies from 0 to 1, and the larger the value of $Sim(Q,G)$, the higher the similarity between $Q$ and $G$. We test different problem instances as the similarity varies from 0.5 to 1 on \textsf{BI} and \textsf{LV}, and report the average running time in Figures~\ref{fig:all_vary_S}(a)-(b) and the average number of formed branches in Figures~\ref{fig:all_vary_S}(c)-(d). {\Yui The results on \textsf{CV} and \textsf{PR} show similar trends, complete details of which appear in the 
\ifx \CR\undefined
Appendix. 
\else
technical report~\cite{TR}. 
\fi
}  We can see that \texttt{RRSplit} consistently outperforms \texttt{McSplitDAL} {\chengC in} various settings, e.g., \texttt{RRSplit} runs several orders of magnitude faster and forms fewer branches than \texttt{McSplitDAL}. This demonstrates that our designed reductions are effective for pruning the redundant branches on problem instances with various similarities. Besides, we observe that both \texttt{RRSplit} and \texttt{McSplitDAL} have the running time and the number of formed branches first increase and then decrease as the similarity grows. {\revision The possible reasons are as follows. (1) The maximum common subgraphs become larger as the similarity increases according to Equation~(\ref{eq:sim}) and typically more common subgraphs will be explored for finding a large maximum common subgraph. Therefore, the running time firstly increases; (2) the upper-bound based reduction {\chengE performs} better as the similarity grows. {\chengE For example, in the setting of} $Sim(Q,G)=1$, the algorithm can be terminated directly once a common subgraph with $\min\{|V_Q|,|V_G|\}$ vertices is found. Therefore, the running time then decreases.}

%Possible reasons include (1) the number of common subgraphs (i.e., search space) first increases and then decreases as the similarity grows and/or (2) the proposed reductions performs better on those problem instances with the similarity {\cheng close to} 0.5 or 1.   


\begin{figure}[]
		\subfigure[\textsf{\Yui Varying time limits (BI)}]{
			\includegraphics[width=4.0cm]{figure/BI_RT.pdf}
		}	
		\subfigure[\textsf{\Yui Varying time limits (LV)}]{
			\includegraphics[width=4.0cm]{figure/LV_RT.pdf}
		}
        \subfigure[\textsf{\Yui Varying limit of \#branches (BI)}]{
			\includegraphics[width=4.0cm]{figure/BI_BT.pdf}
		}	
		\subfigure[\textsf{\Yui Varying limit of \#branches (LV)}]{
			\includegraphics[width=4.0cm]{figure/LV_BT.pdf}
		}
        \vspace{-0.15in}
	\caption{Comparison among various reductions}
	\label{fig:all_vary_R}
\end{figure}

{\revision
\smallskip
\noindent\textbf{Scalability test.} We test the scalability of our \texttt{RRSplit} on two large datasets, i.e., \textsf{Twitter} and \textsf{DBLP}, which are collected from different domains (http://konect.cc/). Here, \textsf{Twitter} is a social network with 465,017 vertices and 833,540 edges, and \textsf{DBLP} is a collaboration network with 317,080 vertices and 1,049,866 edges. Following existing studies~\cite{arai2023gup,jin2023circinus,sun2023efficient}, we generate the problem instances (i.e., $Q$ and $G$) as follows. Let \textsf{Twitter} or \textsf{DBLP} be the graph $G$. We first extract a set of graphs $Q$ from $G$. Specifically, we conduct a random walk on $G$ and extract a subgraph induced by the visited vertices. By varying the size of the extracted graph $Q$ (among $\{20,30,40,50,60\}$), we extract 5 sets and each of them contains 100 different graphs $Q$.
%
Then, we generate different problems by pairing the graph $G$ (i.e., \textsf{Twitter} or \textsf{DBLP}) with different graphs $Q$ in the set. In summary, for each dataset, we have 500 different problem instances. 

We compare our \texttt{RRSplit} with \texttt{McSplitDAL} by varying the size of $Q$, and report the average running time in Figure~\ref{fig:scalability_test}. We observe that our \texttt{RRSplit} outperforms \texttt{McSplitDAL} significantly.
% , which demonstrates the scalability of the proposed method. 
Besides, \texttt{McSplitDAL} cannot handle almost all the problem instances within the time and/or space limit (INF/OOM). This is because the implementation of \texttt{McSplitDAL} highly relies on the adjacent matrix of $Q$ and $G$, which introduces huge space and time costs when $G$ is very large. Finally, we observe that \texttt{RRSplit} has the running time increase as the size of $Q$ grows. This is also consistent with the theoretical analysis.
}

\begin{figure}[]
		\subfigure[\textsf{Running time (Twitter)}]{
			\includegraphics[width=4.0cm]{figure/TW_S.pdf}
		}	
		\subfigure[\textsf{Running time (DBLP)}]{
			\includegraphics[width=4.0cm]{figure/DBLP_S.pdf}
		}
        \vspace{-0.2in}
	\caption{\revision Scalability test on large datasets}
	\label{fig:scalability_test}
\end{figure}

\subsection{Ablation studies}

We study the effects of various reductions on reducing the redundant computations. In specific, we compare \texttt{RRSplit} with three variants, namely \texttt{RRSplit-VE}: the full version without vertex-equivalence based reductions, \texttt{RRSplit-MB}: the full version without maximality based reductions and \texttt{RRSplit-UB}: the full version without the vertex-equivalence based upper bound, on \textsf{BI} and \textsf{LV}. We report the number of solved problem instances in Figure~\ref{fig:all_vary_R}(a,b) for varying the time limit and in Figure~\ref{fig:all_vary_R}(c,d) for varying the limit of number of formed branches. {\Yui The results on \textsf{CV} and \textsf{PR} show similar clues, which we put in the 
\ifx \CR\undefined
Appendix. 
\else
technical report~\cite{TR}. 
\fi
First, we can see that all four algorithms perform better than the baseline \texttt{McSplitDAL}, among which \texttt{RRSplit} performs the best. This demonstrates the effectiveness of vertex-equivalence-based reductions, maximality-based reductions and vertex-equivalence-based upper bound. Second, \texttt{RRSplit-VE} and \texttt{RRSplit-MB} {\chengB achieve} comparable performance and {\chengB both} contribute to the improvements. Specifically, we note that \texttt{RRSplit-VE} runs slightly faster than \texttt{RRSplit-MB} on \textsf{BI} while \texttt{RRSplit-MB} runs slightly faster than \texttt{RRSplit-VE} on \textsf{LV}. }  {\revision This is possibly because graphs in \textsf{BI} are relatively small biochemical networks where two vertices are more likely to be structural equivalent and thus the vertex-equivalence based reductions outperform other reductions, while graphs in \texttt{LV} are synthetic networks. }
\section{Conclusion }
This paper introduces the Latent Radiance Field (LRF), which to our knowledge, is the first work to construct radiance field representations directly in the 2D latent space for 3D reconstruction. We present a novel framework for incorporating 3D awareness into 2D representation learning, featuring a correspondence-aware autoencoding method and a VAE-Radiance Field (VAE-RF) alignment strategy to bridge the domain gap between the 2D latent space and the natural 3D space, thereby significantly enhancing the visual quality of our LRF.
Future work will focus on incorporating our method with more compact 3D representations, efficient NVS, few-shot NVS in latent space, as well as exploring its application with potential 3D latent diffusion models.

% Bibliography entries for the entire Anthology, followed by custom entries
%\bibliography{anthology,custom}
% Custom bibliography entries only
%%\clearpage
\bibliography{custom}

\clearpage
\appendix

\onecolumn

\section{Additional Experimental Settings}
\label{sec:appendix:exp_setting}

\subsection{Hyper-parameters}
The default sampling parameters for LLM generation are \{temperature=0.75, top\_k=40, top\_p=0.95\}.
Across all MCTS experiments, we set the number of rollouts to 8, the number of sub-questions per node to 4, the number of candidate sub-answers per sub-question to 4, the maximum depth allowed to 5.

\subsection{Dataset Statistics}
Since we make use of the rStar code base~\footnote{\url{https://github.com/zhentingqi/rStar/}} to implement \mosa, we directly adopt the data files released in their git repository.
There are 1,319 instances in GSM8K, 1,000 instances in SVAMP, 500 instances in MATH-500, and 687 instances in StrategyQA.

\section{Additional Implementation Details for Aggregators}
\label{sec:appendix:prompt}

In this section, we will show the basic instruction and several in-context learning demonstrations for aggregators.

% The aggregating operation that was introduced in the main content of this paper is mainly conducted within each search step, which is termed as Action Aggregation.
% Apart from it, we also aggregate all search trajectories after the entire search process terminates, which is named as Final Aggregation.
% In this section, we first give an ablation for the two types of aggregations and show that Action Aggregation is more essential than the other.
% Then, we will show the aggregating prompts and several in-context learning demonstrations.

% \subsection{Ablation on Aggregating}

% \begin{table*}[t]
% \caption{
% }
% \label{tab:appendix:action_or_final_agg}
% \vskip 0.15in
% \begin{center}
% \begin{small}
% % \begin{sc}
% \begin{tabular}{l|cccc|cr}
% \toprule
% Method & GSM8K & SVAMP & MATH & StrategyQA & Avg. \\
% \midrule
% \textsc{MoSA} as Proposers \& Aggregators  &  91.96 &  94.90 &  56.60 &  76.42 & 79.97  \\
% \hspace{10pt} $\circ$ \emph{without} Action Aggregation & 91.51 & 95.20 & 54.40 & 75.69 & 79.20 \\
% \hspace{10pt} $\circ$ \emph{without} Final Aggregation & 91.96 & 94.60 & 56.20 & 75.69 & 79.61 \\
% \bottomrule
% \end{tabular}
% % \end{sc}
% \end{small}
% \end{center}
% \vskip -0.1in
% \end{table*}

% As shwon in Table~\ref{tab:appendix:action_or_final_agg}, removing Action Aggregation leads to an average decrease of 0.77\% while that of removing Final Aggregation is only 0.36\%.
% This suggests that aggregation during each search step is more essential than aggregation after the search terminates.

% \subsection{Prompt Templates for Aggregators}

% % \paragraph{Basic Prompt}


% \textbf{Basic Prompt for Action Aggregation}

% \begin{lstlisting}
% **TASK**:\nYou are an intelligent and supportive AI assistant. You will receive a collection of responses from various AI assistants regarding a query. Your goal is to synthesize these responses into a single, high-quality response. You should first write down your thoughts as \"\#\#\# Thoughts: \textless{}THOUGHTS\textgreater{}.\", and then write down the synthesized response in the format of \"\#\#\# Synthesized Response: \textless{}SYNTHESIZED RESPONSE\textgreater{}.\". If \"The answer is: \textless{}ANSWER\textgreater{}\" appears in the set of responses, you should also write down \"The answer is: \textless{}ANSWER\textgreater{}\"; otherwise, do not write it. \n1. Critically assess the information in the responses, as some may be biased, incorrect, or nonsensical. \n2. Your synthesized response should not merely echo the provided answers but should present a refined, accurate, and thorough response to the prompt. \n3.Ensure that your response is well-organized, coherent, and meets the highest standards of accuracy and reliability.\n
% \end{lstlisting}

% \textbf{Basic Prompt for Final Aggregation}

% \begin{lstlisting}
% **TASK**:\nYou are an intelligent and supportive AI assistant. You will receive a collection of responses from various AI assistants regarding a user's query. Your goal is to synthesize these responses into a single, high-quality response. You should write down your thoughts, and then write down the final answer in the format of \"The answer is: \textless{}ANSWER\textgreater{}.\".\n1. Critically assess the information in the collection of responses, as some may be biased, incorrect, or nonsensical. \n2. Your synthesized response should not merely echo the provided responses but should present a refined, accurate, and thorough response to the user's query. \n3.Ensure that your response is well-organized, coherent, and meets the highest standards of accuracy and reliability.\n
% \end{lstlisting}

% \subsection{ICL Demonstrations for Aggregators}

% \textbf{An ICL Demonstration for Action Aggregation}

% \begin{lstlisting}
% \end{lstlisting}

\newpage

\begin{center}
	\small
	\begin{tcolorbox}[width=1\linewidth,title={\textbf{Basic Instruction for Aggregators}}]
		**TASK**:
        
        You are an intelligent and supportive AI assistant. You will receive a collection of responses from various AI assistants regarding a query. Your goal is to synthesize these responses into a single, high-quality response. You should first write down your thoughts as "\#\#\# Thoughts: \textless{}THOUGHTS\textgreater{}.", and then write down the synthesized response in the format of "\#\#\# Synthesized Response: \textless{}SYNTHESIZED RESPONSE\textgreater{}.". If "The answer is: \textless{}ANSWER\textgreater{}" appears in the set of responses, you should also write down "The answer is: \textless{}ANSWER\textgreater{}"; otherwise, do not write it. 
        
        1. Critically assess the information in the responses, as some may be biased, incorrect, or nonsensical. 
        
        2. Your synthesized response should not merely echo the provided answers but should present a refined, accurate, and thorough response to the prompt. 
        
        3.Ensure that your response is well-organized, coherent, and meets the highest standards of accuracy and reliability.
	\end{tcolorbox}
\end{center}

\newpage


\begin{center}
	\small
	\begin{tcolorbox}[width=1\linewidth,title={\textbf{An Example ICL Demonstration of MATH-500 for Aggregators: User Query}}]

\textbf{User:}


\textbf{[Here is the user query that was fed into those AI assistants]:}

Given a question, please decompose it into sub-questions. For each sub-question, please answer it in a complete sentence, ending with "The answer is \textless a numeric answer \textgreater". When the original question is answerable, please start the subquestion with "Now we can answer the question: \textless{}original question\textgreater{}".

Question 5: Find the product $CD$ of the integers $C$ and $D$ for which

\[\frac{C}{x-3}+\frac{D}{x+8}=\frac{4x-23}{x^2+5x-24}\]for all real values of $x$ except $-8$ and $3$.

Question 5.1: What is the equation obtained by multiplying both sides of the given equation by $(x - 3)(x + 8)$?

Answer 5.1: $(x - 3)(x + 8) \cdot \frac{C}{x-3} + (x - 3)(x + 8) \cdot \frac{D}{x+8} = (x - 3)(x + 8) \cdot \frac{4x-23}{x^2+5x-24}$

This simplifies to $C(x + 8) + D(x - 3) = 4x - 23$.

The answer is $C(x + 8) + D(x - 3) = 4x - 23$.

Question 5.2: What are the values of $C$ and $D$ that satisfy the equation $C(x + 8) + D(x - 3) = 4x - 23$ for all $x$ except $-8$ and $3$?

Answer 5.2:


\textbf{[Here are the responses from various AI assistants]:}

Response 1:

To find the values of $C$ and $D$, we can substitute specific values of $x$ into the equation $C(x + 8) + D(x - 3) = 4x - 23$ to obtain a system of equations. 

Question 5.2.1: What happens if we substitute $x = 3$ into the equation $C(x + 8) + D(x - 3) = 4x - 23$?

Answer 5.2.1: Substituting $x = 3$ gives $C(3 + 8) + D(3 - 3) = 4 \cdot 3 - 23$, which simplifies to $11C = -11$. The answer is $11C = -11$.

Question 5.2.2: What happens if we substitute $x = -8$ into the equation $C(x + 8) + D(x - 3) = 4x - 23$?

Answer 5.2.2: Substituting $x = -8$ gives $C(-8 + 8) + D(-8 - 3) = 4 \cdot (-8) - 23$, which simplifies to $-11D = -59$. The answer is $-11D = -59$.

Question 5.2.3: Now we can answer the question: What are the values of $C$ and $D$?

Answer 5.2.3: Solving the system of equations $11C = -11$ and $-11D = -59$ gives $C = -1$ and $D = 7/11 \cdot 11 = 7$. The answer is -1 and 7 respectively.

Response 2:

To find the values of $C$ and $D$, we can compare the coefficients of the corresponding terms on both sides of the equation.

For the $x^2$ terms, we have $C + D = 0$.

For the $x$ terms, we have $8C - 3D = 4$.

Solving these equations simultaneously, we find that $C = 3$ and $D = -3$.

Therefore, the product $CD = 3 \cdot (-3) = \boxed{-9}$.

Response 3:

To find $C$ and $D$, we can substitute $x = -8$ and $x = 3$ into the equation $C(x + 8) + D(x - 3) = 4x - 23$.

For $x = -8$:
\[C(-8 + 8) + D(-8 - 3) = 4(-8) - 23\]
\[-11D = -55\]
\[D = 5\]

For $x = 3$:
\[C(3 + 8) + D(3 - 3) = 4(3) - 23\]
\[11C = -13\]
\[C = -\frac{13}{11}\]

Now we can answer the question: What is the product $CD$ of the integers $C$ and $D$?
\[CD = \left(-\frac{13}{11}\right)(5) = -\frac{65}{11}\]
The answer is $-\frac{65}{11}$.

Response 4:

To find $C$ and $D$, we can set $x = 3$ and $x = -8$ to get two equations:

1. $C(3 + 8) + D(3 - 3) = 4(3) - 23 \Rightarrow 11C = -11 \Rightarrow C = -1$

2. $C(-8 + 8) + D(-8 - 3) = 4(-8) - 23 \Rightarrow -11D = -55 \Rightarrow D = 5$

The answer is $C = -1$ and $D = 5$.


	\end{tcolorbox}
\end{center}



\newpage


\begin{center}
	\small
	\begin{tcolorbox}[width=1\linewidth,title={\textbf{An Example ICL Demonstration of MATH-500 for Aggregators: Assistant Response}}]

\textbf{Assistant:}

\#\#\# Thoughts:

1. The responses generally attempt to solve the equation \(C(x + 8) + D(x - 3) = 4x - 23\) by substituting specific values for \(x\), namely \(x = 3\) and \(x = -8\).

2. The correct method involves creating a system of linear equations by substituting these values and solving for \(C\) and \(D\).

3. There is a discrepancy in the values of \(C\) and \(D\) obtained in the responses:

   - Some responses correctly find \(C = -1\) and \(D = 5\).
   
   - Another response incorrectly finds \(C = -13/11\) and \(D = 5\).
   
   - Another response finds \(C = 3\) and \(D = -3\), which is not consistent with the given equation.

4. The correct values should satisfy both equations derived from the substitutions.

\#\#\# Synthesized Response:
 
 The equation \(C(x + 8) + D(x - 3) = 4x - 23\) can be solved by substituting specific values for \(x\):

1. Substituting \(x = 3\):
\[C(3 + 8) + D(3 - 3) = 4(3) - 23\]
\[11C = -11 \]
\[C = -1\]

2. Substituting \(x = -8\):
\[C(-8 + 8) + D(-8 - 3) = 4(-8) - 23\]
\[-11D = -55 \]
\[D = 5\]

Thus, the values of \(C\) and \(D\) are \(C = -1\) and \(D = 5\). The product \(CD\) is:
\[CD = (-1) \cdot 5 = -5\]

The answer is: -5

	\end{tcolorbox}
\end{center}




\newpage





\begin{center}
	\small
	\begin{tcolorbox}[width=1\linewidth,title={\textbf{An Example ICL Demonstration of StrategyQA for Aggregators: User Query}}]

\textbf{User:}


\textbf{[Here is the user query that was fed into those AI assistants]:}

Given a question, please decompose it into sub-questions. For each sub-question, please answer it in a complete sentence. When the original question is answerable, please start the sub-question with "Now we can answer the question: \textless{}original question\textgreater{}".

Question 5: Johnny Carson: American talk show host and comedian. Could Johnny Carson's children fill out a water polo team?

Question 5.1: How many children did Johnny Carson have?

Answer 5.1: Johnny Carson had three children: two daughters, Catherine and Christie, and a son, Jeff.

Question 5.2: How many players are needed to form a water polo team?

Answer 5.2:

\textbf{[Here are the responses from various AI assistants]:}

Response 1:

A standard water polo team consists of 13 players.

Response 2:

A water polo team consists of seven players, including one goalkeeper and six field players.

Response 3:

A water polo team consists of 13 players: 1 goalkeeper and 12 field players (7 field players and 5 substitutes).

Response 4:

A standard water polo team consists of 13 players, including 7 field players and 6 substitutes.

	\end{tcolorbox}
\end{center}


\newpage



\begin{center}
	\small
	\begin{tcolorbox}[width=1\linewidth,title={\textbf{An Example ICL Demonstration of StrategyQA for Aggregators: Assistant Response}}]

\textbf{Assistant:}

\#\#\# Thoughts:

1. The responses contain valuable information about the composition of a water polo team. 

2. There is a consensus that a water polo team consists of 13 players in total. However, there is some variation in how these 13 players are categorized.

3. The key point to clarify is how many players are in the water at one time and how many are substitutes.

4. It appears that in the water during play, there are 7 players (including the goalkeeper), and the remaining players are substitutes.

\#\#\# Synthesized Response:

The synthesized response is: 

To form a water polo team, a total of 13 players are required. This team includes 7 players who are in the water during play, comprising one goalkeeper and six field players. The remaining 6 players serve as substitutes.

	\end{tcolorbox}
\end{center}

\end{document}