\section{Related Works}
\label{sec: relate}

\paragraph{Contextual Integrity Theory}
Contextual Integrity (CI)~\cite{Nissenbaum-2010-CI} claims that privacy is about information flows and information flows must adhere to the informational norms of the context to protect privacy.
Both information flows and their governed norms can be well-formed by specifying five key parameters: sender, recipient, information subject, information types (transmitted attributes, topics and other sensitive information about the subject), and transmission principle~\cite{Benthall-CI-2017}.
From the linguistic view, CI aligns with frame semantics~\cite{baker-etal-1998-berkeley-framenet, palmer-etal-2005-proposition} where the structured social contexts can be represented as frames and CI's contextual roles correspond to frame elements.
Accordingly, information flows can be structured into a standardized template as shown in Figure~\ref{fig:model}:
\begin{center}
\vspace{-10pt}
\resizebox{1\linewidth}{!}{
\begin{tabular}{l}
\noindent{\textcolor{stepcolor}{SENDER} \textcolor{contentcolor} {shares} \textcolor{stepcolor}{SUBJECT}\textcolor{contentcolor}{'s} \textcolor{stepcolor}{ATTRIBUTES} \textcolor{contentcolor}{to}} \\
\noindent{\textcolor{stepcolor}{RECEIVER} \textcolor{contentcolor}{under} \textcolor{stepcolor}{TP}
\textcolor{contentcolor}{transmission principle.} 
}\\
\end{tabular}
}
\end{center}
%\vspace{-5pt}
The transmission principle conditions the flow of information, such as \textit{consent of the data subject}, \textit{confidentiality} and \textit{purpose}.
In this work, we apply the CI template to parse information flows from evaluation data and informational norms specified in legal regulations.

%\{SENDER\} shares \{SUBJECT\}’s \{ATTRIBUTES\} to \{RECEIVER\} under \{TP\} transmission principle.
\begin{figure*}[!t]
    \centering
    \includegraphics[width=0.9\linewidth]{figures/pipeline.pdf}
    \caption{\xmsfm: structure from motion pipeline with \nameshort.}
    \label{fig:pipeline}
    \vspace{-3mm}
\end{figure*}

\paragraph{Existing Works on CI}
Existing works on CI can be categorized into two main approaches.
The first approach aims to transform the context into formal logic languages such as first-order logic to explicitly model the context~\cite{Barth-2006-CI}.
Various access control languages such as Binder~\cite{DeTreville-Binder-2002}, Cassandra~\cite{becker2004cassandra}, and EPAL~\cite{Ashley-EPAL-2003} are proposed to describe the task-specific context.
The second approach leverages LLMs' reasoning capabilities to address the inherent flexibility and ambiguity presented in real-world contexts.
LLMs are capable of analyzing information flows inside the context and reason about ethical legitimacy given existing privacy standards and expectations~\cite{mireshghallah2024can, fan2024goldcoin,li-2024-privacychecklist, shao2024privacylens}.
In addition, \citet{shvartzshnaider2025position} surveyed existing works on using LLMs for contextual integrity and proposed four fundamental tenets
of CI theory.
Our \name builds upon these existing LLM-based approaches by addressing a broader range of real-world contextual scenarios across various domains.
We collect the most extensive evaluation data and construct necessary knowledge bases to facilitate the reasoning process with the CI theory.