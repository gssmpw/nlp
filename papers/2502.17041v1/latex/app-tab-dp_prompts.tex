
\begin{table*}[t!]

\small
\centering
\begin{tabular}{p{2\columnwidth}}
\toprule

\textbf{DP}\\
    Determine the relationship between the provided event and the \textcolor{contentcolor}{<domain>} regulations. You have three options for your conclusion:\\
(A) The event is prohibited by \textcolor{contentcolor}{<domain>}.\\
(B) The event is permitted by \textcolor{contentcolor}{<domain>}.\\
(C) The event is not related to \textcolor{contentcolor}{<domain>}.\\

Event:
\textcolor{contentcolor}{<event>}

Output Format:

Choice: [A. Prohibited | B. Permitted | C. Not related ]
\\

\midrule
\textbf{CoT} \\
Determine the relationship between the provided event and the \textcolor{contentcolor}{<domain>} regulations. You have three options for your conclusion:\\
(A) The event is prohibited by \textcolor{contentcolor}{<domain>}.\\
(B) The event is permitted by \textcolor{contentcolor}{<domain>}.\\
(C) The event is not related to \textcolor{contentcolor}{<domain>}.\\

Event:
\textcolor{contentcolor}{<event>}

Let's approach this step by step:

**Planning**: Outline the steps (no more than 3) necessary to analyze the relationship of the event with \textcolor{contentcolor}{<domain>} standards.\\
**Execution**: Implement these steps to gather information and assess the specific aspects of the event.\\
**Decision Making**: Based on the information collected and analyzed, determine if there was a \textcolor{contentcolor}{<domain>} violation.\\
Output Format:

**Plans**: [List the steps planned to evaluate the event:]\\
plan 1 - ...\\
plan 2 - ...\\
....\\
plan N - ...\\

**Execution**: [Document the outcomes from executing your plans:]\\
plan 1 - ....\\
plan 2 - ....\\
....\\
plan N - ...\\

**Choice**: [A. Prohibited | B. Permitted | C. Not related ]\\

\midrule
\textbf{Multiple-choice Template} \\

Here is a case about \textcolor{contentcolor}{<domain>}.\\
Now imagine that you are an expert in contextual integrity, and you now want to analyze the flow of information for the following situation\\
\textcolor{contentcolor}{<scenario>}\\
Given the above scenario, Which \textcolor{contentcolor}{<attr>} is requested in this scenario? \\
-A: \textcolor{contentcolor}{<choice1>}, -B: \textcolor{contentcolor}{<choice2>}, -C: \textcolor{contentcolor}{<choice3>}, -D: \textcolor{contentcolor}{<choice4>}\\
Your output format should strictly follow:\\
**Choice**: A or B or C or D\\ 

\bottomrule
\end{tabular}
\vspace{-0.1in}
\caption{Prompt templates for DP, CoT and multiple-choice questions. Light blue texts inside each ``\textcolor{contentcolor}{<>}'' block denote a string variable.}
\label{tabs:non_rag_prompt}
\vspace{-0.1in}
\end{table*}
