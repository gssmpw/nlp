\begin{table*}[t!]

\small
\centering
\begin{tabular}{p{2\columnwidth}}
\toprule
\textbf{1. LLM Context Explanation before Calculating BM25 Similarity}.\\
I will provide you with an event concerning the delivery of information. Your task is to generate content related to this event by applying your knowledge of the \textcolor{contentcolor}{<domain>} regulations.

To ensure the content is relevant and accurate, follow these steps:

1. Understand the Event: Clearly define and understand the specifics of the event. Identify the key players involved, the type of information being handled, and the context in which it is being delivered.\\ 
2. Apply \textcolor{contentcolor}{<domain>} Knowledge: Utilize your understanding of \textcolor{contentcolor}{<domain>} regulations, focusing on privacy, security, and the minimum necessary information principles. Ensure that your content addresses these aspects in the context of the event.\\ 

Event Details:
\textcolor{contentcolor}{<event>}

Output Format:

**Execution**:

1. Identify the key players, type of information, and context.\\ 
2. Apply relevant \textcolor{contentcolor}{<domain>} principles to the event.

Generated \textcolor{contentcolor}{<domain>} Content:\\ 
1. The \textcolor{contentcolor}{<domain>} Rule with its content: ...\\ 
2. Other Necessary Standard:...\\ 


**References**:\\ 
List the specific \textcolor{contentcolor}{<domain>} regulations you consulted to generate the content.\\
\midrule
\textbf{2. Run BM25 to retrieve relevant regulations from our structured legal documents given listed regulations generated in step 1}.\\
\midrule
\textbf{3. Law Filter Template}.\\
Given a regulation and an information sending or requesting event, identify if the given regulation is relevant to the event. The process involves the following steps:

1. **Understand the Event**:
   - Extract key details from the description of the event, such as the type of information being exchanged, the parties involved, and the context or domain (e.g., healthcare, finance, education).

2. **Analyze The Regulation**:
   - For given regulation, determine its scope and main focus by reading the summary or key sections. Identify the primary subject matter, applicable contexts, or targeted stakeholders.

3. **Relevance Matching**:
   - Compare the key details of the event with the main focus of the regulation. Consider:
     - Does the regulation explicitly address scenarios similar to the event?
     - Is the regulation intended for the context or domain of the event?
     - Are the parties involved in the event the type of entities the regulation aims to govern?

4. **Filtering Decision**:
   - If a regulation's focus aligns well with the event’s details, please answer yes.
   - If there is little to no alignment, such as different contexts, unrelated subjects, or inappropriate scopes, please answer no.

Event:\textcolor{contentcolor}{<event>}

\textcolor{contentcolor}{<domain>} Regulation Candidates:
\textcolor{contentcolor}{<candidates>}

Note:
It is possible that the event may be completely unrelated to the \textcolor{contentcolor}{<domain>} regulations provided. In such cases, just answer NONE.

%Question:
%Is the given HIPAA Regulation relevant with the given event?

Output Format:\\
Seleted Related \textcolor{contentcolor}{<domain>} Regulations:\\
1. section id - content.\\
....\\
N. section id - content.\\

\midrule
\textbf{Decision-making Template}.\\
Assess the relationship between the provided event and \textcolor{contentcolor}{<domain>} regulations, and select one of the following conclusions:

(A) The event is prohibited by \textcolor{contentcolor}{<domain>}.\\
(B) The event is permitted by \textcolor{contentcolor}{<domain>}.\\
(C) The event is not related to \textcolor{contentcolor}{<domain>}.\\

You will be given some sections of \textcolor{contentcolor}{<domain>} regulations for reference (Note: these references may not be 100\% accurate). Your task is to consider the specifics of the event in light of these sections. Remember that the provided references are not exhaustive, and the presence of a regulation does not automatically imply relevance or violation.

Event:
\textcolor{contentcolor}{<event>}

\textcolor{contentcolor}{<domain>} Regulations Reference:
\textcolor{contentcolor}{<reference\_regulations>}

Let's complete it step by step:\\
1. Analyze the specific details of the event by identifying who is involved (sender and recipient), what information is being sent or requested, and for what purpose.\\
2. Compare key elements of the event with \textcolor{contentcolor}{<domain>} rules, identifying if they involve the use, disclosure, or sensitive information as defined by \textcolor{contentcolor}{<domain>}.\\
3. Evaluate the provided \textcolor{contentcolor}{<domain>} regulation excerpts to see if they directly relate to the event.\\
%4. Consider if there are other \textcolor{contentcolor}{<domain>} rules not mentioned in the excerpts that might apply.\\
4. Conclude based on the comprehensive analysis whether the event is in compliance, in violation, or unrelated to \textcolor{contentcolor}{<domain>}.\\

Output Format:\\

**Execution**: [Document the outcomes from executing each step]:\\
1. - ...\\
2. - ...\\
...\\



**Choice**: [A. Prohibited | B. Permitted | C. Not related]\\
\bottomrule
\end{tabular}
\vspace{-0.1in}
\caption{Workflows and prompt templates used for \textbf{RAG}. Light blue texts inside each ``\textcolor{contentcolor}{<>}'' block denote a string variable.}
\label{tabs:prompt_BM25}
\end{table*}
