\section{Data Statistics}
\label{app:stat}
%%
\paragraph{Legal Compliance Data}
To better illustrate our dataset's details, we present the following table summarizing the privacy compliance cases. Table~\ref{tab:data_cases} presents a quantitative comparison of different regulations and labels within the dataset. Table~\ref{tab:data_word_count} compares sentence lengths across different regulations and labels within the privacy compliance dataset.



\begin{table}[h]
\small
        \centering
        \renewcommand{\arraystretch}{1.2}
        \setlength{\tabcolsep}{2pt}  
        \begin{tabular}{l|c|c|c|c|c}
            \toprule
            \textbf{Category} & \textbf{HIPAA} & \textbf{GDPR} & \textbf{AI Act} & \textbf{ACLU} & \textbf{Total} \\
            \midrule
            Permitted & 86 & 675 & 1,029 & 11 & 1,801 \\
            Prohibited & 19 & 2,462 & 971 & 58 & 3,510 \\
            Not Applicable & 106 & - & 1,000 & - & 1,106 \\
            % \midrule
            % \textbf{Total} & \textbf{211} & \textbf{3,137} & \textbf{3,000} & \textbf{69} & \textbf{6,417} \\
            \midrule
            Total & 211 & 3,137 & 3,000 & 69 & 6,417 \\
            \bottomrule
        \end{tabular}
        \vspace{-0.1in}
        \caption{Evaluation data for privacy compliance.}
        \label{tab:data_cases}
        \vspace{-0.15in}
\end{table}


\paragraph{Multiple-Choice Questions}
For the multiple-choice questions dataset mentioned in Section~\ref{sec: data processing}, we use the BERT\_base~\cite{bert} model to embed words. Additionally, We provide additional details in Table~\ref{tab:data_mc} showing the number of questions. The problem distribution remains consistent across different difficulty levels, as the only variation lies in the strategy for selecting options.


\begin{table}[h]
\small
    \centering
    \setlength{\tabcolsep}{2pt}  
    \begin{tabular}{l|c|c|c|c}
        \toprule
        \textbf{Category} & \textbf{HIPAA} & \textbf{GDPR} & \textbf{AI Act} & \textbf{Total} \\
        \midrule
        Easy Questions  & 86 & 675 & 1,029  & 49,280 \\
        Medium Questions  & 86 & 675 & 1,029  & 49,280 \\
        Hard Questions  & 86 & 675 & 1,029  & 49,280 \\
        \midrule
        Total & 49,280 & 49,280 & 49,280  & 147,840 \\
        \bottomrule
    \end{tabular}
    \vspace{-0.1in}
    \caption{Multiple-choice questions statistics.}
    \label{tab:data_mc}
    \vspace{-0.15in}
\end{table}

\paragraph{Auxiliary Knowledge Bases}
For the parsed regulation dataset we produced in Section~\ref{sec: Auxiliary Knowledge Bases}, Table~\ref{tab:data_laws} summarizes the regulation dataset size and composition. 
And Table~\ref{tab:data_graphs} lists the size of the knowledge graphs we build.

\begin{table}[h]
\small
    \centering
    \begin{tabular}{l|c|c|c}
        \toprule
        \textbf{Category} & \textbf{HIPAA} & \textbf{GDPR} & \textbf{AI Act} \\
        \midrule
        Nodes & 591 & 679 & 842 \\
        Positive Norms & 230 & 146 & 365 \\
        Negative Norms & 31 & 30 & 65 \\
        \bottomrule
    \end{tabular}
    \vspace{-0.1in}
    \caption{Statistics of parsed regulations.}  \label{tab:data_laws}
    \vspace{-0.15in}
\end{table}

\begin{table}[h]
\small
    \centering
    \begin{tabular}{l|c c}
        \toprule
        \textbf{Knowledge Graph} & \textbf{Node \#} & \textbf{Edge \#} \\
        \midrule
        Role KG ($\mathcal{R}$) & 8,993 & 91,876  \\
        Attribute KG ($\mathcal{A}$) & 7,875 & 176,999  \\
        \bottomrule
    \end{tabular}
    \vspace{-0.1in}
    \caption{Statistics of annotated hierarchical graphs.}    \label{tab:data_graphs}
    \vspace{-0.15in}
\end{table}


\begin{table*}[t]
\small
    \centering
    \begin{tabular}{l|c|c|c|c|c}
        \toprule
        \textbf{Category} & \textbf{HIPAA} & \textbf{GDPR} & \textbf{AI Act} & \textbf{ACLU} & \textbf{Weighted average} \\
        \midrule
        Permitted & 312.91 & 66.41 & 133.03 & 340.70 & 118.03 \\
        Prohibited & 307.35 & 56.31 & 129.51 & 319.76 & 82.33 \\
        Not Applicable & 360.56 & - & 122.17 & - & 145.21 \\
        \midrule
        Weighted Average & 336.21 & 58.48 & 128.17 & 323.10 & 103.20 \\
        \bottomrule
    \end{tabular}
    \vspace{-0.1in}
    \caption{Privacy compliance data word statistics}
\label{tab:data_word_count}
\vspace{-0.1in}
\end{table*}



\section{Experimental Details}


\paragraph{Generation Details}
For open-source models, we generate the models' responses with the recommended configurations in their model cards.
For close-source models, we use their official APIs to obtain the responses with temperature = 0.2.
For each generation among all models, we set the max\_new\_token = 512 with max\_retry = 3.



\paragraph{Prompt Templates}

We follow the Privacy Checklist's prompt templates~\cite{li-2024-privacychecklist} with modifications to build our prompt templates.
Our full prompts used for \textbf{DP}, \textbf{CoT} and Multiple-choice questions are listed in Table~\ref{tabs:non_rag_prompt}.
For the \textbf{RAG} method, we detailedly illustrate its whole workflow in Table~\ref{tabs:prompt_BM25}.

\paragraph{Licenses}
For the HIPAA domain, we use data provided by GoldCoin's official GitHub implementation~\cite{fan2024goldcoin} under the Apache-2.0 license.
For other domains, we double-check the licenses and copyright policies of our collected data from web pages. 
These data are under the CC BY-NC-SA 4.0 license and the U.S. copyright laws, and we are able to use them for non-commercial and research purposes.
In terms of used models, we have agreed with all their specific licenses to use their models for research purposes.
For example, we follow the Llama Community License Agreement to use the Llama-3.1-8B-Instruct to run our experiments.



\paragraph{Computational Resources} During our experiment, we use 2 NVIDIA H800 to run our codes for open-source models, and it takes 6-week GPU hours to complete all experiments.
In terms of API cost, our overall cost for calling APIs is approximately \$1,000 USD.

%\paragraph{Annotation Costs}
\begin{table*}[t]

\small
\centering
\begin{tabular}{p{2\columnwidth}}
\toprule
% he
\textbf{A Synthesized EU AI ACT Case.} \\
\midrule

\textbf{Question-Answer Chain from the Official Compliance Checker: } \\
(1) What kind of entity is your organization? \\
\hspace{3.5mm} Answer: Provider.

(2) Has a downstream deployer, distributor, or importer made any modifications to your system? \\
\hspace{3.5mm} Answer: Modifying the intended purpose of the system; performing a substantial modification to the system.

(3) Is the system used within the scope of Europe? \\
\hspace{3.5mm} Answer: It is placed on the EU market, and the output is used in the EU.

(4) Is the system excluded from the EU AI ACT (e.g., for military, research, or educational purposes)? \\
\hspace{3.5mm} Answer: No.

(5) Is the system a prohibited system? \\
\hspace{3.5mm} Answer: Yes, it incorporates biometric categorization and social scoring. \\
\textbf{Norm Type: } Prohibited (determined by the question-answer chain). \\
\midrule

\textbf{GPT-4o Synthesized Case: }\\
The AI system, named 'EmotionSense', is a General Purpose AI model designed to perform emotion recognition and real-time remote biometrics. It is provided under free and open-source licenses. The system has undergone substantial modifications and its intended purpose has been altered since its initial deployment. EmotionSense is placed on the market and put into service within the European Union, and its outputs are utilized within the EU. The system is capable of analyzing facial expressions and physiological signals to determine emotional states in real-time. This functionality is used in various applications, including customer service, security, and healthcare. The system's deployment involves the collection and processing of biometric data from individuals without their explicit consent, raising concerns about privacy and data protection.\\
\midrule
\textbf{GPT-4o Annotation for Contextual Integrity Parameters:} \\
\textbf{Sender: }: AI Developer, Service Provider. \\
\textbf{Receiver: } Customer Service Company, Security Firm, Healthcare Institution. \\
\textbf{Subject: }: EU Citizens. \\
\textbf{Information Type}: Biometric Data. \\
\textbf{Purpose}: Emotion Analysis. \\
% \midrule
% \textbf{Norm Type: } Prohibited (Deterministic result from the compliance checker). \\

\bottomrule
\end{tabular}
\vspace{-0.1in}
\caption{An example of GPT-4o Synthesized EU AI ACT Study Cases.}
\label{tabs:synthesized_ai_act_case}
\vspace{-0.1in}
\end{table*}
\begin{table*}[htbp]
    \centering
    \small
    \setlength{\tabcolsep}{3pt}
    \begin{tabular}{lccc|ccc|ccc|ccc|cc}
        \toprule
        \multirow{2}{*}{}  & \multicolumn{3}{c|}{\textbf{EU AI Act}} & \multicolumn{3}{c|}{\textbf{GDPR}} & \multicolumn{3}{c|}{\textbf{HIPAA}} & \multicolumn{2}{c}{\textbf{ACLU}} \\ 
        \cmidrule(lr){2-4} \cmidrule(lr){5-7} \cmidrule(lr){8-10} \cmidrule(lr){11-12}
        \textbf{Model} & DP & CoT & RAG & DP & CoT & RAG & DP & CoT & RAG & DP & CoT \\ 
        \midrule
        Mistral-7B-Instruct & 49.84 & 44.62 & 46.69 & 82.47 & 78.36 & 54.27 & 49.62 & 63.26 & 67.48 & 54.39 & 73.53 \\
        Qwen-2.5-7B-Instruct & 49.90 & 65.33 & 59.18 & 89.77 & 90.25 & 86.09 & 68.69 & 77.11 & 77.47 & 50.72 & 52.17 \\
        Llama-3.1-8B-Instruct & 61.32 & 60.62 & 54.53 & 85.45 & 90.55 & 77.23 & 77.57 & 85.71 & 88.52 & 67.15 & 66.67 \\
        GPT-4o-mini & 73.77 & 66.60 & - & 92.59 & 77.66 & - & 80.84 & 67.76 & - & 69.57 & 31.88 \\
        QwQ-32B & 77.14 & 75.30 & - & 85.47 & 90.62 & - & 77.92 & 88.32 & - & 55.47 & 62.81 \\
        Deepseek R1 & 33.17 & 32.73 & - & 91.48 & 49.88 & - & 89.25 & 38.32 & - & 65.22 & 59.42 \\
        % \midrule
        % Average & 57.52 & 57.53 & 53.47 & 87.87 & 79.55 & 72.53 & 73.98 & 70.08 & 77.82 & - & -  \\
        \bottomrule
    \end{tabular}
    \vspace{-0.1in}
    \caption{Micro-F1 Evaluation results of the legal compliance task. All results are reported in \%.}%Evaluation result in micro f1 criterion}
\label{tab:compliance_f1}
\vspace{-0.1in}
\end{table*}


\section{More Evaluation Results}


\subsection{F1 Scores of Legal Compliance Task}

In addition to reporting only the accuracies for the legal compliance task, we further report the micro-averaged F1 scores in Table~\ref{tab:compliance_f1}.
The micro-F1 scores share similar results as Table~\ref{tab:privacy_result}.





\subsection{Cases Studies on the EU AI Act}
\label{app: case}
In the absence of real court cases under the EU AI Act, we utilized GPT-4o to synthesize study cases based on the outputs from its official compliance checker. 
We enumerated all possible choices from the compliance checker and created essential question-answer pairs for case generation, which were then provided to GPT-4o to generate realistic court case scenarios. 
The norm type of a case is determined by the question-answer chain, which can be categorized into three classes: permitted, prohibited, and not applicable.

Specifically, prohibited cases involve dangerous systems, such as those that exploit vulnerabilities, conduct biometric categorization, or predict political outcomes. Not applicable cases fall outside the scope of the EU AI Act, for instance, an AI system is not deployed in Europe. Permitted cases comply with the EU AI Act.

Furthermore, we leveraged GPT-4o to annotate the CI parameters. An example of the synthesized cases is provided in Table \ref{tabs:synthesized_ai_act_case}. This example demonstrates that the GPT-4o generation process generally adheres to the information from the question-answer chain and annotates the CI parameters accurately. Besides, in the synthesized scenario, the entities exhibit realistic names and behaviors. However, there is still room for improvement. The synthesized cases lack comprehensiveness, and the narrative development is not coherent. We plan to enhance the quality of the synthesized cases by introducing additional constraints and guidance for future work.




\begin{table*}[t!]

\small
\centering
\begin{tabular}{p{2\columnwidth}}
\toprule

\textbf{DP}\\
    Determine the relationship between the provided event and the \textcolor{contentcolor}{<domain>} regulations. You have three options for your conclusion:\\
(A) The event is prohibited by \textcolor{contentcolor}{<domain>}.\\
(B) The event is permitted by \textcolor{contentcolor}{<domain>}.\\
(C) The event is not related to \textcolor{contentcolor}{<domain>}.\\

Event:
\textcolor{contentcolor}{<event>}

Output Format:

Choice: [A. Prohibited | B. Permitted | C. Not related ]
\\

\midrule
\textbf{CoT} \\
Determine the relationship between the provided event and the \textcolor{contentcolor}{<domain>} regulations. You have three options for your conclusion:\\
(A) The event is prohibited by \textcolor{contentcolor}{<domain>}.\\
(B) The event is permitted by \textcolor{contentcolor}{<domain>}.\\
(C) The event is not related to \textcolor{contentcolor}{<domain>}.\\

Event:
\textcolor{contentcolor}{<event>}

Let's approach this step by step:

**Planning**: Outline the steps (no more than 3) necessary to analyze the relationship of the event with \textcolor{contentcolor}{<domain>} standards.\\
**Execution**: Implement these steps to gather information and assess the specific aspects of the event.\\
**Decision Making**: Based on the information collected and analyzed, determine if there was a \textcolor{contentcolor}{<domain>} violation.\\
Output Format:

**Plans**: [List the steps planned to evaluate the event:]\\
plan 1 - ...\\
plan 2 - ...\\
....\\
plan N - ...\\

**Execution**: [Document the outcomes from executing your plans:]\\
plan 1 - ....\\
plan 2 - ....\\
....\\
plan N - ...\\

**Choice**: [A. Prohibited | B. Permitted | C. Not related ]\\

\midrule
\textbf{Multiple-choice Template} \\

Here is a case about \textcolor{contentcolor}{<domain>}.\\
Now imagine that you are an expert in contextual integrity, and you now want to analyze the flow of information for the following situation\\
\textcolor{contentcolor}{<scenario>}\\
Given the above scenario, Which \textcolor{contentcolor}{<attr>} is requested in this scenario? \\
-A: \textcolor{contentcolor}{<choice1>}, -B: \textcolor{contentcolor}{<choice2>}, -C: \textcolor{contentcolor}{<choice3>}, -D: \textcolor{contentcolor}{<choice4>}\\
Your output format should strictly follow:\\
**Choice**: A or B or C or D\\ 

\bottomrule
\end{tabular}
\vspace{-0.1in}
\caption{Prompt templates for DP, CoT and multiple-choice questions. Light blue texts inside each ``\textcolor{contentcolor}{<>}'' block denote a string variable.}
\label{tabs:non_rag_prompt}
\vspace{-0.1in}
\end{table*}

\begin{table*}[t!]

\small
\centering
\begin{tabular}{p{2\columnwidth}}
\toprule
\textbf{1. LLM Context Explanation before Calculating BM25 Similarity}.\\
I will provide you with an event concerning the delivery of information. Your task is to generate content related to this event by applying your knowledge of the \textcolor{contentcolor}{<domain>} regulations.

To ensure the content is relevant and accurate, follow these steps:

1. Understand the Event: Clearly define and understand the specifics of the event. Identify the key players involved, the type of information being handled, and the context in which it is being delivered.\\ 
2. Apply \textcolor{contentcolor}{<domain>} Knowledge: Utilize your understanding of \textcolor{contentcolor}{<domain>} regulations, focusing on privacy, security, and the minimum necessary information principles. Ensure that your content addresses these aspects in the context of the event.\\ 

Event Details:
\textcolor{contentcolor}{<event>}

Output Format:

**Execution**:

1. Identify the key players, type of information, and context.\\ 
2. Apply relevant \textcolor{contentcolor}{<domain>} principles to the event.

Generated \textcolor{contentcolor}{<domain>} Content:\\ 
1. The \textcolor{contentcolor}{<domain>} Rule with its content: ...\\ 
2. Other Necessary Standard:...\\ 


**References**:\\ 
List the specific \textcolor{contentcolor}{<domain>} regulations you consulted to generate the content.\\
\midrule
\textbf{2. Run BM25 to retrieve relevant regulations from our structured legal documents given listed regulations generated in step 1}.\\
\midrule
\textbf{3. Law Filter Template}.\\
Given a regulation and an information sending or requesting event, identify if the given regulation is relevant to the event. The process involves the following steps:

1. **Understand the Event**:
   - Extract key details from the description of the event, such as the type of information being exchanged, the parties involved, and the context or domain (e.g., healthcare, finance, education).

2. **Analyze The Regulation**:
   - For given regulation, determine its scope and main focus by reading the summary or key sections. Identify the primary subject matter, applicable contexts, or targeted stakeholders.

3. **Relevance Matching**:
   - Compare the key details of the event with the main focus of the regulation. Consider:
     - Does the regulation explicitly address scenarios similar to the event?
     - Is the regulation intended for the context or domain of the event?
     - Are the parties involved in the event the type of entities the regulation aims to govern?

4. **Filtering Decision**:
   - If a regulation's focus aligns well with the event’s details, please answer yes.
   - If there is little to no alignment, such as different contexts, unrelated subjects, or inappropriate scopes, please answer no.

Event:\textcolor{contentcolor}{<event>}

\textcolor{contentcolor}{<domain>} Regulation Candidates:
\textcolor{contentcolor}{<candidates>}

Note:
It is possible that the event may be completely unrelated to the \textcolor{contentcolor}{<domain>} regulations provided. In such cases, just answer NONE.

%Question:
%Is the given HIPAA Regulation relevant with the given event?

Output Format:\\
Seleted Related \textcolor{contentcolor}{<domain>} Regulations:\\
1. section id - content.\\
....\\
N. section id - content.\\

\midrule
\textbf{Decision-making Template}.\\
Assess the relationship between the provided event and \textcolor{contentcolor}{<domain>} regulations, and select one of the following conclusions:

(A) The event is prohibited by \textcolor{contentcolor}{<domain>}.\\
(B) The event is permitted by \textcolor{contentcolor}{<domain>}.\\
(C) The event is not related to \textcolor{contentcolor}{<domain>}.\\

You will be given some sections of \textcolor{contentcolor}{<domain>} regulations for reference (Note: these references may not be 100\% accurate). Your task is to consider the specifics of the event in light of these sections. Remember that the provided references are not exhaustive, and the presence of a regulation does not automatically imply relevance or violation.

Event:
\textcolor{contentcolor}{<event>}

\textcolor{contentcolor}{<domain>} Regulations Reference:
\textcolor{contentcolor}{<reference\_regulations>}

Let's complete it step by step:\\
1. Analyze the specific details of the event by identifying who is involved (sender and recipient), what information is being sent or requested, and for what purpose.\\
2. Compare key elements of the event with \textcolor{contentcolor}{<domain>} rules, identifying if they involve the use, disclosure, or sensitive information as defined by \textcolor{contentcolor}{<domain>}.\\
3. Evaluate the provided \textcolor{contentcolor}{<domain>} regulation excerpts to see if they directly relate to the event.\\
%4. Consider if there are other \textcolor{contentcolor}{<domain>} rules not mentioned in the excerpts that might apply.\\
4. Conclude based on the comprehensive analysis whether the event is in compliance, in violation, or unrelated to \textcolor{contentcolor}{<domain>}.\\

Output Format:\\

**Execution**: [Document the outcomes from executing each step]:\\
1. - ...\\
2. - ...\\
...\\



**Choice**: [A. Prohibited | B. Permitted | C. Not related]\\
\bottomrule
\end{tabular}
\vspace{-0.1in}
\caption{Workflows and prompt templates used for \textbf{RAG}. Light blue texts inside each ``\textcolor{contentcolor}{<>}'' block denote a string variable.}
\label{tabs:prompt_BM25}
\end{table*}
