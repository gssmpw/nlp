%%%%%%%% ICML 2025 EXAMPLE LATEX SUBMISSION FILE %%%%%%%%%%%%%%%%%

\documentclass{article}

% Recommended, but optional, packages for figures and better typesetting:
\usepackage{microtype}
\usepackage{graphicx}
\usepackage{subfigure}
\usepackage{booktabs} % for professional tables
% \usepackage{geometry}
\usepackage{multirow}
\usepackage{makecell}
\usepackage[usestackEOL]{stackengine}
% hyperref makes hyperlinks in the resulting PDF.
% If your build breaks (sometimes temporarily if a hyperlink spans a page)
% please comment out the following usepackage line and replace
% \usepackage{icml2025} with \usepackage[nohyperref]{icml2025} above.
\usepackage{hyperref}


% Attempt to make hyperref and algorithmic work together better:
\newcommand{\theHalgorithm}{\arabic{algorithm}}

% Use the following line for the initial blind version submitted for review:
% \usepackage{icml2025}

% If accepted, instead use the following line for the camera-ready submission:
\usepackage[accepted]{icml2025}

% For theorems and such
\usepackage{amsmath}
\usepackage{amssymb}
\usepackage{mathtools}
\usepackage{amsthm}

% if you use cleveref..
\usepackage[capitalize,noabbrev]{cleveref}

%%%%%%%%%%%%%%%%%%%%%%%%%%%%%%%%
% THEOREMS
%%%%%%%%%%%%%%%%%%%%%%%%%%%%%%%%
\theoremstyle{plain}
\newtheorem{theorem}{Theorem}[section]
\newtheorem{proposition}[theorem]{Proposition}
\newtheorem{lemma}[theorem]{Lemma}
\newtheorem{corollary}[theorem]{Corollary}
\theoremstyle{definition}
\newtheorem{definition}[theorem]{Definition}
\newtheorem{assumption}[theorem]{Assumption}
\theoremstyle{remark}
\newtheorem{remark}[theorem]{Remark}

% Todonotes is useful during development; simply uncomment the next line
%    and comment out the line below the next line to turn off comments
%\usepackage[disable,textsize=tiny]{todonotes}
\usepackage[textsize=tiny]{todonotes}


% The \icmltitle you define below is probably too long as a header.
% Therefore, a short form for the running title is supplied here:
% \icmltitlerunning{Submission and Formatting Instructions for ICML 2025}
\icmltitlerunning{An Investigation of FP8 Across Accelerators for LLM Inference}

\begin{document}

\twocolumn[
\icmltitle{An Investigation of FP8 Across Accelerators for LLM Inference}

% It is OKAY to include author information, even for blind
% submissions: the style file will automatically remove it for you
% unless you've provided the [accepted] option to the icml2025
% package.

% List of affiliations: The first argument should be a (short)
% identifier you will use later to specify author affiliations
% Academic affiliations should list Department, University, City, Region, Country
% Industry affiliations should list Company, City, Region, Country

% You can specify symbols, otherwise they are numbered in order.
% Ideally, you should not use this facility. Affiliations will be numbered
% in order of appearance and this is the preferred way.
\icmlsetsymbol{equal}{*}

\begin{icmlauthorlist}
\icmlauthor{Jiwoo Kim}{equal,POSTECH}
\icmlauthor{Joonhyung Lee}{equal,NAVER Cloud}
\icmlauthor{Gunho Park}{NAVER Cloud}
\icmlauthor{Byeongwook Kim}{NAVER Cloud}
\icmlauthor{Se Jung Kwon}{NAVER Cloud}
\icmlauthor{Dongsoo Lee}{NAVER Cloud}
\icmlauthor{Youngjoo Lee}{POSTECH}
\end{icmlauthorlist}

\icmlaffiliation{POSTECH}{Department of Electrical Engineering, Pohang University of Science and Technology, Pohang, Republic of Korea}
\icmlaffiliation{NAVER Cloud}{NAVER Cloud, Seongnam, Republic of Korea}

% \icmlcorrespondingauthor{Dongsoo Lee}{dongsoo.lee@navercorp.com}
% \icmlcorrespondingauthor{Youngjoo Lee}{youngjoo.lee@postech.ac.kr}

% You may provide any keywords that you
% find helpful for describing your paper; these are used to populate
% the "keywords" metadata in the PDF but will not be shown in the document
\icmlkeywords{FP8, LLM, Quantization, GEMM, GEMV}

\vskip 0.3in
]

% this must go after the closing bracket ] following \twocolumn[ ...

% This command actually creates the footnote in the first column
% listing the affiliations and the copyright notice.
% The command takes one argument, which is text to display at the start of the footnote.
% The \icmlEqualContribution command is standard text for equal contribution.
% Remove it (just {}) if you do not need this facility.

%\printAffiliationsAndNotice{}  % leave blank if no need to mention equal contribution
\printAffiliationsAndNotice{\icmlEqualContribution} % otherwise use the standard text.

\begin{abstract}
The introduction of 8-bit floating-point (FP8) computation units in modern AI accelerators has generated significant interest in FP8-based large language model (LLM) inference. Unlike 16-bit floating-point formats, FP8 in deep learning requires a shared scaling factor. Additionally, while E4M3 and E5M2 are well-defined at the individual value level, their scaling and accumulation methods remain unspecified and vary across hardware and software implementations. As a result, FP8 behaves more like a quantization format than a standard numeric representation.
In this work, we provide the first comprehensive analysis of FP8 computation and acceleration on two AI accelerators: the NVIDIA H100 and Intel Gaudi 2. Our findings highlight that the Gaudi 2, by leveraging FP8, achieves higher throughput-to-power efficiency during LLM inference, offering valuable insights into the practical implications of FP8 adoption for datacenter-scale LLM serving.
\end{abstract}

\section{Introduction}
\label{section:introduction}

% redirection is unique and important in VR
Virtual Reality (VR) systems enable users to embody virtual avatars by mirroring their physical movements and aligning their perspective with virtual avatars' in real time. 
As the head-mounted displays (HMDs) block direct visual access to the physical world, users primarily rely on visual feedback from the virtual environment and integrate it with proprioceptive cues to control the avatar’s movements and interact within the VR space.
Since human perception is heavily influenced by visual input~\cite{gibson1933adaptation}, 
VR systems have the unique capability to control users' perception of the virtual environment and avatars by manipulating the visual information presented to them.
Leveraging this, various redirection techniques have been proposed to enable novel VR interactions, 
such as redirecting users' walking paths~\cite{razzaque2005redirected, suma2012impossible, steinicke2009estimation},
modifying reaching movements~\cite{gonzalez2022model, azmandian2016haptic, cheng2017sparse, feick2021visuo},
and conveying haptic information through visual feedback to create pseudo-haptic effects~\cite{samad2019pseudo, dominjon2005influence, lecuyer2009simulating}.
Such redirection techniques enable these interactions by manipulating the alignment between users' physical movements and their virtual avatar's actions.

% % what is hand/arm redirection, motivation of study arm-offset
% \change{\yj{i don't understand the purpose of this paragraph}
% These illusion-based techniques provide users with unique experiences in virtual environments that differ from the physical world yet maintain an immersive experience. 
% A key example is hand redirection, which shifts the virtual hand’s position away from the real hand as the user moves to enhance ergonomics during interaction~\cite{feuchtner2018ownershift, wentzel2020improving} and improve interaction performance~\cite{montano2017erg, poupyrev1996go}. 
% To increase the realism of virtual movements and strengthen the user’s sense of embodiment, hand redirection techniques often incorporate a complete virtual arm or full body alongside the redirected virtual hand, using inverse kinematics~\cite{hartfill2021analysis, ponton2024stretch} or adjustments to the virtual arm's movement as well~\cite{li2022modeling, feick2024impact}.
% }

% noticeability, motivation of predicting a probability, not a classification
However, these redirection techniques are most effective when the manipulation remains undetected~\cite{gonzalez2017model, li2022modeling}. 
If the redirection becomes too large, the user may not mitigate the conflict between the visual sensory input (redirected virtual movement) and their proprioception (actual physical movement), potentially leading to a loss of embodiment with the virtual avatar and making it difficult for the user to accurately control virtual movements to complete interaction tasks~\cite{li2022modeling, wentzel2020improving, feuchtner2018ownershift}. 
While proprioception is not absolute, users only have a general sense of their physical movements and the likelihood that they notice the redirection is probabilistic. 
This probability of detecting the redirection is referred to as \textbf{noticeability}~\cite{li2022modeling, zenner2024beyond, zenner2023detectability} and is typically estimated based on the frequency with which users detect the manipulation across multiple trials.

% version B
% Prior research has explored factors influencing the noticeability of redirected motion, including the redirection's magnitude~\cite{wentzel2020improving, poupyrev1996go}, direction~\cite{li2022modeling, feuchtner2018ownershift}, and the visual characteristics of the virtual avatar~\cite{ogawa2020effect, feick2024impact}.
% While these factors focus on the avatars, the surrounding virtual environment can also influence the users' behavior and in turn affect the noticeability of redirection.
% One such prominent external influence is through the visual channel - the users' visual attention is constantly distracted by complex visual effects and events in practical VR scenarios.
% Although some prior studies have explored how to leverage user blindness caused by visual distractions to redirect users' virtual hand~\cite{zenner2023detectability}, there remains a gap in understanding how to quantify the noticeability of redirection under visual distractions.

% visual stimuli and gaze behavior
Prior research has explored factors influencing the noticeability of redirected motion, including the redirection's magnitude~\cite{wentzel2020improving, poupyrev1996go}, direction~\cite{li2022modeling, feuchtner2018ownershift}, and the visual characteristics of the virtual avatar~\cite{ogawa2020effect, feick2024impact}.
While these factors focus on the avatars, the surrounding virtual environment can also influence the users' behavior and in turn affect the noticeability of redirection.
This, however, remains underexplored.
One such prominent external influence is through the visual channel - the users' visual attention is constantly distracted by complex visual effects and events in practical VR scenarios.
We thus want to investigate how \textbf{visual stimuli in the virtual environment} affect the noticeability of redirection.
With this, we hope to complement existing works that focus on avatars by incorporating environmental visual influences to enable more accurate control over the noticeability of redirected motions in practical VR scenarios.
% However, in realistic VR applications, the virtual environment often contains complex visual effects beyond the virtual avatar itself. 
% We argue that these visual effects can \textbf{distract users’ visual attention and thus affect the noticeability of redirection offsets}, while current research has yet taken into account.
% For instance, in a VR boxing scenario, a user’s visual attention is likely focused on their opponent rather than on their virtual body, leading to a lower noticeability of redirection offsets on their virtual movements. 
% Conversely, when reaching for an object in the center of their field of view, the user’s attention is more concentrated on the virtual hand’s movement and position to ensure successful interaction, resulting in a higher noticeability of offsets.

Since each visual event is a complex choreography of many underlying factors (type of visual effect, location, duration, etc.), it is extremely difficult to quantify or parameterize visual stimuli.
Furthermore, individuals respond differently to even the same visual events.
Prior neuroscience studies revealed that factors like age, gender, and personality can influence how quickly someone reacts to visual events~\cite{gillon2024responses, gale1997human}. 
Therefore, aiming to model visual stimuli in a way that is generalizable and applicable to different stimuli and users, we propose to use users' \textbf{gaze behavior} as an indicator of how they respond to visual stimuli.
In this paper, we used various gaze behaviors, including gaze location, saccades~\cite{krejtz2018eye}, fixations~\cite{perkhofer2019using}, and the Index of Pupil Activity (IPA)~\cite{duchowski2018index}.
These behaviors indicate both where users are looking and their cognitive activity, as looking at something does not necessarily mean they are attending to it.
Our goal is to investigate how these gaze behaviors stimulated by various visual stimuli relate to the noticeability of redirection.
With this, we contribute a model that allows designers and content creators to adjust the redirection in real-time responding to dynamic visual events in VR.

To achieve this, we conducted user studies to collect users' noticeability of redirection under various visual stimuli.
To simulate realistic VR scenarios, we adopted a dual-task design in which the participants performed redirected movements while monitoring the visual stimuli.
Specifically, participants' primary task was to report if they noticed an offset between the avatar's movement and their own, while their secondary task was to monitor and report the visual stimuli.
As realistic virtual environments often contain complex visual effects, we started with simple and controlled visual stimulus to manage the influencing factors.

% first user study, confirmation study
% collect data under no visual stimuli, different basic visual stimuli
We first conducted a confirmation study (N=16) to test whether applying visual stimuli (opacity-based) actually affects their noticeability of redirection. 
The results showed that participants were significantly less likely to detect the redirection when visual stimuli was presented $(F_{(1,15)}=5.90,~p=0.03)$.
Furthermore, by analyzing the collected gaze data, results revealed a correlation between the proposed gaze behaviors and the noticeability results $(r=-0.43)$, confirming that the gaze behaviors could be leveraged to compute the noticeability.

% data collection study
We then conducted a data collection study to obtain more accurate noticeability results through repeated measurements to better model the relationship between visual stimuli-triggered gaze behaviors and noticeability of redirection.
With the collected data, we analyzed various numerical features from the gaze behaviors to identify the most effective ones. 
We tested combinations of these features to determine the most effective one for predicting noticeability under visual stimuli.
Using the selected features, our regression model achieved a mean squared error (MSE) of 0.011 through leave-one-user-out cross-validation. 
Furthermore, we developed both a binary and a three-class classification model to categorize noticeability, which achieved an accuracy of 91.74\% and 85.62\%, respectively.

% evaluation study
To evaluate the generalizability of the regression model, we conducted an evaluation study (N=24) to test whether the model could accurately predict noticeability with new visual stimuli (color- and scale-based animations).
Specifically, we evaluated whether the model's predictions aligned with participants' responses under these unseen stimuli.
The results showed that our model accurately estimated the noticeability, achieving mean squared errors (MSE) of 0.014 and 0.012 for the color- and scale-based visual stimili, respectively, compared to participants' responses.
Since the tested visual stimuli data were not included in the training, the results suggested that the extracted gaze behavior features capture a generalizable pattern and can effectively indicate the corresponding impact on the noticeability of redirection.

% application
Based on our model, we implemented an adaptive redirection technique and demonstrated it through two applications: adaptive VR action game and opportunistic rendering.
We conducted a proof-of-concept user study (N=8) to compare our adaptive redirection technique with a static redirection, evaluating the usability and benefits of our adaptive redirection technique.
The results indicated that participants experienced less physical demand and stronger sense of embodiment and agency when using the adaptive redirection technique. 
These results demonstrated the effectiveness and usability of our model.

In summary, we make the following contributions.
% 
\begin{itemize}
    \item 
    We propose to use users' gaze behavior as a medium to quantify how visual stimuli influences the noticebility of redirection. 
    Through two user studies, we confirm that visual stimuli significantly influences noticeability and identify key gaze behavior features that are closely related to this impact.
    \item 
    We build a regression model that takes the user's gaze behavioral data as input, then computes the noticeability of redirection.
    Through an evaluation study, we verify that our model can estimate the noticeability with new participants under unseen visual stimuli.
    These findings suggest that the extracted gaze behavior features effectively capture the influence of visual stimuli on noticeability and can generalize across different users and visual stimuli.
    \item 
    We develop an adaptive redirection technique based on our regression model and implement two applications with it.
    With a proof-of-concept study, we demonstrate the effectiveness and potential usability of our regression model on real-world use cases.

\end{itemize}

% \delete{
% Virtual Reality (VR) allows the user to embody a virtual avatar by mirroring their physical movements through the avatar.
% As the user's visual access to the physical world is blocked in tasks involving motion control, they heavily rely on the visual representation of the avatar's motions to guide their proprioception.
% Similar to real-world experiences, the user is able to resolve conflicts between different sensory inputs (e.g., vision and motor control) through multisensory integration, which is essential for mitigating the sensory noise that commonly arises.
% However, it also enables unique manipulations in VR, as the system can intentionally modify the avatar's movements in relation to the user's motions to achieve specific functional outcomes,
% for example, 
% % the manipulations on the avatar's movements can 
% enabling novel interaction techniques of redirected walking~\cite{razzaque2005redirected}, redirected reaching~\cite{gonzalez2022model}, and pseudo haptics~\cite{samad2019pseudo}.
% With small adjustments to the avatar's movements, the user can maintain their sense of embodiment, due to their ability to resolve the perceptual differences.
% % However, a large mismatch between the user and avatar's movements can result in the user losing their sense of embodiment, due to an inability to resolve the perceptual differences.
% }

% \delete{
% However, multisensory integration can break when the manipulation is so intense that the user is aware of the existence of the motion offset and no longer maintains the sense of embodiment.
% Prior research studied the intensity threshold of the offset applied on the avatar's hand, beyond which the embodiment will break~\cite{li2022modeling}. 
% Studies also investigated the user's sensitivity to the offsets over time~\cite{kohm2022sensitivity}.
% Based on the findings, we argue that one crucial factor that affects to what extent the user notices the offset (i.e., \textit{noticeability}) that remains under-explored is whether the user directs their visual attention towards or away from the virtual avatar.
% Related work (e.g., Mise-unseen~\cite{marwecki2019mise}) has showcased applications where adjustments in the environment can be made in an unnoticeable manner when they happen in the area out of the user's visual field.
% We hypothesize that directing the user's visual attention away from the avatar's body, while still partially keeping the avatar within the user's field-of-view, can reduce the noticeability of the offset.
% Therefore, we conduct two user studies and implement a regression model to systematically investigate this effect.
% }

% \delete{
% In the first user study (N = 16), we test whether drawing the user's visual attention away from their body impacts the possibility of them noticing an offset that we apply to their arm motion in VR.
% We adopt a dual-task design to enable the alteration of the user's visual attention and a yes/no paradigm to measure the noticeability of motion offset. 
% The primary task for the user is to perform an arm motion and report when they perceive an offset between the avatar's virtual arm and their real arm.
% In the secondary task, we randomly render a visual animation of a ball turning from transparent to red and becoming transparent again and ask them to monitor and report when it appears.
% We control the strength of the visual stimuli by changing the duration and location of the animation.
% % By changing the time duration and location of the visual animation, we control the strengths of attraction to the users.
% As a result, we found significant differences in the noticeability of the offsets $(F_{(1,15)}=5.90,~p=0.03)$ between conditions with and without visual stimuli.
% Based on further analysis, we also identified the behavioral patterns of the user's gaze (including pupil dilation, fixations, and saccades) to be correlated with the noticeability results $(r=-0.43)$ and they may potentially serve as indicators of noticeability.
% }

% \delete{
% To further investigate how visual attention influences the noticeability, we conduct a data collection study (N = 12) and build a regression model based on the data.
% The regression model is able to calculate the noticeability of the offset applied on the user's arm under various visual stimuli based on their gaze behaviors.
% Our leave-one-out cross-validation results show that the proposed method was able to achieve a mean-squared error (MSE) of 0.012 in the probability regression task.
% }

% \delete{
% To verify the feasibility and extendability of the regression model, we conduct an evaluation study where we test new visual animations based on adjustments on scale and color and invite 24 new participants to attend the study.
% Results show that the proposed method can accurately estimate the noticeability with an MSE of 0.014 and 0.012 in the conditions of the color- and scale-based visual effects.
% Since these animations were not included in the dataset that the regression model was built on, the study demonstrates that the gaze behavioral features we extracted from the data capture a generalizable pattern of the user's visual attention and can indicate the corresponding impact on the noticeability of the offset.
% }

% \delete{
% Finally, we demonstrate applications that can benefit from the noticeability prediction model, including adaptive motion offsets and opportunistic rendering, considering the user's visual attention. 
% We conclude with discussions of our work's limitations and future research directions.
% }

% \delete{
% In summary, we make the following contributions.
% }
% % 
% \begin{itemize}
%     \item 
%     \delete{
%     We quantify the effects of the user's visual attention directed away by stimuli on their noticeability of an offset applied to the avatar's arm motion with respect to the user's physical arm. 
%     Through two user studies, we identified gaze behavioral features that are indicative of the changes in noticeability.
%     }
%     \item 
%     \delete{We build a regression model that takes the user's gaze behavioral data and the offset applied to the arm motion as input, then computes the probability of the user noticing the offset.
%     Through an evaluation study, we verified that the model needs no information about the source attracting the user's visual attention and can be generalizable in different scenarios.
%     }
%     \item 
%     \delete{We demonstrate two applications that potentially benefit from the regression model, including adaptive motion offsets and opportunistic rendering.
%     }

% \end{itemize}

\begin{comment}
However, users will lose the sense of embodiment to the virtual avatars if they notice the offset between the virtual and physical movements.
To address this, researchers have been exploring the noticing threshold of offsets with various magnitudes and proposing various redirection techniques that maintain the sense of embodiment~\cite{}.

However, when users embody virtual avatars to explore virtual environments, they encounter various visual effects and content that can attract their attention~\cite{}.
During this, the user may notice an offset when he observes the virtual movement carefully while ignoring it when the virtual contents attract his attention from the movements.
Therefore, static offset thresholds are not appropriate in dynamic scenarios.

Past research has proposed dynamic mapping techniques that adapted to users' state, such as hand moving speed~\cite{frees2007prism} or ergonomically comfortable poses~\cite{montano2017erg}, but not considering the influence of virtual content.
More specifically, PRISM~\cite{frees2007prism} proposed adjusting the C/D ratio with a non-linear mapping according to users' hand moving speed, but it might not be optimal for various virtual scenarios.
While Erg-O~\cite{montano2017erg} redirected users' virtual hands according to the virtual target's relative position to reduce physical fatigue, neglecting the change of virtual environments. 

Therefore, how to design redirection techniques in various scenarios with different visual attractions remains unknown.
To address this, we investigate how visual attention affects the noticing probability of movement offsets.
Based on our experiments, we implement a computational model that automatically computes the noticing probability of offsets under certain visual attractions.
VR application designers and developers can easily leverage our model to design redirection techniques maintaining the sense of embodiment adapt to the user's visual attention.
We implement a dynamic redirection technique with our model and demonstrate that it effectively reduces the target reaching time without reducing the sense of embodiment compared to static redirection techniques.

% Need to be refined
This paper offers the following contributions.
\begin{itemize}
    \item We investigate how visual attractions affect the noticing probability of redirection offsets.
    \item We construct a computational model to predict the noticing probability of an offset with a given visual background.
    \item We implement a dynamic redirection technique adapting to the visual background. We evaluate the technique and develop three applications to demonstrate the benefits. 
\end{itemize}



First, we conducted a controlled experiment to understand how users perceived the movement offset while subjected to various distractions.
Since hand redirection is one of the most frequently used redirections in VR interactions, we focused on the dynamic arm movements and manually added angular offsets to the' elbow joint~\cite{li2022modeling, gonzalez2022model, zenner2019estimating}. 
We employed flashing spheres in the user's field of view as distractions to attract users' visual attention.
Participants were instructed to report the appearing location of the spheres while simultaneously performing the arm movements and reporting if they perceived an offset during the movement. 
(\zhipeng{Add the results of data collection. Analyze the influence of the distance between the gaze map and the offset.}
We measured the visual attraction's magnitude with the gaze distribution on it.
Results showed that stronger distractions made it harder for users to notice the offset.)
\zhipeng{Need to rewrite. Not sure to use gaze distribution or a metric obtained from the visual content.}
Secondly, we constructed a computational model to predict the noticing probability of offsets with given visual content.
We analyzed the data from the user studies to measure the influence of visual attractions on the noticing probability of offsets.
We built a statistical model to predict the offset's noticing probability with a given visual content.
Based on the model, we implement a dynamic redirection technique to adjust the redirection offset adapted to the user's current field of view.
We evaluated the technique in a target selection task compared to no hand redirection and static hand redirection.
\zhipeng{Add the results of the evaluation.}
Results showed that the dynamic hand redirection technique significantly reduced the target selection time with similar accuracy and a comparable sense of embodiment.
Finally, we implemented three applications to demonstrate the potential benefits of the visual attention adapted dynamic redirection technique.
\end{comment}

% This one modifies arm length, not redirection
% \citeauthor{mcintosh2020iteratively} proposed an adaptation method to iteratively change the virtual avatar arm's length based on the primary tasks' performance~\cite{mcintosh2020iteratively}.



% \zhipeng{TO ADD: what is redirection}
% Redirection enables novel interactions in Virtual Reality, including redirected walking, haptic redirection, and pseudo haptics by introducing an offset to users' movement.
% \zhipeng{TO ADD: extend this sentence}
% The price of this is that users' immersiveness and embodiment in VR can be compromised when they notice the offset and perceive the virtual movement not as theirs~\cite{}.
% \zhipeng{TO ADD: extend this sentence, elaborate how the virtual environment attracts users' attention}
% Meanwhile, the visual content in the virtual environment is abundant and consistently captures users' attention, making it harder to notice the offset~\cite{}.
% While previous studies explored the noticing threshold of the offsets and optimized the redirection techniques to maintain the sense of embodiment~\cite{}, the influence of visual content on the probability of perceiving offsets remains unknown.  
% Therefore, we propose to investigate how users perceive the redirection offset when they are facing various visual attractions.


% We conducted a user study to understand how users notice the shift with visual attractions.
% We used a color-changing ball to attract the user's attention while instructing users to perform different poses with their arms and observe it meanwhile.
% \zhipeng{(Which one should be the primary task? Observe the ball should be the primary one, but if the primary task is too simple, users might allocate more attention on the secondary task and this makes the secondary task primary.)}
% \zhipeng{(We need a good and reasonable dual-task design in which users care about both their pose and the visual content, at least in the evaluation study. And we need to be able to control the visual content's magnitude and saliency maybe?)}
% We controlled the shift magnitude and direction, the user's pose, the ball's size, and the color range.
% We set the ball's color-changing interval as the independent factor.
% We collect the user's response to each shift and the color-changing times.
% Based on the collected data, we constructed a statistical model to describe the influence of visual attraction on the noticing probability.
% \zhipeng{(Are we actually controlling the attention allocation? How do we measure the attracting effect? We need uniform metrics, otherwise it is also hard for others to use our knowledge.)}
% \zhipeng{(Try to use eye gaze? The eye gaze distribution in the last five seconds to decide the attention allocation? Basically constructing a model with eye gaze distribution and noticing probability. But the user's head is moving, so the eye gaze distribution is not aligned well with the current view.)}

% \zhipeng{Saliency and EMD}
% \zhipeng{Gaze is more than just a point: Rethinking visual attention
% analysis using peripheral vision-based gaze mapping}

% Evaluation study(ideal case): based on the visual content, adjusting the redirection magnitude dynamically.

% \zhipeng{(The risk is our model's effect is trivial.)}

% Applications:
% Playing Lego while watching demo videos, we can accelerate the reaching process of bricks, and forbid the redirection during the manipulation.

% Beat saber again: but not make a lot of sense? Difficult game has complicated visual effects, while allows larger shift, but do not need large shift with high difficulty




\section{Related Work}
\label{lit_review}

\begin{highlight}
{

Our research builds upon {\em (i)} Assessing Web Accessibility, {\em (ii)} End-User Accessibility Repair, and {\em (iii)} Developer Tools for Accessibility.

\subsection{Assessing Web Accessibility}
From the earliest attempts to set standards and guidelines, web accessibility has been shaped by a complex interplay of technical challenges, legal imperatives, and educational campaigns. Over the past 25 years, stakeholders have sought to improve digital inclusion by establishing foundational standards~\cite{chisholm2001web, caldwell2008web}, enforcing legal obligations~\cite{sierkowski2002achieving, yesilada2012understanding}, and promoting a broader culture of accessibility awareness among developers~\cite{sloan2006contextual, martin2022landscape, pandey2023blending}. 
Despite these longstanding efforts, systemic accessibility issues persist. According to the 2024 WebAIM Million report~\cite{webaim2024}, 95.9\% of the top one million home pages contained detectable WCAG violations, averaging nearly 57 errors per page. 
These errors take many forms: low color contrast makes the interface difficult for individuals with color deficiency or low vision to read text; missing alternative text leaves users relying on screen readers without crucial visual context; and unlabeled form inputs or empty links and buttons hinder people who navigate with assistive technologies from completing basic tasks. 
Together, these accessibility issues not only limit user access to critical online resources such as healthcare, education, and employment but also result in significant legal risks and lost opportunities for businesses to engage diverse audiences. Addressing these pervasive issues requires systematic methods to identify, measure, and prioritize accessibility barriers, which is the first step toward achieving meaningful improvements.

Prior research has introduced methods blending automation and human evaluation to assess web accessibility. Hybrid approaches like SAMBA combine automated tools with expert reviews to measure the severity and impact of barriers, enhancing evaluation reliability~\cite{brajnik2007samba}. Quantitative metrics, such as Failure Rate and Unified Web Evaluation Methodology, support large-scale monitoring and comparative analysis, enabling cost-effective insights~\cite{vigo2007quantitative, martins2024large}. However, automated tools alone often detect less than half of WCAG violations and generate false positives, emphasizing the need for human interpretation~\cite{freire2008evaluation, vigo2013benchmarking}. Recent progress with large pretrained models like Large Language Models (LLMs)~\cite{dubey2024llama,bai2023qwen} and Large Multimodal Models (LMMs)~\cite{liu2024visual, bai2023qwenvl} offers a promising step forward, automating complex checks like non-text content evaluation and link purposes, achieving higher detection rates than traditional tools~\cite{lopez2024turning, delnevo2024interaction}. Yet, these large models face challenges, including dependence on training data, limited contextual judgment, and the inability to simulate real user experiences. These limitations underscore the necessity of combining models with human oversight for reliable, user-centered evaluations~\cite{brajnik2007samba, vigo2013benchmarking, delnevo2024interaction}. 

Our work builds on these prior efforts and recent advancements by leveraging the capabilities of large pretrained models while addressing their limitations through a developer-centric approach. CodeA11y integrates LLM-powered accessibility assessments, tailored accessibility-aware system prompts, and a dedicated accessibility checker directly into GitHub Copilot---one of the most widely used coding assistants. Unlike standalone evaluation tools, CodeA11y actively supports developers throughout the coding process by reinforcing accessibility best practices, prompting critical manual validations, and embedding accessibility considerations into existing workflows.
% This pervasive shortfall reflects the difficulty of scaling traditional approaches---such as manual audits and automated tools---that either demand immense human effort or lack the nuanced understanding needed to capture real-world user experiences. 
%
% In response, a new wave of AI-driven methods, many powered by large language models (LLMs), is emerging to bridge these accessibility detection and assessment gaps. Early explorations, such as those by Morillo et al.~\cite{morillo2020system}, introduced AI-assisted recommendations capable of automatic corrections, illustrating how computational intelligence can tackle the repetitive, common errors that plague large swaths of the web. Building on this foundation, Huang et al.~\cite{huang2024access} proposed ACCESS, a prompt-engineering framework that streamlines the identification and remediation of accessibility violations, while López-Gil et al.~\cite{lopez2024turning} demonstrated how LLMs can help apply WCAG success criteria more consistently---reducing the reliance on manual effort. Beyond these direct interventions, recent work has also begun integrating user experiences more seamlessly into the evaluation process. For example, Huq et al.~\cite{huq2024automated} translate user transcripts and corresponding issues into actionable test reports, ensuring that accessibility improvements align more closely with authentic user needs.
% However, as these AI-driven solutions evolve, researchers caution against uncritical adoption. Othman et al.~\cite{othman2023fostering} highlight that while LLMs can accelerate remediation, they may also introduce biases or encourage over-reliance on automated processes. Similarly, Delnevo et al.~\cite{delnevo2024interaction} emphasize the importance of contextual understanding and adaptability, pointing to the current limitations of LLM-based systems in serving the full spectrum of user needs. 
% In contrast to this backdrop, our work introduces and evaluates CodeA11y, an LLM-augmented extension for GitHub Copilot that not only mitigates these challenges by providing more consistent guidance and manual validation prompts, but also aligns AI-driven assistance with developers’ workflows, ultimately contributing toward more sustainable propulsion for building accessible web.

% Broader implications of inaccessibility—legal compliance, ethical concerns, and user experience
% A Historical Review of Web Accessibility Using WAVE
% "I tend to view ads almost like a pestilence": On the Accessibility Implications of Mobile Ads for Blind Users

% In the research domain, several methods have been developed to assess and enhance web accessibility. These include incorporating feedback into developer tools~\cite{adesigner, takagi2003accessibility, bigham2010accessibility} and automating the creation of accessibility tests and reports for user interfaces~\cite{swearngin2024towards, taeb2024axnav}. 

% Prior work has also studied accessibility scanners as another avenue of AI to improve web development practices~\cite{}.
% However, a persistent challenge is that developers need to be aware of these tools to utilize them effectively. With recent advancements in LLMs, developers might now build accessible websites with less effort using AI assistants. However, the impact of these assistants on the accessibility of their generated code remains unclear. This study aims to investigate these effects.

\subsection{End-user Accessibility Repair}
In addition to detecting accessibility errors and measuring web accessibility, significant research has focused on fixing these problems.
Since end-users are often the first to notice accessibility problems and have a strong incentive to address them, systems have been developed to help them report or fix these problems.

Collaborative, or social accessibility~\cite{takagi2009collaborative,sato2010social}, enabled these end-user contributions to be scaled through crowd-sourcing.
AccessMonkey~\cite{bigham2007accessmonkey} and Accessibility Commons~\cite{kawanaka2008accessibility} were two examples of repositories that store accessibility-related scripts and metadata, respectively.
Other work has developed browser extensions that leverage crowd-sourced databases to automatically correct reading order, alt-text, color contrast, and interaction-related issues~\cite{sato2009s,huang2015can}.

One drawback of collaborative accessibility approaches is that they cannot fix problems for an ``unseen'' web page on-demand, so many projects aim to automatically detect and improve interfaces without the need for an external source of fixes.
A large body of research has focused on making specific web media (e.g., images~\cite{gleason2019making,guinness2018caption, twitterally, gleason2020making, lee2021image}, design~\cite{potluri2019ai,li2019editing, peng2022diffscriber, peng2023slide}, and videos~\cite{pavel2020rescribe,peng2021say,peng2021slidecho,huh2023avscript}) accessible through a combination of machine learning (ML) and user-provided fixes.
Other work has focused on applying more general fixes across all websites.

Opportunity accessibility addressed a common accessibility problem of most websites: by default, content is often hard to see for people with visual impairments, and many users, especially older adults, do not know how to adjust or enable content zooming~\cite{bigham2014making}.
To this end, a browser script (\texttt{oppaccess.js}) was developed that automatically adjusted the browser's content zoom to maximally enlarge content without introducing adverse side-effects (\textit{e.g.,} content overlap).
While \texttt{oppaccess.js} primarily targeted zoom-related accessibility, recent work aimed to enable larger types of changes, by using LLMs to modify the source code of web pages based on user questions or directives~\cite{li2023using}.

Several efforts have been focused on improving access to desktop and mobile applications, which present additional challenges due to the unavailability of app source code (\textit{e.g.,} HTML).
Prefab is an approach that allows graphical UIs to be modified at runtime by detecting existing UI widgets, then replacing them~\cite{dixon2010prefab}.
Interaction Proxies used these runtime modification strategies to ``repair'' Android apps by replacing inaccessible widgets with improved alternatives~\cite{zhang2017interaction, zhang2018robust}.
The widget detection strategies used by these systems previously relied on a combination of heuristics and system metadata (\textit{e.g.,} the view hierarchy), which are incomplete or missing in the accessible apps.
To this end, ML has been employed to better localize~\cite{chen2020object} and repair UI elements~\cite{chen2020unblind,zhang2021screen,wu2023webui,peng2025dreamstruct}.

In general, end-user solutions to repairing application accessibility are limited due to the lack of underlying code and knowledge of the semantics of the intended content.

\subsection{Developer Tools for Accessibility}
Ultimately, the best solution for ensuring an accessible experience lies with front-end developers. Many efforts have focused on building adequate tooling and support to help developers with ensuring that their UI code complies with accessibility standards.

Numerous automated accessibility testing tools have been created to help developers identify accessibility issues in their code: i) static analysis tools, such as IBM Equal Access Accessibility Checker~\cite{ibm2024toolkit} or Microsoft Accessibility Insights~\cite{accessibilityinsights2024}, scan the UI code's compliance with predefined rules derived from accessibility guidelines; and ii) dynamic or runtime accessibility scanners, such as Chrome Devtools~\cite{chromedevtools2024} or axe-Core Accessibility Engine~\cite{deque2024axe}, perform real-time testing on user interfaces to detect interaction issues not identifiable from the code structure. While these tools greatly reduce the manual effort required for accessibility testing, they are often criticized for their limited coverage. Thus, experts often recommend manually testing with assistive technologies to uncover more complex interaction issues. Prior studies have created accessibility crawlers that either assist in developer testing~\cite{swearngin2024towards,taeb2024axnav} or simulate how assistive technologies interact with UIs~\cite{10.1145/3411764.3445455, 10.1145/3551349.3556905, 10.1145/3544548.3580679}.

Similar to end-user accessibility repair, research has focused on generating fixes to remediate accessibility issues in the UI source code. Initial attempts developed heuristic-based algorithms for fixing specific issues, for instance, by replacing text or background color attributes~\cite{10.1145/3611643.3616329}. More recent work has suggested that the code-understanding capabilities of LLMs allow them to suggest more targeted fixes.
For example, a study demonstrated that prompting ChatGPT to fix identified WCAG compliance issues in source code could automatically resolve a significant number of them~\cite{othman2023fostering}. Researchers have sought to leverage this capability by employing a multi-agent LLM architecture to automatically identify and localize issues in source code and suggest potential code fixes~\cite{mehralian2024automated}.

While the approaches mentioned above focus on assessing UI accessibility of already-authored code (\textit{i.e.,} fixing existing code), there is potential for more proactive approaches.
For example, LLMs are often used by developers to generate UI source code from natural language descriptions or tab completions~\cite{chen2021evaluating,GitHubCopilot,lozhkov2024starcoder,hui2024qwen2,roziere2023code,zheng2023codegeex}, but LLMs frequently produce inaccessible code by default~\cite{10.1145/3677846.3677854,mowar2024tab}, leading to inaccessible output when used by developers without sufficient awareness of accessibility knowledge.
The primary focus of this paper is to design a more accessibility-aware coding assistant that both produces more accessible code without manual intervention (\textit{e.g.,} specific user prompting) and gradually enables developers to implement and improve accessibility of automatically-generated code through IDE UI modifications (\textit{e.g.}, reminder notifications).

}
\end{highlight}



% Work related to this paper includes {\em (i)} Web Accessibility and {\em (ii)} Developer Practices in AI-Assisted Programming.

% \ipstart{Web Accessibility: Practice, Evaluation, and Improvements} Substantial efforts have been made to set accessibility standards~\cite{chisholm2001web, caldwell2008web}, establish legal requirements~\cite{sierkowski2002achieving, yesilada2012understanding}, and promote education and advocacy among developers~\cite{sloan2006contextual, martin2022landscape, pandey2023blending}. In the research domain, several methods have been developed to assess and enhance web accessibility. These include incorporating feedback into developer tools~\cite{adesigner, takagi2003accessibility, bigham2010accessibility} and automating the creation of accessibility tests and reports for user interfaces~\cite{swearngin2024towards, taeb2024axnav}. 
% % Prior work has also studied accessibility scanners as another avenue of AI to improve web development practices~\cite{}.
% However, a persistent challenge is that developers need to be aware of these tools to utilize them effectively. With recent advancements in LLMs, developers might now build accessible websites with less effort using AI assistants. However, the impact of these assistants on the accessibility of their generated code remains unclear. This study aims to investigate these effects.

% \ipstart{Developer Practices in AI-Assisted Programming}
% Recent usability research on AI-assisted development has examined the interaction strategies of developers while using AI coding assistants~\cite{barke2023grounded}.
% They observed developers interacted with these assistants in two modes -- 1) \textit{acceleration mode}: associated with shorter completions and 2) \textit{exploration mode}: associated with long completions.
% Liang {\em et al.} \cite{liang2024large} found that developers are driven to use AI assistants to reduce their keystrokes, finish tasks faster, and recall the syntax of programming languages. On the other hand, developers' reason for rejecting autocomplete suggestions was the need for more consideration of appropriate software requirements. This is because primary research on code generation models has mainly focused on functional correctness while often sidelining non-functional requirements such as latency, maintainability, and security~\cite{singhal2024nofuneval}. Consequently, there have been increasing concerns about the security implications of AI-generated code~\cite{sandoval2023lost}. Similarly, this study focuses on the effectiveness and uptake of code suggestions among developers in mitigating accessibility-related vulnerabilities. 


% ============================= additional rw ============================================
% - Paulina Morillo, Diego Chicaiza-Herrera, and Diego Vallejo-Huanga. 2020. System of Recommendation and Automatic Correction of Web Accessibility Using Artificial Intelligence. In Advances in Usability and User Experience, Tareq Ahram and Christianne Falcão (Eds.). Springer International Publishing, Cham, 479–489
% - Juan-Miguel López-Gil and Juanan Pereira. 2024. Turning manual web accessibility success criteria into automatic: an LLM-based approach. Universal Access in the Information Society (2024). https://doi.org/10.1007/s10209-024-01108-z
% - s
% - Calista Huang, Alyssa Ma, Suchir Vyasamudri, Eugenie Puype, Sayem Kamal, Juan Belza Garcia, Salar Cheema, and Michael Lutz. 2024. ACCESS: Prompt Engineering for Automated Web Accessibility Violation Corrections. arXiv:2401.16450 [cs.HC] https://arxiv.org/abs/2401.16450
% - Syed Fatiul Huq, Mahan Tafreshipour, Kate Kalcevich, and Sam Malek. 2025. Automated Generation of Accessibility Test Reports from Recorded User Transcripts. In Proceedings of the 47th International Conference on Software Engineering (ICSE) (Ottawa, Ontario, Canada). IEEE. https://ics.uci.edu/~seal/publications/2025_ICSE_reca11.pdf To appear in IEEE Xplore
% - Achraf Othman, Amira Dhouib, and Aljazi Nasser Al Jabor. 2023. Fostering websites accessibility: A case study on the use of the Large Language Models ChatGPT for automatic remediation. In Proceedings of the 16th International Conference on PErvasive Technologies Related to Assistive Environments (Corfu, Greece) (PETRA ’23). Association for Computing Machinery, New York, NY, USA, 707–713. https://doi.org/10.1145/3594806.3596542
% - Zsuzsanna B. Palmer and Sushil K. Oswal. 0. Constructing Websites with Generative AI Tools: The Accessibility of Their Workflows and Products for Users With Disabilities. Journal of Business and Technical Communication 0, 0 (0), 10506519241280644. https://doi.org/10.1177/10506519241280644
% ============================= additional rw ============================================

\section{Matrix multiplication comparison}\label{sec:matmul}

\subsection{Differences in hardware capabilities}
% I need to cite the FP8 Intel white paper here.

We compare implementations of FP8 GEMM between NVIDIA GPUs and Intel Gaudi HPUs, identifying key differences despite both adhering to the FP8 specification \citep{micikevicius2022fp8formatsdeeplearning}.

\textbf{(Accumulation precision)} Hopper GPUs use a 14-bit accumulator for FP8 GEMMs \citep{deepseekai2024deepseekv3technicalreport}, requiring casting to CUDA cores for higher precision. Software optimizations, such as applying FP32 accumulation to only one in four WGMMA instructions, reduce error but increase kernel complexity and remain less precise than full FP32 accumulation. In contrast, Gaudi HPUs always accumulate in FP32, ensuring higher numerical precision.

\textbf{(E4M3 range)} The Gaudi 2 follows the IEEE specification for special values, unlike NVIDIA GPUs, which use a single special value representation \citep{noune20228bitnumericalformatsdeep}. This results in seven fewer magnitude representations and a maximum value of 240 for E4M3 in the Gaudi 2 compared to 448 on NVIDIA GPUs. This has been addressed in the Gaudi 3, but we were unable to test these in our experiments.

\textbf{(Power-of-2 scaling)} Gaudi HPUs allow modification of floating-point exponent biases to accelerate scaling factor application. The Gaudi 2 supports fixed hardware-accelerated scaling factors of  $2^{-8}, 2^{-4}, 2^{0}, 2^{4}$ for E4M3 while the Gaudi 3 extends this to arbitrary powers of 2 between $2^{-32}$ and $2^{31}$. However, this feature is limited to GEMMs with per-tensor scaling factors.

\textbf{(Stochastic rounding)} During FP8 quantization, stochastic rounding can be applied when converting 16/32-bit floating-point values to FP8, similar to the technique proposed for FP32-to-BF16 conversion in \citep{hlat}. This method is distinct from stochastic rounding applied at the inner-product \citep{doi:10.1137/22M1510819}.

\textbf{(Sparsity)} NVIDIA GPUs support semi-structured sparsity acceleration, potentially doubling tensor core peak throughput. However, despite attempts to leverage sparsity in LLMs \citep{sparsegpt, wanda_sun2024a}, dense GEMMs remain dominant due to accuracy loss and limited speedups. Gaudi HPUs do not support sparsity acceleration.

\subsection{GEMM throughput measurements}




For a GEMM between matrices of size $(M \times K) \times (K \times N)$, the total number of floating-point operations (FLOPs) performed is $2MKN$. This is derived from the $M \times N$ dot products of length $K$, where each element undergoes one multiplication and one addition.
Following the convention where FLOPS denotes FLOPs per second, we compute throughput in FLOPS using the theoretical FLOPs and the measured latency. We can thus calculate the MFU by dividing observed throughput by peak throughput.

As LLM computations are dominated by matrix multiplications, GEMM throughput serves as an upper bound for model end-to-end performance. While accelerator specifications list peak throughput values, actual throughput rarely reaches this limit, particularly for small matrices.

Table \ref{tab:gemm_tflops_power} presents GEMM throughput and power consumption measurements for row-wise scaled FP8 GEMM. We conducted measurements on NVIDIA H100 GPUs using the NGC PyTorch 24.12 image and on Intel Gaudi 2 HPUs using the Synapse AI v1.19.0 image. Power consumption was recorded via the "nvidia-smi" and "hl-smi" utilities, respectively. Input casting overheads were excluded from the measurements. Additional results for GEMM throughputs under different conditions are provided in Tables \ref{tab:gaudi2_fp8_tflops} and \ref{tab:h100_fp8_tflops}.
Our results show that the Gaudi 2 achieves higher utilization than the H100, particularly for small matrices. The performance drop for smaller matrices is steeper on the H100, with the Gaudi 2 providing a higher TFLOPS at matrix sizes of 1K. Also, the Gaudi 2 exhibits lower power consumption than its stated 600W TDP, whereas the H100 approaches its peak 700W TDP even at moderate utilizations.

Unlike NVIDIA GPUs, which incorporate multiple small matrix multiplication accelerators, Gaudi HPUs feature a few large matrix accelerators. As a result, Gaudi HPUs requires fewer input elements per cycle to fully utilize compute resources, thereby reducing first-level memory bandwidth requirements and improving efficiency \citep{gaudi2_whitepaper, gaudi3_whitepaper}. Additionally, Gaudi HPUs leverage a graph compiler to dynamically reconfigure the MME size based on the target GEMM dimensions, optimizing resource utilization.

This superior efficiency of the Gaudi 2 for small matrices is particularly relevant given that LLM hidden dimensions typically range between 1K (\textit{e.g.}, Llama 1B) and 8K (\textit{e.g.}, Llama 70B), with tensor parallelism further reducing matrix sizes. Moreover, the lower power draw of the Gaudi 2 relative to its TDP suggests that naïve TDP comparisons can be misleading, emphasizing the need for empirical evaluation.

\begin{table}
\small
\centering
\caption{Throughput and power measurements for square FP8 GEMMs with row-wise scaling. The ratio of measured TFLOPS and power draw relative to their peak values are included in parentheses. We include the TFLOPS/Watt ratio on the right. H100 has 1989.9 peak FP8 GEMM TFLOPS and 700W TDP. Gaudi 2 has 865 peak FP8 GEMM TFLOPS and 600W TDP.}
\vskip 0.15in
\begin{tabular}{@{}ccrcc@{}}
\toprule
Device & (M,K,N) & \multicolumn{1}{c}{TFLOPS} & \multicolumn{1}{r}{Power (W)} & TF/W \\ \midrule
\multirow{4}{*}{Gaudi 2} & 1K & 367.9  (42.5\%) & 375 (63\%) & 1.0 \\
                         & 2K & 586.2  (67.8\%) & 460 (77\%) & 1.3 \\
                         & 4K & 817.1  (94.5\%) & 460 (77\%) & 1.8 \\
                         & 8K & 741.8  (85.8\%) & 490 (82\%) & 1.5 \\ \midrule
\multirow{4}{*}{H100}    & 1K & 218.3  (11.0\%) & 350 (50\%) & 0.6 \\
                         & 2K & 879.7  (44.2\%) & 690 (99\%) & 1.3 \\
                         & 4K & 1167.6 (58.7\%) & 690 (99\%) & 1.7 \\
                         & 8K & 1084.7 (54.5\%) & 690 (99\%) & 1.6 \\ \bottomrule
\end{tabular}
\vskip -0.1in
\label{tab:gemm_tflops_power}
\end{table}


\section{Comparing end-to-end results}\label{sec:end2end}

\begin{table}
\centering
\caption{The ``Cited" columns include results from \citet{sqzb_fp8_blog} using static ``FP8 (S)" scaling from the Intel Neural Compressor (INC) library using the UltraChat 200K \citep{ultrachat_200k} dataset for calibration. The ``Measured" column includes results for dynamic ``FP8 (D)" scaling. Reference BF16 accuracies are included for both columns to control for evaluation condition differences.}
% The dynamic FP8 scaling experiments were conducted on an H100 due to HPUs not being supported for most tasks in LM Evaluation Harness v0.4.7.
\vskip 0.15in
\small
\begin{tabular}{@{}lcccc@{}}
\toprule
\multicolumn{1}{c}{\multirow{2}{*}{Llama v3.1 8B Inst.}} & \multicolumn{2}{c}{Cited} & \multicolumn{2}{c}{Measured} \\ \cmidrule(l){2-5} 
\multicolumn{1}{c}{}                      & BF16     & FP8 (S)     & BF16      & FP8 (D)      \\ \midrule
MMLU CoT 5-shot  & 68.4\% & 66.3\% & 68.8\% & 68.3\% \\
GSM8K CoT 5-shot & 75.7\% & 70.4\% & 83.1\% & 84.5\% \\
Winogrande       & 78.1\% & 77.6\% & 73.8\% & 73.8\% \\
TruthfulQA mc1   & 37.1\% & 34.9\% & 39.9\% & 39.4\% \\
TruthfulQA mc2   & 54.0\% & 52.2\% & 55.1\% & 54.3\% \\ \bottomrule
\end{tabular}
\vskip -0.1in
\label{tab:static_vs_dynamic}
\end{table}

We conduct evaluations on different FP8 features and come to the following conclusions. First, E4M3 consistently outperforms E5M2 in quantization accuracy across nearly all tested configurations.
Second, stochastic rounding during quantization has minimal impact on model accuracy and, in some cases, may even be detrimental.
Third, dynamic scaling achieves accuracy comparable to BF16 models while eliminating the need for a calibration set.

For evaluation, we use instruction-tuned versions of Llama v3.2 1B, v3.2 3B, v3.1 8B, and v3.3 70B. Unless stated otherwise, models employ dynamic row-wise scaling for all linear layers, while LM head remains in BF16.




\subsection{Dynamic vs static scaling}

Section \ref{sec:background} describes the trade-offs between static and dynamic scaling for activation quantization. Static scaling offers higher throughput for large matrices as scaling factors do not need to be computed for each input. It also enables per-tensor quantization, leading to improved GEMM utilization. In contrast, dynamic scaling enhances output quality by assigning separate scaling factors to each token. 


% We also need an analysis of why this degradation happens, most likely because the scaling factors are too small for certain inputs.
% However, this probably cannot be done before the revision period.

Table  \ref{tab:static_vs_dynamic} presents a comparative analysis of model output quality on static vs. dynamic FP8 scaling. 
The results indicate that dynamic quantization obtains results similar to the original BF16 models, whereas static quantization leads to noticeable accuracy degradation. While various techniques, such as those proposed by \citet{xiao2024smoothquantaccurateefficientposttraining, liu2024spinquant, ashkboos2024quarot}, can potentially mitigate this degradation, they still pose a risk as the scaling factors obtained during calibration may be suboptimal for unseen inputs. Therefore, dynamic quantization is the preferred approach, provided that the associated throughput reduction remains within acceptable limits.


\begin{table}
\centering
\caption{Comparison between different FP8 data types and rounding modes for MMLU CoT 5-shot performance on instruction-tuned Llama models. Stochastic rounding provides little or no benefit to accuracy while E5M2 is detrimental, especially for smaller models. Experiments were conducted on a Gaudi 2 HPU.}
\vskip 0.15in
\small
\begin{tabular}{@{}lccc@{}}
\toprule
\multicolumn{1}{l}{\textbf{Model}} & \textbf{Data Type} &  \textbf{Rounding} & \textbf{MMLU} \\ \midrule
\multirow{3}{*}{Llama v3.2}  & BF16 & -   & 46.3\% \\
\multirow{3}{*}{1B Instruct}              & E4M3 & SR  & 45.7\% \\
                                        & E4M3 & RTN & 45.5\% \\
                                        & E5M2 & RTN & 44.5\% \\ \midrule
\multirow{3}{*}{Llama v3.2}  & BF16 & -   & 61.8\% \\
\multirow{3}{*}{3B Instruct}          & E4M3 & SR  & 61.7\% \\
                                        & E4M3 & RTN & 61.6\% \\
                                        & E5M2 & RTN & 60.7\% \\ \midrule
\multirow{3}{*}{Llama v3.1}  & BF16 & -   & 68.8\% \\
\multirow{3}{*}{8B Instruct}           & E4M3 & SR  & 68.3\% \\
                                        & E4M3 & RTN & 68.3\% \\
                                        & E5M2 & RTN & 67.5\% \\ \midrule
\multirow{3}{*}{Llama v3.3} & BF16 & -   & 82.0\% \\
\multirow{3}{*}{70B Instruct}              & E4M3 & SR  & 82.0\% \\
                                        & E4M3 & RTN & 82.0\% \\
                                        & E5M2 & RTN & 82.2\% \\ \bottomrule
\end{tabular}
\vskip -0.1in
\label{tab:e5m2_sr}
\end{table}

\subsection{E4M3 vs. E5M2}




Following on the findings of \citet{MLSYS2024_dea9b4b6}, which demonstrated that the E4M3 format achieves superior accuracy on language tasks compared to E5M2, we extend this analysis to instruction-tuned models in Table \ref{tab:e5m2_sr}.
 
To determine the optimal format for inference, we conducted a comparative evaluation of E4M3 and E5M2, measuring MMLU accuracy using LM Evaluation Harness (v0.4.7, \citep{eval-harness}).
Our experimental results consistently indicate that E4M3 outperforms E5M2 across all evaluated scenarios.
Furthermore, as shown in Table \ref{tab:gaudi2_fp8_tflops}, the GEMM throughputs using E4M3 and E5M2 are comparable, making E4M3 the preferred choice for inference even considering the smaller representational capacity of E4M3 on Gaudi 2.




\subsection{Stochastic rounding}

Hardware-accelerated stochastic rounding during FP8 quantization is a distinctive feature of Gaudi HPUs, similar to the approach proposed by \citet{hlat}, and is unavailable in NVIDIA GPUs.
Equation  \ref{eq:stochastic_rounding} formalizes the concept, where a higher-precision value  $x$ is rounded up to $x_{up}$ or down to $x_{down}$ stochastically based on the distance from $x$. 

\begin{equation}
  x_{quantized} =
    \begin{cases}
      x_{up}   \quad (p=\frac{x-x_{down}}{x_{up}-x_{down}}) \\
      x_{down} \quad (p=\frac{x_{up}-x}{x_{up}-x_{down}})
    \end{cases}
\label{eq:stochastic_rounding}
\end{equation}

Stochastic rounding is expected to preserve more of the original numerical information post-quantization, potentially leading to improved model accuracy.
However, empirical results in Table \ref{tab:e5m2_sr} indicate that stochastic rounding during quantization does not significantly enhance model accuracy. Furthermore, as shown in Table \ref{tab:sr_mmlu}, it may even lead to accuracy degradation in certain cases. These findings suggest that while stochastic rounding during quantization theoretically retains more information, its practical benefits for FP8 inference in LLMs remains inconclusive.

\begin{figure*}[]
    \centering
    \includegraphics[width=\textwidth]{figures/LLM_phase_5.png}
    \vspace{-1em}
    \caption{Process and utilization characterization of the two phases of generative LLM inference.}
    \label{fig:decode_diagram}
    \vskip -0.1in 
\end{figure*}

\section{LLM inference comparison}\label{sec:g2_vs_h100}

\subsection{LLM inference phases}

Generative LLM inference comprises two distinct phases: a compute-bound prefill phase and a memory-bound decode phase \citep{splitwise, distserve_ZhongLCHZL0024}. Figure  \ref{fig:decode_diagram} illustrates the key differences between these phases.
% We include a diagram highlighting the differences between the two phases in Figure \ref{fig:decode_diagram}.

During the prefill phase, the LLM generates the first token based on the provided input prompt. 
Throughout this process, key-value (KV) pairs are stored in the KV cache for reuse in subsequent token generations.
% The key-value pairs are stored in the KV cache for reuse during the generation of subsequent tokens. 
The prefill phase processes the entire input in parallel, utilizing matrix-matrix operations that are typically compute-bound. This high degree of parallelism allows for effective hardware utilization, enabling efficient input processing.

Once the first token is generated, the process transitions to the decode phase, where output tokens are generated sequentially in an autoregressive manner based on the previous tokens. 
Unlike the prefill phase, the decode phase is constrained by memory bandwidth, as it involves GEMV operations, which inherently limit hardware utilization compared to the more parallelizable GEMMs in the prefill phase.
% The decode phase involves GEMV operations, inherently limiting hardware utilization compared to the compute-bound GEMMs in the prefill phase.

Modern inference frameworks such as vLLM \citep{kwon2023efficient} and TensorRT-LLM \citep{TensorRT_LLM} partially remedy this issue by batching multiple decoding requests, improving the computational intensity of the linear layers. However, the batch size is limited by the memory capacity as each sequence in a batch requires its own KV cache. Furthermore, the computational intensity of the attention operation during decoding is unaffected by batch size, instead requiring a GEMV for each sequence. Grouped query attention (GQA) \citep{ainslie2023gqa} converts the operation into a thin GEMM, but it remains memory-bound. This slowdown becomes more evident at longer sequence lengths, where the attention FLOPs continue to increase in proportion to sequence length while the FLOPs required for linear layers remain constant.

\subsection{Calculating inference FLOPs}

While previous works \citep{splitwise} have evaluated inference performance using time to first token (TTFT) and time per output token (TPOT), these metrics do not facilitate comparisons across different inference stages, model sizes, or sequence lengths. To ensure a consistent comparison, we directly compute model FLOPs.
Following the approach of \citet{pytorch2, megatron_lm}, we calculate only the FLOPs associated with matrix multiplication, as these dominate LLM computations. However, unlike previous works, we exclude FLOPs related to autoregressive attention masking, which can be skipped \citep{dao2024flashattention}. This approach aligns more closely with how the KV cache is leveraged when generating new tokens.

Using this method, the FLOPs required for a forward pass of a Llama model with $l$ transformer blocks, hidden size $h$, intermediate size $ah$, head size $d$,  head count $H=h/d$, GQA group size $g$, vocabulary size $v$, and input sequence of length $s$ can be calculated as follows:

\begin{equation}
  f_{llama}(s)=2sh^2l(3a+2+\frac{2}{g})+2s^2hl+2vsh
\label{eq:model_flops}
\end{equation}

By denoting $A=(3a+2+\frac{2}{g})$ as a constant for each model, the equation can be simplified to the following:

\begin{equation}
  f_{llama}(s)=2s(Ah^2l+vh)+2s^2hl
\label{eq:model_flops_simple}
\end{equation}

When the model generates $t$ tokens in a single decoding iteration where $t \ll s$, we can calculate Equation \ref{eq:model_flops} using $s'=s+t$ and obtain the approximation:

\begin{equation}
\begin{aligned}
  f_{llama}(s+t)-f_{llama}(s) \\
  \approx 2t(Ah^2l+vh)+4sthl
\label{eq:model_flops_delta}
\end{aligned}
\end{equation}

From Equation \ref{eq:model_flops_delta}, we observe that linear layer computations, including those in the LM head, remain independent of the previous sequence length $s$. However, attention computations scale proportionally with $s$.
In the autoregressive case where $t=1$ and considering a batch of sequences with batch size $b$ and sequence lengths $s_1, ..., s_b$, the FLOPs per decoding step can be approximated as follows:

\begin{equation}
\begin{aligned}
    2b(Ah^2l+vh)+4hl\sum_{i=1}^{b}s_i
\label{eq:model_flops_delta_batch}
\end{aligned}
\end{equation}

One challenge in interpreting these results is that only the $2bAh^2l$ term, representing linear layers except the LM head, is computed in FP8, whereas the $2bvh$ term for the LM head and the $4hl\sum_{i=1}^{b}s_i$ term for attention are computed in BF16.
Another is that online KV cache dequantization would add non-trivial overhead to Equation \ref{eq:model_flops_delta_batch}. However, such overheads do not constitute additional model FLOPs as per the definition of MFU in \citet{palm}.

A more significant limitation is that each FLOP cannot be executed at full efficiency due to hardware utilization constraints. For example, the Gaudi 2 has a peak HBM bandwidth of 2.4 TB/s and a peak FP8 GEMM throughput of 865 TFLOPS, requiring a FLOP/byte ratio (computational intensity, CI) of at least 360 for optimal FP8 execution.
However, in the decoding phase, a thin GEMM of  $(b \times h)\times (h \times ah)$, where $b \ll h$ and $a \ge 1$, results in a CI of approximately $2b$ for FP8 and $b$ for BF16, far below the required intensity for peak throughput on realistic batch sizes $b$.
Additionally, matrix multiplication units operate with fixed block sizes, meaning that full utilization is only achieved when input dimensions align with hardware-friendly shapes such as multiples of 128 \citep{lee2024debunkingcudamythgpubased}.

Furthermore, KV cache computations present a unique bottleneck. Increasing batch size does not improve the CI since each sequence in a batch maintains a separate KV cache.
For a BF16 KV cache using GQA with $g$ groups, the CI is thus limited to $g$ FLOPs/byte. 
Consequently, even with a perfectly optimized kernel, the theoretical throughput of applying the query to the KV cache is capped by the memory bandwidth multiplied by the CI. For Llama v3 models with $g=8$ on a Gaudi 2 with a peak 2.4 TB/s HBM, this theoretical ceiling is 19.2 TFLOPS, a small fraction of peak GEMM throughput. As attention computations are both inherently memory-bound and scale linearly with sequence length, decoding at longer sequence lengths ultimately converges to the attention throughput, as demonstrated in Figure \ref{fig:decode_efficiency}.

By distinguishing these phases and understanding their computational characteristics, optimization strategies can be tailored to enhance the efficiency of LLM inference, especially during the resource-intensive decode phase.

\subsection{Prefill phase}

\begin{figure}[]
    \centering
    \includegraphics[width=0.48\textwidth]{figures/prefill_roofline_3.png}
    \vspace{-1.6em}
    \caption{Roofline diagram during the prefill phase for batch size 1. Throughput improves with longer sequence length until it begins to decline as the proportion of attention computations, which are slower than GEMMs, take up a greater share of the computation. We use static FP8 scaling for both the H100 and the Gaudi 2 as optimized dynamic FP8 scaling is unavailable for the Gaudi.}
    \label{fig:prefill_roofline}
    \vskip -0.1in
\end{figure}




We first conducted an analysis of the operational characteristics of the two accelerators in the prefill phase.
Figure \ref{fig:prefill_roofline} compares the prefill TFLOPS achieved for three models across varying sequence lengths, using a roofline diagram for the H100 and Gaudi 2 accelerators.
The results demonstrate that the H100 achieves significantly higher throughput during the prefill phase than the Gaudi 2. For Llama 8B, the H100 achieves double the throughput of the Gaudi 2.

During the prefill phase, attention layers process long input sequences, enabling large matrix computations to be executed in a single step. This maximizes compute engine utilization, leading to high efficiency. As a result, performance in this phase is primarily dictated by GEMM throughput, rather than utilization or memory bandwidth constraints.

As model sizes increase, matrix dimensions grow proportionally with hidden dimensions, making computations more compute-bound. This leads to a clear trend of increasing prefill throughput with larger models, underscoring the importance of compute performance in the prefill phase. A similar pattern is observed for longer sequence lengths, where utilization improves until reaching saturation.


\begin{figure}[]
    \centering
    \includegraphics[width=0.48\textwidth]{figures/static_dynamic_2.png}
    \vspace{-1.6em}
    \caption{Decode throughput comparison between BF16 and FP8 on Llama v3.1 8B Instruct using a batch size of 64 using static FP8 scaling on the Gaudi 2 (left); static and dynamic FP8 scaling on the H100 (right). Throughput differences between FP8 and BF16 in the Gaudi 2 are 50\% or greater, whereas they are under 25\% for the H100. Dynamic scaling has higher throughput despite its greater overhead because row-wise GEMM is faster than per-tensor GEMM for small matrices on the H100 (see Table \ref{tab:h100_fp8_tflops}).}
    \label{fig:static_dynamic}
\end{figure}

\subsection{Decode phase}

In Figure \ref{fig:decode_efficiency}, we compare decode throughputs between the Gaudi 2 and H100. Surprisingly, the Gaudi 2 not only demonstrates superior power efficiency across all sequence lengths but also achieves higher absolute throughput for short sequences below 4K tokens. At longer sequence lengths, however, memory bandwidth becomes the primary bottleneck. Our measurements show HBM bandwidths of 2.0 TB/s for the Gaudi 2 and 2.6 TB/s for the H100, giving the latter an advantage as the KV cache size increases.

We also compare decode throughputs for BF16, dynamic FP8 scaling, and static FP8 scaling on the H100 in Figure \ref{fig:static_dynamic} and find that the throughput gain from FP8 is below 25\%. In contrast, the gain for the Gaudi 2 approaches 50\%.






To investigate the lack of performance gains in the H100, we analyze its throughput on thin matrices, identifying it as the primary bottleneck. Table \ref{tab:thin_gemm} shows that while FP8 GEMM throughput on the Gaudi 2 is nearly twice that of BF16 GEMM, there is minimal improvement on the H100. As shown in Equation \ref{eq:model_flops_delta_batch}, the decoding stage at short sequence lengths ($s < h$) is dominated by linear operations, making thin GEMM performance critical.

\begin{table}[]
\small
\centering
\caption{Thin GEMM throughputs in TFLOPS for the Gaudi 2 and H100 measuring $(M \times K)\times(K \times N)$ GEMMs, where $M$ corresponds to typical batch sizes encountered during inference and $K,N$ represent hidden dimension sizes. Row-wise scaling is used for FP8 GEMMs.
Throughput scales linearly with $M$ on both devices, but the Gaudi 2 consistently outperforms the H100, even for BF16. These results highlight the superior efficiency of the Gaudi 2 for thin GEMMs, crucial for LLM decoding.
}
\vskip 0.15in
% We used 1,024 GEMMs per function call to hide function calling overhead.
\begin{tabular}{@{}rrrrrrr@{}}
\toprule
GEMM TFLOPS & \multicolumn{2}{c}{Gaudi 2} & \multicolumn{2}{c}{H100} \\ 
\multicolumn{1}{c}{Shape: (M,K,N)} & BF16 & FP8 & BF16 & FP8 \\ \midrule
(\enspace 8, 1024, 1024) &   3.3 &   3.8 &   1.7 &   1.7 \\
(16, 1024, 1024) &   6.5 &  11.4 &   3.4 &   3.9 \\
(32, 1024, 1024) &  12.8 &  23.8 &   6.5 &   7.0 \\
(64, 1024, 1024) &  26.7 &  54.0 &  12.6 &  14.9 \\ \midrule
(\enspace 8, 2048, 2048) &  12.4 &  26.1 &   6.7 &   7.5 \\
(16, 2048, 2048) &  20.6 &  48.6 &  12.9 &  15.0 \\
(32, 2048, 2048) &  48.0 &  87.6 &  27.1 &  28.2 \\
(64, 2048, 2048) &  91.3 & 163.2 &  52.3 &  60.5 \\ \midrule
(\enspace 8, 4096, 4096) &  18.8 &  35.4 &  14.4 &  16.8 \\
(16, 4096, 4096) &  37.4 &  67.9 &  28.6 &  33.5 \\
(32, 4096, 4096) &  73.6 & 132.0 &  68.3 &  68.1 \\
(64, 4096, 4096) & 144.5 & 253.4 & 133.3 & 133.9 \\ \bottomrule
\end{tabular}
\vskip -0.1in
\label{tab:thin_gemm}
\end{table}

\begin{figure}[]
    \centering
    \includegraphics[width=0.48\textwidth]{figures/GEMV_MFU_4.png}
    \vspace{-1.6em}
    \caption{Thin GEMM MFU comparison between the Gaudi 2 and H100 for both BF16 and FP8. The Gaudi 2 maintains a similar MFU for BF16 and FP8 of similar shapes while there is a noticeable drop for the H100. The MFU differences at the same shapes are enough to provide superior TFLOPS for the Gaudi 2 over the H100 as shown in Table \ref{tab:thin_gemm}.}
    \label{fig:Decode_efficiency}
    \vskip -0.1in
\end{figure}



\section{Conclusion}
\label{sec:Conclusion}
In this paper, we proposed a complete real-time planning and control approach for continuous, reliable, and fast online generation of dynamically feasible Bernstein trajectories and control for FW aircrafts. The generated trajectories span kilometers, navigating through multiple waypoints. By leveraging differential flatness equations for coordinated flight, we ensure precise trajectory tracking. Our approach guarantees smooth transitions from simulation to real-world applications, enabling timely field deployment. The system also features a user-friendly mission planning interface. Continuous replanning  maintains the rajectory curvature 
$\kappa$ within limits, preventing abrupt roll changes.

Future works will include the ability to add  a higher-level kinodynamic path planner to optimize waypoint spatial allocation and improve replanning success, and enhancing the trajectory-tracking algorithm by refining the aerodynamic coefficient estimation. 



% % Acknowledgements should only appear in the accepted version.
% \section*{Acknowledgements}

% \textbf{Do not} include acknowledgements in the initial version of
% the paper submitted for blind review.

% If a paper is accepted, the final camera-ready version can (and
% usually should) include acknowledgements.  Such acknowledgements
% should be placed at the end of the section, in an unnumbered section
% that does not count towards the paper page limit. Typically, this will 
% include thanks to reviewers who gave useful comments, to colleagues 
% who contributed to the ideas, and to funding agencies and corporate 
% sponsors that provided financial support.

% \section*{Impact Statement}

% Authors are \textbf{required} to include a statement of the potential 
% broader impact of their work, including its ethical aspects and future 
% societal consequences. This statement should be in an unnumbered 
% section at the end of the paper (co-located with Acknowledgements -- 
% the two may appear in either order, but both must be before References), 
% and does not count toward the paper page limit. In many cases, where 
% the ethical impacts and expected societal implications are those that 
% are well established when advancing the field of Machine Learning, 
% substantial discussion is not required, and a simple statement such 
% as the following will suffice:

% ``This paper presents work whose goal is to advance the field of 
% Machine Learning. There are many potential societal consequences 
% of our work, none which we feel must be specifically highlighted here.''

% The above statement can be used verbatim in such cases, but we 
% encourage authors to think about whether there is content which does 
% warrant further discussion, as this statement will be apparent if the 
% paper is later flagged for ethics review.

\bibliography{references}
\bibliographystyle{icml2025}

%%%%%%%%%%%%%%%%%%%%%%%%%%%%%%%%%%%%%%%%%%%%%%%%%%%%%%%%%%%%%%%%%%%%%%%%%%%%%%%
%%%%%%%%%%%%%%%%%%%%%%%%%%%%%%%%%%%%%%%%%%%%%%%%%%%%%%%%%%%%%%%%%%%%%%%%%%%%%%%
% APPENDIX
%%%%%%%%%%%%%%%%%%%%%%%%%%%%%%%%%%%%%%%%%%%%%%%%%%%%%%%%%%%%%%%%%%%%%%%%%%%%%%%
%%%%%%%%%%%%%%%%%%%%%%%%%%%%%%%%%%%%%%%%%%%%%%%%%%%%%%%%%%%%%%%%%%%%%%%%%%%%%%%
\newpage
\appendix
\onecolumn

\section{Related Work} \label{app:related_work}
\textbf{Personalized Generation} 
Due to the considerable success of large text-to-image models \cite{ramesh2022hierarchical, ramesh2021zero, saharia2022photorealistic, rombach2022high}, the field of personalized generation has been actively developed. The challenge is to customize a text-to-image model to generate specific concepts that are specified using several input images. Many different approaches \cite{DB, TI, CD, svdiff, ortogonal, profusion, elite, r1e} have been proposed to solve this problem and can be divided into the following groups: pseudo-token optimization \cite{TI, profusion, disenbooth, r1e}, diffusion fune-tuning \cite{DB, CD, profusion}, and encoder-based \cite{elite}. The pseudo-token paradigm adjusts the text encoder to convert the concept token into the proper embedding for the diffusion model. Such embedding can be optimized directly \cite{TI, r1e} or can be generated by other neural networks \cite{disenbooth, profusion}. Such approaches usually require a small number of parameters to optimize but lose the visual features of the target concept. Diffusion fine-tuning-based methods optimize almost all \cite{DB} or parts \cite{CD} of the model to reconstruct the training images of the concept. This allows the model to learn the input concept with high accuracy, but the model due to overfitting may lose the ability to edit it when generated with different text prompts. To reduce overfitting and memory usage, lightweight parameterizations \cite{svdiff, r1e, lora} have been proposed that preserve edibility but at the cost of degrading concept fidelity. Encoder-based methods \cite{elite} allow one forward pass of an encoder that has been trained on a large dataset of many different objects to embed the input concept. This dramatically speeds up the process of learning a new concept and such a model is highly editable, but the quality of recovering concept details may be low. Generally, the main problem with existing personalized generation approaches is that they struggle to simultaneously recover a concept with high quality and generate it in a variety of scenes.

\textbf{Sampling strategies}
Much research has been devoted to sampling techniques for text-to-image diffusion models, focusing not only on personalized generation but also on image editing. In this paper, we address a more specific question: how can the two trajectories -- superclass and concept -- be optimally combined to achieve both high concept fidelity and high editability? The ProFusion paper \cite{profusion} considered one way of combining these trajectories (Mixed sampling), which we analyze in detail in our paper (see Section \ref{sec:mixed_sampling}) and show its properties and problems. In ProFusion, authors additionally proposed a more complex sampling procedure, which we observed to be redundant compared to Mixed sampling, as can be seen in our experiments (see Section \ref{sec:experiments}). In Photoswap \cite{photoswap}, authors consider another way of combining trajectories by superclass and concept, which turns out to be almost identical to the Switching sampling strategy that we discuss in detail in Section \ref{sec:switching_sampling}. We show why this strategy fails to achieve simultaneous improvements in concept reconstruction and editability. In the paper, we propose a more efficient way of combining these two trajectories that achieves an optimal balance between the two key features of personalized generation: concept reconstruction and editability.

\section{Training details} \label{sec:training-details}
The Stable Diffusion-2-base model is used for all experiments. For the Dreambooth, Custom Diffusion, and Textual Inversion methods, we used the implementation from \url{https://github.com/huggingface/diffusers}.

\textbf{SVDiff} We implement the method based on \url{https://github.com/mkshing/svdiff-pytorch}. The parameterization is applied to all Text Encoder and U-Net layers. The models for all concepts were trained for $1600$ using Adam optimizer with $\text{batch size} = 1$, $\text{learning rate} = 0.001$, $\text{learning rate 1d} = 0.000001$, $\text{betas} = (0.9, 0.999)$, $\text{epsilon} = 1e\!-\!8$, and $\text{weight decay} = 0.01$. 

\textbf{Dreambooth} All query, key, and value layers in Text Encoder and U-Net were trained during fine-tuning. The models for all concepts were trained for $400$ steps using Adam optimizer with $\text{batch size} = 1$, $\text{learning rate} = 2e\!-\!5$, $\text{betas} = (0.9, 0.999)$, $\text{epsilon} = 1e\!-\!8$, and $\text{weight decay} = 0.01$. 

\textbf{Custom Diffusion} The models for all concepts were trained for $1600$ steps using Adam optimizer with $\text{batch size} = 1$, $\text{learning rate} = 0.00001$, $\text{betas} = (0.9, 0.999)$, $\text{epsilon} = 1e\!-\!8$, and $\text{weight decay} = 0.01$. 

\textbf{Textual Inversion} The models for all concepts were trained for $10000$ steps using Adam optimizer with $\text{batch size} = 1$, $\text{learning rate} = 0.005$, $\text{betas} = (0.9, 0.999)$, $\text{epsilon} = 1e\!-\!8$, and $\text{weight decay} = 0.01$. 

\textbf{ELITE} We used the pre-trained model from the official repo \url{https://github.com/csyxwei/ELITE} with $\lambda=0.6$ and inference hyperparams from the original paper.

\clearpage
\section{Superclass and concept trajectory choice}\label{app:hyper_theta}

 \begin{wrapfigure}{r}{0.45\textwidth}
    % \centering
    \includegraphics[trim={3cm 10cm 3cm 10cm},clip,width=\linewidth]{imgs/mixed_noft_nosup.pdf}
    \caption{The Pareto frontiers for original Mixed sampling and Mixed sampling in the Superclass, NoFT, and Empty Prompt setups. Mixed NoFT and Mixed Empty Prompt configurations overlap with the Pareto frontier of the original mixed sampling, but primarily in regions associated with low image similarity, which compromises concept fidelity.} \label{fig:mixed_noft_ep}
    \vspace{-0.14in}
\end{wrapfigure}

There are multiple ways to define sampling with maximized textual alignment to the prompt. However, the arbitrary choice can harm the alignment between Base sampling~\ref{eq:concept_sampling} and the selected trajectory. We use the Sampling with superclass (\ref{eq:superclass_sampling}) as it's the default choice in the literature and guarantees the maximized alignment between noise predictions $\tilde{\varepsilon}_{\theta}(p^C)$ and $\tilde{\varepsilon}_{\theta}(p^S)$. 

The several natural ways to adjust Sampling with superclass can be presented by varying $\theta$ and $p^{S}$ in (\ref{eq:superclass_sampling}). We explore two additional options with decreased alignment with (\ref{eq:concept_sampling}): (1) NoFT -- weights of base model $\theta^{\text{orig}}$ instead fine-tuned weights, (2) Empty Prompt -- prompt without any reference to a concept, even to its superclass category, i.e. $p^{\hat{S}} = \textit{"with a city in the background"}$ instead of $p^{S} = \textit{"a backpack with a city in the background"}$.

To validate the robustness of our framework for sampling method selection, we employ the original experimental protocol, supplementing the results shown in Figures~\ref{fig:examples} and~\ref{fig:profusion-photoswap}. Our analysis of Figures~\ref{fig:multi-stage_noft_ep} and~\ref{fig:masked_noft_ep} reveals that trajectories generated under the NoFT and Empty Prompt configurations (second and third columns, respectively) maintain identical method ordering to those produced by Superclass sampling ((\ref{eq:superclass_sampling}), first column).

Notably, Figure~\ref{fig:mixed_noft_ep} shows that Empty Prompt configuration demonstrates weaker alignment with Base sampling compared to NoFT, particularly at higher values of the superclass guidance scale $\omega_{s}$. This divergence manifests as reduced concept fidelity for Empty Prompt under large $\omega_{s}$. These findings highlight a practical adjustment: prioritizing smaller $\omega_{s}$ values in Empty Prompt setup preserves concept fidelity without altering the framework’s core selection logic. 

A key limitation of increased misalignment is the gradual erosion of superclass category information from generated images, which can lead to semantically inconsistent outputs. For instance, Figure~\ref{fig:examples_noft_ep} illustrates how the Mixed Empty Prompt setup, despite the strong animal prior in Base sampling, can produce human-like features in an image of a cat described as \textit{"in a chef outfit"}. This suggests that when superclass information is weakened, the model may introduce unexpected visual artifacts, impacting the fidelity of the intended concept.

Concept sampling (\ref{eq:concept_sampling}) can also be adjusted to better capture a concept’s visual characteristics, further decoupling fidelity from editability. For example, this can be achieved by (1) using the weights of a highly overfitted model (e.g., DreamBooth) or (2) selecting a prompt that omits contextual details, such as $p^{\hat{C}} = \textit{"a photo of V*"}$ instead of $p^{C} = \textit{"a V* with a city in the background"}$. Combining superclass sampling under NoFT or Empty Prompt with Base sampling configured via (1) or (2) could enhance both image and text similarity. We leave this direction for future work.

\begin{figure}[b]
    \centering
    \includegraphics[trim={3cm 10cm 3cm 10cm},clip,width=0.32\linewidth]{imgs/multi-stage_original.pdf}
    \hfill
    \includegraphics[trim={3cm 10cm 3cm 10cm},clip,width=0.32\linewidth]{imgs/multi-stage_noft.pdf}
    \hfill
    \includegraphics[trim={3cm 10cm 3cm 10cm},clip,width=0.32\linewidth]{imgs/multi-stage_nosup.pdf}
    \caption{Pareto Frontier Curves for Mixed, Switching, and Multi-Stage Sampling Methods in the Superclass, NoFT and Empty Prompt setups.
The NoFT and Empty Prompt configurations (second and third columns, respectively) preserve the same method ordering as those produced by Superclass sampling (first column).} \label{fig:multi-stage_noft_ep}
\end{figure}
\begin{figure}[t]
    \centering
    \includegraphics[trim={3cm 10cm 3cm 10cm},clip,width=0.32\linewidth]{imgs/masked_profusion.pdf}
    \hfill
    \includegraphics[trim={3cm 10cm 3cm 10cm},clip,width=0.32\linewidth]{imgs/masked_noft.pdf}
    \hfill
    \includegraphics[trim={3cm 10cm 3cm 10cm},clip,width=0.32\linewidth]{imgs/masked_nosup.pdf}
    \caption{Pareto Frontier Curves for Mixed, Switching, Masked, and ProFusion Sampling Methods in the Superclass, NoFT, and Empty Prompt setups.
The NoFT and Empty Prompt configurations (second and third columns, respectively) preserve the same method ordering as those produced by Superclass sampling (first column).} \label{fig:masked_noft_ep}
\end{figure}

\begin{figure}[b]
    \centering
    \includegraphics[width=\linewidth]{imgs/examples_noft_ep.pdf}
    \caption{Examples of the generation outputs for Mixed and ProFusion sampling methods for their optimal metrics point in the Superclass, NoFT, and Empty Prompt (EP) setups.} \label{fig:examples_noft_ep}
\end{figure}

\clearpage

\begin{figure}[ht!]
  \centering
  \includegraphics[trim={0 5cm 0 5cm},clip,width=0.95\linewidth]{imgs/us_example_new.pdf}
  \caption{An example of a task in the user study}
  \label{fig:us_ex}
  \vspace{-0.19in}
\end{figure}

\section{Data preparation}\label{app:data}
For each concept, we used inpainting augmentations to create the training dataset. We took an original image and automatically segmented it using the Segment Anything model on top of the CLIP cross-attention maps. Then we crop the concept from the original image, apply affine transformations to it, and inpaint the background. We used $10$ augmentation prompts, different from the evaluation prompts, and sampled $3$ images per prompt, resulting in a total of $30$ training images per concept. We commit to open-source the augmented datasets for each concept after publication.

\section{User Study}\label{app:us}

An example task from the user study is shown in Figure~\ref{fig:us_ex}. In total, we collected 48,864 responses from 200 unique users for 16,000 unique pairs. For each task, users were asked three questions: 1) "Which image is more consistent with the text prompt?" 2) "Which image better represents the original image?" 3) "Which image is generally better in terms of alignment with the prompt and concept identity preservation?" For each question, users selected one of three responses: "1", "2", or "Can't decide."

\section{Complex Prompts Setting}\label{app:long_prompts}

We conduct a comparison of different sampling methods using a set of complex prompts. For this analysis, we collected 10 prompts, each featuring multiple scene changes simultaneously, including stylization, background, and outfit:

\adjustbox{max width=\linewidth}{
\begin{lstlisting}
live_long = [
  "V* in a chief outfit in a nostalgic kitchen filled with vintage furniture and scattered biscuit",
  "V* sitting on a windowsill in Tokyo at dusk, illuminated by neon city lights, using neon color palette",
  "a vintage-style illustration of a V* sitting on a cobblestone street in Paris during a rainy evening, showcasing muted tones and soft grays",
  "an anime drawing of a V* dressed in a superhero cape, soaring through the skies above a bustling city during a sunset",
  "a cartoonish illustration of a V* dressed as a ballerina performing on a stage in the spotlight",
  "oil painting of a V* in Seattle during a snowy full moon night",
  "a digital painting of a V* in a wizard's robe in a magical forest at midnight, accented with purples and sparkling silver tones",
  "a drawing of a V* wearing a space helmet, floating among stars in a cosmic landscape during a starry night",
  "a V* in a detective outfit in a foggy London street during a rainy evening, using muted grays and blues",
  "a V* wearing a pirate hat exploring a sandy beach at the sunset with a boat floating in the background",
]

object_long = [
  "a digital illustration of a V* on a windowsill in Tokyo at dusk, illuminated by neon city lights, using neon color palette",
  "a sketch of a V* on a sofa in a cozy living room, rendered in warm tones",
  "a watercolor painting of a V* on a wooden table in a sunny backyard, surrounded by flowers and butterflies",
  "a V* floating in a bathtub filled with bubbles and illuminated by the warm glow of evening sunlight filtering through a nearby window",
  "a charcoal sketch of a giant V* surrounded by floating clouds during a starry night, where the moonlight creates an ethereal glow",
  "oil painting of a V* in Seattle during a snowy full moon night",
  "a drawing of a V* floating among stars in a cosmic landscape during a starry night with a spacecraft in the background",
  "a V* on a sandy beach next to the sand castle at the sunset with a floaing boat in the background",
  "an anime drawing V* on top of a white rug in the forest with a small wooden house in the background",
  "a vintage-style illustration of a V* on a cobblestone street in Paris during a rainy evening, showcasing muted tones and soft grays",
]
\end{lstlisting}
}

The results of this comparison are presented in Figures~\ref{fig:add_long},~\ref{fig:add_long_metrics}. We observe that Base sampling may struggle to preserve all the features specified by the prompts, whereas advanced sampling techniques effectively restore them. The overall arrangement of methods in the metric space closely mirrors that observed in the setting with simple prompts.

\begin{figure}[h!]
  \centering
  \includegraphics[width=\linewidth]{imgs/long_prompts_examples.pdf}
  \caption{Additional examples of the generation outputs for different sampling methods with \textbf{complex prompts}. We highlight parts of the prompt that are missing in Base sampling while appearing in other methods.}
  \label{fig:add_long}
\end{figure}

\clearpage
\section{Dreambooth results}\label{app:dreambooth}

We conduct additional analysis of different sampling methods in combination with Dreambooth. Figure~\ref{fig:add_db_metrics} shows that Mixed Sampling still overperforms Switching and Photoswap,  while Multi-stage and Masked struggle to provide an additional improvement over the simple baseline. Figure~\ref{fig:add_db} shows that all methods allow for improvement TS with a negligent decrease in IS while Mixed Sampling provides the best IS among all samplings.

\begin{figure*}[!ht]
\centering
\begin{minipage}{.477\textwidth}
  \centering
  \includegraphics[trim={3cm 10cm 3cm 10cm},clip,width=\linewidth]{imgs/long_prompts.pdf}
  \caption{CLIP metrics for different sampling methods estimated on \textbf{complex prompts}.}
  \label{fig:add_long_metrics}
\end{minipage}
\hfill
\begin{minipage}{.477\textwidth}
  \centering
  \includegraphics[trim={3cm 10cm 3cm 10cm},clip,width=\linewidth]{imgs/db_samplings.pdf}
  \caption{CLIP metrics for different sampling strategies on top of a Dreambooth fine-tuning method.}
  \label{fig:add_db_metrics}
\end{minipage}
\end{figure*} 

\begin{figure}[h!]
  \centering
  \includegraphics[trim={0 1cm 0 1cm},clip,width=\linewidth]{imgs/db_sampling_examples.pdf}
  \caption{Additional examples of the generation outputs for different sampling methods on top of a Dreambooth fine-tuning method.}
  \label{fig:add_db}
\end{figure}

\section{PixArt-alpha \& SD-XL}\label{app:add_backbones}
We conducted a series of experiments using different backbones. For SD-XL~\cite{podell2023sdxlimprovinglatentdiffusion}, we used SVDDiff as the fine-tuning method, while PixArt-alpha~\citep{chen2023pixartalphafasttrainingdiffusion} employed standard Dreambooth training. Hyperparameters for Switching, Masked, and ProFusion were selected in the same manner as in the experiments with SD2.

Figures~\ref{fig:pixart} and~\ref{fig:sdxl} demonstrate that Mixed Sampling follows a similar pattern to SD2, improving TS without a significant loss in IS. Notably, Mixed Sampling for SD-XL achieves simultaneous improvements in both IS and TS. ProFusion exhibits behavior consistent with SD2, enhancing IS more effectively than Mixed Sampling but performing worse at improving TS while also requiring twice the computational resources. 

\begin{figure}[h]
\centering
\begin{minipage}{.49\textwidth}
  \centering
  \includegraphics[trim={3cm 10cm 3cm 10cm},clip,width=\linewidth]{imgs/pixart.pdf}
  \captionof{figure}{CLIP metrics for different sampling methods estimated on PixArt model.}
  \label{fig:pixart}
\end{minipage}%
\hfill
\begin{minipage}{.49\textwidth}
  \centering
  \includegraphics[trim={3cm 10cm 3cm 10cm},clip,width=\linewidth]{imgs/sdxl.pdf}
  \captionof{figure}{CLIP metrics for different sampling methods estimated on SD-XL model.}
  \label{fig:sdxl}
\end{minipage}
\end{figure}

\clearpage
\section{Cross-Attention Masks}\label{app:cross_attn}

\begin{figure}[h!]
  \centering
  \includegraphics[trim={3cm 0cm 3cm 0cm},clip,width=\linewidth]{imgs/cross_attention_masks.pdf}
  \caption{Visualization of the cross-attention masks for Masked sampling examples. Here, $q$ defines the thresholding quantile and $t$ the denoising step.}
  \label{fig:cross_attn_add_ex}
\end{figure}

\clearpage
\section{Additional Examples}\label{app:add_example}

\begin{figure}[h!]
  \centering
  \includegraphics[trim={0 2cm 0 2cm},clip,width=\linewidth]{imgs/additional_examples.pdf}
  \caption{Additional examples of the generation outputs for different sampling methods.}
  \label{fig:add_ex}
\end{figure}

\begin{figure}[h!]
  \centering
  \includegraphics[trim={0 2cm 0 2cm},clip,width=\linewidth]{imgs/additional_examples_all.pdf}
  \caption{Additional examples of the generation outputs for Mixed and ProFusion sampling methods in comparison to the main personalized generation baselines.}
  \label{fig:add_ex_all}
\end{figure}

\clearpage
\section{DINO Image Similarity}\label{app:add_dino}

We compare CLIP-IS (left column) and DINO-IS~\citep{oquab2024dinov2learningrobustvisual} (right column) in Figures~\ref{fig:profusion_photoswap_dino},~\ref{fig:all_methods_dino}. We observe that despite the choice of metric, different sampling techniques and finetuning strategies have the same arrangement. The most noticeable difference is that SVDDiff superiority over ELITE and TI is more pronounced. That strengthens our motivation to select SVDDiff as the main backbone.

\begin{figure}[h]
\centering
\begin{minipage}{.49\textwidth}
  \centering
  \includegraphics[trim={3cm 10cm 3cm 10cm},clip,width=\linewidth]{imgs/profusion_photoswap.pdf}
\end{minipage}%
\hfill
\begin{minipage}{.49\textwidth}
  \centering
  \includegraphics[trim={3cm 10cm 3cm 10cm},clip,width=\linewidth]{imgs/profusion_photoswap_dino.pdf}
\end{minipage}
\caption{Pareto frontiers curves for Photoswap~\citep{photoswap} and ProFusion~\citep{profusion}.}\label{fig:profusion_photoswap_dino}
\end{figure}

\begin{figure}[h]
\centering
\begin{minipage}{.49\textwidth}
  \centering
  \includegraphics[trim={3cm 10cm 3cm 10cm},clip,width=\linewidth]{imgs/all_methods.pdf}
\end{minipage}%
\hfill
\begin{minipage}{.49\textwidth}
  \centering
  \includegraphics[trim={3cm 10cm 3cm 10cm},clip,width=\linewidth]{imgs/all_methods_dino.pdf}
\end{minipage}
\caption{The overall results of different sampling methods against main personalized generation baselines.}\label{fig:all_methods_dino}
\end{figure}

\end{document}


% This document was modified from the file originally made available by
% Pat Langley and Andrea Danyluk for ICML-2K. This version was created
% by Iain Murray in 2018, and modified by Alexandre Bouchard in
% 2019 and 2021 and by Csaba Szepesvari, Gang Niu and Sivan Sabato in 2022.
% Modified again in 2023 and 2024 by Sivan Sabato and Jonathan Scarlett.
% Previous contributors include Dan Roy, Lise Getoor and Tobias
% Scheffer, which was slightly modified from the 2010 version by
% Thorsten Joachims & Johannes Fuernkranz, slightly modified from the
% 2009 version by Kiri Wagstaff and Sam Roweis's 2008 version, which is
% slightly modified from Prasad Tadepalli's 2007 version which is a
% lightly changed version of the previous year's version by Andrew
% Moore, which was in turn edited from those of Kristian Kersting and
% Codrina Lauth. Alex Smola contributed to the algorithmic style files.
