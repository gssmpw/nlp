%%%%%%%% ICML 2025 EXAMPLE LATEX SUBMISSION FILE %%%%%%%%%%%%%%%%%

\documentclass{article}

% Recommended, but optional, packages for figures and better typesetting:
\usepackage{microtype}
\usepackage{graphicx}
\usepackage{subfigure}
\usepackage{booktabs} % for professional tables
% \usepackage{geometry}
\usepackage{multirow}
\usepackage{makecell}
\usepackage[usestackEOL]{stackengine}
% hyperref makes hyperlinks in the resulting PDF.
% If your build breaks (sometimes temporarily if a hyperlink spans a page)
% please comment out the following usepackage line and replace
% \usepackage{icml2025} with \usepackage[nohyperref]{icml2025} above.
\usepackage{hyperref}


% Attempt to make hyperref and algorithmic work together better:
\newcommand{\theHalgorithm}{\arabic{algorithm}}

% Use the following line for the initial blind version submitted for review:
% \usepackage{icml2025}

% If accepted, instead use the following line for the camera-ready submission:
\usepackage[accepted]{icml2025}

% For theorems and such
\usepackage{amsmath}
\usepackage{amssymb}
\usepackage{mathtools}
\usepackage{amsthm}

% if you use cleveref..
\usepackage[capitalize,noabbrev]{cleveref}

%%%%%%%%%%%%%%%%%%%%%%%%%%%%%%%%
% THEOREMS
%%%%%%%%%%%%%%%%%%%%%%%%%%%%%%%%
\theoremstyle{plain}
\newtheorem{theorem}{Theorem}[section]
\newtheorem{proposition}[theorem]{Proposition}
\newtheorem{lemma}[theorem]{Lemma}
\newtheorem{corollary}[theorem]{Corollary}
\theoremstyle{definition}
\newtheorem{definition}[theorem]{Definition}
\newtheorem{assumption}[theorem]{Assumption}
\theoremstyle{remark}
\newtheorem{remark}[theorem]{Remark}

% Todonotes is useful during development; simply uncomment the next line
%    and comment out the line below the next line to turn off comments
%\usepackage[disable,textsize=tiny]{todonotes}
\usepackage[textsize=tiny]{todonotes}


% The \icmltitle you define below is probably too long as a header.
% Therefore, a short form for the running title is supplied here:
% \icmltitlerunning{Submission and Formatting Instructions for ICML 2025}
\icmltitlerunning{An Investigation of FP8 Across Accelerators for LLM Inference}

\begin{document}

\twocolumn[
\icmltitle{An Investigation of FP8 Across Accelerators for LLM Inference}

% It is OKAY to include author information, even for blind
% submissions: the style file will automatically remove it for you
% unless you've provided the [accepted] option to the icml2025
% package.

% List of affiliations: The first argument should be a (short)
% identifier you will use later to specify author affiliations
% Academic affiliations should list Department, University, City, Region, Country
% Industry affiliations should list Company, City, Region, Country

% You can specify symbols, otherwise they are numbered in order.
% Ideally, you should not use this facility. Affiliations will be numbered
% in order of appearance and this is the preferred way.
\icmlsetsymbol{equal}{*}

\begin{icmlauthorlist}
\icmlauthor{Jiwoo Kim}{equal,POSTECH}
\icmlauthor{Joonhyung Lee}{equal,NAVER Cloud}
\icmlauthor{Gunho Park}{NAVER Cloud}
\icmlauthor{Byeongwook Kim}{NAVER Cloud}
\icmlauthor{Se Jung Kwon}{NAVER Cloud}
\icmlauthor{Dongsoo Lee}{NAVER Cloud}
\icmlauthor{Youngjoo Lee}{POSTECH}
\end{icmlauthorlist}

\icmlaffiliation{POSTECH}{Department of Electrical Engineering, Pohang University of Science and Technology, Pohang, Republic of Korea}
\icmlaffiliation{NAVER Cloud}{NAVER Cloud, Seongnam, Republic of Korea}

% \icmlcorrespondingauthor{Dongsoo Lee}{dongsoo.lee@navercorp.com}
% \icmlcorrespondingauthor{Youngjoo Lee}{youngjoo.lee@postech.ac.kr}

% You may provide any keywords that you
% find helpful for describing your paper; these are used to populate
% the "keywords" metadata in the PDF but will not be shown in the document
\icmlkeywords{FP8, LLM, Quantization, GEMM, GEMV}

\vskip 0.3in
]

% this must go after the closing bracket ] following \twocolumn[ ...

% This command actually creates the footnote in the first column
% listing the affiliations and the copyright notice.
% The command takes one argument, which is text to display at the start of the footnote.
% The \icmlEqualContribution command is standard text for equal contribution.
% Remove it (just {}) if you do not need this facility.

%\printAffiliationsAndNotice{}  % leave blank if no need to mention equal contribution
\printAffiliationsAndNotice{\icmlEqualContribution} % otherwise use the standard text.

\begin{abstract}
The introduction of 8-bit floating-point (FP8) computation units in modern AI accelerators has generated significant interest in FP8-based large language model (LLM) inference. Unlike 16-bit floating-point formats, FP8 in deep learning requires a shared scaling factor. Additionally, while E4M3 and E5M2 are well-defined at the individual value level, their scaling and accumulation methods remain unspecified and vary across hardware and software implementations. As a result, FP8 behaves more like a quantization format than a standard numeric representation.
In this work, we provide the first comprehensive analysis of FP8 computation and acceleration on two AI accelerators: the NVIDIA H100 and Intel Gaudi 2. Our findings highlight that the Gaudi 2, by leveraging FP8, achieves higher throughput-to-power efficiency during LLM inference, offering valuable insights into the practical implications of FP8 adoption for datacenter-scale LLM serving.
\end{abstract}

\begin{figure}[ht]
    \centering
    \includegraphics[width=0.8\linewidth]{graphs/greater_than_naive.pdf}
    \vspace{0.5cm}
    \includegraphics[width=0.8\linewidth]{graphs/p1_bottom.png}
    \vspace{-5pt}
    \caption{\textcolor{positional}{Positional} vs.\ \textcolor{nonpositional}{non-positional} circuits. In a \textcolor{nonpositional}{non-positional} circuit, the same edges must be included at all positions. A \textcolor{positional}{positional} circuit can distinguish between the same edge at different positions. This specificity yields better trade-offs between circuit size and faithfulness. It can also increase both precision and recall.}
    \label{fig:p1}
    \vspace{-5pt}
\end{figure}

\section{Introduction}

\looseness=-1
A primary goal of interpretability research is to characterize the internal mechanisms in language models (LMs) and other NLP models. 
A core approach in this area is \textbf{circuit discovery}---identifying the minimal subgraph within the model's computation graph that performs a specific task \citep{olah2021framework,olah-mech}.
Typically, the nodes of a circuit represent model components (e.g., attention heads, neurons, or layers).
While manual circuit discovery methods can yield position-specific insights \citep{wanginterpretability,goldowskydill2023localizingmodelbehaviorpath}, \emph{automatic methods often overlook positional information}, treating components as uniformly relevant across all input token positions \citep{conmytowards,syed2023attribution}. 
For instance, if an attention head is included in a circuit, it is assumed to contribute equally to the computation for every position in the input sequence.
The assumption that circuits are position-invariant ignores the fact that different positions often require distinct computations.
By ignoring positions, current methods limit their ability to capture mechanisms that operate across positions, such as interactions between attention heads across positions.

In this study, we start by demonstrating that positional agnosticism is a significant limitation (\S\ref{sec:motivating}). Then, to address these limitations, we introduce a new approach: position-aware edge attribution patching (PEAP; \S\ref{sec:full_circ_discovery}; Figure~\ref{fig:p1}). Current approaches  assume that if an edge is in a circuit, then the same edge will be in the circuit at all positions, thus leading to low precision. It is also assumed that an edge's importance should be aggregated across positions before deciding whether it should be included in the circuit; this can lead to cancellation effects, and thus low recall. PEAP instead allows us to compute the importance of cross-positional edges, and separately evaluates edge importance at each position. We show that this leads to smaller and more accurate circuits; see Figure~\ref{fig:p1}.

Incorporating positional information into circuit discovery is straightforward when inputs have the same length and structure across examples.

However, realistic datasets are not nearly this templatic.
How, then, can we incorporate positional information into automatic circuit discovery?
To address this challenge, we propose \textbf{schemas} (\S\ref{sec:schema}). 
Schemas assign semantic labels to spans of tokens, enabling information aggregation across examples even when the spans differ in length.

For example, in the input ``The \textcolor{positional}{war} lasted from 1453 to 14\underline{\hspace{1em}},'' the span ``\textcolor{positional}{war}'' could be labeled as ``\emph{Subject}''.
This enables handling spans with varying lengths: the phrase ``\textcolor{positional}{Black Plague}'' in another example can be treated as a single positional span with the same role as ``\textcolor{positional}{war}''.
In experiments with two LMs and three tasks, we find that circuits discovered using schemas achieve a better trade-off between circuit size and faithfulness to the model's behavior than position-agnostic circuits.
Importantly, position-aware circuits offer a more precise representation of the underlying mechanisms, providing a more concise foundation for mechanistic explanations.

We also present a fully automated pipeline for schema generation and application (\S\ref{sec:schema-generation}) using large language models (LLMs). 
We evaluate the quality of the generated schemas and their utility in discovering position-aware circuits (\S\ref{sec:schema-eval}).
Notably, circuits derived using automatically generated and applied schemas achieve comparable faithfulness scores to circuits discovered with human-designed and manually applied schemas.

We summarize our contributions as follows:
\begin{itemize}[noitemsep,leftmargin=*,topsep=1pt,parsep=1pt]
    \item Introduce a position-aware circuit discovery method, which obtains better faithfulness than position-agnostic discovery.  
    \item Introduce dataset schemas,  facilitating positional circuit discovery in more naturalistic settings. 
    \item Develop an automated schema generation and application pipeline with LLMs, yielding schemas that are comparable to manually-annotated ones.
\end{itemize}



\section{Related work}


Recent advances in single-image animatable head avatar generation can be categorized into mainly 2D-based and 3D-based approaches. 

\paragraph{\bf Image to 2D Animatable Avatar.}
2D-based methods, leveraging the power of convolutional neural networks (CNNs)~\cite{DBLP:conf/cvpr/KarrasLAHLA20,DBLP:conf/cvpr/IsolaZZE17,DBLP:conf/nips/GoodfellowPMXWOCB14}, often employ generative adversarial networks (GANs)~\cite{DBLP:conf/cvpr/StyleGAN} for direct image synthesis. Early approaches~\cite{DBLP:conf/cvpr/WangDYSW23,DBLP:conf/cvpr/BurkovPGL20,DBLP:conf/iccv/ZakharovSBL19} focus on injecting expression and pose features into the generator network, often utilizing architectures like U-Net or StyleGAN~\cite{DBLP:conf/cvpr/StyleGAN}.
Some other 2D methods~\cite{DBLP:journals/corr/abs-2407-03168,DBLP:conf/cvpr/ZhangQZZW0CW023,DBLP:conf/cvpr/HongZS022,DBLP:conf/mm/DrobyshevCKILZ22,DBLP:conf/cvpr/BurkovPGL20,DBLP:conf/nips/SiarohinLT0S19} represent expressions and poses as warping fields applied to the source image. 
Benefiting from advances in image and video diffusion networks, more recent 2D-based works~\cite{DBLP:journals/corr/abs-2410-07718,DBLP:journals/corr/abs-2406-08801,DBLP:conf/eccv/TianWZB24} get improved results with diffusion techniques. 
However, these methods still face challenges related to long generation times and significant computational resource demands. Audio-driven 2D control methods~\cite{DBLP:conf/cvpr/ZhangCWZSGSW23,DBLP:journals/corr/abs-2211-12368,DBLP:conf/iccv/GuoCLLBZ21} are easy to use but cannot explicitly control facial expressions and poses. 2D-based techniques often struggle with large pose or expression variations due to the lack of an explicit 3D structure, sometimes producing unrealistic distortions or identity changes. While some 2D methods~\cite{SadTalker,StyleHEAT,Pirenderer,DBLP:conf/cvpr/WangM021,MegaPortraits} incorporate 3D Morphable Models (3DMMs)~\cite{DBLP:conf/fgr/GerigMBELSV18,DBLP:journals/tog/LiBBL017,DBLP:conf/avss/PaysanKARV09,DBLP:conf/siggraph/BlanzV99} to mitigate these issues, they typically cannot achieve free-viewpoint rendering. 

\vspace{-0.1in}

\begin{figure*}[h]
    \centering
    \includegraphics[width=0.9\linewidth]{images/framework.pdf}
    \caption{\textbf{Overall Framework.} Our framework utilizes learnable query features attached to FLAME vertices to perform cross-attention with the extracted multi-level image features. The extracted features are then decoded to reconstruct the Gaussian avatar in the canonical space, which can be animated utilizing standard linear blend skinning (LBS) and corrective blendshapes as the FLAME model did and rendered in real-time on various platforms.}
    \label{fig:framework}
\end{figure*}

\paragraph{\bf Image to 3D Animatable Avatar.}
3D-aware methods offer improved geometric consistency and free-viewpoint rendering capabilities. Early 3D approaches~\cite{DBLP:conf/eccv/KhakhulinSLZ22,DBLP:conf/cvpr/XuYCWDJT20} utilize 3DMMs for head avatar reconstruction. With the advent of Neural Radiance Fields (NeRFs)~\cite{DBLP:conf/eccv/MildenhallSTBRN20}, many recent methods~\cite{DBLP:conf/siggraph/YuFZWYBCSWSW23,DBLP:conf/cvpr/MaZQLZ23,DBLP:conf/cvpr/LiZWZ0CZWB023,GPAvatar,ye2024real3d,deng2024portrait4d,deng2024portrait4d2,DBLP:conf/eccv/KiMC24,DBLP:conf/cvpr/BaiFWZSYS23,PointAvatar,Nerfies,INSTA} have adopted this representation for higher fidelity, particularly in modeling fine details like hair. However, NeRF-based~\cite{DBLP:conf/cvpr/ZhangZLHLWGCL024,HAvatar,DBLP:conf/cvpr/BaiTHSTQMDDOPTB23,AD-NeRF,DBLP:journals/tog/GaoZXHGZ22,DBLP:journals/tog/ParkSHBBGMS21,DBLP:conf/cvpr/AtharXSSS22,DBLP:journals/corr/abs-2112-05637,DBLP:conf/iccv/TretschkTGZLT21,DBLP:conf/cvpr/GafniTZN21,DBLP:conf/eccv/KiMC24,DBLP:conf/cvpr/BaiFWZSYS23,PointAvatar,Nerfies,DBLP:conf/siggraph/YuFZWYBCSWSW23,DBLP:conf/cvpr/MaZQLZ23,DBLP:conf/cvpr/LiZWZ0CZWB023} approaches often require extensive training data, including multi-view or single-view videos, raising privacy concerns and limiting generalization to unseen identities. Some methods~\cite{DBLP:conf/cvpr/SunWWLZZL23,DBLP:conf/3dim/ZhuangMKS22,DBLP:journals/pami/SunWZHWL24,DBLP:journals/tvcg/TangZYZCMW24,DBLP:conf/iclr/XuZLZBFS23} bypass this data requirement by training generators with random noise and then inverting them for identity-specific reconstruction, but inversion accuracy remains a challenge. Test-time optimization offers another alternative, but its computational cost limits practical applications. Several recent works~\cite{goha2023,hidenerf2023,gpavatar2024,ye2024real3d,ma2024cvthead,deng2024portrait4d,deng2024portrait4d2,GGHead} have explored one-shot 3D head reconstruction to address the limitations of data requirements and computational cost. These methods employ various techniques, such as tri-plane features, deformation fields, point-based expression fields, and vertex-feature transformers. Despite these advancements, NeRF-based methods often struggle with real-time rendering. 
Recently, 3D Gaussian Splatting~\cite{GaussianSplatting} has emerged as a promising alternative, offering both high-quality results and fast rendering speeds. However, existing Gaussian Splatting methods~\cite{GaussianAvatar,DBLP:conf/cvpr/XuCL00ZL24} typically rely on video data for training for each person, limiting their ability to generalize to new identities. Instead, the most recent work, GAGAvatar~\cite{GAGAvatar}, proposes a one-shot 3D Gaussian-based head avatar generation method. However, it still relies heavily on complex 2D neural post-processing to achieve optimal animation outcomes, thus it is not a pure 3D solution and the extra neural network hinders its application on various platforms. In contrast, our work generates Gaussian heads that are immediately animatable and renderable without additional networks or post-processing steps, enabling seamless integration into existing rendering pipelines for real-time animation and rendering across a wide range of platforms, including mobile phones. 

\section{Matrix multiplication comparison}\label{sec:matmul}

\subsection{Differences in hardware capabilities}
% I need to cite the FP8 Intel white paper here.

We compare implementations of FP8 GEMM between NVIDIA GPUs and Intel Gaudi HPUs, identifying key differences despite both adhering to the FP8 specification \citep{micikevicius2022fp8formatsdeeplearning}.

\textbf{(Accumulation precision)} Hopper GPUs use a 14-bit accumulator for FP8 GEMMs \citep{deepseekai2024deepseekv3technicalreport}, requiring casting to CUDA cores for higher precision. Software optimizations, such as applying FP32 accumulation to only one in four WGMMA instructions, reduce error but increase kernel complexity and remain less precise than full FP32 accumulation. In contrast, Gaudi HPUs always accumulate in FP32, ensuring higher numerical precision.

\textbf{(E4M3 range)} The Gaudi 2 follows the IEEE specification for special values, unlike NVIDIA GPUs, which use a single special value representation \citep{noune20228bitnumericalformatsdeep}. This results in seven fewer magnitude representations and a maximum value of 240 for E4M3 in the Gaudi 2 compared to 448 on NVIDIA GPUs. This has been addressed in the Gaudi 3, but we were unable to test these in our experiments.

\textbf{(Power-of-2 scaling)} Gaudi HPUs allow modification of floating-point exponent biases to accelerate scaling factor application. The Gaudi 2 supports fixed hardware-accelerated scaling factors of  $2^{-8}, 2^{-4}, 2^{0}, 2^{4}$ for E4M3 while the Gaudi 3 extends this to arbitrary powers of 2 between $2^{-32}$ and $2^{31}$. However, this feature is limited to GEMMs with per-tensor scaling factors.

\textbf{(Stochastic rounding)} During FP8 quantization, stochastic rounding can be applied when converting 16/32-bit floating-point values to FP8, similar to the technique proposed for FP32-to-BF16 conversion in \citep{hlat}. This method is distinct from stochastic rounding applied at the inner-product \citep{doi:10.1137/22M1510819}.

\textbf{(Sparsity)} NVIDIA GPUs support semi-structured sparsity acceleration, potentially doubling tensor core peak throughput. However, despite attempts to leverage sparsity in LLMs \citep{sparsegpt, wanda_sun2024a}, dense GEMMs remain dominant due to accuracy loss and limited speedups. Gaudi HPUs do not support sparsity acceleration.

\subsection{GEMM throughput measurements}




For a GEMM between matrices of size $(M \times K) \times (K \times N)$, the total number of floating-point operations (FLOPs) performed is $2MKN$. This is derived from the $M \times N$ dot products of length $K$, where each element undergoes one multiplication and one addition.
Following the convention where FLOPS denotes FLOPs per second, we compute throughput in FLOPS using the theoretical FLOPs and the measured latency. We can thus calculate the MFU by dividing observed throughput by peak throughput.

As LLM computations are dominated by matrix multiplications, GEMM throughput serves as an upper bound for model end-to-end performance. While accelerator specifications list peak throughput values, actual throughput rarely reaches this limit, particularly for small matrices.

Table \ref{tab:gemm_tflops_power} presents GEMM throughput and power consumption measurements for row-wise scaled FP8 GEMM. We conducted measurements on NVIDIA H100 GPUs using the NGC PyTorch 24.12 image and on Intel Gaudi 2 HPUs using the Synapse AI v1.19.0 image. Power consumption was recorded via the "nvidia-smi" and "hl-smi" utilities, respectively. Input casting overheads were excluded from the measurements. Additional results for GEMM throughputs under different conditions are provided in Tables \ref{tab:gaudi2_fp8_tflops} and \ref{tab:h100_fp8_tflops}.
Our results show that the Gaudi 2 achieves higher utilization than the H100, particularly for small matrices. The performance drop for smaller matrices is steeper on the H100, with the Gaudi 2 providing a higher TFLOPS at matrix sizes of 1K. Also, the Gaudi 2 exhibits lower power consumption than its stated 600W TDP, whereas the H100 approaches its peak 700W TDP even at moderate utilizations.

Unlike NVIDIA GPUs, which incorporate multiple small matrix multiplication accelerators, Gaudi HPUs feature a few large matrix accelerators. As a result, Gaudi HPUs requires fewer input elements per cycle to fully utilize compute resources, thereby reducing first-level memory bandwidth requirements and improving efficiency \citep{gaudi2_whitepaper, gaudi3_whitepaper}. Additionally, Gaudi HPUs leverage a graph compiler to dynamically reconfigure the MME size based on the target GEMM dimensions, optimizing resource utilization.

This superior efficiency of the Gaudi 2 for small matrices is particularly relevant given that LLM hidden dimensions typically range between 1K (\textit{e.g.}, Llama 1B) and 8K (\textit{e.g.}, Llama 70B), with tensor parallelism further reducing matrix sizes. Moreover, the lower power draw of the Gaudi 2 relative to its TDP suggests that naïve TDP comparisons can be misleading, emphasizing the need for empirical evaluation.

\begin{table}
\small
\centering
\caption{Throughput and power measurements for square FP8 GEMMs with row-wise scaling. The ratio of measured TFLOPS and power draw relative to their peak values are included in parentheses. We include the TFLOPS/Watt ratio on the right. H100 has 1989.9 peak FP8 GEMM TFLOPS and 700W TDP. Gaudi 2 has 865 peak FP8 GEMM TFLOPS and 600W TDP.}
\vskip 0.15in
\begin{tabular}{@{}ccrcc@{}}
\toprule
Device & (M,K,N) & \multicolumn{1}{c}{TFLOPS} & \multicolumn{1}{r}{Power (W)} & TF/W \\ \midrule
\multirow{4}{*}{Gaudi 2} & 1K & 367.9  (42.5\%) & 375 (63\%) & 1.0 \\
                         & 2K & 586.2  (67.8\%) & 460 (77\%) & 1.3 \\
                         & 4K & 817.1  (94.5\%) & 460 (77\%) & 1.8 \\
                         & 8K & 741.8  (85.8\%) & 490 (82\%) & 1.5 \\ \midrule
\multirow{4}{*}{H100}    & 1K & 218.3  (11.0\%) & 350 (50\%) & 0.6 \\
                         & 2K & 879.7  (44.2\%) & 690 (99\%) & 1.3 \\
                         & 4K & 1167.6 (58.7\%) & 690 (99\%) & 1.7 \\
                         & 8K & 1084.7 (54.5\%) & 690 (99\%) & 1.6 \\ \bottomrule
\end{tabular}
\vskip -0.1in
\label{tab:gemm_tflops_power}
\end{table}


\section{Comparing end-to-end results}\label{sec:end2end}

\begin{table}
\centering
\caption{The ``Cited" columns include results from \citet{sqzb_fp8_blog} using static ``FP8 (S)" scaling from the Intel Neural Compressor (INC) library using the UltraChat 200K \citep{ultrachat_200k} dataset for calibration. The ``Measured" column includes results for dynamic ``FP8 (D)" scaling. Reference BF16 accuracies are included for both columns to control for evaluation condition differences.}
% The dynamic FP8 scaling experiments were conducted on an H100 due to HPUs not being supported for most tasks in LM Evaluation Harness v0.4.7.
\vskip 0.15in
\small
\begin{tabular}{@{}lcccc@{}}
\toprule
\multicolumn{1}{c}{\multirow{2}{*}{Llama v3.1 8B Inst.}} & \multicolumn{2}{c}{Cited} & \multicolumn{2}{c}{Measured} \\ \cmidrule(l){2-5} 
\multicolumn{1}{c}{}                      & BF16     & FP8 (S)     & BF16      & FP8 (D)      \\ \midrule
MMLU CoT 5-shot  & 68.4\% & 66.3\% & 68.8\% & 68.3\% \\
GSM8K CoT 5-shot & 75.7\% & 70.4\% & 83.1\% & 84.5\% \\
Winogrande       & 78.1\% & 77.6\% & 73.8\% & 73.8\% \\
TruthfulQA mc1   & 37.1\% & 34.9\% & 39.9\% & 39.4\% \\
TruthfulQA mc2   & 54.0\% & 52.2\% & 55.1\% & 54.3\% \\ \bottomrule
\end{tabular}
\vskip -0.1in
\label{tab:static_vs_dynamic}
\end{table}

We conduct evaluations on different FP8 features and come to the following conclusions. First, E4M3 consistently outperforms E5M2 in quantization accuracy across nearly all tested configurations.
Second, stochastic rounding during quantization has minimal impact on model accuracy and, in some cases, may even be detrimental.
Third, dynamic scaling achieves accuracy comparable to BF16 models while eliminating the need for a calibration set.

For evaluation, we use instruction-tuned versions of Llama v3.2 1B, v3.2 3B, v3.1 8B, and v3.3 70B. Unless stated otherwise, models employ dynamic row-wise scaling for all linear layers, while LM head remains in BF16.




\subsection{Dynamic vs static scaling}

Section \ref{sec:background} describes the trade-offs between static and dynamic scaling for activation quantization. Static scaling offers higher throughput for large matrices as scaling factors do not need to be computed for each input. It also enables per-tensor quantization, leading to improved GEMM utilization. In contrast, dynamic scaling enhances output quality by assigning separate scaling factors to each token. 


% We also need an analysis of why this degradation happens, most likely because the scaling factors are too small for certain inputs.
% However, this probably cannot be done before the revision period.

Table  \ref{tab:static_vs_dynamic} presents a comparative analysis of model output quality on static vs. dynamic FP8 scaling. 
The results indicate that dynamic quantization obtains results similar to the original BF16 models, whereas static quantization leads to noticeable accuracy degradation. While various techniques, such as those proposed by \citet{xiao2024smoothquantaccurateefficientposttraining, liu2024spinquant, ashkboos2024quarot}, can potentially mitigate this degradation, they still pose a risk as the scaling factors obtained during calibration may be suboptimal for unseen inputs. Therefore, dynamic quantization is the preferred approach, provided that the associated throughput reduction remains within acceptable limits.


\begin{table}
\centering
\caption{Comparison between different FP8 data types and rounding modes for MMLU CoT 5-shot performance on instruction-tuned Llama models. Stochastic rounding provides little or no benefit to accuracy while E5M2 is detrimental, especially for smaller models. Experiments were conducted on a Gaudi 2 HPU.}
\vskip 0.15in
\small
\begin{tabular}{@{}lccc@{}}
\toprule
\multicolumn{1}{l}{\textbf{Model}} & \textbf{Data Type} &  \textbf{Rounding} & \textbf{MMLU} \\ \midrule
\multirow{3}{*}{Llama v3.2}  & BF16 & -   & 46.3\% \\
\multirow{3}{*}{1B Instruct}              & E4M3 & SR  & 45.7\% \\
                                        & E4M3 & RTN & 45.5\% \\
                                        & E5M2 & RTN & 44.5\% \\ \midrule
\multirow{3}{*}{Llama v3.2}  & BF16 & -   & 61.8\% \\
\multirow{3}{*}{3B Instruct}          & E4M3 & SR  & 61.7\% \\
                                        & E4M3 & RTN & 61.6\% \\
                                        & E5M2 & RTN & 60.7\% \\ \midrule
\multirow{3}{*}{Llama v3.1}  & BF16 & -   & 68.8\% \\
\multirow{3}{*}{8B Instruct}           & E4M3 & SR  & 68.3\% \\
                                        & E4M3 & RTN & 68.3\% \\
                                        & E5M2 & RTN & 67.5\% \\ \midrule
\multirow{3}{*}{Llama v3.3} & BF16 & -   & 82.0\% \\
\multirow{3}{*}{70B Instruct}              & E4M3 & SR  & 82.0\% \\
                                        & E4M3 & RTN & 82.0\% \\
                                        & E5M2 & RTN & 82.2\% \\ \bottomrule
\end{tabular}
\vskip -0.1in
\label{tab:e5m2_sr}
\end{table}

\subsection{E4M3 vs. E5M2}




Following on the findings of \citet{MLSYS2024_dea9b4b6}, which demonstrated that the E4M3 format achieves superior accuracy on language tasks compared to E5M2, we extend this analysis to instruction-tuned models in Table \ref{tab:e5m2_sr}.
 
To determine the optimal format for inference, we conducted a comparative evaluation of E4M3 and E5M2, measuring MMLU accuracy using LM Evaluation Harness (v0.4.7, \citep{eval-harness}).
Our experimental results consistently indicate that E4M3 outperforms E5M2 across all evaluated scenarios.
Furthermore, as shown in Table \ref{tab:gaudi2_fp8_tflops}, the GEMM throughputs using E4M3 and E5M2 are comparable, making E4M3 the preferred choice for inference even considering the smaller representational capacity of E4M3 on Gaudi 2.




\subsection{Stochastic rounding}

Hardware-accelerated stochastic rounding during FP8 quantization is a distinctive feature of Gaudi HPUs, similar to the approach proposed by \citet{hlat}, and is unavailable in NVIDIA GPUs.
Equation  \ref{eq:stochastic_rounding} formalizes the concept, where a higher-precision value  $x$ is rounded up to $x_{up}$ or down to $x_{down}$ stochastically based on the distance from $x$. 

\begin{equation}
  x_{quantized} =
    \begin{cases}
      x_{up}   \quad (p=\frac{x-x_{down}}{x_{up}-x_{down}}) \\
      x_{down} \quad (p=\frac{x_{up}-x}{x_{up}-x_{down}})
    \end{cases}
\label{eq:stochastic_rounding}
\end{equation}

Stochastic rounding is expected to preserve more of the original numerical information post-quantization, potentially leading to improved model accuracy.
However, empirical results in Table \ref{tab:e5m2_sr} indicate that stochastic rounding during quantization does not significantly enhance model accuracy. Furthermore, as shown in Table \ref{tab:sr_mmlu}, it may even lead to accuracy degradation in certain cases. These findings suggest that while stochastic rounding during quantization theoretically retains more information, its practical benefits for FP8 inference in LLMs remains inconclusive.

\begin{figure*}[]
    \centering
    \includegraphics[width=\textwidth]{figures/LLM_phase_5.png}
    \vspace{-1em}
    \caption{Process and utilization characterization of the two phases of generative LLM inference.}
    \label{fig:decode_diagram}
    \vskip -0.1in 
\end{figure*}

\section{LLM inference comparison}\label{sec:g2_vs_h100}

\subsection{LLM inference phases}

Generative LLM inference comprises two distinct phases: a compute-bound prefill phase and a memory-bound decode phase \citep{splitwise, distserve_ZhongLCHZL0024}. Figure  \ref{fig:decode_diagram} illustrates the key differences between these phases.
% We include a diagram highlighting the differences between the two phases in Figure \ref{fig:decode_diagram}.

During the prefill phase, the LLM generates the first token based on the provided input prompt. 
Throughout this process, key-value (KV) pairs are stored in the KV cache for reuse in subsequent token generations.
% The key-value pairs are stored in the KV cache for reuse during the generation of subsequent tokens. 
The prefill phase processes the entire input in parallel, utilizing matrix-matrix operations that are typically compute-bound. This high degree of parallelism allows for effective hardware utilization, enabling efficient input processing.

Once the first token is generated, the process transitions to the decode phase, where output tokens are generated sequentially in an autoregressive manner based on the previous tokens. 
Unlike the prefill phase, the decode phase is constrained by memory bandwidth, as it involves GEMV operations, which inherently limit hardware utilization compared to the more parallelizable GEMMs in the prefill phase.
% The decode phase involves GEMV operations, inherently limiting hardware utilization compared to the compute-bound GEMMs in the prefill phase.

Modern inference frameworks such as vLLM \citep{kwon2023efficient} and TensorRT-LLM \citep{TensorRT_LLM} partially remedy this issue by batching multiple decoding requests, improving the computational intensity of the linear layers. However, the batch size is limited by the memory capacity as each sequence in a batch requires its own KV cache. Furthermore, the computational intensity of the attention operation during decoding is unaffected by batch size, instead requiring a GEMV for each sequence. Grouped query attention (GQA) \citep{ainslie2023gqa} converts the operation into a thin GEMM, but it remains memory-bound. This slowdown becomes more evident at longer sequence lengths, where the attention FLOPs continue to increase in proportion to sequence length while the FLOPs required for linear layers remain constant.

\subsection{Calculating inference FLOPs}

While previous works \citep{splitwise} have evaluated inference performance using time to first token (TTFT) and time per output token (TPOT), these metrics do not facilitate comparisons across different inference stages, model sizes, or sequence lengths. To ensure a consistent comparison, we directly compute model FLOPs.
Following the approach of \citet{pytorch2, megatron_lm}, we calculate only the FLOPs associated with matrix multiplication, as these dominate LLM computations. However, unlike previous works, we exclude FLOPs related to autoregressive attention masking, which can be skipped \citep{dao2024flashattention}. This approach aligns more closely with how the KV cache is leveraged when generating new tokens.

Using this method, the FLOPs required for a forward pass of a Llama model with $l$ transformer blocks, hidden size $h$, intermediate size $ah$, head size $d$,  head count $H=h/d$, GQA group size $g$, vocabulary size $v$, and input sequence of length $s$ can be calculated as follows:

\begin{equation}
  f_{llama}(s)=2sh^2l(3a+2+\frac{2}{g})+2s^2hl+2vsh
\label{eq:model_flops}
\end{equation}

By denoting $A=(3a+2+\frac{2}{g})$ as a constant for each model, the equation can be simplified to the following:

\begin{equation}
  f_{llama}(s)=2s(Ah^2l+vh)+2s^2hl
\label{eq:model_flops_simple}
\end{equation}

When the model generates $t$ tokens in a single decoding iteration where $t \ll s$, we can calculate Equation \ref{eq:model_flops} using $s'=s+t$ and obtain the approximation:

\begin{equation}
\begin{aligned}
  f_{llama}(s+t)-f_{llama}(s) \\
  \approx 2t(Ah^2l+vh)+4sthl
\label{eq:model_flops_delta}
\end{aligned}
\end{equation}

From Equation \ref{eq:model_flops_delta}, we observe that linear layer computations, including those in the LM head, remain independent of the previous sequence length $s$. However, attention computations scale proportionally with $s$.
In the autoregressive case where $t=1$ and considering a batch of sequences with batch size $b$ and sequence lengths $s_1, ..., s_b$, the FLOPs per decoding step can be approximated as follows:

\begin{equation}
\begin{aligned}
    2b(Ah^2l+vh)+4hl\sum_{i=1}^{b}s_i
\label{eq:model_flops_delta_batch}
\end{aligned}
\end{equation}

One challenge in interpreting these results is that only the $2bAh^2l$ term, representing linear layers except the LM head, is computed in FP8, whereas the $2bvh$ term for the LM head and the $4hl\sum_{i=1}^{b}s_i$ term for attention are computed in BF16.
Another is that online KV cache dequantization would add non-trivial overhead to Equation \ref{eq:model_flops_delta_batch}. However, such overheads do not constitute additional model FLOPs as per the definition of MFU in \citet{palm}.

A more significant limitation is that each FLOP cannot be executed at full efficiency due to hardware utilization constraints. For example, the Gaudi 2 has a peak HBM bandwidth of 2.4 TB/s and a peak FP8 GEMM throughput of 865 TFLOPS, requiring a FLOP/byte ratio (computational intensity, CI) of at least 360 for optimal FP8 execution.
However, in the decoding phase, a thin GEMM of  $(b \times h)\times (h \times ah)$, where $b \ll h$ and $a \ge 1$, results in a CI of approximately $2b$ for FP8 and $b$ for BF16, far below the required intensity for peak throughput on realistic batch sizes $b$.
Additionally, matrix multiplication units operate with fixed block sizes, meaning that full utilization is only achieved when input dimensions align with hardware-friendly shapes such as multiples of 128 \citep{lee2024debunkingcudamythgpubased}.

Furthermore, KV cache computations present a unique bottleneck. Increasing batch size does not improve the CI since each sequence in a batch maintains a separate KV cache.
For a BF16 KV cache using GQA with $g$ groups, the CI is thus limited to $g$ FLOPs/byte. 
Consequently, even with a perfectly optimized kernel, the theoretical throughput of applying the query to the KV cache is capped by the memory bandwidth multiplied by the CI. For Llama v3 models with $g=8$ on a Gaudi 2 with a peak 2.4 TB/s HBM, this theoretical ceiling is 19.2 TFLOPS, a small fraction of peak GEMM throughput. As attention computations are both inherently memory-bound and scale linearly with sequence length, decoding at longer sequence lengths ultimately converges to the attention throughput, as demonstrated in Figure \ref{fig:decode_efficiency}.

By distinguishing these phases and understanding their computational characteristics, optimization strategies can be tailored to enhance the efficiency of LLM inference, especially during the resource-intensive decode phase.

\subsection{Prefill phase}

\begin{figure}[]
    \centering
    \includegraphics[width=0.48\textwidth]{figures/prefill_roofline_3.png}
    \vspace{-1.6em}
    \caption{Roofline diagram during the prefill phase for batch size 1. Throughput improves with longer sequence length until it begins to decline as the proportion of attention computations, which are slower than GEMMs, take up a greater share of the computation. We use static FP8 scaling for both the H100 and the Gaudi 2 as optimized dynamic FP8 scaling is unavailable for the Gaudi.}
    \label{fig:prefill_roofline}
    \vskip -0.1in
\end{figure}




We first conducted an analysis of the operational characteristics of the two accelerators in the prefill phase.
Figure \ref{fig:prefill_roofline} compares the prefill TFLOPS achieved for three models across varying sequence lengths, using a roofline diagram for the H100 and Gaudi 2 accelerators.
The results demonstrate that the H100 achieves significantly higher throughput during the prefill phase than the Gaudi 2. For Llama 8B, the H100 achieves double the throughput of the Gaudi 2.

During the prefill phase, attention layers process long input sequences, enabling large matrix computations to be executed in a single step. This maximizes compute engine utilization, leading to high efficiency. As a result, performance in this phase is primarily dictated by GEMM throughput, rather than utilization or memory bandwidth constraints.

As model sizes increase, matrix dimensions grow proportionally with hidden dimensions, making computations more compute-bound. This leads to a clear trend of increasing prefill throughput with larger models, underscoring the importance of compute performance in the prefill phase. A similar pattern is observed for longer sequence lengths, where utilization improves until reaching saturation.


\begin{figure}[]
    \centering
    \includegraphics[width=0.48\textwidth]{figures/static_dynamic_2.png}
    \vspace{-1.6em}
    \caption{Decode throughput comparison between BF16 and FP8 on Llama v3.1 8B Instruct using a batch size of 64 using static FP8 scaling on the Gaudi 2 (left); static and dynamic FP8 scaling on the H100 (right). Throughput differences between FP8 and BF16 in the Gaudi 2 are 50\% or greater, whereas they are under 25\% for the H100. Dynamic scaling has higher throughput despite its greater overhead because row-wise GEMM is faster than per-tensor GEMM for small matrices on the H100 (see Table \ref{tab:h100_fp8_tflops}).}
    \label{fig:static_dynamic}
\end{figure}

\subsection{Decode phase}

In Figure \ref{fig:decode_efficiency}, we compare decode throughputs between the Gaudi 2 and H100. Surprisingly, the Gaudi 2 not only demonstrates superior power efficiency across all sequence lengths but also achieves higher absolute throughput for short sequences below 4K tokens. At longer sequence lengths, however, memory bandwidth becomes the primary bottleneck. Our measurements show HBM bandwidths of 2.0 TB/s for the Gaudi 2 and 2.6 TB/s for the H100, giving the latter an advantage as the KV cache size increases.

We also compare decode throughputs for BF16, dynamic FP8 scaling, and static FP8 scaling on the H100 in Figure \ref{fig:static_dynamic} and find that the throughput gain from FP8 is below 25\%. In contrast, the gain for the Gaudi 2 approaches 50\%.






To investigate the lack of performance gains in the H100, we analyze its throughput on thin matrices, identifying it as the primary bottleneck. Table \ref{tab:thin_gemm} shows that while FP8 GEMM throughput on the Gaudi 2 is nearly twice that of BF16 GEMM, there is minimal improvement on the H100. As shown in Equation \ref{eq:model_flops_delta_batch}, the decoding stage at short sequence lengths ($s < h$) is dominated by linear operations, making thin GEMM performance critical.

\begin{table}[]
\small
\centering
\caption{Thin GEMM throughputs in TFLOPS for the Gaudi 2 and H100 measuring $(M \times K)\times(K \times N)$ GEMMs, where $M$ corresponds to typical batch sizes encountered during inference and $K,N$ represent hidden dimension sizes. Row-wise scaling is used for FP8 GEMMs.
Throughput scales linearly with $M$ on both devices, but the Gaudi 2 consistently outperforms the H100, even for BF16. These results highlight the superior efficiency of the Gaudi 2 for thin GEMMs, crucial for LLM decoding.
}
\vskip 0.15in
% We used 1,024 GEMMs per function call to hide function calling overhead.
\begin{tabular}{@{}rrrrrrr@{}}
\toprule
GEMM TFLOPS & \multicolumn{2}{c}{Gaudi 2} & \multicolumn{2}{c}{H100} \\ 
\multicolumn{1}{c}{Shape: (M,K,N)} & BF16 & FP8 & BF16 & FP8 \\ \midrule
(\enspace 8, 1024, 1024) &   3.3 &   3.8 &   1.7 &   1.7 \\
(16, 1024, 1024) &   6.5 &  11.4 &   3.4 &   3.9 \\
(32, 1024, 1024) &  12.8 &  23.8 &   6.5 &   7.0 \\
(64, 1024, 1024) &  26.7 &  54.0 &  12.6 &  14.9 \\ \midrule
(\enspace 8, 2048, 2048) &  12.4 &  26.1 &   6.7 &   7.5 \\
(16, 2048, 2048) &  20.6 &  48.6 &  12.9 &  15.0 \\
(32, 2048, 2048) &  48.0 &  87.6 &  27.1 &  28.2 \\
(64, 2048, 2048) &  91.3 & 163.2 &  52.3 &  60.5 \\ \midrule
(\enspace 8, 4096, 4096) &  18.8 &  35.4 &  14.4 &  16.8 \\
(16, 4096, 4096) &  37.4 &  67.9 &  28.6 &  33.5 \\
(32, 4096, 4096) &  73.6 & 132.0 &  68.3 &  68.1 \\
(64, 4096, 4096) & 144.5 & 253.4 & 133.3 & 133.9 \\ \bottomrule
\end{tabular}
\vskip -0.1in
\label{tab:thin_gemm}
\end{table}

\begin{figure}[]
    \centering
    \includegraphics[width=0.48\textwidth]{figures/GEMV_MFU_4.png}
    \vspace{-1.6em}
    \caption{Thin GEMM MFU comparison between the Gaudi 2 and H100 for both BF16 and FP8. The Gaudi 2 maintains a similar MFU for BF16 and FP8 of similar shapes while there is a noticeable drop for the H100. The MFU differences at the same shapes are enough to provide superior TFLOPS for the Gaudi 2 over the H100 as shown in Table \ref{tab:thin_gemm}.}
    \label{fig:Decode_efficiency}
    \vskip -0.1in
\end{figure}


\section{Conclusion and future work}
In this study, we examined the ability of LLMs to produce self-generated counterfactual explanations (SCEs).
We design a prompt-based setup for evaluating the efficacy of \SCEs.
Our results show that LLMs consistently struggle with generating valid \SCEs. In many cases model prediction on a \SCE does not yield the same target prediction for which the model crafted the \SCE.
Surprisingly, we find that LLMs put significant emphasis on the context---the prediction on \SCE is significantly impacted by the presence of original prediction and instructions for generating the \SCE.
Based on this empirical evidence, we argue that LLMs are still far from being able to explain their own predictions counterfactually.
Our findings add to similar insights from recent studies on other forms of self-explanations~\cite{lanham2023measuring,tanneru2024quantifying}.



Our work opens several avenues for future work. Inspired by counterfactual data augmentation~\cite{sachdeva2023catfood}, one could include the counterfactual explanation capabilities a part of the LLM training process. This inclusion may enhance the counterfactual reasoning capabilities of the LLM. Follow ups should also explore the effect of prompt tuning, specifically, model-tailored prompts for generating \SCEs. These approaches might lead to better quality \SCEs.


We limited our investigation to open source models of upto 70B parameters. Extending our analysis to larger and more recent models, \eg, DeepSeek R1 671B, and closed source models like OpenAI o3 would be an interesting avenue for future work.

Finally, our experiments were limited to relatively simple tasks: classification and mathematics problems where the solution is an integer. This limitation was mainly due to the fact that it is difficult to automatically judge validity of answers for more open-ended language generation tasks like search and information retrieval. Scaling our analysis to such tasks would require significant human-annotation resources, and is an important direction for future investigations.



% % Acknowledgements should only appear in the accepted version.
% \section*{Acknowledgements}

% \textbf{Do not} include acknowledgements in the initial version of
% the paper submitted for blind review.

% If a paper is accepted, the final camera-ready version can (and
% usually should) include acknowledgements.  Such acknowledgements
% should be placed at the end of the section, in an unnumbered section
% that does not count towards the paper page limit. Typically, this will 
% include thanks to reviewers who gave useful comments, to colleagues 
% who contributed to the ideas, and to funding agencies and corporate 
% sponsors that provided financial support.

% \section*{Impact Statement}

% Authors are \textbf{required} to include a statement of the potential 
% broader impact of their work, including its ethical aspects and future 
% societal consequences. This statement should be in an unnumbered 
% section at the end of the paper (co-located with Acknowledgements -- 
% the two may appear in either order, but both must be before References), 
% and does not count toward the paper page limit. In many cases, where 
% the ethical impacts and expected societal implications are those that 
% are well established when advancing the field of Machine Learning, 
% substantial discussion is not required, and a simple statement such 
% as the following will suffice:

% ``This paper presents work whose goal is to advance the field of 
% Machine Learning. There are many potential societal consequences 
% of our work, none which we feel must be specifically highlighted here.''

% The above statement can be used verbatim in such cases, but we 
% encourage authors to think about whether there is content which does 
% warrant further discussion, as this statement will be apparent if the 
% paper is later flagged for ethics review.

\bibliography{references}
\bibliographystyle{icml2025}

\newpage
\appendix
\onecolumn

\renewcommand{\thetable}{A\arabic{table}} % Prefix table numbers with 'A'
\renewcommand{\thefigure}{A\arabic{figure}} % Prefix figure numbers with 'A'
\renewcommand{\theequation}{A\arabic{equation}} % Prefix equation numbers with 'A'

\setcounter{table}{0} % Reset table counter
\setcounter{figure}{0} % Reset figure counter
\setcounter{equation}{0} % Reset equation counter

\section*{Appendix}

\section{Optimal Brain Surgeon Derivation}
\label{OBS_ALGORITHM}

In the original setup in OBS, we have a local quadratic model for the loss $L$ given by:
$$
    \delta L = L(w + \delta w) \approx L(w) + \nabla_w L^T \delta w + \frac{1}{2} \delta w^T H \delta w
$$
Since OBS is a pruning-after-training approach, they discarded the 1-st order component. Reducing the expression for saliency as:
$$
    \delta L = \frac{1}{2} \delta w^T H \delta w
$$
To remove a single parameter, the authors of OBS introduced the constraint $e_q^T \delta w + w_q = 0$, with $e_q$ being the $q^{\text{th}}$ canonical basis vector. The pruning is defined as a constrained optimization problem of the form:
$$
    \min_{\delta w \in \mathbb{R^d}} \left( \frac{1}{2} \delta w^T H \delta w\right),
    ~~\text{s.t}~~
    e_q^T \delta w + w_q = 0.
$$
And the choice of which parameter to remove becomes:
$$
    \min_{q \in \mathcal{Q}} \left\{
        \min_{\delta w \in \mathbb{R^d}} \left( \frac{1}{2} \delta w^T H \delta w\right),
        ~~\text{s.t}~~
        e_q^T \delta w + w_q = 0
    \right\}.
$$
To solve the internal problem, we use a Lagrange multiplier $\lambda$ to write the problem as an unconstrained optimization case as follows:
$$
    \mathcal{L}(\delta w, \lambda) =
    \frac{1}{2} \delta w^T H \delta w +
    \lambda(e_q^T \delta w + w_q).
$$
Then, to find the stationary conditions, we compute the partial derivatives with respect to $\delta w$ and $\lambda$, and equate them to 0, obtaining:
$$
    \nabla_{\delta w} \mathcal{L} = 
    H \delta w + \lambda e_q = 0 
    \rightarrow
    \delta w = - \lambda H^{-1} e_q
$$
$$
    \nabla_{\lambda} \mathcal{L} =
    e_q^T \delta w + w_q = 0
    \rightarrow
    e_q^T \delta w = -w_q
$$
With some replacements, we get:
$$
    e_q^T \delta w = -w_q
    \rightarrow
    e_q^T \left( 
        - \lambda H^{-1} e_q
    \right) = -w_q
    \rightarrow
    - \lambda e_q^T H^{-1} e_q = -w_q
    \rightarrow
    \lambda = \frac{w_q}{e_q^T H^{-1} e_q} = \frac{w_q}{[H^{-1}]_{qq}}
$$
$$
    \delta w = - \frac{w_q H^{-1} e_q}{[H^{-1}]_{qq}}
$$
Replacing the expression for $\delta w$ in the saliency expression, we have:
\begin{align*}
    \delta L = \frac{1}{2} \delta w^T H \delta w
    &= \frac{1}{2}\left(
        - \frac{w_q H^{-1} e_q}{[H^{-1}]_{qq}}
    \right)^T
    H
    \left(
        - \frac{w_q H^{-1} e_q}{[H^{-1}]_{qq}}
    \right)
    \nonumber \\
    &= 
    \frac{w_q^2}{2[H^{-1}]_{qq}^2}
    \left(
        H^{-1} e_q
    \right)^T
    H
    \left(
        H^{-1} e_q
    \right)
    \nonumber \\
    &= 
    \frac{w_q^2}{2[H^{-1}]_{qq}^2}
    e_q ^T
    H^{-1}
    e_q
    = 
    \frac{w_q^2}{2[H^{-1}]_{qq}^2}
    [H^{-1}]_{qq}
    = 
    \frac{w_q^2}{2[H^{-1}]_{qq}}
    \nonumber \\
\end{align*}
%------------------------------------------------------------------------------------------------
\newpage
\section{Fisher Brain Surgeon Sensitivity Derivation}
\label{FBSS_ALGORITHM}
As we considered a PBT setting, it is not possible to ignore the first-order term in the local quadratic approximation of the error as it could still be informative. In this case, our model for sensitivity is given by: 
$$
    \delta L = \nabla_w L^T \delta w + \frac{1}{2} \delta w^T H \delta w
$$
The process to remove a single parameter remains similar; the constraint $e_q^T \delta w + w_q = 0$, with $e_q$ is still valid, redefining the optimization problem as:
$$
    \min_{\delta w \in \mathbb{R^d}} \left(
        \nabla_w L^T \delta w +  \frac{1}{2} \delta w^T H \delta w
    \right),
    ~~\text{s.t}~~
    e_q^T \delta w + w_q = 0.
$$
And the choice of which parameter to remove becomes:
$$
    \min_{q \in \mathcal{Q}} \left\{
        \min_{\delta w \in \mathbb{R^d}} \left(
            \nabla_w L^T \delta w + \frac{1}{2} \delta w^T H \delta w
        \right),
        ~~\text{s.t}~~
        e_q^T \delta w + w_q = 0
    \right\}.
$$
Using a Lagrange multiplier $\lambda$ as in the reference case, we solve the following unconstrained optimization problem:
$$
    \mathcal{L}(\delta w, \lambda) =
    \nabla_w L^T \delta w + 
    \frac{1}{2} \delta w^T H \delta w +
    \lambda(e_q^T \delta w + w_q).
$$
With the following stationary conditions:
$$
    \nabla_{\delta w} \mathcal{L} = 
    \nabla_w L + H \delta w + \lambda e_q = 0 
    \rightarrow
    \delta w = - (\lambda H^{-1}e_q + H^{-1} \nabla_w L)
$$
$$
    \nabla_{\lambda} \mathcal{L} =
    e_q^T \delta w + w_q = 0
    \rightarrow
    e_q^T \delta w = -w_q
$$
The expression for $\lambda$ is redefined as follows:
\begin{align*}
    e_q^T \left(
        - (\lambda H^{-1}e_q + H^{-1} \nabla_w L)
    \right) 
    &= -w_q
    \nonumber \\
    \lambda e_q^T H^{-1} e_q + e_q^T H^{-1} \nabla_w L
    &= w_q
    \nonumber \\
    \lambda [H^{-1}]_{qq} 
    &= w_q - e_q^T H^{-1} \nabla_w L
    \nonumber \\
    \lambda
    &= \frac{w_q - e_q^T H^{-1} \nabla_w L}{[H^{-1}]_{qq}}
\end{align*}
Replacing the expression for $\delta w$ in our sensitivity expression, we have:
\begin{align*}
    \delta L = \nabla_w L^T \delta w + \frac{1}{2} \delta w^T H \delta w
    &= 
    \nabla_w L^T \left[
        - (\lambda H^{-1}e_q + H^{-1} \nabla_w L)
    \right]
    \nonumber \\
    &+
    \frac{1}{2}\left[
        - (\lambda H^{-1}e_q + H^{-1} \nabla_w L)
    \right]^T
    H
    \left[
        - (\lambda H^{-1}e_q + H^{-1} \nabla_w L)
    \right]
    \nonumber \\
    &= 
    - \lambda \nabla_w L^T H^{-1}e_q - \nabla_w L^T H^{-1} \nabla_w L
    \nonumber \\
    &+
    \frac{1}{2}\left[
        (\lambda H^{-1}e_q)^T + (H^{-1} \nabla_w L)^T
    \right]
    \left[
        \lambda H H^{-1}e_q + H H^{-1} \nabla_w L)
    \right]
    \nonumber \\
    &= 
    - \lambda \nabla_w L^T H^{-1}e_q - \nabla_w L^T H^{-1} \nabla_w L
    \nonumber \\
    &+
    \frac{1}{2}\left[
        (\lambda H^{-1}e_q)^T + (H^{-1} \nabla_w L)^T
    \right]
    \left[
        \lambda e_q + \nabla_w L
    \right]
    \nonumber \\
    &= 
    - \lambda \nabla_w L^T H^{-1}e_q - \nabla_w L^T H^{-1} \nabla_w L
    \nonumber \\
    &+
    \frac{1}{2}\left[
        (\lambda H^{-1}e_q)^T \lambda e_q
        + (H^{-1} \nabla_w L)^T \lambda e_q
        + (\lambda H^{-1}e_q)^T \nabla_w L
        + (H^{-1} \nabla_w L)^T \nabla_w L
    \right]
    \nonumber \\
    &= 
    - \lambda \nabla_w L^T H^{-1}e_q - \nabla_w L^T H^{-1} \nabla_w L
    \nonumber \\
    &+
    \frac{1}{2}\left[
        \lambda^2 e_q^T H^{-1} e_q
        + \lambda \nabla_w L^T H^{-1} e_q
        + \lambda e_q^T H^{-1} \nabla_w L
        + \nabla_w L^T H^{-1} \nabla_w L
    \right]
    \nonumber \\
    &= 
    \frac{1}{2}\left[
        \lambda^2 [H^{-1}]_{qq}
        - \lambda \nabla_w L^T H^{-1} e_q
        + \lambda e_q^T H^{-1} \nabla_w L
        - \nabla_w L^T H^{-1} \nabla_w L
    \right]
    \nonumber \\
\end{align*}
Finally, replacing the $\lambda$:
\begin{align*}
    \delta L 
    &= 
    \frac{1}{2}\left[
        \lambda^2 [H^{-1}]_{qq}
        - \lambda \nabla_w L^T H^{-1} e_q
        + \lambda e_q^T H^{-1} \nabla_w L
        - \nabla_w L^T H^{-1} \nabla_w L
    \right]
    \nonumber \\
    &= 
    \frac{1}{2[H^{-1}]_{qq}}\left[
        (w_q - e_q^T H^{-1} \nabla_w L)^2 
        + (w_q - e_q^T H^{-1} \nabla_w L)(e_q^T H^{-1} \nabla_w L - \nabla_w L^T H^{-1} e_q)
        - \nabla_w L^T H^{-1} \nabla_w L
    \right]
    \nonumber \\
    &= 
    \frac{1}{2[H^{-1}]_{qq}}[
        w_q^2
        - 2 w_q (e_q^T H^{-1} \nabla_w L)
        + (e_q^T H^{-1} \nabla_w L)^2
        + w_q (e_q^T H^{-1} \nabla_w L)
    \nonumber \\
        &- w_q (\nabla_w L^T H^{-1} e_q)
        - (e_q^T H^{-1} \nabla_w L)(e_q^T H^{-1} \nabla_w L)
        + (e_q^T H^{-1} \nabla_w L)(\nabla_w L^T H^{-1} e_q)
        - \nabla_w L^T H^{-1} \nabla_w L
    ]
    \nonumber \\
    &= 
    \frac{1}{2[H^{-1}]_{qq}}[
        w_q^2
        - w_q (e_q^T H^{-1} \nabla_w L)
        + (e_q^T H^{-1} \nabla_w L)^2
    \nonumber \\
        &- w_q (\nabla_w L^T H^{-1} e_q)
        - (e_q^T H^{-1} \nabla_w L)^2
        + (e_q^T H^{-1} \nabla_w L)(\nabla_w L^T H^{-1} e_q)
        - \nabla_w L^T H^{-1} \nabla_w L
    ]
    \nonumber \\
    &= 
    \frac{1}{2[H^{-1}]_{qq}}\left[
        w_q^2
        - 2 w_q (e_q^T H^{-1} \nabla_w L)
        + (e_q^T H^{-1} \nabla_w L)^2
        - \nabla_w L^T H^{-1} \nabla_w L
    \right]
    \nonumber \\
    &= 
    \frac{1}{2[\hat{F}^{-1}]_{qq}}
    \left[
        w_q - (e_q^T \hat{F}^{-1} \nabla \mathcal{L}(w_0))
    \right]^2
\end{align*}

%------------------------------------------------------------------------------------------------

\newpage
\section{Training and Testing Details}
\label{appendix:training_parameters}

We perform an 80:20 stratified split, with a constant seed, on the CIFAR10/100 training dataset to obtain a validation set with the same class distribution. For both datasets, we have a training set with 40,000 samples, a validation set with 10,000 samples, and a testing set of 10,000 samples. Validation is performed after each training step, and the weights of the best-performing validation step (based on top-1 accuracy) are utilized for the final evaluation on the testing set. Table \ref{tab:table_training_parameters} summarizes the training parameters.

\begin{table}[h]
\caption{Training parameters used for ResNet18 and VGG19 on the CIFAR-10/100 datasets.}
\label{tab:table_training_parameters}
\vskip 0.15in
\begin{center}
\begin{small}
\begin{sc}
\begin{tabular}{lcc}
\toprule
Parameter & ResNet18 & VGG19 \\
\midrule
Number of steps       & 160 & 160 \\
Criterion             & CE & CE \\
Optimizer             & SGD & SGD \\
Learning rate         & 0.01 & 0.1 \\
Momentum              & 0.9 & 0.9 \\
Weight decay          & $5 \times 10^{-4}$ & $1 \times 10^{-4}$ \\
Learning rate drops   & [60, 120] & [60, 120] \\
Learning rate drop factor & 0.2 & 0.1 \\
\bottomrule
\end{tabular}
\end{sc}
\end{small}
\end{center}
\vskip -0.1in
\end{table}

%------------------------------------------------------------------------------------------------

\newpage
\section{Results CIFAR10}
\subsection{ResNet18}
\label{appendix:CIFAR10_ResNet18}

\begin{table}[h]
\caption{Performance of different sensitivity methods for pruning evaluated using ResNet18 on the CIFAR-10 testset. The right side of the table presents our proposed criteria. The mean accuracy and standard deviation are reported across three initialization seeds for various sparsity levels. Baseline, no pruning: $91.78 \pm 0.09$.}
\label{tab:resnet18_cifar10_compressors}
\vskip 0.15in
\begin{center}
\begin{small}
\begin{sc}
\resizebox{\textwidth}{!}{%
\begin{tabular}{lccccc|cccc}
\toprule
Sparsity  & Random & Magnitude & GN & SNIP & GraSP & FD & FP & FTS & FBSS \\
\midrule
0.10  & 91.71 ± 0.21 & 91.72 ± 0.07 & 91.57 ± 0.15 & 91.72 ± 0.07 & 89.16 ± 0.05 & 91.87 ± 0.13 & 91.63 ± 0.21 & 91.53 ± 0.12 & 91.76 ± 0.08 \\
0.20  & 91.63 ± 0.11 & 91.42 ± 0.12 & 91.51 ± 0.09 & 91.64 ± 0.16 & 88.69 ± 0.34 & 91.50 ± 0.12 & 91.65 ± 0.14 & 91.53 ± 0.15 & 91.54 ± 0.13 \\
0.30  & 91.45 ± 0.18 & 91.61 ± 0.13 & 91.68 ± 0.20 & 91.65 ± 0.08 & 88.67 ± 0.26 & 91.65 ± 0.18 & 91.44 ± 0.27 & 91.49 ± 0.05 & 91.62 ± 0.07 \\
0.40  & 91.59 ± 0.18 & 91.06 ± 0.16 & 91.61 ± 0.09 & 91.55 ± 0.08 & 88.24 ± 0.33 & 91.51 ± 0.05 & 91.38 ± 0.13 & 91.56 ± 0.28 & 91.39 ± 0.05 \\
0.50  & 91.60 ± 0.06 & 91.32 ± 0.13 & 91.44 ± 0.13 & 91.22 ± 0.07 & 87.69 ± 0.15 & 91.30 ± 0.18 & 91.58 ± 0.16 & 91.46 ± 0.19 & 91.41 ± 0.05 \\
0.60  & 91.10 ± 0.16 & 91.18 ± 0.16 & 91.59 ± 0.13 & 91.24 ± 0.04 & 87.48 ± 0.55 & 91.34 ± 0.07 & 91.35 ± 0.16 & 91.40 ± 0.11 & 91.38 ± 0.18 \\
0.70  & 91.17 ± 0.04 & 91.07 ± 0.07 & 91.19 ± 0.17 & 91.33 ± 0.18 & 87.26 ± 0.34 & 91.34 ± 0.23 & 91.42 ± 0.23 & 91.18 ± 0.18 & 91.27 ± 0.14 \\
0.80  & 90.78 ± 0.08 & 91.10 ± 0.12 & 90.95 ± 0.35 & 90.74 ± 0.10 & 87.18 ± 0.51 & 90.95 ± 0.11 & 91.08 ± 0.06 & 90.94 ± 0.22 & 90.73 ± 0.33 \\
0.90  & 89.35 ± 0.13 & 89.88 ± 0.28 & 90.39 ± 0.23 & 90.36 ± 0.34 & 86.60 ± 0.51 & 90.04 ± 0.21 & 90.20 ± 0.08 & 90.55 ± 0.23 & 89.22 ± 0.30 \\
0.95  & 87.59 ± 0.11 & 89.23 ± 0.19 & 89.00 ± 0.05 & 89.31 ± 0.17 & 86.50 ± 0.05 & 88.61 ± 0.28 & 89.50 ± 0.18 & 89.47 ± 0.32 & 87.58 ± 0.25 \\
0.98  & 83.47 ± 0.20 & 85.70 ± 0.33 & 86.43 ± 0.05 & 87.26 ± 0.28 & 85.99 ± 0.08 & 85.61 ± 0.20 & 86.97 ± 0.22 & 87.24 ± 0.32 & 83.40 ± 0.74 \\
0.99  & 78.28 ± 0.45 & 71.99 ± 0.28 & 83.47 ± 0.15 & 84.54 ± 0.04 & 84.56 ± 0.46 & 82.13 ± 0.28 & 83.74 ± 0.48 & 84.85 ± 0.18 & 77.60 ± 1.02 \\
\bottomrule
\end{tabular}}
\end{sc}
\end{small}
\end{center}
\vskip -0.1in
\end{table}

%------------------------------------------------------------------------------------------------
\clearpage
\subsection{VGG19}
\label{appendix:CIFAR10_VGG19}

As discussed earlier, introducing a warm-up phase effectively mitigates layer collapse in data-dependent pruning methods. Here, we evaluate the impact of different warm-up durations by comparing no warm-up, a single warm-up epoch, and five warm-up epochs. Table \ref{tab:VGG19_cifar10_compressors} demonstrates how performance drastically degrades with increasing sparsity, ultimately leading to layer collapse at 0.90 sparsity. However, as shown in the results, a single warm-up epoch is sufficient to prevent collapse and stabilize pruning performance. Moreover, as seen in Table \ref{tab:VGG19_cifar10_compressors_warmup5}, increasing the warm-up period to five epochs provides no substantial additional improvement. This indicates that prolonged warm-up training is not necessary; a single training step is enough to achieve gradient stabilization and overcome layer collapse.

\begin{table}[h]
\caption{Performance of different sensitivity methods for pruning evaluated using VGG19 on the CIFAR-10 test set. The right side of the table presents our proposed criteria. The mean accuracy and standard deviation are reported across three initialization seeds for various sparsity levels. Baseline, no pruning: $89.21 \pm 0.22$.}
\label{tab:VGG19_cifar10_compressors}
\vskip 0.15in
\begin{center}
\begin{small}
\begin{sc}
\resizebox{\textwidth}{!}{%
\begin{tabular}{lccccc|cccc}
\toprule
Sparsity  & Random & Magnitude & GN & SNIP & GraSP & FD & FP & FTS & FBSS \\
\midrule
0.10  & 88.40 ± 0.95 & 89.12 ± 0.55 & 90.14 ± 0.10 & 90.16 ± 0.18 & 87.81 ± 1.66 & 90.20 ± 0.29 & 90.21 ± 0.37 & 90.25 ± 0.38 & 89.06 ± 0.75 \\
0.20  & 89.19 ± 0.22 & 89.65 ± 0.60 & 89.59 ± 0.69 & 90.06 ± 0.04 & 89.57 ± 0.34 & 89.91 ± 0.28 & 90.28 ± 0.55 & 89.80 ± 0.28 & 88.89 ± 0.76 \\
0.30  & 88.93 ± 0.83 & 88.77 ± 1.07 & 90.23 ± 0.09 & 89.88 ± 0.59 & 89.14 ± 0.19 & 90.25 ± 0.09 & 89.97 ± 0.26 & 90.46 ± 0.41 & 89.06 ± 0.36 \\
0.40  & 88.28 ± 1.08 & 89.38 ± 0.53 & 90.50 ± 0.23 & 89.79 ± 0.67 & 88.20 ± 0.31 & 90.51 ± 0.12 & 90.37 ± 0.24 & 90.23 ± 0.14 & 10.00 ± 0.00 \\
0.50  & 88.96 ± 0.82 & 89.03 ± 0.59 & 90.46 ± 0.60 & 90.38 ± 0.25 & 88.67 ± 0.23 & 89.54 ± 0.86 & 90.47 ± 0.52 & 90.19 ± 0.31 & 10.00 ± 0.00 \\
0.60  & 88.15 ± 0.68 & 89.47 ± 0.18 & 89.95 ± 0.30 & 90.32 ± 0.25 & 88.82 ± 0.32 & 90.02 ± 0.40 & 90.18 ± 0.33 & 90.14 ± 0.36 & 10.00 ± 0.00 \\
0.70  & 88.02 ± 0.53 & 89.63 ± 0.44 & 89.69 ± 0.42 & 89.23 ± 0.19 & 89.62 ± 0.81 & 89.85 ± 0.08 & 90.01 ± 0.34 & 10.00 ± 0.00 & 10.00 ± 0.00 \\
0.80  & 88.28 ± 0.34 & 89.62 ± 0.91 & 85.72 ± 0.63 & 89.39 ± 0.43 & 88.82 ± 0.14 & 10.00 ± 0.00 & 88.29 ± 0.11 & 10.00 ± 0.00 & 10.00 ± 0.00 \\
0.90  & 85.82 ± 0.19 & 89.29 ± 0.79 & 10.00 ± 0.00 & 80.85 ± 0.62 & 24.28 ± 20.2 & 10.00 ± 0.00 & 10.00 ± 0.00 & 10.00 ± 0.00 & 10.00 ± 0.00 \\
0.95  & 84.41 ± 0.05 & 10.00 ± 0.00 & 10.00 ± 0.00 & 10.00 ± 0.00 & 10.00 ± 0.00 & 10.00 ± 0.00 & 10.00 ± 0.00 & 10.00 ± 0.00 & 10.00 ± 0.00 \\
0.98  & 80.04 ± 0.90 & 10.00 ± 0.00 & 10.00 ± 0.00 & 10.00 ± 0.00 & 10.00 ± 0.00 & 10.00 ± 0.00 & 10.00 ± 0.00 & 10.00 ± 0.00 & 10.00 ± 0.00 \\
0.99  & 76.89 ± 0.26 & 10.00 ± 0.00 & 10.00 ± 0.00 & 10.00 ± 0.00 & 10.00 ± 0.00 & 10.00 ± 0.00 & 10.00 ± 0.00 & 10.00 ± 0.00 & 10.00 ± 0.00 \\
\bottomrule
\end{tabular}}
\end{sc}
\end{small}
\end{center}
\vskip -0.1in
\end{table}
\newpage
%------------------------------------------------------------------------------------------------
\begin{table*}[h]
\caption{Performance of different compression methods evaluated after 1 warmup epoch using VGG19 on the CIFAR-10 dataset. We report the mean accuracy between three initialization seeds across various sparsity levels. Baseline, no pruning: $89.21 \pm 0.22$.}
\label{tab:VGG19_cifar10_compressors_warmup1}
\vskip 0.15in
\begin{center}
\begin{small}
\begin{sc}
\resizebox{\textwidth}{!}{%
\begin{tabular}{lccccc|cccc}
\toprule
Sparsity  & Random & Magnitude & GN & SNIP & GraSP & FD & FP & FTS & FBSS \\
\midrule
0.80  & 88.73 ± 0.38 & 88.35 ± 0.54 & 86.76 ± 0.27 & 87.39 ± 0.66 & 87.24 ± 0.25 & 87.14 ± 0.45 & 87.00 ± 0.87 & 87.68 ± 0.33 & 64.33 ± 15.91 \\
0.90  & 87.26 ± 0.42 & 88.62 ± 0.49 & 85.96 ± 0.75 & 86.75 ± 0.76 & 87.47 ± 0.33 & 86.69 ± 0.72 & 87.09 ± 0.31 & 87.42 ± 0.21 & 46.16 ± 7.62 \\
0.95  & 85.47 ± 0.64 & 87.68 ± 0.49 & 86.66 ± 0.27 & 86.00 ± 1.10 & 86.71 ± 1.24 & 85.71 ± 1.35 & 86.73 ± 0.36 & 87.56 ± 0.62 & 46.30 ± 5.32 \\
0.98  & 80.44 ± 0.30 & 86.61 ± 0.62 & 84.72 ± 1.69 & 87.22 ± 0.23 & 86.45 ± 0.64 & 80.34 ± 6.43 & 86.07 ± 0.39 & 86.36 ± 0.29 & 49.05 ± 4.31 \\
0.99  & 77.24 ± 0.73 & 83.69 ± 1.36 & 80.28 ± 2.04 & 83.49 ± 1.77 & 85.39 ± 0.43 & 75.11 ± 7.80 & 84.40 ± 1.27 & 85.35 ± 1.05 & 47.10 ± 4.41 \\
\bottomrule
\end{tabular}}
\end{sc}
\end{small}
\end{center}
\vskip -0.1in
\end{table*} 
%------------------------------------------------------------------------------------------------

\begin{table}[h]
\caption{Performance of different sensitivity methods for pruning evaluated after 5 warmup epochs using VGG19 on the CIFAR-10 testset. The right side of the table presents our proposed criteria. The mean accuracy and standard deviation are reported across three initialization seeds for various sparsity levels. Baseline, no pruning: $89.21 \pm 0.22$.}
\label{tab:VGG19_cifar10_compressors_warmup5}
\vskip 0.15in
\begin{center}
\begin{small}
\begin{sc}
\resizebox{\textwidth}{!}{%
\begin{tabular}{lccccc|cccc}
\toprule
Sparsity  & Random & Magnitude & GN & SNIP & GraSP & FD & FP & FTS & FBSS \\
\midrule
0.80  & 88.84 ± 0.43 & 88.41 ± 0.47 & 87.58 ± 0.52 & 88.15 ± 1.09 & 86.77 ± 1.14 & 87.28 ± 0.90 & 88.22 ± 0.82 & 86.68 ± 0.61 & 70.52 ± 9.25 \\
0.90  & 87.56 ± 0.62 & 88.60 ± 0.93 & 86.73 ± 0.37 & 87.89 ± 0.25 & 87.10 ± 0.47 & 87.50 ± 1.42 & 88.18 ± 0.47 & 86.98 ± 0.14 & 47.78 ± 1.26 \\
0.95 & 85.51 ± 0.69 & 87.66 ± 1.19 & 87.44 ± 0.46 & 87.71 ± 0.82 & 87.05 ± 0.16 & 86.83 ± 1.47 & 87.36 ± 0.52 & 87.00 ± 0.74 & 48.83 ± 2.52 \\
0.98 & 82.09 ± 0.17 & 86.24 ± 0.52 & 84.66 ± 1.33 & 86.55 ± 0.84 & 86.04 ± 0.66 & 85.44 ± 0.64 & 86.64 ± 0.13 & 84.89 ± 0.51 & 49.48 ± 0.85 \\
0.99 & 77.22 ± 1.03 & 83.93 ± 1.80 & 81.62 ± 2.17 & 84.53 ± 0.70 & 81.33 ± 5.77 & 81.71 ± 1.41 & 85.02 ± 0.69 & 83.78 ± 0.80 & 41.24 ± 1.55 \\
\bottomrule
\end{tabular}}
\end{sc}
\end{small}
\end{center}
\vskip -0.1in
\end{table}

%------------------------------------------------------------------------------------------------

\newpage
\section{Results CIFAR100}
\subsection{ResNet18}
\label{sec:resnet_cifar-100}

CIFAR-100 results exhibit a similar trend to those observed on CIFAR-10, further reinforcing the robustness of our proposed Fisher-Taylor Sensitivity (FTS) criterion. Across all evaluated sparsity levels, FTS consistently maintains strong performance, frequently ranking among the top-performing methods. This trend is particularly evident at extreme sparsities, where many pruning approaches suffer significant performance degradation. The stability of FTS across both datasets highlights its effectiveness in preserving network expressivity despite aggressive pruning.

\begin{table}[h]
\caption{Performance of different compression methods evaluated using ResNet18 on the CIFAR-100 dataset. We report the mean accuracy between three initialization seeds across various sparsity levels. Baseline, no pruning: $69.57 \pm 0.19$.}
\label{tab:resnet18_cifar100_compressors}
\vskip 0.15in
\begin{center}
\begin{small}
\begin{sc}
\resizebox{\textwidth}{!}{%
\begin{tabular}{lccccc|cccc}
\toprule
Sparsity  & Random & Magnitude & GN & SNIP & GraSP & FD & FP & FTS & FBSS \\
\midrule
0.10  & 69.16 ± 0.11 & 69.37 ± 0.14 & 69.63 ± 0.34 & 69.42 ± 0.07 & 64.26 ± 0.27 & 69.66 ± 0.30 & 69.08 ± 0.21 & 69.16 ± 0.11 & 69.07 ± 0.10 \\
0.20  & 69.16 ± 0.30 & 69.06 ± 0.24 & 69.19 ± 0.11 & 69.30 ± 0.08 & 63.28 ± 0.58 & 69.60 ± 0.30 & 69.35 ± 0.35 & 69.41 ± 0.43 & 69.07 ± 0.20 \\
0.30  & 69.36 ± 0.18 & 68.58 ± 0.36 & 69.37 ± 0.13 & 68.82 ± 0.17 & 62.02 ± 0.43 & 69.24 ± 0.40 & 68.84 ± 0.13 & 68.80 ± 0.55 & 68.96 ± 0.11 \\
0.40  & 69.41 ± 0.20 & 68.50 ± 0.29 & 69.16 ± 0.26 & 68.95 ± 0.19 & 61.18 ± 0.19 & 69.17 ± 0.16 & 68.88 ± 0.25 & 69.02 ± 0.21 & 68.92 ± 0.25 \\
0.50  & 69.12 ± 0.46 & 68.17 ± 0.20 & 68.94 ± 0.20 & 68.63 ± 0.11 & 61.11 ± 0.40 & 69.13 ± 0.13 & 68.68 ± 0.12 & 68.71 ± 0.12 & 68.71 ± 0.57 \\
0.60  & 68.66 ± 0.27 & 67.78 ± 0.35 & 68.77 ± 0.17 & 68.63 ± 0.42 & 61.40 ± 0.78 & 68.34 ± 0.43 & 67.98 ± 0.23 & 68.41 ± 0.14 & 68.60 ± 0.15 \\
0.70  & 67.95 ± 0.43 & 67.51 ± 0.24 & 68.29 ± 0.39 & 68.08 ± 0.18 & 59.43 ± 0.76 & 68.03 ± 0.46 & 67.96 ± 0.15 & 68.29 ± 0.06 & 68.16 ± 0.07 \\
0.80  & 67.26 ± 0.48 & 66.55 ± 0.19 & 67.20 ± 0.37 & 67.21 ± 0.38 & 59.08 ± 0.22 & 66.70 ± 0.05 & 67.05 ± 0.06 & 66.77 ± 0.65 & 66.62 ± 0.43 \\
0.90  & 64.75 ± 0.16 & 64.48 ± 0.18 & 64.87 ± 0.27 & 65.70 ± 0.08 & 59.16 ± 0.91 & 64.74 ± 0.44 & 65.46 ± 0.30 & 65.41 ± 0.13 & 63.90 ± 0.31 \\
0.95  & 61.01 ± 0.32 & 62.20 ± 0.06 & 62.20 ± 0.23 & 63.20 ± 0.20 & 57.91 ± 0.09 & 62.14 ± 0.42 & 63.22 ± 0.25 & 63.21 ± 0.47 & 61.25 ± 0.44 \\
0.98  & 54.72 ± 0.22 & 55.44 ± 0.18 & 57.34 ± 0.31 & 58.83 ± 0.35 & 54.85 ± 0.35 & 55.57 ± 0.17 & 58.05 ± 0.18 & 58.59 ± 0.12 & 55.02 ± 0.34 \\
0.99  & 45.62 ± 0.55 & 40.39 ± 0.36 & 50.46 ± 0.61 & 52.96 ± 0.10 & 49.13 ± 0.19 & 48.02 ± 0.32 & 49.98 ± 0.60 & 52.85 ± 0.24 & 44.91 ± 0.52 \\
\bottomrule
\end{tabular}}
\end{sc}
\end{small}
\end{center}
\vskip -0.1in
\end{table}

%------------------------------------------------------------------------------------------------
\clearpage
\subsection{VGG19}
The results on VGG19 with CIFAR-100 exhibit a similar trend to those observed on CIFAR-10, reinforcing the effectiveness of our proposed approach. Once again, we identify the occurrence of layer collapse at extreme sparsities when no warm-up is applied, leading to a significant drop in accuracy. Introducing a single warm-up epoch effectively resolves this issue, restoring pruning performance across all evaluated criteria. However, increasing the warm-up phase to five epochs does not yield any additional advantage, indicating that a brief warm-up period is sufficient to stabilize gradient-based importance scores and prevent collapse.

\label{sec:vgg_cifar-100}

\begin{table}[h]
\caption{Performance of different compression methods evaluated using VGG19 on the CIFAR-100 dataset. We report the mean accuracy between three initialization seeds across various sparsity levels. Baseline, no pruning: $58.96 \pm 2.30$.}
\label{tab:VGG19_cifar100_compressors}
\vskip 0.15in
\begin{center}
\begin{small}
\begin{sc}
\resizebox{\textwidth}{!}{%
\begin{tabular}{lccccc|cccc}
\toprule
Sparsity & Random & Magnitude & GN & SNIP & GraSP & FD & FP & FTS & FBSS \\
\midrule
0.10  & 60.31 ± 0.40 & 59.13 ± 1.29 & 61.93 ± 0.48 & 61.98 ± 0.29 & 59.32 ± 0.63 & 62.13 ± 0.61 & 60.45 ± 3.47 & 61.56 ± 1.04 & 58.79 ± 0.98 \\
0.20  & 60.43 ± 1.14 & 59.27 ± 0.34 & 62.64 ± 0.21 & 62.68 ± 0.24 & 61.21 ± 0.41 & 63.04 ± 0.43 & 62.71 ± 1.02 & 62.24 ± 0.44 & 60.48 ± 0.48 \\
0.30  & 58.32 ± 0.60 & 59.35 ± 1.43 & 62.61 ± 0.23 & 63.11 ± 0.35 & 59.30 ± 0.43 & 62.85 ± 0.42 & 61.43 ± 0.61 & 62.65 ± 0.54 & 58.77 ± 1.02 \\
0.40  & 56.50 ± 3.20 & 60.04 ± 1.02 & 62.36 ± 0.02 & 62.39 ± 0.55 & 56.34 ± 1.49 & 62.38 ± 0.75 & 61.56 ± 1.25 & 62.67 ± 0.06 & 1.00 ± 0.00 \\
0.50  & 58.47 ± 1.49 & 61.49 ± 1.22 & 62.02 ± 0.64 & 62.76 ± 0.50 & 54.43 ± 0.84 & 62.84 ± 0.33 & 62.25 ± 0.33 & 62.47 ± 0.42 & 1.00 ± 0.00 \\
0.60  & 57.54 ± 0.74 & 61.50 ± 0.30 & 62.55 ± 0.13 & 63.08 ± 0.55 & 56.76 ± 0.69 & 62.40 ± 0.57 & 62.70 ± 0.63 & 62.17 ± 0.23 & 1.00 ± 0.00 \\
0.70  & 57.63 ± 0.80 & 61.71 ± 0.25 & 60.85 ± 0.79 & 60.58 ± 0.39 & 57.76 ± 0.84 & 60.44 ± 0.34 & 60.92 ± 0.41 & 60.51 ± 1.67 & 1.00 ± 0.00 \\
0.80  & 57.84 ± 0.57 & 61.89 ± 1.02 & 55.09 ± 0.49 & 59.84 ± 0.29 & 58.39 ± 0.74 & 1.00 ± 0.00 & 43.16 ± 1.02 & 58.66 ± 2.28 & 1.00 ± 0.00 \\
0.90  & 58.41 ± 0.41 & 62.60 ± 0.91 & 1.00 ± 0.00 & 8.35 ± 10.39 & 42.88 ± 1.64 & 1.00 ± 0.00 & 1.00 ± 0.00 & 8.87 ± 11.13 & 1.00 ± 0.00 \\
0.95  & 54.84 ± 1.08 & 1.00 ± 0.00 & 1.00 ± 0.00 & 1.00 ± 0.00 & 1.00 ± 0.00 & 1.00 ± 0.00 & 1.00 ± 0.00 & 1.00 ± 0.00 & 1.00 ± 0.00 \\
0.98  & 50.21 ± 0.72 & 1.00 ± 0.00 & 1.00 ± 0.00 & 1.00 ± 0.00 & 1.00 ± 0.00 & 1.00 ± 0.00 & 1.00 ± 0.00 & 1.00 ± 0.00 & 1.00 ± 0.00 \\
0.99  & 46.69 ± 0.45 & 1.00 ± 0.00 & 1.00 ± 0.00 & 1.00 ± 0.00 & 1.00 ± 0.00 & 1.00 ± 0.00 & 1.00 ± 0.00 & 1.00 ± 0.00 & 1.00 ± 0.00 \\
\bottomrule
\end{tabular}}
\end{sc}
\end{small}
\end{center}
\vskip -0.1in
\end{table}

%------------------------------------------------------------------------------------------------

\begin{table}[h]
\caption{Performance of different compression methods evaluated after 1 warmup epoch using VGG19 on the CIFAR-100 dataset. We report the mean accuracy between three initialization seeds across various sparsity levels. Baseline, no pruning: $58.96 \pm 2.30$.}
\label{tab:VGG19_cifar100_compressors_warmup1}
\vskip 0.15in
\begin{center}
\begin{small}
\begin{sc}
\resizebox{\textwidth}{!}{%
\begin{tabular}{lccccc|cccc}
\toprule
Sparsity & Random & Magnitude & GN & SNIP & GraSP & FD & FP & FTS & FBSS \\
\midrule
0.80  & 60.39 ± 1.16 & 58.91 ± 0.41 & 52.81 ± 1.32 & 55.62 ± 2.27 & 55.15 ± 2.25 & 56.71 ± 0.31 & 58.03 ± 0.93 & 52.41 ± 3.07 & 52.74 ± 5.16 \\
0.90  & 58.90 ± 0.98 & 60.95 ± 0.81 & 50.56 ± 4.59 & 55.89 ± 2.05 & 56.01 ± 1.58 & 52.07 ± 3.24 & 53.65 ± 0.57 & 52.45 ± 3.75 & 19.65 ± 1.68 \\
0.95  & 56.10 ± 0.85 & 57.64 ± 2.63 & 50.34 ± 1.00 & 53.70 ± 3.60 & 56.16 ± 0.41 & 54.44 ± 1.38 & 53.24 ± 3.54 & 53.56 ± 1.26 & 17.24 ± 0.44 \\
0.98  & 50.97 ± 0.40 & 54.66 ± 2.56 & 43.43 ± 5.32 & 50.19 ± 1.59 & 54.64 ± 1.50 & 42.75 ± 1.91 & 50.59 ± 3.39 & 48.56 ± 5.25 & 16.42 ± 0.64 \\
0.99  & 46.52 ± 0.45 & 43.33 ± 5.83 & 33.90 ± 5.35 & 42.65 ± 5.32 & 45.98 ± 4.48 & 29.67 ± 8.49 & 49.11 ± 3.46 & 48.70 ± 2.59 & 13.25 ± 0.84 \\
\bottomrule
\end{tabular}}
\end{sc}
\end{small}
\end{center}
\vskip -0.1in
\end{table}


%------------------------------------------------------------------------------------------------

\begin{table}[h]
\caption{Performance of different compression methods evaluated after 5 warmup epochs using VGG19 on the CIFAR-100 dataset. We report the mean accuracy between three initialization seeds across various sparsity levels. Baseline, no pruning: $58.96 \pm 2.30$.}
\label{tab:VGG19_cifar100_compressors_warmup5}
\vskip 0.15in
\begin{center}
\begin{small}
\begin{sc}
\resizebox{\textwidth}{!}{%
\begin{tabular}{lccccc|cccc}
\toprule
Sparsity & Random & Magnitude & GN & SNIP & GraSP & FD & FP & FTS & FBSS \\
\midrule
0.80  & 60.41 ± 1.39 & 58.38 ± 0.85 & 60.86 ± 0.79 & 61.63 ± 0.45 & 56.25 ± 0.49 & 59.59 ± 0.76 & 59.37 ± 3.50 & 60.86 ± 0.53 & 46.93 ± 9.04 \\
0.90  & 60.32 ± 0.09 & 57.74 ± 1.64 & 57.77 ± 2.41 & 58.23 ± 4.07 & 56.27 ± 1.02 & 60.19 ± 0.63 & 61.23 ± 0.50 & 60.52 ± 0.37 & 21.66 ± 1.95 \\
0.95 & 57.86 ± 0.53 & 59.55 ± 1.15 & 56.09 ± 0.97 & 58.83 ± 0.65 & 55.26 ± 1.25 & 55.80 ± 2.77 & 59.83 ± 0.94 & 58.52 ± 1.32 & 19.98 ± 2.62 \\
0.98 & 51.75 ± 0.43 & 47.75 ± 7.63 & 52.26 ± 4.06 & 55.27 ± 1.69 & 54.59 ± 0.96 & 49.46 ± 4.98 & 57.40 ± 1.26 & 56.00 ± 1.08 & 17.59 ± 1.36 \\
0.99 & 47.59 ± 0.80 & 42.46 ± 7.95 & 46.58 ± 2.00 & 53.13 ± 0.84 & 53.91 ± 1.53 & 42.87 ± 4.63 & 53.17 ± 1.18 & 53.05 ± 2.14 & 13.92 ± 0.14 \\
\bottomrule
\end{tabular}}
\end{sc}
\end{small}
\end{center}
\vskip -0.1in
\end{table}


%------------------------------------------------------------------------------------------------
\clearpage

\section{Mask Batch Size for Other Sparsities}
The Effect of batch size on pruning performance across different sparsities. 
As sparsity increases, the effect of batch size on pruning performance becomes more pronounced. 
At lower sparsities (0.90, 0.95), the differences across batch sizes are less evident, suggesting that even smaller batches provide a reasonable estimation of parameter importance. However, at extreme sparsities (0.98, 0.99), we observe a clear trend where larger batch sizes consistently lead to better parameter selection, ultimately improving accuracy. This aligns with our hypothesis that larger batches help reduce variance in gradient estimation, leading to more stable and effective pruning decisions. 
\label{batch_size_heatmaps}

\begin{figure}[h]
    \centering
    \includegraphics[width=0.8\linewidth]{imgs/cifar10_resnet18_heatmap_warmup_0.png}
    \caption{Effect of batch size on pruning performance at increasing sparsities.}
    \label{fig:enter-label}
\end{figure}

%------------------------------------------------------------------------------------------------

\clearpage
\section{Comparison of our criteria with magnitude-based pruning}

Figure \ref{fig:our_criterion_vs_magnitude} illustrates the relationship between parameter magnitude and different sensitivity-based pruning metrics. Each point represents a model parameter, with red points indicating the top-ranked parameters selected for retention by each criterion. The green dashed line marks the 99th percentile of parameter magnitudes.

A key observation is that the most effective pruning criteria, such as Fisher-Taylor Sensitivity, tend to retain parameters with a broad range of magnitudes, including many that are relatively small (left of the green line). This shows that the estimated importance does not always prioritize parameters based on their magnitude. 


\begin{figure}[htp]
    \centering
    \includegraphics[width=0.9\linewidth]{imgs/cifar_10_mag_vs_criteria_s_99.png}
    \caption{Our criteria vs. Magnitude parameter selection for 99\% sparsity (ResNet18, CIFAR-10, Seed 0)} 
    \label{fig:our_criterion_vs_magnitude}
\end{figure}

\end{document}


% This document was modified from the file originally made available by
% Pat Langley and Andrea Danyluk for ICML-2K. This version was created
% by Iain Murray in 2018, and modified by Alexandre Bouchard in
% 2019 and 2021 and by Csaba Szepesvari, Gang Niu and Sivan Sabato in 2022.
% Modified again in 2023 and 2024 by Sivan Sabato and Jonathan Scarlett.
% Previous contributors include Dan Roy, Lise Getoor and Tobias
% Scheffer, which was slightly modified from the 2010 version by
% Thorsten Joachims & Johannes Fuernkranz, slightly modified from the
% 2009 version by Kiri Wagstaff and Sam Roweis's 2008 version, which is
% slightly modified from Prasad Tadepalli's 2007 version which is a
% lightly changed version of the previous year's version by Andrew
% Moore, which was in turn edited from those of Kristian Kersting and
% Codrina Lauth. Alex Smola contributed to the algorithmic style files.
