In subsection~\ref{subsec:sml_rec}, we have shown how to obtain a global rSVD basis $\Phi_{\mathbf u}\in\mathbb{R}^{d_{\mathbf u}\times r_{\mathbf u}}$ for the velocity field on the reference geometry, combining CFD solutions from a database of patient geometries with registration. 
In this section, we address the reconstruction of the velocity field associated to a new patient geometry from a set of velocity observations acquired via 4D flow MRI on a lower resolution.
%
The data assimilation problem is solved using the Parametrized-Background Data-Weak (PBDW) method, in which, given the observations,
the velocity reconstruction is computed solving a modified least squares problem minimizing the distance of the reconstruction from a physics-informed
linear space and with an additional correction that accounts for the discrepancy with the available measurements.
The method was originally proposed in \cite{MPPY2015} and further analyzed and extended in \cite{cohen2022_nonlinearSpaces,gong2019pbdw}. 

We consider a physics-informed space defined by the global rSVD basis on the template, obtained from different shapes and CFD solutions. Additionally, we extend the approach 
proposed in~\cite{gong2019pbdw} for homogeneous noise to the case of heteroscedastic noise, to handle real applications which require assimilation techniques robust against real data.

% In Section \ref{sec:sml}, we have shown how to obtain a global rSVD basis for the velocity field on the reference geometry combining CFD solutions from a database of patient geometries. 
% In this Section, we address the reconstruction of the velocity field of a new patient geometry from a set of velocity observations acquired via 4DMRI on a lower resolution.

% We will employ PBDW~\cite{gong2019pbdw} with heteroscedastic noise for the reconstruction of the velocity field from noisy 4D flow MRI measurements. The method was originally proposed in \cite{MPPY2015} and further analyzed and extended in \cite{BCDDPW2017, maday-taddei-2019, cohen2022_nonlinearSpaces}. The extension for noisy measurements is studied here, as real applications call for assimilation techniques which are robust against real data.

% In short, PBDW tackles the reconstruction problem by means of a least-squares fit between the measurements and a linear reduced space, including an additional correction term in the space of observations. In the previous section, we have shown how to obtain a global rSVD basis $\Phi_{\mathbf u}\in\mathbb{R}^{d_{\mathbf u}\times r_{\mathbf u}}$ for the velocity field on the template geometry.

% Let us suppose, that we want to reconstruct the high-resolution velocity field of a new patient with corresponding test geometry from 4D flow MRI measurements of the velocity field. To do so, we execute the following steps:
% \begin{enumerate}
% \item We register the new geometry on the template through the registration map $\phi:[0,1]\times\mathbb{R}^3\rightarrow\mathbb{R}^3$, from definition~\ref{def:resnetlddmm}.
% \item We transport the velocity rSVD basis $\Phi_{\mathbf u}\in\mathbb{R}^{d_{\mathbf u}\times r_{\mathbf u}}$ from the template geometry to the test geometry via the registration map $\phi_1$ and interpolate it with RBF interpolation onto the dofs $\widehat{d}_{\mathbf u}$ of the test geometry, $\widehat{\Phi}_{\mathbf u}\in\mathbb{R}^{\widehat{d}_{\mathbf u}\times r_{\mathbf u}}$: informally we can write $\widehat{\Phi}_{\mathbf u}=(\phi_{\text{RBF}})^{\#}(\Phi_{\mathbf u})$.
% \item We apply PBDW with the basis $\widehat{\Phi}_{\mathbf u}$ after orthonormalization.
% \end{enumerate}
% All quantities in the test geometry will be represented with a hat above, e.g. $\widehat{\Phi}_{\mathbf u}$ for the transported velocity basis.

% \subsection{Velocity data assimilation from 4D flow MRI}
\newcommand{\pbdw}[1]{#1_{\rm PBDW}}
\subsection{Parametrized-Background Data-Weak with heteroscedastic noise}
Let us consider a patient geometry $\mathcal T$ and its computational mesh $\Omega_{\mathcal T}$.
%
We denote with $\phi:[0,1]\times\mathbb{R}^3\rightarrow\mathbb{R}^3$
the map to register the patient $\mathcal T$ on the reference shape. % $\Shat$ (i.e., $\phi_1(\mathcal T) = \Shat$). 
Given the velocity rSVD basis $\Phi_{\mathbf u}\in\mathbb{R}^{d_{\mathbf u}\times r_{\mathbf u}}$ on the reference shape, we use the registration map $\phi_1$ and RBF interpolation 
onto the finite element space on $\Omega_{\mathcal T}$  to compute the transported basis $\widehat{\Phi}_{\mathbf u}\in\mathbb{R}^{\widehat{d}_{\mathbf u}\times r_{\mathbf u}}$  on the new patient shape using the pushforward operator \eqref{eq:pushforward}. In particular, $\widehat{d}_{\mathbf u}$ denotes the corresponding number of velocity degrees of freedom of the velocity in the
computational domain $\Omega_{\mathcal T}$.

We assume to have available a set of velocity observations gathered from medical imaging, modelled as linear operators. Given a grid of voxels $\{Q_i\}_{i=1}^{M_{\text{voxels}}}$ s.t. $Q_i=\times_{i=1}^3 [a_i, b_i]$, $b_i>a_i,\ a_i,b_i\in\mathbb{R}_+$, with centers $\mathbf{c}^{\text{vox}}_i\in\mathbb{R}^3$ and vertices $\{\mathbf{x}_i^{\text{vox}}\}_{i=1}^8\subset\mathbb{R}^3$:
\begin{equation}
  \label{eq:voxel}
  l_i(\widehat{\mathbf{v}}) = \frac{1}{9}\left(\sum_{i=1}^{8}\widehat{\mathbf{v}}(\x^{\text{vox}}_i)+\widehat{\mathbf{v}}(\mathbf{c}^{\text{vox}}_i)\right)\approx \int_{Q_i}\widehat{\mathbf{v}}(\x)\ d\x,\qquad l_i:\mathbb{R}^{\widehat{d}_{\mathbf u}}\rightarrow\mathbb{R},
\end{equation}
with $\widehat{\mathbf{v}}:\Omega_{\mathcal{T}}\rightarrow\mathbb{R}^3$. Moreover, we introduce the divergence operator
\begin{equation}
  \label{eq:div}
  l_{\text{div}}(\widehat{\mathbf{v}})=\int_{\Omega_h}\text{div}(\widehat{\mathbf{v}}(\x))\ d\x,\qquad l_{\text{div}}:\mathbb{R}^{\widehat{d}_{\mathbf u}}\rightarrow\mathbb{R},
\end{equation}

\begin{rmk}
  The divergence operator \eqref{eq:div} can be evaluated exactly for velocity fields belonging to piecewise linear finite element space.
\end{rmk}

% Alternatively, the divergence constraint can be imposed exactly with the Piola transform~\cite{guibert2014group} acting on the registration map $\phi$. The divergence operator can be evaluated exactly as the velocity fields $\widehat{u}$ are functions in the polynomial P1 Lagrangian finite elements space of dimension $\widehat{d}_{\mathbf u}$. We can collect the Riesz representatives of $(\{l_i\}_{i=1}^{M_{\text{voxels}}}, l_{\text{div}})$ with respect to the discrete $\ell^2$ norm in a matrix $\mathcal{Z}_{\mathbf u}\in\mathbb{R}^{\widehat{d}_{\mathbf u}\times(M_{\text{voxels}}+1)}$ . We introduce the PBDW matrices
% \begin{align}
%   L=\mathcal{Z}_{\mathbf u}^T\widehat{\Phi}_{u},\quad L\in\mathbb{R}^{ M\times r_{\mathbf u}},\qquad K=\mathcal{Z}_{\mathbf u}^T\mathcal{Z}_{\mathbf u},\quad K\in\mathbb{R}^{ M\times  M},
% \end{align}
% with $M=M_{\text{voxels}}+1$, and define the true velocity field $\widehat{\mathbf{u}}^{\text{true}}\in\mathbb{R}^{\widehat{d}_{\mathbf u}}$ of which we know only the measurements
% \begin{equation}
%   l(\widehat{\mathbf{u}}^{\text{true}})=(\{l_i(\widehat{\mathbf{u}}^{\text{true}})\}_{i=1}^{M_{\text{voxels}}}, l_{\text{div}}(\widehat{\mathbf{u}}^{\text{true}}))=y,\qquad l:\mathbb{R}^{\widehat{d}_{\mathbf u}}\rightarrow\mathbb{R}^M,
% \end{equation}
% where each coordinate corresponds to the output of an observation operator $\{\{l_i\}_{i=1}^{M_{\text{voxels}}}, l_{\text{div}}\}$, with $l_i:\mathbb{R}^{\widehat{d}_{\mathbf u}}\rightarrow\mathbb{R}^3$ and $l_{\text{div}}:\mathbb{R}^{\widehat{d}_{\mathbf u}}\rightarrow\mathbb{R}$.

% In~\cite{HAIK2023115868, gong2019pbdw}, they propose a PBDW formulation with observations affected by homogeneous noise:
% \begin{equation}
%   \label{eq:pbdw_homo}
%   (z_{\text{PBDW}}, \eta_{\text{PBDW}}) = \argmin_{(z, \eta)\in\mathbb{R}^r\times \mathbb{R}^M}\xi\lVert\eta\rVert^2+\lVert Lz + K\eta-y\rVert^{2}_{S^{-1}},
% \end{equation}
% where $S\in\mathbb{R}^{M\times M}$ is the measurements diagonal covariance matrix, and where the parameter $\xi>0$ needs to be set from a validation dataset: $\xi=0$ and $S=\text{Id}$ corresponds to the standard PBDW formulation. In general, $\xi$ should be proportional to $\lVert y-l(\widehat{u}^{\text{true}})\rVert_2/\lVert\mathcal{Z}_{\mathbf u}^{\perp}\widehat{u}^{\text{true}}\rVert_2$. In contrast, we will use a heteroscedastic noise formulation. Namely, for measurements $y\approx Lz + K\eta+\epsilon_{y}$ with $\epsilon_y\sim\mathcal{N}(0, S)$, we estimate $u_{\text{PBDW}} = (z_{\text{PBDW}}, \eta_{\text{PBDW}})$ as follows:
%   \begin{equation}
%     \label{eq:pbdw_hetero}
%     (z_{\text{PBDW}}, \eta_{\text{PBDW}}) = \argmin_{(z, \eta)\in\mathbb{R}^r\times \mathbb{R}^M}\lVert\eta\rVert^2_{R^{-1}}+\lVert Lz + K\eta-y\rVert^{2}_{S^{-1}},
%   \end{equation}
% where $R\in\mathbb{R}^{M\times M}$ needs to be set, instead of $\xi\in\mathbb{R}$.

% \begin{theorem}[PBDW with heteroscedastic noise]
%   \label{theo:pbdw}
%   Problem \eqref{eq:pbdw_hetero} can be split into the two sub-problems:
%   \begin{equation}
%     z_{\text{PBDW}} = \argmin_{\mathbb{R}^{r_{\mathbf u}}} \lVert Lz-y\rVert^2_{S^{-1}W^{-1}},\qquad\eta_{\text{PBDW}} = \argmin_{\eta\in \mathbb{R}^M}\lVert\eta\rVert^2_{R^{-1}}+\lVert K\eta-y_{\text{err}}\rVert^2_{S^{-1}},
%   \end{equation}
%   with $W = (K + \text{Id})$ and $y_{\text{err}}=y-Lz_{\text{PBDW}}$, \textbf{assuming} that $R^{-1}=KS^{-1}$. %\fg{Since we dont consider it, we may say something about model bias (and thus model covariance), so that we avoid any comment on that matter from a not friendly reviewer}
% \end{theorem}

Let $M := M_{\text{voxels}}+1$ and let us denote with $\mathcal{Z}_{\mathbf u}\in\mathbb{R}^{\widehat{d}_{\mathbf u}\times M}$ the matrix whose columns are the Riesz representers of 
the operators $\{l_i\}_{i=1}^{M_{\text{voxels}}}$ and $ l_{\text{div}}$ with respect to the discrete $\ell^2$ norm.
Moreover, we will denote with 
$y \in \mathbb R^M$ the available set of measurements, and  with $\widehat{\mathbf u}^{\text{true}}\in\mathbb{R}^{\widehat{d}_{\mathbf u}}$ the \textit{true} velocity field, i.e., the unknown field
from which the available measurements are taken, i.e., such that
\begin{equation*}\label{eq:noisy_y}
y = \mathcal{Z}_{\mathbf u} \widehat{\mathbf u}^{\text{true}} + \epsilon_{y},\;\epsilon_y\sim\mathcal{N}(0, S),
%=(\{l_i(\widehat{\mathbf u}^{\text{true}})\}_{i=1}^{M_{\text{voxels}}}, l_{\text{div}}(\widehat{\mathbf u}^{\text{true}}))\,.
\end{equation*}
where $S\in\mathbb{R}^{M\times M}$ is the measurements noise covariance matrix.
 
\begin{rmk}
The operator \eqref{eq:div} will be used to impose incompressibility of the reconstructed velocity field as a fictitious measurement.
Alternatively, the divergence constraint can be imposed exactly with the Piola transform (see, e.g.~\cite{guibert2014group}) acting on the registration map $\phi$.
\end{rmk}

Let us now define the matrices
%
\begin{align*}
  L=\mathcal{Z}_{\mathbf u}^T\widehat{\Phi}_{u},\quad L\in\mathbb{R}^{ M\times r_{\mathbf u}},\qquad K=\mathcal{Z}_{\mathbf u}^T\mathcal{Z}_{\mathbf u},\quad K\in\mathbb{R}^{ M\times  M}\,.
\end{align*}

In the case of homogeneous noise, the PBDW approach proposed in~\cite{gong2019pbdw} seeks the reconstruction 
in the form of $\pbdw{\widehat{\mathbf u}} = \widehat{\Phi}_{u} \pbdw{z} +  \mathcal{Z}_{\mathbf u} \pbdw{\eta}$ solving
  \begin{equation*}
    \label{eq:pbdw_homo}
    (z_{\text{PBDW}}, \eta_{\text{PBDW}}) = \argmin_{(z, \eta)\in\mathbb{R}^{r_{\mathbf u}}\times \mathbb{R}^M}\xi^{-1} \lVert\eta\rVert^2+\lVert Lz + K\eta-y\rVert^{2}_{S^{-1}},
  \end{equation*}
where $S\in\mathbb{R}^{M\times M}$ is the measurements diagonal covariance matrix and $\xi >0$ is set from a validation dataset and proportional to 
$\lVert y-l(\widehat{\mathbf u}^{\text{true}})\rVert_2/\lVert\mathcal{Z}_{\mathbf u}^{\perp}\widehat{\mathbf u}^{\text{true}}\rVert_2$. The special choices
$\xi=0$ and $S=\text{Id}$ yield the original PBDW formulation.

\begin{problem}[PBDW with heteroscedastic noise] We consider the following extension: given $y \in \mathbb R^M$, find $\pbdw{\widehat{\mathbf u}} = \widehat{\Phi}_{u} \pbdw{z} +  \mathcal{Z}_{\mathbf u} \pbdw{\eta}$ such that 
  \begin{equation}
    \label{eq:pbdw_hetero}
    (z_{\text{PBDW}}, \eta_{\text{PBDW}}) = \argmin_{(z, \eta)\in\mathbb{R}^{r_{\mathbf u}}\times \mathbb{R}^M}\lVert\eta\rVert^2_{R^{-1}}+\lVert Lz + K\eta-y\rVert^{2}_{S^{-1}},
  \end{equation}
where a matrix $R\in\mathbb{R}^{M\times M}$, instead of a single parameter, needs to be set.
\end{problem}


\begin{theorem}[PBDW reconstruction]
  \label{theo:pbdw}
Let us assume that $R$ is chosen such as $R^{-1}=KS^{-1}$. Then the solution to problem \eqref{eq:pbdw_hetero} can be obtained solving the sub-problems:
  \begin{eqnarray}
    z_{\text{PBDW}} & = \argmin_{z\in\mathbb{R}^{r_{\mathbf u}}} \lVert Lz-y\rVert^2_{S^{-1}W^{-1}},\label{eq:pbdw_sub1}\\
    \eta_{\text{PBDW}} & = \argmin_{\eta\in \mathbb{R}^M}\lVert\eta\rVert^2_{R^{-1}}+\lVert K\eta-y_{\text{err}}\rVert^2_{S^{-1}}, \label{eq:pbdw_sub2}
  \end{eqnarray}
  where $W = (K + \text{Id})$ and $y_{\text{err}}=y-Lz_{\text{PBDW}}$.
\end{theorem}
\begin{proof}
  The proof is reported in the appendix~\ref{appendix:pbdw}.
\end{proof}
\begin{rmk}
 Our choice for $R^{-1}$ results in the choice of the prior distribution for $\eta$. According to the Gauss-Markov theorem, the solution of \eqref{eq:pbdw_sub1} is 
 $z_{\text{PBDW}}\sim\mathcal{N}(m_{z_{\text{PBDW}}}, \Sigma_{z_{\text{PBDW}}})$ with
 \begin{equation}
       \label{eq:1pbdw}
       m_{z_{\text{PBDW}}}=\underbrace{(L^T S^{-1}W^{-1} L)^{-1} L^T S^{-1}W^{-1}}_{:=H_{z_{\text{PBDW}}}} y,\;
       \Sigma_{z_{\text{PBDW}}}=(L^T S^{-1}W^{-1} L)^{-1}.
   \end{equation} 
       This results can be interpreted as follows. Let us assume that there exists a reconstruction $z_{\text{true}}$ on the reduced-order space that fits the measurements, i.e.,  
       $y\approx Lz_{\text{true}}+\epsilon_{z}$, up to a noise $\epsilon_{z}\sim\mathcal{N}(0, WS)$. Then, the estimate is unbiased, i.e.,
        $\mathbb{E}[z_{\text{PBDW}}]=z_{\text{true}}$, and it minimizes $\mathbb{E}[\lVert z-z_{\text{true}}\rVert^2_2]$ as well as the covariance $\mathbb{E}[(z-z_{\text{true}})\otimes(z-z_{\text{true}})]$.
 %
 
 Let $y_{\text{err}}=y-Lz_{\text{PBDW}}=(I-LH_{z_{\text{PBDW}}})y$. Then, the solution to the problem \eqref{eq:pbdw_sub2} 
 can be interpreted as an inverse problem in the Bayesian framework with resulting posterior distribution:
 $$
 \eta_{\text{PBDW}}|y_{\text{err}}\sim\mathcal{N}(m_{\eta_{\text{PBDW}}}, \Sigma_{\eta_{\text{PBDW}}}),
 $$ 
 where
 \begin{equation}\label{eq:2pbdw}
 \begin{aligned}
 m_{\eta_{\text{PBDW}}} & = \underbrace{(KS^{-1}K+KS^{-1})^{-1}KS^{-1}}_{:= H_{\eta_{\text{PBDW}}}}y_{\text{err}},\\
 \Sigma_{\eta_{\text{PBDW}}} & =(KS^{-1}K+KS^{-1})^{-1} = \left[R^{-1}\left(K + \text{Id}\right)\right]^{-1} = W^{-1} R.
 \end{aligned}
 \end{equation}
 Equation \eqref{eq:2pbdw} is equivalent to assuming that there exists a correction on the measurement space 
 $\eta_{\text{true}}$, such that $y_{\text{err}}\approx K\eta_{\text{true}}+\epsilon_{\eta}$, with $\epsilon_{\eta}\sim\mathcal{N}(0, S)$.
 
 Using \eqref{eq:1pbdw} and \eqref{eq:2pbdw}, one obtains that the solution to \eqref{eq:pbdw_hetero} is Gaussian distributed,
 $\pbdw{\widehat{\mathbf u}} = \widehat{\Phi}_{\mathbf u}z_{\text{PBDW}}+\mathcal{Z}_{\mathbf u}\eta_{\text{PBDW}}$,
 \[
 \pbdw{\widehat{\mathbf u}} \sim\mathcal{N}(m_{\pbdw{\widehat{\mathbf u}}}, \Sigma_{\pbdw{\widehat{\mathbf u}}})
 \]
 with 
   \begin{equation}\label{eq:cov}
   \begin{aligned}
       &m_{\widehat{\mathbf{u}}_{\text{PBDW}}} = \widehat{\Phi}_{\mathbf u}  m_{z_{\text{PBDW}}} + \mathcal{Z}_{\mathbf u} m_{\eta_{\text{PBDW}}} = 
 [\widehat{\Phi}_{\mathbf u}  H_{z_{\text{PBDW}}}+\mathcal{Z}_{\mathbf u} H_{\eta_{\text{PBDW}}}-\mathcal{Z}_{\mathbf u} H_{\eta_{\text{PBDW}}}LH_{z_{\text{PBDW}}}]y = H_{\widehat{\mathbf{u}}_{\text{PBDW}}}y  \\
     &\Sigma_{\widehat{\mathbf{u}}_{\text{PBDW}}} = H_{\widehat{\mathbf{u}}_{\text{PBDW}}}S H_{\widehat{\mathbf{u}}_{\text{PBDW}}}^T.
   \end{aligned}
   \end{equation}
\end{rmk}

The next result builds on the estimate of \cite{gong2019pbdw} and, taking into account additional sources of error coming from the registration step, 
provides an error estimate for the PBDW reconstruction.
\begin{theorem}[Error estimate for PBDW reconstruction with heterogeneous noise]
  \label{theo:pbdwmsq}
Let $P_X:\mathbb{R}^{\widehat{d}_{\mathbf u}}\rightarrow\mathbb{R}^{\widehat{d}_{\mathbf u}}$ denote the linear projections onto a linear subspace $X\subset\mathbb{R}^{\widehat{d}_{\mathbf u}}$. Let $H_{\widehat{\mathbf{u}}_{\text{PBDW}}}:\mathbb{R}^{\widehat{d}_{\mathbf u}}\rightarrow\mathbb{R}^{\widehat{d}_{\mathbf u}}$ be the matrix defined in \eqref{eq:cov}, and
$H_l := H_{\widehat{\mathbf{u}}^{\text{PBDW}}} \, \mathcal Z_{\mathbf u}^T$. %\circ l:\mathbb{R}^{\widehat{d}_{\mathbf u}}\rightarrow\mathbb{R}^{\widehat{d}_{\mathbf u}}$.
%
The following estimate holds: 
  \begin{linenomath}\begin{align*}
    \mathbb{E}[\lVert &\widehat{\mathbf{u}}^{\text{true}}-\widehat{\mathbf{u}}_{\text{PBDW}}\rVert_2]\leq &\\
    &\leq\text{tr}(H_{\widehat{\mathbf{u}}_{\text{PBDW}}}SH_{\widehat{\mathbf{u}}_{\text{PBDW}}}^T)^{\tfrac{1}{2}} &\text{(noise error)}\\
    &+\lVert(\text{Id}-H_l)\circ P_{\text{Im}(H_l)}\rVert_2\lVert \widehat{\mathbf{u}}^{\text{true}}\rVert_2&\text{(PBDW bias)}\\
    &+\lVert\text{Id}-H_l\rVert_2 \ \cdot \inf_{{\mathbf u}^{\text{best}}\in\text{col}(X_{\text{train}}^{\mathbf{u}})}\big(&\text{(PBDW stability constant)}\\
    &\lVert(\phi_{\text{RBF}})^{\#}({\mathbf u}^{\text{best}})-\widehat{\mathbf{u}}^{\text{true}}-P_{{\text{Im}(H_l)}}((\phi_{\text{RBF}})^{\#}({\mathbf u}^{\text{best}})-\widehat{\mathbf{u}}^{\text{true}})\rVert_2&\text{(manifold approximation error)}\\
    &+ C\cdot\lVert {\mathbf u}^{\text{best}}-P_{\text{Im}(\Phi_{\mathbf u})} {\mathbf u}^{\text{best}}\rVert_2&\text{(template rSVD approximation error)}\\
    &+\lVert P_{\text{Im}(H_l)}\left((\phi_{\text{RBF}})^{\#}({\mathbf u}^{\text{best}})\right)-(\phi_{\text{RBF}})^{\#}(P_{\text{Im}(\Phi_{\mathbf u})}{\mathbf u}^{\text{best}})\rVert_2\big)&\text{(registration degradation error)}
  \end{align*}\end{linenomath}
  where the matrix $X_{\text{train}}^{u}\in\mathbb{R}^{d_{\mathbf u}\times (n_{\text{train}}n_T)}$ contains, columnwise, the set of training snapshots registered on the reference shape.
  %  $\text{col}(X_{\text{train}}^{\mathbf{u}})$ is the set of registered training snapshots ordered column-wise in the matrix $X_{\text{train}}^{\mathbf{u}}\in\mathbb{R}^{d_{\mathbf u}\times (n_{\text{train}}n_T)}$ from equation~\eqref{eq:trainsnap}, , 
  
\end{theorem}
\begin{proof}
  The proof is reported in appendix~\ref{appendix:pbdw} along with an interpretation of the various sources of error.
\end{proof}

\newcommand{\snrho}{\text{SNR-ho}}
\newcommand{\snrhe}{\text{SNR-he}}

\subsection{Heteroscedastic noise model}
In the case of 4DMRI images, the observations often present velocity gradients whose accuracy degrades close to the vessel boundaries (see, e.g.~\cite{zingaro2024advancing, IRARRAZAVAL2019250}).
To account for this aspect, we consider a heteroscedastic noise model depending on three parameters:
the signal-to-noise homoscedastic ration ($\snrho$), the signal-to-noise heteroscedastic ratio ($\snrhe$), and the divergence observations operator's variance ($\sigma_{\text{div}}^2$).
We then subdivide the domain $\Omega_{\mathcal T}$ into a boundary layer $\Omega_{T, \text{he}}$, where measurements are affected by non-homogeneous variance noise, and 
an inner domain $\Omega_{\mathcal T, \text{ho}}$, whose measurements are characterized by standard homogeneous noise.


The covariance matrix $S\in\mathbb{R}^{(3M_{\text{voxels}}+1)\times (3M_{\text{voxels}}+1)}$ in equation~\eqref{eq:pbdw_hetero} is defined as a block matrix
\begin{equation*}
  S = \begin{pmatrix}
    S_{\text{obs}} & 0\\
    0 & \sigma_{\text{div}}^2
    \end{pmatrix},\end{equation*}
where the first block $S_{\text{obs}}\in\mathbb{R}^{3M_{\text{voxels}}\times 3M_{\text{voxels}}}$ is associated to observation operators $\{\{l_i\}_{i=1}^{M_{\text{voxels}}}\}$, and the last diagonal 
entry is associated to the divergence operator ($l_{\text{div}}$).
Let $M_{\text{ho}}$ and $M_{\text{he}}$ denote the amount of voxels whose centers belong to $\Omega_{\mathcal T, \text{ho}}$ and $\Omega_{\mathcal T, \text{he}}$, respectively.
Moreover, let us denote with $\mathbf{c}^{\text{vox}}_i$ the center of voxel $i$.
In our approach, the heteroscedastic noise is modeled as a spatially correlated multiplicative scalar to each velocity observation in the boundary layer, which
only affects the magnitude of the observed velocity vectors.

The covariance matrix $S_{\text{obs}}$ is then split into a homoscedastic ($S_{\text{ho}}$) and a heteroscedastic ($S_{\text{he}}$) block, 
associated to the observation operators in each subdomain
\begin{align*}
  S_{\text{obs}} = \begin{pmatrix}
    S_{\text{ho}} & 0\\
    0 & S_{\text{he}}
    \end{pmatrix},\quad
  S_{\text{ho}} := \left(\frac{\bar{\widehat{\mathbf{u}}}}{\text{SNR-ho}}\right)^2\text{Id}_{3M_{\text{ho}}},\quad
  S_{\text{he}} :=\left(\frac{\bar{\widehat{\mathbf{u}}}}{\text{SNR-he}}\right)^2PCP^T,\;
\end{align*}
where $\bar{\widehat{\mathbf{u}}}=\frac{1}{M_{\text{voxels}}}\sum_{i=1}^{M_{\text{voxels}}}l_i(\widehat{\mathbf{u}})$, $P\in\mathbb{R}^{3M_{\text{he}}\times M_{\text{he}}}$ is the operator 
projecting the velocity vector into its norm and $C\in\mathbb{R}^{M_{\text{he}}\times M_{\text{he}}}$ is the Gramian matrix of the radial basis function kernel
\begin{equation*}
  k(\mathbf{c}^{\text{vox}}_i, \mathbf{c}^{\text{vox}}_j) = \exp\left(-\frac{\lVert\mathbf{c}^{\text{vox}}_i-\mathbf{c}^{\text{vox}}_j\rVert^2_2}{2l_{\mathcal T}}\right)+\epsilon^2\delta_{ij},\qquad
  \forall (i,j) \mid \mathbf{c}^{\text{vox}}_i, \mathbf{c}^{\text{vox}}_j \in \Omega_{T, \text{he}},
\end{equation*}
with length scale $l_{\mathcal T}>0$ and additive homogeneous noise variance $\epsilon^2>0$. 

These parameters depend on the measurement procedure employed. In what follows, we set 
$l_{\mathcal T}=\text{diam}(\Omega_{\mathcal T})/12$ and $\epsilon^2=0.1$.

\subsection{Numerical results and comparison with GNNs}
We consider three levels of noise: $(\text{SNR-ho}, \text{SNR-he})\in \{(10, 0.5), (0.4, 0.1), (0.2, 0.05)\}$ and voxels of resolution $\SI{2e-3}{\meter^3}$.
The velocity observations are computed approximating the integral in equation~\eqref{eq:voxel} with the average of the values of the finite element function on the voxel centers and on the vertices.
Figure \ref{fig:noise_pbdw} shows the resulting observations for the test geometry $n=12$
(the closest to the training velocity solution manifold with respect to the Grassmann metric on the velocity fields, see figure~\ref{fig:cluster_v}),
at a selected time instant for the different noise intensities, together with the corresponding PBDW reconstruction.
%
\begin{figure}[!htp]
  \centering
  \includegraphics[width=0.8\textwidth]{img/noise_fields.pdf}
  \caption{Data assimilation of the velocity field at time $0.1$s from noisy velocity measurements for the test geometry $n=12$ (see figure~\ref{fig:cluster_v}). 
\textbf{Left.} High-fidelity velocity field from the CFD simulations on the corresponding domain. 
\textbf{Right.} Observations with the three considered noise intensities (top) and PBDW predictions with fixed number of template velocity modes $r_{\mathbf u} = 2000$ (bottom).}
  \label{fig:noise_pbdw}
\end{figure}

We compare the reconstructed velocity field with those obtained with the EPD-GNN surrogate models from the inference problem \textit{gnn-gv} (geometry $\mapsto$ velocity) introduced in Section~\ref{ssec:pres-gnn}, which delivers a velocity prediction solely from the geometry data.
The results are shown in figure~\ref{fig:pbdw_vs_gnn_v}, depicting the $L^2$ average relative error $\epsilon_{\mathbf u}$ on the test dataset of $52$ geometries, using PBDW and EPD-GNNs.
The errors are evaluated on the target geometry, after transporting the predicted velocity fields with the registration map in the case of EPD-GNNs.
\begin{figure}[!ht]
  \centering
  \includegraphics[width=0.9\textwidth]{img/pbdw_0.pdf}\\
  \includegraphics[width=0.9\textwidth]{img/pbdw_1.pdf}\\
  \includegraphics[width=0.9\textwidth]{img/pbdw_2.pdf}\\
  \caption{Average $L^2$-relative error $\epsilon_{\mathbf u}$ of the velocity fields evaluated on the $52$ target-test geometries: the accuracy of data assimilation with PBDW from velocity observations is compared with respect to direct inference with EPD-GNNs. The rSVD reconstruction error (\textit{rec}) and the error of the noise-free observations (\textit{obs noise-free}) are also shown. \textbf{Top: } $(\text{SNR-ho}, \text{SNR-he})=(10, 0.5)$. \textbf{Middle: } $(\text{SNR-ho}, \text{SNR-he})=(0.4, 0.1)$. \textbf{Bottom: } $(\text{SNR-ho}, \text{SNR-he})=(0.2, 0.05)$.}
  \label{fig:pbdw_vs_gnn_v}
\end{figure}

Notice that the 4DMRI data are not used in the inference problem with EPD-GNNs. The comparison with PBDW is shown to underline, in the case of limited data, the difference in accuracy between a purely data-driven inference problem, such as the EPD-GNN,  and a physics-based data assimilation method that incorporates a state space
constructed using geometrical and physical information, as well as an additional set of observations.

Figure~\ref{fig:pbdw_vs_gnn_v} shows that increasing the level of noise affects the stability properties of the rSVD basis used in PBDW. For the lowest noise
$(\text{SNR-ho}, \text{SNR-he})=(10, 0.5)$  the best results are obtained with $r_{\mathbf u} = 2000$, while $r_{\mathbf u} = 500$ is the best performing case for $(\text{SNR-ho}, \text{SNR-he})=(0.2, 0.05)$.
%
In the ideal, noise-free case, the optimality properties of the PBDW guarantee that the reconstruction error is lower than the sole rSVD approximation error 
for specific $r<M$. For the case with the lowest noise $(\text{SNR-ho}, \text{SNR-he})=(10, 0.5)$, it can be observed that the PBDW reconstruction error
is lower than the rSVD approximation error for $r_{\mathbf u}=500$, but higher for $r_{\mathbf u}=2000$.

From the high-resolution reconstructed velocity field, obtained through PBDW or EPD-GNNs, clinically relevant biomarkers such as the time-averaged wall shear stress (TWSS)
\begin{align*}
  \tau_{\text{wss}}(\mathbf{x}, t)&=\mu\frac{\partial}{\partial\mathbf{n}(\mathbf{x}, t)}\left(\mathbf{u}(\mathbf{x}, t)-(\mathbf{u}(\mathbf{x}, t)\cdot\mathbf{n}(\mathbf{x}, t))\mathbf{n}(\mathbf{x}, t)\right),\\
  \tau_{\text{twss}}(\mathbf{x})&=\int_{t=0.05s}^{t=0.225s}\tau_{\text{wss}}(\mathbf{x}, t)\,dt\approx\frac{1}{8}\sum_{i=0}^{7}\tau_{\text{wss}}(\mathbf{x}, 0.05+i\cdot\Delta t),
\end{align*}
and the oscillatory shear index OSI$_I$ relative to the time interval $I=[0.05s,0.225s]$
\[
  OSI_{I}(\mathbf{x})=\frac{1}{2}\left(1-\frac{\left.|\int_{t=0.05s}^{t=0.225s}\tau_{\text{wss}}(\mathbf{x}, t)\,dt\right|}{\int_{t=0.05s}^{t=0.225s}|\tau_{\text{wss}}(\mathbf{x}, t)|\,dt}\right)\approx\frac{1}{2}\left(1-\frac{\left.|\frac{1}{8}\sum_{i=0}^{7}\tau_{\text{wss}}(\mathbf{x},  0.05+i\cdot\Delta t)\right|}{\frac{1}{8}\sum_{i=0}^{7}|\tau_{\text{wss}}(\mathbf{x},  0.05+i\cdot\Delta t)|}\right)
\]
can be computed.
The $L^2$-relative error for the test geometry $n=12$ are shown in figure~\ref{fig:twss_12} for different noise intensities,
while the results of the quantitative study on all the $52$ test geometries are presented in figure~\ref{fig:twss_osi}. 
PBDW achieves a satisfactory accuracy in all considered cases, while the prediction with EPD-GNNs fails, confirming that the model is not able to capture the
high geometric variability. As previously noticed, the EPD-GNN only relies on the geometry data. 
The purpose of the comparison with PBWD is to underline the problematics of the EPD-GNN approach in this clinical context, 
in order to address them in future studies with a higher computational budget and a higher amount of data.
%
\begin{figure}[!htp]
  \centering
  \includegraphics[width=0.85\textwidth]{img/wss_new.pdf}
  \caption{Time-averaged wall shear stress (TWSS, \textbf{top}) and oscillatory index (OSI, \textbf{bottom}) for test geometry $n=12$, figure~\ref{fig:cluster_v}. The results are shown for different noise levels $(\text{SNR-ho}, \text{SNR-he})\in \{(10, 0.5), (0.4, 0.1), (0.2, 0.05)\}$ and fixed rSVD rank $r_{\mathbf u}=2000$.}
  \label{fig:twss_12}
\end{figure}
%
\begin{figure}[!htp]
  \centering
  \includegraphics[width=0.8\textwidth]{img/twss.pdf}
  \caption{Average $L^2$-relative error of the time averaged wall shear stress (TWSS, top) and of the oscillatory index (OSI, bottom) on the $52$ test geometries,
calculated from the voxel observations (\textit{obs}), the PBDW reconstruction (\textit{pbdw}) and the EPD-GNN reconstruction (\textit{gnn-gv}).
  The results are shown for different noise levels $(\text{SNR-ho}, \text{SNR-he})\in \{(10, 0.5), (0.4, 0.1), (0.2, 0.05)\}$ and rSVD ranks $r_{\mathbf u}\in\{500, 1000, 2000\}$. The $25\%$ and $75\%$ percentile are shown with shaded regions.}
  \label{fig:twss_osi}
\end{figure}



