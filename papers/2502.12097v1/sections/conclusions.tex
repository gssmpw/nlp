In this work, we propose a robust shape registration method for complex geometries, such as aortic shapes, and its application to reduced-order modeling, data assimilation problems, and the training of EPD-GNNs for the inference of velocity and pressure fields from available observations.

The registration is based on a multigrid ResNet-LDDMM approach, trained using a shape database created with Statistical Shape Modeling (SSM) from an initial cohort of healthy subjects and aortic coarctation patients. The optimization is based on a modified Chamfer distance, tailored to computational meshes. By refining the surface mesh over the epochs, we show that realistic computational meshes can be efficiently handled.

The registration allows the definition of a bijective and non-parametric mapping between shapes, independent of mesh connectivity.
This enables the development of projection-based ROMs on different geometries after the pushforward of the SVD modes from a reference shape.
Our results showed that this might be challenging in practice due to the high number of velocity and pressure modes needed to achieve satisfactory accuracy.
However, in other applications with a larger amount of data or a solution manifold characterized by a lower intrinsic dimensionality, registration could be effectively used to develop physics-based surrogates, as in~\cite{guibert2014group}. 
We studied the correlation between geometric encodings and velocity/pressure fields on a common reference geometry,
emphasizing the impact of solution manifold learning metrics (Hausdorff vs. Grassmann). 

The employment of registration leads to a substantial improvement in the training of EPD-GNNs for the inference of pressure and velocity fields from geometry encoding, as well as for the inference of the pressure field from velocity data.
%
Next, we studied the application of the global rSVD basis in the context of the Parametrized-Background Data-Weak (PBDW) method.
Focusing on measurement data mimicking 4D MRI measurements, we proposed a natural extension of the PBDW formulation to heteroscedastic noise models, accounting for higher uncertainties in velocity gradients near vessel walls.
%
The results showed that physics-driven methods, such as PBDW, yield better results for velocity reconstruction than purely data-driven methods, such as EPD-GNN, which infers the velocity field solely from the geometry encoding.
%
Regarding the reconstruction of the pressure field and pressure drops from velocity data, the results of EPD-GNN—whether from coarse observations or in combination with the PBDW reconstruction of a high-resolution velocity field—outperformed widely used approaches such as the Pressure-Poisson estimator (PPE) and the Stokes estimator (STE).
%
Finally, we explored the use of the PPE and STE, combined with projection-based reduced-order modeling, for forward uncertainty quantification.

For relevance in clinical applications and image-based diagnostic, surrogates models and data assimilation methods shall be able to predict biomarkers of interest accurately and efficiently, 
%the main objective is to increase the accuracy and efficiency of the models employed in the prediction of 
validating the results with medical data. Towards this direction, in the specific context of estimating pressure drops in patients affected by aortic coarctation, 
the high variability of the three-dimensional fluid dynamics requires a higher amount of data. Reducing the surrogate models to one-dimensional centerline supports instead of predicting 
full three-dimensional fields might yield a more efficient and sufficiently accurate approach. 
However, although three-dimensional models are not more computationally expensive and with a higher intrinsic dimensionality, these are also more interpretable and have the possibility to evaluate
additional quantity of interests such as secondary flow degree, normalized flow displacement, as well as wall shear stresses.

Our pipeline could be extended to other anatomical shapes (e.g., heart or liver), as well as beyond biomedical applications (e.g., for shape optimization in general context). 
From the point of view of surrogate modelling, different approaches to tackle the shape variability could be compared considering the reconstruction errors on the template geometry, thus measuring the complexity of the different solution manifolds. 
%
In this way, approaches that are effective on test cases that require few SVD modes to be well-approximated, but struggle on more difficult ones, could be identified.

Future directions of research include greedy data augmentation strategies to efficiently increase the training dataset and design better solution manifold approximants, the implementation of nonlinear shape generative models from machine learning instead of SSM (which is a linear method based on PCA), and the optimization of EPD-GNNs architectures for large-scale applications, with a focus on memory efficiency.