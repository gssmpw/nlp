Computational fluid dynamics (CFD) of the cardiovascular system can provide valuable tools for analyzing complex biological phenomena in the absence or scarcity of medical data,  supporting the image-based diagnosis and diseases staging, as well as the treatment selection and postoperative monitoring (see, e.g.~\cite{quarteroni24_cardio, lesage2023mapping,Morris18}). 
%
While computational models are becoming increasingly relevant in clinical practice, computational costs remain a significant bottleneck. In specific cases, such as the onset of turbulent flows in large vessels or the simulation of complex flows in realistic geometries, CFD simulations demand substantial computational resources.

Physics-based reduced-order models (ROMs) (see, e.g.~\cite{benner_model_2017, hesthaven2016certified, rozza2022advanced}) and data-driven models obtained from machine learning have proven to be able to reduce the computational costs in different applications. However, especially in the context of hemodynamics, these models are often highly patient-specific and cannot be efficiently extrapolated to computational domains of new patients without requiring a substantial amount of new training simulation data, thereby hindering the development of effective predictive models: they may be employed to monitor post-surgery biomarkers, but generally they could not be extended to inter-patient studies. Often the shape variability is introduced locally via small parametrized deformations, and on a fixed computational domain~\cite{Ballarin2016}. Projection-based ROMs are applied for CFD simulations with parametric inflows or optimal control problems~\cite{Girfoglio2022, Zainib2019} on fixed geometries, and they cannot be extrapolated to different patients without recomputing all the training snapshots. Purely data-driven surrogates are also often designed on single geometries~\cite{Fresca2020} and are affected by the same limitations. All these methodologies could benefit from the registration techniques we are going to introduce.

In this work, we present a non-parametric shape registration approach to handle the geometric variability in the case of realistic aortic coarctation and
use it to design a data assimilation pipeline to efficiently predict the three-dimensional patient-specific velocity and pressure fields from velocity measurements or from a patient-specific geometrical encoding
and additional boundary conditions.

The method is composed of an \textit{offline stage}, in which a database of training solutions is prepared and preprocessed on a template geometry, and of an \textit{online stage} in which 
velocity and pressure fields and related quantities of interest (e.g. time-averaged wall shear stress, oscillatory shear index, pressure drops) are inferred on a new patient. 
An overview of the data assimilation processes presented in this paper is shown in Figure~\ref{fig:scheme}. 
%

% offline
\paragraph*{Offline stage} In the offline stage, the first step is the design of a Statistical Shape Model (SSM) based on a centerline encoding of aorta geometries directly acquired from healthy subjects and aortic coarctation patients, which is used to generate additional synthetic geometries representing realistic anatomical features. An initial database containing 776 
geometries -- including both healthy and stenotic aortas -- is employed for generating velocity and pressure fields solving the three-dimensional
incompressible Navier--Stokes equations with 3-elements (lumped-parameter) Windkessel models on the main outlet branches, and with a variational multiscale turbulence model \cite{bazilevs2007variational}. The individual simulations are set up using patient-specific boundary conditions (b.c.),
based on measured inlet flow rates and tuning the parameters of the Windkessel models to control available or estimated flow rates.
%
Next, we use a registration algorithm to map the point clouds of the available patient geometries onto a reference shape, chosen within the cohort.
The registration algorithm is based on a large deformation diffeomorphic metric mapping (LDDMM)~\cite{trouve1998diffeomorphisms, trouve2005local}, which defines
the map between source and target points as the flow of a differential equation, whose vector field is parametrized by a residual neural network (ResNet)~\cite{amor2022resnet}.
%
We propose a cost functional tailored to the case of aortic meshes, taking into account additional information on the vessel centerline, the common topology of the shapes (inlet, outlets, and vessel wall),
and the normals to the boundary faces. 
%
As reported in \cite{pajaziti2023shape}, handling large meshes, as those required in the simulation of blood flows, leads to prohibitive computational costs for the training of the neural networks.
To overcome this problem we employ a multigrid optimization strategy, gradually
increasing the mesh size over the epochs. 
%
Crucial for the computational efficiency of our methodology is also the employment of GPUs with \texttt{pytorch3d}~\cite{ravi2020pytorch3d}. 
%
The multigrid ResNet-LDDMM registration is used to pullback the database of blood flow solutions onto the same reference shape. These data can be used for the solution manifold learning, to enable efficient linear dimension reduction methods such as randomized singular value decomposition (rSVD), and to define a geometrical encoding across shapes.
This implicit geometry encoding is used to accelerate the training of different Encode-Process-Decode GNNs~\cite{pfaff2020learning} that infer velocity and pressure solely from the geometry, or the pressure from the velocity data.

% online
\paragraph*{Online stage} In the online stage, we assume to have available a new patient shape, possibly also with related velocity measurements. 
The geometry is first mapped on the reference shape via the ResNet-LDDMM registration, using the computed map to transport the global rSVD basis on the new domain: this enables the implementation of efficient data assimilation techniques to infer velocity, pressure, and related quantities of interest.
%
First, we focus on the Parametrized-Background Data-Weak (PBDW) method to reconstruct the high-resolution velocity field from partial observations minimizing the distance from the physics-informed space defined by the global rSVD basis. The PBDW, originally proposed in~\cite{MPPY2015}, has been recently used in different contexts to tackle data assimilation problems in hemodynamics \cite{galarce2022state} as well as to handle different noise models~\cite{gong2019pbdw}. We consider a generalized PBDW formulation extending the approach of~\cite{gong2019pbdw} to the case of heteroscedastic noise, to account for measurement data whose quality degrades close to the vessels' boundaries. 
%
Next, we investigate the estimation of pressure fields and pressure drops from velocity observations or solely from the geometry and b.c., and compare different approaches based on registration: EPD-GNNs against pressure-Poisson equation and Stokes pressure estimators (see, e.g.~\cite{bertoglio2018relative}). Using the global rSVD basis, the uncertainty quantification of these methodologies is also carried out.\newline

A classical LDDMM registration method to handle shape variability in model-order reduction for hemodynamics was first proposed in~\cite{guibert2014group} in the context of pulmonary blood flow, where a dataset of $17$ patients has been registered to a common template and then used to design a time-dependent ROM with a proper orthogonal decomposition.
The approach, however, was restricted to computational meshes with the same topology and validated only with limited variability of boundary conditions.
%
In~\cite{pajaziti2023shape} a dataset of $2800$ truncated healthy aortic arches has been generated with SSM and mapped with parametric non-rigid deformations, which, 
unlike LDDMM, do not guarantee the bijectivity of the registration maps. CFD simulations, limited to the stationary case, have been considered, and a shallow neural network was trained as a surrogate model in the space of proper orthogonal decomposition coordinates, in order to predict the velocity and pressure fields from the encoded description of the geometry in a shape vector. 
%
A proof-of-concept pipeline to handle shape variability was recently proposed in~\cite{galarce2022state}, employing PBDW and registration. The approach involves parametric deformations of cylinders and a registration based on LDDMM, while a nearest neighbor criteria is used to select at the online stage the proximal ROMs with a Hausdorff-like metric for the data assimilation.
%
A computational framework to address shape variability in hemodynamics was recently proposed in~\cite{Tenderini2024}, where registration was used to train
deep neural operators from a geometrical encoding obtained with conditioned neural ODEs, in a LDDMM fashion. The dataset was generated with the Radial Basis Function interpolation (RBF) of small deformations from a dataset of $20$ healthy patient-specific aortas. 
The procedure, tested on $358$ training and $39$ testing healthy geometries, was limited to the prediction of velocity field at systolic peak and considered limited variability of boundary conditions.
%
The use of machine learning methods for estimating the pressure field from reduced geometric representations of the hemodynamics has been discussed, among others, in~\cite{pegolotti2024learning,iacovelli2023novel, yevtushenko2021deep, versnjak2024deep},
focusing on the inference of pressure from one-dimensional vessel centerline information, and in~\cite{10.1115/1.4055285}, considering two-dimensional representations.
A deep learning frameworks for estimation of pressure for three-dimensional hemodynamics has been recently 
proposed in~\cite{nannini2025learninghemodynamicscalarfields}, without employing registration. An alternative approach, employing domain decomposable projection-based ROMs, has been proposed in~\cite{PEGOLOTTI2021113762}.

We focus on a realistic three-dimensional patient-specific representation of the computational domain as a general framework of interest in other research fields,
since some clinically relevant biomarkers are inherently two- or three-dimensional, for example the wall shear stress or the full description of the velocity field in proximity of a stenosis. Moreover, the evaluation of biomarkers from 3d fields is more interpretable than the use of surrogate models based on reduced 1d representations: non-physiological results can be assessed with more confidence looking at the 3d fields.

\begin{figure}[!htp]
  \centering
  \includegraphics[width=1\textwidth]{img/scheme.pdf}
  \caption{Overview of the registration-based data assimilation process.}
  \label{fig:scheme}
\end{figure}

The main contributions of this work can be summarized as follows. 
%
Firstly, we propose a registration algorithm tailored to the case of aortic coarctation meshes including the main branches, that employs a multigrid optimization strategy, and that can handle realistic computational meshes.
%
Secondly, we investigate in detail the approximation properties and the potential of reduced rSVD bases created across different patient shapes for realistic meshes, time-dependent, three-dimensional hemodynamics, and patient-specific boundary conditions. 
%
Thirdly, we propose an EPD-GNN where the registration is used as a pre-processing step to speed up the training, in order to infer the velocity and pressure fields directly from the geometrical encoding, as well as the pressure field from velocity data.
%
Finally, we use the PBDW method built on the global rSVD basis and extended to the case of heteroscedastic noise, to account for 
the higher uncertainty of 4DMRI images close to the vessel boundaries.
%
We then validate and compare the different algorithms for data assimilation to reconstruct velocity and pressure solutions, as well as related biomarkers, from coarse velocity observations 
mimicking 4DMRI data. In particular, we investigate a pressure estimator that combines PBDW velocity reconstruction with EPD-GNN inference to reconstruct pressure quantities of interest
from coarse-grained and noisy velocity data, validating it against state--of--the art pressure estimators.
Finally, we assess the benefit of employing registration as a pre-processing step to speed up the training of GNNs and compare their prediction accuracy with PBDW for the inference of the velocity field and with PPE and STE for the inference of the pressure field.


The results show that (i) the physical variability of the problem is more complex than what has been shown in the literature, (ii) the intrinsic dimensionality of the solution manifold is high, more than $500-1000$ rSVD modes are used to estimate, with a relative accuracy under $10\%$, the pressure and velocity fields, (iii) while the employment of linear ROMs is possibly not feasible, EPD-GNNs have good potential but require more data and a higher computational budget with respect to that available for our studies, (iv) direct inference from the geometry is a harder problem than exploiting velocity observations, (v) standard pressure estimators are outperformed even in our limited data regime.

% overview
The rest of the paper is structured as follows. Section~\ref{sec:setting} introduces the key elements of the computational framework for aortic shape modelling and blood flow simulations. The registration algorithm is described in section~\ref{sec:registration}, while the detailed validation and employment of registration for solution manifold learning is presented in section~\ref{sec:sml}. In section~\ref{ssec:pres-gnn}, we present the hyperparameter studies for the EPD-GNNs architecture and how we employ the registered datasets in this context. Sections~\ref{sec:da} and~\ref{sec:prec} focus on the estimation of velocity and pressure fields from medical imaging data. Finally, in section~\ref{sec:discussions} we discuss the results and limitations of the approach and in section~\ref{sec:conclusions} we draw the conclusions and present possible future directions of research.
