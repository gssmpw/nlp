\paragraph{Increasing the accuracy}
The correlations between velocity and pressure, and the comparison of the PPE and STE with GNNs (\textit{gnn-vp-pbdw}) suggest that GNNs can be used to infer the pressure from the velocity field in presence of a sufficiently high amount of data. Employing 4DMRI to acquire the velocity field observations should reduce the source of errors due to intrinsic uncertainties on the geometry and 
boundary condition acquisition and also on the choice of physical model and related approximations (i.e. rigid walls, turbulence model). The inference of the pressure and the velocity from the geometrical encoding (\textit{gnn-gp}, \textit{gnn-gv}) represents a more difficult task that may require a substantial increment of the available training dataset. Local linear and nonlinear dimension reduction techniques for solution manifold learning should be employed.

\paragraph{Limitations of the model} Apart from physical modelling assumptions, such as the neglecting fluid-structure interaction, the modelling of the aortic valve, and external tissue support on the vessel walls, we considered only parabolic inflow and a Windkessel model tuned according to an estimated flow split. These last two approximations, could be removed enriching the training dataset at the price of an additional computational cost as the intrinsic dimensionality of the solution manifold would also increase. Arbitrarily increasing the accuracy of the physical model does not necessarily results in more accurate predictions in a data assimilation context due to the epistemic uncertainties associated to the boundary conditions, to the physics, and to the geometry definition.

\paragraph{GNNs memory-bound} We employed coarse meshes for GNNs instead of the full-mesh due to memory constraints: GNNs built on graphs that correspond to meshes of large-scale applications have a high memory footprint limiting the size of the NN models. This is an active field of research: possible solutions include \textit{multigrid} strategies, distributed training, mini-batches subsamplers and gradient checkpointing. We remark that despite our limited computational budget, the low accuracy on the test dataset depends on the high geometric variability rather than on the employment of coarse meshes, as can be deduced from the high overfitting error and high training error on the coarse mesh in figures~\ref{fig:overfitting} and ~\ref{fig:overfitting_edges}.
