\subsection{Statistical shape modelling of patients with aortic coarctation}
\label{subsec:ssm}

The data used in this study were obtained from a cohort of patients with coarctation of the aorta (CoA), augmented synthetically using statistical shape models (SSM). The procedure is briefly outlined below. For the detailed methodology, we refer the reader, e.g. to 
~\cite{goubergrits2022ct, thamsen2021synthetic,thamsen2020unsupervised,versnjak2024deep, yevtushenko2021deep}.

The initial database contained $228$ surfaces acquired from 3D steady-state free-precession (SSFP) magnetic resonance imaging (MRI)
(acquired resolution $\SI{2}{mm}\times \SI{2}{mm}\times \SI{4}{mm}$, reconstructed resolution used for surface reconstruction $\SI{1}{mm}\times \SI{1}{mm}\times \SI{2}{mm}$) and segmented with \texttt{ZIB Amira}~\cite{stalling2005amira}. In total, 106 CoA patients (32 female) and 85 healthy subjects were acquired (25 female). 
For 37 (8 female) of the 106 CoA patients also post-treatment image data were available, thus increasing the database. The median age was 21 years with interquartile range (IQR) of 32 years.
The considered region of interest comprises the vessel surface of aortic arch up to the thoracic aorta (TA), including three main branches 
and the corresponding boundary surfaces (brachiocephalic artery, BCA, left common carotid artery, LCCA, left subclavian artery, LSA). Few available cases with two or four branches of the aortic arch were not included into the database.

Additionally, pointwise linear centerlines for the aorta and the three branching vessels have been obtained along with the radii of the inscribed spheres using the vascular modelling toolkit \texttt{VMTK}~\cite{antiga2008image} (see the sketch in figure~\ref{fig:clustergeometries} (left) for an example).
%
\begin{figure}[!htp]
  \centering
  \includegraphics[width=0.9\textwidth]{img/cluster_g.pdf}
  \caption{
  \textbf{Left:} Sketch of the centerline encoding (points $p_i$ and radius $r_i$ of the associated inscribed sphere, for $i\in\{1,\dots,390\}$). 
  \textbf{Center:} Clustering of the considered training ($n=724$) and test ($n=52$) shapes using t-SNE
  with the Euclidean distance on a geometrical encoding, based on the distance of each point from the centerline after shape registration (see section~\ref{subsec:sml_correlations} for details).
  \textbf{Right:} Visualization of the furthest shapes (top, test cases $34$, $50$, and $44$) and the closest ones (bottom, test cases $3$, $21$, $15$) according to the metric in the center plot.}
  \label{fig:clustergeometries}
\end{figure}

The procedure resulted in $300$ centerline points for the aorta and $30$ centerline points for each branching vessel, for a total of $n_{\text{cntrl}}=390$ points and corresponding radii of inscribed spheres for each considered shape.
These data allow to encode the morphology of each shape into a matrix $S_{\text{SSM}}\in\mathbb{R}^{n_{\text{cntrl}}\times (3+1)}$ containing the spatial coordinates of the $n_{\text{cntrl}}$ centerline points and the associated radii. Closed triangulated surfaces are then generated from this skeletal representation~\cite{yevtushenko2021deep}. 
Each geometry is rigidly moved towards the mean shape $S_{\rm mean}$, %$\bar{\Tilde{S}}\in\mathbb{R}^{n_{\text{cntrl}}\times (3+1)}$, 
minimizing the least-squares distance between points with the closest point algorithm using \textsf{mcAlignPoints} package of the \texttt{ZIB Amira} software.
%
No scaling is performed. New shapes 
are generated through SSM with Principal Component Analysis (PCA):
\begin{equation*}
  \Tilde{S}_{\text{SSM}} = S_{\rm mean}  + P_{\text{SSM}}\ b_{\text{SSM}}, 
\end{equation*}
where $P_{\text{SSM}}\in\mathbb{R}^{(n_{\text{cntrl}}\times (3+1))\times k}$ contains $k>0$ truncated modes of the correlation matrix of the training shapes used for SSM, and 
$b_{\text{SSM}}\in\mathbb{R}^k$ is the vector of coefficients. For the SSM development only pre-treatment CoA shapes ($93$ cases) and healthy aorta ($65$ cases) were used.


A database of more than $10000$ shapes is generated sampling $b_{\text{SSM}}$ from a normal distribution. Unrealistic shapes, e.g., containing self-intersection, small vessel radius (below $1.0$ mm), or 
excessive degree of stenosis (less than $20\%$ or greater than $80\%$) have been removed. 
%
As aortic length and aortic inlet diameter are correlated with age, when age ranges deduced from these two morphological parameters did not overlap, the corresponding shapes were discarded. Further shapes have been removed performing a preliminary CFD analysis of peak systole flow using \texttt{STAR-CCM+}~\cite{yevtushenko2021deep}, discarding 
those resulting in unphysical quantities of interest.

This procedure resulted in a cleaned database of $1312$ (real and synthetic) shapes, represented by triangulated surface meshes, centerline points, and centerline radii.
Along the curse of this study, $437$ additional geometries have been removed based on the results of time dependent simulations (see section \ref{ssec:blood_flow}) and further $99$ due to 
inaccurate registrations (see section \ref{sec:registration}). The remaining $776$ cases have been split in $724$ training and $52$ testing shapes.
Figure \ref{fig:clustergeometries}, center and right plots, show qualitatively the extent of the training and test datasets, based on a 
T-distributed Stochastic Neighbor Embedding~\cite{van2008visualizing} (t-SNE) with the Euclidean distance on a geometrical encoding of the shapes
that relies on the shape registration map (further details will be given in section~\ref{subsec:sml_correlations}).


\subsection{Blood flow modelling}\label{ssec:blood_flow}
Let us denote with $\Omega \subset \mathbb R^3$ the computational domain representing a generic shape from the considered dataset, whose boundaries can be decomposed as
\begin{equation}\label{eq:omega_bnd}
\partial \Omega = \Gamma_{\text{wall}} \cup \Gamma_{\rm in} \cup \left( \bigcup_{i=1}^4 \Gamma_i \right),
\end{equation}
distinguishing between the vessel \text{wall} $\Gamma_{wall}$, the inlet boundary $\Gamma_{\rm in}$, and the four outlet boundaries (BCA, LCCA, LSA, TA), as depicted in figure \ref{fig:domain}. We assume that the blood flow in the considered vessels behaves as an incompressible Newtonian fluid and thus describes the hemodynamics via the incompressible Navier--Stokes equations for the velocity $\bu:\Omega \to \mathbb R^3$ and the pressure fields $p:\Omega \to \mathbb R$:
\begin{equation}\label{eq:3dnse}
\left\{
\begin{aligned}
\rho \partial_t \mathbf{u}+\rho \mathbf{u}\cdot\nabla\mathbf{u}+\mu\Delta\mathbf{u}-\nabla p=\mathbf{0},\qquad&\text{in}\ \Omega,\\
    \nabla\cdot\mathbf{u}=0,\qquad&\text{in}\ \Omega,
\end{aligned}
\right.
\end{equation}
%
where $\rho=\SI{1.06e3}{\kilogram\per\meter^3}$ stands for the blood density, and $\mu=\SI{3.5e-3}{\second\cdot\pascal}$ is the dynamic viscosity. 


Equations~\eqref{eq:3dnse} are complemented by homogeneous Dirichlet boundary conditions for the velocity on $\Gamma_{\rm \text{wall}}$, i.e., neglecting  fluid-structure interactions between the blood flow and the vessel wall, by a Dirichlet boundary condition on $\Gamma_{\rm in}$, imposing a parabolic flow profile at the inlet~\cite{katz2023impact}, and by lumped parameter models on the four outlet boundaries, i.e., 
\begin{equation}
  \label{eq:3dnse-bc}
\left\{
\begin{aligned}
\mathbf{u}&=\mathbf{u}_{\rm in},\qquad &&\text{on}\ \Gamma_{\rm in},\\
\mathbf{u}&=\mathbf{0},\qquad &&\text{on}\ \Gamma_{\text{wall}},\\
    -pI+\mu\left(\nabla\mathbf{u}+\nabla\mathbf{u}^T\right)&=-P_i\mathbf{n},\qquad &&\text{on}\ \Gamma_i,\ i\in\{1,2,3,4\}.
\end{aligned}
\right.
\end{equation}
In the last equation, $\mathbf{n}$ denotes the outward normal vector to the fluid boundary and $P_i$ stands for an approximation of the outlet pressure imposed on the boundary $\Gamma_i$, which is evaluated as a function of the boundary flow rates $Q_i:=\int_{\Gamma_i} \mathbf{u}\cdot\mathbf{n}$, $i=\in\{1,2,3,4\}$, 
via a 3-elements (RCR) Windkessel model~\cite{westerhof2009arterial}
\begin{equation}\label{eq:wk-rcr-i}
\left\{
\begin{aligned}
C_{d,i}\frac{d\pi_i}{dt}+\frac{\pi_i}{R_{d,i}}=Q_i,\qquad\qquad\qquad\qquad\ \;\,\quad&\text{on}\ \Gamma_i,\ i\in\{1,2,3,4\},\\
   P_i=R_{p,i}Q_i+\pi_i,\qquad\qquad\qquad\qquad\ \;\,\quad&\text{on}\ \Gamma_i,\ i\in\{1,2,3,4\},\\
\end{aligned}
\right.
\end{equation}
depending on an auxiliary \textit{distal} pressure $\pi_i$, a \textit{proximal} resistance $R_p$ (modeling the resistance to the flow of the arteries close to the open boundary), a \textit{distal} resistance $R_d$ (modeling the downstream resistance of the rest of the cardiovascular system), and a \textit{capacitance} $C_d$ (modeling the compliance of the cardiovascular system).
The tuning of the Windkessel parameters will be discussed in detail in section \ref{ssec:bc-calibration}.
%
Peak inlet flow rates for each shape were provided as part of the patient cohort data for each considered shape. These values  were adjusted to match a parabolic shape on the inlet boundary, and multiplied by a time dependent function to obtain the inlet boundary condition over time.

\begin{figure}[!htp]
  \centering
  \includegraphics[width=0.4\textwidth, trim={0 0 0 20}, clip]{img/domain.pdf}
  \caption{Example of a computational domain. A parabolic profile is imposed on the inlet boundary (ascending aorta), based on a given peak flow rate. 
  Windkessel models are used at the outlets (thoracic aorta, brachiocephalic artery, left common carotid artery, and left subclavian artery).}
  \label{fig:domain
}
  \label{fig:domain}
\end{figure}

\subsection{Calibration of boundary conditions across the cohort of patients}\label{ssec:bc-calibration}
The Windkessel parameters might have a considerable impact on the solution and the calibration typically depends on the flow regime of interest, on the
particular anatomical details, and on available data. For the purpose of this study, we opted for an approach driven by the flow split across
the different branches.

We introduce the total resistances
$R_i := R_{p,i} + R_{d,i}$ and the equivalent
\textit{systemic} resistance $R_S^{-1}  := \sum_{i=1}^N R_i^{-1}$. Neglecting the contribution of the 3D domain to the total resistance, the quantities
$R_i/ R_S$, for $i=1,2,3,4$, can be used to control the \textit{flow split}, i.e., the ratio of the inlet flow $Q_{\rm in}$ that flows, on average, through each outlet.

We have tuned the template geometry's parameters considering a flow split of $50\%$ for the BCA, and of $25\%$ for LCCA and LSA, as in~\cite{katz2023impact}. For the systematic calibration of the Windkessel parameters on other geometries, we consider a model for the average flow split based on the following steps:
\begin{enumerate}[itemsep=2pt, left=0pt, labelsep=5pt]
  \item A rescaling of the systemic resistance, based on the patient specific inlet,
  \begin{equation*}%\label{eq:calibrated_RS}
  R_S :=  \frac{\hat Q_{\rm in}}{Q_{\rm in}} \hat R_S,
  \end{equation*}
  \item A shape-specific flow split based on the reference mean velocities and the outlet areas of the new patient,
  \begin{equation*}%\label{eq:calibrated_sigma_i}
      \sigma_i := \hat Q_i \frac{A_i}{\hat A_i} \frac{1}{\sum_{j=1}^4 \hat Q_j \frac{A_j}{\hat A_j}} = \frac{\hat u^{\mathrm{mean}}_i A_i }{ \sum_{j=1}^4 \hat u^{\mathrm{mean}}_j A_j},
  \end{equation*}
      \item The approximation of the patient-specific total resistance 
      \begin{equation*}
        R_i = \sigma_i R_S,
      \end{equation*}
      \item A splitting between proximal and distal resistance (the same for all  patients),
  \begin{equation*}%\label{eq:R_p-and-R_d}
  R_{p,i}  = 0.1 R_i,\ R_{d,i}=0.9 R_i\,.
  \end{equation*}    
\end{enumerate}
 Finally, capacitances are defined proportionally to the area of the outlet boundaries, i.e., 
\begin{equation*}
  C_i = \frac{A_i}{A_{\rm tot}} C_{\rm tot},
\end{equation*}
as a fraction of the total capacitance $C_{\rm tot} = 10^{-8}$ (the same for all shapes).

The distributions of the total resistances, distal capacities, and outlets areas, for the complete shape database, are shown in figure~\ref{fig:param_distr}.
\begin{figure}[!htp]
  \centering
  \includegraphics[width=0.49\textwidth]{img/capacity.pdf}
  \includegraphics[width=0.49\textwidth]{img/area.pdf}\\
  \includegraphics[width=0.49\textwidth]{img/resistance.pdf}
  \caption{Distribution of Windkessel parameters $C_{d,i}$ (distal capacity), $R_i=R_{d,i}+R_{p,i}$ (Total resistance), and $A_i$ (boundary area) for the different outlets, across the training and test datasets.}
  \label{fig:param_distr}
\end{figure}


\begin{rmk}[Calibration based on flow split]
The main motivation behind this approach is the fact that the flow split can be experimentally measured non-invasively on the different sections, or inferred according 
to existing literature data and patient anatomy. Moreover, using the flow split as parameter allows modelling different physiological (rest/exercise)
or pathological (e.g., obstruction of vessels downstream) conditions, and can be used to enrich the solution dataset depending on the context of interest.
%
\end{rmk}


\subsection{Synthetic dataset of aortic shapes and numerical simulations}
\label{ref:numsim}
We solve numerically system \eqref{eq:3dnse}-\eqref{eq:3dnse-bc} for each of the $776$ considered geometries, discretizing the corresponding volume with a tetrahedral mesh reated from the original surface shape
and imposing shape specific boundary conditions as described in section \eqref{ssec:bc-calibration}. 
%
We use stabilized equal-order linear finite elements for velocity and pressure and a BDF2 time marching scheme, 
with a semi-implicit treatment of the non-linear convective term and of the VMS turbulence model~\cite{bazilevs2007variational, forti2015semi}.
Further details on the discretization and on the numerical method are provided in  appendix \ref{appendix:weak}.
%
The ODEs \eqref{eq:wk-rcr-i} are solved using an implicit Euler scheme, and the coupling at the boundary is implemented explicitly, i.e., using the boundary pressures at the previous time iteration to impose Neumann boundary conditions on each outlet. Some velocity snapshots are shown in figure~\ref{fig:snapsu}.
%
The solver is implemented in the computational framework \texttt{lifex-cfd}~\cite{AFRICA2024109039}, based on the open-source library \texttt{deal.II}~\cite{arndt2022deal}.
%
Simulations have been run for five hearth beats, only the last period is considered, in order to ensure a quasi-periodic state.

The results (figure~\ref{fig:flow_and_pressure}) show a rather uniform distribution of flow and pressure values at boundaries across the dataset.
%
\begin{figure}[!htp]
  \centering
  \includegraphics[width=0.9\textwidth]{img/flows_and_pressure.pdf}
  \caption{\textbf{Top: } Numerical results for te flow at inlet (AAo, with opposite sign) and outlets (BCA, LCCA, LSA, TA) boundaries. 
  \textbf{Bottom: }  Numerical results for the pressure at inlet (AAo) and outlets (BCA, LCCA, LSA, TA) boundaries. The $25$-th and $75$-th percentile across all the $724$ and $52$ training and test data are shown. The red vertical bands correspond to the time window $t\in[0.05s, 0.25s]$ which is the focus of our data assimilation studies, see remark~\ref{rmk:timewindow}.}
\label{fig:flow_and_pressure}
\end{figure}

