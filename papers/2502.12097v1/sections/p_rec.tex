This section focuses on the problem of estimating the pressure field from velocity data.
We benchmark the pressure reconstruction method based on the EPD-GNN (section \ref{ssec:pres-gnn}) against two variational-based approaches, the pressure Poisson estimator (PPE) and the Stokes estimator (STE).
A complete comparison of methodologies is presented in \cite{bertoglio2018relative}. We choose to consider PPE due to its simple implementation and already popularized use, and STE, because it has been benchmarked as the best method in \cite{bertoglio2018relative}. As it has been already discussed, joint reconstructions with PBDW for velocity and pressure, as in \cite{galarce2023displacement}, will not be applied due to the loss of accuracy in the overall reconstruction.


The pressure estimation problem is considered in a given target shape $\mathcal T$. 
We will denote with $\Omega_{\mathcal T}$ the corresponding computational domain, with $\mathcal{T}_h$ its triangulation, 
and with $\widehat{d}_{\mathbf u}$ and $\widehat{d}_p$ the degrees of freedom of the underlying finite element spaces for velocity and pressure, respectively. The diameter of a generic element $K \in \mathcal T_h$ is $h_K$. 
%
The input velocity field at a time $t_n$ will be denoted by $\widehat{\mathbf u}^n \sim\mathcal{N}(\mathbf m^n_{\widehat{\mathbf u}}, \Sigma^n_{\widehat{\mathbf u}})$ and it is assumed to be Gaussian distributed, 
with $\mathbf m^n_{\widehat{\mathbf u}}\in\mathbb{R}^{\widehat{d}_{\mathbf u}}$ and $\Sigma^n_{\widehat{\mathbf u}}\in\mathbb{R}^{\widehat{d}_{\mathbf u}\times \widehat{d}_{\mathbf u}}$. For example, $\widehat{\mathbf u}^n$ could be obtained from 4D-flow MRI data with heteroscedastic PBDW.

\subsection{Variational-based pressure estimators}
\label{subsec:ppestedef}

  
In the pressure-Poisson estimator (PPE) \cite{ebbers2001}, the velocity field is directly inserted in the right-hand-side variational form of the incompressible Navier--Stokes equations, and a suitable pressure field is recovered solving the resulting problem.
\begin{problem}[PPE]
  \label{def:ppe}
  Given three consecutive velocity time steps $\widehat{\ub}^{n}$, $\widehat{\ub}^{n+1/2}$, and $\widehat{\ub}^{n+1}$, find the
pressure at the intermediate time $\widehat{p}^{n+1/2}_{\text{PPE}}\in\mathbb{P}^1(\mathcal{T}_h)$ such that
  \begin{align}
    \int_{\Omega_{\mathcal T}} \nabla \widehat{p}^{n+1/2}_{\text{PPE}}\cdot\nabla q  =
     -\frac{\rho}{\tau}\int_{\Omega_{\mathcal T}} (\widehat{\ub}^{n+1}-\widehat{\ub}^{n})\cdot\nabla q - \rho\int_{\Omega_{\mathcal T}} (\widehat{\ub}^{n+1/2}\cdot\nabla\widehat{\ub}^{n+1/2})\cdot\nabla q,\;
     \forall q\in\mathbb{P}^1(\mathcal{T}_h),%\\
    % A_{\text{PPE}}\,\widehat{p}^{n+1/2}_{\text{PPE}} = M_{\text{PPE}}^{n+1}\widehat{\ub}_{n+1} - M_{\text{PPE}}^{n}\widehat{\ub}_{n} + Q_{\text{PPE}}(\widehat{\ub}^{n+1/2}, \widehat{\ub}^{n+1/2}),
     \label{eq:ppe-weak}
  \end{align}
  with boundary conditions $\widehat{p}^{n+1/2}_{\text{PPE}}=q=0$ on $\partial \Omega_{\mathcal T}$. %Gamma=\Gamma_{\rm in} \cup \left(\cup_{i=1}^4\Gamma_i\right)\cup\Gamma_{wall}$.
  Problem \eqref{eq:ppe-weak} can be equivalently written in matrix form as
  \begin{equation} \label{eq:ppe}
  A_{\text{PPE}}\,\widehat{p}^{n+1/2}_{\text{PPE}} = M_{\text{PPE}}\widehat{\ub}_{n+1} - M_{\text{PPE}}\widehat{\ub}_{n} + Q_{\text{PPE}}(\widehat{\ub}^{n+1/2}, \widehat{\ub}^{n+1/2}),
  \end{equation}
with natural definition of the stiffness matrix $A_{\text{PPE}}$, the mass matrix $M_{\text{PPE}}$,  and the advection term $Q_{\text{PPE}}$. 
 \end{problem} 

A bias correction for the estimator is obtained as the solution to the following problem:
Find $b_{\text{PPE}}\in\mathbb{P}^1(\mathcal{T}_h)$ such that %$\forall\mathbf{z}\in [\mathbb{P}^1(\mathcal{T}_h)]^d,\ $
  \begin{equation}\label{eq:ppe-bias}
    \int_{\Omega_{\mathcal T}} \nabla b_{\text{PPE}}\cdot\nabla q =-\rho\sum_{T\in\mathcal{T}_h}\sum_{i=1}^{\widehat{d}_{\mathbf u}}\sum_{j=1}^{\widehat{d}_{\mathbf u}}\int_T \Psi_{i,j}\nabla q,
    \; \forall q\in\mathbb{P}^1(\mathcal{T}_h)\,,
  \end{equation}
where we introduced the notaton
\begin{equation*}\label{eq:psi-bias-corr}
\Psi_{i,j} : = \left((\phib_i\cdot\nabla\phib_j)\odot\Sigma^{n+1/2}_{ij}\right), \; i,j=1,\hdots,\widehat{d}_{\mathbf u}
\end{equation*}
and $\Sigma^{n+1/2}_{ij}$ stands for the $3\times3$ $(i,j)$-subblock of the covariance matrix $\Sigma^{n+1/2}_{\widehat{\mathbf u}}$ corresponding to the support points 
of the finite element shape functions $\phib_i$,and $\phib_j$, and $\odot$ stands for the element-wise Hadamard product of two matrices.



%Combining \eqref{eq:ppe} and  \eqref{eq:ppe-bias}, the unbiased PPE bilinear estimator is defined by
% \begin{equation}
%  \widehat{p}^{n+1/2}_{\text{PPE}} = A_{\text{PPE}}^{-1}M_{\text{PPE}}^{n+1}\widehat{\ub}_{n+1} - A_{\text{PPE}}^{-1}M_{\text{PPE}}^{n}\widehat{\ub}_{n} + A_{\text{PPE}}^{-1}Q_{\text{PPE}}(\widehat{\ub}%^{n+1/2}, \widehat{\ub}^{n+1/2}) -b_{\text{PPE}}\,.
%   \end{equation}
%   \ac{If we don't need $H_{\text{PPE}}$ later, I would not introduce the additional notation}
%    $\widehat{p}^{n+1/2}_{\text{PPE}} = H_{\text{PPE}}x^{n}-b_{\text{STE}}$ with $x^{n}=(\widehat{\ub}^{n}, \widehat{\ub}^{n+1/2}, \widehat{\ub}^{n+1})$:
%  \begin{equation}
%    H_{\text{PPE}}x^{n} \coloneqq A_{\text{PPE}}^{-1}M_{\text{PPE}}^{n+1}\widehat{\ub}_{n+1} - A_{\text{PPE}}^{-1}M_{\text{PPE}}^{n}\widehat{\ub}_{n} + A_{\text{PPE}}^{-1}Q_{\text{PPE}}(\widehat{\ub}^{n+1/2}, \widehat{\ub}^{n+1/2}).
%  \end{equation}

In the Stokes estimator (STE) \cite{svihlova_2016}, the velocity field is inserted in the right-hand-side of the Navier--Stokes equations, but, unlike the PPE, the 
variational problem for the pressure field is formulated as a Stokes projection, including an additional corrector. As shown, e.g.,  in \cite{bertoglio2018relative}, this approach allows to 
%Stokes field is computed which is used to test the governyng equations and thus providing 
obtain more robust results, especially against noisy velocity data.
\begin{problem}[STE]
  \label{def:ste}
    Given three consecutive velocity time steps $\widehat{\ub}^{n}$, $\widehat{\ub}^{n+1/2}$, and $\widehat{\ub}^{n+1}$, find $(\wb,\widehat{p}^{n+1/2}_{\text{STE}})\in [\mathbb{P}^1(\mathcal{T}_h)]^d \times \mathbb{P}^1(\mathcal{T}_h)$ such that
  \begin{equation}\label{eq:ste-weak}
  \begin{aligned}
    \int_{\Omega_{\mathcal T}} \nabla\wb:\nabla\mathbf{z} &- \int_{\Omega_{\mathcal T}} \widehat{p}^{n+1/2}_{\text{STE}}(\nabla\cdot\mathbf{z})+\int(\nabla\cdot\wb)q\\
     &+ \sum_{K\in\mathcal{T}_h}C_s h^2_K\int_K \nabla \widehat{p}^{n+1/2}_{\text{STE}}\cdot\nabla q=\\
     &-\frac{\rho}{\tau}\int_{\Omega_{\mathcal T}} (\widehat{\ub}^{n+1}-\widehat{\ub}^{n})\cdot\mathbf{z} - \rho\int_{\Omega_{\mathcal T}} (\widehat{\ub}^{n+1/2}\cdot\nabla\widehat{\widehat{\ub}^{n+1/2}})\cdot\wb-\mu\int_{\Omega_{\mathcal T}} \nabla\widehat{\ub}^{n+1/2}:\nabla\mathbf{z}\\
     &+\sum_{K\in\mathcal{T}_h}C_s h^2_K\rho\int_K (\mu\Delta\widehat{\ub}^{n+1/2}-\widehat{\ub}^{n+1/2}\cdot\nabla\widehat{\ub}^{n+1/2})\cdot\nabla q,
\;     \forall (\mathbf{z},q) \in [\mathbb{P}^1(\mathcal{T}_h)]^d \times \mathbb{P}^1(\mathcal{T}_h)
  \end{aligned}
  \end{equation}
    with boundary conditions $\wb=\mathbf{z}=0$ on $\partial \Omega_{\mathcal T}$. %$\Gamma_{\rm in} \cup \left(\cup_{i=1}^4\Gamma_i\right)\cup\Gamma_{wall}$.
    %
Problem \eqref{eq:ste} can be equivalently formulated in matrix form as
  \begin{equation}\label{eq:ste}
    \begin{aligned}
     A_{\text{STE}}\,\wb + B \widehat{p}^{n+1/2}_{\text{STE}} &= M_{\text{STE}}\widehat{\ub}_{n+1} - M_{\text{STE}}\widehat{\ub}_{n} + Q_{\text{STE}}(\widehat{\ub}^{n+1/2}, \widehat{\ub}^{n+1/2}) + M_{\text{STE}}\widehat{\ub}_{n+1/2},\\
     B^T\wb &= 0,
  \end{aligned}
  \end{equation}
with natural definition of the stiffness matrix $A_{\text{STE}}$, of the mass matrix $M_{\text{STE}}$, of 
the matrix associated to the grad-div term $B$, and of the advection term $Q_{\text{STE}}$. 
 \end{problem} 
  
A bias correction for the STE can be obtained as solution to the following problem: Find $(\wb,b_{\text{STE}}) \in[\mathbb{P}^1(\mathcal{T}_h)]^d \times \mathbb{P}^1(\mathcal{T}_h)$ such that 
  \begin{equation}\label{eq:ste-bias}
    \begin{aligned}
    \int_{\Omega_{\mathcal T}} \nabla\wb:\nabla\mathbf{z} &- \int_{\Omega_{\mathcal T}} b_{\text{STE}}(\nabla\cdot\mathbf{z})+\int(\nabla\cdot\wb)q =\rho\sum_{T\in\mathcal{T}_h}\sum_{i=1}^{\widehat{d}_{\mathbf u}}\sum_{j=1}^{\widehat{d}_{\mathbf u}}\int_{\Omega_{\mathcal T}} \Psi_{i,j} \cdot\mathbf{z}.
   \end{aligned}
    \end{equation}

The bias corrections introduced here, have been extended to the general case of heteroscedastic noise starting from~\cite{bertoglio2018relative}.

\subsection{Numerical results}
\label{subsec:resultspressureestimators}
This section is dedicated to the comparison of the performance of the variational-based estimators (PPE and STE)
against the GNNs. The results are evaluated considering a global error on the pressure fluctuation on the whole target domain $\epsilon_{\widehat{p}}$ (equation~\eqref{eq:l2relerr}).

A further biomarker of clinical interest is the pressure drop across the coarctation. We consider two cross-sections $\Gamma^{\text{sec}}_\text{in}$ close to the inlet $\Gamma_{\rm in}$ and a cross-section $\Gamma^{\text{sec}}_4$ close to the outlet $\Gamma_4$ of the target geometries (the position of the cross-sections depends on the centerline encoding of the geometries, as in~\cite{katz2023impact}) and define the pressure drop as
\begin{equation}
  \widehat{p}_{4\text{-}\text{in}} = \frac{1}{\left|\Gamma^{\text{sec}}_4\right|}\int_{\Gamma^{\text{sec}}_4}\widehat{p}\,d\sigma-\frac{1}{|\Gamma^{\text{sec}}_\text{in}|}\int_{\Gamma^{\text{sec}}_\text{in}}\widehat{p}\,d\sigma,
\end{equation}
where $| \cdot |$ stands for the area of the corresponding surface. 
%A_{\Gamma^{\text{sec}}_5},A_{\Gamma^{\text{sec}}_\text{in}}$ are the areas of the cross-sections $\Gamma^{\text{sec}}_5,\Gamma^{\text{sec}}_\text{in}$. 
%As highlighted by the definition of the pressure drop, we are often more interested on the pressure gradient rather than the pressure field itself. From this reasoning, we employ as metric to compare the prediction error of the pressure field on the whole target domains:
%\begin{equation}
%  \label{eq:l2relerr}
%  \epsilon_{\widehat{p}} = \frac{\lVert \widehat{p}_{\text{true}}-\widehat{p}-\overline{\widehat{p}}_{\text{true}}+\overline{\widehat{p}}\rVert_2}{\lVert \widehat{p}_{\text{true}}-\overline{{\widehat{p}}}_{\text{true}} \rVert_2}
%\end{equation}
%where $\widehat{p}_{\text{true}}$ is the high-fidelity pressure field obtained from 3D-INS  numerical simulations, $\overline{\widehat{p}}_{\text{true}}\in\mathbb{R}$ is its average, $\widehat{p}$ is the predicted velocity field, $\widehat{p}_{\text{true}}\in\mathbb{R}$ is its average.

% In the following results, the high-fidelity pressure field ($\widehat{p}_{\text{true}}$) is compared with PPE and STE reconstructions based on different velocity inputs
% (high-fidelity solution, low resolution observations, and PBDW reconstruction), as well as 
% with the pressure inferred with GNNs from the geometrical encoding (\textit{gnn-gp}), from  the high-fidelity velocity field (\textit{gnn-vp}), and from the velocity reconstructed with PBDW (\textit{gnn-vp-pbdw}).

We will consider the pressure fields computed with the PPE and STE from the high-fidelity velocity $\widehat{\mathbf{u}}^{\text{true}}$ (\textit{ppe}, \textit{ste}), 
the pressure fields computed with the PPE and STE from the observed velocity field with values $\{l_i\}_{i=1}^{M_{\text{voxels}}}$ and support points $\{\mathbf{c}_i^{\text{vox}}\}_{i=1}^{M_{\text{voxels}}}$ (\textit{ppeobs}, \textit{steobs}), 
the pressure fields computed with the PPE and STE from the PBDW velocity $\widehat{\mathbf{u}}^{\text{true}}_{\text{PBDW}}$ (\textit{ppefom}, \textit{stefom}), and the pressure fields 
computed with the reduced order models of the PPE and STE from the PBDW velocity $\widehat{\mathbf{u}}^{\text{true}}_{\text{PBDW}}$ (\textit{pperom}, \textit{sterom}), as described in appendix~\ref{apendix:rom}. 
%
Additionally, the pressure inferred with GNNs from the geometrical encoding is denoted by \textit{gnn-gp}, the pressure inferred with GNNs from the high-fidelity velocity field $\widehat{\mathbf{u}}^{\text{true}}$ by \textit{gnn-vp}, and the pressure inferred with GNNs from the PBDW velocity field $\widehat{\mathbf{u}}^{\text{true}}_{\text{PBDW}}$ by \textit{gnn-vp-pbdw}. The inference problem \textit{gnn-vp} and \textit{gnn-vp-pbdw} employ the same architecture defined in section~\ref{ssec:pres-gnn}.

The average pressure drops, median absolute pressure drops errors and the average $L^2$-relative errors (equation \eqref{eq:l2relerr}) across all test geometries are shown in figure~\ref{fig:pressure_res}.
For a qualitative comparison, the different predicted pressure fields are shown for the test geometry $n=12$ in figure~\ref{fig:pres_12}. 
%
In this case the results of the EPD-GNNs are comparable to those of PPE and STE. However, GNN are expected to deliver better results increasing the amount of training data,
due to the high geometric variability of the dataset: a symptom is the low training error in figure~\ref{fig:overfitting}. The test cases corresponding to the minimum, maximum and median $L^2$-relative errors are reported in figure~\ref{fig:gp} for the \textit{gnn-gp} problem. 

The best accuracy on the pressure field approximation and pressure drop corresponds to the time instants $t=0.125s$ and $t=0.15s$ for the PPE and STE estimators, the same time instants associated also to the best rSVD reconstruction error for the pressure fields in figure~\ref{fig:recerr}. However, the worse accuracy is not due to the rSVD reconstruction error but to the definition of the PPE and STE, as the same accuracy is associated to the estimated pressure fields \textit{ppe} and \textit{ste} obtained from the high-fidelity velocity. Moreover, the accuracy of PPE and STE does not seem to depend on the choice of observed velocity field: be it obtained from 4D-flow MRI observations (\textit{ppeobs}, \textit{steobs}), high-fidelity velocity fields (\textit{ppe}, \textit{ste}), PBDW velocity (\textit{ppefom}, \textit{stefom}), or reduced-order PPE and STE (\textit{pperom}, \textit{sterom}), the predictions achieve almost the same accuracy. Possibly, to improve the accuracy, the time resolution of $\Delta t=0.025s$ should be reduced to the high-fidelity simulations' time step $\Delta t=0.0025s$.

\begin{figure}[!ht]
  \centering
  \includegraphics[width=0.9\textwidth]{img/pressure_drop.pdf}\\
  \includegraphics[width=0.45\textwidth]{img/pressure_l2.pdf}
  \caption{\textbf{Top left}: average value of the pressure drop $\overline{\widehat{p}_{4\text{-}\text{in}}-\widehat{p}_{\text{true},4\text{-}\text{in}}}$ between cross-sections $\Gamma_4^{\text{sec}}-\Gamma_\text{in}^{\text{sec}}$ over all the $52$ test geometries at time instants $t\in\{0.075s, 0.1s, 0.125s, 0.15s, 0.175s, 0.2s\}$. \textbf{Top right}: median of the absolute error $|\widehat{p}_{4\text{-}\text{in}}-\widehat{p}_{\text{true},4\text{-}\text{in}}|$ of the predicted pressure drop with respect to the high-fidelity pressure drop over all the $52$ test geometries. \textbf{Bottom: } average of the $L^2$-relative error $\epsilon_{\widehat{p}}$ defined in equation~\ref{eq:l2relerr} over all the $52$ test geometries. The description of the labels is reported in the text.}
  \label{fig:pressure_res}
\end{figure}

\begin{figure}[!ht]
  \centering
  \includegraphics[width=0.9\textwidth]{img/pressure.pdf}\\
  \caption{Predicted pressure at time $t=0.125s$ for test geometry $n=12$ from figure~\ref{fig:cluster_v} and associated $L^2$-relative errors $\epsilon_{\widehat{p}}$.
  % The pressure fields computed with the PPE and STE from the high-fidelity velocity $\widehat{\mathbf{u}}^{\text{true}}$ (\textit{ppe}, \textit{ste}), 
  %the pressure fields computed with the PPE and STE from the observed velocity field with values $\{l_i\}_{i=1}^{M_{\text{voxels}}}$ and support points $\{\mathbf{c}_i^{\text{vox}}\}_{i=1}^{M_{\text{voxels}}}$ (\textit{ppeobs}, \textit{steobs}), 
  %the pressure fields computed with the PPE and STE from the PBDW velocity $\widehat{\mathbf{u}}^{\text{true}}_{\text{PBDW}}$ (\textit{ppefom}, \textit{stefom}), and the pressure fields 
  %computed with the reduced order models of the PPE and STE from the PBDW velocity $\widehat{\mathbf{u}}^{\text{true}}_{\text{PBDW}}$ (\textit{pperom}, \textit{sterom}). 
  %%
  %Additionally, the pressure inferred with GNNs from the geometrical encoding is denoted by \textit{gnn-gp}, the pressure inferred with GNNs from the high-fidleity velocity field $\widehat{\mathbf{u}}^{\text{true}}$ by \textit{gnn-vp}, and the pressure inferred with GNNs from the PBDW velocity field $\widehat{\mathbf{u}}^{\text{true}}_{\text{PBDW}}$ by \textit{gnn-vp-pbdw}.
  }
  \label{fig:pres_12}
\end{figure}

\subsection{Forward uncertainty quantification}
\label{subsec:uqpressureestimators}
Since PBDW models the uncertainty on the predicted velocity field from the coarse measurements $\widehat{\mathbf{u}}_{\text{PBDW}}\sim\mathcal{N}(m_{\widehat{\mathbf{u}}_{\text{PBDW}}}, \Sigma_{\widehat{\mathbf{u}}_{\text{PBDW}}})$, we want to study the uncertainty propagation to the pressure field through the pressure estimators and the inference with GNNs. In the following studies, we will keep the test geometry fixed and equal to test case $n=12$ from figure~\ref{fig:cluster_v}.

To measure the velocity field variability, we evaluate the normalized standard deviation at different values of $(\text{SNR-ho}, \text{SNR-he})\in \{(10, 0.5), (0.4, 0.1), (0.2, 0.05)\}$ in figure~\ref{fig:snr}:
\begin{equation}
  \label{eq:std}
  \text{std}_{\widehat{\mathbf{u}}} = \frac{\sum^{n_{\text{samples}}}_{i=1} \lVert \widehat{\mathbf{u}}_i-\tfrac{1}{n_{\text{samples}}}\sum_{i=1}^{n_{\text{samples}}}\widehat{\mathbf{u}}_i\rVert_2}{\sum^{n_{\text{samples}}}_{i=1} \lVert \tfrac{1}{n_{\text{samples}}}\sum_{i=1}^{n_{\text{samples}}}\widehat{\mathbf{u}}_i\rVert_2},
\end{equation}
where $n_{\text{samples}}=100$ are the number of samples from the Gaussian distribution $\widehat{\mathbf{u}}_{\text{PBDW}}\sim\mathcal{N}(m_{\widehat{\mathbf{u}}_{\text{PBDW}}}, \Sigma_{\widehat{\mathbf{u}}_{\text{PBDW}}})$ of the velocity predicted by PBDW with $r_{\mathbf u}\in\{500, 1000, 2000\}$ velocity modes.

Since the computational cost of a forward uncertainty problem is high due to the high number of forward evaluations, $n_{\text{samples}}=100$ in our case, we employ a reduced order model of the PPE and STE (\textit{pperom, sterom}), as described in appendix~\ref{apendix:rom}. We keep the number of pressure modes $r_p=1000$ fixed and vary the number of velocity modes $r_{\mathbf u}\in\{500, 1000, 2000\}$ and signal-to-noise ratios $(\text{SNR-ho}, \text{SNR-he})\in \{(10, 0.5), (0.4, 0.1), (0.2, 0.05)\}$.
%Since uncertainty quantification problems requires a large number of forward evaluations ($n_{\text{samples}}=100$ in our case), to reduce the overall computational cost 
%we employ a reduced-order version of the pressure estimators PPE and STE, obtained projecting the corresponding formulations ~\eqref{eq:ppe} and~\eqref{eq:ste} onto reduced-order spaces
%defined by the rSVD bases for velocity and pressure registered on the target geometry. 
%Since the computational cost of a forward uncertainty problem are high due to the high number of forward evaluations, $n_{\text{samples}}=100$ in our case, we employ a reduced order model of the PPE and STE (\textit{pperom, sterom}).
%\subsection{Reduced-order formulation of PPE and STE}
%A reduced-order formulation of PPE and STE can be defined projecting 
%their matrix forms Equations~\eqref{eq:ppe} and~\eqref{eq:ste}, respectively, onto reduced spaces, i.e., 
%\begin{equation}\label{eq:ppe-rom}
%  \widehat{\Phi}_{p}^T A_{\text{PPE}}\,\widehat{\Phi}_{p}z_{\widehat{p}^{n+1/2}_{\text{PPE}}} = \widehat{\Phi}_{p}^T M_{\text{PPE}}^{n+1}\widehat{\ub}_{n+1} - \widehat{\Phi}_{p}^T M_{\text{PPE}}^{n}\widehat{\ub}_{n} + \widehat{\Phi}_{p}^T Q_{\text{PPE}}(\widehat{\ub}^{n+1/2}, \widehat{\ub}^{n+1/2}),
%\end{equation}
%\begin{equation}\label{eq:ste-rom}
%\begin{aligned}
%  \widehat{\Phi}_{u}^T A_{\text{STE}}\,\widehat{\Phi}_{u}\mathbf{z}_{\wb} + \widehat{\Phi}_{u}^T B \widehat{\Phi}_{p}z_{\widehat{p}^{n+1/2}_{\text{STE}}} &= \widehat{\Phi}_{u}^T M_{\text{STE}}%^{n+1}\widehat{\ub}_{n+1} - \widehat{\Phi}_{u}^T M_{\text{STE}}^{n}\widehat{\ub}_{n} + \widehat{\Phi}_{u}^T Q_{\text{STE}}(\widehat{\ub}^{n+1/2}, \widehat{\ub}^{n+1/2}) \\
%  & \quad + \widehat{\Phi}_{u}^T M_{\text{STE}}^{n+1/2}\widehat{\ub}_{n+1/2},\\
%     \widehat{\Phi}_{p}^T B^T\widehat{\Phi}_{u}\mathbf{z}_{\wb} &= 0,
%\end{aligned}
%\end{equation}
%where $\widehat{\Phi}_{u}$ and $\widehat{\Phi}_{p}$ denote the 
%rSVD bases for velocity and pressure, registered on the target geometry.
%
%
%$\widehat{\Phi}_{u}\in\mathbb{R}^{r_{\mathbf u}\times \widehat{d}_{\mathbf u}}$ and $\widehat{\Phi}_{p}\in\mathbb{R}^{r_p\times \widehat{d}_p}$ and subsitution of $\widehat{p}^{n+1/2}_{\text{PPE}}\in\mathbb{R}^{\widehat{d}_p},\widehat{p}^{n+1/2}_{\text{STE}}\in\mathbb{R}^{\widehat{d}_p}, \wb\in\mathbb{R}^{\widehat{d}_{\mathbf u}}$ with the respective reduced variables $z_{\widehat{p}^{n+1/2}_{\text{PPE}}}\in\mathbb{R}^{r_p},z_{\widehat{p}^{n+1/2}_{\text{STE}}}\in\mathbb{R}^{r_p}, z_{\wb}\in\mathbb{R}^{r_{\mathbf u}}$. 
\begin{rmk}
Notice that, unlike for classical reduced-order models, the assembly of the matrices cannot be performed offline, since the registration map is needed to transport the rSVD modes from the reference to the target geometry. 
%
However, once the registration map is evaluated and the rSVD bases have been transported to the target geometry, the assembly of the reduced systems can be performed in parallel and solved with less computational costs thanks to the lower dimensionality of the reduced systems. %, related to the reduced dimensions $r_{\mathbf u}, r_p$.
\end{rmk}
%
%The velocity rSVD modes $\widehat{\Phi}_{u}$ are enriched with the supremizer technique~\cite{ballarin2015supremizer} with additional $r_p$ modes computed from the pressure rSVD modes $\widehat{\Phi}_{p}$, for a total of $r_{u,\text{sup}}=r_{\mathbf u}+r_p$ velocity modes $\Phi_{\widehat{\mathbf{u}},\text{sup}}\in\mathbb{R}^{r_{u,\text{sup}}\times \widehat{d}_{\mathbf u}}$:
%\begin{align*}
%  \Phi_{\widehat{\mathbf{u}},\text{sup}}^T A_{\text{STE}}\,\Phi_{\widehat{\mathbf{u}},\text{sup}}\mathbf{z}_{\wb} &+ \Phi_{\widehat{\mathbf{u}},\text{sup}}^T B \widehat{\Phi}_{p}z_{\widehat{p}^{n+1/2}_{\text{STE}}} = \\
%  &=\Phi_{\widehat{\mathbf{u}},\text{sup}}^T M_{\text{STE}}^{n+1}\widehat{\ub}_{n+1} - \Phi_{\widehat{\mathbf{u}},\text{sup}}^T M_{\text{STE}}^{n}\widehat{\ub}_{n} + \Phi_{\widehat{\mathbf{u}},\text{sup}}^T Q_{\text{STE}}(\widehat{\ub}^{n+1/2}, \widehat{\ub}^{n+1/2}) + \Phi_{\widehat{\mathbf{u}},\text{sup}}^T M_{\text{STE}}^{n+1/2}\widehat{\ub}_{n+1/2},\\
%    &\qquad\,\,\widehat{\Phi}_{p}^T B^T\Phi_{\widehat{\mathbf{u}},\text{sup}}\mathbf{z}_{\wb} = 0.
%\end{align*}

\begin{rmk}
For enhancing the stability of the reduced STE, the velocity rSVD modes are enriched with the supremizer technique~\cite{ballarin2015supremizer}. 
In this work, we evaluated the supremizers directly on the test geometries. The enrichment could be also performed offline on the reference geometry and transported on the new target, but this might
affect the overall stability of the formulation. 
\end{rmk}
%In principle, the supremizer enrichment could be performed offline on the template geometry and then transported with the registration map on the new patient geometry. It may occur that the method is yet not stable and additional corrections to the supremizers must be implemented. This is a future research direction. For the moment, we evaluted the supremizers directly on the test geometries.

In figure~\ref{fig:uq_p}, we compare the results of the PPE and STE with the predictions from the GNNs that compute the pressure field based on the PBDW velocity field as input, \textit{gnn-vp-pbdw}. The errors are evaluated on the test/target geometry with respect to the metrics in equation~\ref{eq:l2relerr}. The PPE estimator is omitted for $(\text{SNR-ho}, \text{SNR-he})\in \{(0.4, 0.1), (0.2, 0.05)\}$ as the relative error $\epsilon_{\widehat{\mathbf{u}}}$ goes above the value $1$ for all time instants.

In comparison to PPE and STE results, the GNNs' predictions are robust (or rather overconfident), to the uncertainty in the velocity field. It can be shown also looking at the standard deviation of the pressure field predicted with  \textit{pperom, sterom} or \textit{gnn-vp-pbdw} from $n_{\text{samples}}=100$ PBDW velocity samples $\widehat{\mathbf{u}}_{\text{PBDW}}$, in figure~\ref{fig:uq_cfd}: the magnitude of the normalized standard deviation in equation~\eqref{eq:std} is high only on a small subdomain for the GNN models, underlying that perturbations of the inputs do not considerably affect the outputs.

\begin{figure}[!htp]
  \centering
  \includegraphics[width=0.9\textwidth]{img/uq_v.pdf}\\
  \caption{Normalized standard deviation ($\text{std}_{\widehat{\mathbf{u}}}$, equation~\ref{eq:std}) of PBDW velocity $\widehat{}_{\text{PBDW}}$ for different values of signal-to-noise ratios $(\text{SNR-ho}, \text{SNR-he})\in \{(10, 0.5), (0.4, 0.1), (0.2, 0.05)\}$ and number of velocity modes $r_{\mathbf u}\in\{500, 1000, 2000\}$. We consider only test case $n=12$, $n_{\text{samples}}=100$ velocity fields were sampled to compute $\text{std}_{\widehat{\mathbf{u}}}$.}
  \label{fig:snr}
\end{figure}

\begin{figure}[!htp]
  \centering
  \includegraphics[width=0.65\textwidth]{img/uqp.pdf}\\
  \caption{Average and standard deviation of pressure field over $n_{\text{samples}}=100$ samples, obtained forwarding the uncertainty with \textit{pperom, sterom} or \textit{gnn-vp-pbdw} from $n_{\text{samples}}=100$ PBDW velocity samples $\widehat{\mathbf{u}}_{\text{PBDW}}$. The test/target geometry is fixed $n=12$. The results correspond to the upper left block of figure~\ref{fig:uq_p}.}
  \label{fig:uq_p}
\end{figure}


\begin{figure}[!htp]
  \centering
  \includegraphics[width=0.85\textwidth]{img/uq.pdf}\\
  \caption{$L^2$-relative error $\epsilon_{\widehat{p}}$ and $0.95$ confidence intervals of the pressure predicted on the target geometry $n=12$ with PPE and STE and a GNN model, for different values of signal-to-noise ratios $(\text{SNR-ho}, \text{SNR-he})$ and number of velocity modes $r_{\mathbf u}\in\{500, 1000, 2000\}$. The pressure modes are fixed at $r_p=1000$.}
  \label{fig:uq_cfd}
\end{figure}
