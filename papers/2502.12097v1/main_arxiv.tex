\documentclass[a4paper, 10pt ]{article}

\usepackage[a4paper,left=2cm,right=2cm,top=2.5cm,bottom=2.5cm]{geometry}
% %%%%%%%%%%%%%%%%%%%%%%%%%%%%%%%%%%%%%%%%%%%%%%%%%%%%%%%%%%%%%%%%%%
\usepackage[backend=biber,style=numeric]{biblatex} % Replace "numeric" with your preferred style
\usepackage{hyperref}
\hypersetup{colorlinks,linkcolor={blue},citecolor={blue},urlcolor={blue}}
\usepackage{subfig}
\usepackage{pgfplots}
\usepackage{pgfplotstable}
\usepackage{mathtools}
\usepackage{multicol}
\usepackage{comment}
\usepackage{booktabs}
\pgfplotsset{compat=1.5}
\usepackage{amssymb}
\usepackage{url}
\usepackage{bm}
\usepackage{enumitem}
% \usepackage{enumerate}
%\usepackage[doublespacing]{setspace}
\usepackage{siunitx}
\usepackage{pdfsync}
\usepackage{float}
\usepackage{tabularx}
\usepackage{array}
\usepackage{xspace}
\usepackage{tikz}
\usepackage{tikz-cd}
\usepackage{tikzsymbols}
\usepackage[labelfont=bf]{caption}
\usepackage[normalem]{ulem}
\usetikzlibrary{calc,trees,positioning,arrows,chains,shapes.geometric,%
    decorations.pathreplacing,decorations.pathmorphing,shapes,%
    matrix,shapes.symbols, decorations.markings, patterns,fit}

\usepackage{siunitx}

\usepackage{authblk}

\usepackage[draft,inline,marginclue]{fixme}
\usepackage{mathrsfs}
\usepackage{color, colortbl}
\usepackage{multirow}

% added packages
\usepackage{amsthm}
\usepackage{amsmath}
\usepackage{stmaryrd}
\usepackage[ruled,vlined,linesnumbered]{algorithm2e}
\usepackage{graphicx}
\usepackage{booktabs}
\usepackage{cleveref}
\usepackage{lineno}

% add yours
\FXRegisterAuthor{fr}{afr}{\color{blue}FR}
\newcommand{\FR}[1]{\frnote{#1}}

\FXRegisterAuthor{fg}{afg}{\color{orange}FG}
\newcommand{\fg}[1]{\fgnote{#1}}

\FXRegisterAuthor{ac}{aac}{\color{purple}AC}
\newcommand{\ac}[1]{\acnote{#1}}

\newcommand{\TODO}[1]{{\color{red}#1}}
\newcommand\deleted{\bgroup\markoverwith{\textcolor{red}{\rule[0.5ex]{2pt}{0.4pt}}}\ULon}

\addbibresource{biblio.bib} % Your .bib file

% Ensure DOIs are displayed
\ExecuteBibliographyOptions{doi=true, url=false}


%
\setlength\unitlength{1mm}
\newcommand{\twodots}{\mathinner {\ldotp \ldotp}}
% bb font symbols
\newcommand{\Rho}{\mathrm{P}}
\newcommand{\Tau}{\mathrm{T}}

\newfont{\bbb}{msbm10 scaled 700}
\newcommand{\CCC}{\mbox{\bbb C}}

\newfont{\bb}{msbm10 scaled 1100}
\newcommand{\CC}{\mbox{\bb C}}
\newcommand{\PP}{\mbox{\bb P}}
\newcommand{\RR}{\mbox{\bb R}}
\newcommand{\QQ}{\mbox{\bb Q}}
\newcommand{\ZZ}{\mbox{\bb Z}}
\newcommand{\FF}{\mbox{\bb F}}
\newcommand{\GG}{\mbox{\bb G}}
\newcommand{\EE}{\mbox{\bb E}}
\newcommand{\NN}{\mbox{\bb N}}
\newcommand{\KK}{\mbox{\bb K}}
\newcommand{\HH}{\mbox{\bb H}}
\newcommand{\SSS}{\mbox{\bb S}}
\newcommand{\UU}{\mbox{\bb U}}
\newcommand{\VV}{\mbox{\bb V}}


\newcommand{\yy}{\mathbbm{y}}
\newcommand{\xx}{\mathbbm{x}}
\newcommand{\zz}{\mathbbm{z}}
\newcommand{\sss}{\mathbbm{s}}
\newcommand{\rr}{\mathbbm{r}}
\newcommand{\pp}{\mathbbm{p}}
\newcommand{\qq}{\mathbbm{q}}
\newcommand{\ww}{\mathbbm{w}}
\newcommand{\hh}{\mathbbm{h}}
\newcommand{\vvv}{\mathbbm{v}}

% Vectors

\newcommand{\av}{{\bf a}}
\newcommand{\bv}{{\bf b}}
\newcommand{\cv}{{\bf c}}
\newcommand{\dv}{{\bf d}}
\newcommand{\ev}{{\bf e}}
\newcommand{\fv}{{\bf f}}
\newcommand{\gv}{{\bf g}}
\newcommand{\hv}{{\bf h}}
\newcommand{\iv}{{\bf i}}
\newcommand{\jv}{{\bf j}}
\newcommand{\kv}{{\bf k}}
\newcommand{\lv}{{\bf l}}
\newcommand{\mv}{{\bf m}}
\newcommand{\nv}{{\bf n}}
\newcommand{\ov}{{\bf o}}
\newcommand{\pv}{{\bf p}}
\newcommand{\qv}{{\bf q}}
\newcommand{\rv}{{\bf r}}
\newcommand{\sv}{{\bf s}}
\newcommand{\tv}{{\bf t}}
\newcommand{\uv}{{\bf u}}
\newcommand{\wv}{{\bf w}}
\newcommand{\vv}{{\bf v}}
\newcommand{\xv}{{\bf x}}
\newcommand{\yv}{{\bf y}}
\newcommand{\zv}{{\bf z}}
\newcommand{\zerov}{{\bf 0}}
\newcommand{\onev}{{\bf 1}}

% Matrices

\newcommand{\Am}{{\bf A}}
\newcommand{\Bm}{{\bf B}}
\newcommand{\Cm}{{\bf C}}
\newcommand{\Dm}{{\bf D}}
\newcommand{\Em}{{\bf E}}
\newcommand{\Fm}{{\bf F}}
\newcommand{\Gm}{{\bf G}}
\newcommand{\Hm}{{\bf H}}
\newcommand{\Id}{{\bf I}}
\newcommand{\Jm}{{\bf J}}
\newcommand{\Km}{{\bf K}}
\newcommand{\Lm}{{\bf L}}
\newcommand{\Mm}{{\bf M}}
\newcommand{\Nm}{{\bf N}}
\newcommand{\Om}{{\bf O}}
\newcommand{\Pm}{{\bf P}}
\newcommand{\Qm}{{\bf Q}}
\newcommand{\Rm}{{\bf R}}
\newcommand{\Sm}{{\bf S}}
\newcommand{\Tm}{{\bf T}}
\newcommand{\Um}{{\bf U}}
\newcommand{\Wm}{{\bf W}}
\newcommand{\Vm}{{\bf V}}
\newcommand{\Xm}{{\bf X}}
\newcommand{\Ym}{{\bf Y}}
\newcommand{\Zm}{{\bf Z}}

% Calligraphic

\newcommand{\Ac}{{\cal A}}
\newcommand{\Bc}{{\cal B}}
\newcommand{\Cc}{{\cal C}}
\newcommand{\Dc}{{\cal D}}
\newcommand{\Ec}{{\cal E}}
\newcommand{\Fc}{{\cal F}}
\newcommand{\Gc}{{\cal G}}
\newcommand{\Hc}{{\cal H}}
\newcommand{\Ic}{{\cal I}}
\newcommand{\Jc}{{\cal J}}
\newcommand{\Kc}{{\cal K}}
\newcommand{\Lc}{{\cal L}}
\newcommand{\Mc}{{\cal M}}
\newcommand{\Nc}{{\cal N}}
\newcommand{\nc}{{\cal n}}
\newcommand{\Oc}{{\cal O}}
\newcommand{\Pc}{{\cal P}}
\newcommand{\Qc}{{\cal Q}}
\newcommand{\Rc}{{\cal R}}
\newcommand{\Sc}{{\cal S}}
\newcommand{\Tc}{{\cal T}}
\newcommand{\Uc}{{\cal U}}
\newcommand{\Wc}{{\cal W}}
\newcommand{\Vc}{{\cal V}}
\newcommand{\Xc}{{\cal X}}
\newcommand{\Yc}{{\cal Y}}
\newcommand{\Zc}{{\cal Z}}

% Bold greek letters

\newcommand{\alphav}{\hbox{\boldmath$\alpha$}}
\newcommand{\betav}{\hbox{\boldmath$\beta$}}
\newcommand{\gammav}{\hbox{\boldmath$\gamma$}}
\newcommand{\deltav}{\hbox{\boldmath$\delta$}}
\newcommand{\etav}{\hbox{\boldmath$\eta$}}
\newcommand{\lambdav}{\hbox{\boldmath$\lambda$}}
\newcommand{\epsilonv}{\hbox{\boldmath$\epsilon$}}
\newcommand{\nuv}{\hbox{\boldmath$\nu$}}
\newcommand{\muv}{\hbox{\boldmath$\mu$}}
\newcommand{\zetav}{\hbox{\boldmath$\zeta$}}
\newcommand{\phiv}{\hbox{\boldmath$\phi$}}
\newcommand{\psiv}{\hbox{\boldmath$\psi$}}
\newcommand{\thetav}{\hbox{\boldmath$\theta$}}
\newcommand{\tauv}{\hbox{\boldmath$\tau$}}
\newcommand{\omegav}{\hbox{\boldmath$\omega$}}
\newcommand{\xiv}{\hbox{\boldmath$\xi$}}
\newcommand{\sigmav}{\hbox{\boldmath$\sigma$}}
\newcommand{\piv}{\hbox{\boldmath$\pi$}}
\newcommand{\rhov}{\hbox{\boldmath$\rho$}}
\newcommand{\upsilonv}{\hbox{\boldmath$\upsilon$}}

\newcommand{\Gammam}{\hbox{\boldmath$\Gamma$}}
\newcommand{\Lambdam}{\hbox{\boldmath$\Lambda$}}
\newcommand{\Deltam}{\hbox{\boldmath$\Delta$}}
\newcommand{\Sigmam}{\hbox{\boldmath$\Sigma$}}
\newcommand{\Phim}{\hbox{\boldmath$\Phi$}}
\newcommand{\Pim}{\hbox{\boldmath$\Pi$}}
\newcommand{\Psim}{\hbox{\boldmath$\Psi$}}
\newcommand{\Thetam}{\hbox{\boldmath$\Theta$}}
\newcommand{\Omegam}{\hbox{\boldmath$\Omega$}}
\newcommand{\Xim}{\hbox{\boldmath$\Xi$}}


% Sans Serif small case

\newcommand{\Gsf}{{\sf G}}

\newcommand{\asf}{{\sf a}}
\newcommand{\bsf}{{\sf b}}
\newcommand{\csf}{{\sf c}}
\newcommand{\dsf}{{\sf d}}
\newcommand{\esf}{{\sf e}}
\newcommand{\fsf}{{\sf f}}
\newcommand{\gsf}{{\sf g}}
\newcommand{\hsf}{{\sf h}}
\newcommand{\isf}{{\sf i}}
\newcommand{\jsf}{{\sf j}}
\newcommand{\ksf}{{\sf k}}
\newcommand{\lsf}{{\sf l}}
\newcommand{\msf}{{\sf m}}
\newcommand{\nsf}{{\sf n}}
\newcommand{\osf}{{\sf o}}
\newcommand{\psf}{{\sf p}}
\newcommand{\qsf}{{\sf q}}
\newcommand{\rsf}{{\sf r}}
\newcommand{\ssf}{{\sf s}}
\newcommand{\tsf}{{\sf t}}
\newcommand{\usf}{{\sf u}}
\newcommand{\wsf}{{\sf w}}
\newcommand{\vsf}{{\sf v}}
\newcommand{\xsf}{{\sf x}}
\newcommand{\ysf}{{\sf y}}
\newcommand{\zsf}{{\sf z}}


% mixed symbols

\newcommand{\sinc}{{\hbox{sinc}}}
\newcommand{\diag}{{\hbox{diag}}}
\renewcommand{\det}{{\hbox{det}}}
\newcommand{\trace}{{\hbox{tr}}}
\newcommand{\sign}{{\hbox{sign}}}
\renewcommand{\arg}{{\hbox{arg}}}
\newcommand{\var}{{\hbox{var}}}
\newcommand{\cov}{{\hbox{cov}}}
\newcommand{\Ei}{{\rm E}_{\rm i}}
\renewcommand{\Re}{{\rm Re}}
\renewcommand{\Im}{{\rm Im}}
\newcommand{\eqdef}{\stackrel{\Delta}{=}}
\newcommand{\defines}{{\,\,\stackrel{\scriptscriptstyle \bigtriangleup}{=}\,\,}}
\newcommand{\<}{\left\langle}
\renewcommand{\>}{\right\rangle}
\newcommand{\herm}{{\sf H}}
\newcommand{\trasp}{{\sf T}}
\newcommand{\transp}{{\sf T}}
\renewcommand{\vec}{{\rm vec}}
\newcommand{\Psf}{{\sf P}}
\newcommand{\SINR}{{\sf SINR}}
\newcommand{\SNR}{{\sf SNR}}
\newcommand{\MMSE}{{\sf MMSE}}
\newcommand{\REF}{{\RED [REF]}}

% Markov chain
\usepackage{stmaryrd} % for \mkv 
\newcommand{\mkv}{-\!\!\!\!\minuso\!\!\!\!-}

% Colors

\newcommand{\RED}{\color[rgb]{1.00,0.10,0.10}}
\newcommand{\BLUE}{\color[rgb]{0,0,0.90}}
\newcommand{\GREEN}{\color[rgb]{0,0.80,0.20}}

%%%%%%%%%%%%%%%%%%%%%%%%%%%%%%%%%%%%%%%%%%
\usepackage{hyperref}
\hypersetup{
    bookmarks=true,         % show bookmarks bar?
    unicode=false,          % non-Latin characters in AcrobatÕs bookmarks
    pdftoolbar=true,        % show AcrobatÕs toolbar?
    pdfmenubar=true,        % show AcrobatÕs menu?
    pdffitwindow=false,     % window fit to page when opened
    pdfstartview={FitH},    % fits the width of the page to the window
%    pdftitle={My title},    % title
%    pdfauthor={Author},     % author
%    pdfsubject={Subject},   % subject of the document
%    pdfcreator={Creator},   % creator of the document
%    pdfproducer={Producer}, % producer of the document
%    pdfkeywords={keyword1} {key2} {key3}, % list of keywords
    pdfnewwindow=true,      % links in new window
    colorlinks=true,       % false: boxed links; true: colored links
    linkcolor=red,          % color of internal links (change box color with linkbordercolor)
    citecolor=green,        % color of links to bibliography
    filecolor=blue,      % color of file links
    urlcolor=blue           % color of external links
}
%%%%%%%%%%%%%%%%%%%%%%%%%%%%%%%%%%%%%%%%%%%


\begin{document}



\title{
  % Robust shape registration method for aortic geometries and applications to data assimilation with graph neural network}
Data assimilation performed with robust shape registration and graph neural networks: application to aortic coarctation}

\author[1]{Francesco~Romor\footnote{francesco.romor@wias-berlin.de}}
\author[2]{Felipe Galarce}
\author[3]{Jan Brüning}
\author[3]{Leonid Goubergrits}
\author[1]{Alfonso Caiazzo\footnote{alfonso.caiazzo@wias-berlin.de}}

\affil[1]{Weierstraß Institute, Mohrenstr. 39 10117, Berlin, Germany}
\affil[2]{School of Civil Engineering, Pontificia Universidad Católica de Valparaíso, Valparaíso, Chile.}
\affil[3]{Institute of Computer-assisted Cardiovascular Medicine, Deutsches Herzzentrum der Charité, Augustenburger Platz 1, 13353 Berlin, Germany}

\maketitle

\begin{abstract}
  Image-based, patient-specific modelling of hemodynamics can improve diagnostic capabilities and provide complementary insights to better understand the hemodynamic treatment outcomes. However, computational fluid dynamics simulations remain relatively costly in a clinical context. Moreover, projection-based reduced-order models and purely data-driven surrogate models struggle due to the high variability of anatomical shapes in a population. A possible solution is shape registration: a reference template geometry is designed from a cohort of available geometries, which can then be diffeomorphically mapped onto it. This provides a natural encoding that can be exploited by machine learning architectures and, at the same time, a reference computational domain in which efficient dimension-reduction strategies can be performed. We compare state-of-the-art graph neural network models with recent data assimilation strategies for the prediction of physical quantities and clinically relevant biomarkers in the context of aortic coarctation.
\end{abstract}

\tableofcontents
% \listoffixmes


\section{Introduction}
\label{sec:intro}
\section{Introduction}


\begin{figure}[t]
\centering
\includegraphics[width=0.6\columnwidth]{figures/evaluation_desiderata_V5.pdf}
\vspace{-0.5cm}
\caption{\systemName is a platform for conducting realistic evaluations of code LLMs, collecting human preferences of coding models with real users, real tasks, and in realistic environments, aimed at addressing the limitations of existing evaluations.
}
\label{fig:motivation}
\end{figure}

\begin{figure*}[t]
\centering
\includegraphics[width=\textwidth]{figures/system_design_v2.png}
\caption{We introduce \systemName, a VSCode extension to collect human preferences of code directly in a developer's IDE. \systemName enables developers to use code completions from various models. The system comprises a) the interface in the user's IDE which presents paired completions to users (left), b) a sampling strategy that picks model pairs to reduce latency (right, top), and c) a prompting scheme that allows diverse LLMs to perform code completions with high fidelity.
Users can select between the top completion (green box) using \texttt{tab} or the bottom completion (blue box) using \texttt{shift+tab}.}
\label{fig:overview}
\end{figure*}

As model capabilities improve, large language models (LLMs) are increasingly integrated into user environments and workflows.
For example, software developers code with AI in integrated developer environments (IDEs)~\citep{peng2023impact}, doctors rely on notes generated through ambient listening~\citep{oberst2024science}, and lawyers consider case evidence identified by electronic discovery systems~\citep{yang2024beyond}.
Increasing deployment of models in productivity tools demands evaluation that more closely reflects real-world circumstances~\citep{hutchinson2022evaluation, saxon2024benchmarks, kapoor2024ai}.
While newer benchmarks and live platforms incorporate human feedback to capture real-world usage, they almost exclusively focus on evaluating LLMs in chat conversations~\citep{zheng2023judging,dubois2023alpacafarm,chiang2024chatbot, kirk2024the}.
Model evaluation must move beyond chat-based interactions and into specialized user environments.



 

In this work, we focus on evaluating LLM-based coding assistants. 
Despite the popularity of these tools---millions of developers use Github Copilot~\citep{Copilot}---existing
evaluations of the coding capabilities of new models exhibit multiple limitations (Figure~\ref{fig:motivation}, bottom).
Traditional ML benchmarks evaluate LLM capabilities by measuring how well a model can complete static, interview-style coding tasks~\citep{chen2021evaluating,austin2021program,jain2024livecodebench, white2024livebench} and lack \emph{real users}. 
User studies recruit real users to evaluate the effectiveness of LLMs as coding assistants, but are often limited to simple programming tasks as opposed to \emph{real tasks}~\citep{vaithilingam2022expectation,ross2023programmer, mozannar2024realhumaneval}.
Recent efforts to collect human feedback such as Chatbot Arena~\citep{chiang2024chatbot} are still removed from a \emph{realistic environment}, resulting in users and data that deviate from typical software development processes.
We introduce \systemName to address these limitations (Figure~\ref{fig:motivation}, top), and we describe our three main contributions below.


\textbf{We deploy \systemName in-the-wild to collect human preferences on code.} 
\systemName is a Visual Studio Code extension, collecting preferences directly in a developer's IDE within their actual workflow (Figure~\ref{fig:overview}).
\systemName provides developers with code completions, akin to the type of support provided by Github Copilot~\citep{Copilot}. 
Over the past 3 months, \systemName has served over~\completions suggestions from 10 state-of-the-art LLMs, 
gathering \sampleCount~votes from \userCount~users.
To collect user preferences,
\systemName presents a novel interface that shows users paired code completions from two different LLMs, which are determined based on a sampling strategy that aims to 
mitigate latency while preserving coverage across model comparisons.
Additionally, we devise a prompting scheme that allows a diverse set of models to perform code completions with high fidelity.
See Section~\ref{sec:system} and Section~\ref{sec:deployment} for details about system design and deployment respectively.



\textbf{We construct a leaderboard of user preferences and find notable differences from existing static benchmarks and human preference leaderboards.}
In general, we observe that smaller models seem to overperform in static benchmarks compared to our leaderboard, while performance among larger models is mixed (Section~\ref{sec:leaderboard_calculation}).
We attribute these differences to the fact that \systemName is exposed to users and tasks that differ drastically from code evaluations in the past. 
Our data spans 103 programming languages and 24 natural languages as well as a variety of real-world applications and code structures, while static benchmarks tend to focus on a specific programming and natural language and task (e.g. coding competition problems).
Additionally, while all of \systemName interactions contain code contexts and the majority involve infilling tasks, a much smaller fraction of Chatbot Arena's coding tasks contain code context, with infilling tasks appearing even more rarely. 
We analyze our data in depth in Section~\ref{subsec:comparison}.



\textbf{We derive new insights into user preferences of code by analyzing \systemName's diverse and distinct data distribution.}
We compare user preferences across different stratifications of input data (e.g., common versus rare languages) and observe which affect observed preferences most (Section~\ref{sec:analysis}).
For example, while user preferences stay relatively consistent across various programming languages, they differ drastically between different task categories (e.g. frontend/backend versus algorithm design).
We also observe variations in user preference due to different features related to code structure 
(e.g., context length and completion patterns).
We open-source \systemName and release a curated subset of code contexts.
Altogether, our results highlight the necessity of model evaluation in realistic and domain-specific settings.






\section{Forward computational hemodynamics}
\label{sec:setting}
\subsection{Statistical shape modelling of patients with aortic coarctation}
\label{subsec:ssm}

The data used in this study were obtained from a cohort of patients with coarctation of the aorta (CoA), augmented synthetically using statistical shape models (SSM). The procedure is briefly outlined below. For the detailed methodology, we refer the reader, e.g. to 
~\cite{goubergrits2022ct, thamsen2021synthetic,thamsen2020unsupervised,versnjak2024deep, yevtushenko2021deep}.

The initial database contained $228$ surfaces acquired from 3D steady-state free-precession (SSFP) magnetic resonance imaging (MRI)
(acquired resolution $\SI{2}{mm}\times \SI{2}{mm}\times \SI{4}{mm}$, reconstructed resolution used for surface reconstruction $\SI{1}{mm}\times \SI{1}{mm}\times \SI{2}{mm}$) and segmented with \texttt{ZIB Amira}~\cite{stalling2005amira}. In total, 106 CoA patients (32 female) and 85 healthy subjects were acquired (25 female). 
For 37 (8 female) of the 106 CoA patients also post-treatment image data were available, thus increasing the database. The median age was 21 years with interquartile range (IQR) of 32 years.
The considered region of interest comprises the vessel surface of aortic arch up to the thoracic aorta (TA), including three main branches 
and the corresponding boundary surfaces (brachiocephalic artery, BCA, left common carotid artery, LCCA, left subclavian artery, LSA). Few available cases with two or four branches of the aortic arch were not included into the database.

Additionally, pointwise linear centerlines for the aorta and the three branching vessels have been obtained along with the radii of the inscribed spheres using the vascular modelling toolkit \texttt{VMTK}~\cite{antiga2008image} (see the sketch in figure~\ref{fig:clustergeometries} (left) for an example).
%
\begin{figure}[!htp]
  \centering
  \includegraphics[width=0.9\textwidth]{img/cluster_g.pdf}
  \caption{
  \textbf{Left:} Sketch of the centerline encoding (points $p_i$ and radius $r_i$ of the associated inscribed sphere, for $i\in\{1,\dots,390\}$). 
  \textbf{Center:} Clustering of the considered training ($n=724$) and test ($n=52$) shapes using t-SNE
  with the Euclidean distance on a geometrical encoding, based on the distance of each point from the centerline after shape registration (see section~\ref{subsec:sml_correlations} for details).
  \textbf{Right:} Visualization of the furthest shapes (top, test cases $34$, $50$, and $44$) and the closest ones (bottom, test cases $3$, $21$, $15$) according to the metric in the center plot.}
  \label{fig:clustergeometries}
\end{figure}

The procedure resulted in $300$ centerline points for the aorta and $30$ centerline points for each branching vessel, for a total of $n_{\text{cntrl}}=390$ points and corresponding radii of inscribed spheres for each considered shape.
These data allow to encode the morphology of each shape into a matrix $S_{\text{SSM}}\in\mathbb{R}^{n_{\text{cntrl}}\times (3+1)}$ containing the spatial coordinates of the $n_{\text{cntrl}}$ centerline points and the associated radii. Closed triangulated surfaces are then generated from this skeletal representation~\cite{yevtushenko2021deep}. 
Each geometry is rigidly moved towards the mean shape $S_{\rm mean}$, %$\bar{\Tilde{S}}\in\mathbb{R}^{n_{\text{cntrl}}\times (3+1)}$, 
minimizing the least-squares distance between points with the closest point algorithm using \textsf{mcAlignPoints} package of the \texttt{ZIB Amira} software.
%
No scaling is performed. New shapes 
are generated through SSM with Principal Component Analysis (PCA):
\begin{equation*}
  \Tilde{S}_{\text{SSM}} = S_{\rm mean}  + P_{\text{SSM}}\ b_{\text{SSM}}, 
\end{equation*}
where $P_{\text{SSM}}\in\mathbb{R}^{(n_{\text{cntrl}}\times (3+1))\times k}$ contains $k>0$ truncated modes of the correlation matrix of the training shapes used for SSM, and 
$b_{\text{SSM}}\in\mathbb{R}^k$ is the vector of coefficients. For the SSM development only pre-treatment CoA shapes ($93$ cases) and healthy aorta ($65$ cases) were used.


A database of more than $10000$ shapes is generated sampling $b_{\text{SSM}}$ from a normal distribution. Unrealistic shapes, e.g., containing self-intersection, small vessel radius (below $1.0$ mm), or 
excessive degree of stenosis (less than $20\%$ or greater than $80\%$) have been removed. 
%
As aortic length and aortic inlet diameter are correlated with age, when age ranges deduced from these two morphological parameters did not overlap, the corresponding shapes were discarded. Further shapes have been removed performing a preliminary CFD analysis of peak systole flow using \texttt{STAR-CCM+}~\cite{yevtushenko2021deep}, discarding 
those resulting in unphysical quantities of interest.

This procedure resulted in a cleaned database of $1312$ (real and synthetic) shapes, represented by triangulated surface meshes, centerline points, and centerline radii.
Along the curse of this study, $437$ additional geometries have been removed based on the results of time dependent simulations (see section \ref{ssec:blood_flow}) and further $99$ due to 
inaccurate registrations (see section \ref{sec:registration}). The remaining $776$ cases have been split in $724$ training and $52$ testing shapes.
Figure \ref{fig:clustergeometries}, center and right plots, show qualitatively the extent of the training and test datasets, based on a 
T-distributed Stochastic Neighbor Embedding~\cite{van2008visualizing} (t-SNE) with the Euclidean distance on a geometrical encoding of the shapes
that relies on the shape registration map (further details will be given in section~\ref{subsec:sml_correlations}).


\subsection{Blood flow modelling}\label{ssec:blood_flow}
Let us denote with $\Omega \subset \mathbb R^3$ the computational domain representing a generic shape from the considered dataset, whose boundaries can be decomposed as
\begin{equation}\label{eq:omega_bnd}
\partial \Omega = \Gamma_{\text{wall}} \cup \Gamma_{\rm in} \cup \left( \bigcup_{i=1}^4 \Gamma_i \right),
\end{equation}
distinguishing between the vessel \text{wall} $\Gamma_{wall}$, the inlet boundary $\Gamma_{\rm in}$, and the four outlet boundaries (BCA, LCCA, LSA, TA), as depicted in figure \ref{fig:domain}. We assume that the blood flow in the considered vessels behaves as an incompressible Newtonian fluid and thus describes the hemodynamics via the incompressible Navier--Stokes equations for the velocity $\bu:\Omega \to \mathbb R^3$ and the pressure fields $p:\Omega \to \mathbb R$:
\begin{equation}\label{eq:3dnse}
\left\{
\begin{aligned}
\rho \partial_t \mathbf{u}+\rho \mathbf{u}\cdot\nabla\mathbf{u}+\mu\Delta\mathbf{u}-\nabla p=\mathbf{0},\qquad&\text{in}\ \Omega,\\
    \nabla\cdot\mathbf{u}=0,\qquad&\text{in}\ \Omega,
\end{aligned}
\right.
\end{equation}
%
where $\rho=\SI{1.06e3}{\kilogram\per\meter^3}$ stands for the blood density, and $\mu=\SI{3.5e-3}{\second\cdot\pascal}$ is the dynamic viscosity. 


Equations~\eqref{eq:3dnse} are complemented by homogeneous Dirichlet boundary conditions for the velocity on $\Gamma_{\rm \text{wall}}$, i.e., neglecting  fluid-structure interactions between the blood flow and the vessel wall, by a Dirichlet boundary condition on $\Gamma_{\rm in}$, imposing a parabolic flow profile at the inlet~\cite{katz2023impact}, and by lumped parameter models on the four outlet boundaries, i.e., 
\begin{equation}
  \label{eq:3dnse-bc}
\left\{
\begin{aligned}
\mathbf{u}&=\mathbf{u}_{\rm in},\qquad &&\text{on}\ \Gamma_{\rm in},\\
\mathbf{u}&=\mathbf{0},\qquad &&\text{on}\ \Gamma_{\text{wall}},\\
    -pI+\mu\left(\nabla\mathbf{u}+\nabla\mathbf{u}^T\right)&=-P_i\mathbf{n},\qquad &&\text{on}\ \Gamma_i,\ i\in\{1,2,3,4\}.
\end{aligned}
\right.
\end{equation}
In the last equation, $\mathbf{n}$ denotes the outward normal vector to the fluid boundary and $P_i$ stands for an approximation of the outlet pressure imposed on the boundary $\Gamma_i$, which is evaluated as a function of the boundary flow rates $Q_i:=\int_{\Gamma_i} \mathbf{u}\cdot\mathbf{n}$, $i=\in\{1,2,3,4\}$, 
via a 3-elements (RCR) Windkessel model~\cite{westerhof2009arterial}
\begin{equation}\label{eq:wk-rcr-i}
\left\{
\begin{aligned}
C_{d,i}\frac{d\pi_i}{dt}+\frac{\pi_i}{R_{d,i}}=Q_i,\qquad\qquad\qquad\qquad\ \;\,\quad&\text{on}\ \Gamma_i,\ i\in\{1,2,3,4\},\\
   P_i=R_{p,i}Q_i+\pi_i,\qquad\qquad\qquad\qquad\ \;\,\quad&\text{on}\ \Gamma_i,\ i\in\{1,2,3,4\},\\
\end{aligned}
\right.
\end{equation}
depending on an auxiliary \textit{distal} pressure $\pi_i$, a \textit{proximal} resistance $R_p$ (modeling the resistance to the flow of the arteries close to the open boundary), a \textit{distal} resistance $R_d$ (modeling the downstream resistance of the rest of the cardiovascular system), and a \textit{capacitance} $C_d$ (modeling the compliance of the cardiovascular system).
The tuning of the Windkessel parameters will be discussed in detail in section \ref{ssec:bc-calibration}.
%
Peak inlet flow rates for each shape were provided as part of the patient cohort data for each considered shape. These values  were adjusted to match a parabolic shape on the inlet boundary, and multiplied by a time dependent function to obtain the inlet boundary condition over time.

\begin{figure}[!htp]
  \centering
  \includegraphics[width=0.4\textwidth, trim={0 0 0 20}, clip]{img/domain.pdf}
  \caption{Example of a computational domain. A parabolic profile is imposed on the inlet boundary (ascending aorta), based on a given peak flow rate. 
  Windkessel models are used at the outlets (thoracic aorta, brachiocephalic artery, left common carotid artery, and left subclavian artery).}
  \label{fig:domain
}
  \label{fig:domain}
\end{figure}

\subsection{Calibration of boundary conditions across the cohort of patients}\label{ssec:bc-calibration}
The Windkessel parameters might have a considerable impact on the solution and the calibration typically depends on the flow regime of interest, on the
particular anatomical details, and on available data. For the purpose of this study, we opted for an approach driven by the flow split across
the different branches.

We introduce the total resistances
$R_i := R_{p,i} + R_{d,i}$ and the equivalent
\textit{systemic} resistance $R_S^{-1}  := \sum_{i=1}^N R_i^{-1}$. Neglecting the contribution of the 3D domain to the total resistance, the quantities
$R_i/ R_S$, for $i=1,2,3,4$, can be used to control the \textit{flow split}, i.e., the ratio of the inlet flow $Q_{\rm in}$ that flows, on average, through each outlet.

We have tuned the template geometry's parameters considering a flow split of $50\%$ for the BCA, and of $25\%$ for LCCA and LSA, as in~\cite{katz2023impact}. For the systematic calibration of the Windkessel parameters on other geometries, we consider a model for the average flow split based on the following steps:
\begin{enumerate}[itemsep=2pt, left=0pt, labelsep=5pt]
  \item A rescaling of the systemic resistance, based on the patient specific inlet,
  \begin{equation*}%\label{eq:calibrated_RS}
  R_S :=  \frac{\hat Q_{\rm in}}{Q_{\rm in}} \hat R_S,
  \end{equation*}
  \item A shape-specific flow split based on the reference mean velocities and the outlet areas of the new patient,
  \begin{equation*}%\label{eq:calibrated_sigma_i}
      \sigma_i := \hat Q_i \frac{A_i}{\hat A_i} \frac{1}{\sum_{j=1}^4 \hat Q_j \frac{A_j}{\hat A_j}} = \frac{\hat u^{\mathrm{mean}}_i A_i }{ \sum_{j=1}^4 \hat u^{\mathrm{mean}}_j A_j},
  \end{equation*}
      \item The approximation of the patient-specific total resistance 
      \begin{equation*}
        R_i = \sigma_i R_S,
      \end{equation*}
      \item A splitting between proximal and distal resistance (the same for all  patients),
  \begin{equation*}%\label{eq:R_p-and-R_d}
  R_{p,i}  = 0.1 R_i,\ R_{d,i}=0.9 R_i\,.
  \end{equation*}    
\end{enumerate}
 Finally, capacitances are defined proportionally to the area of the outlet boundaries, i.e., 
\begin{equation*}
  C_i = \frac{A_i}{A_{\rm tot}} C_{\rm tot},
\end{equation*}
as a fraction of the total capacitance $C_{\rm tot} = 10^{-8}$ (the same for all shapes).

The distributions of the total resistances, distal capacities, and outlets areas, for the complete shape database, are shown in figure~\ref{fig:param_distr}.
\begin{figure}[!htp]
  \centering
  \includegraphics[width=0.49\textwidth]{img/capacity.pdf}
  \includegraphics[width=0.49\textwidth]{img/area.pdf}\\
  \includegraphics[width=0.49\textwidth]{img/resistance.pdf}
  \caption{Distribution of Windkessel parameters $C_{d,i}$ (distal capacity), $R_i=R_{d,i}+R_{p,i}$ (Total resistance), and $A_i$ (boundary area) for the different outlets, across the training and test datasets.}
  \label{fig:param_distr}
\end{figure}


\begin{rmk}[Calibration based on flow split]
The main motivation behind this approach is the fact that the flow split can be experimentally measured non-invasively on the different sections, or inferred according 
to existing literature data and patient anatomy. Moreover, using the flow split as parameter allows modelling different physiological (rest/exercise)
or pathological (e.g., obstruction of vessels downstream) conditions, and can be used to enrich the solution dataset depending on the context of interest.
%
\end{rmk}


\subsection{Synthetic dataset of aortic shapes and numerical simulations}
\label{ref:numsim}
We solve numerically system \eqref{eq:3dnse}-\eqref{eq:3dnse-bc} for each of the $776$ considered geometries, discretizing the corresponding volume with a tetrahedral mesh reated from the original surface shape
and imposing shape specific boundary conditions as described in section \eqref{ssec:bc-calibration}. 
%
We use stabilized equal-order linear finite elements for velocity and pressure and a BDF2 time marching scheme, 
with a semi-implicit treatment of the non-linear convective term and of the VMS turbulence model~\cite{bazilevs2007variational, forti2015semi}.
Further details on the discretization and on the numerical method are provided in  appendix \ref{appendix:weak}.
%
The ODEs \eqref{eq:wk-rcr-i} are solved using an implicit Euler scheme, and the coupling at the boundary is implemented explicitly, i.e., using the boundary pressures at the previous time iteration to impose Neumann boundary conditions on each outlet. Some velocity snapshots are shown in figure~\ref{fig:snapsu}.
%
The solver is implemented in the computational framework \texttt{lifex-cfd}~\cite{AFRICA2024109039}, based on the open-source library \texttt{deal.II}~\cite{arndt2022deal}.
%
Simulations have been run for five hearth beats, only the last period is considered, in order to ensure a quasi-periodic state.

The results (figure~\ref{fig:flow_and_pressure}) show a rather uniform distribution of flow and pressure values at boundaries across the dataset.
%
\begin{figure}[!htp]
  \centering
  \includegraphics[width=0.9\textwidth]{img/flows_and_pressure.pdf}
  \caption{\textbf{Top: } Numerical results for te flow at inlet (AAo, with opposite sign) and outlets (BCA, LCCA, LSA, TA) boundaries. 
  \textbf{Bottom: }  Numerical results for the pressure at inlet (AAo) and outlets (BCA, LCCA, LSA, TA) boundaries. The $25$-th and $75$-th percentile across all the $724$ and $52$ training and test data are shown. The red vertical bands correspond to the time window $t\in[0.05s, 0.25s]$ which is the focus of our data assimilation studies, see remark~\ref{rmk:timewindow}.}
\label{fig:flow_and_pressure}
\end{figure}



\section{Registration with ResNet-LDDMM}
\label{sec:registration}
\subsection{Large deformation diffeomorphic metric mapping}\label{ssec:resnet-lddmm-intro}
The registration, or image matching problem, consists in smoothly mapping a \textit{source} (or template) image into a \textit{target} image. 
Our approach is based on the so-called Large Deformation Diffeomorphic Metric Mapping (LDDMM), in which
the map between the source and the target is sought as a diffeomorphic flow of an ODE~\cite{bruveris2017completeness,dupuis1998variational}.
In this section, we present the formulation of the method from a continuous perspective and the main theoretical background. 
The specific implementation to the case of three-dimensional
meshes of aortic shapes will be described in detail in section \ref{subsec:resnetlddmm}.

Formally, let us consider the source and target images defined by the characteristic functions $\chi_S:\mathbb{R}^3\rightarrow\mathbb{R}$ and $\chi_T:\mathbb{R}^3\rightarrow\mathbb{R}$, respectively.
We assume that both images are contained in an open bounded set, i.e., 
\begin{equation*}
\text{supp}(\chi_S)\cup\text{supp}(\chi_T)\subset G\subset\mathbb{R}^3.
\end{equation*}

The goal of LDDMM is to find a diffeomorphic map between the source and the target as a one-parameter group of diffeomorphisms  $\{\phi(t, \cdot)\}_{t\in I}$, $I :=[0,1]$, defined
as the flow of an ODE depending on a vector field $f:I\times\mathbb{R}^3\rightarrow\mathbb{R}^3$, i.e., such that
\begin{equation}\label{eq:phi_t_lddmm}
  d_t \phi(t, \x) = f(t, \phi(t, \x)),\qquad \phi(0, \x) = \x,\qquad\forall\x\in G.
\end{equation}
In particular, $\chi_S\left(\phi(0,\cdot)\right) = \chi_S$ coincides with the source, and  $\chi_S\left(\phi(1,\cdot)\right)$ is the mapped image.
%
The problem is addressed in an optimal control framework, where the control is the vector field $f$, minimizing the matching error between the mapped source and the target.
%
The following result ensures the existence of an optimal solution in the considered setting for arbitrary dimension $d$. 


\begin{theorem}[LDDMM registration, theorem 3.1~\cite{dupuis1998variational}]
  \label{def:regpb}
  Assume that $S:G\rightarrow\mathbb{R}$ and $T:G\rightarrow\mathbb{R}$ are two bounded measurable functions, and that $S$ 
 \textit{(e.g., the source image)} is continuous almost everywhere. 
 Let us suppose that $f\in L^2(I, H^s)$, with $s>d/2 + 1$ (the Sobolev embedding implies $f\in \mathcal{C}^{1, \alpha}(I, \mathcal{C}^{1, \alpha})$ with $\alpha=s - \tfrac{d}{2} - 1>0$), then the following registration problem
  \begin{equation}\label{eq:reg_miminization}
    \min_{f\in L^2(I, H^s)} \int_{0}^{1}\lVert f(t, \cdot)\rVert^2_{H^s}\ dt + \int_G |S(\phi(1, \x))-T(\x)|^2\ d\x
  \end{equation}
  has a minimizer.
\end{theorem}
%

In particular, theorem \ref{def:regpb} holds for signed distance functions, as well as for the particular case of characteristic functions. % $\chi_S$ and $\chi_T$.
The regularity condition $f\in L^2(I, H^s)$, with $s>d/2 + 1$ from the Sobolev embedding theorem, is sufficient to obtain that $\{\phi(t, \cdot)\}_{t\in I}\subset\text{Diff}^1(G)$ is a one-parameter group of diffeomorphisms. 
%
In general,  for a generic Hilbert space $\mathcal{H}$ such that $\mathcal{H}\hookrightarrow C^1_b(\mathbb{R}^d)$, the group $\mathcal{G}_\mathcal{H}(\mathbb{R}^d)$ of diffeomorphisms generated by vector fields in $L^2(I, \mathcal{H})$ is only strictly contained in $\text{Diff}^1(\mathbb{R}^d)$.
However, if $s>d/2+1$, it holds (see ~\cite{bruveris2017completeness}) 
\[
\mathcal{G}_{H^s}=\mathcal{D}^s(\mathbb{R}^d) =\{\phi\in\text{Diff}^1(\mathbb{R}^d) \mid \phi\in\text{Id}+H^s(\mathbb{R}^d,\mathbb{R}^d)\}.
\]


Using the universal approximation theorems of neural networks, one can infer the existence of neural networks that approximate the vector field $f$ arbitrarily well, 
with convergence estimates depending on its regularity.
\begin{theorem}[Existence of ResNet-LDDMM vector field]
  \label{theo:existreg}
Let  $f:I\times\mathbb{R}^d\rightarrow\mathbb{R}^d$ be a minimizer for problem \eqref{eq:reg_miminization}, let
$\{\phi(t, \cdot)\}_{t\in I}$ the corresponding group of diffeomorphisms \eqref{eq:phi_t_lddmm}, and let $\epsilon>0$. 
\begin{itemize}
\item[(i)]There exists a neural network (NN) $f_\epsilon: I\times \mathbb{R}^d\rightarrow  I\times \mathbb{R}^d$, $f_\epsilon\in\mathcal{C}^{0, 1}(I, \mathcal{C}^{0, 1})$,  with ReLU activations and one hidden layer, such that:
  \begin{equation*}
    \lVert \phi_\epsilon - \phi\rVert_{L^2(I, L^2)}\rVert \leq C_1 \lVert f_\epsilon - f\rVert_{L^2(I, L^2)} \leq \epsilon,
  \end{equation*}
  where  $\{\phi_\epsilon(t, \cdot)\}_{t\in I}$ satisfy
 \begin{equation*}
    d_t \phi_\epsilon(t, \x) = f_\epsilon(t, \phi_\epsilon(t, \x)),\qquad \phi_\epsilon(0, \x) = \x,\qquad\forall\x\in G,
  \end{equation*}
and $C_1>0$ is independent of $\epsilon$.
  
\item[(ii)] Let $f_N\in\mathcal{C}^{2, 1}(I, \mathcal{C}^{2, 1})$ denote a deep NN, with a  fixed number of layers, ReCU activations, $N$ non-zero weights, and
let $\{\phi_N(t, \cdot)\}_{t\in I}$ be the flow of the corresponding ODE, i.e., 
  \begin{equation*}
    d_t \phi_N(t, \x) = f_N(t, \phi_N(t, \x)),\qquad \phi_N(0, \x) = \x,\qquad\forall\x\in G.
  \end{equation*}
For any $n\in\{0, 1\}$, and $\forall m\in\mathbb{N}$, $m\geq n+1$, the following estimate holds
  \begin{equation*}
    \lVert \phi_N - \phi\rVert_{L^2(I, H^n)}\rVert \leq C_2 \lVert f_N - f\rVert_{L^2(I, H^n)} \leq C_3\cdot N^{-\frac{m-n}{2(d+1)}},
  \end{equation*}
  with positive constants $C_2, C_3>0$ independent of $N$.
  \end{itemize}
\end{theorem}

\begin{proof}
See appendix~\ref{appendix:convergence}.
\end{proof}

In the practical implementation (see section~\ref{subsec:resnetlddmm}) we will 
consider only autonomous vector fields $f:\mathbb{R}^3\rightarrow \mathbb{R}^3$, i.e., that do not depend on $t\in I$,  and
NNs with ReLU activations, $7$ hidden layers, and where the first layer's input is augmented with Random Fourier Features (RFF).
%
Moreover, the matching error between mapped source and target images will be measured using a metric based on the Chamfer distance, which is a natural choice
when considering discrete point clouds, rather than the $L^2$-norm between characteristic functions used in \eqref{eq:reg_miminization}.
Up to our best knowledge, general existence results using the Chamfer distance are not available. However, it can be
shown that the solutions to a discrete version of~\eqref{eq:reg_miminization} with the discrete $L^2$-norm as discrepancy metrics convergence to the solution of the continuous registration problem (see appendix~\ref{appendix:convergence}). This result motivates also the usage of a multigrid optimization.

\subsection{Multigrid ResNet-LDDMM for aortic shape meshes}
\label{subsec:resnetlddmm}

\paragraph{Preliminaries} 
In this section, we address the shape registration between 3d computational domains representing different aortic surfaces, discretized by triangular meshes.  
%
We assume that, for all considered shapes, it is possible to subdivide each surface mesh into an inlet boundary $\Gamma_{\rm in}$, four outlet boundaries $\Gamma_1, \hdots,\Gamma_4$, 
and the wall boundary $\Gamma_{wall}$ (see equation \eqref{eq:omega_bnd}).
%
All shapes have are also characterized by a piecewise centerline $l\in\mathbb{R}^{n_{\text{cntrl}}\times 3}$ with a main branch (ascending and thoracic aorta) and three minor branches.
By construction (see section \ref{subsec:ssm}), we also assume that all centerlines have the same number of points $n_{\text{cntrl}}=390$.
%
These assumptions are motivated by the fact that all computational domains shall provide suitable discretizations of the physical model of interest 
(Equations \eqref{eq:3dnse}, with boundary conditions \eqref{eq:3dnse-bc}).

However, \textit{no assumptions} are made on the number of vertices, edges, or faces in each mesh nor on the connectivity of the elements.
In what follows, a shape $\mathcal S$ will be generally defined by the corresponding surface mesh, i.e., a pair $(X_{\mathcal S}, F_{\mathcal S})$, where
%
\begin{itemize}
  \item $\XS=\{\mathbf x_i\}_{i=1}^{n_{p,\mathcal S}}\subset\mathbb{R}^3$ is a point cloud with cardinality $n_{p,\mathcal S}>0$.
  \item $\FS = \{(a_i, b_i, c_i), a_i \neq b_i, a_i \neq c_i, b_i \neq c_i \}_{i=1}^{n_{f,\mathcal S}}\subset\mathbb{N}^3$, is a set of triangular faces with cardinality $n_{f,\mathcal S}>0$, where the element $(a_i, b_i, c_i)$ corresponds to the face 
defined by three (different) vertices $(\mathbf x_{a_i}, \mathbf x_{b_i}, \mathbf x_{c_i})$,
\end{itemize}
and its centerline $l_{\mathcal S}\in\mathbb{R}^{n_{\text{cntrl}}\times 3}$ ($n_{\text{cntrl}} = 90$, equal amount for all shapes).
We also introduce the set 
\begin{equation*}
\NS := \{\mathbf n_i^{\mathcal S}, i=1,\hdots,n_{p,\mathcal S}\} 
\end{equation*}
of normal vectors to the mesh vertices defined, for each $\mathbf x_i\in \XS$, as the linear combination of 
the normals of all faces adjacent to the vertex $\mathbf x_i$, weighted by the arccosine of the angles corresponding to $\mathbf x_i$ in the respective adjacent triangular faces, and renormalized such that $\lVert \mathbf n_i^{\mathcal S}\rVert=1$.

%
Let us also introduce, for two arbitrary point clouds $X_{\mathcal{S}}$ and $X_{\mathcal{T}}$, the \textit{Chamfer} distance
\begin{equation}
  \label{eq:classical_Chamfer}
  \mathcal{D}_{\text{Chamfer}}(X_{\mathcal{S}}, X_{\mathcal{T}}) = \frac{1}{n_{p,\mathcal{S}}}\left(\sum_{\mathbf x\in X_{\mathcal{S}}} \min_{ \mathbf y\in X_{\mathcal{T}}} \lVert \mathbf{\mathbf x} - \mathbf{\mathbf y} \rVert_2\right) + \frac{1}{n_{p, \mathcal{T}}}\left(\sum_{\mathbf y \in X_{\mathcal{T}}} \min_{\mathbf x \in X_{\mathcal{S}}} \lVert \mathbf{x} - \mathbf{y} \rVert_2\right).
\end{equation}
We consider a metric inspired by \eqref{eq:classical_Chamfer} but tailored to the case of closed meshes, accounting
for the comon anatomical features of all shapes and for the common subdivision of the boundary.  

Specifically, for a pair of shapes $(\XS,\FS;l_{\mathcal S})$ and $(X_{\mathcal T},F_{\mathcal T}; l_{\mathcal T})$, 
we define a modified Chamfer distance computed separately on each subdomain, and with an additional term related to the orientation of the faces on the vessel wall:
\begin{equation}\label{eq:mesh-chamfer}
\begin{aligned}
  \mathcal{D}_{\text{Chamfer}}^*&\left((\XS,\FS),(X_{\mathcal T},F_{\mathcal T})\right) := \\
% wall
& \mathcal{D}_{\text{Chamfer}}(\XS(\Gamma_{\text{wall}}), \XT(\Gamma_{\text{wall}})) \\
& +   \lambda_n \left( 
  \frac{1}{n_{p,\mathcal{S}}}\sum_{\mathbf x\in \XS(\Gamma_{\text{wall}})}   \left(1-\lvert \mathbf{n}_{\mathbf x}\cdot\mathbf{n}^*_{(\XT(\Gamma_{\text{wall}}),{\mathbf x})} \rvert \right)
   +    \frac{1}{n_{p,\mathcal{T}}}\sum_{\mathbf x\in \XT(\Gamma_{\text{wall}})}   \left(1-\lvert \mathbf{n}_{\mathbf x}\cdot\mathbf{n}^*_{(\XS(\Gamma_{\text{wall}}),{\mathbf x})} \rvert \right)
   \right)\\
 % open boundaries
&  + \sum_{i=1}^{4} \mathcal{D}_{\text{Chamfer}}(\XS(\Gamma_i), \XT(\Gamma_i)) + \mathcal{D}_{\text{Chamfer}}(\XS(\Gamma_{\text{in}}), \XT(\Gamma_{\text{in}})) \\
& +  \sum_{i=1}^{4} \mathcal{D}_{\text{Chamfer}}(\XS(\Gamma_i\cap\Gamma_{\text{wall}}), \XT(\Gamma_i\cap\Gamma_{\text{wall}}))+\mathcal{D}_{\text{Chamfer}}(\XS(\Gamma_{\text{in}}\cap\Gamma_{\text{wall}}), \XT(\Gamma_{\text{in}}\cap\Gamma_{\text{wall}})),
\end{aligned}
\end{equation}
where $n_{p, \mathcal S}$ and $n_{p,\mathcal T}$ denote the cardinalities of the two point clouds,
$X(\Gamma)$ stands for the subset of point clouds whose vertices belong to the boundary subset $\Gamma$, 
$\mathbf{n}^*_{(X,{\mathbf x})}$ is the normal at the point of $X$ closest to $\mathbf x$, and the constant $\lambda_n>0$ is an additional hyperparameter.

\paragraph*{Transformation map} 
Following the approach introduced in section \ref{ssec:resnet-lddmm-intro}, we seek a map between source and target shapes as a diffeomorphism $\phi:[0,1]\times\mathbb{R}^3\rightarrow\mathbb{R}^3$ defined as the flow of an autonomous ordinary differential equation:
\begin{equation}\label{eq:resnet-fnn-ode}
  \dot{\mathbf x} = f(\mathbf x;\theta),\qquad \frac{d}{dt}\phi(t, \mathbf x;\theta)=f(\mathbf  x;\theta),
\end{equation}
where the vector field $f:\mathbb{R}^3\rightarrow\mathbb{R}^3$ is approximated by a feed-forward neural network (FNN) with ReLU activations as in~\cite{amor2022resnet},
and where $\theta$ represents the dependency on a generic set of hyperparameters of the FNN. We will use the abbreviation $\phi_1=\phi(1,\cdot):\mathbb{R}^3\rightarrow\mathbb{R}^3$.

In practice, the ODE \eqref{eq:resnet-fnn-ode} is discretized with a forward Euler method with $10$ time steps in the interval $I=[0,1]$ ($\Delta t = \frac{1}{10}$), resulting in a ResNet architecture~\cite{amor2022resnet} taking as input only the points of the source surface mesh $X|_{t=0}=\XS$:
\begin{equation}
  \label{eq:resnet}
  X|_{t=\Delta t\cdot (i+1)} = X|_{t=\Delta t\cdot i} + \Delta t\cdot f_{FNN}(\psi(X|_{t=\Delta t\cdot i});\theta),\quad \forall i\in\{0, \dots, N-1\}.
\end{equation} 
The map  $\psi:\mathbb{R}^3\rightarrow\mathbb{R}^{3+6\cdot n_{rff}}$, 
\begin{equation*}
  \psi(X|_{t=\Delta t\cdot i}) = (X|_{t=\Delta t\cdot i}, \{\cos(2^i\cdot X|_{t=\Delta t\cdot i}), \sin(2^i\cdot X|_{t=\Delta t\cdot i})\}_{i=0}^{7}),
\end{equation*}
is used to augment the inpus,  at each time iteration, with random fourier features~\cite{tancik2020fourier}. We used $n_{rff}=8$ in our implementation.

We employ vectorization, so the feed-forward neural network $f_{FNN}:\mathbb{R}^{3+6\cdot n_{rff}}\rightarrow\mathbb{R}^3$ that defines the vector field $f:\mathbb{R}^3\rightarrow\mathbb{R}^3$, $f(x)\mapsto f_{FNN}(\psi(x); \theta)$ is evaluated on the rows of $\psi(X|_{t=\Delta t\cdot i})\in\mathbb{R}^{n_S\times (3+6\cdot n_{rff})}$ in equation~\eqref{eq:resnet}, and $\psi:\mathbb{R}^3\rightarrow\mathbb{R}^{3+6\cdot n_{rff}}$ acts on the rows of $X|_{t=\Delta t\cdot i}\in\mathbb{R}^{n_{p, S}\times 3}$, for every $i$. 

The architecture of the FNN that we use is fixed, but its weights change for every pair of source-target aorta geometries: it has six hidden layer of dimension $500$ with ReLU activations, an input dimension of $51$ and an output dimension of $3$.

\begin{rmk}[Bijectivity of the transformation map]
In general, ResNets as defined in equation~\eqref{eq:resnet} are not invertible. 
  Using the Banach fixed point theorem, a sufficient condition to have bijectivity is 
  $\{f_{FNN}(\psi(X|_{t=\Delta t\cdot i});\theta)\}_{i\in\{0, \dots, N-1\}}$ to be 1-Lipshitz. 
  In practice, this condition can be verified as a post-processing step once the registration map has been computed, without the need of 
  additional computational costs associated with techniques, architectures, and optimization methods that enforce the invertibility exactly. The bijectivity is necessary from the theoretical point of view to define the \textit{pllback} and \textit{pushforward} of the velocity and pressure fields, and from the practical point of view to implement more robust registration algorithms, since the bijectivity property acts as an additional regularization with respect to non-rigid deformations~\cite{scarpolini2023enabling}.
\end{rmk}

\paragraph{Objective function} 
Let us denote with $\mathcal A:=\{(X, F, l)\in\mathbb{R}^{\nvert \times 3}\times\mathbb{N}^{n_f\times 3}\times\mathbb{R}^{n_{\text{cntrl}}\times 3}\}$ the set of all possible combinations of aortic shapes (vertices, faces, centerlines) in $\mathbb R^3$.

The objective function $\mathcal{L}:\mathcal{A}\times\mathcal{A}\rightarrow\mathbb{R}$ has the form
\begin{equation}\label{eq:L_AA}
\begin{aligned}
  \mathcal{L}\left((\XS,\FS, \lS),(\XT,\FT, \lT)\right)  & = 
  \mathcal{D}_{\text{Chamfer}}^*(\phi(1, \XS; \theta), \XT) + \lambda_C\cdot\sum_{i=1}^{n_{\text{cntrl}}}\lVert \phi(1, l_{\mathcal S, i}; \theta)-l_{\mathcal T, i} \rVert^2_2 \\
  & \quad + 
\lambda_{\text{edges}}  \mathcal{R}_{\text{edges}}(\phi(1, \XS; \theta),\FS) +  \lambda_{\text{en}} \mathcal{R}_{\text{energy}}(X_{\mathcal S}; \theta)
\end{aligned}
\end{equation}
and it is composed by the modified Chamfer distance \eqref{eq:mesh-chamfer}
between the mapped source $(\phi(1, \XS),\FS)$ and the target $(\XT,\FT)$, 
the distance in $L^2$-norm between the mapped source centerline $l_{\mathcal{S}}=\{l_{\mathcal{S}, i}\}_{i=1}^{390}$ and the target centerline $l_{\mathcal{T}}=\{l_{\mathcal{T}, i}\}_{i=1}^{390}$, and 
two regularizers.
The first term %$\mathcal{R}_{\text{edges}}:\mathcal{A}\rightarrow\mathbb{R}$
%
%\begin{equation}
%  \label{eq:reg}
%  \mathcal{R}((\XS,\FS); \theta) = \lambda_{\text{edges}}\cdot\mathcal{R}_{\text{edges}}(\phi(1, A_S; \theta)) + \lambda_{\text{en}}\cdot\mathcal{R}_{\text{energy}}(A_S; %\theta),
%\end{equation}
\begin{equation*}
  \mathcal{R}_{\text{edges}}\left(X,F\right) := 
  \sum_{(a_i,b_i,c_i) \in F} \left(
  \sum_{(e,f) \in \{(a_i, b_i), (b_i,c_i),(c_i,a_i)\} } \lVert \mathbf x_e - \mathbf x_f \rVert^2_2
   \right),
   %\\
  %\sum_{i=1}^{n_{f, S}}&\left(\lVert\phi(1, x_{S, a_i})-\phi(1, x_{S, b_i})\rVert^2_2+ \lVert\phi(1, x_{S,b_i})-\phi(1, x_{S, c_i})\rVert^2_2 + \lVert\phi(1, x_{S, c_i})-\phi(1, x_{S, a_i})\rVert^2_2\right),
\end{equation*}
penalizes the presence of stretched edges in each face of the mesh, whilst the second term imposes the minimization of the kinetic energy along the discrete trajectory 
$\{X|_{t=\Delta t\cdot i}\}_{i=0}^{N-1}=\{\phi(\Delta t\cdot i, X_S)\}_{i=0}^{N-1}$:
\begin{equation*}
  \mathcal{R}_{\text{energy}}(\XS; \theta) := \sum_{i=0}^{N-1} \lVert f_{FNN}(\psi(X|_{t=\Delta t\cdot i});\theta)\rVert^2_2.
\end{equation*}

The constants $\lambda_C>0$, $\lambda_{\text{en}}$, $\lambda_{\text{edges}}$ are positive parameters. Notice that in the above definitions we have used the fact that, when applying the diffeomorphism $\phi$, only the point clouds (coordinates of the mesh vertices) are mapped, whilst the set of faces $\FS$ remains the same.


\begin{problem}[Shape registration with ResNet-LDDMM]
With the above definitions, we formulate the following discrete surface registration problem. Given a source and a target meshes 
$\mathcal S=(\XS,\FS, \lS)$ and $\mathcal T = (\XT,\FT, \lT)$, find 
  \label{def:resnetlddmm}
  \begin{align*}
    &\argmin_{\theta}\ 
    \mathcal{D}_{\text{Chamfer}}^*(\phi(1, \XS; \theta), \XT) + \lambda_C\cdot\sum_{i=1}^{n_{\text{cntrl}}}\lVert \phi(1, l_{\mathcal S, i}; \theta)-l_{\mathcal T, i} \rVert^2_2 \\
  & \quad + 
 \lambda_{\text{edges}} \mathcal{R}_{\text{edges}}(\phi(1, \XS; \theta),\FS) +  \lambda_{\text{en}} \mathcal{R}_{\text{energy}}(\{\phi(\Delta t\cdot (i+1), A_S;\theta)\}_{i=0}^{N-1})\\
%    \mathcal{D}_{\text{Chamfer}}(\phi(1, A_S;\theta), A_T) + \lambda_C\cdot\mathcal{D}_{\text{cntrl}}(\phi(1, l_S;\theta), l_T) \\
%    &\qquad+ \lambda_{\text{edges}}\cdot\mathcal{R}_{\text{edges}}(\phi(1, A_S;\theta)) + \lambda_{\text{en}}\cdot\mathcal{R}_{\text{energy}}(\{\phi(\Delta t\cdot (i+1), A_S;\theta)\}_{i=0}^{N-1})\\
  \end{align*}
such that   $\phi(t,\cdot;\theta)$ is the flow of the discretized ODE defined by the corresponding ResNet vector field:
%
\begin{align*}
    &\phi(\Delta t\cdot (i+1), \XS;\theta) = X|_{t=\Delta t\cdot i} + \Delta t\cdot f_{FNN}(\psi(X|_{t=\Delta t\cdot i});\theta),\quad \forall i\in\{0, \dots, N-1\},\\
    &\phi(0, X_S;\theta) = \XS.
\end{align*}
\end{problem}

\paragraph{Optimization} To solve Problem \eqref{def:resnetlddmm}, we consider the ADAM optimizer~\cite{kingma2014adam} combined with a \textit{multigrid} strategy, i.e., considering
three level of refinement for the source mesh, while the target mesh is kept fixed. 
This approach is crucial to guarantee 
the convergence of the discrete registration problem: on one hand it speeds up the procedure and on the other hand guarantees an arbitrary small registration error. 
In practice, for a source mesh $(\XS,\FS)$, we will denote as
\begin{equation*}
(\XS^{j}, \FS^{j})\in\mathbb{R}^{\nvert^{j}\times 3}\times\mathbb{N}^{n_f^{j}\times 3},\qquad j\in\{0, 1 , 2\},
\end{equation*}
with an increasing number of vertices $\nvert^{j=0} < \nvert^{j=1} < \nvert^{j=2}$ the three considered refinements, assuming that the set of faces is consistently refined.
The different refinements are obtained imposing a different upper bound for the lengths of face edges and an upper bound for the radii of the surface Delaunay balls. 


An analogous \textit{multigrid} approach has been proposed in~\cite{scarpolini2023enabling}. However, since both the transformation map 
$\phi(1, \cdot):\mathbb{R}^3\rightarrow\mathbb{R}^3$ and the vector field $f:\mathbb{R}^3\rightarrow\mathbb{R}^3$ act on the ambient space $\mathbb{R}^3$, we do not need to interpolate from one source mesh refinement to the other. 

\paragraph{Registration of the computational domain: \textit{pullback} and \textit{pushforward} operators}
For each couple of source and target meshes, we store the registration map as the image of the source point cloud, i.e.,  $\phi_1(X_{\mathcal{S}})\subset\mathbb{R}^3$. 
%
We denote with $\Omega_{\mathcal{S}}\subset\mathbb{R}^3$ and $\Omega_{\mathcal{T}}\subset\mathbb{R}^3$ the computational domains used for the CFD simulations for the source/template and target geometries, respectively. The surface registration maps are computed and interpolated on the source domain $\Omega_{\mathcal{S}}$ 
through RBF interpolation $\phi_{RBF}:\mathbb{R}^3\rightarrow\mathbb{R}^3$ with thin splines as kernels and $\phi_1(X_{\mathcal{S}})$ as centers.
Let $g_{\mathcal S}:\Omega_{\mathcal{S}}\rightarrow\mathbb{R}$ be a function defined on the source domain. The \textit{\textit{pushforward}} of $g_{\mathcal S}$ through the registration map
is defined as
\begin{equation}\label{eq:pushforward}
\phi_{RBF}^{\#}(g_{\mathcal S}):\Omega_{\mathcal{T}}\rightarrow\mathbb{R},\quad  \phi_{RBF}^{\#}(g_{\mathcal S}):=g_{\mathcal S} \circ\phi_{RBF}^{-1},
\end{equation}
Conversely, for a function  $g_{\mathcal T}:\Omega_{\mathcal{T}}\rightarrow\mathbb{R}$, the \textit{pullback} $(\phi_{RBF})_{\#} \left(g_{\mathcal T}\right):\Omega_{\mathcal{S}}\rightarrow\mathbb{R}$ is defined as:
\begin{equation}\label{eq:pullback}
  (\phi_{RBF})_{\#} \left(g_{\mathcal T}\right) :=\left(g_{\mathcal T} \right) \circ\phi_{RBF}.
\end{equation}



\subsection{Shape registration results}
To train the ResNet, the source (or template) mesh is chosen in the training set of shapes and kept fixed throughout the offline stage, 
whilst the target mesh varies among the remaining $723$ training shapes. 
In the online stage, the source is unchanged whilst the target mesh varies among the $52$ test geometries. 

\begin{rmk}[Choice of template geometry]
 The source mesh has been chosen within the training set without any particular criteria. In general, the selection can also be optimized with respect to the reconstruction error of the velocity and pressure fields, see section~\ref{sec:sml}.  
\end{rmk}


The hyperparameters for the regularization in \eqref{eq:L_AA} are chosen as $\lambda_n=5\cdot 10^{-5}$, $\lambda_C = 10^{-5}$, 
and   $\lambda_{\text{en}}=\lambda_{\text{edges}}=1$, for the terms related to face orientation, centerline, energy of the trajectory, and edges, respectively.


We consider a total number of epochs $n_{\text{epochs}}=5000$. The change of source mesh is performed at the epochs $e_{0,1} = 3000$, 
from $\left(\XS^{0}, \FS^{0}\right)$ to $\left(\XS^{1}, \FS^{1}\right)$, and  $e_{1,2} = 4000$, from $\left(\XS^{1}, \FS^{1}\right)$ 
to $\left(\XS^{2}, \FS^{2}\right)$. The source meshes have a number of vertices $n^{j=0}_v=4127$, $n^{j=1}_v=11402$ and  $n^{j=2}_v=110676$.
These values are kept fixed. However, one could also use an adaptive strategy to select the refinement epochs, e.g., based on the ratio between Chamfer distances at consecutive steps.

A sketch of the \textit{multigrid} optimization is shown in figure~\ref{fig:multigrid}, displaying both the mesh refinement on the aortic arch and the 
displacement field $\phi(1, X_S^i;\theta)\text{-}X_S^i$ for the different refinement levels. Figure~\ref{fig:multigrid} (left) shows also 
the loss decay on a sample geometry, highlighting the influence of the \textit{multigrid} strategy for convergence.
\begin{figure}[!htp]
  \centering
  \includegraphics[width=0.95\textwidth]{img/registration_multigrid.pdf}
  \caption{Application of the \textit{multigrid} ResNet-LDDMM. The source surface mesh is refined progressively during the training to guarantee the convergence of the discrete registration problem. 
\textbf{Left}: Decay of the Chamfer loss over the epochs. At epochs $3000$ and $4000$ the mesh is refined. 
\textbf{Right}: Displacement field $\phi(1, X_\mathcal{S}^i;\theta)\text{-}X_\mathcal{S}$ and surface mesh for refinement levels $j\in\{0, 1, 2\}$.}
  \label{fig:multigrid}
\end{figure}
Figure~\ref{fig:flow} depicts different steps of the registration process between two surface meshes showing
both the registration field $\phi(t_i, X_S;\theta)$ and the vector field $f(x;\theta)$ at different intermediate deformed configurations. See figure~\ref{fig:12_42} for an example of velocity and pressure fields' registration on the template geometry at systolic peak, corresponding to the best $n=12$ and worst $n=42$ test cases from figure~\ref{fig:cluster_v}. 
%
\begin{figure}[!htp]
  \centering
  \includegraphics[width=0.83\textwidth]{img/flow.pdf}
  \caption{Registration of a source surface mesh $X_\mathcal{S}$ onto the target surface mesh $X_\mathcal{T}$ through the ResNet-LDDMM vector field $f$. 
  \textbf{Top}: registration field $\phi(t_i, \mathbf x;\theta)$ at different intermediate steps $t_i$. \textbf{Bottom}: Vector field $f(\mathbf x;\theta)$ in the configuration $\phi(t_i, X_\mathcal{S};\theta)$.}
  \label{fig:flow}
\end{figure}

For validating the registration algorithm, we evaluate the classical Chamfer distance~\eqref{eq:classical_Chamfer} between the point clouds of the 
registered and the target geometries, normalized by the diameter of the target geometry, for each considered shape. The results, shown in Table \ref{tab:reg-chamfer}, confirm the 
robustness and the accuracy of the registration across the whole database.
The computational cost for registering the $723$ training geometries was of circa $\SI{114}{\hour}\approx 723\times\SI{9.45}{\minute}$ (embarrassingly parallel tasks), while the online registration of the source on a single
target required, on average, $9.45$ minutes.
\begin{table}[H]
    \centering
    \begin{tabular}{lcc}
         & \textbf{Training set} (n=723) & \textbf{Test set} (n=52) \\
         \hline
        Average Chamfer Distance & $0.00367$ & $0.00347$ \\
        \hline
        Maximum Chamfer Distance & $0.00605$ & $0.00470$ \\
        \hline
    \end{tabular}
    \caption{Chamfer distance~\eqref{eq:classical_Chamfer} between the registered source and target shapes, normalized by the diameter of the target mesh.}
    \label{tab:reg-chamfer}
\end{table}
%The average Chamfer distance~\eqref{eq:classical_Chamfer} normalized over the diameter of the target geometry is $\mathbf{0.00367}$ and $\mathbf{0.00347}$ for the training and test datasets, respectively. The maximum Chamfer distance normalized over the diameter of the target geometry is $\mathbf{0.00605}$ and $\mathbf{0.00470}$ for the training and test datasets, respectively. 

%

\begin{figure}[!htp]
  \centering
  \includegraphics[width=0.95\textwidth, trim={0 0 0 20}, clip]{img/12_reg.pdf}\\
  \includegraphics[width=0.95\textwidth, trim={0 0 0 20}, clip]{img/42_reg.pdf}
  \caption{Registration of test case $n=12$ (top row, best) and $n=42$ (bottom row, worst) velocity and pressure fields onto the template. The displacement fields $\phi(1,X_\mathcal{S};\theta)\text{-}X_\mathcal{S}$ are shown on the third column: the registered geometry $\phi(1,X_\mathcal{S})$ is shaded and compared with original template $X_\mathcal{T}$, under the title \textit{registered template on target}.}
  \label{fig:12_42}
\end{figure}





\section{Application of shape registration for solution manifold learning}
\label{sec:sml}

We formally refer to \textsl{solution manifold} as the set of time-dependent velocity and pressure fields which are solutions of the Navier--Stokes equations~\eqref{eq:3dnse}
for different computational domains and boundary conditions~\eqref{eq:3dnse-bc}. 
The goal of solution manifold learning is to accurately and efficiently approximate the solution manifold using the available training data. 
%
Widely used approaches consider linear global reduced bases of the discrete finite element spaces 
(see e.g. ~\cite{benner_model_2017,hesthaven2016certified,rozza2022advanced}), time- or geometry-dependent linear bases, as well as 
non-linear approximants, including autoencoders~\cite{Fresca2020, romor2023nonlinear, ROMOR2025113729} or other non-linear dimension reduction techniques from machine learning. 

This section presents two applications of the shape registration method from section \ref{sec:registration} to solution manifold learning. 
First, in subsection~\ref{subsec:sml_correlations}, we propose different metrics to study the correlation between geometries and solutions (pressure/velocity), as well 
as between velocity and pressure, based on mapping the dataset of patient-specific flow data on the same reference shape.
The results are used in the context of data assimilation problems,  where correlations are particularly relevant when considering the estimation of pressure-related quantities from velocity measurements (see sections \ref{sec:da} and \ref{sec:prec}).  
Next, in section~\ref{subsec:sml_rec}, we investigate the accuracy of the global and local rSVD bases constructed registering the snapshots of pressure and velocity solutions from different shapes onto the same reference.

\subsection{Analysis of physics-based and geometric-based correlations}
\label{subsec:sml_correlations}
We evaluate the correlation between dissimilarity matrices computed from the database of shapes and corresponding numerical solutions, exploiting
the fact that all relevant fields can be encoded in the same discrete space, i.e., the finite element space on the computational domain of the reference shape.
%
In what follows, let us denote with $\mathcal{S}$ the fixed shape, and with $n_{p, \mathcal{S}}$, $\dofu$, and $\dofp$ the number of vertices of the
corresponding finite element mesh and the degrees of freedom of the velocity and pressure solutions, respectively ($n_{p, \mathcal{S}}=110676$, $\dofu = 332028$, $\dofp = 110676$ for the particular selection of the reference mesh considered in this study).
The remaining \textit{target} geometries, that can be registered to $\mathcal{S}$ via the registration maps
$\phi_1^i$, will be denoted
as $\shape{i}$, $i=1,\hdots,n_{\rm geo}$, with $n_{\rm geo} = 724 + 52 = 776$ (both the test and training sets used in section  \ref{sec:registration}).
%
Each target shape $\shape{i}$ can be encoded using the three spatial coordinates of the reference mesh vertices mapped to the target domain i.e. through the displacement fields
\begin{equation*}
  Y_{\shape{i}} = \phi^i_1(\XS)-\XS, \qquad Y_{\shape{i}}\in\mathbb{R}^{\nvertref \times 3}, 
\end{equation*}
and a further coordinate based on the distance from the centerline $l_{\shape{i}}$ computed on the mapped reference vertices
\begin{equation}
  \label{eq:enc_geo}
  Z_{\shape{i}} = d\left(\phi^i_1\left(\XS\right),l_{\shape{i}} \right),\qquad Z_{\shape{i}} \in \mathbb{R}^{\nvertref}\,,
\end{equation}
where $d$ stands for the Euclidean distance.

% then physics
For each $\shape{i}$, let us now define the matrices $X^{\mathbf u}_{\shape{i}} \in\mathbb{R}^{\dofu \times n_T}$  and 
$X^{p}_{\shape{i}}\in\mathbb{R}^{\dofp \times n_T}$ containing $n_T=80$ equally spaced snapshots of the velocity and pressure fields in the time interval $[0.05s, 0.25 s]$ (see remark~\ref{rmk:timewindow}) registered on $\mathcal{S}$ and evaluated on the reference finite element mesh.
%

We then introduce two metrics. Let $A$ and $B$ be two matrices, $A, B \in O(D, r)$  with $r>0$ orthonormal columns of dimension $D>0$. 
Following ~\cite{galarce2022state}, we define the Hausdorff distance as
\begin{equation}\label{eq:d_hausdorff}
 d_H^2(A, B) := \max\left(\max_{a\in \text{col}(A)} \frac{\lVert a-P_{B}a\rVert^2}{\lVert a\rVert^2};\max_{b\in \text{col}(B)} \frac{\lVert b-P_{A}b\rVert^2}{\lVert b\rVert^2}\right),
\end{equation}
where $\text{col}(A), \text{col}(B)$ are the set of columns of the matrices $A$ and $B$, respectively, 
and $P_C$, ($C=A,B$), is the finite-dimensional orthogonal projector onto the space spanned by the columns of $C$.
%
We also define the Grassmann distance (see e.g.~\cite{daniel2020model}) as
\begin{equation}\label{eq:d_grassman}
  d_{\text{Gr}}^2(A, B) = \sum^{r}_{i=1}\arccos^2(\sigma_i),\qquad A^TB = %U\Sigma V^T = 
  U\begin{bmatrix}
    \sigma_{1} & & \\
    & \ddots & \\
    & & \sigma_{r}
  \end{bmatrix}V^T,
\end{equation}
where $U\Sigma V^T$ stands for the singular value decomposition of $A^TB$. 


We consider then the following \textit{dissimilarity} matrices of dimension $\mathbb{R}^{n_{\text{geo}}\times n_{\text{geo}}}$:
\begin{equation}\label{eq:geo_matrices}
\begin{aligned}
(K^{\text{enc}})_{ij} & = d\left(Z_{\shape{i}}, Z_{\shape{j}}\right), \\
(K^{\phi})_{ij} & = d\left(\phi(1, X_{\mathcal S_i})-X_{\mathcal S_i}, \phi(1, X_{\shape{j}})-X_{\shape{j}}\right) = d\left(Y_{\shape{i}}, Y_{\shape{j}}\right), \\
\end{aligned}
\end{equation}
based on the Euclidean distance between geometric encodings, and 
\begin{equation}\label{eq:up_matrices}
\begin{aligned}
(K_{H}^{\mathbf u})_{ij} & = d_H(X_{\shape{i}}^{\mathbf u}, X_{\shape{j}}^{\mathbf u}),
\quad(K_{{\rm Gr}}^{\mathbf u})_{ij} = d_{\rm Gr}(X_{\shape{i}}^{\mathbf u}, X_{\shape{j}}^{\mathbf u}),\\
(K_{H}^{p})_{ij} & = d_H(X^p_{\shape{i}}, X^p_{\shape{j}}),
\quad(K_{{\rm Gr}}^{p})_{ij} = d_{\rm Gr}(X^p_{\shape{i}}, X^p_{\shape{j}}),
\end{aligned}
\end{equation}
based on the distances between the solutions introduced in \eqref{eq:d_hausdorff} and \eqref{eq:d_grassman}.
%
These matrices are used to evaluate the correlation between the geometry and the solution, as well as the velocity and the pressure fields using a Mantel test with 
Pearson’s product-moment correlation coefficient $r_m\in[-1, 1]$ and $999$ permutations.

Table~\ref{tab:mantel1} shows the results for the correlation between geometry and velocity/pressure fields, whilst Table~\ref{tab:mantel2} shows the results for the correlation 
between velocity and pressure.
Since different metrics are used, the dissimilarity matrices are centered, before the correlation coefficient is computed. 
%
For a qualitative comparison, figure~\ref{fig:mantel} shows the dissimilarity matrices entries omitting the diagonal ones. %and plotting one entry every $100$.

\begin{table}[htp!]
  \centering
  \footnotesize
  \caption{Mantel test results for the correlation between geometry dissimilarity matrices \eqref{eq:geo_matrices} and velocity/pressure dissimilarity matrices \eqref{eq:up_matrices}. Correlation coefficients with less statistical significance ($p$-value$>0.05$) have been omitted. }
  \begin{tabular}{
      l 
      |>{\centering\arraybackslash}p{3.cm} 
      |>{\centering\arraybackslash}p{3.cm} 
      |>{\centering\arraybackslash}p{3.2cm} 
      |>{\centering\arraybackslash}p{3.cm} }
      \textbf{} &$K_{H}^{\mathbf u}$ & $K^{\mathbf u}_{\text{Gr}}$ & $K^p_{d_H}$ &$K^p_{d_{\text{Gr}}}$ \\[3pt]
      \hline
      \hline 
     $K^{\text{enc}}$ & $r_m=0.172, p=0.001$ & $r_m=0.156, p=0.001$ & - & $r_m=0.110, p=0.001$ \\
      \hline
    $K^{\phi}$ & - & $r_m=0.267, p=0.001$ & $r_m=-0.094, p=0.001$ & $r_m=0.217, p=0.001$ \\
      \hline
  \end{tabular}
  \label{tab:mantel1}
\end{table}
%
\begin{table}[htp!]
  \caption{Mantel test results for the correlation between the velocity and pressure dissimilarity matrices \eqref{eq:up_matrices}. Correlation coefficients with less statistical significance ($p$-value$>0.05$) have been omitted. }
  \centering
    \footnotesize
  \begin{tabular}{
      l 
      |>{\centering\arraybackslash}p{3.5cm} 
      |>{\centering\arraybackslash}p{3.5cm} }
     & $K^p_{H}$ & $K^p_{\text{Gr}}$\\[3pt]
    \hline
    \hline
    $K_{H}^{\mathbf u}$ & $r_m=0.433, p=0.033$ & -\\[2pt]
    \hline
    $K^{\mathbf u}_{\text{Gr}}$ & $r_m=-0.169, p=0.001$ & $r_m=0.943, p=0.001$ \\[2pt]
    \hline
\end{tabular}
  \label{tab:mantel2}
\end{table}
%
The results suggests that the Grassmann metric is more granular than the Hausdorff metrics proposed in~\cite{galarce2022state}. 
In particular, the correlation plots $K^{\text{enc}}$ \textit{vs.} $K^{\mathbf{u}}_{H}$, $K^{\phi}$ \textit{vs.} $K^{\mathbf{u}}_{H}$, 
$K^{\mathbf{u}}_{H}$ \textit{vs.} $K^{p}_{H}$, and $K^{\mathbf{u}}_{H}$ \textit{vs.} $K^{p}_{\text{Gr}}$ show two clusters of more and less correlated pairs. 
%
We can also observe that there exists a choice of metrics ($K^{\mathbf{u}}_{\text{Gr}}$ \textit{vs.} $K^{p}_{\text{Gr}}$) for which velocity and pressure fields 
are highly correlated, suggesting that in this setting the data assimilation of pressure from velocity measurements may be more feasible. 
At the same time, based on the correlation study, inferring the velocity or the pressure solution solely from the geometry seems to be a more challenging task.
%
\begin{figure}[!htp]
  \centering
  \includegraphics[width=0.59\textwidth]{img/corr.pdf}
  \includegraphics[width=.335\textwidth]{img/corr2.pdf}
  \caption{Correlation among dissimilarity matrices: each dot corresponds to an entry of the matrices indicated in the $x$- and $y$-axis. 
Diagonal entries have been omitted.  Only one every $100$ entries among the $n_{\text{geo}}^2\text{-}n_{\text{geo}}$ off-diagonal entries are shown. 
\textbf{Left}: correlation between geometry encoding and velocity/pressure fields. \textbf{Right}: Correlation between velocity and pressure solutions.}
  \label{fig:mantel}
\end{figure}

In figure~\ref{fig:cluster_v}, we show the clustering of the available training and test geometries with MDS and dissimilarity matrix $K^{u}_{d_{G_r}}$ from equation~\eqref{eq:up_matrices}. In particular, we can detect the test geometries with the best ($n=12$) and worst ($n=42$) approximable velocity field.

\begin{figure}[!htp]
 \centering
 \includegraphics[width=0.9\textwidth]{img/cluster_v.pdf}
 \caption{\textbf{Left: } Clustering with MDS of the $724$ training and $52$ test geomeries. \textbf{Right: }  Test case $12$ and $42$ represent the closest and furthest geometries to the training set with respect to the Grassmann distance on the velocity field.}
 \label{fig:cluster_v}
\end{figure}


\subsection{Approximation properties of  global and local rSVD bases}
\label{subsec:sml_rec}
The shape dataset has been split into a  training ($n_{\text{train}}=724$) and a test  ($n_{\text{test}}=52$) set. To obtain a global linear reduced basis, we apply randomized SVD~\cite{halko2011finding} with a given rank $r>0$ to the matrices of velocity and pressure fields on the template geometry ordered column-wise $X^{\mathbf u}_{\text{train}}\in\mathbb{R}^{\dofu\times(n_{\text{train}}n_T)}$ and $X^{p}_{\text{train}}\in\mathbb{R}^{\dofp\times(n_{\text{train}}n_T)}$, respectively: 
\begin{equation*}
\begin{aligned}
X^{\mathbf u}_{\text{train}} \ &\rsvd\  \Phi_{\mathbf u}\Sigma^{r}_{\mathbf u}\Psi_{\mathbf u}
,\quad \Phi_{\mathbf u} \in\mathbb{R}^{\dofu\times r},\ \Sigma^{r}_{\mathbf u}\in\mathbb{R}^{r\times r},\ \Psi_{\mathbf u}\in\mathbb{R}^{r\times (n_{\text{train}}n_T)},\\
%
X^{p}_{\text{train}} \ &\rsvd\  \Phi_p\Sigma^{r}_p\Psi_p,\quad \Phi_p\in\mathbb{R}^{\dofp\times r},\ \Sigma^{r}_p\in\mathbb{R}^{r\times r},\ \Psi_p\in\mathbb{R}^{r\times (n_{\text{train}}n_T)}.
\end{aligned}
\end{equation*}
The columns of the matrices $\Phi_{\mathbf u}$ and $\Phi_p$ define the orthonormal global rSVD basis. For a $r$-dimensional (reduced) representation of a velocity field
$z_{\mathbf u}^{r} \in \mathbb R^r$ (resp. of a pressure field $z_p^{r}\in\mathbb{R}^r$ ), the corresponding approximation in the full finite element space will be defined by $\Phi_{\mathbf u} z_{\mathbf u}^{r}$ (resp. $\Phi_p z_p^{r}$).


\begin{rmk}[Partitioned \textit{vs.}  monolithic rSVD]
 We considered a partitioned rSVD global basis, i.e., computing the rSVD modes for velocity and pressure from two snapshot matrices. An alternative monolithic approach consists in 
computing the rSVD on a single snapshot matrix of dimension $(\dofu+\dofp) \times n_{\rm train} n_T$ where velocity and pressure solutions are stacked row-wise.
%
In our case, the choice was dictated by the better performance in terms of accuracy of the partitioned rSVD basis.
%
However, especially in the context of data assimilation for inferring pressure fields from velocity observations, a monolithic rSVD might have the advantage of handling the coupled latent representation in a single $r$-dimensional variable, which allows to automatically obtain the pressure field from the same reduced variable~\cite{galarce2023displacement}. 
\end{rmk}

Depending on the reduced dimension $r$, we consider the relative $L^2$-reconstruction errors to evaluate the accuracy of the reduced approximation 
\begin{equation}
  \label{eq:rec}
  \epsilon^{r}_{u} (\mathbf u_i(t)) := \frac{\lVert \mathbf u_i(t) -\Phi_{\mathbf u}\Phi_{\mathbf u}^T \mathbf u_i(t)\rVert_2}{\lVert \mathbf u_i(t) \rVert_2},\quad 
  \epsilon^{r}_p (p_i(t)) := \frac{\lVert p_i(t)-\Phi_p\Phi_p^T p_i(t)\rVert_2}{\lVert p_i(t)-\bar{p}_i \rVert_2},
\end{equation}
varying $u_i$ and $p_i$ among the numerical solutions of the training and test geometries, for $i=1,\hdots,724 + 52$, and time instances $t\in \{0.05s+n\cdot\Delta t\ | n\in\{0,\dots, n_T=80\}\}$.
In \eqref{eq:rec}, $\bar{p}_i\in\mathbb{R}$ stands for the average of the considered pressure solution.

The relative $L^2$-reconstruction errors \eqref{eq:rec} are shown in figure~\ref{fig:recerr} for $r\in\{500, 1000, 2000, 4000\}$, showing that a very high number of modes
is required to obtain approximation errors of the order of 10\% for the velocity field. The error for the pressure field is of the order of 1-5\% for all considered dimensions.
%
\begin{rmk}[Time window]
  \label{rmk:timewindow}
  Outside the considered time window $t \in [0.05s,0.25s]$, the velocity was poorly approximated. This might be due to the additional complexity in the flow patterns during flow deceleration, such as arise of vortices, which depend very strongly on the geometrical details and are not accurately reprodicible with a linear basis such as $\Phi_{\mathbf u}$. This is the reason why we restrict our data assimilation studies to the time window $t \in [0.05s,0.25s]$.
\end{rmk}
  %

\begin{figure}[!htp]
  \centering
  \includegraphics[width=0.85\textwidth]{img/rec_err.pdf}
  \caption{Average among the training or test datasets, of the relative $L^2$-reconstruction errors \eqref{eq:rec} of the velocity 
  and pressure fields in the time interval $[0.05s,0.25]$ for different rSVD ranks.}
  \label{fig:recerr}
\end{figure}
The current results suggest that the size of the reduced space might not be suitable for designing reduced order models, restricting the finite element spaces of the variational formulation to the linear space spanned by the rSVD basis.
Although this approach has been proposed and applied in related contexts using much lower reduced space dimensions~\cite{guibert2014group,PEGOLOTTI2021113762}, the additional complexity
considered in the current setting (general geometries, high variability of outlet boundary dimension, variable Windkessel parameters depending on flow split and measured inlet flow rate, presence of 
stenosis which leads to more complex patterns, and the incorporation of turbulence modelling) leads to the need of a much larger space for satisfactory approximations.

Handling shape variability in the context of reduced-order modeling and data assimilation has been also recently discussed in~\cite{galarce2022state} in the context of parametric domains, 
proposing to employ  a \textit{local} rSVD basis, i.e., first clustering the different shapes using a multidimensional scaling (MDS) clustering algorithm, and then assembling the reduced-order model only considering the closest instances.
%
%, but in our case the global rSVD basis performs best, possibly due to the high geometric variability and our limited computational budget to increase the training dataset. 
To test an analogous approach, we clustered the training geometries based on the MDS using the dissimilarity matrices $K^{\mathbf u}_{\text{Gr}}$ and $K^{\mathbf u}_{d_{H}}$. Then, given a test geometry
and a fixed value or $r$, we collect the snapshots of the closest $n_{\text{local}}:=\left\lceil r/n_T \right \rceil$ training shapes 
\begin{equation}
  \label{eq:trainsnap}
X^{\mathbf u}_{\text{train}, H}\in\mathbb{R}^{\dofu \times(n_{\text{local}}\,n_T)},\quad X^p_{\text{train}, H}\in\mathbb{R}^{\dofp\times(n_{\text{local}}\,n_T)} ,
\end{equation}
and
\begin{equation}
X^{\mathbf u}_{\text{train}, {\rm Gr}}\in\mathbb{R}^{\dofu \times(n_{\text{local}}\,n_T)},\quad X^p_{\text{train}, {\rm Gr}}\in\mathbb{R}^{\dofp\times(n_{\text{local}}\,n_T)} ,
\end{equation}
depending on the considered dissimilarity metric for the clustering, and compute the corresponding local rSVD bases.
Figure~\ref{fig:reclocal} shows the $L^2$-reconstruction errors \eqref{eq:rec} for the global rSVD basis and for the local ones
%the resulting local rSVD modes $\Phi_{u, H}$, $\Phi_{p, H}$ and $\Phi_{u, G_r}$, $\Phi_{p, G_r}$ are employed to compute the relative $L^2$-reconstruction error for that specific test geometry. We show the performance of a global SVD basis against local basis found with MDS clustering from $K^u_{d_{\text{Gr}}}, K^u_{d_{H}}$, in figure~\ref{fig:reclocal}. 
In the considered range of dimensions, the global rSVD achieves a better accuracy than the local approaches, using either the Hausdorff or the Grassmann metric. 
As observed above, this might reflect the high geometrical variability of the considered dataset, for which a larger amount of local geometries are required. 

%
\begin{figure}[!htp]
  \centering
  \includegraphics[width=0.7\textwidth]{img/rec_err_local.pdf}
  \caption{Relative $L^2$-reconstruction errors \eqref{eq:rec} for velocity and pressure averaged over the computational domain and over the considered
  time interval $[0.05s,0.25]$ varying the rSVD rank $r$ and considering a global rSVD and local Hausdorff and Grassman rSVD bases.
	The $25\%$ and $75\%$ percentile are also shown.}
  \label{fig:reclocal}
\end{figure}

Figure~\ref{fig:recerr_ntrain} shows the dependency of the reconstruction error on the number of training data employed. We observe that the global rSVD basis 
is not an efficient approximant of the solution manifold, yielding a decay of the velocity reconstruciton error of the order of $n_{\text{train}}^{-1/2}$. 
Increasing the training dataset with additional geometries is expected to improve the local rSVD error, and local rSVD basis and non-linear dimension reduction methods should be preferred. 

\begin{figure}[!htp]
  \centering
  \includegraphics[width=1\textwidth]{img/ml_studies.pdf}
  \caption{Study of the asymptotic behaviour of the relative $L^2$ reconstruction errors \eqref{eq:rec} increasing the size of the training dataset.
  A nonlinear fit with the function $n_{\text{train}}^{-a}+b$ and multiple initial values for the parameters $a>0,b$ is also shown: red crosses represent the extrapolated values. 
}
  \label{fig:recerr_ntrain}
\end{figure}


\section{EPD-GNN trained with registered solutions}\label{ssec:pres-gnn}
Building on the shape registration algorithm, we propose a new framework for inference with neural networks on different meshes. 
The dataset is represented by the collection of registered velocity and pressure fields supported on the reference shape.

We employ encode-process-decode graph neural networks (EPD-GNN), introduced in~\cite{pfaff2020learning}, that 
represent the state of the art GNN architectures to perform inference on computational meshes. 
To reduce the computational cost, the reference mesh has been coarsened using TetGen~\cite{Si2015}, reducing the number of vertices from $n_{p, \mathcal{S}}=110676$ to $n_{\text{vertices}}=5181$.
%
The velocity and pressure fields are transported from the fine to the coarse template meshes and back through RBF interpolation. The metrics (equation~\eqref{eq:l2relerr}) used to validate the results are always evaluated on the fine target meshes. We employ the nearest-neighbours algorithm to enrich each vertex with $e\in\{6, 9, 12\}$ edges to the closest vertices, for a total of $n_{\text{edges}}\in\{36648, 53748, 66135\}$ edges, respectively. Only undirected graphs will be employed.
%
We consider two inference problems.
\paragraph*{Geometry to velocity (\textit{gnn-gv}) and geometry to pressure (\textit{gnn-gp}) inference}
The input represents a geometrical encoding of the target computational domains with additional velocity b.c.. For each target domain, we evaluate the scalar field that represents the distance from the centerline $Z_{\shape{i}}\in\mathbb{R}^{n_{\text{vertices}}}$, from equation~\eqref{eq:enc_geo}. The pullback of this scalar field to the reference geometry through the registration map together with the pushforwarded coordinates of the vertices of the template geometry $\phi^i_1(\XS)\in\mathbb{R}^{n_{\text{vertices}\times 3}}$ is our $4$-dimensional geometrical encoding. This geometrical encoding is then embedded with a Fourier positional encoding~\cite{sutherland2015error} with $10$ features, through the maps $\{\cos(2^{i}z_j), \sin(2^{i}z_j)\}_{i=0, j=0}^{9, 3}$, where $\mathbf{z}=\{z_i\}_{i=0}^3\in\mathbb{R}^4$ is an arbitrary input vector, for a total of $n_{\text{feat}}=80=10\cdot 2\cdot 4$ geometrical input features. To these it is added the velocity field at $n_{t,\text{GNN}}=8$ times
\begin{equation}\label{eq:int}
  t\in\{0.05s, 0.075s, 0.1s, 0.125s, 0.15s, 0.2s, 0.225s\}=I_t,
\end{equation}
restricted at the boundaries $\Gamma_{\text{in}}$ and $\Gamma_i$ with $i\in\{1, 2, 3, 4\}$ of the target domains and then pulled back to the reference geometry. The value of the velocity boundary field is zero inside the computational domain $\overline{\Omega}\backslash \left(\Gamma_{\text{in}}\cup \left(\cup_{i=1}^{4}\Gamma_i\right)\right)$. The total dimension of the inputs is thus $n_{\text{fnodes}} = 104 =n_{\text{feat}}+n_{t,\text{GNN}}\cdot 3$, where $3$ refers to the number of components of the velocity field. Each edge between the vertices $\mathbf{x}_i$ and $\mathbf{x}_j$ has as features, the vector $\mathbf{x}_i-\mathbf{x}_j$, its $L^2$-norm $\lVert\mathbf{x}\rVert_2$, and the difference between the values of the input vector at the nodes $\mathbf{x}_i$ and $\mathbf{x}_j$, for a total of $n_{\text{fedges}}=4+n_{\text{fnodes}}$ edge features. The output of the GNNs for the problems \textit{gnn-gv} or \textit{gnn-gp} are the velocity field $u\in\mathbb{R}^{n_{\text{vertices}}\times 3n_{t,\text{GNN}}}$ or the pressure field $p\in\mathbb{R}^{n_{\text{vertices}}\times n_{t,\text{GNN}}}$ respectively, evaluated at $n_{t,\text{GNN}}=8$ time instants $t\in I_t$ and supported on the coarse reference mesh.

\paragraph*{Velocity to pressure (\textit{gnn-vp}) inference}
The input is the velocity field $u\in\mathbb{R}^{n_{\text{vertices}}\times 3n_{t,\text{GNN}}}$ at $n_{t,\text{GNN}}=8$ times $t\in I_t$ supported on the coarse reference mesh with $n_{\text{vertices}}=5181$ vertices and $n_{\text{edges}}\in\{36648, 53748, 66135\}$ edges, depending on the number of adjacent nodes $e\in\{6, 9, 12\}$. Each edge between the vertices $\mathbf{x}_i$ and $\mathbf{x}_j$ has, as features, the vector $\mathbf{x}_i-\mathbf{x}_j$, its $L^2$-norm $\lVert\mathbf{x}\rVert_2$, and the difference between the values of the velocity field at $n_{t,\text{GNN}}=8$ time instances at the nodes $\mathbf{x}_i$ and $\mathbf{x}_j$, for a total of $n_{\text{fedges}}=28=4+n_{t,\text{GNN}}\cdot 3$ edge features, $3$ stands for the components of the velocity field. The output is the pressure field $p\in\mathbb{R}^{n_{\text{vertices}}\times n_{t,\text{GNN}}}$ at $n_{t,\text{GNN}}=8$ times $t\in I_t$ supported on the coarse reference mesh. \newline

As our EPD-GNN model we choose the \textit{MeshGraphNet} architecture implemented in NVIDIA-Modulus~\cite{modulus}, based on \texttt{pytorch}~\cite{NEURIPS2019_9015}. The hyperparameters are the width of the network $w$, i.e., the number of consecutive EPD layers, the common hidden dimension $h$ of the node encoder and decoders and the edge encoder, and also the number of edges $e$ of each node. The loss is the relative mean squared error. We apply the ADAM stochastic optimization method~\cite{kingma2014adam} to train the EPD-GNNs with a scheduler used to halve the learning rate when the validation error does not decrease after $200$ epochs. The initial learning rate value is $0.001$. The $52$ test geometries are not employed during the training. We perform the optimization on a single GPU NVIDIA A100-SXM4 with 40GB of graphics RAM size.

We perform two hyperparameter studies.  Firstly, we consider the values $$(w,h)\in\{(10, 64), (15, 128), (20, 256)\}$$ and fix $e=9$, with $n_{\text{train}}=720$ training geometries and $n_{\text{val}}=4$ validation geometries, and $n_{\text{epochs}}=500$. We then select the best model as the one with lowest validation error. 
Secondly, we fix $(w,h)=(30, 256)$ and choose $e\in\{6, 9, 12\}$, with $n_{\text{train}}=680$ and $n_{\text{val}}=44$, and $n_{\text{epochs}}=1000$. We then select the best model as the one at the last epoch.

\begin{figure}[!htp]
  \centering
  \includegraphics[width=.85\textwidth]{img/gnn_train.pdf}
  \caption{First hyperparameter study: $e=9$, $(w,h)\in\{(10, 64), (15, 128), (20, 256)\}$, $n_{\text{train}}=720$, $n_{\text{val}}=4$, and $n_{\text{epoch}}=500$.}
  \label{fig:overfitting}
\end{figure}

\begin{figure}[!htp]
  \centering
  \includegraphics[width=.85\textwidth]{img/gnn_train_edges.pdf}
  \caption{Second hyperparameter study: $e\in\{6, 9, 12\}$, $(w,h)=(30, 256)$, $n_{\text{train}}=680$, $n_{\text{val}}=44$, and $n_{\text{epoch}}=1000$.}
  \label{fig:overfitting_edges}
\end{figure}

The values of the loss during the training for the first and second hyperparameter studies are reported in figures~\ref{fig:overfitting} and~\ref{fig:overfitting_edges}, respectively. A comparison of the training and validation mean squared loss, computed on the coarse mesh with $n_{\text{vertices}}=5181$ vertices, highlights a clear overfitting phenomenon in our limited data regime, with a high generalization error compared to the training error. From the convergence behavior of the training error, we are hopeful that increasing the training dataset would bring better results. The optimal way to increase the training dataset is a future direction of research.

To evaluate the prediction errors we will consider the $L^2$-relative errors for the velocity $\epsilon_{\widehat{\mathbf{u}}}$ and pressure $\epsilon_{\widehat{p}}$ evaluated on target shapes $\mathcal{T}$:
\begin{equation}
  \label{eq:l2relerr}
  \epsilon_{\widehat{\mathbf{u}}} = \frac{\lVert \widehat{\mathbf{u}}_{\text{true}}-\widehat{\mathbf{u}}\rVert_2}{\lVert \widehat{\mathbf{u}}_{\text{true}}\rVert_2},\qquad\epsilon_{\widehat{p}} = \frac{\lVert \widehat{p}_{\text{true}}-\overline{\widehat{p}}_{\text{true}} -(\widehat{p}-\overline{\widehat{p}})\rVert_2}{\lVert \widehat{p}_{\text{true}}-\overline{{\widehat{p}}}_{\text{true}} \rVert_2},
\end{equation}
where $\widehat{\mathbf{u}}_{\text{true}}$ and $\widehat{p}_{\text{true}}$ are the high-fidelity velocity and pressure fields obtained from the solution of the Navier--Stokes equation \eqref{eq:3dnse} on the target domain, $\widehat{\mathbf{u}}$ and $\widehat{p}$ are the predicted velocity and pressure fields, and
$\overline{\widehat{p}}_{\text{true}}\in\mathbb{R}$ and $\overline{\widehat{p}}\in\mathbb{R}$ denote the averages of the pressure fields. Hat symbols denote quantities defined on the target geometries. The minimum, maximum and median relative $L^2$-errors for the problems $\textit{gnn-gv}$, $\textit{gnn-gp}$, and $\textit{gnn-vp}$, are reported in Table~\ref{tab:gnns}: we evaluated these errors from the best models selected from the hyperparameter studies. The fields associated to the minimum, maximum and median values are shown in figure~\ref{fig:gv} for \textit{gnn-gv}, figure~\ref{fig:gp} for \textit{gnn-vp}, and figure~\ref{fig:vp} for \textit{gnn-vp}. In the following sections we will compare these results with PBDW (section~\ref{sec:da}) and pressure estimators (section~\ref{sec:prec}), using only the best model selected from the first hyperparameter study.
\begin{table}[H]
  \centering
  \begin{tabular}{l|ccc|ccc|ccc}
 & \multicolumn{3}{c}{$\textit{gnn-gv}~(\epsilon_{\widehat{\mathbf{u}}})$} & \multicolumn{3}{|c}{$\textit{gnn-gv}~(\epsilon_{\widehat{\mathbf{u}}})$} & \multicolumn{3}{|c}{$\textit{gnn-gv}~(\epsilon_{\widehat{\mathbf{u}}})$}\\ \hline
& \textbf{min} & \textbf{max} & \textbf{median} & \textbf{min} & \textbf{max} & \textbf{median} & \textbf{min} & \textbf{max} & \textbf{median}\\
       \hline
      First hp study  & $0.26$ & $0.44$ & $0.31$ & $0.11$ & $0.83$ & $0.27$ & $0.11$ & $0.54$ & $0.22$ \\
      \hline
      Second hp study & $0.25$ & $0.45$ & $0.31$ & $0.11$ & $0.93$ & $0.30$ & $0.11$ & $0.67$ & $0.24$ \\
      \hline
  \end{tabular}\hspace{5mm}
  \caption{Relative $L^2$-errors of the velocity and pressure predictions corresponding the best architectures from the first and second hyperparameter (hp) studies.}
  \label{tab:gnns}
\end{table}
%\begin{table}[H]
%  \centering
%  \begin{tabular}{lccc}
%      $\textit{gnn-gv}~(\epsilon_{\widehat{\mathbf{u}}})$ & \textbf{min} & \textbf{max} & \textbf{median} \\
%       \hline
%      first hp study  & $0.26$ & $0.44$ & $0.31$\\
%      \hline
%      second hp study & $0.25$ & $0.45$ & $0.31$\\
%      \hline
%  \end{tabular}\hspace{5mm}
%  \begin{tabular}{lccc}
%    $\textit{gnn-gp} ~(\epsilon_{\widehat{p}})$ & \textbf{min} & \textbf{max} & \textbf{median} \\
%    \hline
%    first hp study & $0.11$ & $0.83$ & $0.27$\\
%   \hline
%   second hp study & $0.11$ & $0.93$ & $0.30$\\
%   \hline
%\end{tabular}\vspace{3mm}
%\begin{tabular}{lccc}
%  $\textit{gnn-vp} ~(\epsilon_{\widehat{p}})$ & \textbf{min} & \textbf{max} & \textbf{median} \\
%  \hline
%  first hp study & $0.11$ & $0.54$ & $0.22$\\
% \hline
% second hp study & $0.11$ & $0.67$ & $0.24$\\
% \hline
%\end{tabular}
%  \caption{Relative $L^2$-errors of the velocity and pressure predictions corresponding the best architectures from the first and second hyperparameter (hp) studies.}
%  \label{tab:gnns}
%\end{table}

\begin{figure}[!htp]
  \centering
  \includegraphics[width=1\textwidth]{img/gv.pdf}
  \caption{Results on the test dataset of the EPD-GNNs for the problem \textit{gnn-gv} at systolic peak $t=0.125s$: true velocity field magnitude, predicted velocity field magnitude and difference between the two scalar fields. \textbf{Left: }Minimum test $L^2$-relative error $\boldsymbol{\epsilon}_{\widehat{\mathbf u}}=\textbf{0.26}$. \textbf{Right: } Maximum test $L^2$-relative error $\boldsymbol{\epsilon}_{\widehat{\mathbf u}}=\textbf{0.44}$. \textbf{Bottom: } Median test $L^2$-relative error $\boldsymbol{\epsilon}_{\widehat{\mathbf u}}=\textbf{0.31}$.}
  \label{fig:gv}
\end{figure}

\begin{figure}[!ht]
  \centering
  \includegraphics[width=1\textwidth]{img/gp.pdf}
  \caption{Results on the test dataset of the EPD-GNNs for the problem \textit{gnn-gp} at systolic peak $t=0.125s$: true pressure field, predicted pressure field and difference between the two scalar fields. The values of the pressure fields are rescaled to the same average. \textbf{Left: }Minimum test $L^2$-relative error $\boldsymbol{\epsilon}_{\widehat{p}}=\textbf{0.11}$. \textbf{Right: } Maximum test $L^2$-relative error $\boldsymbol{\epsilon}_{\widehat{p}}=\textbf{0.83}$. \textbf{Bottom: } Median test $L^2$-relative error $\boldsymbol{\epsilon}_{\widehat{p}}=\textbf{0.27}$.}
  \label{fig:gp}
\end{figure}

\begin{figure}[!ht]
  \centering
  \includegraphics[width=1\textwidth]{img/vp.pdf}
  \caption{Results on the test dataset of the EPD-GNNs for the problem \textit{gnn-vp} at systolic peak $t=0.125s$: true pressure field, predicted pressure field and difference between the two scalar fields. The values of the pressure fields are rescaled to the same average. \textbf{Left: }Minimum test $L^2$-relative error $\boldsymbol{\epsilon}_{\widehat{\mathbf p}}=\textbf{0.11}$. \textbf{Right: } Maximum test $L^2$-relative error $\boldsymbol{\epsilon}_{\widehat{\mathbf p}}=\textbf{0.54}$. \textbf{Bottom: } Median test $L^2$-relative error $\boldsymbol{\epsilon}_{\widehat{\mathbf p}}=\textbf{0.22}$.}
  \label{fig:vp}
\end{figure}

EPD-GNNs could be seen as a potential approach to reduce the effort and the time required for accurate experimental acquisition of 4DMRI data. However, in our case, the results suggest that the training of EPD-GNNs requires more data to achieve a better accuracy. 
Similar problems have been recently addressed considering GNNs or a combination of NNs and SVD as data-driven surrogate models in simpler settings, i.e., 
considering only healthy geometries, neglecting the secondary branches (LBCA, LCCA, LSA), employing a simplified physical model~\cite{pajaziti2023shape}(where $20$ and $57$ SVD modes for pressure and velocity are sufficient in their case to achieve a good reconstruction error with a different registration method), or employing 1D graphs instead of full 3D geometries~\cite{iacovelli2023novel,pegolotti2024learning}.

The overall low level of accuracy of the EPD-GNNs predictions is confirmed in figures~\ref{fig:gv},~\ref{fig:gp}, and~\ref{fig:vp}, showing the minimum, the maximum and the median $L^2$-relative error for the EPD-GNNs predictions.
These results might suggest that the variability of stenotic aortic geometries requires necessarily a larger training dataset
than only the $724$ training geometries used in the present study. 
This causes noticeable overfitting, see, e.g.,  the value of the loss on the validation set in figure~\ref{fig:overfitting}, and a still high training error.

In figure~\ref{fig:epdgnn_reg_no_reg}, we compare the mean $L^2$-relative error of the new framework over the $52$ test target geometries against the results of an EPD-GNN architecture trained on the original target geometries without registration, and coarsening them with TetGen~\cite{Si2015} as it has been done previously for the template mesh. The definition of the loss, inputs, outputs, and optimization are the same for registered and non-registered datasets. The difference is that the datasets are not supported on the same graph anymore. 
We can observe that the registration represents an efficient encoding of the geometrical and physical features of the problem: for each inference problem \textit{gnn-vp}, \textit{gnn-gp}, and \textit{gnn-gv}, the EPD-GNNs trained on registered data perform better than their alternatives on non-registered data \textit{gnn-gv-no-reg}, \textit{gnn-gp-no-reg}, and \textit{gnn-vp-no-reg}.


%mean $L^2$-relative error over the $52$ test target geometries for the predictions of the EPD-GNNs architectures trained with datasets registered on the template or supported on the original target geometries. Also in the case, we will consider only coarsened meshes obtained with Tetgen~\cite{Si2015}, with the same coarsening parameters, for computational budget limits. 
\begin{figure}[!htp]
  \centering
  \includegraphics[width=0.7\textwidth]{img/gnn_reg.pdf}
  \caption{$L^2$-relative errors $\epsilon_{\widehat{\mathbf{u}}}$ and $\epsilon_{\widehat{p}}$ computed on the target computational domains of the predicted velocity (\textit{gnn-gv},\textit{gnn-gv-no-reg}) and pressure fields (\textit{gnn-gp},\textit{gnn-gp-no-reg}, \textit{gnn-vp},\textit{gnn-vp-no-reg}) using EPD-GNN models with and without registration of the datasets.}
  \label{fig:epdgnn_reg_no_reg}
\end{figure}

\section{Data assimilation of the velocity field}
\label{sec:da}
In subsection~\ref{subsec:sml_rec}, we have shown how to obtain a global rSVD basis $\Phi_{\mathbf u}\in\mathbb{R}^{d_{\mathbf u}\times r_{\mathbf u}}$ for the velocity field on the reference geometry, combining CFD solutions from a database of patient geometries with registration. 
In this section, we address the reconstruction of the velocity field associated to a new patient geometry from a set of velocity observations acquired via 4D flow MRI on a lower resolution.
%
The data assimilation problem is solved using the Parametrized-Background Data-Weak (PBDW) method, in which, given the observations,
the velocity reconstruction is computed solving a modified least squares problem minimizing the distance of the reconstruction from a physics-informed
linear space and with an additional correction that accounts for the discrepancy with the available measurements.
The method was originally proposed in \cite{MPPY2015} and further analyzed and extended in \cite{cohen2022_nonlinearSpaces,gong2019pbdw}. 

We consider a physics-informed space defined by the global rSVD basis on the template, obtained from different shapes and CFD solutions. Additionally, we extend the approach 
proposed in~\cite{gong2019pbdw} for homogeneous noise to the case of heteroscedastic noise, to handle real applications which require assimilation techniques robust against real data.

% In Section \ref{sec:sml}, we have shown how to obtain a global rSVD basis for the velocity field on the reference geometry combining CFD solutions from a database of patient geometries. 
% In this Section, we address the reconstruction of the velocity field of a new patient geometry from a set of velocity observations acquired via 4DMRI on a lower resolution.

% We will employ PBDW~\cite{gong2019pbdw} with heteroscedastic noise for the reconstruction of the velocity field from noisy 4D flow MRI measurements. The method was originally proposed in \cite{MPPY2015} and further analyzed and extended in \cite{BCDDPW2017, maday-taddei-2019, cohen2022_nonlinearSpaces}. The extension for noisy measurements is studied here, as real applications call for assimilation techniques which are robust against real data.

% In short, PBDW tackles the reconstruction problem by means of a least-squares fit between the measurements and a linear reduced space, including an additional correction term in the space of observations. In the previous section, we have shown how to obtain a global rSVD basis $\Phi_{\mathbf u}\in\mathbb{R}^{d_{\mathbf u}\times r_{\mathbf u}}$ for the velocity field on the template geometry.

% Let us suppose, that we want to reconstruct the high-resolution velocity field of a new patient with corresponding test geometry from 4D flow MRI measurements of the velocity field. To do so, we execute the following steps:
% \begin{enumerate}
% \item We register the new geometry on the template through the registration map $\phi:[0,1]\times\mathbb{R}^3\rightarrow\mathbb{R}^3$, from definition~\ref{def:resnetlddmm}.
% \item We transport the velocity rSVD basis $\Phi_{\mathbf u}\in\mathbb{R}^{d_{\mathbf u}\times r_{\mathbf u}}$ from the template geometry to the test geometry via the registration map $\phi_1$ and interpolate it with RBF interpolation onto the dofs $\widehat{d}_{\mathbf u}$ of the test geometry, $\widehat{\Phi}_{\mathbf u}\in\mathbb{R}^{\widehat{d}_{\mathbf u}\times r_{\mathbf u}}$: informally we can write $\widehat{\Phi}_{\mathbf u}=(\phi_{\text{RBF}})^{\#}(\Phi_{\mathbf u})$.
% \item We apply PBDW with the basis $\widehat{\Phi}_{\mathbf u}$ after orthonormalization.
% \end{enumerate}
% All quantities in the test geometry will be represented with a hat above, e.g. $\widehat{\Phi}_{\mathbf u}$ for the transported velocity basis.

% \subsection{Velocity data assimilation from 4D flow MRI}
\newcommand{\pbdw}[1]{#1_{\rm PBDW}}
\subsection{Parametrized-Background Data-Weak with heteroscedastic noise}
Let us consider a patient geometry $\mathcal T$ and its computational mesh $\Omega_{\mathcal T}$.
%
We denote with $\phi:[0,1]\times\mathbb{R}^3\rightarrow\mathbb{R}^3$
the map to register the patient $\mathcal T$ on the reference shape. % $\Shat$ (i.e., $\phi_1(\mathcal T) = \Shat$). 
Given the velocity rSVD basis $\Phi_{\mathbf u}\in\mathbb{R}^{d_{\mathbf u}\times r_{\mathbf u}}$ on the reference shape, we use the registration map $\phi_1$ and RBF interpolation 
onto the finite element space on $\Omega_{\mathcal T}$  to compute the transported basis $\widehat{\Phi}_{\mathbf u}\in\mathbb{R}^{\widehat{d}_{\mathbf u}\times r_{\mathbf u}}$  on the new patient shape using the pushforward operator \eqref{eq:pushforward}. In particular, $\widehat{d}_{\mathbf u}$ denotes the corresponding number of velocity degrees of freedom of the velocity in the
computational domain $\Omega_{\mathcal T}$.

We assume to have available a set of velocity observations gathered from medical imaging, modelled as linear operators. Given a grid of voxels $\{Q_i\}_{i=1}^{M_{\text{voxels}}}$ s.t. $Q_i=\times_{i=1}^3 [a_i, b_i]$, $b_i>a_i,\ a_i,b_i\in\mathbb{R}_+$, with centers $\mathbf{c}^{\text{vox}}_i\in\mathbb{R}^3$ and vertices $\{\mathbf{x}_i^{\text{vox}}\}_{i=1}^8\subset\mathbb{R}^3$:
\begin{equation}
  \label{eq:voxel}
  l_i(\widehat{\mathbf{v}}) = \frac{1}{9}\left(\sum_{i=1}^{8}\widehat{\mathbf{v}}(\x^{\text{vox}}_i)+\widehat{\mathbf{v}}(\mathbf{c}^{\text{vox}}_i)\right)\approx \int_{Q_i}\widehat{\mathbf{v}}(\x)\ d\x,\qquad l_i:\mathbb{R}^{\widehat{d}_{\mathbf u}}\rightarrow\mathbb{R},
\end{equation}
with $\widehat{\mathbf{v}}:\Omega_{\mathcal{T}}\rightarrow\mathbb{R}^3$. Moreover, we introduce the divergence operator
\begin{equation}
  \label{eq:div}
  l_{\text{div}}(\widehat{\mathbf{v}})=\int_{\Omega_h}\text{div}(\widehat{\mathbf{v}}(\x))\ d\x,\qquad l_{\text{div}}:\mathbb{R}^{\widehat{d}_{\mathbf u}}\rightarrow\mathbb{R},
\end{equation}

\begin{rmk}
  The divergence operator \eqref{eq:div} can be evaluated exactly for velocity fields belonging to piecewise linear finite element space.
\end{rmk}

% Alternatively, the divergence constraint can be imposed exactly with the Piola transform~\cite{guibert2014group} acting on the registration map $\phi$. The divergence operator can be evaluated exactly as the velocity fields $\widehat{u}$ are functions in the polynomial P1 Lagrangian finite elements space of dimension $\widehat{d}_{\mathbf u}$. We can collect the Riesz representatives of $(\{l_i\}_{i=1}^{M_{\text{voxels}}}, l_{\text{div}})$ with respect to the discrete $\ell^2$ norm in a matrix $\mathcal{Z}_{\mathbf u}\in\mathbb{R}^{\widehat{d}_{\mathbf u}\times(M_{\text{voxels}}+1)}$ . We introduce the PBDW matrices
% \begin{align}
%   L=\mathcal{Z}_{\mathbf u}^T\widehat{\Phi}_{u},\quad L\in\mathbb{R}^{ M\times r_{\mathbf u}},\qquad K=\mathcal{Z}_{\mathbf u}^T\mathcal{Z}_{\mathbf u},\quad K\in\mathbb{R}^{ M\times  M},
% \end{align}
% with $M=M_{\text{voxels}}+1$, and define the true velocity field $\widehat{\mathbf{u}}^{\text{true}}\in\mathbb{R}^{\widehat{d}_{\mathbf u}}$ of which we know only the measurements
% \begin{equation}
%   l(\widehat{\mathbf{u}}^{\text{true}})=(\{l_i(\widehat{\mathbf{u}}^{\text{true}})\}_{i=1}^{M_{\text{voxels}}}, l_{\text{div}}(\widehat{\mathbf{u}}^{\text{true}}))=y,\qquad l:\mathbb{R}^{\widehat{d}_{\mathbf u}}\rightarrow\mathbb{R}^M,
% \end{equation}
% where each coordinate corresponds to the output of an observation operator $\{\{l_i\}_{i=1}^{M_{\text{voxels}}}, l_{\text{div}}\}$, with $l_i:\mathbb{R}^{\widehat{d}_{\mathbf u}}\rightarrow\mathbb{R}^3$ and $l_{\text{div}}:\mathbb{R}^{\widehat{d}_{\mathbf u}}\rightarrow\mathbb{R}$.

% In~\cite{HAIK2023115868, gong2019pbdw}, they propose a PBDW formulation with observations affected by homogeneous noise:
% \begin{equation}
%   \label{eq:pbdw_homo}
%   (z_{\text{PBDW}}, \eta_{\text{PBDW}}) = \argmin_{(z, \eta)\in\mathbb{R}^r\times \mathbb{R}^M}\xi\lVert\eta\rVert^2+\lVert Lz + K\eta-y\rVert^{2}_{S^{-1}},
% \end{equation}
% where $S\in\mathbb{R}^{M\times M}$ is the measurements diagonal covariance matrix, and where the parameter $\xi>0$ needs to be set from a validation dataset: $\xi=0$ and $S=\text{Id}$ corresponds to the standard PBDW formulation. In general, $\xi$ should be proportional to $\lVert y-l(\widehat{u}^{\text{true}})\rVert_2/\lVert\mathcal{Z}_{\mathbf u}^{\perp}\widehat{u}^{\text{true}}\rVert_2$. In contrast, we will use a heteroscedastic noise formulation. Namely, for measurements $y\approx Lz + K\eta+\epsilon_{y}$ with $\epsilon_y\sim\mathcal{N}(0, S)$, we estimate $u_{\text{PBDW}} = (z_{\text{PBDW}}, \eta_{\text{PBDW}})$ as follows:
%   \begin{equation}
%     \label{eq:pbdw_hetero}
%     (z_{\text{PBDW}}, \eta_{\text{PBDW}}) = \argmin_{(z, \eta)\in\mathbb{R}^r\times \mathbb{R}^M}\lVert\eta\rVert^2_{R^{-1}}+\lVert Lz + K\eta-y\rVert^{2}_{S^{-1}},
%   \end{equation}
% where $R\in\mathbb{R}^{M\times M}$ needs to be set, instead of $\xi\in\mathbb{R}$.

% \begin{theorem}[PBDW with heteroscedastic noise]
%   \label{theo:pbdw}
%   Problem \eqref{eq:pbdw_hetero} can be split into the two sub-problems:
%   \begin{equation}
%     z_{\text{PBDW}} = \argmin_{\mathbb{R}^{r_{\mathbf u}}} \lVert Lz-y\rVert^2_{S^{-1}W^{-1}},\qquad\eta_{\text{PBDW}} = \argmin_{\eta\in \mathbb{R}^M}\lVert\eta\rVert^2_{R^{-1}}+\lVert K\eta-y_{\text{err}}\rVert^2_{S^{-1}},
%   \end{equation}
%   with $W = (K + \text{Id})$ and $y_{\text{err}}=y-Lz_{\text{PBDW}}$, \textbf{assuming} that $R^{-1}=KS^{-1}$. %\fg{Since we dont consider it, we may say something about model bias (and thus model covariance), so that we avoid any comment on that matter from a not friendly reviewer}
% \end{theorem}

Let $M := M_{\text{voxels}}+1$ and let us denote with $\mathcal{Z}_{\mathbf u}\in\mathbb{R}^{\widehat{d}_{\mathbf u}\times M}$ the matrix whose columns are the Riesz representers of 
the operators $\{l_i\}_{i=1}^{M_{\text{voxels}}}$ and $ l_{\text{div}}$ with respect to the discrete $\ell^2$ norm.
Moreover, we will denote with 
$y \in \mathbb R^M$ the available set of measurements, and  with $\widehat{\mathbf u}^{\text{true}}\in\mathbb{R}^{\widehat{d}_{\mathbf u}}$ the \textit{true} velocity field, i.e., the unknown field
from which the available measurements are taken, i.e., such that
\begin{equation*}\label{eq:noisy_y}
y = \mathcal{Z}_{\mathbf u} \widehat{\mathbf u}^{\text{true}} + \epsilon_{y},\;\epsilon_y\sim\mathcal{N}(0, S),
%=(\{l_i(\widehat{\mathbf u}^{\text{true}})\}_{i=1}^{M_{\text{voxels}}}, l_{\text{div}}(\widehat{\mathbf u}^{\text{true}}))\,.
\end{equation*}
where $S\in\mathbb{R}^{M\times M}$ is the measurements noise covariance matrix.
 
\begin{rmk}
The operator \eqref{eq:div} will be used to impose incompressibility of the reconstructed velocity field as a fictitious measurement.
Alternatively, the divergence constraint can be imposed exactly with the Piola transform (see, e.g.~\cite{guibert2014group}) acting on the registration map $\phi$.
\end{rmk}

Let us now define the matrices
%
\begin{align*}
  L=\mathcal{Z}_{\mathbf u}^T\widehat{\Phi}_{u},\quad L\in\mathbb{R}^{ M\times r_{\mathbf u}},\qquad K=\mathcal{Z}_{\mathbf u}^T\mathcal{Z}_{\mathbf u},\quad K\in\mathbb{R}^{ M\times  M}\,.
\end{align*}

In the case of homogeneous noise, the PBDW approach proposed in~\cite{gong2019pbdw} seeks the reconstruction 
in the form of $\pbdw{\widehat{\mathbf u}} = \widehat{\Phi}_{u} \pbdw{z} +  \mathcal{Z}_{\mathbf u} \pbdw{\eta}$ solving
  \begin{equation*}
    \label{eq:pbdw_homo}
    (z_{\text{PBDW}}, \eta_{\text{PBDW}}) = \argmin_{(z, \eta)\in\mathbb{R}^{r_{\mathbf u}}\times \mathbb{R}^M}\xi^{-1} \lVert\eta\rVert^2+\lVert Lz + K\eta-y\rVert^{2}_{S^{-1}},
  \end{equation*}
where $S\in\mathbb{R}^{M\times M}$ is the measurements diagonal covariance matrix and $\xi >0$ is set from a validation dataset and proportional to 
$\lVert y-l(\widehat{\mathbf u}^{\text{true}})\rVert_2/\lVert\mathcal{Z}_{\mathbf u}^{\perp}\widehat{\mathbf u}^{\text{true}}\rVert_2$. The special choices
$\xi=0$ and $S=\text{Id}$ yield the original PBDW formulation.

\begin{problem}[PBDW with heteroscedastic noise] We consider the following extension: given $y \in \mathbb R^M$, find $\pbdw{\widehat{\mathbf u}} = \widehat{\Phi}_{u} \pbdw{z} +  \mathcal{Z}_{\mathbf u} \pbdw{\eta}$ such that 
  \begin{equation}
    \label{eq:pbdw_hetero}
    (z_{\text{PBDW}}, \eta_{\text{PBDW}}) = \argmin_{(z, \eta)\in\mathbb{R}^{r_{\mathbf u}}\times \mathbb{R}^M}\lVert\eta\rVert^2_{R^{-1}}+\lVert Lz + K\eta-y\rVert^{2}_{S^{-1}},
  \end{equation}
where a matrix $R\in\mathbb{R}^{M\times M}$, instead of a single parameter, needs to be set.
\end{problem}


\begin{theorem}[PBDW reconstruction]
  \label{theo:pbdw}
Let us assume that $R$ is chosen such as $R^{-1}=KS^{-1}$. Then the solution to problem \eqref{eq:pbdw_hetero} can be obtained solving the sub-problems:
  \begin{eqnarray}
    z_{\text{PBDW}} & = \argmin_{z\in\mathbb{R}^{r_{\mathbf u}}} \lVert Lz-y\rVert^2_{S^{-1}W^{-1}},\label{eq:pbdw_sub1}\\
    \eta_{\text{PBDW}} & = \argmin_{\eta\in \mathbb{R}^M}\lVert\eta\rVert^2_{R^{-1}}+\lVert K\eta-y_{\text{err}}\rVert^2_{S^{-1}}, \label{eq:pbdw_sub2}
  \end{eqnarray}
  where $W = (K + \text{Id})$ and $y_{\text{err}}=y-Lz_{\text{PBDW}}$.
\end{theorem}
\begin{proof}
  The proof is reported in the appendix~\ref{appendix:pbdw}.
\end{proof}
\begin{rmk}
 Our choice for $R^{-1}$ results in the choice of the prior distribution for $\eta$. According to the Gauss-Markov theorem, the solution of \eqref{eq:pbdw_sub1} is 
 $z_{\text{PBDW}}\sim\mathcal{N}(m_{z_{\text{PBDW}}}, \Sigma_{z_{\text{PBDW}}})$ with
 \begin{equation}
       \label{eq:1pbdw}
       m_{z_{\text{PBDW}}}=\underbrace{(L^T S^{-1}W^{-1} L)^{-1} L^T S^{-1}W^{-1}}_{:=H_{z_{\text{PBDW}}}} y,\;
       \Sigma_{z_{\text{PBDW}}}=(L^T S^{-1}W^{-1} L)^{-1}.
   \end{equation} 
       This results can be interpreted as follows. Let us assume that there exists a reconstruction $z_{\text{true}}$ on the reduced-order space that fits the measurements, i.e.,  
       $y\approx Lz_{\text{true}}+\epsilon_{z}$, up to a noise $\epsilon_{z}\sim\mathcal{N}(0, WS)$. Then, the estimate is unbiased, i.e.,
        $\mathbb{E}[z_{\text{PBDW}}]=z_{\text{true}}$, and it minimizes $\mathbb{E}[\lVert z-z_{\text{true}}\rVert^2_2]$ as well as the covariance $\mathbb{E}[(z-z_{\text{true}})\otimes(z-z_{\text{true}})]$.
 %
 
 Let $y_{\text{err}}=y-Lz_{\text{PBDW}}=(I-LH_{z_{\text{PBDW}}})y$. Then, the solution to the problem \eqref{eq:pbdw_sub2} 
 can be interpreted as an inverse problem in the Bayesian framework with resulting posterior distribution:
 $$
 \eta_{\text{PBDW}}|y_{\text{err}}\sim\mathcal{N}(m_{\eta_{\text{PBDW}}}, \Sigma_{\eta_{\text{PBDW}}}),
 $$ 
 where
 \begin{equation}\label{eq:2pbdw}
 \begin{aligned}
 m_{\eta_{\text{PBDW}}} & = \underbrace{(KS^{-1}K+KS^{-1})^{-1}KS^{-1}}_{:= H_{\eta_{\text{PBDW}}}}y_{\text{err}},\\
 \Sigma_{\eta_{\text{PBDW}}} & =(KS^{-1}K+KS^{-1})^{-1} = \left[R^{-1}\left(K + \text{Id}\right)\right]^{-1} = W^{-1} R.
 \end{aligned}
 \end{equation}
 Equation \eqref{eq:2pbdw} is equivalent to assuming that there exists a correction on the measurement space 
 $\eta_{\text{true}}$, such that $y_{\text{err}}\approx K\eta_{\text{true}}+\epsilon_{\eta}$, with $\epsilon_{\eta}\sim\mathcal{N}(0, S)$.
 
 Using \eqref{eq:1pbdw} and \eqref{eq:2pbdw}, one obtains that the solution to \eqref{eq:pbdw_hetero} is Gaussian distributed,
 $\pbdw{\widehat{\mathbf u}} = \widehat{\Phi}_{\mathbf u}z_{\text{PBDW}}+\mathcal{Z}_{\mathbf u}\eta_{\text{PBDW}}$,
 \[
 \pbdw{\widehat{\mathbf u}} \sim\mathcal{N}(m_{\pbdw{\widehat{\mathbf u}}}, \Sigma_{\pbdw{\widehat{\mathbf u}}})
 \]
 with 
   \begin{equation}\label{eq:cov}
   \begin{aligned}
       &m_{\widehat{\mathbf{u}}_{\text{PBDW}}} = \widehat{\Phi}_{\mathbf u}  m_{z_{\text{PBDW}}} + \mathcal{Z}_{\mathbf u} m_{\eta_{\text{PBDW}}} = 
 [\widehat{\Phi}_{\mathbf u}  H_{z_{\text{PBDW}}}+\mathcal{Z}_{\mathbf u} H_{\eta_{\text{PBDW}}}-\mathcal{Z}_{\mathbf u} H_{\eta_{\text{PBDW}}}LH_{z_{\text{PBDW}}}]y = H_{\widehat{\mathbf{u}}_{\text{PBDW}}}y  \\
     &\Sigma_{\widehat{\mathbf{u}}_{\text{PBDW}}} = H_{\widehat{\mathbf{u}}_{\text{PBDW}}}S H_{\widehat{\mathbf{u}}_{\text{PBDW}}}^T.
   \end{aligned}
   \end{equation}
\end{rmk}

The next result builds on the estimate of \cite{gong2019pbdw} and, taking into account additional sources of error coming from the registration step, 
provides an error estimate for the PBDW reconstruction.
\begin{theorem}[Error estimate for PBDW reconstruction with heterogeneous noise]
  \label{theo:pbdwmsq}
Let $P_X:\mathbb{R}^{\widehat{d}_{\mathbf u}}\rightarrow\mathbb{R}^{\widehat{d}_{\mathbf u}}$ denote the linear projections onto a linear subspace $X\subset\mathbb{R}^{\widehat{d}_{\mathbf u}}$. Let $H_{\widehat{\mathbf{u}}_{\text{PBDW}}}:\mathbb{R}^{\widehat{d}_{\mathbf u}}\rightarrow\mathbb{R}^{\widehat{d}_{\mathbf u}}$ be the matrix defined in \eqref{eq:cov}, and
$H_l := H_{\widehat{\mathbf{u}}^{\text{PBDW}}} \, \mathcal Z_{\mathbf u}^T$. %\circ l:\mathbb{R}^{\widehat{d}_{\mathbf u}}\rightarrow\mathbb{R}^{\widehat{d}_{\mathbf u}}$.
%
The following estimate holds: 
  \begin{linenomath}\begin{align*}
    \mathbb{E}[\lVert &\widehat{\mathbf{u}}^{\text{true}}-\widehat{\mathbf{u}}_{\text{PBDW}}\rVert_2]\leq &\\
    &\leq\text{tr}(H_{\widehat{\mathbf{u}}_{\text{PBDW}}}SH_{\widehat{\mathbf{u}}_{\text{PBDW}}}^T)^{\tfrac{1}{2}} &\text{(noise error)}\\
    &+\lVert(\text{Id}-H_l)\circ P_{\text{Im}(H_l)}\rVert_2\lVert \widehat{\mathbf{u}}^{\text{true}}\rVert_2&\text{(PBDW bias)}\\
    &+\lVert\text{Id}-H_l\rVert_2 \ \cdot \inf_{{\mathbf u}^{\text{best}}\in\text{col}(X_{\text{train}}^{\mathbf{u}})}\big(&\text{(PBDW stability constant)}\\
    &\lVert(\phi_{\text{RBF}})^{\#}({\mathbf u}^{\text{best}})-\widehat{\mathbf{u}}^{\text{true}}-P_{{\text{Im}(H_l)}}((\phi_{\text{RBF}})^{\#}({\mathbf u}^{\text{best}})-\widehat{\mathbf{u}}^{\text{true}})\rVert_2&\text{(manifold approximation error)}\\
    &+ C\cdot\lVert {\mathbf u}^{\text{best}}-P_{\text{Im}(\Phi_{\mathbf u})} {\mathbf u}^{\text{best}}\rVert_2&\text{(template rSVD approximation error)}\\
    &+\lVert P_{\text{Im}(H_l)}\left((\phi_{\text{RBF}})^{\#}({\mathbf u}^{\text{best}})\right)-(\phi_{\text{RBF}})^{\#}(P_{\text{Im}(\Phi_{\mathbf u})}{\mathbf u}^{\text{best}})\rVert_2\big)&\text{(registration degradation error)}
  \end{align*}\end{linenomath}
  where the matrix $X_{\text{train}}^{u}\in\mathbb{R}^{d_{\mathbf u}\times (n_{\text{train}}n_T)}$ contains, columnwise, the set of training snapshots registered on the reference shape.
  %  $\text{col}(X_{\text{train}}^{\mathbf{u}})$ is the set of registered training snapshots ordered column-wise in the matrix $X_{\text{train}}^{\mathbf{u}}\in\mathbb{R}^{d_{\mathbf u}\times (n_{\text{train}}n_T)}$ from equation~\eqref{eq:trainsnap}, , 
  
\end{theorem}
\begin{proof}
  The proof is reported in appendix~\ref{appendix:pbdw} along with an interpretation of the various sources of error.
\end{proof}

\newcommand{\snrho}{\text{SNR-ho}}
\newcommand{\snrhe}{\text{SNR-he}}

\subsection{Heteroscedastic noise model}
In the case of 4DMRI images, the observations often present velocity gradients whose accuracy degrades close to the vessel boundaries (see, e.g.~\cite{zingaro2024advancing, IRARRAZAVAL2019250}).
To account for this aspect, we consider a heteroscedastic noise model depending on three parameters:
the signal-to-noise homoscedastic ration ($\snrho$), the signal-to-noise heteroscedastic ratio ($\snrhe$), and the divergence observations operator's variance ($\sigma_{\text{div}}^2$).
We then subdivide the domain $\Omega_{\mathcal T}$ into a boundary layer $\Omega_{T, \text{he}}$, where measurements are affected by non-homogeneous variance noise, and 
an inner domain $\Omega_{\mathcal T, \text{ho}}$, whose measurements are characterized by standard homogeneous noise.


The covariance matrix $S\in\mathbb{R}^{(3M_{\text{voxels}}+1)\times (3M_{\text{voxels}}+1)}$ in equation~\eqref{eq:pbdw_hetero} is defined as a block matrix
\begin{equation*}
  S = \begin{pmatrix}
    S_{\text{obs}} & 0\\
    0 & \sigma_{\text{div}}^2
    \end{pmatrix},\end{equation*}
where the first block $S_{\text{obs}}\in\mathbb{R}^{3M_{\text{voxels}}\times 3M_{\text{voxels}}}$ is associated to observation operators $\{\{l_i\}_{i=1}^{M_{\text{voxels}}}\}$, and the last diagonal 
entry is associated to the divergence operator ($l_{\text{div}}$).
Let $M_{\text{ho}}$ and $M_{\text{he}}$ denote the amount of voxels whose centers belong to $\Omega_{\mathcal T, \text{ho}}$ and $\Omega_{\mathcal T, \text{he}}$, respectively.
Moreover, let us denote with $\mathbf{c}^{\text{vox}}_i$ the center of voxel $i$.
In our approach, the heteroscedastic noise is modeled as a spatially correlated multiplicative scalar to each velocity observation in the boundary layer, which
only affects the magnitude of the observed velocity vectors.

The covariance matrix $S_{\text{obs}}$ is then split into a homoscedastic ($S_{\text{ho}}$) and a heteroscedastic ($S_{\text{he}}$) block, 
associated to the observation operators in each subdomain
\begin{align*}
  S_{\text{obs}} = \begin{pmatrix}
    S_{\text{ho}} & 0\\
    0 & S_{\text{he}}
    \end{pmatrix},\quad
  S_{\text{ho}} := \left(\frac{\bar{\widehat{\mathbf{u}}}}{\text{SNR-ho}}\right)^2\text{Id}_{3M_{\text{ho}}},\quad
  S_{\text{he}} :=\left(\frac{\bar{\widehat{\mathbf{u}}}}{\text{SNR-he}}\right)^2PCP^T,\;
\end{align*}
where $\bar{\widehat{\mathbf{u}}}=\frac{1}{M_{\text{voxels}}}\sum_{i=1}^{M_{\text{voxels}}}l_i(\widehat{\mathbf{u}})$, $P\in\mathbb{R}^{3M_{\text{he}}\times M_{\text{he}}}$ is the operator 
projecting the velocity vector into its norm and $C\in\mathbb{R}^{M_{\text{he}}\times M_{\text{he}}}$ is the Gramian matrix of the radial basis function kernel
\begin{equation*}
  k(\mathbf{c}^{\text{vox}}_i, \mathbf{c}^{\text{vox}}_j) = \exp\left(-\frac{\lVert\mathbf{c}^{\text{vox}}_i-\mathbf{c}^{\text{vox}}_j\rVert^2_2}{2l_{\mathcal T}}\right)+\epsilon^2\delta_{ij},\qquad
  \forall (i,j) \mid \mathbf{c}^{\text{vox}}_i, \mathbf{c}^{\text{vox}}_j \in \Omega_{T, \text{he}},
\end{equation*}
with length scale $l_{\mathcal T}>0$ and additive homogeneous noise variance $\epsilon^2>0$. 

These parameters depend on the measurement procedure employed. In what follows, we set 
$l_{\mathcal T}=\text{diam}(\Omega_{\mathcal T})/12$ and $\epsilon^2=0.1$.

\subsection{Numerical results and comparison with GNNs}
We consider three levels of noise: $(\text{SNR-ho}, \text{SNR-he})\in \{(10, 0.5), (0.4, 0.1), (0.2, 0.05)\}$ and voxels of resolution $\SI{2e-3}{\meter^3}$.
The velocity observations are computed approximating the integral in equation~\eqref{eq:voxel} with the average of the values of the finite element function on the voxel centers and on the vertices.
Figure \ref{fig:noise_pbdw} shows the resulting observations for the test geometry $n=12$
(the closest to the training velocity solution manifold with respect to the Grassmann metric on the velocity fields, see figure~\ref{fig:cluster_v}),
at a selected time instant for the different noise intensities, together with the corresponding PBDW reconstruction.
%
\begin{figure}[!htp]
  \centering
  \includegraphics[width=0.8\textwidth]{img/noise_fields.pdf}
  \caption{Data assimilation of the velocity field at time $0.1$s from noisy velocity measurements for the test geometry $n=12$ (see figure~\ref{fig:cluster_v}). 
\textbf{Left.} High-fidelity velocity field from the CFD simulations on the corresponding domain. 
\textbf{Right.} Observations with the three considered noise intensities (top) and PBDW predictions with fixed number of template velocity modes $r_{\mathbf u} = 2000$ (bottom).}
  \label{fig:noise_pbdw}
\end{figure}

We compare the reconstructed velocity field with those obtained with the EPD-GNN surrogate models from the inference problem \textit{gnn-gv} (geometry $\mapsto$ velocity) introduced in Section~\ref{ssec:pres-gnn}, which delivers a velocity prediction solely from the geometry data.
The results are shown in figure~\ref{fig:pbdw_vs_gnn_v}, depicting the $L^2$ average relative error $\epsilon_{\mathbf u}$ on the test dataset of $52$ geometries, using PBDW and EPD-GNNs.
The errors are evaluated on the target geometry, after transporting the predicted velocity fields with the registration map in the case of EPD-GNNs.
\begin{figure}[!ht]
  \centering
  \includegraphics[width=0.9\textwidth]{img/pbdw_0.pdf}\\
  \includegraphics[width=0.9\textwidth]{img/pbdw_1.pdf}\\
  \includegraphics[width=0.9\textwidth]{img/pbdw_2.pdf}\\
  \caption{Average $L^2$-relative error $\epsilon_{\mathbf u}$ of the velocity fields evaluated on the $52$ target-test geometries: the accuracy of data assimilation with PBDW from velocity observations is compared with respect to direct inference with EPD-GNNs. The rSVD reconstruction error (\textit{rec}) and the error of the noise-free observations (\textit{obs noise-free}) are also shown. \textbf{Top: } $(\text{SNR-ho}, \text{SNR-he})=(10, 0.5)$. \textbf{Middle: } $(\text{SNR-ho}, \text{SNR-he})=(0.4, 0.1)$. \textbf{Bottom: } $(\text{SNR-ho}, \text{SNR-he})=(0.2, 0.05)$.}
  \label{fig:pbdw_vs_gnn_v}
\end{figure}

Notice that the 4DMRI data are not used in the inference problem with EPD-GNNs. The comparison with PBDW is shown to underline, in the case of limited data, the difference in accuracy between a purely data-driven inference problem, such as the EPD-GNN,  and a physics-based data assimilation method that incorporates a state space
constructed using geometrical and physical information, as well as an additional set of observations.

Figure~\ref{fig:pbdw_vs_gnn_v} shows that increasing the level of noise affects the stability properties of the rSVD basis used in PBDW. For the lowest noise
$(\text{SNR-ho}, \text{SNR-he})=(10, 0.5)$  the best results are obtained with $r_{\mathbf u} = 2000$, while $r_{\mathbf u} = 500$ is the best performing case for $(\text{SNR-ho}, \text{SNR-he})=(0.2, 0.05)$.
%
In the ideal, noise-free case, the optimality properties of the PBDW guarantee that the reconstruction error is lower than the sole rSVD approximation error 
for specific $r<M$. For the case with the lowest noise $(\text{SNR-ho}, \text{SNR-he})=(10, 0.5)$, it can be observed that the PBDW reconstruction error
is lower than the rSVD approximation error for $r_{\mathbf u}=500$, but higher for $r_{\mathbf u}=2000$.

From the high-resolution reconstructed velocity field, obtained through PBDW or EPD-GNNs, clinically relevant biomarkers such as the time-averaged wall shear stress (TWSS)
\begin{align*}
  \tau_{\text{wss}}(\mathbf{x}, t)&=\mu\frac{\partial}{\partial\mathbf{n}(\mathbf{x}, t)}\left(\mathbf{u}(\mathbf{x}, t)-(\mathbf{u}(\mathbf{x}, t)\cdot\mathbf{n}(\mathbf{x}, t))\mathbf{n}(\mathbf{x}, t)\right),\\
  \tau_{\text{twss}}(\mathbf{x})&=\int_{t=0.05s}^{t=0.225s}\tau_{\text{wss}}(\mathbf{x}, t)\,dt\approx\frac{1}{8}\sum_{i=0}^{7}\tau_{\text{wss}}(\mathbf{x}, 0.05+i\cdot\Delta t),
\end{align*}
and the oscillatory shear index OSI$_I$ relative to the time interval $I=[0.05s,0.225s]$
\[
  OSI_{I}(\mathbf{x})=\frac{1}{2}\left(1-\frac{\left.|\int_{t=0.05s}^{t=0.225s}\tau_{\text{wss}}(\mathbf{x}, t)\,dt\right|}{\int_{t=0.05s}^{t=0.225s}|\tau_{\text{wss}}(\mathbf{x}, t)|\,dt}\right)\approx\frac{1}{2}\left(1-\frac{\left.|\frac{1}{8}\sum_{i=0}^{7}\tau_{\text{wss}}(\mathbf{x},  0.05+i\cdot\Delta t)\right|}{\frac{1}{8}\sum_{i=0}^{7}|\tau_{\text{wss}}(\mathbf{x},  0.05+i\cdot\Delta t)|}\right)
\]
can be computed.
The $L^2$-relative error for the test geometry $n=12$ are shown in figure~\ref{fig:twss_12} for different noise intensities,
while the results of the quantitative study on all the $52$ test geometries are presented in figure~\ref{fig:twss_osi}. 
PBDW achieves a satisfactory accuracy in all considered cases, while the prediction with EPD-GNNs fails, confirming that the model is not able to capture the
high geometric variability. As previously noticed, the EPD-GNN only relies on the geometry data. 
The purpose of the comparison with PBWD is to underline the problematics of the EPD-GNN approach in this clinical context, 
in order to address them in future studies with a higher computational budget and a higher amount of data.
%
\begin{figure}[!htp]
  \centering
  \includegraphics[width=0.85\textwidth]{img/wss_new.pdf}
  \caption{Time-averaged wall shear stress (TWSS, \textbf{top}) and oscillatory index (OSI, \textbf{bottom}) for test geometry $n=12$, figure~\ref{fig:cluster_v}. The results are shown for different noise levels $(\text{SNR-ho}, \text{SNR-he})\in \{(10, 0.5), (0.4, 0.1), (0.2, 0.05)\}$ and fixed rSVD rank $r_{\mathbf u}=2000$.}
  \label{fig:twss_12}
\end{figure}
%
\begin{figure}[!htp]
  \centering
  \includegraphics[width=0.8\textwidth]{img/twss.pdf}
  \caption{Average $L^2$-relative error of the time averaged wall shear stress (TWSS, top) and of the oscillatory index (OSI, bottom) on the $52$ test geometries,
calculated from the voxel observations (\textit{obs}), the PBDW reconstruction (\textit{pbdw}) and the EPD-GNN reconstruction (\textit{gnn-gv}).
  The results are shown for different noise levels $(\text{SNR-ho}, \text{SNR-he})\in \{(10, 0.5), (0.4, 0.1), (0.2, 0.05)\}$ and rSVD ranks $r_{\mathbf u}\in\{500, 1000, 2000\}$. The $25\%$ and $75\%$ percentile are shown with shaded regions.}
  \label{fig:twss_osi}
\end{figure}





\section{Pressure estimation}
\label{sec:prec}
This section focuses on the problem of estimating the pressure field from velocity data.
We benchmark the pressure reconstruction method based on the EPD-GNN (section \ref{ssec:pres-gnn}) against two variational-based approaches, the pressure Poisson estimator (PPE) and the Stokes estimator (STE).
A complete comparison of methodologies is presented in \cite{bertoglio2018relative}. We choose to consider PPE due to its simple implementation and already popularized use, and STE, because it has been benchmarked as the best method in \cite{bertoglio2018relative}. As it has been already discussed, joint reconstructions with PBDW for velocity and pressure, as in \cite{galarce2023displacement}, will not be applied due to the loss of accuracy in the overall reconstruction.


The pressure estimation problem is considered in a given target shape $\mathcal T$. 
We will denote with $\Omega_{\mathcal T}$ the corresponding computational domain, with $\mathcal{T}_h$ its triangulation, 
and with $\widehat{d}_{\mathbf u}$ and $\widehat{d}_p$ the degrees of freedom of the underlying finite element spaces for velocity and pressure, respectively. The diameter of a generic element $K \in \mathcal T_h$ is $h_K$. 
%
The input velocity field at a time $t_n$ will be denoted by $\widehat{\mathbf u}^n \sim\mathcal{N}(\mathbf m^n_{\widehat{\mathbf u}}, \Sigma^n_{\widehat{\mathbf u}})$ and it is assumed to be Gaussian distributed, 
with $\mathbf m^n_{\widehat{\mathbf u}}\in\mathbb{R}^{\widehat{d}_{\mathbf u}}$ and $\Sigma^n_{\widehat{\mathbf u}}\in\mathbb{R}^{\widehat{d}_{\mathbf u}\times \widehat{d}_{\mathbf u}}$. For example, $\widehat{\mathbf u}^n$ could be obtained from 4D-flow MRI data with heteroscedastic PBDW.

\subsection{Variational-based pressure estimators}
\label{subsec:ppestedef}

  
In the pressure-Poisson estimator (PPE) \cite{ebbers2001}, the velocity field is directly inserted in the right-hand-side variational form of the incompressible Navier--Stokes equations, and a suitable pressure field is recovered solving the resulting problem.
\begin{problem}[PPE]
  \label{def:ppe}
  Given three consecutive velocity time steps $\widehat{\ub}^{n}$, $\widehat{\ub}^{n+1/2}$, and $\widehat{\ub}^{n+1}$, find the
pressure at the intermediate time $\widehat{p}^{n+1/2}_{\text{PPE}}\in\mathbb{P}^1(\mathcal{T}_h)$ such that
  \begin{align}
    \int_{\Omega_{\mathcal T}} \nabla \widehat{p}^{n+1/2}_{\text{PPE}}\cdot\nabla q  =
     -\frac{\rho}{\tau}\int_{\Omega_{\mathcal T}} (\widehat{\ub}^{n+1}-\widehat{\ub}^{n})\cdot\nabla q - \rho\int_{\Omega_{\mathcal T}} (\widehat{\ub}^{n+1/2}\cdot\nabla\widehat{\ub}^{n+1/2})\cdot\nabla q,\;
     \forall q\in\mathbb{P}^1(\mathcal{T}_h),%\\
    % A_{\text{PPE}}\,\widehat{p}^{n+1/2}_{\text{PPE}} = M_{\text{PPE}}^{n+1}\widehat{\ub}_{n+1} - M_{\text{PPE}}^{n}\widehat{\ub}_{n} + Q_{\text{PPE}}(\widehat{\ub}^{n+1/2}, \widehat{\ub}^{n+1/2}),
     \label{eq:ppe-weak}
  \end{align}
  with boundary conditions $\widehat{p}^{n+1/2}_{\text{PPE}}=q=0$ on $\partial \Omega_{\mathcal T}$. %Gamma=\Gamma_{\rm in} \cup \left(\cup_{i=1}^4\Gamma_i\right)\cup\Gamma_{wall}$.
  Problem \eqref{eq:ppe-weak} can be equivalently written in matrix form as
  \begin{equation} \label{eq:ppe}
  A_{\text{PPE}}\,\widehat{p}^{n+1/2}_{\text{PPE}} = M_{\text{PPE}}\widehat{\ub}_{n+1} - M_{\text{PPE}}\widehat{\ub}_{n} + Q_{\text{PPE}}(\widehat{\ub}^{n+1/2}, \widehat{\ub}^{n+1/2}),
  \end{equation}
with natural definition of the stiffness matrix $A_{\text{PPE}}$, the mass matrix $M_{\text{PPE}}$,  and the advection term $Q_{\text{PPE}}$. 
 \end{problem} 

A bias correction for the estimator is obtained as the solution to the following problem:
Find $b_{\text{PPE}}\in\mathbb{P}^1(\mathcal{T}_h)$ such that %$\forall\mathbf{z}\in [\mathbb{P}^1(\mathcal{T}_h)]^d,\ $
  \begin{equation}\label{eq:ppe-bias}
    \int_{\Omega_{\mathcal T}} \nabla b_{\text{PPE}}\cdot\nabla q =-\rho\sum_{T\in\mathcal{T}_h}\sum_{i=1}^{\widehat{d}_{\mathbf u}}\sum_{j=1}^{\widehat{d}_{\mathbf u}}\int_T \Psi_{i,j}\nabla q,
    \; \forall q\in\mathbb{P}^1(\mathcal{T}_h)\,,
  \end{equation}
where we introduced the notaton
\begin{equation*}\label{eq:psi-bias-corr}
\Psi_{i,j} : = \left((\phib_i\cdot\nabla\phib_j)\odot\Sigma^{n+1/2}_{ij}\right), \; i,j=1,\hdots,\widehat{d}_{\mathbf u}
\end{equation*}
and $\Sigma^{n+1/2}_{ij}$ stands for the $3\times3$ $(i,j)$-subblock of the covariance matrix $\Sigma^{n+1/2}_{\widehat{\mathbf u}}$ corresponding to the support points 
of the finite element shape functions $\phib_i$,and $\phib_j$, and $\odot$ stands for the element-wise Hadamard product of two matrices.



%Combining \eqref{eq:ppe} and  \eqref{eq:ppe-bias}, the unbiased PPE bilinear estimator is defined by
% \begin{equation}
%  \widehat{p}^{n+1/2}_{\text{PPE}} = A_{\text{PPE}}^{-1}M_{\text{PPE}}^{n+1}\widehat{\ub}_{n+1} - A_{\text{PPE}}^{-1}M_{\text{PPE}}^{n}\widehat{\ub}_{n} + A_{\text{PPE}}^{-1}Q_{\text{PPE}}(\widehat{\ub}%^{n+1/2}, \widehat{\ub}^{n+1/2}) -b_{\text{PPE}}\,.
%   \end{equation}
%   \ac{If we don't need $H_{\text{PPE}}$ later, I would not introduce the additional notation}
%    $\widehat{p}^{n+1/2}_{\text{PPE}} = H_{\text{PPE}}x^{n}-b_{\text{STE}}$ with $x^{n}=(\widehat{\ub}^{n}, \widehat{\ub}^{n+1/2}, \widehat{\ub}^{n+1})$:
%  \begin{equation}
%    H_{\text{PPE}}x^{n} \coloneqq A_{\text{PPE}}^{-1}M_{\text{PPE}}^{n+1}\widehat{\ub}_{n+1} - A_{\text{PPE}}^{-1}M_{\text{PPE}}^{n}\widehat{\ub}_{n} + A_{\text{PPE}}^{-1}Q_{\text{PPE}}(\widehat{\ub}^{n+1/2}, \widehat{\ub}^{n+1/2}).
%  \end{equation}

In the Stokes estimator (STE) \cite{svihlova_2016}, the velocity field is inserted in the right-hand-side of the Navier--Stokes equations, but, unlike the PPE, the 
variational problem for the pressure field is formulated as a Stokes projection, including an additional corrector. As shown, e.g.,  in \cite{bertoglio2018relative}, this approach allows to 
%Stokes field is computed which is used to test the governyng equations and thus providing 
obtain more robust results, especially against noisy velocity data.
\begin{problem}[STE]
  \label{def:ste}
    Given three consecutive velocity time steps $\widehat{\ub}^{n}$, $\widehat{\ub}^{n+1/2}$, and $\widehat{\ub}^{n+1}$, find $(\wb,\widehat{p}^{n+1/2}_{\text{STE}})\in [\mathbb{P}^1(\mathcal{T}_h)]^d \times \mathbb{P}^1(\mathcal{T}_h)$ such that
  \begin{equation}\label{eq:ste-weak}
  \begin{aligned}
    \int_{\Omega_{\mathcal T}} \nabla\wb:\nabla\mathbf{z} &- \int_{\Omega_{\mathcal T}} \widehat{p}^{n+1/2}_{\text{STE}}(\nabla\cdot\mathbf{z})+\int(\nabla\cdot\wb)q\\
     &+ \sum_{K\in\mathcal{T}_h}C_s h^2_K\int_K \nabla \widehat{p}^{n+1/2}_{\text{STE}}\cdot\nabla q=\\
     &-\frac{\rho}{\tau}\int_{\Omega_{\mathcal T}} (\widehat{\ub}^{n+1}-\widehat{\ub}^{n})\cdot\mathbf{z} - \rho\int_{\Omega_{\mathcal T}} (\widehat{\ub}^{n+1/2}\cdot\nabla\widehat{\widehat{\ub}^{n+1/2}})\cdot\wb-\mu\int_{\Omega_{\mathcal T}} \nabla\widehat{\ub}^{n+1/2}:\nabla\mathbf{z}\\
     &+\sum_{K\in\mathcal{T}_h}C_s h^2_K\rho\int_K (\mu\Delta\widehat{\ub}^{n+1/2}-\widehat{\ub}^{n+1/2}\cdot\nabla\widehat{\ub}^{n+1/2})\cdot\nabla q,
\;     \forall (\mathbf{z},q) \in [\mathbb{P}^1(\mathcal{T}_h)]^d \times \mathbb{P}^1(\mathcal{T}_h)
  \end{aligned}
  \end{equation}
    with boundary conditions $\wb=\mathbf{z}=0$ on $\partial \Omega_{\mathcal T}$. %$\Gamma_{\rm in} \cup \left(\cup_{i=1}^4\Gamma_i\right)\cup\Gamma_{wall}$.
    %
Problem \eqref{eq:ste} can be equivalently formulated in matrix form as
  \begin{equation}\label{eq:ste}
    \begin{aligned}
     A_{\text{STE}}\,\wb + B \widehat{p}^{n+1/2}_{\text{STE}} &= M_{\text{STE}}\widehat{\ub}_{n+1} - M_{\text{STE}}\widehat{\ub}_{n} + Q_{\text{STE}}(\widehat{\ub}^{n+1/2}, \widehat{\ub}^{n+1/2}) + M_{\text{STE}}\widehat{\ub}_{n+1/2},\\
     B^T\wb &= 0,
  \end{aligned}
  \end{equation}
with natural definition of the stiffness matrix $A_{\text{STE}}$, of the mass matrix $M_{\text{STE}}$, of 
the matrix associated to the grad-div term $B$, and of the advection term $Q_{\text{STE}}$. 
 \end{problem} 
  
A bias correction for the STE can be obtained as solution to the following problem: Find $(\wb,b_{\text{STE}}) \in[\mathbb{P}^1(\mathcal{T}_h)]^d \times \mathbb{P}^1(\mathcal{T}_h)$ such that 
  \begin{equation}\label{eq:ste-bias}
    \begin{aligned}
    \int_{\Omega_{\mathcal T}} \nabla\wb:\nabla\mathbf{z} &- \int_{\Omega_{\mathcal T}} b_{\text{STE}}(\nabla\cdot\mathbf{z})+\int(\nabla\cdot\wb)q =\rho\sum_{T\in\mathcal{T}_h}\sum_{i=1}^{\widehat{d}_{\mathbf u}}\sum_{j=1}^{\widehat{d}_{\mathbf u}}\int_{\Omega_{\mathcal T}} \Psi_{i,j} \cdot\mathbf{z}.
   \end{aligned}
    \end{equation}

The bias corrections introduced here, have been extended to the general case of heteroscedastic noise starting from~\cite{bertoglio2018relative}.

\subsection{Numerical results}
\label{subsec:resultspressureestimators}
This section is dedicated to the comparison of the performance of the variational-based estimators (PPE and STE)
against the GNNs. The results are evaluated considering a global error on the pressure fluctuation on the whole target domain $\epsilon_{\widehat{p}}$ (equation~\eqref{eq:l2relerr}).

A further biomarker of clinical interest is the pressure drop across the coarctation. We consider two cross-sections $\Gamma^{\text{sec}}_\text{in}$ close to the inlet $\Gamma_{\rm in}$ and a cross-section $\Gamma^{\text{sec}}_4$ close to the outlet $\Gamma_4$ of the target geometries (the position of the cross-sections depends on the centerline encoding of the geometries, as in~\cite{katz2023impact}) and define the pressure drop as
\begin{equation}
  \widehat{p}_{4\text{-}\text{in}} = \frac{1}{\left|\Gamma^{\text{sec}}_4\right|}\int_{\Gamma^{\text{sec}}_4}\widehat{p}\,d\sigma-\frac{1}{|\Gamma^{\text{sec}}_\text{in}|}\int_{\Gamma^{\text{sec}}_\text{in}}\widehat{p}\,d\sigma,
\end{equation}
where $| \cdot |$ stands for the area of the corresponding surface. 
%A_{\Gamma^{\text{sec}}_5},A_{\Gamma^{\text{sec}}_\text{in}}$ are the areas of the cross-sections $\Gamma^{\text{sec}}_5,\Gamma^{\text{sec}}_\text{in}$. 
%As highlighted by the definition of the pressure drop, we are often more interested on the pressure gradient rather than the pressure field itself. From this reasoning, we employ as metric to compare the prediction error of the pressure field on the whole target domains:
%\begin{equation}
%  \label{eq:l2relerr}
%  \epsilon_{\widehat{p}} = \frac{\lVert \widehat{p}_{\text{true}}-\widehat{p}-\overline{\widehat{p}}_{\text{true}}+\overline{\widehat{p}}\rVert_2}{\lVert \widehat{p}_{\text{true}}-\overline{{\widehat{p}}}_{\text{true}} \rVert_2}
%\end{equation}
%where $\widehat{p}_{\text{true}}$ is the high-fidelity pressure field obtained from 3D-INS  numerical simulations, $\overline{\widehat{p}}_{\text{true}}\in\mathbb{R}$ is its average, $\widehat{p}$ is the predicted velocity field, $\widehat{p}_{\text{true}}\in\mathbb{R}$ is its average.

% In the following results, the high-fidelity pressure field ($\widehat{p}_{\text{true}}$) is compared with PPE and STE reconstructions based on different velocity inputs
% (high-fidelity solution, low resolution observations, and PBDW reconstruction), as well as 
% with the pressure inferred with GNNs from the geometrical encoding (\textit{gnn-gp}), from  the high-fidelity velocity field (\textit{gnn-vp}), and from the velocity reconstructed with PBDW (\textit{gnn-vp-pbdw}).

We will consider the pressure fields computed with the PPE and STE from the high-fidelity velocity $\widehat{\mathbf{u}}^{\text{true}}$ (\textit{ppe}, \textit{ste}), 
the pressure fields computed with the PPE and STE from the observed velocity field with values $\{l_i\}_{i=1}^{M_{\text{voxels}}}$ and support points $\{\mathbf{c}_i^{\text{vox}}\}_{i=1}^{M_{\text{voxels}}}$ (\textit{ppeobs}, \textit{steobs}), 
the pressure fields computed with the PPE and STE from the PBDW velocity $\widehat{\mathbf{u}}^{\text{true}}_{\text{PBDW}}$ (\textit{ppefom}, \textit{stefom}), and the pressure fields 
computed with the reduced order models of the PPE and STE from the PBDW velocity $\widehat{\mathbf{u}}^{\text{true}}_{\text{PBDW}}$ (\textit{pperom}, \textit{sterom}), as described in appendix~\ref{apendix:rom}. 
%
Additionally, the pressure inferred with GNNs from the geometrical encoding is denoted by \textit{gnn-gp}, the pressure inferred with GNNs from the high-fidelity velocity field $\widehat{\mathbf{u}}^{\text{true}}$ by \textit{gnn-vp}, and the pressure inferred with GNNs from the PBDW velocity field $\widehat{\mathbf{u}}^{\text{true}}_{\text{PBDW}}$ by \textit{gnn-vp-pbdw}. The inference problem \textit{gnn-vp} and \textit{gnn-vp-pbdw} employ the same architecture defined in section~\ref{ssec:pres-gnn}.

The average pressure drops, median absolute pressure drops errors and the average $L^2$-relative errors (equation \eqref{eq:l2relerr}) across all test geometries are shown in figure~\ref{fig:pressure_res}.
For a qualitative comparison, the different predicted pressure fields are shown for the test geometry $n=12$ in figure~\ref{fig:pres_12}. 
%
In this case the results of the EPD-GNNs are comparable to those of PPE and STE. However, GNN are expected to deliver better results increasing the amount of training data,
due to the high geometric variability of the dataset: a symptom is the low training error in figure~\ref{fig:overfitting}. The test cases corresponding to the minimum, maximum and median $L^2$-relative errors are reported in figure~\ref{fig:gp} for the \textit{gnn-gp} problem. 

The best accuracy on the pressure field approximation and pressure drop corresponds to the time instants $t=0.125s$ and $t=0.15s$ for the PPE and STE estimators, the same time instants associated also to the best rSVD reconstruction error for the pressure fields in figure~\ref{fig:recerr}. However, the worse accuracy is not due to the rSVD reconstruction error but to the definition of the PPE and STE, as the same accuracy is associated to the estimated pressure fields \textit{ppe} and \textit{ste} obtained from the high-fidelity velocity. Moreover, the accuracy of PPE and STE does not seem to depend on the choice of observed velocity field: be it obtained from 4D-flow MRI observations (\textit{ppeobs}, \textit{steobs}), high-fidelity velocity fields (\textit{ppe}, \textit{ste}), PBDW velocity (\textit{ppefom}, \textit{stefom}), or reduced-order PPE and STE (\textit{pperom}, \textit{sterom}), the predictions achieve almost the same accuracy. Possibly, to improve the accuracy, the time resolution of $\Delta t=0.025s$ should be reduced to the high-fidelity simulations' time step $\Delta t=0.0025s$.

\begin{figure}[!ht]
  \centering
  \includegraphics[width=0.9\textwidth]{img/pressure_drop.pdf}\\
  \includegraphics[width=0.45\textwidth]{img/pressure_l2.pdf}
  \caption{\textbf{Top left}: average value of the pressure drop $\overline{\widehat{p}_{4\text{-}\text{in}}-\widehat{p}_{\text{true},4\text{-}\text{in}}}$ between cross-sections $\Gamma_4^{\text{sec}}-\Gamma_\text{in}^{\text{sec}}$ over all the $52$ test geometries at time instants $t\in\{0.075s, 0.1s, 0.125s, 0.15s, 0.175s, 0.2s\}$. \textbf{Top right}: median of the absolute error $|\widehat{p}_{4\text{-}\text{in}}-\widehat{p}_{\text{true},4\text{-}\text{in}}|$ of the predicted pressure drop with respect to the high-fidelity pressure drop over all the $52$ test geometries. \textbf{Bottom: } average of the $L^2$-relative error $\epsilon_{\widehat{p}}$ defined in equation~\ref{eq:l2relerr} over all the $52$ test geometries. The description of the labels is reported in the text.}
  \label{fig:pressure_res}
\end{figure}

\begin{figure}[!ht]
  \centering
  \includegraphics[width=0.9\textwidth]{img/pressure.pdf}\\
  \caption{Predicted pressure at time $t=0.125s$ for test geometry $n=12$ from figure~\ref{fig:cluster_v} and associated $L^2$-relative errors $\epsilon_{\widehat{p}}$.
  % The pressure fields computed with the PPE and STE from the high-fidelity velocity $\widehat{\mathbf{u}}^{\text{true}}$ (\textit{ppe}, \textit{ste}), 
  %the pressure fields computed with the PPE and STE from the observed velocity field with values $\{l_i\}_{i=1}^{M_{\text{voxels}}}$ and support points $\{\mathbf{c}_i^{\text{vox}}\}_{i=1}^{M_{\text{voxels}}}$ (\textit{ppeobs}, \textit{steobs}), 
  %the pressure fields computed with the PPE and STE from the PBDW velocity $\widehat{\mathbf{u}}^{\text{true}}_{\text{PBDW}}$ (\textit{ppefom}, \textit{stefom}), and the pressure fields 
  %computed with the reduced order models of the PPE and STE from the PBDW velocity $\widehat{\mathbf{u}}^{\text{true}}_{\text{PBDW}}$ (\textit{pperom}, \textit{sterom}). 
  %%
  %Additionally, the pressure inferred with GNNs from the geometrical encoding is denoted by \textit{gnn-gp}, the pressure inferred with GNNs from the high-fidleity velocity field $\widehat{\mathbf{u}}^{\text{true}}$ by \textit{gnn-vp}, and the pressure inferred with GNNs from the PBDW velocity field $\widehat{\mathbf{u}}^{\text{true}}_{\text{PBDW}}$ by \textit{gnn-vp-pbdw}.
  }
  \label{fig:pres_12}
\end{figure}

\subsection{Forward uncertainty quantification}
\label{subsec:uqpressureestimators}
Since PBDW models the uncertainty on the predicted velocity field from the coarse measurements $\widehat{\mathbf{u}}_{\text{PBDW}}\sim\mathcal{N}(m_{\widehat{\mathbf{u}}_{\text{PBDW}}}, \Sigma_{\widehat{\mathbf{u}}_{\text{PBDW}}})$, we want to study the uncertainty propagation to the pressure field through the pressure estimators and the inference with GNNs. In the following studies, we will keep the test geometry fixed and equal to test case $n=12$ from figure~\ref{fig:cluster_v}.

To measure the velocity field variability, we evaluate the normalized standard deviation at different values of $(\text{SNR-ho}, \text{SNR-he})\in \{(10, 0.5), (0.4, 0.1), (0.2, 0.05)\}$ in figure~\ref{fig:snr}:
\begin{equation}
  \label{eq:std}
  \text{std}_{\widehat{\mathbf{u}}} = \frac{\sum^{n_{\text{samples}}}_{i=1} \lVert \widehat{\mathbf{u}}_i-\tfrac{1}{n_{\text{samples}}}\sum_{i=1}^{n_{\text{samples}}}\widehat{\mathbf{u}}_i\rVert_2}{\sum^{n_{\text{samples}}}_{i=1} \lVert \tfrac{1}{n_{\text{samples}}}\sum_{i=1}^{n_{\text{samples}}}\widehat{\mathbf{u}}_i\rVert_2},
\end{equation}
where $n_{\text{samples}}=100$ are the number of samples from the Gaussian distribution $\widehat{\mathbf{u}}_{\text{PBDW}}\sim\mathcal{N}(m_{\widehat{\mathbf{u}}_{\text{PBDW}}}, \Sigma_{\widehat{\mathbf{u}}_{\text{PBDW}}})$ of the velocity predicted by PBDW with $r_{\mathbf u}\in\{500, 1000, 2000\}$ velocity modes.

Since the computational cost of a forward uncertainty problem is high due to the high number of forward evaluations, $n_{\text{samples}}=100$ in our case, we employ a reduced order model of the PPE and STE (\textit{pperom, sterom}), as described in appendix~\ref{apendix:rom}. We keep the number of pressure modes $r_p=1000$ fixed and vary the number of velocity modes $r_{\mathbf u}\in\{500, 1000, 2000\}$ and signal-to-noise ratios $(\text{SNR-ho}, \text{SNR-he})\in \{(10, 0.5), (0.4, 0.1), (0.2, 0.05)\}$.
%Since uncertainty quantification problems requires a large number of forward evaluations ($n_{\text{samples}}=100$ in our case), to reduce the overall computational cost 
%we employ a reduced-order version of the pressure estimators PPE and STE, obtained projecting the corresponding formulations ~\eqref{eq:ppe} and~\eqref{eq:ste} onto reduced-order spaces
%defined by the rSVD bases for velocity and pressure registered on the target geometry. 
%Since the computational cost of a forward uncertainty problem are high due to the high number of forward evaluations, $n_{\text{samples}}=100$ in our case, we employ a reduced order model of the PPE and STE (\textit{pperom, sterom}).
%\subsection{Reduced-order formulation of PPE and STE}
%A reduced-order formulation of PPE and STE can be defined projecting 
%their matrix forms Equations~\eqref{eq:ppe} and~\eqref{eq:ste}, respectively, onto reduced spaces, i.e., 
%\begin{equation}\label{eq:ppe-rom}
%  \widehat{\Phi}_{p}^T A_{\text{PPE}}\,\widehat{\Phi}_{p}z_{\widehat{p}^{n+1/2}_{\text{PPE}}} = \widehat{\Phi}_{p}^T M_{\text{PPE}}^{n+1}\widehat{\ub}_{n+1} - \widehat{\Phi}_{p}^T M_{\text{PPE}}^{n}\widehat{\ub}_{n} + \widehat{\Phi}_{p}^T Q_{\text{PPE}}(\widehat{\ub}^{n+1/2}, \widehat{\ub}^{n+1/2}),
%\end{equation}
%\begin{equation}\label{eq:ste-rom}
%\begin{aligned}
%  \widehat{\Phi}_{u}^T A_{\text{STE}}\,\widehat{\Phi}_{u}\mathbf{z}_{\wb} + \widehat{\Phi}_{u}^T B \widehat{\Phi}_{p}z_{\widehat{p}^{n+1/2}_{\text{STE}}} &= \widehat{\Phi}_{u}^T M_{\text{STE}}%^{n+1}\widehat{\ub}_{n+1} - \widehat{\Phi}_{u}^T M_{\text{STE}}^{n}\widehat{\ub}_{n} + \widehat{\Phi}_{u}^T Q_{\text{STE}}(\widehat{\ub}^{n+1/2}, \widehat{\ub}^{n+1/2}) \\
%  & \quad + \widehat{\Phi}_{u}^T M_{\text{STE}}^{n+1/2}\widehat{\ub}_{n+1/2},\\
%     \widehat{\Phi}_{p}^T B^T\widehat{\Phi}_{u}\mathbf{z}_{\wb} &= 0,
%\end{aligned}
%\end{equation}
%where $\widehat{\Phi}_{u}$ and $\widehat{\Phi}_{p}$ denote the 
%rSVD bases for velocity and pressure, registered on the target geometry.
%
%
%$\widehat{\Phi}_{u}\in\mathbb{R}^{r_{\mathbf u}\times \widehat{d}_{\mathbf u}}$ and $\widehat{\Phi}_{p}\in\mathbb{R}^{r_p\times \widehat{d}_p}$ and subsitution of $\widehat{p}^{n+1/2}_{\text{PPE}}\in\mathbb{R}^{\widehat{d}_p},\widehat{p}^{n+1/2}_{\text{STE}}\in\mathbb{R}^{\widehat{d}_p}, \wb\in\mathbb{R}^{\widehat{d}_{\mathbf u}}$ with the respective reduced variables $z_{\widehat{p}^{n+1/2}_{\text{PPE}}}\in\mathbb{R}^{r_p},z_{\widehat{p}^{n+1/2}_{\text{STE}}}\in\mathbb{R}^{r_p}, z_{\wb}\in\mathbb{R}^{r_{\mathbf u}}$. 
\begin{rmk}
Notice that, unlike for classical reduced-order models, the assembly of the matrices cannot be performed offline, since the registration map is needed to transport the rSVD modes from the reference to the target geometry. 
%
However, once the registration map is evaluated and the rSVD bases have been transported to the target geometry, the assembly of the reduced systems can be performed in parallel and solved with less computational costs thanks to the lower dimensionality of the reduced systems. %, related to the reduced dimensions $r_{\mathbf u}, r_p$.
\end{rmk}
%
%The velocity rSVD modes $\widehat{\Phi}_{u}$ are enriched with the supremizer technique~\cite{ballarin2015supremizer} with additional $r_p$ modes computed from the pressure rSVD modes $\widehat{\Phi}_{p}$, for a total of $r_{u,\text{sup}}=r_{\mathbf u}+r_p$ velocity modes $\Phi_{\widehat{\mathbf{u}},\text{sup}}\in\mathbb{R}^{r_{u,\text{sup}}\times \widehat{d}_{\mathbf u}}$:
%\begin{align*}
%  \Phi_{\widehat{\mathbf{u}},\text{sup}}^T A_{\text{STE}}\,\Phi_{\widehat{\mathbf{u}},\text{sup}}\mathbf{z}_{\wb} &+ \Phi_{\widehat{\mathbf{u}},\text{sup}}^T B \widehat{\Phi}_{p}z_{\widehat{p}^{n+1/2}_{\text{STE}}} = \\
%  &=\Phi_{\widehat{\mathbf{u}},\text{sup}}^T M_{\text{STE}}^{n+1}\widehat{\ub}_{n+1} - \Phi_{\widehat{\mathbf{u}},\text{sup}}^T M_{\text{STE}}^{n}\widehat{\ub}_{n} + \Phi_{\widehat{\mathbf{u}},\text{sup}}^T Q_{\text{STE}}(\widehat{\ub}^{n+1/2}, \widehat{\ub}^{n+1/2}) + \Phi_{\widehat{\mathbf{u}},\text{sup}}^T M_{\text{STE}}^{n+1/2}\widehat{\ub}_{n+1/2},\\
%    &\qquad\,\,\widehat{\Phi}_{p}^T B^T\Phi_{\widehat{\mathbf{u}},\text{sup}}\mathbf{z}_{\wb} = 0.
%\end{align*}

\begin{rmk}
For enhancing the stability of the reduced STE, the velocity rSVD modes are enriched with the supremizer technique~\cite{ballarin2015supremizer}. 
In this work, we evaluated the supremizers directly on the test geometries. The enrichment could be also performed offline on the reference geometry and transported on the new target, but this might
affect the overall stability of the formulation. 
\end{rmk}
%In principle, the supremizer enrichment could be performed offline on the template geometry and then transported with the registration map on the new patient geometry. It may occur that the method is yet not stable and additional corrections to the supremizers must be implemented. This is a future research direction. For the moment, we evaluted the supremizers directly on the test geometries.

In figure~\ref{fig:uq_p}, we compare the results of the PPE and STE with the predictions from the GNNs that compute the pressure field based on the PBDW velocity field as input, \textit{gnn-vp-pbdw}. The errors are evaluated on the test/target geometry with respect to the metrics in equation~\ref{eq:l2relerr}. The PPE estimator is omitted for $(\text{SNR-ho}, \text{SNR-he})\in \{(0.4, 0.1), (0.2, 0.05)\}$ as the relative error $\epsilon_{\widehat{\mathbf{u}}}$ goes above the value $1$ for all time instants.

In comparison to PPE and STE results, the GNNs' predictions are robust (or rather overconfident), to the uncertainty in the velocity field. It can be shown also looking at the standard deviation of the pressure field predicted with  \textit{pperom, sterom} or \textit{gnn-vp-pbdw} from $n_{\text{samples}}=100$ PBDW velocity samples $\widehat{\mathbf{u}}_{\text{PBDW}}$, in figure~\ref{fig:uq_cfd}: the magnitude of the normalized standard deviation in equation~\eqref{eq:std} is high only on a small subdomain for the GNN models, underlying that perturbations of the inputs do not considerably affect the outputs.

\begin{figure}[!htp]
  \centering
  \includegraphics[width=0.9\textwidth]{img/uq_v.pdf}\\
  \caption{Normalized standard deviation ($\text{std}_{\widehat{\mathbf{u}}}$, equation~\ref{eq:std}) of PBDW velocity $\widehat{}_{\text{PBDW}}$ for different values of signal-to-noise ratios $(\text{SNR-ho}, \text{SNR-he})\in \{(10, 0.5), (0.4, 0.1), (0.2, 0.05)\}$ and number of velocity modes $r_{\mathbf u}\in\{500, 1000, 2000\}$. We consider only test case $n=12$, $n_{\text{samples}}=100$ velocity fields were sampled to compute $\text{std}_{\widehat{\mathbf{u}}}$.}
  \label{fig:snr}
\end{figure}

\begin{figure}[!htp]
  \centering
  \includegraphics[width=0.65\textwidth]{img/uqp.pdf}\\
  \caption{Average and standard deviation of pressure field over $n_{\text{samples}}=100$ samples, obtained forwarding the uncertainty with \textit{pperom, sterom} or \textit{gnn-vp-pbdw} from $n_{\text{samples}}=100$ PBDW velocity samples $\widehat{\mathbf{u}}_{\text{PBDW}}$. The test/target geometry is fixed $n=12$. The results correspond to the upper left block of figure~\ref{fig:uq_p}.}
  \label{fig:uq_p}
\end{figure}


\begin{figure}[!htp]
  \centering
  \includegraphics[width=0.85\textwidth]{img/uq.pdf}\\
  \caption{$L^2$-relative error $\epsilon_{\widehat{p}}$ and $0.95$ confidence intervals of the pressure predicted on the target geometry $n=12$ with PPE and STE and a GNN model, for different values of signal-to-noise ratios $(\text{SNR-ho}, \text{SNR-he})$ and number of velocity modes $r_{\mathbf u}\in\{500, 1000, 2000\}$. The pressure modes are fixed at $r_p=1000$.}
  \label{fig:uq_cfd}
\end{figure}


\section{Discussion and limitations}
\label{sec:discussions}
\section{Discussion of Assumptions}\label{sec:discussion}
In this paper, we have made several assumptions for the sake of clarity and simplicity. In this section, we discuss the rationale behind these assumptions, the extent to which these assumptions hold in practice, and the consequences for our protocol when these assumptions hold.

\subsection{Assumptions on the Demand}

There are two simplifying assumptions we make about the demand. First, we assume the demand at any time is relatively small compared to the channel capacities. Second, we take the demand to be constant over time. We elaborate upon both these points below.

\paragraph{Small demands} The assumption that demands are small relative to channel capacities is made precise in \eqref{eq:large_capacity_assumption}. This assumption simplifies two major aspects of our protocol. First, it largely removes congestion from consideration. In \eqref{eq:primal_problem}, there is no constraint ensuring that total flow in both directions stays below capacity--this is always met. Consequently, there is no Lagrange multiplier for congestion and no congestion pricing; only imbalance penalties apply. In contrast, protocols in \cite{sivaraman2020high, varma2021throughput, wang2024fence} include congestion fees due to explicit congestion constraints. Second, the bound \eqref{eq:large_capacity_assumption} ensures that as long as channels remain balanced, the network can always meet demand, no matter how the demand is routed. Since channels can rebalance when necessary, they never drop transactions. This allows prices and flows to adjust as per the equations in \eqref{eq:algorithm}, which makes it easier to prove the protocol's convergence guarantees. This also preserves the key property that a channel's price remains proportional to net money flow through it.

In practice, payment channel networks are used most often for micro-payments, for which on-chain transactions are prohibitively expensive; large transactions typically take place directly on the blockchain. For example, according to \cite{river2023lightning}, the average channel capacity is roughly $0.1$ BTC ($5,000$ BTC distributed over $50,000$ channels), while the average transaction amount is less than $0.0004$ BTC ($44.7k$ satoshis). Thus, the small demand assumption is not too unrealistic. Additionally, the occasional large transaction can be treated as a sequence of smaller transactions by breaking it into packets and executing each packet serially (as done by \cite{sivaraman2020high}).
Lastly, a good path discovery process that favors large capacity channels over small capacity ones can help ensure that the bound in \eqref{eq:large_capacity_assumption} holds.

\paragraph{Constant demands} 
In this work, we assume that any transacting pair of nodes have a steady transaction demand between them (see Section \ref{sec:transaction_requests}). Making this assumption is necessary to obtain the kind of guarantees that we have presented in this paper. Unless the demand is steady, it is unreasonable to expect that the flows converge to a steady value. Weaker assumptions on the demand lead to weaker guarantees. For example, with the more general setting of stochastic, but i.i.d. demand between any two nodes, \cite{varma2021throughput} shows that the channel queue lengths are bounded in expectation. If the demand can be arbitrary, then it is very hard to get any meaningful performance guarantees; \cite{wang2024fence} shows that even for a single bidirectional channel, the competitive ratio is infinite. Indeed, because a PCN is a decentralized system and decisions must be made based on local information alone, it is difficult for the network to find the optimal detailed balance flow at every time step with a time-varying demand.  With a steady demand, the network can discover the optimal flows in a reasonably short time, as our work shows.

We view the constant demand assumption as an approximation for a more general demand process that could be piece-wise constant, stochastic, or both (see simulations in Figure \ref{fig:five_nodes_variable_demand}).
We believe it should be possible to merge ideas from our work and \cite{varma2021throughput} to provide guarantees in a setting with random demands with arbitrary means. We leave this for future work. In addition, our work suggests that a reasonable method of handling stochastic demands is to queue the transaction requests \textit{at the source node} itself. This queuing action should be viewed in conjunction with flow-control. Indeed, a temporarily high unidirectional demand would raise prices for the sender, incentivizing the sender to stop sending the transactions. If the sender queues the transactions, they can send them later when prices drop. This form of queuing does not require any overhaul of the basic PCN infrastructure and is therefore simpler to implement than per-channel queues as suggested by \cite{sivaraman2020high} and \cite{varma2021throughput}.

\subsection{The Incentive of Channels}
The actions of the channels as prescribed by the DEBT control protocol can be summarized as follows. Channels adjust their prices in proportion to the net flow through them. They rebalance themselves whenever necessary and execute any transaction request that has been made of them. We discuss both these aspects below.

\paragraph{On Prices}
In this work, the exclusive role of channel prices is to ensure that the flows through each channel remains balanced. In practice, it would be important to include other components in a channel's price/fee as well: a congestion price  and an incentive price. The congestion price, as suggested by \cite{varma2021throughput}, would depend on the total flow of transactions through the channel, and would incentivize nodes to balance the load over different paths. The incentive price, which is commonly used in practice \cite{river2023lightning}, is necessary to provide channels with an incentive to serve as an intermediary for different channels. In practice, we expect both these components to be smaller than the imbalance price. Consequently, we expect the behavior of our protocol to be similar to our theoretical results even with these additional prices.

A key aspect of our protocol is that channel fees are allowed to be negative. Although the original Lightning network whitepaper \cite{poon2016bitcoin} suggests that negative channel prices may be a good solution to promote rebalancing, the idea of negative prices in not very popular in the literature. To our knowledge, the only prior work with this feature is \cite{varma2021throughput}. Indeed, in papers such as \cite{van2021merchant} and \cite{wang2024fence}, the price function is explicitly modified such that the channel price is never negative. The results of our paper show the benefits of negative prices. For one, in steady state, equal flows in both directions ensure that a channel doesn't loose any money (the other price components mentioned above ensure that the channel will only gain money). More importantly, negative prices are important to ensure that the protocol selectively stifles acyclic flows while allowing circulations to flow. Indeed, in the example of Section \ref{sec:flow_control_example}, the flows between nodes $A$ and $C$ are left on only because the large positive price over one channel is canceled by the corresponding negative price over the other channel, leading to a net zero price.

Lastly, observe that in the DEBT control protocol, the price charged by a channel does not depend on its capacity. This is a natural consequence of the price being the Lagrange multiplier for the net-zero flow constraint, which also does not depend on the channel capacity. In contrast, in many other works, the imbalance price is normalized by the channel capacity \cite{ren2018optimal, lin2020funds, wang2024fence}; this is shown to work well in practice. The rationale for such a price structure is explained well in \cite{wang2024fence}, where this fee is derived with the aim of always maintaining some balance (liquidity) at each end of every channel. This is a reasonable aim if a channel is to never rebalance itself; the experiments of the aforementioned papers are conducted in such a regime. In this work, however, we allow the channels to rebalance themselves a few times in order to settle on a detailed balance flow. This is because our focus is on the long-term steady state performance of the protocol. This difference in perspective also shows up in how the price depends on the channel imbalance. \cite{lin2020funds} and \cite{wang2024fence} advocate for strictly convex prices whereas this work and \cite{varma2021throughput} propose linear prices.

\paragraph{On Rebalancing} 
Recall that the DEBT control protocol ensures that the flows in the network converge to a detailed balance flow, which can be sustained perpetually without any rebalancing. However, during the transient phase (before convergence), channels may have to perform on-chain rebalancing a few times. Since rebalancing is an expensive operation, it is worthwhile discussing methods by which channels can reduce the extent of rebalancing. One option for the channels to reduce the extent of rebalancing is to increase their capacity; however, this comes at the cost of locking in more capital. Each channel can decide for itself the optimum amount of capital to lock in. Another option, which we discuss in Section \ref{sec:five_node}, is for channels to increase the rate $\gamma$ at which they adjust prices. 

Ultimately, whether or not it is beneficial for a channel to rebalance depends on the time-horizon under consideration. Our protocol is based on the assumption that the demand remains steady for a long period of time. If this is indeed the case, it would be worthwhile for a channel to rebalance itself as it can make up this cost through the incentive fees gained from the flow of transactions through it in steady state. If a channel chooses not to rebalance itself, however, there is a risk of being trapped in a deadlock, which is suboptimal for not only the nodes but also the channel.

\section{Conclusion}
This work presents DEBT control: a protocol for payment channel networks that uses source routing and flow control based on channel prices. The protocol is derived by posing a network utility maximization problem and analyzing its dual minimization. It is shown that under steady demands, the protocol guides the network to an optimal, sustainable point. Simulations show its robustness to demand variations. The work demonstrates that simple protocols with strong theoretical guarantees are possible for PCNs and we hope it inspires further theoretical research in this direction.

\section{Conclusions}
\label{sec:conclusions}
\vspace{-0.2cm}
\section{Impact: Why Free Scientific Knowledge?}
\vspace{-0.1cm}

Historically, making knowledge widely available has driven transformative progress. Gutenberg’s printing press broke medieval monopolies on information, increasing literacy and contributing to the Renaissance and Scientific Revolution. In today's world, open source projects such as GNU/Linux and Wikipedia show that freely accessible and modifiable knowledge fosters innovation while ensuring creators are credited through copyleft licenses. These examples highlight a key idea: \textit{access to essential knowledge supports overall advancement.} 

This aligns with the arguments made by Prabhakaran et al. \cite{humanrightsbasedapproachresponsible}, who specifically highlight the \textbf{ human right to participate in scientific advancement} as enshrined in the Universal Declaration of Human Rights. They emphasize that this right underscores the importance of \textit{ equal access to the benefits of scientific progress for all}, a principle directly supported by our proposal for Knowledge Units. The UN Special Rapporteur on Cultural Rights further reinforces this, advocating for the expansion of copyright exceptions to broaden access to scientific knowledge as a crucial component of the right to science and culture \cite{scienceright}. 

However, current intellectual property regimes often create ``patently unfair" barriers to this knowledge, preventing innovation and access, especially in areas critical to human rights, as Hale compellingly argues \cite{patentlyunfair}. Finding a solution requires carefully balancing the imperative of open access with the legitimate rights of authors. As Austin and Ginsburg remind us, authors' rights are also human rights, necessitating robust protection \cite{authorhumanrights}. Shareable knowledge entities like Knowledge Units offer a potential mechanism to achieve this delicate balance in the scientific domain, enabling wider dissemination of research findings while respecting authors' fundamental rights.

\vspace{-0.2cm}
\subsection{Impact Across Sectors}

\textbf{Researchers:} Collaboration across different fields becomes easier when knowledge is shared openly. For instance, combining machine learning with biology or applying quantum principles to cryptography can lead to important breakthroughs. Removing copyright restrictions allows researchers to freely use data and methods, speeding up discoveries while respecting original contributions.

\textbf{Practitioners:} Professionals, especially in healthcare, benefit from immediate access to the latest research. Quick access to newer insights on the effectiveness of drugs, and alternative treatments speeds up adoption and awareness, potentially saving lives. Additionally, open knowledge helps developing countries gain access to health innovations.

\textbf{Education:} Education becomes more accessible when teachers use the latest research to create up-to-date curricula without prohibitive costs. Students can access high-quality research materials and use LM assistance to better understand complex topics, enhancing their learning experience and making high-quality education more accessible.

\textbf{Public Trust:} When information is transparent and accessible, the public can better understand and trust decision-making processes. Open access to government policies and industry practices allows people to review and verify information, helping to reduce misinformation. This transparency encourages critical thinking and builds trust in scientific and governmental institutions.

Overall, making scientific knowledge accessible supports global fairness. By viewing knowledge as a common resource rather than a product to be sold, we can speed up innovation, encourage critical thinking, and empower communities to address important challenges.

\vspace{-0.2cm}
\section{Open Problems}
\vspace{-0.1cm}

Moving forward, we identify key research directions to further exploit the potential of converting original texts into shareable knowledge entities such as demonstrated by the conversion into Knowledge Units in this work:


\textbf{1. Enhancing Factual Accuracy and Reliability:}  Refining KUs through cross-referencing with source texts and incorporating community-driven correction mechanisms, similar to Wikipedia, can minimize hallucinations and ensure the long-term accuracy of knowledge-based datasets at scale.

\textbf{2. Developing Applications for Education and Research:}  Using KU-based conversion for datasets to be employed in practical tools, such as search interfaces and learning platforms, can ensure rapid dissemination of any new knowledge into shareable downstream resources, significantly improving the accessibility, spread, and impact of KUs.

\textbf{3. Establishing Standards for Knowledge Interoperability and Reuse:}  Future research should focus on defining standardized formats for entities like KU and knowledge graph layouts \citep{lenat1990cyc}. These standards are essential to unlock seamless interoperability, facilitate reuse across diverse platforms, and foster a vibrant ecosystem of open scientific knowledge. 

\textbf{4. Interconnecting Shareable Knowledge for Scientific Workflow Assistance and Automation:} There might be further potential in constructing a semantic web that interconnects publicly shared knowledge, together with mechanisms that continually update and validate all shareable knowledge units. This can be starting point for a platform that uses all collected knowledge to assist scientific workflows, for instance by feeding such a semantic web into recently developed reasoning models equipped with retrieval augmented generation. Such assistance could assemble knowledge across multiple scientific papers, guiding scientists more efficiently through vast research landscapes. Given further progress in model capabilities, validation, self-repair and evolving new knowledge from already existing vast collection in the semantic web can lead to automation of scientific discovery, assuming that knowledge data in the semantic web can be freely shared.

We open-source our code and encourage collaboration to improve extraction pipelines, enhance Knowledge Unit capabilities, and expand coverage to additional fields.

\vspace{-0.2cm}
\section{Conclusion}
\vspace{-0.1cm}

In this paper, we highlight the potential of systematically separating factual scientific knowledge from protected artistic or stylistic expression. By representing scientific insights as structured facts and relationships, prototypes like Knowledge Units (KUs) offer a pathway to broaden access to scientific knowledge without infringing copyright, aligning with legal principles like German \S 24(1) UrhG and U.S. fair use standards. Extensive testing across a range of domains and models shows evidence that Knowledge Units (KUs) can feasibly retain core information. These findings offer a promising way forward for openly disseminating scientific information while respecting copyright constraints.

\section*{Author Contributions}

Christoph conceived the project and led organization. Christoph and Gollam led all the experiments. Nick and Huu led the legal aspects. Tawsif led the data collection. Ameya and Andreas led the manuscript writing. Ludwig, Sören, Robert, Jenia and Matthias provided feedback. advice and scientific supervision throughout the project. 

\section*{Acknowledgements}

The authors would like to thank (in alphabetical order): Sebastian Dziadzio, Kristof Meding, Tea Mustać, Shantanu Prabhat for insightful feedback and suggestions. Special thanks to Andrej Radonjic for help in scaling up data collection. GR and SA acknowledge financial support by the German Research Foundation (DFG) for the NFDI4DataScience Initiative (project number 460234259). AP and MB acknowledge financial support by the Federal Ministry of Education and Research (BMBF), FKZ: 011524085B and Open Philanthropy Foundation funded by the Good Ventures Foundation. AH acknowledges financial support by the Federal Ministry of Education and Research (BMBF), FKZ: 01IS24079A and the Carl Zeiss Foundation through the project "Certification and Foundations of Safe ML Systems" as well as the support from the International Max Planck Research School for Intelligent Systems (IMPRS-IS). JJ acknowledges funding by the Federal Ministry of Education and Research of Germany (BMBF) under grant no. 01IS22094B (WestAI - AI Service Center West), under grant no. 01IS24085C (OPENHAFM) and under the grant DE002571 (MINERVA), as well as co-funding by EU from EuroHPC Joint Undertaking programm under grant no. 101182737 (MINERVA) and from Digital Europe Programme under grant no. 101195233 (openEuroLLM) 

\section*{Acknowledgements}
This research has been partially funded by the Deutsche Forschungsgemeinschaft (DFG, German Research Foundation) under Germany's Excellence Strategy - MATH+: the Berlin Mathematical Research Center [EXC-2046/1 - project ID: 390685689], project AA5--10 \textit{Robust data-driven reduced-order models for cardiovascular
imaging of turbulent flows}.

\appendix
\setcounter{equation}{0}
\renewcommand\theequation{\arabic{equation}}
\subsection{Lloyd-Max Algorithm}
\label{subsec:Lloyd-Max}
For a given quantization bitwidth $B$ and an operand $\bm{X}$, the Lloyd-Max algorithm finds $2^B$ quantization levels $\{\hat{x}_i\}_{i=1}^{2^B}$ such that quantizing $\bm{X}$ by rounding each scalar in $\bm{X}$ to the nearest quantization level minimizes the quantization MSE. 

The algorithm starts with an initial guess of quantization levels and then iteratively computes quantization thresholds $\{\tau_i\}_{i=1}^{2^B-1}$ and updates quantization levels $\{\hat{x}_i\}_{i=1}^{2^B}$. Specifically, at iteration $n$, thresholds are set to the midpoints of the previous iteration's levels:
\begin{align*}
    \tau_i^{(n)}=\frac{\hat{x}_i^{(n-1)}+\hat{x}_{i+1}^{(n-1)}}2 \text{ for } i=1\ldots 2^B-1
\end{align*}
Subsequently, the quantization levels are re-computed as conditional means of the data regions defined by the new thresholds:
\begin{align*}
    \hat{x}_i^{(n)}=\mathbb{E}\left[ \bm{X} \big| \bm{X}\in [\tau_{i-1}^{(n)},\tau_i^{(n)}] \right] \text{ for } i=1\ldots 2^B
\end{align*}
where to satisfy boundary conditions we have $\tau_0=-\infty$ and $\tau_{2^B}=\infty$. The algorithm iterates the above steps until convergence.

Figure \ref{fig:lm_quant} compares the quantization levels of a $7$-bit floating point (E3M3) quantizer (left) to a $7$-bit Lloyd-Max quantizer (right) when quantizing a layer of weights from the GPT3-126M model at a per-tensor granularity. As shown, the Lloyd-Max quantizer achieves substantially lower quantization MSE. Further, Table \ref{tab:FP7_vs_LM7} shows the superior perplexity achieved by Lloyd-Max quantizers for bitwidths of $7$, $6$ and $5$. The difference between the quantizers is clear at 5 bits, where per-tensor FP quantization incurs a drastic and unacceptable increase in perplexity, while Lloyd-Max quantization incurs a much smaller increase. Nevertheless, we note that even the optimal Lloyd-Max quantizer incurs a notable ($\sim 1.5$) increase in perplexity due to the coarse granularity of quantization. 

\begin{figure}[h]
  \centering
  \includegraphics[width=0.7\linewidth]{sections/figures/LM7_FP7.pdf}
  \caption{\small Quantization levels and the corresponding quantization MSE of Floating Point (left) vs Lloyd-Max (right) Quantizers for a layer of weights in the GPT3-126M model.}
  \label{fig:lm_quant}
\end{figure}

\begin{table}[h]\scriptsize
\begin{center}
\caption{\label{tab:FP7_vs_LM7} \small Comparing perplexity (lower is better) achieved by floating point quantizers and Lloyd-Max quantizers on a GPT3-126M model for the Wikitext-103 dataset.}
\begin{tabular}{c|cc|c}
\hline
 \multirow{2}{*}{\textbf{Bitwidth}} & \multicolumn{2}{|c|}{\textbf{Floating-Point Quantizer}} & \textbf{Lloyd-Max Quantizer} \\
 & Best Format & Wikitext-103 Perplexity & Wikitext-103 Perplexity \\
\hline
7 & E3M3 & 18.32 & 18.27 \\
6 & E3M2 & 19.07 & 18.51 \\
5 & E4M0 & 43.89 & 19.71 \\
\hline
\end{tabular}
\end{center}
\end{table}

\subsection{Proof of Local Optimality of LO-BCQ}
\label{subsec:lobcq_opt_proof}
For a given block $\bm{b}_j$, the quantization MSE during LO-BCQ can be empirically evaluated as $\frac{1}{L_b}\lVert \bm{b}_j- \bm{\hat{b}}_j\rVert^2_2$ where $\bm{\hat{b}}_j$ is computed from equation (\ref{eq:clustered_quantization_definition}) as $C_{f(\bm{b}_j)}(\bm{b}_j)$. Further, for a given block cluster $\mathcal{B}_i$, we compute the quantization MSE as $\frac{1}{|\mathcal{B}_{i}|}\sum_{\bm{b} \in \mathcal{B}_{i}} \frac{1}{L_b}\lVert \bm{b}- C_i^{(n)}(\bm{b})\rVert^2_2$. Therefore, at the end of iteration $n$, we evaluate the overall quantization MSE $J^{(n)}$ for a given operand $\bm{X}$ composed of $N_c$ block clusters as:
\begin{align*}
    \label{eq:mse_iter_n}
    J^{(n)} = \frac{1}{N_c} \sum_{i=1}^{N_c} \frac{1}{|\mathcal{B}_{i}^{(n)}|}\sum_{\bm{v} \in \mathcal{B}_{i}^{(n)}} \frac{1}{L_b}\lVert \bm{b}- B_i^{(n)}(\bm{b})\rVert^2_2
\end{align*}

At the end of iteration $n$, the codebooks are updated from $\mathcal{C}^{(n-1)}$ to $\mathcal{C}^{(n)}$. However, the mapping of a given vector $\bm{b}_j$ to quantizers $\mathcal{C}^{(n)}$ remains as  $f^{(n)}(\bm{b}_j)$. At the next iteration, during the vector clustering step, $f^{(n+1)}(\bm{b}_j)$ finds new mapping of $\bm{b}_j$ to updated codebooks $\mathcal{C}^{(n)}$ such that the quantization MSE over the candidate codebooks is minimized. Therefore, we obtain the following result for $\bm{b}_j$:
\begin{align*}
\frac{1}{L_b}\lVert \bm{b}_j - C_{f^{(n+1)}(\bm{b}_j)}^{(n)}(\bm{b}_j)\rVert^2_2 \le \frac{1}{L_b}\lVert \bm{b}_j - C_{f^{(n)}(\bm{b}_j)}^{(n)}(\bm{b}_j)\rVert^2_2
\end{align*}

That is, quantizing $\bm{b}_j$ at the end of the block clustering step of iteration $n+1$ results in lower quantization MSE compared to quantizing at the end of iteration $n$. Since this is true for all $\bm{b} \in \bm{X}$, we assert the following:
\begin{equation}
\begin{split}
\label{eq:mse_ineq_1}
    \tilde{J}^{(n+1)} &= \frac{1}{N_c} \sum_{i=1}^{N_c} \frac{1}{|\mathcal{B}_{i}^{(n+1)}|}\sum_{\bm{b} \in \mathcal{B}_{i}^{(n+1)}} \frac{1}{L_b}\lVert \bm{b} - C_i^{(n)}(b)\rVert^2_2 \le J^{(n)}
\end{split}
\end{equation}
where $\tilde{J}^{(n+1)}$ is the the quantization MSE after the vector clustering step at iteration $n+1$.

Next, during the codebook update step (\ref{eq:quantizers_update}) at iteration $n+1$, the per-cluster codebooks $\mathcal{C}^{(n)}$ are updated to $\mathcal{C}^{(n+1)}$ by invoking the Lloyd-Max algorithm \citep{Lloyd}. We know that for any given value distribution, the Lloyd-Max algorithm minimizes the quantization MSE. Therefore, for a given vector cluster $\mathcal{B}_i$ we obtain the following result:

\begin{equation}
    \frac{1}{|\mathcal{B}_{i}^{(n+1)}|}\sum_{\bm{b} \in \mathcal{B}_{i}^{(n+1)}} \frac{1}{L_b}\lVert \bm{b}- C_i^{(n+1)}(\bm{b})\rVert^2_2 \le \frac{1}{|\mathcal{B}_{i}^{(n+1)}|}\sum_{\bm{b} \in \mathcal{B}_{i}^{(n+1)}} \frac{1}{L_b}\lVert \bm{b}- C_i^{(n)}(\bm{b})\rVert^2_2
\end{equation}

The above equation states that quantizing the given block cluster $\mathcal{B}_i$ after updating the associated codebook from $C_i^{(n)}$ to $C_i^{(n+1)}$ results in lower quantization MSE. Since this is true for all the block clusters, we derive the following result: 
\begin{equation}
\begin{split}
\label{eq:mse_ineq_2}
     J^{(n+1)} &= \frac{1}{N_c} \sum_{i=1}^{N_c} \frac{1}{|\mathcal{B}_{i}^{(n+1)}|}\sum_{\bm{b} \in \mathcal{B}_{i}^{(n+1)}} \frac{1}{L_b}\lVert \bm{b}- C_i^{(n+1)}(\bm{b})\rVert^2_2  \le \tilde{J}^{(n+1)}   
\end{split}
\end{equation}

Following (\ref{eq:mse_ineq_1}) and (\ref{eq:mse_ineq_2}), we find that the quantization MSE is non-increasing for each iteration, that is, $J^{(1)} \ge J^{(2)} \ge J^{(3)} \ge \ldots \ge J^{(M)}$ where $M$ is the maximum number of iterations. 
%Therefore, we can say that if the algorithm converges, then it must be that it has converged to a local minimum. 
\hfill $\blacksquare$


\begin{figure}
    \begin{center}
    \includegraphics[width=0.5\textwidth]{sections//figures/mse_vs_iter.pdf}
    \end{center}
    \caption{\small NMSE vs iterations during LO-BCQ compared to other block quantization proposals}
    \label{fig:nmse_vs_iter}
\end{figure}

Figure \ref{fig:nmse_vs_iter} shows the empirical convergence of LO-BCQ across several block lengths and number of codebooks. Also, the MSE achieved by LO-BCQ is compared to baselines such as MXFP and VSQ. As shown, LO-BCQ converges to a lower MSE than the baselines. Further, we achieve better convergence for larger number of codebooks ($N_c$) and for a smaller block length ($L_b$), both of which increase the bitwidth of BCQ (see Eq \ref{eq:bitwidth_bcq}).


\subsection{Additional Accuracy Results}
%Table \ref{tab:lobcq_config} lists the various LOBCQ configurations and their corresponding bitwidths.
\begin{table}
\setlength{\tabcolsep}{4.75pt}
\begin{center}
\caption{\label{tab:lobcq_config} Various LO-BCQ configurations and their bitwidths.}
\begin{tabular}{|c||c|c|c|c||c|c||c|} 
\hline
 & \multicolumn{4}{|c||}{$L_b=8$} & \multicolumn{2}{|c||}{$L_b=4$} & $L_b=2$ \\
 \hline
 \backslashbox{$L_A$\kern-1em}{\kern-1em$N_c$} & 2 & 4 & 8 & 16 & 2 & 4 & 2 \\
 \hline
 64 & 4.25 & 4.375 & 4.5 & 4.625 & 4.375 & 4.625 & 4.625\\
 \hline
 32 & 4.375 & 4.5 & 4.625& 4.75 & 4.5 & 4.75 & 4.75 \\
 \hline
 16 & 4.625 & 4.75& 4.875 & 5 & 4.75 & 5 & 5 \\
 \hline
\end{tabular}
\end{center}
\end{table}

%\subsection{Perplexity achieved by various LO-BCQ configurations on Wikitext-103 dataset}

\begin{table} \centering
\begin{tabular}{|c||c|c|c|c||c|c||c|} 
\hline
 $L_b \rightarrow$& \multicolumn{4}{c||}{8} & \multicolumn{2}{c||}{4} & 2\\
 \hline
 \backslashbox{$L_A$\kern-1em}{\kern-1em$N_c$} & 2 & 4 & 8 & 16 & 2 & 4 & 2  \\
 %$N_c \rightarrow$ & 2 & 4 & 8 & 16 & 2 & 4 & 2 \\
 \hline
 \hline
 \multicolumn{8}{c}{GPT3-1.3B (FP32 PPL = 9.98)} \\ 
 \hline
 \hline
 64 & 10.40 & 10.23 & 10.17 & 10.15 &  10.28 & 10.18 & 10.19 \\
 \hline
 32 & 10.25 & 10.20 & 10.15 & 10.12 &  10.23 & 10.17 & 10.17 \\
 \hline
 16 & 10.22 & 10.16 & 10.10 & 10.09 &  10.21 & 10.14 & 10.16 \\
 \hline
  \hline
 \multicolumn{8}{c}{GPT3-8B (FP32 PPL = 7.38)} \\ 
 \hline
 \hline
 64 & 7.61 & 7.52 & 7.48 &  7.47 &  7.55 &  7.49 & 7.50 \\
 \hline
 32 & 7.52 & 7.50 & 7.46 &  7.45 &  7.52 &  7.48 & 7.48  \\
 \hline
 16 & 7.51 & 7.48 & 7.44 &  7.44 &  7.51 &  7.49 & 7.47  \\
 \hline
\end{tabular}
\caption{\label{tab:ppl_gpt3_abalation} Wikitext-103 perplexity across GPT3-1.3B and 8B models.}
\end{table}

\begin{table} \centering
\begin{tabular}{|c||c|c|c|c||} 
\hline
 $L_b \rightarrow$& \multicolumn{4}{c||}{8}\\
 \hline
 \backslashbox{$L_A$\kern-1em}{\kern-1em$N_c$} & 2 & 4 & 8 & 16 \\
 %$N_c \rightarrow$ & 2 & 4 & 8 & 16 & 2 & 4 & 2 \\
 \hline
 \hline
 \multicolumn{5}{|c|}{Llama2-7B (FP32 PPL = 5.06)} \\ 
 \hline
 \hline
 64 & 5.31 & 5.26 & 5.19 & 5.18  \\
 \hline
 32 & 5.23 & 5.25 & 5.18 & 5.15  \\
 \hline
 16 & 5.23 & 5.19 & 5.16 & 5.14  \\
 \hline
 \multicolumn{5}{|c|}{Nemotron4-15B (FP32 PPL = 5.87)} \\ 
 \hline
 \hline
 64  & 6.3 & 6.20 & 6.13 & 6.08  \\
 \hline
 32  & 6.24 & 6.12 & 6.07 & 6.03  \\
 \hline
 16  & 6.12 & 6.14 & 6.04 & 6.02  \\
 \hline
 \multicolumn{5}{|c|}{Nemotron4-340B (FP32 PPL = 3.48)} \\ 
 \hline
 \hline
 64 & 3.67 & 3.62 & 3.60 & 3.59 \\
 \hline
 32 & 3.63 & 3.61 & 3.59 & 3.56 \\
 \hline
 16 & 3.61 & 3.58 & 3.57 & 3.55 \\
 \hline
\end{tabular}
\caption{\label{tab:ppl_llama7B_nemo15B} Wikitext-103 perplexity compared to FP32 baseline in Llama2-7B and Nemotron4-15B, 340B models}
\end{table}

%\subsection{Perplexity achieved by various LO-BCQ configurations on MMLU dataset}


\begin{table} \centering
\begin{tabular}{|c||c|c|c|c||c|c|c|c|} 
\hline
 $L_b \rightarrow$& \multicolumn{4}{c||}{8} & \multicolumn{4}{c||}{8}\\
 \hline
 \backslashbox{$L_A$\kern-1em}{\kern-1em$N_c$} & 2 & 4 & 8 & 16 & 2 & 4 & 8 & 16  \\
 %$N_c \rightarrow$ & 2 & 4 & 8 & 16 & 2 & 4 & 2 \\
 \hline
 \hline
 \multicolumn{5}{|c|}{Llama2-7B (FP32 Accuracy = 45.8\%)} & \multicolumn{4}{|c|}{Llama2-70B (FP32 Accuracy = 69.12\%)} \\ 
 \hline
 \hline
 64 & 43.9 & 43.4 & 43.9 & 44.9 & 68.07 & 68.27 & 68.17 & 68.75 \\
 \hline
 32 & 44.5 & 43.8 & 44.9 & 44.5 & 68.37 & 68.51 & 68.35 & 68.27  \\
 \hline
 16 & 43.9 & 42.7 & 44.9 & 45 & 68.12 & 68.77 & 68.31 & 68.59  \\
 \hline
 \hline
 \multicolumn{5}{|c|}{GPT3-22B (FP32 Accuracy = 38.75\%)} & \multicolumn{4}{|c|}{Nemotron4-15B (FP32 Accuracy = 64.3\%)} \\ 
 \hline
 \hline
 64 & 36.71 & 38.85 & 38.13 & 38.92 & 63.17 & 62.36 & 63.72 & 64.09 \\
 \hline
 32 & 37.95 & 38.69 & 39.45 & 38.34 & 64.05 & 62.30 & 63.8 & 64.33  \\
 \hline
 16 & 38.88 & 38.80 & 38.31 & 38.92 & 63.22 & 63.51 & 63.93 & 64.43  \\
 \hline
\end{tabular}
\caption{\label{tab:mmlu_abalation} Accuracy on MMLU dataset across GPT3-22B, Llama2-7B, 70B and Nemotron4-15B models.}
\end{table}


%\subsection{Perplexity achieved by various LO-BCQ configurations on LM evaluation harness}

\begin{table} \centering
\begin{tabular}{|c||c|c|c|c||c|c|c|c|} 
\hline
 $L_b \rightarrow$& \multicolumn{4}{c||}{8} & \multicolumn{4}{c||}{8}\\
 \hline
 \backslashbox{$L_A$\kern-1em}{\kern-1em$N_c$} & 2 & 4 & 8 & 16 & 2 & 4 & 8 & 16  \\
 %$N_c \rightarrow$ & 2 & 4 & 8 & 16 & 2 & 4 & 2 \\
 \hline
 \hline
 \multicolumn{5}{|c|}{Race (FP32 Accuracy = 37.51\%)} & \multicolumn{4}{|c|}{Boolq (FP32 Accuracy = 64.62\%)} \\ 
 \hline
 \hline
 64 & 36.94 & 37.13 & 36.27 & 37.13 & 63.73 & 62.26 & 63.49 & 63.36 \\
 \hline
 32 & 37.03 & 36.36 & 36.08 & 37.03 & 62.54 & 63.51 & 63.49 & 63.55  \\
 \hline
 16 & 37.03 & 37.03 & 36.46 & 37.03 & 61.1 & 63.79 & 63.58 & 63.33  \\
 \hline
 \hline
 \multicolumn{5}{|c|}{Winogrande (FP32 Accuracy = 58.01\%)} & \multicolumn{4}{|c|}{Piqa (FP32 Accuracy = 74.21\%)} \\ 
 \hline
 \hline
 64 & 58.17 & 57.22 & 57.85 & 58.33 & 73.01 & 73.07 & 73.07 & 72.80 \\
 \hline
 32 & 59.12 & 58.09 & 57.85 & 58.41 & 73.01 & 73.94 & 72.74 & 73.18  \\
 \hline
 16 & 57.93 & 58.88 & 57.93 & 58.56 & 73.94 & 72.80 & 73.01 & 73.94  \\
 \hline
\end{tabular}
\caption{\label{tab:mmlu_abalation} Accuracy on LM evaluation harness tasks on GPT3-1.3B model.}
\end{table}

\begin{table} \centering
\begin{tabular}{|c||c|c|c|c||c|c|c|c|} 
\hline
 $L_b \rightarrow$& \multicolumn{4}{c||}{8} & \multicolumn{4}{c||}{8}\\
 \hline
 \backslashbox{$L_A$\kern-1em}{\kern-1em$N_c$} & 2 & 4 & 8 & 16 & 2 & 4 & 8 & 16  \\
 %$N_c \rightarrow$ & 2 & 4 & 8 & 16 & 2 & 4 & 2 \\
 \hline
 \hline
 \multicolumn{5}{|c|}{Race (FP32 Accuracy = 41.34\%)} & \multicolumn{4}{|c|}{Boolq (FP32 Accuracy = 68.32\%)} \\ 
 \hline
 \hline
 64 & 40.48 & 40.10 & 39.43 & 39.90 & 69.20 & 68.41 & 69.45 & 68.56 \\
 \hline
 32 & 39.52 & 39.52 & 40.77 & 39.62 & 68.32 & 67.43 & 68.17 & 69.30  \\
 \hline
 16 & 39.81 & 39.71 & 39.90 & 40.38 & 68.10 & 66.33 & 69.51 & 69.42  \\
 \hline
 \hline
 \multicolumn{5}{|c|}{Winogrande (FP32 Accuracy = 67.88\%)} & \multicolumn{4}{|c|}{Piqa (FP32 Accuracy = 78.78\%)} \\ 
 \hline
 \hline
 64 & 66.85 & 66.61 & 67.72 & 67.88 & 77.31 & 77.42 & 77.75 & 77.64 \\
 \hline
 32 & 67.25 & 67.72 & 67.72 & 67.00 & 77.31 & 77.04 & 77.80 & 77.37  \\
 \hline
 16 & 68.11 & 68.90 & 67.88 & 67.48 & 77.37 & 78.13 & 78.13 & 77.69  \\
 \hline
\end{tabular}
\caption{\label{tab:mmlu_abalation} Accuracy on LM evaluation harness tasks on GPT3-8B model.}
\end{table}

\begin{table} \centering
\begin{tabular}{|c||c|c|c|c||c|c|c|c|} 
\hline
 $L_b \rightarrow$& \multicolumn{4}{c||}{8} & \multicolumn{4}{c||}{8}\\
 \hline
 \backslashbox{$L_A$\kern-1em}{\kern-1em$N_c$} & 2 & 4 & 8 & 16 & 2 & 4 & 8 & 16  \\
 %$N_c \rightarrow$ & 2 & 4 & 8 & 16 & 2 & 4 & 2 \\
 \hline
 \hline
 \multicolumn{5}{|c|}{Race (FP32 Accuracy = 40.67\%)} & \multicolumn{4}{|c|}{Boolq (FP32 Accuracy = 76.54\%)} \\ 
 \hline
 \hline
 64 & 40.48 & 40.10 & 39.43 & 39.90 & 75.41 & 75.11 & 77.09 & 75.66 \\
 \hline
 32 & 39.52 & 39.52 & 40.77 & 39.62 & 76.02 & 76.02 & 75.96 & 75.35  \\
 \hline
 16 & 39.81 & 39.71 & 39.90 & 40.38 & 75.05 & 73.82 & 75.72 & 76.09  \\
 \hline
 \hline
 \multicolumn{5}{|c|}{Winogrande (FP32 Accuracy = 70.64\%)} & \multicolumn{4}{|c|}{Piqa (FP32 Accuracy = 79.16\%)} \\ 
 \hline
 \hline
 64 & 69.14 & 70.17 & 70.17 & 70.56 & 78.24 & 79.00 & 78.62 & 78.73 \\
 \hline
 32 & 70.96 & 69.69 & 71.27 & 69.30 & 78.56 & 79.49 & 79.16 & 78.89  \\
 \hline
 16 & 71.03 & 69.53 & 69.69 & 70.40 & 78.13 & 79.16 & 79.00 & 79.00  \\
 \hline
\end{tabular}
\caption{\label{tab:mmlu_abalation} Accuracy on LM evaluation harness tasks on GPT3-22B model.}
\end{table}

\begin{table} \centering
\begin{tabular}{|c||c|c|c|c||c|c|c|c|} 
\hline
 $L_b \rightarrow$& \multicolumn{4}{c||}{8} & \multicolumn{4}{c||}{8}\\
 \hline
 \backslashbox{$L_A$\kern-1em}{\kern-1em$N_c$} & 2 & 4 & 8 & 16 & 2 & 4 & 8 & 16  \\
 %$N_c \rightarrow$ & 2 & 4 & 8 & 16 & 2 & 4 & 2 \\
 \hline
 \hline
 \multicolumn{5}{|c|}{Race (FP32 Accuracy = 44.4\%)} & \multicolumn{4}{|c|}{Boolq (FP32 Accuracy = 79.29\%)} \\ 
 \hline
 \hline
 64 & 42.49 & 42.51 & 42.58 & 43.45 & 77.58 & 77.37 & 77.43 & 78.1 \\
 \hline
 32 & 43.35 & 42.49 & 43.64 & 43.73 & 77.86 & 75.32 & 77.28 & 77.86  \\
 \hline
 16 & 44.21 & 44.21 & 43.64 & 42.97 & 78.65 & 77 & 76.94 & 77.98  \\
 \hline
 \hline
 \multicolumn{5}{|c|}{Winogrande (FP32 Accuracy = 69.38\%)} & \multicolumn{4}{|c|}{Piqa (FP32 Accuracy = 78.07\%)} \\ 
 \hline
 \hline
 64 & 68.9 & 68.43 & 69.77 & 68.19 & 77.09 & 76.82 & 77.09 & 77.86 \\
 \hline
 32 & 69.38 & 68.51 & 68.82 & 68.90 & 78.07 & 76.71 & 78.07 & 77.86  \\
 \hline
 16 & 69.53 & 67.09 & 69.38 & 68.90 & 77.37 & 77.8 & 77.91 & 77.69  \\
 \hline
\end{tabular}
\caption{\label{tab:mmlu_abalation} Accuracy on LM evaluation harness tasks on Llama2-7B model.}
\end{table}

\begin{table} \centering
\begin{tabular}{|c||c|c|c|c||c|c|c|c|} 
\hline
 $L_b \rightarrow$& \multicolumn{4}{c||}{8} & \multicolumn{4}{c||}{8}\\
 \hline
 \backslashbox{$L_A$\kern-1em}{\kern-1em$N_c$} & 2 & 4 & 8 & 16 & 2 & 4 & 8 & 16  \\
 %$N_c \rightarrow$ & 2 & 4 & 8 & 16 & 2 & 4 & 2 \\
 \hline
 \hline
 \multicolumn{5}{|c|}{Race (FP32 Accuracy = 48.8\%)} & \multicolumn{4}{|c|}{Boolq (FP32 Accuracy = 85.23\%)} \\ 
 \hline
 \hline
 64 & 49.00 & 49.00 & 49.28 & 48.71 & 82.82 & 84.28 & 84.03 & 84.25 \\
 \hline
 32 & 49.57 & 48.52 & 48.33 & 49.28 & 83.85 & 84.46 & 84.31 & 84.93  \\
 \hline
 16 & 49.85 & 49.09 & 49.28 & 48.99 & 85.11 & 84.46 & 84.61 & 83.94  \\
 \hline
 \hline
 \multicolumn{5}{|c|}{Winogrande (FP32 Accuracy = 79.95\%)} & \multicolumn{4}{|c|}{Piqa (FP32 Accuracy = 81.56\%)} \\ 
 \hline
 \hline
 64 & 78.77 & 78.45 & 78.37 & 79.16 & 81.45 & 80.69 & 81.45 & 81.5 \\
 \hline
 32 & 78.45 & 79.01 & 78.69 & 80.66 & 81.56 & 80.58 & 81.18 & 81.34  \\
 \hline
 16 & 79.95 & 79.56 & 79.79 & 79.72 & 81.28 & 81.66 & 81.28 & 80.96  \\
 \hline
\end{tabular}
\caption{\label{tab:mmlu_abalation} Accuracy on LM evaluation harness tasks on Llama2-70B model.}
\end{table}

%\section{MSE Studies}
%\textcolor{red}{TODO}


\subsection{Number Formats and Quantization Method}
\label{subsec:numFormats_quantMethod}
\subsubsection{Integer Format}
An $n$-bit signed integer (INT) is typically represented with a 2s-complement format \citep{yao2022zeroquant,xiao2023smoothquant,dai2021vsq}, where the most significant bit denotes the sign.

\subsubsection{Floating Point Format}
An $n$-bit signed floating point (FP) number $x$ comprises of a 1-bit sign ($x_{\mathrm{sign}}$), $B_m$-bit mantissa ($x_{\mathrm{mant}}$) and $B_e$-bit exponent ($x_{\mathrm{exp}}$) such that $B_m+B_e=n-1$. The associated constant exponent bias ($E_{\mathrm{bias}}$) is computed as $(2^{{B_e}-1}-1)$. We denote this format as $E_{B_e}M_{B_m}$.  

\subsubsection{Quantization Scheme}
\label{subsec:quant_method}
A quantization scheme dictates how a given unquantized tensor is converted to its quantized representation. We consider FP formats for the purpose of illustration. Given an unquantized tensor $\bm{X}$ and an FP format $E_{B_e}M_{B_m}$, we first, we compute the quantization scale factor $s_X$ that maps the maximum absolute value of $\bm{X}$ to the maximum quantization level of the $E_{B_e}M_{B_m}$ format as follows:
\begin{align}
\label{eq:sf}
    s_X = \frac{\mathrm{max}(|\bm{X}|)}{\mathrm{max}(E_{B_e}M_{B_m})}
\end{align}
In the above equation, $|\cdot|$ denotes the absolute value function.

Next, we scale $\bm{X}$ by $s_X$ and quantize it to $\hat{\bm{X}}$ by rounding it to the nearest quantization level of $E_{B_e}M_{B_m}$ as:

\begin{align}
\label{eq:tensor_quant}
    \hat{\bm{X}} = \text{round-to-nearest}\left(\frac{\bm{X}}{s_X}, E_{B_e}M_{B_m}\right)
\end{align}

We perform dynamic max-scaled quantization \citep{wu2020integer}, where the scale factor $s$ for activations is dynamically computed during runtime.

\subsection{Vector Scaled Quantization}
\begin{wrapfigure}{r}{0.35\linewidth}
  \centering
  \includegraphics[width=\linewidth]{sections/figures/vsquant.jpg}
  \caption{\small Vectorwise decomposition for per-vector scaled quantization (VSQ \citep{dai2021vsq}).}
  \label{fig:vsquant}
\end{wrapfigure}
During VSQ \citep{dai2021vsq}, the operand tensors are decomposed into 1D vectors in a hardware friendly manner as shown in Figure \ref{fig:vsquant}. Since the decomposed tensors are used as operands in matrix multiplications during inference, it is beneficial to perform this decomposition along the reduction dimension of the multiplication. The vectorwise quantization is performed similar to tensorwise quantization described in Equations \ref{eq:sf} and \ref{eq:tensor_quant}, where a scale factor $s_v$ is required for each vector $\bm{v}$ that maps the maximum absolute value of that vector to the maximum quantization level. While smaller vector lengths can lead to larger accuracy gains, the associated memory and computational overheads due to the per-vector scale factors increases. To alleviate these overheads, VSQ \citep{dai2021vsq} proposed a second level quantization of the per-vector scale factors to unsigned integers, while MX \citep{rouhani2023shared} quantizes them to integer powers of 2 (denoted as $2^{INT}$).

\subsubsection{MX Format}
The MX format proposed in \citep{rouhani2023microscaling} introduces the concept of sub-block shifting. For every two scalar elements of $b$-bits each, there is a shared exponent bit. The value of this exponent bit is determined through an empirical analysis that targets minimizing quantization MSE. We note that the FP format $E_{1}M_{b}$ is strictly better than MX from an accuracy perspective since it allocates a dedicated exponent bit to each scalar as opposed to sharing it across two scalars. Therefore, we conservatively bound the accuracy of a $b+2$-bit signed MX format with that of a $E_{1}M_{b}$ format in our comparisons. For instance, we use E1M2 format as a proxy for MX4.

\begin{figure}
    \centering
    \includegraphics[width=1\linewidth]{sections//figures/BlockFormats.pdf}
    \caption{\small Comparing LO-BCQ to MX format.}
    \label{fig:block_formats}
\end{figure}

Figure \ref{fig:block_formats} compares our $4$-bit LO-BCQ block format to MX \citep{rouhani2023microscaling}. As shown, both LO-BCQ and MX decompose a given operand tensor into block arrays and each block array into blocks. Similar to MX, we find that per-block quantization ($L_b < L_A$) leads to better accuracy due to increased flexibility. While MX achieves this through per-block $1$-bit micro-scales, we associate a dedicated codebook to each block through a per-block codebook selector. Further, MX quantizes the per-block array scale-factor to E8M0 format without per-tensor scaling. In contrast during LO-BCQ, we find that per-tensor scaling combined with quantization of per-block array scale-factor to E4M3 format results in superior inference accuracy across models. 


\printbibliography
% \bibliographystyle{abbrv-doi}
% \bibliography{biblio}

% \begin{thebibliography}{10} \end{thebibliography}

\end{document}
