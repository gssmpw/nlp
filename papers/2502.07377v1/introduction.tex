Nowadays, users increasingly share food-related content online by posting recipes, meal plans, or dietary advice.
While over one million recipes are already available on the Web \cite{salvador_learning_2017}, social media platforms further amplify this trend. 
For example, as of November $2024$, Instagram alone has more than $535$ million posts with the hashtag ``food.'' 
%\footnote{\url{https://www.instagram.com/explore/search/keyword/?q=\%23food}}. 
Several studies have analyzed these high levels of user activity trying to identify factors driving such high user engagement with food-related online postings. While individual post features such as positive language and emotion are generally related to user engagement \cite{barklamb_learning_2020}, the engagement factors related to food content are typically more intricate.
For example, temporal and seasonal effects along with the reputation and the social network of the user who created the posting strongly affect how people engage with food content \cite{rokicki_how_2017, starke_nudging_2021}. 
In addition, intrinsic food factors such as visual appeal are also typically associated with the user's food preferences and their engagement levels \cite{lee_visual_2023}. 

Such high levels of user involvement with food on social media raise a question about the health implications of this activity. For instance, it is still mostly unclear whether users engage more with food postings that promote healthy eating practices or with postings that, for instance, contain high-caloric meals \cite{garaus2021unhealthy}.
Recently, some initial studies explored the association between nutritional content and online engagement showing a positive correlation between engagement and nutritional density of food \cite{pancer_content_2022}. 
In this study, the authors analyzed $700$ Facebook posts featuring Buzzfeed's Tasty videos, showing that posts featuring calorie-dense meals receive more likes, shares, and comments.
These small-scale studies analyzing a few hundred food posts, give fruitful insight into the health implications of high user engagement with certain food-related content online.
However, it is not clear whether these findings generalize to a global online community that generates large amounts of food content and attracts vast amounts of user attention.

In this paper, we extend on those previous studies by examining engagement with food-related posts on Reddit, an online social platform hosting various communities of interest.
In particular, we focus on r/Food---a Reddit community related to sharing food online---by computing nutritional density from food post titles and investigating the association of this density to user engagement. 
With our paper, we expand the previous work in two important ways. 
First, we apply a robust embedding-based method for estimating nutritional content, such as calorie density, from the food post title only. 
Second, we conduct a large-scale analysis of over half a million Reddit food-related posts and analyze the factors of user engagement in these posts.
Specifically, we collect almost $600,000$ posts from the r/Food (all posts from $2017$ until $2022$) to analyze whether the macro-nutrient and caloric content of food shared in a posting are predictive of user engagement measured by the number of comments to that post.
Particularly, we focus on whether nutritional density is predictive of (i) engagement, operationalized as posts receiving at least one comment, and (ii) resonance, defined as the top 1\% of posts by the number of comments.
To isolate the relation between nutritional factors and engagement, we control for several non-food-related features such as seasonality, user tenure, or textual features. 

To estimate the association between nutritional content and user engagement we train a series of XGBoost classifiers \cite{chen_xgboost_2016} and use SHAP values \cite{lundberg_unified_2017} for a detailed explanation of the predictive power of nutritional content. Our findings indicate that even after controlling for temporal features and user tenure, posts featuring more nutrient-dense meals are positively associated with both engagement and resonance. In particular, these posts are more likely to obtain comments and, more specifically, they are also more likely to resonate with the community. We also find that some specific words in the post title such as ``cheese'', ``fried'', or ``pizza'' are predictive of strong engagement, while others such as ``potato'' or ``rice'' are associated with weak engagement. However, nutritional features are still able to improve the prediction performance of the XGBoost models that include such textual features, indicating a strong association between nutritional density and user engagement in food-related postings.

Our work provides a deeper understanding of the driving factors behind user engagement, particularly in relation to the nutritional content of food.
Moreover, the explainability of our models via SHAP values reveals the structure of posts, which resonate strongly with the users.
This enables the design of more engaging online initiatives aimed at encouraging healthy eating habits.
In addition, our approach to calculating the nutritional content of a meal from just the textual description can be used in dietary education, helping people understand the nutritional profile of their meals.
Beyond these applications, we demonstrate how large-scale analyses can unveil patterns of online user behavior.
In addition, we publish all of our code and data \footnote{\url{https://github.com/gabrielaozegovic/Food-posts-engagement-analysis}}.

