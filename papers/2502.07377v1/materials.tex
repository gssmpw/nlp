\subsection{Dataset}

\xhdr{Reddit}
%Reddit is an online platform consisting of subreddits,
Reddit is an online platform consisting of multiple topical communities, called ``subreddits,'' in which users engage in discussions.
Typically, subreddits are focused on a specific topic, and users write posts or comment on existing posts, forming a shared interest-centric community. 
Each subreddit has its own rules and guidelines on how to participate in that community. 
These rules typically define what to include in the post title and body, formatting instructions, or general instructions on communication tone. 

\xhdr{Food subreddit}
In this paper, we focus on r/Food, a subreddit dedicated to sharing meals. As of January $2025$, it is the $21$st largest subreddit, with around $24$ million subscribers\footnote{\url{https://www.reddit.com/best/communities/1/\#t5_2qh55}}. 
In particular, users post meals, following the rules of the subreddit: the post title must describe the meal.
Additionally, each post must include an original image of the meal, taken by the user who creates the post.
These rules ensure consistency across user posts and their focus on food. Even though the rules slightly changed over the years, the meal name had to be always included in the post title.


\xhdr{Data collection}
We collect data with Pushshift, a service that conducts large-scale crawls of Reddit \cite{baumgartner_pushshift_2020}.
We retrieve all submissions ($594,842$ posts) from r/Food subreddit from January $2017$ up to the end of December $2022$. 
For each post, we collect the number of comments the post received as a basic measurement of community engagement with a particular post. In addition, we collect further post information such as username or submission time.

\begin{table}[b]
\caption{\textit{Data filtering.} Number of posts at each stage of data preparation, from initial collection to posts included in the final analysis. We define resonant posts as the top 1\% of posts by the number of comments.}
\begin{tabular}{l|r}
                                         & \textbf{Value} \\ \hline
Collected posts                          & $594,842$        \\
Posts after preprocessing                & $509,479$        \\
Posts with macronutrient estimates       & $416,779$         \\
Posts with comments                      & $320,125$        \\ %todo check
Resonant posts                           & $3,219$          \\
\end{tabular}
\end{table}

\xhdr{Preprocessing}
In the first preprocessing step, we remove empty and deleted posts, as the community does not engage with such posts. Next, we remove duplicate posts, which we define as those made by the same user with the same title within five minutes. 
This leaves us with $509,479$ posts.
Lastly, to prepare the Reddit posts for the calculation of calories, we clean up the titles by removing special characters and emojis. 

\subsection{Nutritional Content Estimation}
To calculate the nutritional content of each meal, we use USDA’s FoodData Central database \cite{mckillop_fooddata_2021}. Specifically, we utilize three of its sources: (i) Foundation Foods, (ii) SR Legacy, and (iii) The Food and Nutrient Database for Dietary Studies.
In the database, nutrient information for each food entry is provided as density per $100$g.
We compute the nutritional content from the titles of Reddit posts by adapting the NutriTransform method \cite{ruprechter_2025}. 

Hence, we start by computing sentence embeddings \cite{reimers_sentence-bert_2019} for both Reddit post titles and the food database items. 
Using these embeddings, we compute the cosine similarity between a given Reddit post and all meals from the food database. We then select the five closest matches to the Reddit post given that they exceed a specified similarity threshold. We compute the similarity threshold by first taking a random sample of $5,000$ Reddit posts and computing their similarities to all the meals ($11,801$ food items) from the food database. As sentence embeddings typically result in vectors with substantial overall similarity (median similarity in our sample is $0.25$), we set the similarity threshold by computing the $99.9$th quantile of the similarity distribution as this quantile results in a sufficiently large number of highly similar meals. Thus, as the median similarity for the distribution of the $99.9$th quantile over our sample is $61.59$\%, we set the similarity threshold to $62$\%. 
We test the robustness of this similarity threshold by making additional computations with varying quantiles (e.g., $99.99$, $99$, $95$) and find no significant impact of the alternative similarity thresholds on our results.
After selecting the most similar meals from the database, we extract the calorie and macronutrient information for selected database matches and aggregate this information by computing similarity-weighted mean to obtain the nutritional content estimate of a given post. 
As the entries in the USDA's FoodData database are given per $100$g of a meal, all calculated calorie and macronutrient information also represent densities per $100$g of food.

Using our method, we compute the nutritional information for $307,799$ different meals, as multiple posts can contain the same meal (e.g. $1,591$ posts have the title ``Pizza'').
We exclude posts for which we did not find any matches in the food database, i.e., that exceeded the threshold, and posts where no suitable match is found in the food database or where the similarity score does not exceed the threshold.
Next, we check for potential outliers, which are all meals with less than $32$ calories (equivalent to $100$g of strawberries) or more than $717$ calories (equivalent to $100$g of butter).
After this final filtering step, we have a total of $306,592$ meals in $416,779$ posts that we use for further analysis.

\subsection{Explorative analysis}

\xhdr{Users}
A total of $146,203$ unique users contributed posts to the subreddit, with $62.6$\% posting only once.
The most active user made $882$ posts. Typically, more active users have more experience and the community engages stronger with their posts \cite{rokicki_how_2017}. 
In our dataset, the top $5$\% of users ($6,417$ users) according to the number of posts have at least $10$ posts each.

\xhdr{Comments}
The mean number of comments per post is nine, with a standard deviation of $43.4$, indicating significant variability in comment counts. In total, $320,125$ ($62.8$\%) posts received at least one comment.  
The maximal number of comments on a post is $2,447$, while the median is only two, and the third quartile is only six comments, indicating a strongly skewed distribution.  
This highlights the disparity between engagement and resonance---while the community engages with the majority of the posts, posts that strongly resonate with the community (top 1\%) receive at least $150$ comments.

\begin{figure*}[t]
    \centering
    \begin{subfigure}[t]{0.348\textwidth}
        \centering
        \vspace{0pt}
          \includegraphics[width=\textwidth, height=3.85cm]{images/num_posts_over_time_line.pdf}
          \caption{Number of Posts by Year}
          \Description{num_posts_over_time_line}
        \label{fig:posts_per_year}
    \end{subfigure}
    \hfill
    \begin{subfigure}[t]{0.372\textwidth}
        \centering
        \vspace{0pt}
          \includegraphics[width=\textwidth, height=4cm]{images/num_posts_per_year_line.pdf}
          \caption{Number of posts by Month}
          \Description{num_posts_per_year_line}
        \label{fig:posts_per_month}
        \end{subfigure}
    \hfill
    \begin{subfigure}[t]{0.24\textwidth}
        \centering
        \vspace{0pt}
        \includegraphics[width = \textwidth, height=4cm]{images/weekend_day_bar_correct.pdf}
        \caption{Number of Posts by Day Type and Time of Day Quartiles}
        \Description{weekend}
        \label{fig:posts_weekend}
        \end{subfigure}

    \medskip

    \centering
    \begin{subfigure}[t]{0.348\textwidth}
        \centering
        \vspace{0pt}
          \includegraphics[width=\textwidth, height=3.85cm]{images/num_comm_over_time_line.pdf}
          \caption{Engagement Level by Year}
          \Description{num_comm_over_time_line}
        \label{fig:comm_per_year}
    \end{subfigure}
    \hfill
    \begin{subfigure}[t]{0.372\textwidth}
        \centering
        \vspace{0pt}
          \includegraphics[width=\textwidth, height=4cm]{images/num_comm_per_year_line.pdf}
          \caption{Engagement Level by Month}
          \Description{num_comm_per_year_line}
        \label{fig:comm_per_month}
        \end{subfigure}
    \hfill
    \begin{subfigure}[t]{0.24\textwidth}
        \centering
        \vspace{0pt}
        \includegraphics[width = \textwidth, height=4cm]{images/weekend_day_bar_comments_correct.pdf}
        \caption{Engagement Level by Day Type and Time of Day Quartiles}
        \Description{comm weekend}
        \label{fig:comm_weekend}
    \end{subfigure}
    \caption{
    \textit{Posts and comments in r/Food over time.} We present how postings and comments developed from $2017$ until $2023$ across different temporal scales including yearly, monthly, weekly, and daily trends.
    In (\subref{fig:posts_per_year}) we present the number of posts over the years. We observe a positive trend before the COVID-19 pandemic with a noticeable peak during the pandemic and a drop afterwards to pre-pandemic levels.
    Monthly posting activity in (\subref{fig:posts_per_month}) is rather consistent except for a peak between March and June $2020$ during the pandemic.
    In (\subref{fig:posts_weekend}) we observe that more posts are created on weekdays than on weekends (left) and that most posts are created in the afternoon in the eastern USA (Q4, right). The bottom row shows the same diagrams for comments.
    In (\subref{fig:comm_per_year}) we observe a gradual increase in commenting activity over time with the highest activity levels during the pandemic and a sharp drop after the pandemic.
    This observation is also reflected in (\subref{fig:comm_per_month}), where we see constant high levels of comments in $2020$. We also see a seasonal spike in January possibly due to the holiday season.
    In (\subref{fig:comm_weekend}), comments mirror posting activity, with more comments over the weekdays (left). On the other hand, the peak in comments is in the morning (Q3, right).
    }
    \label{fig:temporal}
\end{figure*}


\xhdr{Scores}
Each Reddit post has a score, determined by the difference of community ``upvotes'' and ``downvotes''. 
The mean score is $245$, with a standard deviation of $1,740.03$, indicating high variability. The median score is only $23$, signaling again a skewed distribution where most posts receive relatively modest scores while the highest score is $70,308$.
In this paper, we do not use score as an engagement metric and opt for comments, which require more effort from the users. In addition, the score has a strong positive correlation with the number of comments ($\rho = 0.627, p < 0.001$), indicating that comments are a comprehensive representation of engagement. %Spearman


\xhdr{Temporal characteristics}
In Figure \ref{fig:temporal} we depict the temporal development of activity and user engagement in r/Food. The number of posts steadily increased over time (Fig. \ref{fig:posts_per_year}), peaking in $2020$, likely due to the COVID-19 pandemic as this surge can be potentially attributed to the widespread lockdowns, increased interest in food, and the shift towards consuming more meals at home \cite{gligoric_population-scale_2022}.
Following a few months of the COVID-19 outbreak, the number of posts rapidly dropped, with a brief increase at the beginning of $2021$.
Subsequently, the number of posts continued to decrease, returning to pre-pandemic levels and even falling further. 

When comparing the monthly post counts across years (Fig. \ref{fig:posts_per_month}), a similar pattern is observed for most years, except for $2020$.
Specifically, there is a noticeable spike in the number of posts during March $2020$, corresponding to the onset of the pandemic and lockdowns.
We show the distribution of postings over weekdays and weekends as well as the time of the day in Fig. \ref{fig:posts_weekend}. In the dataset, the posting time is stored in UTC time. 
As the majority of Reddit traffic comes from the USA (cf. Reddit traffic as of March 2024\footnote{\url{https://www.statista.com/statistics/325144/reddit-global-active-user-distribution/}}) with time zones ranging from EST (UTC - $5$) to PST (UTC - $8$), we interpret the time of the day results using USA eastern times. The exact time ranges in different time zones are presented in Table \ref{table:timezones}.
Hence, we observe that more posts are made on weekdays ($279,299$) than on weekends ($137,480$). Further, posting activity peaks in the afternoon in the eastern USA (likely reflecting users' lunchtime), accounting for $33.3$\% of total posts. 
This is followed by $29.3$\% of posts made in the evening and $26.4$\% in the morning in the eastern USA. The least amount of activity occurs during the night, with only $11.6$\% of posts.

\begin{table}[b]
\centering
\caption{\textit{Time of Day Quartiles.} The dataset contains times in UTC. As most Reddit users are from the USA, we interpret these times in both EST and PST.}
\begin{tabular}{@{}l l l l@{}}
\toprule
Quartile & UTC              & EST              & PST              \\ \midrule
Q1 (evening) & $12$ AM - $6$ AM  & $7$ PM - $1$ AM & $4$ PM - $10$ PM \\
Q2 (night)   & $6$ AM - $12$ PM  & $1$ AM - $7$ AM & $10$ PM - $4$ AM \\
Q3 (morning) & $12$ PM - $6$ PM & $7$ AM - $1$ PM & $4$ AM - $10$ AM \\
Q4 (afternoon) & $6$ PM - $12$ AM & $1$ PM - $7$ PM & $10$ AM - $4$ PM \\
\bottomrule
\end{tabular}
\label{table:timezones}
\end{table}

In the bottom row of Figure \ref{fig:temporal} we show the same temporal analysis for comments. In these figures, we categorize comments according to the time of their original postings. For instance, if a post is made in June $2020$, we treat all its comments as if they were made in June $2020$, even if they are posted at a later date. The commenting activity shows a gradual increase until $2020$ (Fig. \ref{fig:comm_per_year}).
Just before $2020$, there was an abrupt drop in the number of comments, followed by a sharp increase.
This trend aligns with the rise in posting behavior during that time and the onset of the COVID-19 pandemic. Another notable increase occurred at the beginning of $2021$.
Subsequently, there has been a steady decline, with the number of comments falling below pre-pandemic levels. There is no clear seasonality observed when comparing the monthly number of comments (Fig. \ref{fig:comm_per_month}).
The highest number of comments is recorded throughout $2020$, likely due to the pandemic.

More comments were made on posts published on weekdays compared to weekends (Fig. \ref{fig:comm_weekend}), which is consistent with the higher number of posts being made during the weekdays. Posts made in the morning in the eastern USA receive the highest number of comments ($1,280,826$ comments, $34.05$\% of total). This large number of comments could be attributed to users' activity during lunchtime and time after work, where they engage with posts made previously in the day.
Posts made in the afternoon ($1,093,762$ comments, $29.08$\%) closely follow. Next are posts made in the evening ($928,398$ comments, $24.68$\%), while posts made during the night receive the least comments ($458,086$ comments, $12.18$\%).


\xhdr{Tags}
According to the current subreddit rules, each post must include a tag indicating the context of the meal: whether the user prepared it at home, whether the user works in the food industry and prepared it, or whether the user purchased it without personal preparation. The majority of meals, $75$\%, were prepared at home by the users, while $19$\% were purchased without any preparation, and $1.5$\% were prepared by food industry professionals. 
The remaining $4.5$\% of posts either lack a tag, most likely due to earlier subreddit policies of not enforcing the tag structure, or these posts include a user-chosen tag.

\begin{figure*}[t]
    \centering
    \begin{subfigure}{0.25\textwidth}
        \centering
        \includegraphics[width=\textwidth]{images/n/rq1_calorie_distribution.pdf}
        \caption{Calorie Densities}
        \Description{calorie distribution}
        \label{fig:rq1_calorie}
    \end{subfigure}
    \hfill
    \begin{subfigure}{0.24\textwidth}
        \centering
        \includegraphics[width =\textwidth]{images/n/rq1_weighted_protein_distribution.pdf}
        \caption{Protein Densities}
        \Description{protein distribution}
        \label{fig:rq1_protein}
        \end{subfigure}
    \hfill
    \begin{subfigure}{0.24\textwidth}
        \centering
        \includegraphics[width =\textwidth]{images/n/rq1_weighted_carb_distribution.pdf}
        \caption{Carbohydrate Densities}
        \Description{carb distribution}
        \label{fig:rq1_carb}
        \end{subfigure}
    \hfill
    \begin{subfigure}{0.24\textwidth}
        \centering
        \includegraphics[width = \textwidth]{images/n/rq1_weighted_fat_distribution.pdf}
        \caption{Fat Densities}
        \Description{fat distribution}
        \label{fig:rq1_fat}
        \end{subfigure}


    \medskip


    \centering
    \begin{subfigure}{0.25\textwidth}
        \centering
        \includegraphics[width=\textwidth]{images/n/rq2_calorie_distribution.pdf}
        \caption{Calorie Densities}
        \Description{calorie distribution}
        \label{fig:rq2_calorie}
    \end{subfigure}
    \hfill
    \begin{subfigure}{0.24\textwidth}
        \centering
        \includegraphics[width =\textwidth]{images/n/rq2_weighted_protein_distribution.pdf}
        \caption{Protein Densities}
        \Description{protein distribution}
        \label{fig:rq2_protein}
        \end{subfigure}
    \hfill
    \begin{subfigure}{0.24\textwidth}
        \centering
        \includegraphics[width =\textwidth]{images/n/rq2_weighted_carb_distribution.pdf}
        \caption{Carbohydrate Densities}
        \Description{carb distribution}
        \label{fig:rq2_carb}
        \end{subfigure}
    \hfill
    \begin{subfigure}{0.24\textwidth}
        \centering
        \includegraphics[width = \textwidth]{images/n/rq2_weighted_fat_distribution.pdf}
        \caption{Fat Densities}
        \Description{fat distribution}
        \label{fig:rq2_fat}
        \end{subfigure}

        
    \caption{
    \textit{Nutritional content distribution of food in r/Food posts}. We illustrate the distribution of calories (\subref{fig:rq1_calorie}, \subref{fig:rq2_calorie}) and macro-nutrients (\subref{fig:rq1_protein}--\subref{fig:rq1_fat}, \subref{fig:rq2_protein}--\subref{fig:rq2_fat}) per $100$g of food, across meal in (i) engaging (red) and non-engaging (blue) posts, and (ii) resonant (red) and non-resonant (blue) posts.
    The calorie content is measured in kCal per $100$g, while macro-nutrients are measured in grams as fractions of $100$g total.
    We observe that the majority of posts fall within the moderate calorie range, between $100$ and $300$ kCal.
    \textit{Top row:} Calorie densities of posts with comments and without comments appear similar but differ significantly in means (\subref{fig:rq1_calorie}).
    We observe a steep decline in the protein (\subref{fig:rq1_protein}) density, with most posts having less than $20$g of protein, suggesting a prevalence of low-to-moderate protein meals.
    Carbohydrates (\subref{fig:rq1_carb}) span over a wider range. While most posts have less than $20$g, there is a consistent amount of carb-rich food as well, as indicated by the long tail in their distributions. 
    Fat (\subref{fig:rq1_fat}) distribution peaks around $10$-$15$g, with most posts containing moderate fat content.
    \textit{Bottom row:} Distribution disparities are more prominent when comparing resonant vs. non-resonant posts. Posts that do not resonate with the community peak at around $150$ kCal, while posts that do resonate peak at $300$ kCal (\subref{fig:rq2_calorie}). We observe similar behavior in all other macronutrient densities, with distributions for resonant posts being shifted to the right as compared to non-resonant posts. (\subref{fig:rq2_protein}--\subref{fig:rq2_fat}).
    }
    \label{fig:macros}
\end{figure*}

\xhdr{Engagement levels}
We operationalize engagement by the number of comments and define posts with at least one comment to be engaging posts and without comments to be non-engaging posts. Further, we define resonant and non-resonant posts by categorizing posts in the top $1$\% by comment count (i.e., the first percentile) as resonant ($3,219$ posts), while those with $0$ or $1$ comment are considered non-resonant ($157,470$ posts).
Note that slight variations in the definition of low resonance, such as considering only posts without comments, posts with just one comment, or posts with up to five comments, did not impact the results. To obtain balanced classes for our prediction experiment (cf. Sec. \ref{sec:prediction})
we randomly sample $3,219$ non-resonant posts, resulting in $6,438$ posts for further analysis. 

\xhdr{Nutritional Content Analysis}
We show the distributions of macronutrient content of meals in Figure \ref{fig:macros}. 
Specifically, we compare the nutritional content distributions of posts with and without user engagement (top row Fig. \ref{fig:macros}), as well as resonant vs. non-resonant posts (bottom row Fig. \ref{fig:macros}). 
The majority of posts fall within the moderate calorie range, from $100$ to $300$ kcal per $100$g of food.
When comparing posts with and without engagement, the calorie distribution appears similar (Fig. \ref{fig:rq1_calorie}), although the difference in means is statistically significant ($p < 10^{-235}$, Mann-Whitney-U test).
The calorie distribution is bimodal, with one peak at around $150$ calories and another at around $300$ calories.

Similarly to the calorie distributions the distributions of other macronutrients appear similar to each other, but all the differences in means are statistically significant (all $p < 10^{-6}$).
For example, the vast majority of meals contain up to $20$g of protein (Fig. \ref{fig:rq1_protein}).
Posts without engagement show a peak at around $5$g of protein, with a gradual decline in posts as protein content increases.
In contrast, posts with engagement exhibit a spike at around $5$g of protein, followed by another increase at just over $10$g of protein. 
Posts with engagement generally feature meals with higher carbohydrate content, while posts without engagement tend to feature meals with lower carbohydrate content (Fig. \ref{fig:rq1_carb}).
The fat distribution is similar, with posts without engagement tending to feature lower-fat meals (Fig. \ref{fig:rq1_fat}).

There is a clear difference when comparing resonant to non-resonant posts (bottom row Fig. \ref{fig:macros}).
A significant difference in means is observed in the calorie distribution ($p < 10^{-53}$).
Most non-resonant posts contain meals with fewer than $150$ calories, while the majority of resonant posts contain meals with around $300$ calories (Fig. \ref{fig:rq2_calorie}).
Beyond $300$ calories, the number of posts that resonate with users is constantly higher than the number of posts that do not resonate.
While both types of posts tend to feature low-protein meals, higher-protein meals are more often found in resonant posts (Fig. \ref{fig:rq2_protein}). However, there is no significant difference in means between the protein distributions ($p = 0.1$).
Similarly, both low-carbohydrate and low-fat values are associated with non-resonant posts (Fig. \ref{fig:rq2_carb} and Fig. \ref{fig:rq2_fat}).
Conversely, higher carbohydrate and fat values are linked to posts that resonate with the community.
The means of both these macronutrients are significantly different between resonant and non-resonant posts ($p < 10^{-27}$).