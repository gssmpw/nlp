\xhdr{Food preferences and food choices} Physical food features affect how individuals respond to food.
Brain fMRI studies have shown that people's response to food depends on the food calorie density. 
Specifically, exposure to calorie-dense food provokes palatable and satiating feelings, while low-calorie food provokes feelings of hunger \cite{killgore_affect_2006, killgore_cortical_2003}.
This aligns with the findings that food framed as healthy tends to make people feel hungrier compared to the same food framed as tasty \cite{finkelstein_when_2010}.
Apart from food features, social influence plays an important part in choices, with individuals often opting for healthier options when their eating partners choose food perceived as healthier based on food-pyramid recommendations \cite{gligoric_formation_2021}. 
This reflects the broader impact of social cues on food decision-making.
For example, even interactions with strangers can shape food choices.
People tend to mimic a large portion size ordered by a thin person but opt for smaller portions when a person appears obese \cite{mcferran_ill_2010}.
Similarly, students are more likely to purchase a food item if the person ahead of them in line buys the same item \cite{gligoric_food_2024}.
Social networks can further intensify these effects. 
A long-term study revealed that a person is up to $57$\% more likely to become obese if someone in their close social circle becomes obese \cite{christakis_spread_2007}.
Such traditional studies usually require active participant involvement, which can result in small sample sizes (e.g., n=$59$, n$=139$) \cite{houben_guilty_2010, serrano-gonzalez_developmental_2021}.
Moreover, controlled experiments might fail to catch nuances of real-world food decisions.
Also, a notable pattern suggests that behavior online reflects offline behavior.
Research indicates that individuals who decide to go on a diet (signified by considering the purchase of a diet-related book) tend to search for lower-calorie meals.
Furthermore, the same study found the nutrient content of commonly searched meals over time, such as sodium levels, has been shown to correlate with hospital admissions for health issues caused by high sodium consumption \cite{west_cookies_2013}.
Similarly, discussions about high-calorie foods on Twitter correlate with state-wide obesity rates in the United States \cite{abbar_you_2015}. 

Similar to other online studies and in contrast to controlled studies, we utilize online data to explore food preferences on a large scale.
By analyzing online interactions, we aim to gain insights into food preferences using a larger and more diverse online sample.

\xhdr{Food and social media} 
Social media engagement is a complex phenomenon, influenced by various factors such as content or social media algorithms \cite{yan_evolution_2024}. 
Studies have shown that visual content generates higher engagement than text-based posts \cite{generator_examination_2024}.
Moreover, persuasive content including emotional and humorous content, is also linked to higher engagement \cite{lee_effect_2013}. 
While image-based content is more successful than video content, all posts benefit from including interaction cues \cite{moran_message_2019}. 
Specifically on Reddit, images and captions combined are the most accurate predictor of user engagement \cite{hessel_cats_2017}. 

In research on food content in social media, a study by Philp et al. \cite{philp_predicting_2022} analyzed user engagement with social media restaurant accounts. 
The authors investigated how visual characteristics of posts, particularly food appearance, are related to engagement rates.
The findings suggested that users are more likely to interact with posts featuring food with a more typical appearance.
They defined typical appearance as easily recognized, using Google Vision AI's classification confidence scores, with higher scores indicating more typical food appearances.
Such findings are corroborated by further studies that investigated engagement with food content prioritizing visual aspects \cite{lee_visual_2023, starke_nudging_2021}, suggesting that more visually appealing food increases purchase intentions and people are more likely to choose healthier foods if those foods are presented in a more visually appealing way.
Another study by Barklamb et al. \cite{barklamb_learning_2020} investigated engagement with nutrition-related social media accounts, finding that on Facebook most engagement is driven by announcements, whereas on Instagram longer captions and providing health information provoked more engagement.
Another study analyzing $61$ popular nutrition-focused Instagram accounts with more than $100,000$ followers, found that these accounts typically post about healthy eating and recipes, weight loss, and physique-related goals \cite{denniss_nutrition-related_2023}. %\cite{rogers_communication_2022} - review
However, research involving almost 200 celebrities \cite{turnwald_nutritional_2022}, including athletes, music artists, actors, actresses, and television personalities showed that
food with less healthy Nutrient Profile Index (NPI) ratings received more engagement (more likes and comments). 
Notably, less than $5$\% of these posts were sponsored, suggesting that the appeal of unhealthy nutrition goes beyond advertisements and sponsorships.

Nevertheless, the role of nutritional content alone in driving engagement remains under-investigated, despite its potential implications for promoting healthier eating behaviors. Therefore, in our study, we focus on calorie and macronutrient content as a primary aspect to provide insights into how nutritional information impacts engagement patterns.
Moreover, we conduct an observational study of user engagement in settings where users directly interact with other users rather than brands or influencers.
These user-to-user interactions offer more authentic insights into user behavior, without the influence of advertisements and marketing strategies.


\xhdr{Estimating nutritional content}  
Discussions about calorie content on Twitter increased following the federal calorie labeling law implementation in the United States \cite{hswen_federal_2021}, highlighting an increase in public engagement with nutritional information.
Still, 
a recent study investigating over $1,000$ food-related Instagram posts revealed that the vast majority (over $90$\%) of the posts are missing nutritional information or provide low-quality information \cite{kabata_can_2022}.
This aligns with another Instagram study exploring popular diet hashtags, which found that less than $4$\% of images contained nutrition information \cite{lister_what_2024}. 
Therefore, some initial studies estimated the nutritional information from social media postings. For example, Turnwald et al. \cite{turnwald_nutritional_2022} manually labeled food using images and captions and matched them to entries in the food database.
In additional approaches, researchers have calculated calorie information by performing keyword matches between posts and entries in food databases or nutritional information websites \cite{sharma_measuring_2015, abbar_you_2015}. 
While convenient, this method is susceptible to issues arising from inconsistent phrasing and may require manual verification.

For our study, we use a calorie calculation method based on text embedding techniques, allowing us to aggregate similar meals while calculating the nutritional density. 
This makes our calorie and macronutrient estimation approach more robust to (random) variations in user-generated post titles.