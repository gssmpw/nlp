%%
%% This is file `sample-sigconf.tex',
%% generated with the docstrip utility.
%%
%% The original source files were:
%%
%% samples.dtx  (with options: `all,proceedings,bibtex,sigconf')
%% 
%% IMPORTANT NOTICE:
%% 
%% For the copyright see the source file.
%% 
%% Any modified versions of this file must be renamed
%% with new filenames distinct from sample-sigconf.tex.
%% 
%% For distribution of the original source see the terms
%% for copying and modification in the file samples.dtx.
%% 
%% This generated file may be distributed as long as the
%% original source files, as listed above, are part of the
%% same distribution. (The sources need not necessarily be
%% in the same archive or directory.)
%%
%%
%% Commands for TeXCount
%TC:macro \cite [option:text,text]
%TC:macro \citep [option:text,text]
%TC:macro \citet [option:text,text]
%TC:envir table 0 1
%TC:envir table* 0 1
%TC:envir tabular [ignore] word
%TC:envir displaymath 0 word
%TC:envir math 0 word
%TC:envir comment 0 0
%%
%%
%% The first command in your LaTeX source must be the \documentclass
%% command.
%%
%% For submission and review of your manuscript please change the
%% command to \documentclass[manuscript, screen, review]{acmart}.
%%
%% When submitting camera ready or to TAPS, please change the command
%% to \documentclass[sigconf]{acmart} or whichever template is required
%% for your publication.
%%
%%
%\documentclass[sigconf]{acmart}
%\documentclass[manuscript,screen,review]{acmart}
\documentclass[manuscript,screen]{acmart}

%%
%% \BibTeX command to typeset BibTeX logo in the docs
\AtBeginDocument{%
  \providecommand\BibTeX{{%
    Bib\TeX}}}

%% Rights management information.  This information is sent to you
%% when you complete the rights form.  These commands have SAMPLE
%% values in them; it is your responsibility as an author to replace
%% the commands and values with those provided to you when you
%% complete the rights form.
\setcopyright{acmlicensed}
\copyrightyear{2025}
%\acmYear{2025}
\acmDOI{XXXXXXX.XXXXXXX}

%% These commands are for a PROCEEDINGS abstract or paper.
\acmConference[Conference acronym 'XX]{Make sure to enter the correct
  conference title from your rights confirmation emai}{June 03--05,
  2018}{Woodstock, NY}
%%
%%  Uncomment \acmBooktitle if the title of the proceedings is different
%%  from ``Proceedings of ...''!
%%
%%\acmBooktitle{Woodstock '18: ACM Symposium on Neural Gaze Detection,
%%  June 03--05, 2018, Woodstock, NY}
\acmISBN{978-1-4503-XXXX-X/18/06}


%%
%% Submission ID.
%% Use this when submitting an article to a sponsored event. You'll
%% receive a unique submission ID from the organizers
%% of the event, and this ID should be used as the parameter to this command.
%%\acmSubmissionID{123-A56-BU3}

%%
%% For managing citations, it is recommended to use bibliography
%% files in BibTeX format.
%%
%% You can then either use BibTeX with the ACM-Reference-Format style,
%% or BibLaTeX with the acmnumeric or acmauthoryear sytles, that include
%% support for advanced citation of software artefact from the
%% biblatex-software package, also separately available on CTAN.
%%
%% Look at the sample-*-biblatex.tex files for templates showcasing
%% the biblatex styles.
%%

%%
%% The majority of ACM publications use numbered citations and
%% references.  The command \citestyle{authoryear} switches to the
%% "author year" style.
%%
%% If you are preparing content for an event
%% sponsored by ACM SIGGRAPH, you must use the "author year" style of
%% citations and references.
%% Uncommenting
%% the next command will enable that style.
%%\citestyle{acmauthoryear}

% Academic text is often much more legible if you give important paragraphs a
% concise name that describes what the paragraph is about. Use the \xhdr
% command for this.
\newcommand{\xhdr}[1]{\vspace{0.7mm}\noindent{{\bf #1.}}}

% Same as \xhdr, but without a period after the heading. Use this version if
% the heading is directly integrated into the first sentence of the paragraph;
% e.g., "\xhdrNoPeriod{Results} are shown in \Figref{fig}."
\newcommand{\xhdrNoPeriod}[1]{\vspace{1mm}\noindent{{\bf #1}}}


\usepackage{booktabs}
\usepackage{multirow}
\usepackage{tabularray}
\usepackage{subcaption}

\newcounter{gozcounter}
\DeclareRobustCommand{\goz}[1]{\textbf{/* {\color{ACMDarkBlue} #1} (goz) */}\stepcounter{gozcounter}\typeout{LaTeX Warning: goz comment \thegozcounter: #1 (line \the\inputlineno)}}
\newcounter{trucounter}
\DeclareRobustCommand{\tru}[1]{\textbf{/* {\color{ACMPurple} #1} (tru) */}\stepcounter{trucounter}\typeout{LaTeX Warning: tru comment \thetrucounter: #1 (line \the\inputlineno)}}
\newcounter{dhecounter}
\DeclareRobustCommand{\dhe}[1]{\textbf{/* {\color{ACMOrange} #1} (dhe) */}\stepcounter{dhecounter}\typeout{LaTeX Warning: dhe comment \thedhecounter: #1 (line \the\inputlineno)}}

\iffalse
    \renewcommand{\goz}[1]{}
    \renewcommand{\tru}[1]{}
    \renewcommand{\dhe}[1]{}
\fi

\definecolor{Gray}{rgb}{0.501,0.501,0.501}

%%
%% end of the preamble, start of the body of the document source.
\begin{document}

%%
%% The "title" command has an optional parameter,
%% allowing the author to define a "short title" to be used in page headers.
% \title{The Effect of Calorie Density on Engagement in Reddit's Food Communities}
\title{Reddit's Appetite: Predicting User Engagement with Nutritional Content}

%%
%% The "author" command and its associated commands are used to define
%% the authors and their affiliations.
%% Of note is the shared affiliation of the first two authors, and the
%% "authornote" and "authornotemark" commands
%% used to denote shared contribution to the research.

\author{Gabriela Ozegovic}
\affiliation{%
  \institution{Graz University of Technology}
  \city{Graz}
  \country{Austria}}
\email{ozegovic@tugraz.at}

\author{Thorsten Ruprechter}
\affiliation{%
  \institution{Graz University of Technology}
  \city{Graz}
  \country{Austria}}
\email{ruprechter@tugraz.at}


\author{Denis Helic}
\affiliation{%
  \institution{Graz University of Technology}
  \city{Graz}
  \country{Austria}}
\email{dhelic@tugraz.at}

%%
%% By default, the full list of authors will be used in the page
%% headers. Often, this list is too long, and will overlap
%% other information printed in the page headers. This command allows
%% the author to define a more concise list
%% of authors' names for this purpose.
\renewcommand{\shortauthors}{Ozegovic et al.}

%%
%% The abstract is a short summary of the work to be presented in the
%% article.
\begin{abstract}
The increased popularity of food communities on social media shapes the way people engage with food-related content. Due to the extensive consequences of such content on users' eating behavior, researchers have started studying the factors that drive user engagement with food in online platforms. However, while most studies focus on visual aspects of food content in social media, there exist only initial studies exploring the impact of nutritional content on user engagement. In this paper, we set out to close this gap and analyze food-related posts on Reddit, focusing on the association between the nutritional density of a meal and engagement levels, particularly the number of comments. Hence, we collect and empirically analyze almost 600,000 food-related posts and uncover differences in nutritional content between engaging and non-engaging posts. Moreover, we train a series of XGBoost models, and evaluate the importance of nutritional density while predicting whether users will comment on a post or whether a post will substantially resonate with the community. We find that nutritional features improve the baseline model’s accuracy by 4\%, with a positive contribution of calorie density towards prediction of engagement, suggesting that higher nutritional content is associated with higher user engagement in food-related posts. Our results provide valuable insights for the design of more engaging online initiatives aimed at, for example, encouraging healthy eating habits.
\end{abstract}

%%
%% The code below is generated by the tool at http://dl.acm.org/ccs.cfm.
%% Please copy and paste the code instead of the example below.
%%
\begin{CCSXML}
<ccs2012>
   <concept>
       <concept_id>10003120.10003130.10011762</concept_id>
       <concept_desc>Human-centered computing~Empirical studies in collaborative and social computing</concept_desc>
       <concept_significance>500</concept_significance>
       </concept>
 </ccs2012>
\end{CCSXML}

\ccsdesc[500]{Human-centered computing~Empirical studies in collaborative and social computing}

%%
%% Keywords. The author(s) should pick words that accurately describe
%% the work being presented. Separate the keywords with commas.
\keywords{Nutrition, Dietary Analysis, User Engagement, Reddit, Social Media, Online Food Communities}
%% A "teaser" image appears between the author and affiliation
%% information and the body of the document, and typically spans the
%% page.


%\received{31 January 2025}
%\received[revised]{12 March 2009}
%\received[accepted]{5 June 2009}

%%
%% This command processes the author and affiliation and title
%% information and builds the first part of the formatted document.
\maketitle

\section{Introduction}
\section{Introduction}
\label{sec:introduction}
The business processes of organizations are experiencing ever-increasing complexity due to the large amount of data, high number of users, and high-tech devices involved \cite{martin2021pmopportunitieschallenges, beerepoot2023biggestbpmproblems}. This complexity may cause business processes to deviate from normal control flow due to unforeseen and disruptive anomalies \cite{adams2023proceddsriftdetection}. These control-flow anomalies manifest as unknown, skipped, and wrongly-ordered activities in the traces of event logs monitored from the execution of business processes \cite{ko2023adsystematicreview}. For the sake of clarity, let us consider an illustrative example of such anomalies. Figure \ref{FP_ANOMALIES} shows a so-called event log footprint, which captures the control flow relations of four activities of a hypothetical event log. In particular, this footprint captures the control-flow relations between activities \texttt{a}, \texttt{b}, \texttt{c} and \texttt{d}. These are the causal ($\rightarrow$) relation, concurrent ($\parallel$) relation, and other ($\#$) relations such as exclusivity or non-local dependency \cite{aalst2022pmhandbook}. In addition, on the right are six traces, of which five exhibit skipped, wrongly-ordered and unknown control-flow anomalies. For example, $\langle$\texttt{a b d}$\rangle$ has a skipped activity, which is \texttt{c}. Because of this skipped activity, the control-flow relation \texttt{b}$\,\#\,$\texttt{d} is violated, since \texttt{d} directly follows \texttt{b} in the anomalous trace.
\begin{figure}[!t]
\centering
\includegraphics[width=0.9\columnwidth]{images/FP_ANOMALIES.png}
\caption{An example event log footprint with six traces, of which five exhibit control-flow anomalies.}
\label{FP_ANOMALIES}
\end{figure}

\subsection{Control-flow anomaly detection}
Control-flow anomaly detection techniques aim to characterize the normal control flow from event logs and verify whether these deviations occur in new event logs \cite{ko2023adsystematicreview}. To develop control-flow anomaly detection techniques, \revision{process mining} has seen widespread adoption owing to process discovery and \revision{conformance checking}. On the one hand, process discovery is a set of algorithms that encode control-flow relations as a set of model elements and constraints according to a given modeling formalism \cite{aalst2022pmhandbook}; hereafter, we refer to the Petri net, a widespread modeling formalism. On the other hand, \revision{conformance checking} is an explainable set of algorithms that allows linking any deviations with the reference Petri net and providing the fitness measure, namely a measure of how much the Petri net fits the new event log \cite{aalst2022pmhandbook}. Many control-flow anomaly detection techniques based on \revision{conformance checking} (hereafter, \revision{conformance checking}-based techniques) use the fitness measure to determine whether an event log is anomalous \cite{bezerra2009pmad, bezerra2013adlogspais, myers2018icsadpm, pecchia2020applicationfailuresanalysispm}. 

The scientific literature also includes many \revision{conformance checking}-independent techniques for control-flow anomaly detection that combine specific types of trace encodings with machine/deep learning \cite{ko2023adsystematicreview, tavares2023pmtraceencoding}. Whereas these techniques are very effective, their explainability is challenging due to both the type of trace encoding employed and the machine/deep learning model used \cite{rawal2022trustworthyaiadvances,li2023explainablead}. Hence, in the following, we focus on the shortcomings of \revision{conformance checking}-based techniques to investigate whether it is possible to support the development of competitive control-flow anomaly detection techniques while maintaining the explainable nature of \revision{conformance checking}.
\begin{figure}[!t]
\centering
\includegraphics[width=\columnwidth]{images/HIGH_LEVEL_VIEW.png}
\caption{A high-level view of the proposed framework for combining \revision{process mining}-based feature extraction with dimensionality reduction for control-flow anomaly detection.}
\label{HIGH_LEVEL_VIEW}
\end{figure}

\subsection{Shortcomings of \revision{conformance checking}-based techniques}
Unfortunately, the detection effectiveness of \revision{conformance checking}-based techniques is affected by noisy data and low-quality Petri nets, which may be due to human errors in the modeling process or representational bias of process discovery algorithms \cite{bezerra2013adlogspais, pecchia2020applicationfailuresanalysispm, aalst2016pm}. Specifically, on the one hand, noisy data may introduce infrequent and deceptive control-flow relations that may result in inconsistent fitness measures, whereas, on the other hand, checking event logs against a low-quality Petri net could lead to an unreliable distribution of fitness measures. Nonetheless, such Petri nets can still be used as references to obtain insightful information for \revision{process mining}-based feature extraction, supporting the development of competitive and explainable \revision{conformance checking}-based techniques for control-flow anomaly detection despite the problems above. For example, a few works outline that token-based \revision{conformance checking} can be used for \revision{process mining}-based feature extraction to build tabular data and develop effective \revision{conformance checking}-based techniques for control-flow anomaly detection \cite{singh2022lapmsh, debenedictis2023dtadiiot}. However, to the best of our knowledge, the scientific literature lacks a structured proposal for \revision{process mining}-based feature extraction using the state-of-the-art \revision{conformance checking} variant, namely alignment-based \revision{conformance checking}.

\subsection{Contributions}
We propose a novel \revision{process mining}-based feature extraction approach with alignment-based \revision{conformance checking}. This variant aligns the deviating control flow with a reference Petri net; the resulting alignment can be inspected to extract additional statistics such as the number of times a given activity caused mismatches \cite{aalst2022pmhandbook}. We integrate this approach into a flexible and explainable framework for developing techniques for control-flow anomaly detection. The framework combines \revision{process mining}-based feature extraction and dimensionality reduction to handle high-dimensional feature sets, achieve detection effectiveness, and support explainability. Notably, in addition to our proposed \revision{process mining}-based feature extraction approach, the framework allows employing other approaches, enabling a fair comparison of multiple \revision{conformance checking}-based and \revision{conformance checking}-independent techniques for control-flow anomaly detection. Figure \ref{HIGH_LEVEL_VIEW} shows a high-level view of the framework. Business processes are monitored, and event logs obtained from the database of information systems. Subsequently, \revision{process mining}-based feature extraction is applied to these event logs and tabular data input to dimensionality reduction to identify control-flow anomalies. We apply several \revision{conformance checking}-based and \revision{conformance checking}-independent framework techniques to publicly available datasets, simulated data of a case study from railways, and real-world data of a case study from healthcare. We show that the framework techniques implementing our approach outperform the baseline \revision{conformance checking}-based techniques while maintaining the explainable nature of \revision{conformance checking}.

In summary, the contributions of this paper are as follows.
\begin{itemize}
    \item{
        A novel \revision{process mining}-based feature extraction approach to support the development of competitive and explainable \revision{conformance checking}-based techniques for control-flow anomaly detection.
    }
    \item{
        A flexible and explainable framework for developing techniques for control-flow anomaly detection using \revision{process mining}-based feature extraction and dimensionality reduction.
    }
    \item{
        Application to synthetic and real-world datasets of several \revision{conformance checking}-based and \revision{conformance checking}-independent framework techniques, evaluating their detection effectiveness and explainability.
    }
\end{itemize}

The rest of the paper is organized as follows.
\begin{itemize}
    \item Section \ref{sec:related_work} reviews the existing techniques for control-flow anomaly detection, categorizing them into \revision{conformance checking}-based and \revision{conformance checking}-independent techniques.
    \item Section \ref{sec:abccfe} provides the preliminaries of \revision{process mining} to establish the notation used throughout the paper, and delves into the details of the proposed \revision{process mining}-based feature extraction approach with alignment-based \revision{conformance checking}.
    \item Section \ref{sec:framework} describes the framework for developing \revision{conformance checking}-based and \revision{conformance checking}-independent techniques for control-flow anomaly detection that combine \revision{process mining}-based feature extraction and dimensionality reduction.
    \item Section \ref{sec:evaluation} presents the experiments conducted with multiple framework and baseline techniques using data from publicly available datasets and case studies.
    \item Section \ref{sec:conclusions} draws the conclusions and presents future work.
\end{itemize}

\section{Related Work}

%\section{Related Work}
%\label{sec:related-work}

%\subsection{Background}

%Defect detection is critical to ensure the yield of integrated circuit manufacturing lines and reduce faults. Previous research has primarily focused on wafer map data, which engineers produce by marking faulty chips with different colors based on test results. The specific spatial distribution of defects on a wafer can provide insights into the causes, thereby helping to determine which stage of the manufacturing process is responsible for the issues. Although such research is relatively mature, the continual miniaturization of integrated circuits and the increasing complexity and density of chip components have made chip-level detection more challenging, leading to potential risks\cite{ma2023review}. Consequently, there is a need to combine this approach with magnified imaging of the wafer surface using scanning electron microscopes (SEMs) to detect, classify, and analyze specific microscopic defects, thus helping to identify the particular process steps where defects originate.

%Previously, wafer surface defect classification and detection were primarily conducted by experienced engineers. However, this method relies heavily on the engineers' expertise and involves significant time expenditure and subjectivity, lacking uniform standards. With the ongoing development of artificial intelligence, deep learning methods using multi-layer neural networks to extract and learn target features have proven highly effective for this task\cite{gao2022review}.

%In the task of defect classification, it is typical to use a model structure that initially extracts features through convolutional and pooling layers, followed by classification via fully connected layers. Researchers have recently developed numerous classification model structures tailored to specific problems. These models primarily focus on how to extract defect features effectively. For instance, Chen et al. presented a defect recognition and classification algorithm rooted in PCA and classification SVM\cite{chen2008defect}. Chang et al. utilized SVM, drawing on features like smoothness and texture intricacy, for classifying high-intensity defect images\cite{chang2013hybrid}. The classification of defect images requires the formulation of numerous classifiers tailored for myriad inspection steps and an Abundance of accurately labeled data, making data acquisition challenging. Cheon et al. proposed a single CNN model adept at feature extraction\cite{cheon2019convolutional}. They achieved a granular classification of wafer surface defects by recognizing misclassified images and employing a k-nearest neighbors (k-NN) classifier algorithm to gauge the aggregate squared distance between each image feature vector and its k-neighbors within the same category. However, when applied to new or unseen defects, such models necessitate retraining, incurring computational overheads. Moreover, with escalating CNN complexity, the computational demands surge.

%Segmentation of defects is necessary to locate defect positions and gather information such as the size of defects. Unlike classification networks, segmentation networks often use classic encoder-decoder structures such as UNet\cite{ronneberger2015u} and SegNet\cite{badrinarayanan2017segnet}, which focus on effectively leveraging both local and global feature information. Han Hui et al. proposed integrating a Region Proposal Network (RPN) with a UNet architecture to suggest defect areas before conducting defect segmentation \cite{han2020polycrystalline}. This approach enables the segmentation of various defects in wafers with only a limited set of roughly labeled images, enhancing the efficiency of training and application in environments where detailed annotations are scarce. Subhrajit Nag et al. introduced a new network structure, WaferSegClassNet, which extracts multi-scale local features in the encoder and performs classification and segmentation tasks in the decoder \cite{nag2022wafersegclassnet}. This model represents the first detection system capable of simultaneously classifying and segmenting surface defects on wafers. However, it relies on extensive data training and annotation for high accuracy and reliability. 

%Recently, Vic De Ridder et al. introduced a novel approach for defect segmentation using diffusion models\cite{de2023semi}. This approach treats the instance segmentation task as a denoising process from noise to a filter, utilizing diffusion models to predict and reconstruct instance masks for semiconductor defects. This method achieves high precision and improved defect classification and segmentation detection performance. However, the complex network structure and the computational process of the diffusion model require substantial computational resources. Moreover, the performance of this model heavily relies on high-quality and large amounts of training data. These issues make it less suitable for industrial applications. Additionally, the model has only been applied to detecting and segmenting a single type of defect(bridges) following a specific manufacturing process step, limiting its practical utility in diverse industrial scenarios.

%\subsection{Few-shot Anomaly Detection}
%Traditional anomaly detection techniques typically rely on extensive training data to train models for identifying and locating anomalies. However, these methods often face limitations in rapidly changing production environments and diverse anomaly types. Recent research has started exploring effective anomaly detection using few or zero samples to address these challenges.

%Huang et al. developed the anomaly detection method RegAD, based on image registration technology. This method pre-trains an object-agnostic registration network with various images to establish the normality of unseen objects. It identifies anomalies by aligning image features and has achieved promising results. Despite these advancements, implementing few-shot settings in anomaly detection remains an area ripe for further exploration. Recent studies show that pre-trained vision-language models such as CLIP and MiniGPT can significantly enhance performance in anomaly detection tasks.

%Dong et al. introduced the MaskCLIP framework, which employs masked self-distillation to enhance contrastive language-image pretraining\cite{zhou2022maskclip}. This approach strengthens the visual encoder's learning of local image patches and uses indirect language supervision to enhance semantic understanding. It significantly improves transferability and pretraining outcomes across various visual tasks, although it requires substantial computational resources.
%Jeong et al. crafted the WinCLIP framework by integrating state words and prompt templates to characterize normal and anomalous states more accurately\cite{Jeong_2023_CVPR}. This framework introduces a novel window-based technique for extracting and aggregating multi-scale spatial features, significantly boosting the anomaly detection performance of the pre-trained CLIP model.
%Subsequently, Li et al. have further contributed to the field by creating a new expansive multimodal model named Myriad\cite{li2023myriad}. This model, which incorporates a pre-trained Industrial Anomaly Detection (IAD) model to act as a vision expert, embeds anomaly images as tokens interpretable by the language model, thus providing both detailed descriptions and accurate anomaly detection capabilities.
%Recently, Chen et al. introduced CLIP-AD\cite{chen2023clip}, and Li et al. proposed PromptAD\cite{li2024promptad}, both employing language-guided, tiered dual-path model structures and feature manipulation strategies. These approaches effectively address issues encountered when directly calculating anomaly maps using the CLIP model, such as reversed predictions and highlighting irrelevant areas. Specifically, CLIP-AD optimizes the utilization of multi-layer features, corrects feature misalignment, and enhances model performance through additional linear layer fine-tuning. PromptAD connects normal prompts with anomaly suffixes to form anomaly prompts, enabling contrastive learning in a single-class setting.

%These studies extend the boundaries of traditional anomaly detection techniques and demonstrate how to effectively address rapidly changing and sample-scarce production environments through the synergy of few-shot learning and deep learning models. Building on this foundation, our research further explores wafer surface defect detection based on the CLIP model, especially focusing on achieving efficient and accurate anomaly detection in the highly specialized and variable semiconductor manufacturing process using a minimal amount of labeled data.


\section{Materials and Methods}
\subsection{Dataset}

\xhdr{Reddit}
%Reddit is an online platform consisting of subreddits,
Reddit is an online platform consisting of multiple topical communities, called ``subreddits,'' in which users engage in discussions.
Typically, subreddits are focused on a specific topic, and users write posts or comment on existing posts, forming a shared interest-centric community. 
Each subreddit has its own rules and guidelines on how to participate in that community. 
These rules typically define what to include in the post title and body, formatting instructions, or general instructions on communication tone. 

\xhdr{Food subreddit}
In this paper, we focus on r/Food, a subreddit dedicated to sharing meals. As of January $2025$, it is the $21$st largest subreddit, with around $24$ million subscribers\footnote{\url{https://www.reddit.com/best/communities/1/\#t5_2qh55}}. 
In particular, users post meals, following the rules of the subreddit: the post title must describe the meal.
Additionally, each post must include an original image of the meal, taken by the user who creates the post.
These rules ensure consistency across user posts and their focus on food. Even though the rules slightly changed over the years, the meal name had to be always included in the post title.


\xhdr{Data collection}
We collect data with Pushshift, a service that conducts large-scale crawls of Reddit \cite{baumgartner_pushshift_2020}.
We retrieve all submissions ($594,842$ posts) from r/Food subreddit from January $2017$ up to the end of December $2022$. 
For each post, we collect the number of comments the post received as a basic measurement of community engagement with a particular post. In addition, we collect further post information such as username or submission time.

\begin{table}[b]
\caption{\textit{Data filtering.} Number of posts at each stage of data preparation, from initial collection to posts included in the final analysis. We define resonant posts as the top 1\% of posts by the number of comments.}
\begin{tabular}{l|r}
                                         & \textbf{Value} \\ \hline
Collected posts                          & $594,842$        \\
Posts after preprocessing                & $509,479$        \\
Posts with macronutrient estimates       & $416,779$         \\
Posts with comments                      & $320,125$        \\ %todo check
Resonant posts                           & $3,219$          \\
\end{tabular}
\end{table}

\xhdr{Preprocessing}
In the first preprocessing step, we remove empty and deleted posts, as the community does not engage with such posts. Next, we remove duplicate posts, which we define as those made by the same user with the same title within five minutes. 
This leaves us with $509,479$ posts.
Lastly, to prepare the Reddit posts for the calculation of calories, we clean up the titles by removing special characters and emojis. 

\subsection{Nutritional Content Estimation}
To calculate the nutritional content of each meal, we use USDA’s FoodData Central database \cite{mckillop_fooddata_2021}. Specifically, we utilize three of its sources: (i) Foundation Foods, (ii) SR Legacy, and (iii) The Food and Nutrient Database for Dietary Studies.
In the database, nutrient information for each food entry is provided as density per $100$g.
We compute the nutritional content from the titles of Reddit posts by adapting the NutriTransform method \cite{ruprechter_2025}. 

Hence, we start by computing sentence embeddings \cite{reimers_sentence-bert_2019} for both Reddit post titles and the food database items. 
Using these embeddings, we compute the cosine similarity between a given Reddit post and all meals from the food database. We then select the five closest matches to the Reddit post given that they exceed a specified similarity threshold. We compute the similarity threshold by first taking a random sample of $5,000$ Reddit posts and computing their similarities to all the meals ($11,801$ food items) from the food database. As sentence embeddings typically result in vectors with substantial overall similarity (median similarity in our sample is $0.25$), we set the similarity threshold by computing the $99.9$th quantile of the similarity distribution as this quantile results in a sufficiently large number of highly similar meals. Thus, as the median similarity for the distribution of the $99.9$th quantile over our sample is $61.59$\%, we set the similarity threshold to $62$\%. 
We test the robustness of this similarity threshold by making additional computations with varying quantiles (e.g., $99.99$, $99$, $95$) and find no significant impact of the alternative similarity thresholds on our results.
After selecting the most similar meals from the database, we extract the calorie and macronutrient information for selected database matches and aggregate this information by computing similarity-weighted mean to obtain the nutritional content estimate of a given post. 
As the entries in the USDA's FoodData database are given per $100$g of a meal, all calculated calorie and macronutrient information also represent densities per $100$g of food.

Using our method, we compute the nutritional information for $307,799$ different meals, as multiple posts can contain the same meal (e.g. $1,591$ posts have the title ``Pizza'').
We exclude posts for which we did not find any matches in the food database, i.e., that exceeded the threshold, and posts where no suitable match is found in the food database or where the similarity score does not exceed the threshold.
Next, we check for potential outliers, which are all meals with less than $32$ calories (equivalent to $100$g of strawberries) or more than $717$ calories (equivalent to $100$g of butter).
After this final filtering step, we have a total of $306,592$ meals in $416,779$ posts that we use for further analysis.

\subsection{Explorative analysis}

\xhdr{Users}
A total of $146,203$ unique users contributed posts to the subreddit, with $62.6$\% posting only once.
The most active user made $882$ posts. Typically, more active users have more experience and the community engages stronger with their posts \cite{rokicki_how_2017}. 
In our dataset, the top $5$\% of users ($6,417$ users) according to the number of posts have at least $10$ posts each.

\xhdr{Comments}
The mean number of comments per post is nine, with a standard deviation of $43.4$, indicating significant variability in comment counts. In total, $320,125$ ($62.8$\%) posts received at least one comment.  
The maximal number of comments on a post is $2,447$, while the median is only two, and the third quartile is only six comments, indicating a strongly skewed distribution.  
This highlights the disparity between engagement and resonance---while the community engages with the majority of the posts, posts that strongly resonate with the community (top 1\%) receive at least $150$ comments.

\begin{figure*}[t]
    \centering
    \begin{subfigure}[t]{0.348\textwidth}
        \centering
        \vspace{0pt}
          \includegraphics[width=\textwidth, height=3.85cm]{images/num_posts_over_time_line.pdf}
          \caption{Number of Posts by Year}
          \Description{num_posts_over_time_line}
        \label{fig:posts_per_year}
    \end{subfigure}
    \hfill
    \begin{subfigure}[t]{0.372\textwidth}
        \centering
        \vspace{0pt}
          \includegraphics[width=\textwidth, height=4cm]{images/num_posts_per_year_line.pdf}
          \caption{Number of posts by Month}
          \Description{num_posts_per_year_line}
        \label{fig:posts_per_month}
        \end{subfigure}
    \hfill
    \begin{subfigure}[t]{0.24\textwidth}
        \centering
        \vspace{0pt}
        \includegraphics[width = \textwidth, height=4cm]{images/weekend_day_bar_correct.pdf}
        \caption{Number of Posts by Day Type and Time of Day Quartiles}
        \Description{weekend}
        \label{fig:posts_weekend}
        \end{subfigure}

    \medskip

    \centering
    \begin{subfigure}[t]{0.348\textwidth}
        \centering
        \vspace{0pt}
          \includegraphics[width=\textwidth, height=3.85cm]{images/num_comm_over_time_line.pdf}
          \caption{Engagement Level by Year}
          \Description{num_comm_over_time_line}
        \label{fig:comm_per_year}
    \end{subfigure}
    \hfill
    \begin{subfigure}[t]{0.372\textwidth}
        \centering
        \vspace{0pt}
          \includegraphics[width=\textwidth, height=4cm]{images/num_comm_per_year_line.pdf}
          \caption{Engagement Level by Month}
          \Description{num_comm_per_year_line}
        \label{fig:comm_per_month}
        \end{subfigure}
    \hfill
    \begin{subfigure}[t]{0.24\textwidth}
        \centering
        \vspace{0pt}
        \includegraphics[width = \textwidth, height=4cm]{images/weekend_day_bar_comments_correct.pdf}
        \caption{Engagement Level by Day Type and Time of Day Quartiles}
        \Description{comm weekend}
        \label{fig:comm_weekend}
    \end{subfigure}
    \caption{
    \textit{Posts and comments in r/Food over time.} We present how postings and comments developed from $2017$ until $2023$ across different temporal scales including yearly, monthly, weekly, and daily trends.
    In (\subref{fig:posts_per_year}) we present the number of posts over the years. We observe a positive trend before the COVID-19 pandemic with a noticeable peak during the pandemic and a drop afterwards to pre-pandemic levels.
    Monthly posting activity in (\subref{fig:posts_per_month}) is rather consistent except for a peak between March and June $2020$ during the pandemic.
    In (\subref{fig:posts_weekend}) we observe that more posts are created on weekdays than on weekends (left) and that most posts are created in the afternoon in the eastern USA (Q4, right). The bottom row shows the same diagrams for comments.
    In (\subref{fig:comm_per_year}) we observe a gradual increase in commenting activity over time with the highest activity levels during the pandemic and a sharp drop after the pandemic.
    This observation is also reflected in (\subref{fig:comm_per_month}), where we see constant high levels of comments in $2020$. We also see a seasonal spike in January possibly due to the holiday season.
    In (\subref{fig:comm_weekend}), comments mirror posting activity, with more comments over the weekdays (left). On the other hand, the peak in comments is in the morning (Q3, right).
    }
    \label{fig:temporal}
\end{figure*}


\xhdr{Scores}
Each Reddit post has a score, determined by the difference of community ``upvotes'' and ``downvotes''. 
The mean score is $245$, with a standard deviation of $1,740.03$, indicating high variability. The median score is only $23$, signaling again a skewed distribution where most posts receive relatively modest scores while the highest score is $70,308$.
In this paper, we do not use score as an engagement metric and opt for comments, which require more effort from the users. In addition, the score has a strong positive correlation with the number of comments ($\rho = 0.627, p < 0.001$), indicating that comments are a comprehensive representation of engagement. %Spearman


\xhdr{Temporal characteristics}
In Figure \ref{fig:temporal} we depict the temporal development of activity and user engagement in r/Food. The number of posts steadily increased over time (Fig. \ref{fig:posts_per_year}), peaking in $2020$, likely due to the COVID-19 pandemic as this surge can be potentially attributed to the widespread lockdowns, increased interest in food, and the shift towards consuming more meals at home \cite{gligoric_population-scale_2022}.
Following a few months of the COVID-19 outbreak, the number of posts rapidly dropped, with a brief increase at the beginning of $2021$.
Subsequently, the number of posts continued to decrease, returning to pre-pandemic levels and even falling further. 

When comparing the monthly post counts across years (Fig. \ref{fig:posts_per_month}), a similar pattern is observed for most years, except for $2020$.
Specifically, there is a noticeable spike in the number of posts during March $2020$, corresponding to the onset of the pandemic and lockdowns.
We show the distribution of postings over weekdays and weekends as well as the time of the day in Fig. \ref{fig:posts_weekend}. In the dataset, the posting time is stored in UTC time. 
As the majority of Reddit traffic comes from the USA (cf. Reddit traffic as of March 2024\footnote{\url{https://www.statista.com/statistics/325144/reddit-global-active-user-distribution/}}) with time zones ranging from EST (UTC - $5$) to PST (UTC - $8$), we interpret the time of the day results using USA eastern times. The exact time ranges in different time zones are presented in Table \ref{table:timezones}.
Hence, we observe that more posts are made on weekdays ($279,299$) than on weekends ($137,480$). Further, posting activity peaks in the afternoon in the eastern USA (likely reflecting users' lunchtime), accounting for $33.3$\% of total posts. 
This is followed by $29.3$\% of posts made in the evening and $26.4$\% in the morning in the eastern USA. The least amount of activity occurs during the night, with only $11.6$\% of posts.

\begin{table}[b]
\centering
\caption{\textit{Time of Day Quartiles.} The dataset contains times in UTC. As most Reddit users are from the USA, we interpret these times in both EST and PST.}
\begin{tabular}{@{}l l l l@{}}
\toprule
Quartile & UTC              & EST              & PST              \\ \midrule
Q1 (evening) & $12$ AM - $6$ AM  & $7$ PM - $1$ AM & $4$ PM - $10$ PM \\
Q2 (night)   & $6$ AM - $12$ PM  & $1$ AM - $7$ AM & $10$ PM - $4$ AM \\
Q3 (morning) & $12$ PM - $6$ PM & $7$ AM - $1$ PM & $4$ AM - $10$ AM \\
Q4 (afternoon) & $6$ PM - $12$ AM & $1$ PM - $7$ PM & $10$ AM - $4$ PM \\
\bottomrule
\end{tabular}
\label{table:timezones}
\end{table}

In the bottom row of Figure \ref{fig:temporal} we show the same temporal analysis for comments. In these figures, we categorize comments according to the time of their original postings. For instance, if a post is made in June $2020$, we treat all its comments as if they were made in June $2020$, even if they are posted at a later date. The commenting activity shows a gradual increase until $2020$ (Fig. \ref{fig:comm_per_year}).
Just before $2020$, there was an abrupt drop in the number of comments, followed by a sharp increase.
This trend aligns with the rise in posting behavior during that time and the onset of the COVID-19 pandemic. Another notable increase occurred at the beginning of $2021$.
Subsequently, there has been a steady decline, with the number of comments falling below pre-pandemic levels. There is no clear seasonality observed when comparing the monthly number of comments (Fig. \ref{fig:comm_per_month}).
The highest number of comments is recorded throughout $2020$, likely due to the pandemic.

More comments were made on posts published on weekdays compared to weekends (Fig. \ref{fig:comm_weekend}), which is consistent with the higher number of posts being made during the weekdays. Posts made in the morning in the eastern USA receive the highest number of comments ($1,280,826$ comments, $34.05$\% of total). This large number of comments could be attributed to users' activity during lunchtime and time after work, where they engage with posts made previously in the day.
Posts made in the afternoon ($1,093,762$ comments, $29.08$\%) closely follow. Next are posts made in the evening ($928,398$ comments, $24.68$\%), while posts made during the night receive the least comments ($458,086$ comments, $12.18$\%).


\xhdr{Tags}
According to the current subreddit rules, each post must include a tag indicating the context of the meal: whether the user prepared it at home, whether the user works in the food industry and prepared it, or whether the user purchased it without personal preparation. The majority of meals, $75$\%, were prepared at home by the users, while $19$\% were purchased without any preparation, and $1.5$\% were prepared by food industry professionals. 
The remaining $4.5$\% of posts either lack a tag, most likely due to earlier subreddit policies of not enforcing the tag structure, or these posts include a user-chosen tag.

\begin{figure*}[t]
    \centering
    \begin{subfigure}{0.25\textwidth}
        \centering
        \includegraphics[width=\textwidth]{images/n/rq1_calorie_distribution.pdf}
        \caption{Calorie Densities}
        \Description{calorie distribution}
        \label{fig:rq1_calorie}
    \end{subfigure}
    \hfill
    \begin{subfigure}{0.24\textwidth}
        \centering
        \includegraphics[width =\textwidth]{images/n/rq1_weighted_protein_distribution.pdf}
        \caption{Protein Densities}
        \Description{protein distribution}
        \label{fig:rq1_protein}
        \end{subfigure}
    \hfill
    \begin{subfigure}{0.24\textwidth}
        \centering
        \includegraphics[width =\textwidth]{images/n/rq1_weighted_carb_distribution.pdf}
        \caption{Carbohydrate Densities}
        \Description{carb distribution}
        \label{fig:rq1_carb}
        \end{subfigure}
    \hfill
    \begin{subfigure}{0.24\textwidth}
        \centering
        \includegraphics[width = \textwidth]{images/n/rq1_weighted_fat_distribution.pdf}
        \caption{Fat Densities}
        \Description{fat distribution}
        \label{fig:rq1_fat}
        \end{subfigure}


    \medskip


    \centering
    \begin{subfigure}{0.25\textwidth}
        \centering
        \includegraphics[width=\textwidth]{images/n/rq2_calorie_distribution.pdf}
        \caption{Calorie Densities}
        \Description{calorie distribution}
        \label{fig:rq2_calorie}
    \end{subfigure}
    \hfill
    \begin{subfigure}{0.24\textwidth}
        \centering
        \includegraphics[width =\textwidth]{images/n/rq2_weighted_protein_distribution.pdf}
        \caption{Protein Densities}
        \Description{protein distribution}
        \label{fig:rq2_protein}
        \end{subfigure}
    \hfill
    \begin{subfigure}{0.24\textwidth}
        \centering
        \includegraphics[width =\textwidth]{images/n/rq2_weighted_carb_distribution.pdf}
        \caption{Carbohydrate Densities}
        \Description{carb distribution}
        \label{fig:rq2_carb}
        \end{subfigure}
    \hfill
    \begin{subfigure}{0.24\textwidth}
        \centering
        \includegraphics[width = \textwidth]{images/n/rq2_weighted_fat_distribution.pdf}
        \caption{Fat Densities}
        \Description{fat distribution}
        \label{fig:rq2_fat}
        \end{subfigure}

        
    \caption{
    \textit{Nutritional content distribution of food in r/Food posts}. We illustrate the distribution of calories (\subref{fig:rq1_calorie}, \subref{fig:rq2_calorie}) and macro-nutrients (\subref{fig:rq1_protein}--\subref{fig:rq1_fat}, \subref{fig:rq2_protein}--\subref{fig:rq2_fat}) per $100$g of food, across meal in (i) engaging (red) and non-engaging (blue) posts, and (ii) resonant (red) and non-resonant (blue) posts.
    The calorie content is measured in kCal per $100$g, while macro-nutrients are measured in grams as fractions of $100$g total.
    We observe that the majority of posts fall within the moderate calorie range, between $100$ and $300$ kCal.
    \textit{Top row:} Calorie densities of posts with comments and without comments appear similar but differ significantly in means (\subref{fig:rq1_calorie}).
    We observe a steep decline in the protein (\subref{fig:rq1_protein}) density, with most posts having less than $20$g of protein, suggesting a prevalence of low-to-moderate protein meals.
    Carbohydrates (\subref{fig:rq1_carb}) span over a wider range. While most posts have less than $20$g, there is a consistent amount of carb-rich food as well, as indicated by the long tail in their distributions. 
    Fat (\subref{fig:rq1_fat}) distribution peaks around $10$-$15$g, with most posts containing moderate fat content.
    \textit{Bottom row:} Distribution disparities are more prominent when comparing resonant vs. non-resonant posts. Posts that do not resonate with the community peak at around $150$ kCal, while posts that do resonate peak at $300$ kCal (\subref{fig:rq2_calorie}). We observe similar behavior in all other macronutrient densities, with distributions for resonant posts being shifted to the right as compared to non-resonant posts. (\subref{fig:rq2_protein}--\subref{fig:rq2_fat}).
    }
    \label{fig:macros}
\end{figure*}

\xhdr{Engagement levels}
We operationalize engagement by the number of comments and define posts with at least one comment to be engaging posts and without comments to be non-engaging posts. Further, we define resonant and non-resonant posts by categorizing posts in the top $1$\% by comment count (i.e., the first percentile) as resonant ($3,219$ posts), while those with $0$ or $1$ comment are considered non-resonant ($157,470$ posts).
Note that slight variations in the definition of low resonance, such as considering only posts without comments, posts with just one comment, or posts with up to five comments, did not impact the results. To obtain balanced classes for our prediction experiment (cf. Sec. \ref{sec:prediction})
we randomly sample $3,219$ non-resonant posts, resulting in $6,438$ posts for further analysis. 

\xhdr{Nutritional Content Analysis}
We show the distributions of macronutrient content of meals in Figure \ref{fig:macros}. 
Specifically, we compare the nutritional content distributions of posts with and without user engagement (top row Fig. \ref{fig:macros}), as well as resonant vs. non-resonant posts (bottom row Fig. \ref{fig:macros}). 
The majority of posts fall within the moderate calorie range, from $100$ to $300$ kcal per $100$g of food.
When comparing posts with and without engagement, the calorie distribution appears similar (Fig. \ref{fig:rq1_calorie}), although the difference in means is statistically significant ($p < 10^{-235}$, Mann-Whitney-U test).
The calorie distribution is bimodal, with one peak at around $150$ calories and another at around $300$ calories.

Similarly to the calorie distributions the distributions of other macronutrients appear similar to each other, but all the differences in means are statistically significant (all $p < 10^{-6}$).
For example, the vast majority of meals contain up to $20$g of protein (Fig. \ref{fig:rq1_protein}).
Posts without engagement show a peak at around $5$g of protein, with a gradual decline in posts as protein content increases.
In contrast, posts with engagement exhibit a spike at around $5$g of protein, followed by another increase at just over $10$g of protein. 
Posts with engagement generally feature meals with higher carbohydrate content, while posts without engagement tend to feature meals with lower carbohydrate content (Fig. \ref{fig:rq1_carb}).
The fat distribution is similar, with posts without engagement tending to feature lower-fat meals (Fig. \ref{fig:rq1_fat}).

There is a clear difference when comparing resonant to non-resonant posts (bottom row Fig. \ref{fig:macros}).
A significant difference in means is observed in the calorie distribution ($p < 10^{-53}$).
Most non-resonant posts contain meals with fewer than $150$ calories, while the majority of resonant posts contain meals with around $300$ calories (Fig. \ref{fig:rq2_calorie}).
Beyond $300$ calories, the number of posts that resonate with users is constantly higher than the number of posts that do not resonate.
While both types of posts tend to feature low-protein meals, higher-protein meals are more often found in resonant posts (Fig. \ref{fig:rq2_protein}). However, there is no significant difference in means between the protein distributions ($p = 0.1$).
Similarly, both low-carbohydrate and low-fat values are associated with non-resonant posts (Fig. \ref{fig:rq2_carb} and Fig. \ref{fig:rq2_fat}).
Conversely, higher carbohydrate and fat values are linked to posts that resonate with the community.
The means of both these macronutrients are significantly different between resonant and non-resonant posts ($p < 10^{-27}$).

\section{Predicting Engagement}
\section{Posterior Predictive Distributions of PCTM}

$\mathbf{W}_{iq}$ is a vector of length $n_{iq}$, the number of words in paragraph $q$ of document $i$, and its $l$ th element, $W_{iql}$, is the $l$ the word in paragraph $p$ of document $i$.
$\mathbf{D}_{iq}$ is a vector of length $i-1$, the number of all possible documents to be cited, and its $j$ th element, $D_{iqj}$ is 1 if paragraph $p$ of document $i$ cited document $j$; 0 otherwise.
We use $\mathbf{W}_{iq}$ and $\mathbf{D}_{iq}$ as the test data to be predicted, and all words and citations in other paragraph than $q$ in document $i$ as well as all data in previous documents as the training data.
Thus, $\mathbf{W}^{train}$ be the set of $\mathbf{W}_{ir, r \neq q}$ and $\mathbf{W}_{jp}$ for all $j < i$ and $p \in \{1 , \ldots , n_j\}$ where $n_j$ is the number of paragraphs in document $j$.
Likewise, $\mathbf{D}^{train}$ be the set of $\mathbf{D}_{ir, r \neq q}$ and $\mathbf{D}_{jp}$ for all $j < i$ and $p \in \{1 , \ldots , n_j\}$.

\begin{equation}
\begin{split}
  &p(\mathbf{W}_{iq}, \mathbf{D}_{iq} \vert \mathbf{W}^{train}, \mathbf{D}^{train}) \\
  &\propto \int_{\eta, \Psi, \tau} \sum_{\mathbf{Z}} p(\mathbf{W}_{iq}, \mathbf{D}_{iq} \vert \mathbf{Z}, \eta, \Psi, \tau) 
  \times p(\mathbf{Z}, \eta, \Psi, \tau, \vert \mathbf{W}^{train}, \mathbf{D}^{train}) d\eta d\Psi d\tau \\
  &\propto \int_{\eta, \Psi, \tau} \sum_{\mathbf{Z}} p(\mathbf{W}_{iq}, \mathbf{D}_{iq} \vert \mathbf{Z}, \eta, \Psi, \tau) 
  \times p(\mathbf{Z} \vert \eta, \Psi, \tau, \mathbf{W}^{train}, \mathbf{D}^{train}) p( \eta, \Psi, \tau \vert \mathbf{W}^{train}, \mathbf{D}^{train}) d\eta d\Psi d\tau \\
  &\approx \sum_{Z_{iq}} p(\mathbf{W}_{iq}, \mathbf{D}_{iq} \vert Z_{iq}, \hat{\mathbf{Z}}^{train}, \hat{\eta}, \hat{\Psi}, \hat{\tau}) \times p(Z_{iq} \vert \hat{\eta})  \\
  &= \sum_{k=1}^K \Big\{ p(\mathbf{W}_{iq}, \mathbf{D}_{iq} \vert Z_{iq} = k, \hat{\mathbf{Z}}^{train}, \hat{\eta}, \hat{\Psi}, \hat{\tau}) \times p(Z_{iq} = k \vert \hat{\eta}) \Big\} \\
  &= \sum_{k=1}^K \Big\{ p(\mathbf{W}_{iq} \vert Z_{iq} = k, \hat{\Psi})
  \times \prod_{j=1}^{i-1} p(D_{iqj} \vert \hat{\tau}, \hat{\eta}, Z_{ip})
  \times p(Z_{iq} = k \vert \hat{\eta}) \Big\} \\
  &= \sum_{k=1}^K \Big\{ p(\mathbf{W}_{iq} \vert Z_{iq} = k, \hat{\Psi})
  \times \prod_{j=1}^{i-1} \mathbb{P}(D_{iqj}^* > 0 \vert \hat{\tau}, \hat{\eta}, Z_{ip} = k)^{\mathbb{I}\{D_{iqj}=1\}}\mathbb{P}(D_{iqj}^* < 0 \vert \hat{\tau}, \hat{\eta}, Z_{ip} = k)^{\mathbb{I}\{D_{iqj}=0\}} \\
  &\quad \times p(Z_{iq} = k \vert \hat{\eta}) \Big\} \\
  &\propto \sum_{k=1}^K \Bigg\{ \prod_{v=1}^V \Psi_{vk}^{W_{iqv}} 
  \times \prod_{j=1}^{i-1} \Big[\int_{t=0}^{\infty} p(D_{iqj}^* = t | \tau_0 + \tau_1 \kappa_j^{(i)} + \tau_2\eta_{jk}) dt\Big]^{\mathbb{I}\{D_{iqj}=1\}} \\
  &\quad \times \Big[\int_{t=-\infty}^{0} p(D_{iqj}^* = t | \tau_0 + \tau_1 \kappa_j^{(i)} + \tau_2\eta_{jk}) dt \Big]^{\mathbb{I}\{D_{iqj}=0\}} 
  \times \frac{\exp(\eta_{ik})}{\sum_{k'=1}^K \exp(\eta_{ik'})} \Bigg\} \\
\end{split}
\end{equation}

The approximation part can use Monte Carlo simulation - We draw parameters $\eta, \tau$ from the posterior distributions and take the average.
Alternatively, we could use the point estimate of $\eta, \tau$ to match the computation in RTM and LDA+logistic.


%\section{Results and Discussion}
\subsection{Results}

\begin{table}[b]
\centering
\caption{\textit{Results.} ROC-AUC scores across engagement and resonance predictions.}
%\resizebox{\columnwidth}{!}{
\begin{tabular}{@{}l c c@{}}
\toprule
Model         & ROC-AUC (Engagement)        & ROC-AUC (Resonance)        \\ \midrule
Control (C)             & $0.593$ [$0.587$, $0.599$] & $0.669$ [$0.641$, $0.698$] \\
C + Nutrition (N)       & $0.603$ [$0.597$, $0.609$] & $0.709$ [$0.680$, $0.737$] \\
C + Food Descriptors (F)     & $0.594$ [$0.588$, $0.600$] & $0.671$ [$0.642$, $0.700$] \\
C + Engagement Discriminators (E)         & $0.602$ [$0.596$, $0.608$] & $0.699$ [$0.672$, $0.726$] \\
C + N + F           & $0.602$ [$0.597$, $0.608$] & $0.710$ [$0.682$, $0.739$] \\
C + N + E           & $0.607$ [$0.602$, $0.613$] & $0.717$ [$0.690$, $0.745$] \\
C + F + E         & $0.602$ [$0.596$, $0.607$] & $0.694$ [$0.666$, $0.723$] \\
C + N + F + E        & $0.608$ [$0.603$, $0.614$] & $0.713$ [$0.687$,$ 0.743$] \\ \bottomrule
\end{tabular}
%}
\label{table:auc_roc_scores}
\end{table}



We present our results in Table \ref{table:auc_roc_scores} where we summarize our main findings, with ROC-AUC scores and their corresponding bootstrap confidence intervals for each model.

\xhdr{Predicting engagement}
Using only control feature set our classification model achieves a ROC-AUC score of $0.593$.
Adding the nutritional density features improves the ROC-AUC score by $1$\%, to $0.603$.
In contrast, adding food descriptors to controls does not improve prediction performance (ROC-AUC of $0.594$).
Similarly to nutritional features, adding engagement discriminators to the controls raises ROC-AUC to $0.602$.
Further, combining the nutritional density and food descriptors, or the nutritional density and engagement discriminators with the controls achieves ROC-AUC scores of $0.602$ and $0.607$, respectively. 
Similarly, adding the food descriptors and engagement discriminators to the controls also achieves ROC-AUC of $0.602$.
Finally, combining all feature sets achieves the best ROC-AUC score of $0.608$.

\xhdr{Predicting resonance}
Due to the resonant and non-resonant classes being more distinctly separated than in the previous experiment, the model with the controls already achieves the ROC-AUC score of $0.669$.
Moreover, adding nutritional density to the controls improves the score to $0.709$, or by $4$\%.
When adding food descriptors we observe a $1$\% improvement in performance (ROC-AUC of $0.671$) and
when adding engagement discriminators the performance improves by $3$\% to $0.699$. 
Further, when we combine discriminators and the descriptors with the controls the model improves to $0.694$.
Combining nutritional density and food descriptors with the controls also increases the performance by $4$\% indicating that the food descriptors are not predictive of resonance.
Finally, adding nutritional content and engagement discriminators we achieve ROC-AUC of $0.713$, the same as the model combining all feature sets. Hence, those models achieve the best overall performance as compared to the controls, improving that base model by $5$\%.


\begin{figure}[t]
    \centering
    \begin{subfigure}[t]{0.51\linewidth}
        \centering
        \includegraphics[width=\linewidth]{images/shap_rq1.pdf}
        \caption{The beeswarm plot displaying how certain features impact whether a post will get a comment or not}
        \label{fig:shap_rq1}
    \end{subfigure}
    \hfill
    \begin{subfigure}[t]{0.45\linewidth}
        \centering
        \includegraphics[width = \linewidth]{images/shap_bar_rq1.pdf}
        \caption{Features with a large contribution to the model predictions}
        \label{fig:shap_bar_rq1}
        \end{subfigure}
    %\caption{Combined figure}
    %\label{fig:combined}
    %\centering
    \vfill
    \begin{subfigure}[t]{0.48\linewidth}
        \centering
        \includegraphics[width=\linewidth]{images/shap_rq1_pred_0.pdf}
        \caption{``[I ate] Pork Belly, Sweet Potato Purée, Maple Gastrique, Micro Celery'' post, not achieving engagement}
        \label{fig:shap_rq1_pred_0}
    \end{subfigure}
    \hfill
    \begin{subfigure}[t]{0.48\linewidth}
        \centering
        \includegraphics[width = \linewidth]{images/shap_rq1_pred_1.pdf}
        \caption{``Red Velvet Cupcakes with Cream Cheese Frosting and Chocolate Decoration'' post, achieving engagement}
        \label{fig:shap_rq1_pred_1}
        \end{subfigure}
    \caption{
        \textit{SHAP visualizations for classifier predicting post engagement.}
        Looking at SHAP values of different features, we can understand to which degree they influence the probability of a post receiving engagement. 
        In the beeswarm plot (\subref{fig:shap_rq1}) we display how the top features impact the model's output, with each dot representing one post. 
        Posting after COVID-19, being an experienced user, and having higher calorie meals strongly increases the likelihood of engagement prediction.   
        Additionally, the absence of no-engagement discriminators, posting on the weekday and later in the day further increases those odds.
        Foods with either high or low protein content have a higher probability of engagement than foods with moderate protein content.
        In the bar plot (\subref{fig:shap_bar_rq1}) we present the feature importance in absolute values. The calorie density ranks third after controlling for COVID-19 and the user tenure.
        We also present two examples (\subref{fig:shap_rq1_pred_0}-\subref{fig:shap_rq1_pred_1}) with local feature importance, highlighting the concrete values of each feature and the way they contributed to the prediction of a post receving comments.
        }
    \label{fig:shap_all_rq1}
    \Description{SHAP plot rq1}
\end{figure}

\xhdr{Feature importance with SHAP values}
To better understand the associations of features and user engagement and their effects on 
the engagement prediction we calculate SHapley Additive exPlanations (SHAP) values.
SHAP values represent the contributions of each individual feature value to the prediction score.
In particular, a positive SHAP value for a feature value indicates an increase in the probability of a positive prediction, while a negative SHAP value for a particular feature value suggests a decrease in the probability of a positive class prediction. For example, in Figure \ref{fig:shap_rq1} we illustrate the contribution of features to the prediction of whether a post will receive comments by plotting the SHAP values across posts.
In particular, in this diagram, we observe how different values of a particular feature affect the model's prediction.
Hence, each point represents a SHAP value for an individual post given its corresponding feature value.
The SHAP values are given on the x-axis reflecting their impact on the model's output.
Specifically, positive SHAP values push the prediction towards engagement, while negative values reduce the probability of the positive class.
The color gradient represents the feature value, with blue indicating lower feature values and red indicating higher values. For example, blue points in the third row of Fig. \ref{fig:shap_rq1} represent lower calorie and red higher calorie density.
Additionally, Figure \ref{fig:shap_bar_rq1} displays the mean of absolute SHAP values across the range of possible feature values. 
While the plot does not differentiate the direction of the feature value, it depicts the overall feature importance in the prediction model.
Lastly, SHAP values allow to inspect how individual predictions are computed to further gain an understanding of the associations between features and the model output. For example, Figure \ref{fig:shap_rq1_pred_0} shows a waterfall plot visualizing the breakdown of the model's prediction for a single instance and highlighting how each feature influences the outcome.
In the waterfall plots, contributions from each feature are depicted as colored bars, showing their cumulative additive influence on the outcome.
To understand the model's prediction for a specific observation, we start with the baseline prediction without features, and iterate trough the features summing their SHAP values to obtain the final prediction (a number between $-1$ and $1$, with $0$ being the class threshold).

\xhdr{Role of nutritional content in engagement prediction}
In Figure \ref{fig:shap_rq1} we observe that calorie density is positively associated with engagement, with a mean absolute SHAP value of $0.09$ (Fig. \ref{fig:shap_bar_rq1}). 
This value is lower than the contributions of the COVID and user experience features, which are control features. In particular, a mean absolute SHAP value of $0.18$ indicates that, on average, the COVID feature contributes $0.18$ in absolute terms to the model's prediction of a post receiving comments, hence, suggesting that posts made after the onset of COVID-19 (March $15$, $2021$) are more likely to receive engagement.
Similarly, user experience also positively affects engagement, with more experienced users generally being associated with engagement. However, calorie density still ranks as the third most important feature even after controlling for COVID and user experience, highlighting its strong impact on engagement prediction. 

Moreover, the calorie density has a stronger effect on the model's prediction than the absence of no-engagement discriminators (cf. row four in \ref{fig:shap_rq1} and \ref{fig:shap_bar_rq1}) or the presence of engagement discriminators (cf. row five in \ref{fig:shap_rq1} and \ref{fig:shap_bar_rq1}), which were selected as the words distinctively used in different classes. This further underlines the strong predictive power of calorie density on user engagement with a particular post.
Regarding other nutritional densities, we observe that lower protein content does not drastically affect engagement while higher protein content has a polarizing effect, either significantly boosting or detracting from engagement prediction. 
In other words, high protein can be a critical factor in determining the prediction outcome, either positively or negatively.
In addition, we also observe that posting on weekends or later in the day appears (both are control features) to negatively influence the likelihood of engagement.


Further, in Figure \ref{fig:shap_rq1_pred_0} we show the waterfall plot for a post without comments as an individual prediction example. The model's baseline is $-0.011$.
The title of this particular post is ``[I ate] Pork Belly, Sweet Potato Purée, Maple Gastrique, Micro Celery'' and it was posted after the initial COVID outbreak and by an experienced user. Consistent with the previously discussed results, those two features increase the probability of the positive class.
Nevertheless, other features such as the presence of the word ``sweet'' (food descriptor) as well as no-engagement discriminators and the absence of the engagement discriminators reduce the prediction probability.
Additionally, low protein content ($7$g) and low carbohydrate content ($5$g) are also negatively associated with the likelihood of engagement resulting in the final prediction for comments to be  $-0.183$. We show another waterfall plot in Figure \ref{fig:shap_rq1_pred_1} with an example of a post that received comments.
The title of the post is ``Red Velvet Cupcakes with Cream Cheese Frosting and Chocolate Decoration''.
The figure demonstrates that the user's experience and the calorie content of the dish ($358$ kCal) positively contribute to the likelihood of engagement prediction. 
In contrast, the fact that the post was made before the onset of COVID-19 reduces the prediction probability. 
Conversely to the previous example, the very low protein content ($3$g) has in this case a positive effect on the likelihood of engagement prediction.
The presence of an engagement discriminator further boosts the model's probability, whereas the presence of a no-engagement discriminator negatively influences the outcome. 
Overall, the predicted value of a post receiving engagement is $0.282$.

\begin{figure}[t]
    \centering
    \begin{subfigure}[t]{0.51\linewidth}
        \centering
        \includegraphics[width=\linewidth]{images/shap_rq2.pdf}
        \caption{The beeswarm plot displaying how certain features impact whether a post will resonate or not}
        \label{fig:shap_rq2}
    \end{subfigure}
    \hfill
    \begin{subfigure}[t]{0.45\linewidth}
        \centering
        \includegraphics[width = \linewidth]{images/shap_bar_rq2.pdf}
        \caption{Features with a large contribution to the model predictions}
        \label{fig:shap_bar_rq2}
        \end{subfigure}
    %\caption{Combined figure}
    %\label{fig:combined}
    %\centering
    \vfill
    \begin{subfigure}[t]{0.48\linewidth}
        \centering
        \includegraphics[width=\linewidth]{images/shap_rq2_pred_0.pdf}
        \caption{``[homemade] minestrone soup'' post, not resonating with the community}
        \label{fig:shap_rq2_pred_0}
    \end{subfigure}
    \hfill
    \begin{subfigure}[t]{0.48\linewidth}
        \centering
        \includegraphics[width = \linewidth]{images/shap_rq2_pred_1.pdf}
        \caption{``[Homemade] Unicorn and Dinosaur cookies'' post, resonating with the community}
        \label{fig:shap_rq2_pred_1}
        \end{subfigure}
    \caption{
        \textit{SHAP visualizations for classifier predicting post's resonance level.}
        SHAP values of features provide us with the transparency of a classifier and allow us to understand which features are beneficial to achieve resonance. In the beeswarm plot (\subref{fig:shap_rq2}) we present how the values of features impact the prediction, while the bar plot (\subref{fig:shap_bar_rq2}) presents each top feature's importance. Calorie density is the fourth most important feature after several control features. Its importance is more than two times higher than for the engagement predictor. Similarly, a higher carbohydrate value indicates the prediction of resonance, and a higher protein content, although smaller, still has a positive influence. Fat content has less of a prediction power. Additionally, we present two concrete examples (\subref{fig:shap_rq2_pred_0}-\subref{fig:shap_rq2_pred_1}), with local feature importance and values that drive the prediction of a post resonating with the community.
        }
    \label{fig:shap_all_rq2}
    \Description{SHAP plot rq2}
\end{figure}


\xhdr{Role of nutritional content in resonance prediction}
In Figure \ref{fig:shap_rq2} and \ref{fig:shap_bar_rq2} we present how features are associated with post resonance. We observe, that the mean absolute SHAP value for calorie density is $0.19$, which is more than twice higher than the value for calorie density when predicting engagement. Although, the absence of low resonance discriminators, high user experience, and posting midday have the strongest influence on resonance prediction (all are control features), the calorie density ranks as the fourth most important feature. Hence, posts featuring high-calorie meals are more likely to resonate with the community even after accounting for the influence of the control features. On the other hand, low-calorie meals tend to be associated with non-resonant posts.
Similarly, SHAP values suggest that high carbohydrate content is also linked with positive resonance prediction.
In contrast, protein content has a more nuanced negative correlation with the resonance. 
While high protein density is associated with non-engaging posts, low protein content tends to appear more likely in resonant posts. 
Besides nutritional content, another control feature (posting on weekdays) is more beneficial for resonance than posting on weekends.

To further illustrate our findings, we again examine two specific examples in more detail.
The first example (Fig. \ref{fig:shap_rq2_pred_0}) visualizes the features of a post titled ``[homemade] minestrone soup'', which does not resonate well with the community.
According to the chart, the post lacked a low resonance discriminator and was posted by an experienced user, which both favored resonance prediction.
Regardless, the majority of other features contributed negatively to the prediction.
The meal’s very low calorie ($53$ kCal), protein ($3$g), and carbohydrate ($8$g) content all had a negative effect.
Additionally, the post was made during the fourth quartile of the day and after COVID, both of which further reduced its likelihood of resonance.
These combined factors result in the prediction value of $-0.95$, classifying the post as a non-resonant post.
The second example (Fig. \ref{fig:shap_rq2_pred_1}) represents a post titled ``[Homemade] Unicorn and Dinosaur cookies'', which resonated well with the community.
The meal's high-calorie content ($432$ kCal) and high carbohydrate content ($75$g) significantly increased the likelihood of resonance.
In this case, the low protein content ($7$g) also positively influenced the outcome.
However, several features negatively impacted the prediction, including the user’s low experience, the presence of a low engagement discriminator, and the post being made in the evening (Q$1$). 
Despite these negative factors, the overall prediction value was $0.414$, classifying the post as one resonating with the community.


\subsection{Discussion}

Our findings on the relation between nutritional content and engagement in food-related social media posts provide several key insights into user behavior as well as the context and content of engaging posts. First, including the nutritional content as a feature set in our engagement prediction models significantly enhances the model's classification accuracy, suggesting a strong predictive power of these nutritional features for engagement. Additionally, we uncover the direction of this strong association: more calorie-dense meals increase the prediction probability for user engagement and, in particular, for post resonance.

The influence of calorie content aligns with prior research suggesting that users are more drawn to calorie-dense meals \cite{pancer_content_2022}.
Posts with higher calorie content consistently demonstrate higher SHAP values, emphasizing their role in engagement prediction.
This finding is further corroborated by the interaction between calorie and carbohydrate density. In particular, posts with both high-calorie and high-carbohydrate meals are typically more likely to reach top engagement levels.
Conversely, protein content's nuanced effect suggests that high-protein meals may appeal to a more specific audience and, hence, not resonate well enough with a broader user community.

We note again that the associations between nutritional content and user engagement are still significant even after controlling for a range of non-food-related features. 
For example, user experience appears as a critical feature strongly related to engagement \cite{bakshy_everyones_2011}. 
For example, 
Posts by older and more experienced users had higher likelihoods of engagement, confirming findings from studies on social media websites such as Usenet and Twitter, 
where contributions by long-term users were more likely to receive responses \cite{arguello_talk_2006, suh_want_2010, turkoglu_improving_2023}.
This phenomenon may be related to the community perceiving experienced users' content as being of higher quality, or to the experienced users being able to understand the community and their expectations better than inexperienced ones. Further, the timing of the posts is also significantly related to engagement. 
Posts made after the onset of COVID-19 experienced higher engagement, potentially due to increased digital screen time during lockdowns \cite{wong_digital_2021} and the rise in internet traffic \cite{feldmann_lockdown_2020}.
However, posts made later in the day or during weekends were less likely to engage users, agreeing with the findings that weekday posts during busier hours attract more interaction \cite{wahid_social_2020, hanifawati_managing_2019}.
Finally, our findings align with studies indicating that captions and post titles significantly influence engagement \cite{hessel_cats_2017, chen2021drives} reinforcing the importance of carefully crafting titles and captions to resonate with audiences.

\xhdr{Limitations}
Even though Reddit is an anonymous platform that allows more authentic behavior, it comes with several limitations.
First, we miss the detailed user demographics as well as their individual interests and nutritional goals. 
%Thus, considering the complexity of human behavior, modeling it as a whole might be too broad of a generalization.
Second, we do not account for bots on the platform, which are easily created by users \cite{long2017could}.
The presence of bots can influence user engagement and the dynamics of online interaction.
Third, Reddit's algorithm that curates feeds may influence user engagement with specific, assumed-relevant, posts.

Another limitation of our work is that we exclude visual features, which in combination with captions are accurate predictors of engagement on Reddit \cite{hessel_cats_2017}.
While focusing on the nutritional density of the meals instead of their appearance is the goal of our research, we are aware of the impact of the visual elements.
Given Reddit's focus on images, and the general influence of food aesthetics on engagement rates \cite{philp_predicting_2022}, the aesthetic of a meal could serve as an additional control feature, providing deeper insights. We plan to extend our analysis by including visual features as another set of control features in our future work.

Although we aimed for a robust estimation method of the nutritional content by using pre-trained embeddings, a robust similarity threshold, and a similarity-weighted density aggregation, we acknowledge potential inaccuracies in our estimation due to, among others, variations in ingredient ratios. In particular, we estimate the nutritional densities and not the total amounts of nutritional content.
Additionally, common meals, such as pizza, can have numerous variations, and users may not feel the need to specify these differences in the title, as they accompany their title with a picture.
Since our estimation of nutritional content relies solely on the title, our approach may overlook valuable information that could enhance the accuracy of the estimation.

Moreover, our work is a large-scale study of a single, although large, community (i.e. Reddit's r/Food). 
While we believe that the amount of data (almost 600,000 posts) and the huge user base of this subreddit (24 million users) is sufficient for an analysis of this kind, we still acknowledge potential sample bias in users who post and engage with this community.
Therefore, our findings might not necessarily generalize to other communities. 
However, we see this as an opportunity to extend our work to other social media platforms that garner a large number of users, such as Instagram.

Finally, we caution that our work indicates an associative link between nutritional density and different levels of engagement, and does not establish causality.
Albeit we control for several confounding features, which makes the evidence we find for this link stronger, we still conduct an observational study without a particular setup needed for causal inference.

%%%%



\section{Conclusion}
\section{Conclusion}
In this work, we propose a simple yet effective approach, called SMILE, for graph few-shot learning with fewer tasks. Specifically, we introduce a novel dual-level mixup strategy, including within-task and across-task mixup, for enriching the diversity of nodes within each task and the diversity of tasks. Also, we incorporate the degree-based prior information to learn expressive node embeddings. Theoretically, we prove that SMILE effectively enhances the model's generalization performance. Empirically, we conduct extensive experiments on multiple benchmarks and the results suggest that SMILE significantly outperforms other baselines, including both in-domain and cross-domain few-shot settings.


%%
%% The acknowledgments section is defined using the "acks" environment
%% (and NOT an unnumbered section). This ensures the proper
%% identification of the section in the article metadata, and the
%% consistent spelling of the heading.
%\begin{acks}
%To Robert, for the bagels and explaining CMYK and color spaces.
%\end{acks}

%%
%% The next two lines define the bibliography style to be used, and
%% the bibliography file.
\bibliographystyle{ACM-Reference-Format}
%\bibliography{sample-base}
\bibliography{references-new}


%%
%% If your work has an appendix, this is the place to put it.
%\appendix

%\section{Research Methods}



\end{document}
\endinput
%%
%% End of file `sample-sigconf.tex'.
