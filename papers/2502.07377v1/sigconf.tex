%%
%% This is file `sample-sigconf.tex',
%% generated with the docstrip utility.
%%
%% The original source files were:
%%
%% samples.dtx  (with options: `all,proceedings,bibtex,sigconf')
%% 
%% IMPORTANT NOTICE:
%% 
%% For the copyright see the source file.
%% 
%% Any modified versions of this file must be renamed
%% with new filenames distinct from sample-sigconf.tex.
%% 
%% For distribution of the original source see the terms
%% for copying and modification in the file samples.dtx.
%% 
%% This generated file may be distributed as long as the
%% original source files, as listed above, are part of the
%% same distribution. (The sources need not necessarily be
%% in the same archive or directory.)
%%
%%
%% Commands for TeXCount
%TC:macro \cite [option:text,text]
%TC:macro \citep [option:text,text]
%TC:macro \citet [option:text,text]
%TC:envir table 0 1
%TC:envir table* 0 1
%TC:envir tabular [ignore] word
%TC:envir displaymath 0 word
%TC:envir math 0 word
%TC:envir comment 0 0
%%
%%
%% The first command in your LaTeX source must be the \documentclass
%% command.
%%
%% For submission and review of your manuscript please change the
%% command to \documentclass[manuscript, screen, review]{acmart}.
%%
%% When submitting camera ready or to TAPS, please change the command
%% to \documentclass[sigconf]{acmart} or whichever template is required
%% for your publication.
%%
%%
%\documentclass[sigconf]{acmart}
%\documentclass[manuscript,screen,review]{acmart}
\documentclass[manuscript,screen]{acmart}

%%
%% \BibTeX command to typeset BibTeX logo in the docs
\AtBeginDocument{%
  \providecommand\BibTeX{{%
    Bib\TeX}}}

%% Rights management information.  This information is sent to you
%% when you complete the rights form.  These commands have SAMPLE
%% values in them; it is your responsibility as an author to replace
%% the commands and values with those provided to you when you
%% complete the rights form.
\setcopyright{acmlicensed}
\copyrightyear{2025}
%\acmYear{2025}
\acmDOI{XXXXXXX.XXXXXXX}

%% These commands are for a PROCEEDINGS abstract or paper.
\acmConference[Conference acronym 'XX]{Make sure to enter the correct
  conference title from your rights confirmation emai}{June 03--05,
  2018}{Woodstock, NY}
%%
%%  Uncomment \acmBooktitle if the title of the proceedings is different
%%  from ``Proceedings of ...''!
%%
%%\acmBooktitle{Woodstock '18: ACM Symposium on Neural Gaze Detection,
%%  June 03--05, 2018, Woodstock, NY}
\acmISBN{978-1-4503-XXXX-X/18/06}


%%
%% Submission ID.
%% Use this when submitting an article to a sponsored event. You'll
%% receive a unique submission ID from the organizers
%% of the event, and this ID should be used as the parameter to this command.
%%\acmSubmissionID{123-A56-BU3}

%%
%% For managing citations, it is recommended to use bibliography
%% files in BibTeX format.
%%
%% You can then either use BibTeX with the ACM-Reference-Format style,
%% or BibLaTeX with the acmnumeric or acmauthoryear sytles, that include
%% support for advanced citation of software artefact from the
%% biblatex-software package, also separately available on CTAN.
%%
%% Look at the sample-*-biblatex.tex files for templates showcasing
%% the biblatex styles.
%%

%%
%% The majority of ACM publications use numbered citations and
%% references.  The command \citestyle{authoryear} switches to the
%% "author year" style.
%%
%% If you are preparing content for an event
%% sponsored by ACM SIGGRAPH, you must use the "author year" style of
%% citations and references.
%% Uncommenting
%% the next command will enable that style.
%%\citestyle{acmauthoryear}

% Academic text is often much more legible if you give important paragraphs a
% concise name that describes what the paragraph is about. Use the \xhdr
% command for this.
\newcommand{\xhdr}[1]{\vspace{0.7mm}\noindent{{\bf #1.}}}

% Same as \xhdr, but without a period after the heading. Use this version if
% the heading is directly integrated into the first sentence of the paragraph;
% e.g., "\xhdrNoPeriod{Results} are shown in \Figref{fig}."
\newcommand{\xhdrNoPeriod}[1]{\vspace{1mm}\noindent{{\bf #1}}}


\usepackage{booktabs}
\usepackage{multirow}
\usepackage{tabularray}
\usepackage{subcaption}

\newcounter{gozcounter}
\DeclareRobustCommand{\goz}[1]{\textbf{/* {\color{ACMDarkBlue} #1} (goz) */}\stepcounter{gozcounter}\typeout{LaTeX Warning: goz comment \thegozcounter: #1 (line \the\inputlineno)}}
\newcounter{trucounter}
\DeclareRobustCommand{\tru}[1]{\textbf{/* {\color{ACMPurple} #1} (tru) */}\stepcounter{trucounter}\typeout{LaTeX Warning: tru comment \thetrucounter: #1 (line \the\inputlineno)}}
\newcounter{dhecounter}
\DeclareRobustCommand{\dhe}[1]{\textbf{/* {\color{ACMOrange} #1} (dhe) */}\stepcounter{dhecounter}\typeout{LaTeX Warning: dhe comment \thedhecounter: #1 (line \the\inputlineno)}}

\iffalse
    \renewcommand{\goz}[1]{}
    \renewcommand{\tru}[1]{}
    \renewcommand{\dhe}[1]{}
\fi

\definecolor{Gray}{rgb}{0.501,0.501,0.501}

%%
%% end of the preamble, start of the body of the document source.
\begin{document}

%%
%% The "title" command has an optional parameter,
%% allowing the author to define a "short title" to be used in page headers.
% \title{The Effect of Calorie Density on Engagement in Reddit's Food Communities}
\title{Reddit's Appetite: Predicting User Engagement with Nutritional Content}

%%
%% The "author" command and its associated commands are used to define
%% the authors and their affiliations.
%% Of note is the shared affiliation of the first two authors, and the
%% "authornote" and "authornotemark" commands
%% used to denote shared contribution to the research.

\author{Gabriela Ozegovic}
\affiliation{%
  \institution{Graz University of Technology}
  \city{Graz}
  \country{Austria}}
\email{ozegovic@tugraz.at}

\author{Thorsten Ruprechter}
\affiliation{%
  \institution{Graz University of Technology}
  \city{Graz}
  \country{Austria}}
\email{ruprechter@tugraz.at}


\author{Denis Helic}
\affiliation{%
  \institution{Graz University of Technology}
  \city{Graz}
  \country{Austria}}
\email{dhelic@tugraz.at}

%%
%% By default, the full list of authors will be used in the page
%% headers. Often, this list is too long, and will overlap
%% other information printed in the page headers. This command allows
%% the author to define a more concise list
%% of authors' names for this purpose.
\renewcommand{\shortauthors}{Ozegovic et al.}

%%
%% The abstract is a short summary of the work to be presented in the
%% article.
\begin{abstract}
The increased popularity of food communities on social media shapes the way people engage with food-related content. Due to the extensive consequences of such content on users' eating behavior, researchers have started studying the factors that drive user engagement with food in online platforms. However, while most studies focus on visual aspects of food content in social media, there exist only initial studies exploring the impact of nutritional content on user engagement. In this paper, we set out to close this gap and analyze food-related posts on Reddit, focusing on the association between the nutritional density of a meal and engagement levels, particularly the number of comments. Hence, we collect and empirically analyze almost 600,000 food-related posts and uncover differences in nutritional content between engaging and non-engaging posts. Moreover, we train a series of XGBoost models, and evaluate the importance of nutritional density while predicting whether users will comment on a post or whether a post will substantially resonate with the community. We find that nutritional features improve the baseline model’s accuracy by 4\%, with a positive contribution of calorie density towards prediction of engagement, suggesting that higher nutritional content is associated with higher user engagement in food-related posts. Our results provide valuable insights for the design of more engaging online initiatives aimed at, for example, encouraging healthy eating habits.
\end{abstract}

%%
%% The code below is generated by the tool at http://dl.acm.org/ccs.cfm.
%% Please copy and paste the code instead of the example below.
%%
\begin{CCSXML}
<ccs2012>
   <concept>
       <concept_id>10003120.10003130.10011762</concept_id>
       <concept_desc>Human-centered computing~Empirical studies in collaborative and social computing</concept_desc>
       <concept_significance>500</concept_significance>
       </concept>
 </ccs2012>
\end{CCSXML}

\ccsdesc[500]{Human-centered computing~Empirical studies in collaborative and social computing}

%%
%% Keywords. The author(s) should pick words that accurately describe
%% the work being presented. Separate the keywords with commas.
\keywords{Nutrition, Dietary Analysis, User Engagement, Reddit, Social Media, Online Food Communities}
%% A "teaser" image appears between the author and affiliation
%% information and the body of the document, and typically spans the
%% page.


%\received{31 January 2025}
%\received[revised]{12 March 2009}
%\received[accepted]{5 June 2009}

%%
%% This command processes the author and affiliation and title
%% information and builds the first part of the formatted document.
\maketitle

\section{Introduction}
\documentclass[../main.tex]{subfiles}
\graphicspath{{../images/}}
\makeatletter
\def\input@path{{../images/}}
\makeatother
\begin{document}
\section{Introduction}
\begin{figure}
\centering
\begin{tikzpicture}
\node[inner sep=0pt] (ws) at (0, 0) {
\includegraphics[height=.4\textwidth, trim={10cm 0 10cm 0},clip]{world_space.png}};
\node[inner sep=0pt] (cs) at (6,0) {\includegraphics[height=.4\textwidth, trim={10cm 1cm 10cm 4cm},clip]{conf_space.png}};
\end{tikzpicture}
\vspace{-5pt}
\label{fig:pbrm_intro}
\caption{\textbf{Left}: Shows world space obstacles as grey spheres. Robots start and goal configuration is colored red and green, respectively. Configurations along the computed path are colored transparent blue. \textbf{Right:} Mapped world space scenario to configuration space. Obstacle region is the grey mesh. Red spheres are collision-free regions computed by the neural SCDF. The optimized shortest path in the convex corridor is the blue curve.}
\vspace{-25pt}
\end{figure}
Motion planning is the problem of finding a collision-free trajectory that connects a given start and goal configuration. The planning takes place in the configuration space of the robot. For single body robots, like mobile robots or drones, the configuration space and the world space are usually the same. This simplifies the planning, since explicit obstacle representations are available which enables geometrical tools like separating hyperplanes, smallest distance to obstacles etc., to be used when designing motion planning algorithms. For multi-body robots like manipulators, the situation is completely different. The world space obstacles are usually mapped to non-convex regions, and to make the problem even harder, the mapping is usually not known. Forming explicit representations of the obstacle region in the configuration space is usually too expensive or intractable. Despite all of this, sampling based planners are used with great success, which mainly is due to their use of implicit representations of the obstacle region. The basic idea is to construct a graph in the configuration space that covers and connects the collision-free region. From this graph, a path can be extracted that connects a given start and goal configuration. The approach is computationally expensive, since the graph is constructed with the smallest geometrical building block available, points, which represents a collision-check. Furthermore, the extracted paths from the graph are non-smooth and jagged due to the stochastic nature of the approach. This adds an additional post-processing step to the process, where the paths are shortcutted and smoothened, before the path can be used for tracking. Clearly a lot of time is invested to form this graph and produce smooth paths. Thus, if the obstacles start to move, then all of this work is done in no use, since all points that make up this graph need to be re-verified, which is simply too time consuming to be done in real time.
\\\\
In this work, we want to address the existing drawbacks of the sampling based planners. Our main contribution is an improved motion planner where each vertex in the graph covers a collision-free region in the form of a sphere instead of a point and where the edges are formed with neighboring intersecting spheres. This representation has the advantage of instead of returning piecewise linear paths, returning a sequence of overlapping spheres, i.e. a convex corridor, that connects a given start and goal configuration, illustrated in Figure \ref{fig:pbrm_intro}. This convex corridor allows us to use convex optimization to produce smooth trajectories, instead of computationally expensive post-processing methods. The representation further allows us to estimate the coverage of the collision-free space, which gives us awareness and feedback in the offline roadmap construction phase. Finally, our representation is simple to adapt to moving obstacles, simply requery for the new radii and recheck for intersections. 
\\\\
The spherical collision-free regions are formed using a signed distance function (SDF), which is a function that returns the smallest distance from an arbitrary point to the boundary of an obstacle. As the name implies, the distance is signed, thus if the point is inside the obstacle it is negative otherwise positive. If the distance is positive, a sphere with radius equal to the distance is guaranteed to cover a collision-free region. Using an SDF in motion planning is not new, but what is novel about our approach is that we express the distance in the configuration space instead of the world space and by doing so allows us to form these convex collision-free regions. We refer to the resulting SDF as a signed configuration distance function (SCDF). Computing an SCDF analytically is non-trivial, our approach is therefore to parameterize the SCDF with a deep neural network and learn the mapping by supervised learning. Our resulting neural SCDF can compute distances for different parameter values of obstacle shapes and we also show how multiple distances can be combined, thus making our approach flexible.
\section{Related work}
Motion planning algorithms can roughly be divided into three families, grid-based, sampling based and optimization based methods. Grid-based methods (GBM) discretize the planning space from which a graph is then compiled. A standard search method is A$^\star$ \citep{a_star}, which is classified as an \textit{informed} search method, since it employs a heuristic function to speed up the search. A$^\star$ guarantees to return an optimal path at the level of discretization used. GBMs usually discretize the planning space by a regular lattice and this limits the GBMs to problems with low dimensionality due to the curse of dimensionality. Thus, GBMs are usually limited to single-body robots where the degrees of freedom (DOF) are low. To overcome the inherent scaling problem with the GBMs, stochastic methods are usually used for multi-body robots. These methods are termed as sampling-based methods (SBM) and core members within this family are the rapidly-exploring random trees (RRT) \citep{rrt} and the probabilistic roadmap (PRM) \citep{prm}. RRT grows a tree from the start configuration and explores the collision-free region in a rapid way until it is able to connect to the goal region. RRT is usually improved by bi-directional planning \citep{rrt_connect}, i.e. an additional tree is grown from the goal configuration and the trees are tested for connection after any tree has been expanded. RRT is a single-query method, thus it searches for a path from scratch each time it is queried. Contrary to this, PRM is a multi-query method, which solves for multiple queries without starting from scratch. PRM does this by creating a roadmap (graph) that covers the collision-free space as an offline step. The graph is then used to solve for multiple queries. PRMs are used in cases where the environment does not change since the extra offline step is too computationally costly and needs to be re-done if the environment is changed. In our work, we address this inherent issue by using a different roadmap representation. Our vertices in the graph cover a collision-free region in the form of spheres and we form the edges by checking for intersecting spheres. If something in the environment changes, we recompute the spheres radii and recheck the intersections, without relying on collision detection. We use a trained neural network to compute the sphere radius, therefore querying for the radius can be done fast, hence our representation enables the PRM for dynamic environments.
\\\\
In the recent decades, optimization based methods (OBM) \citep{chomp, schulman, itomp, stomp} have been introduced as an alternative to SBM for multi-body robots. Like the SBM, the OBMs scale well to higher dimensional problems and produce smoother motion. It is common to use a SDF in the optimization since it is a smooth function, thus enabling gradient-based methods. However, the standard way of expressing the SDF is in world space. The distance therefore needs to be mapped to the configuration space by the forward kinematics. This mapping makes the optimization problem a non-linear program (NLP), which is computationally expensive to solve. Recently, a different approach has been proposed. In \cite{mp_gcs} motion planning is formulated as a convex optimization problem by using the graph of convex sets framework \citep{gcs}. The underlying idea is to decompose the collision-free space into intersecting convex sets from which a convex optimization problem is formulated. In cases where an explicit representation of the obstacles in the configuration space exists, like for single-body robots, creating collision-free convex regions can be done fast \citep{iris}. For multi-body robots, this is non-trivial. Existing work does this successfully \citep{iris_nlp, iris_c} by an optimization based approach, but the methods are still too time consuming to be used in the presence of moving obstacles. Our approach is instead to use deep learning to learn an SDF expressed in the configuration space. With this, we can query for shortest distances to the collision boundary, which allows us to expand spherical regions which are collision-free. Our approach is fast and therefore enables our suggested roadmap planner to be used in dynamic environments.
\\\\
Recent research has focused on learning collision detection \citep{fk_kernel_distance, diffco, graphdistnet} by predicting the signed distance between the robot links and the surrounding obstacles in the world space. The learned SDF is used in trajectory optimization but since the distance is expressed in the world space, the problem becomes an NLP and therefore takes a long time to solve. We take a novel approach and suggest to instead express the signed distance in the configuration space. This allows us to improve the PRM at the same time as it enables convex optimization for trajectory optimization, which runs faster and is more reliable than NLP solvers. In \cite{cspf} a learned signed distance function in the configuration space is proposed similar to our approach. However, their approach is restricted to point cloud representations, while we propose to represent the obstacles as parameterized geometric shapes, e.g. spheres. Furthermore, we also show how to use our learned SCDF to improve an existing roadmap planner.
\section{Problem formulation}
A robot is located in the world space, $\W \subset \R^3 $. The unique location of the robot is given by its configuration $\q \in \C$, where $\C$ is the configuration space. The set of points covered by the robots bodies at a certain configuration is expressed as $\B(\q) \subset \W$. The robot is surrounded by $\NrObst$ obstacles $\O = \bigcup_{i=1}^{\NrObst} \O_i$, where  $\O_i \subset \W$. The representation of the obstacle in the configuration space is the set $\C\O_i = \{\q \in \C \: |\: \B(\q) \cap \O_i \neq \emptyset \}$. The obstacle space is formed as $\Co = \bigcup_{i=1}^{\NrObst} \C \O_i$. The complement is referred to as the free space, $\Cf = \C \setminus \Co$. The path planning problem is a tuple, ($\Cf$, $\qStart$, $\qGoal$), where we want to connect a query pair, consisting of a start, $\qStart$, and goal configuration, $\qGoal$, with a geometric path, $\q(s): [0, 1] \mapsto \Cf$, such that $\q(0)=\qStart$ and $\q(1)=\qGoal$, or report correctly when such a path does not exist.
\end{document}


\section{Related Work}
\section{Related Work}
\label{sec:related-work}
%We now contextualize our work with related literature so that our contributions are highlighted. We cover FMTS, perturbations in time-series, 
% robustness testing of FMs, 
%and rating of AI systems. 

\noindent \textbf{Foundation Models Supporting Time Series} 
The use of FMs for time series forecasting has advanced significantly. 
% \cite{lu2022frozen} first demonstrated that transformers pre-trained on text data (LLMs) can effectively solve sequence modeling tasks in other modalities, paving the way for leveraging language pre-trained transformers for time series analysis. Recent studies have focused on reprogramming LLMs for time series tasks through parameter-efficient fine-tuning and suitable tokenization strategies \cite{zhou2023one, gruver2024large, jin2023time, cao2023tempo, ekambaram2024tiny}. These methods have successfully adapted transformers to the unique challenges of time series forecasting. \cite{zhou2023one} and \cite{jin2023time} further illustrate the versatility and robustness of fine-tuned language pre-trained transformers for diverse time series tasks.
\cite{lu2022frozen} showed that transformers pre-trained on text data can solve sequence modeling tasks in other modalities, enabling their application to time series analysis. Recent studies have reprogrammed LLMs for time series tasks through parameter-efficient fine-tuning and tokenization strategies \cite{zhou2023one, gruver2024large, jin2023time, cao2023tempo, ekambaram2024tiny}. 
% These methods have successfully adapted transformers to the unique challenges of time series forecasting. 
\cite{zhou2023one} and \cite{jin2023time} further illustrate the versatility and robustness of fine-tuned language pre-trained transformers for diverse time series tasks.
% Several models have contributed to the advancement of time series forecasting. \cite{ansari2024chronos} and \cite{woo2024unified} have improved forecasting accuracy and model generalization.  
% % \cite{ansari2024chronos} and \cite{woo2024unified} have pushed the boundaries of forecasting accuracy and model generalization. 
% \cite{rasul2023lag} and \cite{das2023decoder} have explored new tokenization strategies and fine-tuning methods to improve model performance. Additionally, \cite{garza2023timegpt} and \cite{ekambaram2024tiny} have focused on creating lightweight and efficient models for real-time applications. \cite{talukder2024totem} stands out with its unique approach to integrating multiple temporal patterns, enhancing forecasting precision.
% FMs trained from scratch have achieved SOTA on time series tasks. Zero-shot forecasting, exemplified by \cite{gruver2024large}, showcases the ability of these models to make accurate predictions without domain-specific training. \cite{cao2023tempo} and \cite{goswami2024moment} have introduced approaches to enhance the performance and efficiency of time series models, leveraging transformer architectures to capture temporal dependencies more effectively. In our experiments, we select Gemini-V and Phi-3 as the GP models and Chronos and MOMENT as TS models due to their SOTA performance in their respective categories.
Several models have advanced time series forecasting. \cite{ansari2024chronos} and \cite{woo2024unified} have improved forecasting accuracy and model generalization, while
% \cite{ansari2024chronos} and \cite{woo2024unified} have pushed the boundaries of forecasting accuracy and model generalization. 
\cite{rasul2023lag} and \cite{das2023decoder} have explored new tokenization strategies and fine-tuning methods. \cite{garza2023timegpt} and \cite{ekambaram2024tiny} developed lightweight models for real-time applications, and \cite{talukder2024totem} integrated multiple temporal patterns to improve precision. FMs trained from scratch, like \cite{gruver2024large}, achieved SOTA in zero-shot forecasting, with \cite{cao2023tempo} and \cite{goswami2024moment} further improving model performance. 
%In our experiments, we select Gemini-V and Phi-3 as the GP models and Chronos and MOMENT as TS models due to their SOTA performance in their respective categories.
Please see Section~\ref{sec:exp_app} for the FMTS we selected due to their SOTA performance in their respective categories.

%The use of FMs for time series forecasting has seen significant advancements in recent years. \cite{lu2022frozen} first demonstrated that transformers pre-trained on text data (LLMs) can effectively solve sequence modeling tasks in other modalities. This work opened the door to leveraging language pre-trained transformers for time series analysis. Recent studies have built on this foundation, focusing on reprogramming LLMs for time series tasks through parameter-efficient fine-tuning and suitable tokenization strategies \cite{zhou2023one, gruver2024large, jin2023time, cao2023tempo, ekambaram2024tiny}. These methods have proven successful in adapting the powerful capabilities of transformers to the unique challenges of time series forecasting. OneFitsAll \cite{zhou2023one} and Time-LLM \cite{jin2023time} further illustrate how language pre-trained transformers can be fine-tuned for diverse time series tasks, demonstrating their versatility and robustness. 
% \zhen{reason why we didn't include these models in our study, weights not available? or other justification, to prevent that naturally raised question from readers.}\kl{Good point. We need to discuss. I added 2 sentences at the bottom but they are probably not very convincing.}
%Several other models have contributed to the advancement of time series forecasting. Chronos \cite{ansari2024chronos} and Moirai \cite{woo2024unified} have pushed the boundaries of forecasting accuracy and model generalization. Lag-llama \cite{rasul2023lag} and TimesFM \cite{das2023decoder} have explored new tokenization strategies and fine-tuning methods to improve model performance. Additionally, Time-GPT1 \cite{garza2023timegpt} and Tiny-Time Mixers \cite{ekambaram2024tiny} have focused on creating lightweight and efficient models suitable for real-time applications. TOTEM \cite{talukder2024totem} stands out with its unique approach to integrating multiple temporal patterns, further enhancing forecasting precision.
%Aside from reprogramming LLMs for time series, FMs trained from scratch have achieved SOTA on times series tasks. 
%Zero-shot forecasting, exemplified by \cite{gruver2024large}, showcases the ability of these models to make accurate predictions without domain-specific training.  TEMPO \cite{cao2023tempo} and MOMENT \cite{goswami2024moment} have introduced approaches to enhance the performance and efficiency of time series models, leveraging transformer architectures to capture temporal dependencies more effectively.
% \zhen{and these are on various time series tasks including time series forecasting?}
% \zhen{These are models that are specifically trained for time series forecasting, I'd suggest mentioning them first after the LLM reprogramming, and then expanding to the models that are trained across time series tasks instead. The flow of this subsection feels a bit odd as of now.} \kl{Done.}
%In our experiments, we select Gemini-V and Phi-3 as the GP models and Chronos and MOMENT as TS models due to their SOTA performance in their respective categories. 

%\vspace{-0.3em}
\noindent \textbf{Perturbations in Time Series Data} TS data is commonly stored in spreadsheets and databases, which are prone to changes due to acts of omission (e.g., negligence, data-entry errors) or commission (e.g., adversarial attacks, sabotage). Omission errors are most common \cite{spreadsheets-errors-risks-survey}. Tools like Microsoft Excel and Google Sheets are widely used for data collection and analysis, allowing end-user programming \cite{spreadsheets-future-workshop}. However, over 90\% of spreadsheets contain errors due to issues like incorrect formulae, leading to multi-billion dollar losses \cite{spreadsheet-qa-survey}.
%\cite{spreadsheet-qa-survey,spreadsheets-errors-risks-survey}.
Adversarial attacks are also increasing in data stores and AI models for tasks like forecasting.
% \cite{papernot2016transferability} introduced a black-box attack method using a substitute model to generate adversarial examples, demonstrating transferability across tasks. \cite{baluja2017adversarial} focused on white-box attacks using gradient information. 
\cite{karim2019adversarial} adapted these concepts to time series, exploring both black-box and white-box attacks. \cite{oregi2018adversarial} revealed the vulnerability of distance-based classifiers. \cite{rathore2020untargeted} examined various adversarial attacks on time series classifiers. TSFool \cite{li2022tsfool} introduced a multi-objective black-box attack to craft imperceptible adversarial time series to fool RNN classifiers.
%Time series (TS) data is widely stored and manipulated in spreadsheets and databases. These are also the tools which see considerable changes or perturbations due to acts of omission that are unintended (e.g., negligence, data-entry errors) or commission which are deliberate (e.g., adversarial attacks, sabotage). 
%Among these, changes due to omission are most common \cite{spreadsheets-errors-risks-survey}.
%For example, a spreadsheet, implemented in tools like Microsoft Excel and Google Sheets, is a common data collection and analysis environment that also allows end-user programming \cite{spreadsheets-future-workshop}. They are used widely at the workplace and are often a door opener to more advanced scientific tools. But gaining expertise in them needs practice since a large proportion of spreadsheets ($\succ$ 90\%) are known to have errors due to issues like incorrect formulae caused by improper understanding of behavior during routine operations like copy-paste and end-user programming, which have caused losses of multi-billion dollars \cite{spreadsheet-qa-survey,spreadsheets-errors-risks-survey}.
% \zhen{do we need to relate our perturbations to these attacks? otherwise, we must manage the readers' expectations on what types of perturbations we focus on other than adversarial attacks, and motivate it properly}
%\zhen{Play down this a bit, and emphasize and justify why we focus on the type of perturbations we consider in the paper, to mimic operational errors in practice apart from adversarial attacks, citing the 2024 and 1996 papers Biplav added.} 
%Furthermore, adversarial attacks are also increasing both in data stores and in AI models created to solve tasks like forecasting.
%Foundational work by ~\cite{papernot2016transferability} introduced a black-box attack method that involved training a substitute model to generate adversarial examples capable of misleading the target model, demonstrating the transferability property across similar tasks. In contrast, research by ~\cite{baluja2017adversarial} focused on white-box attacks, using gradient information and probabilistic outputs to craft adversarial examples. Researchers~\cite{karim2019adversarial} have adapted these concepts to the time series domain, exploring both black-box and white-box attacks on time series classification models. In addition, ~\cite{oregi2018adversarial} revealed the susceptibility of distance-based time-series classifiers to adversarial examples. ~\cite{rathore2020untargeted} examined untargeted, targeted, and universal adversarial attacks on time series classifiers, demonstrating the effectiveness of these attacks across various datasets. TSFool~\cite{li2022tsfool} introduced a multi-objective black-box attack to craft highly imperceptible adversarial time series to fool RNN classifiers.
%Adversarial attacks on time-series data are initially focused on time-series classification tasks, leveraging concepts adapted from adversarial attacks in other domains.
%explored adversarial sample crafting for time series classification using elastic similarity measures,  %These works collectively underscore the ongoing efforts to understand and mitigate the risks posed by adversarial attacks on time series classification models.
% More recently, research into adversarial attacks on time series forecasting models has revealed distinct challenges and novel attack strategies. One primary challenge is targeted attacks. While targeted adversarial attacks on time series classification aim to misclassify specific instances, achieving similar precision in time series forecasting is more complex due to the sequential nature of the data. Perturbations must be designed to influence specific aspects of the forecast (e.g., directional shifts or amplitude changes) without disrupting the overall temporal dependencies, making precise control more challenging~\cite{govindarajulu2023targeted}. Another challenge is attacks on multivariate forecasting. Adversarial attacks could exploit the inter-dependences between variables. ~\cite{liu2022robust} introduced sparse and indirect cross-time-series attacks in multivariate settings, which are more effective and realistic than direct attacks in univariate cases.
% \zhen{Biplav, could we make a quick comment here as well that we focus more on data error side in practice, other than attacks? and cite the paper that you mentioned on data errors? Otherwise this section of adversarial attacks feel a bit standalone to other sections}
%These challenges underscore the need for ongoing research to develop effective adversarial attack strategies and robust defense mechanisms tailored to the unique characteristics of time series forecasting models.
% -----


\noindent \textbf{Rating AI Systems} Several works have assessed and rated AI systems for trustworthiness from a third-party perspective without access to training data. \cite{srivastava2020rating} proposed a method to rate AI systems for bias, specifically targeting gender bias in machine translators \cite{srivastava2018towards}, and used visualizations to communicate these ratings \cite{bernagozzi2021vega}. They conducted user studies on trust perception through visualizations \cite{vega-userstudy-translatorbias}, but these lacked causal interpretation. \cite{kausik2024rating} introduced a causal analysis approach to rate bias in sentiment analysis systems, extending it to assess their impact when used with translators \cite{kausik2023the}. We extend their method to rate MM-TSFM for robustness against perturbations. Causal analysis offers advantages over statistical analysis by determining accountability, aligning with humanistic values, and quantifying the direct influence of various attributes on forecasting accuracy.



\section{Materials and Methods}
\subsection{Dataset}

\xhdr{Reddit}
%Reddit is an online platform consisting of subreddits,
Reddit is an online platform consisting of multiple topical communities, called ``subreddits,'' in which users engage in discussions.
Typically, subreddits are focused on a specific topic, and users write posts or comment on existing posts, forming a shared interest-centric community. 
Each subreddit has its own rules and guidelines on how to participate in that community. 
These rules typically define what to include in the post title and body, formatting instructions, or general instructions on communication tone. 

\xhdr{Food subreddit}
In this paper, we focus on r/Food, a subreddit dedicated to sharing meals. As of January $2025$, it is the $21$st largest subreddit, with around $24$ million subscribers\footnote{\url{https://www.reddit.com/best/communities/1/\#t5_2qh55}}. 
In particular, users post meals, following the rules of the subreddit: the post title must describe the meal.
Additionally, each post must include an original image of the meal, taken by the user who creates the post.
These rules ensure consistency across user posts and their focus on food. Even though the rules slightly changed over the years, the meal name had to be always included in the post title.


\xhdr{Data collection}
We collect data with Pushshift, a service that conducts large-scale crawls of Reddit \cite{baumgartner_pushshift_2020}.
We retrieve all submissions ($594,842$ posts) from r/Food subreddit from January $2017$ up to the end of December $2022$. 
For each post, we collect the number of comments the post received as a basic measurement of community engagement with a particular post. In addition, we collect further post information such as username or submission time.

\begin{table}[b]
\caption{\textit{Data filtering.} Number of posts at each stage of data preparation, from initial collection to posts included in the final analysis. We define resonant posts as the top 1\% of posts by the number of comments.}
\begin{tabular}{l|r}
                                         & \textbf{Value} \\ \hline
Collected posts                          & $594,842$        \\
Posts after preprocessing                & $509,479$        \\
Posts with macronutrient estimates       & $416,779$         \\
Posts with comments                      & $320,125$        \\ %todo check
Resonant posts                           & $3,219$          \\
\end{tabular}
\end{table}

\xhdr{Preprocessing}
In the first preprocessing step, we remove empty and deleted posts, as the community does not engage with such posts. Next, we remove duplicate posts, which we define as those made by the same user with the same title within five minutes. 
This leaves us with $509,479$ posts.
Lastly, to prepare the Reddit posts for the calculation of calories, we clean up the titles by removing special characters and emojis. 

\subsection{Nutritional Content Estimation}
To calculate the nutritional content of each meal, we use USDA’s FoodData Central database \cite{mckillop_fooddata_2021}. Specifically, we utilize three of its sources: (i) Foundation Foods, (ii) SR Legacy, and (iii) The Food and Nutrient Database for Dietary Studies.
In the database, nutrient information for each food entry is provided as density per $100$g.
We compute the nutritional content from the titles of Reddit posts by adapting the NutriTransform method \cite{ruprechter_2025}. 

Hence, we start by computing sentence embeddings \cite{reimers_sentence-bert_2019} for both Reddit post titles and the food database items. 
Using these embeddings, we compute the cosine similarity between a given Reddit post and all meals from the food database. We then select the five closest matches to the Reddit post given that they exceed a specified similarity threshold. We compute the similarity threshold by first taking a random sample of $5,000$ Reddit posts and computing their similarities to all the meals ($11,801$ food items) from the food database. As sentence embeddings typically result in vectors with substantial overall similarity (median similarity in our sample is $0.25$), we set the similarity threshold by computing the $99.9$th quantile of the similarity distribution as this quantile results in a sufficiently large number of highly similar meals. Thus, as the median similarity for the distribution of the $99.9$th quantile over our sample is $61.59$\%, we set the similarity threshold to $62$\%. 
We test the robustness of this similarity threshold by making additional computations with varying quantiles (e.g., $99.99$, $99$, $95$) and find no significant impact of the alternative similarity thresholds on our results.
After selecting the most similar meals from the database, we extract the calorie and macronutrient information for selected database matches and aggregate this information by computing similarity-weighted mean to obtain the nutritional content estimate of a given post. 
As the entries in the USDA's FoodData database are given per $100$g of a meal, all calculated calorie and macronutrient information also represent densities per $100$g of food.

Using our method, we compute the nutritional information for $307,799$ different meals, as multiple posts can contain the same meal (e.g. $1,591$ posts have the title ``Pizza'').
We exclude posts for which we did not find any matches in the food database, i.e., that exceeded the threshold, and posts where no suitable match is found in the food database or where the similarity score does not exceed the threshold.
Next, we check for potential outliers, which are all meals with less than $32$ calories (equivalent to $100$g of strawberries) or more than $717$ calories (equivalent to $100$g of butter).
After this final filtering step, we have a total of $306,592$ meals in $416,779$ posts that we use for further analysis.

\subsection{Explorative analysis}

\xhdr{Users}
A total of $146,203$ unique users contributed posts to the subreddit, with $62.6$\% posting only once.
The most active user made $882$ posts. Typically, more active users have more experience and the community engages stronger with their posts \cite{rokicki_how_2017}. 
In our dataset, the top $5$\% of users ($6,417$ users) according to the number of posts have at least $10$ posts each.

\xhdr{Comments}
The mean number of comments per post is nine, with a standard deviation of $43.4$, indicating significant variability in comment counts. In total, $320,125$ ($62.8$\%) posts received at least one comment.  
The maximal number of comments on a post is $2,447$, while the median is only two, and the third quartile is only six comments, indicating a strongly skewed distribution.  
This highlights the disparity between engagement and resonance---while the community engages with the majority of the posts, posts that strongly resonate with the community (top 1\%) receive at least $150$ comments.

\begin{figure*}[t]
    \centering
    \begin{subfigure}[t]{0.348\textwidth}
        \centering
        \vspace{0pt}
          \includegraphics[width=\textwidth, height=3.85cm]{images/num_posts_over_time_line.pdf}
          \caption{Number of Posts by Year}
          \Description{num_posts_over_time_line}
        \label{fig:posts_per_year}
    \end{subfigure}
    \hfill
    \begin{subfigure}[t]{0.372\textwidth}
        \centering
        \vspace{0pt}
          \includegraphics[width=\textwidth, height=4cm]{images/num_posts_per_year_line.pdf}
          \caption{Number of posts by Month}
          \Description{num_posts_per_year_line}
        \label{fig:posts_per_month}
        \end{subfigure}
    \hfill
    \begin{subfigure}[t]{0.24\textwidth}
        \centering
        \vspace{0pt}
        \includegraphics[width = \textwidth, height=4cm]{images/weekend_day_bar_correct.pdf}
        \caption{Number of Posts by Day Type and Time of Day Quartiles}
        \Description{weekend}
        \label{fig:posts_weekend}
        \end{subfigure}

    \medskip

    \centering
    \begin{subfigure}[t]{0.348\textwidth}
        \centering
        \vspace{0pt}
          \includegraphics[width=\textwidth, height=3.85cm]{images/num_comm_over_time_line.pdf}
          \caption{Engagement Level by Year}
          \Description{num_comm_over_time_line}
        \label{fig:comm_per_year}
    \end{subfigure}
    \hfill
    \begin{subfigure}[t]{0.372\textwidth}
        \centering
        \vspace{0pt}
          \includegraphics[width=\textwidth, height=4cm]{images/num_comm_per_year_line.pdf}
          \caption{Engagement Level by Month}
          \Description{num_comm_per_year_line}
        \label{fig:comm_per_month}
        \end{subfigure}
    \hfill
    \begin{subfigure}[t]{0.24\textwidth}
        \centering
        \vspace{0pt}
        \includegraphics[width = \textwidth, height=4cm]{images/weekend_day_bar_comments_correct.pdf}
        \caption{Engagement Level by Day Type and Time of Day Quartiles}
        \Description{comm weekend}
        \label{fig:comm_weekend}
    \end{subfigure}
    \caption{
    \textit{Posts and comments in r/Food over time.} We present how postings and comments developed from $2017$ until $2023$ across different temporal scales including yearly, monthly, weekly, and daily trends.
    In (\subref{fig:posts_per_year}) we present the number of posts over the years. We observe a positive trend before the COVID-19 pandemic with a noticeable peak during the pandemic and a drop afterwards to pre-pandemic levels.
    Monthly posting activity in (\subref{fig:posts_per_month}) is rather consistent except for a peak between March and June $2020$ during the pandemic.
    In (\subref{fig:posts_weekend}) we observe that more posts are created on weekdays than on weekends (left) and that most posts are created in the afternoon in the eastern USA (Q4, right). The bottom row shows the same diagrams for comments.
    In (\subref{fig:comm_per_year}) we observe a gradual increase in commenting activity over time with the highest activity levels during the pandemic and a sharp drop after the pandemic.
    This observation is also reflected in (\subref{fig:comm_per_month}), where we see constant high levels of comments in $2020$. We also see a seasonal spike in January possibly due to the holiday season.
    In (\subref{fig:comm_weekend}), comments mirror posting activity, with more comments over the weekdays (left). On the other hand, the peak in comments is in the morning (Q3, right).
    }
    \label{fig:temporal}
\end{figure*}


\xhdr{Scores}
Each Reddit post has a score, determined by the difference of community ``upvotes'' and ``downvotes''. 
The mean score is $245$, with a standard deviation of $1,740.03$, indicating high variability. The median score is only $23$, signaling again a skewed distribution where most posts receive relatively modest scores while the highest score is $70,308$.
In this paper, we do not use score as an engagement metric and opt for comments, which require more effort from the users. In addition, the score has a strong positive correlation with the number of comments ($\rho = 0.627, p < 0.001$), indicating that comments are a comprehensive representation of engagement. %Spearman


\xhdr{Temporal characteristics}
In Figure \ref{fig:temporal} we depict the temporal development of activity and user engagement in r/Food. The number of posts steadily increased over time (Fig. \ref{fig:posts_per_year}), peaking in $2020$, likely due to the COVID-19 pandemic as this surge can be potentially attributed to the widespread lockdowns, increased interest in food, and the shift towards consuming more meals at home \cite{gligoric_population-scale_2022}.
Following a few months of the COVID-19 outbreak, the number of posts rapidly dropped, with a brief increase at the beginning of $2021$.
Subsequently, the number of posts continued to decrease, returning to pre-pandemic levels and even falling further. 

When comparing the monthly post counts across years (Fig. \ref{fig:posts_per_month}), a similar pattern is observed for most years, except for $2020$.
Specifically, there is a noticeable spike in the number of posts during March $2020$, corresponding to the onset of the pandemic and lockdowns.
We show the distribution of postings over weekdays and weekends as well as the time of the day in Fig. \ref{fig:posts_weekend}. In the dataset, the posting time is stored in UTC time. 
As the majority of Reddit traffic comes from the USA (cf. Reddit traffic as of March 2024\footnote{\url{https://www.statista.com/statistics/325144/reddit-global-active-user-distribution/}}) with time zones ranging from EST (UTC - $5$) to PST (UTC - $8$), we interpret the time of the day results using USA eastern times. The exact time ranges in different time zones are presented in Table \ref{table:timezones}.
Hence, we observe that more posts are made on weekdays ($279,299$) than on weekends ($137,480$). Further, posting activity peaks in the afternoon in the eastern USA (likely reflecting users' lunchtime), accounting for $33.3$\% of total posts. 
This is followed by $29.3$\% of posts made in the evening and $26.4$\% in the morning in the eastern USA. The least amount of activity occurs during the night, with only $11.6$\% of posts.

\begin{table}[b]
\centering
\caption{\textit{Time of Day Quartiles.} The dataset contains times in UTC. As most Reddit users are from the USA, we interpret these times in both EST and PST.}
\begin{tabular}{@{}l l l l@{}}
\toprule
Quartile & UTC              & EST              & PST              \\ \midrule
Q1 (evening) & $12$ AM - $6$ AM  & $7$ PM - $1$ AM & $4$ PM - $10$ PM \\
Q2 (night)   & $6$ AM - $12$ PM  & $1$ AM - $7$ AM & $10$ PM - $4$ AM \\
Q3 (morning) & $12$ PM - $6$ PM & $7$ AM - $1$ PM & $4$ AM - $10$ AM \\
Q4 (afternoon) & $6$ PM - $12$ AM & $1$ PM - $7$ PM & $10$ AM - $4$ PM \\
\bottomrule
\end{tabular}
\label{table:timezones}
\end{table}

In the bottom row of Figure \ref{fig:temporal} we show the same temporal analysis for comments. In these figures, we categorize comments according to the time of their original postings. For instance, if a post is made in June $2020$, we treat all its comments as if they were made in June $2020$, even if they are posted at a later date. The commenting activity shows a gradual increase until $2020$ (Fig. \ref{fig:comm_per_year}).
Just before $2020$, there was an abrupt drop in the number of comments, followed by a sharp increase.
This trend aligns with the rise in posting behavior during that time and the onset of the COVID-19 pandemic. Another notable increase occurred at the beginning of $2021$.
Subsequently, there has been a steady decline, with the number of comments falling below pre-pandemic levels. There is no clear seasonality observed when comparing the monthly number of comments (Fig. \ref{fig:comm_per_month}).
The highest number of comments is recorded throughout $2020$, likely due to the pandemic.

More comments were made on posts published on weekdays compared to weekends (Fig. \ref{fig:comm_weekend}), which is consistent with the higher number of posts being made during the weekdays. Posts made in the morning in the eastern USA receive the highest number of comments ($1,280,826$ comments, $34.05$\% of total). This large number of comments could be attributed to users' activity during lunchtime and time after work, where they engage with posts made previously in the day.
Posts made in the afternoon ($1,093,762$ comments, $29.08$\%) closely follow. Next are posts made in the evening ($928,398$ comments, $24.68$\%), while posts made during the night receive the least comments ($458,086$ comments, $12.18$\%).


\xhdr{Tags}
According to the current subreddit rules, each post must include a tag indicating the context of the meal: whether the user prepared it at home, whether the user works in the food industry and prepared it, or whether the user purchased it without personal preparation. The majority of meals, $75$\%, were prepared at home by the users, while $19$\% were purchased without any preparation, and $1.5$\% were prepared by food industry professionals. 
The remaining $4.5$\% of posts either lack a tag, most likely due to earlier subreddit policies of not enforcing the tag structure, or these posts include a user-chosen tag.

\begin{figure*}[t]
    \centering
    \begin{subfigure}{0.25\textwidth}
        \centering
        \includegraphics[width=\textwidth]{images/n/rq1_calorie_distribution.pdf}
        \caption{Calorie Densities}
        \Description{calorie distribution}
        \label{fig:rq1_calorie}
    \end{subfigure}
    \hfill
    \begin{subfigure}{0.24\textwidth}
        \centering
        \includegraphics[width =\textwidth]{images/n/rq1_weighted_protein_distribution.pdf}
        \caption{Protein Densities}
        \Description{protein distribution}
        \label{fig:rq1_protein}
        \end{subfigure}
    \hfill
    \begin{subfigure}{0.24\textwidth}
        \centering
        \includegraphics[width =\textwidth]{images/n/rq1_weighted_carb_distribution.pdf}
        \caption{Carbohydrate Densities}
        \Description{carb distribution}
        \label{fig:rq1_carb}
        \end{subfigure}
    \hfill
    \begin{subfigure}{0.24\textwidth}
        \centering
        \includegraphics[width = \textwidth]{images/n/rq1_weighted_fat_distribution.pdf}
        \caption{Fat Densities}
        \Description{fat distribution}
        \label{fig:rq1_fat}
        \end{subfigure}


    \medskip


    \centering
    \begin{subfigure}{0.25\textwidth}
        \centering
        \includegraphics[width=\textwidth]{images/n/rq2_calorie_distribution.pdf}
        \caption{Calorie Densities}
        \Description{calorie distribution}
        \label{fig:rq2_calorie}
    \end{subfigure}
    \hfill
    \begin{subfigure}{0.24\textwidth}
        \centering
        \includegraphics[width =\textwidth]{images/n/rq2_weighted_protein_distribution.pdf}
        \caption{Protein Densities}
        \Description{protein distribution}
        \label{fig:rq2_protein}
        \end{subfigure}
    \hfill
    \begin{subfigure}{0.24\textwidth}
        \centering
        \includegraphics[width =\textwidth]{images/n/rq2_weighted_carb_distribution.pdf}
        \caption{Carbohydrate Densities}
        \Description{carb distribution}
        \label{fig:rq2_carb}
        \end{subfigure}
    \hfill
    \begin{subfigure}{0.24\textwidth}
        \centering
        \includegraphics[width = \textwidth]{images/n/rq2_weighted_fat_distribution.pdf}
        \caption{Fat Densities}
        \Description{fat distribution}
        \label{fig:rq2_fat}
        \end{subfigure}

        
    \caption{
    \textit{Nutritional content distribution of food in r/Food posts}. We illustrate the distribution of calories (\subref{fig:rq1_calorie}, \subref{fig:rq2_calorie}) and macro-nutrients (\subref{fig:rq1_protein}--\subref{fig:rq1_fat}, \subref{fig:rq2_protein}--\subref{fig:rq2_fat}) per $100$g of food, across meal in (i) engaging (red) and non-engaging (blue) posts, and (ii) resonant (red) and non-resonant (blue) posts.
    The calorie content is measured in kCal per $100$g, while macro-nutrients are measured in grams as fractions of $100$g total.
    We observe that the majority of posts fall within the moderate calorie range, between $100$ and $300$ kCal.
    \textit{Top row:} Calorie densities of posts with comments and without comments appear similar but differ significantly in means (\subref{fig:rq1_calorie}).
    We observe a steep decline in the protein (\subref{fig:rq1_protein}) density, with most posts having less than $20$g of protein, suggesting a prevalence of low-to-moderate protein meals.
    Carbohydrates (\subref{fig:rq1_carb}) span over a wider range. While most posts have less than $20$g, there is a consistent amount of carb-rich food as well, as indicated by the long tail in their distributions. 
    Fat (\subref{fig:rq1_fat}) distribution peaks around $10$-$15$g, with most posts containing moderate fat content.
    \textit{Bottom row:} Distribution disparities are more prominent when comparing resonant vs. non-resonant posts. Posts that do not resonate with the community peak at around $150$ kCal, while posts that do resonate peak at $300$ kCal (\subref{fig:rq2_calorie}). We observe similar behavior in all other macronutrient densities, with distributions for resonant posts being shifted to the right as compared to non-resonant posts. (\subref{fig:rq2_protein}--\subref{fig:rq2_fat}).
    }
    \label{fig:macros}
\end{figure*}

\xhdr{Engagement levels}
We operationalize engagement by the number of comments and define posts with at least one comment to be engaging posts and without comments to be non-engaging posts. Further, we define resonant and non-resonant posts by categorizing posts in the top $1$\% by comment count (i.e., the first percentile) as resonant ($3,219$ posts), while those with $0$ or $1$ comment are considered non-resonant ($157,470$ posts).
Note that slight variations in the definition of low resonance, such as considering only posts without comments, posts with just one comment, or posts with up to five comments, did not impact the results. To obtain balanced classes for our prediction experiment (cf. Sec. \ref{sec:prediction})
we randomly sample $3,219$ non-resonant posts, resulting in $6,438$ posts for further analysis. 

\xhdr{Nutritional Content Analysis}
We show the distributions of macronutrient content of meals in Figure \ref{fig:macros}. 
Specifically, we compare the nutritional content distributions of posts with and without user engagement (top row Fig. \ref{fig:macros}), as well as resonant vs. non-resonant posts (bottom row Fig. \ref{fig:macros}). 
The majority of posts fall within the moderate calorie range, from $100$ to $300$ kcal per $100$g of food.
When comparing posts with and without engagement, the calorie distribution appears similar (Fig. \ref{fig:rq1_calorie}), although the difference in means is statistically significant ($p < 10^{-235}$, Mann-Whitney-U test).
The calorie distribution is bimodal, with one peak at around $150$ calories and another at around $300$ calories.

Similarly to the calorie distributions the distributions of other macronutrients appear similar to each other, but all the differences in means are statistically significant (all $p < 10^{-6}$).
For example, the vast majority of meals contain up to $20$g of protein (Fig. \ref{fig:rq1_protein}).
Posts without engagement show a peak at around $5$g of protein, with a gradual decline in posts as protein content increases.
In contrast, posts with engagement exhibit a spike at around $5$g of protein, followed by another increase at just over $10$g of protein. 
Posts with engagement generally feature meals with higher carbohydrate content, while posts without engagement tend to feature meals with lower carbohydrate content (Fig. \ref{fig:rq1_carb}).
The fat distribution is similar, with posts without engagement tending to feature lower-fat meals (Fig. \ref{fig:rq1_fat}).

There is a clear difference when comparing resonant to non-resonant posts (bottom row Fig. \ref{fig:macros}).
A significant difference in means is observed in the calorie distribution ($p < 10^{-53}$).
Most non-resonant posts contain meals with fewer than $150$ calories, while the majority of resonant posts contain meals with around $300$ calories (Fig. \ref{fig:rq2_calorie}).
Beyond $300$ calories, the number of posts that resonate with users is constantly higher than the number of posts that do not resonate.
While both types of posts tend to feature low-protein meals, higher-protein meals are more often found in resonant posts (Fig. \ref{fig:rq2_protein}). However, there is no significant difference in means between the protein distributions ($p = 0.1$).
Similarly, both low-carbohydrate and low-fat values are associated with non-resonant posts (Fig. \ref{fig:rq2_carb} and Fig. \ref{fig:rq2_fat}).
Conversely, higher carbohydrate and fat values are linked to posts that resonate with the community.
The means of both these macronutrients are significantly different between resonant and non-resonant posts ($p < 10^{-27}$).

\section{Predicting Engagement}
\section{Prediction-based Model Selection}\label{appendix:prediction}

This section provides a comparative analysis of several TTFT prediction methods. For selecting the endpoint with a lower TTFT for each request, TTFT prediction is imperative. For on-device inference, TTFT prediction is straightforward, as TTFT exhibits a linear relationship with prompt length. Conversely, on-server inference TTFT is characterized by high variability, rendering prediction challenging. Moreover, the prediction method itself must be computationally efficient, as its overhead also contributes to end-to-end TTFT.

Table~\ref{tab:model-comparison} presents a comparative analysis of four common lightweight time-series-based prediction methods applied to traces collected from three prevalent LLM services. Our correlation analysis (Table~\ref{tab:correlation-analysis}) revealed no significant correlation between prompt length and TTFT; thus, prompt length is omitted as a feature in these prediction methods. We demonstrate that none of these methods offers sufficient accuracy for TTFT prediction.

\begin{table}[t]
    \centering
    \footnotesize
    \begin{tabular}{p{3.5cm}cc}
    \toprule
    \textbf{Model} & \textbf{MAPE(\%)} & \textbf{MAE(s)} \\
    \midrule
    \multicolumn{3}{c}{\textbf{Command}} \\
    \midrule
    Moving Average & 39.40 & 0.0899 \\
    ExponentialSmoothing & 53.51 & 0.1047 \\
    Random Forest & 39.33 & 0.0966 \\
    XGBoost & 35.43 & 0.0905 \\
    \midrule
    \multicolumn{3}{c}{\textbf{DeepSeek-V2.5}} \\
    \midrule
    Moving Average & 27.80 & 0.3959 \\
    ExponentialSmoothing & 27.39 & 0.3771 \\
    Random Forest & 32.97 & 0.4745 \\
    XGBoost & 27.51 & 0.4001 \\
    \midrule
    \multicolumn{3}{c}{\textbf{GPT-4o-mini}} \\
    \midrule
    Moving Average & 24.55 & 0.0995 \\
    ExponentialSmoothing & 20.88 & 0.0844 \\
    Random Forest & 28.68 & 0.1128 \\
    XGBoost & 24.83 & 0.0997 \\
    \midrule
    \multicolumn{3}{c}{\textbf{LLaMA-3-70b-Instruct}} \\
    \midrule
    Moving Average & 42.18 & 0.3312 \\
    ExponentialSmoothing & 40.27 & 0.3154 \\
    Random Forest & 49.67 & 0.3875 \\
    XGBoost & 43.94 & 0.3451 \\
    \bottomrule
    \end{tabular}
    \caption{Comparative analysis of Moving Average, Exponential Smoothing, Random Forest, and XGBoost prediction models across Command, DeepSeek, GPT, and LLaMA model traces. Metrics include Mean Absolute Percentage Error (MAPE) and Mean Absolute Error (MAE).}
    \label{tab:model-comparison}
\end{table}

%\section{Results and Discussion}
\subsection{Results}

\begin{table}[b]
\centering
\caption{\textit{Results.} ROC-AUC scores across engagement and resonance predictions.}
%\resizebox{\columnwidth}{!}{
\begin{tabular}{@{}l c c@{}}
\toprule
Model         & ROC-AUC (Engagement)        & ROC-AUC (Resonance)        \\ \midrule
Control (C)             & $0.593$ [$0.587$, $0.599$] & $0.669$ [$0.641$, $0.698$] \\
C + Nutrition (N)       & $0.603$ [$0.597$, $0.609$] & $0.709$ [$0.680$, $0.737$] \\
C + Food Descriptors (F)     & $0.594$ [$0.588$, $0.600$] & $0.671$ [$0.642$, $0.700$] \\
C + Engagement Discriminators (E)         & $0.602$ [$0.596$, $0.608$] & $0.699$ [$0.672$, $0.726$] \\
C + N + F           & $0.602$ [$0.597$, $0.608$] & $0.710$ [$0.682$, $0.739$] \\
C + N + E           & $0.607$ [$0.602$, $0.613$] & $0.717$ [$0.690$, $0.745$] \\
C + F + E         & $0.602$ [$0.596$, $0.607$] & $0.694$ [$0.666$, $0.723$] \\
C + N + F + E        & $0.608$ [$0.603$, $0.614$] & $0.713$ [$0.687$,$ 0.743$] \\ \bottomrule
\end{tabular}
%}
\label{table:auc_roc_scores}
\end{table}



We present our results in Table \ref{table:auc_roc_scores} where we summarize our main findings, with ROC-AUC scores and their corresponding bootstrap confidence intervals for each model.

\xhdr{Predicting engagement}
Using only control feature set our classification model achieves a ROC-AUC score of $0.593$.
Adding the nutritional density features improves the ROC-AUC score by $1$\%, to $0.603$.
In contrast, adding food descriptors to controls does not improve prediction performance (ROC-AUC of $0.594$).
Similarly to nutritional features, adding engagement discriminators to the controls raises ROC-AUC to $0.602$.
Further, combining the nutritional density and food descriptors, or the nutritional density and engagement discriminators with the controls achieves ROC-AUC scores of $0.602$ and $0.607$, respectively. 
Similarly, adding the food descriptors and engagement discriminators to the controls also achieves ROC-AUC of $0.602$.
Finally, combining all feature sets achieves the best ROC-AUC score of $0.608$.

\xhdr{Predicting resonance}
Due to the resonant and non-resonant classes being more distinctly separated than in the previous experiment, the model with the controls already achieves the ROC-AUC score of $0.669$.
Moreover, adding nutritional density to the controls improves the score to $0.709$, or by $4$\%.
When adding food descriptors we observe a $1$\% improvement in performance (ROC-AUC of $0.671$) and
when adding engagement discriminators the performance improves by $3$\% to $0.699$. 
Further, when we combine discriminators and the descriptors with the controls the model improves to $0.694$.
Combining nutritional density and food descriptors with the controls also increases the performance by $4$\% indicating that the food descriptors are not predictive of resonance.
Finally, adding nutritional content and engagement discriminators we achieve ROC-AUC of $0.713$, the same as the model combining all feature sets. Hence, those models achieve the best overall performance as compared to the controls, improving that base model by $5$\%.


\begin{figure}[t]
    \centering
    \begin{subfigure}[t]{0.51\linewidth}
        \centering
        \includegraphics[width=\linewidth]{images/shap_rq1.pdf}
        \caption{The beeswarm plot displaying how certain features impact whether a post will get a comment or not}
        \label{fig:shap_rq1}
    \end{subfigure}
    \hfill
    \begin{subfigure}[t]{0.45\linewidth}
        \centering
        \includegraphics[width = \linewidth]{images/shap_bar_rq1.pdf}
        \caption{Features with a large contribution to the model predictions}
        \label{fig:shap_bar_rq1}
        \end{subfigure}
    %\caption{Combined figure}
    %\label{fig:combined}
    %\centering
    \vfill
    \begin{subfigure}[t]{0.48\linewidth}
        \centering
        \includegraphics[width=\linewidth]{images/shap_rq1_pred_0.pdf}
        \caption{``[I ate] Pork Belly, Sweet Potato Purée, Maple Gastrique, Micro Celery'' post, not achieving engagement}
        \label{fig:shap_rq1_pred_0}
    \end{subfigure}
    \hfill
    \begin{subfigure}[t]{0.48\linewidth}
        \centering
        \includegraphics[width = \linewidth]{images/shap_rq1_pred_1.pdf}
        \caption{``Red Velvet Cupcakes with Cream Cheese Frosting and Chocolate Decoration'' post, achieving engagement}
        \label{fig:shap_rq1_pred_1}
        \end{subfigure}
    \caption{
        \textit{SHAP visualizations for classifier predicting post engagement.}
        Looking at SHAP values of different features, we can understand to which degree they influence the probability of a post receiving engagement. 
        In the beeswarm plot (\subref{fig:shap_rq1}) we display how the top features impact the model's output, with each dot representing one post. 
        Posting after COVID-19, being an experienced user, and having higher calorie meals strongly increases the likelihood of engagement prediction.   
        Additionally, the absence of no-engagement discriminators, posting on the weekday and later in the day further increases those odds.
        Foods with either high or low protein content have a higher probability of engagement than foods with moderate protein content.
        In the bar plot (\subref{fig:shap_bar_rq1}) we present the feature importance in absolute values. The calorie density ranks third after controlling for COVID-19 and the user tenure.
        We also present two examples (\subref{fig:shap_rq1_pred_0}-\subref{fig:shap_rq1_pred_1}) with local feature importance, highlighting the concrete values of each feature and the way they contributed to the prediction of a post receving comments.
        }
    \label{fig:shap_all_rq1}
    \Description{SHAP plot rq1}
\end{figure}

\xhdr{Feature importance with SHAP values}
To better understand the associations of features and user engagement and their effects on 
the engagement prediction we calculate SHapley Additive exPlanations (SHAP) values.
SHAP values represent the contributions of each individual feature value to the prediction score.
In particular, a positive SHAP value for a feature value indicates an increase in the probability of a positive prediction, while a negative SHAP value for a particular feature value suggests a decrease in the probability of a positive class prediction. For example, in Figure \ref{fig:shap_rq1} we illustrate the contribution of features to the prediction of whether a post will receive comments by plotting the SHAP values across posts.
In particular, in this diagram, we observe how different values of a particular feature affect the model's prediction.
Hence, each point represents a SHAP value for an individual post given its corresponding feature value.
The SHAP values are given on the x-axis reflecting their impact on the model's output.
Specifically, positive SHAP values push the prediction towards engagement, while negative values reduce the probability of the positive class.
The color gradient represents the feature value, with blue indicating lower feature values and red indicating higher values. For example, blue points in the third row of Fig. \ref{fig:shap_rq1} represent lower calorie and red higher calorie density.
Additionally, Figure \ref{fig:shap_bar_rq1} displays the mean of absolute SHAP values across the range of possible feature values. 
While the plot does not differentiate the direction of the feature value, it depicts the overall feature importance in the prediction model.
Lastly, SHAP values allow to inspect how individual predictions are computed to further gain an understanding of the associations between features and the model output. For example, Figure \ref{fig:shap_rq1_pred_0} shows a waterfall plot visualizing the breakdown of the model's prediction for a single instance and highlighting how each feature influences the outcome.
In the waterfall plots, contributions from each feature are depicted as colored bars, showing their cumulative additive influence on the outcome.
To understand the model's prediction for a specific observation, we start with the baseline prediction without features, and iterate trough the features summing their SHAP values to obtain the final prediction (a number between $-1$ and $1$, with $0$ being the class threshold).

\xhdr{Role of nutritional content in engagement prediction}
In Figure \ref{fig:shap_rq1} we observe that calorie density is positively associated with engagement, with a mean absolute SHAP value of $0.09$ (Fig. \ref{fig:shap_bar_rq1}). 
This value is lower than the contributions of the COVID and user experience features, which are control features. In particular, a mean absolute SHAP value of $0.18$ indicates that, on average, the COVID feature contributes $0.18$ in absolute terms to the model's prediction of a post receiving comments, hence, suggesting that posts made after the onset of COVID-19 (March $15$, $2021$) are more likely to receive engagement.
Similarly, user experience also positively affects engagement, with more experienced users generally being associated with engagement. However, calorie density still ranks as the third most important feature even after controlling for COVID and user experience, highlighting its strong impact on engagement prediction. 

Moreover, the calorie density has a stronger effect on the model's prediction than the absence of no-engagement discriminators (cf. row four in \ref{fig:shap_rq1} and \ref{fig:shap_bar_rq1}) or the presence of engagement discriminators (cf. row five in \ref{fig:shap_rq1} and \ref{fig:shap_bar_rq1}), which were selected as the words distinctively used in different classes. This further underlines the strong predictive power of calorie density on user engagement with a particular post.
Regarding other nutritional densities, we observe that lower protein content does not drastically affect engagement while higher protein content has a polarizing effect, either significantly boosting or detracting from engagement prediction. 
In other words, high protein can be a critical factor in determining the prediction outcome, either positively or negatively.
In addition, we also observe that posting on weekends or later in the day appears (both are control features) to negatively influence the likelihood of engagement.


Further, in Figure \ref{fig:shap_rq1_pred_0} we show the waterfall plot for a post without comments as an individual prediction example. The model's baseline is $-0.011$.
The title of this particular post is ``[I ate] Pork Belly, Sweet Potato Purée, Maple Gastrique, Micro Celery'' and it was posted after the initial COVID outbreak and by an experienced user. Consistent with the previously discussed results, those two features increase the probability of the positive class.
Nevertheless, other features such as the presence of the word ``sweet'' (food descriptor) as well as no-engagement discriminators and the absence of the engagement discriminators reduce the prediction probability.
Additionally, low protein content ($7$g) and low carbohydrate content ($5$g) are also negatively associated with the likelihood of engagement resulting in the final prediction for comments to be  $-0.183$. We show another waterfall plot in Figure \ref{fig:shap_rq1_pred_1} with an example of a post that received comments.
The title of the post is ``Red Velvet Cupcakes with Cream Cheese Frosting and Chocolate Decoration''.
The figure demonstrates that the user's experience and the calorie content of the dish ($358$ kCal) positively contribute to the likelihood of engagement prediction. 
In contrast, the fact that the post was made before the onset of COVID-19 reduces the prediction probability. 
Conversely to the previous example, the very low protein content ($3$g) has in this case a positive effect on the likelihood of engagement prediction.
The presence of an engagement discriminator further boosts the model's probability, whereas the presence of a no-engagement discriminator negatively influences the outcome. 
Overall, the predicted value of a post receiving engagement is $0.282$.

\begin{figure}[t]
    \centering
    \begin{subfigure}[t]{0.51\linewidth}
        \centering
        \includegraphics[width=\linewidth]{images/shap_rq2.pdf}
        \caption{The beeswarm plot displaying how certain features impact whether a post will resonate or not}
        \label{fig:shap_rq2}
    \end{subfigure}
    \hfill
    \begin{subfigure}[t]{0.45\linewidth}
        \centering
        \includegraphics[width = \linewidth]{images/shap_bar_rq2.pdf}
        \caption{Features with a large contribution to the model predictions}
        \label{fig:shap_bar_rq2}
        \end{subfigure}
    %\caption{Combined figure}
    %\label{fig:combined}
    %\centering
    \vfill
    \begin{subfigure}[t]{0.48\linewidth}
        \centering
        \includegraphics[width=\linewidth]{images/shap_rq2_pred_0.pdf}
        \caption{``[homemade] minestrone soup'' post, not resonating with the community}
        \label{fig:shap_rq2_pred_0}
    \end{subfigure}
    \hfill
    \begin{subfigure}[t]{0.48\linewidth}
        \centering
        \includegraphics[width = \linewidth]{images/shap_rq2_pred_1.pdf}
        \caption{``[Homemade] Unicorn and Dinosaur cookies'' post, resonating with the community}
        \label{fig:shap_rq2_pred_1}
        \end{subfigure}
    \caption{
        \textit{SHAP visualizations for classifier predicting post's resonance level.}
        SHAP values of features provide us with the transparency of a classifier and allow us to understand which features are beneficial to achieve resonance. In the beeswarm plot (\subref{fig:shap_rq2}) we present how the values of features impact the prediction, while the bar plot (\subref{fig:shap_bar_rq2}) presents each top feature's importance. Calorie density is the fourth most important feature after several control features. Its importance is more than two times higher than for the engagement predictor. Similarly, a higher carbohydrate value indicates the prediction of resonance, and a higher protein content, although smaller, still has a positive influence. Fat content has less of a prediction power. Additionally, we present two concrete examples (\subref{fig:shap_rq2_pred_0}-\subref{fig:shap_rq2_pred_1}), with local feature importance and values that drive the prediction of a post resonating with the community.
        }
    \label{fig:shap_all_rq2}
    \Description{SHAP plot rq2}
\end{figure}


\xhdr{Role of nutritional content in resonance prediction}
In Figure \ref{fig:shap_rq2} and \ref{fig:shap_bar_rq2} we present how features are associated with post resonance. We observe, that the mean absolute SHAP value for calorie density is $0.19$, which is more than twice higher than the value for calorie density when predicting engagement. Although, the absence of low resonance discriminators, high user experience, and posting midday have the strongest influence on resonance prediction (all are control features), the calorie density ranks as the fourth most important feature. Hence, posts featuring high-calorie meals are more likely to resonate with the community even after accounting for the influence of the control features. On the other hand, low-calorie meals tend to be associated with non-resonant posts.
Similarly, SHAP values suggest that high carbohydrate content is also linked with positive resonance prediction.
In contrast, protein content has a more nuanced negative correlation with the resonance. 
While high protein density is associated with non-engaging posts, low protein content tends to appear more likely in resonant posts. 
Besides nutritional content, another control feature (posting on weekdays) is more beneficial for resonance than posting on weekends.

To further illustrate our findings, we again examine two specific examples in more detail.
The first example (Fig. \ref{fig:shap_rq2_pred_0}) visualizes the features of a post titled ``[homemade] minestrone soup'', which does not resonate well with the community.
According to the chart, the post lacked a low resonance discriminator and was posted by an experienced user, which both favored resonance prediction.
Regardless, the majority of other features contributed negatively to the prediction.
The meal’s very low calorie ($53$ kCal), protein ($3$g), and carbohydrate ($8$g) content all had a negative effect.
Additionally, the post was made during the fourth quartile of the day and after COVID, both of which further reduced its likelihood of resonance.
These combined factors result in the prediction value of $-0.95$, classifying the post as a non-resonant post.
The second example (Fig. \ref{fig:shap_rq2_pred_1}) represents a post titled ``[Homemade] Unicorn and Dinosaur cookies'', which resonated well with the community.
The meal's high-calorie content ($432$ kCal) and high carbohydrate content ($75$g) significantly increased the likelihood of resonance.
In this case, the low protein content ($7$g) also positively influenced the outcome.
However, several features negatively impacted the prediction, including the user’s low experience, the presence of a low engagement discriminator, and the post being made in the evening (Q$1$). 
Despite these negative factors, the overall prediction value was $0.414$, classifying the post as one resonating with the community.


\subsection{Discussion}

Our findings on the relation between nutritional content and engagement in food-related social media posts provide several key insights into user behavior as well as the context and content of engaging posts. First, including the nutritional content as a feature set in our engagement prediction models significantly enhances the model's classification accuracy, suggesting a strong predictive power of these nutritional features for engagement. Additionally, we uncover the direction of this strong association: more calorie-dense meals increase the prediction probability for user engagement and, in particular, for post resonance.

The influence of calorie content aligns with prior research suggesting that users are more drawn to calorie-dense meals \cite{pancer_content_2022}.
Posts with higher calorie content consistently demonstrate higher SHAP values, emphasizing their role in engagement prediction.
This finding is further corroborated by the interaction between calorie and carbohydrate density. In particular, posts with both high-calorie and high-carbohydrate meals are typically more likely to reach top engagement levels.
Conversely, protein content's nuanced effect suggests that high-protein meals may appeal to a more specific audience and, hence, not resonate well enough with a broader user community.

We note again that the associations between nutritional content and user engagement are still significant even after controlling for a range of non-food-related features. 
For example, user experience appears as a critical feature strongly related to engagement \cite{bakshy_everyones_2011}. 
For example, 
Posts by older and more experienced users had higher likelihoods of engagement, confirming findings from studies on social media websites such as Usenet and Twitter, 
where contributions by long-term users were more likely to receive responses \cite{arguello_talk_2006, suh_want_2010, turkoglu_improving_2023}.
This phenomenon may be related to the community perceiving experienced users' content as being of higher quality, or to the experienced users being able to understand the community and their expectations better than inexperienced ones. Further, the timing of the posts is also significantly related to engagement. 
Posts made after the onset of COVID-19 experienced higher engagement, potentially due to increased digital screen time during lockdowns \cite{wong_digital_2021} and the rise in internet traffic \cite{feldmann_lockdown_2020}.
However, posts made later in the day or during weekends were less likely to engage users, agreeing with the findings that weekday posts during busier hours attract more interaction \cite{wahid_social_2020, hanifawati_managing_2019}.
Finally, our findings align with studies indicating that captions and post titles significantly influence engagement \cite{hessel_cats_2017, chen2021drives} reinforcing the importance of carefully crafting titles and captions to resonate with audiences.

\xhdr{Limitations}
Even though Reddit is an anonymous platform that allows more authentic behavior, it comes with several limitations.
First, we miss the detailed user demographics as well as their individual interests and nutritional goals. 
%Thus, considering the complexity of human behavior, modeling it as a whole might be too broad of a generalization.
Second, we do not account for bots on the platform, which are easily created by users \cite{long2017could}.
The presence of bots can influence user engagement and the dynamics of online interaction.
Third, Reddit's algorithm that curates feeds may influence user engagement with specific, assumed-relevant, posts.

Another limitation of our work is that we exclude visual features, which in combination with captions are accurate predictors of engagement on Reddit \cite{hessel_cats_2017}.
While focusing on the nutritional density of the meals instead of their appearance is the goal of our research, we are aware of the impact of the visual elements.
Given Reddit's focus on images, and the general influence of food aesthetics on engagement rates \cite{philp_predicting_2022}, the aesthetic of a meal could serve as an additional control feature, providing deeper insights. We plan to extend our analysis by including visual features as another set of control features in our future work.

Although we aimed for a robust estimation method of the nutritional content by using pre-trained embeddings, a robust similarity threshold, and a similarity-weighted density aggregation, we acknowledge potential inaccuracies in our estimation due to, among others, variations in ingredient ratios. In particular, we estimate the nutritional densities and not the total amounts of nutritional content.
Additionally, common meals, such as pizza, can have numerous variations, and users may not feel the need to specify these differences in the title, as they accompany their title with a picture.
Since our estimation of nutritional content relies solely on the title, our approach may overlook valuable information that could enhance the accuracy of the estimation.

Moreover, our work is a large-scale study of a single, although large, community (i.e. Reddit's r/Food). 
While we believe that the amount of data (almost 600,000 posts) and the huge user base of this subreddit (24 million users) is sufficient for an analysis of this kind, we still acknowledge potential sample bias in users who post and engage with this community.
Therefore, our findings might not necessarily generalize to other communities. 
However, we see this as an opportunity to extend our work to other social media platforms that garner a large number of users, such as Instagram.

Finally, we caution that our work indicates an associative link between nutritional density and different levels of engagement, and does not establish causality.
Albeit we control for several confounding features, which makes the evidence we find for this link stronger, we still conduct an observational study without a particular setup needed for causal inference.

%%%%



\section{Conclusion}
\section*{Conclusion}
This paper aims to enhance our understanding of the computational complexity of computing various Shapley value variants. We found that for various ML models --- including decision trees, regression tree ensembles, weighted automata, and linear regression --- both local and global interventional and baseline SHAP can be computed in polynomial time under HMM modeled distributions. This extends popular algorithms, such as TreeSHAP, beyond their empirical distributional scope. We also establish strict complexity gaps between the various SHAP variants (baseline, interventional, and conditional) and prove the intractability of computing SHAP for tree ensembles and neural networks in simplified scenarios. Overall, we present SHAP as a versatile framework whose complexity depends on four key factors: \begin{inparaenum}[(i)] \item model type, \item SHAP variant, \item distribution modeling approach, \item and local vs. global explanations\end{inparaenum}. We believe this perspective provides deeper insight into the computational complexity of SHAP, paving the way for future work.




%We believe that our framework provides a more intricate understanding of SHAP computation complexity across different models, distributions, and variants, paving the way for further research.

Our work opens promising directions for future research. First, expanding our computational analysis to other SHAP-related metrics, such as asymmetric SHAP~\citep{frye20} and SAGE~\citep{covert2020understanding}, would be valuable. Additionally, we aim to explore more expressive distribution classes and relaxed assumptions beyond those in Section \ref{sec:tractable} while maintaining tractable SHAP computation. Finally, when exact computation is intractable (Section \ref{sec:intractable}), investigating the approximability of SHAP metrics through approximation and parameterized complexity theory~\citep{downey2012parameterized} is an important direction.

%Our work opens several promising avenues for future research on the computational properties of explainable AI methods, with a particular focus on SHAP. First, it would be interesting to broaden the computational analysis conducted in this work to include other popular SHAP-related metrics in the literature, such as asymmetric SHAP \cite{frye20} and SAGE \cite{covert2020understanding}. Also, in the future, we aim to explore more expressive distribution classes and relaxed distributional assumptions—extending beyond those examined in Section \ref{sec:tractable} —that still yield tractable SHAP computation. Finally, when exact computation proves intractable (Section \ref{sec:intractable}), it is worthwhile to theoretically investigate the question of the approximability of computing the SHAP metrics across various configurations, through the lens of approximation and parametrized complexity theory \cite{arora2009computational}.

%This paper aims to deepen our understanding of the computational complexity involved in obtaining different Shapley value variants. We found that for a variety of ML models, including decision trees, tree ensembles for regression, weighted automata, and linear regression models — computing both local and global interventional and baseline SHAP can be done in polynomial time when distributions are modeled by HMMs. This extends the distributional scope of popular algorithms like TreeSHAP, which is limited to empirical distributions. Additionally, we demonstrate a strict complexity gap between SHAP variants, showing that interventional and baseline SHAP can be strictly easier to compute than conditional SHAP. Despite these positive results, we uncovered intractability for various SHAP variants in neural networks and tree ensembles. Finally, we provided generalized complexity relations across SHAP variants. We believe that our framework offers a deeper understanding of the complexity involved in computing SHAP across various variants, models, distributions, as well as in both local and global computations, laying the groundwork for future research.


%%
%% The acknowledgments section is defined using the "acks" environment
%% (and NOT an unnumbered section). This ensures the proper
%% identification of the section in the article metadata, and the
%% consistent spelling of the heading.
%\begin{acks}
%To Robert, for the bagels and explaining CMYK and color spaces.
%\end{acks}

%%
%% The next two lines define the bibliography style to be used, and
%% the bibliography file.
\bibliographystyle{ACM-Reference-Format}
%\bibliography{sample-base}
\bibliography{references-new}


%%
%% If your work has an appendix, this is the place to put it.
%\appendix

%\section{Research Methods}



\end{document}
\endinput
%%
%% End of file `sample-sigconf.tex'.
