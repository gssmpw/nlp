\begin{abstract}
  % oversmoothing
  % revisit the concept of signed graphs and show
  Oversmoothing is a common issue in graph neural networks (GNNs), where node representations become excessively homogeneous as the number of layers increases, resulting in degraded performance. 
  Various strategies have been proposed to combat oversmoothing in practice, yet they are based on different heuristics and lack a unified understanding of their inherent mechanisms. 
In this paper, we  
show that three major classes of anti-oversmoothing techniques can be mathematically interpreted as message-passing over signed graphs comprising both positive and negative edges.
By analyzing the asymptotic behavior of signed graph propagation, we demonstrate that negative edges can repel nodes to a certain extent, providing deeper insights into how these methods mitigate oversmoothing. Furthermore, our results suggest that the structural balance of a signed graph---where positive edges exist only within clusters and negative edges appear only between clusters---is crucial for clustering node representations in the long term through signed graph propagation.
% Furthermore, we reveal that negative edges can repel nodes and thereby revel these methods uniformly alleviate oversmoothing to a certain extent by analyzing the asymptotic behaviors of the generic signed graph propagation. 
% , providing a unified understanding of their inherent mechanisms.
% Moreover, we investigate
% Inspired by this observation, we further analyze the asymptotic behaviors of the generic signed graph propagation and find that negative edges can repel nodes and thereby alleviate oversmoothing to a certain extent. 
% 
% though arbitrary signed propagation does not gurrenteend to resolve oversmoothing in the long term.
% while negative signs can partially disperse nodes, 
% However, 
Motivated by these observations, we propose a solution to mitigate oversmoothing with theoretical guarantees---\oursfull (\ours), by incorporating label and feature information to create a structurally balanced graph for message-passing. 
% We present our method, \oursfull (\ours), 
% By incorporating label and feature information to the adjacency matrix, 
% \ours implicitly clusters the intra-label nodes by the positive edges and repels inter-label nodes by the negative edges. 
Experiments on nine datasets against twelve baselines demonstrate the effectiveness of our method, highlighting the value of our signed graph perspective.
% theoretically and empirically enhance structural balance to alleviate oversmoothing under specific conditions.
% demonstrate the effectiveness of our methods, highlighting the value of our signed graph explanation framework.
% the propagation of any arbitrary signed graph does not definitively address oversmoothing, providing insights into the limitations of these approaches. Leveraging the classic theory of structurally balanced graphs, we characterize their asymptotic behaviors and prove that the ideal structurally balanced signed graph provably prevents oversmoothing.
  % Leveraging the classic theory of signed graphs, we characterize the asymptotic behaviors of existing methods and reveal that they deviate from the ideal state of structural balance that provably prevents oversmoothing.  
  % and improves node classification performance. 
    % Driven by this unified analysis and theoretical insights, we propose \oursfull (\ours) where we explicitly enhance the structural balance of the signed graph with the help of label and feature information. 
  % Experiments on synthetic and real-world datasets demonstrate the effectiveness of our methods, highlighting the value of our signed graph framework.
  % Oversmoothing is a common phenomenon in a wide range of graph neural networks (GNNs), where model performance worsens as the number of layers increases.
  % % previous methods
  % Various strategies are proposed to combat oversmoothing, \jq{one simple and effective class is normalization methods}, yet based on different heuristics and lack a unified understanding of their inherent mechanisms. 
  % % signed graph
  % % In this paper, we propose a signed graph framework by revisiting and utilizing the concept and theory of \textit{signed graphs}.
  % In this paper, we revisit the concept of \textit{signed graphs} and show that 
  % % analysis results
  % % We observe that 
  % these \jq{normalization} classes of anti-oversmoothing techniques can be unified as the method-specific signed graph propagation.
  % \jq{todo}
  % Theoretically, we found that they deviate far from the ideal state of \emph{structural balance}.
  % with the adjacency matrix as the positive graph and the method-specific negative graph. 
  % Leveraging classic theories of signed graphs, 
  % We also theoretically characterize the asymptotic behaviors of existing methods and reveal that they deviate far from the ideal state of \emph{structural balance} that provably prevents oversmoothing and improves node classification performance. 
  % of signed graph,
  % graph where edge weights are negative.
  % a signed graph propagation, with the same positive graph but variations of negative graphs, critically influencing the convergence of repeated propagation. 
  % from the signed graph perspective
  % Driven by the surprising unify, we dive into the 
  % asymptotic behaviors of 
  % Driven by this unified analysis, we propose 
  % \textit{structurally balanced} theory and design a powerful method: 
  % \oursfull (\ours) where we explicitly enhance structural balance of the signed graph with the help of label and feature information.
  % by designing the label-enhanced negative graph.
  % ---to propose two powerful strategies: \textbf{Label}-enhanced \oursfull (\ours-Label) and \textbf{Feature}-enhanced \oursfull (\ours-Feature) by designing the negative graph.
  % theory
  % We theoretically and empirically prove that \ours can improve the structural balance to alleviate oversmoothing under certain conditions.
  % and empirically observe that \ours can .
  % experiments
  % Experiments on various synthetic and real-world datasets demonstrate the effectiveness of our methods, highlighting the value of our signed graph framework. 
\end{abstract}