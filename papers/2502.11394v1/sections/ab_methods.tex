% \section{Theoretical Characterization via Signed Graph Theory}
% \label{sec: main results}

% In Section~\ref{sec: signed pers}, our formulation explains why all these methods can help alleviate oversmoothing from a unified perspective: they introduce repulsion among nodes in message-passing by implicitly injecting a negative graph.
% Nevertheless, we will show in this section, not all instances of negative graph-injected propagation could effectively address the issue of oversmoothing.

% \subsection{Structural Balance}
% In this paper, we propose the optimal theoretical solution to neither covergence nor divergence of node features---\oursfull (\ours) based on the structural balance property. 
% Formally, following~\citep{signed_dynamics_paper_review,structuralbalance}, we first define a structural balance graph as follows. 

% \begin{definition}[Structural Balance]
%     A signed graph \( \mathcal{G} \) is \textbf{structurally balanced} if there is a partition of the node set into \( V = V_1 \cup V_2 \) with \( V_1 \) and \( V_2 \) being nonempty and mutually disjoint, where any edge between the two node subsets \( V_1 \) and \( V_2 \) is negative, and any edge within each \( V_i \) is positive.
% % \end{definition}
% % The notion of structural balance can be weakened in the following definition \ref{def: weak struct}.
% % \begin{definition}
%     % A signed graph \( G \) is \textbf{weakly structurally balanced} if there is a partition of \( V \) into \( V = V_1 \cup V_2 \cup \ldots \cup V_m \), \( m \geq 2 \) with \( V_1, \ldots, V_m \) being nonempty and mutually disjoint, where any edge between different \( V_i \)'s is negative, and any edge within each \( V_i \) is positive.
%     \label{def: weak struct}
% \end{definition}
% The structural balance property divides the graph into two disjoint groups (\( V_1 \) and \( V_2 \)) and separate intra-group and inter-group edges by their signs. 
% In a structurally balanced graph, the node sets $V_1$ and $V_2$ represent distinct groups. 
% Thanks to the distribution of signed edges in the structurally balanced signed graph, we can balance the convergence and divergence behaviors of node features.
% Within the same set there are only positive edges indicating attraction, while between the two different sets there are only negative edges indicating repulsion.
% The distinct property of structurally balanced graphs yields valuable insights into balancing the convergence and divergence behaviors of node features. 
% Moreover, 
% to constrain node features from diverging to $\infty$ in Theorem~\ref{thm: connected positive graph}, 
% to constrain node features from diverging, we introduce a bounded function $\mathcal{F}_c(z)$: 
% to constrain the value of node features to be within $[A,-A]$, where $A > 0$ is a constant and 
% \( \mathcal{P}_A(z) = -A, z < -A \), \( \mathcal{P}_A(z) = z, z \in [-A, A] \), and \( \mathcal{P}_A(z) = A, z > A \). 
% \begin{equation}
% \mathcal{F}_c(z)=\left\{
% \begin{aligned}
% -c,& \,  z < -c  \\
%  z,& \, z \in [-c, c] \\
%  c,& \, z > c 
% \end{aligned}
% \right.
% \end{equation}
% In practice, such an effect of $\mathcal{F}_c(z)$ can be achieved by normalization layers such as LayerNorm~\citep{layernorm}. 
% Then we prove that 

% \xinyic{ The function $\mathcal{P}_A(z)$ needs introdution: to constrain the node value from diverging (Theorem 3.2) to be within $[-A,A]$. In practice, such a effect of $\mathcal{P}_A(z)$ can be achieved by normalization layers such as LayerNorm...}
% \begin{theorem}
% % [\cite{signed_dynamics_paper_review}, Theorem 9]
% \label{thm: repel_struct} 
% Assume that node $i$ is connected to node $j$ and $x_i(t)$ represents the value of node $i$ after $t$ round of propagation. 
% Let $\theta=\alpha$ if the edge $\{i,j\}$ is positive and $\theta=-\beta$ if the edge $\{i,j\}$ is negative.
% Consider the constrained signed propagation update:
% \begin{equation}
% \label{eq: constrained repel dyn}
%     x_i(t + 1) = \mathcal{F}_c((1-\theta) x_i(t)+\theta x_j(t)).
% \end{equation}
% Let \(\alpha \in (0,1/2)\). 
% Assume that \(\mathcal{G}\) is a structurally balanced complete graph under the partition \(V = V_1 \cup V_2\). 
% When \(\beta\) is sufficiently large, for almost all initial values \(x(0)\) w.r.t. Lebesgue measure, there exists a binary random variable \(l(x(0))\) taking values in \(\{-c,c\}\) such that
% \begin{equation}
%     \mathbb{P}\left(\lim_{t \to \infty} x_i(t) = l(x(0)), i \in V_1; \lim_{t \to \infty} x_i(t) = -l(x(0)), i \in V_2 \right) = 1.
% \end{equation}
% \end{theorem}
% Elaborate meanings; 
% Theorem~\ref{thm: repel_struct} shows that convergence only occurs within each node group when the signed graph exhibits the structural balance property with sufficiently large $\beta$. Furthermore, different groups will repel each other to the extreme opposite boundaries, regardless of the depth of the layer.
% Such a convergence within individual groups holds promise for attaining optimal node classification by aggregating node features from the same label class while simultaneously repelling those from different label classes.
% The function $\mathcal{P}_A(z)$ is to constrain the node value from diverging to be $\infty$. 
% \jq{how to connect the metric value in discussion with the definition?}



% Finally, the two label classes assumption might be too strong to be satisfied in real scenarios.
% In that case, a more general notion---weakly structural balance---extending the number of distinct node sets from $2$ to $n$ could be introduced to address this issue. We discuss similar convergence behaviors of the weakly balanced graph in Appendix \ref{app:weak-balance}.


