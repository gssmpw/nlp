
\section{More Discussion on \oursfull}
\label{app: sbp}
The overall update of \oursfull is as following:
\begin{equation}
    X^{(k)} = \text{Layernorm}\{(1-\lambda)X^{(k-1)} + \lambda(\alpha \pgh{A^+} X^{(k-1)} - \beta \ngh{A^-} X^{(k-1)})\},
\end{equation}
Our methods adopt the normalized adjacency matrix as the positive graph
$\pgh{A^+=\hat{A}}$, while use different negative graphs.
Although both the positive and negative graphs have hyperparameters, we do not carefully adjust the hyperparameters. 
Instead, we fix $\alpha=1$ and only select the best value for $\beta$. 
You can also change $\alpha$ and $\beta$ together to achieve the best performance.
\paragraph{Label-Induced Negative Graph}
The negative graph for Label-\ours is: 
\begin{equation}
    \ngh{A^-}_{ij}=
    \begin{cases}
      1 & \text{if i,j has the different labels,} \\
      -1 & \text{if i,j has the same labels,} \\
      0 & \text{if i,j has the unknown labels.}
    \end{cases}
\end{equation}
For practice, we apply softmax to it: 
\begin{equation}
    \ngh{\tilde{A}^-}= \text{softmax}(\ngh{A^-}).
\end{equation}
Applying softmax makes the negative graph the row-stochastic which is a non-negative matrix with row sum equal to one.
We also tried the normalization method, which is not as good as the softmax. This may be because the softmax method is more aligned with the row-stochastic adjacency, where every element is non-negative. 

\paragraph{Similarity-Induced Negative Graph}
The negative graph for Feature-\ours is: 
\begin{equation}
    \ngh{A^-} = - X^{(0)} X^{(0)T}
\end{equation}
We also attempted using the last layer node features for the negative graph, but they are not as effective as the initial layer node features. 
This may be due to oversmoothing as the layers go deeper.
For practice, we apply softmax as the Label-\ours to it: 
\begin{equation}
    \ngh{\tilde{A}^-} = \text{softmax}(- X X^T)
\end{equation}


\section{\jq{Time Complexity Analysis and the Modified \ours}}
\label{app: time complexity of sbp}
\paragraph{Label-\ours}
As shown in~\eqref{eq: sbp}, we maintain the positive adjacency matrix $A^+=\hat{A}$ and construct the negative adjacency matrix $A_{l}$ by assigning 1 when nodes $i,j$ have different labels, -1 when they share the same label, and 0 when either label is unknown.
We then apply softmax to $A_{l}$ to normalize the negative adjacency matrix. The overall signed adjacency matrix is $A_{sign}= \alpha A^+ - \beta softmax(A_{l})$, where $\alpha$ and $\beta$ are hyperparameters.
Given $n_t$ training nodes and $d$ edges in the graph, our Label-SBP increases the edge count from $O(d)$ to $O(n_t^2)$, thereby increasing the computational complexity to $O(n_t^2d)$.

\paragraph{Feature-\ours}
When labels are unavailable, we propose Feature-SBP, which uses the similarity matrix of node features to create the negative adjacency matrix.
As depicted in~\eqref{eq: sbp}, we design the negative adjacency matrix as $A_{f}=-X_{0}X_{0}^T$. We then apply softmax to $A_{f}$ to normalize it. The overall matrix follows the same format as Label-SBP: $A_{sign}= \alpha A^+ - \beta softmax(A_{f})$, where $\alpha$ and $\beta$ are hyperparameters.
The additional computational complexity primarily stems from the negative graph propagation, which involves $X_{0}X_{0}^T \in \mathbb{R}^{n\times n}$, increasing the overall complexity to $O(n^2d)$.

We show the computation time of different methods in the Table~\ref{tab: time sbp}. On average, we improve performance on 8 out of 9 datasets (as shown in Table~\ref{table: sgc results}) with less than 0.05s overhead—even faster than three other baselines. 
We believe this time overhead is acceptable given the benefits it provides.

\begin{table}[htbp]
\centering
\caption{Estimated training time of \ours on Cora dataset. All experiments are run under 2 layers. s is the abbreviation for second. Precompute time is the aggregation time across layers, train time is the update time of the SGC weight $W$, total time is the sum of them.}
\label{tab: time sbp}
\resizebox{\linewidth}{!}{
\begin{tabular}{ccccccccc}
\hline
& Label-\ours & Feature-\ours & BatchNorm & ContraNorm & Residual & JKNET & DAGNN & SGC \\ \hline
Precompute time & 0.1809s & 0.1520s & 0.1860s & 0.1888s & 0.0604s & 0.0577s & 0.1438s & 0.1307s \\ 
Train time & 0.1071s & 0.1060s & 0.1076s & 0.1038s & 0.1368s & 0.1446s & 0.1348s & 0.1034s \\ 
Total time & 0.2879s & 0.2580s & 0.2935s & 0.2926s & 0.1972s & 0.2023s & 0.2786s & 0.2341s \\ 
Rank & 6 & 4 & 8 & 7 & 1 & 2 & 5 & 3 \\ \hline
\end{tabular}
}
\end{table}


\paragraph{Scalability of \ours on large-scale graph}
For large-scale graphs, we introduce a modified version Label-\ours-v2 by only removing edges when pairs of nodes belong to different classes.
This approach allows Label-\ours-v2 to eliminate the computational overhead of the negative graph, further enhancing the sparsity of large-scale graphs.
For Feature-\ours, as the number of nodes $n$ increases, the complexity of this matrix operation grows quadratically, i.e., $\mathcal{O}(n^2d)$.
To address this, we reorder the matrix multiplication from $-X_{0}X_{0}^T \in \mathbb{R}^{n\times n}$ to $-X_{0}^TX_{0} \in \mathbb{R}^{d\times d}$. This preserves the distinctiveness of node representations across the feature dimension, rather than across the node dimension as in the original node-level repulsion.
The modified version of Feature-\ours can be expressed as:
\begin{equation}
 (\text{Feature-\ours-v2})\,\,\,\,\,   X^{k}=(1-\lambda)X^{(k-1)}+\lambda(\alpha \hat{A}X^{(K)} - \beta X^{(K)} \text{softmax}(-X_{0}^TX_{0}))
\end{equation}
This transposed alternative has a linear complexity in the number of samples, i.e., $\mathcal{O}(nd^2)$, significantly reducing the computational burden in cases where $n \gg d$.

We compare the compute time \ours with other baselines on ogbn-arxiv dataset over 100 epochs for a fair comparison. 
Among all the training times of the baselines, our Label-\ours-v2 achieves the 3rd fastest time while Feature-\ours-v2 ranks 5th. Therefore, we recommend using Label-\ours-v2 for large-scale graphs since they typically have a sufficient number of node labels. We believe that although there is a slight time increase, it is acceptable given the benefits.
\begin{table}[htbp]
\centering
\caption{Estimated training time of \ours on ogbn-arixv dataset. All experiments are run under 2 layers and 100 epochs. s is the abbreviation for second.}
\label{tab:my-table}
\resizebox{0.9\linewidth}{!}{%
\begin{tabular}{ccccccc}
\hline
 & Label-\ours & Feature-\ours & BatchNorm & ContraNorm & DropEdge & SGC \\ \hline
Train time (s) & 5.5850 & 6.1333 & 5.3872 & 5.8375 & 9.5727 & 5.3097 \\ 
Rank & 3 & 5 & 2 & 4 & 6 & 1 \\ \hline
\end{tabular}%
}
\end{table}

\section{Details of Experiments}
\label{app: exp}
% \begin{wrapfigure}{r}{.75\textwidth}

% \end{wrapfigure}

% \begin{table}[h]
\centering
% \vspace{-0.15in}
\caption{CSBM test accuracy (\%) comparison results. The best results are marked in blue on each layer. The second best results are marked in gray on each layer. We run 10 runs for the seed from $0-9$ and demonstrate the mean $\pm$ std in the table.}
\begin{adjustbox}{width=0.99\textwidth}
\begin{tabular}{lccccccc}
\toprule
 Model             & \#L=2              & \#L=5              & \#L=10             & \#L=20             & \#L=50        & \#L=100    & \#L=200    \\
\midrule
SGC & 73.25 {\footnotesize $\pm$ 6.90} & 44.50 {\footnotesize $\pm$ 9.34} & 45.75 {\footnotesize $\pm$ 9.36} & 45.75 {\footnotesize $\pm$ 9.36} & 45.75 {\footnotesize $\pm$ 9.36} & 45.75 {\footnotesize $\pm$ 9.36} & 45.75 {\footnotesize $\pm$ 9.36} \\
Feature-\ourst &\cellcolor{secondbest}48.75 {\footnotesize $\pm$ 5.62} &\cellcolor{secondbest} 53.75 {\footnotesize $\pm$ 6.45} & \cellcolor{secondbest}63.75 {\footnotesize $\pm$ 6.25} & \cellcolor{secondbest}77.00 {\footnotesize $\pm$ 5.45} &\cellcolor{secondbest} 82.00 {\footnotesize $\pm$ 4.58} &\cellcolor{secondbest} 82.50 {\footnotesize $\pm$ 5.12} &\cellcolor{secondbest} 82.00 {\footnotesize $\pm$ 5.45} \\
Label-\ourst & \cellcolor{best}85.75 {\footnotesize $\pm$ 4.04} & \cellcolor{best}93.50 {\footnotesize $\pm$ 4.06} & \cellcolor{best}93.50 {\footnotesize $\pm$ 3.57} & \cellcolor{best}93.50 {\footnotesize $\pm$ 3.57} & \cellcolor{best}92.25 {\footnotesize $\pm$ 3.44} & \cellcolor{best}93.25 {\footnotesize $\pm$ 3.72} & \cellcolor{best}91.25 {\footnotesize $\pm$ 6.05} \\


\bottomrule
\end{tabular}
\end{adjustbox}
\label{table: app_sbm_results}
\end{table}
% \vspace{-0.1in}
The code for the experiments will be available when our paper is acceptable.
% \begin{quote}
% \centering
%     https://xxxxxxxx
% \end{quote}
We will replace this anonymous link with a non-anonymous GitHub link after the acceptance. 
We implement all experiments in Python 3.9 with PyTorch Geometric on one NVIDIA Tesla V100 GPU.

\subsection{Details of the Dataset}
\label{app: data}

\begin{table}[h]
% \vspace{-0.1cm}
\caption{Summary of datasets. $H(G)$ refers to the edge homophily level: the higher, the more homophilic the dataset is.}
\centering
\resizebox{0.6\linewidth}{!}{
\begin{tabular}{lcccc}
\toprule
Dataset    & $H(G)$        & Classes             & Nodes &Edges \\
% & Train/Dev/Test Nodes\\
\midrule
\textbf{Cora} & 0.81 & 7 & 2,708 &  5,429 \\ %&140/500/1,000\\
\textbf{Citeseer} & 0.74 & 6 &3,327  &4,732 \\ % & 120/500/1,000 \\
\textbf{PubMed} & 0.80 & 3 & 19,717 & 44,338 \\ %& 60/500/1,000\\

\midrule
\textbf{Texas} & 0.21 & 5 & 183 & 295 \\
\textbf{Cornell} & 0.30 & 5 & 183 & 280\\
\textbf{Amazon-ratings} & 0.38 & 5 & 24,492 & 93,050 \\
\textbf{Wisconsin} & 0.11 &5 & 251 & 466\\
% \textbf{Chameleon}& 0.23 & 6 & 2,277 & 31,421 & 1,092/729/456\\
\textbf{Squirrel} & 0.22 & 4 & 198,493 & 2,089 \\ %& 2,596/1,664/1,041\\
\midrule
\textbf{Ogbn-Arxiv} & 0.65 & 40 & 16,9343 & 1,166,243 \\ %  & 90,941/29,799/48,603\\
\bottomrule
\end{tabular}
}
% \end{adjustbox}
\label{tab: main_data}
% }
% \vspace{-1cm}
\end{table}
We consider two types of datasets: Homophilic and Heterophilic. 
They are differentiated by the \emph{homophily level} of a graph.
$$
\mathcal{H}
=
\frac{1}{|V|} \sum_{v \in V} \frac{\hbox{ Number of } v\hbox{'s neighbors who have the same label as } v}{\hbox{ Number of } v\hbox{'s neighbors }}.
$$
The low homophily level means that the dataset is more heterophilic when most of the neighbors are not in the same class, and the high homophily level indicates that the dataset is close to homophilic when similar nodes tend to be connected. 
In the experiments, we use four homophilic datasets, including Cora, CiteSeer, PubMed, and Ogbn-Arxiv, and four heterophilic datasets, including Texas, Wisconsin, Cornell, Squirrel, and
Amazon-rating~\citep{platonov2023critical}).
% Moreover, we list the numbers of classes, nodes, edges, the splits of each dataset, and their homophily level in Table~\ref{tab: data}. 
The datasets we used covers various homophily levels.

% \section{Dataset Generation}
\label{sec:dataset}
\revise{
To train the proposed GNN, we constructed a dataset of building structures and a subset of these structures were subjected to fire simulations using FEA. The dataset generation process is illustrated in \figref{fig:dataset_generation_procedure}. Initially, a total of 33,000 building structures with geometrical details, material properties, and gravity loads were created. Due to randomness in generating these structures, a filter is applied to remove unreasonable data after gravity load simulation, which included 15,377 structures. A trade-off between computational feasibility and model performance is made among the remaining 17,623 structures. As further labeling structures with MIDR requires resource-intensive fire simulations via OpenSeesRT, a large proportion of 16,050 structures is selected as unlabeled dataset. On the other hand, each of the other 1,573 structures was further subjected to 30 different fire simulations, forming the labeled dataset containing $1,573\times 30 = 47,190$ fire cases.} This section details the step-by-step process for generating the dataset, including geometry creation, material property assignment, and simulations due to gravity loads and fire scenarios. 
% To train the proposed neural network, we constructed a dataset comprising building structure data and a subset of fire scenario data. The dataset generation process is illustrated in \figref{fig:dataset_generation_procedure}. 
% A total of 33,000 building structures with geometric details, material properties, and gravity loads were initially created. Out of these, 3,000 structures were selected as labeled data, and the remaining 30,000 were designated as unlabeled data. Further, about half of them filtered out due to instability under gravity loads only. 
\begin{figure*}[h!]
    \centering
    \includegraphics[width=0.8\linewidth]{figures/dataset_filter_procedure.pdf}
    \caption{Workflow for dataset generation (geometry, material property, gravity loads, and fire scenarios).}
    \label{fig:dataset_generation_procedure}
\end{figure*}

\subsection{Geometry Generation}
\label{subsec:geometry_generation}
The geometry of the building structures forms the foundation of the dataset. Regular 
\revise{3D structures} resembling multi-story parking structures or shopping malls were generated, with parameters such as building floor dimensions and story heights selected randomly. Each building structure is composed of multiple rooms, which serve as the basic unit in this study. A room herein is a cuboid space defined by specific length, width, and height. Within a structure, rooms of the same dimensions are uniformly arranged along the length, width, and height, corresponding to the $x$-, $y$-, and $z$-axes, respectively. Structures vary in room size and number of rooms along each axis. Specifically, the room length, width, and height are independently sampled from a uniform distribution within the interval $[2, 5]$ meters along the three directions of the structure. Similarly, the room number along each axis is uniformly sampled independently as an integer within the interval $[2, 7]$, i.e., the maximum number of stories of the buildings simulated in this study is 7.

To introduce variability and simulate real-world scenarios, approximately $8\%$ of structural elements (beams or columns) are randomly removed after initial geometry creation. 
\revise{Such removal is not fire-induced damage, but reflects functional diversity often observed in real buildings, such as open spaces designed for activities in shopping malls, e.g., ice skating rinks. Examples of the generated geometries are illustrated in \figref{fig:example_generated_geometry}, showcasing the diversity and realism of the dataset. This element removal does not affect the definition of room's geometry in the structure and nor does it affect the number of considered fire scenarios.} 

\revise{A range of coefficient of variation values ($3.3\%$ to $17.5\%$) was derived from prior studies that investigated the statistics of geometrical and material properties of structural components of buildings (e.g., \cite{mirza1979variations, lee2004probabilistic}). These studies provide empirical data on the natural variability in parameters such as Young's modulus, yield strength, and dimensions of structural elements due to manufacturing tolerances and material inconsistencies. By selecting $8\%$ for the removal of structural elements in our database, we aimed to maintain a level of variability that is representative of real-world uncertainties while ensuring computational feasibility. This choice ensures that the database captures realistic deviations without introducing extreme cases that may not be commonly encountered in practice.}

\begin{figure*}[h!]
    \centering
    \includegraphics[width=\linewidth]{figures/example_generated_geometry.pdf}
    \caption{Examples of generated structural geometry of different sizes (all dimensions in meters).}
    \label{fig:example_generated_geometry} 
\end{figure*}

{\blockRevise

In this study, we opted for a deterministic square, dimension of $0.1$ m, solid cross-sectional steel elements due to their simplicity in modeling and analysis. Square sections exhibit uniform geometrical properties in all directions, simplifying the computation of structural responses and avoiding complications associated with more complex shapes, such as wide-flange sections, facilitating the computational efficiency and scalability to generate a large dataset. This choice also helps to mitigate issues related to stress concentrations and facilitates a more straightforward representation of structural behavior under thermal loads. 

\textit{Remark:} The selected cross-section provides a comparable flexural rigidity to a $W 130 \times 130 \times 28.1$ wide-flange section (metric units), albeit with significantly higher axial rigidity. This cross-section is acceptable for gravity-load-designed frames under service loading conditions where the models assume fully rigid, moment-resisting beam-column connections for the evaluation of the IDR under thermal loading. This assumption is reasonable in this computational study where the primary interest is to understand the global deformation response of frames under fire conditions. The selection of uniform square cross-sections for both beams and columns, rather than adherence to standard capacity design principles, was made here primarily for computational efficiency and to reduce design parameters in the database generation process. This choice allows for simplified and scalable approach to analyze the fire-induced response of generic steel frames without the need for large section variations, where this study mainly focuses on the fire vulnerability assessment using ML-based predictions. However, if additional loading conditions, e.g., seismic or wind loads, were to be considered, larger sections, strong-column/weak-beam principle, and ductile detailing would be required in the generated buildings for realistic structural behavior under combined loading conditions. Future studies may also consider investigating the influence of variable cross-sectional dimensions and semi-rigid connections on the structural performance under fire conditions. 
} % blockRevise

\subsection{Material Properties}
Steel is chosen as the material for the structures. To reflect real-world variations, we randomly assign one of five slightly different steel material types to each structural element. \revise{
The ranges of material properties are provided in \tabref{tab:material_property_ranges} and the properties are sampled from uniform distributions of the corresponding ranges. These variations simulate differences arising from manufacturing batches or regional material properties. That these properties are at ambient temperature and change when the temperature rises due to a fire. The selection of materials with varying properties is aimed at increasing the diversity of the data. Our goal is to represent as wide a range of data as possible with a limited amount of building structure data, thereby enhancing the generalization ability of the GNN. Our assumed material property ranges are expected to be wider than the real-world conditions based on findings in \cite{mirza1979variations, lee2004probabilistic}. Therefore, we are essentially tackling a more challenging and general task. If we can solve this problem, we are confident that our method will perform equally well or even better in real-world scenarios.
}
\begin{table}[h!]
    \centering
    \caption{Material properties ranges for considered steel structures.}
    \begin{tabular}{lc}
        \toprule
        Property & Range \\
        \midrule
        Young's modulus & [168, 252] GPa \\
        Yield strength & [220, 330] MPa \\
        Strain-hardening ratio & [0.8, 1.2] \% \\
        \bottomrule
    \end{tabular}
    \label{tab:material_property_ranges}
\end{table}

\subsection{Gravity Loads}
Gravity loads are applied to columns and beams based on their \revise{influence (tributary) areas as typically conducted in structural analysis. The considered ``service'' load conditions include the column self-weight and the additional loads directly supported on the beams from their self-weight and weights of the reinforced concrete slabs, people as live load, and building content. An edge beam typically carries approximately half the gravity load supported by a parallel interior beam}. The ranges of gravity loads are listed in \tabref{tab:gravity_load_ranges}. \revise{The loads are sampled from uniform distributions of the corresponding ranges.} Structures that failed to meet an MIDR threshold of $1\%$ under gravity loads were deemed unacceptable designs and filtered out, as such configurations of randomly chosen geometry, material, and gravity load combinations were considered unrealistic from a regulatory and practicality points of view.
\begin{table}[h!]
    \centering
    \caption{Gravity load ranges for considered beams and columns.}
    \begin{tabular}{lc}
        \toprule
        Element & Range (kN/m)  \\
        \midrule
        Column & [0.5, 1.0]  \\
        Edge beam & [1.5, 4.5]  \\
        Interior beam & [3.0, 7.5]  \\
        \bottomrule
    \end{tabular}
    \label{tab:gravity_load_ranges}
\end{table} 

\subsection{Rule-based Thermal Load Generation}
\label{subsec:thermal_load_generation}
To evaluate a building's structural response during a fire event, we employed a simplified rule-based approach for thermal load generation. 
% Previous studies \cite{nan_structuralfire_2023} have demonstrated that steel structures rapidly equilibrate with surrounding gases temperatures due to efficient heat exchange. Consequently, gas temperatures can be directly used as inputs for FEA tools, e.g., OpenSees, simplifying the process of modeling thermal loads. 
% Accurately simulating temperature fields in fire scenarios poses significant challenges. Advanced thermodynamic simulations, such as those performed using Fire Dynamics Simulator (FDS) \cite{mcgrattan_fire_2000}, provide precise temperature predictions. However, these methods are hindered by high computational costs, prolonging execution times, and limited scalability, making them impractical for generating large datasets. Additionally, real-world fire loads often display substantial spatial variability across different rooms \cite{dundar_fire_2023}, resulting in scenario-specific temperature fields with limited generalizability. For example, studies on bridge fires \cite{he_study_2024} have demonstrated that environmental factors, such as wind speeds, can significantly influence temperature distributions. Furthermore, even within identical scenarios, variations in fire modeling methodologies can produce distinctly different temperature fields \cite{zhang_temperature_2020, du_new_2012}. These challenges emphasize the need for efficient and adaptable methods to generate fire temperature data.
% To address these issues, we adopted a rule-based approach to model temperature variations. 
According to \cite{spearpoint_fire_2008}, a typical fire development follows a predictable pattern. During the {\em{growth stage}}, the temperature rises slowly and approximately linearly after ignition. This is followed by the {\em{flashover stage}}, where temperatures increase rapidly to peak values. After reaching the peak, the temperature either stabilizes or continues to rise slowly until the {\em{decay stage}} begins. Inspired by this fire development pattern, we describe the temperature evolution in time, $t$, prior to the decay stage in two distinct stages:
\begin{enumerate}
    \item {\bf{Initial linear increase stage}}: For $t \in [0, t_1)$, temperature increases gradually and linearly as the fire spreads through the building. This stage represents the time before the fire directly affects a structural element.  
    \item {\bf{ISO 834 fire curve stage}}: For $t \in [t_1, t_{\thre}]$, temperature rises rapidly following the ISO 834 curve \cite{ISO834}, modeling the direct impact of the fire on the structural element. 
\end{enumerate}
The slope of the linear temperature increase, $c$, and the transition time, $t_1$, are influenced by the spatial relationship between the fire source and the structural element. For the second stage of temperature evolution, we utilize the ISO 834 curve, a widely accepted standard for fire resistance testing. This standardized fire curve describes the temperature rise over time, enabling rapid and consistent thermal fields across various scenarios. The duration of fire simulation in this study is set to $t_{\thre}=60$ minutes. This value represents the upper limit for the temperature evolution of each structural element, providing a consistent basis for analyzing the structural response to fire.

Let $(x, y, z)$ represents the midpoint of a structural element and $(x_{\subfire}, y_{\subfire}, z_{\subfire})$ the fire source point. \revise{Integer parameters $h$ and $h_{\subfire}$ correspond to the respective floor levels of the element and the fire source}. The temperature evolution for each element is expressed as follows:
\begin{enumerate}
    \item Linear increase stage ($0 < t < t_1$):
    \begin{equation}
    T(t) = c \cdot t,
    \end{equation}
    where $c$, the rate of temperature increase ($^\circ\mathrm{C}/\mathrm{min}$), depends on the height difference between the element, $h$, and the fire source, $h_{\subfire}$:
    \begin{equation}
        c = 
        \begin{cases} 
        5\left/\left(h - h_{\subfire} + 1\right)\right., & h \geq h_{\subfire}, \\
        2\left/\left(h_{\subfire} - h\right)\right., & h < h_{\subfire}.
        \end{cases}
    \end{equation}
     \item ISO 834 stage ($t \geq t_1$):
\begin{equation}
    T(t) = c \cdot t_1 + 345 \log_{10} \left(8 \left(t - t_1\right) + 1\right).
\end{equation}
\end{enumerate}

The transition (arrival) time $t_1$, marking the end of the linear stage, depends on the spatial distance between the fire source and the element. We define the following two Euclidean distances $L_p$ in the $xy$ plane and $L_s$ in the $xyz$ space:
\begin{eqnarray}
L_p & \triangleq & \sqrt{(x - x_{\subfire})^2 + (y - y_{\subfire})^2}, \\
\label{eq:Lp}
L_s & \triangleq & \sqrt{(x - x_{\subfire})^2 + (y - y_{\subfire})^2 + (z - z_{\subfire})^2}.
\label{eq:Ls}
\end{eqnarray}
Accordingly, the transition time, $t_1$, is expressed as follows:
\begin{equation}
    t_1 = 
    \begin{cases}
    \beta_{1} \cdot \left(1 - \exp\left\{- L_s\left/\alpha_{1}\right.\right\}\right), & h > h_{\subfire}, \\
    \beta_{2} \cdot \left(1 - \exp\left\{- L_p\left/\alpha_{2}\right.\right\}\right), & h = h_{\subfire}, \\
    \beta_{3} \cdot \left(1 - \exp\left\{- L_s\left/\alpha_{3}\right.\right\}\right), & h < h_{\subfire} .
    \end{cases}
    \label{eq:t1}
\end{equation}
The parameters $\beta_i$ and $\alpha_i$ for determining $t_1$ are summarized in Table~\ref{tab:fire_spread_parameters}. In this study, we take $r_{\mathrm{up}}=0.95$ and $r_{\mathrm{down}}=0.97$.
\begin{table}[ht]
    \centering
    \caption{Fire spread parameters for $t_1$ calculations.}
    \begin{tabular}{lcc}
        \toprule
        Case  & $\beta_i$ & $\alpha_i$  \\
        \midrule
        $i=1$, Upward spread & $16 \left.\left(1-r_{\mathrm{up}}^{\left|h-h_{\subfire}\right|}\right)\right/\left(1-r_{\mathrm{up}}\right)$ & $10$  \\
        $i=2$, Horizontal spread & $18$ & $18$  \\
        $i=3$, Downward spread & $30 \left.\left(1-r_{\mathrm{down}}^{\left|h-h_{\subfire}\right|}\right)\right/\left(1-r_{\mathrm{down}}\right)$ & $5$  \\
        \bottomrule
    \end{tabular}
    \label{tab:fire_spread_parameters}
\end{table}

\figref{fig:t1_curve} illustrates the $t_1$ curves for various fire scenarios: (1) fire originating on the lower floor, $h-h_{\subfire}=1$ with rapid upward spread, (2) fire on the same floor, $h=h_{\subfire}$ with the fastest spread, and (3) fire on the upper floor, $h_{\subfire}-h=1$ with slow downward spread. The exponential decay in $t_1$ reflects the accelerating fire propagation speed as the distance increases. \figref{fig:t1_curve} also indicates that the employed simplified model is consistent with the Markov chain-based dynamic model given by \cite{cheng_dynamic_2011}, where the rooms at the same floor of the fire point start flashover slightly before the corresponding upper floors. Additionally, $\beta_{1}$ and $\beta_{3}$ are the summation of a geometric sequence, where story level $h$ is the index. The common ratios $r_{\mathrm{up}}<1$ in $\beta_{1}$ and $r_{\mathrm{down}}<1$ in $\beta_{3}$ indicate that the fire speeds up to spread through the next story, which is consistent with the real-world fire spread mechanism given in \cite{hokugo_mechanism_2000}. The temperature profile within the range $t \in [0, t_{\thre}]$ is subsequently used as the thermal load in OpenSeesRT simulations to compute displacements at each structural node at time $t_{\thre}$.
\begin{figure}[h!]
    \centering
    \includegraphics[width=0.8\linewidth]{figures/m204_t1_curve.pdf}
    \caption{Three examples for the $t_1$ curve.}
    \label{fig:t1_curve}
\end{figure}

\revise{
\textit{Remark:} The effects of structural elements, such as concrete floor slabs and partitions, are not explicitly modeled in our approach. Instead, their influence is implicitly captured through the careful selection of the parameters $ \alpha, \beta, r_\mathrm{up} $, and $ r_\mathrm{down} $. This parameterization provides a unified framework for generating temperature fields. Indeed, fire propagation is governed by a multitude of factors and remains an open research question. For instance, if the fire resistance of a floor slab is enhanced by fire protective coating, the corresponding model can account for this by decreasing $\alpha_1$ \& $\alpha_3$, increasing $\beta_1$ \& $\beta_3$, and adopting larger values for $r_\mathrm{up}$ \& $r_\mathrm{down}$, which collectively slow down the vertical spread of fire. Conversely, scenarios involving higher amounts of combustible materials would warrant the opposite adjustments. This flexible and integrated approach avoids the need to design separate models for different fire propagation scenarios while still capturing the essential effects.
}

\revise{
In conclusion, our rule-based approach is a computationally efficient method for approximating fire temperature fields, enabling large-scale dataset generation to train predictive models. By combining ISO 834 fire curves with spatial considerations and embedding structural effects through parameter calibration, the method achieves a balanced trade-off between accuracy and scalability, making it a practical solution for thermal load modeling in fire scenarios. After generating the temperature of each beam or column according to the middle point, the temperature is applied as uniform thermal load to the elements of the structure in question using OpenSeesRT. 
}

% In conclusion, this rule-based approach is a computationally efficient method to approximate fire temperature fields, enabling large-scale dataset generation to train predictive models. By combining ISO 834 fire curves with spatial considerations, the method balances accuracy and scalability, making it a practical solution for thermal load modeling in fire scenarios.

% \subsection{Interstory Drift Ratio}
\subsection{OpenSeesRT Simulation}
\label{subsec:opensees_simulation}

The thermal and mechanical responses of 3D frame structures under combined fire and gravity loads are simulated using OpenSeesRT \cite{perez2024openseesrt}. \revise{In the simulation, the IDR of each node at $t_{\thre}$ is computed using the computed nodal displacements. Each structural model features six degrees of freedom per node (3 translational  and 3 rotational), with linear geometrical transformations (\texttt{geomTransf: Linear}) defining how the element local coordinate systems are mapped to the global coordinate system and assuming small displacements and rotations. Although OpenSeesRT allows a variety of options for modeling finite deformations, in the present simulations and mainly for simplicity, we did not consider large deformations. All bottom nodes (nodes on the ground) are fully constrained in all six degrees of freedom, while degrees of freedom os all other nodes are free.} Material behavior is temperature-dependent and modeled with \texttt{Steel01Thermal}, while fiber-based sections (\texttt{FiberThermal}) capture nonlinear interactions between thermal and mechanical responses at the cross-section level. \revise{Structural elements are represented as displacement-based Euler-Bernoulli beam-columns (\texttt{dispBeamColumnThermal}). This element  formulation accounts for thermal strains (temperature gradients) in the section, which is discretized into fibers. Numerical integration is used along the length of each element using three integration (Gauss) points, one at each end and the third in the middle of the element.}

{\revise{Thermal expansion of steel members plays a crucial role in IDR development. In reality, reinforced concrete floor slabs heat at a different rate than steel members due to their higher thermal mass and lower thermal conductivity. This differential heating can lead to restrained thermal expansion, introducing axial compression in beams and affecting the overall structural response. In this study, explicit {\em{composite action}} between steel members and concrete slabs is not modeled. Instead, our approach focuses on isolating the response of the steel structural frame, which is often the critical load-bearing component in fire scenarios. This assumption aligns with prior studies \cite{Possidente_2024} demonstrating that steel structures reach thermal equilibrium with surrounding gases quickly, allowing the use of uniform thermal loading in fire analysis. Future work could enhance this framework by incorporating slab-beam interaction effects, through a refined FEA for an extended dataset where constraints imposed by floor slabs are explicitly considered.}

The analysis begins with the application of gravity loads, followed by incremental thermal loads simulating the fire exposure. A static nonlinear solver using  \texttt{ExpressNewton} algorithm ensures convergence, while the \texttt{NormDispIncr} test maintains accuracy. An incremental \texttt{LoadControl} scheme with small step sizes is employed to guarantee numerical stability, using 10\% for gravity loads and 1\% for thermal loads. 

\revise{
In the thermal load analysis, uniform thermal load is applied to each beam or column, i.e., the temperature of each element is set to be that at the middle point, according to \secref{subsec:thermal_load_generation}. The \texttt{Steel01Thermal} material allows the properties (e.g., Young's modulus and yield strength) to be adjusted at increasing temperatures according to \cite{EN1993} using its Table 3.1: Reduction factors for the stress-strain relationship of carbon steel at elevated temperatures. For example, if the Young’s modulus at ambient temperature is $E_0$, then as the temperature ($T$) increases, the modulus changes as $E(T) = \eta (T) \times E_0$. \cite{EN1993} directly provides the values of $\eta(T) \in \left[0,1\right] $ at every $100 ^\circ\mathrm{C}$ interval and recommends using linear interpolation to obtain $\eta(T)$ for intermediate values of $T$.
} OpenSeesRT documentation \cite{OpenSeesThermalExamples} provides several examples of thermal analyses.

This modeling framework accommodates variations in material properties, cross-sectional geometries, and temperature profiles, providing robust simulations of structural behavior under fire conditions. The primary settings and configurations for the OpenSeesRT simulations are summarized in \tabref{tab:ops_detail}.
\begin{table}[h!]
    \centering
        \caption{Key settings of OpenSeesRT simulations.}
    \begin{tabular}{l|>{\raggedright\arraybackslash}p{0.6\linewidth}} %
    \toprule
    Modeling Aspect     & Details \\
    \midrule
    Geometry            & 3D models; 6 degrees of freedom per node \\
    Transformation      & geomTransf: Linear \\ 
    Material            & Steel01Thermal \\
    Section             & FiberThermal; Cross-section: $0.1$ m $\times$ $0.1$ m \\ 
    Element type        & {dispBeamColumnThermal} \\ 
    Loading             & Gravity loads: {beamUniform}; Thermal loads: {beamThermal} \\
    Integration scheme  & Incremental {LoadControl}; Step size: $10\%$ (gravity analysis), $1\%$ (thermal analysis) \\
    Nonlinear solver    & {ExpressNewton} algorithm; {UmfPack} solver; Convergence test: {NormDispIncr} tolerance: $10^{-8}$; Maximum \# iterations per step: $1000$. \\ 
    \bottomrule
    \end{tabular}
    \label{tab:ops_detail}
\end{table}

For each structure in the labeled dataset, 30 fire points are selected using a dual-granularity approach, \revise{i.e., two-stage sampling strategy,} to ensure they are well-distributed. Specifically, rooms are sequentially selected, with one fire point randomly chosen within each selected room. If a building is large and contains more than 30 rooms, we randomly select 30 rooms without replacement, i.e., ensuring that no more than one fire point is located in the same room. Conversely, if the building is small and has fewer than 30 rooms, all rooms are initially selected, with one fire point randomly assigned to each room. Additionally, rooms are then selected with replacement until a total of 30 fire points are assigned. \revise{The room-level sampling prioritizes selecting distinct rooms to avoid spatial clustering of fire points, while the point-level sampling ensures intra-room variability. This approach aligns with stratified sampling principles commonly used for efficient spatial representation, where multi-stage sampling strategies optimize coverage and variability, e.g., \cite{arunachalam_generalized_2023}, and enables a more comprehensive characterizing of how the structures respond under fire conditions.}
% This selection method prevents fire points from clustering too closely while maintaining an element of randomness. By distributing fire points in this manner, the 30 fire scenarios are effectively utilized, enabling a more comprehensive characterizing of how the structures respond under fire conditions.

\subsection{Summary of the Dataset Generation}
As discussed in this section and related to  \figref{fig:dataset_generation_procedure}, three key steps were considered in the development of the dataset: 
\begin{enumerate}
    \item {\bf{Filtering process}}: Structures with MIDR exceeding $1\%$ under gravity loads were excluded,  resulting in $1,573$ labeled structures retained for fire simulation and $16,050$ unlabeled structures for training the MFSP predictor.
    \item {\bf{Fire simulations}}: For each retained labeled structure, 30 fire scenarios were simulated using OpenSeesRT, yielding $47,190$ fire cases.
    \item {\bf{Data distribution check}}: MIDR distributions for labeled and unlabeled data under gravity loads were highly similar, because both datasets were generated using the same method. Under fire conditions, the MIDR distribution shifted, reflecting significant structural deformation with values reaching a maximum of about 6\%, an average of 1.70\%, and a standard deviation of 1.12\%. This step ensured a diverse and comprehensive dataset for the proposed predictive framework.
\end{enumerate}
The statistical distribution histograms for MIDR (after applying the $1\%$ filtering threshold \revise{for gravity load responses}) under different loading conditions are plotted in \figref{fig:histogram_mdr}. Figures \ref{fig:histogram_mdr}(a) and \ref{fig:histogram_mdr}(b) show the MIDR distributions of the labeled and unlabeled data, respectively, under gravity loads only. \figref{fig:histogram_mdr}(c) shows the MIDR distribution of the labeled data under the combined effects of gravity and fire loads. Fire load causes the structures to significantly deform, leading to a noticeably \revise{right-skewed} MIDR distribution.

\begin{figure*}[h!]
    \centering
    \includegraphics[width=\linewidth]{figures/histogram_mdr.pdf}
    \caption{Histograms of MIDR for labeled and unlabeled structures with gravity loads and fire cases.}
    \label{fig:histogram_mdr}
\end{figure*}

\revise{
This dataset provides the basis for training and testing the performance of the GNN-based framework. Although we employed a simplified rule-based thermal load generation method compared with conventional CFD-based simulations, the temperature field, the changes of the material properties, and the response of the structures, are all still highly nonlinear and complex. Therefore, it is still a challenging task for the NN to predict the MIDRs based on this dataset.
}

% In detail, Cora and CiteSeer are two popular citation graph datasets. In these graphs, nodes represent papers and edges correspond to the citation relationship between two papers. Nodes are classified according to academic topics.
% Chameleon and Squirrel are Wikipedia page networks on specific topics, where nodes represent web pages and edges are the mutual links between them. Node features are the bag-of-words representation of informative nouns. 

\subsection{Experiments Setup}
\label{app: setup}
For the SGC backbone, we follow the~\cite{dgc} setting where we run $10$ runs for the fixed seed $42$ and calculate the mean and the standard deviation. 
Furthermore, we fix the learning rate and weight decay in the same dataset and run $100$ epochs for every dataset. 
For the GCN backbone, we follow the~\cite{contranorm} settings where we run $5$ runs from the seed $\{0,1,2,3,4\}$ and calculate the mean and the standard deviation. We fix the hidden dimension to $32$ and dropout rate to $0.6$.
Furthermore, we fix the learning rate to be $0.005$ and weight decay to be $5e-4$ and run $200$ epochs for every dataset. 
We use the default splits in torch\_geometric.
We use Tesla-V100-SXM2-32GB in all experiments.
% The residual hyperparameter $\alpha$ selects from $\{0.3, 0.5, 0.7\}$ and the DropEdge hyperparameter $\alpha$ selects from $\{0.3,0.5,0.7\}$.
% We select the best settings for PairNorm and ContraNorm based on their respective default hyperparameters. 
% We choose the best of scale controller of Label-\ours and Feature-\ours $\alpha,\ \beta$ from  $\{ 0.1, 0.2, 0.5, 0.7, 0.9\}$.
% We apply both the SGC and GCN backbones.



\subsection{Results Analysis}


\subsubsection{CSBM results}
The comparative results of Label-\ours and Feature-\ours against SGC are presented in Table \ref{table: app_sbm_results}. 
As the number of layers increases, SGC's node features suffer from oversmoothing, causing the two classes to converge and accuracy to drop by nearly $30$ points from its peak at $2$ layers, down to $45\%$. 
Conversely, after $300$ layers, \ours maintains strong performance, with node features of different classes repelling each other. 
This effect limits oversmoothing to within-class interactions, and improves performance from $85$ to $91$ in Label-\ours and from $48$ to $82$ in Feature-\ours, further substantiating our approach to mitigating oversmoothing.

We visualize the node features learned by Label-\ours in Figure~\ref{fig: SBM_Label}.
We can see that from layer $0$ to layer $200$, the node features from different labels repel each other and aggregate the node features from the same labels.
And we also visualize the adjacency matrix of Label-\ours and Feature-\ours in Figure~\ref{fig:adj label} and Figure~\ref{fig:adj feature} respectively, further verifying the effectiveness of our theorem and insights.
\begin{figure}[h]
    \centering
    \begin{subfigure}[b]{0.28\textwidth}
        \centering
        \includegraphics[width=0.95\textwidth]{figures/sbm_SGC_299.pdf} % Adjust the path and filename as necessary
        \caption{SGC, acc$=47.50$}
        \label{fig:overall matrix}
    \end{subfigure}
    \quad
    \begin{subfigure}[b]{0.28\textwidth}
        \centering
        \includegraphics[width=0.95\textwidth]{figures/sbm_Sign_299.pdf}
        \caption{Feature-\ours, acc$=80.00$}
        \label{fig:sbm_sgc}
    \end{subfigure}
    % \hfill
    \quad
    \begin{subfigure}[b]{0.28\textwidth}
        \centering
        \includegraphics[width=0.95\textwidth]{figures/sbm_Label_299.pdf} % Adjust the path and filename as necessary
        \caption{Label-\ours, acc$=97.50$}
        \label{fig:sbm_ours}
    \end{subfigure}
    
    \caption{The t-SNE visualization of the node features and the classification accuracy from $2$-CSBM and Layer$=300$. 
    Left is the result of the vallina SGC, and the middle and right are the results of \ours. }
    \label{fig: sbm overall}
\end{figure}
\begin{figure}[h]
\centering
\captionsetup[subfigure]{labelformat=empty} % Optional: removes labels (a), (b), etc.
\begin{subfigure}{0.3\textwidth}
    \includegraphics[width=\linewidth]{figures/sbm/node_sbm_0.png}
    \caption{L=0} % Optional caption
\end{subfigure}\hfill % Ensures no extra spaces between the images
\begin{subfigure}{0.3\textwidth}
    \includegraphics[width=\linewidth]{figures/sbm/node_Label_sbm_layernorm_1.png}
    \caption{L=1} % Optional caption
\end{subfigure}\hfill % Ensures no extra spaces between the images
\begin{subfigure}{0.3\textwidth}
    \includegraphics[width=\linewidth]{figures/sbm/node_Label_sbm_layernorm_10.png}
    \caption{L=10} % Optional caption
\end{subfigure}\hfill
\begin{subfigure}{0.3\textwidth}
    \includegraphics[width=\linewidth]{figures/sbm/node_Label_sbm_layernorm_50.png}
    \caption{L=50} % Optional caption
\end{subfigure}\hfill
\begin{subfigure}{0.3\textwidth}
    \includegraphics[width=\linewidth]{figures/sbm/node_Label_sbm_layernorm_100.png}
    \caption{L=100} % Optional caption
\end{subfigure}\hfill
\begin{subfigure}{0.3\textwidth}
    \includegraphics[width=\linewidth]{figures/sbm/node_Label_sbm_layernorm_199.png}
    \caption{L=200} % Optional caption
\end{subfigure}
\caption{CSBM node features visualization. We update the features by Label-\ours. L is the propagation layer number. 0,1 represent different classes.}
\label{fig: SBM_Label} % Corrected label placement
\end{figure}
\begin{figure}[h]
    \centering
    \includegraphics[width=1\textwidth]
    {figures/adj_label.pdf}
    \caption{The visualization of the adjacency matrix of Label-\ours. Here left is the positive graph; middle is the negative graph; right is the overall signed graph.}
    \label{fig:adj label}
\end{figure}
\begin{figure}[h]
    \centering
    \includegraphics[width=1\textwidth]
    {figures/adj_feature.pdf}
    \caption{The visualization of the adjacency matrix of Feature-\ours. Here left is the positive graph; middle is the negative graph; right is the overall signed graph.}
    \label{fig:adj feature}
\end{figure}

\begin{table}[h]
\centering
% \vspace{-0.15in}
\caption{CSBM test accuracy (\%) comparison results. The best results are marked in blue on each layer. The second best results are marked in gray on each layer. We run 10 runs for the seed from $0-9$ and demonstrate the mean $\pm$ std in the table.}
\begin{adjustbox}{width=0.99\textwidth}
\begin{tabular}{lccccccc}
\toprule
 Model             & \#L=2              & \#L=5              & \#L=10             & \#L=20             & \#L=50        & \#L=100    & \#L=200    \\
\midrule
SGC & 73.25 {\footnotesize $\pm$ 6.90} & 44.50 {\footnotesize $\pm$ 9.34} & 45.75 {\footnotesize $\pm$ 9.36} & 45.75 {\footnotesize $\pm$ 9.36} & 45.75 {\footnotesize $\pm$ 9.36} & 45.75 {\footnotesize $\pm$ 9.36} & 45.75 {\footnotesize $\pm$ 9.36} \\
Feature-\ourst &\cellcolor{secondbest}48.75 {\footnotesize $\pm$ 5.62} &\cellcolor{secondbest} 53.75 {\footnotesize $\pm$ 6.45} & \cellcolor{secondbest}63.75 {\footnotesize $\pm$ 6.25} & \cellcolor{secondbest}77.00 {\footnotesize $\pm$ 5.45} &\cellcolor{secondbest} 82.00 {\footnotesize $\pm$ 4.58} &\cellcolor{secondbest} 82.50 {\footnotesize $\pm$ 5.12} &\cellcolor{secondbest} 82.00 {\footnotesize $\pm$ 5.45} \\
Label-\ourst & \cellcolor{best}85.75 {\footnotesize $\pm$ 4.04} & \cellcolor{best}93.50 {\footnotesize $\pm$ 4.06} & \cellcolor{best}93.50 {\footnotesize $\pm$ 3.57} & \cellcolor{best}93.50 {\footnotesize $\pm$ 3.57} & \cellcolor{best}92.25 {\footnotesize $\pm$ 3.44} & \cellcolor{best}93.25 {\footnotesize $\pm$ 3.72} & \cellcolor{best}91.25 {\footnotesize $\pm$ 6.05} \\


\bottomrule
\end{tabular}
\end{adjustbox}
\label{table: app_sbm_results}
\end{table}




% \begin{figure}[ht]
% \centering
% \captionsetup[subfigure]{labelformat=empty} % Optional: removes labels (a), (b), etc.
% \begin{subfigure}{0.3\textwidth}
%     \includegraphics[width=\linewidth]{figures/sbm/node_sbm_0.png}
%     \caption{L=0} % Optional caption
% \end{subfigure}\hfill % Ensures no extra 
% \begin{subfigure}{0.3\textwidth}
%     \includegraphics[width=\linewidth]{figures/sbm/node_SGC_sbm_1.png}
%     \caption{L=1} % Optional caption
% \end{subfigure}\hfill % Ensures no extra spaces between the images
% \begin{subfigure}{0.3\textwidth}
%     \includegraphics[width=\linewidth]{figures/sbm/node_SGC_sbm_10.png}
%     \caption{L=10} % Optional caption
% \end{subfigure}\hfill
% \begin{subfigure}{0.3\textwidth}
%     \includegraphics[width=\linewidth]{figures/sbm/node_SGC_sbm_50.png}
%     \caption{L=50} % Optional caption
% \end{subfigure}\hfill
% \begin{subfigure}{0.3\textwidth}
%     \includegraphics[width=\linewidth]{figures/sbm/node_SGC_sbm_100.png}
%     \caption{L=100} % Optional caption
% \end{subfigure}\hfill
% \begin{subfigure}{0.3\textwidth}
%     \includegraphics[width=\linewidth]{figures/sbm/node_SGC_sbm_199.png}
%     \caption{L=200} % Optional caption
% \end{subfigure}
% \caption{2-SBM node features visualization. We update the features by SGC \cite{sgc}. L is the propagation layer number. 0,1 represent different classes.}
% \label{fig:SBM_SGC} % Corrected label placement
% \end{figure}





\subsubsection{GCN Results}
\begin{table}[h]
\centering
\small
\caption{GCN test accuracy (\%) comparison results. The best results are marked in blue and the second best results are marked in gray on every layer. We run 5 runs for the seed from $0-4$ and demonstrate the mean $\pm$ std in the table.}
\begin{adjustbox}{width=0.99\textwidth}
\begin{tabular}{lcccccc}
\toprule
 Model             & \#L=2              & \#L=4              & \#L=8              & \#L=16             & \#L=32             & \#L=64\\

\midrule
\rowcolor{gray!8}\multicolumn{7}{c}{\textit{Cora}~\citep{cora}}\\
\midrule
  GCN~\citep{gcn} & \cellcolor{secondbest}80.68 {\footnotesize$\pm 0.09$} & \cellcolor{secondbest}79.69 {\footnotesize$\pm 0.00$} & 74.32 {\footnotesize$\pm 0.00$} & 30.95 {\footnotesize$\pm 0.00$} & 30.95 {\footnotesize$\pm 0.00$} & 24.85 {\footnotesize$\pm 7.46$} \\
GAT~\citep{gat} & 81.48 {\footnotesize$\pm$ 0.48} & 80.69 {\footnotesize$\pm$ 0.93} & 58.59 {\footnotesize$\pm$ 1.95} & 25.17 {\footnotesize$\pm$ 5.67} & 31.93 {\footnotesize$\pm$ 0.21} & 28.38 {\footnotesize$\pm$ 0.00} \\
wGCN~\citep{wGCN} & 80.97 {\footnotesize$\pm$ 0.28} & 80.51 {\footnotesize$\pm$ 0.00} & 80.46 {\footnotesize$\pm$ 1.77} & 70.53 {\footnotesize$\pm$ 22.09} & 80.02 {\footnotesize$\pm$ 0.12} & 27.90 {\footnotesize$\pm$ 6.09} \\
    % +LayerNorm~\citep{layernorm} & 80.51 {\footnotesize$\pm 0.12$} & \cellcolor{best}80.28 {\footnotesize$\pm 0.66$} & 75.05 {\footnotesize$\pm 0.00$} & 30.95 {\footnotesize$\pm 0.00$} & 30.95 {\footnotesize$\pm 0.00$} & 24.85 {\footnotesize$\pm 7.46$} \\
    BatchNorm~\citep{batchnorm} & 78.09 {\footnotesize$\pm 0.00$} & 77.87 {\footnotesize$\pm 0.02$} & 73.62 {\footnotesize$\pm 0.57$} & 70.79 {\footnotesize$\pm 0.00$} & 53.90 {\footnotesize$\pm 2.19$} & 35.32 {\footnotesize$\pm 3.41$}\\
    PairNorm~\citep{pairnorm} & 79.01 {\footnotesize$\pm 0.00$} & 78.26 {\footnotesize$\pm 0.50$} & 73.21 {\footnotesize$\pm 0.00$} & 62.96 {\footnotesize$\pm 0.00$} & 48.13 {\footnotesize$\pm 0.91$} & 44.01 {\footnotesize$\pm 3.46$} \\
    ContraNorm~\citep{contranorm} & \cellcolor{best}81.55 {\footnotesize$\pm 0.21$} & 79.61 {\footnotesize$\pm 0.75$} & 77.71 {\footnotesize$\pm 0.00$} & 63.35 {\footnotesize$\pm 0.00$} & 44.56 {\footnotesize$\pm 4.83$} & 38.97 {\footnotesize$\pm 0.00$} \\
    DropEdge~\citep{dropedge} & 78.38 {\footnotesize$\pm 0.00$} & 74.47 {\footnotesize$\pm 0.00$} & 26.91 {\footnotesize$\pm 0.83$} & 22.24 {\footnotesize$\pm 3.04$} & 27.18 {\footnotesize$\pm 0.00$} & 25.98 {\footnotesize$\pm 6.00$}\\
    Residual & 80.68 {\footnotesize$\pm 0.09$} & 78.77 {\footnotesize$\pm 0.00$} & \cellcolor{secondbest}79.26 {\footnotesize$\pm 0.21$} & 40.91 {\footnotesize$\pm 0.00$} & 30.95 {\footnotesize$\pm 0.00$} & 27.90 {\footnotesize$\pm 6.09$}\\
\midrule
     Feature-\ourst & 80.44 {\footnotesize$\pm 0.83$} & 79.26 {\footnotesize$\pm 1.18$} & 78.56 {\footnotesize$\pm 0.59$} & \cellcolor{secondbest}77.22 {\footnotesize$\pm 0.55$} & \cellcolor{secondbest}73.65 {\footnotesize$\pm 0.48$} & \cellcolor{secondbest}61.62 {\footnotesize$\pm 5.24$}\\
     Label-\ourst & 80.31 {\footnotesize$\pm 0.70$} & 79.16 {\footnotesize$\pm 1.30$} & \cellcolor{best}79.50 {\footnotesize$\pm 0.00$} & \cellcolor{best}77.43 {\footnotesize$\pm 1.49$} & \cellcolor{best}74.52 {\footnotesize$\pm 0.36$} & \cellcolor{best}65.02 {\footnotesize$\pm 2.97$} \\
\midrule
\rowcolor{gray!8}\multicolumn{7}{c}{\textit{CiteSeer}~\citep{citeseer}}\\
\midrule
   GCN~\citep{gcn} & \cellcolor{best}67.45 {\footnotesize$\pm 0.54$} & 65.62 {\footnotesize$\pm 0.25$} & 37.22 {\footnotesize$\pm 2.46$} & 22.03 {\footnotesize$\pm 4.76$} & 19.65 {\footnotesize$\pm 0.00$} & 19.65 {\footnotesize$\pm 0.00$} \\
GAT~\cite{gat} & 69.91 {\footnotesize$\pm$ 0.86} & 67.47 {\footnotesize$\pm$ 0.22} & 44.71 {\footnotesize$\pm$ 3.07} & 23.48 {\footnotesize$\pm$ 1.36} & 24.40 {\footnotesize$\pm$ 0.40} & 25.95 {\footnotesize$\pm$ 2.17} \\
wGCN~\citep{wGCN} & 66.21 {\footnotesize$\pm$ 0.63} & 66.49 {\footnotesize$\pm$ 0.69} & 66.79 {\footnotesize$\pm$ 0.00} & 57.54 {\footnotesize$\pm$ 18.94} & 19.65 {\footnotesize$\pm$ 0.00} & 19.65 {\footnotesize$\pm$ 0.00} \\% +LayerNorm~\citep{layernorm} & 67.24 {\footnotesize$\pm 0.66$} & 64.95 {\footnotesize$\pm 0.72$} & 38.87 {\footnotesize$\pm 4.12$} & 24.29 {\footnotesize$\pm 5.68$} & 19.65 {\footnotesize$\pm 0.00$} & 19.65 {\footnotesize$\pm 0.00$} \\
     BatchNorm~\citep{batchnorm} & 63.44 {\footnotesize$\pm 0.94$} & 62.34 {\footnotesize$\pm 0.25$} & 61.36 {\footnotesize$\pm 0.00$} & 50.58 {\footnotesize$\pm 1.24$} & 41.41 {\footnotesize$\pm 0.00$} & 35.00 {\footnotesize$\pm 1.09$} \\
    PairNorm~\citep{pairnorm} & 63.58 {\footnotesize$\pm 0.63$} & 64.32 {\footnotesize$\pm 0.95$} & 61.95 {\footnotesize$\pm 1.24$} & 50.06 {\footnotesize$\pm 0.00$} & 37.21 {\footnotesize$\pm 1.87$} & 36.09 {\footnotesize$\pm 0.07$} \\
    ContraNorm~\citep{contranorm} & 66.83 {\footnotesize$\pm 0.49$} & 64.78 {\footnotesize$\pm 0.92$} & 60.70 {\footnotesize$\pm 0.60$} & 44.79 {\footnotesize$\pm 1.65$} & 37.36 {\footnotesize$\pm 0.25$} & 30.85 {\footnotesize$\pm 0.81$} \\
    DropEdge~\citep{dropedge} & 63.86 {\footnotesize$\pm 0.03$} & 62.24 {\footnotesize$\pm 0.90$} & 24.73 {\footnotesize$\pm 5.72$} & 20.65 {\footnotesize$\pm 0.00$} & 20.04 {\footnotesize$\pm 0.19$} & 19.95 {\footnotesize$\pm 0.09$}\\
    Residual & \cellcolor{secondbest}67.45 {\footnotesize$\pm 0.54$} & 66.21 {\footnotesize$\pm 0.16$} & \cellcolor{best}67.34 {\footnotesize$\pm 0.00$} & 33.21 {\footnotesize$\pm 0.00$} & 19.65 {\footnotesize$\pm 0.00$} & 19.65 {\footnotesize$\pm 0.00$} \\
\midrule
    Feature-\ourst &  67.38 {\footnotesize$\pm 0.66$} & \cellcolor{best}66.94 {\footnotesize$\pm 0.00$} & 66.29 {\footnotesize$\pm 0.02$} & \cellcolor{secondbest}65.35 {\footnotesize$\pm 1.99$} & \cellcolor{best}61.43 {\footnotesize$\pm 0.00$} & \cellcolor{secondbest}42.09 {\footnotesize$\pm 1.65$}\\
     Label-\ourst & 67.23 {\footnotesize$\pm 0.64$} & \cellcolor{secondbest} 66.72 {\footnotesize$\pm 0.00$} & \cellcolor{secondbest}66.29 {\footnotesize$\pm 0.89$} & \cellcolor{best}65.50 {\footnotesize$\pm 2.13$} & \cellcolor{secondbest}59.93 {\footnotesize$\pm 0.85$} & \cellcolor{best}44.41 {\footnotesize$\pm 1.57$} \\
\midrule
\rowcolor{gray!8}\multicolumn{7}{c}{\textit{PubMed}~\citep{pubmed}}\\
\midrule
   GCN~\citep{gcn} & \cellcolor{best}76.44 {\footnotesize$\pm 0.34$} & 76.52 {\footnotesize$\pm 0.32$} & 69.58 {\footnotesize$\pm 5.89$} & 39.92 {\footnotesize$\pm 0.00$} & 39.92 {\footnotesize$\pm 0.00$} & 39.92 {\footnotesize$\pm 0.00$} \\

    % +LayerNorm~\citep{layernorm} & 76.27 {\footnotesize$\pm 0.51$} & 76.71 {\footnotesize$\pm 0.24$} & 76.95 {\footnotesize$\pm 0.17$} & 39.92 {\footnotesize$\pm 0.00$} & 39.92 {\footnotesize$\pm 0.00$} & 39.92 {\footnotesize$\pm 0.00$} \\
    +BatchNorm~\citep{batchnorm} & 75.52 {\footnotesize$\pm 0.12$} & \cellcolor{secondbest}77.15 {\footnotesize$\pm 0.00$} & 77.10 {\footnotesize$\pm 0.00$} & 76.92 {\footnotesize$\pm 0.00$} & 75.43 {\footnotesize$\pm 0.00$} & 69.33 {\footnotesize$\pm 1.01$} \\
    +PairNorm~\citep{pairnorm} & 75.66 {\footnotesize$\pm 0.11$} & 76.71 {\footnotesize$\pm 0.00$} & \cellcolor{secondbest}77.99 {\footnotesize$\pm 0.00$} & \cellcolor{secondbest}77.22 {\footnotesize$\pm 0.39$} & 75.52 {\footnotesize$\pm 2.02$} & 71.22 {\footnotesize$\pm 3.68$} \\
    +ContraNorm~\citep{contranorm} & 76.05 {\footnotesize$\pm 0.33$} & \cellcolor{best}78.42 {\footnotesize$\pm 0.00$} & OOM & OOM & OOM & OOM \\
    +DropEdge~\citep{dropedge}& 73.41 {\footnotesize$\pm 0.03$} & 73.96 {\footnotesize$\pm 0.79$} & 52.51 {\footnotesize$\pm 10.91$} & 40.27 {\footnotesize$\pm 0.00$} & 39.90 {\footnotesize$\pm 0.59$} & 40.08 {\footnotesize$\pm 0.39$} \\
    +Residual & \cellcolor{secondbest}76.44 {\footnotesize$\pm 0.34$} & 77.28 {\footnotesize$\pm 0.00$} & 77.38 {\footnotesize$\pm 0.00$} & 63.14 {\footnotesize$\pm 3.05$} & 39.92 {\footnotesize$\pm 0.00$} & 39.92 {\footnotesize$\pm 0.00$} \\
\midrule
    Feature-\ourst & 75.72 {\footnotesize$\pm 0.06$} & 76.84 {\footnotesize$\pm 0.00$} & \cellcolor{best}78.39 {\footnotesize$\pm 0.00$} & \cellcolor{best}79.71 {\footnotesize$\pm 0.00$} & \cellcolor{best}77.59 {\footnotesize$\pm 0.23$} & \cellcolor{best}78.06 {\footnotesize$\pm 0.13$}\\
    Label-\ourst & 76.33 {\footnotesize$\pm 0.25$} & 76.91 {\footnotesize$\pm 0.00$} & 77.60 {\footnotesize$\pm 0.49$} & 76.31 {\footnotesize$\pm 0.00$} & \cellcolor{secondbest}77.17 {\footnotesize$\pm 0.67$} & \cellcolor{secondbest}78.01 {\footnotesize$\pm 0.16$}\\
\bottomrule
\end{tabular}
\end{adjustbox}
\label{table: gcn result}
\end{table}

% \begin{table}[h]
% \centering
% \caption{GCN test accuracy (\%) comparison results. The best results are marked in blue and the second best results are marked in gray on every layer.}
% \begin{adjustbox}{width=0.99\textwidth}
% \begin{tabular}{lcccccc}
% \toprule
%  Model             & \#L=2              & \#L=4              & \#L=8              & \#L=16             & \#L=32             & \#L=64\\

% \midrule
% \rowcolor{gray!8}\multicolumn{7}{c}{\textit{Cora}~\citepp{cora}}\\
% \midrule
%   GCN~\citep{gcn} & \cellcolor{secondbest}80.68 $\pm$ 0.09 & \cellcolor{secondbest}79.69 $\pm$ 0.00 & 74.32 $\pm$ 0.00 & 30.95 $\pm$ 0.00 & 30.95 $\pm$ 0.00 & 24.85 $\pm$ 7.46 \\


% % & Center & 79.85 $\pm$ 0.46 & 77.32 $\pm$ 1.33 & 75.25 $\pm$ 0.40 & 57.75 $\pm$ 3.53 & 41.73 $\pm$ 0.92 & 39.34 $\pm$ 2.29 \\
%     LayerNorm~\citep{layernorm} & 80.51 $\pm$ 0.12 & \cellcolor{best}80.28 $\pm$ 0.66 & 75.05 $\pm$ 0.00 & 30.95 $\pm$ 0.00 & 30.95 $\pm$ 0.00 & 24.85 $\pm$ 7.46 \\
%     BatchNorm~\citep{batchnorm} & 78.09 $\pm$ 0.00 & 77.87 $\pm$ 0.02 & 73.62 $\pm$ 0.57 & 70.79 $\pm$ 0.00 & 53.90 $\pm$ 2.19 & 35.32 $\pm$ 3.41\\
%     PairNorm~\citep{pairnorm} & 79.01 $\pm$ 0.00 & 78.26 $\pm$ 0.50 & 73.21 $\pm$ 0.00 & 62.96 $\pm$ 0.00 & 48.13 $\pm$ 0.91 & 44.01 $\pm$ 3.46 \\
%     ContraNorm~\citep{contranorm} & \cellcolor{best}81.55 $\pm$ 0.21 & 79.61 $\pm$ 0.75 & 77.71 $\pm$ 0.00 & 63.35 $\pm$ 0.00 & 44.56 $\pm$ 4.83 & 38.97 $\pm$ 0.00 \\
%     DropEdge~\citep{dropedge} & 78.38 $\pm$ 0.00 & 74.47 $\pm$ 0.00 & 26.91 $\pm$ 0.83 & 22.24 $\pm$ 3.04 & 27.18 $\pm$ 0.00 & 25.98 $\pm$ 6.00\\
%     Residual& 80.68 $\pm$ 0.09 & 78.77 $\pm$ 0.00 & \cellcolor{secondbest}79.26 $\pm$ 0.21 & 40.91 $\pm$ 0.00 & 30.95 $\pm$ 0.00 & 27.90 $\pm$ 6.09\\
% \midrule
%      Feature-\ourst & 80.44 $\pm$ 0.83 & 79.26 $\pm$ 1.18 &  78.56 $\pm$ 0.59 & \cellcolor{secondbest} 77.22 $\pm$ 0.55 &\cellcolor{secondbest} 73.65 $\pm$ 0.48 &\cellcolor{secondbest} 61.62 $\pm$ 5.24\\
%      Label-\ourst &80.31 $\pm$ 0.70 & 79.16 $\pm$ 1.30 & \cellcolor{best}79.50 $\pm$ 0.00 & \cellcolor{best}77.43 $\pm$ 1.49 & \cellcolor{best}74.52 $\pm$ 0.36 & \cellcolor{best}65.02 $\pm$ 2.97 \\
% \midrule
% \rowcolor{gray!8}\multicolumn{7}{c}{\textit{CiteSeer}~\citepp{citeseer}}\\
% \midrule
%    GCN~\citep{gcn} &\cellcolor{best} 67.45 $\pm$ 0.54 & 65.62 $\pm$ 0.25 & 37.22 $\pm$ 2.46 & 22.03 $\pm$ 4.76 & 19.65 $\pm$ 0.00 & 19.65 $\pm$ 0.00 \\


% % & Center & 67.21 $\pm$ 0.64 & 65.50 $\pm$ 0.99 & 59.25 $\pm$ 3.18 & 40.29 $\pm$ 1.18 & 41.73 $\pm$ 0.92 & 35.81 \pm 1.21\\
%     LayerNorm~\citep{layernorm} & 67.24 $\pm$ 0.66 &64.95 $\pm$ 0.72 & 38.87 $\pm$ 4.12 & 24.29 $\pm$ 5.68 & 19.65 $\pm$ 0.00 & 19.65 $\pm$ 0.00 \\
%      BatchNorm~\citep{batchnorm} &63.44 $\pm$ 0.94 & 62.34 $\pm$ 0.25 & 61.36 $\pm$ 0.00 & 50.58 $\pm$ 1.24 & 41.41 $\pm$ 0.00 & 35.00 $\pm$ 1.09 \\
%     PairNorm~\citep{pairnorm} & 63.58 $\pm$ 0.63 & 64.32 $\pm$ 0.95 & 61.95 $\pm$ 1.24 & 50.06 $\pm$ 0.00 & 37.21 $\pm$ 1.87 & 36.09 $\pm$ 0.07 \\
%     ContraNorm~\citep{contranorm} & 66.83 $\pm$ 0.49 & 64.78 $\pm$ 0.92 & 60.70 $\pm$ 0.60 & 44.79 $\pm$ 1.65 & 37.36 $\pm$ 0.25 & 30.85 $\pm$ 0.81 \\
%     DropEdge~\citep{dropedge} & 63.86 $\pm$ 0.03 & 62.24 $\pm$ 0.90 & 24.73 $\pm$ 5.72 & 20.65 $\pm$ 0.00 & 20.04 $\pm$ 0.19 & 19.95 $\pm$ 0.09\\
%     Residual & \cellcolor{secondbest} 67.45 $\pm$ 0.54 & 66.21 $\pm$ 0.16 & \cellcolor{best}67.34 $\pm$ 0.00 & 33.21 $\pm$ 0.00 & 19.65 $\pm$ 0.00 & 19.65 $\pm$ 0.00 \\
% \midrule
%     Feature-\ourst &  67.38 $\pm$ 0.66 & \cellcolor{best}66.94 $\pm$ 0.00 & 66.29 $\pm$ 0.02 & \cellcolor{secondbest}65.35 $\pm$ 1.99 & \cellcolor{best}61.43 $\pm$ 0.00 & \cellcolor{secondbest}42.09 $\pm$ 1.65\\
%      Label-\ourst & 67.23 $\pm$ 0.64 & \cellcolor{secondbest} 66.72 $\pm$ 0.00 & \cellcolor{secondbest}66.29 $\pm$ 0.89 & \cellcolor{best}65.50 $\pm$ 2.13 & \cellcolor{secondbest}59.93 $\pm$ 0.85 & \cellcolor{best}44.41 $\pm$ 1.57 \\
% \midrule
% \rowcolor{gray!8}\multicolumn{7}{c}{\textit{PubMed}~\citepp{pubmed}}\\
% \midrule
%    GCN~\citep{gcn} & \cellcolor{best}76.44 $\pm$ 0.34 & 76.52 $\pm$ 0.32 & 69.58 $\pm$ 5.89 & 39.92 $\pm$ 0.00 & 39.92 $\pm$ 0.00 & 39.92 $\pm$ 0.00 \\


% % & Center & 75.19 $\pm$0.26	& 76.67 \pm	0.00 &OOM &OOM &OOM & OOM\\
%     LayerNorm~\citep{layernorm} & 76.27 $\pm$ 0.51 & 76.71 $\pm$ 0.24 & 76.95 $\pm$ 0.17 & 39.92 $\pm$ 0.00 & 39.92 $\pm$ 0.00 & 39.92 $\pm$ 0.00 \\
%     BatchNorm~\citep{batchnorm} & 75.52 $\pm$ 0.12 & \cellcolor{secondbest}77.15 $\pm$ 0.00 & 77.10 $\pm$ 0.00 & 76.92 $\pm$ 0.00 & 75.43 $\pm$ 0.00 & 69.33 $\pm$ 1.01 \\
%     PairNorm~\citep{pairnorm} & 75.66 $\pm$ 0.11 & 76.71 $\pm$ 0.00 & \cellcolor{secondbest}77.99 $\pm$ 0.00 & \cellcolor{secondbest}77.22 $\pm$ 0.39 & 75.52 $\pm$ 2.02 & 71.22 $\pm$ 3.68 \\
%     ContraNorm~\citep{contranorm} & 76.05 $\pm$ 0.33 & \cellcolor{best}78.42 $\pm$ 0.00 & OOM & OOM & OOM & OOM \\
%     DropEdge~\citep{dropedge}& 73.41 $\pm$ 0.03 & 73.96 $\pm$ 0.79 & 52.51 $\pm$ 10.91 & 40.27 $\pm$ 0.00 & 39.90 $\pm$ 0.59 & 40.08 $\pm$ 0.39 \\
%     Residual & \cellcolor{secondbest} 76.44 $\pm$ 0.34 & 77.28 $\pm$ 0.00 & 77.38 $\pm$ 0.00 & 63.14 $\pm$ 3.05 & 39.92 $\pm$ 0.00 & 39.92 $\pm$ 0.00 \\
% \midrule
%     Feature-\ourst & 75.72 $\pm$ 0.06 & 76.84 $\pm$ 0.00 & \cellcolor{best}78.39 $\pm$ 0.00 &\cellcolor{best} 79.71 $\pm$ 0.00 & \cellcolor{best}77.59 $\pm$ 0.23 & \cellcolor{best}78.06 $\pm$ 0.13\\
%     Label-\ourst & 76.33 $\pm$ 0.25 & 76.91 $\pm$ 0.00 & 77.60 $\pm$ 0.49 & 76.31 $\pm$ 0.00 & \cellcolor{secondbest}77.17 $\pm$ 0.67 & \cellcolor{secondbest}78.01 $\pm$ 0.16\\
% % Add more rows as needed
% % \midrule

% % Arxiv & GCN & 70.20 $\pm$ 0.36 & 70.84 $\pm$ 0.12 & 69.73 $\pm$ 0.28\\
% %     & APPNP \\
% %     & Center \\
% %     & LayerNorm \\
% %     & PairNorm \\
% %     & ContraNorm \\
% %     & Feature-\ourst \\
% %     & Label-\ourst \\
% \bottomrule
% \end{tabular}
% \end{adjustbox}
% \end{table}
The results for GCN are detailed in Table \ref{table: gcn result}, respectively. 
Overall, \ours consistently outperforms all previous methods, especially in deeper layers. 
Beyond $16$ layers in GCN, \ours maintains superior performance, affirming the effectiveness of our approach. 
Notably, \ours exceeds the best results of prior methods by at least $10\%$ and up to $30\%$ points in GCN's deepest layers, marking significant improvements.
Moreover, unlike previous methods that perform best in shallow layers, \ours excels in moderately deep layers, as observed in GCN across all datasets. 
This further confirms the effectiveness of \ours.


% \subsection{Ablation Experiments}
% \label{app: ablation}
% % \begin{table}[t]
\vspace{-0.1in}
\centering
\small
\caption{GCN test accuracy (\%) comparison results on heterophilic datasets. The best results are marked in blue and the second best results are marked underline on every layer.
We run 5 runs and demonstrate the mean $\pm$ std in the table.}%for the seed from $0~4$
\begin{adjustbox}{width=0.91\textwidth}
\begin{tabular}{lcccccc}
\toprule
 Model             & \#L=2              & \#L=4              & \#L=8              & \#L=16             & \#L=32             & \#L=64\\

\midrule
\rowcolor{gray!8}\multicolumn{7}{c}{\textit{Chameleon}~\cite{heter_dataset}}\\
\midrule
   GCN & 66.01{\footnotesize$\pm$0.72} & 54.21{\footnotesize$\pm$0.53} & 35.48{\footnotesize$\pm$3.09} & 22.37{\footnotesize$\pm$0.00} & 22.37{\footnotesize$\pm$0.00} & 22.37{\footnotesize$\pm$0.00} \\
    +BatchNorm & \underline{65.83{\footnotesize$\pm$0.58}} & 56.40{\footnotesize$\pm$0.35} & 36.36{\footnotesize$\pm$2.04} & 22.37{\footnotesize$\pm$0.00} & 22.37{\footnotesize$\pm$0.00} & 22.37{\footnotesize$\pm$0.00}\\
    +PairNorm & 66.01{\footnotesize$\pm$0.72} & 54.12{\footnotesize$\pm$0.79} & 36.75{\footnotesize$\pm$0.38} & 22.37{\footnotesize$\pm$0.00} & 22.37{\footnotesize$\pm$0.00} & 22.37{\footnotesize$\pm$0.00}\\
    +ContraNorm & 66.01{\footnotesize$\pm$0.72} & 58.16{\footnotesize$\pm$1.76} & 37.15{\footnotesize$\pm$4.91} & 22.37{\footnotesize$\pm$0.00} & 22.37{\footnotesize$\pm$0.00} & 22.37{\footnotesize$\pm$0.00}\\
    +DropEdge & 62.50{\footnotesize$\pm$0.00} & 53.07{\footnotesize$\pm$1.61} & 32.15{\footnotesize$\pm$1.49} & 21.71{\footnotesize$\pm$0.00} & \underline{27.19{\footnotesize$\pm$1.75}} & \underline{23.68{\footnotesize$\pm$0.00}} \\
    +Residual & 66.01{\footnotesize$\pm$0.72} & \cellcolor{best}62.94{\footnotesize$\pm$0.00} & \underline{57.59{\footnotesize$\pm$2.58}} & \underline{41.27{\footnotesize$\pm$0.32}} & 22.37{\footnotesize$\pm$0.00} & 22.37{\footnotesize$\pm$0.00} \\
\midrule
    Feature-\ourst & 62.98{\footnotesize$\pm$0.75} & \underline{62.89{\footnotesize$\pm$1.29}} & \cellcolor{best}65.35{\footnotesize$\pm$0.00} & \cellcolor{best}62.28{\footnotesize$\pm$0.00} & \cellcolor{best}55.31{\footnotesize$\pm$0.53} & \cellcolor{best}35.13{\footnotesize$\pm$0.93}\\
    Label-\ourst & \cellcolor{best}66.01{\footnotesize$\pm$0.72} & 57.11{\footnotesize$\pm$0.11} & 38.11{\footnotesize$\pm$1.87} & 22.37{\footnotesize$\pm$0.00} & 22.37{\footnotesize$\pm$0.00} & 22.37{\footnotesize$\pm$0.00}\\
\midrule
\rowcolor{gray!8}\multicolumn{7}{c}{\textit{Squirrel}~\cite{heter_dataset}}\\
\midrule
   GCN  & 42.38{\footnotesize$\pm$0.04} & 32.20{\footnotesize$\pm$3.05} & 22.57{\footnotesize$\pm$0.00} & 20.46{\footnotesize$\pm$0.00} & 20.46{\footnotesize$\pm$0.00} & 20.46{\footnotesize$\pm$0.00}\\
    +BatchNorm & 41.77{\footnotesize$\pm$0.35} & 32.37{\footnotesize$\pm$3.46} & 22.67{\footnotesize$\pm$0.00} & 20.46{\footnotesize$\pm$0.00} & 20.46{\footnotesize$\pm$0.00} & 20.46{\footnotesize$\pm$0.00}\\
    +PairNorm & 42.75{\footnotesize$\pm$0.00} & 32.12{\footnotesize$\pm$3.00} & 22.57{\footnotesize$\pm$0.00} & 20.46{\footnotesize$\pm$0.00} & 20.46{\footnotesize$\pm$0.00} & 20.46{\footnotesize$\pm$0.00}\\
    +ContraNorm & \underline{43.78{\footnotesize$\pm$1.08}} & 32.80{\footnotesize$\pm$3.76} & 22.57{\footnotesize$\pm$0.00} & 20.46{\footnotesize$\pm$0.00} & 20.46{\footnotesize$\pm$0.00} & 20.46{\footnotesize$\pm$0.00} \\
    +DropEdge & 40.54{\footnotesize$\pm$0.00} & 22.57{\footnotesize$\pm$0.00} & 22.77{\footnotesize$\pm$2.12} & 22.19{\footnotesize$\pm$0.58} & \underline{22.61{\footnotesize$\pm$1.36}} & 20.46{\footnotesize$\pm$0.00}\\
    +Residual & 41.92{\footnotesize$\pm$0.65} & \underline{42.23{\footnotesize$\pm$0.08}} & \underline{39.15{\footnotesize$\pm$0.07}} & \underline{33.41{\footnotesize$\pm$2.73}} & 20.46{\footnotesize$\pm$0.00} & 20.46{\footnotesize$\pm$0.00} \\
\midrule
    Feature-\ourst& \cellcolor{best}44.48{\footnotesize$\pm$0.00} &\cellcolor{best} 45.01{\footnotesize$\pm$0.72} &\cellcolor{best} 44.03{\footnotesize$\pm$0.63} &\cellcolor{best} 41.42{\footnotesize$\pm$0.78} &\cellcolor{best} 36.79{\footnotesize$\pm$0.00} &\cellcolor{best} 29.20{\footnotesize$\pm$0.00}\\
    Label-\ourst& 43.61{\footnotesize$\pm$0.58} & 32.78{\footnotesize$\pm$3.49} & 22.79{\footnotesize$\pm$0.09} & 20.46{\footnotesize$\pm$0.00} & 20.46{\footnotesize$\pm$0.00} & \underline{20.46{\footnotesize$\pm$0.00}} \\

\bottomrule
\end{tabular}
\end{adjustbox}
\label{table: gcn heter}
\vspace{-0.1in}
\end{table}

% \begin{table}[h]
% \centering
% \caption{GCN test accuracy (\%) comparison results. The best results are marked in blue and the second best results are marked in gray on every layer.}
% \begin{adjustbox}{width=0.99\textwidth}
% \begin{tabular}{lcccccc}
% \toprule
%  Model             & \#L=2              & \#L=4              & \#L=8              & \#L=16             & \#L=32             & \#L=64\\

% \midrule
% \rowcolor{gray!8}\multicolumn{7}{c}{\textit{Cora}~\citep{cora}}\\
% \midrule
%   GCN~\cite{gcn} & \cellcolor{secondbest}80.68 $\pm$ 0.09 & \cellcolor{secondbest}79.69 $\pm$ 0.00 & 74.32 $\pm$ 0.00 & 30.95 $\pm$ 0.00 & 30.95 $\pm$ 0.00 & 24.85 $\pm$ 7.46 \\


% % & Center & 79.85 $\pm$ 0.46 & 77.32 $\pm$ 1.33 & 75.25 $\pm$ 0.40 & 57.75 $\pm$ 3.53 & 41.73 $\pm$ 0.92 & 39.34 $\pm$ 2.29 \\
%     LayerNorm~\cite{layernorm} & 80.51 $\pm$ 0.12 & \cellcolor{best}80.28 $\pm$ 0.66 & 75.05 $\pm$ 0.00 & 30.95 $\pm$ 0.00 & 30.95 $\pm$ 0.00 & 24.85 $\pm$ 7.46 \\
%     BatchNorm~\cite{batchnorm} & 78.09 $\pm$ 0.00 & 77.87 $\pm$ 0.02 & 73.62 $\pm$ 0.57 & 70.79 $\pm$ 0.00 & 53.90 $\pm$ 2.19 & 35.32 $\pm$ 3.41\\
%     PairNorm~\cite{pairnorm} & 79.01 $\pm$ 0.00 & 78.26 $\pm$ 0.50 & 73.21 $\pm$ 0.00 & 62.96 $\pm$ 0.00 & 48.13 $\pm$ 0.91 & 44.01 $\pm$ 3.46 \\
%     ContraNorm~\cite{contranorm} & \cellcolor{best}81.55 $\pm$ 0.21 & 79.61 $\pm$ 0.75 & 77.71 $\pm$ 0.00 & 63.35 $\pm$ 0.00 & 44.56 $\pm$ 4.83 & 38.97 $\pm$ 0.00 \\
%     DropEdge~\cite{dropedge} & 78.38 $\pm$ 0.00 & 74.47 $\pm$ 0.00 & 26.91 $\pm$ 0.83 & 22.24 $\pm$ 3.04 & 27.18 $\pm$ 0.00 & 25.98 $\pm$ 6.00\\
%     Residual& 80.68 $\pm$ 0.09 & 78.77 $\pm$ 0.00 & \cellcolor{secondbest}79.26 $\pm$ 0.21 & 40.91 $\pm$ 0.00 & 30.95 $\pm$ 0.00 & 27.90 $\pm$ 6.09\\
% \midrule
%      \ourst-Feature & 80.44 $\pm$ 0.83 & 79.26 $\pm$ 1.18 &  78.56 $\pm$ 0.59 & \cellcolor{secondbest} 77.22 $\pm$ 0.55 &\cellcolor{secondbest} 73.65 $\pm$ 0.48 &\cellcolor{secondbest} 61.62 $\pm$ 5.24\\
%      \ourst-Label &80.31 $\pm$ 0.70 & 79.16 $\pm$ 1.30 & \cellcolor{best}79.50 $\pm$ 0.00 & \cellcolor{best}77.43 $\pm$ 1.49 & \cellcolor{best}74.52 $\pm$ 0.36 & \cellcolor{best}65.02 $\pm$ 2.97 \\
% \midrule
% \rowcolor{gray!8}\multicolumn{7}{c}{\textit{CiteSeer}~\citep{citeseer}}\\
% \midrule
%    GCN~\cite{gcn} &\cellcolor{best} 67.45 $\pm$ 0.54 & 65.62 $\pm$ 0.25 & 37.22 $\pm$ 2.46 & 22.03 $\pm$ 4.76 & 19.65 $\pm$ 0.00 & 19.65 $\pm$ 0.00 \\


% % & Center & 67.21 $\pm$ 0.64 & 65.50 $\pm$ 0.99 & 59.25 $\pm$ 3.18 & 40.29 $\pm$ 1.18 & 41.73 $\pm$ 0.92 & 35.81 \pm 1.21\\
%     LayerNorm~\cite{layernorm} & 67.24 $\pm$ 0.66 &64.95 $\pm$ 0.72 & 38.87 $\pm$ 4.12 & 24.29 $\pm$ 5.68 & 19.65 $\pm$ 0.00 & 19.65 $\pm$ 0.00 \\
%      BatchNorm~\cite{batchnorm} &63.44 $\pm$ 0.94 & 62.34 $\pm$ 0.25 & 61.36 $\pm$ 0.00 & 50.58 $\pm$ 1.24 & 41.41 $\pm$ 0.00 & 35.00 $\pm$ 1.09 \\
%     PairNorm~\cite{pairnorm} & 63.58 $\pm$ 0.63 & 64.32 $\pm$ 0.95 & 61.95 $\pm$ 1.24 & 50.06 $\pm$ 0.00 & 37.21 $\pm$ 1.87 & 36.09 $\pm$ 0.07 \\
%     ContraNorm~\cite{contranorm} & 66.83 $\pm$ 0.49 & 64.78 $\pm$ 0.92 & 60.70 $\pm$ 0.60 & 44.79 $\pm$ 1.65 & 37.36 $\pm$ 0.25 & 30.85 $\pm$ 0.81 \\
%     DropEdge~\cite{dropedge} & 63.86 $\pm$ 0.03 & 62.24 $\pm$ 0.90 & 24.73 $\pm$ 5.72 & 20.65 $\pm$ 0.00 & 20.04 $\pm$ 0.19 & 19.95 $\pm$ 0.09\\
%     Residual & \cellcolor{secondbest} 67.45 $\pm$ 0.54 & 66.21 $\pm$ 0.16 & \cellcolor{best}67.34 $\pm$ 0.00 & 33.21 $\pm$ 0.00 & 19.65 $\pm$ 0.00 & 19.65 $\pm$ 0.00 \\
% \midrule
%     \ourst-Feature &  67.38 $\pm$ 0.66 & \cellcolor{best}66.94 $\pm$ 0.00 & 66.29 $\pm$ 0.02 & \cellcolor{secondbest}65.35 $\pm$ 1.99 & \cellcolor{best}61.43 $\pm$ 0.00 & \cellcolor{secondbest}42.09 $\pm$ 1.65\\
%      \ourst-Label & 67.23 $\pm$ 0.64 & \cellcolor{secondbest} 66.72 $\pm$ 0.00 & \cellcolor{secondbest}66.29 $\pm$ 0.89 & \cellcolor{best}65.50 $\pm$ 2.13 & \cellcolor{secondbest}59.93 $\pm$ 0.85 & \cellcolor{best}44.41 $\pm$ 1.57 \\
% \midrule
% \rowcolor{gray!8}\multicolumn{7}{c}{\textit{PubMed}~\citep{pubmed}}\\
% \midrule
%    GCN~\cite{gcn} & \cellcolor{best}76.44 $\pm$ 0.34 & 76.52 $\pm$ 0.32 & 69.58 $\pm$ 5.89 & 39.92 $\pm$ 0.00 & 39.92 $\pm$ 0.00 & 39.92 $\pm$ 0.00 \\


% % & Center & 75.19 $\pm$0.26	& 76.67 \pm	0.00 &OOM &OOM &OOM & OOM\\
%     LayerNorm~\cite{layernorm} & 76.27 $\pm$ 0.51 & 76.71 $\pm$ 0.24 & 76.95 $\pm$ 0.17 & 39.92 $\pm$ 0.00 & 39.92 $\pm$ 0.00 & 39.92 $\pm$ 0.00 \\
%     BatchNorm~\cite{batchnorm} & 75.52 $\pm$ 0.12 & \cellcolor{secondbest}77.15 $\pm$ 0.00 & 77.10 $\pm$ 0.00 & 76.92 $\pm$ 0.00 & 75.43 $\pm$ 0.00 & 69.33 $\pm$ 1.01 \\
%     PairNorm~\cite{pairnorm} & 75.66 $\pm$ 0.11 & 76.71 $\pm$ 0.00 & \cellcolor{secondbest}77.99 $\pm$ 0.00 & \cellcolor{secondbest}77.22 $\pm$ 0.39 & 75.52 $\pm$ 2.02 & 71.22 $\pm$ 3.68 \\
%     ContraNorm~\cite{contranorm} & 76.05 $\pm$ 0.33 & \cellcolor{best}78.42 $\pm$ 0.00 & OOM & OOM & OOM & OOM \\
%     DropEdge~\cite{dropedge}& 73.41 $\pm$ 0.03 & 73.96 $\pm$ 0.79 & 52.51 $\pm$ 10.91 & 40.27 $\pm$ 0.00 & 39.90 $\pm$ 0.59 & 40.08 $\pm$ 0.39 \\
%     Residual & \cellcolor{secondbest} 76.44 $\pm$ 0.34 & 77.28 $\pm$ 0.00 & 77.38 $\pm$ 0.00 & 63.14 $\pm$ 3.05 & 39.92 $\pm$ 0.00 & 39.92 $\pm$ 0.00 \\
% \midrule
%     \ourst-Feature & 75.72 $\pm$ 0.06 & 76.84 $\pm$ 0.00 & \cellcolor{best}78.39 $\pm$ 0.00 &\cellcolor{best} 79.71 $\pm$ 0.00 & \cellcolor{best}77.59 $\pm$ 0.23 & \cellcolor{best}78.06 $\pm$ 0.13\\
%     \ourst-Label & 76.33 $\pm$ 0.25 & 76.91 $\pm$ 0.00 & 77.60 $\pm$ 0.49 & 76.31 $\pm$ 0.00 & \cellcolor{secondbest}77.17 $\pm$ 0.67 & \cellcolor{secondbest}78.01 $\pm$ 0.16\\
% % Add more rows as needed
% % \midrule

% % Arxiv & GCN & 70.20 $\pm$ 0.36 & 70.84 $\pm$ 0.12 & 69.73 $\pm$ 0.28\\
% %     & APPNP \\
% %     & Center \\
% %     & LayerNorm \\
% %     & PairNorm \\
% %     & ContraNorm \\
% %     & \ourst-Feature \\
% %     & \ourst-Label \\
% \bottomrule
% \end{tabular}
% \end{adjustbox}
% \end{table}
% % \begin{table*}
  [t]
  \centering
  \resizebox{\textwidth}{!}{%
  \begin{tabular}{cccccccccccc}
    \toprule \multicolumn{2}{c}{Components}                                                             & \multicolumn{5}{c}{Re-executability Rate (\%)} & \multicolumn{5}{c}{Readability (\#)} \\
    \cmidrule(lr){1-2} \cmidrule(lr){3-7} \cmidrule(lr){8-12}        \hspace{8pt}\labelemoji\hspace{8pt}                                                                & \hspace{8pt}\toolemoji\hspace{8pt}                                      & O0                                 & O1             & O2             & O3             & AVG            & O0             & O1             & O2             & O3             & AVG            \\
    \hline
    \rowcolor[rgb]{0.93,0.93,0.93}\multicolumn{12}{c}{\textbf{Initialize with LLM4Decompile-End-6.7B~\citep{llm4decompile}}}   \\
    \xmark                                                                                              & \xmark                                    & 69.51                              & 46.95          & 50.61          & 46.34          & 53.35          & 3.98 & 3.41 & 3.44 & 3.38 & 3.55 \\
    \cmark                                                                                              & \xmark                                    & 75.61                              & 50.61          & 50.00          & 50.00          & 56.55          & 4.01 & 3.44 & 3.39 & \textbf{3.49} & 3.58 \\
    \xmark                                                                                              & \cmark                                    & 83.54                     & \textbf{56.10}          & 51.22          & 50.61 & 60.37 & 4.05 & 3.51 & 3.51 & 3.42 & 3.62 \\
    \cmark                                                                                              & \cmark                                    & \textbf{85.37}                            & \textbf{56.10}                     & \textbf{51.83} & \textbf{52.43}          & \textbf{61.43} & \textbf{4.13} & \textbf{3.60} & \textbf{3.54} & \textbf{3.49} & \textbf{3.69} \\

    \rowcolor[rgb]{0.93,0.93,0.93}\multicolumn{12}{c}{\textbf{Initialize with Deepseek-Coder-6.7B-base~\citep{deepseekcoder}}} \\
    \xmark                                                                                              & \xmark                                    & 59.15                              & 35.98          & 39.02          & 37.80          & 42.99          & 3.71 & 3.05 & 3.16 & 3.05 & 3.24 \\
    \cmark                                                                                              & \xmark                                    & 66.46                              & 41.46          & 38.41          & 36.59          & 45.73          & 3.76 & 3.17 & \textbf{3.21} & 3.08 & 3.31 \\
    \xmark                                                                                              & \cmark                                    & 70.73                              & 39.63          & 39.02          & 40.24          & 47.41          & 3.90 & 3.17 & 3.08 & 3.11 & 3.31 \\
    \cmark                                                                                              & \cmark                                    & \textbf{79.88}                     & \textbf{45.73} & \textbf{43.90} & \textbf{42.68} & \textbf{53.05} & \textbf{3.96} & \textbf{3.21} & 3.18 & \textbf{3.19} & \textbf{3.38} \\
    \bottomrule
  \end{tabular}%
  }
  \caption{The ablation study of different methods across four optimization levels
  (O0, O1, O2, O3), as well as their average scores (AVG). The results in bold represent the optimal performance. The ~\labelemoji~ and ~\toolemoji~ means Relabedling and Function Call. \textbf{Bold} denotes the best performance.}
  \label{tab:ablation}
\end{table*}
% The test accuracy comparison for the heterophilic datasets is shown in Table~\ref{table: gcn heter}.
% We strengthen the weight $\beta$ of the negative subgraph selected from $\{1,2,5,10,20,50\}$.
% We can see that Label-\ours and Feature-\ours are still effective across all methods and all layers, verifying our signed graph propagation insight. 
% Note that our feature-based approach (Feature-\ours) is more effective than our label-based approach (Label-\ours). 
% The reason for this difference may be attributed to the relatively small proportion of the training set indicating that Feature-\ours can be a good supplement for Label-\ours when the train label rate is small.




% \subsection{Details of Experiments}

\subsubsection{\jq{Repulsion ablation on heterophilic datasets}}
Our method SBP can outperform other baselines under $\beta=1$ across different layers, so we do not tune our hyper-parameters carefully.
However, since $\beta$ is the weight of the negative adjacency matrix (\eqref{eq: sbp}) representing the repulsion between different nodes, as seen in Figure~\ref{fig:beta csbm} and~\ref{fig:beta real}, the best performance of \ours appears when $\beta$ is larger in the heterophilic graphs, so the result in Figure~\ref{fig: layer depth}(a) is not the best performance of our SBP.
To further show the effectiveness of our SBP, we conduct experiments on Cornell with different $\beta$ in Table~\ref{tab: beta on cornell}, the best $\beta$ is 20 where the performance increases 25 points in deep layer 50.

\begin{table}[htbp]
\centering
\caption{Ablation study of negative weight $\beta$ on Cornell dataset.}
\label{tab: beta on cornell}
\resizebox{\linewidth}{!}{%
\begin{tabular}{ccccccc}
\hline
 Layer & 2 & 5 & 10 & 20 & 50 \\
\hline
$\beta=0.1$ & 72.97 $\pm$ 0.00 & 67.57 $\pm$ 0.00 & 51.53 $\pm$ 0.00 & 35.14 $\pm$ 0.00 & 29.73 $\pm$ 0.00 \\
$\beta=1$ (default) & 72.97 $\pm$ 0.00 & 67.57 $\pm$ 0.00 & 51.53 $\pm$ 0.00 & 45.95 $\pm$ 0.00 & 35.14 $\pm$ 0.00 \\
$\beta=10$ & 70.27 $\pm$ 0.00 & 67.57 $\pm$ 0.00 & 58.11 $\pm$ 1.35 & 51.53 $\pm$ 0.00 & 51.53 $\pm$ 0.00 \\
$\beta=20$ (best) & 70.27 $\pm$ 0.00 & 70.27 $\pm$ 0.00 & 67.57 $\pm$ 0.00 & 59.46 $\pm$ 0.00 & 59.46 $\pm$ 0.00 \\
$\beta=50$ & 64.60 $\pm$ 0.00 & 40.54 $\pm$ 0.00 & 40.54 $\pm$ 0.00 & 40.54 $\pm$ 0.00 & 40.54 $\pm$ 0.00 \\
\hline
\end{tabular}%
}
\end{table}



\subsubsection{\jq{Performance of \ours on more benchmarks}}
We further compare our \ours with SGC on six additional datasets~\citep{platonov2023critical} in Table~\ref{tab: app more bench}. Our \ours outperforms SGC on five out of these six datasets. We believe that these six datasets, combined with the nine datasets presented in Table~\ref{table: sgc results} of our paper, provide sufficient evidence to demonstrate the effectiveness of our approach.
\begin{table}[htbp]
\centering
\caption{Performance Comparison on more datasets}
\resizebox{\linewidth}{!}{
\label{tab: app more bench}
\begin{tabular}{ccccccc}
\hline
 & actor & penny94 & roman-empire & Tolokers & Questions & Minesweeper \\
\hline
SGC & 29.18 $\pm$ 0.10 & 72.56 $\pm$ 0.05 & 40.83 $\pm$ 0.03 & 78.18 $\pm$ 0.02 & 97.09 $\pm$ 0.00 & 80.43 $\pm$ 0.00 \\
Feature-SBP & 34.93 $\pm$ 0.02 & 75.68 $\pm$ 0.01 & 66.48 $\pm$ 0.02 & 78.24 $\pm$ 0.04 & 97.14 $\pm$ 0.02 & 80.00 $\pm$ 0.00 \\
Label-SBP & 34.94 $\pm$ 0.00 & 75.74 $\pm$ 0.01 & 66.32 $\pm$ 0.01 & 78.46 $\pm$ 0.08 & 97.15 $\pm$ 0.02 & 80.00 $\pm$ 0.00 \\
\hline
\end{tabular}
}
\end{table}

\subsubsection{\jq{Combine \ours to other methods}}
\label{app: gcnii}
In this paper, we focus on introducing a novel theoretic signed graph perspective for oversmoothing analysis, so we do not take many tricks into account or carefully fine-tune our hyperparameters. 
Thus, our results in the paper are not as comparable to previous baselines~\citep{GCNII,ACM-GCN,PDE-GCN}.
% , adapting many tricks, such as random dropout features, normalizing the features.
\jq{However, existing oversmoothing researches are indeed hard to compare, because they are often composed of multiple techniques — such as residual, BatchNorm, data augmentation — and the parameters are often heavily (over-)tuned on small-scale datasets. And it becomes clear that to attain SOTA performance, one needs to essentially compose multiple such techniques without fully understanding their individual roles. For example, GCNII uses both initial residual connection and identity map, futher combined with dropout.}

\jq{Our goal is to provide a new unified understanding of these techniques, so we justified it by showing that SBP as a single simple technique can attain good performance. 
And we believe that it would work complementarily with other techniques in the field, because oversmoothing is still challenging to solve with a very larger depth. }

To further verify the effectiveness, we combine our SBP to one of the SOTA settings GCNII~\citep{GCNII} and the results are as seen in Table~\ref{tab:gcnii-performance}. 
\jq{The results indicate that after combining our method, GCNII demonstrates greater robustness as the layers go deeper, particularly in the middle layers (layer=8), highlighting the efficacy of our signed graph insight.}
% achieves higher scores than GCNII in deep layers but and 



\begin{table}[htbp]
\centering
\caption{Performance Comparison between \ours and GCNII under the GCNII settings on Cora and Citesser datasets}
\label{tab:gcnii-performance}
\resizebox{\linewidth}{!}{
\begin{tabular}{cccccccc}
\hline
 & & 2 & 4 & 8 & 16 & 32 & 64 \\
\hline
\multirow{3}{*}{Cora} & GCNII & 78.58 $\pm$ 0.00 & 77.76 $\pm$ 0.24 & 73.47 $\pm$ 3.82 & 78.12 $\pm$ 1.32 & 82.54 $\pm$ 0.00 & 81.34 $\pm$ 0.53 \\
 & Label-\ours & 78.74 $\pm$ 1.54 & 78.87 $\pm$ 0.00 & \cellcolor{best}79.14 $\pm$ 0.35 & 79.17 $\pm$ 0.41 & 80.86 $\pm$ 0.32 & 81.38 $\pm$ 0.30 \\
 & Feature-\ours & 77.95 $\pm$ 0.91 & 78.82 $\pm$ 0.00 & 78.11 $\pm$ 1.62 & 78.82 $\pm$ 0.29 & 81.82 $\pm$ 0.47 & 81.65 $\pm$ 0.40 \\
\hline
\multirow{3}{*}{Citesser} & GCNII & 61.66 $\pm$ 0.67 & 63.23 $\pm$ 2.31 & 64.58 $\pm$ 2.66 & 66.21 $\pm$ 0.64 & 69.38 $\pm$ 0.83 & 69.73 $\pm$ 0.26 \\
 & Label-\ours & 65.31 $\pm$ 0.63 & 63.93 $\pm$ 3.66 & 68.33 $\pm$ 0.99 & 66.46 $\pm$ 0.00 & 70.00 $\pm$ 0.81 & 69.47 $\pm$ 0.25 \\
 & Feature-\ours & 65.63 $\pm$ 0.87 & 64.43 $\pm$ 3.55 & \cellcolor{best}68.44 $\pm$ 1.19 & 66.94 $\pm$ 0.00 & 69.98 $\pm$ 0.93 & 69.66 $\pm$ 0.28 \\
\hline
\end{tabular}
}
\end{table}
\subsubsection{\jq{Performance of \ours on Large-scale graphs}}
We conducted experiments with a larger graph ogbn-products than ogbn-arxiv under 100 epochs and 2 layers in Table~\ref{tab: ogbn-products}. 
The results indicate that our \ours outperforms the initial GCN baselines. Given the results presented for ogbn-arxiv in Table 5 of our paper, we believe these findings adequately demonstrate the performance of our \ours on large-scale graphs.
\begin{table}[h]
\centering
\caption{Performance of different models on ogbn-products dataset. Time means the runtime, the format is (hour: minutes: seconds).}
\label{tab: ogbn-products}
\resizebox{0.45\linewidth}{!}{
\begin{tabular}{lcc}
\hline
Method & Accuracy & Time \\
\hline
GCN & 73.96 & 00:06:33 \\
BatchNorm & 74.93 & 00:06:18 \\
Feature-SBP & 74.90 & 00:06:43 \\
Label-SBP & 76.62 & 00:06:39 \\
\hline
\end{tabular}
}
\end{table}

\subsubsection{\jq{Further Optimization based on \ours}}
Based on the experiment results, we want to propose 2 strategies for further optimization. 

1) hyper-parameter tuning on the negative weight $\beta$. As seen in Figures~\ref{fig:beta csbm} and~\ref{fig:beta real}, we found that $\beta$ influences the performance a lot, our default $\beta=1$ for Table~\ref{table: sgc results} and~\ref{tab: large} is certainly not optimal for the above 4 homophilic datasets. We suggest tuning higher $\beta$ for the heterophilic graphs since they need more repulsion and smaller for the homophilic datasets.  As the layer deepens, maybe greater weight should be placed on the negative adjacency graphs to alleviate oversmoothing. 

2) adapt our SBP to more effective GNNs. Our method is simple, architecture-free, without additional learnable parameters, and thus can be flexibly applied in various architectures. As seen in Appendix~\ref{app: gcnii}, we adapt our SBP to the GCNII models, and the results increase more than adaptation in vanilla GNN as shown in Table~\ref{table: sgc results} and~\ref{tab: large}. Besides, compared to the GCNII, our SBP is more robust and stable to the layers as seen in Table~\ref{tab:gcnii-performance}.

\begin{table}[t]
\centering
% \vspace{-0.15in}
\caption{SGC test accuracy (\%) comparison results. The best results are marked in blue and the second best results are marked in gray on every layer. We run 10 runs and demonstrate the mean $\pm$ std in the table.} % for the seed $0$\~$9$ 
% \resizebox{\textwidth}{!}{
\begin{adjustbox}{width=0.99\textwidth}
\begin{tabular}{lccccccc}
\toprule
 Model             & \#L=2              & \#L=5              & \#L=10             & \#L=20             & \#L=50        & \#L=100    & \#L=300    \\
\midrule
\rowcolor{gray!8}\multicolumn{8}{c}{\textit{Cora}~\citep{cora}}\\
\midrule
% \midrule
% \rowcolor{gray!8}\textit{cora}~\citep{cora}\\
% \midrule
 SGC      & 80.21 {\footnotesize $\pm$ 0.07}& 81.45 {\footnotesize $\pm$ 0.14 }& 81.53 {\footnotesize $\pm$ 0.19 }& 79.53 {\footnotesize $\pm$ 0.14 }& 79.20 {\footnotesize $\pm$ 0.21 }& 76.13 {\footnotesize $\pm$ 0.24 }& 65.64 {\footnotesize $\pm$ 1.15 }\\

 % +LayerNorm~\cite{layernorm}       & 80.07 {\footnotesize $\pm$ 0.22 }& \cellcolor{secondbest}81.60 {\footnotesize $\pm$ 0.22 }& 81.20 {\footnotesize $\pm$ 0.29 }& 79.52 {\footnotesize $\pm$ 0.16  }& 79.21 {\footnotesize $\pm$ 0.23 }& 76.44 {\footnotesize $\pm$ 0.11 }& 68.38 {\footnotesize $\pm$ 0.66}  \\
 +BatchNorm & 77.90 {\footnotesize $\pm$ 0.00 }& 78.02 {\footnotesize $\pm$ 0.04 }& 76.94 {\footnotesize $\pm$ 0.08 }& 75.18 {\footnotesize $\pm$ 0.09 }& 74.54 {\footnotesize $\pm$ 0.05 }& 72.64 {\footnotesize $\pm$ 0.05 }& 63.12 {\footnotesize $\pm$ 0.06} \\
+PairNorm     & 80.30 {\footnotesize $\pm$ 0.05 }& 78.57 {\footnotesize $\pm$ 0.00 }&78.14 {\footnotesize $\pm$ 0.07 }& 76.90 {\footnotesize $\pm$ 0.00 }& 77.49 {\footnotesize $\pm$ 0.03}  & 72.01 {\footnotesize $\pm$ 0.03 }& 40.93 {\footnotesize $\pm$ 0.11} \\
 +ContraNorm      & \cellcolor{best}81.60 {\footnotesize $\pm$ 0.00 }& 80.67 {\footnotesize $\pm$ 0.06 }& 79.11 {\footnotesize $\pm$ 0.03 }& 74.28 {\footnotesize $\pm$ 0.15 }& 69.67 {\footnotesize $\pm$ 1.23 }& 65.58 {\footnotesize $\pm$ 2.11 }&47.21 {\footnotesize $\pm$ 10.80} \\
+DropEdge & 73.58 {\footnotesize $\pm$ 2.76 }& 62.11 {\footnotesize $\pm$ 5.10 }& 39.21 {\footnotesize $\pm$ 7.54 }& 15.07 {\footnotesize $\pm$ 6.22 }& 11.16 {\footnotesize $\pm$ 2.73 }& 11.15 {\footnotesize $\pm$ 2.81 }& 11.15 {\footnotesize $\pm$ 2.81 }\\
 +Residual& 77.81 {\footnotesize $\pm$ 0.03 }& 81.47 {\footnotesize $\pm$ 0.05 }& \cellcolor{best}82.90 {\footnotesize $\pm$ 0.00 }& 79.87 {\footnotesize $\pm$ 0.05 }& 75.64 {\footnotesize $\pm$ 0.05 }& 66.90 {\footnotesize $\pm$ 0.10 }& 25.33 {\footnotesize $\pm$ 0.46}\\
\midrule
 Feature-\ourst &78.10 {\footnotesize $\pm$ 0.11 }& 80.88 {\footnotesize $\pm$ 0.23 }& 80.83 {\footnotesize $\pm$ 0.37 }& \cellcolor{secondbest}82.46 {\footnotesize $\pm$ 0.07 }& \cellcolor{secondbest}80.47 {\footnotesize $\pm$ 0.25 }& \cellcolor{secondbest}80.23 {\footnotesize $\pm$ 0.51 }& \cellcolor{secondbest}77.49 {\footnotesize $\pm$ 0.23 }\\

 Label-\ourst    & \cellcolor{secondbest}81.14 {\footnotesize $\pm$ 0.49 }& \cellcolor{best}82.90 {\footnotesize $\pm$ 0.00}&	\cellcolor{secondbest}82.54 {\footnotesize $\pm$ 0.05}&	\cellcolor{best}82.44 {\footnotesize $\pm$ 0.05}&	\cellcolor{best}82.60 {\footnotesize $\pm$ 0.00}&	\cellcolor{best}81.10 {\footnotesize $\pm$ 0.00}&	\cellcolor{best}74.98 {\footnotesize $\pm$ 0.11 }\\

\midrule
\rowcolor{gray!8}\multicolumn{8}{c}{\textit{CiteSeer}~\citep{citeseer}}\\
\midrule

SGC & 71.88 {\footnotesize $\pm$ 0.27 }& \cellcolor{secondbest}72.55 {\footnotesize $\pm$ 0.25 }& 72.53 {\footnotesize $\pm$ 0.15 }& 72.07 {\footnotesize $\pm$ 0.21 }& 69.83 {\footnotesize $\pm$ 0.20 }& 65.42 {\footnotesize $\pm$ 0.43 }& 54.69 {\footnotesize $\pm$ 0.98} \\

% +LayerNorm~\cite{layernorm} &66.92 {\footnotesize $\pm$ 7.97 }& 65.78 {\footnotesize $\pm$ 2.22 }& 65.82 {\footnotesize $\pm$ 2.39 }& 64.83 {\footnotesize $\pm$ 1.83 }& 62.96 {\footnotesize $\pm$ 2.52 }& 56.67 {\footnotesize $\pm$ 6.55 }& 48.87 {\footnotesize $\pm$ 7.06} \\
+BatchNorm &60.85 {\footnotesize $\pm$ 0.09 }& 60.45 {\footnotesize $\pm$ 0.07 }& 61.74 {\footnotesize $\pm$ 0.27 }& 63.29 {\footnotesize $\pm$ 0.18 }& 63.71 {\footnotesize $\pm$ 0.18 }& 64.28 {\footnotesize $\pm$ 0.27 }& 59.42 {\footnotesize $\pm$ 0.20} \\
 +PairNorm  &70.83 {\footnotesize $\pm$ 0.06 }& 69.68 {\footnotesize $\pm$ 0.32 }& 70.54 {\footnotesize $\pm$ 0.04 }& 69.86 {\footnotesize $\pm$ 0.08 }& 70.51 {\footnotesize $\pm$ 0.07 }& \cellcolor{secondbest}69.86 {\footnotesize $\pm$ 0.06 }& \cellcolor{secondbest}65.22 {\footnotesize $\pm$ 0.16 }\\  
 +ContraNorm  &\cellcolor{best}72.25 {\footnotesize $\pm$ 0.08 }& 71.9 {\footnotesize $\pm$ 0.06 }& 71.52 {\footnotesize $\pm$ 0.04 }& 59.82 {\footnotesize $\pm$ 2.30 }& 52.87 {\footnotesize $\pm$ 1.86 }& 45.93 {\footnotesize $\pm$ 1.40 }& 35.67 {\footnotesize $\pm$ 1.62}\\
+DropEdge & 65.63 {\footnotesize $\pm$ 1.76 }& 51.80 {\footnotesize $\pm$ 4.61 }& 25.36 {\footnotesize $\pm$ 2.54 }& 18.60 {\footnotesize $\pm$ 3.78 }& 16.52 {\footnotesize $\pm$ 3.97 }& 16.49 {\footnotesize $\pm$ 4.03  }& 16.49 {\footnotesize $\pm$ 4.03 }\\
 +Residual & 71.61 {\footnotesize $\pm$ 0.17 }& 72.31 {\footnotesize $\pm$ 0.15  }& \cellcolor{best}72.78 {\footnotesize $\pm$ 0.12 }& \cellcolor{secondbest}72.50 {\footnotesize $\pm$ 0.14 }& \cellcolor{secondbest}71.24 {\footnotesize $\pm$ 0.21 }& 69.85 {\footnotesize $\pm$ 0.22 }& 62.11 {\footnotesize $\pm$ 0.42}\\
\midrule
  Feature-\ourst & 70.63 {\footnotesize $\pm$ 0.52 }& 70.85 {\footnotesize $\pm$ 0.09 }& 70.52 {\footnotesize $\pm$ 0.14 }& 70.76 {\footnotesize $\pm$ 0.22 }& 68.25 {\footnotesize $\pm$ 0.46 }& 67.20 {\footnotesize $\pm$ 1.15 }& 65.12 {\footnotesize $\pm$ 1.95 }\\
 Label-\ourst &\cellcolor{secondbest}72.01 {\footnotesize $\pm$ 0.10 }& \cellcolor{best}72.87 {\footnotesize $\pm$ 0.05 }& \cellcolor{secondbest}72.72 {\footnotesize $\pm$ 0.28 }& \cellcolor{best}73.04 {\footnotesize $\pm$ 0.10 }& \cellcolor{best}72.52 {\footnotesize $\pm$ 0.17 }& \cellcolor{best}72.45 {\footnotesize $\pm$ 0.11 }& \cellcolor{best}70.97 {\footnotesize $\pm$ 0.22 }\\

\midrule
\rowcolor{gray!8}\multicolumn{8}{c}{\textit{PubMed }~\citep{pubmed}}\\
\midrule
SGC &76.99 {\footnotesize $\pm$ 0.38 }& 75.92 {\footnotesize $\pm$ 0.30 }& 76.18 {\footnotesize $\pm$ 0.70 }& 77.13 {\footnotesize $\pm$ 0.34 }&76.09 {\footnotesize $\pm$ 0.43 }& 76.19 {\footnotesize $\pm$ 0.19 }& 70.58 {\footnotesize $\pm$ 0.52 }\\

 % +LayerNorm~\cite{layernorm} &77.67 {\footnotesize $\pm$ 0.40 }& 76.43 {\footnotesize $\pm$ 0.36 }& 76.26 {\footnotesize $\pm$ 0.34 }& 76.27 {\footnotesize $\pm$ 0.41 }& 75.95 {\footnotesize $\pm$ 0.24 }& 74.79 {\footnotesize $\pm$ 0.53 }& 71.77 {\footnotesize $\pm$ 0.45} \\ 
 +BatchNorm &77.15 {\footnotesize $\pm$ 0.09 }& 77.87 {\footnotesize $\pm$ 0.05 }& 78.47 {\footnotesize $\pm$ 0.05 }& 77.90 {\footnotesize $\pm$ 1.10 }& 76.85 {\footnotesize $\pm$ 0.08 }& 74.35 {\footnotesize $\pm$ 0.08 }& 69.61 {\footnotesize $\pm$ 0.08} \\
 +PairNorm &77.69 {\footnotesize $\pm$ 0.26 }& 75.78 {\footnotesize $\pm$ 0.37 }& 75.13 {\footnotesize $\pm$ 0.13 }& 74.75 {\footnotesize $\pm$ 0.33 }& 72.13 {\footnotesize $\pm$ 0.11 }& 69.79 {\footnotesize $\pm$ 0.16 }& 71.75 {\footnotesize $\pm$ 0.51 }\\       
+ContraNorm &\cellcolor{best}79.30 {\footnotesize $\pm$ 0.10 }& 78.69 {\footnotesize $\pm$ 0.07 }& 77.54 {\footnotesize $\pm$ 0.09 }& 73.67 {\footnotesize $\pm$ 0.12 }& 71.37 {\footnotesize $\pm$ 3.15 }& 67.96 {\footnotesize $\pm$ 3.24 }& 65.00 {\footnotesize $\pm$ 4.12 }\\
 +DropEdge & 74.64 {\footnotesize $\pm$ 1.37 }& 69.83 {\footnotesize $\pm$ 3.19 }& 60.28 {\footnotesize $\pm$ 2.70 }& 32.62 {\footnotesize $\pm$ 10.95 }& 33.95 {\footnotesize $\pm$ 10.44 }& 33.95 {\footnotesize $\pm$ 10.44 }& 33.95 {\footnotesize $\pm$ 10.44 }\\
+Residual & 77.40 {\footnotesize $\pm$ 0.06 }& \cellcolor{secondbest}79.30 {\footnotesize $\pm$ 0.10 }& \cellcolor{secondbest}79.83 {\footnotesize $\pm$ 0.09 }& \cellcolor{secondbest}79.44 {\footnotesize $\pm$ 0.09 }& 74.96 {\footnotesize $\pm$ 0.09 }& 71.72 {\footnotesize $\pm$ 0.13 }& 55.57 {\footnotesize $\pm$ 0.21 }\\
\midrule
 Feature-\ourst & 73.99 {\footnotesize $\pm$ 1.44 }& 74.36 {\footnotesize $\pm$ 0.63 }& 75.61 {\footnotesize $\pm$ 0.24 }& 77.09 {\footnotesize $\pm$ 0.35 }& \cellcolor{secondbest}77.41 {\footnotesize $\pm$ 0.21 }& \cellcolor{secondbest}77.10 {\footnotesize $\pm$ 0.36 }& \cellcolor{secondbest}76.87 {\footnotesize $\pm$ 0.49 }\\
 Label-\ourst &\cellcolor{secondbest}78.98 {\footnotesize $\pm$ 0.14 }& \cellcolor{best}80.14 {\footnotesize $\pm$ 0.05 }& \cellcolor{best}80.22 {\footnotesize $\pm$ 0.04 }& \cellcolor{best}80.32 {\footnotesize $\pm$ 0.04 }& \cellcolor{best}80.20 {\footnotesize $\pm$ 0.00 }& \cellcolor{best}79.60 {\footnotesize $\pm$ 0.00 }& \cellcolor{best}73.96 {\footnotesize $\pm$ 0.05} \\
% \midrule

% Arxiv }& SGC \\
%     }& +LayerNorm \\
%     }& PairNorm \\
%     }& ContraNorm \\
%     }& Feature-\ourst \\
%     }& Label-\ourst \\
% % Add more rows as needed
\bottomrule
\end{tabular}
% }

\end{adjustbox}
\label{table: sgc results}
% \vspace{-0.15in}
\end{table}