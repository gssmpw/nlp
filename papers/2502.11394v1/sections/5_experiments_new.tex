
% \vspace{-2ex}
\section{Experiments}
\label{sec: exp_main}


% \jq{almost rewrite}
In this section, we conduct a comprehensive evaluation of
\ours on various benchmark datasets, including both
homophilic and heterophilic graphs. We aim to answer the following three
key research questions: \textbf{RQ1} How does \ours perform in node classification tasks? \textbf{RQ2} How effectively does \ours mitigate oversmoothing? \textbf{RQ3} How sensitive, robust, and scalable is \ours?   

% In this section, we first verify our theoretical insights on the synthetic datasets.
% Then we demonstrate the effectiveness of \ours in three widely used datasets, and then extend the experimental evaluation to one large-scale dataset and two heterophilous datasets. 
% We apply Label-\ours and Feature-\ours in both linear SGC and non-linear GCN. 




% % \textbf{Contextual Stochastic Block Models.} 
% % Following~\cite{sbm_xinyi}, We focus on the CSBM$(N, p, q, \mu_1, \mu_2, \sigma^2 )$.
% % It consists of two classes $\mathcal{C}_1$ and $\mathcal{C}_2$ of nodes of equal size, in total with $N$ nodes. 
% % For any two nodes in the graph, if they are from the same class, they are connected by an edge independently with probability $p$, or if they are from different classes, the probability is $q$. For each node $v \in \mathcal{C}_i, i\in\{1,-1\}$, the initial feature $X_v$ is sampled independently from a Gaussian distribution $\mathcal{N}(\mu_i, {\sigma^2})$, where $\mu_i =\mathcal{C}_i, \sigma = I $. Specially, let $N=200$, $p=0.092$, $q=0.046$, $d=8$.

% \paragraph{Setup}  
% We apply our methods to node classification on the synthetic dataset CSBM$(n, p, q, \mu_1, \mu_2, \sigma^2 )$.
% Specially, let $n=200$, $p=0.092$, $q=0.046$, $d=8$, $\mu_1=1$, $\mu_2=-1$ and $\sigma^2=I$ following~\cite{sbm_xinyi}.
% We used $60\%$, $20\%$ and $20 \%$ random splits for train, valid and test data, respectively.
% Since our theoretical analysis is primarily based on linear propagation, we apply the linear backbone SGC, which removes the non-linear activation function compared to the GCN, to verify our insight.
% We introduce SGC and GCN in Appendix~\ref{app: GNNs} in details.
% % We further apply both SGC and the GCN in the real world dataset.



% \paragraph{Results} 
% The visualization of node features using Label-\ours and Feature-\ours compared to SGC are shown in Figure \ref{fig: sbm overall}. As the number of layers increases, SGC's node features suffer from oversmoothing, causing the two classes to converge and the classification accuracy to drop to $47.50\%$, which is worse than random guessing ($50\%$). However, Label-\ours and Feature-\ours effectively repel nodes from different classes, achieving high accuracy of $80\%$ and $97.5\%$, respectively, even with $300$ layers.


\begin{figure*}[t]
% \captionsetup{font=small}
    \begin{subfigure}{0.69\textwidth}
        \centering
        % \captionsetup{font=small}
        \includegraphics[width=0.99\textwidth]{figures/eval_sgc_layer.pdf}
        \caption{Oversmoothing performance.}
        \label{fig: layer depth}
    \end{subfigure}
    % \quad
    \begin{subfigure}{0.3\textwidth}
        \centering
        % \captionsetup{font=small}
        \includegraphics[width=0.99\textwidth]{figures/eval_train.pdf} % Adjust the path and filename as necessary
        \caption{Training ratio ablation study.}
        \label{fig: train ratio}
    \end{subfigure}
    \caption{Left is the performance comparison of \ours against Normalization GNNs under various model depths where the X-axis has the number of layers, and the Y-axis has node classification accuracy. Right is the ablation study on Label-\ours where the X-axis indicates the ratio of the training node numbers.}
    % \vspace{-0.55cm}
    % \vspace{-0.2in}
\end{figure*}



% \subsection{Real World Benchmark}

% \begin{table}[t]
\centering
% \vspace{-0.15in}
\caption{SGC test accuracy (\%) comparison results. The best results are marked in blue and the second best results are marked in gray on every layer. We run 10 runs and demonstrate the mean $\pm$ std in the table.} % for the seed $0$\~$9$ 
% \resizebox{\textwidth}{!}{
\begin{adjustbox}{width=0.99\textwidth}
\begin{tabular}{lccccccc}
\toprule
 Model             & \#L=2              & \#L=5              & \#L=10             & \#L=20             & \#L=50        & \#L=100    & \#L=300    \\
\midrule
\rowcolor{gray!8}\multicolumn{8}{c}{\textit{Cora}~\citep{cora}}\\
\midrule
% \midrule
% \rowcolor{gray!8}\textit{cora}~\citep{cora}\\
% \midrule
 SGC      & 80.21 {\footnotesize $\pm$ 0.07}& 81.45 {\footnotesize $\pm$ 0.14 }& 81.53 {\footnotesize $\pm$ 0.19 }& 79.53 {\footnotesize $\pm$ 0.14 }& 79.20 {\footnotesize $\pm$ 0.21 }& 76.13 {\footnotesize $\pm$ 0.24 }& 65.64 {\footnotesize $\pm$ 1.15 }\\

 % +LayerNorm~\cite{layernorm}       & 80.07 {\footnotesize $\pm$ 0.22 }& \cellcolor{secondbest}81.60 {\footnotesize $\pm$ 0.22 }& 81.20 {\footnotesize $\pm$ 0.29 }& 79.52 {\footnotesize $\pm$ 0.16  }& 79.21 {\footnotesize $\pm$ 0.23 }& 76.44 {\footnotesize $\pm$ 0.11 }& 68.38 {\footnotesize $\pm$ 0.66}  \\
 +BatchNorm & 77.90 {\footnotesize $\pm$ 0.00 }& 78.02 {\footnotesize $\pm$ 0.04 }& 76.94 {\footnotesize $\pm$ 0.08 }& 75.18 {\footnotesize $\pm$ 0.09 }& 74.54 {\footnotesize $\pm$ 0.05 }& 72.64 {\footnotesize $\pm$ 0.05 }& 63.12 {\footnotesize $\pm$ 0.06} \\
+PairNorm     & 80.30 {\footnotesize $\pm$ 0.05 }& 78.57 {\footnotesize $\pm$ 0.00 }&78.14 {\footnotesize $\pm$ 0.07 }& 76.90 {\footnotesize $\pm$ 0.00 }& 77.49 {\footnotesize $\pm$ 0.03}  & 72.01 {\footnotesize $\pm$ 0.03 }& 40.93 {\footnotesize $\pm$ 0.11} \\
 +ContraNorm      & \cellcolor{best}81.60 {\footnotesize $\pm$ 0.00 }& 80.67 {\footnotesize $\pm$ 0.06 }& 79.11 {\footnotesize $\pm$ 0.03 }& 74.28 {\footnotesize $\pm$ 0.15 }& 69.67 {\footnotesize $\pm$ 1.23 }& 65.58 {\footnotesize $\pm$ 2.11 }&47.21 {\footnotesize $\pm$ 10.80} \\
+DropEdge & 73.58 {\footnotesize $\pm$ 2.76 }& 62.11 {\footnotesize $\pm$ 5.10 }& 39.21 {\footnotesize $\pm$ 7.54 }& 15.07 {\footnotesize $\pm$ 6.22 }& 11.16 {\footnotesize $\pm$ 2.73 }& 11.15 {\footnotesize $\pm$ 2.81 }& 11.15 {\footnotesize $\pm$ 2.81 }\\
 +Residual& 77.81 {\footnotesize $\pm$ 0.03 }& 81.47 {\footnotesize $\pm$ 0.05 }& \cellcolor{best}82.90 {\footnotesize $\pm$ 0.00 }& 79.87 {\footnotesize $\pm$ 0.05 }& 75.64 {\footnotesize $\pm$ 0.05 }& 66.90 {\footnotesize $\pm$ 0.10 }& 25.33 {\footnotesize $\pm$ 0.46}\\
\midrule
 Feature-\ourst &78.10 {\footnotesize $\pm$ 0.11 }& 80.88 {\footnotesize $\pm$ 0.23 }& 80.83 {\footnotesize $\pm$ 0.37 }& \cellcolor{secondbest}82.46 {\footnotesize $\pm$ 0.07 }& \cellcolor{secondbest}80.47 {\footnotesize $\pm$ 0.25 }& \cellcolor{secondbest}80.23 {\footnotesize $\pm$ 0.51 }& \cellcolor{secondbest}77.49 {\footnotesize $\pm$ 0.23 }\\

 Label-\ourst    & \cellcolor{secondbest}81.14 {\footnotesize $\pm$ 0.49 }& \cellcolor{best}82.90 {\footnotesize $\pm$ 0.00}&	\cellcolor{secondbest}82.54 {\footnotesize $\pm$ 0.05}&	\cellcolor{best}82.44 {\footnotesize $\pm$ 0.05}&	\cellcolor{best}82.60 {\footnotesize $\pm$ 0.00}&	\cellcolor{best}81.10 {\footnotesize $\pm$ 0.00}&	\cellcolor{best}74.98 {\footnotesize $\pm$ 0.11 }\\

\midrule
\rowcolor{gray!8}\multicolumn{8}{c}{\textit{CiteSeer}~\citep{citeseer}}\\
\midrule

SGC & 71.88 {\footnotesize $\pm$ 0.27 }& \cellcolor{secondbest}72.55 {\footnotesize $\pm$ 0.25 }& 72.53 {\footnotesize $\pm$ 0.15 }& 72.07 {\footnotesize $\pm$ 0.21 }& 69.83 {\footnotesize $\pm$ 0.20 }& 65.42 {\footnotesize $\pm$ 0.43 }& 54.69 {\footnotesize $\pm$ 0.98} \\

% +LayerNorm~\cite{layernorm} &66.92 {\footnotesize $\pm$ 7.97 }& 65.78 {\footnotesize $\pm$ 2.22 }& 65.82 {\footnotesize $\pm$ 2.39 }& 64.83 {\footnotesize $\pm$ 1.83 }& 62.96 {\footnotesize $\pm$ 2.52 }& 56.67 {\footnotesize $\pm$ 6.55 }& 48.87 {\footnotesize $\pm$ 7.06} \\
+BatchNorm &60.85 {\footnotesize $\pm$ 0.09 }& 60.45 {\footnotesize $\pm$ 0.07 }& 61.74 {\footnotesize $\pm$ 0.27 }& 63.29 {\footnotesize $\pm$ 0.18 }& 63.71 {\footnotesize $\pm$ 0.18 }& 64.28 {\footnotesize $\pm$ 0.27 }& 59.42 {\footnotesize $\pm$ 0.20} \\
 +PairNorm  &70.83 {\footnotesize $\pm$ 0.06 }& 69.68 {\footnotesize $\pm$ 0.32 }& 70.54 {\footnotesize $\pm$ 0.04 }& 69.86 {\footnotesize $\pm$ 0.08 }& 70.51 {\footnotesize $\pm$ 0.07 }& \cellcolor{secondbest}69.86 {\footnotesize $\pm$ 0.06 }& \cellcolor{secondbest}65.22 {\footnotesize $\pm$ 0.16 }\\  
 +ContraNorm  &\cellcolor{best}72.25 {\footnotesize $\pm$ 0.08 }& 71.9 {\footnotesize $\pm$ 0.06 }& 71.52 {\footnotesize $\pm$ 0.04 }& 59.82 {\footnotesize $\pm$ 2.30 }& 52.87 {\footnotesize $\pm$ 1.86 }& 45.93 {\footnotesize $\pm$ 1.40 }& 35.67 {\footnotesize $\pm$ 1.62}\\
+DropEdge & 65.63 {\footnotesize $\pm$ 1.76 }& 51.80 {\footnotesize $\pm$ 4.61 }& 25.36 {\footnotesize $\pm$ 2.54 }& 18.60 {\footnotesize $\pm$ 3.78 }& 16.52 {\footnotesize $\pm$ 3.97 }& 16.49 {\footnotesize $\pm$ 4.03  }& 16.49 {\footnotesize $\pm$ 4.03 }\\
 +Residual & 71.61 {\footnotesize $\pm$ 0.17 }& 72.31 {\footnotesize $\pm$ 0.15  }& \cellcolor{best}72.78 {\footnotesize $\pm$ 0.12 }& \cellcolor{secondbest}72.50 {\footnotesize $\pm$ 0.14 }& \cellcolor{secondbest}71.24 {\footnotesize $\pm$ 0.21 }& 69.85 {\footnotesize $\pm$ 0.22 }& 62.11 {\footnotesize $\pm$ 0.42}\\
\midrule
  Feature-\ourst & 70.63 {\footnotesize $\pm$ 0.52 }& 70.85 {\footnotesize $\pm$ 0.09 }& 70.52 {\footnotesize $\pm$ 0.14 }& 70.76 {\footnotesize $\pm$ 0.22 }& 68.25 {\footnotesize $\pm$ 0.46 }& 67.20 {\footnotesize $\pm$ 1.15 }& 65.12 {\footnotesize $\pm$ 1.95 }\\
 Label-\ourst &\cellcolor{secondbest}72.01 {\footnotesize $\pm$ 0.10 }& \cellcolor{best}72.87 {\footnotesize $\pm$ 0.05 }& \cellcolor{secondbest}72.72 {\footnotesize $\pm$ 0.28 }& \cellcolor{best}73.04 {\footnotesize $\pm$ 0.10 }& \cellcolor{best}72.52 {\footnotesize $\pm$ 0.17 }& \cellcolor{best}72.45 {\footnotesize $\pm$ 0.11 }& \cellcolor{best}70.97 {\footnotesize $\pm$ 0.22 }\\

\midrule
\rowcolor{gray!8}\multicolumn{8}{c}{\textit{PubMed }~\citep{pubmed}}\\
\midrule
SGC &76.99 {\footnotesize $\pm$ 0.38 }& 75.92 {\footnotesize $\pm$ 0.30 }& 76.18 {\footnotesize $\pm$ 0.70 }& 77.13 {\footnotesize $\pm$ 0.34 }&76.09 {\footnotesize $\pm$ 0.43 }& 76.19 {\footnotesize $\pm$ 0.19 }& 70.58 {\footnotesize $\pm$ 0.52 }\\

 % +LayerNorm~\cite{layernorm} &77.67 {\footnotesize $\pm$ 0.40 }& 76.43 {\footnotesize $\pm$ 0.36 }& 76.26 {\footnotesize $\pm$ 0.34 }& 76.27 {\footnotesize $\pm$ 0.41 }& 75.95 {\footnotesize $\pm$ 0.24 }& 74.79 {\footnotesize $\pm$ 0.53 }& 71.77 {\footnotesize $\pm$ 0.45} \\ 
 +BatchNorm &77.15 {\footnotesize $\pm$ 0.09 }& 77.87 {\footnotesize $\pm$ 0.05 }& 78.47 {\footnotesize $\pm$ 0.05 }& 77.90 {\footnotesize $\pm$ 1.10 }& 76.85 {\footnotesize $\pm$ 0.08 }& 74.35 {\footnotesize $\pm$ 0.08 }& 69.61 {\footnotesize $\pm$ 0.08} \\
 +PairNorm &77.69 {\footnotesize $\pm$ 0.26 }& 75.78 {\footnotesize $\pm$ 0.37 }& 75.13 {\footnotesize $\pm$ 0.13 }& 74.75 {\footnotesize $\pm$ 0.33 }& 72.13 {\footnotesize $\pm$ 0.11 }& 69.79 {\footnotesize $\pm$ 0.16 }& 71.75 {\footnotesize $\pm$ 0.51 }\\       
+ContraNorm &\cellcolor{best}79.30 {\footnotesize $\pm$ 0.10 }& 78.69 {\footnotesize $\pm$ 0.07 }& 77.54 {\footnotesize $\pm$ 0.09 }& 73.67 {\footnotesize $\pm$ 0.12 }& 71.37 {\footnotesize $\pm$ 3.15 }& 67.96 {\footnotesize $\pm$ 3.24 }& 65.00 {\footnotesize $\pm$ 4.12 }\\
 +DropEdge & 74.64 {\footnotesize $\pm$ 1.37 }& 69.83 {\footnotesize $\pm$ 3.19 }& 60.28 {\footnotesize $\pm$ 2.70 }& 32.62 {\footnotesize $\pm$ 10.95 }& 33.95 {\footnotesize $\pm$ 10.44 }& 33.95 {\footnotesize $\pm$ 10.44 }& 33.95 {\footnotesize $\pm$ 10.44 }\\
+Residual & 77.40 {\footnotesize $\pm$ 0.06 }& \cellcolor{secondbest}79.30 {\footnotesize $\pm$ 0.10 }& \cellcolor{secondbest}79.83 {\footnotesize $\pm$ 0.09 }& \cellcolor{secondbest}79.44 {\footnotesize $\pm$ 0.09 }& 74.96 {\footnotesize $\pm$ 0.09 }& 71.72 {\footnotesize $\pm$ 0.13 }& 55.57 {\footnotesize $\pm$ 0.21 }\\
\midrule
 Feature-\ourst & 73.99 {\footnotesize $\pm$ 1.44 }& 74.36 {\footnotesize $\pm$ 0.63 }& 75.61 {\footnotesize $\pm$ 0.24 }& 77.09 {\footnotesize $\pm$ 0.35 }& \cellcolor{secondbest}77.41 {\footnotesize $\pm$ 0.21 }& \cellcolor{secondbest}77.10 {\footnotesize $\pm$ 0.36 }& \cellcolor{secondbest}76.87 {\footnotesize $\pm$ 0.49 }\\
 Label-\ourst &\cellcolor{secondbest}78.98 {\footnotesize $\pm$ 0.14 }& \cellcolor{best}80.14 {\footnotesize $\pm$ 0.05 }& \cellcolor{best}80.22 {\footnotesize $\pm$ 0.04 }& \cellcolor{best}80.32 {\footnotesize $\pm$ 0.04 }& \cellcolor{best}80.20 {\footnotesize $\pm$ 0.00 }& \cellcolor{best}79.60 {\footnotesize $\pm$ 0.00 }& \cellcolor{best}73.96 {\footnotesize $\pm$ 0.05} \\
% \midrule

% Arxiv }& SGC \\
%     }& +LayerNorm \\
%     }& PairNorm \\
%     }& ContraNorm \\
%     }& Feature-\ourst \\
%     }& Label-\ourst \\
% % Add more rows as needed
\bottomrule
\end{tabular}
% }

\end{adjustbox}
\label{table: sgc results}
% \vspace{-0.15in}
\end{table}
% \vspace{-0.1in}
% \begin{table}[t]
\vspace{-0.1in}
\centering
\small
\caption{GCN test accuracy (\%) comparison results on heterophilic datasets. The best results are marked in blue and the second best results are marked underline on every layer.
We run 5 runs and demonstrate the mean $\pm$ std in the table.}%for the seed from $0~4$
\begin{adjustbox}{width=0.91\textwidth}
\begin{tabular}{lcccccc}
\toprule
 Model             & \#L=2              & \#L=4              & \#L=8              & \#L=16             & \#L=32             & \#L=64\\

\midrule
\rowcolor{gray!8}\multicolumn{7}{c}{\textit{Chameleon}~\cite{heter_dataset}}\\
\midrule
   GCN & 66.01{\footnotesize$\pm$0.72} & 54.21{\footnotesize$\pm$0.53} & 35.48{\footnotesize$\pm$3.09} & 22.37{\footnotesize$\pm$0.00} & 22.37{\footnotesize$\pm$0.00} & 22.37{\footnotesize$\pm$0.00} \\
    +BatchNorm & \underline{65.83{\footnotesize$\pm$0.58}} & 56.40{\footnotesize$\pm$0.35} & 36.36{\footnotesize$\pm$2.04} & 22.37{\footnotesize$\pm$0.00} & 22.37{\footnotesize$\pm$0.00} & 22.37{\footnotesize$\pm$0.00}\\
    +PairNorm & 66.01{\footnotesize$\pm$0.72} & 54.12{\footnotesize$\pm$0.79} & 36.75{\footnotesize$\pm$0.38} & 22.37{\footnotesize$\pm$0.00} & 22.37{\footnotesize$\pm$0.00} & 22.37{\footnotesize$\pm$0.00}\\
    +ContraNorm & 66.01{\footnotesize$\pm$0.72} & 58.16{\footnotesize$\pm$1.76} & 37.15{\footnotesize$\pm$4.91} & 22.37{\footnotesize$\pm$0.00} & 22.37{\footnotesize$\pm$0.00} & 22.37{\footnotesize$\pm$0.00}\\
    +DropEdge & 62.50{\footnotesize$\pm$0.00} & 53.07{\footnotesize$\pm$1.61} & 32.15{\footnotesize$\pm$1.49} & 21.71{\footnotesize$\pm$0.00} & \underline{27.19{\footnotesize$\pm$1.75}} & \underline{23.68{\footnotesize$\pm$0.00}} \\
    +Residual & 66.01{\footnotesize$\pm$0.72} & \cellcolor{best}62.94{\footnotesize$\pm$0.00} & \underline{57.59{\footnotesize$\pm$2.58}} & \underline{41.27{\footnotesize$\pm$0.32}} & 22.37{\footnotesize$\pm$0.00} & 22.37{\footnotesize$\pm$0.00} \\
\midrule
    Feature-\ourst & 62.98{\footnotesize$\pm$0.75} & \underline{62.89{\footnotesize$\pm$1.29}} & \cellcolor{best}65.35{\footnotesize$\pm$0.00} & \cellcolor{best}62.28{\footnotesize$\pm$0.00} & \cellcolor{best}55.31{\footnotesize$\pm$0.53} & \cellcolor{best}35.13{\footnotesize$\pm$0.93}\\
    Label-\ourst & \cellcolor{best}66.01{\footnotesize$\pm$0.72} & 57.11{\footnotesize$\pm$0.11} & 38.11{\footnotesize$\pm$1.87} & 22.37{\footnotesize$\pm$0.00} & 22.37{\footnotesize$\pm$0.00} & 22.37{\footnotesize$\pm$0.00}\\
\midrule
\rowcolor{gray!8}\multicolumn{7}{c}{\textit{Squirrel}~\cite{heter_dataset}}\\
\midrule
   GCN  & 42.38{\footnotesize$\pm$0.04} & 32.20{\footnotesize$\pm$3.05} & 22.57{\footnotesize$\pm$0.00} & 20.46{\footnotesize$\pm$0.00} & 20.46{\footnotesize$\pm$0.00} & 20.46{\footnotesize$\pm$0.00}\\
    +BatchNorm & 41.77{\footnotesize$\pm$0.35} & 32.37{\footnotesize$\pm$3.46} & 22.67{\footnotesize$\pm$0.00} & 20.46{\footnotesize$\pm$0.00} & 20.46{\footnotesize$\pm$0.00} & 20.46{\footnotesize$\pm$0.00}\\
    +PairNorm & 42.75{\footnotesize$\pm$0.00} & 32.12{\footnotesize$\pm$3.00} & 22.57{\footnotesize$\pm$0.00} & 20.46{\footnotesize$\pm$0.00} & 20.46{\footnotesize$\pm$0.00} & 20.46{\footnotesize$\pm$0.00}\\
    +ContraNorm & \underline{43.78{\footnotesize$\pm$1.08}} & 32.80{\footnotesize$\pm$3.76} & 22.57{\footnotesize$\pm$0.00} & 20.46{\footnotesize$\pm$0.00} & 20.46{\footnotesize$\pm$0.00} & 20.46{\footnotesize$\pm$0.00} \\
    +DropEdge & 40.54{\footnotesize$\pm$0.00} & 22.57{\footnotesize$\pm$0.00} & 22.77{\footnotesize$\pm$2.12} & 22.19{\footnotesize$\pm$0.58} & \underline{22.61{\footnotesize$\pm$1.36}} & 20.46{\footnotesize$\pm$0.00}\\
    +Residual & 41.92{\footnotesize$\pm$0.65} & \underline{42.23{\footnotesize$\pm$0.08}} & \underline{39.15{\footnotesize$\pm$0.07}} & \underline{33.41{\footnotesize$\pm$2.73}} & 20.46{\footnotesize$\pm$0.00} & 20.46{\footnotesize$\pm$0.00} \\
\midrule
    Feature-\ourst& \cellcolor{best}44.48{\footnotesize$\pm$0.00} &\cellcolor{best} 45.01{\footnotesize$\pm$0.72} &\cellcolor{best} 44.03{\footnotesize$\pm$0.63} &\cellcolor{best} 41.42{\footnotesize$\pm$0.78} &\cellcolor{best} 36.79{\footnotesize$\pm$0.00} &\cellcolor{best} 29.20{\footnotesize$\pm$0.00}\\
    Label-\ourst& 43.61{\footnotesize$\pm$0.58} & 32.78{\footnotesize$\pm$3.49} & 22.79{\footnotesize$\pm$0.09} & 20.46{\footnotesize$\pm$0.00} & 20.46{\footnotesize$\pm$0.00} & \underline{20.46{\footnotesize$\pm$0.00}} \\

\bottomrule
\end{tabular}
\end{adjustbox}
\label{table: gcn heter}
\vspace{-0.1in}
\end{table}

% \begin{table}[h]
% \centering
% \caption{GCN test accuracy (\%) comparison results. The best results are marked in blue and the second best results are marked in gray on every layer.}
% \begin{adjustbox}{width=0.99\textwidth}
% \begin{tabular}{lcccccc}
% \toprule
%  Model             & \#L=2              & \#L=4              & \#L=8              & \#L=16             & \#L=32             & \#L=64\\

% \midrule
% \rowcolor{gray!8}\multicolumn{7}{c}{\textit{Cora}~\citep{cora}}\\
% \midrule
%   GCN~\cite{gcn} & \cellcolor{secondbest}80.68 $\pm$ 0.09 & \cellcolor{secondbest}79.69 $\pm$ 0.00 & 74.32 $\pm$ 0.00 & 30.95 $\pm$ 0.00 & 30.95 $\pm$ 0.00 & 24.85 $\pm$ 7.46 \\


% % & Center & 79.85 $\pm$ 0.46 & 77.32 $\pm$ 1.33 & 75.25 $\pm$ 0.40 & 57.75 $\pm$ 3.53 & 41.73 $\pm$ 0.92 & 39.34 $\pm$ 2.29 \\
%     LayerNorm~\cite{layernorm} & 80.51 $\pm$ 0.12 & \cellcolor{best}80.28 $\pm$ 0.66 & 75.05 $\pm$ 0.00 & 30.95 $\pm$ 0.00 & 30.95 $\pm$ 0.00 & 24.85 $\pm$ 7.46 \\
%     BatchNorm~\cite{batchnorm} & 78.09 $\pm$ 0.00 & 77.87 $\pm$ 0.02 & 73.62 $\pm$ 0.57 & 70.79 $\pm$ 0.00 & 53.90 $\pm$ 2.19 & 35.32 $\pm$ 3.41\\
%     PairNorm~\cite{pairnorm} & 79.01 $\pm$ 0.00 & 78.26 $\pm$ 0.50 & 73.21 $\pm$ 0.00 & 62.96 $\pm$ 0.00 & 48.13 $\pm$ 0.91 & 44.01 $\pm$ 3.46 \\
%     ContraNorm~\cite{contranorm} & \cellcolor{best}81.55 $\pm$ 0.21 & 79.61 $\pm$ 0.75 & 77.71 $\pm$ 0.00 & 63.35 $\pm$ 0.00 & 44.56 $\pm$ 4.83 & 38.97 $\pm$ 0.00 \\
%     DropEdge~\cite{dropedge} & 78.38 $\pm$ 0.00 & 74.47 $\pm$ 0.00 & 26.91 $\pm$ 0.83 & 22.24 $\pm$ 3.04 & 27.18 $\pm$ 0.00 & 25.98 $\pm$ 6.00\\
%     Residual& 80.68 $\pm$ 0.09 & 78.77 $\pm$ 0.00 & \cellcolor{secondbest}79.26 $\pm$ 0.21 & 40.91 $\pm$ 0.00 & 30.95 $\pm$ 0.00 & 27.90 $\pm$ 6.09\\
% \midrule
%      \ourst-Feature & 80.44 $\pm$ 0.83 & 79.26 $\pm$ 1.18 &  78.56 $\pm$ 0.59 & \cellcolor{secondbest} 77.22 $\pm$ 0.55 &\cellcolor{secondbest} 73.65 $\pm$ 0.48 &\cellcolor{secondbest} 61.62 $\pm$ 5.24\\
%      \ourst-Label &80.31 $\pm$ 0.70 & 79.16 $\pm$ 1.30 & \cellcolor{best}79.50 $\pm$ 0.00 & \cellcolor{best}77.43 $\pm$ 1.49 & \cellcolor{best}74.52 $\pm$ 0.36 & \cellcolor{best}65.02 $\pm$ 2.97 \\
% \midrule
% \rowcolor{gray!8}\multicolumn{7}{c}{\textit{CiteSeer}~\citep{citeseer}}\\
% \midrule
%    GCN~\cite{gcn} &\cellcolor{best} 67.45 $\pm$ 0.54 & 65.62 $\pm$ 0.25 & 37.22 $\pm$ 2.46 & 22.03 $\pm$ 4.76 & 19.65 $\pm$ 0.00 & 19.65 $\pm$ 0.00 \\


% % & Center & 67.21 $\pm$ 0.64 & 65.50 $\pm$ 0.99 & 59.25 $\pm$ 3.18 & 40.29 $\pm$ 1.18 & 41.73 $\pm$ 0.92 & 35.81 \pm 1.21\\
%     LayerNorm~\cite{layernorm} & 67.24 $\pm$ 0.66 &64.95 $\pm$ 0.72 & 38.87 $\pm$ 4.12 & 24.29 $\pm$ 5.68 & 19.65 $\pm$ 0.00 & 19.65 $\pm$ 0.00 \\
%      BatchNorm~\cite{batchnorm} &63.44 $\pm$ 0.94 & 62.34 $\pm$ 0.25 & 61.36 $\pm$ 0.00 & 50.58 $\pm$ 1.24 & 41.41 $\pm$ 0.00 & 35.00 $\pm$ 1.09 \\
%     PairNorm~\cite{pairnorm} & 63.58 $\pm$ 0.63 & 64.32 $\pm$ 0.95 & 61.95 $\pm$ 1.24 & 50.06 $\pm$ 0.00 & 37.21 $\pm$ 1.87 & 36.09 $\pm$ 0.07 \\
%     ContraNorm~\cite{contranorm} & 66.83 $\pm$ 0.49 & 64.78 $\pm$ 0.92 & 60.70 $\pm$ 0.60 & 44.79 $\pm$ 1.65 & 37.36 $\pm$ 0.25 & 30.85 $\pm$ 0.81 \\
%     DropEdge~\cite{dropedge} & 63.86 $\pm$ 0.03 & 62.24 $\pm$ 0.90 & 24.73 $\pm$ 5.72 & 20.65 $\pm$ 0.00 & 20.04 $\pm$ 0.19 & 19.95 $\pm$ 0.09\\
%     Residual & \cellcolor{secondbest} 67.45 $\pm$ 0.54 & 66.21 $\pm$ 0.16 & \cellcolor{best}67.34 $\pm$ 0.00 & 33.21 $\pm$ 0.00 & 19.65 $\pm$ 0.00 & 19.65 $\pm$ 0.00 \\
% \midrule
%     \ourst-Feature &  67.38 $\pm$ 0.66 & \cellcolor{best}66.94 $\pm$ 0.00 & 66.29 $\pm$ 0.02 & \cellcolor{secondbest}65.35 $\pm$ 1.99 & \cellcolor{best}61.43 $\pm$ 0.00 & \cellcolor{secondbest}42.09 $\pm$ 1.65\\
%      \ourst-Label & 67.23 $\pm$ 0.64 & \cellcolor{secondbest} 66.72 $\pm$ 0.00 & \cellcolor{secondbest}66.29 $\pm$ 0.89 & \cellcolor{best}65.50 $\pm$ 2.13 & \cellcolor{secondbest}59.93 $\pm$ 0.85 & \cellcolor{best}44.41 $\pm$ 1.57 \\
% \midrule
% \rowcolor{gray!8}\multicolumn{7}{c}{\textit{PubMed}~\citep{pubmed}}\\
% \midrule
%    GCN~\cite{gcn} & \cellcolor{best}76.44 $\pm$ 0.34 & 76.52 $\pm$ 0.32 & 69.58 $\pm$ 5.89 & 39.92 $\pm$ 0.00 & 39.92 $\pm$ 0.00 & 39.92 $\pm$ 0.00 \\


% % & Center & 75.19 $\pm$0.26	& 76.67 \pm	0.00 &OOM &OOM &OOM & OOM\\
%     LayerNorm~\cite{layernorm} & 76.27 $\pm$ 0.51 & 76.71 $\pm$ 0.24 & 76.95 $\pm$ 0.17 & 39.92 $\pm$ 0.00 & 39.92 $\pm$ 0.00 & 39.92 $\pm$ 0.00 \\
%     BatchNorm~\cite{batchnorm} & 75.52 $\pm$ 0.12 & \cellcolor{secondbest}77.15 $\pm$ 0.00 & 77.10 $\pm$ 0.00 & 76.92 $\pm$ 0.00 & 75.43 $\pm$ 0.00 & 69.33 $\pm$ 1.01 \\
%     PairNorm~\cite{pairnorm} & 75.66 $\pm$ 0.11 & 76.71 $\pm$ 0.00 & \cellcolor{secondbest}77.99 $\pm$ 0.00 & \cellcolor{secondbest}77.22 $\pm$ 0.39 & 75.52 $\pm$ 2.02 & 71.22 $\pm$ 3.68 \\
%     ContraNorm~\cite{contranorm} & 76.05 $\pm$ 0.33 & \cellcolor{best}78.42 $\pm$ 0.00 & OOM & OOM & OOM & OOM \\
%     DropEdge~\cite{dropedge}& 73.41 $\pm$ 0.03 & 73.96 $\pm$ 0.79 & 52.51 $\pm$ 10.91 & 40.27 $\pm$ 0.00 & 39.90 $\pm$ 0.59 & 40.08 $\pm$ 0.39 \\
%     Residual & \cellcolor{secondbest} 76.44 $\pm$ 0.34 & 77.28 $\pm$ 0.00 & 77.38 $\pm$ 0.00 & 63.14 $\pm$ 3.05 & 39.92 $\pm$ 0.00 & 39.92 $\pm$ 0.00 \\
% \midrule
%     \ourst-Feature & 75.72 $\pm$ 0.06 & 76.84 $\pm$ 0.00 & \cellcolor{best}78.39 $\pm$ 0.00 &\cellcolor{best} 79.71 $\pm$ 0.00 & \cellcolor{best}77.59 $\pm$ 0.23 & \cellcolor{best}78.06 $\pm$ 0.13\\
%     \ourst-Label & 76.33 $\pm$ 0.25 & 76.91 $\pm$ 0.00 & 77.60 $\pm$ 0.49 & 76.31 $\pm$ 0.00 & \cellcolor{secondbest}77.17 $\pm$ 0.67 & \cellcolor{secondbest}78.01 $\pm$ 0.16\\
% % Add more rows as needed
% % \midrule

% % Arxiv & GCN & 70.20 $\pm$ 0.36 & 70.84 $\pm$ 0.12 & 69.73 $\pm$ 0.28\\
% %     & APPNP \\
% %     & Center \\
% %     & LayerNorm \\
% %     & PairNorm \\
% %     & ContraNorm \\
% %     & \ourst-Feature \\
% %     & \ourst-Label \\
% \bottomrule
% \end{tabular}
% \end{adjustbox}
% \end{table}
\textbf{Datasets.} 
% \section{Dataset}
\label{sec:dataset}

\subsection{Data Collection}

To analyze political discussions on Discord, we followed the methodology in \cite{singh2024Cross-Platform}, collecting messages from politically-oriented public servers in compliance with Discord's platform policies.

Using Discord's Discovery feature, we employed a web scraper to extract server invitation links, names, and descriptions, focusing on public servers accessible without participation. Invitation links were used to access data via the Discord API. To ensure relevance, we filtered servers using keywords related to the 2024 U.S. elections (e.g., Trump, Kamala, MAGA), as outlined in \cite{balasubramanian2024publicdatasettrackingsocial}. This resulted in 302 server links, further narrowed to 81 English-speaking, politics-focused servers based on their names and descriptions.

Public messages were retrieved from these servers using the Discord API, collecting metadata such as \textit{content}, \textit{user ID}, \textit{username}, \textit{timestamp}, \textit{bot flag}, \textit{mentions}, and \textit{interactions}. Through this process, we gathered \textbf{33,373,229 messages} from \textbf{82,109 users} across \textbf{81 servers}, including \textbf{1,912,750 messages} from \textbf{633 bots}. Data collection occurred between November 13th and 15th, covering messages sent from January 1st to November 12th, just after the 2024 U.S. election.

\subsection{Characterizing the Political Spectrum}
\label{sec:timeline}

A key aspect of our research is distinguishing between Republican- and Democratic-aligned Discord servers. To categorize their political alignment, we relied on server names and self-descriptions, which often include rules, community guidelines, and references to key ideologies or figures. Each server's name and description were manually reviewed based on predefined, objective criteria, focusing on explicit political themes or mentions of prominent figures. This process allowed us to classify servers into three categories, ensuring a systematic and unbiased alignment determination.

\begin{itemize}
    \item \textbf{Republican-aligned}: Servers referencing Republican and right-wing and ideologies, movements, or figures (e.g., MAGA, Conservative, Traditional, Trump).  
    \item \textbf{Democratic-aligned}: Servers mentioning Democratic and left-wing ideologies, movements, or figures (e.g., Progressive, Liberal, Socialist, Biden, Kamala).  
    \item \textbf{Unaligned}: Servers with no defined spectrum and ideologies or opened to general political debate from all orientations.
\end{itemize}

To ensure the reliability and consistency of our classification, three independent reviewers assessed the classification following the specified set of criteria. The inter-rater agreement of their classifications was evaluated using Fleiss' Kappa \cite{fleiss1971measuring}, with a resulting Kappa value of \( 0.8191 \), indicating an almost perfect agreement among the reviewers. Disagreements were resolved by adopting the majority classification, as there were no instances where a server received different classifications from all three reviewers. This process guaranteed the consistency and accuracy of the final categorization.

Through this process, we identified \textbf{7 Republican-aligned servers}, \textbf{9 Democratic-aligned servers}, and \textbf{65 unaligned servers}.

Table \ref{tab:statistics} shows the statistics of the collected data. Notably, while Democratic- and Republican-aligned servers had a comparable number of user messages, users in the latter servers were significantly more active, posting more than double the number of messages per user compared to their Democratic counterparts. 
This suggests that, in our sample, Democratic-aligned servers attract more users, but these users were less engaged in text-based discussions. Additionally, around 10\% of the messages across all server categories were posted by bots. 

\subsection{Temporal Data} 

Throughout this paper, we refer to the election candidates using the names adopted by their respective campaigns: \textit{Kamala}, \textit{Biden}, and \textit{Trump}. To examine how the content of text messages evolves based on the political alignment of servers, we divided the 2024 election year into three periods: \textbf{Biden vs Trump} (January 1 to July 21), \textbf{Kamala vs Trump} (July 21 to September 20), and the \textbf{Voting Period} (after September 20). These periods reflect key phases of the election: the early campaign dominated by Biden and Trump, the shift in dynamics with Kamala Harris replacing Joe Biden as the Democratic candidate, and the final voting stage focused on electoral outcomes and their implications. This segmentation enables an analysis of how discourse responds to pivotal electoral moments.

Figure \ref{fig:line-plot} illustrates the distribution of messages over time, highlighting trends in total messages volume and mentions of each candidate. Prior to Biden's withdrawal on July 21, mentions of Biden and Trump were relatively balanced. However, following Kamala's entry into the race, mentions of Trump surged significantly, a trend further amplified by an assassination attempt on him, solidifying his dominance in the discourse. The only instance where Trump’s mentions were exceeded occurred during the first debate, as concerns about Biden’s age and cognitive abilities temporarily shifted the focus. In the final stages of the election, mentions of all three candidates rose, with Trump’s mentions peaking as he emerged as the victor.
We use nine widely-used node classification
benchmark datasets (Table~\ref{tab: main_data}), where four of them are heterophilic (Texas, Wisconsin, Cornell, Squirrel, and
Amazon-rating~\citep{platonov2023critical}), and the remaining four are homophilic (Cora~\citep{cora},
Citeseer~\citep{citeseer}, and Pubmed~\citep{pubmed}) including one large-scale dataset (Ogbn-Arxiv~\cite{hu2020ogb}). 
Further information about the datasets and splits are provided in Appendix~\ref{app: exp}.
% and experimental results on more datasets
% In line with prior research, we employ the default training/validation/test splits provided by Pytorch Geometric (PyG). 
% For details of the datasets, including their sources and construction methods.
% We evaluate \ours for the semi-supervised node classification on Cora~\cite{cora}, CiteSeer~\cite{citeseer} and PubMed~\cite{pubmed} and one large-scale dataset ogbn-arxiv from OGB benchmarks~\citep{openbenchmark}.
% We also extend our models to two heterophilous datasets: Chameleon and Squirrel~\cite{heter_dataset}.
% We show the details of the dataset in Appendix~\ref{app: data}.




\textbf{Baselines and experiment settings.}
We compare the performance of \ours against the following $12$ baseline models. 
% \begin{itemize}
1) \textbf{Classic models}: MLP, SGC~\citep{sgc}.
2) \textbf{GNNs with normalization}: BatchNorm~\citep{batchnorm}, PairNorm~\citep{pairnorm} and ContraNorm~\citep{contranorm}.
3) \textbf{Augmenation-based GNNs}: DropEdge~\citep{dropedge}.
4) \textbf{GNNs with residual connections}: Residual, APPNP~\citep{appap}, JKNET~\citep{jknet} and DAGNN~\citep{dagnn}. 
5) \textbf{Other baselines}: GCNII~\citep{GCNII} and \(\omega\)GCN~\citep{wGCN}.
% \xw{你忘记cite gcnii 跟wgcn了}
For the sake of fair comparison, we do not deploy specific training techniques used in some prior works for benchmarking.
All models are trained under the same setting on the pure SGC backbone and we choose the best of scale controller in the range of $\{ 0.1, 0.5, 0.9\}$ for ContraNorm, DropEdge, and residual connections.
% For both Label-\ours and Feature-\ours, 
We choose the best of $\lambda$ in the range of $\{0.1, 0.5, 0.9\}$, fix $\alpha=1$ and select the best value for $\beta$ from $\{ 0.1, 0.5, 0.9\}$ for \ours. 
More experiment results with hyperparameter tuning and optimization strategies can be found in Appendix~\ref{app: exp}.
% fix $\alpha=1$ and only select $\beta$ from $\{0.1, 1,10, 20, 50, 100\}$ for simplify .
% \end{itemize}
% For more details of the classic anti-oversmoothing methods seen in Appendix~\ref{xx}.
% We apply both the linear SGC~\cite{sgc} and non-linear GCN~\cite{gcn} backbones.
% For fair comparison, we fix the hidden dimension to $32$ and dropout rate to $0.6$ following \cite{contranorm}.
% % The residual $\alpha=0.5$, the dropedge present is selecting from $\{0.3,0.5,0.7\}$.
% % We choose the best of scale controller $\alpha,\ \beta \in \{ 0.1, 0.2, 0.5, 0.7, 0.9\}$.
% We select the best settings for PairNorm, Residual, DropEdge, and ContraNorm based on their default hyperparameters. 

% We use Tesla-V100-SXM2-32GB in all experiments.
\textbf{RQ1: Node classification performance.}
In Table~\ref{tab: main_data}, we provide the mean of the node classification accuracy along with their corresponding standard deviations across 10 random seeds under the same 2-layer SGC backbone following~\citet{dgc}.
Overall, \ours achieves the best performance across $8$ datasets in the shallow layers, as Label/Feature-\ours performs the best on 7 out of the 8 datasets. 
% In particular, we make the following three observations:
% First, \ours outperforms all normalization methods. 
% Since our theoretical findings suggest that these normalization methods are essentially implicitly signed graph propagation, the theoretical properties of structural balance (Section~\ref{subsec: sb theory}) contribute to the enhanced classification accuracy of \ours.
% Second, \ours outperforms random argumentation based GNNs. Since DropEdge randomly drops edges, it isn't easy to characterize their exact behaviors, but we highlight that it works when it happens to remove edges between different classes of nodes thanks to our structurally balanced theory, as 
% DropEdge still follows the unified signed graph analysis in its message-passing scheme.
% Lastly, \ours outperforms residual connection based GNNs, including the last layer connection: residual and multilayer feature connection: APPNP, JKNET, and DAGNN. 
% In our analysis, GNNs with residual connections can be seen as a special case of signed graph propagation, where their positive and negative adjacency matrices are the linear combination of adjacency matrices of different orders, yet they are not the theoretically best solution to alleviate oversmoothing.
% This validates the effectiveness of our novel insight from a signed graph perspective. 

% \paragraph{Heterophilic datasets}
% Besides the three homophilic datasets, we also conduct experiments on four heterophilic datasets~\citep{heter_dataset}.
% We find that our method is still the most effective one across all of the methods for alleviating oversmoothing as indicated in Table~\ref{table: gcn heter}.
% Interestingly, we observe that Feature-\ours performs better than Label-\ours on the heterophilic datasets, which is the opposite of the results on the homophilic datasets.

\textbf{RQ2: Anti-oversmoothing analysis.}
% \paragraph{Results} 
% The results for SGC are detailed in Table \ref{table: sgc results} and we give the GCN results in Appendix (Table~\ref{table: gcn result}). 
We further evaluate the robustness of \ours by assessing its performance at deeper model depths: $K \in \{2, 10, 50, 100, 300\}$ for homophilic datasets and $K \in \{2, 5, 10, 20, 50\}$ for heterophilic datasets.
% To provide a comparative analysis against other GNNs, we also evaluate two best-performed normalization-based GNNs: BatchNorm and ContraNorm. the performances of these methods
% We evaluate on one heterophilic graph and two homophilic graphs.
Figure~\ref{fig: layer depth} shows that the performance of Feature/Label-\ours remains relatively stable with varying
% ($K = 50$) 
numbers of layers, achieving its best performance when the model gets deeper.
In contrast, the normalization methods considered exhibit a substantial decrease in performance as the number of layers increases, indicating their persistent susceptibility to the oversmoothing problem.
Note that we find that for \ours to maintain performance in the heterophilic dataset, \(\beta\) needs to be larger than the uniform range considered in Figure~\ref{fig: layer depth}. 
See Appendix~\ref{app: exp} for the result under larger \(\beta\), where \ours on deep layers remains $\approx60\%$ in Cornell.   

% \subsection{RQ3: Ablation Study}
\textbf{RQ3.1: Sensitivity analysis of training ratio.}
% Since Label-\ours leverages the ground truth label information to construct the negative graph, we conduct an ablation study examining the impact of different training data ratios. 
As shown in Figure~\ref{fig: train ratio}, Label-\ours's performance on the CSBM and Cora datasets improves as the training ratio increases. Even with a modest training ratio of 20\%, the worst-performing models still achieve an impressive 80\% accuracy, while the best models approach 100\% accuracy when the training ratio is increased to 80\%. This is in line with our theoretical insights that increasing the training ratio leads to more structural balance resulting from our method~\ours. 
% Moreover, our main experiments detailed in Table~\ref{table: sgc results} demonstrate that Label-\ours outperforms other methods, even when adopting the default training set ratios in those datasets, indicating its effectiveness in real-world graph settings.
% % \begin{table}[!t]
% \centering
% \scalebox{0.68}{
%     \begin{tabular}{ll cccc}
%       \toprule
%       & \multicolumn{4}{c}{\textbf{Intellipro Dataset}}\\
%       & \multicolumn{2}{c}{Rank Resume} & \multicolumn{2}{c}{Rank Job} \\
%       \cmidrule(lr){2-3} \cmidrule(lr){4-5} 
%       \textbf{Method}
%       &  Recall@100 & nDCG@100 & Recall@10 & nDCG@10 \\
%       \midrule
%       \confitold{}
%       & 71.28 &34.79 &76.50 &52.57 
%       \\
%       \cmidrule{2-5}
%       \confitsimple{}
%     & 82.53 &48.17
%        & 85.58 &64.91
     
%        \\
%        +\RunnerUpMiningShort{}
%     &85.43 &50.99 &91.38 &71.34 
%       \\
%       +\HyReShort
%         &- & -
%        &-&-\\
       
%       \bottomrule

%     \end{tabular}
%   }
% \caption{Ablation studies using Jina-v2-base as the encoder. ``\confitsimple{}'' refers using a simplified encoder architecture. \framework{} trains \confitsimple{} with \RunnerUpMiningShort{} and \HyReShort{}.}
% \label{tbl:ablation}
% \end{table}
\begin{table*}[!t]
\centering
\scalebox{0.75}{
    \begin{tabular}{l cccc cccc}
      \toprule
      & \multicolumn{4}{c}{\textbf{Recruiting Dataset}}
      & \multicolumn{4}{c}{\textbf{AliYun Dataset}}\\
      & \multicolumn{2}{c}{Rank Resume} & \multicolumn{2}{c}{Rank Job} 
      & \multicolumn{2}{c}{Rank Resume} & \multicolumn{2}{c}{Rank Job}\\
      \cmidrule(lr){2-3} \cmidrule(lr){4-5} 
      \cmidrule(lr){6-7} \cmidrule(lr){8-9} 
      \textbf{Method}
      & Recall@100 & nDCG@100 & Recall@10 & nDCG@10
      & Recall@100 & nDCG@100 & Recall@10 & nDCG@10\\
      \midrule
      \confitold{}
      & 71.28 & 34.79 & 76.50 & 52.57 
      & 87.81 & 65.06 & 72.39 & 56.12
      \\
      \cmidrule{2-9}
      \confitsimple{}
      & 82.53 & 48.17 & 85.58 & 64.91
      & 94.90&78.40 & 78.70& 65.45
       \\
      +\HyReShort{}
       &85.28 & 49.50
       &90.25 & 70.22
       & 96.62&81.99 & \textbf{81.16}& 67.63
       \\
      +\RunnerUpMiningShort{}
       % & 85.14& 49.82
       % &90.75&72.51
       & \textbf{86.13}&\textbf{51.90} & \textbf{94.25}&\textbf{73.32}
       & \textbf{97.07}&\textbf{83.11} & 80.49& \textbf{68.02}
       \\
   %     +\RunnerUpMiningShort{}
   %    & 85.43 & 50.99 & 91.38 & 71.34 
   %    & 96.24 & 82.95 & 80.12 & 66.96
   %    \\
   %    +\HyReShort{} old
   %     &85.28 & 49.50
   %     &90.25 & 70.22
   %     & 96.62&81.99 & 81.16& 67.63
   %     \\
   % +\HyReShort{} 
   %     % & 85.14& 49.82
   %     % &90.75&72.51
   %     & 86.83&51.77 &92.00 &72.04
   %     & 97.07&83.11 & 80.49& 68.02
   %     \\
      \bottomrule

    \end{tabular}
  }
\caption{\framework{} ablation studies. ``\confitsimple{}'' refers using a simplified encoder architecture. \framework{} trains \confitsimple{} with \RunnerUpMiningShort{} and \HyReShort{}. We use Jina-v2-base as the encoder due to its better performance.
}
\label{tbl:ablation}
\end{table*}


% \begin{figure}[t]
%     \centering
%     \begin{subfigure}{0.63\textwidth}
%         \centering
%         \includegraphics[width=0.99\textwidth]{figures/eval_negative (3).pdf} % Adjust the path and filename as necessary
%         \caption{ Significance plot for $\beta$ in terms of test accuracy on cora (left) and texas (right) with fixed $\alpha=1$}
%         \label{fig: beta}
%     \end{subfigure}
%     \quad
%     \begin{subfigure}{0.33\textwidth}
%         \centering
%         \captionsetup{font=small}
%         \includegraphics[width=0.99\textwidth]{figures/eval_train.pdf} % Adjust the path and filename as necessary
%         \caption{Ablation study on Label-SBP. X-axis indicates the ratio of the training node numbers.}
%         \label{}
%     \end{subfigure}
%     \caption{Ablation study}
%     \label{fig: train ratio}
% \end{figure}




% \begin{figure*}[t]
% % \hspace{-10pt}
%     \begin{minipage}{.48\textwidth}
%     \captionof{table}{Node classification accuracy (\%) on the large-scale dataset~\textit{ogbn-arxiv}.}
%     % Test accuracy (\%) comparison results on large scale dataset (Ogbn-ArXiv). The best results are marked in blue and the second best results are marked in gray on every layer.
%     \centering
%     \resizebox{0.99\linewidth}{!}{
%     \begin{tabular}{lcccc}
%     \toprule
%      Model             & \#L=2              & \#L=4              & \#L=8            & \#L=16  \\
%     \midrule
%     GCN & 67.32 {\footnotesize $\pm$ 0.28} & 67.79 {\footnotesize $\pm$ 0.25} & 65.54 {\footnotesize $\pm$ 0.31} & 59.13 {\footnotesize $\pm$ 0.95}  \\
%          BatchNorm & 70.14 {\footnotesize $\pm$ 0.28} & 70.93 {\footnotesize $\pm$ 0.15} & 70.14 {\footnotesize $\pm$ 0.43} & 63.24 {\footnotesize $\pm$ 1.40} \\
%          % +LayerNorm& \cellcolor{secondbest}70.53 {\footnotesize $\pm$ 0.19} & \cellcolor{best}71.66 {\footnotesize $\pm$ 0.17} & \cellcolor{secondbest}71.23 {\footnotesize $\pm$ 0.16} & 68.62 {\footnotesize $\pm$ 0.47} \\
%          PairNorm & 70.48 {\footnotesize $\pm$ 0.20} & \cellcolor{best}71.59 {\footnotesize $\pm$ 0.17} & \cellcolor{best}71.24 {\footnotesize $\pm$ 0.07} & 68.92 {\footnotesize $\pm$ 0.43} \\
%          ContraNorm & OOM & OOM & OOM & OOM \\
%          DropEdge & 64.07 {\footnotesize $\pm$ 0.32} & 63.92 {\footnotesize $\pm$ 0.27} & 60.74 {\footnotesize $\pm$ 0.45} & 52.52 {\footnotesize $\pm$ 0.34} \\
%          Residual & 66.90 {\footnotesize $\pm$ 0.14} & 66.67 {\footnotesize $\pm$ 0.25} & 61.76 {\footnotesize $\pm$ 0.62} & 53.25 {\footnotesize $\pm$ 0.75} \\
%     % \midrule
%          Feature-\ourst & 67.89 {\footnotesize $\pm$ 0.10} & 68.47 {\footnotesize $\pm$ 0.26} & 65.09 {\footnotesize $\pm$ 0.30} & 60.34 {\footnotesize $\pm$ 0.94} \\
%          Label-\ourst & \cellcolor{best}70.55 {\footnotesize $\pm$ 0.22} & 71.54 {\footnotesize $\pm$ 0.18} & 71.07 {\footnotesize $\pm$ 0.28} & \cellcolor{best}69.33 {\footnotesize $\pm$ 0.59}  \\
%     \bottomrule
%     \end{tabular}
%     \label{tab: large}
%     }\hfill
%     \end{minipage}
%    \begin{minipage}{.52\textwidth}
%    % \vspace{0.2cm}
%        \centering
%        \includegraphics[width=\linewidth]{figures/eval_negative (3).pdf}
%        \captionof{figure}{Significance of negative graph weight $\beta$ on Cora and Texas datasets where we fix the positive graph weight $\alpha=1$.}
%        \label{fig:beta real} 
%    \end{minipage}
%    % \vspace{-0.5cm}
%     % \hspace{-10pt}
% %     \begin{minipage}{.35\textwidth}
% %     \centering
% %     \captionsetup{font=small}
% %     \caption{Accuracy on different splits of train/valid/test dataset. SGC in CSBM and GCN in Cora.}
% %     % SGC test accuracy (\%) comparison results on sbm of \ourst-Label on different splits of train/valid/test dataset. GCN test accuracy (\%) comparison results on Cora of \ourst-Label on different splits of train/valid/test dataset.
% % % The best results are marked in blue on every dataset.
% %     \centering
% %      \resizebox{0.95\linewidth}{!}{
% %     \begin{tabular}{lcc}
% %     \toprule
% %      Splits & CSBM  & Cora   \\
% %     \midrule
% %     % \midrule
% %     % sbm 0/5/5& 57.00 {\footnotesize $\pm$ 13.36}
% %       2/4/4 & 86.25 {\footnotesize $\pm$ 3.01} & 82.80 {\footnotesize $\pm$ 0.81}\\
% %       4/3/3 & 91.50 {\footnotesize $\pm$ 2.52} & 85.39 {\footnotesize $\pm$ 0.18 }\\
% %       6/4/4 & 91.50 {\footnotesize $\pm$ 6.05} & 87.64 {\footnotesize $\pm$ 0.37 }\\
% %       8/1/1 & \cellcolor{best}99.05 {\footnotesize $\pm$ 1.90} & \cellcolor{best} 94.10 {\footnotesize $\pm$ 0.74} \\
% %     \bottomrule
% %     \end{tabular}
% %     \label{tab: ablation}
% %     }
% %     \end{minipage}
%     % \caption{Caption}
%     % \label{tab:my_label}
% \vspace{-0.15in}
% \end{figure*}
\begin{figure}
   % \vspace{0.2cm}
   % \captionsetup{font=small}
       \centering
       \includegraphics[width=0.7\linewidth]{figures/eval_negative_3.pdf}
       \captionof{figure}{Significance of negative graph weight $\beta$ on Cora and Texas datasets where we fix the positive graph weight $\alpha=1$ and vary a large range of \(\beta\).}
       \label{fig:beta real} 
    % \vspace{-0.2in}
\end{figure}
\textbf{RQ3.2: Performance under varying graph homophily and heterophily levels.}
In order to test the performance of \ours on graphs with arbitrary
levels of homophily and heterophily, we conduct an ablation study in the CSBM setting with the controllable homophilic and heterophilic levels following~\citet{GRP-GNN}.
As shown in Figure~\ref{fig:beta csbm}, Feature/Label-\ours performs best in homophilic graphs when all nodes are effectively attracted to one another, i.e., when the repulsion strength $\beta$ is small. As $\beta$ increases, the performance of the model degrades.
% The parameter $\phi$ in the CSBM controls the relative importance of node features and graph topology in determining the homophily level.
% Specifically, $\phi$ ranges from -1 to 1, with lower values corresponding to strongly heterophilic graphs and higher values indicating strongly homophilic graphs. 
% Specially, $\phi=1$ corresponds to strongly homophilic graphs while $\phi=-1$ corresponds to strongly heterophilic graphs.
% We fix $\lambda=0.5$ and then vary $\beta$ which indicates the strength of the repulsive force between the two nodes introduced by the negative edge connecting them.
In contrast, for heterophilic graphs, when the attraction power of the positive graph dominates, \ours achieves only $50\%$ accuracy. 
As $\beta$ increases, the negative graph becomes more dominant, and the model's performance gets significantly better. We observe similar phenomena in the real homophilic and heterophilic graph datasets as shown in Figure~\ref{fig:beta real}.

% \vspace{-1ex}
\textbf{RQ3.3: Performance on large-scale dataset.} 
Finally, we conduct an evaluation of \ours on the large-scale ogbn-arxiv dataset, and the results are presented in Table \ref{tab: large}. 
% To maintain the sparsity of the graph structure and avoid additional computational overhead, we adopt variants of the \ours approach mentioned in Section~\ref{sec: method}. 
Overall, the results demonstrate that Label-\ours-v2 achieves comparable or even superior performance compared to previous normalization methods, particularly in the deep layer setting  ($L=16$).
This verifies the empirical superiority and robustness of our proposed signed graph construction in \ours, which effectively leverages the available label information to alleviate oversmoothing, even at scale.
\begin{table}
    \captionof{table}{Node classification accuracy (\%) on the large-scale dataset~\textit{ogbn-arxiv}.}
    % Test accuracy (\%) comparison results on large scale dataset (Ogbn-ArXiv). The best results are marked in blue and the second best results are marked in gray on every layer.
    \centering
    \resizebox{0.7\linewidth}{!}{
    \begin{tabular}{lcccc}
    \toprule
     Model             & \#L=2              & \#L=4              & \#L=8            & \#L=16  \\
    \midrule
    GCN & 67.32 {\footnotesize $\pm$ 0.28} & 67.79 {\footnotesize $\pm$ 0.25} & 65.54 {\footnotesize $\pm$ 0.31} & 59.13 {\footnotesize $\pm$ 0.95}  \\
         BatchNorm & 70.14 {\footnotesize $\pm$ 0.28} & 70.93 {\footnotesize $\pm$ 0.15} & 70.14 {\footnotesize $\pm$ 0.43} & 63.24 {\footnotesize $\pm$ 1.40} \\
         % +LayerNorm& \cellcolor{secondbest}70.53 {\footnotesize $\pm$ 0.19} & \cellcolor{best}71.66 {\footnotesize $\pm$ 0.17} & \cellcolor{secondbest}71.23 {\footnotesize $\pm$ 0.16} & 68.62 {\footnotesize $\pm$ 0.47} \\
         PairNorm & 70.48 {\footnotesize $\pm$ 0.20} & \cellcolor{best}71.59 {\footnotesize $\pm$ 0.17} & \cellcolor{best}71.24 {\footnotesize $\pm$ 0.07} & 68.92 {\footnotesize $\pm$ 0.43} \\
         ContraNorm & OOM & OOM & OOM & OOM \\
         DropEdge & 64.07 {\footnotesize $\pm$ 0.32} & 63.92 {\footnotesize $\pm$ 0.27} & 60.74 {\footnotesize $\pm$ 0.45} & 52.52 {\footnotesize $\pm$ 0.34} \\
         Residual & 66.90 {\footnotesize $\pm$ 0.14} & 66.67 {\footnotesize $\pm$ 0.25} & 61.76 {\footnotesize $\pm$ 0.62} & 53.25 {\footnotesize $\pm$ 0.75} \\
    % \midrule
         % Feature-\ourst-v2 & 67.89 {\footnotesize $\pm$ 0.10} & 68.47 {\footnotesize $\pm$ 0.26} & 65.09 {\footnotesize $\pm$ 0.30} & 60.34 {\footnotesize $\pm$ 0.94} \\
         Label-\ourst-v2 & \cellcolor{best}70.55 {\footnotesize $\pm$ 0.22} & 71.54 {\footnotesize $\pm$ 0.18} & 71.07 {\footnotesize $\pm$ 0.28} & \cellcolor{best}69.33 {\footnotesize $\pm$ 0.59}  \\
    \bottomrule
    \end{tabular}
    \label{tab: large}
    }
    % \vspace{-0.2in}
\end{table}

% verifying the empirical advantages of our proposed technique.
% \phi 
% Hyperparameter $\beta$ indicating strength of the repulsive force between the two nodes introduced by negative edges connecting them
% To illustrate the impact of $\beta$, we conduct ablation studies on the synthetic CSBM graphs under various settings as well as real datasets. 
%
% Following~\cite{GRP-GNN}, we reconstruct the CSBM with the 
% The parameter $\phi$ to control for the the information given by the node features and the graph topology and the homophily level. 
% Specifically $\phi=0$ indicates that only node features are informative, while $|\phi|=1$ indicates that only the graph topology is informative. Moreover, $\phi=1$ corresponds to strongly homophilic graphs while $\phi=-1$ corresponds to strongly heterophilic graphs.
%
% \begin{figure}[t]
% % \hspace{-10pt}
%     \begin{minipage}{.48\textwidth}
%     \captionof{table}{GCN test accuracy on the large-scale dataset~\textit{ogbn-arxiv}.}
%     % Test accuracy (\%) comparison results on large scale dataset (Ogbn-ArXiv). The best results are marked in blue and the second best results are marked in gray on every layer.
%     \centering
%     \resizebox{0.99\linewidth}{!}{
%     \begin{tabular}{lcccc}
%     \toprule
%      Model             & \#L=2              & \#L=4              & \#L=8            & \#L=16  \\
%     \midrule
%     GCN & 67.32 {\footnotesize $\pm$ 0.28} & 67.79 {\footnotesize $\pm$ 0.25} & 65.54 {\footnotesize $\pm$ 0.31} & 59.13 {\footnotesize $\pm$ 0.95}  \\
%          BatchNorm & 70.14 {\footnotesize $\pm$ 0.28} & 70.93 {\footnotesize $\pm$ 0.15} & 70.14 {\footnotesize $\pm$ 0.43} & 63.24 {\footnotesize $\pm$ 1.40} \\
%          % +LayerNorm& \cellcolor{secondbest}70.53 {\footnotesize $\pm$ 0.19} & \cellcolor{best}71.66 {\footnotesize $\pm$ 0.17} & \cellcolor{secondbest}71.23 {\footnotesize $\pm$ 0.16} & 68.62 {\footnotesize $\pm$ 0.47} \\
%          PairNorm & 70.48 {\footnotesize $\pm$ 0.20} & \cellcolor{best}71.59 {\footnotesize $\pm$ 0.17} & \cellcolor{best}71.24 {\footnotesize $\pm$ 0.07} & 68.92 {\footnotesize $\pm$ 0.43} \\
%          ContraNorm & OOM & OOM & OOM & OOM \\
%          DropEdge & 64.07 {\footnotesize $\pm$ 0.32} & 63.92 {\footnotesize $\pm$ 0.27} & 60.74 {\footnotesize $\pm$ 0.45} & 52.52 {\footnotesize $\pm$ 0.34} \\
%          Residual & 66.90 {\footnotesize $\pm$ 0.14} & 66.67 {\footnotesize $\pm$ 0.25} & 61.76 {\footnotesize $\pm$ 0.62} & 53.25 {\footnotesize $\pm$ 0.75} \\
%     % \midrule
%          % Feature-\ourst & 67.89 {\footnotesize $\pm$ 0.10} & 68.47 {\footnotesize $\pm$ 0.26} & 65.09 {\footnotesize $\pm$ 0.30} & 60.34 {\footnotesize $\pm$ 0.94} \\
%          Label-\ourst & \cellcolor{best}70.55 {\footnotesize $\pm$ 0.22} & 71.54 {\footnotesize $\pm$ 0.18} & 71.07 {\footnotesize $\pm$ 0.28} & \cellcolor{best}69.33 {\footnotesize $\pm$ 0.59}  \\
%     \bottomrule
%     \end{tabular}
%     \label{tab: large}
%     }\hfill
%     \end{minipage}
%    \begin{minipage}{.52\textwidth}
%    \vspace{0.2cm}
%        \centering
%        \includegraphics[width=\linewidth]{figures/eval_negative (3).pdf}
%        \captionof{figure}{Significance of negative graph weight $\beta$ on Cora and Texas datasets where we fix the positive graph weight $\alpha=1$.}
%        \label{fig:beta real} 
%    \end{minipage}
%    % \vspace{-0.5cm}
%     % \hspace{-10pt}
% %     \begin{minipage}{.35\textwidth}
% %     \centering
% %     \captionsetup{font=small}
% %     \caption{Accuracy on different splits of train/valid/test dataset. SGC in CSBM and GCN in Cora.}
% %     % SGC test accuracy (\%) comparison results on sbm of \ourst-Label on different splits of train/valid/test dataset. GCN test accuracy (\%) comparison results on Cora of \ourst-Label on different splits of train/valid/test dataset.
% % % The best results are marked in blue on every dataset.
% %     \centering
% %      \resizebox{0.95\linewidth}{!}{
% %     \begin{tabular}{lcc}
% %     \toprule
% %      Splits & CSBM  & Cora   \\
% %     \midrule
% %     % \midrule
% %     % sbm 0/5/5& 57.00 {\footnotesize $\pm$ 13.36}
% %       2/4/4 & 86.25 {\footnotesize $\pm$ 3.01} & 82.80 {\footnotesize $\pm$ 0.81}\\
% %       4/3/3 & 91.50 {\footnotesize $\pm$ 2.52} & 85.39 {\footnotesize $\pm$ 0.18 }\\
% %       6/4/4 & 91.50 {\footnotesize $\pm$ 6.05} & 87.64 {\footnotesize $\pm$ 0.37 }\\
% %       8/1/1 & \cellcolor{best}99.05 {\footnotesize $\pm$ 1.90} & \cellcolor{best} 94.10 {\footnotesize $\pm$ 0.74} \\
% %     \bottomrule
% %     \end{tabular}
% %     \label{tab: ablation}
% %     }
% %     \end{minipage}
%     % \caption{Caption}
%     % \label{tab:my_label}
% % \vspace{-0.12in}
% \end{figure}

% \begin{wraptable}{r}{.5\linewidth}
% \begin{table}[h]
\centering
% \small
\caption{Test accuracy (\%) comparison results on large scale dataset (Ogbn-ArXiv). 
The best results are marked in blue and the second best results are marked in gray on every layer.}
\begin{adjustbox}{width=0.7\textwidth}
\begin{tabular}{lcccc}
\toprule
 Model             & \#L=2              & \#L=4              & \#L=8            & \#L=16  \\
\midrule


% \midrule

GCN & 67.32 {\footnotesize $\pm$ 0.28} & 67.79 {\footnotesize $\pm$ 0.25} & 65.54 {\footnotesize $\pm$ 0.31} & 59.13 {\footnotesize $\pm$ 0.95}  \\
     BatchNorm & 70.14 {\footnotesize $\pm$ 0.28} & 70.93 {\footnotesize $\pm$ 0.15} & 70.14 {\footnotesize $\pm$ 0.43} & 63.24 {\footnotesize $\pm$ 1.40} \\
     % +LayerNorm& \cellcolor{secondbest}70.53 {\footnotesize $\pm$ 0.19} & \cellcolor{best}71.66 {\footnotesize $\pm$ 0.17} & \cellcolor{secondbest}71.23 {\footnotesize $\pm$ 0.16} & 68.62 {\footnotesize $\pm$ 0.47} \\
     PairNorm & 70.48 {\footnotesize $\pm$ 0.20} & \cellcolor{best}71.59 {\footnotesize $\pm$ 0.17} & \cellcolor{best}71.24 {\footnotesize $\pm$ 0.07} & \cellcolor{best}68.92 {\footnotesize $\pm$ 0.43} \\
     ContraNorm & OOM & OOM & OOM & OOM \\
     DropEdge & 64.07 {\footnotesize $\pm$ 0.32} & 63.92 {\footnotesize $\pm$ 0.27} & 60.74 {\footnotesize $\pm$ 0.45} & 52.52 {\footnotesize $\pm$ 0.34} \\
     Residual & 66.90 {\footnotesize $\pm$ 0.14} & 66.67 {\footnotesize $\pm$ 0.25} & 61.76 {\footnotesize $\pm$ 0.62} & 53.25 {\footnotesize $\pm$ 0.75} \\

     % Feature-\ourst & 67.89 {\footnotesize $\pm$ 0.10} & 68.47 {\footnotesize $\pm$ 0.26} & 65.09 {\footnotesize $\pm$ 0.30} & 60.34 {\footnotesize $\pm$ 0.94} \\
     Label-\ourst & \cellcolor{best}70.55 {\footnotesize $\pm$ 0.22} & 71.54 {\footnotesize $\pm$ 0.18} & 71.07 {\footnotesize $\pm$ 0.28} & \cellcolor{best}69.33 {\footnotesize $\pm$ 0.59}  \\

% % Add more rows as needed
\bottomrule
\end{tabular}
\end{adjustbox}
\label{table: large result}
% \end{table}
\end{wraptable}





% \subsection{Ablation Study}





% Appedix~\ref{app: ablation}. 



