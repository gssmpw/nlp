
% \vspace{-2ex}
\section{Experiments}
\label{sec: exp_main}


% \jq{almost rewrite}
In this section, we conduct a comprehensive evaluation of
\ours on various benchmark datasets, including both
homophilic and heterophilic graphs. We aim to answer the following three
key research questions: \textbf{RQ1} How does \ours perform in node classification tasks? \textbf{RQ2} How effectively does \ours mitigate oversmoothing? \textbf{RQ3} How sensitive, robust, and scalable is \ours?   

% In this section, we first verify our theoretical insights on the synthetic datasets.
% Then we demonstrate the effectiveness of \ours in three widely used datasets, and then extend the experimental evaluation to one large-scale dataset and two heterophilous datasets. 
% We apply Label-\ours and Feature-\ours in both linear SGC and non-linear GCN. 




% % \textbf{Contextual Stochastic Block Models.} 
% % Following~\cite{sbm_xinyi}, We focus on the CSBM$(N, p, q, \mu_1, \mu_2, \sigma^2 )$.
% % It consists of two classes $\mathcal{C}_1$ and $\mathcal{C}_2$ of nodes of equal size, in total with $N$ nodes. 
% % For any two nodes in the graph, if they are from the same class, they are connected by an edge independently with probability $p$, or if they are from different classes, the probability is $q$. For each node $v \in \mathcal{C}_i, i\in\{1,-1\}$, the initial feature $X_v$ is sampled independently from a Gaussian distribution $\mathcal{N}(\mu_i, {\sigma^2})$, where $\mu_i =\mathcal{C}_i, \sigma = I $. Specially, let $N=200$, $p=0.092$, $q=0.046$, $d=8$.

% \paragraph{Setup}  
% We apply our methods to node classification on the synthetic dataset CSBM$(n, p, q, \mu_1, \mu_2, \sigma^2 )$.
% Specially, let $n=200$, $p=0.092$, $q=0.046$, $d=8$, $\mu_1=1$, $\mu_2=-1$ and $\sigma^2=I$ following~\cite{sbm_xinyi}.
% We used $60\%$, $20\%$ and $20 \%$ random splits for train, valid and test data, respectively.
% Since our theoretical analysis is primarily based on linear propagation, we apply the linear backbone SGC, which removes the non-linear activation function compared to the GCN, to verify our insight.
% We introduce SGC and GCN in Appendix~\ref{app: GNNs} in details.
% % We further apply both SGC and the GCN in the real world dataset.



% \paragraph{Results} 
% The visualization of node features using Label-\ours and Feature-\ours compared to SGC are shown in Figure \ref{fig: sbm overall}. As the number of layers increases, SGC's node features suffer from oversmoothing, causing the two classes to converge and the classification accuracy to drop to $47.50\%$, which is worse than random guessing ($50\%$). However, Label-\ours and Feature-\ours effectively repel nodes from different classes, achieving high accuracy of $80\%$ and $97.5\%$, respectively, even with $300$ layers.


\begin{figure*}[t]
% \captionsetup{font=small}
    \begin{subfigure}{0.69\textwidth}
        \centering
        % \captionsetup{font=small}
        \includegraphics[width=0.99\textwidth]{figures/eval_sgc_layer.pdf}
        \caption{Oversmoothing performance.}
        \label{fig: layer depth}
    \end{subfigure}
    % \quad
    \begin{subfigure}{0.3\textwidth}
        \centering
        % \captionsetup{font=small}
        \includegraphics[width=0.99\textwidth]{figures/eval_train.pdf} % Adjust the path and filename as necessary
        \caption{Training ratio ablation study.}
        \label{fig: train ratio}
    \end{subfigure}
    \caption{Left is the performance comparison of \ours against Normalization GNNs under various model depths where the X-axis has the number of layers, and the Y-axis has node classification accuracy. Right is the ablation study on Label-\ours where the X-axis indicates the ratio of the training node numbers.}
    % \vspace{-0.55cm}
    % \vspace{-0.2in}
\end{figure*}



% \subsection{Real World Benchmark}

% \begin{table}[t]
\centering
% \vspace{-0.15in}
\caption{SGC test accuracy (\%) comparison results. The best results are marked in blue and the second best results are marked in gray on every layer. We run 10 runs and demonstrate the mean $\pm$ std in the table.} % for the seed $0$\~$9$ 
% \resizebox{\textwidth}{!}{
\begin{adjustbox}{width=0.99\textwidth}
\begin{tabular}{lccccccc}
\toprule
 Model             & \#L=2              & \#L=5              & \#L=10             & \#L=20             & \#L=50        & \#L=100    & \#L=300    \\
\midrule
\rowcolor{gray!8}\multicolumn{8}{c}{\textit{Cora}~\citep{cora}}\\
\midrule
% \midrule
% \rowcolor{gray!8}\textit{cora}~\citep{cora}\\
% \midrule
 SGC      & 80.21 {\footnotesize $\pm$ 0.07}& 81.45 {\footnotesize $\pm$ 0.14 }& 81.53 {\footnotesize $\pm$ 0.19 }& 79.53 {\footnotesize $\pm$ 0.14 }& 79.20 {\footnotesize $\pm$ 0.21 }& 76.13 {\footnotesize $\pm$ 0.24 }& 65.64 {\footnotesize $\pm$ 1.15 }\\

 % +LayerNorm~\cite{layernorm}       & 80.07 {\footnotesize $\pm$ 0.22 }& \cellcolor{secondbest}81.60 {\footnotesize $\pm$ 0.22 }& 81.20 {\footnotesize $\pm$ 0.29 }& 79.52 {\footnotesize $\pm$ 0.16  }& 79.21 {\footnotesize $\pm$ 0.23 }& 76.44 {\footnotesize $\pm$ 0.11 }& 68.38 {\footnotesize $\pm$ 0.66}  \\
 +BatchNorm & 77.90 {\footnotesize $\pm$ 0.00 }& 78.02 {\footnotesize $\pm$ 0.04 }& 76.94 {\footnotesize $\pm$ 0.08 }& 75.18 {\footnotesize $\pm$ 0.09 }& 74.54 {\footnotesize $\pm$ 0.05 }& 72.64 {\footnotesize $\pm$ 0.05 }& 63.12 {\footnotesize $\pm$ 0.06} \\
+PairNorm     & 80.30 {\footnotesize $\pm$ 0.05 }& 78.57 {\footnotesize $\pm$ 0.00 }&78.14 {\footnotesize $\pm$ 0.07 }& 76.90 {\footnotesize $\pm$ 0.00 }& 77.49 {\footnotesize $\pm$ 0.03}  & 72.01 {\footnotesize $\pm$ 0.03 }& 40.93 {\footnotesize $\pm$ 0.11} \\
 +ContraNorm      & \cellcolor{best}81.60 {\footnotesize $\pm$ 0.00 }& 80.67 {\footnotesize $\pm$ 0.06 }& 79.11 {\footnotesize $\pm$ 0.03 }& 74.28 {\footnotesize $\pm$ 0.15 }& 69.67 {\footnotesize $\pm$ 1.23 }& 65.58 {\footnotesize $\pm$ 2.11 }&47.21 {\footnotesize $\pm$ 10.80} \\
+DropEdge & 73.58 {\footnotesize $\pm$ 2.76 }& 62.11 {\footnotesize $\pm$ 5.10 }& 39.21 {\footnotesize $\pm$ 7.54 }& 15.07 {\footnotesize $\pm$ 6.22 }& 11.16 {\footnotesize $\pm$ 2.73 }& 11.15 {\footnotesize $\pm$ 2.81 }& 11.15 {\footnotesize $\pm$ 2.81 }\\
 +Residual& 77.81 {\footnotesize $\pm$ 0.03 }& 81.47 {\footnotesize $\pm$ 0.05 }& \cellcolor{best}82.90 {\footnotesize $\pm$ 0.00 }& 79.87 {\footnotesize $\pm$ 0.05 }& 75.64 {\footnotesize $\pm$ 0.05 }& 66.90 {\footnotesize $\pm$ 0.10 }& 25.33 {\footnotesize $\pm$ 0.46}\\
\midrule
 Feature-\ourst &78.10 {\footnotesize $\pm$ 0.11 }& 80.88 {\footnotesize $\pm$ 0.23 }& 80.83 {\footnotesize $\pm$ 0.37 }& \cellcolor{secondbest}82.46 {\footnotesize $\pm$ 0.07 }& \cellcolor{secondbest}80.47 {\footnotesize $\pm$ 0.25 }& \cellcolor{secondbest}80.23 {\footnotesize $\pm$ 0.51 }& \cellcolor{secondbest}77.49 {\footnotesize $\pm$ 0.23 }\\

 Label-\ourst    & \cellcolor{secondbest}81.14 {\footnotesize $\pm$ 0.49 }& \cellcolor{best}82.90 {\footnotesize $\pm$ 0.00}&	\cellcolor{secondbest}82.54 {\footnotesize $\pm$ 0.05}&	\cellcolor{best}82.44 {\footnotesize $\pm$ 0.05}&	\cellcolor{best}82.60 {\footnotesize $\pm$ 0.00}&	\cellcolor{best}81.10 {\footnotesize $\pm$ 0.00}&	\cellcolor{best}74.98 {\footnotesize $\pm$ 0.11 }\\

\midrule
\rowcolor{gray!8}\multicolumn{8}{c}{\textit{CiteSeer}~\citep{citeseer}}\\
\midrule

SGC & 71.88 {\footnotesize $\pm$ 0.27 }& \cellcolor{secondbest}72.55 {\footnotesize $\pm$ 0.25 }& 72.53 {\footnotesize $\pm$ 0.15 }& 72.07 {\footnotesize $\pm$ 0.21 }& 69.83 {\footnotesize $\pm$ 0.20 }& 65.42 {\footnotesize $\pm$ 0.43 }& 54.69 {\footnotesize $\pm$ 0.98} \\

% +LayerNorm~\cite{layernorm} &66.92 {\footnotesize $\pm$ 7.97 }& 65.78 {\footnotesize $\pm$ 2.22 }& 65.82 {\footnotesize $\pm$ 2.39 }& 64.83 {\footnotesize $\pm$ 1.83 }& 62.96 {\footnotesize $\pm$ 2.52 }& 56.67 {\footnotesize $\pm$ 6.55 }& 48.87 {\footnotesize $\pm$ 7.06} \\
+BatchNorm &60.85 {\footnotesize $\pm$ 0.09 }& 60.45 {\footnotesize $\pm$ 0.07 }& 61.74 {\footnotesize $\pm$ 0.27 }& 63.29 {\footnotesize $\pm$ 0.18 }& 63.71 {\footnotesize $\pm$ 0.18 }& 64.28 {\footnotesize $\pm$ 0.27 }& 59.42 {\footnotesize $\pm$ 0.20} \\
 +PairNorm  &70.83 {\footnotesize $\pm$ 0.06 }& 69.68 {\footnotesize $\pm$ 0.32 }& 70.54 {\footnotesize $\pm$ 0.04 }& 69.86 {\footnotesize $\pm$ 0.08 }& 70.51 {\footnotesize $\pm$ 0.07 }& \cellcolor{secondbest}69.86 {\footnotesize $\pm$ 0.06 }& \cellcolor{secondbest}65.22 {\footnotesize $\pm$ 0.16 }\\  
 +ContraNorm  &\cellcolor{best}72.25 {\footnotesize $\pm$ 0.08 }& 71.9 {\footnotesize $\pm$ 0.06 }& 71.52 {\footnotesize $\pm$ 0.04 }& 59.82 {\footnotesize $\pm$ 2.30 }& 52.87 {\footnotesize $\pm$ 1.86 }& 45.93 {\footnotesize $\pm$ 1.40 }& 35.67 {\footnotesize $\pm$ 1.62}\\
+DropEdge & 65.63 {\footnotesize $\pm$ 1.76 }& 51.80 {\footnotesize $\pm$ 4.61 }& 25.36 {\footnotesize $\pm$ 2.54 }& 18.60 {\footnotesize $\pm$ 3.78 }& 16.52 {\footnotesize $\pm$ 3.97 }& 16.49 {\footnotesize $\pm$ 4.03  }& 16.49 {\footnotesize $\pm$ 4.03 }\\
 +Residual & 71.61 {\footnotesize $\pm$ 0.17 }& 72.31 {\footnotesize $\pm$ 0.15  }& \cellcolor{best}72.78 {\footnotesize $\pm$ 0.12 }& \cellcolor{secondbest}72.50 {\footnotesize $\pm$ 0.14 }& \cellcolor{secondbest}71.24 {\footnotesize $\pm$ 0.21 }& 69.85 {\footnotesize $\pm$ 0.22 }& 62.11 {\footnotesize $\pm$ 0.42}\\
\midrule
  Feature-\ourst & 70.63 {\footnotesize $\pm$ 0.52 }& 70.85 {\footnotesize $\pm$ 0.09 }& 70.52 {\footnotesize $\pm$ 0.14 }& 70.76 {\footnotesize $\pm$ 0.22 }& 68.25 {\footnotesize $\pm$ 0.46 }& 67.20 {\footnotesize $\pm$ 1.15 }& 65.12 {\footnotesize $\pm$ 1.95 }\\
 Label-\ourst &\cellcolor{secondbest}72.01 {\footnotesize $\pm$ 0.10 }& \cellcolor{best}72.87 {\footnotesize $\pm$ 0.05 }& \cellcolor{secondbest}72.72 {\footnotesize $\pm$ 0.28 }& \cellcolor{best}73.04 {\footnotesize $\pm$ 0.10 }& \cellcolor{best}72.52 {\footnotesize $\pm$ 0.17 }& \cellcolor{best}72.45 {\footnotesize $\pm$ 0.11 }& \cellcolor{best}70.97 {\footnotesize $\pm$ 0.22 }\\

\midrule
\rowcolor{gray!8}\multicolumn{8}{c}{\textit{PubMed }~\citep{pubmed}}\\
\midrule
SGC &76.99 {\footnotesize $\pm$ 0.38 }& 75.92 {\footnotesize $\pm$ 0.30 }& 76.18 {\footnotesize $\pm$ 0.70 }& 77.13 {\footnotesize $\pm$ 0.34 }&76.09 {\footnotesize $\pm$ 0.43 }& 76.19 {\footnotesize $\pm$ 0.19 }& 70.58 {\footnotesize $\pm$ 0.52 }\\

 % +LayerNorm~\cite{layernorm} &77.67 {\footnotesize $\pm$ 0.40 }& 76.43 {\footnotesize $\pm$ 0.36 }& 76.26 {\footnotesize $\pm$ 0.34 }& 76.27 {\footnotesize $\pm$ 0.41 }& 75.95 {\footnotesize $\pm$ 0.24 }& 74.79 {\footnotesize $\pm$ 0.53 }& 71.77 {\footnotesize $\pm$ 0.45} \\ 
 +BatchNorm &77.15 {\footnotesize $\pm$ 0.09 }& 77.87 {\footnotesize $\pm$ 0.05 }& 78.47 {\footnotesize $\pm$ 0.05 }& 77.90 {\footnotesize $\pm$ 1.10 }& 76.85 {\footnotesize $\pm$ 0.08 }& 74.35 {\footnotesize $\pm$ 0.08 }& 69.61 {\footnotesize $\pm$ 0.08} \\
 +PairNorm &77.69 {\footnotesize $\pm$ 0.26 }& 75.78 {\footnotesize $\pm$ 0.37 }& 75.13 {\footnotesize $\pm$ 0.13 }& 74.75 {\footnotesize $\pm$ 0.33 }& 72.13 {\footnotesize $\pm$ 0.11 }& 69.79 {\footnotesize $\pm$ 0.16 }& 71.75 {\footnotesize $\pm$ 0.51 }\\       
+ContraNorm &\cellcolor{best}79.30 {\footnotesize $\pm$ 0.10 }& 78.69 {\footnotesize $\pm$ 0.07 }& 77.54 {\footnotesize $\pm$ 0.09 }& 73.67 {\footnotesize $\pm$ 0.12 }& 71.37 {\footnotesize $\pm$ 3.15 }& 67.96 {\footnotesize $\pm$ 3.24 }& 65.00 {\footnotesize $\pm$ 4.12 }\\
 +DropEdge & 74.64 {\footnotesize $\pm$ 1.37 }& 69.83 {\footnotesize $\pm$ 3.19 }& 60.28 {\footnotesize $\pm$ 2.70 }& 32.62 {\footnotesize $\pm$ 10.95 }& 33.95 {\footnotesize $\pm$ 10.44 }& 33.95 {\footnotesize $\pm$ 10.44 }& 33.95 {\footnotesize $\pm$ 10.44 }\\
+Residual & 77.40 {\footnotesize $\pm$ 0.06 }& \cellcolor{secondbest}79.30 {\footnotesize $\pm$ 0.10 }& \cellcolor{secondbest}79.83 {\footnotesize $\pm$ 0.09 }& \cellcolor{secondbest}79.44 {\footnotesize $\pm$ 0.09 }& 74.96 {\footnotesize $\pm$ 0.09 }& 71.72 {\footnotesize $\pm$ 0.13 }& 55.57 {\footnotesize $\pm$ 0.21 }\\
\midrule
 Feature-\ourst & 73.99 {\footnotesize $\pm$ 1.44 }& 74.36 {\footnotesize $\pm$ 0.63 }& 75.61 {\footnotesize $\pm$ 0.24 }& 77.09 {\footnotesize $\pm$ 0.35 }& \cellcolor{secondbest}77.41 {\footnotesize $\pm$ 0.21 }& \cellcolor{secondbest}77.10 {\footnotesize $\pm$ 0.36 }& \cellcolor{secondbest}76.87 {\footnotesize $\pm$ 0.49 }\\
 Label-\ourst &\cellcolor{secondbest}78.98 {\footnotesize $\pm$ 0.14 }& \cellcolor{best}80.14 {\footnotesize $\pm$ 0.05 }& \cellcolor{best}80.22 {\footnotesize $\pm$ 0.04 }& \cellcolor{best}80.32 {\footnotesize $\pm$ 0.04 }& \cellcolor{best}80.20 {\footnotesize $\pm$ 0.00 }& \cellcolor{best}79.60 {\footnotesize $\pm$ 0.00 }& \cellcolor{best}73.96 {\footnotesize $\pm$ 0.05} \\
% \midrule

% Arxiv }& SGC \\
%     }& +LayerNorm \\
%     }& PairNorm \\
%     }& ContraNorm \\
%     }& Feature-\ourst \\
%     }& Label-\ourst \\
% % Add more rows as needed
\bottomrule
\end{tabular}
% }

\end{adjustbox}
\label{table: sgc results}
% \vspace{-0.15in}
\end{table}
% \vspace{-0.1in}
% \begin{table}[t]
\vspace{-0.1in}
\centering
\small
\caption{GCN test accuracy (\%) comparison results on heterophilic datasets. The best results are marked in blue and the second best results are marked underline on every layer.
We run 5 runs and demonstrate the mean $\pm$ std in the table.}%for the seed from $0~4$
\begin{adjustbox}{width=0.91\textwidth}
\begin{tabular}{lcccccc}
\toprule
 Model             & \#L=2              & \#L=4              & \#L=8              & \#L=16             & \#L=32             & \#L=64\\

\midrule
\rowcolor{gray!8}\multicolumn{7}{c}{\textit{Chameleon}~\cite{heter_dataset}}\\
\midrule
   GCN & 66.01{\footnotesize$\pm$0.72} & 54.21{\footnotesize$\pm$0.53} & 35.48{\footnotesize$\pm$3.09} & 22.37{\footnotesize$\pm$0.00} & 22.37{\footnotesize$\pm$0.00} & 22.37{\footnotesize$\pm$0.00} \\
    +BatchNorm & \underline{65.83{\footnotesize$\pm$0.58}} & 56.40{\footnotesize$\pm$0.35} & 36.36{\footnotesize$\pm$2.04} & 22.37{\footnotesize$\pm$0.00} & 22.37{\footnotesize$\pm$0.00} & 22.37{\footnotesize$\pm$0.00}\\
    +PairNorm & 66.01{\footnotesize$\pm$0.72} & 54.12{\footnotesize$\pm$0.79} & 36.75{\footnotesize$\pm$0.38} & 22.37{\footnotesize$\pm$0.00} & 22.37{\footnotesize$\pm$0.00} & 22.37{\footnotesize$\pm$0.00}\\
    +ContraNorm & 66.01{\footnotesize$\pm$0.72} & 58.16{\footnotesize$\pm$1.76} & 37.15{\footnotesize$\pm$4.91} & 22.37{\footnotesize$\pm$0.00} & 22.37{\footnotesize$\pm$0.00} & 22.37{\footnotesize$\pm$0.00}\\
    +DropEdge & 62.50{\footnotesize$\pm$0.00} & 53.07{\footnotesize$\pm$1.61} & 32.15{\footnotesize$\pm$1.49} & 21.71{\footnotesize$\pm$0.00} & \underline{27.19{\footnotesize$\pm$1.75}} & \underline{23.68{\footnotesize$\pm$0.00}} \\
    +Residual & 66.01{\footnotesize$\pm$0.72} & \cellcolor{best}62.94{\footnotesize$\pm$0.00} & \underline{57.59{\footnotesize$\pm$2.58}} & \underline{41.27{\footnotesize$\pm$0.32}} & 22.37{\footnotesize$\pm$0.00} & 22.37{\footnotesize$\pm$0.00} \\
\midrule
    Feature-\ourst & 62.98{\footnotesize$\pm$0.75} & \underline{62.89{\footnotesize$\pm$1.29}} & \cellcolor{best}65.35{\footnotesize$\pm$0.00} & \cellcolor{best}62.28{\footnotesize$\pm$0.00} & \cellcolor{best}55.31{\footnotesize$\pm$0.53} & \cellcolor{best}35.13{\footnotesize$\pm$0.93}\\
    Label-\ourst & \cellcolor{best}66.01{\footnotesize$\pm$0.72} & 57.11{\footnotesize$\pm$0.11} & 38.11{\footnotesize$\pm$1.87} & 22.37{\footnotesize$\pm$0.00} & 22.37{\footnotesize$\pm$0.00} & 22.37{\footnotesize$\pm$0.00}\\
\midrule
\rowcolor{gray!8}\multicolumn{7}{c}{\textit{Squirrel}~\cite{heter_dataset}}\\
\midrule
   GCN  & 42.38{\footnotesize$\pm$0.04} & 32.20{\footnotesize$\pm$3.05} & 22.57{\footnotesize$\pm$0.00} & 20.46{\footnotesize$\pm$0.00} & 20.46{\footnotesize$\pm$0.00} & 20.46{\footnotesize$\pm$0.00}\\
    +BatchNorm & 41.77{\footnotesize$\pm$0.35} & 32.37{\footnotesize$\pm$3.46} & 22.67{\footnotesize$\pm$0.00} & 20.46{\footnotesize$\pm$0.00} & 20.46{\footnotesize$\pm$0.00} & 20.46{\footnotesize$\pm$0.00}\\
    +PairNorm & 42.75{\footnotesize$\pm$0.00} & 32.12{\footnotesize$\pm$3.00} & 22.57{\footnotesize$\pm$0.00} & 20.46{\footnotesize$\pm$0.00} & 20.46{\footnotesize$\pm$0.00} & 20.46{\footnotesize$\pm$0.00}\\
    +ContraNorm & \underline{43.78{\footnotesize$\pm$1.08}} & 32.80{\footnotesize$\pm$3.76} & 22.57{\footnotesize$\pm$0.00} & 20.46{\footnotesize$\pm$0.00} & 20.46{\footnotesize$\pm$0.00} & 20.46{\footnotesize$\pm$0.00} \\
    +DropEdge & 40.54{\footnotesize$\pm$0.00} & 22.57{\footnotesize$\pm$0.00} & 22.77{\footnotesize$\pm$2.12} & 22.19{\footnotesize$\pm$0.58} & \underline{22.61{\footnotesize$\pm$1.36}} & 20.46{\footnotesize$\pm$0.00}\\
    +Residual & 41.92{\footnotesize$\pm$0.65} & \underline{42.23{\footnotesize$\pm$0.08}} & \underline{39.15{\footnotesize$\pm$0.07}} & \underline{33.41{\footnotesize$\pm$2.73}} & 20.46{\footnotesize$\pm$0.00} & 20.46{\footnotesize$\pm$0.00} \\
\midrule
    Feature-\ourst& \cellcolor{best}44.48{\footnotesize$\pm$0.00} &\cellcolor{best} 45.01{\footnotesize$\pm$0.72} &\cellcolor{best} 44.03{\footnotesize$\pm$0.63} &\cellcolor{best} 41.42{\footnotesize$\pm$0.78} &\cellcolor{best} 36.79{\footnotesize$\pm$0.00} &\cellcolor{best} 29.20{\footnotesize$\pm$0.00}\\
    Label-\ourst& 43.61{\footnotesize$\pm$0.58} & 32.78{\footnotesize$\pm$3.49} & 22.79{\footnotesize$\pm$0.09} & 20.46{\footnotesize$\pm$0.00} & 20.46{\footnotesize$\pm$0.00} & \underline{20.46{\footnotesize$\pm$0.00}} \\

\bottomrule
\end{tabular}
\end{adjustbox}
\label{table: gcn heter}
\vspace{-0.1in}
\end{table}

% \begin{table}[h]
% \centering
% \caption{GCN test accuracy (\%) comparison results. The best results are marked in blue and the second best results are marked in gray on every layer.}
% \begin{adjustbox}{width=0.99\textwidth}
% \begin{tabular}{lcccccc}
% \toprule
%  Model             & \#L=2              & \#L=4              & \#L=8              & \#L=16             & \#L=32             & \#L=64\\

% \midrule
% \rowcolor{gray!8}\multicolumn{7}{c}{\textit{Cora}~\citep{cora}}\\
% \midrule
%   GCN~\cite{gcn} & \cellcolor{secondbest}80.68 $\pm$ 0.09 & \cellcolor{secondbest}79.69 $\pm$ 0.00 & 74.32 $\pm$ 0.00 & 30.95 $\pm$ 0.00 & 30.95 $\pm$ 0.00 & 24.85 $\pm$ 7.46 \\


% % & Center & 79.85 $\pm$ 0.46 & 77.32 $\pm$ 1.33 & 75.25 $\pm$ 0.40 & 57.75 $\pm$ 3.53 & 41.73 $\pm$ 0.92 & 39.34 $\pm$ 2.29 \\
%     LayerNorm~\cite{layernorm} & 80.51 $\pm$ 0.12 & \cellcolor{best}80.28 $\pm$ 0.66 & 75.05 $\pm$ 0.00 & 30.95 $\pm$ 0.00 & 30.95 $\pm$ 0.00 & 24.85 $\pm$ 7.46 \\
%     BatchNorm~\cite{batchnorm} & 78.09 $\pm$ 0.00 & 77.87 $\pm$ 0.02 & 73.62 $\pm$ 0.57 & 70.79 $\pm$ 0.00 & 53.90 $\pm$ 2.19 & 35.32 $\pm$ 3.41\\
%     PairNorm~\cite{pairnorm} & 79.01 $\pm$ 0.00 & 78.26 $\pm$ 0.50 & 73.21 $\pm$ 0.00 & 62.96 $\pm$ 0.00 & 48.13 $\pm$ 0.91 & 44.01 $\pm$ 3.46 \\
%     ContraNorm~\cite{contranorm} & \cellcolor{best}81.55 $\pm$ 0.21 & 79.61 $\pm$ 0.75 & 77.71 $\pm$ 0.00 & 63.35 $\pm$ 0.00 & 44.56 $\pm$ 4.83 & 38.97 $\pm$ 0.00 \\
%     DropEdge~\cite{dropedge} & 78.38 $\pm$ 0.00 & 74.47 $\pm$ 0.00 & 26.91 $\pm$ 0.83 & 22.24 $\pm$ 3.04 & 27.18 $\pm$ 0.00 & 25.98 $\pm$ 6.00\\
%     Residual& 80.68 $\pm$ 0.09 & 78.77 $\pm$ 0.00 & \cellcolor{secondbest}79.26 $\pm$ 0.21 & 40.91 $\pm$ 0.00 & 30.95 $\pm$ 0.00 & 27.90 $\pm$ 6.09\\
% \midrule
%      \ourst-Feature & 80.44 $\pm$ 0.83 & 79.26 $\pm$ 1.18 &  78.56 $\pm$ 0.59 & \cellcolor{secondbest} 77.22 $\pm$ 0.55 &\cellcolor{secondbest} 73.65 $\pm$ 0.48 &\cellcolor{secondbest} 61.62 $\pm$ 5.24\\
%      \ourst-Label &80.31 $\pm$ 0.70 & 79.16 $\pm$ 1.30 & \cellcolor{best}79.50 $\pm$ 0.00 & \cellcolor{best}77.43 $\pm$ 1.49 & \cellcolor{best}74.52 $\pm$ 0.36 & \cellcolor{best}65.02 $\pm$ 2.97 \\
% \midrule
% \rowcolor{gray!8}\multicolumn{7}{c}{\textit{CiteSeer}~\citep{citeseer}}\\
% \midrule
%    GCN~\cite{gcn} &\cellcolor{best} 67.45 $\pm$ 0.54 & 65.62 $\pm$ 0.25 & 37.22 $\pm$ 2.46 & 22.03 $\pm$ 4.76 & 19.65 $\pm$ 0.00 & 19.65 $\pm$ 0.00 \\


% % & Center & 67.21 $\pm$ 0.64 & 65.50 $\pm$ 0.99 & 59.25 $\pm$ 3.18 & 40.29 $\pm$ 1.18 & 41.73 $\pm$ 0.92 & 35.81 \pm 1.21\\
%     LayerNorm~\cite{layernorm} & 67.24 $\pm$ 0.66 &64.95 $\pm$ 0.72 & 38.87 $\pm$ 4.12 & 24.29 $\pm$ 5.68 & 19.65 $\pm$ 0.00 & 19.65 $\pm$ 0.00 \\
%      BatchNorm~\cite{batchnorm} &63.44 $\pm$ 0.94 & 62.34 $\pm$ 0.25 & 61.36 $\pm$ 0.00 & 50.58 $\pm$ 1.24 & 41.41 $\pm$ 0.00 & 35.00 $\pm$ 1.09 \\
%     PairNorm~\cite{pairnorm} & 63.58 $\pm$ 0.63 & 64.32 $\pm$ 0.95 & 61.95 $\pm$ 1.24 & 50.06 $\pm$ 0.00 & 37.21 $\pm$ 1.87 & 36.09 $\pm$ 0.07 \\
%     ContraNorm~\cite{contranorm} & 66.83 $\pm$ 0.49 & 64.78 $\pm$ 0.92 & 60.70 $\pm$ 0.60 & 44.79 $\pm$ 1.65 & 37.36 $\pm$ 0.25 & 30.85 $\pm$ 0.81 \\
%     DropEdge~\cite{dropedge} & 63.86 $\pm$ 0.03 & 62.24 $\pm$ 0.90 & 24.73 $\pm$ 5.72 & 20.65 $\pm$ 0.00 & 20.04 $\pm$ 0.19 & 19.95 $\pm$ 0.09\\
%     Residual & \cellcolor{secondbest} 67.45 $\pm$ 0.54 & 66.21 $\pm$ 0.16 & \cellcolor{best}67.34 $\pm$ 0.00 & 33.21 $\pm$ 0.00 & 19.65 $\pm$ 0.00 & 19.65 $\pm$ 0.00 \\
% \midrule
%     \ourst-Feature &  67.38 $\pm$ 0.66 & \cellcolor{best}66.94 $\pm$ 0.00 & 66.29 $\pm$ 0.02 & \cellcolor{secondbest}65.35 $\pm$ 1.99 & \cellcolor{best}61.43 $\pm$ 0.00 & \cellcolor{secondbest}42.09 $\pm$ 1.65\\
%      \ourst-Label & 67.23 $\pm$ 0.64 & \cellcolor{secondbest} 66.72 $\pm$ 0.00 & \cellcolor{secondbest}66.29 $\pm$ 0.89 & \cellcolor{best}65.50 $\pm$ 2.13 & \cellcolor{secondbest}59.93 $\pm$ 0.85 & \cellcolor{best}44.41 $\pm$ 1.57 \\
% \midrule
% \rowcolor{gray!8}\multicolumn{7}{c}{\textit{PubMed}~\citep{pubmed}}\\
% \midrule
%    GCN~\cite{gcn} & \cellcolor{best}76.44 $\pm$ 0.34 & 76.52 $\pm$ 0.32 & 69.58 $\pm$ 5.89 & 39.92 $\pm$ 0.00 & 39.92 $\pm$ 0.00 & 39.92 $\pm$ 0.00 \\


% % & Center & 75.19 $\pm$0.26	& 76.67 \pm	0.00 &OOM &OOM &OOM & OOM\\
%     LayerNorm~\cite{layernorm} & 76.27 $\pm$ 0.51 & 76.71 $\pm$ 0.24 & 76.95 $\pm$ 0.17 & 39.92 $\pm$ 0.00 & 39.92 $\pm$ 0.00 & 39.92 $\pm$ 0.00 \\
%     BatchNorm~\cite{batchnorm} & 75.52 $\pm$ 0.12 & \cellcolor{secondbest}77.15 $\pm$ 0.00 & 77.10 $\pm$ 0.00 & 76.92 $\pm$ 0.00 & 75.43 $\pm$ 0.00 & 69.33 $\pm$ 1.01 \\
%     PairNorm~\cite{pairnorm} & 75.66 $\pm$ 0.11 & 76.71 $\pm$ 0.00 & \cellcolor{secondbest}77.99 $\pm$ 0.00 & \cellcolor{secondbest}77.22 $\pm$ 0.39 & 75.52 $\pm$ 2.02 & 71.22 $\pm$ 3.68 \\
%     ContraNorm~\cite{contranorm} & 76.05 $\pm$ 0.33 & \cellcolor{best}78.42 $\pm$ 0.00 & OOM & OOM & OOM & OOM \\
%     DropEdge~\cite{dropedge}& 73.41 $\pm$ 0.03 & 73.96 $\pm$ 0.79 & 52.51 $\pm$ 10.91 & 40.27 $\pm$ 0.00 & 39.90 $\pm$ 0.59 & 40.08 $\pm$ 0.39 \\
%     Residual & \cellcolor{secondbest} 76.44 $\pm$ 0.34 & 77.28 $\pm$ 0.00 & 77.38 $\pm$ 0.00 & 63.14 $\pm$ 3.05 & 39.92 $\pm$ 0.00 & 39.92 $\pm$ 0.00 \\
% \midrule
%     \ourst-Feature & 75.72 $\pm$ 0.06 & 76.84 $\pm$ 0.00 & \cellcolor{best}78.39 $\pm$ 0.00 &\cellcolor{best} 79.71 $\pm$ 0.00 & \cellcolor{best}77.59 $\pm$ 0.23 & \cellcolor{best}78.06 $\pm$ 0.13\\
%     \ourst-Label & 76.33 $\pm$ 0.25 & 76.91 $\pm$ 0.00 & 77.60 $\pm$ 0.49 & 76.31 $\pm$ 0.00 & \cellcolor{secondbest}77.17 $\pm$ 0.67 & \cellcolor{secondbest}78.01 $\pm$ 0.16\\
% % Add more rows as needed
% % \midrule

% % Arxiv & GCN & 70.20 $\pm$ 0.36 & 70.84 $\pm$ 0.12 & 69.73 $\pm$ 0.28\\
% %     & APPNP \\
% %     & Center \\
% %     & LayerNorm \\
% %     & PairNorm \\
% %     & ContraNorm \\
% %     & \ourst-Feature \\
% %     & \ourst-Label \\
% \bottomrule
% \end{tabular}
% \end{adjustbox}
% \end{table}
\textbf{Datasets.} 
% \section{Dataset Generation}
\label{sec:dataset}
\revise{
To train the proposed GNN, we constructed a dataset of building structures and a subset of these structures were subjected to fire simulations using FEA. The dataset generation process is illustrated in \figref{fig:dataset_generation_procedure}. Initially, a total of 33,000 building structures with geometrical details, material properties, and gravity loads were created. Due to randomness in generating these structures, a filter is applied to remove unreasonable data after gravity load simulation, which included 15,377 structures. A trade-off between computational feasibility and model performance is made among the remaining 17,623 structures. As further labeling structures with MIDR requires resource-intensive fire simulations via OpenSeesRT, a large proportion of 16,050 structures is selected as unlabeled dataset. On the other hand, each of the other 1,573 structures was further subjected to 30 different fire simulations, forming the labeled dataset containing $1,573\times 30 = 47,190$ fire cases.} This section details the step-by-step process for generating the dataset, including geometry creation, material property assignment, and simulations due to gravity loads and fire scenarios. 
% To train the proposed neural network, we constructed a dataset comprising building structure data and a subset of fire scenario data. The dataset generation process is illustrated in \figref{fig:dataset_generation_procedure}. 
% A total of 33,000 building structures with geometric details, material properties, and gravity loads were initially created. Out of these, 3,000 structures were selected as labeled data, and the remaining 30,000 were designated as unlabeled data. Further, about half of them filtered out due to instability under gravity loads only. 
\begin{figure*}[h!]
    \centering
    \includegraphics[width=0.8\linewidth]{figures/dataset_filter_procedure.pdf}
    \caption{Workflow for dataset generation (geometry, material property, gravity loads, and fire scenarios).}
    \label{fig:dataset_generation_procedure}
\end{figure*}

\subsection{Geometry Generation}
\label{subsec:geometry_generation}
The geometry of the building structures forms the foundation of the dataset. Regular 
\revise{3D structures} resembling multi-story parking structures or shopping malls were generated, with parameters such as building floor dimensions and story heights selected randomly. Each building structure is composed of multiple rooms, which serve as the basic unit in this study. A room herein is a cuboid space defined by specific length, width, and height. Within a structure, rooms of the same dimensions are uniformly arranged along the length, width, and height, corresponding to the $x$-, $y$-, and $z$-axes, respectively. Structures vary in room size and number of rooms along each axis. Specifically, the room length, width, and height are independently sampled from a uniform distribution within the interval $[2, 5]$ meters along the three directions of the structure. Similarly, the room number along each axis is uniformly sampled independently as an integer within the interval $[2, 7]$, i.e., the maximum number of stories of the buildings simulated in this study is 7.

To introduce variability and simulate real-world scenarios, approximately $8\%$ of structural elements (beams or columns) are randomly removed after initial geometry creation. 
\revise{Such removal is not fire-induced damage, but reflects functional diversity often observed in real buildings, such as open spaces designed for activities in shopping malls, e.g., ice skating rinks. Examples of the generated geometries are illustrated in \figref{fig:example_generated_geometry}, showcasing the diversity and realism of the dataset. This element removal does not affect the definition of room's geometry in the structure and nor does it affect the number of considered fire scenarios.} 

\revise{A range of coefficient of variation values ($3.3\%$ to $17.5\%$) was derived from prior studies that investigated the statistics of geometrical and material properties of structural components of buildings (e.g., \cite{mirza1979variations, lee2004probabilistic}). These studies provide empirical data on the natural variability in parameters such as Young's modulus, yield strength, and dimensions of structural elements due to manufacturing tolerances and material inconsistencies. By selecting $8\%$ for the removal of structural elements in our database, we aimed to maintain a level of variability that is representative of real-world uncertainties while ensuring computational feasibility. This choice ensures that the database captures realistic deviations without introducing extreme cases that may not be commonly encountered in practice.}

\begin{figure*}[h!]
    \centering
    \includegraphics[width=\linewidth]{figures/example_generated_geometry.pdf}
    \caption{Examples of generated structural geometry of different sizes (all dimensions in meters).}
    \label{fig:example_generated_geometry} 
\end{figure*}

{\blockRevise

In this study, we opted for a deterministic square, dimension of $0.1$ m, solid cross-sectional steel elements due to their simplicity in modeling and analysis. Square sections exhibit uniform geometrical properties in all directions, simplifying the computation of structural responses and avoiding complications associated with more complex shapes, such as wide-flange sections, facilitating the computational efficiency and scalability to generate a large dataset. This choice also helps to mitigate issues related to stress concentrations and facilitates a more straightforward representation of structural behavior under thermal loads. 

\textit{Remark:} The selected cross-section provides a comparable flexural rigidity to a $W 130 \times 130 \times 28.1$ wide-flange section (metric units), albeit with significantly higher axial rigidity. This cross-section is acceptable for gravity-load-designed frames under service loading conditions where the models assume fully rigid, moment-resisting beam-column connections for the evaluation of the IDR under thermal loading. This assumption is reasonable in this computational study where the primary interest is to understand the global deformation response of frames under fire conditions. The selection of uniform square cross-sections for both beams and columns, rather than adherence to standard capacity design principles, was made here primarily for computational efficiency and to reduce design parameters in the database generation process. This choice allows for simplified and scalable approach to analyze the fire-induced response of generic steel frames without the need for large section variations, where this study mainly focuses on the fire vulnerability assessment using ML-based predictions. However, if additional loading conditions, e.g., seismic or wind loads, were to be considered, larger sections, strong-column/weak-beam principle, and ductile detailing would be required in the generated buildings for realistic structural behavior under combined loading conditions. Future studies may also consider investigating the influence of variable cross-sectional dimensions and semi-rigid connections on the structural performance under fire conditions. 
} % blockRevise

\subsection{Material Properties}
Steel is chosen as the material for the structures. To reflect real-world variations, we randomly assign one of five slightly different steel material types to each structural element. \revise{
The ranges of material properties are provided in \tabref{tab:material_property_ranges} and the properties are sampled from uniform distributions of the corresponding ranges. These variations simulate differences arising from manufacturing batches or regional material properties. That these properties are at ambient temperature and change when the temperature rises due to a fire. The selection of materials with varying properties is aimed at increasing the diversity of the data. Our goal is to represent as wide a range of data as possible with a limited amount of building structure data, thereby enhancing the generalization ability of the GNN. Our assumed material property ranges are expected to be wider than the real-world conditions based on findings in \cite{mirza1979variations, lee2004probabilistic}. Therefore, we are essentially tackling a more challenging and general task. If we can solve this problem, we are confident that our method will perform equally well or even better in real-world scenarios.
}
\begin{table}[h!]
    \centering
    \caption{Material properties ranges for considered steel structures.}
    \begin{tabular}{lc}
        \toprule
        Property & Range \\
        \midrule
        Young's modulus & [168, 252] GPa \\
        Yield strength & [220, 330] MPa \\
        Strain-hardening ratio & [0.8, 1.2] \% \\
        \bottomrule
    \end{tabular}
    \label{tab:material_property_ranges}
\end{table}

\subsection{Gravity Loads}
Gravity loads are applied to columns and beams based on their \revise{influence (tributary) areas as typically conducted in structural analysis. The considered ``service'' load conditions include the column self-weight and the additional loads directly supported on the beams from their self-weight and weights of the reinforced concrete slabs, people as live load, and building content. An edge beam typically carries approximately half the gravity load supported by a parallel interior beam}. The ranges of gravity loads are listed in \tabref{tab:gravity_load_ranges}. \revise{The loads are sampled from uniform distributions of the corresponding ranges.} Structures that failed to meet an MIDR threshold of $1\%$ under gravity loads were deemed unacceptable designs and filtered out, as such configurations of randomly chosen geometry, material, and gravity load combinations were considered unrealistic from a regulatory and practicality points of view.
\begin{table}[h!]
    \centering
    \caption{Gravity load ranges for considered beams and columns.}
    \begin{tabular}{lc}
        \toprule
        Element & Range (kN/m)  \\
        \midrule
        Column & [0.5, 1.0]  \\
        Edge beam & [1.5, 4.5]  \\
        Interior beam & [3.0, 7.5]  \\
        \bottomrule
    \end{tabular}
    \label{tab:gravity_load_ranges}
\end{table} 

\subsection{Rule-based Thermal Load Generation}
\label{subsec:thermal_load_generation}
To evaluate a building's structural response during a fire event, we employed a simplified rule-based approach for thermal load generation. 
% Previous studies \cite{nan_structuralfire_2023} have demonstrated that steel structures rapidly equilibrate with surrounding gases temperatures due to efficient heat exchange. Consequently, gas temperatures can be directly used as inputs for FEA tools, e.g., OpenSees, simplifying the process of modeling thermal loads. 
% Accurately simulating temperature fields in fire scenarios poses significant challenges. Advanced thermodynamic simulations, such as those performed using Fire Dynamics Simulator (FDS) \cite{mcgrattan_fire_2000}, provide precise temperature predictions. However, these methods are hindered by high computational costs, prolonging execution times, and limited scalability, making them impractical for generating large datasets. Additionally, real-world fire loads often display substantial spatial variability across different rooms \cite{dundar_fire_2023}, resulting in scenario-specific temperature fields with limited generalizability. For example, studies on bridge fires \cite{he_study_2024} have demonstrated that environmental factors, such as wind speeds, can significantly influence temperature distributions. Furthermore, even within identical scenarios, variations in fire modeling methodologies can produce distinctly different temperature fields \cite{zhang_temperature_2020, du_new_2012}. These challenges emphasize the need for efficient and adaptable methods to generate fire temperature data.
% To address these issues, we adopted a rule-based approach to model temperature variations. 
According to \cite{spearpoint_fire_2008}, a typical fire development follows a predictable pattern. During the {\em{growth stage}}, the temperature rises slowly and approximately linearly after ignition. This is followed by the {\em{flashover stage}}, where temperatures increase rapidly to peak values. After reaching the peak, the temperature either stabilizes or continues to rise slowly until the {\em{decay stage}} begins. Inspired by this fire development pattern, we describe the temperature evolution in time, $t$, prior to the decay stage in two distinct stages:
\begin{enumerate}
    \item {\bf{Initial linear increase stage}}: For $t \in [0, t_1)$, temperature increases gradually and linearly as the fire spreads through the building. This stage represents the time before the fire directly affects a structural element.  
    \item {\bf{ISO 834 fire curve stage}}: For $t \in [t_1, t_{\thre}]$, temperature rises rapidly following the ISO 834 curve \cite{ISO834}, modeling the direct impact of the fire on the structural element. 
\end{enumerate}
The slope of the linear temperature increase, $c$, and the transition time, $t_1$, are influenced by the spatial relationship between the fire source and the structural element. For the second stage of temperature evolution, we utilize the ISO 834 curve, a widely accepted standard for fire resistance testing. This standardized fire curve describes the temperature rise over time, enabling rapid and consistent thermal fields across various scenarios. The duration of fire simulation in this study is set to $t_{\thre}=60$ minutes. This value represents the upper limit for the temperature evolution of each structural element, providing a consistent basis for analyzing the structural response to fire.

Let $(x, y, z)$ represents the midpoint of a structural element and $(x_{\subfire}, y_{\subfire}, z_{\subfire})$ the fire source point. \revise{Integer parameters $h$ and $h_{\subfire}$ correspond to the respective floor levels of the element and the fire source}. The temperature evolution for each element is expressed as follows:
\begin{enumerate}
    \item Linear increase stage ($0 < t < t_1$):
    \begin{equation}
    T(t) = c \cdot t,
    \end{equation}
    where $c$, the rate of temperature increase ($^\circ\mathrm{C}/\mathrm{min}$), depends on the height difference between the element, $h$, and the fire source, $h_{\subfire}$:
    \begin{equation}
        c = 
        \begin{cases} 
        5\left/\left(h - h_{\subfire} + 1\right)\right., & h \geq h_{\subfire}, \\
        2\left/\left(h_{\subfire} - h\right)\right., & h < h_{\subfire}.
        \end{cases}
    \end{equation}
     \item ISO 834 stage ($t \geq t_1$):
\begin{equation}
    T(t) = c \cdot t_1 + 345 \log_{10} \left(8 \left(t - t_1\right) + 1\right).
\end{equation}
\end{enumerate}

The transition (arrival) time $t_1$, marking the end of the linear stage, depends on the spatial distance between the fire source and the element. We define the following two Euclidean distances $L_p$ in the $xy$ plane and $L_s$ in the $xyz$ space:
\begin{eqnarray}
L_p & \triangleq & \sqrt{(x - x_{\subfire})^2 + (y - y_{\subfire})^2}, \\
\label{eq:Lp}
L_s & \triangleq & \sqrt{(x - x_{\subfire})^2 + (y - y_{\subfire})^2 + (z - z_{\subfire})^2}.
\label{eq:Ls}
\end{eqnarray}
Accordingly, the transition time, $t_1$, is expressed as follows:
\begin{equation}
    t_1 = 
    \begin{cases}
    \beta_{1} \cdot \left(1 - \exp\left\{- L_s\left/\alpha_{1}\right.\right\}\right), & h > h_{\subfire}, \\
    \beta_{2} \cdot \left(1 - \exp\left\{- L_p\left/\alpha_{2}\right.\right\}\right), & h = h_{\subfire}, \\
    \beta_{3} \cdot \left(1 - \exp\left\{- L_s\left/\alpha_{3}\right.\right\}\right), & h < h_{\subfire} .
    \end{cases}
    \label{eq:t1}
\end{equation}
The parameters $\beta_i$ and $\alpha_i$ for determining $t_1$ are summarized in Table~\ref{tab:fire_spread_parameters}. In this study, we take $r_{\mathrm{up}}=0.95$ and $r_{\mathrm{down}}=0.97$.
\begin{table}[ht]
    \centering
    \caption{Fire spread parameters for $t_1$ calculations.}
    \begin{tabular}{lcc}
        \toprule
        Case  & $\beta_i$ & $\alpha_i$  \\
        \midrule
        $i=1$, Upward spread & $16 \left.\left(1-r_{\mathrm{up}}^{\left|h-h_{\subfire}\right|}\right)\right/\left(1-r_{\mathrm{up}}\right)$ & $10$  \\
        $i=2$, Horizontal spread & $18$ & $18$  \\
        $i=3$, Downward spread & $30 \left.\left(1-r_{\mathrm{down}}^{\left|h-h_{\subfire}\right|}\right)\right/\left(1-r_{\mathrm{down}}\right)$ & $5$  \\
        \bottomrule
    \end{tabular}
    \label{tab:fire_spread_parameters}
\end{table}

\figref{fig:t1_curve} illustrates the $t_1$ curves for various fire scenarios: (1) fire originating on the lower floor, $h-h_{\subfire}=1$ with rapid upward spread, (2) fire on the same floor, $h=h_{\subfire}$ with the fastest spread, and (3) fire on the upper floor, $h_{\subfire}-h=1$ with slow downward spread. The exponential decay in $t_1$ reflects the accelerating fire propagation speed as the distance increases. \figref{fig:t1_curve} also indicates that the employed simplified model is consistent with the Markov chain-based dynamic model given by \cite{cheng_dynamic_2011}, where the rooms at the same floor of the fire point start flashover slightly before the corresponding upper floors. Additionally, $\beta_{1}$ and $\beta_{3}$ are the summation of a geometric sequence, where story level $h$ is the index. The common ratios $r_{\mathrm{up}}<1$ in $\beta_{1}$ and $r_{\mathrm{down}}<1$ in $\beta_{3}$ indicate that the fire speeds up to spread through the next story, which is consistent with the real-world fire spread mechanism given in \cite{hokugo_mechanism_2000}. The temperature profile within the range $t \in [0, t_{\thre}]$ is subsequently used as the thermal load in OpenSeesRT simulations to compute displacements at each structural node at time $t_{\thre}$.
\begin{figure}[h!]
    \centering
    \includegraphics[width=0.8\linewidth]{figures/m204_t1_curve.pdf}
    \caption{Three examples for the $t_1$ curve.}
    \label{fig:t1_curve}
\end{figure}

\revise{
\textit{Remark:} The effects of structural elements, such as concrete floor slabs and partitions, are not explicitly modeled in our approach. Instead, their influence is implicitly captured through the careful selection of the parameters $ \alpha, \beta, r_\mathrm{up} $, and $ r_\mathrm{down} $. This parameterization provides a unified framework for generating temperature fields. Indeed, fire propagation is governed by a multitude of factors and remains an open research question. For instance, if the fire resistance of a floor slab is enhanced by fire protective coating, the corresponding model can account for this by decreasing $\alpha_1$ \& $\alpha_3$, increasing $\beta_1$ \& $\beta_3$, and adopting larger values for $r_\mathrm{up}$ \& $r_\mathrm{down}$, which collectively slow down the vertical spread of fire. Conversely, scenarios involving higher amounts of combustible materials would warrant the opposite adjustments. This flexible and integrated approach avoids the need to design separate models for different fire propagation scenarios while still capturing the essential effects.
}

\revise{
In conclusion, our rule-based approach is a computationally efficient method for approximating fire temperature fields, enabling large-scale dataset generation to train predictive models. By combining ISO 834 fire curves with spatial considerations and embedding structural effects through parameter calibration, the method achieves a balanced trade-off between accuracy and scalability, making it a practical solution for thermal load modeling in fire scenarios. After generating the temperature of each beam or column according to the middle point, the temperature is applied as uniform thermal load to the elements of the structure in question using OpenSeesRT. 
}

% In conclusion, this rule-based approach is a computationally efficient method to approximate fire temperature fields, enabling large-scale dataset generation to train predictive models. By combining ISO 834 fire curves with spatial considerations, the method balances accuracy and scalability, making it a practical solution for thermal load modeling in fire scenarios.

% \subsection{Interstory Drift Ratio}
\subsection{OpenSeesRT Simulation}
\label{subsec:opensees_simulation}

The thermal and mechanical responses of 3D frame structures under combined fire and gravity loads are simulated using OpenSeesRT \cite{perez2024openseesrt}. \revise{In the simulation, the IDR of each node at $t_{\thre}$ is computed using the computed nodal displacements. Each structural model features six degrees of freedom per node (3 translational  and 3 rotational), with linear geometrical transformations (\texttt{geomTransf: Linear}) defining how the element local coordinate systems are mapped to the global coordinate system and assuming small displacements and rotations. Although OpenSeesRT allows a variety of options for modeling finite deformations, in the present simulations and mainly for simplicity, we did not consider large deformations. All bottom nodes (nodes on the ground) are fully constrained in all six degrees of freedom, while degrees of freedom os all other nodes are free.} Material behavior is temperature-dependent and modeled with \texttt{Steel01Thermal}, while fiber-based sections (\texttt{FiberThermal}) capture nonlinear interactions between thermal and mechanical responses at the cross-section level. \revise{Structural elements are represented as displacement-based Euler-Bernoulli beam-columns (\texttt{dispBeamColumnThermal}). This element  formulation accounts for thermal strains (temperature gradients) in the section, which is discretized into fibers. Numerical integration is used along the length of each element using three integration (Gauss) points, one at each end and the third in the middle of the element.}

{\revise{Thermal expansion of steel members plays a crucial role in IDR development. In reality, reinforced concrete floor slabs heat at a different rate than steel members due to their higher thermal mass and lower thermal conductivity. This differential heating can lead to restrained thermal expansion, introducing axial compression in beams and affecting the overall structural response. In this study, explicit {\em{composite action}} between steel members and concrete slabs is not modeled. Instead, our approach focuses on isolating the response of the steel structural frame, which is often the critical load-bearing component in fire scenarios. This assumption aligns with prior studies \cite{Possidente_2024} demonstrating that steel structures reach thermal equilibrium with surrounding gases quickly, allowing the use of uniform thermal loading in fire analysis. Future work could enhance this framework by incorporating slab-beam interaction effects, through a refined FEA for an extended dataset where constraints imposed by floor slabs are explicitly considered.}

The analysis begins with the application of gravity loads, followed by incremental thermal loads simulating the fire exposure. A static nonlinear solver using  \texttt{ExpressNewton} algorithm ensures convergence, while the \texttt{NormDispIncr} test maintains accuracy. An incremental \texttt{LoadControl} scheme with small step sizes is employed to guarantee numerical stability, using 10\% for gravity loads and 1\% for thermal loads. 

\revise{
In the thermal load analysis, uniform thermal load is applied to each beam or column, i.e., the temperature of each element is set to be that at the middle point, according to \secref{subsec:thermal_load_generation}. The \texttt{Steel01Thermal} material allows the properties (e.g., Young's modulus and yield strength) to be adjusted at increasing temperatures according to \cite{EN1993} using its Table 3.1: Reduction factors for the stress-strain relationship of carbon steel at elevated temperatures. For example, if the Young’s modulus at ambient temperature is $E_0$, then as the temperature ($T$) increases, the modulus changes as $E(T) = \eta (T) \times E_0$. \cite{EN1993} directly provides the values of $\eta(T) \in \left[0,1\right] $ at every $100 ^\circ\mathrm{C}$ interval and recommends using linear interpolation to obtain $\eta(T)$ for intermediate values of $T$.
} OpenSeesRT documentation \cite{OpenSeesThermalExamples} provides several examples of thermal analyses.

This modeling framework accommodates variations in material properties, cross-sectional geometries, and temperature profiles, providing robust simulations of structural behavior under fire conditions. The primary settings and configurations for the OpenSeesRT simulations are summarized in \tabref{tab:ops_detail}.
\begin{table}[h!]
    \centering
        \caption{Key settings of OpenSeesRT simulations.}
    \begin{tabular}{l|>{\raggedright\arraybackslash}p{0.6\linewidth}} %
    \toprule
    Modeling Aspect     & Details \\
    \midrule
    Geometry            & 3D models; 6 degrees of freedom per node \\
    Transformation      & geomTransf: Linear \\ 
    Material            & Steel01Thermal \\
    Section             & FiberThermal; Cross-section: $0.1$ m $\times$ $0.1$ m \\ 
    Element type        & {dispBeamColumnThermal} \\ 
    Loading             & Gravity loads: {beamUniform}; Thermal loads: {beamThermal} \\
    Integration scheme  & Incremental {LoadControl}; Step size: $10\%$ (gravity analysis), $1\%$ (thermal analysis) \\
    Nonlinear solver    & {ExpressNewton} algorithm; {UmfPack} solver; Convergence test: {NormDispIncr} tolerance: $10^{-8}$; Maximum \# iterations per step: $1000$. \\ 
    \bottomrule
    \end{tabular}
    \label{tab:ops_detail}
\end{table}

For each structure in the labeled dataset, 30 fire points are selected using a dual-granularity approach, \revise{i.e., two-stage sampling strategy,} to ensure they are well-distributed. Specifically, rooms are sequentially selected, with one fire point randomly chosen within each selected room. If a building is large and contains more than 30 rooms, we randomly select 30 rooms without replacement, i.e., ensuring that no more than one fire point is located in the same room. Conversely, if the building is small and has fewer than 30 rooms, all rooms are initially selected, with one fire point randomly assigned to each room. Additionally, rooms are then selected with replacement until a total of 30 fire points are assigned. \revise{The room-level sampling prioritizes selecting distinct rooms to avoid spatial clustering of fire points, while the point-level sampling ensures intra-room variability. This approach aligns with stratified sampling principles commonly used for efficient spatial representation, where multi-stage sampling strategies optimize coverage and variability, e.g., \cite{arunachalam_generalized_2023}, and enables a more comprehensive characterizing of how the structures respond under fire conditions.}
% This selection method prevents fire points from clustering too closely while maintaining an element of randomness. By distributing fire points in this manner, the 30 fire scenarios are effectively utilized, enabling a more comprehensive characterizing of how the structures respond under fire conditions.

\subsection{Summary of the Dataset Generation}
As discussed in this section and related to  \figref{fig:dataset_generation_procedure}, three key steps were considered in the development of the dataset: 
\begin{enumerate}
    \item {\bf{Filtering process}}: Structures with MIDR exceeding $1\%$ under gravity loads were excluded,  resulting in $1,573$ labeled structures retained for fire simulation and $16,050$ unlabeled structures for training the MFSP predictor.
    \item {\bf{Fire simulations}}: For each retained labeled structure, 30 fire scenarios were simulated using OpenSeesRT, yielding $47,190$ fire cases.
    \item {\bf{Data distribution check}}: MIDR distributions for labeled and unlabeled data under gravity loads were highly similar, because both datasets were generated using the same method. Under fire conditions, the MIDR distribution shifted, reflecting significant structural deformation with values reaching a maximum of about 6\%, an average of 1.70\%, and a standard deviation of 1.12\%. This step ensured a diverse and comprehensive dataset for the proposed predictive framework.
\end{enumerate}
The statistical distribution histograms for MIDR (after applying the $1\%$ filtering threshold \revise{for gravity load responses}) under different loading conditions are plotted in \figref{fig:histogram_mdr}. Figures \ref{fig:histogram_mdr}(a) and \ref{fig:histogram_mdr}(b) show the MIDR distributions of the labeled and unlabeled data, respectively, under gravity loads only. \figref{fig:histogram_mdr}(c) shows the MIDR distribution of the labeled data under the combined effects of gravity and fire loads. Fire load causes the structures to significantly deform, leading to a noticeably \revise{right-skewed} MIDR distribution.

\begin{figure*}[h!]
    \centering
    \includegraphics[width=\linewidth]{figures/histogram_mdr.pdf}
    \caption{Histograms of MIDR for labeled and unlabeled structures with gravity loads and fire cases.}
    \label{fig:histogram_mdr}
\end{figure*}

\revise{
This dataset provides the basis for training and testing the performance of the GNN-based framework. Although we employed a simplified rule-based thermal load generation method compared with conventional CFD-based simulations, the temperature field, the changes of the material properties, and the response of the structures, are all still highly nonlinear and complex. Therefore, it is still a challenging task for the NN to predict the MIDRs based on this dataset.
}
We use nine widely-used node classification
benchmark datasets (Table~\ref{tab: main_data}), where four of them are heterophilic (Texas, Wisconsin, Cornell, Squirrel, and
Amazon-rating~\citep{platonov2023critical}), and the remaining four are homophilic (Cora~\citep{cora},
Citeseer~\citep{citeseer}, and Pubmed~\citep{pubmed}) including one large-scale dataset (Ogbn-Arxiv~\cite{hu2020ogb}). 
Further information about the datasets and splits are provided in Appendix~\ref{app: exp}.
% and experimental results on more datasets
% In line with prior research, we employ the default training/validation/test splits provided by Pytorch Geometric (PyG). 
% For details of the datasets, including their sources and construction methods.
% We evaluate \ours for the semi-supervised node classification on Cora~\cite{cora}, CiteSeer~\cite{citeseer} and PubMed~\cite{pubmed} and one large-scale dataset ogbn-arxiv from OGB benchmarks~\citep{openbenchmark}.
% We also extend our models to two heterophilous datasets: Chameleon and Squirrel~\cite{heter_dataset}.
% We show the details of the dataset in Appendix~\ref{app: data}.




\textbf{Baselines and experiment settings.}
We compare the performance of \ours against the following $12$ baseline models. 
% \begin{itemize}
1) \textbf{Classic models}: MLP, SGC~\citep{sgc}.
2) \textbf{GNNs with normalization}: BatchNorm~\citep{batchnorm}, PairNorm~\citep{pairnorm} and ContraNorm~\citep{contranorm}.
3) \textbf{Augmenation-based GNNs}: DropEdge~\citep{dropedge}.
4) \textbf{GNNs with residual connections}: Residual, APPNP~\citep{appap}, JKNET~\citep{jknet} and DAGNN~\citep{dagnn}. 
5) \textbf{Other baselines}: GCNII~\citep{GCNII} and \(\omega\)GCN~\citep{wGCN}.
% \xw{你忘记cite gcnii 跟wgcn了}
For the sake of fair comparison, we do not deploy specific training techniques used in some prior works for benchmarking.
All models are trained under the same setting on the pure SGC backbone and we choose the best of scale controller in the range of $\{ 0.1, 0.5, 0.9\}$ for ContraNorm, DropEdge, and residual connections.
% For both Label-\ours and Feature-\ours, 
We choose the best of $\lambda$ in the range of $\{0.1, 0.5, 0.9\}$, fix $\alpha=1$ and select the best value for $\beta$ from $\{ 0.1, 0.5, 0.9\}$ for \ours. 
More experiment results with hyperparameter tuning and optimization strategies can be found in Appendix~\ref{app: exp}.
% fix $\alpha=1$ and only select $\beta$ from $\{0.1, 1,10, 20, 50, 100\}$ for simplify .
% \end{itemize}
% For more details of the classic anti-oversmoothing methods seen in Appendix~\ref{xx}.
% We apply both the linear SGC~\cite{sgc} and non-linear GCN~\cite{gcn} backbones.
% For fair comparison, we fix the hidden dimension to $32$ and dropout rate to $0.6$ following \cite{contranorm}.
% % The residual $\alpha=0.5$, the dropedge present is selecting from $\{0.3,0.5,0.7\}$.
% % We choose the best of scale controller $\alpha,\ \beta \in \{ 0.1, 0.2, 0.5, 0.7, 0.9\}$.
% We select the best settings for PairNorm, Residual, DropEdge, and ContraNorm based on their default hyperparameters. 

% We use Tesla-V100-SXM2-32GB in all experiments.
\textbf{RQ1: Node classification performance.}
In Table~\ref{tab: main_data}, we provide the mean of the node classification accuracy along with their corresponding standard deviations across 10 random seeds under the same 2-layer SGC backbone following~\citet{dgc}.
Overall, \ours achieves the best performance across $8$ datasets in the shallow layers, as Label/Feature-\ours performs the best on 7 out of the 8 datasets. 
% In particular, we make the following three observations:
% First, \ours outperforms all normalization methods. 
% Since our theoretical findings suggest that these normalization methods are essentially implicitly signed graph propagation, the theoretical properties of structural balance (Section~\ref{subsec: sb theory}) contribute to the enhanced classification accuracy of \ours.
% Second, \ours outperforms random argumentation based GNNs. Since DropEdge randomly drops edges, it isn't easy to characterize their exact behaviors, but we highlight that it works when it happens to remove edges between different classes of nodes thanks to our structurally balanced theory, as 
% DropEdge still follows the unified signed graph analysis in its message-passing scheme.
% Lastly, \ours outperforms residual connection based GNNs, including the last layer connection: residual and multilayer feature connection: APPNP, JKNET, and DAGNN. 
% In our analysis, GNNs with residual connections can be seen as a special case of signed graph propagation, where their positive and negative adjacency matrices are the linear combination of adjacency matrices of different orders, yet they are not the theoretically best solution to alleviate oversmoothing.
% This validates the effectiveness of our novel insight from a signed graph perspective. 

% \paragraph{Heterophilic datasets}
% Besides the three homophilic datasets, we also conduct experiments on four heterophilic datasets~\citep{heter_dataset}.
% We find that our method is still the most effective one across all of the methods for alleviating oversmoothing as indicated in Table~\ref{table: gcn heter}.
% Interestingly, we observe that Feature-\ours performs better than Label-\ours on the heterophilic datasets, which is the opposite of the results on the homophilic datasets.

\textbf{RQ2: Anti-oversmoothing analysis.}
% \paragraph{Results} 
% The results for SGC are detailed in Table \ref{table: sgc results} and we give the GCN results in Appendix (Table~\ref{table: gcn result}). 
We further evaluate the robustness of \ours by assessing its performance at deeper model depths: $K \in \{2, 10, 50, 100, 300\}$ for homophilic datasets and $K \in \{2, 5, 10, 20, 50\}$ for heterophilic datasets.
% To provide a comparative analysis against other GNNs, we also evaluate two best-performed normalization-based GNNs: BatchNorm and ContraNorm. the performances of these methods
% We evaluate on one heterophilic graph and two homophilic graphs.
Figure~\ref{fig: layer depth} shows that the performance of Feature/Label-\ours remains relatively stable with varying
% ($K = 50$) 
numbers of layers, achieving its best performance when the model gets deeper.
In contrast, the normalization methods considered exhibit a substantial decrease in performance as the number of layers increases, indicating their persistent susceptibility to the oversmoothing problem.
Note that we find that for \ours to maintain performance in the heterophilic dataset, \(\beta\) needs to be larger than the uniform range considered in Figure~\ref{fig: layer depth}. 
See Appendix~\ref{app: exp} for the result under larger \(\beta\), where \ours on deep layers remains $\approx60\%$ in Cornell.   

% \subsection{RQ3: Ablation Study}
\textbf{RQ3.1: Sensitivity analysis of training ratio.}
% Since Label-\ours leverages the ground truth label information to construct the negative graph, we conduct an ablation study examining the impact of different training data ratios. 
As shown in Figure~\ref{fig: train ratio}, Label-\ours's performance on the CSBM and Cora datasets improves as the training ratio increases. Even with a modest training ratio of 20\%, the worst-performing models still achieve an impressive 80\% accuracy, while the best models approach 100\% accuracy when the training ratio is increased to 80\%. This is in line with our theoretical insights that increasing the training ratio leads to more structural balance resulting from our method~\ours. 
% Moreover, our main experiments detailed in Table~\ref{table: sgc results} demonstrate that Label-\ours outperforms other methods, even when adopting the default training set ratios in those datasets, indicating its effectiveness in real-world graph settings.
% \begin{table*}
  [t]
  \centering
  \resizebox{\textwidth}{!}{%
  \begin{tabular}{cccccccccccc}
    \toprule \multicolumn{2}{c}{Components}                                                             & \multicolumn{5}{c}{Re-executability Rate (\%)} & \multicolumn{5}{c}{Readability (\#)} \\
    \cmidrule(lr){1-2} \cmidrule(lr){3-7} \cmidrule(lr){8-12}        \hspace{8pt}\labelemoji\hspace{8pt}                                                                & \hspace{8pt}\toolemoji\hspace{8pt}                                      & O0                                 & O1             & O2             & O3             & AVG            & O0             & O1             & O2             & O3             & AVG            \\
    \hline
    \rowcolor[rgb]{0.93,0.93,0.93}\multicolumn{12}{c}{\textbf{Initialize with LLM4Decompile-End-6.7B~\citep{llm4decompile}}}   \\
    \xmark                                                                                              & \xmark                                    & 69.51                              & 46.95          & 50.61          & 46.34          & 53.35          & 3.98 & 3.41 & 3.44 & 3.38 & 3.55 \\
    \cmark                                                                                              & \xmark                                    & 75.61                              & 50.61          & 50.00          & 50.00          & 56.55          & 4.01 & 3.44 & 3.39 & \textbf{3.49} & 3.58 \\
    \xmark                                                                                              & \cmark                                    & 83.54                     & \textbf{56.10}          & 51.22          & 50.61 & 60.37 & 4.05 & 3.51 & 3.51 & 3.42 & 3.62 \\
    \cmark                                                                                              & \cmark                                    & \textbf{85.37}                            & \textbf{56.10}                     & \textbf{51.83} & \textbf{52.43}          & \textbf{61.43} & \textbf{4.13} & \textbf{3.60} & \textbf{3.54} & \textbf{3.49} & \textbf{3.69} \\

    \rowcolor[rgb]{0.93,0.93,0.93}\multicolumn{12}{c}{\textbf{Initialize with Deepseek-Coder-6.7B-base~\citep{deepseekcoder}}} \\
    \xmark                                                                                              & \xmark                                    & 59.15                              & 35.98          & 39.02          & 37.80          & 42.99          & 3.71 & 3.05 & 3.16 & 3.05 & 3.24 \\
    \cmark                                                                                              & \xmark                                    & 66.46                              & 41.46          & 38.41          & 36.59          & 45.73          & 3.76 & 3.17 & \textbf{3.21} & 3.08 & 3.31 \\
    \xmark                                                                                              & \cmark                                    & 70.73                              & 39.63          & 39.02          & 40.24          & 47.41          & 3.90 & 3.17 & 3.08 & 3.11 & 3.31 \\
    \cmark                                                                                              & \cmark                                    & \textbf{79.88}                     & \textbf{45.73} & \textbf{43.90} & \textbf{42.68} & \textbf{53.05} & \textbf{3.96} & \textbf{3.21} & 3.18 & \textbf{3.19} & \textbf{3.38} \\
    \bottomrule
  \end{tabular}%
  }
  \caption{The ablation study of different methods across four optimization levels
  (O0, O1, O2, O3), as well as their average scores (AVG). The results in bold represent the optimal performance. The ~\labelemoji~ and ~\toolemoji~ means Relabedling and Function Call. \textbf{Bold} denotes the best performance.}
  \label{tab:ablation}
\end{table*}


% \begin{figure}[t]
%     \centering
%     \begin{subfigure}{0.63\textwidth}
%         \centering
%         \includegraphics[width=0.99\textwidth]{figures/eval_negative (3).pdf} % Adjust the path and filename as necessary
%         \caption{ Significance plot for $\beta$ in terms of test accuracy on cora (left) and texas (right) with fixed $\alpha=1$}
%         \label{fig: beta}
%     \end{subfigure}
%     \quad
%     \begin{subfigure}{0.33\textwidth}
%         \centering
%         \captionsetup{font=small}
%         \includegraphics[width=0.99\textwidth]{figures/eval_train.pdf} % Adjust the path and filename as necessary
%         \caption{Ablation study on Label-SBP. X-axis indicates the ratio of the training node numbers.}
%         \label{}
%     \end{subfigure}
%     \caption{Ablation study}
%     \label{fig: train ratio}
% \end{figure}




% \begin{figure*}[t]
% % \hspace{-10pt}
%     \begin{minipage}{.48\textwidth}
%     \captionof{table}{Node classification accuracy (\%) on the large-scale dataset~\textit{ogbn-arxiv}.}
%     % Test accuracy (\%) comparison results on large scale dataset (Ogbn-ArXiv). The best results are marked in blue and the second best results are marked in gray on every layer.
%     \centering
%     \resizebox{0.99\linewidth}{!}{
%     \begin{tabular}{lcccc}
%     \toprule
%      Model             & \#L=2              & \#L=4              & \#L=8            & \#L=16  \\
%     \midrule
%     GCN & 67.32 {\footnotesize $\pm$ 0.28} & 67.79 {\footnotesize $\pm$ 0.25} & 65.54 {\footnotesize $\pm$ 0.31} & 59.13 {\footnotesize $\pm$ 0.95}  \\
%          BatchNorm & 70.14 {\footnotesize $\pm$ 0.28} & 70.93 {\footnotesize $\pm$ 0.15} & 70.14 {\footnotesize $\pm$ 0.43} & 63.24 {\footnotesize $\pm$ 1.40} \\
%          % +LayerNorm& \cellcolor{secondbest}70.53 {\footnotesize $\pm$ 0.19} & \cellcolor{best}71.66 {\footnotesize $\pm$ 0.17} & \cellcolor{secondbest}71.23 {\footnotesize $\pm$ 0.16} & 68.62 {\footnotesize $\pm$ 0.47} \\
%          PairNorm & 70.48 {\footnotesize $\pm$ 0.20} & \cellcolor{best}71.59 {\footnotesize $\pm$ 0.17} & \cellcolor{best}71.24 {\footnotesize $\pm$ 0.07} & 68.92 {\footnotesize $\pm$ 0.43} \\
%          ContraNorm & OOM & OOM & OOM & OOM \\
%          DropEdge & 64.07 {\footnotesize $\pm$ 0.32} & 63.92 {\footnotesize $\pm$ 0.27} & 60.74 {\footnotesize $\pm$ 0.45} & 52.52 {\footnotesize $\pm$ 0.34} \\
%          Residual & 66.90 {\footnotesize $\pm$ 0.14} & 66.67 {\footnotesize $\pm$ 0.25} & 61.76 {\footnotesize $\pm$ 0.62} & 53.25 {\footnotesize $\pm$ 0.75} \\
%     % \midrule
%          Feature-\ourst & 67.89 {\footnotesize $\pm$ 0.10} & 68.47 {\footnotesize $\pm$ 0.26} & 65.09 {\footnotesize $\pm$ 0.30} & 60.34 {\footnotesize $\pm$ 0.94} \\
%          Label-\ourst & \cellcolor{best}70.55 {\footnotesize $\pm$ 0.22} & 71.54 {\footnotesize $\pm$ 0.18} & 71.07 {\footnotesize $\pm$ 0.28} & \cellcolor{best}69.33 {\footnotesize $\pm$ 0.59}  \\
%     \bottomrule
%     \end{tabular}
%     \label{tab: large}
%     }\hfill
%     \end{minipage}
%    \begin{minipage}{.52\textwidth}
%    % \vspace{0.2cm}
%        \centering
%        \includegraphics[width=\linewidth]{figures/eval_negative (3).pdf}
%        \captionof{figure}{Significance of negative graph weight $\beta$ on Cora and Texas datasets where we fix the positive graph weight $\alpha=1$.}
%        \label{fig:beta real} 
%    \end{minipage}
%    % \vspace{-0.5cm}
%     % \hspace{-10pt}
% %     \begin{minipage}{.35\textwidth}
% %     \centering
% %     \captionsetup{font=small}
% %     \caption{Accuracy on different splits of train/valid/test dataset. SGC in CSBM and GCN in Cora.}
% %     % SGC test accuracy (\%) comparison results on sbm of \ourst-Label on different splits of train/valid/test dataset. GCN test accuracy (\%) comparison results on Cora of \ourst-Label on different splits of train/valid/test dataset.
% % % The best results are marked in blue on every dataset.
% %     \centering
% %      \resizebox{0.95\linewidth}{!}{
% %     \begin{tabular}{lcc}
% %     \toprule
% %      Splits & CSBM  & Cora   \\
% %     \midrule
% %     % \midrule
% %     % sbm 0/5/5& 57.00 {\footnotesize $\pm$ 13.36}
% %       2/4/4 & 86.25 {\footnotesize $\pm$ 3.01} & 82.80 {\footnotesize $\pm$ 0.81}\\
% %       4/3/3 & 91.50 {\footnotesize $\pm$ 2.52} & 85.39 {\footnotesize $\pm$ 0.18 }\\
% %       6/4/4 & 91.50 {\footnotesize $\pm$ 6.05} & 87.64 {\footnotesize $\pm$ 0.37 }\\
% %       8/1/1 & \cellcolor{best}99.05 {\footnotesize $\pm$ 1.90} & \cellcolor{best} 94.10 {\footnotesize $\pm$ 0.74} \\
% %     \bottomrule
% %     \end{tabular}
% %     \label{tab: ablation}
% %     }
% %     \end{minipage}
%     % \caption{Caption}
%     % \label{tab:my_label}
% \vspace{-0.15in}
% \end{figure*}
\begin{figure}
   % \vspace{0.2cm}
   % \captionsetup{font=small}
       \centering
       \includegraphics[width=0.7\linewidth]{figures/eval_negative_3.pdf}
       \captionof{figure}{Significance of negative graph weight $\beta$ on Cora and Texas datasets where we fix the positive graph weight $\alpha=1$ and vary a large range of \(\beta\).}
       \label{fig:beta real} 
    % \vspace{-0.2in}
\end{figure}
\textbf{RQ3.2: Performance under varying graph homophily and heterophily levels.}
In order to test the performance of \ours on graphs with arbitrary
levels of homophily and heterophily, we conduct an ablation study in the CSBM setting with the controllable homophilic and heterophilic levels following~\citet{GRP-GNN}.
As shown in Figure~\ref{fig:beta csbm}, Feature/Label-\ours performs best in homophilic graphs when all nodes are effectively attracted to one another, i.e., when the repulsion strength $\beta$ is small. As $\beta$ increases, the performance of the model degrades.
% The parameter $\phi$ in the CSBM controls the relative importance of node features and graph topology in determining the homophily level.
% Specifically, $\phi$ ranges from -1 to 1, with lower values corresponding to strongly heterophilic graphs and higher values indicating strongly homophilic graphs. 
% Specially, $\phi=1$ corresponds to strongly homophilic graphs while $\phi=-1$ corresponds to strongly heterophilic graphs.
% We fix $\lambda=0.5$ and then vary $\beta$ which indicates the strength of the repulsive force between the two nodes introduced by the negative edge connecting them.
In contrast, for heterophilic graphs, when the attraction power of the positive graph dominates, \ours achieves only $50\%$ accuracy. 
As $\beta$ increases, the negative graph becomes more dominant, and the model's performance gets significantly better. We observe similar phenomena in the real homophilic and heterophilic graph datasets as shown in Figure~\ref{fig:beta real}.

% \vspace{-1ex}
\textbf{RQ3.3: Performance on large-scale dataset.} 
Finally, we conduct an evaluation of \ours on the large-scale ogbn-arxiv dataset, and the results are presented in Table \ref{tab: large}. 
% To maintain the sparsity of the graph structure and avoid additional computational overhead, we adopt variants of the \ours approach mentioned in Section~\ref{sec: method}. 
Overall, the results demonstrate that Label-\ours-v2 achieves comparable or even superior performance compared to previous normalization methods, particularly in the deep layer setting  ($L=16$).
This verifies the empirical superiority and robustness of our proposed signed graph construction in \ours, which effectively leverages the available label information to alleviate oversmoothing, even at scale.
\begin{table}
    \captionof{table}{Node classification accuracy (\%) on the large-scale dataset~\textit{ogbn-arxiv}.}
    % Test accuracy (\%) comparison results on large scale dataset (Ogbn-ArXiv). The best results are marked in blue and the second best results are marked in gray on every layer.
    \centering
    \resizebox{0.7\linewidth}{!}{
    \begin{tabular}{lcccc}
    \toprule
     Model             & \#L=2              & \#L=4              & \#L=8            & \#L=16  \\
    \midrule
    GCN & 67.32 {\footnotesize $\pm$ 0.28} & 67.79 {\footnotesize $\pm$ 0.25} & 65.54 {\footnotesize $\pm$ 0.31} & 59.13 {\footnotesize $\pm$ 0.95}  \\
         BatchNorm & 70.14 {\footnotesize $\pm$ 0.28} & 70.93 {\footnotesize $\pm$ 0.15} & 70.14 {\footnotesize $\pm$ 0.43} & 63.24 {\footnotesize $\pm$ 1.40} \\
         % +LayerNorm& \cellcolor{secondbest}70.53 {\footnotesize $\pm$ 0.19} & \cellcolor{best}71.66 {\footnotesize $\pm$ 0.17} & \cellcolor{secondbest}71.23 {\footnotesize $\pm$ 0.16} & 68.62 {\footnotesize $\pm$ 0.47} \\
         PairNorm & 70.48 {\footnotesize $\pm$ 0.20} & \cellcolor{best}71.59 {\footnotesize $\pm$ 0.17} & \cellcolor{best}71.24 {\footnotesize $\pm$ 0.07} & 68.92 {\footnotesize $\pm$ 0.43} \\
         ContraNorm & OOM & OOM & OOM & OOM \\
         DropEdge & 64.07 {\footnotesize $\pm$ 0.32} & 63.92 {\footnotesize $\pm$ 0.27} & 60.74 {\footnotesize $\pm$ 0.45} & 52.52 {\footnotesize $\pm$ 0.34} \\
         Residual & 66.90 {\footnotesize $\pm$ 0.14} & 66.67 {\footnotesize $\pm$ 0.25} & 61.76 {\footnotesize $\pm$ 0.62} & 53.25 {\footnotesize $\pm$ 0.75} \\
    % \midrule
         % Feature-\ourst-v2 & 67.89 {\footnotesize $\pm$ 0.10} & 68.47 {\footnotesize $\pm$ 0.26} & 65.09 {\footnotesize $\pm$ 0.30} & 60.34 {\footnotesize $\pm$ 0.94} \\
         Label-\ourst-v2 & \cellcolor{best}70.55 {\footnotesize $\pm$ 0.22} & 71.54 {\footnotesize $\pm$ 0.18} & 71.07 {\footnotesize $\pm$ 0.28} & \cellcolor{best}69.33 {\footnotesize $\pm$ 0.59}  \\
    \bottomrule
    \end{tabular}
    \label{tab: large}
    }
    % \vspace{-0.2in}
\end{table}

% verifying the empirical advantages of our proposed technique.
% \phi 
% Hyperparameter $\beta$ indicating strength of the repulsive force between the two nodes introduced by negative edges connecting them
% To illustrate the impact of $\beta$, we conduct ablation studies on the synthetic CSBM graphs under various settings as well as real datasets. 
%
% Following~\cite{GRP-GNN}, we reconstruct the CSBM with the 
% The parameter $\phi$ to control for the the information given by the node features and the graph topology and the homophily level. 
% Specifically $\phi=0$ indicates that only node features are informative, while $|\phi|=1$ indicates that only the graph topology is informative. Moreover, $\phi=1$ corresponds to strongly homophilic graphs while $\phi=-1$ corresponds to strongly heterophilic graphs.
%
% \begin{figure}[t]
% % \hspace{-10pt}
%     \begin{minipage}{.48\textwidth}
%     \captionof{table}{GCN test accuracy on the large-scale dataset~\textit{ogbn-arxiv}.}
%     % Test accuracy (\%) comparison results on large scale dataset (Ogbn-ArXiv). The best results are marked in blue and the second best results are marked in gray on every layer.
%     \centering
%     \resizebox{0.99\linewidth}{!}{
%     \begin{tabular}{lcccc}
%     \toprule
%      Model             & \#L=2              & \#L=4              & \#L=8            & \#L=16  \\
%     \midrule
%     GCN & 67.32 {\footnotesize $\pm$ 0.28} & 67.79 {\footnotesize $\pm$ 0.25} & 65.54 {\footnotesize $\pm$ 0.31} & 59.13 {\footnotesize $\pm$ 0.95}  \\
%          BatchNorm & 70.14 {\footnotesize $\pm$ 0.28} & 70.93 {\footnotesize $\pm$ 0.15} & 70.14 {\footnotesize $\pm$ 0.43} & 63.24 {\footnotesize $\pm$ 1.40} \\
%          % +LayerNorm& \cellcolor{secondbest}70.53 {\footnotesize $\pm$ 0.19} & \cellcolor{best}71.66 {\footnotesize $\pm$ 0.17} & \cellcolor{secondbest}71.23 {\footnotesize $\pm$ 0.16} & 68.62 {\footnotesize $\pm$ 0.47} \\
%          PairNorm & 70.48 {\footnotesize $\pm$ 0.20} & \cellcolor{best}71.59 {\footnotesize $\pm$ 0.17} & \cellcolor{best}71.24 {\footnotesize $\pm$ 0.07} & 68.92 {\footnotesize $\pm$ 0.43} \\
%          ContraNorm & OOM & OOM & OOM & OOM \\
%          DropEdge & 64.07 {\footnotesize $\pm$ 0.32} & 63.92 {\footnotesize $\pm$ 0.27} & 60.74 {\footnotesize $\pm$ 0.45} & 52.52 {\footnotesize $\pm$ 0.34} \\
%          Residual & 66.90 {\footnotesize $\pm$ 0.14} & 66.67 {\footnotesize $\pm$ 0.25} & 61.76 {\footnotesize $\pm$ 0.62} & 53.25 {\footnotesize $\pm$ 0.75} \\
%     % \midrule
%          % Feature-\ourst & 67.89 {\footnotesize $\pm$ 0.10} & 68.47 {\footnotesize $\pm$ 0.26} & 65.09 {\footnotesize $\pm$ 0.30} & 60.34 {\footnotesize $\pm$ 0.94} \\
%          Label-\ourst & \cellcolor{best}70.55 {\footnotesize $\pm$ 0.22} & 71.54 {\footnotesize $\pm$ 0.18} & 71.07 {\footnotesize $\pm$ 0.28} & \cellcolor{best}69.33 {\footnotesize $\pm$ 0.59}  \\
%     \bottomrule
%     \end{tabular}
%     \label{tab: large}
%     }\hfill
%     \end{minipage}
%    \begin{minipage}{.52\textwidth}
%    \vspace{0.2cm}
%        \centering
%        \includegraphics[width=\linewidth]{figures/eval_negative (3).pdf}
%        \captionof{figure}{Significance of negative graph weight $\beta$ on Cora and Texas datasets where we fix the positive graph weight $\alpha=1$.}
%        \label{fig:beta real} 
%    \end{minipage}
%    % \vspace{-0.5cm}
%     % \hspace{-10pt}
% %     \begin{minipage}{.35\textwidth}
% %     \centering
% %     \captionsetup{font=small}
% %     \caption{Accuracy on different splits of train/valid/test dataset. SGC in CSBM and GCN in Cora.}
% %     % SGC test accuracy (\%) comparison results on sbm of \ourst-Label on different splits of train/valid/test dataset. GCN test accuracy (\%) comparison results on Cora of \ourst-Label on different splits of train/valid/test dataset.
% % % The best results are marked in blue on every dataset.
% %     \centering
% %      \resizebox{0.95\linewidth}{!}{
% %     \begin{tabular}{lcc}
% %     \toprule
% %      Splits & CSBM  & Cora   \\
% %     \midrule
% %     % \midrule
% %     % sbm 0/5/5& 57.00 {\footnotesize $\pm$ 13.36}
% %       2/4/4 & 86.25 {\footnotesize $\pm$ 3.01} & 82.80 {\footnotesize $\pm$ 0.81}\\
% %       4/3/3 & 91.50 {\footnotesize $\pm$ 2.52} & 85.39 {\footnotesize $\pm$ 0.18 }\\
% %       6/4/4 & 91.50 {\footnotesize $\pm$ 6.05} & 87.64 {\footnotesize $\pm$ 0.37 }\\
% %       8/1/1 & \cellcolor{best}99.05 {\footnotesize $\pm$ 1.90} & \cellcolor{best} 94.10 {\footnotesize $\pm$ 0.74} \\
% %     \bottomrule
% %     \end{tabular}
% %     \label{tab: ablation}
% %     }
% %     \end{minipage}
%     % \caption{Caption}
%     % \label{tab:my_label}
% % \vspace{-0.12in}
% \end{figure}

% \begin{wraptable}{r}{.5\linewidth}
% \begin{table}[h]
\centering
% \small
\caption{Test accuracy (\%) comparison results on large scale dataset (Ogbn-ArXiv). 
The best results are marked in blue and the second best results are marked in gray on every layer.}
\begin{adjustbox}{width=0.7\textwidth}
\begin{tabular}{lcccc}
\toprule
 Model             & \#L=2              & \#L=4              & \#L=8            & \#L=16  \\
\midrule


% \midrule

GCN & 67.32 {\footnotesize $\pm$ 0.28} & 67.79 {\footnotesize $\pm$ 0.25} & 65.54 {\footnotesize $\pm$ 0.31} & 59.13 {\footnotesize $\pm$ 0.95}  \\
     BatchNorm & 70.14 {\footnotesize $\pm$ 0.28} & 70.93 {\footnotesize $\pm$ 0.15} & 70.14 {\footnotesize $\pm$ 0.43} & 63.24 {\footnotesize $\pm$ 1.40} \\
     % +LayerNorm& \cellcolor{secondbest}70.53 {\footnotesize $\pm$ 0.19} & \cellcolor{best}71.66 {\footnotesize $\pm$ 0.17} & \cellcolor{secondbest}71.23 {\footnotesize $\pm$ 0.16} & 68.62 {\footnotesize $\pm$ 0.47} \\
     PairNorm & 70.48 {\footnotesize $\pm$ 0.20} & \cellcolor{best}71.59 {\footnotesize $\pm$ 0.17} & \cellcolor{best}71.24 {\footnotesize $\pm$ 0.07} & \cellcolor{best}68.92 {\footnotesize $\pm$ 0.43} \\
     ContraNorm & OOM & OOM & OOM & OOM \\
     DropEdge & 64.07 {\footnotesize $\pm$ 0.32} & 63.92 {\footnotesize $\pm$ 0.27} & 60.74 {\footnotesize $\pm$ 0.45} & 52.52 {\footnotesize $\pm$ 0.34} \\
     Residual & 66.90 {\footnotesize $\pm$ 0.14} & 66.67 {\footnotesize $\pm$ 0.25} & 61.76 {\footnotesize $\pm$ 0.62} & 53.25 {\footnotesize $\pm$ 0.75} \\

     % Feature-\ourst & 67.89 {\footnotesize $\pm$ 0.10} & 68.47 {\footnotesize $\pm$ 0.26} & 65.09 {\footnotesize $\pm$ 0.30} & 60.34 {\footnotesize $\pm$ 0.94} \\
     Label-\ourst & \cellcolor{best}70.55 {\footnotesize $\pm$ 0.22} & 71.54 {\footnotesize $\pm$ 0.18} & 71.07 {\footnotesize $\pm$ 0.28} & \cellcolor{best}69.33 {\footnotesize $\pm$ 0.59}  \\

% % Add more rows as needed
\bottomrule
\end{tabular}
\end{adjustbox}
\label{table: large result}
% \end{table}
\end{wraptable}





% \subsection{Ablation Study}





% Appedix~\ref{app: ablation}. 



