\documentclass{article}

% if you need to pass options to natbib, use, e.g.:
%     \PassOptionsToPackage{numbers, compress}{natbib}
% before loading neurips_2024
\PassOptionsToPackage{sort,numbers}{natbib}
% \linespread{0.98}
% ready for submission
% \usepackage{icml2025}
\usepackage[preprint]{neurips_2024}

%%%%% NEW MATH DEFINITIONS %%%%%

\usepackage{amsmath,amsfonts,bm}
\usepackage{derivative}
% Mark sections of captions for referring to divisions of figures
\newcommand{\figleft}{{\em (Left)}}
\newcommand{\figcenter}{{\em (Center)}}
\newcommand{\figright}{{\em (Right)}}
\newcommand{\figtop}{{\em (Top)}}
\newcommand{\figbottom}{{\em (Bottom)}}
\newcommand{\captiona}{{\em (a)}}
\newcommand{\captionb}{{\em (b)}}
\newcommand{\captionc}{{\em (c)}}
\newcommand{\captiond}{{\em (d)}}

% Highlight a newly defined term
\newcommand{\newterm}[1]{{\bf #1}}

% Derivative d 
\newcommand{\deriv}{{\mathrm{d}}}

% Figure reference, lower-case.
\def\figref#1{figure~\ref{#1}}
% Figure reference, capital. For start of sentence
\def\Figref#1{Figure~\ref{#1}}
\def\twofigref#1#2{figures \ref{#1} and \ref{#2}}
\def\quadfigref#1#2#3#4{figures \ref{#1}, \ref{#2}, \ref{#3} and \ref{#4}}
% Section reference, lower-case.
\def\secref#1{section~\ref{#1}}
% Section reference, capital.
\def\Secref#1{Section~\ref{#1}}
% Reference to two sections.
\def\twosecrefs#1#2{sections \ref{#1} and \ref{#2}}
% Reference to three sections.
\def\secrefs#1#2#3{sections \ref{#1}, \ref{#2} and \ref{#3}}
% Reference to an equation, lower-case.
\def\eqref#1{equation~\ref{#1}}
% Reference to an equation, upper case
\def\Eqref#1{Equation~\ref{#1}}
% A raw reference to an equation---avoid using if possible
\def\plaineqref#1{\ref{#1}}
% Reference to a chapter, lower-case.
\def\chapref#1{chapter~\ref{#1}}
% Reference to an equation, upper case.
\def\Chapref#1{Chapter~\ref{#1}}
% Reference to a range of chapters
\def\rangechapref#1#2{chapters\ref{#1}--\ref{#2}}
% Reference to an algorithm, lower-case.
\def\algref#1{algorithm~\ref{#1}}
% Reference to an algorithm, upper case.
\def\Algref#1{Algorithm~\ref{#1}}
\def\twoalgref#1#2{algorithms \ref{#1} and \ref{#2}}
\def\Twoalgref#1#2{Algorithms \ref{#1} and \ref{#2}}
% Reference to a part, lower case
\def\partref#1{part~\ref{#1}}
% Reference to a part, upper case
\def\Partref#1{Part~\ref{#1}}
\def\twopartref#1#2{parts \ref{#1} and \ref{#2}}

\def\ceil#1{\lceil #1 \rceil}
\def\floor#1{\lfloor #1 \rfloor}
\def\1{\bm{1}}
\newcommand{\train}{\mathcal{D}}
\newcommand{\valid}{\mathcal{D_{\mathrm{valid}}}}
\newcommand{\test}{\mathcal{D_{\mathrm{test}}}}

\def\eps{{\epsilon}}


% Random variables
\def\reta{{\textnormal{$\eta$}}}
\def\ra{{\textnormal{a}}}
\def\rb{{\textnormal{b}}}
\def\rc{{\textnormal{c}}}
\def\rd{{\textnormal{d}}}
\def\re{{\textnormal{e}}}
\def\rf{{\textnormal{f}}}
\def\rg{{\textnormal{g}}}
\def\rh{{\textnormal{h}}}
\def\ri{{\textnormal{i}}}
\def\rj{{\textnormal{j}}}
\def\rk{{\textnormal{k}}}
\def\rl{{\textnormal{l}}}
% rm is already a command, just don't name any random variables m
\def\rn{{\textnormal{n}}}
\def\ro{{\textnormal{o}}}
\def\rp{{\textnormal{p}}}
\def\rq{{\textnormal{q}}}
\def\rr{{\textnormal{r}}}
\def\rs{{\textnormal{s}}}
\def\rt{{\textnormal{t}}}
\def\ru{{\textnormal{u}}}
\def\rv{{\textnormal{v}}}
\def\rw{{\textnormal{w}}}
\def\rx{{\textnormal{x}}}
\def\ry{{\textnormal{y}}}
\def\rz{{\textnormal{z}}}

% Random vectors
\def\rvepsilon{{\mathbf{\epsilon}}}
\def\rvphi{{\mathbf{\phi}}}
\def\rvtheta{{\mathbf{\theta}}}
\def\rva{{\mathbf{a}}}
\def\rvb{{\mathbf{b}}}
\def\rvc{{\mathbf{c}}}
\def\rvd{{\mathbf{d}}}
\def\rve{{\mathbf{e}}}
\def\rvf{{\mathbf{f}}}
\def\rvg{{\mathbf{g}}}
\def\rvh{{\mathbf{h}}}
\def\rvu{{\mathbf{i}}}
\def\rvj{{\mathbf{j}}}
\def\rvk{{\mathbf{k}}}
\def\rvl{{\mathbf{l}}}
\def\rvm{{\mathbf{m}}}
\def\rvn{{\mathbf{n}}}
\def\rvo{{\mathbf{o}}}
\def\rvp{{\mathbf{p}}}
\def\rvq{{\mathbf{q}}}
\def\rvr{{\mathbf{r}}}
\def\rvs{{\mathbf{s}}}
\def\rvt{{\mathbf{t}}}
\def\rvu{{\mathbf{u}}}
\def\rvv{{\mathbf{v}}}
\def\rvw{{\mathbf{w}}}
\def\rvx{{\mathbf{x}}}
\def\rvy{{\mathbf{y}}}
\def\rvz{{\mathbf{z}}}

% Elements of random vectors
\def\erva{{\textnormal{a}}}
\def\ervb{{\textnormal{b}}}
\def\ervc{{\textnormal{c}}}
\def\ervd{{\textnormal{d}}}
\def\erve{{\textnormal{e}}}
\def\ervf{{\textnormal{f}}}
\def\ervg{{\textnormal{g}}}
\def\ervh{{\textnormal{h}}}
\def\ervi{{\textnormal{i}}}
\def\ervj{{\textnormal{j}}}
\def\ervk{{\textnormal{k}}}
\def\ervl{{\textnormal{l}}}
\def\ervm{{\textnormal{m}}}
\def\ervn{{\textnormal{n}}}
\def\ervo{{\textnormal{o}}}
\def\ervp{{\textnormal{p}}}
\def\ervq{{\textnormal{q}}}
\def\ervr{{\textnormal{r}}}
\def\ervs{{\textnormal{s}}}
\def\ervt{{\textnormal{t}}}
\def\ervu{{\textnormal{u}}}
\def\ervv{{\textnormal{v}}}
\def\ervw{{\textnormal{w}}}
\def\ervx{{\textnormal{x}}}
\def\ervy{{\textnormal{y}}}
\def\ervz{{\textnormal{z}}}

% Random matrices
\def\rmA{{\mathbf{A}}}
\def\rmB{{\mathbf{B}}}
\def\rmC{{\mathbf{C}}}
\def\rmD{{\mathbf{D}}}
\def\rmE{{\mathbf{E}}}
\def\rmF{{\mathbf{F}}}
\def\rmG{{\mathbf{G}}}
\def\rmH{{\mathbf{H}}}
\def\rmI{{\mathbf{I}}}
\def\rmJ{{\mathbf{J}}}
\def\rmK{{\mathbf{K}}}
\def\rmL{{\mathbf{L}}}
\def\rmM{{\mathbf{M}}}
\def\rmN{{\mathbf{N}}}
\def\rmO{{\mathbf{O}}}
\def\rmP{{\mathbf{P}}}
\def\rmQ{{\mathbf{Q}}}
\def\rmR{{\mathbf{R}}}
\def\rmS{{\mathbf{S}}}
\def\rmT{{\mathbf{T}}}
\def\rmU{{\mathbf{U}}}
\def\rmV{{\mathbf{V}}}
\def\rmW{{\mathbf{W}}}
\def\rmX{{\mathbf{X}}}
\def\rmY{{\mathbf{Y}}}
\def\rmZ{{\mathbf{Z}}}

% Elements of random matrices
\def\ermA{{\textnormal{A}}}
\def\ermB{{\textnormal{B}}}
\def\ermC{{\textnormal{C}}}
\def\ermD{{\textnormal{D}}}
\def\ermE{{\textnormal{E}}}
\def\ermF{{\textnormal{F}}}
\def\ermG{{\textnormal{G}}}
\def\ermH{{\textnormal{H}}}
\def\ermI{{\textnormal{I}}}
\def\ermJ{{\textnormal{J}}}
\def\ermK{{\textnormal{K}}}
\def\ermL{{\textnormal{L}}}
\def\ermM{{\textnormal{M}}}
\def\ermN{{\textnormal{N}}}
\def\ermO{{\textnormal{O}}}
\def\ermP{{\textnormal{P}}}
\def\ermQ{{\textnormal{Q}}}
\def\ermR{{\textnormal{R}}}
\def\ermS{{\textnormal{S}}}
\def\ermT{{\textnormal{T}}}
\def\ermU{{\textnormal{U}}}
\def\ermV{{\textnormal{V}}}
\def\ermW{{\textnormal{W}}}
\def\ermX{{\textnormal{X}}}
\def\ermY{{\textnormal{Y}}}
\def\ermZ{{\textnormal{Z}}}

% Vectors
\def\vzero{{\bm{0}}}
\def\vone{{\bm{1}}}
\def\vmu{{\bm{\mu}}}
\def\vtheta{{\bm{\theta}}}
\def\vphi{{\bm{\phi}}}
\def\va{{\bm{a}}}
\def\vb{{\bm{b}}}
\def\vc{{\bm{c}}}
\def\vd{{\bm{d}}}
\def\ve{{\bm{e}}}
\def\vf{{\bm{f}}}
\def\vg{{\bm{g}}}
\def\vh{{\bm{h}}}
\def\vi{{\bm{i}}}
\def\vj{{\bm{j}}}
\def\vk{{\bm{k}}}
\def\vl{{\bm{l}}}
\def\vm{{\bm{m}}}
\def\vn{{\bm{n}}}
\def\vo{{\bm{o}}}
\def\vp{{\bm{p}}}
\def\vq{{\bm{q}}}
\def\vr{{\bm{r}}}
\def\vs{{\bm{s}}}
\def\vt{{\bm{t}}}
\def\vu{{\bm{u}}}
\def\vv{{\bm{v}}}
\def\vw{{\bm{w}}}
\def\vx{{\bm{x}}}
\def\vy{{\bm{y}}}
\def\vz{{\bm{z}}}

% Elements of vectors
\def\evalpha{{\alpha}}
\def\evbeta{{\beta}}
\def\evepsilon{{\epsilon}}
\def\evlambda{{\lambda}}
\def\evomega{{\omega}}
\def\evmu{{\mu}}
\def\evpsi{{\psi}}
\def\evsigma{{\sigma}}
\def\evtheta{{\theta}}
\def\eva{{a}}
\def\evb{{b}}
\def\evc{{c}}
\def\evd{{d}}
\def\eve{{e}}
\def\evf{{f}}
\def\evg{{g}}
\def\evh{{h}}
\def\evi{{i}}
\def\evj{{j}}
\def\evk{{k}}
\def\evl{{l}}
\def\evm{{m}}
\def\evn{{n}}
\def\evo{{o}}
\def\evp{{p}}
\def\evq{{q}}
\def\evr{{r}}
\def\evs{{s}}
\def\evt{{t}}
\def\evu{{u}}
\def\evv{{v}}
\def\evw{{w}}
\def\evx{{x}}
\def\evy{{y}}
\def\evz{{z}}

% Matrix
\def\mA{{\bm{A}}}
\def\mB{{\bm{B}}}
\def\mC{{\bm{C}}}
\def\mD{{\bm{D}}}
\def\mE{{\bm{E}}}
\def\mF{{\bm{F}}}
\def\mG{{\bm{G}}}
\def\mH{{\bm{H}}}
\def\mI{{\bm{I}}}
\def\mJ{{\bm{J}}}
\def\mK{{\bm{K}}}
\def\mL{{\bm{L}}}
\def\mM{{\bm{M}}}
\def\mN{{\bm{N}}}
\def\mO{{\bm{O}}}
\def\mP{{\bm{P}}}
\def\mQ{{\bm{Q}}}
\def\mR{{\bm{R}}}
\def\mS{{\bm{S}}}
\def\mT{{\bm{T}}}
\def\mU{{\bm{U}}}
\def\mV{{\bm{V}}}
\def\mW{{\bm{W}}}
\def\mX{{\bm{X}}}
\def\mY{{\bm{Y}}}
\def\mZ{{\bm{Z}}}
\def\mBeta{{\bm{\beta}}}
\def\mPhi{{\bm{\Phi}}}
\def\mLambda{{\bm{\Lambda}}}
\def\mSigma{{\bm{\Sigma}}}

% Tensor
\DeclareMathAlphabet{\mathsfit}{\encodingdefault}{\sfdefault}{m}{sl}
\SetMathAlphabet{\mathsfit}{bold}{\encodingdefault}{\sfdefault}{bx}{n}
\newcommand{\tens}[1]{\bm{\mathsfit{#1}}}
\def\tA{{\tens{A}}}
\def\tB{{\tens{B}}}
\def\tC{{\tens{C}}}
\def\tD{{\tens{D}}}
\def\tE{{\tens{E}}}
\def\tF{{\tens{F}}}
\def\tG{{\tens{G}}}
\def\tH{{\tens{H}}}
\def\tI{{\tens{I}}}
\def\tJ{{\tens{J}}}
\def\tK{{\tens{K}}}
\def\tL{{\tens{L}}}
\def\tM{{\tens{M}}}
\def\tN{{\tens{N}}}
\def\tO{{\tens{O}}}
\def\tP{{\tens{P}}}
\def\tQ{{\tens{Q}}}
\def\tR{{\tens{R}}}
\def\tS{{\tens{S}}}
\def\tT{{\tens{T}}}
\def\tU{{\tens{U}}}
\def\tV{{\tens{V}}}
\def\tW{{\tens{W}}}
\def\tX{{\tens{X}}}
\def\tY{{\tens{Y}}}
\def\tZ{{\tens{Z}}}


% Graph
\def\gA{{\mathcal{A}}}
\def\gB{{\mathcal{B}}}
\def\gC{{\mathcal{C}}}
\def\gD{{\mathcal{D}}}
\def\gE{{\mathcal{E}}}
\def\gF{{\mathcal{F}}}
\def\gG{{\mathcal{G}}}
\def\gH{{\mathcal{H}}}
\def\gI{{\mathcal{I}}}
\def\gJ{{\mathcal{J}}}
\def\gK{{\mathcal{K}}}
\def\gL{{\mathcal{L}}}
\def\gM{{\mathcal{M}}}
\def\gN{{\mathcal{N}}}
\def\gO{{\mathcal{O}}}
\def\gP{{\mathcal{P}}}
\def\gQ{{\mathcal{Q}}}
\def\gR{{\mathcal{R}}}
\def\gS{{\mathcal{S}}}
\def\gT{{\mathcal{T}}}
\def\gU{{\mathcal{U}}}
\def\gV{{\mathcal{V}}}
\def\gW{{\mathcal{W}}}
\def\gX{{\mathcal{X}}}
\def\gY{{\mathcal{Y}}}
\def\gZ{{\mathcal{Z}}}

% Sets
\def\sA{{\mathbb{A}}}
\def\sB{{\mathbb{B}}}
\def\sC{{\mathbb{C}}}
\def\sD{{\mathbb{D}}}
% Don't use a set called E, because this would be the same as our symbol
% for expectation.
\def\sF{{\mathbb{F}}}
\def\sG{{\mathbb{G}}}
\def\sH{{\mathbb{H}}}
\def\sI{{\mathbb{I}}}
\def\sJ{{\mathbb{J}}}
\def\sK{{\mathbb{K}}}
\def\sL{{\mathbb{L}}}
\def\sM{{\mathbb{M}}}
\def\sN{{\mathbb{N}}}
\def\sO{{\mathbb{O}}}
\def\sP{{\mathbb{P}}}
\def\sQ{{\mathbb{Q}}}
\def\sR{{\mathbb{R}}}
\def\sS{{\mathbb{S}}}
\def\sT{{\mathbb{T}}}
\def\sU{{\mathbb{U}}}
\def\sV{{\mathbb{V}}}
\def\sW{{\mathbb{W}}}
\def\sX{{\mathbb{X}}}
\def\sY{{\mathbb{Y}}}
\def\sZ{{\mathbb{Z}}}

% Entries of a matrix
\def\emLambda{{\Lambda}}
\def\emA{{A}}
\def\emB{{B}}
\def\emC{{C}}
\def\emD{{D}}
\def\emE{{E}}
\def\emF{{F}}
\def\emG{{G}}
\def\emH{{H}}
\def\emI{{I}}
\def\emJ{{J}}
\def\emK{{K}}
\def\emL{{L}}
\def\emM{{M}}
\def\emN{{N}}
\def\emO{{O}}
\def\emP{{P}}
\def\emQ{{Q}}
\def\emR{{R}}
\def\emS{{S}}
\def\emT{{T}}
\def\emU{{U}}
\def\emV{{V}}
\def\emW{{W}}
\def\emX{{X}}
\def\emY{{Y}}
\def\emZ{{Z}}
\def\emSigma{{\Sigma}}

% entries of a tensor
% Same font as tensor, without \bm wrapper
\newcommand{\etens}[1]{\mathsfit{#1}}
\def\etLambda{{\etens{\Lambda}}}
\def\etA{{\etens{A}}}
\def\etB{{\etens{B}}}
\def\etC{{\etens{C}}}
\def\etD{{\etens{D}}}
\def\etE{{\etens{E}}}
\def\etF{{\etens{F}}}
\def\etG{{\etens{G}}}
\def\etH{{\etens{H}}}
\def\etI{{\etens{I}}}
\def\etJ{{\etens{J}}}
\def\etK{{\etens{K}}}
\def\etL{{\etens{L}}}
\def\etM{{\etens{M}}}
\def\etN{{\etens{N}}}
\def\etO{{\etens{O}}}
\def\etP{{\etens{P}}}
\def\etQ{{\etens{Q}}}
\def\etR{{\etens{R}}}
\def\etS{{\etens{S}}}
\def\etT{{\etens{T}}}
\def\etU{{\etens{U}}}
\def\etV{{\etens{V}}}
\def\etW{{\etens{W}}}
\def\etX{{\etens{X}}}
\def\etY{{\etens{Y}}}
\def\etZ{{\etens{Z}}}

% The true underlying data generating distribution
\newcommand{\pdata}{p_{\rm{data}}}
\newcommand{\ptarget}{p_{\rm{target}}}
\newcommand{\pprior}{p_{\rm{prior}}}
\newcommand{\pbase}{p_{\rm{base}}}
\newcommand{\pref}{p_{\rm{ref}}}

% The empirical distribution defined by the training set
\newcommand{\ptrain}{\hat{p}_{\rm{data}}}
\newcommand{\Ptrain}{\hat{P}_{\rm{data}}}
% The model distribution
\newcommand{\pmodel}{p_{\rm{model}}}
\newcommand{\Pmodel}{P_{\rm{model}}}
\newcommand{\ptildemodel}{\tilde{p}_{\rm{model}}}
% Stochastic autoencoder distributions
\newcommand{\pencode}{p_{\rm{encoder}}}
\newcommand{\pdecode}{p_{\rm{decoder}}}
\newcommand{\precons}{p_{\rm{reconstruct}}}

\newcommand{\laplace}{\mathrm{Laplace}} % Laplace distribution

\newcommand{\E}{\mathbb{E}}
\newcommand{\Ls}{\mathcal{L}}
\newcommand{\R}{\mathbb{R}}
\newcommand{\emp}{\tilde{p}}
\newcommand{\lr}{\alpha}
\newcommand{\reg}{\lambda}
\newcommand{\rect}{\mathrm{rectifier}}
\newcommand{\softmax}{\mathrm{softmax}}
\newcommand{\sigmoid}{\sigma}
\newcommand{\softplus}{\zeta}
\newcommand{\KL}{D_{\mathrm{KL}}}
\newcommand{\Var}{\mathrm{Var}}
\newcommand{\standarderror}{\mathrm{SE}}
\newcommand{\Cov}{\mathrm{Cov}}
% Wolfram Mathworld says $L^2$ is for function spaces and $\ell^2$ is for vectors
% But then they seem to use $L^2$ for vectors throughout the site, and so does
% wikipedia.
\newcommand{\normlzero}{L^0}
\newcommand{\normlone}{L^1}
\newcommand{\normltwo}{L^2}
\newcommand{\normlp}{L^p}
\newcommand{\normmax}{L^\infty}

\newcommand{\parents}{Pa} % See usage in notation.tex. Chosen to match Daphne's book.

\DeclareMathOperator*{\argmax}{arg\,max}
\DeclareMathOperator*{\argmin}{arg\,min}

\DeclareMathOperator{\sign}{sign}
\DeclareMathOperator{\Tr}{Tr}
\let\ab\allowbreak


% to compile a preprint version, e.g., for submission to arXiv, add add the
% [preprint] option:
% \usepackage[preprint]{neurips_2024}


% to compile a camera-ready version, add the [final] option, e.g.:
%     \usepackage[final]{neurips_2024}


% to avoid loading the natbib package, add option nonatbib:
%    \usepackage[nonatbib]{neurips_2024}

% \usepackage{tikz}
% \def\checkmark{\tikz\fill[scale=0.4](0,.35) -- (.25,0) -- (1,.7) -- (.25,.15) -- cycle;} 
\usepackage{bbding}
\usepackage{pifont}
\usepackage{wasysym}
\usepackage{utfsym}
\usepackage[utf8]{inputenc} % allow utf-8 input
\usepackage[T1]{fontenc}    % use 8-bit T1 fonts
\usepackage{hyperref}       % hyperlinks
\usepackage{url}            % simple URL typesetting
\usepackage{booktabs}       % professional-quality tables
\usepackage{amsfonts}       % blackboard math symbols
\usepackage{nicefrac}       % compact symbols for 1/2, etc.
\usepackage{microtype}      % microtypography
% \usepackage{xcolor}         % colors

\usepackage{booktabs}
% \usepackage[table]{xcolor} % For cell coloring
\usepackage{colortbl}
\definecolor{best}{RGB}{173,216,230} % Light blue for the best result
\definecolor{secondbest}{RGB}{220,220,220} % Light gray for the second 

\usepackage{graphicx}
\usepackage{subcaption}
\usepackage{amsfonts}
\usepackage{amssymb}
\usepackage{bbold}
\usepackage{amsmath}
\usepackage{xspace} % for name abbrev
\usepackage{array}
\usepackage{subcaption}
\usepackage{multirow} % 表格中的跨行
\usepackage{adjustbox} % 表格的大小调整
\usepackage[textsize=tiny]{todonotes}
\usepackage{wrapfig}
\usepackage{etoc}
\usepackage{enumitem}
\etocdepthtag.toc{mtchapter}
\etocsettagdepth{mtchapter}{subsection}
\etocsettagdepth{mtappendix}{none}
\usepackage[most]{tcolorbox}



%%%%%%%%%%%%%%%%%%%%%%%%%%%%%%%%
% THEOREMS
%%%%%%%%%%%%%%%%%%%%%%%%%%%%%%%%
% \theoremstyle{plain}
\newtheorem{theorem}{Theorem}[section]
\newtheorem{proposition}[theorem]{Proposition}
\newtheorem{lemma}[theorem]{Lemma}
\newtheorem{corollary}[theorem]{Corollary}
% \theoremstyle{definition}
\newtheorem{definition}[theorem]{Definition}
\newtheorem{assumption}[theorem]{Assumption}
% \theoremstyle{remark}
\newtheorem{remark}[theorem]{Remark}


% A Unified Framework for Understanding and Mitigating Feature Oversmoothing
\newcommand{\ours}[0]{\texttt{SBP}\xspace}
\newcommand{\ourst}[0]{\text{SBP}\xspace}	% ours in normal text 
\newcommand{\oursfull}[0]{\textbf{S}tructural \textbf{B}alance \textbf{P}ropagation
}
\newcommand{\theoremfull}[0]{Signed Graph Framework
}
% \textbf{S}igned \textbf{G}raph \textbf{F}ramework
\newcommand{\pgh}[1]{\textcolor{brown}{#1}}
\newcommand{\ngh}[1]{\textcolor{blue}{#1}}
\DeclareMathOperator{\diag}{diag}
\newcommand{\xinyic}[1]{\todo[color=orange!40]{#1}}
\newcommand{\jq}[1]{\textcolor{black}{#1}} %
% \newcommand{\yf}[1]{\todo[color=orange]{#1}}%
\newcommand{\yf}[1]{\textcolor{red}{\textbf{YW}:#1}} %
\definecolor{tkcolor}{RGB}{224,223,255}
\newtcolorbox{takeaways}[2][]{
	width=\columnwidth,
	colback = tkcolor, 
	colframe = tkcolor, 
	boxsep=0pt,left=10pt,right=10pt,top=0pt,bottom=0pt,
	fontupper=\linespread{0.9}\selectfont,
	title=#2,#1}
% The \author macro works with any number of authors. There are two commands
% used to separate the names and addresses of multiple authors: \And and \AND.
%
% Using \And between authors leaves it to LaTeX to determine where to break the
% lines. Using \AND forces a line break at that point. So, if LaTeX puts 3 of 4
% authors names on the first line, and the last on the second line, try using
% \AND instead of \And before the third author name.
\title{ Oversmoothing as Loss of Sign: Towards Structural Balance in Graph Neural Networks}

\author{Jiaqi Wang$^{1}$ \thanks{Equal contribution.} \qquad Xinyi Wu$^2$\footnotemark[1]  \qquad James Cheng$^{1}$\qquad Yifei Wang$^{3}$\\  
$^{1}$ The Chinese University of Hong Kong \qquad $^2$MIT IDSS \& LIDS \qquad $^3$MIT CSAIL\\
\texttt{
\{jqwang23, jcheng\}@cse.cuhk.edu.hk} \qquad 
\texttt{\{xinyiwu,yifei\_w\}@mit.edu}
}
\begin{document}
\maketitle
% \twocolumn[
% \icmltitle{ Oversmoothing as Loss of Sign: Towards Structural Balance \\ in Graph Neural Networks}
% \author{%
%   Jiaqi Wang $^{*1}$, \, Xinyi Wu \thanks{Equal Contribution}\, $^2$, \,James Cheng $^1$, \,Yifei Wang $^2$\thanks{Corresponding Author}\\
%   $^1$The Chinese University of Hong Kong \, $^2$MIT
% }

% It is OKAY to include author information, even for blind
% submissions: the style file will automatically remove it for you
% unless you've provided the [accepted] option to the icml2025
% package.

% List of affiliations: The first argument should be a (short)
% identifier you will use later to specify author affiliations
% Academic affiliations should list Department, University, City, Region, Country
% Industry affiliations should list Company, City, Region, Country

% You can specify symbols, otherwise they are numbered in order.
% Ideally, you should not use this facility. Affiliations will be numbered
% in order of appearance and this is the preferred way.
% \icmlsetsymbol{equal}{*}

% \begin{icmlauthorlist}
% \icmlauthor{Firstname1 Lastname1}{equal,yyy}
% \icmlauthor{Firstname2 Lastname2}{equal,yyy,comp}
% \icmlauthor{Firstname3 Lastname3}{comp}
% \icmlauthor{Firstname4 Lastname4}{sch}
% \icmlauthor{Firstname5 Lastname5}{yyy}
% \icmlauthor{Firstname6 Lastname6}{sch,yyy,comp}
% \icmlauthor{Firstname7 Lastname7}{comp}
% %\icmlauthor{}{sch}
% \icmlauthor{Firstname8 Lastname8}{sch}
% \icmlauthor{Firstname8 Lastname8}{yyy,comp}
% %\icmlauthor{}{sch}
% %\icmlauthor{}{sch}
% \end{icmlauthorlist}

% \icmlaffiliation{yyy}{Department of XXX, University of YYY, Location, Country}
% \icmlaffiliation{comp}{Company Name, Location, Country}
% \icmlaffiliation{sch}{School of ZZZ, Institute of WWW, Location, Country}

% \icmlcorrespondingauthor{Firstname1 Lastname1}{first1.last1@xxx.edu}
% \icmlcorrespondingauthor{Firstname2 Lastname2}{first2.last2@www.uk}

% You may provide any keywords that you
% find helpful for describing your paper; these are used to populate
% the "keywords" metadata in the PDF but will not be shown in the document
% \icmlkeywords{Machine Learning, ICML}

% \vskip 0.3in
% ]

% this must go after the closing bracket ] following \twocolumn[ ...

% This command actually creates the footnote in the first column
% listing the affiliations and the copyright notice.
% The command takes one argument, which is text to display at the start of the footnote.
% The \icmlEqualContribution command is standard text for equal contribution.
% Remove it (just {}) if you do not need this facility.

%\printAffiliationsAndNotice{}  % leave blank if no need to mention equal contribution
% \printAffiliationsAndNotice{\icmlEqualContribution} % otherwise use the standard text.



% \maketitle

\begin{abstract}


The choice of representation for geographic location significantly impacts the accuracy of models for a broad range of geospatial tasks, including fine-grained species classification, population density estimation, and biome classification. Recent works like SatCLIP and GeoCLIP learn such representations by contrastively aligning geolocation with co-located images. While these methods work exceptionally well, in this paper, we posit that the current training strategies fail to fully capture the important visual features. We provide an information theoretic perspective on why the resulting embeddings from these methods discard crucial visual information that is important for many downstream tasks. To solve this problem, we propose a novel retrieval-augmented strategy called RANGE. We build our method on the intuition that the visual features of a location can be estimated by combining the visual features from multiple similar-looking locations. We evaluate our method across a wide variety of tasks. Our results show that RANGE outperforms the existing state-of-the-art models with significant margins in most tasks. We show gains of up to 13.1\% on classification tasks and 0.145 $R^2$ on regression tasks. All our code and models will be made available at: \href{https://github.com/mvrl/RANGE}{https://github.com/mvrl/RANGE}.

\end{abstract}


\label{sec: abstract}
\section{Introduction}

Video generation has garnered significant attention owing to its transformative potential across a wide range of applications, such media content creation~\citep{polyak2024movie}, advertising~\citep{zhang2024virbo,bacher2021advert}, video games~\citep{yang2024playable,valevski2024diffusion, oasis2024}, and world model simulators~\citep{ha2018world, videoworldsimulators2024, agarwal2025cosmos}. Benefiting from advanced generative algorithms~\citep{goodfellow2014generative, ho2020denoising, liu2023flow, lipman2023flow}, scalable model architectures~\citep{vaswani2017attention, peebles2023scalable}, vast amounts of internet-sourced data~\citep{chen2024panda, nan2024openvid, ju2024miradata}, and ongoing expansion of computing capabilities~\citep{nvidia2022h100, nvidia2023dgxgh200, nvidia2024h200nvl}, remarkable advancements have been achieved in the field of video generation~\citep{ho2022video, ho2022imagen, singer2023makeavideo, blattmann2023align, videoworldsimulators2024, kuaishou2024klingai, yang2024cogvideox, jin2024pyramidal, polyak2024movie, kong2024hunyuanvideo, ji2024prompt}.


In this work, we present \textbf{\ours}, a family of rectified flow~\citep{lipman2023flow, liu2023flow} transformer models designed for joint image and video generation, establishing a pathway toward industry-grade performance. This report centers on four key components: data curation, model architecture design, flow formulation, and training infrastructure optimization—each rigorously refined to meet the demands of high-quality, large-scale video generation.


\begin{figure}[ht]
    \centering
    \begin{subfigure}[b]{0.82\linewidth}
        \centering
        \includegraphics[width=\linewidth]{figures/t2i_1024.pdf}
        \caption{Text-to-Image Samples}\label{fig:main-demo-t2i}
    \end{subfigure}
    \vfill
    \begin{subfigure}[b]{0.82\linewidth}
        \centering
        \includegraphics[width=\linewidth]{figures/t2v_samples.pdf}
        \caption{Text-to-Video Samples}\label{fig:main-demo-t2v}
    \end{subfigure}
\caption{\textbf{Generated samples from \ours.} Key components are highlighted in \textcolor{red}{\textbf{RED}}.}\label{fig:main-demo}
\end{figure}


First, we present a comprehensive data processing pipeline designed to construct large-scale, high-quality image and video-text datasets. The pipeline integrates multiple advanced techniques, including video and image filtering based on aesthetic scores, OCR-driven content analysis, and subjective evaluations, to ensure exceptional visual and contextual quality. Furthermore, we employ multimodal large language models~(MLLMs)~\citep{yuan2025tarsier2} to generate dense and contextually aligned captions, which are subsequently refined using an additional large language model~(LLM)~\citep{yang2024qwen2} to enhance their accuracy, fluency, and descriptive richness. As a result, we have curated a robust training dataset comprising approximately 36M video-text pairs and 160M image-text pairs, which are proven sufficient for training industry-level generative models.

Secondly, we take a pioneering step by applying rectified flow formulation~\citep{lipman2023flow} for joint image and video generation, implemented through the \ours model family, which comprises Transformer architectures with 2B and 8B parameters. At its core, the \ours framework employs a 3D joint image-video variational autoencoder (VAE) to compress image and video inputs into a shared latent space, facilitating unified representation. This shared latent space is coupled with a full-attention~\citep{vaswani2017attention} mechanism, enabling seamless joint training of image and video. This architecture delivers high-quality, coherent outputs across both images and videos, establishing a unified framework for visual generation tasks.


Furthermore, to support the training of \ours at scale, we have developed a robust infrastructure tailored for large-scale model training. Our approach incorporates advanced parallelism strategies~\citep{jacobs2023deepspeed, pytorch_fsdp} to manage memory efficiently during long-context training. Additionally, we employ ByteCheckpoint~\citep{wan2024bytecheckpoint} for high-performance checkpointing and integrate fault-tolerant mechanisms from MegaScale~\citep{jiang2024megascale} to ensure stability and scalability across large GPU clusters. These optimizations enable \ours to handle the computational and data challenges of generative modeling with exceptional efficiency and reliability.


We evaluate \ours on both text-to-image and text-to-video benchmarks to highlight its competitive advantages. For text-to-image generation, \ours-T2I demonstrates strong performance across multiple benchmarks, including T2I-CompBench~\citep{huang2023t2i-compbench}, GenEval~\citep{ghosh2024geneval}, and DPG-Bench~\citep{hu2024ella_dbgbench}, excelling in both visual quality and text-image alignment. In text-to-video benchmarks, \ours-T2V achieves state-of-the-art performance on the UCF-101~\citep{ucf101} zero-shot generation task. Additionally, \ours-T2V attains an impressive score of \textbf{84.85} on VBench~\citep{huang2024vbench}, securing the top position on the leaderboard (as of 2025-01-25) and surpassing several leading commercial text-to-video models. Qualitative results, illustrated in \Cref{fig:main-demo}, further demonstrate the superior quality of the generated media samples. These findings underscore \ours's effectiveness in multi-modal generation and its potential as a high-performing solution for both research and commercial applications.
\label{sec: introduction}
\section{Background} \label{section:LLM}

% \subsection{Large Language Model (LLM)}   

Figure~\ref{fig:LLaMA_model}(a) shows that a decoder-only LLM initially processes a user prompt in the “prefill” stage and subsequently generates tokens sequentially during the “decoding” stage.
Both stages contain an input embedding layer, multiple decoder transformer blocks, an output embedding layer, and a sampling layer.
Figure~\ref{fig:LLaMA_model}(b) demonstrates that the decoder transformer blocks consist of a self attention and a feed-forward network (FFN) layer, each paired with residual connection and normalization layers. 

% Differentiate between encoder/decoder, explain why operation intensity is low, explain the different parts of a transformer block. Discuss Table II here. 

% Explain the architecture with Llama2-70B.

% \begin{table}[thb]
% \renewcommand\arraystretch{1.05}
% \centering
% % \vspace{-5mm}
%     \caption{ML Model Parameter Size and Operational Intensity}
%     \vspace{-2mm}
%     \small
%     \label{tab:ML Model Parameter Size and Operational Intensity}    
%     \scalebox{0.95}{
%         \begin{tabular}{|c|c|c|c|c|}
%             \hline
%             & Llama2 & BLOOM & BERT & ResNet \\
%             Model & (70B) & (176B) & & 152 \\
%             \hline
%             Parameter Size (GB) & 140 & 352 & 0.17 & 0.16 \\
%             \hline
%             Op Intensity (Ops/Byte) & 1 & 1 & 282 & 346 \\
%             \hline
%           \end{tabular}
%     }
% \vspace{-3mm}
% \end{table}

% {\fontsize{8pt}{11pt}\selectfont 8pt font size test Memory Requirement}

\begin{figure}[t]
    \centering
    \includegraphics[width=8cm]{Figure/LLaMA_model_new_new.pdf}
    \caption{(a) Prefill stage encodes prompt tokens in parallel. Decoding stage generates output tokens sequentially.
    (b) LLM contains N$\times$ decoder transformer blocks. 
    (c) Llama2 model architecture.}
    \label{fig:LLaMA_model}
\end{figure}

Figure~\ref{fig:LLaMA_model}(c) demonstrates the Llama2~\cite{touvron2023llama} model architecture as a representative LLM.
% The self attention layer requires three GEMVs\footnote{GEMVs in multi-head attention~\cite{attention}, narrow GEMMs in grouped-query attention~\cite{gqa}.} to generate query, key and value vectors.
In the self-attention layer, query, key and value vectors are generated by multiplying input vector to corresponding weight matrices.
These matrices are segmented into multiple heads, representing different semantic dimensions.
The query and key vectors go though Rotary Positional Embedding (RoPE) to encode the relative positional information~\cite{rope-paper}.
Within each head, the generated key and value vectors are appended to their caches.
The query vector is multiplied by the key cache to produce a score vector.
After the Softmax operation, the score vector is multiplied by the value cache to yield the output vector.
The output vectors from all heads are concatenated and multiplied by output weight matrix, resulting in a vector that undergoes residual connection and Root Mean Square layer Normalization (RMSNorm)~\cite{rmsnorm-paper}.
The residual connection adds up the input and output vectors of a layer to avoid vanishing gradient~\cite{he2016deep}.
The FFN layer begins with two parallel fully connections, followed by a Sigmoid Linear Unit (SiLU), and ends with another fully connection.
\label{sec: background}
% sign framework
\section{A Signed Graph Perspective on Existing  Oversmoothing Countermeasures} 
\label{sec: signed pers}
% \jq{notaion, re-define.}


% In this section, we first introduce the notion of signed graphs and how message-passing over signed graphs works. Then we present the theoretical properties of the asymptotic behaviors of signed graph propagation. 
%
% Finally, we leverage these theoretical tools from signed graphs to provide a unified framework for existing anti-oversmoothing techniques. 
% \yf{we leverage signed graph to analyze oversmoothing ,not to address it}

% In this paper, we define the 

%
% The positive neighbor set is $N_i^+$ and the negative neighbor set is $N_i^-$.
% The positive neighbor set is $N_i^+$ and the positive degree $d_i=|N_i^+|$.
% The negative neighbor set is $N_i^-$ and the negative degree $d_i=|N_i^-|$.
% The raw normalized positive and negative adjacency matrix are $\pgh{\hat{A}^+}$ and $\ngh{\hat{A}^-}$, respectively.
%



% \subsection{Interpreting Regularization Techniques via Signed Graph}
% \label{sec: regularization analysis}




% \textbf{Contextual Stochastic Block Models.} 
% We focus on the $2$-CSBM$(N, p, q, \mu_1, \mu_2, \sigma^2 )$ to explain the methods following~\cite{sbm_xinyi}.

In this section, we revisit three popular types of anti-oversmoothing methods and reinterpret them through the lens of signed graph propagation in the form of (\ref{eq: sign_overall}). We find that all of these methods can be attributed to some kind of signed graph design \(\mathcal{G}_s\) by introducing positive and negative edges to the original graph.
% introduce a signed graph perspective to unify popular anti-oversmoothing techniques.
We summarize eight specific methods with their corresponding positive and negative graphs in Table~\ref{tab: framework}.
% \vspace{-1ex}
\subsection{Normalization Techniques}
Normalization operates the node features after each message-passing step by centering them with zero mean and unit variance (up to a scale) with different strategies. 
% of normalization methods 
A few representative methods include BatchNorm~\citep{batchnorm}, PairNorm~\citep{Zhao2020PairNorm}, 
% LayerNorm~\cite{layernorm} 
and ContraNorm~\citep{contranorm}, where PairNorm and ContraNorm were proposed specifically to address the oversmoothing issue in GNNs.
Further details on these methods are provided in Appendix~\ref{app: previous}.
% Specifically, BatchNorm centers the node representations $X$ to have zero mean and unit variance across nodes for each feature, which can be written as BatchNorm($x_i$) \(=\frac{1}{\sqrt{\sigma^2 + \epsilon}}(x_i - \frac{1}{n}\Sigma_{i=1}^n x_i)\) where $ \epsilon > 0$ 
% and $\sigma^2$ is the variance of the feature across all nodes.
% %
% Meanwhile, PairNorm is a normalization technique specifically developed for GNNs to combat oversmoothing, where its only difference from BatchNorm is that PairNorm scales all the entries in $X$ using the same number rather than scaling each column by its own variance. It can be written as PairNorm($x_i$) \(=\frac{s}{\sqrt{\Gamma
% ^2 + \epsilon}}(x_i - \frac{1}{n}\Sigma_{i=1}^n x_i)\) where $\Gamma = \|(\hat{A}- \mathbb{1}_n \mathbb{1}_n^T/n)X \|_F/\sqrt{n} $ and $s$ is a scalar.
% %
% Apart from these two methods, ContraNorm is inspired by the uniformity loss from contrastive learning, aiming to alleviate dimensional feature collapse.
% For simplicity, we consider the spectral version of ContraNorm that takes the following form: ContraNorm($X$) $= (1 + \alpha) X- \alpha /\tau(X X^{T}) X \,$ where $\alpha\in(0,1)$ and $\tau>0$ are hyperparameters.
% Proposition 2) 
% \xinyic{better with an exact reference: which Theorem in ContraNorm} 
% without additional regularization and LayerNorm that are used additionally in practice, 
% \yf{strength/weakness not discussed}
% Besides, it can assigns suitable weights to positive and negative edges by selecting different $\alpha$ and $\tau$.
% \xinyic{what does "selecting different hyperparameters" mean here}
Despite the differences in motivation and implementation, all the three normalization methods can be seen as a signed graph propagation with different designs of the negative graph: 
\begin{theorem}
     BatchNorm, PairNorm and ContraNorm can be interpreted as signed graph propagation defined in (\ref{eq: sign_overall}), sharing the same raw normalized positive adjacency matrix $\pgh{\hat{A}^+=\hat{A}}$ while having different raw normalized negative adjacency matrices transformed from $\hat{A}^+$, that is,  $\ngh{\hat{A}^-=\frac{\mathbb{1}_n \mathbb{1}_n^T}{n}  \hat{A}}$ for BatchNorm and PairNorm, and $\ngh{\hat{A}^-=(X X^{T}) \hat{A} }$ for ContraNorm.
\end{theorem}
The result shows that PairNorm shares the same fixed positive and negative graphs (up to scale) as BatchNorm. In contrast, ContraNorm extends the negative graph to an adaptive one based on similarities in node features. 
% We can see that $\pgh{\hat{A}}$ is again the positive graph and $\ngh{(X X^T)\hat{A}}$ is the negative graph in the corresponding signed graph propagation.
% In particular, the repulsion of ContraNorm is affected by both feature similarities and node degrees. 
% Consider the update:
% \begin{equation}
%     \label{eq: bn sign}
%     \hat{X}= (\pgh{A}-\ngh{\frac{X X^T}{n} A}) X\,,
% \end{equation}
% After one signed graph propagation, the edge weight changes from $\{0,1\}$ to $\{-\frac{p-q}{2}, 1- \frac{p-q}{2}\}$, so the SB can be expressed as: 
% \begin{equation}
%     SB_{CN}= (1-\frac{p-q}{2})p + \frac{p-q}{2} (1-q) = p + \frac{p-q}{2} (1-p-q).
% \end{equation}







\subsection{Augmentation-Based Methods}
% \paragraph{DropEdge}
Node or edge dropping~\citep{dropedge} is another popular type of method to combat oversmoothing.
% and they can also be interpreted as having signed graph propagation. 
In particular, we denote $A_m\in \{0,-1\}^{n\times n}$ where $(A_m)_{i,j}=1$ if the edge $\{i,j\}$ is dropped and otherwise 0. 
Then the signed graph induced by (randomly) dropping edges can be formulated as \(\mathcal{G}_{drop}=\{A,A_m,X\}\). 
% \begin{equation}
    % \label{drop sign}
    % $\hat{X} =\pgh{A} X - \ngh{A_{m}}X$.
    % =(\pgh{A}  -\ngh{A_{m}})X.
% \end{equation}
The negative adjacency matrix $A_m$, while created through random generation, effectively helps alleviate oversmoothing in practice.
% demonstrates that adding negative edges can 

% Consider the update:
% \begin{equation}
%     \label{eq: bn sign}
%     \hat{X}= (\pgh{A}-\ngh{A_m}) X\,,
% \end{equation}
% Assume that in $\ngh{A_m}$, for any two nodes in the graph, if they are from the same class, they are connected by an edge independently with probability $s$, or if they are from different classes, the probability is $t$.
% % After one signed graph propagation, the edge distribution probabilities are $q-t$ from the same label and $p-s$ from the different labels, so 
% The SB can be expressed as: 
% \begin{equation}
%     SB_{ED}= 1 \times (p-s) - 0 \times (1-q) + 1 \times t = p-s+t.
% \end{equation}


  
% Furthermore, residual connection precisely corresponds to signed graph propagation, as described in~\eqref{eq: sign_overall}, particularly when $\alpha=\beta$.

% \begin{equation}
%     \label{eq: appnp}
%     \begin{split}
%         \hat{X}^{(k+1)} &= (1-\alpha)X^{(0)}  + \alpha \hat{A} X ^{(k)} \\
%         &= \pgh{\Sigma_{i=0}^{k+1}\alpha^i\hat{A}^i} X^{(0)} -\ngh{\alpha \Sigma_{j=0}^{k}\alpha^j\hat{A}^j }X^{(0)}\,.
%     \end{split}
% \end{equation}

\subsection{Residual Connections} % 
Besides normalization layers and edge-dropping, residual connections can also be seen through the lens of signed graph propagation. Based on different combinations of layers in this class, we provide analysis for the following three types of residual connections: First, the standard residual connection~\citep{dgc,Chen2020SimpleAD}, which directly combines the previous and the current layer features together.
% It can be formulated as: \( \hat{X} = (1-\alpha)X  + \alpha \hat{A} X  \).
% \begin{equation}
%     \label{eq: residual sign}
% \end{equation} = X + \alpha \pgh{\hat{A}} X -\alpha \ngh{I} X
% For residual connections, the positive adjacency matrix is $\pgh{\hat{A}}$ and the negative adjacency matrix $\ngh{I}$ in the corresponding signed graph propagation.
% \paragraph{APPNP}
% We reformulate the method APPNP~\citep{appap} as the signed propagation form of the initial node feature. 
Another type combines the current layer features together with the initial features, such as APPNP~\citep{appap} or GCNII~\cite{GCNII}.
% \begin{equation}
% , written as: \( \hat{X}^{(k+1)} = (1-\alpha)X^{(0)}  + \alpha \hat{A} X ^{(k)} \) .
% \end{equation}
% \begin{theorem}
% With $\hat{A}^+=\Sigma_{i=0}^{k+1}\alpha^i\hat{A}^i$ and $\hat{A}^-=\alpha \Sigma_{j=0}^{k}\alpha^j\hat{A}^j$, the propagation process of APPNP following the signed graph propagation.
% \end{theorem}
In addition to combining with the previous or the initial layer features, there is a third type of residual connections which integrates intermediate layer features, such as JKNET~\citep{jknet} and DAGNN~\citep{dagnn}.
% \paragraph{\jq{JKNET and DAGNN}}
% JKNET is a deep graph neural network which exploits information from neighborhoods of differing locality. 
% JKNET selectively combines aggregations from different layers through operations such as concatenation or max-pooling at the output, i.e., the representations "jump" to the last layer.
% Using attention mechanism for combination at the last layer, the $k+1$-layer propagation result of JKNET can be written as:
% \begin{equation}
%     \label{eq:jk-net}
%     \begin{split}
%          X^{(k+1)} &= \alpha_0 X^{(0)}  + \alpha_1  X ^{(1)} + \cdots \alpha_k X^{(k)}\\
%         &= \Sigma_{i=0}^k\alpha_i \hat{A}^i X^{(0)}\,,
%     \end{split}
% \end{equation}
% where $\alpha_0, \alpha_1, \cdots, \alpha_{k}$ are the learnable fusion weights with $\Sigma_{i=0}^k\alpha_i=1$.
% Deep Adaptive Graph Neural Networks (DAGNN) tries to adaptively add all the features from the previous layer to the current layer features with additional learnable coefficients. 
More details about these methods can be found in Appendix~\ref{app: residual}.
Formally, we establish the following result that these three types of residual connections can all be seen as signed graph propagation:
% After decoupling representation transformation and propagation, the propagation mechanism of DAGNN is similar to that of JKNET.
% \begin{equation}
%     \label{eq:dagnn}
%          X^{(k+1)} = \Sigma_{i=0}^k\alpha_i \hat{A}^i H^{(0)}, \,H^{(0)}=f_\theta(X^{(0)})
% \end{equation}
% $ H^{(0)}=f_\theta(X^{(0)})$ ) is the non-linear feature transformation using an MLP
% network, which is conducted before the propagation process and $\alpha_0, \alpha_1, \cdots, \alpha_{k}$ are the learnable fusion weights with $\Sigma_{i=0}^k\alpha_i=1$. \jq{double check the correctness.}
\begingroup
\setlength{\abovedisplayskip}{3pt} % 上方间距缩小
\setlength{\belowdisplayskip}{3pt}
\begin{theorem}
% \vspace{-3ex}
With \pgh{$\hat{A}^+=\hat{A}$} and \ngh{$\hat{A}^-=I$}, the standard residual connections follows the signed graph propagation~(\ref{eq: sign_overall}).
With \pgh{$\hat{A}^+=\Sigma_{i=0}^{k+1}\alpha^i\hat{A}^i$} and \ngh{$\hat{A}^-=\alpha \Sigma_{j=0}^{k}\alpha^j\hat{A}^j$}, APPNP follows the signed graph propagation~(\ref{eq: sign_overall}).
    With \pgh{$\hat{A}^+=\Sigma_{i=0}^{k-1}\alpha^i\hat{A}^i+\hat{A}^k$} and \ngh{$\hat{A}^-=\Sigma_{j=0}^{k-1}\alpha^j\hat{A}^k$}, JKNET and DAGNN follows the signed graph propagation~(\ref{eq: sign_overall}).
\end{theorem}
\endgroup
% \vspace{-5pt}

\textbf{Discussion.} In summary, we establish a unifying perspective in which normalization, edge dropping, and residual connections can all be interpreted as instances of signed graph propagation, even though this structure is not explicitly recognized. Notably, for these methods, while their positive adjacency matrices typically reflect the original graph structure, the negative adjacency matrices are often constructed heuristically. As a result, the interaction between signed graph structures and node feature dynamics remains insufficiently understood motivating a systematic theoretical analysis of the asymptotic behaviors of signed graph propagation.
% , thus inspiring us to theoretically and systematically analyze oversmoothing through the signed lens.
% Moreover, these methods create positive and negative adjacency matrices based on the different heuristics, remaining unclear about the complex interplay between the signed graph structure and the resulting node feature dynamics. 
% thus may initially provide some benefits in terms of preventing oversmoothing.
% However, while empirically constructing the signed graph propagation, 
% there is still a lack of theoretical guidance to fully understand the .
%
\label{sec: sign graph}
Effective human-robot cooperation in CoNav-Maze hinges on efficient communication. Maximizing the human’s information gain enables more precise guidance, which in turn accelerates task completion. Yet for the robot, the challenge is not only \emph{what} to communicate but also \emph{when}, as it must balance gathering information for the human with pursuing immediate goals when confident in its navigation.

To achieve this, we introduce \emph{Information Gain Monte Carlo Tree Search} (IG-MCTS), which optimizes both task-relevant objectives and the transmission of the most informative communication. IG-MCTS comprises three key components:
\textbf{(1)} A data-driven human perception model that tracks how implicit (movement) and explicit (image) information updates the human’s understanding of the maze layout.
\textbf{(2)} Reward augmentation to integrate multiple objectives effectively leveraging on the learned perception model.
\textbf{(3)} An uncertainty-aware MCTS that accounts for unobserved maze regions and human perception stochasticity.
% \begin{enumerate}[leftmargin=*]
%     \item A data-driven human perception model that tracks how implicit (movement) and explicit (image transmission) information updates the human’s understanding of the maze layout.
%     \item Reward augmentation to integrate multiple objectives effectively leveraging on the learned perception model.
%     \item An uncertainty-aware MCTS that accounts for unobserved maze regions and human perception stochasticity.
% \end{enumerate}

\subsection{Human Perception Dynamics}
% IG-MCTS seeks to optimize the expected novel information gained by the human through the robot’s actions, including both movement and communication. Achieving this requires a model of how the human acquires task-relevant information from the robot.

% \subsubsection{Perception MDP}
\label{sec:perception_mdp}
As the robot navigates the maze and transmits images, humans update their understanding of the environment. Based on the robot's path, they may infer that previously assumed blocked locations are traversable or detect discrepancies between the transmitted image and their map.  

To formally capture this process, we model the evolution of human perception as another Markov Decision Process, referred to as the \emph{Perception MDP}. The state space $\mathcal{X}$ represents all possible maze maps. The action space $\mathcal{S}^+ \times \mathcal{O}$ consists of the robot's trajectory between two image transmissions $\tau \in \mathcal{S}^+$ and an image $o \in \mathcal{O}$. The unknown transition function $F: (x, (\tau, o)) \rightarrow x'$ defines the human perception dynamics, which we aim to learn.

\subsubsection{Crowd-Sourced Transition Dataset}
To collect data, we designed a mapping task in the CoNav-Maze environment. Participants were tasked to edit their maps to match the true environment. A button triggers the robot's autonomous movements, after which it captures an image from a random angle.
In this mapping task, the robot, aware of both the true environment and the human’s map, visits predefined target locations and prioritizes areas with mislabeled grid cells on the human’s map.
% We assume that the robot has full knowledge of both the actual environment and the human’s current map. Leveraging this knowledge, the robot autonomously navigates to all predefined target locations. It then randomly selects subsequent goals to reach, prioritizing grid locations that remain mislabeled on the human’s map. This ensures that the robot’s actions are strategically focused on providing useful information to improve map accuracy.

We then recruited over $50$ annotators through Prolific~\cite{palan2018prolific} for the mapping task. Each annotator labeled three randomly generated mazes. They were allowed to proceed to the next maze once the robot had reached all four goal locations. However, they could spend additional time refining their map before moving on. To incentivize accuracy, annotators receive a performance-based bonus based on the final accuracy of their annotated map.


\subsubsection{Fully-Convolutional Dynamics Model}
\label{sec:nhpm}

We propose a Neural Human Perception Model (NHPM), a fully convolutional neural network (FCNN), to predict the human perception transition probabilities modeled in \Cref{sec:perception_mdp}. We denote the model as $F_\theta$ where $\theta$ represents the trainable weights. Such design echoes recent studies of model-based reinforcement learning~\cite{hansen2022temporal}, where the agent first learns the environment dynamics, potentially from image observations~\cite{hafner2019learning,watter2015embed}.

\begin{figure}[t]
    \centering
    \includegraphics[width=0.9\linewidth]{figures/ICML_25_CNN.pdf}
    \caption{Neural Human Perception Model (NHPM). \textbf{Left:} The human's current perception, the robot's trajectory since the last transmission, and the captured environment grids are individually processed into 2D masks. \textbf{Right:} A fully convolutional neural network predicts two masks: one for the probability of the human adding a wall to their map and another for removing a wall.}
    \label{fig:nhpm}
    \vskip -0.1in
\end{figure}

As illustrated in \Cref{fig:nhpm}, our model takes as input the human’s current perception, the robot’s path, and the image captured by the robot, all of which are transformed into a unified 2D representation. These inputs are concatenated along the channel dimension and fed into the CNN, which outputs a two-channel image: one predicting the probability of human adding a new wall and the other predicting the probability of removing a wall.

% Our approach builds on world model learning, where neural networks predict state transitions or environmental updates based on agent actions and observations. By leveraging the local feature extraction capabilities of CNNs, our model effectively captures spatial relationships and interprets local changes within the grid maze environment. Similar to prior work in localization and mapping, the CNN architecture is well-suited for processing spatially structured data and aligning the robot’s observations with human map updates.

To enhance robustness and generalization, we apply data augmentation techniques, including random rotation and flipping of the 2D inputs during training. These transformations are particularly beneficial in the grid maze environment, which is invariant to orientation changes.

\subsection{Perception-Aware Reward Augmentation}
The robot optimizes its actions over a planning horizon \( H \) by solving the following optimization problem:
\begin{subequations}
    \begin{align}
        \max_{a_{0:H-1}} \;
        & \mathop{\mathbb{E}}_{T, F} \left[ \sum_{t=0}^{H-1} \gamma^t \left(\underbrace{R_{\mathrm{task}}(\tau_{t+1}, \zeta)}_{\text{(1) Task reward}} + \underbrace{\|x_{t+1}-x_t\|_1}_{\text{(2) Info reward}}\right)\right] \label{obj}\\ 
        \subjectto \quad
        &x_{t+1} = F(x_t, (\tau_t, a_t)), \quad a_t\in\Ocal \label{const:perception_update}\\ 
        &\tau_{t+1} = \tau_t \oplus T(s_t, a_t), \quad a_t\in \Ucal\label{const:history_update}
    \end{align}
\end{subequations} 

The objective in~\eqref{obj} maximizes the expected cumulative reward over \( T \) and \( F \), reflecting the uncertainty in both physical transitions and human perception dynamics. The reward function consists of two components: 
(1) The \emph{task reward} incentivizes efficient navigation. The specific formulation for the task in this work is outlined in \Cref{appendix:task_reward}.
(2) The \emph{information reward} quantifies the change in the human’s perception due to robot actions, computed as the \( L_1 \)-norm distance between consecutive perception states.  

The constraint in~\eqref{const:history_update} ensures that for movement actions, the trajectory history \( \tau_t \) expands with new states based on the robot’s chosen actions, where \( s_t \) is the most recent state in \( \tau_t \), and \( \oplus \) represents sequence concatenation. 
In constraint~\eqref{const:perception_update}, the robot leverages the learned human perception dynamics \( F \) to estimate the evolution of the human’s understanding of the environment from perception state $x_t$ to $x_{t+1}$ based on the observed trajectory \( \tau_t \) and transmitted image \( a_t\in\Ocal \). 
% justify from a cognitive science perspective
% Cognitive science research has shown that humans read in a way to maximize the information gained from each word, aligning with the efficient coding principle, which prioritizes minimizing perceptual errors and extracting relevant features under limited processing capacity~\cite{kangassalo2020information}. Drawing on this principle, we hypothesize that humans similarly prioritize task-relevant information in multimodal settings. To accommodate this cognitive pattern, our robot policy selects and communicates high information-gain observations to human operators, akin to summarizing key insights from a lengthy article.
% % While the brain naturally seeks to gain information, the brain employs various strategies to manage information overload, including filtering~\cite{quiroga2004reducing}, limiting/working memory, and prioritizing information~\cite{arnold2023dealing}.
% In this context of our setup, we optimize the selection of camera angles to maximize the human operator's information gain about the environment. 

\subsection{Information Gain Monte Carlo Tree Search (IG-MCTS)}
IG-MCTS follows the four stages of Monte Carlo tree search: \emph{selection}, \emph{expansion}, \emph{rollout}, and \emph{backpropagation}, but extends it by incorporating uncertainty in both environment dynamics and human perception. We introduce uncertainty-aware simulations in the \emph{expansion} and \emph{rollout} phases and adjust \emph{backpropagation} with a value update rule that accounts for transition feasibility.

\subsubsection{Uncertainty-Aware Simulation}
As detailed in \Cref{algo:IG_MCTS}, both the \emph{expansion} and \emph{rollout} phases involve forward simulation of robot actions. Each tree node $v$ contains the state $(\tau, x)$, representing the robot's state history and current human perception. We handle the two action types differently as follows:
\begin{itemize}
    \item A movement action $u$ follows the environment dynamics $T$ as defined in \Cref{sec:problem}. Notably, the maze layout is observable up to distance $r$ from the robot's visited grids, while unexplored areas assume a $50\%$ chance of walls. In \emph{expansion}, the resulting search node $v'$ of this uncertain transition is assigned a feasibility value $\delta = 0.5$. In \emph{rollout}, the transition could fail and the robot remains in the same grid.
    
    \item The state transition for a communication step $o$ is governed by the learned stochastic human perception model $F_\theta$ as defined in \Cref{sec:nhpm}. Since transition probabilities are known, we compute the expected information reward $\bar{R_\mathrm{info}}$ directly:
    \begin{align*}
        \bar{R_\mathrm{info}}(\tau_t, x_t, o_t) &= \mathbb{E}_{x_{t+1}}\|x_{t+1}-x_t\|_1 \\
        &= \|p_\mathrm{add}\|_1 + \|p_\mathrm{remove}\|_1,
    \end{align*}
    where $(p_\mathrm{add}, p_\mathrm{remove}) \gets F_\theta(\tau_t, x_t, o_t)$ are the estimated probabilities of adding or removing walls from the map. 
    Directly computing the expected return at a node avoids the high number of visitations required to obtain an accurate value estimate.
\end{itemize}

% We denote a node in the search tree as $v$, where $s(v)$, $r(v)$, and $\delta(v)$ represent the state, reward, and transition feasibility at $v$, respectively. The visit count of $v$ is denoted as $N(v)$, while $Q(v)$ represents its total accumulated return. The set of child nodes of $v$ is denoted by $\mathbb{C}(v)$.

% The goal of each search is to plan a sequence for the robot until it reaches a goal or transmits a new image to the human. We initialize the search tree with the current human guidance $\zeta$, and the robot's approximation of human perception $x_0$. Each search node consists consists of the state information required by our reward augmentation: $(\tau, x)$. A node is terminal if it is the resulting state of a communication step, or if the robot reaches a goal location. 

% A rollout from the expanded node simulates future transitions until reaching a terminal state or a predefined depth $H$. Actions are selected randomly from the available action set $\mathcal{A}(s)$. If an action's feasibility is uncertain due to the environment's unknown structure, the transition occurs with probability $\delta(s, a)$. When a random number draw deems the transition infeasible, the state remains unchanged. On the other hand, for communication steps, we don't resolve the uncertainty but instead compute the expected information gain reward: \philip{TODO: adjust notation}
% \begin{equation}
%     \mathbb{E}\left[R_\mathrm{info}(\tau, x')\right] = \sum \mathrm{NPM(\tau, o)}.
% \end{equation}

\subsubsection{Feasibility-Adjusted Backpropagation}
During backpropagation, the rewards obtained from the simulation phase are propagated back through the tree, updating the total value $Q(v)$ and the visitation count $N(v)$ for all nodes along the path to the root. Due to uncertainty in unexplored environment dynamics, the rollout return depends on the feasibility of the transition from the child node. Given a sample return \(q'_{\mathrm{sample}}\) at child node \(v'\), the parent node's return is:
\begin{equation}
    q_{\mathrm{sample}} = r + \gamma \left[ \delta' q'_{\mathrm{sample}} + (1 - \delta') \frac{Q(v)}{N(v)} \right],
\end{equation}
where $\delta'$ represents the probability of a successful transition. The term \((1 - \delta')\) accounts for failed transitions, relying instead on the current value estimate.

% By incorporating uncertainty-aware rollouts and backpropagation, our approach enables more robust decision-making in scenarios where the environment dynamics is unknown and avoids simulation of the stochastic human perception dynamics.

\label{sec: method}

% \vspace{-2ex}
\section{Experiments}
\label{sec: exp_main}


% \jq{almost rewrite}
In this section, we conduct a comprehensive evaluation of
\ours on various benchmark datasets, including both
homophilic and heterophilic graphs. We aim to answer the following three
key research questions: \textbf{RQ1} How does \ours perform in node classification tasks? \textbf{RQ2} How effectively does \ours mitigate oversmoothing? \textbf{RQ3} How sensitive, robust, and scalable is \ours?   

% In this section, we first verify our theoretical insights on the synthetic datasets.
% Then we demonstrate the effectiveness of \ours in three widely used datasets, and then extend the experimental evaluation to one large-scale dataset and two heterophilous datasets. 
% We apply Label-\ours and Feature-\ours in both linear SGC and non-linear GCN. 




% % \textbf{Contextual Stochastic Block Models.} 
% % Following~\cite{sbm_xinyi}, We focus on the CSBM$(N, p, q, \mu_1, \mu_2, \sigma^2 )$.
% % It consists of two classes $\mathcal{C}_1$ and $\mathcal{C}_2$ of nodes of equal size, in total with $N$ nodes. 
% % For any two nodes in the graph, if they are from the same class, they are connected by an edge independently with probability $p$, or if they are from different classes, the probability is $q$. For each node $v \in \mathcal{C}_i, i\in\{1,-1\}$, the initial feature $X_v$ is sampled independently from a Gaussian distribution $\mathcal{N}(\mu_i, {\sigma^2})$, where $\mu_i =\mathcal{C}_i, \sigma = I $. Specially, let $N=200$, $p=0.092$, $q=0.046$, $d=8$.

% \paragraph{Setup}  
% We apply our methods to node classification on the synthetic dataset CSBM$(n, p, q, \mu_1, \mu_2, \sigma^2 )$.
% Specially, let $n=200$, $p=0.092$, $q=0.046$, $d=8$, $\mu_1=1$, $\mu_2=-1$ and $\sigma^2=I$ following~\cite{sbm_xinyi}.
% We used $60\%$, $20\%$ and $20 \%$ random splits for train, valid and test data, respectively.
% Since our theoretical analysis is primarily based on linear propagation, we apply the linear backbone SGC, which removes the non-linear activation function compared to the GCN, to verify our insight.
% We introduce SGC and GCN in Appendix~\ref{app: GNNs} in details.
% % We further apply both SGC and the GCN in the real world dataset.



% \paragraph{Results} 
% The visualization of node features using Label-\ours and Feature-\ours compared to SGC are shown in Figure \ref{fig: sbm overall}. As the number of layers increases, SGC's node features suffer from oversmoothing, causing the two classes to converge and the classification accuracy to drop to $47.50\%$, which is worse than random guessing ($50\%$). However, Label-\ours and Feature-\ours effectively repel nodes from different classes, achieving high accuracy of $80\%$ and $97.5\%$, respectively, even with $300$ layers.


\begin{figure*}[t]
% \captionsetup{font=small}
    \begin{subfigure}{0.69\textwidth}
        \centering
        % \captionsetup{font=small}
        \includegraphics[width=0.99\textwidth]{figures/eval_sgc_layer.pdf}
        \caption{Oversmoothing performance.}
        \label{fig: layer depth}
    \end{subfigure}
    % \quad
    \begin{subfigure}{0.3\textwidth}
        \centering
        % \captionsetup{font=small}
        \includegraphics[width=0.99\textwidth]{figures/eval_train.pdf} % Adjust the path and filename as necessary
        \caption{Training ratio ablation study.}
        \label{fig: train ratio}
    \end{subfigure}
    \caption{Left is the performance comparison of \ours against Normalization GNNs under various model depths where the X-axis has the number of layers, and the Y-axis has node classification accuracy. Right is the ablation study on Label-\ours where the X-axis indicates the ratio of the training node numbers.}
    % \vspace{-0.55cm}
    % \vspace{-0.2in}
\end{figure*}



% \subsection{Real World Benchmark}

% \begin{table}[t]
\centering
% \vspace{-0.15in}
\caption{SGC test accuracy (\%) comparison results. The best results are marked in blue and the second best results are marked in gray on every layer. We run 10 runs and demonstrate the mean $\pm$ std in the table.} % for the seed $0$\~$9$ 
% \resizebox{\textwidth}{!}{
\begin{adjustbox}{width=0.99\textwidth}
\begin{tabular}{lccccccc}
\toprule
 Model             & \#L=2              & \#L=5              & \#L=10             & \#L=20             & \#L=50        & \#L=100    & \#L=300    \\
\midrule
\rowcolor{gray!8}\multicolumn{8}{c}{\textit{Cora}~\citep{cora}}\\
\midrule
% \midrule
% \rowcolor{gray!8}\textit{cora}~\citep{cora}\\
% \midrule
 SGC      & 80.21 {\footnotesize $\pm$ 0.07}& 81.45 {\footnotesize $\pm$ 0.14 }& 81.53 {\footnotesize $\pm$ 0.19 }& 79.53 {\footnotesize $\pm$ 0.14 }& 79.20 {\footnotesize $\pm$ 0.21 }& 76.13 {\footnotesize $\pm$ 0.24 }& 65.64 {\footnotesize $\pm$ 1.15 }\\

 % +LayerNorm~\cite{layernorm}       & 80.07 {\footnotesize $\pm$ 0.22 }& \cellcolor{secondbest}81.60 {\footnotesize $\pm$ 0.22 }& 81.20 {\footnotesize $\pm$ 0.29 }& 79.52 {\footnotesize $\pm$ 0.16  }& 79.21 {\footnotesize $\pm$ 0.23 }& 76.44 {\footnotesize $\pm$ 0.11 }& 68.38 {\footnotesize $\pm$ 0.66}  \\
 +BatchNorm & 77.90 {\footnotesize $\pm$ 0.00 }& 78.02 {\footnotesize $\pm$ 0.04 }& 76.94 {\footnotesize $\pm$ 0.08 }& 75.18 {\footnotesize $\pm$ 0.09 }& 74.54 {\footnotesize $\pm$ 0.05 }& 72.64 {\footnotesize $\pm$ 0.05 }& 63.12 {\footnotesize $\pm$ 0.06} \\
+PairNorm     & 80.30 {\footnotesize $\pm$ 0.05 }& 78.57 {\footnotesize $\pm$ 0.00 }&78.14 {\footnotesize $\pm$ 0.07 }& 76.90 {\footnotesize $\pm$ 0.00 }& 77.49 {\footnotesize $\pm$ 0.03}  & 72.01 {\footnotesize $\pm$ 0.03 }& 40.93 {\footnotesize $\pm$ 0.11} \\
 +ContraNorm      & \cellcolor{best}81.60 {\footnotesize $\pm$ 0.00 }& 80.67 {\footnotesize $\pm$ 0.06 }& 79.11 {\footnotesize $\pm$ 0.03 }& 74.28 {\footnotesize $\pm$ 0.15 }& 69.67 {\footnotesize $\pm$ 1.23 }& 65.58 {\footnotesize $\pm$ 2.11 }&47.21 {\footnotesize $\pm$ 10.80} \\
+DropEdge & 73.58 {\footnotesize $\pm$ 2.76 }& 62.11 {\footnotesize $\pm$ 5.10 }& 39.21 {\footnotesize $\pm$ 7.54 }& 15.07 {\footnotesize $\pm$ 6.22 }& 11.16 {\footnotesize $\pm$ 2.73 }& 11.15 {\footnotesize $\pm$ 2.81 }& 11.15 {\footnotesize $\pm$ 2.81 }\\
 +Residual& 77.81 {\footnotesize $\pm$ 0.03 }& 81.47 {\footnotesize $\pm$ 0.05 }& \cellcolor{best}82.90 {\footnotesize $\pm$ 0.00 }& 79.87 {\footnotesize $\pm$ 0.05 }& 75.64 {\footnotesize $\pm$ 0.05 }& 66.90 {\footnotesize $\pm$ 0.10 }& 25.33 {\footnotesize $\pm$ 0.46}\\
\midrule
 Feature-\ourst &78.10 {\footnotesize $\pm$ 0.11 }& 80.88 {\footnotesize $\pm$ 0.23 }& 80.83 {\footnotesize $\pm$ 0.37 }& \cellcolor{secondbest}82.46 {\footnotesize $\pm$ 0.07 }& \cellcolor{secondbest}80.47 {\footnotesize $\pm$ 0.25 }& \cellcolor{secondbest}80.23 {\footnotesize $\pm$ 0.51 }& \cellcolor{secondbest}77.49 {\footnotesize $\pm$ 0.23 }\\

 Label-\ourst    & \cellcolor{secondbest}81.14 {\footnotesize $\pm$ 0.49 }& \cellcolor{best}82.90 {\footnotesize $\pm$ 0.00}&	\cellcolor{secondbest}82.54 {\footnotesize $\pm$ 0.05}&	\cellcolor{best}82.44 {\footnotesize $\pm$ 0.05}&	\cellcolor{best}82.60 {\footnotesize $\pm$ 0.00}&	\cellcolor{best}81.10 {\footnotesize $\pm$ 0.00}&	\cellcolor{best}74.98 {\footnotesize $\pm$ 0.11 }\\

\midrule
\rowcolor{gray!8}\multicolumn{8}{c}{\textit{CiteSeer}~\citep{citeseer}}\\
\midrule

SGC & 71.88 {\footnotesize $\pm$ 0.27 }& \cellcolor{secondbest}72.55 {\footnotesize $\pm$ 0.25 }& 72.53 {\footnotesize $\pm$ 0.15 }& 72.07 {\footnotesize $\pm$ 0.21 }& 69.83 {\footnotesize $\pm$ 0.20 }& 65.42 {\footnotesize $\pm$ 0.43 }& 54.69 {\footnotesize $\pm$ 0.98} \\

% +LayerNorm~\cite{layernorm} &66.92 {\footnotesize $\pm$ 7.97 }& 65.78 {\footnotesize $\pm$ 2.22 }& 65.82 {\footnotesize $\pm$ 2.39 }& 64.83 {\footnotesize $\pm$ 1.83 }& 62.96 {\footnotesize $\pm$ 2.52 }& 56.67 {\footnotesize $\pm$ 6.55 }& 48.87 {\footnotesize $\pm$ 7.06} \\
+BatchNorm &60.85 {\footnotesize $\pm$ 0.09 }& 60.45 {\footnotesize $\pm$ 0.07 }& 61.74 {\footnotesize $\pm$ 0.27 }& 63.29 {\footnotesize $\pm$ 0.18 }& 63.71 {\footnotesize $\pm$ 0.18 }& 64.28 {\footnotesize $\pm$ 0.27 }& 59.42 {\footnotesize $\pm$ 0.20} \\
 +PairNorm  &70.83 {\footnotesize $\pm$ 0.06 }& 69.68 {\footnotesize $\pm$ 0.32 }& 70.54 {\footnotesize $\pm$ 0.04 }& 69.86 {\footnotesize $\pm$ 0.08 }& 70.51 {\footnotesize $\pm$ 0.07 }& \cellcolor{secondbest}69.86 {\footnotesize $\pm$ 0.06 }& \cellcolor{secondbest}65.22 {\footnotesize $\pm$ 0.16 }\\  
 +ContraNorm  &\cellcolor{best}72.25 {\footnotesize $\pm$ 0.08 }& 71.9 {\footnotesize $\pm$ 0.06 }& 71.52 {\footnotesize $\pm$ 0.04 }& 59.82 {\footnotesize $\pm$ 2.30 }& 52.87 {\footnotesize $\pm$ 1.86 }& 45.93 {\footnotesize $\pm$ 1.40 }& 35.67 {\footnotesize $\pm$ 1.62}\\
+DropEdge & 65.63 {\footnotesize $\pm$ 1.76 }& 51.80 {\footnotesize $\pm$ 4.61 }& 25.36 {\footnotesize $\pm$ 2.54 }& 18.60 {\footnotesize $\pm$ 3.78 }& 16.52 {\footnotesize $\pm$ 3.97 }& 16.49 {\footnotesize $\pm$ 4.03  }& 16.49 {\footnotesize $\pm$ 4.03 }\\
 +Residual & 71.61 {\footnotesize $\pm$ 0.17 }& 72.31 {\footnotesize $\pm$ 0.15  }& \cellcolor{best}72.78 {\footnotesize $\pm$ 0.12 }& \cellcolor{secondbest}72.50 {\footnotesize $\pm$ 0.14 }& \cellcolor{secondbest}71.24 {\footnotesize $\pm$ 0.21 }& 69.85 {\footnotesize $\pm$ 0.22 }& 62.11 {\footnotesize $\pm$ 0.42}\\
\midrule
  Feature-\ourst & 70.63 {\footnotesize $\pm$ 0.52 }& 70.85 {\footnotesize $\pm$ 0.09 }& 70.52 {\footnotesize $\pm$ 0.14 }& 70.76 {\footnotesize $\pm$ 0.22 }& 68.25 {\footnotesize $\pm$ 0.46 }& 67.20 {\footnotesize $\pm$ 1.15 }& 65.12 {\footnotesize $\pm$ 1.95 }\\
 Label-\ourst &\cellcolor{secondbest}72.01 {\footnotesize $\pm$ 0.10 }& \cellcolor{best}72.87 {\footnotesize $\pm$ 0.05 }& \cellcolor{secondbest}72.72 {\footnotesize $\pm$ 0.28 }& \cellcolor{best}73.04 {\footnotesize $\pm$ 0.10 }& \cellcolor{best}72.52 {\footnotesize $\pm$ 0.17 }& \cellcolor{best}72.45 {\footnotesize $\pm$ 0.11 }& \cellcolor{best}70.97 {\footnotesize $\pm$ 0.22 }\\

\midrule
\rowcolor{gray!8}\multicolumn{8}{c}{\textit{PubMed }~\citep{pubmed}}\\
\midrule
SGC &76.99 {\footnotesize $\pm$ 0.38 }& 75.92 {\footnotesize $\pm$ 0.30 }& 76.18 {\footnotesize $\pm$ 0.70 }& 77.13 {\footnotesize $\pm$ 0.34 }&76.09 {\footnotesize $\pm$ 0.43 }& 76.19 {\footnotesize $\pm$ 0.19 }& 70.58 {\footnotesize $\pm$ 0.52 }\\

 % +LayerNorm~\cite{layernorm} &77.67 {\footnotesize $\pm$ 0.40 }& 76.43 {\footnotesize $\pm$ 0.36 }& 76.26 {\footnotesize $\pm$ 0.34 }& 76.27 {\footnotesize $\pm$ 0.41 }& 75.95 {\footnotesize $\pm$ 0.24 }& 74.79 {\footnotesize $\pm$ 0.53 }& 71.77 {\footnotesize $\pm$ 0.45} \\ 
 +BatchNorm &77.15 {\footnotesize $\pm$ 0.09 }& 77.87 {\footnotesize $\pm$ 0.05 }& 78.47 {\footnotesize $\pm$ 0.05 }& 77.90 {\footnotesize $\pm$ 1.10 }& 76.85 {\footnotesize $\pm$ 0.08 }& 74.35 {\footnotesize $\pm$ 0.08 }& 69.61 {\footnotesize $\pm$ 0.08} \\
 +PairNorm &77.69 {\footnotesize $\pm$ 0.26 }& 75.78 {\footnotesize $\pm$ 0.37 }& 75.13 {\footnotesize $\pm$ 0.13 }& 74.75 {\footnotesize $\pm$ 0.33 }& 72.13 {\footnotesize $\pm$ 0.11 }& 69.79 {\footnotesize $\pm$ 0.16 }& 71.75 {\footnotesize $\pm$ 0.51 }\\       
+ContraNorm &\cellcolor{best}79.30 {\footnotesize $\pm$ 0.10 }& 78.69 {\footnotesize $\pm$ 0.07 }& 77.54 {\footnotesize $\pm$ 0.09 }& 73.67 {\footnotesize $\pm$ 0.12 }& 71.37 {\footnotesize $\pm$ 3.15 }& 67.96 {\footnotesize $\pm$ 3.24 }& 65.00 {\footnotesize $\pm$ 4.12 }\\
 +DropEdge & 74.64 {\footnotesize $\pm$ 1.37 }& 69.83 {\footnotesize $\pm$ 3.19 }& 60.28 {\footnotesize $\pm$ 2.70 }& 32.62 {\footnotesize $\pm$ 10.95 }& 33.95 {\footnotesize $\pm$ 10.44 }& 33.95 {\footnotesize $\pm$ 10.44 }& 33.95 {\footnotesize $\pm$ 10.44 }\\
+Residual & 77.40 {\footnotesize $\pm$ 0.06 }& \cellcolor{secondbest}79.30 {\footnotesize $\pm$ 0.10 }& \cellcolor{secondbest}79.83 {\footnotesize $\pm$ 0.09 }& \cellcolor{secondbest}79.44 {\footnotesize $\pm$ 0.09 }& 74.96 {\footnotesize $\pm$ 0.09 }& 71.72 {\footnotesize $\pm$ 0.13 }& 55.57 {\footnotesize $\pm$ 0.21 }\\
\midrule
 Feature-\ourst & 73.99 {\footnotesize $\pm$ 1.44 }& 74.36 {\footnotesize $\pm$ 0.63 }& 75.61 {\footnotesize $\pm$ 0.24 }& 77.09 {\footnotesize $\pm$ 0.35 }& \cellcolor{secondbest}77.41 {\footnotesize $\pm$ 0.21 }& \cellcolor{secondbest}77.10 {\footnotesize $\pm$ 0.36 }& \cellcolor{secondbest}76.87 {\footnotesize $\pm$ 0.49 }\\
 Label-\ourst &\cellcolor{secondbest}78.98 {\footnotesize $\pm$ 0.14 }& \cellcolor{best}80.14 {\footnotesize $\pm$ 0.05 }& \cellcolor{best}80.22 {\footnotesize $\pm$ 0.04 }& \cellcolor{best}80.32 {\footnotesize $\pm$ 0.04 }& \cellcolor{best}80.20 {\footnotesize $\pm$ 0.00 }& \cellcolor{best}79.60 {\footnotesize $\pm$ 0.00 }& \cellcolor{best}73.96 {\footnotesize $\pm$ 0.05} \\
% \midrule

% Arxiv }& SGC \\
%     }& +LayerNorm \\
%     }& PairNorm \\
%     }& ContraNorm \\
%     }& Feature-\ourst \\
%     }& Label-\ourst \\
% % Add more rows as needed
\bottomrule
\end{tabular}
% }

\end{adjustbox}
\label{table: sgc results}
% \vspace{-0.15in}
\end{table}
% \vspace{-0.1in}
% \begin{table}[t]
\vspace{-0.1in}
\centering
\small
\caption{GCN test accuracy (\%) comparison results on heterophilic datasets. The best results are marked in blue and the second best results are marked underline on every layer.
We run 5 runs and demonstrate the mean $\pm$ std in the table.}%for the seed from $0~4$
\begin{adjustbox}{width=0.91\textwidth}
\begin{tabular}{lcccccc}
\toprule
 Model             & \#L=2              & \#L=4              & \#L=8              & \#L=16             & \#L=32             & \#L=64\\

\midrule
\rowcolor{gray!8}\multicolumn{7}{c}{\textit{Chameleon}~\cite{heter_dataset}}\\
\midrule
   GCN & 66.01{\footnotesize$\pm$0.72} & 54.21{\footnotesize$\pm$0.53} & 35.48{\footnotesize$\pm$3.09} & 22.37{\footnotesize$\pm$0.00} & 22.37{\footnotesize$\pm$0.00} & 22.37{\footnotesize$\pm$0.00} \\
    +BatchNorm & \underline{65.83{\footnotesize$\pm$0.58}} & 56.40{\footnotesize$\pm$0.35} & 36.36{\footnotesize$\pm$2.04} & 22.37{\footnotesize$\pm$0.00} & 22.37{\footnotesize$\pm$0.00} & 22.37{\footnotesize$\pm$0.00}\\
    +PairNorm & 66.01{\footnotesize$\pm$0.72} & 54.12{\footnotesize$\pm$0.79} & 36.75{\footnotesize$\pm$0.38} & 22.37{\footnotesize$\pm$0.00} & 22.37{\footnotesize$\pm$0.00} & 22.37{\footnotesize$\pm$0.00}\\
    +ContraNorm & 66.01{\footnotesize$\pm$0.72} & 58.16{\footnotesize$\pm$1.76} & 37.15{\footnotesize$\pm$4.91} & 22.37{\footnotesize$\pm$0.00} & 22.37{\footnotesize$\pm$0.00} & 22.37{\footnotesize$\pm$0.00}\\
    +DropEdge & 62.50{\footnotesize$\pm$0.00} & 53.07{\footnotesize$\pm$1.61} & 32.15{\footnotesize$\pm$1.49} & 21.71{\footnotesize$\pm$0.00} & \underline{27.19{\footnotesize$\pm$1.75}} & \underline{23.68{\footnotesize$\pm$0.00}} \\
    +Residual & 66.01{\footnotesize$\pm$0.72} & \cellcolor{best}62.94{\footnotesize$\pm$0.00} & \underline{57.59{\footnotesize$\pm$2.58}} & \underline{41.27{\footnotesize$\pm$0.32}} & 22.37{\footnotesize$\pm$0.00} & 22.37{\footnotesize$\pm$0.00} \\
\midrule
    Feature-\ourst & 62.98{\footnotesize$\pm$0.75} & \underline{62.89{\footnotesize$\pm$1.29}} & \cellcolor{best}65.35{\footnotesize$\pm$0.00} & \cellcolor{best}62.28{\footnotesize$\pm$0.00} & \cellcolor{best}55.31{\footnotesize$\pm$0.53} & \cellcolor{best}35.13{\footnotesize$\pm$0.93}\\
    Label-\ourst & \cellcolor{best}66.01{\footnotesize$\pm$0.72} & 57.11{\footnotesize$\pm$0.11} & 38.11{\footnotesize$\pm$1.87} & 22.37{\footnotesize$\pm$0.00} & 22.37{\footnotesize$\pm$0.00} & 22.37{\footnotesize$\pm$0.00}\\
\midrule
\rowcolor{gray!8}\multicolumn{7}{c}{\textit{Squirrel}~\cite{heter_dataset}}\\
\midrule
   GCN  & 42.38{\footnotesize$\pm$0.04} & 32.20{\footnotesize$\pm$3.05} & 22.57{\footnotesize$\pm$0.00} & 20.46{\footnotesize$\pm$0.00} & 20.46{\footnotesize$\pm$0.00} & 20.46{\footnotesize$\pm$0.00}\\
    +BatchNorm & 41.77{\footnotesize$\pm$0.35} & 32.37{\footnotesize$\pm$3.46} & 22.67{\footnotesize$\pm$0.00} & 20.46{\footnotesize$\pm$0.00} & 20.46{\footnotesize$\pm$0.00} & 20.46{\footnotesize$\pm$0.00}\\
    +PairNorm & 42.75{\footnotesize$\pm$0.00} & 32.12{\footnotesize$\pm$3.00} & 22.57{\footnotesize$\pm$0.00} & 20.46{\footnotesize$\pm$0.00} & 20.46{\footnotesize$\pm$0.00} & 20.46{\footnotesize$\pm$0.00}\\
    +ContraNorm & \underline{43.78{\footnotesize$\pm$1.08}} & 32.80{\footnotesize$\pm$3.76} & 22.57{\footnotesize$\pm$0.00} & 20.46{\footnotesize$\pm$0.00} & 20.46{\footnotesize$\pm$0.00} & 20.46{\footnotesize$\pm$0.00} \\
    +DropEdge & 40.54{\footnotesize$\pm$0.00} & 22.57{\footnotesize$\pm$0.00} & 22.77{\footnotesize$\pm$2.12} & 22.19{\footnotesize$\pm$0.58} & \underline{22.61{\footnotesize$\pm$1.36}} & 20.46{\footnotesize$\pm$0.00}\\
    +Residual & 41.92{\footnotesize$\pm$0.65} & \underline{42.23{\footnotesize$\pm$0.08}} & \underline{39.15{\footnotesize$\pm$0.07}} & \underline{33.41{\footnotesize$\pm$2.73}} & 20.46{\footnotesize$\pm$0.00} & 20.46{\footnotesize$\pm$0.00} \\
\midrule
    Feature-\ourst& \cellcolor{best}44.48{\footnotesize$\pm$0.00} &\cellcolor{best} 45.01{\footnotesize$\pm$0.72} &\cellcolor{best} 44.03{\footnotesize$\pm$0.63} &\cellcolor{best} 41.42{\footnotesize$\pm$0.78} &\cellcolor{best} 36.79{\footnotesize$\pm$0.00} &\cellcolor{best} 29.20{\footnotesize$\pm$0.00}\\
    Label-\ourst& 43.61{\footnotesize$\pm$0.58} & 32.78{\footnotesize$\pm$3.49} & 22.79{\footnotesize$\pm$0.09} & 20.46{\footnotesize$\pm$0.00} & 20.46{\footnotesize$\pm$0.00} & \underline{20.46{\footnotesize$\pm$0.00}} \\

\bottomrule
\end{tabular}
\end{adjustbox}
\label{table: gcn heter}
\vspace{-0.1in}
\end{table}

% \begin{table}[h]
% \centering
% \caption{GCN test accuracy (\%) comparison results. The best results are marked in blue and the second best results are marked in gray on every layer.}
% \begin{adjustbox}{width=0.99\textwidth}
% \begin{tabular}{lcccccc}
% \toprule
%  Model             & \#L=2              & \#L=4              & \#L=8              & \#L=16             & \#L=32             & \#L=64\\

% \midrule
% \rowcolor{gray!8}\multicolumn{7}{c}{\textit{Cora}~\citep{cora}}\\
% \midrule
%   GCN~\cite{gcn} & \cellcolor{secondbest}80.68 $\pm$ 0.09 & \cellcolor{secondbest}79.69 $\pm$ 0.00 & 74.32 $\pm$ 0.00 & 30.95 $\pm$ 0.00 & 30.95 $\pm$ 0.00 & 24.85 $\pm$ 7.46 \\


% % & Center & 79.85 $\pm$ 0.46 & 77.32 $\pm$ 1.33 & 75.25 $\pm$ 0.40 & 57.75 $\pm$ 3.53 & 41.73 $\pm$ 0.92 & 39.34 $\pm$ 2.29 \\
%     LayerNorm~\cite{layernorm} & 80.51 $\pm$ 0.12 & \cellcolor{best}80.28 $\pm$ 0.66 & 75.05 $\pm$ 0.00 & 30.95 $\pm$ 0.00 & 30.95 $\pm$ 0.00 & 24.85 $\pm$ 7.46 \\
%     BatchNorm~\cite{batchnorm} & 78.09 $\pm$ 0.00 & 77.87 $\pm$ 0.02 & 73.62 $\pm$ 0.57 & 70.79 $\pm$ 0.00 & 53.90 $\pm$ 2.19 & 35.32 $\pm$ 3.41\\
%     PairNorm~\cite{pairnorm} & 79.01 $\pm$ 0.00 & 78.26 $\pm$ 0.50 & 73.21 $\pm$ 0.00 & 62.96 $\pm$ 0.00 & 48.13 $\pm$ 0.91 & 44.01 $\pm$ 3.46 \\
%     ContraNorm~\cite{contranorm} & \cellcolor{best}81.55 $\pm$ 0.21 & 79.61 $\pm$ 0.75 & 77.71 $\pm$ 0.00 & 63.35 $\pm$ 0.00 & 44.56 $\pm$ 4.83 & 38.97 $\pm$ 0.00 \\
%     DropEdge~\cite{dropedge} & 78.38 $\pm$ 0.00 & 74.47 $\pm$ 0.00 & 26.91 $\pm$ 0.83 & 22.24 $\pm$ 3.04 & 27.18 $\pm$ 0.00 & 25.98 $\pm$ 6.00\\
%     Residual& 80.68 $\pm$ 0.09 & 78.77 $\pm$ 0.00 & \cellcolor{secondbest}79.26 $\pm$ 0.21 & 40.91 $\pm$ 0.00 & 30.95 $\pm$ 0.00 & 27.90 $\pm$ 6.09\\
% \midrule
%      \ourst-Feature & 80.44 $\pm$ 0.83 & 79.26 $\pm$ 1.18 &  78.56 $\pm$ 0.59 & \cellcolor{secondbest} 77.22 $\pm$ 0.55 &\cellcolor{secondbest} 73.65 $\pm$ 0.48 &\cellcolor{secondbest} 61.62 $\pm$ 5.24\\
%      \ourst-Label &80.31 $\pm$ 0.70 & 79.16 $\pm$ 1.30 & \cellcolor{best}79.50 $\pm$ 0.00 & \cellcolor{best}77.43 $\pm$ 1.49 & \cellcolor{best}74.52 $\pm$ 0.36 & \cellcolor{best}65.02 $\pm$ 2.97 \\
% \midrule
% \rowcolor{gray!8}\multicolumn{7}{c}{\textit{CiteSeer}~\citep{citeseer}}\\
% \midrule
%    GCN~\cite{gcn} &\cellcolor{best} 67.45 $\pm$ 0.54 & 65.62 $\pm$ 0.25 & 37.22 $\pm$ 2.46 & 22.03 $\pm$ 4.76 & 19.65 $\pm$ 0.00 & 19.65 $\pm$ 0.00 \\


% % & Center & 67.21 $\pm$ 0.64 & 65.50 $\pm$ 0.99 & 59.25 $\pm$ 3.18 & 40.29 $\pm$ 1.18 & 41.73 $\pm$ 0.92 & 35.81 \pm 1.21\\
%     LayerNorm~\cite{layernorm} & 67.24 $\pm$ 0.66 &64.95 $\pm$ 0.72 & 38.87 $\pm$ 4.12 & 24.29 $\pm$ 5.68 & 19.65 $\pm$ 0.00 & 19.65 $\pm$ 0.00 \\
%      BatchNorm~\cite{batchnorm} &63.44 $\pm$ 0.94 & 62.34 $\pm$ 0.25 & 61.36 $\pm$ 0.00 & 50.58 $\pm$ 1.24 & 41.41 $\pm$ 0.00 & 35.00 $\pm$ 1.09 \\
%     PairNorm~\cite{pairnorm} & 63.58 $\pm$ 0.63 & 64.32 $\pm$ 0.95 & 61.95 $\pm$ 1.24 & 50.06 $\pm$ 0.00 & 37.21 $\pm$ 1.87 & 36.09 $\pm$ 0.07 \\
%     ContraNorm~\cite{contranorm} & 66.83 $\pm$ 0.49 & 64.78 $\pm$ 0.92 & 60.70 $\pm$ 0.60 & 44.79 $\pm$ 1.65 & 37.36 $\pm$ 0.25 & 30.85 $\pm$ 0.81 \\
%     DropEdge~\cite{dropedge} & 63.86 $\pm$ 0.03 & 62.24 $\pm$ 0.90 & 24.73 $\pm$ 5.72 & 20.65 $\pm$ 0.00 & 20.04 $\pm$ 0.19 & 19.95 $\pm$ 0.09\\
%     Residual & \cellcolor{secondbest} 67.45 $\pm$ 0.54 & 66.21 $\pm$ 0.16 & \cellcolor{best}67.34 $\pm$ 0.00 & 33.21 $\pm$ 0.00 & 19.65 $\pm$ 0.00 & 19.65 $\pm$ 0.00 \\
% \midrule
%     \ourst-Feature &  67.38 $\pm$ 0.66 & \cellcolor{best}66.94 $\pm$ 0.00 & 66.29 $\pm$ 0.02 & \cellcolor{secondbest}65.35 $\pm$ 1.99 & \cellcolor{best}61.43 $\pm$ 0.00 & \cellcolor{secondbest}42.09 $\pm$ 1.65\\
%      \ourst-Label & 67.23 $\pm$ 0.64 & \cellcolor{secondbest} 66.72 $\pm$ 0.00 & \cellcolor{secondbest}66.29 $\pm$ 0.89 & \cellcolor{best}65.50 $\pm$ 2.13 & \cellcolor{secondbest}59.93 $\pm$ 0.85 & \cellcolor{best}44.41 $\pm$ 1.57 \\
% \midrule
% \rowcolor{gray!8}\multicolumn{7}{c}{\textit{PubMed}~\citep{pubmed}}\\
% \midrule
%    GCN~\cite{gcn} & \cellcolor{best}76.44 $\pm$ 0.34 & 76.52 $\pm$ 0.32 & 69.58 $\pm$ 5.89 & 39.92 $\pm$ 0.00 & 39.92 $\pm$ 0.00 & 39.92 $\pm$ 0.00 \\


% % & Center & 75.19 $\pm$0.26	& 76.67 \pm	0.00 &OOM &OOM &OOM & OOM\\
%     LayerNorm~\cite{layernorm} & 76.27 $\pm$ 0.51 & 76.71 $\pm$ 0.24 & 76.95 $\pm$ 0.17 & 39.92 $\pm$ 0.00 & 39.92 $\pm$ 0.00 & 39.92 $\pm$ 0.00 \\
%     BatchNorm~\cite{batchnorm} & 75.52 $\pm$ 0.12 & \cellcolor{secondbest}77.15 $\pm$ 0.00 & 77.10 $\pm$ 0.00 & 76.92 $\pm$ 0.00 & 75.43 $\pm$ 0.00 & 69.33 $\pm$ 1.01 \\
%     PairNorm~\cite{pairnorm} & 75.66 $\pm$ 0.11 & 76.71 $\pm$ 0.00 & \cellcolor{secondbest}77.99 $\pm$ 0.00 & \cellcolor{secondbest}77.22 $\pm$ 0.39 & 75.52 $\pm$ 2.02 & 71.22 $\pm$ 3.68 \\
%     ContraNorm~\cite{contranorm} & 76.05 $\pm$ 0.33 & \cellcolor{best}78.42 $\pm$ 0.00 & OOM & OOM & OOM & OOM \\
%     DropEdge~\cite{dropedge}& 73.41 $\pm$ 0.03 & 73.96 $\pm$ 0.79 & 52.51 $\pm$ 10.91 & 40.27 $\pm$ 0.00 & 39.90 $\pm$ 0.59 & 40.08 $\pm$ 0.39 \\
%     Residual & \cellcolor{secondbest} 76.44 $\pm$ 0.34 & 77.28 $\pm$ 0.00 & 77.38 $\pm$ 0.00 & 63.14 $\pm$ 3.05 & 39.92 $\pm$ 0.00 & 39.92 $\pm$ 0.00 \\
% \midrule
%     \ourst-Feature & 75.72 $\pm$ 0.06 & 76.84 $\pm$ 0.00 & \cellcolor{best}78.39 $\pm$ 0.00 &\cellcolor{best} 79.71 $\pm$ 0.00 & \cellcolor{best}77.59 $\pm$ 0.23 & \cellcolor{best}78.06 $\pm$ 0.13\\
%     \ourst-Label & 76.33 $\pm$ 0.25 & 76.91 $\pm$ 0.00 & 77.60 $\pm$ 0.49 & 76.31 $\pm$ 0.00 & \cellcolor{secondbest}77.17 $\pm$ 0.67 & \cellcolor{secondbest}78.01 $\pm$ 0.16\\
% % Add more rows as needed
% % \midrule

% % Arxiv & GCN & 70.20 $\pm$ 0.36 & 70.84 $\pm$ 0.12 & 69.73 $\pm$ 0.28\\
% %     & APPNP \\
% %     & Center \\
% %     & LayerNorm \\
% %     & PairNorm \\
% %     & ContraNorm \\
% %     & \ourst-Feature \\
% %     & \ourst-Label \\
% \bottomrule
% \end{tabular}
% \end{adjustbox}
% \end{table}
\textbf{Datasets.} 
% \section{Dataset}
\label{sec:dataset}

\subsection{Data Collection}

To analyze political discussions on Discord, we followed the methodology in \cite{singh2024Cross-Platform}, collecting messages from politically-oriented public servers in compliance with Discord's platform policies.

Using Discord's Discovery feature, we employed a web scraper to extract server invitation links, names, and descriptions, focusing on public servers accessible without participation. Invitation links were used to access data via the Discord API. To ensure relevance, we filtered servers using keywords related to the 2024 U.S. elections (e.g., Trump, Kamala, MAGA), as outlined in \cite{balasubramanian2024publicdatasettrackingsocial}. This resulted in 302 server links, further narrowed to 81 English-speaking, politics-focused servers based on their names and descriptions.

Public messages were retrieved from these servers using the Discord API, collecting metadata such as \textit{content}, \textit{user ID}, \textit{username}, \textit{timestamp}, \textit{bot flag}, \textit{mentions}, and \textit{interactions}. Through this process, we gathered \textbf{33,373,229 messages} from \textbf{82,109 users} across \textbf{81 servers}, including \textbf{1,912,750 messages} from \textbf{633 bots}. Data collection occurred between November 13th and 15th, covering messages sent from January 1st to November 12th, just after the 2024 U.S. election.

\subsection{Characterizing the Political Spectrum}
\label{sec:timeline}

A key aspect of our research is distinguishing between Republican- and Democratic-aligned Discord servers. To categorize their political alignment, we relied on server names and self-descriptions, which often include rules, community guidelines, and references to key ideologies or figures. Each server's name and description were manually reviewed based on predefined, objective criteria, focusing on explicit political themes or mentions of prominent figures. This process allowed us to classify servers into three categories, ensuring a systematic and unbiased alignment determination.

\begin{itemize}
    \item \textbf{Republican-aligned}: Servers referencing Republican and right-wing and ideologies, movements, or figures (e.g., MAGA, Conservative, Traditional, Trump).  
    \item \textbf{Democratic-aligned}: Servers mentioning Democratic and left-wing ideologies, movements, or figures (e.g., Progressive, Liberal, Socialist, Biden, Kamala).  
    \item \textbf{Unaligned}: Servers with no defined spectrum and ideologies or opened to general political debate from all orientations.
\end{itemize}

To ensure the reliability and consistency of our classification, three independent reviewers assessed the classification following the specified set of criteria. The inter-rater agreement of their classifications was evaluated using Fleiss' Kappa \cite{fleiss1971measuring}, with a resulting Kappa value of \( 0.8191 \), indicating an almost perfect agreement among the reviewers. Disagreements were resolved by adopting the majority classification, as there were no instances where a server received different classifications from all three reviewers. This process guaranteed the consistency and accuracy of the final categorization.

Through this process, we identified \textbf{7 Republican-aligned servers}, \textbf{9 Democratic-aligned servers}, and \textbf{65 unaligned servers}.

Table \ref{tab:statistics} shows the statistics of the collected data. Notably, while Democratic- and Republican-aligned servers had a comparable number of user messages, users in the latter servers were significantly more active, posting more than double the number of messages per user compared to their Democratic counterparts. 
This suggests that, in our sample, Democratic-aligned servers attract more users, but these users were less engaged in text-based discussions. Additionally, around 10\% of the messages across all server categories were posted by bots. 

\subsection{Temporal Data} 

Throughout this paper, we refer to the election candidates using the names adopted by their respective campaigns: \textit{Kamala}, \textit{Biden}, and \textit{Trump}. To examine how the content of text messages evolves based on the political alignment of servers, we divided the 2024 election year into three periods: \textbf{Biden vs Trump} (January 1 to July 21), \textbf{Kamala vs Trump} (July 21 to September 20), and the \textbf{Voting Period} (after September 20). These periods reflect key phases of the election: the early campaign dominated by Biden and Trump, the shift in dynamics with Kamala Harris replacing Joe Biden as the Democratic candidate, and the final voting stage focused on electoral outcomes and their implications. This segmentation enables an analysis of how discourse responds to pivotal electoral moments.

Figure \ref{fig:line-plot} illustrates the distribution of messages over time, highlighting trends in total messages volume and mentions of each candidate. Prior to Biden's withdrawal on July 21, mentions of Biden and Trump were relatively balanced. However, following Kamala's entry into the race, mentions of Trump surged significantly, a trend further amplified by an assassination attempt on him, solidifying his dominance in the discourse. The only instance where Trump’s mentions were exceeded occurred during the first debate, as concerns about Biden’s age and cognitive abilities temporarily shifted the focus. In the final stages of the election, mentions of all three candidates rose, with Trump’s mentions peaking as he emerged as the victor.
We use nine widely-used node classification
benchmark datasets (Table~\ref{tab: main_data}), where four of them are heterophilic (Texas, Wisconsin, Cornell, Squirrel, and
Amazon-rating~\citep{platonov2023critical}), and the remaining four are homophilic (Cora~\citep{cora},
Citeseer~\citep{citeseer}, and Pubmed~\citep{pubmed}) including one large-scale dataset (Ogbn-Arxiv~\cite{hu2020ogb}). 
Further information about the datasets and splits are provided in Appendix~\ref{app: exp}.
% and experimental results on more datasets
% In line with prior research, we employ the default training/validation/test splits provided by Pytorch Geometric (PyG). 
% For details of the datasets, including their sources and construction methods.
% We evaluate \ours for the semi-supervised node classification on Cora~\cite{cora}, CiteSeer~\cite{citeseer} and PubMed~\cite{pubmed} and one large-scale dataset ogbn-arxiv from OGB benchmarks~\citep{openbenchmark}.
% We also extend our models to two heterophilous datasets: Chameleon and Squirrel~\cite{heter_dataset}.
% We show the details of the dataset in Appendix~\ref{app: data}.




\textbf{Baselines and experiment settings.}
We compare the performance of \ours against the following $12$ baseline models. 
% \begin{itemize}
1) \textbf{Classic models}: MLP, SGC~\citep{sgc}.
2) \textbf{GNNs with normalization}: BatchNorm~\citep{batchnorm}, PairNorm~\citep{pairnorm} and ContraNorm~\citep{contranorm}.
3) \textbf{Augmenation-based GNNs}: DropEdge~\citep{dropedge}.
4) \textbf{GNNs with residual connections}: Residual, APPNP~\citep{appap}, JKNET~\citep{jknet} and DAGNN~\citep{dagnn}. 
5) \textbf{Other baselines}: GCNII~\citep{GCNII} and \(\omega\)GCN~\citep{wGCN}.
% \xw{你忘记cite gcnii 跟wgcn了}
For the sake of fair comparison, we do not deploy specific training techniques used in some prior works for benchmarking.
All models are trained under the same setting on the pure SGC backbone and we choose the best of scale controller in the range of $\{ 0.1, 0.5, 0.9\}$ for ContraNorm, DropEdge, and residual connections.
% For both Label-\ours and Feature-\ours, 
We choose the best of $\lambda$ in the range of $\{0.1, 0.5, 0.9\}$, fix $\alpha=1$ and select the best value for $\beta$ from $\{ 0.1, 0.5, 0.9\}$ for \ours. 
More experiment results with hyperparameter tuning and optimization strategies can be found in Appendix~\ref{app: exp}.
% fix $\alpha=1$ and only select $\beta$ from $\{0.1, 1,10, 20, 50, 100\}$ for simplify .
% \end{itemize}
% For more details of the classic anti-oversmoothing methods seen in Appendix~\ref{xx}.
% We apply both the linear SGC~\cite{sgc} and non-linear GCN~\cite{gcn} backbones.
% For fair comparison, we fix the hidden dimension to $32$ and dropout rate to $0.6$ following \cite{contranorm}.
% % The residual $\alpha=0.5$, the dropedge present is selecting from $\{0.3,0.5,0.7\}$.
% % We choose the best of scale controller $\alpha,\ \beta \in \{ 0.1, 0.2, 0.5, 0.7, 0.9\}$.
% We select the best settings for PairNorm, Residual, DropEdge, and ContraNorm based on their default hyperparameters. 

% We use Tesla-V100-SXM2-32GB in all experiments.
\textbf{RQ1: Node classification performance.}
In Table~\ref{tab: main_data}, we provide the mean of the node classification accuracy along with their corresponding standard deviations across 10 random seeds under the same 2-layer SGC backbone following~\citet{dgc}.
Overall, \ours achieves the best performance across $8$ datasets in the shallow layers, as Label/Feature-\ours performs the best on 7 out of the 8 datasets. 
% In particular, we make the following three observations:
% First, \ours outperforms all normalization methods. 
% Since our theoretical findings suggest that these normalization methods are essentially implicitly signed graph propagation, the theoretical properties of structural balance (Section~\ref{subsec: sb theory}) contribute to the enhanced classification accuracy of \ours.
% Second, \ours outperforms random argumentation based GNNs. Since DropEdge randomly drops edges, it isn't easy to characterize their exact behaviors, but we highlight that it works when it happens to remove edges between different classes of nodes thanks to our structurally balanced theory, as 
% DropEdge still follows the unified signed graph analysis in its message-passing scheme.
% Lastly, \ours outperforms residual connection based GNNs, including the last layer connection: residual and multilayer feature connection: APPNP, JKNET, and DAGNN. 
% In our analysis, GNNs with residual connections can be seen as a special case of signed graph propagation, where their positive and negative adjacency matrices are the linear combination of adjacency matrices of different orders, yet they are not the theoretically best solution to alleviate oversmoothing.
% This validates the effectiveness of our novel insight from a signed graph perspective. 

% \paragraph{Heterophilic datasets}
% Besides the three homophilic datasets, we also conduct experiments on four heterophilic datasets~\citep{heter_dataset}.
% We find that our method is still the most effective one across all of the methods for alleviating oversmoothing as indicated in Table~\ref{table: gcn heter}.
% Interestingly, we observe that Feature-\ours performs better than Label-\ours on the heterophilic datasets, which is the opposite of the results on the homophilic datasets.

\textbf{RQ2: Anti-oversmoothing analysis.}
% \paragraph{Results} 
% The results for SGC are detailed in Table \ref{table: sgc results} and we give the GCN results in Appendix (Table~\ref{table: gcn result}). 
We further evaluate the robustness of \ours by assessing its performance at deeper model depths: $K \in \{2, 10, 50, 100, 300\}$ for homophilic datasets and $K \in \{2, 5, 10, 20, 50\}$ for heterophilic datasets.
% To provide a comparative analysis against other GNNs, we also evaluate two best-performed normalization-based GNNs: BatchNorm and ContraNorm. the performances of these methods
% We evaluate on one heterophilic graph and two homophilic graphs.
Figure~\ref{fig: layer depth} shows that the performance of Feature/Label-\ours remains relatively stable with varying
% ($K = 50$) 
numbers of layers, achieving its best performance when the model gets deeper.
In contrast, the normalization methods considered exhibit a substantial decrease in performance as the number of layers increases, indicating their persistent susceptibility to the oversmoothing problem.
Note that we find that for \ours to maintain performance in the heterophilic dataset, \(\beta\) needs to be larger than the uniform range considered in Figure~\ref{fig: layer depth}. 
See Appendix~\ref{app: exp} for the result under larger \(\beta\), where \ours on deep layers remains $\approx60\%$ in Cornell.   

% \subsection{RQ3: Ablation Study}
\textbf{RQ3.1: Sensitivity analysis of training ratio.}
% Since Label-\ours leverages the ground truth label information to construct the negative graph, we conduct an ablation study examining the impact of different training data ratios. 
As shown in Figure~\ref{fig: train ratio}, Label-\ours's performance on the CSBM and Cora datasets improves as the training ratio increases. Even with a modest training ratio of 20\%, the worst-performing models still achieve an impressive 80\% accuracy, while the best models approach 100\% accuracy when the training ratio is increased to 80\%. This is in line with our theoretical insights that increasing the training ratio leads to more structural balance resulting from our method~\ours. 
% Moreover, our main experiments detailed in Table~\ref{table: sgc results} demonstrate that Label-\ours outperforms other methods, even when adopting the default training set ratios in those datasets, indicating its effectiveness in real-world graph settings.
% % \begin{table}[!t]
% \centering
% \scalebox{0.68}{
%     \begin{tabular}{ll cccc}
%       \toprule
%       & \multicolumn{4}{c}{\textbf{Intellipro Dataset}}\\
%       & \multicolumn{2}{c}{Rank Resume} & \multicolumn{2}{c}{Rank Job} \\
%       \cmidrule(lr){2-3} \cmidrule(lr){4-5} 
%       \textbf{Method}
%       &  Recall@100 & nDCG@100 & Recall@10 & nDCG@10 \\
%       \midrule
%       \confitold{}
%       & 71.28 &34.79 &76.50 &52.57 
%       \\
%       \cmidrule{2-5}
%       \confitsimple{}
%     & 82.53 &48.17
%        & 85.58 &64.91
     
%        \\
%        +\RunnerUpMiningShort{}
%     &85.43 &50.99 &91.38 &71.34 
%       \\
%       +\HyReShort
%         &- & -
%        &-&-\\
       
%       \bottomrule

%     \end{tabular}
%   }
% \caption{Ablation studies using Jina-v2-base as the encoder. ``\confitsimple{}'' refers using a simplified encoder architecture. \framework{} trains \confitsimple{} with \RunnerUpMiningShort{} and \HyReShort{}.}
% \label{tbl:ablation}
% \end{table}
\begin{table*}[!t]
\centering
\scalebox{0.75}{
    \begin{tabular}{l cccc cccc}
      \toprule
      & \multicolumn{4}{c}{\textbf{Recruiting Dataset}}
      & \multicolumn{4}{c}{\textbf{AliYun Dataset}}\\
      & \multicolumn{2}{c}{Rank Resume} & \multicolumn{2}{c}{Rank Job} 
      & \multicolumn{2}{c}{Rank Resume} & \multicolumn{2}{c}{Rank Job}\\
      \cmidrule(lr){2-3} \cmidrule(lr){4-5} 
      \cmidrule(lr){6-7} \cmidrule(lr){8-9} 
      \textbf{Method}
      & Recall@100 & nDCG@100 & Recall@10 & nDCG@10
      & Recall@100 & nDCG@100 & Recall@10 & nDCG@10\\
      \midrule
      \confitold{}
      & 71.28 & 34.79 & 76.50 & 52.57 
      & 87.81 & 65.06 & 72.39 & 56.12
      \\
      \cmidrule{2-9}
      \confitsimple{}
      & 82.53 & 48.17 & 85.58 & 64.91
      & 94.90&78.40 & 78.70& 65.45
       \\
      +\HyReShort{}
       &85.28 & 49.50
       &90.25 & 70.22
       & 96.62&81.99 & \textbf{81.16}& 67.63
       \\
      +\RunnerUpMiningShort{}
       % & 85.14& 49.82
       % &90.75&72.51
       & \textbf{86.13}&\textbf{51.90} & \textbf{94.25}&\textbf{73.32}
       & \textbf{97.07}&\textbf{83.11} & 80.49& \textbf{68.02}
       \\
   %     +\RunnerUpMiningShort{}
   %    & 85.43 & 50.99 & 91.38 & 71.34 
   %    & 96.24 & 82.95 & 80.12 & 66.96
   %    \\
   %    +\HyReShort{} old
   %     &85.28 & 49.50
   %     &90.25 & 70.22
   %     & 96.62&81.99 & 81.16& 67.63
   %     \\
   % +\HyReShort{} 
   %     % & 85.14& 49.82
   %     % &90.75&72.51
   %     & 86.83&51.77 &92.00 &72.04
   %     & 97.07&83.11 & 80.49& 68.02
   %     \\
      \bottomrule

    \end{tabular}
  }
\caption{\framework{} ablation studies. ``\confitsimple{}'' refers using a simplified encoder architecture. \framework{} trains \confitsimple{} with \RunnerUpMiningShort{} and \HyReShort{}. We use Jina-v2-base as the encoder due to its better performance.
}
\label{tbl:ablation}
\end{table*}


% \begin{figure}[t]
%     \centering
%     \begin{subfigure}{0.63\textwidth}
%         \centering
%         \includegraphics[width=0.99\textwidth]{figures/eval_negative (3).pdf} % Adjust the path and filename as necessary
%         \caption{ Significance plot for $\beta$ in terms of test accuracy on cora (left) and texas (right) with fixed $\alpha=1$}
%         \label{fig: beta}
%     \end{subfigure}
%     \quad
%     \begin{subfigure}{0.33\textwidth}
%         \centering
%         \captionsetup{font=small}
%         \includegraphics[width=0.99\textwidth]{figures/eval_train.pdf} % Adjust the path and filename as necessary
%         \caption{Ablation study on Label-SBP. X-axis indicates the ratio of the training node numbers.}
%         \label{}
%     \end{subfigure}
%     \caption{Ablation study}
%     \label{fig: train ratio}
% \end{figure}




% \begin{figure*}[t]
% % \hspace{-10pt}
%     \begin{minipage}{.48\textwidth}
%     \captionof{table}{Node classification accuracy (\%) on the large-scale dataset~\textit{ogbn-arxiv}.}
%     % Test accuracy (\%) comparison results on large scale dataset (Ogbn-ArXiv). The best results are marked in blue and the second best results are marked in gray on every layer.
%     \centering
%     \resizebox{0.99\linewidth}{!}{
%     \begin{tabular}{lcccc}
%     \toprule
%      Model             & \#L=2              & \#L=4              & \#L=8            & \#L=16  \\
%     \midrule
%     GCN & 67.32 {\footnotesize $\pm$ 0.28} & 67.79 {\footnotesize $\pm$ 0.25} & 65.54 {\footnotesize $\pm$ 0.31} & 59.13 {\footnotesize $\pm$ 0.95}  \\
%          BatchNorm & 70.14 {\footnotesize $\pm$ 0.28} & 70.93 {\footnotesize $\pm$ 0.15} & 70.14 {\footnotesize $\pm$ 0.43} & 63.24 {\footnotesize $\pm$ 1.40} \\
%          % +LayerNorm& \cellcolor{secondbest}70.53 {\footnotesize $\pm$ 0.19} & \cellcolor{best}71.66 {\footnotesize $\pm$ 0.17} & \cellcolor{secondbest}71.23 {\footnotesize $\pm$ 0.16} & 68.62 {\footnotesize $\pm$ 0.47} \\
%          PairNorm & 70.48 {\footnotesize $\pm$ 0.20} & \cellcolor{best}71.59 {\footnotesize $\pm$ 0.17} & \cellcolor{best}71.24 {\footnotesize $\pm$ 0.07} & 68.92 {\footnotesize $\pm$ 0.43} \\
%          ContraNorm & OOM & OOM & OOM & OOM \\
%          DropEdge & 64.07 {\footnotesize $\pm$ 0.32} & 63.92 {\footnotesize $\pm$ 0.27} & 60.74 {\footnotesize $\pm$ 0.45} & 52.52 {\footnotesize $\pm$ 0.34} \\
%          Residual & 66.90 {\footnotesize $\pm$ 0.14} & 66.67 {\footnotesize $\pm$ 0.25} & 61.76 {\footnotesize $\pm$ 0.62} & 53.25 {\footnotesize $\pm$ 0.75} \\
%     % \midrule
%          Feature-\ourst & 67.89 {\footnotesize $\pm$ 0.10} & 68.47 {\footnotesize $\pm$ 0.26} & 65.09 {\footnotesize $\pm$ 0.30} & 60.34 {\footnotesize $\pm$ 0.94} \\
%          Label-\ourst & \cellcolor{best}70.55 {\footnotesize $\pm$ 0.22} & 71.54 {\footnotesize $\pm$ 0.18} & 71.07 {\footnotesize $\pm$ 0.28} & \cellcolor{best}69.33 {\footnotesize $\pm$ 0.59}  \\
%     \bottomrule
%     \end{tabular}
%     \label{tab: large}
%     }\hfill
%     \end{minipage}
%    \begin{minipage}{.52\textwidth}
%    % \vspace{0.2cm}
%        \centering
%        \includegraphics[width=\linewidth]{figures/eval_negative (3).pdf}
%        \captionof{figure}{Significance of negative graph weight $\beta$ on Cora and Texas datasets where we fix the positive graph weight $\alpha=1$.}
%        \label{fig:beta real} 
%    \end{minipage}
%    % \vspace{-0.5cm}
%     % \hspace{-10pt}
% %     \begin{minipage}{.35\textwidth}
% %     \centering
% %     \captionsetup{font=small}
% %     \caption{Accuracy on different splits of train/valid/test dataset. SGC in CSBM and GCN in Cora.}
% %     % SGC test accuracy (\%) comparison results on sbm of \ourst-Label on different splits of train/valid/test dataset. GCN test accuracy (\%) comparison results on Cora of \ourst-Label on different splits of train/valid/test dataset.
% % % The best results are marked in blue on every dataset.
% %     \centering
% %      \resizebox{0.95\linewidth}{!}{
% %     \begin{tabular}{lcc}
% %     \toprule
% %      Splits & CSBM  & Cora   \\
% %     \midrule
% %     % \midrule
% %     % sbm 0/5/5& 57.00 {\footnotesize $\pm$ 13.36}
% %       2/4/4 & 86.25 {\footnotesize $\pm$ 3.01} & 82.80 {\footnotesize $\pm$ 0.81}\\
% %       4/3/3 & 91.50 {\footnotesize $\pm$ 2.52} & 85.39 {\footnotesize $\pm$ 0.18 }\\
% %       6/4/4 & 91.50 {\footnotesize $\pm$ 6.05} & 87.64 {\footnotesize $\pm$ 0.37 }\\
% %       8/1/1 & \cellcolor{best}99.05 {\footnotesize $\pm$ 1.90} & \cellcolor{best} 94.10 {\footnotesize $\pm$ 0.74} \\
% %     \bottomrule
% %     \end{tabular}
% %     \label{tab: ablation}
% %     }
% %     \end{minipage}
%     % \caption{Caption}
%     % \label{tab:my_label}
% \vspace{-0.15in}
% \end{figure*}
\begin{figure}
   % \vspace{0.2cm}
   % \captionsetup{font=small}
       \centering
       \includegraphics[width=0.7\linewidth]{figures/eval_negative_3.pdf}
       \captionof{figure}{Significance of negative graph weight $\beta$ on Cora and Texas datasets where we fix the positive graph weight $\alpha=1$ and vary a large range of \(\beta\).}
       \label{fig:beta real} 
    % \vspace{-0.2in}
\end{figure}
\textbf{RQ3.2: Performance under varying graph homophily and heterophily levels.}
In order to test the performance of \ours on graphs with arbitrary
levels of homophily and heterophily, we conduct an ablation study in the CSBM setting with the controllable homophilic and heterophilic levels following~\citet{GRP-GNN}.
As shown in Figure~\ref{fig:beta csbm}, Feature/Label-\ours performs best in homophilic graphs when all nodes are effectively attracted to one another, i.e., when the repulsion strength $\beta$ is small. As $\beta$ increases, the performance of the model degrades.
% The parameter $\phi$ in the CSBM controls the relative importance of node features and graph topology in determining the homophily level.
% Specifically, $\phi$ ranges from -1 to 1, with lower values corresponding to strongly heterophilic graphs and higher values indicating strongly homophilic graphs. 
% Specially, $\phi=1$ corresponds to strongly homophilic graphs while $\phi=-1$ corresponds to strongly heterophilic graphs.
% We fix $\lambda=0.5$ and then vary $\beta$ which indicates the strength of the repulsive force between the two nodes introduced by the negative edge connecting them.
In contrast, for heterophilic graphs, when the attraction power of the positive graph dominates, \ours achieves only $50\%$ accuracy. 
As $\beta$ increases, the negative graph becomes more dominant, and the model's performance gets significantly better. We observe similar phenomena in the real homophilic and heterophilic graph datasets as shown in Figure~\ref{fig:beta real}.

% \vspace{-1ex}
\textbf{RQ3.3: Performance on large-scale dataset.} 
Finally, we conduct an evaluation of \ours on the large-scale ogbn-arxiv dataset, and the results are presented in Table \ref{tab: large}. 
% To maintain the sparsity of the graph structure and avoid additional computational overhead, we adopt variants of the \ours approach mentioned in Section~\ref{sec: method}. 
Overall, the results demonstrate that Label-\ours-v2 achieves comparable or even superior performance compared to previous normalization methods, particularly in the deep layer setting  ($L=16$).
This verifies the empirical superiority and robustness of our proposed signed graph construction in \ours, which effectively leverages the available label information to alleviate oversmoothing, even at scale.
\begin{table}
    \captionof{table}{Node classification accuracy (\%) on the large-scale dataset~\textit{ogbn-arxiv}.}
    % Test accuracy (\%) comparison results on large scale dataset (Ogbn-ArXiv). The best results are marked in blue and the second best results are marked in gray on every layer.
    \centering
    \resizebox{0.7\linewidth}{!}{
    \begin{tabular}{lcccc}
    \toprule
     Model             & \#L=2              & \#L=4              & \#L=8            & \#L=16  \\
    \midrule
    GCN & 67.32 {\footnotesize $\pm$ 0.28} & 67.79 {\footnotesize $\pm$ 0.25} & 65.54 {\footnotesize $\pm$ 0.31} & 59.13 {\footnotesize $\pm$ 0.95}  \\
         BatchNorm & 70.14 {\footnotesize $\pm$ 0.28} & 70.93 {\footnotesize $\pm$ 0.15} & 70.14 {\footnotesize $\pm$ 0.43} & 63.24 {\footnotesize $\pm$ 1.40} \\
         % +LayerNorm& \cellcolor{secondbest}70.53 {\footnotesize $\pm$ 0.19} & \cellcolor{best}71.66 {\footnotesize $\pm$ 0.17} & \cellcolor{secondbest}71.23 {\footnotesize $\pm$ 0.16} & 68.62 {\footnotesize $\pm$ 0.47} \\
         PairNorm & 70.48 {\footnotesize $\pm$ 0.20} & \cellcolor{best}71.59 {\footnotesize $\pm$ 0.17} & \cellcolor{best}71.24 {\footnotesize $\pm$ 0.07} & 68.92 {\footnotesize $\pm$ 0.43} \\
         ContraNorm & OOM & OOM & OOM & OOM \\
         DropEdge & 64.07 {\footnotesize $\pm$ 0.32} & 63.92 {\footnotesize $\pm$ 0.27} & 60.74 {\footnotesize $\pm$ 0.45} & 52.52 {\footnotesize $\pm$ 0.34} \\
         Residual & 66.90 {\footnotesize $\pm$ 0.14} & 66.67 {\footnotesize $\pm$ 0.25} & 61.76 {\footnotesize $\pm$ 0.62} & 53.25 {\footnotesize $\pm$ 0.75} \\
    % \midrule
         % Feature-\ourst-v2 & 67.89 {\footnotesize $\pm$ 0.10} & 68.47 {\footnotesize $\pm$ 0.26} & 65.09 {\footnotesize $\pm$ 0.30} & 60.34 {\footnotesize $\pm$ 0.94} \\
         Label-\ourst-v2 & \cellcolor{best}70.55 {\footnotesize $\pm$ 0.22} & 71.54 {\footnotesize $\pm$ 0.18} & 71.07 {\footnotesize $\pm$ 0.28} & \cellcolor{best}69.33 {\footnotesize $\pm$ 0.59}  \\
    \bottomrule
    \end{tabular}
    \label{tab: large}
    }
    % \vspace{-0.2in}
\end{table}

% verifying the empirical advantages of our proposed technique.
% \phi 
% Hyperparameter $\beta$ indicating strength of the repulsive force between the two nodes introduced by negative edges connecting them
% To illustrate the impact of $\beta$, we conduct ablation studies on the synthetic CSBM graphs under various settings as well as real datasets. 
%
% Following~\cite{GRP-GNN}, we reconstruct the CSBM with the 
% The parameter $\phi$ to control for the the information given by the node features and the graph topology and the homophily level. 
% Specifically $\phi=0$ indicates that only node features are informative, while $|\phi|=1$ indicates that only the graph topology is informative. Moreover, $\phi=1$ corresponds to strongly homophilic graphs while $\phi=-1$ corresponds to strongly heterophilic graphs.
%
% \begin{figure}[t]
% % \hspace{-10pt}
%     \begin{minipage}{.48\textwidth}
%     \captionof{table}{GCN test accuracy on the large-scale dataset~\textit{ogbn-arxiv}.}
%     % Test accuracy (\%) comparison results on large scale dataset (Ogbn-ArXiv). The best results are marked in blue and the second best results are marked in gray on every layer.
%     \centering
%     \resizebox{0.99\linewidth}{!}{
%     \begin{tabular}{lcccc}
%     \toprule
%      Model             & \#L=2              & \#L=4              & \#L=8            & \#L=16  \\
%     \midrule
%     GCN & 67.32 {\footnotesize $\pm$ 0.28} & 67.79 {\footnotesize $\pm$ 0.25} & 65.54 {\footnotesize $\pm$ 0.31} & 59.13 {\footnotesize $\pm$ 0.95}  \\
%          BatchNorm & 70.14 {\footnotesize $\pm$ 0.28} & 70.93 {\footnotesize $\pm$ 0.15} & 70.14 {\footnotesize $\pm$ 0.43} & 63.24 {\footnotesize $\pm$ 1.40} \\
%          % +LayerNorm& \cellcolor{secondbest}70.53 {\footnotesize $\pm$ 0.19} & \cellcolor{best}71.66 {\footnotesize $\pm$ 0.17} & \cellcolor{secondbest}71.23 {\footnotesize $\pm$ 0.16} & 68.62 {\footnotesize $\pm$ 0.47} \\
%          PairNorm & 70.48 {\footnotesize $\pm$ 0.20} & \cellcolor{best}71.59 {\footnotesize $\pm$ 0.17} & \cellcolor{best}71.24 {\footnotesize $\pm$ 0.07} & 68.92 {\footnotesize $\pm$ 0.43} \\
%          ContraNorm & OOM & OOM & OOM & OOM \\
%          DropEdge & 64.07 {\footnotesize $\pm$ 0.32} & 63.92 {\footnotesize $\pm$ 0.27} & 60.74 {\footnotesize $\pm$ 0.45} & 52.52 {\footnotesize $\pm$ 0.34} \\
%          Residual & 66.90 {\footnotesize $\pm$ 0.14} & 66.67 {\footnotesize $\pm$ 0.25} & 61.76 {\footnotesize $\pm$ 0.62} & 53.25 {\footnotesize $\pm$ 0.75} \\
%     % \midrule
%          % Feature-\ourst & 67.89 {\footnotesize $\pm$ 0.10} & 68.47 {\footnotesize $\pm$ 0.26} & 65.09 {\footnotesize $\pm$ 0.30} & 60.34 {\footnotesize $\pm$ 0.94} \\
%          Label-\ourst & \cellcolor{best}70.55 {\footnotesize $\pm$ 0.22} & 71.54 {\footnotesize $\pm$ 0.18} & 71.07 {\footnotesize $\pm$ 0.28} & \cellcolor{best}69.33 {\footnotesize $\pm$ 0.59}  \\
%     \bottomrule
%     \end{tabular}
%     \label{tab: large}
%     }\hfill
%     \end{minipage}
%    \begin{minipage}{.52\textwidth}
%    \vspace{0.2cm}
%        \centering
%        \includegraphics[width=\linewidth]{figures/eval_negative (3).pdf}
%        \captionof{figure}{Significance of negative graph weight $\beta$ on Cora and Texas datasets where we fix the positive graph weight $\alpha=1$.}
%        \label{fig:beta real} 
%    \end{minipage}
%    % \vspace{-0.5cm}
%     % \hspace{-10pt}
% %     \begin{minipage}{.35\textwidth}
% %     \centering
% %     \captionsetup{font=small}
% %     \caption{Accuracy on different splits of train/valid/test dataset. SGC in CSBM and GCN in Cora.}
% %     % SGC test accuracy (\%) comparison results on sbm of \ourst-Label on different splits of train/valid/test dataset. GCN test accuracy (\%) comparison results on Cora of \ourst-Label on different splits of train/valid/test dataset.
% % % The best results are marked in blue on every dataset.
% %     \centering
% %      \resizebox{0.95\linewidth}{!}{
% %     \begin{tabular}{lcc}
% %     \toprule
% %      Splits & CSBM  & Cora   \\
% %     \midrule
% %     % \midrule
% %     % sbm 0/5/5& 57.00 {\footnotesize $\pm$ 13.36}
% %       2/4/4 & 86.25 {\footnotesize $\pm$ 3.01} & 82.80 {\footnotesize $\pm$ 0.81}\\
% %       4/3/3 & 91.50 {\footnotesize $\pm$ 2.52} & 85.39 {\footnotesize $\pm$ 0.18 }\\
% %       6/4/4 & 91.50 {\footnotesize $\pm$ 6.05} & 87.64 {\footnotesize $\pm$ 0.37 }\\
% %       8/1/1 & \cellcolor{best}99.05 {\footnotesize $\pm$ 1.90} & \cellcolor{best} 94.10 {\footnotesize $\pm$ 0.74} \\
% %     \bottomrule
% %     \end{tabular}
% %     \label{tab: ablation}
% %     }
% %     \end{minipage}
%     % \caption{Caption}
%     % \label{tab:my_label}
% % \vspace{-0.12in}
% \end{figure}

% \begin{wraptable}{r}{.5\linewidth}
% \begin{table}[h]
\centering
% \small
\caption{Test accuracy (\%) comparison results on large scale dataset (Ogbn-ArXiv). 
The best results are marked in blue and the second best results are marked in gray on every layer.}
\begin{adjustbox}{width=0.7\textwidth}
\begin{tabular}{lcccc}
\toprule
 Model             & \#L=2              & \#L=4              & \#L=8            & \#L=16  \\
\midrule


% \midrule

GCN & 67.32 {\footnotesize $\pm$ 0.28} & 67.79 {\footnotesize $\pm$ 0.25} & 65.54 {\footnotesize $\pm$ 0.31} & 59.13 {\footnotesize $\pm$ 0.95}  \\
     BatchNorm & 70.14 {\footnotesize $\pm$ 0.28} & 70.93 {\footnotesize $\pm$ 0.15} & 70.14 {\footnotesize $\pm$ 0.43} & 63.24 {\footnotesize $\pm$ 1.40} \\
     % +LayerNorm& \cellcolor{secondbest}70.53 {\footnotesize $\pm$ 0.19} & \cellcolor{best}71.66 {\footnotesize $\pm$ 0.17} & \cellcolor{secondbest}71.23 {\footnotesize $\pm$ 0.16} & 68.62 {\footnotesize $\pm$ 0.47} \\
     PairNorm & 70.48 {\footnotesize $\pm$ 0.20} & \cellcolor{best}71.59 {\footnotesize $\pm$ 0.17} & \cellcolor{best}71.24 {\footnotesize $\pm$ 0.07} & \cellcolor{best}68.92 {\footnotesize $\pm$ 0.43} \\
     ContraNorm & OOM & OOM & OOM & OOM \\
     DropEdge & 64.07 {\footnotesize $\pm$ 0.32} & 63.92 {\footnotesize $\pm$ 0.27} & 60.74 {\footnotesize $\pm$ 0.45} & 52.52 {\footnotesize $\pm$ 0.34} \\
     Residual & 66.90 {\footnotesize $\pm$ 0.14} & 66.67 {\footnotesize $\pm$ 0.25} & 61.76 {\footnotesize $\pm$ 0.62} & 53.25 {\footnotesize $\pm$ 0.75} \\

     % Feature-\ourst & 67.89 {\footnotesize $\pm$ 0.10} & 68.47 {\footnotesize $\pm$ 0.26} & 65.09 {\footnotesize $\pm$ 0.30} & 60.34 {\footnotesize $\pm$ 0.94} \\
     Label-\ourst & \cellcolor{best}70.55 {\footnotesize $\pm$ 0.22} & 71.54 {\footnotesize $\pm$ 0.18} & 71.07 {\footnotesize $\pm$ 0.28} & \cellcolor{best}69.33 {\footnotesize $\pm$ 0.59}  \\

% % Add more rows as needed
\bottomrule
\end{tabular}
\end{adjustbox}
\label{table: large result}
% \end{table}
\end{wraptable}





% \subsection{Ablation Study}





% Appedix~\ref{app: ablation}. 




\label{sec: exp}
\paragraph{Summary}
Our findings provide significant insights into the influence of correctness, explanations, and refinement on evaluation accuracy and user trust in AI-based planners. 
In particular, the findings are three-fold: 
(1) The \textbf{correctness} of the generated plans is the most significant factor that impacts the evaluation accuracy and user trust in the planners. As the PDDL solver is more capable of generating correct plans, it achieves the highest evaluation accuracy and trust. 
(2) The \textbf{explanation} component of the LLM planner improves evaluation accuracy, as LLM+Expl achieves higher accuracy than LLM alone. Despite this improvement, LLM+Expl minimally impacts user trust. However, alternative explanation methods may influence user trust differently from the manually generated explanations used in our approach.
% On the other hand, explanations may help refine the trust of the planner to a more appropriate level by indicating planner shortcomings.
(3) The \textbf{refinement} procedure in the LLM planner does not lead to a significant improvement in evaluation accuracy; however, it exhibits a positive influence on user trust that may indicate an overtrust in some situations.
% This finding is aligned with prior works showing that iterative refinements based on user feedback would increase user trust~\cite{kunkel2019let, sebo2019don}.
Finally, the propensity-to-trust analysis identifies correctness as the primary determinant of user trust, whereas explanations provided limited improvement in scenarios where the planner's accuracy is diminished.

% In conclusion, our results indicate that the planner's correctness is the dominant factor for both evaluation accuracy and user trust. Therefore, selecting high-quality training data and optimizing the training procedure of AI-based planners to improve planning correctness is the top priority. Once the AI planner achieves a similar correctness level to traditional graph-search planners, strengthening its capability to explain and refine plans will further improve user trust compared to traditional planners.

\paragraph{Future Research} Future steps in this research include expanding user studies with larger sample sizes to improve generalizability and including additional planning problems per session for a more comprehensive evaluation. Next, we will explore alternative methods for generating plan explanations beyond manual creation to identify approaches that more effectively enhance user trust. 
Additionally, we will examine user trust by employing multiple LLM-based planners with varying levels of planning accuracy to better understand the interplay between planning correctness and user trust. 
Furthermore, we aim to enable real-time user-planner interaction, allowing users to provide feedback and refine plans collaboratively, thereby fostering a more dynamic and user-centric planning process.

\label{sec: conclusion}


%%%%%%%%%%%%%%%%%%%%%%%%%%%%%%%%%%%%%%%%%%%%%%%%%%%%%%%%%%%%

% In the unusual situation where you want a paper to appear in the
% references without citing it in the main text, use \nocite
% \nocite{langley00}

\bibliographystyle{plainnat}
\bibliography{reference}
% \bibliographystyle{plainnat}

\newpage
\appendix
\onecolumn

\begin{center}
	\LARGE \bf {Appendix}
\end{center}


\etocdepthtag.toc{mtappendix}
\etocsettagdepth{mtchapter}{none}
\etocsettagdepth{mtappendix}{subsubsection}
\tableofcontents
\newpage

% \section{Background} \label{section:LLM}

% \subsection{Large Language Model (LLM)}   

Figure~\ref{fig:LLaMA_model}(a) shows that a decoder-only LLM initially processes a user prompt in the “prefill” stage and subsequently generates tokens sequentially during the “decoding” stage.
Both stages contain an input embedding layer, multiple decoder transformer blocks, an output embedding layer, and a sampling layer.
Figure~\ref{fig:LLaMA_model}(b) demonstrates that the decoder transformer blocks consist of a self attention and a feed-forward network (FFN) layer, each paired with residual connection and normalization layers. 

% Differentiate between encoder/decoder, explain why operation intensity is low, explain the different parts of a transformer block. Discuss Table II here. 

% Explain the architecture with Llama2-70B.

% \begin{table}[thb]
% \renewcommand\arraystretch{1.05}
% \centering
% % \vspace{-5mm}
%     \caption{ML Model Parameter Size and Operational Intensity}
%     \vspace{-2mm}
%     \small
%     \label{tab:ML Model Parameter Size and Operational Intensity}    
%     \scalebox{0.95}{
%         \begin{tabular}{|c|c|c|c|c|}
%             \hline
%             & Llama2 & BLOOM & BERT & ResNet \\
%             Model & (70B) & (176B) & & 152 \\
%             \hline
%             Parameter Size (GB) & 140 & 352 & 0.17 & 0.16 \\
%             \hline
%             Op Intensity (Ops/Byte) & 1 & 1 & 282 & 346 \\
%             \hline
%           \end{tabular}
%     }
% \vspace{-3mm}
% \end{table}

% {\fontsize{8pt}{11pt}\selectfont 8pt font size test Memory Requirement}

\begin{figure}[t]
    \centering
    \includegraphics[width=8cm]{Figure/LLaMA_model_new_new.pdf}
    \caption{(a) Prefill stage encodes prompt tokens in parallel. Decoding stage generates output tokens sequentially.
    (b) LLM contains N$\times$ decoder transformer blocks. 
    (c) Llama2 model architecture.}
    \label{fig:LLaMA_model}
\end{figure}

Figure~\ref{fig:LLaMA_model}(c) demonstrates the Llama2~\cite{touvron2023llama} model architecture as a representative LLM.
% The self attention layer requires three GEMVs\footnote{GEMVs in multi-head attention~\cite{attention}, narrow GEMMs in grouped-query attention~\cite{gqa}.} to generate query, key and value vectors.
In the self-attention layer, query, key and value vectors are generated by multiplying input vector to corresponding weight matrices.
These matrices are segmented into multiple heads, representing different semantic dimensions.
The query and key vectors go though Rotary Positional Embedding (RoPE) to encode the relative positional information~\cite{rope-paper}.
Within each head, the generated key and value vectors are appended to their caches.
The query vector is multiplied by the key cache to produce a score vector.
After the Softmax operation, the score vector is multiplied by the value cache to yield the output vector.
The output vectors from all heads are concatenated and multiplied by output weight matrix, resulting in a vector that undergoes residual connection and Root Mean Square layer Normalization (RMSNorm)~\cite{rmsnorm-paper}.
The residual connection adds up the input and output vectors of a layer to avoid vanishing gradient~\cite{he2016deep}.
The FFN layer begins with two parallel fully connections, followed by a Sigmoid Linear Unit (SiLU), and ends with another fully connection.
% \section{Brader Impacts}
% \label{app: impact}
% Considering the high sensitivity of GNNs to the oversmoothing issue and the depth of layers,
% it is important to develop GNNs that can performance well when the layers go deep, especially for realistic scenarios such as the social network where the scale of friends is getting bigger.
% By introducing the concept and theory of \oursfull (\ours), our work can serve as an initiate step towards tacking oversmoothing problem on graphs,
% with the hope to empower GNNs for broader applications and social benefits.
% Besides, this paper does not raise any ethical concerns.
% This study does not involve any human subjects, practices to data set releases, potentially harmful insights, methodologies and applications, potential conflicts of interest and sponsorship, discrimination/bias/fairness concerns, privacy and security issues, legal compliance, and research integrity issues.

% \section{Discussions on limitations of $\ours$ and future directions}
% \label{app_sec: limiatation}
% Although we have conducted comprehensive theoretical analysis and extensive experiments, there are still aspects requiring further investigation. 

% \paragraph{Theoretical limitation.}
% Our theory is based on asymptotic behaviors, it remains to be explored under non-asymptotic conditions.
% We use the classic signed graph theory---structural balance theory, to capture the ideal distribution of the positive and negative edges and further inspire Label-\ours and Feature-\ours, but it still maintains a gap between the theory and the practice as shown that the performance still decreases when the layers go to deepest (e.g., $64$ for GCN~\citep{gcn}, $300$ for SGC~\citep{sgc}).
% % Additionally, we have not fully captured the relationship between the previous methods for alleviating the oversmoothing and the signed graph propagation through the theoretical perspective. 
% Moreover, the label rate for the influence on test performance necessitates further theoretical analysis.

% \paragraph{More sophisticated architectures/parameter tunning.}
% The Label-\ours and Feature-\ours can have multiple implementations.
% We choose the most classic architectures GCN and the linear architecture SGC in our experiments for the purpose of concept verification.
% Moreover, as shown in Appendix \ref{app: exp}, Label-\ours and Feature-\ours still requires certain additional tunning efforts for the objectives.
% Hence we believe it is also a promising future direction to reduce the parameter tunning by leveraging better optimization techniques.

% \paragraph{Better signed graph design.}
% Typical signed graph contains edges with either positive or negative signs~\cite{signedgraph,yan2022two,LRGNN,acmp}.
% Moreover, the value of every edge can be extended from a concrete number, such as $0,1,-1$, to a continuous number.
% Since our implementation of Label-\ours in this work aims to verify the theoretical findings, we do not apply sophisticated edge assignments during the signed graph propagation, simply using the label-enhanced negative subgraph and the feature similarities enhanced negative subgraph while maintaining the positive subgraph as the adjacency matrix.
% Nevertheless, it is promising to leverage better positive and negative subgraphs to improve the performance of the signed graph propagation to alleviate oversmoothing. 
\begin{figure*}
    \centering
    \includegraphics[width=1.0\linewidth]{figures/csbm-b-2.pdf}
    \caption{ Figure (a)-(d) shows the effect of negative graph weight $\beta$ by \ours on CSBM. In all cases, $\lambda=0.5$ and $\alpha=1$. The X-axis is the $\beta$ and the Y-axis is the test accuracy. 
    $\phi$ is the hyperparameter to control the level of homophily and $H(G)$ measure the homophily level. 
    SBP1 indicates Label-\ours and SBP2 indicates Feature-\ours. }
    \label{fig:beta csbm}
    % \vspace{-0.1in}
\end{figure*}



\section{Related Work}

\paragraph{Theory of Oversmoothing}
The concept of oversmoothing was initially introduced by \cite{oversmooth_first}: when the number of layers becomes large, the representations of different nodes tend to converge to a common value after excessively exchanging messages with neighboring nodes. 
\cite{Oono2019GraphNN, wu2023demystifying} rigorously show that the convergence of node representations to a common value happens at an exponential rate as the number of layers increases to infinity, for GCNs and attention-based GNNs, respectively. 
% \cite{zhou2020graph} shows that under specific conditions, that the ultimate convergence point solely encodes information about the graph's structure.
\cite{sbm_xinyi}~theoretically proves that oversmoothing can start to happen even in shallow depth under certain random graph settings.
\jq{\cite{zhou2021dirichlet} proposed an appropriate residual connection according to the lower limit of Dirichlet energy and connected to previous methods qualitatively.}

\paragraph{Signed Graph Inspired Methods}
In the heterophilic graphs, various methods are inspired by the signed graph propagation~\citep{H2GNN,orderedgnn, yan2022two,acmp, GRP-GNN}. In particular,
\citet{yan2022two,acmp} utilize the idea that the negative edges denote connections between nodes that are "not similar to each other" to create repulsion between them during message-passing.
\cite{GRP-GNN} extend the coefficients of the output of different layers in the final aggregation to be learnable and find that the odd layer coefficients tends to be negative for heterophilic graphs, suggesting that learning naturally finds signed-graph message-passing. 
However, \cite{signremedy} show that under some specific random graph settings, the oversmoothing will even happen under signed graph propagation. 
% which aligns with the case in our analysis for Theorem~\ref{thm: small nega} when the repulsion among nodes are not sufficient. 
Nevertheless, we extend the theory to generic graphs and prove that in the ideal state---structural balance, signed edges can indeed serve as a remedy to effectively combat oversmoothing. 

\paragraph{Structural Balance}
Structual balance theory has gained significant attention in recent years~\citep{signedgraph,yan2022two,LRGNN,acmp}. 
%
Inspired by the structural balance theory, \cite{signedgraph} characterizes the balanced path intuitively to learn both balanced and unbalanced representations on each layer.
%
\cite{LRGNN} predicts the signed adjacency matrix by an off-the-shelf neural network classifier to generate pseudo labels with the low-rank assumption.
\cite{signed_dynamics_paper_review} introduces the definition of the Laplacian for signed graphs and develops a comprehensive mathematical theory.
%
In this paper, we rigorously show that structural balance is the theoretical solution to alleviate oversmoothing and propose practical methods based on the property without any additional learnable parameters. 


% In addition to the above methods which explicitly make use of the signed graph propagation, in this paper, we also revisit a wide class of previous anti-oversmoothing methods that do not explicitly claim to use signed message-passing. We find that all of them can be attributed to some kind of design of negative edges to the original graph.
% a theoretical analysis that prioritizes the significance of signed graphs over unsigned graphs, by leveraging the principles of structural balance theory.

% Introduce signed graph and related GNN papers using signed graph. Discuss similar works to ours that also analyze from attractive and repulsive forces. Our key innovation: the formulation and the analysis based on SB.\yf{refined as above}

% On the empirical side, the Simplified Graph Convolution Network (SGC)~\cite{sgc} attributes oversmoothing to non-linear operations and suggests removing non-linear activations, resulting in \(H_{(k)} = Z_{(k-1)}W_{(k-1)}\). Other methods address oversmoothing by incorporating normalization operations, such as LayerNorm~\cite{layernorm}, BatchNorm~\cite{batchnorm}, PairNorm~\cite{pairnorm}, and ContraNorm~\cite{contranorm}. Additionally, DropEdge~\cite{dropedge} mitigates oversmoothing by randomly dropping a percentage \(p\) of edges in the graph, while residual connections~\cite{Chen2020SimpleAD, appap} between the initial and current layers' features further alleviate the issue.

% \textbf{Signed Graph Network.}
% Over the past few years, various signed network models have been proposed~\cite{LRGNN,acmp,yan2022two} while each with notable limitations. 
% %
% \cite{LRGNN} attempts to merge features and the adjacency matrix into a low-rank signed graph using a complex optimization algorithm, which is computationally intensive and difficult to scale. 
% %
% \cite{yan2022two} explores the interaction between oversmoothing and heterophily but limits its approach to merely splitting the cosine similarity matrix into positive and negative matrices based on signs, which might oversimplify the complex graph structure interactions. 
% %
% Additionally, \cite{acmp} addresses signed graph behavior under oversmoothing in continuous, finite GNN layers, but their findings are restricted to limited-layer scenarios. 
% %
% In this paper, we propose a unified signed graph framework that revisits previous methods for mitigating oversmoothing and extends these theories to more generalized, infinite situations, thereby overcoming the limitations identified in earlier research.

% Nevertheless, this form of aggregation treats neighboring nodes indiscriminately across various classes, leading to a decline in node classification accuracy with increasing layer depth, known as the phenomenon of \textit{oversmoothing}.

% \textbf{Oversmoothing.}
% We give he definition of oversmoothing inspired by \cite{graph_oversmoothing_survey}.
% \begin{definition}[Oversmoothing]
%     Let $G$ be an undirected, connected graph and node features $X \in \mathbb{R}^{n \times d}$. We call $\mu : \mathbb{R}^{n \times d} \rightarrow \mathbb{R}_{\geq 0}$ a node-similarity measure if it satisfies the following axioms:
%     \begin{enumerate}
%         \item $\exists c \in \mathbb{R}^d$ with $X_u = c$ for all nodes $u \in V \iff \mu(X) = 0$, for $X \in \mathbb{R}^{n \times d}$
%         \item $\mu(X + Y) \leq \mu(X) + \mu(Y)$, for all $X, Y \in \mathbb{R}^{v \times m}$
%     \end{enumerate}
% \end{definition}
% % \todo{introduce similarity and variance measures and figures.}

% \begin{proposition}

% \end{proposition}


\section{More Discussion on GNNs}
\label{app: GNNs}

\subsection{Message-passing Graph Neural Networks (MP-GNNs)}
Let $\mathcal{G}=(A, X)$ denote a graph with $n$ nodes and $m$ edges, where $A \in \{0,1\}^{n\times n}$ is the adjacency matrix, and $X\in \mathbb{R}^{n \times d}$ is the node feature matrix with a node feature dimension of $d$.
Usually, we will transform the concrete adjacency matrix $A$ to the normalized adjacency matrix $\hat{A}$ by the degree matrix.
Define $D=diag(d_1,d_2, \dots, d_n)$ where $d_i$ is the degree of the node $i$.
Then the normalized adjacency matrix $\hat{A}=D^{-\frac{1}{2}}AD^{-\frac{1}{2}}$.
Moreover, many theoretical works simplified the normalized adjacency matrix to be $D^{-1}A$ or $AD^{-1}$ as the raw-normalized or column-normalized stochastic matrix where the sum of every raw (column) is $1$ and every entry is non-negative. 
In this paper, we use $\hat{A}=D^{-\frac{1}{2}}AD^{-\frac{1}{2}}$.

Different GNNs typically share a common propagation mechanism, where node features are aggregated and transformed along the network's topology to a certain depth.
The $k$-th layer propagation can be formalized as
\begin{equation}
        \label{Ana_eq_propagation_1}
	\begin{aligned}
	H_{(k)} = \textbf{PROPAGATE}(X; \mathcal{G}; k) =\bigg\langle\textbf{\textit{Trans}}\Big(\textbf{\textit{Agg}}\big\{\mathcal{G};H_{(k-1)}\big\}\Big)\bigg\rangle_{k},
	\end{aligned}
\end{equation}
with $H_{(0)} = X$ and $H_{(k)}$ is the output after the $k$-layer propagation. 
The notation \(\langle \rangle_{k}\) generally varies from GNN models and denotes the generalized combination operation following \(k\) convolutions. 
%
\(\textbf{\textit{Agg}}\{\mathcal{G}; H_{(k-1)}\}\) refers to aggregating the $(k-1)$-layer output $\textbf{H}^{(k-1)}$ along graph $\mathcal{G}$. 
%
Meanwhile, \(\textbf{\textit{Trans}}(\cdot)\) is the corresponding layer-wise feature transformation which often includes a non-linear activation function (e.g., ReLU) and a layer-specific learnable weight matrix \(W\) for transformation

\subsection{GCN}
To deal with non-Euclidean graph data, GCNs are proposed for direct convolution operation over graph, and have drawn interests from various domains. GCNisfirstly introduced for a spectral perspective~\cite{gcn}, but soon it becomes popular as a general message-passing algorithm in the spatial domain.
In the feature transformation stage, GCN adopts a non-linear activation function (e.g., ReLU) and a layer-specific learnable weight matrix \(W\) for transformation.
The propagation rule of GCN can formulated as follow:
\begin{equation}
    H_{(k)} = ReLU((\hat{A}H_{(k-1)})W_{(k)})
\end{equation}

\subsection{SGC}
SGC~\cite{sgc} simplifies and separates the two stages of GCNs: feature propagation and (non-linear) feature transformation. 
It finds that utilizing only a simple logistic regression after feature propagation (removing the non-linearities), which makes it a linear GCN, can obtain comparable performance to canonical GCNs. 
The propagation rule of GCN can formulated as follow:
\begin{equation}
    H_{(k)} = \hat{A}H_{(k-1)})W_{(k)}=\hat{A}^{k}H_{(0)})W_{(k)}...W_{(1)}
\end{equation}
Moreover, SGC transforms $W_{(k)}...W_{(1)}$ to a general learnable parameter $W$, so the formula of SGC can be this:
\begin{equation}
    H_{(k)} = \hat{A}^{k}H_{(0)})W
\end{equation}





\section{More Background about Signed Graph}
\label{app: signed graph}

\subsection{Signed Graph Propagation}
Classical GNNs~\citep{gcn,sgc,gat,gin} primarily focused on message-passing on unsigned graphs or graphs composed solely of positive edges.
For example, if there exists a edge $\{i,j\}$ or the sign of edge $\{i,j\}$ is positive, the node $x_i$ updates its value by:
\begin{equation}
\label{app_eq: node attractive}
    \hat{x}_i = x_i + \alpha(x_j-x_i) = (1-\alpha) x_i + \alpha x_j, \alpha \in (0,1). 
\end{equation}
Compared to the unsigned graph, a signed graph extends the edges to either positive or negative.
Notably, if the sign of edge $\{i,j\}$ is negative, the node $x_i$ update its value using the following expression:
% \yf{need to motivate this method. eg we observe that existing techniques can be cast as xx}
\begin{equation}
\label{app_eq: node repel}
    \hat{x}_i = x_i -\beta (x_j-x_i) = (1+\beta) x_i -\beta x_j, \beta \in (0,1).
\end{equation}
In words, the positive interaction~\eqref{app_eq: node attractive} 
indicates the attraction while the negative interaction~\eqref{app_eq: node repel} 
indicates that the nodes will repel their neighbors.

More generally, when considering all of the neighbors of node $x_i$, let $N_i^+$ denote the positive neighbor set while $N_i^-$ denote the negative neighbor set, where $N_i^+ \cup N_i^-= N_i$ and $N_i^+ \cap N_i^-= \emptyset$.
The representation of $x_i$ is thus updated by: 
\begin{equation}
\label{app_eq: sign_node}
    \hat{x}_i = (1-\alpha + \beta) x_i + \frac{\alpha}{|N_i^+|}\sum_{j\in N_i^+}x_j
    -\frac{\beta}{|N_i^-|} \sum_{j\in N_i^-}x_j\,.
\end{equation}
In particular, the two parameters $\alpha$ and $\beta$ mark the strength of positive and negative edges, respectively.
Furthermore, the signed propagation rule~\eqref{app_eq: sign_node} from a single node can be generalized  
over the whole graph $\mathcal{G}$ written in the matrix update form as:
\begin{equation}
\label{app_eq: sign_overall}
    \hat{X} = (1-\alpha + \beta) X + \alpha \pgh{\hat{A}^+ }X - \beta\ngh{\hat{A}^- }X,
\end{equation}
where $\hat{A}^+$ is the raw normalized version of the positive adjacency matrix $A^+ \in \{0,1\}^{n \times n}$ and $\hat{A}^-$ is that of the negative adjacency matrix $A^- \in \{0,1\}^{n \times n}$.


% \section{Details of Signed Graph Framework}

\subsection{Definition of negative graph}
\label{app_sec: negative graph}
For further proofs of the theorems and propositions in the paper, we give a more simple and detailed definition in this section.

Let \(D_{G^+} = \text{diag}(deg_1^+, \ldots, deg_n^+)\) and \(D_{G^-} = \text{diag}(deg_1^-, \ldots, deg_n^-)\) be the degree matrices of the positive subgraph and negative subgraph, respectively. 
%
Let \(A_{G^+}\) be the adjacency matrix of the graph \(G^+\) with \([A_{G^+}]_{ij} = 1\) if \(\{i, j\} \in E^+\) and \([A_{G^+}]_{ij} = 0\) otherwise. 
%
The adjacency matrix \(A_{G^-}\) of the negative subgraph \(G^-\) is defined by \([A_{G^-}]_{ij} = -1\)  for \(\{i, j\} \in E^-\) and \([A_{G^-}]_{ij} = 0\) for \(\{i, j\} \not\in E^-\).

The Laplacian plays a central role in the algebraic representation of structural properties of graphs. 
%
% \jq{ours is $(D+I)^{-\frac{1}{2}}(A+I)(D+I)^{-\frac{1}{2}}$, no form like $D-A$.}
%
In the presence of negative edges, more than one definition of Laplacian is possible; see \cite{signed_dynamics_paper_review}. 
The Laplacian of the positive subgraph \(G^+\) is \(L_{G^+} := D_{G^+} - A_{G^+}\), while for the negative subgraph \(G^-\) the following two variants can be used: \(L_{G^-}^o := D_{G^-} - A_{G^-}\) and \(L_{G^-}^r := -D_{G^-} - A_{G^-}\). 
Consequently, we have the following definitions.

{Definition 1.} Given the signed graph \(G\), its opposing Laplacian is defined as
\begin{equation}
L_{G}^o := L_{G^+} + L_{G^-}^o = D_{G^+} + D_{G^-} - A_{G^+} - A_{G^-},
\end{equation}
and its repelling Laplacian is defined as
\begin{equation}
L_{G}^r = L_{G^+} + L_{G^-}^r = D_{G^+} - D_{G^-} - A_{G^+} - A_{G^-}.
\end{equation}


\subsection{Positive / Negative Interaction}
% \jq{our method doesn't have the same coefficient}

Time is slotted at \(t = 0, 1, \ldots\). 
Each node \(i\) holds a state \(x_i(t) \in \Omega\) at time \(t\) and interacts with its neighbors at each time to revise its state. 
The interaction rule is specified by the sign of the links. 
Let \(\alpha, \beta \geq 0\). 
We first focus on a particular link \(\{i, j\} \in E\) and specify for the moment the dynamics along this link isolating all other interactions.

The DeGroot Rule:
\begin{equation}
    x_s(t + 1) = x_s(t) + \alpha(x_{-s}(t) - x_s(t)) = (1 - \alpha)x_s(t) + \alpha x_{-s}(t),
\end{equation}
where \(-s \in \{i, j\} \setminus \{s\}\) with \(\alpha \in (0, 1)\)

The Opposing Rule:
\begin{equation}
    x_s(t + 1) = x_s(t) + \beta(-x_{-s}(t) - x_s(t)) = (1 - \beta)x_s(t) - \beta x_{-s}(t);
\end{equation}
or
The Repelling Rule:
\begin{equation}
    x_s(t + 1) = x_s(t) - \beta(x_{-s}(t) - x_s(t)) = (1 + \beta)x_s(t) - \beta x_{-s}(t).
\end{equation}



\subsection{Deterministic Networks}
% \jq{our method is more like the repelling negative dynamics intuitively.}

The Repelling Negative Dynamics:
\begin{equation}
\label{eq: repell_neg}
\begin{split}
    x_i(t + 1) &= x_i(t) + \alpha \sum_{j \in N_i^+} (x_j(t) - x_i(t)) - \beta \sum_{j \in N_i^-} (x_j(t) - x_i(t)) \\
    &= (1 - \alpha deg_i^+ + \beta deg_i^-)x_i(t) + \alpha \sum_{j \in N_i^+} x_j(t) - \beta \sum_{j \in N_i^-} x_j(t).
\end{split}
\end{equation}

Denote \(\bold{x}(t) = (x_1(t) \ldots x_n(t))^T\). We can now rewrite \ref{eq: repell_neg} in the compact form

\begin{equation}
\label{eq: over_repell}
\bold{x}(t + 1) = M_{G} \bold{x}(t) = (I - \alpha L_{G_+} - \beta L_{G_-}^r)\bold{x}(t).
\end{equation}
Here,
\begin{equation}
    M_G = I - \alpha L_{G^+} - \beta L_{G^-}^r = I - L_{G}^{rw},
\end{equation}
with \(L_{G}^{rw} = \alpha L_{G^+} + \beta L_{G^-}^r\) being the repelling weighted Laplacian of \(G\). 
From Equation \ref{eq: over_repell}, \(M_G \mathbf{1} = \mathbf{1}\) always holds. 
We present the following result, which by itself is merely a straightforward look into the spectrum of the repelling Laplacian \(L_{G}^{rw}\).

Note that our~\eqref{eq: sign_overall} is consistent with Equation~\eqref{eq: repell_neg}, only need to replace the $\alpha$ and $\beta$ with $\frac{\alpha}{deg_i^+}$ and $\frac{\beta}{deg_i^-}$ respectively.


\label{sec: exp detail}



% \section{Theorem of Signed graph }
\section{Analysis of Previous methods via Signed Graph}
\label{app: previous}

\subsection{Discussion of Normalization}
\label{sec: prof of norm}
\paragraph{BatchNorm} 
% across different nodes in each feature dimension. 
% The update rule of BatchNorm on node representation $x_i$ 
% prevents the denominator from becoming zero 
BatchNorm centers the node representations $X$ to zero mean and unit variance  and can be written as BatchNorm($x_i$) \(=\frac{1}{\sqrt{\sigma^2 + \epsilon}}(x_i - \frac{1}{n}\Sigma_{i=1}^n x_i)\), where $ \epsilon > 0$ 
and $\sigma^2$ is the variance of node features.
We rewrite BatchNorm in the signed graph propagation form as follows: 
\begin{equation}
    \label{eq: bn sign}
    \hat{X}= \pgh{\hat{A}}  X \Gamma_d^{-1}  - \ngh{\frac{\mathbb{1}_n \mathbb{1}_n^T}{n}  \hat{A}} X \Gamma_d^{-1} = \pgh{\hat{A}}  \tilde{X}-\ngh{\frac{\mathbb{1}_n \mathbb{1}_n^T}{n}  \hat{A}} \tilde{X}\,,
\end{equation}
where $\Gamma_d = \diag(\sigma_1,\dots,\sigma_d)$ is a diagonal matrix that represents column-wise variance with $\sigma_i^2=\frac{1}{n}\sum_{j=1}^n ((\hat{A} X)_{_{ji}}- \mathbb{1}_n^\top \hat{A} X/n)^2$, and
$\tilde{X}= X \Gamma_d^{-1}$ is a normalized version of $X$.
We can correspond to the positive graph $\pgh{A^+}$ to $\pgh{\hat{A}}$ and the negative graph $\ngh{A^-}$ to $\ngh{\frac{\mathbb{1}_n \mathbb{1}_n^T}{n} \hat{A}}$ in Eq. \eqref{eq: bn sign}.
% Through the repulsion mechanism introduced by the negative edges, BatchNorm can mitigate the problem of oversmoothing as suggested by Theorem~\ref{thm: connected positive graph}.
% \yf{we should discuss some limitations here}

% Consider the update:
% \begin{equation}
%     \label{eq: bn sign}
%     \hat{X}= (\pgh{A}-\ngh{\frac{\mathbb{1}_n \mathbb{1}_n^T}{n} A}) X\,,
% \end{equation}
% After one signed graph propagation, the edge weight changes from $\{0,1\}$ to $\{-\frac{p+q}{2}, 1- \frac{p+q}{2}\}$, so the SB can be expressed as: 
% \begin{equation}
%     SB_{BN}= (1-\frac{p+q}{2})p + \frac{p+q}{2} (1-q) = p + \frac{p+q}{2} (1-p-q).
% \end{equation}

\paragraph{PairNorm} 
We then introduce another method called PairNorm where the only difference between it and BatchNorm is that PairNorm scales all the entries in $X$ using the same number rather than scaling each column by its own variance.
The formulation of PairNorm can be rewritten as follows:
\begin{equation}
    \label{eq: pn sign}
    \hat{X} = \frac{1}{\Gamma}\pgh{\hat{A}}  X   -  \frac{1}{\Gamma} \ngh{\frac{\mathbb{1}_n \mathbb{1}_n^T}{n}  \hat{A}} X
    =\frac{1}{\Gamma}(\pgh{\hat{A}} X-\ngh{\frac{\mathbb{1}_n \mathbb{1}_n^T}{n}  \hat{A}} X) \,,
\end{equation}
where $\Gamma = \|(\hat{A}- \mathbb{1}_n \mathbb{1}_n^T/n)X \|_F/\sqrt{n} $. 
% Comparing~\eqref{eq: pn sign} to~\eqref{eq: bn sign}, 
We observe that PairNorm shares the same positive and negative graphs (up to scale) as BatchNorm.
Another normalization technique, ContraNorm, turns out to extend the negative graph to an adaptive one based on node feature similarities. 
% which reveals that its effectiveness in alleviating oversmoothing is actually attributed to a mechanism very similar to BatchNorm.

% Through the repulsion mechanism introduced by the negative edges, BatchNorm can mitigate the problem of oversmoothing as suggested by Theorem~\ref{thm: connected positive graph}.
% Meanwhile, from \eqref{eq: bn sign} and~\eqref{eq: pn sign}, we see that BatchNorm and PairNorm inject the negative graph  $-(\mathbb{1}_n \mathbb{1}_n^T/n)  \hat{A}$ through a constant transformation of the positive graph derived from the original graph structure $\hat{A}$.



\paragraph{ContraNorm}
ContraNorm is inspired by the uniformity loss from contrastive learning, aiming to alleviate dimensional feature collapse.
For simplicity, we consider the spectral version of ContraNorm
% Proposition 2) 
% \xinyic{better with an exact reference: which Theorem in ContraNorm} 
% without additional regularization and LayerNorm that are used additionally in practice, 
that takes the following form:
\begin{equation}
    \label{eq: contra sign}
    \hat{X} = (1 + \alpha) \pgh{\hat{A}}X- \alpha /\tau \ngh{(X X^{T}) \hat{A} } X \,,
    % = ((1 + \alpha) \pgh{\hat{A}}- \alpha /\tau \ngh{(X X^{T}) \hat{A} }) X
\end{equation}
where $\alpha\in(0,1)$ and $\tau>0$ are hyperparameters.
We can see that $\pgh{\hat{A}}$ is again the positive graph and $\ngh{(X X^T)\hat{A}}$ is the negative graph in the corresponding signed graph propagation.

\begin{proposition}
    Consider the update:
    \begin{equation}
        \hat{X} = \pgh{A}X-\ngh{\frac{\mathbb{1}_n \mathbb{1}_n^T}{n}A}X,
    \end{equation}
    where $A\in \{0,1\}^{n \times n}$ is the adjacency matrix. Define the overall signed graph adjacency matrix $A_s$ as $A-\frac{\mathbb{1}_n \mathbb{1}_n^T}{n}A$. Then we have that the signed graph is (weakly) structurally balanced only if the original graph can be divided into several isolated complete subgraphs. 
\end{proposition}

\paragraph{Proof.} Assume that there is no isolated node and no node has edges with all the other nodes.
    $(A_s)_{i,j}=(A)_{i,j}-\frac{deg_j}{n}$.
    If $(A)_{i,j}=1$, then we have $(A_s)_{i,j}>0$.
    If $(A)_{i,j}=0$, then we have $(A_s)_{i,j}<0$.
    
    If the nodes can be divided into several isolated complete subgraphs, then the nodes set $V=V_1\cup V_2 \dots V_m$, where $|V_i|>1$, $m$ is the number of the isolated complete subgraphs. 
    So only the nodes within the same set have edges, thus relative entries of $A_s>0$, while nodes from different sets do not, thus relative entries of $A_s <0$.
    
    On the other hand, if $A_s$ is (weakly) structurally balanced, then the nodes set can be expressed as $V=V_1\cup V_2 \dots V_k$, where $|V_i|>1$, $k$ is the number of the separated parties in the signed graph. 
    The entry of $A_s$ in the same parties is positive, while between different parties is negative.
    According to $(A_s)_{i,j}=(A)_{i,j}-\frac{deg_j}{n}$, we know that nodes in the same parties are connected in the original graph while not connected in the original graph between different parties.
    So the graph can be divided into several isolated complete subgraphs.

    Overall, the signed graph is (weakly) structurally balanced only if the original graph can be divided into several isolated complete subgraphs, the proof is over. 

The Proposition shows that in order for the structural balance property to hold for the signed graph of normalization, the graph needs to satisfy an unrealistic condition where the edges strictly cluster the nodes.

\paragraph{Discussion of ContraNorm}
% \label{sec: prof of contra}
% \jq{not very sure how to give the detailed expression of the theorem and proof.}
% \begin{proposition}
    Consider the update:
    \begin{equation}
    \label{app_eq: contra theory sign}
        \hat{X} = \pgh{A}X-\ngh{\frac{XX^T}{n}A}X,
    \end{equation}
    Define the overall signed graph adjacency matrix $A_s = A-\frac{XX^T}{n}A$ where $(A_s)_{i,j}=(A)_{i,j}- \frac{1}{n}\Sigma_{k=1}^n x_ix_k^T(A)_{k,j}$ . 
    % Then we said that under the signed propagation \ref{eq: contra theory sign}, the node features will not converge to a constant $C$ with any initial nodes X(0).
% \end{proposition}

% \paragraph{Proof.} 
Assume that the nodes feature is normalized every update, that is $||x_i||_2=1$ for every $i$.

If $(A)_{i,j}=1$, then we have that
\begin{equation}
\begin{aligned}
    (A_s)_{i,j}&=(A)_{i,j}- \frac{1}{n}\Sigma_{k=1}^n x_ix_k^T(A)_{k,j}\\
    &=1-\frac{1}{n}\Sigma_{k=1}^n x_ix_k^T(A)_{k,j}\\
    &>1-\frac{1}{n}\Sigma_{k=1}^n(A)_{k,j}\\
    &=1-\frac{d_j}{n}>0.\\
\end{aligned}
\end{equation}
That means if $(A)_{i,j}=1$, then  $(A_s)_{i,j}>0$.
However, if $(A)_{i,j}=0$, then we have that
\begin{equation}
    \begin{aligned}
        (A_s)_{i,j}&=(A)_{i,j}- \frac{1}{n}\Sigma_{k=1}^n x_ix_k^T(A)_{k,j}\\
        &= -\frac{1}{n}\Sigma_{k=1}^n x_ix_k^T(A)_{k,j}\\
        &= -\frac{1}{n} \Sigma_{k \in N_j}x_ix_k^T.
    \end{aligned}
\end{equation}

Intuitively, if $x_i$ has similar features to $x_j$'s neighbors, then we have that $(A_s)_{i,j}<0$, which means trying to repel nodes with similar representations. 
If $x_i$ has different features to $x_j$'s neighbors, then we have that $(A_s)_{i,j}>0$, which means trying to aggregate nodes with original different representations. 

If graph $G$ is a completed graph, then all entries of $(A_s)>0$, however, when all of the nodes coverage to each other, $\Sigma_{k=1}^n x_ix_k^T(A)_{k,j}=\Sigma_{k=1}^n x_ix_k^T$ will also become bigger.

% Moreover, if all of the node features converge to $C\in \mathbb{R}^d$, the l2-norm of each node feature will be $||C^2||=1$, then we have that:
% \begin{equation}
% (A_s)_{i,j}=  
% \left\{
% \begin{array}{l}
%   1-\frac{1}{n} \Sigma_{k \in N_j}x_ix_k^T, (A)_{i,j}= 1; \\
%  -\frac{1}{n} \Sigma_{k \in N_j}x_ix_k^T, (A)_{i,j}= 0.
% \end{array}
% \right.
% \end{equation}
% Then, according to the Eq.\eqref{app_eq: contra theory sign}, we have that result of the updated node feature $\hat{x}_i$ as following:
% \begin{equation}
% \begin{aligned}
%     \hat{x}_i =& \Sigma_{j=1}^n (A_s)_{i,j} x_j\\
%     =& [\Sigma_{j\in \mathcal{N}_i}(1-\frac{1}{n} \Sigma_{k \in N_j}x_ix_k^T)-\Sigma_{j\notin \mathcal{N}_i}\frac{1}{n} \Sigma_{k \in N_j}x_ix_k^T]x_j\\
%     =& [d_i - \Sigma_{j\in \mathcal{N}_i}\frac{d_j}{n}-\Sigma_{j\notin \mathcal{N}_i}\frac{d_j}{n}]C\\
%     =& [d_i-\frac{2|E|}{n}]C\\
% \end{aligned}
% \end{equation}

% After $k$-th updates, we have that
% \begin{equation}
%     \hat{x}^{(k)}_i= [d_i-\frac{2|E|}{n}]^kC
% \end{equation}
% If there exists one node satisfied $|d_i-\frac{2|E|}{n} |>1$, then we have that $\hat{x}^{(2k)}_i \rightarrow \infty$, which is contradictory to the oversmoothing assumption.
% If all of the nodes satisfy $|d_i-\frac{2|E|}{n} |\leq 1$


\subsection{Discussion of Residual Connection}
\label{app: residual}
The standard residual connection~\citep{dgc,Chen2020SimpleAD} directly combines the previous and the current layer features together. It can be formulated as:
\begin{equation}
    \label{eq: residual sign}
     \hat{X} = (1-\alpha)X  + \alpha \hat{A} X = X + \alpha \pgh{\hat{A}} X -\alpha \ngh{I} X\,.
\end{equation} 
For residual connections, the positive adjacency matrix is $\pgh{\hat{A}}$ and the negative adjacency matrix $\ngh{I}$ in the corresponding signed graph propagation.
\paragraph{APPNP}
We reformulate the method APPNP~\citep{appap} as the signed propagation form of the initial node feature. 
Another propagation process is APPNP~\citep{appap} which can be viewed as a layer-wise graph convolution with a residual connection to the initial transformed
feature matrix $X^{(0)}$, expressed as: 
\begin{equation}
 \hat{X}^{(k+1)} = (1-\alpha)X^{(0)}  + \alpha \hat{A} X ^{(k)}.
\end{equation}
\begin{theorem}
With $\hat{A}^+=\Sigma_{i=0}^{k+1}\alpha^i\hat{A}^i$ and $\hat{A}^-=\alpha \Sigma_{j=0}^{k}\alpha^j\hat{A}^j$, the propagation process of APPNP following the signed graph propagation.
\end{theorem}
\textbf{Proof.}
Easily prove with mathematical induction.

In addition to combining with the last and initial layer features, the last type integrates several intermediate layer features. The established representations are JKNET~\citep{jknet} and DAGNN~\citep{dagnn}.
\paragraph{JKNET}
JKNET is a deep graph neural network which exploits information from neighborhoods of differing locality. 
JKNET selectively combines aggregations from different layers with Concatenation/Max-pooling/Attention at the output, i.e., the representations "jump" to the last layer.
Using attention mechanism for combination at the last layer, the $k+1$-layer propagation result of JKNET can be written as:
\begin{equation}
    \label{eq:jk-net}
    \begin{split}
         X^{(k+1)} &= \alpha_0 X^{(0)}  + \alpha_1  X ^{(1)} + \cdots \alpha_k X^{(k)}\\
        &= \Sigma_{i=0}^k\alpha_i \hat{A}^i X^{(0)}\,,
    \end{split}
\end{equation}
where $\alpha_0, \alpha_1, \cdots, \alpha_{k}$ are the learnable fusion weights with $\Sigma_{i=0}^k\alpha_i=1$.

\paragraph{DAGNN}
Deep Adaptive Graph Neural Networks (DAGNN)~\citep{dagnn} tries to adaptively add all the features from the previous layer to the current layer features with the additional learnable coefficients. 
After decoupling representation transformation and propagation, the propagation mechanism of DAGNN is similar to that of JKNET.
\begin{equation}
    \label{eq:dagnn}
         X^{(k+1)} = \Sigma_{i=0}^k\alpha_i \hat{A}^i H^{(0)}, \,H^{(0)}=f_\theta(X^{(0)})
\end{equation}
$ H^{(0)}=f_\theta(X^{(0)})$ ) is the non-linear feature transformation using an MLP
network, which is conducted before the propagation process and $\alpha_0, \alpha_1, \cdots, \alpha_{k}$ are the learnable fusion weights with $\Sigma_{i=0}^k\alpha_i=1$. 
\begin{theorem}
    With \pgh{$\hat{A}^+=\Sigma_{i=0}^{k-1}\alpha^i\hat{A}^i+\hat{A}^k$} and \ngh{$\hat{A}^-=\Sigma_{j=0}^{k-1}\alpha^j\hat{A}^k$}, the propagation process of JKNET and DAGNN following the signed graph propagation.
\end{theorem}
\textbf{Proof.}
Easily prove with mathematical induction.
% \jq{double check the correctness.}

As for more residual inspired methods~\citep{GCNII,wGCN,ACM-GCN,PDE-GCN}, we select GCNII and wGCN to give a detailed discussion as follows.
\begin{itemize}
    \item As for GCNII~\citep{GCNII}, it is an improved version of APPNP with the learnable coefficients $\alpha_i$ and changes the learnable weight W to $(1-\beta_i)I+\beta_i W$ each layer, so it shares the same positive and negative graph as APPNP.
    \item As for the wGCN~\citep{wGCN}, it incorporates trainable channel-wise weighting factors $\omega$ to learn and mix multiple smoothing and sharpening propagation operators at each layer, same as the init residual combines but change parameters $\alpha$ to be learnable with a more detailed selection strategy.
\end{itemize}


\subsection{Discussion of DropMessage}
For DropMessage~\citep{Fang2022DropMessageUR}, it is a unified way of DropNode, DropEdge and Dropout but with a more flexible mask strategy. We have discussed the DropNode and DropEdge in our paper. DropMessage can be viewed as randomly dropping some dimension of the aggregated node features instead of the whole node or the whole edge. 
We give the unified positive and negative graph of DropMessage in the term of the signed graph.
The propagation of DropMessage can be expressed as $H^{(k)}= AH^{(k-1)}-M_m,$ where if dropping $AH^{(k-1)}_{ij}$, then $M_{ij}=AH^{(k-1)}_{ij}$ else $M_{ij}=0$.







\section{Proof of Theorem~\ref{thm: small nega}}

Now consider the combined theorem. 

\begin{theorem}
\label{app: theorm_positive connected}
    Suppose that the positive edges are connected. Then along Equation \ref{eq: repell_neg} for any \(0 < \alpha < 1/\max_{i \in V} \deg_i^+\), there exists a critical value \(\beta_* \geq 0\) for \(\beta\) such that
    \begin{enumerate}
        \item[(i)] if \(\beta < \beta_*\), then we have \(\lim_{t \to \infty} x_i(t) = \sum_{j=1}^n x_j(0)/n\) for all initial values \(x(0)\);
        \item[(ii)] if \(\beta > \beta_*\), then \(\lim_{t \to \infty} \|x(t)\| = \infty\) for almost all initial values w.r.t. Lebesgue measure.
    \end{enumerate}
\end{theorem}

\paragraph{Proof.}
we change the signed graph update to the equivalent version of \(x_i(t)\) read as:
\[
x_i(t + 1) = x_i(t) + \alpha \sum_{j \in N_i^+} (x_j(t) - x_i(t)) - \beta \sum_{j \in N_i^-} (x_j(t) - x_i(t)).
\]
This can be expressed as:
\begin{equation}
\label{appendix_eq: nege_node}
    x(t + 1) = (1 - \alpha \deg^+ + \beta \deg^-) x_i(t) + \alpha \sum_{j \in N^+} x_j(t) - \beta \sum_{j \in N^-} x_j(t).
\end{equation}


Algorithm \ref{appendix_eq: nege_node} can be written as:
\begin{equation}
\label{appendix_eq: nege_graph}
    x(t + 1) = M_G x(t) = (I - \alpha L_G^+ - \beta L_G^-) x(t).
\end{equation}


Here, \(M_G = I - \alpha L_G^+ - \beta L_G^-\), with \(L_G^+ = \alpha L_C^+ + \beta L_C^-\) being the repelling weighted Laplacian of \(G\), defined in Sec.\ref{app_sec: negative graph}.  
From Eq.\eqref{appendix_eq: nege_graph}, \(M_G \mathbf{1}= \mathbf{1}\) always holds. We present the following result, which by itself is merely a straightforward look into the spectrum of the repelling Laplacian \(L_G^-\).

So theorem \ref{app: theorm_positive connected} can be changed to the following version:

 Suppose \(G^+\) is connected. Then along Eq.\eqref{appendix_eq: nege_graph} for any \(0 < \alpha < 1/\max_{i \in V} \deg_i^+\), there exists a critical value \(\beta > 0\) for \(\beta\) such that:
\begin{enumerate}
    \item[(i)] if \(\beta < \beta_*\), then average consensus is reached in the sense that \(\lim_{t \to \infty} x_i(t) = \frac{1}{n} \sum_{j=1}^n x_j(0)\) for all initial values \(x(0)\);
    \item[(ii)] if \(\beta > \beta_*\), then \(\lim_{t \to \infty} \|x(t)\| = \infty\) for almost all initial values w.r.t. Lebesgue measure.
\end{enumerate}

\textbf{Proof.}
Define \(J = 11^T/n\). Fix \(\alpha \in (0,1/\max_{i \in V} \deg_i^+)\) and consider \(f(\beta) = \lambda_{\max}(I - \alpha L_G^+ - \beta L_G^- - J)\), and \(g(\beta) = \lambda_{\min}(I - \alpha L_G^+ - \beta L_G^- - J)\). The Courant–Fischer Theorem  implies that both \(f(\cdot)\) and \(g(\cdot)\) are continuous and nondecreasing functions over \([0, \infty)\). The matrix \(J\) always commutes with \(I - \alpha L_G^+ - \beta L_G^-\), and 1 is the only nonzero eigenvalue of \(J\). Moreover, the eigenvalue 1 of \(J\) shares a common eigenvector 1 with the eigenvalue 1 of \(I - \alpha L_G^+ - \beta L_G^-\).

Since \(G^+\) is connected, the second smallest eigenvalue of \(L_{G^+}\) is positive. Since \(0 < \alpha < \frac{1}{\max_{i \in V} \deg^+_i}\), there holds \(\lambda_{\min}(I - \alpha L_{G^+}) \geq -1\), again due to the Gershgorin Circle Theorem. Therefore, \(f(0) < 1\), \(g(0) \geq -1\). Noticing \(f(\infty) = \infty > 1\), there exists \(\beta_* > 0\) satisfying \(f(\beta_*) = 1\). We can then verify the following facts:
\begin{itemize}
  \item There hold \(f(\beta) < 1\) and \(g(\beta) > -1\) if \(\beta < \beta_*\). In this case, along Eq.\eqref{appendix_eq: nege_graph} \(\lim_{t \to \infty} (I - J)x(t) = 0\), which in turn implies that \(x(t)\) converges to the eigenspace corresponding to the eigenvalue 1 of \(M_{G}\). This leads to the average consensus statement in (i).
  \item There holds \(f(\beta) \geq 1\) if \(\beta > \beta_*\). 
  In this case, along Eq.\eqref{appendix_eq: nege_graph} \(x(t)\) diverges as long as the initial value \(x(0)\) has a nonzero projection onto the eigenspace corresponding to \(\lambda_{\max}(M_{G})\) of \(M_{G}\). 
  This leads to the almost everywhere divergence statement in (ii).
\end{itemize}
The proof is now complete.

\section{Proof of Theorem \ref{thm: repel_struct}}

\begin{theorem}
let \( A > 0 \) be a constant and define \( \mathcal{F}(z)_c \) by \( \mathcal{F}(z)_c = -c, z < -c \), \( \mathcal{F}(z)_c = z, z \in [-c, c] \), and \( \mathcal{F}(z)_c = c, z > c \). Define the function \( \theta : E \to \mathbb{R} \) so that \( \theta(\{i,j\}) = \alpha \) if \( \{i,j\} \in E^+ \) and \( \theta(\{i,j\}) = -\beta \) if \( \{i,j\} \in E^- \). 
Assume that node \( i \) interacts with node \( j \) at time \( t \) and consider the following node interaction under the signed propagation rules:
\begin{equation}
\label{app_eq: repel dyn}
    x_s(t + 1) = \mathcal{F}(z)_c((1 - \theta)x_s(t) + \theta x_{-s}(t)), \ s \in \{i,j\}.
\end{equation}

let \(\alpha \in (0,1/2)\). Assume that \(G\) is a structurally balanced complete graph under the partition \(V = V_1 \cup V_2\). 
When \(\beta\) is sufficiently large, for almost all initial values \(x(0)\) w.r.t. Lebesgue measure, there exists a binary random variable \(l(x(0))\) taking values in \(\{-c,c\}\) such that
\begin{equation}
    \mathbb{P}\left(\lim_{t \to \infty} x_i(t) = l(x(0)), i \in V_1; \lim_{t \to \infty} x_i(t) = -l(x(0)), i \in V_2 \right) = 1.
\end{equation}
\end{theorem}


\paragraph{Proof.}
The proof is based on the following lemmas.

\begin{lemma}
\label{ap_lemma: 2 bound}
Fix \(\alpha \in (0, 1)\) with \(\alpha \neq \frac{1}{2}\). For the dynamics \ref{app_eq: repel dyn} with the random pair selection process, there exists \(\beta^*(\alpha) > 0\) such that
\[
\mathbb{P}\left(\limsup_{t \to \infty} \max_{i,j \in V} |x_i(t) - x_j(t)| = 2A\right) = 1
\]
for almost all initial beliefs if \(\beta > \beta^*\).
\end{lemma}

\begin{lemma}
\label{ap_lemma: appro }
    Fix $\alpha \in (1/2, 1)$ and $\beta \geq 2/(2\alpha - 1)$. Consider the dynamics \ref{app_eq: repel dyn} with the random pair selection process. Let $G$ be the complete graph with $\kappa(G^+) \geq 2$. Suppose for time $t$ there are $i_1, j_1 \in V$ with $x_{i_1}(t) = -c$ and $x_{j_1}(t) = c$. Then for any $\epsilon \in [0, (2\alpha - 1)c/2\alpha]$ and any $i_* \in V$, the following statements hold:
\begin{enumerate}
    \item[(i)] There exist an integer $Z(\epsilon)$ and a sequence of node pair realizations, $G_{t+s}(\omega)$, for $s = 0, 1, \dots, Z - 1$, under which $x_{i_*}(t + Z)(\omega) \leq -A + \epsilon$.
    \item[(ii)] There exist an integer $Z(\epsilon)$ and a sequence of node pair realizations, $G_{t+s}(\omega)$, for $s = 0, 1, \dots, Z - 1$, under which $x_{i_*}(t + Z)(\omega) \geq A - \epsilon$.
\end{enumerate}
\end{lemma} 


\textbf{Proof.} From our standing assumption, the negative graph $G^-$ contains at least one edge. Let $k_*, m_* \in V$ share a negative link. We assume the two nodes $i_1, j_1 \in V$ labeled in the lemma are different from \(k_*, m_*\), for ease of presentation. We can then analyze all possible sign patterns among the four nodes \(i_1, j_1, k_*, m_*\). We present here just the analysis for the case with
\[
\{i_1, k_*\} \in E^+, \{i_1, m_*\} \in E^+, \{j_1, k_*\} \in E^+, \{j_1, m_*\} \in E^+.
\]
The other cases are indeed simpler and can be studied via similar techniques.

Without loss of generality we let \(x_{m_*}(t) \geq x_{k_*}(t)\). First of all we select \(G_t = \{i_1, k_*\}\) and \(G_{t+1} = \{j_1, m_*\}\). It is then straightforward to verify that
\[
x_{m_*}(t + 2) \geq x_{k_*}(t + 2) + 2\alpha c.
\]
By selecting \(G_{t+2} = \{m_*, k_*\}\) we know from \(\beta \geq \frac{2}{(2\alpha - 1)} > \frac{1}{\alpha}\) that
\[
x_{k_*}(t + 3) = -c, \quad x_{m_*}(t + 3) = c.
\]
There will be two cases:
\begin{itemize}
    \item[(a)] Let \(i_* \in \{m_*, k_*\}\). Noting that \(\kappa(G^+) \geq 2\), there will be a path connecting to \(k_*\) from \(i_*\) without passing through \(m_*\) in \(G^+\). It is then obvious that we can select a finite number \(Z_1\) of links which alternate between \(\{m_*, k_*\}\) and the edges over that path so that \(x_{i_*}(t + 3 + Z_1) \geq -c + \epsilon\). Here \(Z_1\) depends only on \(\alpha\) and \(n\).
    \item[(b)] Let \(i_* \in \{m_*, k_*\}\). We only need to show that we can select pair realizations so that \(x_{m_*}\) can get close to \(-c\), and \(x_{k_*}\) gets close to \(c\) after \(t + 3\). Since \(G^+\) is connected, either \(m_*\) or \(k_*\) has at least one positive neighbor. For the moment assume \(m'\) is a positive neighbor of \(m_*\) and \(k'\) is a positive neighbor of \(k_*\) with \(m' \neq k'\). Then from part (a) we can select \(Z_2\) node pairs so that
    \[
    x_{m_*}(t + 3 + Z_2) \leq -c + \epsilon, \quad x_{k_*}(t + 3 + Z_2) \geq c - \epsilon.
    \]
\end{itemize}
Thus, selecting the negative edge \(\{m_*, k_*\}\) for \(t + 5 + Z_2\) implies \(x_{m_*}(t + 6 + Z_2) = c\) for \(\beta \geq \frac{2}{(2\alpha - 1)}\). The case with \(m' = k'\) can be dealt with by a similar treatment, leading to the same conclusion.

This concludes the proof of the lemma.

In view of Lemmas \ref{ap_lemma: 2 bound} and \ref{ap_lemma: appro }, the desired theorem is a consequence of the second Borel--Cantelli Lemma.

\section{\jq{The relationship of oversmoothing and Theorem~\ref{thm: small nega} and Theorem~\ref{thm: repel_struct}}}
\label{app: oversmoothing of theorem 4.1 and 4.3}

\paragraph{Discussion with other training methods}
While \cite{Peng2024BeyondOU} questions the existence of oversmoothing in trained GNNs, their observations are primarily based on specific experimental settings that may not fully capture the oversmoothing challenge present in the literature. Specifically, the empirical observations in~\cite{Peng2024BeyondOU} are based on 10-layer GCNs, which, while useful for their argument, may not represent the behavior of deeper networks or other GNN architectures. Moreover, Figure 2 in~\cite{Peng2024BeyondOU}  is based on a normalized metric, which might not be the most appropriate. To see this point, suppose one wants to classify two points. In one case, we have 0.01 vs -0.01 and in the other case, we have 100 vs -100. While the normalized distance considered in~\cite{Peng2024BeyondOU} would be the same for these two cases, the latter case has a much larger margin, and it would be thus much easier to learn a classifier.
On the other hand, \cite{Cong2021OnPB} suggests that the trainability of deep GNNs is more of a problem than over-smoothing. However, over-smoothing naturally presents challenges for training deep GNNs, as when oversmoothing happens, gradients vanish across different nodes. Besides, \cite{Cong2021OnPB}compares 3 models GCN, ResGCN and GCNII, proving that GCNII is the best backbone. We have adapted our SBP to GCNII in Table~\ref{tab:gcnii-performance} and the results showed that our SBP outperforms GCNII on average, especially in the middle layers.

\paragraph{Measure on oversmoothing}
There exist a variety of different approaches to quantify over-smoothing in deep GNNs, here we choose the measure based on the Dirichlet energy on graphs~\citep{wu2023demystifying,graph_oversmoothing_survey}.
\begin{equation}
    \epsilon(X(t))=\frac{1}{v}\Sigma_{i\in V}\Sigma_{j \in N_i}||x_i(t)-x_j(t)||_2^2,
\end{equation}

\jq{where $v$ is the number of the nodes, $x_i(t)$ is the node feature of node $i$ at time $t$. $N_i$ represents the neighbor set of node $i$, In the signed graph, it including nodes connected to $i$ by both positive and negative edges.}
Oversmoothing means that when the layers are infinity, all of the node features will converge, that is to say $\lim_{t \to \infty}\epsilon(X(t))\to 0$.

In Theorem~\ref{thm: small nega}, there are 2 cases: 
\begin{itemize}
    \item $if \beta < \beta_*, \text{then we have }\lim_{t \to \infty} x_i(t) = \sum_{j=1}^n x_j(0)/n     \text{ for all initial values }x(0)$
    \item $if \beta > \beta_*, \text{then} \lim_{t \to \infty} \|x(t)\| = \infty \text{ for almost all initial values w.r.t. Lebesgue measure}.$
\end{itemize}
In the first case, all the node features will coverage to the mean of them and therefore $\lim_{t \to \infty}\epsilon(X(t))\to 0$, then oversmoothing happens.
In the second case, the node features will diverge to infinity and thus $\lim_{t \to \infty}\epsilon(X(t))\to 0 \text{ or } \infty$ which is also not what we want. 

Theorem~\ref{thm: small nega} demonstrated that both insufficient repulsion and excessive repulsion caused by the negative graph can hinder performance in signed graph propagation.
From this, we conclude that relying solely on the negative signs is insufficient to alleviate oversmoothing.
Therefore, we propose the \jq{provable} solution: a structurally balanced graph to efficiently alleviate oversmoothing in Theorem~\ref{thm: repel_struct}.
Specifically, we have the following conclusion from the structurally balanced graph in Theorem~\ref{thm: repel_struct}:
\begin{equation}
    \mathbb{P}\left(\lim_{t \to \infty} x_i(t) = l(x(0)), i \in V_1; \lim_{t \to \infty} x_i(t) = -l(x(0)), i \in V_2 \right) = 1.
\end{equation}
Then we have:
\begin{align}
    \lim_{t \to \infty}\epsilon(X(t))&=\lim_{t \to \infty}\frac{1}{v}\Sigma_{i\in V}\Sigma_{j \in N_i}||x_i(t)-x_j(t)||^2_2 \\
    & =\lim_{t \to \infty}\frac{1}{v}\Sigma_{i \in V_1}\Sigma_{j \in N_i}||x_i(t)-x_j(t)||_2^2+ \frac{1}{v}\Sigma_{i \in V_2}\Sigma_{j \in N_i}||x_i(t)-x_j(t)||_2^2 \\
    & =\lim_{t \to \infty}\frac{1}{v}\Sigma_{i\in V_1}\Sigma_{j \in N_i, y_i \neq y_j}||x_i(t)-x_j(t)||_2^2+ \frac{1}{v}\Sigma_{i\in V_2}\Sigma_{j \in N_i, y_i \neq y_j}||x_i(t)-x_j(t)||_2^2 \\
    & =\lim_{t \to \infty}\frac{1}{v}\Sigma_{i\in V_1}\frac{v}{2}\times2c+ \frac{1}{v}\Sigma_{i\in V_2}\frac{v}{2}\times2c \\
    & =\lim_{t \to \infty}\frac{1}{v}(\frac{v}{2}\times \frac{v}{2}\times2c+ \frac{v}{2}\times\frac{v}{2}\times2c) \\
    & =vc\geq 0 
\end{align}
So Theorem~\ref{thm: repel_struct} proves that under certain conditions, structural balance can alleviate oversmoothing even when the layers are infinity.




\section{Extension of Structural Balance}
\label{app:weak-balance}
\begin{figure}
    \centering
    \includegraphics[width=1.1\textwidth]{figures/sb.pdf}
    \caption{Examples of structural balanced (left), weakly structural balanced (middle), and unbalanced signed graphs (right). Here red lines represent positive edges; black dashed lines represent negative edges; gray and blue circles represent nodes from different labels}
    \label{fig: sb}
\end{figure}
To clarify the concept of structural balance, weakly structural balance and unbalance signed graph, we give the examples as shown in Figure~\ref{fig: sb}.
The notion of structural balance can be weakened in the following definition \ref{def: weak struct}.
\begin{definition}
    A signed graph \( G \) is \textbf{weakly structurally balanced} if there is a partition of \( V \) into \( V = V_1 \cup V_2 \cup \ldots \cup V_m \), \( m \geq 2 \) with \( V_1, \ldots, V_m \) being nonempty and mutually disjoint, where any edge between different \( V_i \)'s is negative, and any edge within each \( V_i \) is positive.
    \label{def: weak struct}
\end{definition}

Then we show that when $\mathcal{G}$ is a complete graph, weak structural balance also leads to clustering of node states.
\begin{theorem}[\cite{signed_dynamics_paper_review}, Theorem 10]
\label{thm: weak_repel_struct} 
Assume that node $i$ interacts with node $j$ and $x_i(t)$ represents the value of node $i$ at time t. 
Let $\theta=\alpha$ if the edge $\{i,j\}$ is positive and $\theta=\beta$ if the edge $\{i,j\}$ is negative.
Consider the constrained signed propagation update:
\begin{equation}
\label{eq: weak constrained repel dyn}
    x_i(t + 1) = \mathcal{F}_c((1-\theta) x_i(t)+\theta x_J(t)).
\end{equation}
Let \(\alpha \in (0,1/2)\). 
Assume that \(\mathcal{G}\) is a weakly structurally balanced complete graph under the partition \(V = V_1 \cup V_2 \dots \cup V_m\). 
When \(\beta\) is sufficiently large, for almost all initial values \(x(0)\) w.r.t. Lebesgue measure, there exists m random variable \(l_1(x(0))\), \(l_2(x(0))\), \dots, \(l_m(x(0))\), each of which taking values in \(\{-c,c\}\) such that
\begin{equation}
    \mathbb{P}\left(\lim_{t \to \infty} x_i(t) = l_j(x(0)), i \in V_j, j=1,\dots, m \right) = 1.
\end{equation}
\end{theorem}





\section{Discussion about $\mathcal{SID}$}
\label{app: SID-csbm}

We give the details of CSBM and a more clear formula of $\mathcal{SID}$, $\mathcal{P}$ and $\mathcal{N}$ as suggested in Tabel~\ref{tab: sid} in this section.

\subsection{Definition of CSBM}
\label{app: csbm}
To quantify the structural balance of the mentioned methods, we simplified the graph to $2$-CSBM$(N, p, q, \mu_1, \mu_2, \sigma^2 )$ following~\cite{sbm_xinyi}. 
It consists of two classes $\mathcal{C}_1$ and $\mathcal{C}_2$ of nodes of equal size, in total with $N$ nodes. 
For any two nodes in the graph, if they are from the same class, they are connected by an edge independently with probability $p$, or if they are from different classes, the probability is $q$. For each node $v \in \mathcal{C}_i, i\in\{1,-1\}$, the initial feature $X_v$ is sampled independently from a Gaussian distribution $\mathcal{N}(\mu_i, {\sigma^2})$, where $\mu_i =\mathcal{C}_i, \sigma = I $. 
In this paper, we assign $N=100$ and the feature dimension is $8$.

\subsection{Measure of $\mathcal{SID}$}
\begin{equation}
    \mathcal{P}
=
\frac{1}{|V|} \sum_{v \in V} \hbox{ Number of nodes who have the same label as}~v~\hbox{and the non-positive edge}.
\end{equation}
\begin{equation}
    \mathcal{N}=\frac{1}{|V|} \sum_{v \in V}\hbox{ Number of nodes who have the different label from}~v~\hbox{and the non-negative edge}.
\end{equation}
\begin{equation}
    \mathcal{SB} = \frac{1}{2}(\mathcal{P} + \mathcal{N})
\end{equation}
\begin{figure}
    \centering
    \includegraphics[width=0.65\textwidth]{figures/sb_cg.pdf}
    \caption{Example of structural complete graph. Here red lines represent positive edges; black dashed lines represent negative edges; gray and blue circles represent nodes from different labels}
    \label{fig:sb_cg}
\end{figure}

\subsection{Proof of Proposition~\ref{pro: sid}}
\label{app: prof of prop sid}
\begin{proposition}
\label{app_pro: sid}
For a structural balanced complete graph $\mathcal{G}$, we have $\mathcal{SID}(\mathcal{G})=0$.
\end{proposition}
\paragraph{Proof}
To better understand, we give an example of the structural balance graph as shown in Figure~\ref{fig:sb_cg}.
we can see that for a node $v$, $\mathcal{P}(v)=0$ and  $\mathcal{N}(v)=0$ due to the structural balance complete graph assumption. So $\mathcal{SID}(\mathcal{G})=0$.

\subsection{\jq{More observations of $\mathcal{SID}$}}
\label{app: obs of sid}
Apart from Table~\ref{tab: sid} on CSBM, we further present the Structural Imbalance Degree ($\mathcal{SID}$) for Cora across different methods in Table~\ref{tab: sid of cora}. As the performance of these methods is similar in shallow layers (2), we focus on layer 16 to showcase the results.

\begin{table}[htbp]
\centering
\caption{$\mathcal{SID}$ on Cora datasets. We implement all of the methods on SGC under 100 epochs and the accuracy is the result.}
\label{tab: sid of cora}
\resizebox{\linewidth}{!}{
\begin{tabular}{ccccccc}
\hline
 & label-\ours & feature-\ours & BatchNorm & ContraNorm & Residual & DropEdge \\
\hline
$\mathcal{P}$ & 482.1123 & 482.1123 & 482.1123 & 482.1123 & 482.5137 & 484.2075 \\
$\mathcal{N}$ & 0.7408 & 0.7408 & 0.7408 & 0.7408 & 2221.7305 & 2221.7305 \\
$\mathcal{SID}$ & 241.4265 & 241.4265 & 241.4265 & 241.4265 & 1352.1221 & 1352.9620 \\
Accuracy & 77.43 $\pm$ 1.49 & 77.22 $\pm$ 0.55  & 70.79 $\pm$ 0.00 & 63.35 $\pm$ 0.00 & 40.91 $\pm$ 0.00 & 22.24 $\pm$ 3.04 \\
\hline
\end{tabular}
}
\end{table}

We have two key observations: 1) Methods with higher $\mathcal{SID}$ generally lead to worse accuracy, while those with lower $\mathcal{SID}$ tend to produce better accuracy. 2) $\mathcal{SID}$ is a coarse-grained metric; different methods can yield the same $\mathcal{SID}$ values while their performance varies. These observations can also align with the experiments in cSBM in Table~\ref{tab: sid}.
The observation may stem from the fact that structural balance is an inherent property of graph structure, which is challenging to measure precisely using a numerical metric like $\mathcal{SID}$. Proposition 4.6 in the paper proves that when $\mathcal{SID}=0$, the graph is structurally balanced. 
However, for graphs that are not structurally balanced, the properties remain unclear. 
For future work, we aim to develop a more nuanced metric to quantify the structural balance property of graphs.





\section{Proof of Proposition \ref{pro: sid} and \ref{pro: ours-label}}
\label{app: proof of label-sbp}

\begin{proposition}
% Consider the update via signed graph:
% \begin{equation}
%     \hat{X} =\pgh{A}X- \ngh{A_l} X.
% \end{equation}
Assume that node label classes are balanced $|Y_1| = |Y_2|$
% \xinyic{unclear what "two balanced classifications" means} 
and denote the ratio of labelled nodes as $p$.
Then we have that the signed graph adjacency matrix $A_s= A-A_l$ and $\mathcal{SID}(\mathcal{G})\leq (1-p)\frac{n}{2}$, where $\mathcal{SID}$ decreases with a larger labelling ratio $p$. In particular, when $p=1$ (full supervision), we have $\mathcal{SID}(\mathcal{G})=0$, i.e., a perfectly balanced graph.
Under the constrained signed propagation \eqref{eq: constrained repel dyn}, the nodes from different classes will converge to distinct constants.
\end{proposition}

\paragraph{Proof.}
Without loss of generality, assume that the node feature has been normalized which means that $||x_i||_2=1$ for every $i$.
If $x_i$ and $x_j$ has the same label, then we have that, $(A_s)_{i,j}=(A)_{i,j}+1>1$.
If $x_i$ and $x_j$ has different labels, then we have that $(A_s)_{i,j}=(A)_{i,j}-1\leq0$.

We first prove that $\mathcal{SID}(\mathcal{G},p)\leq(1-p)\frac{n}{2}$ where $n$ is the nodes number and $p$ is the label ratio.
We have that 
\begin{equation}
    \mathcal{P}(v)+\mathcal{N}(v)\leq(1-p)n\, ,
\end{equation}
because for a single node $v$ only the remaining $(1-p)n$ nodes' labels are unknown and therefore their edges may need to change so that
\begin{equation}
\begin{aligned}
    \mathcal{SID}(\mathcal{G}) &= \frac{1}{2n}\sum_{v\in\mathcal{G}}(|\mathcal{P}(v)| + |\mathcal{N}(v)|)\\
    &\leq \frac{1}{2n}\sum_{v\in\mathcal{G}}(1-p)n\\
    &= (1-p)\frac{n}{2}.
\end{aligned}
\end{equation}

We know that when $\mathcal{SID}(\mathcal{G})=0$, then we have that the nodes $V$ set can be divided into $V_1\cup V_1 \dots \cup V_L$ where $L$ is the number of the node classes.
% The node with the class $i$ belongs to the $V_i$ set.
There are only positive edges with the node subset and only negative edges between the node subset.

Since $C=2$, the node set can be divided into $V_1$ and $V_2$, the signed graph is structurally balanced.
According to Theorem~\ref{thm: repel_struct}, we have that the nodes in $V_1$ will converge to the $c$ where $||c||_2=1$ and the nodes in $V_2$ will converge to $-c$.
Thus under Label-\ours propagation, the oversmoothing will only happen within the same label and repel different labels to the boundary.







\label{sec: proof}



\newpage
\appendix
\section{Applicability of SparseTransX for dense graphs} 
\label{A:density}
Even for fully dense graphs, our KGE computations remain highly sparse. This is because our SpMM leverages the incidence matrix for triplets, rather than the graph's adjacency matrix. In the paper, the sparse matrix $A \in \{-1,0,1\}^{M \times (N+R)}$ represents the triplets, where $N$ is the number of entities, $R$ is the number of relations, and $M$ is the number of triplets. This representation remains extremely sparse, as each row contains exactly three non-zero values (or two in the case of the "ht" representation). Hence, the sparsity of this formulation is independent of the graph's structure, ensuring computational efficiency even for dense graphs.

\section{Computational Complexity}
\label{A:complexity}
 For a sparse matrix $A$ with $m \times k$ having $nnz(A)=$ number of non zeros and dense matrix $X$ with $k \times n$ dimension, the computational complexity of the SpMM is $O(nnz(A) \cdot n)$ since there are a total of $nnz(A)$ number of dot products each involving $n$ components. Since our sparse matrix contains exactly three non-zeros in each row, $nnz(A) = 3m$. Therefore, the complexity of SpMM is $O(3m \cdot n)$ or $O(m \cdot n)$, meaning the complexity increases when triplet counts or embedding dimension is increased. Memory access pattern will change when the number of entities is increased and it will affect the runtime, but the algorithmic complexity will not be affected by the number of entities/relations.

\section{Applicability to Non-translational Models}
\label{A:non_trans}
Our paper focused on translational models using sparse operations, but the concept extends broadly to various other knowledge graph embedding (KGE) methods. Neural network-based models, which are inherently matrix-multiplication-based, can be seamlessly integrated into this framework. Additionally, models such as DistMult, ComplEx, and RotatE can be implemented with simple modifications to the SpMM operations. Implementing these KGE models requires modifying the addition and multiplication operators in SpMM, effectively changing the semiring that governs the multiplication.   

In the paper, the sparse matrix $A \in \{-1,0,1\}^{M \times (N+R)}$ represents the triplets, and the dense matrix $E \in \mathbb{R}^{(N+R) \times d}$ represents the embedding matrix, where $N$ is the number of entities, $R$ is the number of relations, and $M$ is the number of triplets. TransE’s score function, defined as $h + r - t$, is computed by multiplying $A$ and $E$ using an SpMM followed by the L2 norm. This operation can be generalized using a semiring-based SpMM model: $Z_{ij} = \bigoplus_{k=1}^{n} (A_{ik} \otimes E_{kj})$

Here, $\oplus$ represents the semiring addition operator, and $\otimes$ represents the semiring multiplication operator. For TransE, these operators correspond to standard arithmetic addition and multiplication, respectively.

\subsection*{DistMult} 
DistMult’s score function has the expression $h \odot r \odot t$. To adapt SpMM for this model, two key adjustments are required: The sparse matrix $A$ stores $+1$ at the positions corresponding to $h_{\text{idx}}$, $t_{\text{idx}}$, and $r_{\text{idx}}$. Both the semiring addition and multiplication operators are set to arithmetic multiplication. These changes enable the use of SpMM for the DistMult score function.

\subsection*{ComplEx} 
ComplEx’s score function has $h \odot r \odot \bar{t}$, where embeddings are stored as complex numbers (e.g., using PyTorch). In this case, the semiring operations are similar to DistMult, but with complex number multiplication replacing real number multiplication.

\subsection*{RotatE} 
RotatE’s score function has $h \odot r - t$. For this model, the semiring requires both arithmetic multiplication and subtraction for $\oplus$. With minor modifications to our SpMM implementation, the semiring addition operator can be adapted to compute $h \odot r - t$.

\subsection*{Support from other libraries}
Many existing libraries, such as GraphBLAS (Kimmerer, Raye, et al., 2024), Ginkgo (Anzt, Hartwig, et al., 2022), and Gunrock (Wang, Yangzihao, et al., 2017), already support custom semirings in SpMM. We can leverage C++ templates to extend support for KGE models with minimal effort.


\begin{figure*}[t]
\centering     %%% not \center
\includegraphics[width=\textwidth]{figures/all-eval.pdf}
\caption{Loss curve for sparse and non-sparse approach. Sparse approach eventually reaches the same loss value with similar Hits@10 test accuracy.}
\label{fig:loss_curve}
\end{figure*}

\section{Model Performance Evaluation and Convergence}
\label{A:eval}
SpTransX follows a slightly different loss curve (see Figure \ref{fig:loss_curve}) and eventually converges with the same loss as other non-sparse implementations such as TorchKGE. We test SpTransX with the WN18 dataset having embedding size 512 (128 for TransR and TransH due to memory limitation) and run 200-1000 epochs. We compute average Hits@10 of 9 runs with different initial seeds and a learning rate scheduler. The results are shown below. We find that Hits@10 is generally comparable to or better than the Hits@10 achieved by TorchKGE.

\begin{table}[h]
\centering
\caption{Average of 9 Hits@10 Accuracy for WN18 dataset}
\begin{tabular}{|c|c|c|}
\hline
\textbf{Model} & \textbf{TorchKGE} & \textbf{SpTransX} \\ \hline
TransE         & 0.79 ± 0.001700   & 0.79 ± 0.002667   \\ \hline
TransR         & 0.29 ± 0.005735   & 0.33 ± 0.006154   \\ \hline
TransH         & 0.76 ± 0.012285   & 0.79 ± 0.001832   \\ \hline
TorusE         & 0.73 ± 0.003258   & 0.73 ± 0.002780   \\ \hline
\end{tabular}
\label{table:perf_eval}
\end{table}

% We also plot the loss curve for different models in Figure \ref{fig:loss_curve}. We observe that the sparse approach follows a similar loss curve and eventually converges to the same final loss.

\section{Distributed SpTransX and Its Applicability to Large KGs}
\label{A:dist}
SpTransX framework includes several features to support distributed KGE training across multi-CPU, multi-GPU, and multi-node setups. Additionally, it incorporates modules for model and dataset streaming to handle massive datasets efficiently. 

Distributed SpTransX relies on PyTorch Distributed Data Parallel (DDP) and Fully Sharded Data Parallel (FSDP) support to distribute sparse computations across multiple GPUs. 

\begin{table}[h]
\centering
\caption{Average Time of 15 Epochs (seconds). Training time of TransE model with Freebase dataset (250M triplets, 77M entities. 74K relations, batch size 393K)  on 32 NVIDIA A100 GPUs. FSDP enables model training with larger embedding when DDP fails.}
\begin{tabular}{|p{2cm}|p{2.5cm}|p{2.5cm}|}
\hline
\textbf{Embedding Size} & \textbf{DDP (Distributed Data Parallel)} & \textbf{FSDP (Fully Sharded Data Parallel)} \\ \hline
16                      & 65.07 ± 1.641                            & 63.35 ± 1.258                               \\ \hline
20                      & Out of Memory                            & 96.44 ± 1.490                               \\ \hline
\end{tabular}
\end{table}

We run an experiment with a large-scale KG to showcase the performance of distributed SpTransX. Freebase (250M triplets, 77M entities. 74K relations, batch size 393K) dataset is trained using the TransE model on 32 NVIDIA A100 GPUs of NERSC using various distributed settings. SpTransX’s Streaming dataset module allows fetching only the necessary batch from the dataset and enables memory-efficient training. FSDP enables model training with larger embedding when DDP fails.

\section{Scaling and Communication Bottlenecks for Large KG Training}
\label{A:scaling}
Communication can be a significant bottleneck in distributed KGE training when using SpMM. However, by leveraging Distributed Data-Parallel (DDP) in PyTorch, we successfully scale distributed SpTransX to 64 NVIDIA A100 GPUs with reasonable efficiency. The training time for the COVID-19 dataset with 60,820 entities, 62 relations, and 1,032,939 triplets is in Table \ref{table:scaling}. 
% \vspace{-.3cm}
\begin{table}[h]
\centering
\caption{Scaling TransE model on COVID-19 dataset}
\begin{tabular}{|c|c|}
\hline
\textbf{Number of GPUs} & \textbf{500 epoch time (seconds)} \\ \hline
4                       & 706.38                            \\ \hline
8                       & 586.03                            \\ \hline
16                      & 340.00                               \\ \hline
32                      & 246.02                            \\ \hline
64                      & 179.95                            \\ \hline
\end{tabular}
\label{table:scaling}
\end{table}
% \vspace{-.2cm}
It indicates that communication is not a bottleneck up to 64 GPUs. If communication becomes a performance bottleneck at larger scales, we plan to explore alternative communication-reducing algorithms, including 2D and 3D matrix distribution techniques, which are known to minimize communication overhead at extreme scales. Additionally, we will incorporate model parallelism alongside data parallelism for large-scale knowledge graphs.

\section{Backpropagation of SpMM}
\label{A:backprop}
 Our main computational kernel is the sparse-dense matrix multiplication (SpMM). The computation of backpropagation of an SpMM w.r.t. the dense matrix is also another SpMM. To see how, let's consider the sparse-dense matrix multiplication $AX = C$ which is part of the training process. As long as the computational graph reduces to a single scaler loss $\mathfrak{L}$, it can be shown that $\frac{\partial C}{\partial X} = A^T$. Here, $X$ is the learnable parameter (embeddings), and $A$ is the sparse matrix. Since $A^T$ is also a sparse matrix and $\frac{\partial \mathfrak{L}}{\partial C}$ is a dense matrix, the computation $\frac{\partial \mathfrak{L}}{\partial X} = \frac{\partial C}{\partial X} \times \frac{\partial \mathfrak{L}}{\partial C} = A^T \times \frac{\partial \mathfrak{L}}{\partial C} $ is an SpMM. This means that both forward and backward propagation of our approach benefit from the efficiency of a high-performance SpMM.

\subsection*{Proof that $\frac{\partial C}{\partial X} = A^T$}
 To see why $\frac{\partial C}{\partial X} = A^T$ is used in the gradient calculation, we can consider the following small matrix multiplication without loss of generality.
\begin{align*}
A &= \begin{bmatrix}
a_1 & a_2 \\
a_3 & a_4
\end{bmatrix} \\ 
 X &= \begin{bmatrix}
x_1 & x_2 \\
x_3 & x_4
\end{bmatrix} \\
 C &=  \begin{bmatrix}
c_1 & c_2 \\
c_3 & c_4
\end{bmatrix}
\end{align*}
Where $C=AX$, thus-
\begin{align*}
c_1&=f(x_1, x_3) \\
c_2&=f(x_2, x_4) \\
c_3&=f(x_1, x_3) \\
c_4&=f(x_2, x_4) \\
\end{align*}
Therefore-
\begin{align*}
\frac{\partial \mathfrak{L}}{\partial x_1} &= \frac{\partial \mathfrak{L}}{\partial c_1} \times \frac{\partial c_1}{\partial x_1} + \frac{\partial \mathfrak{L}}{\partial c_2} \times \frac{\partial c_2}{\partial x_1} + \frac{\partial \mathfrak{L}}{\partial c_3} \times \frac{\partial c_3}{\partial x_1} + \frac{\partial \mathfrak{L}}{\partial c_4} \times \frac{\partial c_4}{\partial x_1}\\
&= \frac{\partial \mathfrak{L}}{\partial c_1} \times \frac{\partial \mathfrak{c_1}}{\partial x_1} + 0 + \frac{\partial \mathfrak{L}}{\partial c_3} \times \frac{\partial \mathfrak{c_3}}{\partial x_1} + 0\\
&= a_1 \times \frac{\partial \mathfrak{L}}{\partial c_1} + a_3 \times \frac{\partial \mathfrak{L}}{\partial c_3}\\
\end{align*}

Similarly-
\begin{align*}
\frac{\partial \mathfrak{L}}{\partial x_2}
&= a_1 \times \frac{\partial \mathfrak{L}}{\partial c_2} + a_3 \times \frac{\partial \mathfrak{L}}{\partial c_4}\\
\frac{\partial \mathfrak{L}}{\partial x_3}
&= a_2 \times \frac{\partial \mathfrak{L}}{\partial c_1} + a_4 \times \frac{\partial \mathfrak{L}}{\partial c_3}\\
\frac{\partial \mathfrak{L}}{\partial x_4}
&= a_2 \times \frac{\partial \mathfrak{L}}{\partial c_2} + a_4 \times \frac{\partial \mathfrak{L}}{\partial c_4}\\
\end{align*}
This can be expressed as a matrix equation in the following manner-
\begin{align*}
\frac{\partial \mathfrak{L}}{\partial X} &= \frac{\partial C}{\partial X} \times \frac{\partial \mathfrak{L}}{\partial C}\\
\implies \begin{bmatrix}
\frac{\partial \mathfrak{L}}{\partial x_1} & \frac{\partial \mathfrak{L}}{\partial x_2} \\
\frac{\partial \mathfrak{L}}{\partial x_3} & \frac{\partial \mathfrak{L}}{\partial x_4}
\end{bmatrix} &= \frac{\partial C}{\partial X} \times \begin{bmatrix}
\frac{\partial \mathfrak{L}}{\partial c_1} & \frac{\partial \mathfrak{L}}{\partial c_2} \\
\frac{\partial \mathfrak{L}}{\partial c_3} & \frac{\partial \mathfrak{L}}{\partial c_4}
\end{bmatrix}
\end{align*}
By comparing the individual partial derivatives computed earlier, we can say-

\begin{align*}
\begin{bmatrix}
\frac{\partial \mathfrak{L}}{\partial x_1} & \frac{\partial \mathfrak{L}}{\partial x_2} \\
\frac{\partial \mathfrak{L}}{\partial x_3} & \frac{\partial \mathfrak{L}}{\partial x_4}
\end{bmatrix} &= \begin{bmatrix}
a_1 & a_3 \\
a_2 & a_4
\end{bmatrix} \times \begin{bmatrix}
\frac{\partial \mathfrak{L}}{\partial c_1} & \frac{\partial \mathfrak{L}}{\partial c_2} \\
\frac{\partial \mathfrak{L}}{\partial c_3} & \frac{\partial \mathfrak{L}}{\partial c_4}
\end{bmatrix}\\
\implies \begin{bmatrix}
\frac{\partial \mathfrak{L}}{\partial x_1} & \frac{\partial \mathfrak{L}}{\partial x_2} \\
\frac{\partial \mathfrak{L}}{\partial x_3} & \frac{\partial \mathfrak{L}}{\partial x_4}
\end{bmatrix} &= A^T \times \begin{bmatrix}
\frac{\partial \mathfrak{L}}{\partial c_1} & \frac{\partial \mathfrak{L}}{\partial c_2} \\
\frac{\partial \mathfrak{L}}{\partial c_3} & \frac{\partial \mathfrak{L}}{\partial c_4}
\end{bmatrix}\\
\implies \frac{\partial \mathfrak{L}}{\partial X} &= A^T \times \frac{\partial \mathfrak{L}}{\partial C}\\
\therefore \frac{\partial C}{\partial X} &= A^T \qed
\end{align*}



\end{document}