\documentclass{article}

% if you need to pass options to natbib, use, e.g.:
%     \PassOptionsToPackage{numbers, compress}{natbib}
% before loading neurips_2024
\PassOptionsToPackage{sort,numbers}{natbib}
% \linespread{0.98}
% ready for submission
% \usepackage{icml2025}
\usepackage[preprint]{neurips_2024}

%%%%% NEW MATH DEFINITIONS %%%%%

% \usepackage{amsmath,amsfonts,bm}
\usepackage{amsmath,amsfonts}

\usepackage{pifont}


\newcommand{\R}{\mathbb{R}}


\def\va{{\mathbf{a}}}
\def\vg{{\mathbf{g}}}

% Sets
\def\sR{\mathbb{R}}
\def\sC{\mathbb{C}}
\def\sZ{\mathbb{Z}}
\def\sN{\mathbb{N}}
\def\sQ{\mathbb{Q}}

\def\sS{\mathcal{S}}



% Vectors
\def\vzero{{\mathbf{0}}}
\def\vone{{\mathbf{1}}}
\def\vmu{{\mathbf{\mu}}}
\def\vtheta{{\mathbf{\theta}}}
\def\va{{\mathbf{a}}}
\def\vb{{\mathbf{b}}}
\def\vc{{\mathbf{c}}}
\def\vd{{\mathbf{d}}}
\def\ve{{\mathbf{e}}}
\def\vf{{\mathbf{f}}}
\def\vg{{\mathbf{g}}}
\def\vh{{\mathbf{h}}}
\def\vi{{\mathbf{i}}}
\def\vj{{\mathbf{j}}}
\def\vk{{\mathbf{k}}}
\def\vl{{\mathbf{l}}}
\def\vm{{\mathbf{m}}}
\def\vn{{\mathbf{n}}}
\def\vo{{\mathbf{o}}}
\def\vp{{\mathbf{p}}}
\def\vq{{\mathbf{q}}}
\def\vr{{\mathbf{r}}}
\def\vs{{\mathbf{s}}}
\def\vt{{\mathbf{t}}}
\def\vu{{\mathbf{u}}}
\def\vv{{\mathbf{v}}}
\def\vw{{\mathbf{w}}}
\def\vx{{\mathbf{x}}}
\def\vy{{\mathbf{y}}}
\def\vz{{\mathbf{z}}}
\def\vzeta{{\mathbf{\zeta}}}

% Matrix
\def\mA{{\mathbf{A}}}
\def\mB{{\mathbf{B}}}
\def\mC{{\mathbf{C}}}
\def\mD{{\mathbf{D}}}
\def\mE{{\mathbf{E}}}
\def\mF{{\mathbf{F}}}
\def\mG{{\mathbf{G}}}
\def\mH{{\mathbf{H}}}
\def\mI{{\mathbf{I}}}
\def\mJ{{\mathbf{J}}}
\def\mK{{\mathbf{K}}}
\def\mL{{\mathbf{L}}}
\def\mM{{\mathbf{M}}}
\def\mN{{\mathbf{N}}}
\def\mO{{\mathbf{O}}}
\def\mP{{\mathbf{P}}}
\def\mQ{{\mathbf{Q}}}
\def\mR{{\mathbf{R}}}
\def\mS{{\mathbf{S}}}
\def\mT{{\mathbf{T}}}
\def\mU{{\mathbf{U}}}
\def\mV{{\mathbf{V}}}
\def\mW{{\mathbf{W}}}
\def\mX{{\mathbf{X}}}
\def\mY{{\mathbf{Y}}}
\def\mZ{{\mathbf{Z}}}
\def\mBeta{{\mathbf{\beta}}}
\def\mPhi{{\mathbf{\Phi}}}
\def\mLambda{{\mathbf{\Lambda}}}
\def\mSigma{{\mathbf{\Sigma}}}


% Expectation
% \def\eE{\mathop{\mathbb{E}}\limits}
\def\eE{\mathbb{E}}

% Probability
\def\pP{\mathbb{P}}

% Tilde
\def\tf{\tilde{f}}
\def\tS{\tilde{S}}
\def\wtF{\widetilde{\mathcal{F}}}
\def\whR{\widehat{R}}
\def\tvx{\tilde{\mathbf{x}}}
\def\ty{\tilde{y}}


\def\defeq{\overset{\textup{def}}{=}}
% \def\defeq{\overset{.}{=}}
\def\defone{\overset{\text{\ding{172}}}{=}}
\def\deftwo{\overset{\text{\ding{173}}}{=}}
\def\leqone{\overset{\text{\ding{172}}}{\leq}}
\def\leqtwo{\overset{\text{\ding{173}}}{\leq}}
\def\leqthree{\overset{\text{\ding{174}}}{\leq}}
\def\leqfour{\overset{\text{\ding{175}}}{\leq}}
\def\eqone{\overset{\text{\ding{172}}}{=}}
\def\eqtwo{\overset{\text{\ding{173}}}{=}}
\def\eqthree{\overset{\text{\ding{174}}}{=}}
\def\eqfour{\overset{\text{\ding{175}}}{=}}
\def\geqfive{\overset{\text{\ding{176}}}{\geq}}

% to compile a preprint version, e.g., for submission to arXiv, add add the
% [preprint] option:
% \usepackage[preprint]{neurips_2024}


% to compile a camera-ready version, add the [final] option, e.g.:
%     \usepackage[final]{neurips_2024}


% to avoid loading the natbib package, add option nonatbib:
%    \usepackage[nonatbib]{neurips_2024}

% \usepackage{tikz}
% \def\checkmark{\tikz\fill[scale=0.4](0,.35) -- (.25,0) -- (1,.7) -- (.25,.15) -- cycle;} 
\usepackage{bbding}
\usepackage{pifont}
\usepackage{wasysym}
\usepackage{utfsym}
\usepackage[utf8]{inputenc} % allow utf-8 input
\usepackage[T1]{fontenc}    % use 8-bit T1 fonts
\usepackage{hyperref}       % hyperlinks
\usepackage{url}            % simple URL typesetting
\usepackage{booktabs}       % professional-quality tables
\usepackage{amsfonts}       % blackboard math symbols
\usepackage{nicefrac}       % compact symbols for 1/2, etc.
\usepackage{microtype}      % microtypography
% \usepackage{xcolor}         % colors

\usepackage{booktabs}
% \usepackage[table]{xcolor} % For cell coloring
\usepackage{colortbl}
\definecolor{best}{RGB}{173,216,230} % Light blue for the best result
\definecolor{secondbest}{RGB}{220,220,220} % Light gray for the second 

\usepackage{graphicx}
\usepackage{subcaption}
\usepackage{amsfonts}
\usepackage{amssymb}
\usepackage{bbold}
\usepackage{amsmath}
\usepackage{xspace} % for name abbrev
\usepackage{array}
\usepackage{subcaption}
\usepackage{multirow} % 表格中的跨行
\usepackage{adjustbox} % 表格的大小调整
\usepackage[textsize=tiny]{todonotes}
\usepackage{wrapfig}
\usepackage{etoc}
\usepackage{enumitem}
\etocdepthtag.toc{mtchapter}
\etocsettagdepth{mtchapter}{subsection}
\etocsettagdepth{mtappendix}{none}
\usepackage[most]{tcolorbox}



%%%%%%%%%%%%%%%%%%%%%%%%%%%%%%%%
% THEOREMS
%%%%%%%%%%%%%%%%%%%%%%%%%%%%%%%%
% \theoremstyle{plain}
\newtheorem{theorem}{Theorem}[section]
\newtheorem{proposition}[theorem]{Proposition}
\newtheorem{lemma}[theorem]{Lemma}
\newtheorem{corollary}[theorem]{Corollary}
% \theoremstyle{definition}
\newtheorem{definition}[theorem]{Definition}
\newtheorem{assumption}[theorem]{Assumption}
% \theoremstyle{remark}
\newtheorem{remark}[theorem]{Remark}


% A Unified Framework for Understanding and Mitigating Feature Oversmoothing
\newcommand{\ours}[0]{\texttt{SBP}\xspace}
\newcommand{\ourst}[0]{\text{SBP}\xspace}	% ours in normal text 
\newcommand{\oursfull}[0]{\textbf{S}tructural \textbf{B}alance \textbf{P}ropagation
}
\newcommand{\theoremfull}[0]{Signed Graph Framework
}
% \textbf{S}igned \textbf{G}raph \textbf{F}ramework
\newcommand{\pgh}[1]{\textcolor{brown}{#1}}
\newcommand{\ngh}[1]{\textcolor{blue}{#1}}
\DeclareMathOperator{\diag}{diag}
\newcommand{\xinyic}[1]{\todo[color=orange!40]{#1}}
\newcommand{\jq}[1]{\textcolor{black}{#1}} %
% \newcommand{\yf}[1]{\todo[color=orange]{#1}}%
\newcommand{\yf}[1]{\textcolor{red}{\textbf{YW}:#1}} %
\definecolor{tkcolor}{RGB}{224,223,255}
\newtcolorbox{takeaways}[2][]{
	width=\columnwidth,
	colback = tkcolor, 
	colframe = tkcolor, 
	boxsep=0pt,left=10pt,right=10pt,top=0pt,bottom=0pt,
	fontupper=\linespread{0.9}\selectfont,
	title=#2,#1}
% The \author macro works with any number of authors. There are two commands
% used to separate the names and addresses of multiple authors: \And and \AND.
%
% Using \And between authors leaves it to LaTeX to determine where to break the
% lines. Using \AND forces a line break at that point. So, if LaTeX puts 3 of 4
% authors names on the first line, and the last on the second line, try using
% \AND instead of \And before the third author name.
\title{ Oversmoothing as Loss of Sign: Towards Structural Balance in Graph Neural Networks}

\author{Jiaqi Wang$^{1}$ \thanks{Equal contribution.} \qquad Xinyi Wu$^2$\footnotemark[1]  \qquad James Cheng$^{1}$\qquad Yifei Wang$^{3}$\\  
$^{1}$ The Chinese University of Hong Kong \qquad $^2$MIT IDSS \& LIDS \qquad $^3$MIT CSAIL\\
\texttt{
\{jqwang23, jcheng\}@cse.cuhk.edu.hk} \qquad 
\texttt{\{xinyiwu,yifei\_w\}@mit.edu}
}
\begin{document}
\maketitle
% \twocolumn[
% \icmltitle{ Oversmoothing as Loss of Sign: Towards Structural Balance \\ in Graph Neural Networks}
% \author{%
%   Jiaqi Wang $^{*1}$, \, Xinyi Wu \thanks{Equal Contribution}\, $^2$, \,James Cheng $^1$, \,Yifei Wang $^2$\thanks{Corresponding Author}\\
%   $^1$The Chinese University of Hong Kong \, $^2$MIT
% }

% It is OKAY to include author information, even for blind
% submissions: the style file will automatically remove it for you
% unless you've provided the [accepted] option to the icml2025
% package.

% List of affiliations: The first argument should be a (short)
% identifier you will use later to specify author affiliations
% Academic affiliations should list Department, University, City, Region, Country
% Industry affiliations should list Company, City, Region, Country

% You can specify symbols, otherwise they are numbered in order.
% Ideally, you should not use this facility. Affiliations will be numbered
% in order of appearance and this is the preferred way.
% \icmlsetsymbol{equal}{*}

% \begin{icmlauthorlist}
% \icmlauthor{Firstname1 Lastname1}{equal,yyy}
% \icmlauthor{Firstname2 Lastname2}{equal,yyy,comp}
% \icmlauthor{Firstname3 Lastname3}{comp}
% \icmlauthor{Firstname4 Lastname4}{sch}
% \icmlauthor{Firstname5 Lastname5}{yyy}
% \icmlauthor{Firstname6 Lastname6}{sch,yyy,comp}
% \icmlauthor{Firstname7 Lastname7}{comp}
% %\icmlauthor{}{sch}
% \icmlauthor{Firstname8 Lastname8}{sch}
% \icmlauthor{Firstname8 Lastname8}{yyy,comp}
% %\icmlauthor{}{sch}
% %\icmlauthor{}{sch}
% \end{icmlauthorlist}

% \icmlaffiliation{yyy}{Department of XXX, University of YYY, Location, Country}
% \icmlaffiliation{comp}{Company Name, Location, Country}
% \icmlaffiliation{sch}{School of ZZZ, Institute of WWW, Location, Country}

% \icmlcorrespondingauthor{Firstname1 Lastname1}{first1.last1@xxx.edu}
% \icmlcorrespondingauthor{Firstname2 Lastname2}{first2.last2@www.uk}

% You may provide any keywords that you
% find helpful for describing your paper; these are used to populate
% the "keywords" metadata in the PDF but will not be shown in the document
% \icmlkeywords{Machine Learning, ICML}

% \vskip 0.3in
% ]

% this must go after the closing bracket ] following \twocolumn[ ...

% This command actually creates the footnote in the first column
% listing the affiliations and the copyright notice.
% The command takes one argument, which is text to display at the start of the footnote.
% The \icmlEqualContribution command is standard text for equal contribution.
% Remove it (just {}) if you do not need this facility.

%\printAffiliationsAndNotice{}  % leave blank if no need to mention equal contribution
% \printAffiliationsAndNotice{\icmlEqualContribution} % otherwise use the standard text.



% \maketitle

\begin{abstract}

% Recent works to jointly reconstruct 3D human and object from a single RGB image, are mostly model-based, that fail to capture the fine details of the clothed human body and object surface. In this paper, we introduce ReCHOR, a novel, model-free, first-method to produce realistic clothed human-object reconstructions from a monocular view. This is extremely challenging due to human-object occlusions, diverse interactions and depth ambiguity, as it needs to infer both 3D spatial awareness and high resolution details. Our core idea is based on estimating neural implicit representations for human and object respectively by an attention-based neural implicit model that attends to pixel-aligned features from both the global human-object image for spatial awareness and  the local separate view of human and object images for high quality details. Additionally, the network is conditioned on semantic features from an initial estimated human-object pose prior and a generative diffusion model that inpaints occluded regions, thus enabling the retrieval of details from them.
% We also propose a synthetic dataset with rendered scenes of diverse, inter-occluded 3D human and object scans, to train our network. We evaluate our method on the synthetic and real world BEHAVE dataset. Our experiments show that our method outperforms the SOTA in achieving realistic clothed human-object reconstructions.
Recent approaches to jointly reconstruct 3D humans and objects from a single RGB image represent 3D shapes with template-based or coarse models, which fail to capture details of loose clothing on human bodies. In this paper, we introduce a novel implicit approach for jointly reconstructing realistic 3D clothed humans and objects from a monocular view. For the first time, we model both the human and the object with an implicit representation, allowing to capture more realistic details such as clothing. This task is extremely challenging due to human-object occlusions and the lack of 3D information in 2D images, often leading to poor detail reconstruction and depth ambiguity. To address these problems, we propose a novel attention-based neural implicit model that leverages image pixel alignment from both the input human-object image for a global understanding of the human-object scene and from local separate views of the human and object images to improve realism with, for example, clothing details. Additionally, the network is conditioned on semantic features derived from an estimated human-object pose prior, which provides 3D spatial information about the shared space of humans and objects. To handle human occlusion caused by objects, we use a generative diffusion model that inpaints the occluded regions, recovering otherwise lost details. For training and evaluation, we introduce a synthetic dataset featuring rendered scenes of inter-occluded 3D human scans and diverse objects. Extensive evaluation on both synthetic and real-world datasets demonstrates the superior quality of the proposed human-object reconstructions over competitive methods.
\end{abstract}
\label{sec: abstract}
\section{Introduction}\label{sec:intro}

In computational finance, Monte Carlo simulations are used extensively to estimate the expected value of financial payoffs based on the solution of stochastic differential equations (SDEs) which model the evolution of stock prices, interest rates, exchange rates and other quantities \cite{glasserman04}.  Monte Carlo methods are very general and flexible, but for high accuracy it requires generating a large number of costly SDE path approximations, which has motivated research into a number of variance reduction or, equivalently, cost reduction techniques. One such method is
Multilevel Monte Carlo (MLMC), which was proposed in \cite{GILES2008} and was adapted for various applications that are summarised in \cite{Giles_overview17} and successfully combined with other methods such as quasi-Monte Carlo methods. The main idea of MLMC is to approximate the payoff using different time stepping resolutions when numerically solving the underlying SDE and to generate an optimal number of samples on each level, such that the overall computational cost is minimised subject to the desired bound on the variance. %, such that the total computational cost is minimised. 
The computational savings come from the fact that most samples are computed on the coarser levels and hence are less expensive while only a few samples from the finest levels are required \cite{GILES2008}.


Among the directions in which the computational cost 
of MLMC methods could further be reduced, an important avenue is the use of lower precision calculations, especially for the first Monte Carlo levels where the targeted accuracy is relatively low. 
 An overview of the research on mixed precision for the standard Monte Carlo (MC) framework is provided in \cite{ChowMixedPrecisionStandardMC} but only a few references study the potential of low precision computation in the MLMC framework \cite{Rounding_error_oliver}. To the best of our knowledge, the only MLMC framework with customised precision in the literature is \cite{brugger2014mixed}, but they use a uniform precision for all operations on each Monte Carlo level instead of optimising 
 the precision of each intermediary variable to reduce as much as possible the cost of path generation.
 
An important motivation for an MLMC framework with variable precision would be performing the low precision computations on reconfigurable hardware devices such as Field Programmable Gate Arrays (FPGAs). FPGAs contain customizable logic blocks and connectors that make it easy to adapt the digital circuit architecture for a specific application, leading to a highly parallel and optimised implementation. Therefore they are successfully exploited in applications that require high speed and have high computational workload, such as signal processing \cite{woods2008fpga}, and real time applications like high frequency trading \cite{HFT1,HFT2}. That is why a number of previous works in hardware architecture design implemented the MLMC algorithm to price financial options using FPGAs as accelerators, which resulted in improved speed and power efficiency compared to full CPU architectures \cite{Schryver2013AMM}. The paper \cite{lindsey2016domain} also proposed 
a Domain Specific Language to automate the configuration of FPGAs for this specific application. However, only \cite{brugger2014mixed} proposed a heuristic to reduce the precision in calculations.

In addition, all aforementioned works considered that the random number generation (RNG) is performed in single or double precision. Yet in most cases an important portion of the workload in the overall MLMC simulation comes from the RNG and in \cite{brugger2014mixed} this limited the total computational savings.
To reduce the cost of MLMC simulations in particular those based on the Geometric Brownian Motion (GBM), \cite{approximateICDF_Oliver, NestedOliver} have proposed to use approximate random numbers that are generated by applying an approximation of the inverse CDF to uniform random numbers. In \cite{NestedOliver}, the authors proposed a way to integrate these lower precision random variables into a \textit{nested} MLMC framework and completed a numerical analysis to bound the resulting error at each MC level by a product of the time step and the error in the random number approximation. The same authors show in \cite{approximateICDF_Oliver} that using approximate random variables reduces the cost of path generation by a factor 7.


In this paper we propose a nested MLMC framework that combines the use of approximate random normal variables and lower precision calculations to reduce the computational cost of MLMC even further than \cite{brugger2014mixed,NestedOliver}. We illustrate the efficiency of our framework in Matlab, after making several assumptions on the cost of operations and size of the errors that we carefully justify. We focus on the case of GBM and use the approximate RNG methods presented in \cite{approximateICDF_Oliver} as well as a new slightly modified method that combines CDF inversion and the central limit theorem. To choose the precision of the variables in the low precision path generation, we introduce a novel method to optimise the bit-widths. This optimisation is performed before the main path generation loop is executed and is based on a linear model of the payoff error  
due to rounding when computing in low precision. The error model relies on algorithmic differentiation in a similar manner to \cite{unifying-bwoptim,bitwidth-AD,ADAPT}. The bit-width optimisation procedure can be performed off-line, so this stage can be excluded from the on-line time complexity of our framework. The user specified desired accuracy is then enforced by calculating on-line the number of samples that need to be generated.

In terms of hardware design, we suggest implementing the low precision path generation on FPGAs and the full-precision ones on a CPU or GPU. 
The FPGA offers enough flexibility to define a separate bit-width for every variable in the low precision path generation, and can be reconfigured periodically to update the bit-widths when the market parameters have changed considerably. 


The paper is organized as follows : \Cref{sec:MLMC} introduces MLMC and nested MLMC to make clear the estimator that is implemented in our framework. Then in \Cref{sec:RNG} we detail the methods that could be used to obtain approximate random normally distributed numbers very cheaply for the low precision path generation. In \Cref{sec:error_model} and \Cref{sec:costModel} we propose an error model and a cost model (resp.) that we then use to formulate the optimisation problem that is solved to obtain the optimal bit-widths of fixed point variables in \Cref{sec:optimisation}. Finally we summarise our results and future directions in \Cref{sec:conclusion}.



\label{sec: introduction}
\section{Background}
\label{sec:background}


\subsection{Preliminaries}

{\color{red}[TODO: LLMs? in-context learning?]}

\subsection{Problem Definition}

{\color{red}[TODO: define the problem of citation intent]}

\label{sec: background}
% sign framework
\section{A Signed Graph Perspective on Existing  Oversmoothing Countermeasures} 
\label{sec: signed pers}
% \jq{notaion, re-define.}


% In this section, we first introduce the notion of signed graphs and how message-passing over signed graphs works. Then we present the theoretical properties of the asymptotic behaviors of signed graph propagation. 
%
% Finally, we leverage these theoretical tools from signed graphs to provide a unified framework for existing anti-oversmoothing techniques. 
% \yf{we leverage signed graph to analyze oversmoothing ,not to address it}

% In this paper, we define the 

%
% The positive neighbor set is $N_i^+$ and the negative neighbor set is $N_i^-$.
% The positive neighbor set is $N_i^+$ and the positive degree $d_i=|N_i^+|$.
% The negative neighbor set is $N_i^-$ and the negative degree $d_i=|N_i^-|$.
% The raw normalized positive and negative adjacency matrix are $\pgh{\hat{A}^+}$ and $\ngh{\hat{A}^-}$, respectively.
%



% \subsection{Interpreting Regularization Techniques via Signed Graph}
% \label{sec: regularization analysis}




% \textbf{Contextual Stochastic Block Models.} 
% We focus on the $2$-CSBM$(N, p, q, \mu_1, \mu_2, \sigma^2 )$ to explain the methods following~\cite{sbm_xinyi}.

In this section, we revisit three popular types of anti-oversmoothing methods and reinterpret them through the lens of signed graph propagation in the form of (\ref{eq: sign_overall}). We find that all of these methods can be attributed to some kind of signed graph design \(\mathcal{G}_s\) by introducing positive and negative edges to the original graph.
% introduce a signed graph perspective to unify popular anti-oversmoothing techniques.
We summarize eight specific methods with their corresponding positive and negative graphs in Table~\ref{tab: framework}.
% \vspace{-1ex}
\subsection{Normalization Techniques}
Normalization operates the node features after each message-passing step by centering them with zero mean and unit variance (up to a scale) with different strategies. 
% of normalization methods 
A few representative methods include BatchNorm~\citep{batchnorm}, PairNorm~\citep{Zhao2020PairNorm}, 
% LayerNorm~\cite{layernorm} 
and ContraNorm~\citep{contranorm}, where PairNorm and ContraNorm were proposed specifically to address the oversmoothing issue in GNNs.
Further details on these methods are provided in Appendix~\ref{app: previous}.
% Specifically, BatchNorm centers the node representations $X$ to have zero mean and unit variance across nodes for each feature, which can be written as BatchNorm($x_i$) \(=\frac{1}{\sqrt{\sigma^2 + \epsilon}}(x_i - \frac{1}{n}\Sigma_{i=1}^n x_i)\) where $ \epsilon > 0$ 
% and $\sigma^2$ is the variance of the feature across all nodes.
% %
% Meanwhile, PairNorm is a normalization technique specifically developed for GNNs to combat oversmoothing, where its only difference from BatchNorm is that PairNorm scales all the entries in $X$ using the same number rather than scaling each column by its own variance. It can be written as PairNorm($x_i$) \(=\frac{s}{\sqrt{\Gamma
% ^2 + \epsilon}}(x_i - \frac{1}{n}\Sigma_{i=1}^n x_i)\) where $\Gamma = \|(\hat{A}- \mathbb{1}_n \mathbb{1}_n^T/n)X \|_F/\sqrt{n} $ and $s$ is a scalar.
% %
% Apart from these two methods, ContraNorm is inspired by the uniformity loss from contrastive learning, aiming to alleviate dimensional feature collapse.
% For simplicity, we consider the spectral version of ContraNorm that takes the following form: ContraNorm($X$) $= (1 + \alpha) X- \alpha /\tau(X X^{T}) X \,$ where $\alpha\in(0,1)$ and $\tau>0$ are hyperparameters.
% Proposition 2) 
% \xinyic{better with an exact reference: which Theorem in ContraNorm} 
% without additional regularization and LayerNorm that are used additionally in practice, 
% \yf{strength/weakness not discussed}
% Besides, it can assigns suitable weights to positive and negative edges by selecting different $\alpha$ and $\tau$.
% \xinyic{what does "selecting different hyperparameters" mean here}
Despite the differences in motivation and implementation, all the three normalization methods can be seen as a signed graph propagation with different designs of the negative graph: 
\begin{theorem}
     BatchNorm, PairNorm and ContraNorm can be interpreted as signed graph propagation defined in (\ref{eq: sign_overall}), sharing the same raw normalized positive adjacency matrix $\pgh{\hat{A}^+=\hat{A}}$ while having different raw normalized negative adjacency matrices transformed from $\hat{A}^+$, that is,  $\ngh{\hat{A}^-=\frac{\mathbb{1}_n \mathbb{1}_n^T}{n}  \hat{A}}$ for BatchNorm and PairNorm, and $\ngh{\hat{A}^-=(X X^{T}) \hat{A} }$ for ContraNorm.
\end{theorem}
The result shows that PairNorm shares the same fixed positive and negative graphs (up to scale) as BatchNorm. In contrast, ContraNorm extends the negative graph to an adaptive one based on similarities in node features. 
% We can see that $\pgh{\hat{A}}$ is again the positive graph and $\ngh{(X X^T)\hat{A}}$ is the negative graph in the corresponding signed graph propagation.
% In particular, the repulsion of ContraNorm is affected by both feature similarities and node degrees. 
% Consider the update:
% \begin{equation}
%     \label{eq: bn sign}
%     \hat{X}= (\pgh{A}-\ngh{\frac{X X^T}{n} A}) X\,,
% \end{equation}
% After one signed graph propagation, the edge weight changes from $\{0,1\}$ to $\{-\frac{p-q}{2}, 1- \frac{p-q}{2}\}$, so the SB can be expressed as: 
% \begin{equation}
%     SB_{CN}= (1-\frac{p-q}{2})p + \frac{p-q}{2} (1-q) = p + \frac{p-q}{2} (1-p-q).
% \end{equation}







\subsection{Augmentation-Based Methods}
% \paragraph{DropEdge}
Node or edge dropping~\citep{dropedge} is another popular type of method to combat oversmoothing.
% and they can also be interpreted as having signed graph propagation. 
In particular, we denote $A_m\in \{0,-1\}^{n\times n}$ where $(A_m)_{i,j}=1$ if the edge $\{i,j\}$ is dropped and otherwise 0. 
Then the signed graph induced by (randomly) dropping edges can be formulated as \(\mathcal{G}_{drop}=\{A,A_m,X\}\). 
% \begin{equation}
    % \label{drop sign}
    % $\hat{X} =\pgh{A} X - \ngh{A_{m}}X$.
    % =(\pgh{A}  -\ngh{A_{m}})X.
% \end{equation}
The negative adjacency matrix $A_m$, while created through random generation, effectively helps alleviate oversmoothing in practice.
% demonstrates that adding negative edges can 

% Consider the update:
% \begin{equation}
%     \label{eq: bn sign}
%     \hat{X}= (\pgh{A}-\ngh{A_m}) X\,,
% \end{equation}
% Assume that in $\ngh{A_m}$, for any two nodes in the graph, if they are from the same class, they are connected by an edge independently with probability $s$, or if they are from different classes, the probability is $t$.
% % After one signed graph propagation, the edge distribution probabilities are $q-t$ from the same label and $p-s$ from the different labels, so 
% The SB can be expressed as: 
% \begin{equation}
%     SB_{ED}= 1 \times (p-s) - 0 \times (1-q) + 1 \times t = p-s+t.
% \end{equation}


  
% Furthermore, residual connection precisely corresponds to signed graph propagation, as described in~\eqref{eq: sign_overall}, particularly when $\alpha=\beta$.

% \begin{equation}
%     \label{eq: appnp}
%     \begin{split}
%         \hat{X}^{(k+1)} &= (1-\alpha)X^{(0)}  + \alpha \hat{A} X ^{(k)} \\
%         &= \pgh{\Sigma_{i=0}^{k+1}\alpha^i\hat{A}^i} X^{(0)} -\ngh{\alpha \Sigma_{j=0}^{k}\alpha^j\hat{A}^j }X^{(0)}\,.
%     \end{split}
% \end{equation}

\subsection{Residual Connections} % 
Besides normalization layers and edge-dropping, residual connections can also be seen through the lens of signed graph propagation. Based on different combinations of layers in this class, we provide analysis for the following three types of residual connections: First, the standard residual connection~\citep{dgc,Chen2020SimpleAD}, which directly combines the previous and the current layer features together.
% It can be formulated as: \( \hat{X} = (1-\alpha)X  + \alpha \hat{A} X  \).
% \begin{equation}
%     \label{eq: residual sign}
% \end{equation} = X + \alpha \pgh{\hat{A}} X -\alpha \ngh{I} X
% For residual connections, the positive adjacency matrix is $\pgh{\hat{A}}$ and the negative adjacency matrix $\ngh{I}$ in the corresponding signed graph propagation.
% \paragraph{APPNP}
% We reformulate the method APPNP~\citep{appap} as the signed propagation form of the initial node feature. 
Another type combines the current layer features together with the initial features, such as APPNP~\citep{appap} or GCNII~\cite{GCNII}.
% \begin{equation}
% , written as: \( \hat{X}^{(k+1)} = (1-\alpha)X^{(0)}  + \alpha \hat{A} X ^{(k)} \) .
% \end{equation}
% \begin{theorem}
% With $\hat{A}^+=\Sigma_{i=0}^{k+1}\alpha^i\hat{A}^i$ and $\hat{A}^-=\alpha \Sigma_{j=0}^{k}\alpha^j\hat{A}^j$, the propagation process of APPNP following the signed graph propagation.
% \end{theorem}
In addition to combining with the previous or the initial layer features, there is a third type of residual connections which integrates intermediate layer features, such as JKNET~\citep{jknet} and DAGNN~\citep{dagnn}.
% \paragraph{\jq{JKNET and DAGNN}}
% JKNET is a deep graph neural network which exploits information from neighborhoods of differing locality. 
% JKNET selectively combines aggregations from different layers through operations such as concatenation or max-pooling at the output, i.e., the representations "jump" to the last layer.
% Using attention mechanism for combination at the last layer, the $k+1$-layer propagation result of JKNET can be written as:
% \begin{equation}
%     \label{eq:jk-net}
%     \begin{split}
%          X^{(k+1)} &= \alpha_0 X^{(0)}  + \alpha_1  X ^{(1)} + \cdots \alpha_k X^{(k)}\\
%         &= \Sigma_{i=0}^k\alpha_i \hat{A}^i X^{(0)}\,,
%     \end{split}
% \end{equation}
% where $\alpha_0, \alpha_1, \cdots, \alpha_{k}$ are the learnable fusion weights with $\Sigma_{i=0}^k\alpha_i=1$.
% Deep Adaptive Graph Neural Networks (DAGNN) tries to adaptively add all the features from the previous layer to the current layer features with additional learnable coefficients. 
More details about these methods can be found in Appendix~\ref{app: residual}.
Formally, we establish the following result that these three types of residual connections can all be seen as signed graph propagation:
% After decoupling representation transformation and propagation, the propagation mechanism of DAGNN is similar to that of JKNET.
% \begin{equation}
%     \label{eq:dagnn}
%          X^{(k+1)} = \Sigma_{i=0}^k\alpha_i \hat{A}^i H^{(0)}, \,H^{(0)}=f_\theta(X^{(0)})
% \end{equation}
% $ H^{(0)}=f_\theta(X^{(0)})$ ) is the non-linear feature transformation using an MLP
% network, which is conducted before the propagation process and $\alpha_0, \alpha_1, \cdots, \alpha_{k}$ are the learnable fusion weights with $\Sigma_{i=0}^k\alpha_i=1$. \jq{double check the correctness.}
\begingroup
\setlength{\abovedisplayskip}{3pt} % 上方间距缩小
\setlength{\belowdisplayskip}{3pt}
\begin{theorem}
% \vspace{-3ex}
With \pgh{$\hat{A}^+=\hat{A}$} and \ngh{$\hat{A}^-=I$}, the standard residual connections follows the signed graph propagation~(\ref{eq: sign_overall}).
With \pgh{$\hat{A}^+=\Sigma_{i=0}^{k+1}\alpha^i\hat{A}^i$} and \ngh{$\hat{A}^-=\alpha \Sigma_{j=0}^{k}\alpha^j\hat{A}^j$}, APPNP follows the signed graph propagation~(\ref{eq: sign_overall}).
    With \pgh{$\hat{A}^+=\Sigma_{i=0}^{k-1}\alpha^i\hat{A}^i+\hat{A}^k$} and \ngh{$\hat{A}^-=\Sigma_{j=0}^{k-1}\alpha^j\hat{A}^k$}, JKNET and DAGNN follows the signed graph propagation~(\ref{eq: sign_overall}).
\end{theorem}
\endgroup
% \vspace{-5pt}

\textbf{Discussion.} In summary, we establish a unifying perspective in which normalization, edge dropping, and residual connections can all be interpreted as instances of signed graph propagation, even though this structure is not explicitly recognized. Notably, for these methods, while their positive adjacency matrices typically reflect the original graph structure, the negative adjacency matrices are often constructed heuristically. As a result, the interaction between signed graph structures and node feature dynamics remains insufficiently understood motivating a systematic theoretical analysis of the asymptotic behaviors of signed graph propagation.
% , thus inspiring us to theoretically and systematically analyze oversmoothing through the signed lens.
% Moreover, these methods create positive and negative adjacency matrices based on the different heuristics, remaining unclear about the complex interplay between the signed graph structure and the resulting node feature dynamics. 
% thus may initially provide some benefits in terms of preventing oversmoothing.
% However, while empirically constructing the signed graph propagation, 
% there is still a lack of theoretical guidance to fully understand the .
%
\label{sec: sign graph}
\section{Causal IL as CMRs}\label{sec:method}

In this section, we demonstrate that performing causal IL in our framework is possible using trajectory histories as instruments. In the next step, we show that the problem can be described as CMRs and propose an effective algorithm to solve it.

The typical target for IL would be the expert policy $\pi_E$ itself. However, since the expert has access to information, namely $u^o_t$, which the imitator does not, the best thing an imitator can do is to learn a history-dependent policy $\pi_h$ that is the closest to the expert. A natural choice is the conditional expectation of $\pi_E(s_t,u^o_t)$ on the history $h_t$:
\begin{align}
\pi_h(h_t)\coloneqq \expectE_{\probP(u^o_t\mid h_t)}[\pi_E(s_t,u^o_t)]=\expectE[\pi_E(s_t,u^o_t)\mid h_t],\nonumber
\end{align}
% where $p(u^o_t\mid h_t)$ is a distribution over expert-observable confounders and captures the information about $u^o_t$ can be inferred from the trajectory history. 
because the conditional expectation minimizes the least squares criterion~\citep{hastie01statisticallearning} and $\pi_h$ is the best predictor of $\pi_E$ given $h_t$. In $\pi_h$, the distribution $\probP(u^o_t\mid h_t)$ captures the information about $u^o_t$ that can be inferred from trajectory histories.
\begin{remark}
\emph{Learning $\pi_h$ is not trivial. Policies learnt naively using behaviour cloning (i.e., $\expectE[a_t\mid h_t]$) fail to match $\pi_E$. In view of~\cref{eq:action}, we have that
\begin{align} 
\expectE[a_t\mid h_t]&=\expectE[\pi_E(s_t,u^o_t) \mid h_{t}]+\expectE[u^\epsilon_t\mid h_{t}]\nonumber\\
&=\pi_h(h_t)+\expectE[u^\epsilon_t\mid h_{t}],\label{eq:history_policy}
\end{align}
where $\expectE[u^\epsilon_t\mid h_{t}]\neq 0$ due to the spurious correlation between $u^\epsilon_t$ and the trajectory history $h_t$. As a result, $\expectE[a_t\mid h_t]$ becomes biased, which can lead to arbitrarily worse performance compared to $\pi_E$.   }
\end{remark}

\vspace{-5pt}
\paragraph{Derivation of CMRs.} 
Leveraging the confounding horizon from Assumption~\ref{assump:horizon}, it becomes possible to break the spurious correlation using the independence of $u^\epsilon_t$ and $u^\epsilon_{t-k}$. We propose to use the $k$-step trajectory history $h_{t-k}=(s_{1},a_{1},...,s_{t-k})$ as an instrument for the current state $s_t$. Taking the expectation conditional on $h_{t-k}$ in~\cref{eq:history_policy} yields
\begin{align*}
    \expectE[a_t\mid h_{t-k}] & = \expectE\left[\expectE[a_t\mid h_{t}]\mid h_{t-k}\right] \\ & = \expectE[\pi_h(h_t)\mid h_{t-k}]+\expectE[\expectE[u^\epsilon_t\mid h_{t}]\mid h_{t-k}] \\
    & = \expectE[\pi_h(h_t) \mid h_{t-k}]+\expectE[u^\epsilon_t\mid h_{t-k}]
\end{align*}
where we use the fact that $h_{t-k}$ is $\sigma(h_t)$-measurable because $h_{t-k}\subseteq h_t$. Next, recall that $u^\epsilon_t\indep u^\epsilon_{t-k}$ by Assumption~\ref{assump:horizon}, which implies $u^\epsilon_t\indep h_{t-k}$, so that % Hence, since $\expectE[u^\epsilon_t] = 0$, we obtain
\begin{align}
    \expectE[a_t\mid h_{t-k}] &= \expectE[\pi_h(h_t) \mid h_{t-k}]+\expectE[u^\epsilon_t]\nonumber\\
    &=\expectE[\pi_h(h_t) \mid h_{t-k}].
\end{align}

As a result, the problem of learning $\pi_h$ reduces to solving for $\pi_h$ that satisfies the following identity
\begin{align}
    \expectE[a_t-\pi_h(h_t)\mid h_{t-k}]=0,\label{eq:CMR}
\end{align}
which is a CMR problem as defined in~\cref{sec:cmr}. In this case, both $a_t$ and $h_t$ are observed in the confounded expert demonstrations, and $h_{t-k}$ acts as the instrument. 

To make sure the instrument $h_{t-k}$ is valid, we check that it satisfies the conditions of~\cref{assump:iv}. Firstly, we have checked that $u^\epsilon_t\indep h_{t-k}$. Secondly, the environment and the expert policy are non-trivial, which means $\probP(h_t\mid h_{t-k})$ is not constant in $h_{t-k}$. Finally, $h_{t-k}$ indeed only affects $a_t$ through $s_t$ by the Markovian property. However, the strength of the instrument, which informally represents the correlation between the instrument $h_{t-k}$ and $h_t$, plays an important role in how well we can identify $\pi_h(h_t)$ by solving the CMRs in~\cref{eq:CMR}. In particular, we see that, as the confounding horizon $k$ increases, the correlation between $h_{t-k}$ and $h_t$ weakens and $h_{t-k}$ becomes a weaker instrument. This means that it is less able to identify $\pi_h$ via the CMR in~\cref{eq:CMR} and the final learnt imitator will have poorer performance. This is confirmed theoretically in Proposition~\ref{prop:ill-posed} and experimentally in~\cref{sec:exps}, and we will formalise this notion of instrument strength in~\cref{sec:theory}.


% Note this problem is equivalent to solving an IV regression on~\cref{eq:history_policy}, where $Y=\expectE[a_t\lvert h_t]$, $f(x)=\pi_h(h_t)$, $\epsilon=\expectE[u^\epsilon_t$ and the instrument $Z=h_{t-k}$.




\subsection{Practical Algorithms for Solving the CMRs}

\begin{algorithm}[tb]
   \caption{DML-IL}
   \label{alg:DML-IL}
\begin{algorithmic}[1]
   \STATE {\bfseries input} Dataset $\dataset_E$ of expert demonstrations, Confounding noise horizon $k$
   \STATE Initialize the roll-out model $\hat{M}$ as a Gaussian mixture model\label{algo:roll_out_1}
    \REPEAT
   \STATE Sample $(h_{t},a_t)$ from data $\dataset_E$
   \STATE Fit the roll-out model $(h_t,a_t)\sim\hat{M}(h_{t-k})$ to maximize the log likelihood 
\UNTIL{convergence}\label{algo:roll_out_2}
   \STATE Initialize the expert model $\hat \pi_h$ as a neural network
   \REPEAT
   % \FOR{$k=1$ {\bfseries to} $K$}
   \STATE Sample $h_{t-k}$ from $\dataset_E$
   \STATE Generate $\hat{h}_t$ and $\hat{a}_t$ using the roll-out model $\hat{M}$
   \STATE Update $\hat \pi_h$ to minimise the loss $\ell:= \norm{\hat{a}_t - \hat{\pi}_h (\hat h_t)}_2$
   % \ENDFOR
    \UNTIL{convergence}
    \STATE {\bfseries return} A history-dependent imitator policy $\hat{\pi}_h$
\end{algorithmic}
\end{algorithm}

There are various techniques~\citep{Shao2024,Bennett2019,Xu2020,Dikkala2020} for solving the CMRs $\expectE[a_t\lvert h_{t-k}]=\expectE[\pi_h(h_t) \lvert h_{t-k}]$. Here, the \textit{CMR error} that we aim to minimise is given by 
\begin{align*}
\sqrt{\expectE\big[\expectE[a_t-\hat{\pi}_h(h_t)\lvert h_{t-k}]^2\big]}=\norm{\expectE[a_t-\hat{\pi}_h(h_t)\lvert h_{t-k}]}_{2}.    
\end{align*}
In~\cref{alg:DML-IL}, we introduce DML-IL, an algorithm adapted from the IV regression algorithm DML-IV~\citep{Shao2024}\footnote{DML stands for double machine learning~\citep{Chernozhukov2018Double}, which is a statistical technique to ensure fast convergence rate for two-step regression, as is the case in~\cref{alg:DML-IL}.}, which solves our CMRs by minimising the CMR error. The first part of the algorithm (line 3-7) learns a roll-out model $\hat{M}$ that generates a trajectory $k$ steps ahead given $h_{t-k}$. Then, the roll-out model $\hat{M}$ is used to train the policy model $\hat{\pi}_h$ (line 8-13). $\hat{\pi}_h$ takes the generated trajectory $\hat{h}_t$ from $\hat{M}(h_{t-k})$ as inputs, and minimises the mean squared error to the next action. Using generated trajectories is crucial in breaking the spurious correlation caused by $u^\epsilon_t$ between past states and actions, and using the trajectory history before $h_{t-k}$ allows the imitator to infer information about $u^o_t$.

DML-IL can also be implemented with $K$-fold cross-fitting, where the dataset is partitioned into $K$ folds, with each fold alternately used to train $\hat{\pi}_h$ and the remaining folds to train $\hat{M}$. This ensures unbiased estimation and improves the stability of training. The base IV algorithm DML-IV with $K$-fold cross-fitting is theoretically shown to converge at the rate of $O(N^{-1/2})$~\citep{Shao2024}, where $N$ is the sample size, under regularity conditions. DML-IL with $K$-fold cross-fitting (see~\cref{appendix:dmlil} for details) will thus inherit this convergence rate guarantee. 

Note that~\cref{alg:DML-IL} requires the confounding noise horizon $k$ as input. While the exact value of $k$ can be difficult to obtain in reality, any upper bound $\bar{k}$ of $k$ is sufficient to guarantee the correctness of ~\cref{alg:DML-IL}, since $h_{t-\bar{k}}$ is also a valid instrument. Ideally, we would like a data-driven approach to determine $k$. Unfortunately, it is generally intractable to empirically verify whether $h_{t-k}$ is a valid instrument from a static dataset, especially the unconfounded instrument condition (i.e., $h_{t-k}\indep u^\epsilon_t$). Therefore, we rely on the user to provide a sensible choice of $\bar{k}$ based on the environment that does not substantially overestimate $k$.


\subsection{Theoretical Analysis}\label{sec:theory}

% \begin{align}
% p(u_t\lvert do(a_{t-k+1}),...,do(a_{t-1}),s_{t-k+1},...,s_{t-1})&\propto p(h_t)p_{\mu_0}(s_{t-k+1})\prod_{i=t-k+1}^{t-1} \transitions(s_{i+1}\lvert s_i,a_i,u_i)
% \end{align}

% since $$(u_t\indep a_{(t-k+1)...(t-1)} \lvert s_{(t-k+1)...(t_1)})_{\mathcal{G}_{\underline{a{(t-k+1)...(t-1)}}}}$$
% on the causal graph $\mathcal{G}_{\underline{a{(t-k+1)...(t-1)}}}$ where the arrows going into $a_{(t-k+1)...(t-1)}$ are removed.



In this section, we derive theoretical guarantees for our algorithm, focusing on the imitation gap and its relationship with existing work.


On a high level, in order to bound the imitation gap of the learnt policy $\hat{\pi}_h$, i.e., $J(\pi_E)-J(\hat{\pi}_h)$, we need to control:
\begin{enumerate}
    \item[($i$)] The amount of information about the hidden confounders that can be inferred from trajectory histories;
    \item[($ii$)] The ill-posedness (or identifiability) of the set of CMRs, which intuitively measures the strength of the instrument $h_{t-k}$;
    \item[($iii$)] The disturbance of the confounding noise to the states and actions at test time.
\end{enumerate}
These factors are all determined by the environment and the expert policy. To control ($i$), we measure how much information about $u^o_t$ is captured by the trajectory history $h_t$ by analysing the Total Variation (TV) distance between the distribution of $u^o_t$ and $\expectE[u^o_t\lvert h_t]$ along the trajectories of $\pi_E$. To control ($ii$) and ($iii$), we need to introduce the following two key concepts.

\begin{definition}[The ill-posedness of CMRs~\citep{Dikkala2020,Chen2012}]

Given the derived CMRs in~\cref{eq:CMR}, for a policy $\pi\in\Pi$, $\norm{\pi_E-\pi}_2$ is the root mean squared error to the expert and $\norm{\expectE[a_t-\pi(s_t)\lvert s_{t-k}]}_2$ is the CMR error we aim to minimise. Then, the \emph{ill-posedness} $\ill(\Pi,k)$ of the policy space with confounding noise horizon $k$ is given by
\begin{align*}
    \ill(\Pi,k)=\sup_{\pi\in\Pi} \frac{\norm{\pi_E-\pi}_{2}}{\norm{\expectE[a_t-\pi(h_t)\lvert h_{t-k}]}_{2}}.
\end{align*}
\end{definition}
The ill-posedness $\ill(\Pi,k)$ measures the strength of the instrument where a higher $\ill(\Pi,k)$ indicates a weaker instrument. It bounds the ratio between the learning error of the imitator following our CMR objective and its $L_2$ error to the expert policy. 

As discussed previously, intuitively, the strength of the instrument would decrease as the confounding horizon $k$ increases. This is in fact true and is confirmed by the following proposition. The proof is deferred to~\cref{appendix:prop}. 
\begin{proposition}\label{prop:ill-posed}
The ill-posedness $\ill(\Pi,k)$ is monotonically increasing as the confounded horizon $k$ increases.
\end{proposition}

Next, we introduce the notion of c-TV stability.
\begin{definition}[c-total variation stability~\citep{Bassily2021,Swamy2022_temporal}]
Let $P(X)$ be the distribution of a random variable $X:\Omega\rightarrow \mathcal{X}$. $P(X)$ is c-TV stable if for $a_1,a_2\in \mathcal{X}$ and $\Delta>0$,
\begin{align*}
\norm{a_1-a_2}\leq\Delta \implies \delta_{TV}(a_1+X,a_2+X)\leq c\Delta.
\end{align*}
where $\norm{\cdot}$ is some norm defined on $\mathcal{X}$ and $\delta_{TV}$ is the total variation distance.
\end{definition}
A wide range of distributions are c-TV stable. For example, standard normal distributions are $\frac{1}{2}$-TV stable. We apply this notion to the distribution over $u^\epsilon_t$ to bound the disturbance it induces in the trajectory and the expected return.

With the notion of ill-posedness and c-TV stability, we can now analyse and upper bound the imitation gap $J(\pi_E)-J(\hat{\pi}_h)$ by controlling the three components $(i)-(iii)$ discussed above. 
% We present the main result for this paper, where t
The full proof is deferred to~\cref{appendix:gap}.

\begin{theorem}[Imitation Gap Bound]\label{thm:gap}
Let $\hat{\pi}_h$ be the learnt policy with CMR error $\epsilon$ and let $\ill(\Pi,k)$ be the ill-posedness of the problem. Assume that $\delta_{TV}(u^o_t,\expectE_{\pi_E}[u^o_t\lvert h_t])\leq\delta$ for $\delta\in\realNumber^+$, $P(u^\epsilon_t)$ is c-TV stable and $\pi_E$ is deterministic. Then, the imitation gap is upper bounded by 
\begin{align*}
    J(\pi_E)-J(\hat{\pi}_h)\leq T^2\big(c\epsilon\ill(\Pi,k)+2\delta\big)=\mathcal{O}\big(T^2(\delta+\epsilon)\big).
\end{align*}
\end{theorem}
This upper bound scales at the rate of $T^2$, which aligns with the expected behaviour of imitation learning without an interactive expert~\citep{Ross2010}.
Next, we show that the upper bounds on the imitation gap from prior work~\citep{Swamy2022_temporal, Swamy2022} are special cases of
% of  subsumed by the unifying causal IL framework introduced in Section~\ref{sec:setting} are special cases of 
Theorem~\ref{thm:gap}. The proofs are deferred to~\cref{appendix:corollaries}.
\begin{corollary}\label{corollary:noUo}
In the special case that $u^o_t = 0$, i.e., there are no expert-observable confounders, or $u^o_t=\expectE_{\pi_E}[u^o_t\lvert h_t]$, i.e., $u^o_t$ is $\sigma(h_t)$ measurable (all information about $u^o_t$ is contained in the history), the imitation gap is upper bounded by
\begin{align*}
    J(\pi_E)-J(\hat{\pi}_h)\leq T^2\big(c\epsilon\ill(\Pi,k)\big)=\mathcal{O}\big(T^2\epsilon\big),
\end{align*}
which coincides with Theorem 5.1 of~\citet{Swamy2022_temporal}.
\end{corollary}

When there are no hidden confounders, i.e, $u^\epsilon_t=0$, our framework is reduced to that of~\citet{Swamy2022}. However, \citet{Swamy2022} provided an abstract bound that directly uses the supremum of key components in the imitation gap over all possible Q functions to bound the imitation gap. We further extend and concretise the bound using the learning error $\epsilon$ and the TV distance bound $\delta$ instead of relying on the suprema.


\begin{corollary}\label{corollary:unconfounded}
In the special case that $u^\epsilon_t=0$, if the learnt policy has optimisation error $\epsilon$,  the imitation gap is upper bounded by
\begin{align*}
    J(\pi_E)-J(\hat{\pi}_h)\leq T^2\left(\frac{2}{\sqrt{\dim(A)}}\epsilon+2\delta \right),
\end{align*}
which is a concrete bound that extends the abstract bound in Theorem 5.4 of~\cite{Swamy2022}.
\end{corollary}

\begin{remark}
\emph{If both $u^\epsilon_t$ and $u^o_t$ are zero, we then recover the classic setting of IL without confounders~\citep{Ross2010}, and the imitation gap bound is $T^2\epsilon$, where $\epsilon$ is the optimisation error of the algorithm.}
\end{remark}
\label{sec: method}

% \vspace{-2ex}
\section{Experiments}
\label{sec: exp_main}


% \jq{almost rewrite}
In this section, we conduct a comprehensive evaluation of
\ours on various benchmark datasets, including both
homophilic and heterophilic graphs. We aim to answer the following three
key research questions: \textbf{RQ1} How does \ours perform in node classification tasks? \textbf{RQ2} How effectively does \ours mitigate oversmoothing? \textbf{RQ3} How sensitive, robust, and scalable is \ours?   

% In this section, we first verify our theoretical insights on the synthetic datasets.
% Then we demonstrate the effectiveness of \ours in three widely used datasets, and then extend the experimental evaluation to one large-scale dataset and two heterophilous datasets. 
% We apply Label-\ours and Feature-\ours in both linear SGC and non-linear GCN. 




% % \textbf{Contextual Stochastic Block Models.} 
% % Following~\cite{sbm_xinyi}, We focus on the CSBM$(N, p, q, \mu_1, \mu_2, \sigma^2 )$.
% % It consists of two classes $\mathcal{C}_1$ and $\mathcal{C}_2$ of nodes of equal size, in total with $N$ nodes. 
% % For any two nodes in the graph, if they are from the same class, they are connected by an edge independently with probability $p$, or if they are from different classes, the probability is $q$. For each node $v \in \mathcal{C}_i, i\in\{1,-1\}$, the initial feature $X_v$ is sampled independently from a Gaussian distribution $\mathcal{N}(\mu_i, {\sigma^2})$, where $\mu_i =\mathcal{C}_i, \sigma = I $. Specially, let $N=200$, $p=0.092$, $q=0.046$, $d=8$.

% \paragraph{Setup}  
% We apply our methods to node classification on the synthetic dataset CSBM$(n, p, q, \mu_1, \mu_2, \sigma^2 )$.
% Specially, let $n=200$, $p=0.092$, $q=0.046$, $d=8$, $\mu_1=1$, $\mu_2=-1$ and $\sigma^2=I$ following~\cite{sbm_xinyi}.
% We used $60\%$, $20\%$ and $20 \%$ random splits for train, valid and test data, respectively.
% Since our theoretical analysis is primarily based on linear propagation, we apply the linear backbone SGC, which removes the non-linear activation function compared to the GCN, to verify our insight.
% We introduce SGC and GCN in Appendix~\ref{app: GNNs} in details.
% % We further apply both SGC and the GCN in the real world dataset.



% \paragraph{Results} 
% The visualization of node features using Label-\ours and Feature-\ours compared to SGC are shown in Figure \ref{fig: sbm overall}. As the number of layers increases, SGC's node features suffer from oversmoothing, causing the two classes to converge and the classification accuracy to drop to $47.50\%$, which is worse than random guessing ($50\%$). However, Label-\ours and Feature-\ours effectively repel nodes from different classes, achieving high accuracy of $80\%$ and $97.5\%$, respectively, even with $300$ layers.


\begin{figure*}[t]
% \captionsetup{font=small}
    \begin{subfigure}{0.69\textwidth}
        \centering
        % \captionsetup{font=small}
        \includegraphics[width=0.99\textwidth]{figures/eval_sgc_layer.pdf}
        \caption{Oversmoothing performance.}
        \label{fig: layer depth}
    \end{subfigure}
    % \quad
    \begin{subfigure}{0.3\textwidth}
        \centering
        % \captionsetup{font=small}
        \includegraphics[width=0.99\textwidth]{figures/eval_train.pdf} % Adjust the path and filename as necessary
        \caption{Training ratio ablation study.}
        \label{fig: train ratio}
    \end{subfigure}
    \caption{Left is the performance comparison of \ours against Normalization GNNs under various model depths where the X-axis has the number of layers, and the Y-axis has node classification accuracy. Right is the ablation study on Label-\ours where the X-axis indicates the ratio of the training node numbers.}
    % \vspace{-0.55cm}
    % \vspace{-0.2in}
\end{figure*}



% \subsection{Real World Benchmark}

% \begin{table}[t]
\centering
% \vspace{-0.15in}
\caption{SGC test accuracy (\%) comparison results. The best results are marked in blue and the second best results are marked in gray on every layer. We run 10 runs and demonstrate the mean $\pm$ std in the table.} % for the seed $0$\~$9$ 
% \resizebox{\textwidth}{!}{
\begin{adjustbox}{width=0.99\textwidth}
\begin{tabular}{lccccccc}
\toprule
 Model             & \#L=2              & \#L=5              & \#L=10             & \#L=20             & \#L=50        & \#L=100    & \#L=300    \\
\midrule
\rowcolor{gray!8}\multicolumn{8}{c}{\textit{Cora}~\citep{cora}}\\
\midrule
% \midrule
% \rowcolor{gray!8}\textit{cora}~\citep{cora}\\
% \midrule
 SGC      & 80.21 {\footnotesize $\pm$ 0.07}& 81.45 {\footnotesize $\pm$ 0.14 }& 81.53 {\footnotesize $\pm$ 0.19 }& 79.53 {\footnotesize $\pm$ 0.14 }& 79.20 {\footnotesize $\pm$ 0.21 }& 76.13 {\footnotesize $\pm$ 0.24 }& 65.64 {\footnotesize $\pm$ 1.15 }\\

 % +LayerNorm~\cite{layernorm}       & 80.07 {\footnotesize $\pm$ 0.22 }& \cellcolor{secondbest}81.60 {\footnotesize $\pm$ 0.22 }& 81.20 {\footnotesize $\pm$ 0.29 }& 79.52 {\footnotesize $\pm$ 0.16  }& 79.21 {\footnotesize $\pm$ 0.23 }& 76.44 {\footnotesize $\pm$ 0.11 }& 68.38 {\footnotesize $\pm$ 0.66}  \\
 +BatchNorm & 77.90 {\footnotesize $\pm$ 0.00 }& 78.02 {\footnotesize $\pm$ 0.04 }& 76.94 {\footnotesize $\pm$ 0.08 }& 75.18 {\footnotesize $\pm$ 0.09 }& 74.54 {\footnotesize $\pm$ 0.05 }& 72.64 {\footnotesize $\pm$ 0.05 }& 63.12 {\footnotesize $\pm$ 0.06} \\
+PairNorm     & 80.30 {\footnotesize $\pm$ 0.05 }& 78.57 {\footnotesize $\pm$ 0.00 }&78.14 {\footnotesize $\pm$ 0.07 }& 76.90 {\footnotesize $\pm$ 0.00 }& 77.49 {\footnotesize $\pm$ 0.03}  & 72.01 {\footnotesize $\pm$ 0.03 }& 40.93 {\footnotesize $\pm$ 0.11} \\
 +ContraNorm      & \cellcolor{best}81.60 {\footnotesize $\pm$ 0.00 }& 80.67 {\footnotesize $\pm$ 0.06 }& 79.11 {\footnotesize $\pm$ 0.03 }& 74.28 {\footnotesize $\pm$ 0.15 }& 69.67 {\footnotesize $\pm$ 1.23 }& 65.58 {\footnotesize $\pm$ 2.11 }&47.21 {\footnotesize $\pm$ 10.80} \\
+DropEdge & 73.58 {\footnotesize $\pm$ 2.76 }& 62.11 {\footnotesize $\pm$ 5.10 }& 39.21 {\footnotesize $\pm$ 7.54 }& 15.07 {\footnotesize $\pm$ 6.22 }& 11.16 {\footnotesize $\pm$ 2.73 }& 11.15 {\footnotesize $\pm$ 2.81 }& 11.15 {\footnotesize $\pm$ 2.81 }\\
 +Residual& 77.81 {\footnotesize $\pm$ 0.03 }& 81.47 {\footnotesize $\pm$ 0.05 }& \cellcolor{best}82.90 {\footnotesize $\pm$ 0.00 }& 79.87 {\footnotesize $\pm$ 0.05 }& 75.64 {\footnotesize $\pm$ 0.05 }& 66.90 {\footnotesize $\pm$ 0.10 }& 25.33 {\footnotesize $\pm$ 0.46}\\
\midrule
 Feature-\ourst &78.10 {\footnotesize $\pm$ 0.11 }& 80.88 {\footnotesize $\pm$ 0.23 }& 80.83 {\footnotesize $\pm$ 0.37 }& \cellcolor{secondbest}82.46 {\footnotesize $\pm$ 0.07 }& \cellcolor{secondbest}80.47 {\footnotesize $\pm$ 0.25 }& \cellcolor{secondbest}80.23 {\footnotesize $\pm$ 0.51 }& \cellcolor{secondbest}77.49 {\footnotesize $\pm$ 0.23 }\\

 Label-\ourst    & \cellcolor{secondbest}81.14 {\footnotesize $\pm$ 0.49 }& \cellcolor{best}82.90 {\footnotesize $\pm$ 0.00}&	\cellcolor{secondbest}82.54 {\footnotesize $\pm$ 0.05}&	\cellcolor{best}82.44 {\footnotesize $\pm$ 0.05}&	\cellcolor{best}82.60 {\footnotesize $\pm$ 0.00}&	\cellcolor{best}81.10 {\footnotesize $\pm$ 0.00}&	\cellcolor{best}74.98 {\footnotesize $\pm$ 0.11 }\\

\midrule
\rowcolor{gray!8}\multicolumn{8}{c}{\textit{CiteSeer}~\citep{citeseer}}\\
\midrule

SGC & 71.88 {\footnotesize $\pm$ 0.27 }& \cellcolor{secondbest}72.55 {\footnotesize $\pm$ 0.25 }& 72.53 {\footnotesize $\pm$ 0.15 }& 72.07 {\footnotesize $\pm$ 0.21 }& 69.83 {\footnotesize $\pm$ 0.20 }& 65.42 {\footnotesize $\pm$ 0.43 }& 54.69 {\footnotesize $\pm$ 0.98} \\

% +LayerNorm~\cite{layernorm} &66.92 {\footnotesize $\pm$ 7.97 }& 65.78 {\footnotesize $\pm$ 2.22 }& 65.82 {\footnotesize $\pm$ 2.39 }& 64.83 {\footnotesize $\pm$ 1.83 }& 62.96 {\footnotesize $\pm$ 2.52 }& 56.67 {\footnotesize $\pm$ 6.55 }& 48.87 {\footnotesize $\pm$ 7.06} \\
+BatchNorm &60.85 {\footnotesize $\pm$ 0.09 }& 60.45 {\footnotesize $\pm$ 0.07 }& 61.74 {\footnotesize $\pm$ 0.27 }& 63.29 {\footnotesize $\pm$ 0.18 }& 63.71 {\footnotesize $\pm$ 0.18 }& 64.28 {\footnotesize $\pm$ 0.27 }& 59.42 {\footnotesize $\pm$ 0.20} \\
 +PairNorm  &70.83 {\footnotesize $\pm$ 0.06 }& 69.68 {\footnotesize $\pm$ 0.32 }& 70.54 {\footnotesize $\pm$ 0.04 }& 69.86 {\footnotesize $\pm$ 0.08 }& 70.51 {\footnotesize $\pm$ 0.07 }& \cellcolor{secondbest}69.86 {\footnotesize $\pm$ 0.06 }& \cellcolor{secondbest}65.22 {\footnotesize $\pm$ 0.16 }\\  
 +ContraNorm  &\cellcolor{best}72.25 {\footnotesize $\pm$ 0.08 }& 71.9 {\footnotesize $\pm$ 0.06 }& 71.52 {\footnotesize $\pm$ 0.04 }& 59.82 {\footnotesize $\pm$ 2.30 }& 52.87 {\footnotesize $\pm$ 1.86 }& 45.93 {\footnotesize $\pm$ 1.40 }& 35.67 {\footnotesize $\pm$ 1.62}\\
+DropEdge & 65.63 {\footnotesize $\pm$ 1.76 }& 51.80 {\footnotesize $\pm$ 4.61 }& 25.36 {\footnotesize $\pm$ 2.54 }& 18.60 {\footnotesize $\pm$ 3.78 }& 16.52 {\footnotesize $\pm$ 3.97 }& 16.49 {\footnotesize $\pm$ 4.03  }& 16.49 {\footnotesize $\pm$ 4.03 }\\
 +Residual & 71.61 {\footnotesize $\pm$ 0.17 }& 72.31 {\footnotesize $\pm$ 0.15  }& \cellcolor{best}72.78 {\footnotesize $\pm$ 0.12 }& \cellcolor{secondbest}72.50 {\footnotesize $\pm$ 0.14 }& \cellcolor{secondbest}71.24 {\footnotesize $\pm$ 0.21 }& 69.85 {\footnotesize $\pm$ 0.22 }& 62.11 {\footnotesize $\pm$ 0.42}\\
\midrule
  Feature-\ourst & 70.63 {\footnotesize $\pm$ 0.52 }& 70.85 {\footnotesize $\pm$ 0.09 }& 70.52 {\footnotesize $\pm$ 0.14 }& 70.76 {\footnotesize $\pm$ 0.22 }& 68.25 {\footnotesize $\pm$ 0.46 }& 67.20 {\footnotesize $\pm$ 1.15 }& 65.12 {\footnotesize $\pm$ 1.95 }\\
 Label-\ourst &\cellcolor{secondbest}72.01 {\footnotesize $\pm$ 0.10 }& \cellcolor{best}72.87 {\footnotesize $\pm$ 0.05 }& \cellcolor{secondbest}72.72 {\footnotesize $\pm$ 0.28 }& \cellcolor{best}73.04 {\footnotesize $\pm$ 0.10 }& \cellcolor{best}72.52 {\footnotesize $\pm$ 0.17 }& \cellcolor{best}72.45 {\footnotesize $\pm$ 0.11 }& \cellcolor{best}70.97 {\footnotesize $\pm$ 0.22 }\\

\midrule
\rowcolor{gray!8}\multicolumn{8}{c}{\textit{PubMed }~\citep{pubmed}}\\
\midrule
SGC &76.99 {\footnotesize $\pm$ 0.38 }& 75.92 {\footnotesize $\pm$ 0.30 }& 76.18 {\footnotesize $\pm$ 0.70 }& 77.13 {\footnotesize $\pm$ 0.34 }&76.09 {\footnotesize $\pm$ 0.43 }& 76.19 {\footnotesize $\pm$ 0.19 }& 70.58 {\footnotesize $\pm$ 0.52 }\\

 % +LayerNorm~\cite{layernorm} &77.67 {\footnotesize $\pm$ 0.40 }& 76.43 {\footnotesize $\pm$ 0.36 }& 76.26 {\footnotesize $\pm$ 0.34 }& 76.27 {\footnotesize $\pm$ 0.41 }& 75.95 {\footnotesize $\pm$ 0.24 }& 74.79 {\footnotesize $\pm$ 0.53 }& 71.77 {\footnotesize $\pm$ 0.45} \\ 
 +BatchNorm &77.15 {\footnotesize $\pm$ 0.09 }& 77.87 {\footnotesize $\pm$ 0.05 }& 78.47 {\footnotesize $\pm$ 0.05 }& 77.90 {\footnotesize $\pm$ 1.10 }& 76.85 {\footnotesize $\pm$ 0.08 }& 74.35 {\footnotesize $\pm$ 0.08 }& 69.61 {\footnotesize $\pm$ 0.08} \\
 +PairNorm &77.69 {\footnotesize $\pm$ 0.26 }& 75.78 {\footnotesize $\pm$ 0.37 }& 75.13 {\footnotesize $\pm$ 0.13 }& 74.75 {\footnotesize $\pm$ 0.33 }& 72.13 {\footnotesize $\pm$ 0.11 }& 69.79 {\footnotesize $\pm$ 0.16 }& 71.75 {\footnotesize $\pm$ 0.51 }\\       
+ContraNorm &\cellcolor{best}79.30 {\footnotesize $\pm$ 0.10 }& 78.69 {\footnotesize $\pm$ 0.07 }& 77.54 {\footnotesize $\pm$ 0.09 }& 73.67 {\footnotesize $\pm$ 0.12 }& 71.37 {\footnotesize $\pm$ 3.15 }& 67.96 {\footnotesize $\pm$ 3.24 }& 65.00 {\footnotesize $\pm$ 4.12 }\\
 +DropEdge & 74.64 {\footnotesize $\pm$ 1.37 }& 69.83 {\footnotesize $\pm$ 3.19 }& 60.28 {\footnotesize $\pm$ 2.70 }& 32.62 {\footnotesize $\pm$ 10.95 }& 33.95 {\footnotesize $\pm$ 10.44 }& 33.95 {\footnotesize $\pm$ 10.44 }& 33.95 {\footnotesize $\pm$ 10.44 }\\
+Residual & 77.40 {\footnotesize $\pm$ 0.06 }& \cellcolor{secondbest}79.30 {\footnotesize $\pm$ 0.10 }& \cellcolor{secondbest}79.83 {\footnotesize $\pm$ 0.09 }& \cellcolor{secondbest}79.44 {\footnotesize $\pm$ 0.09 }& 74.96 {\footnotesize $\pm$ 0.09 }& 71.72 {\footnotesize $\pm$ 0.13 }& 55.57 {\footnotesize $\pm$ 0.21 }\\
\midrule
 Feature-\ourst & 73.99 {\footnotesize $\pm$ 1.44 }& 74.36 {\footnotesize $\pm$ 0.63 }& 75.61 {\footnotesize $\pm$ 0.24 }& 77.09 {\footnotesize $\pm$ 0.35 }& \cellcolor{secondbest}77.41 {\footnotesize $\pm$ 0.21 }& \cellcolor{secondbest}77.10 {\footnotesize $\pm$ 0.36 }& \cellcolor{secondbest}76.87 {\footnotesize $\pm$ 0.49 }\\
 Label-\ourst &\cellcolor{secondbest}78.98 {\footnotesize $\pm$ 0.14 }& \cellcolor{best}80.14 {\footnotesize $\pm$ 0.05 }& \cellcolor{best}80.22 {\footnotesize $\pm$ 0.04 }& \cellcolor{best}80.32 {\footnotesize $\pm$ 0.04 }& \cellcolor{best}80.20 {\footnotesize $\pm$ 0.00 }& \cellcolor{best}79.60 {\footnotesize $\pm$ 0.00 }& \cellcolor{best}73.96 {\footnotesize $\pm$ 0.05} \\
% \midrule

% Arxiv }& SGC \\
%     }& +LayerNorm \\
%     }& PairNorm \\
%     }& ContraNorm \\
%     }& Feature-\ourst \\
%     }& Label-\ourst \\
% % Add more rows as needed
\bottomrule
\end{tabular}
% }

\end{adjustbox}
\label{table: sgc results}
% \vspace{-0.15in}
\end{table}
% \vspace{-0.1in}
% \begin{table}[t]
\vspace{-0.1in}
\centering
\small
\caption{GCN test accuracy (\%) comparison results on heterophilic datasets. The best results are marked in blue and the second best results are marked underline on every layer.
We run 5 runs and demonstrate the mean $\pm$ std in the table.}%for the seed from $0~4$
\begin{adjustbox}{width=0.91\textwidth}
\begin{tabular}{lcccccc}
\toprule
 Model             & \#L=2              & \#L=4              & \#L=8              & \#L=16             & \#L=32             & \#L=64\\

\midrule
\rowcolor{gray!8}\multicolumn{7}{c}{\textit{Chameleon}~\cite{heter_dataset}}\\
\midrule
   GCN & 66.01{\footnotesize$\pm$0.72} & 54.21{\footnotesize$\pm$0.53} & 35.48{\footnotesize$\pm$3.09} & 22.37{\footnotesize$\pm$0.00} & 22.37{\footnotesize$\pm$0.00} & 22.37{\footnotesize$\pm$0.00} \\
    +BatchNorm & \underline{65.83{\footnotesize$\pm$0.58}} & 56.40{\footnotesize$\pm$0.35} & 36.36{\footnotesize$\pm$2.04} & 22.37{\footnotesize$\pm$0.00} & 22.37{\footnotesize$\pm$0.00} & 22.37{\footnotesize$\pm$0.00}\\
    +PairNorm & 66.01{\footnotesize$\pm$0.72} & 54.12{\footnotesize$\pm$0.79} & 36.75{\footnotesize$\pm$0.38} & 22.37{\footnotesize$\pm$0.00} & 22.37{\footnotesize$\pm$0.00} & 22.37{\footnotesize$\pm$0.00}\\
    +ContraNorm & 66.01{\footnotesize$\pm$0.72} & 58.16{\footnotesize$\pm$1.76} & 37.15{\footnotesize$\pm$4.91} & 22.37{\footnotesize$\pm$0.00} & 22.37{\footnotesize$\pm$0.00} & 22.37{\footnotesize$\pm$0.00}\\
    +DropEdge & 62.50{\footnotesize$\pm$0.00} & 53.07{\footnotesize$\pm$1.61} & 32.15{\footnotesize$\pm$1.49} & 21.71{\footnotesize$\pm$0.00} & \underline{27.19{\footnotesize$\pm$1.75}} & \underline{23.68{\footnotesize$\pm$0.00}} \\
    +Residual & 66.01{\footnotesize$\pm$0.72} & \cellcolor{best}62.94{\footnotesize$\pm$0.00} & \underline{57.59{\footnotesize$\pm$2.58}} & \underline{41.27{\footnotesize$\pm$0.32}} & 22.37{\footnotesize$\pm$0.00} & 22.37{\footnotesize$\pm$0.00} \\
\midrule
    Feature-\ourst & 62.98{\footnotesize$\pm$0.75} & \underline{62.89{\footnotesize$\pm$1.29}} & \cellcolor{best}65.35{\footnotesize$\pm$0.00} & \cellcolor{best}62.28{\footnotesize$\pm$0.00} & \cellcolor{best}55.31{\footnotesize$\pm$0.53} & \cellcolor{best}35.13{\footnotesize$\pm$0.93}\\
    Label-\ourst & \cellcolor{best}66.01{\footnotesize$\pm$0.72} & 57.11{\footnotesize$\pm$0.11} & 38.11{\footnotesize$\pm$1.87} & 22.37{\footnotesize$\pm$0.00} & 22.37{\footnotesize$\pm$0.00} & 22.37{\footnotesize$\pm$0.00}\\
\midrule
\rowcolor{gray!8}\multicolumn{7}{c}{\textit{Squirrel}~\cite{heter_dataset}}\\
\midrule
   GCN  & 42.38{\footnotesize$\pm$0.04} & 32.20{\footnotesize$\pm$3.05} & 22.57{\footnotesize$\pm$0.00} & 20.46{\footnotesize$\pm$0.00} & 20.46{\footnotesize$\pm$0.00} & 20.46{\footnotesize$\pm$0.00}\\
    +BatchNorm & 41.77{\footnotesize$\pm$0.35} & 32.37{\footnotesize$\pm$3.46} & 22.67{\footnotesize$\pm$0.00} & 20.46{\footnotesize$\pm$0.00} & 20.46{\footnotesize$\pm$0.00} & 20.46{\footnotesize$\pm$0.00}\\
    +PairNorm & 42.75{\footnotesize$\pm$0.00} & 32.12{\footnotesize$\pm$3.00} & 22.57{\footnotesize$\pm$0.00} & 20.46{\footnotesize$\pm$0.00} & 20.46{\footnotesize$\pm$0.00} & 20.46{\footnotesize$\pm$0.00}\\
    +ContraNorm & \underline{43.78{\footnotesize$\pm$1.08}} & 32.80{\footnotesize$\pm$3.76} & 22.57{\footnotesize$\pm$0.00} & 20.46{\footnotesize$\pm$0.00} & 20.46{\footnotesize$\pm$0.00} & 20.46{\footnotesize$\pm$0.00} \\
    +DropEdge & 40.54{\footnotesize$\pm$0.00} & 22.57{\footnotesize$\pm$0.00} & 22.77{\footnotesize$\pm$2.12} & 22.19{\footnotesize$\pm$0.58} & \underline{22.61{\footnotesize$\pm$1.36}} & 20.46{\footnotesize$\pm$0.00}\\
    +Residual & 41.92{\footnotesize$\pm$0.65} & \underline{42.23{\footnotesize$\pm$0.08}} & \underline{39.15{\footnotesize$\pm$0.07}} & \underline{33.41{\footnotesize$\pm$2.73}} & 20.46{\footnotesize$\pm$0.00} & 20.46{\footnotesize$\pm$0.00} \\
\midrule
    Feature-\ourst& \cellcolor{best}44.48{\footnotesize$\pm$0.00} &\cellcolor{best} 45.01{\footnotesize$\pm$0.72} &\cellcolor{best} 44.03{\footnotesize$\pm$0.63} &\cellcolor{best} 41.42{\footnotesize$\pm$0.78} &\cellcolor{best} 36.79{\footnotesize$\pm$0.00} &\cellcolor{best} 29.20{\footnotesize$\pm$0.00}\\
    Label-\ourst& 43.61{\footnotesize$\pm$0.58} & 32.78{\footnotesize$\pm$3.49} & 22.79{\footnotesize$\pm$0.09} & 20.46{\footnotesize$\pm$0.00} & 20.46{\footnotesize$\pm$0.00} & \underline{20.46{\footnotesize$\pm$0.00}} \\

\bottomrule
\end{tabular}
\end{adjustbox}
\label{table: gcn heter}
\vspace{-0.1in}
\end{table}

% \begin{table}[h]
% \centering
% \caption{GCN test accuracy (\%) comparison results. The best results are marked in blue and the second best results are marked in gray on every layer.}
% \begin{adjustbox}{width=0.99\textwidth}
% \begin{tabular}{lcccccc}
% \toprule
%  Model             & \#L=2              & \#L=4              & \#L=8              & \#L=16             & \#L=32             & \#L=64\\

% \midrule
% \rowcolor{gray!8}\multicolumn{7}{c}{\textit{Cora}~\citep{cora}}\\
% \midrule
%   GCN~\cite{gcn} & \cellcolor{secondbest}80.68 $\pm$ 0.09 & \cellcolor{secondbest}79.69 $\pm$ 0.00 & 74.32 $\pm$ 0.00 & 30.95 $\pm$ 0.00 & 30.95 $\pm$ 0.00 & 24.85 $\pm$ 7.46 \\


% % & Center & 79.85 $\pm$ 0.46 & 77.32 $\pm$ 1.33 & 75.25 $\pm$ 0.40 & 57.75 $\pm$ 3.53 & 41.73 $\pm$ 0.92 & 39.34 $\pm$ 2.29 \\
%     LayerNorm~\cite{layernorm} & 80.51 $\pm$ 0.12 & \cellcolor{best}80.28 $\pm$ 0.66 & 75.05 $\pm$ 0.00 & 30.95 $\pm$ 0.00 & 30.95 $\pm$ 0.00 & 24.85 $\pm$ 7.46 \\
%     BatchNorm~\cite{batchnorm} & 78.09 $\pm$ 0.00 & 77.87 $\pm$ 0.02 & 73.62 $\pm$ 0.57 & 70.79 $\pm$ 0.00 & 53.90 $\pm$ 2.19 & 35.32 $\pm$ 3.41\\
%     PairNorm~\cite{pairnorm} & 79.01 $\pm$ 0.00 & 78.26 $\pm$ 0.50 & 73.21 $\pm$ 0.00 & 62.96 $\pm$ 0.00 & 48.13 $\pm$ 0.91 & 44.01 $\pm$ 3.46 \\
%     ContraNorm~\cite{contranorm} & \cellcolor{best}81.55 $\pm$ 0.21 & 79.61 $\pm$ 0.75 & 77.71 $\pm$ 0.00 & 63.35 $\pm$ 0.00 & 44.56 $\pm$ 4.83 & 38.97 $\pm$ 0.00 \\
%     DropEdge~\cite{dropedge} & 78.38 $\pm$ 0.00 & 74.47 $\pm$ 0.00 & 26.91 $\pm$ 0.83 & 22.24 $\pm$ 3.04 & 27.18 $\pm$ 0.00 & 25.98 $\pm$ 6.00\\
%     Residual& 80.68 $\pm$ 0.09 & 78.77 $\pm$ 0.00 & \cellcolor{secondbest}79.26 $\pm$ 0.21 & 40.91 $\pm$ 0.00 & 30.95 $\pm$ 0.00 & 27.90 $\pm$ 6.09\\
% \midrule
%      \ourst-Feature & 80.44 $\pm$ 0.83 & 79.26 $\pm$ 1.18 &  78.56 $\pm$ 0.59 & \cellcolor{secondbest} 77.22 $\pm$ 0.55 &\cellcolor{secondbest} 73.65 $\pm$ 0.48 &\cellcolor{secondbest} 61.62 $\pm$ 5.24\\
%      \ourst-Label &80.31 $\pm$ 0.70 & 79.16 $\pm$ 1.30 & \cellcolor{best}79.50 $\pm$ 0.00 & \cellcolor{best}77.43 $\pm$ 1.49 & \cellcolor{best}74.52 $\pm$ 0.36 & \cellcolor{best}65.02 $\pm$ 2.97 \\
% \midrule
% \rowcolor{gray!8}\multicolumn{7}{c}{\textit{CiteSeer}~\citep{citeseer}}\\
% \midrule
%    GCN~\cite{gcn} &\cellcolor{best} 67.45 $\pm$ 0.54 & 65.62 $\pm$ 0.25 & 37.22 $\pm$ 2.46 & 22.03 $\pm$ 4.76 & 19.65 $\pm$ 0.00 & 19.65 $\pm$ 0.00 \\


% % & Center & 67.21 $\pm$ 0.64 & 65.50 $\pm$ 0.99 & 59.25 $\pm$ 3.18 & 40.29 $\pm$ 1.18 & 41.73 $\pm$ 0.92 & 35.81 \pm 1.21\\
%     LayerNorm~\cite{layernorm} & 67.24 $\pm$ 0.66 &64.95 $\pm$ 0.72 & 38.87 $\pm$ 4.12 & 24.29 $\pm$ 5.68 & 19.65 $\pm$ 0.00 & 19.65 $\pm$ 0.00 \\
%      BatchNorm~\cite{batchnorm} &63.44 $\pm$ 0.94 & 62.34 $\pm$ 0.25 & 61.36 $\pm$ 0.00 & 50.58 $\pm$ 1.24 & 41.41 $\pm$ 0.00 & 35.00 $\pm$ 1.09 \\
%     PairNorm~\cite{pairnorm} & 63.58 $\pm$ 0.63 & 64.32 $\pm$ 0.95 & 61.95 $\pm$ 1.24 & 50.06 $\pm$ 0.00 & 37.21 $\pm$ 1.87 & 36.09 $\pm$ 0.07 \\
%     ContraNorm~\cite{contranorm} & 66.83 $\pm$ 0.49 & 64.78 $\pm$ 0.92 & 60.70 $\pm$ 0.60 & 44.79 $\pm$ 1.65 & 37.36 $\pm$ 0.25 & 30.85 $\pm$ 0.81 \\
%     DropEdge~\cite{dropedge} & 63.86 $\pm$ 0.03 & 62.24 $\pm$ 0.90 & 24.73 $\pm$ 5.72 & 20.65 $\pm$ 0.00 & 20.04 $\pm$ 0.19 & 19.95 $\pm$ 0.09\\
%     Residual & \cellcolor{secondbest} 67.45 $\pm$ 0.54 & 66.21 $\pm$ 0.16 & \cellcolor{best}67.34 $\pm$ 0.00 & 33.21 $\pm$ 0.00 & 19.65 $\pm$ 0.00 & 19.65 $\pm$ 0.00 \\
% \midrule
%     \ourst-Feature &  67.38 $\pm$ 0.66 & \cellcolor{best}66.94 $\pm$ 0.00 & 66.29 $\pm$ 0.02 & \cellcolor{secondbest}65.35 $\pm$ 1.99 & \cellcolor{best}61.43 $\pm$ 0.00 & \cellcolor{secondbest}42.09 $\pm$ 1.65\\
%      \ourst-Label & 67.23 $\pm$ 0.64 & \cellcolor{secondbest} 66.72 $\pm$ 0.00 & \cellcolor{secondbest}66.29 $\pm$ 0.89 & \cellcolor{best}65.50 $\pm$ 2.13 & \cellcolor{secondbest}59.93 $\pm$ 0.85 & \cellcolor{best}44.41 $\pm$ 1.57 \\
% \midrule
% \rowcolor{gray!8}\multicolumn{7}{c}{\textit{PubMed}~\citep{pubmed}}\\
% \midrule
%    GCN~\cite{gcn} & \cellcolor{best}76.44 $\pm$ 0.34 & 76.52 $\pm$ 0.32 & 69.58 $\pm$ 5.89 & 39.92 $\pm$ 0.00 & 39.92 $\pm$ 0.00 & 39.92 $\pm$ 0.00 \\


% % & Center & 75.19 $\pm$0.26	& 76.67 \pm	0.00 &OOM &OOM &OOM & OOM\\
%     LayerNorm~\cite{layernorm} & 76.27 $\pm$ 0.51 & 76.71 $\pm$ 0.24 & 76.95 $\pm$ 0.17 & 39.92 $\pm$ 0.00 & 39.92 $\pm$ 0.00 & 39.92 $\pm$ 0.00 \\
%     BatchNorm~\cite{batchnorm} & 75.52 $\pm$ 0.12 & \cellcolor{secondbest}77.15 $\pm$ 0.00 & 77.10 $\pm$ 0.00 & 76.92 $\pm$ 0.00 & 75.43 $\pm$ 0.00 & 69.33 $\pm$ 1.01 \\
%     PairNorm~\cite{pairnorm} & 75.66 $\pm$ 0.11 & 76.71 $\pm$ 0.00 & \cellcolor{secondbest}77.99 $\pm$ 0.00 & \cellcolor{secondbest}77.22 $\pm$ 0.39 & 75.52 $\pm$ 2.02 & 71.22 $\pm$ 3.68 \\
%     ContraNorm~\cite{contranorm} & 76.05 $\pm$ 0.33 & \cellcolor{best}78.42 $\pm$ 0.00 & OOM & OOM & OOM & OOM \\
%     DropEdge~\cite{dropedge}& 73.41 $\pm$ 0.03 & 73.96 $\pm$ 0.79 & 52.51 $\pm$ 10.91 & 40.27 $\pm$ 0.00 & 39.90 $\pm$ 0.59 & 40.08 $\pm$ 0.39 \\
%     Residual & \cellcolor{secondbest} 76.44 $\pm$ 0.34 & 77.28 $\pm$ 0.00 & 77.38 $\pm$ 0.00 & 63.14 $\pm$ 3.05 & 39.92 $\pm$ 0.00 & 39.92 $\pm$ 0.00 \\
% \midrule
%     \ourst-Feature & 75.72 $\pm$ 0.06 & 76.84 $\pm$ 0.00 & \cellcolor{best}78.39 $\pm$ 0.00 &\cellcolor{best} 79.71 $\pm$ 0.00 & \cellcolor{best}77.59 $\pm$ 0.23 & \cellcolor{best}78.06 $\pm$ 0.13\\
%     \ourst-Label & 76.33 $\pm$ 0.25 & 76.91 $\pm$ 0.00 & 77.60 $\pm$ 0.49 & 76.31 $\pm$ 0.00 & \cellcolor{secondbest}77.17 $\pm$ 0.67 & \cellcolor{secondbest}78.01 $\pm$ 0.16\\
% % Add more rows as needed
% % \midrule

% % Arxiv & GCN & 70.20 $\pm$ 0.36 & 70.84 $\pm$ 0.12 & 69.73 $\pm$ 0.28\\
% %     & APPNP \\
% %     & Center \\
% %     & LayerNorm \\
% %     & PairNorm \\
% %     & ContraNorm \\
% %     & \ourst-Feature \\
% %     & \ourst-Label \\
% \bottomrule
% \end{tabular}
% \end{adjustbox}
% \end{table}
\textbf{Datasets.} 
% \section{Dataset Generation}
\label{sec:dataset}
\revise{
To train the proposed GNN, we constructed a dataset of building structures and a subset of these structures were subjected to fire simulations using FEA. The dataset generation process is illustrated in \figref{fig:dataset_generation_procedure}. Initially, a total of 33,000 building structures with geometrical details, material properties, and gravity loads were created. Due to randomness in generating these structures, a filter is applied to remove unreasonable data after gravity load simulation, which included 15,377 structures. A trade-off between computational feasibility and model performance is made among the remaining 17,623 structures. As further labeling structures with MIDR requires resource-intensive fire simulations via OpenSeesRT, a large proportion of 16,050 structures is selected as unlabeled dataset. On the other hand, each of the other 1,573 structures was further subjected to 30 different fire simulations, forming the labeled dataset containing $1,573\times 30 = 47,190$ fire cases.} This section details the step-by-step process for generating the dataset, including geometry creation, material property assignment, and simulations due to gravity loads and fire scenarios. 
% To train the proposed neural network, we constructed a dataset comprising building structure data and a subset of fire scenario data. The dataset generation process is illustrated in \figref{fig:dataset_generation_procedure}. 
% A total of 33,000 building structures with geometric details, material properties, and gravity loads were initially created. Out of these, 3,000 structures were selected as labeled data, and the remaining 30,000 were designated as unlabeled data. Further, about half of them filtered out due to instability under gravity loads only. 
\begin{figure*}[h!]
    \centering
    \includegraphics[width=0.8\linewidth]{figures/dataset_filter_procedure.pdf}
    \caption{Workflow for dataset generation (geometry, material property, gravity loads, and fire scenarios).}
    \label{fig:dataset_generation_procedure}
\end{figure*}

\subsection{Geometry Generation}
\label{subsec:geometry_generation}
The geometry of the building structures forms the foundation of the dataset. Regular 
\revise{3D structures} resembling multi-story parking structures or shopping malls were generated, with parameters such as building floor dimensions and story heights selected randomly. Each building structure is composed of multiple rooms, which serve as the basic unit in this study. A room herein is a cuboid space defined by specific length, width, and height. Within a structure, rooms of the same dimensions are uniformly arranged along the length, width, and height, corresponding to the $x$-, $y$-, and $z$-axes, respectively. Structures vary in room size and number of rooms along each axis. Specifically, the room length, width, and height are independently sampled from a uniform distribution within the interval $[2, 5]$ meters along the three directions of the structure. Similarly, the room number along each axis is uniformly sampled independently as an integer within the interval $[2, 7]$, i.e., the maximum number of stories of the buildings simulated in this study is 7.

To introduce variability and simulate real-world scenarios, approximately $8\%$ of structural elements (beams or columns) are randomly removed after initial geometry creation. 
\revise{Such removal is not fire-induced damage, but reflects functional diversity often observed in real buildings, such as open spaces designed for activities in shopping malls, e.g., ice skating rinks. Examples of the generated geometries are illustrated in \figref{fig:example_generated_geometry}, showcasing the diversity and realism of the dataset. This element removal does not affect the definition of room's geometry in the structure and nor does it affect the number of considered fire scenarios.} 

\revise{A range of coefficient of variation values ($3.3\%$ to $17.5\%$) was derived from prior studies that investigated the statistics of geometrical and material properties of structural components of buildings (e.g., \cite{mirza1979variations, lee2004probabilistic}). These studies provide empirical data on the natural variability in parameters such as Young's modulus, yield strength, and dimensions of structural elements due to manufacturing tolerances and material inconsistencies. By selecting $8\%$ for the removal of structural elements in our database, we aimed to maintain a level of variability that is representative of real-world uncertainties while ensuring computational feasibility. This choice ensures that the database captures realistic deviations without introducing extreme cases that may not be commonly encountered in practice.}

\begin{figure*}[h!]
    \centering
    \includegraphics[width=\linewidth]{figures/example_generated_geometry.pdf}
    \caption{Examples of generated structural geometry of different sizes (all dimensions in meters).}
    \label{fig:example_generated_geometry} 
\end{figure*}

{\blockRevise

In this study, we opted for a deterministic square, dimension of $0.1$ m, solid cross-sectional steel elements due to their simplicity in modeling and analysis. Square sections exhibit uniform geometrical properties in all directions, simplifying the computation of structural responses and avoiding complications associated with more complex shapes, such as wide-flange sections, facilitating the computational efficiency and scalability to generate a large dataset. This choice also helps to mitigate issues related to stress concentrations and facilitates a more straightforward representation of structural behavior under thermal loads. 

\textit{Remark:} The selected cross-section provides a comparable flexural rigidity to a $W 130 \times 130 \times 28.1$ wide-flange section (metric units), albeit with significantly higher axial rigidity. This cross-section is acceptable for gravity-load-designed frames under service loading conditions where the models assume fully rigid, moment-resisting beam-column connections for the evaluation of the IDR under thermal loading. This assumption is reasonable in this computational study where the primary interest is to understand the global deformation response of frames under fire conditions. The selection of uniform square cross-sections for both beams and columns, rather than adherence to standard capacity design principles, was made here primarily for computational efficiency and to reduce design parameters in the database generation process. This choice allows for simplified and scalable approach to analyze the fire-induced response of generic steel frames without the need for large section variations, where this study mainly focuses on the fire vulnerability assessment using ML-based predictions. However, if additional loading conditions, e.g., seismic or wind loads, were to be considered, larger sections, strong-column/weak-beam principle, and ductile detailing would be required in the generated buildings for realistic structural behavior under combined loading conditions. Future studies may also consider investigating the influence of variable cross-sectional dimensions and semi-rigid connections on the structural performance under fire conditions. 
} % blockRevise

\subsection{Material Properties}
Steel is chosen as the material for the structures. To reflect real-world variations, we randomly assign one of five slightly different steel material types to each structural element. \revise{
The ranges of material properties are provided in \tabref{tab:material_property_ranges} and the properties are sampled from uniform distributions of the corresponding ranges. These variations simulate differences arising from manufacturing batches or regional material properties. That these properties are at ambient temperature and change when the temperature rises due to a fire. The selection of materials with varying properties is aimed at increasing the diversity of the data. Our goal is to represent as wide a range of data as possible with a limited amount of building structure data, thereby enhancing the generalization ability of the GNN. Our assumed material property ranges are expected to be wider than the real-world conditions based on findings in \cite{mirza1979variations, lee2004probabilistic}. Therefore, we are essentially tackling a more challenging and general task. If we can solve this problem, we are confident that our method will perform equally well or even better in real-world scenarios.
}
\begin{table}[h!]
    \centering
    \caption{Material properties ranges for considered steel structures.}
    \begin{tabular}{lc}
        \toprule
        Property & Range \\
        \midrule
        Young's modulus & [168, 252] GPa \\
        Yield strength & [220, 330] MPa \\
        Strain-hardening ratio & [0.8, 1.2] \% \\
        \bottomrule
    \end{tabular}
    \label{tab:material_property_ranges}
\end{table}

\subsection{Gravity Loads}
Gravity loads are applied to columns and beams based on their \revise{influence (tributary) areas as typically conducted in structural analysis. The considered ``service'' load conditions include the column self-weight and the additional loads directly supported on the beams from their self-weight and weights of the reinforced concrete slabs, people as live load, and building content. An edge beam typically carries approximately half the gravity load supported by a parallel interior beam}. The ranges of gravity loads are listed in \tabref{tab:gravity_load_ranges}. \revise{The loads are sampled from uniform distributions of the corresponding ranges.} Structures that failed to meet an MIDR threshold of $1\%$ under gravity loads were deemed unacceptable designs and filtered out, as such configurations of randomly chosen geometry, material, and gravity load combinations were considered unrealistic from a regulatory and practicality points of view.
\begin{table}[h!]
    \centering
    \caption{Gravity load ranges for considered beams and columns.}
    \begin{tabular}{lc}
        \toprule
        Element & Range (kN/m)  \\
        \midrule
        Column & [0.5, 1.0]  \\
        Edge beam & [1.5, 4.5]  \\
        Interior beam & [3.0, 7.5]  \\
        \bottomrule
    \end{tabular}
    \label{tab:gravity_load_ranges}
\end{table} 

\subsection{Rule-based Thermal Load Generation}
\label{subsec:thermal_load_generation}
To evaluate a building's structural response during a fire event, we employed a simplified rule-based approach for thermal load generation. 
% Previous studies \cite{nan_structuralfire_2023} have demonstrated that steel structures rapidly equilibrate with surrounding gases temperatures due to efficient heat exchange. Consequently, gas temperatures can be directly used as inputs for FEA tools, e.g., OpenSees, simplifying the process of modeling thermal loads. 
% Accurately simulating temperature fields in fire scenarios poses significant challenges. Advanced thermodynamic simulations, such as those performed using Fire Dynamics Simulator (FDS) \cite{mcgrattan_fire_2000}, provide precise temperature predictions. However, these methods are hindered by high computational costs, prolonging execution times, and limited scalability, making them impractical for generating large datasets. Additionally, real-world fire loads often display substantial spatial variability across different rooms \cite{dundar_fire_2023}, resulting in scenario-specific temperature fields with limited generalizability. For example, studies on bridge fires \cite{he_study_2024} have demonstrated that environmental factors, such as wind speeds, can significantly influence temperature distributions. Furthermore, even within identical scenarios, variations in fire modeling methodologies can produce distinctly different temperature fields \cite{zhang_temperature_2020, du_new_2012}. These challenges emphasize the need for efficient and adaptable methods to generate fire temperature data.
% To address these issues, we adopted a rule-based approach to model temperature variations. 
According to \cite{spearpoint_fire_2008}, a typical fire development follows a predictable pattern. During the {\em{growth stage}}, the temperature rises slowly and approximately linearly after ignition. This is followed by the {\em{flashover stage}}, where temperatures increase rapidly to peak values. After reaching the peak, the temperature either stabilizes or continues to rise slowly until the {\em{decay stage}} begins. Inspired by this fire development pattern, we describe the temperature evolution in time, $t$, prior to the decay stage in two distinct stages:
\begin{enumerate}
    \item {\bf{Initial linear increase stage}}: For $t \in [0, t_1)$, temperature increases gradually and linearly as the fire spreads through the building. This stage represents the time before the fire directly affects a structural element.  
    \item {\bf{ISO 834 fire curve stage}}: For $t \in [t_1, t_{\thre}]$, temperature rises rapidly following the ISO 834 curve \cite{ISO834}, modeling the direct impact of the fire on the structural element. 
\end{enumerate}
The slope of the linear temperature increase, $c$, and the transition time, $t_1$, are influenced by the spatial relationship between the fire source and the structural element. For the second stage of temperature evolution, we utilize the ISO 834 curve, a widely accepted standard for fire resistance testing. This standardized fire curve describes the temperature rise over time, enabling rapid and consistent thermal fields across various scenarios. The duration of fire simulation in this study is set to $t_{\thre}=60$ minutes. This value represents the upper limit for the temperature evolution of each structural element, providing a consistent basis for analyzing the structural response to fire.

Let $(x, y, z)$ represents the midpoint of a structural element and $(x_{\subfire}, y_{\subfire}, z_{\subfire})$ the fire source point. \revise{Integer parameters $h$ and $h_{\subfire}$ correspond to the respective floor levels of the element and the fire source}. The temperature evolution for each element is expressed as follows:
\begin{enumerate}
    \item Linear increase stage ($0 < t < t_1$):
    \begin{equation}
    T(t) = c \cdot t,
    \end{equation}
    where $c$, the rate of temperature increase ($^\circ\mathrm{C}/\mathrm{min}$), depends on the height difference between the element, $h$, and the fire source, $h_{\subfire}$:
    \begin{equation}
        c = 
        \begin{cases} 
        5\left/\left(h - h_{\subfire} + 1\right)\right., & h \geq h_{\subfire}, \\
        2\left/\left(h_{\subfire} - h\right)\right., & h < h_{\subfire}.
        \end{cases}
    \end{equation}
     \item ISO 834 stage ($t \geq t_1$):
\begin{equation}
    T(t) = c \cdot t_1 + 345 \log_{10} \left(8 \left(t - t_1\right) + 1\right).
\end{equation}
\end{enumerate}

The transition (arrival) time $t_1$, marking the end of the linear stage, depends on the spatial distance between the fire source and the element. We define the following two Euclidean distances $L_p$ in the $xy$ plane and $L_s$ in the $xyz$ space:
\begin{eqnarray}
L_p & \triangleq & \sqrt{(x - x_{\subfire})^2 + (y - y_{\subfire})^2}, \\
\label{eq:Lp}
L_s & \triangleq & \sqrt{(x - x_{\subfire})^2 + (y - y_{\subfire})^2 + (z - z_{\subfire})^2}.
\label{eq:Ls}
\end{eqnarray}
Accordingly, the transition time, $t_1$, is expressed as follows:
\begin{equation}
    t_1 = 
    \begin{cases}
    \beta_{1} \cdot \left(1 - \exp\left\{- L_s\left/\alpha_{1}\right.\right\}\right), & h > h_{\subfire}, \\
    \beta_{2} \cdot \left(1 - \exp\left\{- L_p\left/\alpha_{2}\right.\right\}\right), & h = h_{\subfire}, \\
    \beta_{3} \cdot \left(1 - \exp\left\{- L_s\left/\alpha_{3}\right.\right\}\right), & h < h_{\subfire} .
    \end{cases}
    \label{eq:t1}
\end{equation}
The parameters $\beta_i$ and $\alpha_i$ for determining $t_1$ are summarized in Table~\ref{tab:fire_spread_parameters}. In this study, we take $r_{\mathrm{up}}=0.95$ and $r_{\mathrm{down}}=0.97$.
\begin{table}[ht]
    \centering
    \caption{Fire spread parameters for $t_1$ calculations.}
    \begin{tabular}{lcc}
        \toprule
        Case  & $\beta_i$ & $\alpha_i$  \\
        \midrule
        $i=1$, Upward spread & $16 \left.\left(1-r_{\mathrm{up}}^{\left|h-h_{\subfire}\right|}\right)\right/\left(1-r_{\mathrm{up}}\right)$ & $10$  \\
        $i=2$, Horizontal spread & $18$ & $18$  \\
        $i=3$, Downward spread & $30 \left.\left(1-r_{\mathrm{down}}^{\left|h-h_{\subfire}\right|}\right)\right/\left(1-r_{\mathrm{down}}\right)$ & $5$  \\
        \bottomrule
    \end{tabular}
    \label{tab:fire_spread_parameters}
\end{table}

\figref{fig:t1_curve} illustrates the $t_1$ curves for various fire scenarios: (1) fire originating on the lower floor, $h-h_{\subfire}=1$ with rapid upward spread, (2) fire on the same floor, $h=h_{\subfire}$ with the fastest spread, and (3) fire on the upper floor, $h_{\subfire}-h=1$ with slow downward spread. The exponential decay in $t_1$ reflects the accelerating fire propagation speed as the distance increases. \figref{fig:t1_curve} also indicates that the employed simplified model is consistent with the Markov chain-based dynamic model given by \cite{cheng_dynamic_2011}, where the rooms at the same floor of the fire point start flashover slightly before the corresponding upper floors. Additionally, $\beta_{1}$ and $\beta_{3}$ are the summation of a geometric sequence, where story level $h$ is the index. The common ratios $r_{\mathrm{up}}<1$ in $\beta_{1}$ and $r_{\mathrm{down}}<1$ in $\beta_{3}$ indicate that the fire speeds up to spread through the next story, which is consistent with the real-world fire spread mechanism given in \cite{hokugo_mechanism_2000}. The temperature profile within the range $t \in [0, t_{\thre}]$ is subsequently used as the thermal load in OpenSeesRT simulations to compute displacements at each structural node at time $t_{\thre}$.
\begin{figure}[h!]
    \centering
    \includegraphics[width=0.8\linewidth]{figures/m204_t1_curve.pdf}
    \caption{Three examples for the $t_1$ curve.}
    \label{fig:t1_curve}
\end{figure}

\revise{
\textit{Remark:} The effects of structural elements, such as concrete floor slabs and partitions, are not explicitly modeled in our approach. Instead, their influence is implicitly captured through the careful selection of the parameters $ \alpha, \beta, r_\mathrm{up} $, and $ r_\mathrm{down} $. This parameterization provides a unified framework for generating temperature fields. Indeed, fire propagation is governed by a multitude of factors and remains an open research question. For instance, if the fire resistance of a floor slab is enhanced by fire protective coating, the corresponding model can account for this by decreasing $\alpha_1$ \& $\alpha_3$, increasing $\beta_1$ \& $\beta_3$, and adopting larger values for $r_\mathrm{up}$ \& $r_\mathrm{down}$, which collectively slow down the vertical spread of fire. Conversely, scenarios involving higher amounts of combustible materials would warrant the opposite adjustments. This flexible and integrated approach avoids the need to design separate models for different fire propagation scenarios while still capturing the essential effects.
}

\revise{
In conclusion, our rule-based approach is a computationally efficient method for approximating fire temperature fields, enabling large-scale dataset generation to train predictive models. By combining ISO 834 fire curves with spatial considerations and embedding structural effects through parameter calibration, the method achieves a balanced trade-off between accuracy and scalability, making it a practical solution for thermal load modeling in fire scenarios. After generating the temperature of each beam or column according to the middle point, the temperature is applied as uniform thermal load to the elements of the structure in question using OpenSeesRT. 
}

% In conclusion, this rule-based approach is a computationally efficient method to approximate fire temperature fields, enabling large-scale dataset generation to train predictive models. By combining ISO 834 fire curves with spatial considerations, the method balances accuracy and scalability, making it a practical solution for thermal load modeling in fire scenarios.

% \subsection{Interstory Drift Ratio}
\subsection{OpenSeesRT Simulation}
\label{subsec:opensees_simulation}

The thermal and mechanical responses of 3D frame structures under combined fire and gravity loads are simulated using OpenSeesRT \cite{perez2024openseesrt}. \revise{In the simulation, the IDR of each node at $t_{\thre}$ is computed using the computed nodal displacements. Each structural model features six degrees of freedom per node (3 translational  and 3 rotational), with linear geometrical transformations (\texttt{geomTransf: Linear}) defining how the element local coordinate systems are mapped to the global coordinate system and assuming small displacements and rotations. Although OpenSeesRT allows a variety of options for modeling finite deformations, in the present simulations and mainly for simplicity, we did not consider large deformations. All bottom nodes (nodes on the ground) are fully constrained in all six degrees of freedom, while degrees of freedom os all other nodes are free.} Material behavior is temperature-dependent and modeled with \texttt{Steel01Thermal}, while fiber-based sections (\texttt{FiberThermal}) capture nonlinear interactions between thermal and mechanical responses at the cross-section level. \revise{Structural elements are represented as displacement-based Euler-Bernoulli beam-columns (\texttt{dispBeamColumnThermal}). This element  formulation accounts for thermal strains (temperature gradients) in the section, which is discretized into fibers. Numerical integration is used along the length of each element using three integration (Gauss) points, one at each end and the third in the middle of the element.}

{\revise{Thermal expansion of steel members plays a crucial role in IDR development. In reality, reinforced concrete floor slabs heat at a different rate than steel members due to their higher thermal mass and lower thermal conductivity. This differential heating can lead to restrained thermal expansion, introducing axial compression in beams and affecting the overall structural response. In this study, explicit {\em{composite action}} between steel members and concrete slabs is not modeled. Instead, our approach focuses on isolating the response of the steel structural frame, which is often the critical load-bearing component in fire scenarios. This assumption aligns with prior studies \cite{Possidente_2024} demonstrating that steel structures reach thermal equilibrium with surrounding gases quickly, allowing the use of uniform thermal loading in fire analysis. Future work could enhance this framework by incorporating slab-beam interaction effects, through a refined FEA for an extended dataset where constraints imposed by floor slabs are explicitly considered.}

The analysis begins with the application of gravity loads, followed by incremental thermal loads simulating the fire exposure. A static nonlinear solver using  \texttt{ExpressNewton} algorithm ensures convergence, while the \texttt{NormDispIncr} test maintains accuracy. An incremental \texttt{LoadControl} scheme with small step sizes is employed to guarantee numerical stability, using 10\% for gravity loads and 1\% for thermal loads. 

\revise{
In the thermal load analysis, uniform thermal load is applied to each beam or column, i.e., the temperature of each element is set to be that at the middle point, according to \secref{subsec:thermal_load_generation}. The \texttt{Steel01Thermal} material allows the properties (e.g., Young's modulus and yield strength) to be adjusted at increasing temperatures according to \cite{EN1993} using its Table 3.1: Reduction factors for the stress-strain relationship of carbon steel at elevated temperatures. For example, if the Young’s modulus at ambient temperature is $E_0$, then as the temperature ($T$) increases, the modulus changes as $E(T) = \eta (T) \times E_0$. \cite{EN1993} directly provides the values of $\eta(T) \in \left[0,1\right] $ at every $100 ^\circ\mathrm{C}$ interval and recommends using linear interpolation to obtain $\eta(T)$ for intermediate values of $T$.
} OpenSeesRT documentation \cite{OpenSeesThermalExamples} provides several examples of thermal analyses.

This modeling framework accommodates variations in material properties, cross-sectional geometries, and temperature profiles, providing robust simulations of structural behavior under fire conditions. The primary settings and configurations for the OpenSeesRT simulations are summarized in \tabref{tab:ops_detail}.
\begin{table}[h!]
    \centering
        \caption{Key settings of OpenSeesRT simulations.}
    \begin{tabular}{l|>{\raggedright\arraybackslash}p{0.6\linewidth}} %
    \toprule
    Modeling Aspect     & Details \\
    \midrule
    Geometry            & 3D models; 6 degrees of freedom per node \\
    Transformation      & geomTransf: Linear \\ 
    Material            & Steel01Thermal \\
    Section             & FiberThermal; Cross-section: $0.1$ m $\times$ $0.1$ m \\ 
    Element type        & {dispBeamColumnThermal} \\ 
    Loading             & Gravity loads: {beamUniform}; Thermal loads: {beamThermal} \\
    Integration scheme  & Incremental {LoadControl}; Step size: $10\%$ (gravity analysis), $1\%$ (thermal analysis) \\
    Nonlinear solver    & {ExpressNewton} algorithm; {UmfPack} solver; Convergence test: {NormDispIncr} tolerance: $10^{-8}$; Maximum \# iterations per step: $1000$. \\ 
    \bottomrule
    \end{tabular}
    \label{tab:ops_detail}
\end{table}

For each structure in the labeled dataset, 30 fire points are selected using a dual-granularity approach, \revise{i.e., two-stage sampling strategy,} to ensure they are well-distributed. Specifically, rooms are sequentially selected, with one fire point randomly chosen within each selected room. If a building is large and contains more than 30 rooms, we randomly select 30 rooms without replacement, i.e., ensuring that no more than one fire point is located in the same room. Conversely, if the building is small and has fewer than 30 rooms, all rooms are initially selected, with one fire point randomly assigned to each room. Additionally, rooms are then selected with replacement until a total of 30 fire points are assigned. \revise{The room-level sampling prioritizes selecting distinct rooms to avoid spatial clustering of fire points, while the point-level sampling ensures intra-room variability. This approach aligns with stratified sampling principles commonly used for efficient spatial representation, where multi-stage sampling strategies optimize coverage and variability, e.g., \cite{arunachalam_generalized_2023}, and enables a more comprehensive characterizing of how the structures respond under fire conditions.}
% This selection method prevents fire points from clustering too closely while maintaining an element of randomness. By distributing fire points in this manner, the 30 fire scenarios are effectively utilized, enabling a more comprehensive characterizing of how the structures respond under fire conditions.

\subsection{Summary of the Dataset Generation}
As discussed in this section and related to  \figref{fig:dataset_generation_procedure}, three key steps were considered in the development of the dataset: 
\begin{enumerate}
    \item {\bf{Filtering process}}: Structures with MIDR exceeding $1\%$ under gravity loads were excluded,  resulting in $1,573$ labeled structures retained for fire simulation and $16,050$ unlabeled structures for training the MFSP predictor.
    \item {\bf{Fire simulations}}: For each retained labeled structure, 30 fire scenarios were simulated using OpenSeesRT, yielding $47,190$ fire cases.
    \item {\bf{Data distribution check}}: MIDR distributions for labeled and unlabeled data under gravity loads were highly similar, because both datasets were generated using the same method. Under fire conditions, the MIDR distribution shifted, reflecting significant structural deformation with values reaching a maximum of about 6\%, an average of 1.70\%, and a standard deviation of 1.12\%. This step ensured a diverse and comprehensive dataset for the proposed predictive framework.
\end{enumerate}
The statistical distribution histograms for MIDR (after applying the $1\%$ filtering threshold \revise{for gravity load responses}) under different loading conditions are plotted in \figref{fig:histogram_mdr}. Figures \ref{fig:histogram_mdr}(a) and \ref{fig:histogram_mdr}(b) show the MIDR distributions of the labeled and unlabeled data, respectively, under gravity loads only. \figref{fig:histogram_mdr}(c) shows the MIDR distribution of the labeled data under the combined effects of gravity and fire loads. Fire load causes the structures to significantly deform, leading to a noticeably \revise{right-skewed} MIDR distribution.

\begin{figure*}[h!]
    \centering
    \includegraphics[width=\linewidth]{figures/histogram_mdr.pdf}
    \caption{Histograms of MIDR for labeled and unlabeled structures with gravity loads and fire cases.}
    \label{fig:histogram_mdr}
\end{figure*}

\revise{
This dataset provides the basis for training and testing the performance of the GNN-based framework. Although we employed a simplified rule-based thermal load generation method compared with conventional CFD-based simulations, the temperature field, the changes of the material properties, and the response of the structures, are all still highly nonlinear and complex. Therefore, it is still a challenging task for the NN to predict the MIDRs based on this dataset.
}
We use nine widely-used node classification
benchmark datasets (Table~\ref{tab: main_data}), where four of them are heterophilic (Texas, Wisconsin, Cornell, Squirrel, and
Amazon-rating~\citep{platonov2023critical}), and the remaining four are homophilic (Cora~\citep{cora},
Citeseer~\citep{citeseer}, and Pubmed~\citep{pubmed}) including one large-scale dataset (Ogbn-Arxiv~\cite{hu2020ogb}). 
Further information about the datasets and splits are provided in Appendix~\ref{app: exp}.
% and experimental results on more datasets
% In line with prior research, we employ the default training/validation/test splits provided by Pytorch Geometric (PyG). 
% For details of the datasets, including their sources and construction methods.
% We evaluate \ours for the semi-supervised node classification on Cora~\cite{cora}, CiteSeer~\cite{citeseer} and PubMed~\cite{pubmed} and one large-scale dataset ogbn-arxiv from OGB benchmarks~\citep{openbenchmark}.
% We also extend our models to two heterophilous datasets: Chameleon and Squirrel~\cite{heter_dataset}.
% We show the details of the dataset in Appendix~\ref{app: data}.




\textbf{Baselines and experiment settings.}
We compare the performance of \ours against the following $12$ baseline models. 
% \begin{itemize}
1) \textbf{Classic models}: MLP, SGC~\citep{sgc}.
2) \textbf{GNNs with normalization}: BatchNorm~\citep{batchnorm}, PairNorm~\citep{pairnorm} and ContraNorm~\citep{contranorm}.
3) \textbf{Augmenation-based GNNs}: DropEdge~\citep{dropedge}.
4) \textbf{GNNs with residual connections}: Residual, APPNP~\citep{appap}, JKNET~\citep{jknet} and DAGNN~\citep{dagnn}. 
5) \textbf{Other baselines}: GCNII~\citep{GCNII} and \(\omega\)GCN~\citep{wGCN}.
% \xw{你忘记cite gcnii 跟wgcn了}
For the sake of fair comparison, we do not deploy specific training techniques used in some prior works for benchmarking.
All models are trained under the same setting on the pure SGC backbone and we choose the best of scale controller in the range of $\{ 0.1, 0.5, 0.9\}$ for ContraNorm, DropEdge, and residual connections.
% For both Label-\ours and Feature-\ours, 
We choose the best of $\lambda$ in the range of $\{0.1, 0.5, 0.9\}$, fix $\alpha=1$ and select the best value for $\beta$ from $\{ 0.1, 0.5, 0.9\}$ for \ours. 
More experiment results with hyperparameter tuning and optimization strategies can be found in Appendix~\ref{app: exp}.
% fix $\alpha=1$ and only select $\beta$ from $\{0.1, 1,10, 20, 50, 100\}$ for simplify .
% \end{itemize}
% For more details of the classic anti-oversmoothing methods seen in Appendix~\ref{xx}.
% We apply both the linear SGC~\cite{sgc} and non-linear GCN~\cite{gcn} backbones.
% For fair comparison, we fix the hidden dimension to $32$ and dropout rate to $0.6$ following \cite{contranorm}.
% % The residual $\alpha=0.5$, the dropedge present is selecting from $\{0.3,0.5,0.7\}$.
% % We choose the best of scale controller $\alpha,\ \beta \in \{ 0.1, 0.2, 0.5, 0.7, 0.9\}$.
% We select the best settings for PairNorm, Residual, DropEdge, and ContraNorm based on their default hyperparameters. 

% We use Tesla-V100-SXM2-32GB in all experiments.
\textbf{RQ1: Node classification performance.}
In Table~\ref{tab: main_data}, we provide the mean of the node classification accuracy along with their corresponding standard deviations across 10 random seeds under the same 2-layer SGC backbone following~\citet{dgc}.
Overall, \ours achieves the best performance across $8$ datasets in the shallow layers, as Label/Feature-\ours performs the best on 7 out of the 8 datasets. 
% In particular, we make the following three observations:
% First, \ours outperforms all normalization methods. 
% Since our theoretical findings suggest that these normalization methods are essentially implicitly signed graph propagation, the theoretical properties of structural balance (Section~\ref{subsec: sb theory}) contribute to the enhanced classification accuracy of \ours.
% Second, \ours outperforms random argumentation based GNNs. Since DropEdge randomly drops edges, it isn't easy to characterize their exact behaviors, but we highlight that it works when it happens to remove edges between different classes of nodes thanks to our structurally balanced theory, as 
% DropEdge still follows the unified signed graph analysis in its message-passing scheme.
% Lastly, \ours outperforms residual connection based GNNs, including the last layer connection: residual and multilayer feature connection: APPNP, JKNET, and DAGNN. 
% In our analysis, GNNs with residual connections can be seen as a special case of signed graph propagation, where their positive and negative adjacency matrices are the linear combination of adjacency matrices of different orders, yet they are not the theoretically best solution to alleviate oversmoothing.
% This validates the effectiveness of our novel insight from a signed graph perspective. 

% \paragraph{Heterophilic datasets}
% Besides the three homophilic datasets, we also conduct experiments on four heterophilic datasets~\citep{heter_dataset}.
% We find that our method is still the most effective one across all of the methods for alleviating oversmoothing as indicated in Table~\ref{table: gcn heter}.
% Interestingly, we observe that Feature-\ours performs better than Label-\ours on the heterophilic datasets, which is the opposite of the results on the homophilic datasets.

\textbf{RQ2: Anti-oversmoothing analysis.}
% \paragraph{Results} 
% The results for SGC are detailed in Table \ref{table: sgc results} and we give the GCN results in Appendix (Table~\ref{table: gcn result}). 
We further evaluate the robustness of \ours by assessing its performance at deeper model depths: $K \in \{2, 10, 50, 100, 300\}$ for homophilic datasets and $K \in \{2, 5, 10, 20, 50\}$ for heterophilic datasets.
% To provide a comparative analysis against other GNNs, we also evaluate two best-performed normalization-based GNNs: BatchNorm and ContraNorm. the performances of these methods
% We evaluate on one heterophilic graph and two homophilic graphs.
Figure~\ref{fig: layer depth} shows that the performance of Feature/Label-\ours remains relatively stable with varying
% ($K = 50$) 
numbers of layers, achieving its best performance when the model gets deeper.
In contrast, the normalization methods considered exhibit a substantial decrease in performance as the number of layers increases, indicating their persistent susceptibility to the oversmoothing problem.
Note that we find that for \ours to maintain performance in the heterophilic dataset, \(\beta\) needs to be larger than the uniform range considered in Figure~\ref{fig: layer depth}. 
See Appendix~\ref{app: exp} for the result under larger \(\beta\), where \ours on deep layers remains $\approx60\%$ in Cornell.   

% \subsection{RQ3: Ablation Study}
\textbf{RQ3.1: Sensitivity analysis of training ratio.}
% Since Label-\ours leverages the ground truth label information to construct the negative graph, we conduct an ablation study examining the impact of different training data ratios. 
As shown in Figure~\ref{fig: train ratio}, Label-\ours's performance on the CSBM and Cora datasets improves as the training ratio increases. Even with a modest training ratio of 20\%, the worst-performing models still achieve an impressive 80\% accuracy, while the best models approach 100\% accuracy when the training ratio is increased to 80\%. This is in line with our theoretical insights that increasing the training ratio leads to more structural balance resulting from our method~\ours. 
% Moreover, our main experiments detailed in Table~\ref{table: sgc results} demonstrate that Label-\ours outperforms other methods, even when adopting the default training set ratios in those datasets, indicating its effectiveness in real-world graph settings.
% \begin{table*}
  [t]
  \centering
  \resizebox{\textwidth}{!}{%
  \begin{tabular}{cccccccccccc}
    \toprule \multicolumn{2}{c}{Components}                                                             & \multicolumn{5}{c}{Re-executability Rate (\%)} & \multicolumn{5}{c}{Readability (\#)} \\
    \cmidrule(lr){1-2} \cmidrule(lr){3-7} \cmidrule(lr){8-12}        \hspace{8pt}\labelemoji\hspace{8pt}                                                                & \hspace{8pt}\toolemoji\hspace{8pt}                                      & O0                                 & O1             & O2             & O3             & AVG            & O0             & O1             & O2             & O3             & AVG            \\
    \hline
    \rowcolor[rgb]{0.93,0.93,0.93}\multicolumn{12}{c}{\textbf{Initialize with LLM4Decompile-End-6.7B~\citep{llm4decompile}}}   \\
    \xmark                                                                                              & \xmark                                    & 69.51                              & 46.95          & 50.61          & 46.34          & 53.35          & 3.98 & 3.41 & 3.44 & 3.38 & 3.55 \\
    \cmark                                                                                              & \xmark                                    & 75.61                              & 50.61          & 50.00          & 50.00          & 56.55          & 4.01 & 3.44 & 3.39 & \textbf{3.49} & 3.58 \\
    \xmark                                                                                              & \cmark                                    & 83.54                     & \textbf{56.10}          & 51.22          & 50.61 & 60.37 & 4.05 & 3.51 & 3.51 & 3.42 & 3.62 \\
    \cmark                                                                                              & \cmark                                    & \textbf{85.37}                            & \textbf{56.10}                     & \textbf{51.83} & \textbf{52.43}          & \textbf{61.43} & \textbf{4.13} & \textbf{3.60} & \textbf{3.54} & \textbf{3.49} & \textbf{3.69} \\

    \rowcolor[rgb]{0.93,0.93,0.93}\multicolumn{12}{c}{\textbf{Initialize with Deepseek-Coder-6.7B-base~\citep{deepseekcoder}}} \\
    \xmark                                                                                              & \xmark                                    & 59.15                              & 35.98          & 39.02          & 37.80          & 42.99          & 3.71 & 3.05 & 3.16 & 3.05 & 3.24 \\
    \cmark                                                                                              & \xmark                                    & 66.46                              & 41.46          & 38.41          & 36.59          & 45.73          & 3.76 & 3.17 & \textbf{3.21} & 3.08 & 3.31 \\
    \xmark                                                                                              & \cmark                                    & 70.73                              & 39.63          & 39.02          & 40.24          & 47.41          & 3.90 & 3.17 & 3.08 & 3.11 & 3.31 \\
    \cmark                                                                                              & \cmark                                    & \textbf{79.88}                     & \textbf{45.73} & \textbf{43.90} & \textbf{42.68} & \textbf{53.05} & \textbf{3.96} & \textbf{3.21} & 3.18 & \textbf{3.19} & \textbf{3.38} \\
    \bottomrule
  \end{tabular}%
  }
  \caption{The ablation study of different methods across four optimization levels
  (O0, O1, O2, O3), as well as their average scores (AVG). The results in bold represent the optimal performance. The ~\labelemoji~ and ~\toolemoji~ means Relabedling and Function Call. \textbf{Bold} denotes the best performance.}
  \label{tab:ablation}
\end{table*}


% \begin{figure}[t]
%     \centering
%     \begin{subfigure}{0.63\textwidth}
%         \centering
%         \includegraphics[width=0.99\textwidth]{figures/eval_negative (3).pdf} % Adjust the path and filename as necessary
%         \caption{ Significance plot for $\beta$ in terms of test accuracy on cora (left) and texas (right) with fixed $\alpha=1$}
%         \label{fig: beta}
%     \end{subfigure}
%     \quad
%     \begin{subfigure}{0.33\textwidth}
%         \centering
%         \captionsetup{font=small}
%         \includegraphics[width=0.99\textwidth]{figures/eval_train.pdf} % Adjust the path and filename as necessary
%         \caption{Ablation study on Label-SBP. X-axis indicates the ratio of the training node numbers.}
%         \label{}
%     \end{subfigure}
%     \caption{Ablation study}
%     \label{fig: train ratio}
% \end{figure}




% \begin{figure*}[t]
% % \hspace{-10pt}
%     \begin{minipage}{.48\textwidth}
%     \captionof{table}{Node classification accuracy (\%) on the large-scale dataset~\textit{ogbn-arxiv}.}
%     % Test accuracy (\%) comparison results on large scale dataset (Ogbn-ArXiv). The best results are marked in blue and the second best results are marked in gray on every layer.
%     \centering
%     \resizebox{0.99\linewidth}{!}{
%     \begin{tabular}{lcccc}
%     \toprule
%      Model             & \#L=2              & \#L=4              & \#L=8            & \#L=16  \\
%     \midrule
%     GCN & 67.32 {\footnotesize $\pm$ 0.28} & 67.79 {\footnotesize $\pm$ 0.25} & 65.54 {\footnotesize $\pm$ 0.31} & 59.13 {\footnotesize $\pm$ 0.95}  \\
%          BatchNorm & 70.14 {\footnotesize $\pm$ 0.28} & 70.93 {\footnotesize $\pm$ 0.15} & 70.14 {\footnotesize $\pm$ 0.43} & 63.24 {\footnotesize $\pm$ 1.40} \\
%          % +LayerNorm& \cellcolor{secondbest}70.53 {\footnotesize $\pm$ 0.19} & \cellcolor{best}71.66 {\footnotesize $\pm$ 0.17} & \cellcolor{secondbest}71.23 {\footnotesize $\pm$ 0.16} & 68.62 {\footnotesize $\pm$ 0.47} \\
%          PairNorm & 70.48 {\footnotesize $\pm$ 0.20} & \cellcolor{best}71.59 {\footnotesize $\pm$ 0.17} & \cellcolor{best}71.24 {\footnotesize $\pm$ 0.07} & 68.92 {\footnotesize $\pm$ 0.43} \\
%          ContraNorm & OOM & OOM & OOM & OOM \\
%          DropEdge & 64.07 {\footnotesize $\pm$ 0.32} & 63.92 {\footnotesize $\pm$ 0.27} & 60.74 {\footnotesize $\pm$ 0.45} & 52.52 {\footnotesize $\pm$ 0.34} \\
%          Residual & 66.90 {\footnotesize $\pm$ 0.14} & 66.67 {\footnotesize $\pm$ 0.25} & 61.76 {\footnotesize $\pm$ 0.62} & 53.25 {\footnotesize $\pm$ 0.75} \\
%     % \midrule
%          Feature-\ourst & 67.89 {\footnotesize $\pm$ 0.10} & 68.47 {\footnotesize $\pm$ 0.26} & 65.09 {\footnotesize $\pm$ 0.30} & 60.34 {\footnotesize $\pm$ 0.94} \\
%          Label-\ourst & \cellcolor{best}70.55 {\footnotesize $\pm$ 0.22} & 71.54 {\footnotesize $\pm$ 0.18} & 71.07 {\footnotesize $\pm$ 0.28} & \cellcolor{best}69.33 {\footnotesize $\pm$ 0.59}  \\
%     \bottomrule
%     \end{tabular}
%     \label{tab: large}
%     }\hfill
%     \end{minipage}
%    \begin{minipage}{.52\textwidth}
%    % \vspace{0.2cm}
%        \centering
%        \includegraphics[width=\linewidth]{figures/eval_negative (3).pdf}
%        \captionof{figure}{Significance of negative graph weight $\beta$ on Cora and Texas datasets where we fix the positive graph weight $\alpha=1$.}
%        \label{fig:beta real} 
%    \end{minipage}
%    % \vspace{-0.5cm}
%     % \hspace{-10pt}
% %     \begin{minipage}{.35\textwidth}
% %     \centering
% %     \captionsetup{font=small}
% %     \caption{Accuracy on different splits of train/valid/test dataset. SGC in CSBM and GCN in Cora.}
% %     % SGC test accuracy (\%) comparison results on sbm of \ourst-Label on different splits of train/valid/test dataset. GCN test accuracy (\%) comparison results on Cora of \ourst-Label on different splits of train/valid/test dataset.
% % % The best results are marked in blue on every dataset.
% %     \centering
% %      \resizebox{0.95\linewidth}{!}{
% %     \begin{tabular}{lcc}
% %     \toprule
% %      Splits & CSBM  & Cora   \\
% %     \midrule
% %     % \midrule
% %     % sbm 0/5/5& 57.00 {\footnotesize $\pm$ 13.36}
% %       2/4/4 & 86.25 {\footnotesize $\pm$ 3.01} & 82.80 {\footnotesize $\pm$ 0.81}\\
% %       4/3/3 & 91.50 {\footnotesize $\pm$ 2.52} & 85.39 {\footnotesize $\pm$ 0.18 }\\
% %       6/4/4 & 91.50 {\footnotesize $\pm$ 6.05} & 87.64 {\footnotesize $\pm$ 0.37 }\\
% %       8/1/1 & \cellcolor{best}99.05 {\footnotesize $\pm$ 1.90} & \cellcolor{best} 94.10 {\footnotesize $\pm$ 0.74} \\
% %     \bottomrule
% %     \end{tabular}
% %     \label{tab: ablation}
% %     }
% %     \end{minipage}
%     % \caption{Caption}
%     % \label{tab:my_label}
% \vspace{-0.15in}
% \end{figure*}
\begin{figure}
   % \vspace{0.2cm}
   % \captionsetup{font=small}
       \centering
       \includegraphics[width=0.7\linewidth]{figures/eval_negative_3.pdf}
       \captionof{figure}{Significance of negative graph weight $\beta$ on Cora and Texas datasets where we fix the positive graph weight $\alpha=1$ and vary a large range of \(\beta\).}
       \label{fig:beta real} 
    % \vspace{-0.2in}
\end{figure}
\textbf{RQ3.2: Performance under varying graph homophily and heterophily levels.}
In order to test the performance of \ours on graphs with arbitrary
levels of homophily and heterophily, we conduct an ablation study in the CSBM setting with the controllable homophilic and heterophilic levels following~\citet{GRP-GNN}.
As shown in Figure~\ref{fig:beta csbm}, Feature/Label-\ours performs best in homophilic graphs when all nodes are effectively attracted to one another, i.e., when the repulsion strength $\beta$ is small. As $\beta$ increases, the performance of the model degrades.
% The parameter $\phi$ in the CSBM controls the relative importance of node features and graph topology in determining the homophily level.
% Specifically, $\phi$ ranges from -1 to 1, with lower values corresponding to strongly heterophilic graphs and higher values indicating strongly homophilic graphs. 
% Specially, $\phi=1$ corresponds to strongly homophilic graphs while $\phi=-1$ corresponds to strongly heterophilic graphs.
% We fix $\lambda=0.5$ and then vary $\beta$ which indicates the strength of the repulsive force between the two nodes introduced by the negative edge connecting them.
In contrast, for heterophilic graphs, when the attraction power of the positive graph dominates, \ours achieves only $50\%$ accuracy. 
As $\beta$ increases, the negative graph becomes more dominant, and the model's performance gets significantly better. We observe similar phenomena in the real homophilic and heterophilic graph datasets as shown in Figure~\ref{fig:beta real}.

% \vspace{-1ex}
\textbf{RQ3.3: Performance on large-scale dataset.} 
Finally, we conduct an evaluation of \ours on the large-scale ogbn-arxiv dataset, and the results are presented in Table \ref{tab: large}. 
% To maintain the sparsity of the graph structure and avoid additional computational overhead, we adopt variants of the \ours approach mentioned in Section~\ref{sec: method}. 
Overall, the results demonstrate that Label-\ours-v2 achieves comparable or even superior performance compared to previous normalization methods, particularly in the deep layer setting  ($L=16$).
This verifies the empirical superiority and robustness of our proposed signed graph construction in \ours, which effectively leverages the available label information to alleviate oversmoothing, even at scale.
\begin{table}
    \captionof{table}{Node classification accuracy (\%) on the large-scale dataset~\textit{ogbn-arxiv}.}
    % Test accuracy (\%) comparison results on large scale dataset (Ogbn-ArXiv). The best results are marked in blue and the second best results are marked in gray on every layer.
    \centering
    \resizebox{0.7\linewidth}{!}{
    \begin{tabular}{lcccc}
    \toprule
     Model             & \#L=2              & \#L=4              & \#L=8            & \#L=16  \\
    \midrule
    GCN & 67.32 {\footnotesize $\pm$ 0.28} & 67.79 {\footnotesize $\pm$ 0.25} & 65.54 {\footnotesize $\pm$ 0.31} & 59.13 {\footnotesize $\pm$ 0.95}  \\
         BatchNorm & 70.14 {\footnotesize $\pm$ 0.28} & 70.93 {\footnotesize $\pm$ 0.15} & 70.14 {\footnotesize $\pm$ 0.43} & 63.24 {\footnotesize $\pm$ 1.40} \\
         % +LayerNorm& \cellcolor{secondbest}70.53 {\footnotesize $\pm$ 0.19} & \cellcolor{best}71.66 {\footnotesize $\pm$ 0.17} & \cellcolor{secondbest}71.23 {\footnotesize $\pm$ 0.16} & 68.62 {\footnotesize $\pm$ 0.47} \\
         PairNorm & 70.48 {\footnotesize $\pm$ 0.20} & \cellcolor{best}71.59 {\footnotesize $\pm$ 0.17} & \cellcolor{best}71.24 {\footnotesize $\pm$ 0.07} & 68.92 {\footnotesize $\pm$ 0.43} \\
         ContraNorm & OOM & OOM & OOM & OOM \\
         DropEdge & 64.07 {\footnotesize $\pm$ 0.32} & 63.92 {\footnotesize $\pm$ 0.27} & 60.74 {\footnotesize $\pm$ 0.45} & 52.52 {\footnotesize $\pm$ 0.34} \\
         Residual & 66.90 {\footnotesize $\pm$ 0.14} & 66.67 {\footnotesize $\pm$ 0.25} & 61.76 {\footnotesize $\pm$ 0.62} & 53.25 {\footnotesize $\pm$ 0.75} \\
    % \midrule
         % Feature-\ourst-v2 & 67.89 {\footnotesize $\pm$ 0.10} & 68.47 {\footnotesize $\pm$ 0.26} & 65.09 {\footnotesize $\pm$ 0.30} & 60.34 {\footnotesize $\pm$ 0.94} \\
         Label-\ourst-v2 & \cellcolor{best}70.55 {\footnotesize $\pm$ 0.22} & 71.54 {\footnotesize $\pm$ 0.18} & 71.07 {\footnotesize $\pm$ 0.28} & \cellcolor{best}69.33 {\footnotesize $\pm$ 0.59}  \\
    \bottomrule
    \end{tabular}
    \label{tab: large}
    }
    % \vspace{-0.2in}
\end{table}

% verifying the empirical advantages of our proposed technique.
% \phi 
% Hyperparameter $\beta$ indicating strength of the repulsive force between the two nodes introduced by negative edges connecting them
% To illustrate the impact of $\beta$, we conduct ablation studies on the synthetic CSBM graphs under various settings as well as real datasets. 
%
% Following~\cite{GRP-GNN}, we reconstruct the CSBM with the 
% The parameter $\phi$ to control for the the information given by the node features and the graph topology and the homophily level. 
% Specifically $\phi=0$ indicates that only node features are informative, while $|\phi|=1$ indicates that only the graph topology is informative. Moreover, $\phi=1$ corresponds to strongly homophilic graphs while $\phi=-1$ corresponds to strongly heterophilic graphs.
%
% \begin{figure}[t]
% % \hspace{-10pt}
%     \begin{minipage}{.48\textwidth}
%     \captionof{table}{GCN test accuracy on the large-scale dataset~\textit{ogbn-arxiv}.}
%     % Test accuracy (\%) comparison results on large scale dataset (Ogbn-ArXiv). The best results are marked in blue and the second best results are marked in gray on every layer.
%     \centering
%     \resizebox{0.99\linewidth}{!}{
%     \begin{tabular}{lcccc}
%     \toprule
%      Model             & \#L=2              & \#L=4              & \#L=8            & \#L=16  \\
%     \midrule
%     GCN & 67.32 {\footnotesize $\pm$ 0.28} & 67.79 {\footnotesize $\pm$ 0.25} & 65.54 {\footnotesize $\pm$ 0.31} & 59.13 {\footnotesize $\pm$ 0.95}  \\
%          BatchNorm & 70.14 {\footnotesize $\pm$ 0.28} & 70.93 {\footnotesize $\pm$ 0.15} & 70.14 {\footnotesize $\pm$ 0.43} & 63.24 {\footnotesize $\pm$ 1.40} \\
%          % +LayerNorm& \cellcolor{secondbest}70.53 {\footnotesize $\pm$ 0.19} & \cellcolor{best}71.66 {\footnotesize $\pm$ 0.17} & \cellcolor{secondbest}71.23 {\footnotesize $\pm$ 0.16} & 68.62 {\footnotesize $\pm$ 0.47} \\
%          PairNorm & 70.48 {\footnotesize $\pm$ 0.20} & \cellcolor{best}71.59 {\footnotesize $\pm$ 0.17} & \cellcolor{best}71.24 {\footnotesize $\pm$ 0.07} & 68.92 {\footnotesize $\pm$ 0.43} \\
%          ContraNorm & OOM & OOM & OOM & OOM \\
%          DropEdge & 64.07 {\footnotesize $\pm$ 0.32} & 63.92 {\footnotesize $\pm$ 0.27} & 60.74 {\footnotesize $\pm$ 0.45} & 52.52 {\footnotesize $\pm$ 0.34} \\
%          Residual & 66.90 {\footnotesize $\pm$ 0.14} & 66.67 {\footnotesize $\pm$ 0.25} & 61.76 {\footnotesize $\pm$ 0.62} & 53.25 {\footnotesize $\pm$ 0.75} \\
%     % \midrule
%          % Feature-\ourst & 67.89 {\footnotesize $\pm$ 0.10} & 68.47 {\footnotesize $\pm$ 0.26} & 65.09 {\footnotesize $\pm$ 0.30} & 60.34 {\footnotesize $\pm$ 0.94} \\
%          Label-\ourst & \cellcolor{best}70.55 {\footnotesize $\pm$ 0.22} & 71.54 {\footnotesize $\pm$ 0.18} & 71.07 {\footnotesize $\pm$ 0.28} & \cellcolor{best}69.33 {\footnotesize $\pm$ 0.59}  \\
%     \bottomrule
%     \end{tabular}
%     \label{tab: large}
%     }\hfill
%     \end{minipage}
%    \begin{minipage}{.52\textwidth}
%    \vspace{0.2cm}
%        \centering
%        \includegraphics[width=\linewidth]{figures/eval_negative (3).pdf}
%        \captionof{figure}{Significance of negative graph weight $\beta$ on Cora and Texas datasets where we fix the positive graph weight $\alpha=1$.}
%        \label{fig:beta real} 
%    \end{minipage}
%    % \vspace{-0.5cm}
%     % \hspace{-10pt}
% %     \begin{minipage}{.35\textwidth}
% %     \centering
% %     \captionsetup{font=small}
% %     \caption{Accuracy on different splits of train/valid/test dataset. SGC in CSBM and GCN in Cora.}
% %     % SGC test accuracy (\%) comparison results on sbm of \ourst-Label on different splits of train/valid/test dataset. GCN test accuracy (\%) comparison results on Cora of \ourst-Label on different splits of train/valid/test dataset.
% % % The best results are marked in blue on every dataset.
% %     \centering
% %      \resizebox{0.95\linewidth}{!}{
% %     \begin{tabular}{lcc}
% %     \toprule
% %      Splits & CSBM  & Cora   \\
% %     \midrule
% %     % \midrule
% %     % sbm 0/5/5& 57.00 {\footnotesize $\pm$ 13.36}
% %       2/4/4 & 86.25 {\footnotesize $\pm$ 3.01} & 82.80 {\footnotesize $\pm$ 0.81}\\
% %       4/3/3 & 91.50 {\footnotesize $\pm$ 2.52} & 85.39 {\footnotesize $\pm$ 0.18 }\\
% %       6/4/4 & 91.50 {\footnotesize $\pm$ 6.05} & 87.64 {\footnotesize $\pm$ 0.37 }\\
% %       8/1/1 & \cellcolor{best}99.05 {\footnotesize $\pm$ 1.90} & \cellcolor{best} 94.10 {\footnotesize $\pm$ 0.74} \\
% %     \bottomrule
% %     \end{tabular}
% %     \label{tab: ablation}
% %     }
% %     \end{minipage}
%     % \caption{Caption}
%     % \label{tab:my_label}
% % \vspace{-0.12in}
% \end{figure}

% \begin{wraptable}{r}{.5\linewidth}
% \begin{table}[h]
\centering
% \small
\caption{Test accuracy (\%) comparison results on large scale dataset (Ogbn-ArXiv). 
The best results are marked in blue and the second best results are marked in gray on every layer.}
\begin{adjustbox}{width=0.7\textwidth}
\begin{tabular}{lcccc}
\toprule
 Model             & \#L=2              & \#L=4              & \#L=8            & \#L=16  \\
\midrule


% \midrule

GCN & 67.32 {\footnotesize $\pm$ 0.28} & 67.79 {\footnotesize $\pm$ 0.25} & 65.54 {\footnotesize $\pm$ 0.31} & 59.13 {\footnotesize $\pm$ 0.95}  \\
     BatchNorm & 70.14 {\footnotesize $\pm$ 0.28} & 70.93 {\footnotesize $\pm$ 0.15} & 70.14 {\footnotesize $\pm$ 0.43} & 63.24 {\footnotesize $\pm$ 1.40} \\
     % +LayerNorm& \cellcolor{secondbest}70.53 {\footnotesize $\pm$ 0.19} & \cellcolor{best}71.66 {\footnotesize $\pm$ 0.17} & \cellcolor{secondbest}71.23 {\footnotesize $\pm$ 0.16} & 68.62 {\footnotesize $\pm$ 0.47} \\
     PairNorm & 70.48 {\footnotesize $\pm$ 0.20} & \cellcolor{best}71.59 {\footnotesize $\pm$ 0.17} & \cellcolor{best}71.24 {\footnotesize $\pm$ 0.07} & \cellcolor{best}68.92 {\footnotesize $\pm$ 0.43} \\
     ContraNorm & OOM & OOM & OOM & OOM \\
     DropEdge & 64.07 {\footnotesize $\pm$ 0.32} & 63.92 {\footnotesize $\pm$ 0.27} & 60.74 {\footnotesize $\pm$ 0.45} & 52.52 {\footnotesize $\pm$ 0.34} \\
     Residual & 66.90 {\footnotesize $\pm$ 0.14} & 66.67 {\footnotesize $\pm$ 0.25} & 61.76 {\footnotesize $\pm$ 0.62} & 53.25 {\footnotesize $\pm$ 0.75} \\

     % Feature-\ourst & 67.89 {\footnotesize $\pm$ 0.10} & 68.47 {\footnotesize $\pm$ 0.26} & 65.09 {\footnotesize $\pm$ 0.30} & 60.34 {\footnotesize $\pm$ 0.94} \\
     Label-\ourst & \cellcolor{best}70.55 {\footnotesize $\pm$ 0.22} & 71.54 {\footnotesize $\pm$ 0.18} & 71.07 {\footnotesize $\pm$ 0.28} & \cellcolor{best}69.33 {\footnotesize $\pm$ 0.59}  \\

% % Add more rows as needed
\bottomrule
\end{tabular}
\end{adjustbox}
\label{table: large result}
% \end{table}
\end{wraptable}





% \subsection{Ablation Study}





% Appedix~\ref{app: ablation}. 




\label{sec: exp}
\section{Concluding Remarks}
In this paper, we proposed a novel approach utilizing multimodal LLMs to generate gesture-aware speech recognition transcripts for patients with language disorders. Our framework integrates verbal speech and iconic gestures, enabling the generation of enriched transcripts that capture the latent meaning conveyed through both modalities. Through extensive experimentation, we demonstrated that the proposed method effectively contextualizes incomplete or disfluent speech by incorporating gesture information, leading to more accurate and meaningful representations of the speaker's intent. These findings highlight the potential of our approach to significantly contribute to the field of speech and language therapy, offering innovative tools that can enhance the quality of life for individuals with language disorders by facilitating better communication and assessment methods.

\subsection{Ethical Statement} 
Our dataset was obtained from AphasiaBank with the approval of the Institutional Review Board (IRB) and adheres to the data sharing guidelines set by TalkBank\footnote{https://talkbank.org/share/ethics.html}. This includes complying with the Ground Rules for all TalkBank databases, which are based on the American Psychological Association Code of Ethics~\cite{american2002ethical}.

\subsection{Limitation \& Future Work} 
%This study represents a preliminary investigation into using multimodal LLMs to generate gesture-aware speech recognition transcripts. 
While the results are promising, we recognize several limitations and outline our plans to extend this work further.

One primary limitation is the absence of a definitive ground truth for quantitative evaluation. Since our model generates transcripts by synthesizing speech and gesture data from scratch, traditional benchmarks, such as comparisons with standard speech recognition outputs, are insufficient. Moreover, existing original transcripts lack gesture annotations, making direct comparisons challenging. In future work, we aim to address this gap by collaborating with certified pathologists to conduct qualitative assessments, such as A-B preference tests, to evaluate the effectiveness of gesture-enriched transcripts in accurately conveying the speaker's intentions.

To support quantitative evaluations, we plan to develop novel metrics that assess transcript quality, including grammar accuracy, semantic consistency, and the integration of multimodal information. Such metrics will provide a more objective basis for assessing our model's performance and facilitate comparisons with other multimodal and unimodal approaches.

Another limitation of this study is its focus on structured gestures from a specific task, the Peanut Butter Sandwich Task. While this task offers a controlled context for testing our approach, it does not encompass the diversity of gestures and communication patterns seen in everyday scenarios. As part of our future work, we plan to expand the scope of our model to include tasks such as the Cinderella Story Recall Task~\cite{bird1996cinderella}, which involves unstructured and complex narrative gestures. This expansion will allow us to evaluate the adaptability and robustness of our model in handling varied linguistic and gestural contexts.

In summary, while this study establishes a strong foundation for gesture-aware speech recognition, we aim to refine and extend our methods through collaborative qualitative evaluations, the development of robust quantitative metrics, and broader task applications. These efforts will ensure that our approach continues to evolve, ultimately contributing to more effective communication tools and interventions for individuals with language disorders.




\label{sec: conclusion}


%%%%%%%%%%%%%%%%%%%%%%%%%%%%%%%%%%%%%%%%%%%%%%%%%%%%%%%%%%%%

% In the unusual situation where you want a paper to appear in the
% references without citing it in the main text, use \nocite
% \nocite{langley00}

\bibliographystyle{plainnat}
\bibliography{reference}
% \bibliographystyle{plainnat}

\newpage
\appendix
\onecolumn

\begin{center}
	\LARGE \bf {Appendix}
\end{center}


\etocdepthtag.toc{mtappendix}
\etocsettagdepth{mtchapter}{none}
\etocsettagdepth{mtappendix}{subsubsection}
\tableofcontents
\newpage

% \section{Background}
\label{sec:background}


\subsection{Preliminaries}

{\color{red}[TODO: LLMs? in-context learning?]}

\subsection{Problem Definition}

{\color{red}[TODO: define the problem of citation intent]}

% \section{Brader Impacts}
% \label{app: impact}
% Considering the high sensitivity of GNNs to the oversmoothing issue and the depth of layers,
% it is important to develop GNNs that can performance well when the layers go deep, especially for realistic scenarios such as the social network where the scale of friends is getting bigger.
% By introducing the concept and theory of \oursfull (\ours), our work can serve as an initiate step towards tacking oversmoothing problem on graphs,
% with the hope to empower GNNs for broader applications and social benefits.
% Besides, this paper does not raise any ethical concerns.
% This study does not involve any human subjects, practices to data set releases, potentially harmful insights, methodologies and applications, potential conflicts of interest and sponsorship, discrimination/bias/fairness concerns, privacy and security issues, legal compliance, and research integrity issues.

% \section{Discussions on limitations of $\ours$ and future directions}
% \label{app_sec: limiatation}
% Although we have conducted comprehensive theoretical analysis and extensive experiments, there are still aspects requiring further investigation. 

% \paragraph{Theoretical limitation.}
% Our theory is based on asymptotic behaviors, it remains to be explored under non-asymptotic conditions.
% We use the classic signed graph theory---structural balance theory, to capture the ideal distribution of the positive and negative edges and further inspire Label-\ours and Feature-\ours, but it still maintains a gap between the theory and the practice as shown that the performance still decreases when the layers go to deepest (e.g., $64$ for GCN~\citep{gcn}, $300$ for SGC~\citep{sgc}).
% % Additionally, we have not fully captured the relationship between the previous methods for alleviating the oversmoothing and the signed graph propagation through the theoretical perspective. 
% Moreover, the label rate for the influence on test performance necessitates further theoretical analysis.

% \paragraph{More sophisticated architectures/parameter tunning.}
% The Label-\ours and Feature-\ours can have multiple implementations.
% We choose the most classic architectures GCN and the linear architecture SGC in our experiments for the purpose of concept verification.
% Moreover, as shown in Appendix \ref{app: exp}, Label-\ours and Feature-\ours still requires certain additional tunning efforts for the objectives.
% Hence we believe it is also a promising future direction to reduce the parameter tunning by leveraging better optimization techniques.

% \paragraph{Better signed graph design.}
% Typical signed graph contains edges with either positive or negative signs~\cite{signedgraph,yan2022two,LRGNN,acmp}.
% Moreover, the value of every edge can be extended from a concrete number, such as $0,1,-1$, to a continuous number.
% Since our implementation of Label-\ours in this work aims to verify the theoretical findings, we do not apply sophisticated edge assignments during the signed graph propagation, simply using the label-enhanced negative subgraph and the feature similarities enhanced negative subgraph while maintaining the positive subgraph as the adjacency matrix.
% Nevertheless, it is promising to leverage better positive and negative subgraphs to improve the performance of the signed graph propagation to alleviate oversmoothing. 
\begin{figure*}
    \centering
    \includegraphics[width=1.0\linewidth]{figures/csbm-b-2.pdf}
    \caption{ Figure (a)-(d) shows the effect of negative graph weight $\beta$ by \ours on CSBM. In all cases, $\lambda=0.5$ and $\alpha=1$. The X-axis is the $\beta$ and the Y-axis is the test accuracy. 
    $\phi$ is the hyperparameter to control the level of homophily and $H(G)$ measure the homophily level. 
    SBP1 indicates Label-\ours and SBP2 indicates Feature-\ours. }
    \label{fig:beta csbm}
    % \vspace{-0.1in}
\end{figure*}



\section{Related Work}

\paragraph{Theory of Oversmoothing}
The concept of oversmoothing was initially introduced by \cite{oversmooth_first}: when the number of layers becomes large, the representations of different nodes tend to converge to a common value after excessively exchanging messages with neighboring nodes. 
\cite{Oono2019GraphNN, wu2023demystifying} rigorously show that the convergence of node representations to a common value happens at an exponential rate as the number of layers increases to infinity, for GCNs and attention-based GNNs, respectively. 
% \cite{zhou2020graph} shows that under specific conditions, that the ultimate convergence point solely encodes information about the graph's structure.
\cite{sbm_xinyi}~theoretically proves that oversmoothing can start to happen even in shallow depth under certain random graph settings.
\jq{\cite{zhou2021dirichlet} proposed an appropriate residual connection according to the lower limit of Dirichlet energy and connected to previous methods qualitatively.}

\paragraph{Signed Graph Inspired Methods}
In the heterophilic graphs, various methods are inspired by the signed graph propagation~\citep{H2GNN,orderedgnn, yan2022two,acmp, GRP-GNN}. In particular,
\citet{yan2022two,acmp} utilize the idea that the negative edges denote connections between nodes that are "not similar to each other" to create repulsion between them during message-passing.
\cite{GRP-GNN} extend the coefficients of the output of different layers in the final aggregation to be learnable and find that the odd layer coefficients tends to be negative for heterophilic graphs, suggesting that learning naturally finds signed-graph message-passing. 
However, \cite{signremedy} show that under some specific random graph settings, the oversmoothing will even happen under signed graph propagation. 
% which aligns with the case in our analysis for Theorem~\ref{thm: small nega} when the repulsion among nodes are not sufficient. 
Nevertheless, we extend the theory to generic graphs and prove that in the ideal state---structural balance, signed edges can indeed serve as a remedy to effectively combat oversmoothing. 

\paragraph{Structural Balance}
Structual balance theory has gained significant attention in recent years~\citep{signedgraph,yan2022two,LRGNN,acmp}. 
%
Inspired by the structural balance theory, \cite{signedgraph} characterizes the balanced path intuitively to learn both balanced and unbalanced representations on each layer.
%
\cite{LRGNN} predicts the signed adjacency matrix by an off-the-shelf neural network classifier to generate pseudo labels with the low-rank assumption.
\cite{signed_dynamics_paper_review} introduces the definition of the Laplacian for signed graphs and develops a comprehensive mathematical theory.
%
In this paper, we rigorously show that structural balance is the theoretical solution to alleviate oversmoothing and propose practical methods based on the property without any additional learnable parameters. 


% In addition to the above methods which explicitly make use of the signed graph propagation, in this paper, we also revisit a wide class of previous anti-oversmoothing methods that do not explicitly claim to use signed message-passing. We find that all of them can be attributed to some kind of design of negative edges to the original graph.
% a theoretical analysis that prioritizes the significance of signed graphs over unsigned graphs, by leveraging the principles of structural balance theory.

% Introduce signed graph and related GNN papers using signed graph. Discuss similar works to ours that also analyze from attractive and repulsive forces. Our key innovation: the formulation and the analysis based on SB.\yf{refined as above}

% On the empirical side, the Simplified Graph Convolution Network (SGC)~\cite{sgc} attributes oversmoothing to non-linear operations and suggests removing non-linear activations, resulting in \(H_{(k)} = Z_{(k-1)}W_{(k-1)}\). Other methods address oversmoothing by incorporating normalization operations, such as LayerNorm~\cite{layernorm}, BatchNorm~\cite{batchnorm}, PairNorm~\cite{pairnorm}, and ContraNorm~\cite{contranorm}. Additionally, DropEdge~\cite{dropedge} mitigates oversmoothing by randomly dropping a percentage \(p\) of edges in the graph, while residual connections~\cite{Chen2020SimpleAD, appap} between the initial and current layers' features further alleviate the issue.

% \textbf{Signed Graph Network.}
% Over the past few years, various signed network models have been proposed~\cite{LRGNN,acmp,yan2022two} while each with notable limitations. 
% %
% \cite{LRGNN} attempts to merge features and the adjacency matrix into a low-rank signed graph using a complex optimization algorithm, which is computationally intensive and difficult to scale. 
% %
% \cite{yan2022two} explores the interaction between oversmoothing and heterophily but limits its approach to merely splitting the cosine similarity matrix into positive and negative matrices based on signs, which might oversimplify the complex graph structure interactions. 
% %
% Additionally, \cite{acmp} addresses signed graph behavior under oversmoothing in continuous, finite GNN layers, but their findings are restricted to limited-layer scenarios. 
% %
% In this paper, we propose a unified signed graph framework that revisits previous methods for mitigating oversmoothing and extends these theories to more generalized, infinite situations, thereby overcoming the limitations identified in earlier research.

% Nevertheless, this form of aggregation treats neighboring nodes indiscriminately across various classes, leading to a decline in node classification accuracy with increasing layer depth, known as the phenomenon of \textit{oversmoothing}.

% \textbf{Oversmoothing.}
% We give he definition of oversmoothing inspired by \cite{graph_oversmoothing_survey}.
% \begin{definition}[Oversmoothing]
%     Let $G$ be an undirected, connected graph and node features $X \in \mathbb{R}^{n \times d}$. We call $\mu : \mathbb{R}^{n \times d} \rightarrow \mathbb{R}_{\geq 0}$ a node-similarity measure if it satisfies the following axioms:
%     \begin{enumerate}
%         \item $\exists c \in \mathbb{R}^d$ with $X_u = c$ for all nodes $u \in V \iff \mu(X) = 0$, for $X \in \mathbb{R}^{n \times d}$
%         \item $\mu(X + Y) \leq \mu(X) + \mu(Y)$, for all $X, Y \in \mathbb{R}^{v \times m}$
%     \end{enumerate}
% \end{definition}
% % \todo{introduce similarity and variance measures and figures.}

% \begin{proposition}

% \end{proposition}


\section{More Discussion on GNNs}
\label{app: GNNs}

\subsection{Message-passing Graph Neural Networks (MP-GNNs)}
Let $\mathcal{G}=(A, X)$ denote a graph with $n$ nodes and $m$ edges, where $A \in \{0,1\}^{n\times n}$ is the adjacency matrix, and $X\in \mathbb{R}^{n \times d}$ is the node feature matrix with a node feature dimension of $d$.
Usually, we will transform the concrete adjacency matrix $A$ to the normalized adjacency matrix $\hat{A}$ by the degree matrix.
Define $D=diag(d_1,d_2, \dots, d_n)$ where $d_i$ is the degree of the node $i$.
Then the normalized adjacency matrix $\hat{A}=D^{-\frac{1}{2}}AD^{-\frac{1}{2}}$.
Moreover, many theoretical works simplified the normalized adjacency matrix to be $D^{-1}A$ or $AD^{-1}$ as the raw-normalized or column-normalized stochastic matrix where the sum of every raw (column) is $1$ and every entry is non-negative. 
In this paper, we use $\hat{A}=D^{-\frac{1}{2}}AD^{-\frac{1}{2}}$.

Different GNNs typically share a common propagation mechanism, where node features are aggregated and transformed along the network's topology to a certain depth.
The $k$-th layer propagation can be formalized as
\begin{equation}
        \label{Ana_eq_propagation_1}
	\begin{aligned}
	H_{(k)} = \textbf{PROPAGATE}(X; \mathcal{G}; k) =\bigg\langle\textbf{\textit{Trans}}\Big(\textbf{\textit{Agg}}\big\{\mathcal{G};H_{(k-1)}\big\}\Big)\bigg\rangle_{k},
	\end{aligned}
\end{equation}
with $H_{(0)} = X$ and $H_{(k)}$ is the output after the $k$-layer propagation. 
The notation \(\langle \rangle_{k}\) generally varies from GNN models and denotes the generalized combination operation following \(k\) convolutions. 
%
\(\textbf{\textit{Agg}}\{\mathcal{G}; H_{(k-1)}\}\) refers to aggregating the $(k-1)$-layer output $\textbf{H}^{(k-1)}$ along graph $\mathcal{G}$. 
%
Meanwhile, \(\textbf{\textit{Trans}}(\cdot)\) is the corresponding layer-wise feature transformation which often includes a non-linear activation function (e.g., ReLU) and a layer-specific learnable weight matrix \(W\) for transformation

\subsection{GCN}
To deal with non-Euclidean graph data, GCNs are proposed for direct convolution operation over graph, and have drawn interests from various domains. GCNisfirstly introduced for a spectral perspective~\cite{gcn}, but soon it becomes popular as a general message-passing algorithm in the spatial domain.
In the feature transformation stage, GCN adopts a non-linear activation function (e.g., ReLU) and a layer-specific learnable weight matrix \(W\) for transformation.
The propagation rule of GCN can formulated as follow:
\begin{equation}
    H_{(k)} = ReLU((\hat{A}H_{(k-1)})W_{(k)})
\end{equation}

\subsection{SGC}
SGC~\cite{sgc} simplifies and separates the two stages of GCNs: feature propagation and (non-linear) feature transformation. 
It finds that utilizing only a simple logistic regression after feature propagation (removing the non-linearities), which makes it a linear GCN, can obtain comparable performance to canonical GCNs. 
The propagation rule of GCN can formulated as follow:
\begin{equation}
    H_{(k)} = \hat{A}H_{(k-1)})W_{(k)}=\hat{A}^{k}H_{(0)})W_{(k)}...W_{(1)}
\end{equation}
Moreover, SGC transforms $W_{(k)}...W_{(1)}$ to a general learnable parameter $W$, so the formula of SGC can be this:
\begin{equation}
    H_{(k)} = \hat{A}^{k}H_{(0)})W
\end{equation}





\section{More Background about Signed Graph}
\label{app: signed graph}

\subsection{Signed Graph Propagation}
Classical GNNs~\citep{gcn,sgc,gat,gin} primarily focused on message-passing on unsigned graphs or graphs composed solely of positive edges.
For example, if there exists a edge $\{i,j\}$ or the sign of edge $\{i,j\}$ is positive, the node $x_i$ updates its value by:
\begin{equation}
\label{app_eq: node attractive}
    \hat{x}_i = x_i + \alpha(x_j-x_i) = (1-\alpha) x_i + \alpha x_j, \alpha \in (0,1). 
\end{equation}
Compared to the unsigned graph, a signed graph extends the edges to either positive or negative.
Notably, if the sign of edge $\{i,j\}$ is negative, the node $x_i$ update its value using the following expression:
% \yf{need to motivate this method. eg we observe that existing techniques can be cast as xx}
\begin{equation}
\label{app_eq: node repel}
    \hat{x}_i = x_i -\beta (x_j-x_i) = (1+\beta) x_i -\beta x_j, \beta \in (0,1).
\end{equation}
In words, the positive interaction~\eqref{app_eq: node attractive} 
indicates the attraction while the negative interaction~\eqref{app_eq: node repel} 
indicates that the nodes will repel their neighbors.

More generally, when considering all of the neighbors of node $x_i$, let $N_i^+$ denote the positive neighbor set while $N_i^-$ denote the negative neighbor set, where $N_i^+ \cup N_i^-= N_i$ and $N_i^+ \cap N_i^-= \emptyset$.
The representation of $x_i$ is thus updated by: 
\begin{equation}
\label{app_eq: sign_node}
    \hat{x}_i = (1-\alpha + \beta) x_i + \frac{\alpha}{|N_i^+|}\sum_{j\in N_i^+}x_j
    -\frac{\beta}{|N_i^-|} \sum_{j\in N_i^-}x_j\,.
\end{equation}
In particular, the two parameters $\alpha$ and $\beta$ mark the strength of positive and negative edges, respectively.
Furthermore, the signed propagation rule~\eqref{app_eq: sign_node} from a single node can be generalized  
over the whole graph $\mathcal{G}$ written in the matrix update form as:
\begin{equation}
\label{app_eq: sign_overall}
    \hat{X} = (1-\alpha + \beta) X + \alpha \pgh{\hat{A}^+ }X - \beta\ngh{\hat{A}^- }X,
\end{equation}
where $\hat{A}^+$ is the raw normalized version of the positive adjacency matrix $A^+ \in \{0,1\}^{n \times n}$ and $\hat{A}^-$ is that of the negative adjacency matrix $A^- \in \{0,1\}^{n \times n}$.


% \section{Details of Signed Graph Framework}

\subsection{Definition of negative graph}
\label{app_sec: negative graph}
For further proofs of the theorems and propositions in the paper, we give a more simple and detailed definition in this section.

Let \(D_{G^+} = \text{diag}(deg_1^+, \ldots, deg_n^+)\) and \(D_{G^-} = \text{diag}(deg_1^-, \ldots, deg_n^-)\) be the degree matrices of the positive subgraph and negative subgraph, respectively. 
%
Let \(A_{G^+}\) be the adjacency matrix of the graph \(G^+\) with \([A_{G^+}]_{ij} = 1\) if \(\{i, j\} \in E^+\) and \([A_{G^+}]_{ij} = 0\) otherwise. 
%
The adjacency matrix \(A_{G^-}\) of the negative subgraph \(G^-\) is defined by \([A_{G^-}]_{ij} = -1\)  for \(\{i, j\} \in E^-\) and \([A_{G^-}]_{ij} = 0\) for \(\{i, j\} \not\in E^-\).

The Laplacian plays a central role in the algebraic representation of structural properties of graphs. 
%
% \jq{ours is $(D+I)^{-\frac{1}{2}}(A+I)(D+I)^{-\frac{1}{2}}$, no form like $D-A$.}
%
In the presence of negative edges, more than one definition of Laplacian is possible; see \cite{signed_dynamics_paper_review}. 
The Laplacian of the positive subgraph \(G^+\) is \(L_{G^+} := D_{G^+} - A_{G^+}\), while for the negative subgraph \(G^-\) the following two variants can be used: \(L_{G^-}^o := D_{G^-} - A_{G^-}\) and \(L_{G^-}^r := -D_{G^-} - A_{G^-}\). 
Consequently, we have the following definitions.

{Definition 1.} Given the signed graph \(G\), its opposing Laplacian is defined as
\begin{equation}
L_{G}^o := L_{G^+} + L_{G^-}^o = D_{G^+} + D_{G^-} - A_{G^+} - A_{G^-},
\end{equation}
and its repelling Laplacian is defined as
\begin{equation}
L_{G}^r = L_{G^+} + L_{G^-}^r = D_{G^+} - D_{G^-} - A_{G^+} - A_{G^-}.
\end{equation}


\subsection{Positive / Negative Interaction}
% \jq{our method doesn't have the same coefficient}

Time is slotted at \(t = 0, 1, \ldots\). 
Each node \(i\) holds a state \(x_i(t) \in \Omega\) at time \(t\) and interacts with its neighbors at each time to revise its state. 
The interaction rule is specified by the sign of the links. 
Let \(\alpha, \beta \geq 0\). 
We first focus on a particular link \(\{i, j\} \in E\) and specify for the moment the dynamics along this link isolating all other interactions.

The DeGroot Rule:
\begin{equation}
    x_s(t + 1) = x_s(t) + \alpha(x_{-s}(t) - x_s(t)) = (1 - \alpha)x_s(t) + \alpha x_{-s}(t),
\end{equation}
where \(-s \in \{i, j\} \setminus \{s\}\) with \(\alpha \in (0, 1)\)

The Opposing Rule:
\begin{equation}
    x_s(t + 1) = x_s(t) + \beta(-x_{-s}(t) - x_s(t)) = (1 - \beta)x_s(t) - \beta x_{-s}(t);
\end{equation}
or
The Repelling Rule:
\begin{equation}
    x_s(t + 1) = x_s(t) - \beta(x_{-s}(t) - x_s(t)) = (1 + \beta)x_s(t) - \beta x_{-s}(t).
\end{equation}



\subsection{Deterministic Networks}
% \jq{our method is more like the repelling negative dynamics intuitively.}

The Repelling Negative Dynamics:
\begin{equation}
\label{eq: repell_neg}
\begin{split}
    x_i(t + 1) &= x_i(t) + \alpha \sum_{j \in N_i^+} (x_j(t) - x_i(t)) - \beta \sum_{j \in N_i^-} (x_j(t) - x_i(t)) \\
    &= (1 - \alpha deg_i^+ + \beta deg_i^-)x_i(t) + \alpha \sum_{j \in N_i^+} x_j(t) - \beta \sum_{j \in N_i^-} x_j(t).
\end{split}
\end{equation}

Denote \(\bold{x}(t) = (x_1(t) \ldots x_n(t))^T\). We can now rewrite \ref{eq: repell_neg} in the compact form

\begin{equation}
\label{eq: over_repell}
\bold{x}(t + 1) = M_{G} \bold{x}(t) = (I - \alpha L_{G_+} - \beta L_{G_-}^r)\bold{x}(t).
\end{equation}
Here,
\begin{equation}
    M_G = I - \alpha L_{G^+} - \beta L_{G^-}^r = I - L_{G}^{rw},
\end{equation}
with \(L_{G}^{rw} = \alpha L_{G^+} + \beta L_{G^-}^r\) being the repelling weighted Laplacian of \(G\). 
From Equation \ref{eq: over_repell}, \(M_G \mathbf{1} = \mathbf{1}\) always holds. 
We present the following result, which by itself is merely a straightforward look into the spectrum of the repelling Laplacian \(L_{G}^{rw}\).

Note that our~\eqref{eq: sign_overall} is consistent with Equation~\eqref{eq: repell_neg}, only need to replace the $\alpha$ and $\beta$ with $\frac{\alpha}{deg_i^+}$ and $\frac{\beta}{deg_i^-}$ respectively.


\label{sec: exp detail}



% \section{Theorem of Signed graph }
\section{Analysis of Previous methods via Signed Graph}
\label{app: previous}

\subsection{Discussion of Normalization}
\label{sec: prof of norm}
\paragraph{BatchNorm} 
% across different nodes in each feature dimension. 
% The update rule of BatchNorm on node representation $x_i$ 
% prevents the denominator from becoming zero 
BatchNorm centers the node representations $X$ to zero mean and unit variance  and can be written as BatchNorm($x_i$) \(=\frac{1}{\sqrt{\sigma^2 + \epsilon}}(x_i - \frac{1}{n}\Sigma_{i=1}^n x_i)\), where $ \epsilon > 0$ 
and $\sigma^2$ is the variance of node features.
We rewrite BatchNorm in the signed graph propagation form as follows: 
\begin{equation}
    \label{eq: bn sign}
    \hat{X}= \pgh{\hat{A}}  X \Gamma_d^{-1}  - \ngh{\frac{\mathbb{1}_n \mathbb{1}_n^T}{n}  \hat{A}} X \Gamma_d^{-1} = \pgh{\hat{A}}  \tilde{X}-\ngh{\frac{\mathbb{1}_n \mathbb{1}_n^T}{n}  \hat{A}} \tilde{X}\,,
\end{equation}
where $\Gamma_d = \diag(\sigma_1,\dots,\sigma_d)$ is a diagonal matrix that represents column-wise variance with $\sigma_i^2=\frac{1}{n}\sum_{j=1}^n ((\hat{A} X)_{_{ji}}- \mathbb{1}_n^\top \hat{A} X/n)^2$, and
$\tilde{X}= X \Gamma_d^{-1}$ is a normalized version of $X$.
We can correspond to the positive graph $\pgh{A^+}$ to $\pgh{\hat{A}}$ and the negative graph $\ngh{A^-}$ to $\ngh{\frac{\mathbb{1}_n \mathbb{1}_n^T}{n} \hat{A}}$ in Eq. \eqref{eq: bn sign}.
% Through the repulsion mechanism introduced by the negative edges, BatchNorm can mitigate the problem of oversmoothing as suggested by Theorem~\ref{thm: connected positive graph}.
% \yf{we should discuss some limitations here}

% Consider the update:
% \begin{equation}
%     \label{eq: bn sign}
%     \hat{X}= (\pgh{A}-\ngh{\frac{\mathbb{1}_n \mathbb{1}_n^T}{n} A}) X\,,
% \end{equation}
% After one signed graph propagation, the edge weight changes from $\{0,1\}$ to $\{-\frac{p+q}{2}, 1- \frac{p+q}{2}\}$, so the SB can be expressed as: 
% \begin{equation}
%     SB_{BN}= (1-\frac{p+q}{2})p + \frac{p+q}{2} (1-q) = p + \frac{p+q}{2} (1-p-q).
% \end{equation}

\paragraph{PairNorm} 
We then introduce another method called PairNorm where the only difference between it and BatchNorm is that PairNorm scales all the entries in $X$ using the same number rather than scaling each column by its own variance.
The formulation of PairNorm can be rewritten as follows:
\begin{equation}
    \label{eq: pn sign}
    \hat{X} = \frac{1}{\Gamma}\pgh{\hat{A}}  X   -  \frac{1}{\Gamma} \ngh{\frac{\mathbb{1}_n \mathbb{1}_n^T}{n}  \hat{A}} X
    =\frac{1}{\Gamma}(\pgh{\hat{A}} X-\ngh{\frac{\mathbb{1}_n \mathbb{1}_n^T}{n}  \hat{A}} X) \,,
\end{equation}
where $\Gamma = \|(\hat{A}- \mathbb{1}_n \mathbb{1}_n^T/n)X \|_F/\sqrt{n} $. 
% Comparing~\eqref{eq: pn sign} to~\eqref{eq: bn sign}, 
We observe that PairNorm shares the same positive and negative graphs (up to scale) as BatchNorm.
Another normalization technique, ContraNorm, turns out to extend the negative graph to an adaptive one based on node feature similarities. 
% which reveals that its effectiveness in alleviating oversmoothing is actually attributed to a mechanism very similar to BatchNorm.

% Through the repulsion mechanism introduced by the negative edges, BatchNorm can mitigate the problem of oversmoothing as suggested by Theorem~\ref{thm: connected positive graph}.
% Meanwhile, from \eqref{eq: bn sign} and~\eqref{eq: pn sign}, we see that BatchNorm and PairNorm inject the negative graph  $-(\mathbb{1}_n \mathbb{1}_n^T/n)  \hat{A}$ through a constant transformation of the positive graph derived from the original graph structure $\hat{A}$.



\paragraph{ContraNorm}
ContraNorm is inspired by the uniformity loss from contrastive learning, aiming to alleviate dimensional feature collapse.
For simplicity, we consider the spectral version of ContraNorm
% Proposition 2) 
% \xinyic{better with an exact reference: which Theorem in ContraNorm} 
% without additional regularization and LayerNorm that are used additionally in practice, 
that takes the following form:
\begin{equation}
    \label{eq: contra sign}
    \hat{X} = (1 + \alpha) \pgh{\hat{A}}X- \alpha /\tau \ngh{(X X^{T}) \hat{A} } X \,,
    % = ((1 + \alpha) \pgh{\hat{A}}- \alpha /\tau \ngh{(X X^{T}) \hat{A} }) X
\end{equation}
where $\alpha\in(0,1)$ and $\tau>0$ are hyperparameters.
We can see that $\pgh{\hat{A}}$ is again the positive graph and $\ngh{(X X^T)\hat{A}}$ is the negative graph in the corresponding signed graph propagation.

\begin{proposition}
    Consider the update:
    \begin{equation}
        \hat{X} = \pgh{A}X-\ngh{\frac{\mathbb{1}_n \mathbb{1}_n^T}{n}A}X,
    \end{equation}
    where $A\in \{0,1\}^{n \times n}$ is the adjacency matrix. Define the overall signed graph adjacency matrix $A_s$ as $A-\frac{\mathbb{1}_n \mathbb{1}_n^T}{n}A$. Then we have that the signed graph is (weakly) structurally balanced only if the original graph can be divided into several isolated complete subgraphs. 
\end{proposition}

\paragraph{Proof.} Assume that there is no isolated node and no node has edges with all the other nodes.
    $(A_s)_{i,j}=(A)_{i,j}-\frac{deg_j}{n}$.
    If $(A)_{i,j}=1$, then we have $(A_s)_{i,j}>0$.
    If $(A)_{i,j}=0$, then we have $(A_s)_{i,j}<0$.
    
    If the nodes can be divided into several isolated complete subgraphs, then the nodes set $V=V_1\cup V_2 \dots V_m$, where $|V_i|>1$, $m$ is the number of the isolated complete subgraphs. 
    So only the nodes within the same set have edges, thus relative entries of $A_s>0$, while nodes from different sets do not, thus relative entries of $A_s <0$.
    
    On the other hand, if $A_s$ is (weakly) structurally balanced, then the nodes set can be expressed as $V=V_1\cup V_2 \dots V_k$, where $|V_i|>1$, $k$ is the number of the separated parties in the signed graph. 
    The entry of $A_s$ in the same parties is positive, while between different parties is negative.
    According to $(A_s)_{i,j}=(A)_{i,j}-\frac{deg_j}{n}$, we know that nodes in the same parties are connected in the original graph while not connected in the original graph between different parties.
    So the graph can be divided into several isolated complete subgraphs.

    Overall, the signed graph is (weakly) structurally balanced only if the original graph can be divided into several isolated complete subgraphs, the proof is over. 

The Proposition shows that in order for the structural balance property to hold for the signed graph of normalization, the graph needs to satisfy an unrealistic condition where the edges strictly cluster the nodes.

\paragraph{Discussion of ContraNorm}
% \label{sec: prof of contra}
% \jq{not very sure how to give the detailed expression of the theorem and proof.}
% \begin{proposition}
    Consider the update:
    \begin{equation}
    \label{app_eq: contra theory sign}
        \hat{X} = \pgh{A}X-\ngh{\frac{XX^T}{n}A}X,
    \end{equation}
    Define the overall signed graph adjacency matrix $A_s = A-\frac{XX^T}{n}A$ where $(A_s)_{i,j}=(A)_{i,j}- \frac{1}{n}\Sigma_{k=1}^n x_ix_k^T(A)_{k,j}$ . 
    % Then we said that under the signed propagation \ref{eq: contra theory sign}, the node features will not converge to a constant $C$ with any initial nodes X(0).
% \end{proposition}

% \paragraph{Proof.} 
Assume that the nodes feature is normalized every update, that is $||x_i||_2=1$ for every $i$.

If $(A)_{i,j}=1$, then we have that
\begin{equation}
\begin{aligned}
    (A_s)_{i,j}&=(A)_{i,j}- \frac{1}{n}\Sigma_{k=1}^n x_ix_k^T(A)_{k,j}\\
    &=1-\frac{1}{n}\Sigma_{k=1}^n x_ix_k^T(A)_{k,j}\\
    &>1-\frac{1}{n}\Sigma_{k=1}^n(A)_{k,j}\\
    &=1-\frac{d_j}{n}>0.\\
\end{aligned}
\end{equation}
That means if $(A)_{i,j}=1$, then  $(A_s)_{i,j}>0$.
However, if $(A)_{i,j}=0$, then we have that
\begin{equation}
    \begin{aligned}
        (A_s)_{i,j}&=(A)_{i,j}- \frac{1}{n}\Sigma_{k=1}^n x_ix_k^T(A)_{k,j}\\
        &= -\frac{1}{n}\Sigma_{k=1}^n x_ix_k^T(A)_{k,j}\\
        &= -\frac{1}{n} \Sigma_{k \in N_j}x_ix_k^T.
    \end{aligned}
\end{equation}

Intuitively, if $x_i$ has similar features to $x_j$'s neighbors, then we have that $(A_s)_{i,j}<0$, which means trying to repel nodes with similar representations. 
If $x_i$ has different features to $x_j$'s neighbors, then we have that $(A_s)_{i,j}>0$, which means trying to aggregate nodes with original different representations. 

If graph $G$ is a completed graph, then all entries of $(A_s)>0$, however, when all of the nodes coverage to each other, $\Sigma_{k=1}^n x_ix_k^T(A)_{k,j}=\Sigma_{k=1}^n x_ix_k^T$ will also become bigger.

% Moreover, if all of the node features converge to $C\in \mathbb{R}^d$, the l2-norm of each node feature will be $||C^2||=1$, then we have that:
% \begin{equation}
% (A_s)_{i,j}=  
% \left\{
% \begin{array}{l}
%   1-\frac{1}{n} \Sigma_{k \in N_j}x_ix_k^T, (A)_{i,j}= 1; \\
%  -\frac{1}{n} \Sigma_{k \in N_j}x_ix_k^T, (A)_{i,j}= 0.
% \end{array}
% \right.
% \end{equation}
% Then, according to the Eq.\eqref{app_eq: contra theory sign}, we have that result of the updated node feature $\hat{x}_i$ as following:
% \begin{equation}
% \begin{aligned}
%     \hat{x}_i =& \Sigma_{j=1}^n (A_s)_{i,j} x_j\\
%     =& [\Sigma_{j\in \mathcal{N}_i}(1-\frac{1}{n} \Sigma_{k \in N_j}x_ix_k^T)-\Sigma_{j\notin \mathcal{N}_i}\frac{1}{n} \Sigma_{k \in N_j}x_ix_k^T]x_j\\
%     =& [d_i - \Sigma_{j\in \mathcal{N}_i}\frac{d_j}{n}-\Sigma_{j\notin \mathcal{N}_i}\frac{d_j}{n}]C\\
%     =& [d_i-\frac{2|E|}{n}]C\\
% \end{aligned}
% \end{equation}

% After $k$-th updates, we have that
% \begin{equation}
%     \hat{x}^{(k)}_i= [d_i-\frac{2|E|}{n}]^kC
% \end{equation}
% If there exists one node satisfied $|d_i-\frac{2|E|}{n} |>1$, then we have that $\hat{x}^{(2k)}_i \rightarrow \infty$, which is contradictory to the oversmoothing assumption.
% If all of the nodes satisfy $|d_i-\frac{2|E|}{n} |\leq 1$


\subsection{Discussion of Residual Connection}
\label{app: residual}
The standard residual connection~\citep{dgc,Chen2020SimpleAD} directly combines the previous and the current layer features together. It can be formulated as:
\begin{equation}
    \label{eq: residual sign}
     \hat{X} = (1-\alpha)X  + \alpha \hat{A} X = X + \alpha \pgh{\hat{A}} X -\alpha \ngh{I} X\,.
\end{equation} 
For residual connections, the positive adjacency matrix is $\pgh{\hat{A}}$ and the negative adjacency matrix $\ngh{I}$ in the corresponding signed graph propagation.
\paragraph{APPNP}
We reformulate the method APPNP~\citep{appap} as the signed propagation form of the initial node feature. 
Another propagation process is APPNP~\citep{appap} which can be viewed as a layer-wise graph convolution with a residual connection to the initial transformed
feature matrix $X^{(0)}$, expressed as: 
\begin{equation}
 \hat{X}^{(k+1)} = (1-\alpha)X^{(0)}  + \alpha \hat{A} X ^{(k)}.
\end{equation}
\begin{theorem}
With $\hat{A}^+=\Sigma_{i=0}^{k+1}\alpha^i\hat{A}^i$ and $\hat{A}^-=\alpha \Sigma_{j=0}^{k}\alpha^j\hat{A}^j$, the propagation process of APPNP following the signed graph propagation.
\end{theorem}
\textbf{Proof.}
Easily prove with mathematical induction.

In addition to combining with the last and initial layer features, the last type integrates several intermediate layer features. The established representations are JKNET~\citep{jknet} and DAGNN~\citep{dagnn}.
\paragraph{JKNET}
JKNET is a deep graph neural network which exploits information from neighborhoods of differing locality. 
JKNET selectively combines aggregations from different layers with Concatenation/Max-pooling/Attention at the output, i.e., the representations "jump" to the last layer.
Using attention mechanism for combination at the last layer, the $k+1$-layer propagation result of JKNET can be written as:
\begin{equation}
    \label{eq:jk-net}
    \begin{split}
         X^{(k+1)} &= \alpha_0 X^{(0)}  + \alpha_1  X ^{(1)} + \cdots \alpha_k X^{(k)}\\
        &= \Sigma_{i=0}^k\alpha_i \hat{A}^i X^{(0)}\,,
    \end{split}
\end{equation}
where $\alpha_0, \alpha_1, \cdots, \alpha_{k}$ are the learnable fusion weights with $\Sigma_{i=0}^k\alpha_i=1$.

\paragraph{DAGNN}
Deep Adaptive Graph Neural Networks (DAGNN)~\citep{dagnn} tries to adaptively add all the features from the previous layer to the current layer features with the additional learnable coefficients. 
After decoupling representation transformation and propagation, the propagation mechanism of DAGNN is similar to that of JKNET.
\begin{equation}
    \label{eq:dagnn}
         X^{(k+1)} = \Sigma_{i=0}^k\alpha_i \hat{A}^i H^{(0)}, \,H^{(0)}=f_\theta(X^{(0)})
\end{equation}
$ H^{(0)}=f_\theta(X^{(0)})$ ) is the non-linear feature transformation using an MLP
network, which is conducted before the propagation process and $\alpha_0, \alpha_1, \cdots, \alpha_{k}$ are the learnable fusion weights with $\Sigma_{i=0}^k\alpha_i=1$. 
\begin{theorem}
    With \pgh{$\hat{A}^+=\Sigma_{i=0}^{k-1}\alpha^i\hat{A}^i+\hat{A}^k$} and \ngh{$\hat{A}^-=\Sigma_{j=0}^{k-1}\alpha^j\hat{A}^k$}, the propagation process of JKNET and DAGNN following the signed graph propagation.
\end{theorem}
\textbf{Proof.}
Easily prove with mathematical induction.
% \jq{double check the correctness.}

As for more residual inspired methods~\citep{GCNII,wGCN,ACM-GCN,PDE-GCN}, we select GCNII and wGCN to give a detailed discussion as follows.
\begin{itemize}
    \item As for GCNII~\citep{GCNII}, it is an improved version of APPNP with the learnable coefficients $\alpha_i$ and changes the learnable weight W to $(1-\beta_i)I+\beta_i W$ each layer, so it shares the same positive and negative graph as APPNP.
    \item As for the wGCN~\citep{wGCN}, it incorporates trainable channel-wise weighting factors $\omega$ to learn and mix multiple smoothing and sharpening propagation operators at each layer, same as the init residual combines but change parameters $\alpha$ to be learnable with a more detailed selection strategy.
\end{itemize}


\subsection{Discussion of DropMessage}
For DropMessage~\citep{Fang2022DropMessageUR}, it is a unified way of DropNode, DropEdge and Dropout but with a more flexible mask strategy. We have discussed the DropNode and DropEdge in our paper. DropMessage can be viewed as randomly dropping some dimension of the aggregated node features instead of the whole node or the whole edge. 
We give the unified positive and negative graph of DropMessage in the term of the signed graph.
The propagation of DropMessage can be expressed as $H^{(k)}= AH^{(k-1)}-M_m,$ where if dropping $AH^{(k-1)}_{ij}$, then $M_{ij}=AH^{(k-1)}_{ij}$ else $M_{ij}=0$.







\section{Proof of Theorem~\ref{thm: small nega}}

Now consider the combined theorem. 

\begin{theorem}
\label{app: theorm_positive connected}
    Suppose that the positive edges are connected. Then along Equation \ref{eq: repell_neg} for any \(0 < \alpha < 1/\max_{i \in V} \deg_i^+\), there exists a critical value \(\beta_* \geq 0\) for \(\beta\) such that
    \begin{enumerate}
        \item[(i)] if \(\beta < \beta_*\), then we have \(\lim_{t \to \infty} x_i(t) = \sum_{j=1}^n x_j(0)/n\) for all initial values \(x(0)\);
        \item[(ii)] if \(\beta > \beta_*\), then \(\lim_{t \to \infty} \|x(t)\| = \infty\) for almost all initial values w.r.t. Lebesgue measure.
    \end{enumerate}
\end{theorem}

\paragraph{Proof.}
we change the signed graph update to the equivalent version of \(x_i(t)\) read as:
\[
x_i(t + 1) = x_i(t) + \alpha \sum_{j \in N_i^+} (x_j(t) - x_i(t)) - \beta \sum_{j \in N_i^-} (x_j(t) - x_i(t)).
\]
This can be expressed as:
\begin{equation}
\label{appendix_eq: nege_node}
    x(t + 1) = (1 - \alpha \deg^+ + \beta \deg^-) x_i(t) + \alpha \sum_{j \in N^+} x_j(t) - \beta \sum_{j \in N^-} x_j(t).
\end{equation}


Algorithm \ref{appendix_eq: nege_node} can be written as:
\begin{equation}
\label{appendix_eq: nege_graph}
    x(t + 1) = M_G x(t) = (I - \alpha L_G^+ - \beta L_G^-) x(t).
\end{equation}


Here, \(M_G = I - \alpha L_G^+ - \beta L_G^-\), with \(L_G^+ = \alpha L_C^+ + \beta L_C^-\) being the repelling weighted Laplacian of \(G\), defined in Sec.\ref{app_sec: negative graph}.  
From Eq.\eqref{appendix_eq: nege_graph}, \(M_G \mathbf{1}= \mathbf{1}\) always holds. We present the following result, which by itself is merely a straightforward look into the spectrum of the repelling Laplacian \(L_G^-\).

So theorem \ref{app: theorm_positive connected} can be changed to the following version:

 Suppose \(G^+\) is connected. Then along Eq.\eqref{appendix_eq: nege_graph} for any \(0 < \alpha < 1/\max_{i \in V} \deg_i^+\), there exists a critical value \(\beta > 0\) for \(\beta\) such that:
\begin{enumerate}
    \item[(i)] if \(\beta < \beta_*\), then average consensus is reached in the sense that \(\lim_{t \to \infty} x_i(t) = \frac{1}{n} \sum_{j=1}^n x_j(0)\) for all initial values \(x(0)\);
    \item[(ii)] if \(\beta > \beta_*\), then \(\lim_{t \to \infty} \|x(t)\| = \infty\) for almost all initial values w.r.t. Lebesgue measure.
\end{enumerate}

\textbf{Proof.}
Define \(J = 11^T/n\). Fix \(\alpha \in (0,1/\max_{i \in V} \deg_i^+)\) and consider \(f(\beta) = \lambda_{\max}(I - \alpha L_G^+ - \beta L_G^- - J)\), and \(g(\beta) = \lambda_{\min}(I - \alpha L_G^+ - \beta L_G^- - J)\). The Courant–Fischer Theorem  implies that both \(f(\cdot)\) and \(g(\cdot)\) are continuous and nondecreasing functions over \([0, \infty)\). The matrix \(J\) always commutes with \(I - \alpha L_G^+ - \beta L_G^-\), and 1 is the only nonzero eigenvalue of \(J\). Moreover, the eigenvalue 1 of \(J\) shares a common eigenvector 1 with the eigenvalue 1 of \(I - \alpha L_G^+ - \beta L_G^-\).

Since \(G^+\) is connected, the second smallest eigenvalue of \(L_{G^+}\) is positive. Since \(0 < \alpha < \frac{1}{\max_{i \in V} \deg^+_i}\), there holds \(\lambda_{\min}(I - \alpha L_{G^+}) \geq -1\), again due to the Gershgorin Circle Theorem. Therefore, \(f(0) < 1\), \(g(0) \geq -1\). Noticing \(f(\infty) = \infty > 1\), there exists \(\beta_* > 0\) satisfying \(f(\beta_*) = 1\). We can then verify the following facts:
\begin{itemize}
  \item There hold \(f(\beta) < 1\) and \(g(\beta) > -1\) if \(\beta < \beta_*\). In this case, along Eq.\eqref{appendix_eq: nege_graph} \(\lim_{t \to \infty} (I - J)x(t) = 0\), which in turn implies that \(x(t)\) converges to the eigenspace corresponding to the eigenvalue 1 of \(M_{G}\). This leads to the average consensus statement in (i).
  \item There holds \(f(\beta) \geq 1\) if \(\beta > \beta_*\). 
  In this case, along Eq.\eqref{appendix_eq: nege_graph} \(x(t)\) diverges as long as the initial value \(x(0)\) has a nonzero projection onto the eigenspace corresponding to \(\lambda_{\max}(M_{G})\) of \(M_{G}\). 
  This leads to the almost everywhere divergence statement in (ii).
\end{itemize}
The proof is now complete.

\section{Proof of Theorem \ref{thm: repel_struct}}

\begin{theorem}
let \( A > 0 \) be a constant and define \( \mathcal{F}(z)_c \) by \( \mathcal{F}(z)_c = -c, z < -c \), \( \mathcal{F}(z)_c = z, z \in [-c, c] \), and \( \mathcal{F}(z)_c = c, z > c \). Define the function \( \theta : E \to \mathbb{R} \) so that \( \theta(\{i,j\}) = \alpha \) if \( \{i,j\} \in E^+ \) and \( \theta(\{i,j\}) = -\beta \) if \( \{i,j\} \in E^- \). 
Assume that node \( i \) interacts with node \( j \) at time \( t \) and consider the following node interaction under the signed propagation rules:
\begin{equation}
\label{app_eq: repel dyn}
    x_s(t + 1) = \mathcal{F}(z)_c((1 - \theta)x_s(t) + \theta x_{-s}(t)), \ s \in \{i,j\}.
\end{equation}

let \(\alpha \in (0,1/2)\). Assume that \(G\) is a structurally balanced complete graph under the partition \(V = V_1 \cup V_2\). 
When \(\beta\) is sufficiently large, for almost all initial values \(x(0)\) w.r.t. Lebesgue measure, there exists a binary random variable \(l(x(0))\) taking values in \(\{-c,c\}\) such that
\begin{equation}
    \mathbb{P}\left(\lim_{t \to \infty} x_i(t) = l(x(0)), i \in V_1; \lim_{t \to \infty} x_i(t) = -l(x(0)), i \in V_2 \right) = 1.
\end{equation}
\end{theorem}


\paragraph{Proof.}
The proof is based on the following lemmas.

\begin{lemma}
\label{ap_lemma: 2 bound}
Fix \(\alpha \in (0, 1)\) with \(\alpha \neq \frac{1}{2}\). For the dynamics \ref{app_eq: repel dyn} with the random pair selection process, there exists \(\beta^*(\alpha) > 0\) such that
\[
\mathbb{P}\left(\limsup_{t \to \infty} \max_{i,j \in V} |x_i(t) - x_j(t)| = 2A\right) = 1
\]
for almost all initial beliefs if \(\beta > \beta^*\).
\end{lemma}

\begin{lemma}
\label{ap_lemma: appro }
    Fix $\alpha \in (1/2, 1)$ and $\beta \geq 2/(2\alpha - 1)$. Consider the dynamics \ref{app_eq: repel dyn} with the random pair selection process. Let $G$ be the complete graph with $\kappa(G^+) \geq 2$. Suppose for time $t$ there are $i_1, j_1 \in V$ with $x_{i_1}(t) = -c$ and $x_{j_1}(t) = c$. Then for any $\epsilon \in [0, (2\alpha - 1)c/2\alpha]$ and any $i_* \in V$, the following statements hold:
\begin{enumerate}
    \item[(i)] There exist an integer $Z(\epsilon)$ and a sequence of node pair realizations, $G_{t+s}(\omega)$, for $s = 0, 1, \dots, Z - 1$, under which $x_{i_*}(t + Z)(\omega) \leq -A + \epsilon$.
    \item[(ii)] There exist an integer $Z(\epsilon)$ and a sequence of node pair realizations, $G_{t+s}(\omega)$, for $s = 0, 1, \dots, Z - 1$, under which $x_{i_*}(t + Z)(\omega) \geq A - \epsilon$.
\end{enumerate}
\end{lemma} 


\textbf{Proof.} From our standing assumption, the negative graph $G^-$ contains at least one edge. Let $k_*, m_* \in V$ share a negative link. We assume the two nodes $i_1, j_1 \in V$ labeled in the lemma are different from \(k_*, m_*\), for ease of presentation. We can then analyze all possible sign patterns among the four nodes \(i_1, j_1, k_*, m_*\). We present here just the analysis for the case with
\[
\{i_1, k_*\} \in E^+, \{i_1, m_*\} \in E^+, \{j_1, k_*\} \in E^+, \{j_1, m_*\} \in E^+.
\]
The other cases are indeed simpler and can be studied via similar techniques.

Without loss of generality we let \(x_{m_*}(t) \geq x_{k_*}(t)\). First of all we select \(G_t = \{i_1, k_*\}\) and \(G_{t+1} = \{j_1, m_*\}\). It is then straightforward to verify that
\[
x_{m_*}(t + 2) \geq x_{k_*}(t + 2) + 2\alpha c.
\]
By selecting \(G_{t+2} = \{m_*, k_*\}\) we know from \(\beta \geq \frac{2}{(2\alpha - 1)} > \frac{1}{\alpha}\) that
\[
x_{k_*}(t + 3) = -c, \quad x_{m_*}(t + 3) = c.
\]
There will be two cases:
\begin{itemize}
    \item[(a)] Let \(i_* \in \{m_*, k_*\}\). Noting that \(\kappa(G^+) \geq 2\), there will be a path connecting to \(k_*\) from \(i_*\) without passing through \(m_*\) in \(G^+\). It is then obvious that we can select a finite number \(Z_1\) of links which alternate between \(\{m_*, k_*\}\) and the edges over that path so that \(x_{i_*}(t + 3 + Z_1) \geq -c + \epsilon\). Here \(Z_1\) depends only on \(\alpha\) and \(n\).
    \item[(b)] Let \(i_* \in \{m_*, k_*\}\). We only need to show that we can select pair realizations so that \(x_{m_*}\) can get close to \(-c\), and \(x_{k_*}\) gets close to \(c\) after \(t + 3\). Since \(G^+\) is connected, either \(m_*\) or \(k_*\) has at least one positive neighbor. For the moment assume \(m'\) is a positive neighbor of \(m_*\) and \(k'\) is a positive neighbor of \(k_*\) with \(m' \neq k'\). Then from part (a) we can select \(Z_2\) node pairs so that
    \[
    x_{m_*}(t + 3 + Z_2) \leq -c + \epsilon, \quad x_{k_*}(t + 3 + Z_2) \geq c - \epsilon.
    \]
\end{itemize}
Thus, selecting the negative edge \(\{m_*, k_*\}\) for \(t + 5 + Z_2\) implies \(x_{m_*}(t + 6 + Z_2) = c\) for \(\beta \geq \frac{2}{(2\alpha - 1)}\). The case with \(m' = k'\) can be dealt with by a similar treatment, leading to the same conclusion.

This concludes the proof of the lemma.

In view of Lemmas \ref{ap_lemma: 2 bound} and \ref{ap_lemma: appro }, the desired theorem is a consequence of the second Borel--Cantelli Lemma.

\section{\jq{The relationship of oversmoothing and Theorem~\ref{thm: small nega} and Theorem~\ref{thm: repel_struct}}}
\label{app: oversmoothing of theorem 4.1 and 4.3}

\paragraph{Discussion with other training methods}
While \cite{Peng2024BeyondOU} questions the existence of oversmoothing in trained GNNs, their observations are primarily based on specific experimental settings that may not fully capture the oversmoothing challenge present in the literature. Specifically, the empirical observations in~\cite{Peng2024BeyondOU} are based on 10-layer GCNs, which, while useful for their argument, may not represent the behavior of deeper networks or other GNN architectures. Moreover, Figure 2 in~\cite{Peng2024BeyondOU}  is based on a normalized metric, which might not be the most appropriate. To see this point, suppose one wants to classify two points. In one case, we have 0.01 vs -0.01 and in the other case, we have 100 vs -100. While the normalized distance considered in~\cite{Peng2024BeyondOU} would be the same for these two cases, the latter case has a much larger margin, and it would be thus much easier to learn a classifier.
On the other hand, \cite{Cong2021OnPB} suggests that the trainability of deep GNNs is more of a problem than over-smoothing. However, over-smoothing naturally presents challenges for training deep GNNs, as when oversmoothing happens, gradients vanish across different nodes. Besides, \cite{Cong2021OnPB}compares 3 models GCN, ResGCN and GCNII, proving that GCNII is the best backbone. We have adapted our SBP to GCNII in Table~\ref{tab:gcnii-performance} and the results showed that our SBP outperforms GCNII on average, especially in the middle layers.

\paragraph{Measure on oversmoothing}
There exist a variety of different approaches to quantify over-smoothing in deep GNNs, here we choose the measure based on the Dirichlet energy on graphs~\citep{wu2023demystifying,graph_oversmoothing_survey}.
\begin{equation}
    \epsilon(X(t))=\frac{1}{v}\Sigma_{i\in V}\Sigma_{j \in N_i}||x_i(t)-x_j(t)||_2^2,
\end{equation}

\jq{where $v$ is the number of the nodes, $x_i(t)$ is the node feature of node $i$ at time $t$. $N_i$ represents the neighbor set of node $i$, In the signed graph, it including nodes connected to $i$ by both positive and negative edges.}
Oversmoothing means that when the layers are infinity, all of the node features will converge, that is to say $\lim_{t \to \infty}\epsilon(X(t))\to 0$.

In Theorem~\ref{thm: small nega}, there are 2 cases: 
\begin{itemize}
    \item $if \beta < \beta_*, \text{then we have }\lim_{t \to \infty} x_i(t) = \sum_{j=1}^n x_j(0)/n     \text{ for all initial values }x(0)$
    \item $if \beta > \beta_*, \text{then} \lim_{t \to \infty} \|x(t)\| = \infty \text{ for almost all initial values w.r.t. Lebesgue measure}.$
\end{itemize}
In the first case, all the node features will coverage to the mean of them and therefore $\lim_{t \to \infty}\epsilon(X(t))\to 0$, then oversmoothing happens.
In the second case, the node features will diverge to infinity and thus $\lim_{t \to \infty}\epsilon(X(t))\to 0 \text{ or } \infty$ which is also not what we want. 

Theorem~\ref{thm: small nega} demonstrated that both insufficient repulsion and excessive repulsion caused by the negative graph can hinder performance in signed graph propagation.
From this, we conclude that relying solely on the negative signs is insufficient to alleviate oversmoothing.
Therefore, we propose the \jq{provable} solution: a structurally balanced graph to efficiently alleviate oversmoothing in Theorem~\ref{thm: repel_struct}.
Specifically, we have the following conclusion from the structurally balanced graph in Theorem~\ref{thm: repel_struct}:
\begin{equation}
    \mathbb{P}\left(\lim_{t \to \infty} x_i(t) = l(x(0)), i \in V_1; \lim_{t \to \infty} x_i(t) = -l(x(0)), i \in V_2 \right) = 1.
\end{equation}
Then we have:
\begin{align}
    \lim_{t \to \infty}\epsilon(X(t))&=\lim_{t \to \infty}\frac{1}{v}\Sigma_{i\in V}\Sigma_{j \in N_i}||x_i(t)-x_j(t)||^2_2 \\
    & =\lim_{t \to \infty}\frac{1}{v}\Sigma_{i \in V_1}\Sigma_{j \in N_i}||x_i(t)-x_j(t)||_2^2+ \frac{1}{v}\Sigma_{i \in V_2}\Sigma_{j \in N_i}||x_i(t)-x_j(t)||_2^2 \\
    & =\lim_{t \to \infty}\frac{1}{v}\Sigma_{i\in V_1}\Sigma_{j \in N_i, y_i \neq y_j}||x_i(t)-x_j(t)||_2^2+ \frac{1}{v}\Sigma_{i\in V_2}\Sigma_{j \in N_i, y_i \neq y_j}||x_i(t)-x_j(t)||_2^2 \\
    & =\lim_{t \to \infty}\frac{1}{v}\Sigma_{i\in V_1}\frac{v}{2}\times2c+ \frac{1}{v}\Sigma_{i\in V_2}\frac{v}{2}\times2c \\
    & =\lim_{t \to \infty}\frac{1}{v}(\frac{v}{2}\times \frac{v}{2}\times2c+ \frac{v}{2}\times\frac{v}{2}\times2c) \\
    & =vc\geq 0 
\end{align}
So Theorem~\ref{thm: repel_struct} proves that under certain conditions, structural balance can alleviate oversmoothing even when the layers are infinity.




\section{Extension of Structural Balance}
\label{app:weak-balance}
\begin{figure}
    \centering
    \includegraphics[width=1.1\textwidth]{figures/sb.pdf}
    \caption{Examples of structural balanced (left), weakly structural balanced (middle), and unbalanced signed graphs (right). Here red lines represent positive edges; black dashed lines represent negative edges; gray and blue circles represent nodes from different labels}
    \label{fig: sb}
\end{figure}
To clarify the concept of structural balance, weakly structural balance and unbalance signed graph, we give the examples as shown in Figure~\ref{fig: sb}.
The notion of structural balance can be weakened in the following definition \ref{def: weak struct}.
\begin{definition}
    A signed graph \( G \) is \textbf{weakly structurally balanced} if there is a partition of \( V \) into \( V = V_1 \cup V_2 \cup \ldots \cup V_m \), \( m \geq 2 \) with \( V_1, \ldots, V_m \) being nonempty and mutually disjoint, where any edge between different \( V_i \)'s is negative, and any edge within each \( V_i \) is positive.
    \label{def: weak struct}
\end{definition}

Then we show that when $\mathcal{G}$ is a complete graph, weak structural balance also leads to clustering of node states.
\begin{theorem}[\cite{signed_dynamics_paper_review}, Theorem 10]
\label{thm: weak_repel_struct} 
Assume that node $i$ interacts with node $j$ and $x_i(t)$ represents the value of node $i$ at time t. 
Let $\theta=\alpha$ if the edge $\{i,j\}$ is positive and $\theta=\beta$ if the edge $\{i,j\}$ is negative.
Consider the constrained signed propagation update:
\begin{equation}
\label{eq: weak constrained repel dyn}
    x_i(t + 1) = \mathcal{F}_c((1-\theta) x_i(t)+\theta x_J(t)).
\end{equation}
Let \(\alpha \in (0,1/2)\). 
Assume that \(\mathcal{G}\) is a weakly structurally balanced complete graph under the partition \(V = V_1 \cup V_2 \dots \cup V_m\). 
When \(\beta\) is sufficiently large, for almost all initial values \(x(0)\) w.r.t. Lebesgue measure, there exists m random variable \(l_1(x(0))\), \(l_2(x(0))\), \dots, \(l_m(x(0))\), each of which taking values in \(\{-c,c\}\) such that
\begin{equation}
    \mathbb{P}\left(\lim_{t \to \infty} x_i(t) = l_j(x(0)), i \in V_j, j=1,\dots, m \right) = 1.
\end{equation}
\end{theorem}





\section{Discussion about $\mathcal{SID}$}
\label{app: SID-csbm}

We give the details of CSBM and a more clear formula of $\mathcal{SID}$, $\mathcal{P}$ and $\mathcal{N}$ as suggested in Tabel~\ref{tab: sid} in this section.

\subsection{Definition of CSBM}
\label{app: csbm}
To quantify the structural balance of the mentioned methods, we simplified the graph to $2$-CSBM$(N, p, q, \mu_1, \mu_2, \sigma^2 )$ following~\cite{sbm_xinyi}. 
It consists of two classes $\mathcal{C}_1$ and $\mathcal{C}_2$ of nodes of equal size, in total with $N$ nodes. 
For any two nodes in the graph, if they are from the same class, they are connected by an edge independently with probability $p$, or if they are from different classes, the probability is $q$. For each node $v \in \mathcal{C}_i, i\in\{1,-1\}$, the initial feature $X_v$ is sampled independently from a Gaussian distribution $\mathcal{N}(\mu_i, {\sigma^2})$, where $\mu_i =\mathcal{C}_i, \sigma = I $. 
In this paper, we assign $N=100$ and the feature dimension is $8$.

\subsection{Measure of $\mathcal{SID}$}
\begin{equation}
    \mathcal{P}
=
\frac{1}{|V|} \sum_{v \in V} \hbox{ Number of nodes who have the same label as}~v~\hbox{and the non-positive edge}.
\end{equation}
\begin{equation}
    \mathcal{N}=\frac{1}{|V|} \sum_{v \in V}\hbox{ Number of nodes who have the different label from}~v~\hbox{and the non-negative edge}.
\end{equation}
\begin{equation}
    \mathcal{SB} = \frac{1}{2}(\mathcal{P} + \mathcal{N})
\end{equation}
\begin{figure}
    \centering
    \includegraphics[width=0.65\textwidth]{figures/sb_cg.pdf}
    \caption{Example of structural complete graph. Here red lines represent positive edges; black dashed lines represent negative edges; gray and blue circles represent nodes from different labels}
    \label{fig:sb_cg}
\end{figure}

\subsection{Proof of Proposition~\ref{pro: sid}}
\label{app: prof of prop sid}
\begin{proposition}
\label{app_pro: sid}
For a structural balanced complete graph $\mathcal{G}$, we have $\mathcal{SID}(\mathcal{G})=0$.
\end{proposition}
\paragraph{Proof}
To better understand, we give an example of the structural balance graph as shown in Figure~\ref{fig:sb_cg}.
we can see that for a node $v$, $\mathcal{P}(v)=0$ and  $\mathcal{N}(v)=0$ due to the structural balance complete graph assumption. So $\mathcal{SID}(\mathcal{G})=0$.

\subsection{\jq{More observations of $\mathcal{SID}$}}
\label{app: obs of sid}
Apart from Table~\ref{tab: sid} on CSBM, we further present the Structural Imbalance Degree ($\mathcal{SID}$) for Cora across different methods in Table~\ref{tab: sid of cora}. As the performance of these methods is similar in shallow layers (2), we focus on layer 16 to showcase the results.

\begin{table}[htbp]
\centering
\caption{$\mathcal{SID}$ on Cora datasets. We implement all of the methods on SGC under 100 epochs and the accuracy is the result.}
\label{tab: sid of cora}
\resizebox{\linewidth}{!}{
\begin{tabular}{ccccccc}
\hline
 & label-\ours & feature-\ours & BatchNorm & ContraNorm & Residual & DropEdge \\
\hline
$\mathcal{P}$ & 482.1123 & 482.1123 & 482.1123 & 482.1123 & 482.5137 & 484.2075 \\
$\mathcal{N}$ & 0.7408 & 0.7408 & 0.7408 & 0.7408 & 2221.7305 & 2221.7305 \\
$\mathcal{SID}$ & 241.4265 & 241.4265 & 241.4265 & 241.4265 & 1352.1221 & 1352.9620 \\
Accuracy & 77.43 $\pm$ 1.49 & 77.22 $\pm$ 0.55  & 70.79 $\pm$ 0.00 & 63.35 $\pm$ 0.00 & 40.91 $\pm$ 0.00 & 22.24 $\pm$ 3.04 \\
\hline
\end{tabular}
}
\end{table}

We have two key observations: 1) Methods with higher $\mathcal{SID}$ generally lead to worse accuracy, while those with lower $\mathcal{SID}$ tend to produce better accuracy. 2) $\mathcal{SID}$ is a coarse-grained metric; different methods can yield the same $\mathcal{SID}$ values while their performance varies. These observations can also align with the experiments in cSBM in Table~\ref{tab: sid}.
The observation may stem from the fact that structural balance is an inherent property of graph structure, which is challenging to measure precisely using a numerical metric like $\mathcal{SID}$. Proposition 4.6 in the paper proves that when $\mathcal{SID}=0$, the graph is structurally balanced. 
However, for graphs that are not structurally balanced, the properties remain unclear. 
For future work, we aim to develop a more nuanced metric to quantify the structural balance property of graphs.





\section{Proof of Proposition \ref{pro: sid} and \ref{pro: ours-label}}
\label{app: proof of label-sbp}

\begin{proposition}
% Consider the update via signed graph:
% \begin{equation}
%     \hat{X} =\pgh{A}X- \ngh{A_l} X.
% \end{equation}
Assume that node label classes are balanced $|Y_1| = |Y_2|$
% \xinyic{unclear what "two balanced classifications" means} 
and denote the ratio of labelled nodes as $p$.
Then we have that the signed graph adjacency matrix $A_s= A-A_l$ and $\mathcal{SID}(\mathcal{G})\leq (1-p)\frac{n}{2}$, where $\mathcal{SID}$ decreases with a larger labelling ratio $p$. In particular, when $p=1$ (full supervision), we have $\mathcal{SID}(\mathcal{G})=0$, i.e., a perfectly balanced graph.
Under the constrained signed propagation \eqref{eq: constrained repel dyn}, the nodes from different classes will converge to distinct constants.
\end{proposition}

\paragraph{Proof.}
Without loss of generality, assume that the node feature has been normalized which means that $||x_i||_2=1$ for every $i$.
If $x_i$ and $x_j$ has the same label, then we have that, $(A_s)_{i,j}=(A)_{i,j}+1>1$.
If $x_i$ and $x_j$ has different labels, then we have that $(A_s)_{i,j}=(A)_{i,j}-1\leq0$.

We first prove that $\mathcal{SID}(\mathcal{G},p)\leq(1-p)\frac{n}{2}$ where $n$ is the nodes number and $p$ is the label ratio.
We have that 
\begin{equation}
    \mathcal{P}(v)+\mathcal{N}(v)\leq(1-p)n\, ,
\end{equation}
because for a single node $v$ only the remaining $(1-p)n$ nodes' labels are unknown and therefore their edges may need to change so that
\begin{equation}
\begin{aligned}
    \mathcal{SID}(\mathcal{G}) &= \frac{1}{2n}\sum_{v\in\mathcal{G}}(|\mathcal{P}(v)| + |\mathcal{N}(v)|)\\
    &\leq \frac{1}{2n}\sum_{v\in\mathcal{G}}(1-p)n\\
    &= (1-p)\frac{n}{2}.
\end{aligned}
\end{equation}

We know that when $\mathcal{SID}(\mathcal{G})=0$, then we have that the nodes $V$ set can be divided into $V_1\cup V_1 \dots \cup V_L$ where $L$ is the number of the node classes.
% The node with the class $i$ belongs to the $V_i$ set.
There are only positive edges with the node subset and only negative edges between the node subset.

Since $C=2$, the node set can be divided into $V_1$ and $V_2$, the signed graph is structurally balanced.
According to Theorem~\ref{thm: repel_struct}, we have that the nodes in $V_1$ will converge to the $c$ where $||c||_2=1$ and the nodes in $V_2$ will converge to $-c$.
Thus under Label-\ours propagation, the oversmoothing will only happen within the same label and repel different labels to the boundary.







\label{sec: proof}

\appendix
\begin{table}[t!]
  \centering
  


% \renewcommand{\arraystretch}{1.2} % 调整行高
% \setlength{\tabcolsep}{10pt}  % 调整列间距
\resizebox{0.48\textwidth}{!}{%
\begin{tabular}{lcrr}
        \toprule
        \textbf{Dataset} & \textbf{Full Size*} & \textbf{Consistency}  & \textbf{\dataset{}} \\
        \midrule
        HotpotQA  & 5,901 & 2,973 {\footnotesize \textcolor{gray}{(50\%)}}  & 1,476 {\footnotesize \textcolor{gray}{(25\%)}}  \\
        NewsQA    & 4,212 & 1,260 {\footnotesize \textcolor{gray}{(30\%)}} & 934  {\footnotesize \textcolor{gray}{(22\%)}}  \\
        NQ        & 7,314 & 4,419 {\footnotesize \textcolor{gray}{(60\%)}}  & 1,479 {\footnotesize \textcolor{gray}{(20\%)}}  \\
        SearchQA  & 16,980 & 12,133 {\footnotesize \textcolor{gray}{(71\%)}} & 1,497 {\footnotesize \textcolor{gray}{(9\%)}}  \\
        SQuAD     & 10,490 & 5,024 {\footnotesize \textcolor{gray}{(48\%)}}  & 2,351 {\footnotesize \textcolor{gray}{(22\%)}}  \\
        TriviaQA  & 7,785 & 6654 {\footnotesize \textcolor{gray}{(85\%)}}  & 792  {\footnotesize \textcolor{gray}{(10\%)}}  \\
        \bottomrule
    \end{tabular}
}




 \caption{Number of instances at each stage in the \dataset{} construction pipeline.}
 \label{tab:our_bench_stats_each_step}
\end{table}
\section{Appendix}
\subsection{License}
We present the licenses of the datasets used in this study: Natural Questions (CC BY-SA 3.0 license), NewsQA (MIT License), SearchQA and TriviaQA (Apache License 2.0), HotpotQA and SQuAD (CC BY-SA 4.0 license).

All these licenses and agreements permit the use of their data for academic purposes.

\subsection{Details of Data Constructing}
\label{append:prompts}
In this section, we detail the two main steps in constructing \dataset{}. The dataset sizes at each stage of the pipeline are shown in Table~\ref{tab:our_bench_stats_each_step}.


\textbf{Parametric Knowledge Elicitation.} First, we elicit the LLM's parametric knowledge by prompting it in a closed-book setting (i.e., without any context). To ensure the reliability of the elicited knowledge, we apply a consistency-based filtering method. Specifically, for each query, the LLM is prompted five times, and the frequency of each response is recorded. The response with the highest frequency is identified as the majority answer. Queries where the majority answer appears fewer than three times are discarded, in order to filter out inconsistent responses and enhance data quality. The following prompt is used to instruct the LLM:
\begin{tcolorbox}
[title=Prompt for eliciting parametric knowledge,colback=blue!10,colframe=blue!50!black,arc=1mm,boxrule=1pt,left=1mm,right=1mm,top=1mm,bottom=1mm]
Answer the question \textcolor{blue}{\{\textit{brevity\_instruction}\}} and provide supporting evidence.

Question: \textcolor{blue}{\{\textit{question}\}}
\end{tcolorbox}
\noindent The ``\textit{brevity\_instruction}'' is used to guide the LLM to generate responses in a more concise form.

\textbf{Conflict Data Selection.} Next, we filter the data to retain only instances where the LLM's parametric knowledge directly conflicts with the contextual answer. Specifically, we categorize the data obtained from the previous step into two groups, conflicting and non-conflicting instances, based on the detailed results of conflict detection. All non-conflicting instances are discarded. GPT-4o-mini is then used to detect the presence of a conflict, using the following prompt:

\begin{tcolorbox}
[title=Prompt for identifying conflict knowledge,colback=blue!10,colframe=blue!50!black,arc=1mm,boxrule=1pt,left=1mm,right=1mm,top=1mm,bottom=1mm]
\small
You are tasked with evaluating the correctness of a model-generated answer based on the given information. 

\small
Context: \textcolor{blue}{\{\textit{context}\}}

Question: \textcolor{blue}{\{\textit{question}\}}

Contextual Answer: \textcolor{blue}{\{\textit{contextual\_answer}\}}

Model-Generated Answer: \textcolor{blue}{\{\textit{Model-Generated\_answer}\}}

\textcolor{blue}{[\textit{Detailed task description...}]}

Output Format:

Evaluate result: (Correct / Partially Correct / Incorrect) 
\end{tcolorbox}




\subsection{Assessing the Reliability of GPT-4o-mini in Knowledge Conflict Identification}
\label{append:human_eval}
In this subsection, we conduct the human evaluation to assess the reliability of GPT-4o-mini in identifying knowledge conflicts, which is a critical task in our data construction process to guarantee the data quality.

We randomly sampled 100 examples from each of the six subsets of \dataset{}, yielding a total of 600 samples. Six senior computational linguistics researchers were then asked to evaluate whether a knowledge conflict was present in each example. For each instance, the evaluators were provided with the question, the contextual answer, the model-generated response, and the corresponding supporting evidence. The results were classified into three categories: No Conflict, Somewhat Conflict, and High Conflict. The detailed annotation instructions are as follows:

\begin{tcolorbox}
[title=Annotation Instruction,colback=blue!10,colframe=blue!50!black,arc=1mm,boxrule=1pt,left=1mm,right=1mm,top=1mm,bottom=1mm]
\small
You are tasked with determining whether the parametric knowledge of LLMs conflicts with the given context to facilitate the study of knowledge conflicts in large language models.

Each data instance contains the following fields: 

Question: \textcolor{blue}{\{\textit{question}\}}


Answers: \textcolor{blue}{\{\textit{answers}\}}


Context: \textcolor{blue}{\{\textit{context}\}}

Parametric\_knowledge: \textcolor{blue}{\{\textit{LLMs' parametric\_knowledge }\}} 

The annotation process consists of two steps. 

\textbf{Step 1}: Compare the model-generated answer with the ground truth answers, based on the given question and context, to determine whether the model’s parametric knowledge conflicts with the context.

\textbf{Step 2}: Classify the results into one of three categories: 

\textcolor{blue}{\{\textit{No Conflict}\}} if the model-generated answer is consistent with the ground truth answers and context, 

\textcolor{blue}{\{\textit{Somewhat Conflict}\}}  if it is partially inconsistent

\textcolor{blue}{\{\textit{High Conflict}\}} if it significantly contradicts the ground truth answers or context.
\end{tcolorbox}


The evaluation results, shown in Table~\ref{tab:append_human_eval}, reveal a high level of agreement between the human annotators and GPT-4o-mini. Over 85\% of the examples reach consensus among the annotators, with an average agreement rate of 85.6\% across all subsets. These findings underscore the reliability of GPT-4o-mini as an effective tool for identifying knowledge conflicts.




\begin{table}[t]
  \centering
  
\centering
\begin{tabular}{l c}
\toprule
\textbf{Subset} & \textbf{Agreement (\%)} \\ \midrule
HotpotQA        & 81.4                        \\
NewsQA          & 72.7                        \\
NQ              & 88.7                        \\
SearchQA        & 95.3                        \\
SQuAD           & 86.1                        \\
TriviaQA        & 90.7                        \\ \midrule
\textbf{Average} & \textbf{85.6}            \\ \bottomrule
\end{tabular}

 \caption{Agreement between human annotators and GPT-4o-mini across different subsets of our \dataset{} benchmark.}
 \label{tab:append_human_eval}
\end{table}



\subsection{Evaluating the Effectiveness of Our Consistency-Based Filtering Method}
\label{append:data_freq}

In this subsection, we evaluate the effectiveness of our consistency-based knowledge conflict filtering method. As described in Appendix~\ref{append:prompts}, for each query, we prompt the model five times and record the most frequently generated answer along with its occurrence frequency. Based on this frequency, we divide the data into sub-datasets, where all queries within each sub-dataset share the same answer frequency. We then apply ``Conflict Data Selection'' to each sub-dataset, retaining only instances where knowledge conflicts occur. Finally, we evaluate ConR and MemR on these sub-datasets.

As shown in Figure~\ref{fig:diff_freq}, a clear trend emerges: as answer frequency increases, ConR consistently decreases, while MemR increases. This pattern indicates that as answer frequency rises, the model becomes increasingly reliant on its internal knowledge. Notably, for data with an answer frequency of 1, MemR is only 3\%, indicating minimal dependence on internal knowledge. Retaining only high-answer-frequency data improves the quality of \dataset{}. This data construction approach distinguishes our methodology from previous studies~\cite{longpre2021entity,xie2023adaptive}.

\begin{figure}[t!]
  \centering
  \includegraphics[width=0.4\textwidth]{figs/diff_freq.pdf}
  \caption{Performance comparison of ConR and MemR across sub-datasets grouped by the answer frequency of LLMs.}
  \label{fig:diff_freq}
\end{figure}





\subsection{Additional Implementation Details of Our Experiments}
\label{append:implementation}
This subsection outlines the training prompt, describes more details of the training data, and provides details of the experimental setup used in our experiments.

\textbf{Training Prompts.}
We adopt a simple QA-format training prompt following~\citet{zhou2023context} for all methods except \attrprompt{} and \oiprompt{}.
\begin{tcolorbox}
[title=Base Prompt ,colback=blue!10,colframe=blue!50!black,arc=1mm,boxrule=1pt,left=1mm,right=1mm,top=1mm,bottom=1mm]
% \small
\textcolor{blue}{\{\textit{context}\}} 
Q: \textcolor{blue}{\{\textit{question}\}} ? 
A: \textcolor{blue}{\{\textit{answer}\}}.
\end{tcolorbox}


\textbf{Training Datasets.} During \method{}, we randomly sample 32,580 instances from the training set of the MRQA 2019 benchmark~\cite{fisch2019mrqa} to construct our training data.



\textbf{Experimental Setup.} In this work, all models are trained for 2,100 steps with a total batch size of 32 and a learning rate of 1e-4. To enhance training efficiency, we implemented \method{} with LoRA~\cite{hu2021lora}, setting both the rank $\text{r}$ and scaling factor $\text{alpha}$ to 64. For \method{}, we set $\alpha$ to 0.1 (Eq.~\ref{eq:selct_layers}), which determines the minimum activation ratio difference required for a layer to be pruned. Additionally, we adopt a dynamic $\gamma$ in $\mathcal{L}_{\text{KC}}$ (Eq.~\ref{eq:kc_loss}), which linearly transitions from an initial margin ($\gamma_{0}=1$) to a final margin ($\gamma^*=5$) as training progresses. This adaptive strategy gradually reduces the model's reliance on internal parametric knowledge, encouraging it to rely more on external knowledge provided by the KAG system.


\subsection{Implementation Details of Baselines}
\label{append:baseline}
This subsection describes the implementation details of all baseline methods.

We adopt two prompt-based baselines: the attributed prompt ($\text{Attr}_{\text{prompt}}$) and a combination of opinion-based and instruction-based prompts ($\text{O\&I}_{\text{prompt}}$). The corresponding prompt templates are as follows:

\begin{tcolorbox}
[title=Attr based prompt ,colback=blue!10,colframe=blue!50!black,arc=1mm,boxrule=1pt,left=1mm,right=1mm,top=1mm,bottom=1mm]
% \small
\textcolor{blue}{\{\textit{context}\}} Q: \textcolor{blue}{\{\textit{question}\}} based on the given text? A: \textcolor{blue}{\{\textit{answer}\}}.
\end{tcolorbox}

\begin{tcolorbox}
[title=O\&I based prompt ,colback=blue!10,colframe=blue!50!black,arc=1mm,boxrule=1pt,left=1mm,right=1mm,top=1mm,bottom=1mm]

Bob said ``\textcolor{blue}{\{\textit{context}\}}'' Q: \textcolor{blue}{\{\textit{question}\}} in Bob's opinion? A: \textcolor{blue}{\{\textit{answer}\}}.
\end{tcolorbox}
For the SFT baseline, we incorporate context during training, similar to \method{}, while keeping the remaining experimental settings identical. To construct preference pairs for DPO training, we use contextually aligned answers from the dataset as ``preferred responses'' to ensure the consistency with the provided context. The ``rejected responses'' are generated by identifying parametric knowledge conflicts through our data construction methodology (Sec.~\ref{sec:benchmark}).

For KAFT, we employ a hybrid dataset containing both counterfactual and factual data. Specifically, we integrate the counterfactual data developed by \citet{xie2023adaptive}, leveraging their advanced data construction framework.

By maintaining equivalent dataset sizes and ensuring comparable data quality across all baselines, we provide a rigorous and fair comparison with our proposed \method{}.




\subsection{Extending \method{} to More LLMs}
\label{append:diff_model_performance}


\begin{figure}[t!]
  \centering
  
\subfigure[ConR Results]{
        \label{fig:diff_model:llama_conr}
        \includegraphics[width=0.462\linewidth]{append_fig/llama_conr.pdf}
    }
    \hspace{0.0005\linewidth} 
    \subfigure[MemR Results]{
        \label{fig:diff_model:llama_memr}
        \includegraphics[width=0.462\linewidth]{append_fig/llama_memr.pdf}
    }


  % \includegraphics[width=0.48\textwidth]{figs/diff_model_double.pdf}
 \caption{Average ConR and MemR across different models implemented by LLMs of LLaMA series, before and after applying \method{}.
 }
 \label{fig:diff_model_double_llama}
\end{figure}

\begin{figure}[t]
  \centering
  \subfigure[ConR Results]{
        \label{fig:diff_model:qwen_conr}
        \includegraphics[width=0.462\linewidth]{append_fig/qwen_conr.pdf}
    }
    \hspace{0.0005\linewidth} 
    \subfigure[MemR Results]{
        \label{fig:diff_model:qwen_memr}
        \includegraphics[width=0.462\linewidth]{append_fig/qwen_memr.pdf}
    }
  % \includegraphics[width=0.48\textwidth]{figs/diff_model_double.pdf}
 \caption{Average ConR and MemR across different models implemented by LLMs of Qwen series, before and after applying \method{}.
 }
 \label{fig:diff_model_double_qwen}
\end{figure}






We extend \method{} to a diverse range of LLMs, encompassing multiple model families and sizes. 

Specifically, our evaluation includes LLaMA3-8B-Instruct, LLaMA3.2-1B-Instruct, LLaMA3.2-3B-Instruct, Qwen2.5-0.5B-Instruct, Qwen2.5-1.5B-Instruct, Qwen2.5-3B-Instruct, Qwen2.5-7B-Instruct, and Qwen2.5-14B-Instruct. The results on ConR and MemR are summarized in Figures~\ref{fig:diff_model_double_llama} and \ref{fig:diff_model_double_qwen}, while Table~\ref{tab:append:all_model_res} presents the average performance of all models on \dataset{} and ConFiQA. Additionally, Table~\ref{tab:diff_model_param} provides detailed parameter information and specifies the layers selected for pruning for each model. This comprehensive evaluation demonstrates the versatility and scalability of \method{} across a wide spectrum of model architectures and sizes.

\begin{table}[!t]
  
    \resizebox{0.48\textwidth}{!}{%
\begin{tabular}{l|c|c|c}
\toprule
\textbf{Models}     & \textbf{Param.} & \textbf{\method{} Param.} & \textbf{Selected Layers} \\
\midrule
\rowcolor{gray!10}
LLaMA3.2-1B        & 1.24B  & 1.08B \small\textcolor{gray}{(87\%)}   & [12, 14]                 \\
LLaMA3.2-3B        & 3.21B  & 2.60B \small\textcolor{gray}{(81\%)}   &  [18, 25]   \\
\rowcolor{gray!10}
LLaMA3-8B          & 8.03B  & 6.97B \small\textcolor{gray}{(87\%)}   & [24, 29]      \\
LLaMA3.1-8B          & 8.03B  & 6.27B \small\textcolor{gray}{(78\%)}   & [20, 29]      \\
\rowcolor{gray!10}
Qwen2.5-0.5B         & 0.49B  & 0.44B \small\textcolor{gray}{(90\%)}   &  [19, 22]       \\
Qwen2.5-1.5B         & 1.54B  & 1.34B \small\textcolor{gray}{(87\%)}   & [21, 25]        \\
\rowcolor{gray!10}
Qwen2.5-3B         & 3.09B  & 2.68B \small\textcolor{gray}{(87\%)}   & [29, 34]        \\
Qwen2.5-7B         & 7.61B  & 7.21B \small\textcolor{gray}{(95\%)}   &   [25, 26 ]     \\
\rowcolor{gray!10}
Qwen2.5-14B        & 14.70B & 12.43B \small\textcolor{gray}{(85\%)}  &  [35, 45]   \\
\bottomrule
\end{tabular}
}

% \end{sidewaystable}

% \end{document}

  \caption{The total number of parameters for various models before and after applying \method{}. \textcolor{gray}{\small$(\cdot)\%$} represents the proportion relative to the original model, and the last column lists the layers selected for pruning.}
   \label{tab:diff_model_param}
\end{table}

These experimental results illustrate several key insights: 1) Larger models tend to rely more on parametric memory. As model size increases in both the LLaMA and Qwen families, MemR also grows, indicating a tendency to overlook external knowledge in favor of internal parameters. \method{} counteracts this behavior, decreasing larger models' MemR score to even below that of smaller models. 2) \method{} consistently benefits all evaluated models. Across both LLaMA and Qwen model families, \method{} outperforms Vanilla-KAG by boosting accuracy and context faithfulness, underscoring its broad applicability and effectiveness. 3) Not all parameters in KAG models are essential. Pruning parametric knowledge not only reduces computation costs but also fosters better generalization without sacrificing accuracy, highlighting the potential of building a parameter-efficient LLM within the KAG framework.




\begin{table*}[!t]
  
\centering
\resizebox{0.96\textwidth}{!}{%
\begin{tabular}{l|c|cccc|cccc}
\toprule
\multirow{2}{*}{\textbf{Models}} & \multirow{2}{*}{\textbf{Param.}} & \multicolumn{4}{c|}{\textbf{\dataset{}}} & \multicolumn{4}{c}{\textbf{ConFiQA}} \\ 
\cmidrule(lr){3-6}  \cmidrule(lr){7-10}
 &  & ConR $\uparrow$ & MemR $\downarrow$ & MR $\downarrow$ & EM $\uparrow$ & ConR $\uparrow$ & MemR $\downarrow$ & MR $\downarrow$ & EM $\uparrow$ \\ 
\midrule
LLaMA3-8B   & 8.03B  & 66.99  & 11.75  & 14.99  & 13.83  & 22.52  & 31.15  & 59.77  & 2.47 \\
\rowcolor{gray!10}
+\method{}    & 6.97B  & 71.50  & 6.48   & 8.41   & 66.19  & 70.43  & 8.82   & 11.32  & 67.29 \\
LLaMA3.1-8B & 8.03B  & 63.15  & 11.69  & 15.93  & 21.85  & 15.38  & 29.97  & 68.98  & 6.69 \\
\rowcolor{gray!10}
+\method{}   & 6.27B  & 70.41  & 6.95   & 9.17   & 63.58  & 71.12  & 9.01   & 11.44  & 66.61 \\
LLaMA3.2-1B & 1.24B  & 39.06  & 10.49  & 21.83  & 5.13   & 32.09  & 18.32  & 36.28  & 7.15 \\
\rowcolor{gray!10}
+\method{}   & 1.08B  & 51.75  & 6.51   & 11.34  & 47.60  & 62.70  & 7.63   & 11.38  & 61.85 \\
LLaMA3.2-3B & 3.21B  & 56.75  & 11.53  & 17.11  & 12.69  & 26.16  & 23.47  & 49.05  & 9.84 \\
\rowcolor{gray!10}
+\method{}   & 2.60B  & 67.00  & 6.80   & 9.35   & 61.59  & 69.61  & 8.39   & 11.09  & 66.53 \\
Qwen2.5-0.5B & 0.49B  & 47.17  & 11.36  & 19.48  & 2.06   & 50.72  & 17.15  & 26.20  & 3.78 \\
\rowcolor{gray!10}
+\method{}   & 0.44B  & 58.13  & 6.63   & 10.41  & 52.56  & 67.54  & 8.04   & 11.03  & 66.33 \\
Qwen2.5-1.5B & 1.54B  & 58.08  & 11.28  & 16.48  & 10.30  & 51.69  & 19.87  & 28.23  & 10.78 \\
\rowcolor{gray!10}
+\method{}   & 1.34B  & 63.78  & 6.74   & 9.76   & 57.67  & 69.61   & 8.35   & 11.05   & 66.04 \\
Qwen2.5-3B   & 3.09B  & 62.22  & 14.45  & 18.88  & 0.10   & 25.47  & 29.34  & 55.70  & 0.01 \\
\rowcolor{gray!10}
+\method{}     & 2.68B  & 66.31  & 6.75   & 9.38   & 59.42  & 66.30   & 8.62  & 11.94   & 63.03 \\
Qwen2.5-7B    & 7.61B  & 65.46  & 14.93  & 18.57  & 0.80   & 24.75  & 33.09  & 59.04  & 0.10 \\
\rowcolor{gray!10}
+\method{}      & 6.60B  & 67.75  & 6.60   & 9.01   & 61.77  & 69.54  & 8.85   & 11.58  & 66.68 \\
Qwen2.5-14B   & 14.70B & 65.75  & 16.13  & 19.75  & 0.00   & 7.86   & 32.88  & 83.71  & 0.01 \\
\rowcolor{gray!10}
+\method{}     & 12.43B & 70.01  & 6.43   & 8.55   & 64.43  & 71.70  & 8.90   & 11.29  & 68.40 \\
\bottomrule
\end{tabular}%
}


  \caption{Average performance of LLMs on \dataset{} and ConFiQA before and after applying \method{}.}
   \label{tab:append:all_model_res}
\end{table*}

\subsection{Neuron Activations in Different LLMs}\label{app:activation}
We present the neuron activations for the LLaMA family models, including LLaMA-3.2-1B-Instruct, LLaMA-3.2-3B-Instruct, LLaMA-3-8B-Instruct, and LLaMA-3.1-8B-Instruct, as well as the Qwen family models, including Qwen-2.5-0.5B-Instruct, Qwen-2.5-1.5B-Instruct, Qwen-2.5-3B-Instruct, Qwen-2.5-7B-Instruct, and Qwen-2.5-14B-Instruct, in Figures~\ref{fig:act_llama} and \ref{fig:act_qwen}, respectively. 
% 我们发现qwen系列模型


\begin{figure*}[t]
  \centering
  \subfigure[Neuron activations of LLaMA-3.2-1B-Instruct]{
        \label{fig:act_llama:3.2-1b}
        \includegraphics[width=0.9\linewidth]{append_fig/act_llama32_1b_all.pdf}
    }
\subfigure[Neuron activations of LLaMA-3.2-3B-Instruct]{
        \label{fig:act_llama:3.2-3b}
        \includegraphics[width=0.9\linewidth]{append_fig/act_llama32_3b_all.pdf}
    }
 \subfigure[Neuron activations of LLaMA-3-8B-Instruct]{
        \label{fig:act_llama:3-8b}
        \includegraphics[width=0.9\linewidth]{append_fig/act_llama_3_8b.pdf}
    }
 \subfigure[Neuron activations of LLaMA-3.1-8B-Instruct]{
        \label{fig:act_llama:3.1-8b}
        \includegraphics[width=0.9\linewidth]{append_fig/act_llama_31_8b.pdf}
    }
 

 \caption{Neuron activations across different layers of the LLaMA series models. We present the inhibition ratio $\Delta R$ under two conditions: with contextual knowledge input (w/ context) and without it (w/o context).}
 \label{fig:act_llama}
\end{figure*}

\begin{figure*}[t]
  \centering
  \subfigure[Neuron activations of Qwen-2.5-0.5B-Instruct]{
        \label{fig:act_qwen:2.5-0.5b}
        \includegraphics[width=0.75\linewidth]{append_fig/act_qwen25_0_5b_all.pdf}
    }
\subfigure[Neuron activations of Qwen-2.5-1.5B-Instruct]{
        \label{fig:act_qwen:2.5-1.5b}
        \includegraphics[width=0.75\linewidth]{append_fig/act_qwen25_1_5b_all.pdf}
    }
\subfigure[Neuron activations of Qwen-2.5-3B-Instruct]{
        \label{fig:act_qwen:2.5-3b}
        \includegraphics[width=0.75\linewidth]{append_fig/act_qwen25_3b_all.pdf}
    }
\subfigure[Neuron activations of Qwen-2.5-7B-Instruct]{
        \label{fig:act_qwen:2.5-7b}
        \includegraphics[width=0.75\linewidth]{append_fig/act_qwen25_7b_all.pdf}
    }
\subfigure[Neuron activations of Qwen-2.5-14B-Instruct]{
        \label{fig:act_qwen:2.5-14b}
        \includegraphics[width=0.75\linewidth]{append_fig/act_qwen25_14b_all.pdf}
    }


 \caption{Neuron activations across different layers of the Qwen series models. We present the inhibition ratio $\Delta R$ under two conditions: with contextual knowledge input (w/ context) and without it (w/o context). }
 \label{fig:act_qwen}
\end{figure*}



\end{document}