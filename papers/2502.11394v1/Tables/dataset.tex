\begin{table}[t]
% {.5\linewidth}
% \vspace{-0.3cm}
\caption{The statistic of datasets. $H(G)$ refers to the homophilic levels.}
\centering
\resizebox{0.75\linewidth}{!}{
\begin{tabular}{lcccc}
\toprule
Dataset    & $H(G)$        & Classes             & Nodes &Edges \\
% & Train/Dev/Test Nodes\\
\midrule
\textbf{Cora} & 0.81 & 7 & 2,708 &  5,429 \\ %&140/500/1,000\\
\textbf{Citeseer} & 0.74 & 6 &3,327  &4,732 \\ % & 120/500/1,000 \\
\textbf{PubMed} & 0.80 & 3 & 19,717 & 44,338 \\ %& 60/500/1,000\\

\midrule
\textbf{Texas} & 0.21 & 5 & 183 & 295 \\
\textbf{Cornell} & 0.30 & 5 & 183 & 280\\
\textbf{Wisconsin} & 0.11 &5 & 251 & 466\\
% \textbf{Chameleon}& 0.23 & 6 & 2,277 & 31,421 & 1,092/729/456\\
\textbf{Squirrel} & 0.22 & 4 & 198,493 & 2,089 \\ %& 2,596/1,664/1,041\\
\midrule
\textbf{Ogbn-Arxiv} & 0.65 & 40 & 16,9343 & 1,166,243 \\ %  & 90,941/29,799/48,603\\
\bottomrule
\end{tabular}
}
% \end{adjustbox}
\label{tab: data}
% }
% \vspace{-0.4cm}
\end{table}