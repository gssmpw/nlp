\section{Related work}
\label{related}
Many DP algorithms for computing quantiles have been proposed in the literature, though most of them are not frugal and/or cannot be adopted in the streaming setting. Here, we recall the most important ones, beginning with those algorithms designed to release a quantile (or a set of quantiles) of a dataset (i.e., all of the items are available when the algorithm begins its execution, not the streaming setting).

The algorithm proposed in ____ allows computing a single quantile and can be considered an instantiation of the so-called exponential mechanism leveraging a specific utility function. It works by sampling an interval and then sampling an item in the selected interval. Since the items must be sorted, its worst-case complexity is $O(n \log n)$. 

The algorithm may be used to compute $m$ quantiles using well-known composition theorems, since one can easily estimate separately each quantile privately, and then obtain the overall privacy budget by updating it, using a composition theorem, taking into account the privacy budgets used independently for each quantile estimation. Such an approach, while feasible, is anyway sub-optimal since the error would scale polynomially with $m$. Algorithms based on this approach are JointExp, AppindExp and AggTree, proposed in ____.

Better algorithms exist for this task that do not use the straightforward direct independent application of a composition theorem. An example is a recent and clever algorithm, Approximate Quantiles ____ which computes recursively $m$ quantiles with an error that scales logarithmically with $m$. The worst-case complexity of the algorithm is $O(n \lg m)$.

In the streaming setting, a recent work ____ proposed the use of linear sketches to privately compute arbitrary quantiles. In particular, the authors design a private variant of the Dyadic CountSketch algorithm ____ by using their version of private CountSketch based on $\rho$-zCDP. The algorithm retains the same space complexity of Dyadic CountSketch, i.e., $O\left(\frac{1}{\varepsilon} \log ^{1.5} u \log ^{1.5}\left(\frac{\log u}{\varepsilon}\right)\right)$ where $u$ is the size of the universe from which the stream items are drawn ($[u]=\{0, \ldots, u-1\}$) and $\varepsilon$ is the approximation error related to the underlying CountSketch data structure. In contrast, our DP algorithms based on \textsc{Frugal-1U} and \textsc{Frugal-1U} only requires one or two units of memory. Moreover, the update (inserting an incoming stream item into the sketch) and query (returning a quantile estimate) of private Dyadic CountSketch are much slower than those of our DP algorithms.

In ____, starting from the observation that all existing private mechanisms for distribution-independent quantile computation require space at least linear in the input size $n$, the authors design DP algorithms which exhibit strongly sub-linear space complexity, namely DPExpGK and DPHistGK. Then, the authors extend DPExpGK to work in the streaming setting. The algorithm works by updating the private estimate not for each incoming stream item, but at fixed checkpoints. For a stream of $n$ items there are $O\left(\log _{1+\alpha} \frac{n}{n_{\min }}\right)$ checkpoints, where $\alpha$ is a parameter related to the non private quantile estimate accuracy and $n_{\min}$ is a threshold for the stream length: it must be $n>n_{\min }=\Omega\left(\frac{1}{\alpha^2 \epsilon} \log n \log \left(\frac{|\mathcal{X}| \log n}{\alpha \beta}\right)\right)$ where $\epsilon$ is the privacy budget, $|\mathcal{X}|$ is the number of distinct items in the stream and $\beta$ a failure probability. The space complexity of the algorithm is $\Omega\left(\frac{1}{\alpha^2 \epsilon} \log ^2 n \log \left(\frac{|\mathcal{X}| \log n}{\alpha \beta}\right)\right)$. Also in this case, our DP algorithms provide faster updates whilst requiring one or two units of memory. 

Another DP algorithm working in the streaming setting has been proposed in ____, but works under Local Differential Privacy (LDP) using self-normalization. The algorithm requires computing, for each incoming stream item, a corresponding step size used in the update of the quantile estimate. A locally randomized process is used, in which given an input stream item, the current quantile estimate and a so-called response rate $r$ (which corresponds to the privacy budget), the authors verify if the stream item is greater than the current quantile estimate. The result (true or false) is randomized using two Bernoulli distributions, and is then used to update the current $i$-th quantile estimate, along with a corresponding step size $d_i=2 /\left(i^{0.51}+100\right)$.
The space required is $O(1)$, since four units of memory are required to process the input. In particular, only one unit of memory is used to estimate the quantile, whilst the other three units of memory are only required if determining a confidence interval for the quantile estimate is requested as well. The update process is relatively fast, even though it is still much slower than our algorithms. Since this algorithm is the only one matching the features of \textsc{Frugal-1U-L}, \textsc{Frugal-1U-G}, \textsc{Frugal-1U-$\rho$} and \textsc{Frugal-2U-SA}  (i.e., streaming setting, $O(1)$ space required and DP release of a quantile estimate), in the next section in which we provide experimental results, we shall compare its performance versus our proposed algorithms. In the sequel, we shall refer to this algorithm as \textsc{LDPQ}.