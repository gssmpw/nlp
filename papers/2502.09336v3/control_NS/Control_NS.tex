\documentclass[3p]{elsarticle}
\usepackage{stmaryrd}
\usepackage{float}
\usepackage{amsmath,amsfonts,amssymb}
\usepackage{graphicx}
\usepackage{caption}
\usepackage{subcaption}
%\usepackage{showkeys}
\captionsetup[subfigure]{labelformat=empty}
\usepackage{color}
\usepackage{url}
\usepackage{amsmath}
\usepackage{amsfonts}
\usepackage{amssymb}
\usepackage{amsthm}
\usepackage{algorithm}
\usepackage{algorithmic}
\usepackage{hyperref}
\usepackage{siunitx}
\hypersetup{pdfstartview=FitH}
\newtheorem{theorem}{Theorem}
\newtheorem{remark}[theorem]{Remark}
\newtheorem{definition}[theorem]{Definition}
\newtheorem{lemma}[theorem]{Lemma}
\newtheorem{corollary}[theorem]{Corollary}
\newtheorem{proposition}[theorem]{Proposition}
\allowdisplaybreaks
\newcommand{\bff}{\boldsymbol{f}}
\newcommand{\bfg}{\boldsymbol{g}}
\newcommand{\bfn}{\boldsymbol{n}}
\newcommand{\bfr}{\boldsymbol{r}}
\newcommand{\bfs}{\boldsymbol{s}}
\newcommand{\bft}{\boldsymbol{t}}
\newcommand{\bfu}{\boldsymbol{u}}
\newcommand{\bfv}{\boldsymbol{v}}
\newcommand{\bfw}{\boldsymbol{w}}
\newcommand{\bfla}{\boldsymbol{\lambda}}
\newcommand{\bftau}{\boldsymbol{\tau}}

\newcommand{\bfH}{\boldsymbol{H}}
\newcommand{\bfL}{\boldsymbol{L}}
\newcommand{\bfW}{\boldsymbol{W}}

\newcommand{\RR}{\mathbb{R}}
\newcommand{\NN}{\mathbb{N}}

\newcommand{\bfxi}{\boldsymbol{\xi}}
\newcommand{\bfsig}{\boldsymbol{\sigma}}

\newcommand{\jump}[1]{{{\llbracket} #1 {\rrbracket}}}
\begin{document}

\title{A finite element scheme for an optimal control problem on steady Navier-Stokes-Brinkman equations}
\author[a]{Jorge Aguayo}
\ead{jaguayo@dim.uchile.cl}
\address[a]{Center for Mathematical Modeling, Universidad de Chile, Santiago, Chile}
\author[b]{Julie Y. Merten}
\ead{j.y.merten@rug.nl}
\address[b]{Bernoulli Institute, University of Groningen, Groningen, The Netherlands}
\date{\today}


\begin{abstract}
This is an abstract.
\end{abstract}

\maketitle

\pagestyle{myheadings} \thispagestyle{plain}

\section{Introduction and Model problem}

Consider $n\in\{2,3 \}  $ and a nonempty bounded domain $\Omega\subseteq\RR^{n}$. The Lebesgue measure of $\Omega$ is denoted by $\left\vert \Omega\right\vert $, which extends to spaces of lesser dimension. The norm and seminorms for Sobolev spaces $W^{m,p}(\Omega)$ is denoted by $\left\Vert \cdot\right\Vert _{m,p,\Omega}$ and $\left\vert \cdot\right\vert _{m,p,\Omega}$, respectively. For $p=2$, the norm, seminorms, inner product and duality pairing of the space $W^{m,2}(\Omega)=H^{m}(\Omega)$ are denoted by $\left\Vert \cdot\right\Vert _{m,\Omega}$, $\left\vert \cdot\right\vert _{m,\Omega}$, $\left(  \cdot,\cdot\right)  _{m,\Omega}$ and $\left\langle \cdot,\cdot\right\rangle _{m,\Omega}$, respectively. For $m=0$, the inner product is denoted by $(\cdot,\cdot)_\Omega$. Also, $\mathcal{C}^{m}(\Omega)$ and $\mathcal{C}^{\infty}(\Omega)$ denote the space of functions with $m$ continuous derivatives and all continuous derivatives, respectively. For $\Omega_{1}$ and $\Omega_{2}$ two open subsets of $\RR^{n}$, we denote $\Omega_{1}\Subset\Omega_{2}$ when there exists a compact set $K$ such that $\Omega_{1}\subseteq K\subseteq\Omega_{2}$.

The spaces $\mathbb{H}^{m}(\Omega)$, $\mathbb{W}^{m,p}(\Omega)$, $\bfH^{m}(\Omega)$, $\bfw^{m,p}(\Omega)$, $\boldsymbol{C}^{m}(\Omega)$ and $\boldsymbol{C}^{\infty}(\Omega)$ are defined by $\mathbb{H}^{m}(\Omega)=[  H^{m}(\Omega)]  ^{n\times n}$, $\mathbb{W}^{m,p}(\Omega)=[  W^{m,p}(\Omega)]  ^{n\times n}$, $\bfH^{m}(\Omega)=[  H^{m}(\Omega)]  ^{n}$, $\bfw^{m,p}(\Omega)=[  W^{m,p}(\Omega)]  ^{n}$, $\boldsymbol{C}^{m}(\Omega)=[  \mathcal{C}^{m}(\Omega)]  ^{n}$ and $\boldsymbol{C}^{\infty}(\Omega)=[  \mathcal{C}^{\infty}(\Omega)] ^{n}$. The notation for norms, seminorms and inner products will be extended from $W^{m,p}(\Omega)$ or $H^{m}(\Omega)$. Given $\boldsymbol{A},\boldsymbol{B}\in\RR^{n\times n}$, $a_{ij}$ denotes the entry in the $i-$th row and $j- $th column of matrix $\boldsymbol{A}$, $\boldsymbol{A}^{T}$ denotes the transpose matrix of $\boldsymbol{A}$, $\operatorname{tr}(\boldsymbol{A})$ denotes the trace of $\boldsymbol{A}$ and $\boldsymbol{A}:\boldsymbol{B}$ denotes the inner product of $\RR%
^{n\times n}$ given by%
\[
\boldsymbol{A}:\boldsymbol{B}=\operatorname{tr}(\boldsymbol{AB}^{T}%
)=\sum_{i=1}^{n}\sum_{j=1}^{n}a_{ij}b_{ij}\text{,}%
\]
$\operatorname{sym}\boldsymbol{A}=\dfrac{1}{2}(\boldsymbol{A}+\boldsymbol{A}^{T})$ and $\operatorname{skew}\boldsymbol{A}=\dfrac{1}{2}(\boldsymbol{A}-\boldsymbol{A}^{T})$. The identity matrix is denoted by $\boldsymbol{I}\in\RR^{n\times n}$. Analogously, given $\boldsymbol{a},\boldsymbol{b}\in\RR^{n}$, $a_{j}$ denotes the $j-$th component of vector $\boldsymbol{a}$ and $\boldsymbol{a}\cdot\boldsymbol{b}$ denotes the inner product of $\RR^{n}$.

\section{An optimal control problem}
First, we introduce some notation.
\begin{definition}
We define the spaces $H=\bfH^{1}(\Omega)=\left\{  \bfv\in\bfH^{1}(\Omega)\text{ } \mid\text{ } \bfv=\boldsymbol{0}\text{ on }\partial\Omega\right\}$ and $Q=L_{0}^{2}(\Omega):=\left\{  q\in L^{2}(\Omega)\text{ } \mid\text{ } (q,1)=0\right\}$. We denote the norm for the space $V:= H\times Q$as $\Vert(v,q)\Vert=\left(\vert \bfv\vert _{1,\Omega}%
^{2}+\Vert q\Vert_{0,\Omega}^{2}\right)  ^{1/2}$
\end{definition}

\begin{definition}
For $\nu>0$, we define for all $\bfu,\bfv,\bfw\in H$ and $q\in Q$.

\begin{enumerate}
\item $a(\bfu,\bfv)=\nu(\nabla\bfu,\nabla\bfv)_\Omega$

\item $b(\bfv,q)=-(q,\operatorname{div}\bfv)_\Omega$

\item $c(\bfu,\bfv,\bfw)=((\nabla\bfu)\bfv,\bfv)_\Omega$
\end{enumerate}
\end{definition}

Consider $\mathcal{A}=\{\gamma\in L^{2}(\Omega)$ $\mid$\ $a\leq\gamma\leq b$ in $\Omega\}$, for $a,b\in\RR$ such that $0\leq a<b$. Given a constant $\alpha>0$ and an nonempty open subset $\omega\subset\Omega$, the optimal control problem to be studied is given by
\begin{align}
\text{minimize\quad}   &  J(\gamma)=\dfrac{1}{2}\Vert\bfu-\bfu_{0}\Vert_{0,\omega}^{2}+\dfrac{\alpha}{2}\Vert\gamma -\gamma_{0}\Vert_{0,\Omega}^{2}\label{min}\\
\text{subject to\quad} &  (\forall(\bfw,r)\in H\times Q)\text{\quad}a(\bfu,\bfw)-b(\bfw,p)+b(\bfu,r)+c(\bfu,\bfu,\bfw)+(\gamma\bfu,\bfv)=(\bff,\bfw)\label{VF}\\ 
                       &  \gamma\in\mathcal{A}\nonumber
\end{align}
where $\bfu_{0}\in\bfL^{2}(\omega)$, $\gamma_{0}\in L^{2}(\Omega)$ and $\bff\in\bfL^{2}(\Omega)$.

\begin{lemma}\label{bounds-1}
There exist positive constants $\beta$ and $\delta$ such that for all $\bfu,\bfv,\bfw\in H$ and $q\in Q$.

\begin{enumerate}
\item $\vert a(\bfu,\bfv)\vert \leq\nu\vert\bfu\vert_{1,\Omega}\vert\bfv\vert_{1,\Omega}$

\item $\vert b(\bfv,q) \vert \leq\sqrt{n}\vert\boldsymbol{v}\vert_{1,\Omega}\Vert q\Vert_{0,\Omega}$

\item $\vert c(\bfu,\bfv,\bfw) \vert\leq\beta\vert\bfu \vert_{1,\Omega}\vert\bfv\vert_{1,\Omega}\vert\bfw\vert_{1,\Omega}$

\item $\sup\limits_{\bfv\in H\setminus\{\mathbf{0}\}}\dfrac {b(\bfv,q)_\Omega}{\vert \bfv\vert _{1,\Omega}} \geq\delta\Vert q\Vert_{0,\Omega}$
\end{enumerate}
\end{lemma}

\begin{theorem}\label{f-bound}
Let $\gamma\in\mathcal{A}$. If there exists a constant $C\in(0,1)$ such that $\vert(\bff,\bfv)_{0,\Omega}\vert\leq C\dfrac{\nu^{2}}{\beta}\vert\bfv\vert_{1,\Omega}$ for all $\bfv\in H$, then \eqref{VF} has an unique solution $(\bfu,p)\in H\times Q$ and it holds $\vert\bfu\vert_{1,\Omega}\leq C\dfrac{\nu}{\beta}$. Furthermore, there exists $C>0$ such that%
\begin{align*}
\vert \bfu\vert _{1,\Omega}  &  \leq C\Vert\bff\Vert _{0,\Omega}\\
\Vert p\Vert_{0,\Omega}  &  \leq C(\Vert \bff\Vert_{0,\Omega}+\Vert \bff\Vert _{0,\Omega}^{2})
\end{align*}
\end{theorem}

\begin{proof}
See Chapter 9 in \cite{G11}.
\end{proof}

In that follows, we assume that $\bff$ fulfills the hypothesis of Theorem \ref{f-bound}.
\begin{definition}
We denote the control-to-state map as $A:\mathcal{A}\rightarrow H\times Q$, where $A(\gamma)=(\bfu,p)$ is the solution of \eqref{VF}.
\end{definition}

\begin{definition}
Given $\gamma\in\mathcal{A}$, we define the adjoint equation as the variational formulation given by
\textit{Find $(\bfv,q)\in V$ such that for all $(\bfw,r)\in V$}
\begin{align}
a(\bfv,\bfw)-b(\boldsymbol{r},q)+b(\bfv,r)+c(\bfu,\bfw,\bfv)+c(\bfw,\bfu,\bfv)+(\gamma\bfv,\bfw)=(\bfu-\bfu_{0},\bfw)_{\omega}\label{AVF}%
\end{align}
where $(\bfu,p)=A(\gamma)$.
\end{definition}

\begin{lemma}\label{adjoint}
Given $\gamma\in\mathcal{A}$, the adjoint problem \eqref{AVF} has an unique solution $(\bfv,q)\in H\times Q$ such that%
\begin{align*}
\vert\bfv\vert_{1,\Omega}  &  \leq C\Vert\bfu-\bfu%
_{0}\Vert_{0,\omega}\\
&  \leq C(\Vert\bfu_{0}\Vert_{0,\omega}+\Vert\bff%
\Vert_{0,\Omega})\\
\Vert q\Vert_{0,\Omega}  &  \leq C(\Vert\bfu-\bfu_{0}%
\Vert_{0,\omega}+\vert\bfv\vert_{1,\Omega}(1+\Vert\bff%
\Vert_{0,\Omega})\\
&  \leq C\Vert\bff\Vert_{0,\Omega}(\Vert\bfu_{0}%
\Vert_{0,\omega}+\Vert\bff\Vert_{0,\Omega})
\end{align*}
\end{lemma}
\begin{proof}
It is a direct consequence of the Babuzka-Brezzi theory. We omit the details.
\end{proof}

\begin{definition}
We denote the control-to-adjoint-state map as $B:\mathcal{A}\rightarrow H\times Q$, where $B(\gamma)=(\bfv,q)$ is the solution of \eqref{AVF}.
\end{definition}

\begin{lemma}\label{h2}
Let $\gamma\in\mathcal{A}$, and $(\bfu,p),(\bfv,q)\in H\times Q$ such that $A(\gamma)=(\bfu,p)$ and $B(\gamma)=(\bfv,q)$. Then, $\bfu,\bfv\in\bfH%
^{2}(\Omega)$ and $p,q\in H^{1}(\Omega)$. Furthermore, the following estimates are verified
\begin{align*}
\vert\bfu\vert_{2,\Omega}+\Vert p\Vert_{1,\Omega}  &  \leq C(\Vert\bff\Vert_{0,\Omega}+\Vert\bff\Vert_{0,\Omega}^{2})\\
\vert\bfv\vert_{2,\Omega}+\Vert q\Vert_{1,\Omega}  &  \leq C\Vert\bff\Vert_{0,\Omega}(\Vert\bff\Vert_{0,\Omega}+\left\Vert \bfu_{0}\right\Vert _{0,\omega})
\end{align*}
for a positive constant $C>0$ independent on $\bfu$, $\bfv$, $\bff$ and $\bfu_{0}$.
\end{lemma}

\begin{proof}
See \cite{G92}.
\end{proof}

\begin{lemma}\label{lipschitz}
There exist positive constants $C_1$ and $C_2$ such that for all $\gamma_{1},\gamma_{2}\in\mathcal{A}$
\begin{enumerate}
\item $\vert A(\gamma_{1})-A(\gamma_{2})\vert\leq C\Vert\gamma_{1}-\gamma_{2}\Vert_{0,\Omega}$
\item $\vert B(\gamma_{1})-B(\gamma_{2})\vert\leq C\Vert\gamma_{1}-\gamma_{2}\Vert_{0,\Omega}$.
\end{enumerate}
\end{lemma}

\begin{proof}
See Lemmas 11 and 20 in \cite{ABO23}.
\end{proof}

\begin{theorem}\label{existence}
The optimal control problem \eqref{min} admits at least one global solution $(\boldsymbol{\bar{u}},\bar{p},\bar{\gamma})\in H\times Q\times\mathcal{A}$.
\end{theorem}
\begin{proof}
See Theorem 10 in \cite{ABO23}.
\end{proof}

\begin{proposition}
The functional $J$ and the operators $A$ and $B$ and twice Fr\'{e}chet-differentiable with respect to the $L^{\infty}(\Omega)$ topology. Given $\varphi_{1},\varphi_{2}\in L^{\infty}(\Omega)$ and $\gamma\in\mathcal{A}$, the derivatives of $A$ have the following properties

\begin{enumerate}
\item $A^{\prime}(\gamma)[\varphi_{1}]=(\bfu^{\prime},p^{\prime})$
is the solution of
\[
(\forall(\bfv,q)\in H\times Q)\text{\quad}a(\bfu^{\prime},\bfv)-b(\bfv,p^{\prime})+b(\bfu^{\prime},q)+c(\bfu^{\prime},\bfu,\bfv)+c(\bfu,\bfu^{\prime},\bfv)+(\gamma\bfu^{\prime},\bfv)=-(\varphi_{1}\bfu,\bfv)
\]
where $A(\gamma)=(\bfu,p)$.

\item $A^{\prime\prime}(\gamma)[\varphi_{1},\varphi_{2}]=(\bfu^{\prime\prime},p^{\prime\prime})$ is the solution of%
\[
(\forall(\bfv,q)\in H\times Q)\text{\quad}a(\bfu^{\prime\prime},\bfv)-b(\bfv,p^{\prime\prime})+b(\bfu^{\prime\prime},q)+c(\bfu^{\prime\prime},\bfu,\bfv)+c(\bfu,\bfu^{\prime\prime},\bfv)+(\gamma\bfu^{\prime\prime},\bfv)=-\left[  (\varphi_{1}\bfu_{2}^{\prime},\bfv)+(\varphi_{2}\bfu_{1}^{\prime},\bfv)\right]
\]
where $A^{\prime}(\gamma)[\gamma_{1}]=(\bfu_{1}^{\prime},p_{1}^{\prime})$ and $A^{\prime}(\gamma)[\gamma_{2}]=(\bfu_{2}^{\prime},p_{2}^{\prime})$.
\end{enumerate}
\end{proposition}
\begin{proof}
See Theorems 11 and 18 in \cite{ABO23}.
\end{proof}
\begin{proposition}
If $\bar{\gamma}\in\mathcal{A}$ is a local solution of \eqref{min}, then the following inequality holds
\begin{equation}
(\forall\varphi\in\mathcal{A})\text{\quad}J^{\prime}(\bar{\gamma})[\varphi-\bar{\gamma}]\geq0 \label{first-order}
\end{equation}

\end{proposition}

The expression $J^{\prime}(\gamma)[\varphi]$ can be rewritten as
\begin{align*}
J^{\prime}(\gamma)[\varphi]  &  =(\bfu-\bfu_{0},\bfu^{\prime})_{\omega}+\alpha(\gamma-\gamma_{0},\varphi)\\
                             &  =(\varphi\bfu,\bfv)+\alpha(\gamma-\gamma_{0},\varphi)\\
														 &  =(\varphi,\bfu\cdot\bfv+\alpha(\gamma-\gamma_{0}))
\end{align*}
where $A(\gamma)=(\bfu,p)$ and $B(\gamma)=(\bfv,q)$.

\begin{definition}
For the constants $a,b\in\RR$, with $0\leq a<b$, the projection operator $\Pi_{\lbrack a,b]}$ is defined in $L^{2}(\Omega)$ as%
\[
(\forall\varphi\in L^{2}(\Omega))\text{\quad}\Pi_{\lbrack a,b]}(\varphi
)(x)=\min\{b,\max\{\varphi(x),a\}\}
\]
\end{definition}

Every local solution $\bar{\gamma}\in\mathcal{A}$ of \eqref{min} that satisfies \eqref{first-order} verifies the identity%
\[
\bar{\gamma}=\Pi_{\lbrack a,b]}\left(  \gamma_{0}+\dfrac{1}{\alpha
}\boldsymbol{\bar{u}}\cdot\boldsymbol{\bar{v}}\right)
\]
where $A(\bar{\gamma})=(\boldsymbol{\bar{u}},\bar{p})$ and $B(\bar{\gamma})=(\boldsymbol{\bar{v}},\bar{q})$.

\begin{proposition}
Let $\varphi_{1},\varphi_{2}\in L^{\infty}(\Omega)$ and $\gamma\in\mathcal{A}$ such that $A^{\prime}(\gamma)[\varphi_{1}]=(\bfu_{1}^{\prime},p_{1}^{\prime})$, $A^{\prime}(\gamma)[\varphi_{2}]=(\bfu_{2}^{\prime},p_{2}^{\prime})$, $A^{\prime\prime}(\gamma)[\varphi_{1},\varphi_{2}]=(\bfu^{\prime\prime},p^{\prime\prime})$. Then, there exists a constant $C>0$ such that

\begin{enumerate}
\item $\Vert\bfu\Vert_{2,\Omega}\leq C\Vert\bff\Vert_{0,\Omega}$

\item $\Vert\bfu_{j}^{\prime}\Vert_{0,\Omega}\leq C\vert\bfu\vert_{1,\Omega}\Vert\varphi_{j}\Vert_{0,\Omega}$ for $j\in\{1,2\}$

\item $\Vert\bfu_{j}^{\prime}\Vert_{2,\Omega}\leq C\Vert\varphi_{j}\Vert_{0,\Omega}$ for $j\in\{1,2\}$

\item $\Vert\bfu^{\prime\prime}\Vert_{0,\Omega}\leq C\vert\bfu\vert_{1,\Omega}\Vert\varphi_{1}\Vert_{0,\Omega}\Vert\varphi_{2}\Vert_{0,\Omega}$

\item $\Vert\bfu^{\prime\prime}\Vert_{2,\Omega}\leq C\Vert\varphi_{1}\Vert_{0,\Omega}\Vert\varphi_{2}\Vert_{0,\Omega}$
\end{enumerate}
\end{proposition}

\begin{theorem}\label{H2}
Let $\bar{\gamma}\in\mathcal{A}$ a local solution of \eqref{min} for $\gamma_{0}\in H^{1}\left(  \Omega\right)  $. Then, $\gamma\in H^{1}(\Omega)$ and there exists a positive constant $C>0$ such that
\[
\Vert\bar{\gamma}\Vert_{1,\Omega}\leq\Vert\gamma_{0}\Vert_{1,\Omega}+C\Vert\bff%
\Vert_{0,\Omega}(\Vert\bff\Vert_{0,\Omega}+\Vert\bfu%
_{0}\Vert_{0,\Omega})
\]

\end{theorem}

\begin{proof}
From Lemma \ref{h2}, if $A(\bar{\gamma})=(\boldsymbol{\bar{u}},\bar{p})$ and $B(\bar{\gamma})=(\boldsymbol{\bar{v}},\bar{q})$, then $\boldsymbol{\bar{u}},\boldsymbol{\bar{v}}\in\bfH^{2}(\Omega)$ and $\boldsymbol{\bar {u}},\boldsymbol{\bar{v}}\in\bfL^{\infty}(\Omega)$ by Sobolev Embeding Theorem. Since $\bfH^{2}(\Omega)$ is a Banach algebra for $d\in\{2,3\}$, then $\boldsymbol{\bar{u}}\cdot\boldsymbol{\bar{v}}\in H^{2}(\Omega)$ allow us to deduce that $\nabla(\boldsymbol{\bar{u}}\cdot\boldsymbol{\bar{v}})\in\bfL^{2}(\Omega)$. Since $\Pi_{[a,b]}:H^{1}(\Omega)\rightarrow H^{1}(\Omega)$ is continuous, we can conclude that $\bar{\gamma}\in H^{1}(\Omega)$. The bound is a direct consequence of Lemma \ref{H2}.
\end{proof}

\begin{proposition}
Let $\varphi\in L^{\infty}(\Omega)$ and $\gamma\in\mathcal{A}$. If $B(\gamma)=(\bfv,q)$ and $A^{\prime}(\gamma)[\varphi]=(\bfu^{\prime},p^{\prime})$, then
\[
J^{\prime\prime}(\gamma)[\varphi,\varphi]=\Vert\bfu^{\prime}\Vert_{0,\omega}^{2}-2(\varphi\bfu^{\prime}+(\nabla\bfu^{\prime})\bfu^{\prime},\bfv)+\alpha\Vert\varphi\Vert_{0,\Omega}^{2}
\]
Furthermore, there exists a positive constant $C>0$ such that for all $\gamma_{1},\gamma_{2}\in\mathcal{A}$%
\[
\vert J^{\prime\prime}(\gamma_{1})[\varphi,\varphi]-J^{\prime\prime}(\gamma_{2})[\varphi,\varphi]\vert\leq C\left\Vert \gamma_{1}-\gamma_{2}\right\Vert_{0,\Omega}\left\Vert \varphi\right\Vert _{0,\Omega}^{2}%
\]
\end{proposition}

\begin{proof}
See Section 6 in \cite{ABO23}.
\end{proof}

\begin{remark}\label{assumption-1}
In that follows, we assume that there exist a positive constant $\delta_{1}>0$ such that for all $\varphi\in L^{\infty}(\Omega)$ and for every local solution $\bar{\gamma}$ that fulfills \eqref{first-order}
\begin{align}
J^{\prime\prime}(\bar{\gamma})[\varphi,\varphi]\geq\delta_{1}\Vert\varphi\Vert_{0,\Omega}^{2}. \label{second-order}
\end{align}
Indeed, this assumption can be fulfilled when $\Vert\boldsymbol{\bar{u}}-\bfu_{0}\Vert_{0,\omega}$ is small enough or $\alpha$ is large enough, since
\begin{align*}
J^{\prime\prime}(\bar{\gamma})[\varphi,\varphi]  &  =\Vert\boldsymbol{\bar{u}}^{\prime}\Vert_{0,\omega}^{2}-(\boldsymbol{\bar{u}}-\bfu_{0},\boldsymbol{\bar{u}}^{\prime\prime})_{\omega}+\alpha\Vert\varphi\Vert_{0,\Omega}^{2}\\
																								 &  \geq\alpha\Vert\varphi\Vert_{0,\Omega}^{2}-\Vert\boldsymbol{\bar{u}}-\bfu_{0}\Vert_{0,\omega}\Vert\boldsymbol{\bar{u}}^{\prime\prime}\Vert_{0,\Omega}\\
																								 &  \geq\alpha\Vert\varphi\Vert_{0,\Omega}^{2}-C\Vert\boldsymbol{\bar{u}}-\bfu_{0}\Vert_{0,\omega}(1+\Vert\bar{\gamma}\Vert_{0,\Omega})\Vert\varphi\Vert_{0,\Omega}^{2}\\
																								 &  \geq(\alpha-C\Vert\boldsymbol{\bar{u}}-\bfu_{0}\Vert_{0,\omega}(1+\Vert\bar{\gamma}\Vert_{0,\Omega}))\Vert\varphi\Vert_{0,\Omega}^{2}
\end{align*}
\end{remark}

\begin{definition}
Consider $\varepsilon>0$. Given $\bar{\gamma}\in\mathcal{A}$ a local solution for \eqref{min}, we denote%
\[
\mathcal{A}_{\varepsilon}(\bar{\gamma})=\{\gamma\in\mathcal{A}\text{ }%
\mid\text{ }\Vert\gamma-\bar{\gamma}\Vert\leq\varepsilon\}
\]
\end{definition}

\begin{lemma}\label{ellipticity}
Let $\bar{\gamma}\in\mathcal{A}$ a local solution of \eqref{min} that verifies the Assumption \eqref{second-order}. There exists a constant $\varepsilon>0$ such that for all $\varphi\in L^{\infty}(\Omega)$ and all $\gamma\in\mathcal{A}_{\varepsilon}(\bar{\gamma})$
\[
J^{\prime\prime}(\gamma)[\varphi,\varphi]\geq\dfrac{\delta}{2}\Vert
\varphi\Vert_{0,\Omega}^{2}%
\]
\end{lemma}
\begin{proof}
It is a straightforward consequence of Remark \ref{assumption-1}.
\end{proof}

\begin{theorem}
Let $\bar{\gamma}\in\mathcal{A}$ a local solution of \eqref{min} that verifies \eqref{first-order} and the Assumption \eqref{second-order}. There exist positive constants $\sigma$ and $\varepsilon$ such that for all $\gamma\in\mathcal{A}_{\varepsilon}(\bar{\gamma})$.
\[
J(\gamma)\geq J(\bar{\gamma})+\sigma\Vert\gamma-\bar{\gamma}\Vert_{0,\Omega
}^{2}%
\]
\end{theorem}
\begin{proof}
See Theorem 25 in \cite{ABO23}.
\end{proof}

\section{Discretization by finite elements of the states and adjoint equations}
Consider $\Omega\subseteq\RR^{n}$ as a bounded polygonal domain. Let $\left\{  \mathcal{T}_{h}\right\}  _{h>0}$ be a shape-regular family of triangulations of $\overline{\Omega}$ composed of triangles (if $n=2$) or tetrahedron (if $n=3$), with $h:=\max\left\{  h_{T}\mid T\in\mathcal{T}_{h}\right\}  $, where $h_{T}:=\operatorname{diam}(T)$ the diameter of $T\in\mathcal{T}_{h}$. We define $\mathcal{E}_{h}$ as the set of all edges (faces) of a given triangulation $\mathcal{T}_{h}$.

\begin{definition}\label{discrete-spaces}
We denote $H^{h}$ the continuous Lagrange finite element vector space with degree $2$ on $\overline{\Omega}$, that is,%
\[
H^{h}:=\{\bfv^{h}\in\boldsymbol{C}(\overline{\Omega})  \mid(\forall T\in\mathcal{T}_{h})  \quad\bfv^{h}\vert_{T}\in\mathbb{P}_{2}(T)^{n}\} \cap H,
\]
where $\mathbb{P}_{2}(T)$ is the space of polynomials of total degree at most $2$ defined on $T$. Analogously,w e denote $Q^{h}$ the continuous Lagrange finite element space with degree $1$ on $\overline{\Omega}$, that is,
\[
Q^{h}:=\{q^{h}\in C(\overline{\Omega})  \mid(\forall T\in\mathcal{T}_{h})  \quad q^{h}\vert_{T}\in\mathbb{P}_{1}(T)\} \cap Q,
\]
where $\mathbb{P}_{1}(T)$ is the space of polynomials of total degree at most $1$ defined on $T$. We denote the Taylor-Hood finite element as $V^{h}=H^{h}\times Q^{h}$.
\end{definition}

\begin{theorem}
Let $\gamma\in\mathcal{A}$, $\bfu_{0}\in\bfL^{2}(\omega)$, $\gamma_{0}\in L^{2}(\Omega)$ and $\bff\in\bfL^{2} (\Omega)$ such that verifies the hypothesis of Theorem \ref{f-bound}. If the solutions $(\bfu,p)$ for \eqref{VF} and $(\bfv,q)$ for \eqref{AVF} are smooth enough, i.e. $\bfu,\bfv\in\bfH^{2}(\Omega)\cap H$ and $p,q\in H^{1}(\Omega)\cap Q$, then the solution of the discrete variational formulations given by

\textit{Find $(\bfu^h,p^h),(\bfv^h,q^h)\in V^{h}$ such that for all $(\bfw^{h},r^{h})\in V^{h}$}
\begin{align}
a(\bfu^{h},\bfw^{h})-b(\bfw^{h},p^{h})+b(\bfu^{h},r^{h})+c(\bfu^{h},\bfu^{h},\bfw^{h})+(\gamma\bfu^{h},\bfv^{h})  & =(\bff,\bfw^{h})_\Omega\label{VFh}\\
a(\bfv^{h},\bfw^{h})-b(\boldsymbol{r}^{h},q^{h})+b(\bfv^{h},r^{h})+c(\bfu^{h},\bfv^{h},\bfw^{h})+c(\bfv^{h},\bfu^{h},\bfw^{h})+(\gamma\bfv^{h},\bfw^{h})  &  =(\bfu^{h}-\bfu_{0},\bfw^{h})_{\omega} \label{AVFh}%
\end{align}
is unique and the following error estimates hold
\begin{align*}
\Vert (\bfu-\bfu^{h},p-p^{h})\Vert    &  \leq Ch(\vert\bfu\vert_{2,\Omega}+\Vert p\Vert_{1,\Omega})\\
\Vert \bfu-\bfu^{h}\Vert _{0,\Omega}  &  \leq Ch^{2}(\vert\bfu\vert_{2,\Omega}+\Vert p\Vert_{1,\Omega})\\
\Vert (\bfv-\bfv^{h},q-q^{h})\Vert    &  \leq Ch(\vert\bfv\vert_{2,\Omega}+\Vert q\Vert_{1,\Omega})\\
\Vert \bfv-\bfv^{h}\Vert _{0,\Omega}  &  \leq Ch^{2}(\vert\bfv\vert_{2,\Omega}+\Vert q\Vert_{1,\Omega})
\end{align*}
\end{theorem}

\begin{proof}
See Sections 5.2 and 6.2 in \cite{J16}.
\end{proof}

\begin{definition}
We denote as $A_{h}:\mathcal{A}\rightarrow V^h$ and $B_{h}:\mathcal{A}\rightarrow V^h$ are the discretized control-to-state and control-to-adjoint-state operators such that $A_{h}(\gamma)=(\bfu^{h},p^{h})$ and $B_{h}(\gamma)=(\bfv^{h},q^{h})$ are the solutions of \eqref{VFh} and \eqref{AVFh}, respectively. Analogously, we define the functional $J_{h}:\mathcal{A}\rightarrow\RR$ given by $J_{h}(\gamma)=\dfrac{1}{2}\Vert\bfu_{h}-\bfu_{0}\Vert_{0,\omega}^{2}+\dfrac{\alpha}{2}\Vert\gamma-\gamma_{0}\Vert_{0,\Omega}^{2}$, where $A_{h}(\gamma)=(\bfu^{h},p^{h})$.
\end{definition}

\begin{remark}
The Fr\'{e}chet-differenciability of $A_{h}$, $B_{h}$ and $J_{h}$ is inherited from $A$, $B$ and $J$, respectively.
\end{remark}

\begin{lemma}
Let $\gamma\in\mathcal{A}$, $A_{h}(\gamma)=(\bfu^{h},p^{h})$ and $B_{h}(\gamma)=(\bfv^{h},q^{h})$. Then, $\Vert\bfu^{h}\Vert_{0,\infty,\Omega}\leq C\Vert\bff\Vert_{0,\Omega}$ and $\Vert\bfu^{h}\Vert_{0,\infty,\Omega}\leq C(\Vert\bfu_{0}\Vert_{0,\omega}+\Vert\bff\Vert_{0,\Omega})$.
\end{lemma}

\begin{proof}
By Sobolev Embedding Theorem, Theorem 3.24 from \cite{EG04} and Theorem \ref{f-bound}, we have%
\[
\Vert\bfu^{h}\Vert_{0,\infty,\Omega}\leq C\Vert\bfu%
^{h}\Vert_{1,4,\Omega}\leq C\vert\bfu\vert_{1,4,\Omega}\leq
C\vert\bfu\vert_{2,\Omega}\leq C\Vert\bff\Vert_{0,\Omega}%
\]
The second estimate is obtained by the same way.
\end{proof}


\begin{lemma}
Let $\gamma\in\mathcal{A}$ and $\varphi\in L^{\infty}(\Omega)$. Then,

\begin{enumerate}
\item $\vert A^{\prime}(\gamma)[\varphi]-A_{h}^{\prime}(\gamma)[\varphi]\vert\leq Ch\Vert\varphi\Vert_{0,\Omega}(\Vert\bff\Vert_{0,\Omega}+\Vert\bff\Vert_{0,\Omega}^{2})$.

\item $\vert A^{\prime\prime}(\gamma)[\varphi,\varphi]-A_{h}^{\prime\prime}(\gamma)[\varphi,\varphi]\vert\leq Ch^{2}\Vert\varphi\Vert_{0,\Omega}^{2}(\Vert\bff\Vert_{0,\Omega}+\Vert\bff\Vert_{0,\Omega}^{2})^{2}$
\end{enumerate}
\end{lemma}

\begin{proof}
If $A(\gamma)=(\bfu,p)$ and $A_{h}(\gamma)=(\bfu^{h},p^{h})$, we have that $A^{\prime}(\gamma)[\varphi]=(\bfu^{\prime},p^{\prime})$ and $A_{h}^{\prime}(\gamma)[\varphi]=(\bfu_{h}^{\prime},p_{h}^{\prime})$ verifies the identities
\begin{align*}
a(\bfu^{\prime},\bfv)-b(\bfv,p^{\prime})+b(\bfu^{\prime},q)+c(\bfu^{\prime},\bfu,\bfv)+c(\bfu,\bfu^{\prime},\bfv)+(\gamma \bfu^{\prime},\bfv)=-(\varphi\bfu,\bfv)\\
a(\bfu_{h}^{\prime},\bfv^{h})-b(\bfv^{h},p_{h}^{\prime})+b(\bfu_{h}^{\prime},q^{h})+c(\bfu_{h}^{\prime},\bfu^{h},\bfv^{h})+c(\bfu^{h},\bfu_{h}^{\prime},\bfv^{h})+(\gamma\bfu_{h}^{\prime},\bfv^{h})=-(\varphi\bfu^{h},\bfv^{h})
\end{align*}
for all $(\bfv,q) \in V$ and $(\bfv^h,q^h) \in V$. If $(\boldsymbol{\hat{u}}^{h},\hat{p}^{h})$ are the solutions of%
\[
(\forall(\bfv^{h},q^{h})\in H^{h}\times Q^{h})\quad
a(\boldsymbol{\hat{u}}^{h},\bfv^{h})-b(\bfv^{h},\hat
{p}^{h})+b(\boldsymbol{\hat{u}}^{h},q^{h})+c(\boldsymbol{\hat{u}}%
^{h},\bfu,\bfv^{h})+c(\bfu,\boldsymbol{\hat{u}%
}^{h},\bfv^{h})+(\gamma\boldsymbol{\hat{u}}^{h},\bfv%
^{h})=-(\varphi\bfu,\bfv^{h})
\]
Then,%
\begin{align*}
\Vert (\bfu^{\prime}-\boldsymbol{\hat{u}}^{h},p^{\prime}-\hat{p}^{h})\Vert  &  \leq Ch\Vert\varphi\Vert_{0,\Omega}\vert\bfu\vert_{1,\Omega}\\
&  \leq Ch\Vert\varphi\Vert_{0,\Omega}\Vert\bff\Vert_{0,\Omega}\\
\Vert (\bfu_{h}^{\prime}-\boldsymbol{\hat{u}}^{h},p_{h}^{\prime}-\hat{p}^{h})\Vert  &  \leq C\Vert\varphi\Vert_{0,\Omega}\Vert\bfu-\bfu^{h}\Vert_{1,\Omega}\\
&  \leq Ch\Vert\varphi\Vert_{0,\Omega}(\vert\bfu\vert_{2,\Omega}+\Vert p\Vert_{1,\Omega})\\
&  \leq Ch\Vert\varphi\Vert_{0,\Omega}(\Vert\bff\Vert_{0,\Omega}+\Vert\bff\Vert_{0,\Omega}^{2})
\end{align*}
By Triangle inequality, we conclude
\[
\Vert A^{\prime}(\gamma)[\varphi]-A_{h}^{\prime}(\gamma)[\varphi]\Vert\leq
Ch\Vert\varphi\Vert_{0,\Omega}(\Vert\bff\Vert_{0,\Omega}%
+\Vert\bff\Vert_{0,\Omega}^{2})
\]
The second estimate is obtained following similar steps.
\end{proof}

\begin{lemma}\label{aux-1}
Let $\gamma,\gamma_{1},\gamma_{2}\in\mathcal{A}$, $\varphi\in L^{\infty}(\Omega)$. Then,

\begin{enumerate}
\item $\left\vert J^{\prime}(\gamma)[\varphi]-J_{h}^{\prime}(\gamma)[\varphi]\right\vert \leq Ch^{2}\Vert\varphi\Vert_{0,\Omega}\Vert\bff\Vert_{0,\Omega}(\Vert\bff\Vert_{0,\Omega}+\Vert\bfu_{0}\Vert_{0,\omega})$

\item $\left\vert J_{h}^{\prime}(\gamma_{1})[\varphi]-J_{h}^{\prime}(\gamma_2)[\varphi]\right\vert \leq C\Vert\gamma_{1}-\gamma_{2}\Vert_{0,\Omega}\Vert\varphi\Vert_{0,\Omega}$
\end{enumerate}
\end{lemma}

\begin{proof}
If $A(\gamma)=(\bfu,p)$, $A_{h}(\gamma)=(\bfu^{h},p^{h})$, $B(\gamma)=(\bfv,q)$ and $B_{h}(\gamma)=(\bfv^{h},q^{h})$, then
\begin{align*}
\left\vert J^{\prime}(\gamma)[\varphi]-J_{h}^{\prime}(\gamma)[\varphi]\right\vert  &  =\left\vert (\varphi,\bfu\cdot\bfv)-(\varphi,\bfu^{h}\cdot\bfv^{h})\right\vert \\
&  =\left\vert (\varphi,(\bfu-\bfu^{h})\cdot\bfv)+(\varphi,\bfu^{h}\cdot(\bfv-\bfv^{h}))\right\vert \\
&  \leq\left\vert (\varphi,(\bfu-\bfu^{h})\cdot \bfv)\Vert+\Vert(\varphi,\bfu^{h}\cdot(\bfv-\bfv^{h}))\right\vert \\
&  \leq C\Vert\varphi\Vert_{0,\Omega}(\Vert\bfu-\bfu^{h}\Vert_{0,\Omega}\vert\bfv\vert_{0,\infty,\Omega}+\Vert\bfv-\bfv^{h}\Vert_{0,\Omega}\Vert\bfu^{h}\Vert_{0,\infty,\Omega})\\
&  \leq Ch^{2}\Vert\varphi\Vert_{0,\Omega}\Vert\bff\Vert_{0,\Omega}(\Vert\bff\Vert_{0,\Omega}+\vert\bfu\vert_{0,\omega})
\end{align*}
proving the first estimate. The second estimate is similar to the proof of Lemma \ref{lipschitz}.
\end{proof}

\begin{lemma}\label{ellipticity-h}
Let $\bar{\gamma}\in\mathcal{A}$ a local solution of \eqref{min} that verifies the Assumption \eqref{second-order}. There exists $h_0>0$ and a constant $\varepsilon>0$ such that for all $\varphi\in L^{\infty}(\Omega)$ and all $\gamma\in\mathcal{A}_{\varepsilon}(\bar{\gamma})$
\[
J_{h}^{\prime\prime}(\gamma)[\varphi,\varphi]\geq\dfrac{\delta}{4}\Vert
\varphi\Vert_{0,\Omega}^{2}%
\]
that holds for any $h\in(0,h_0)$.
\end{lemma}

\begin{proof}
First, we have $J^{\prime\prime}(\gamma)[\varphi,\varphi]\geq \dfrac{\delta}{2}\Vert\varphi\Vert_{0,\Omega}^{2}$ from Lemma \ref{ellipticity}. Then, from Lemma \ref{aux-1},
\begin{align*}
\vert J_{h}^{\prime\prime}(\gamma)[\varphi,\varphi]-J^{\prime\prime}(\gamma)[\varphi,\varphi]\vert  &  =\vert\Vert\bfu_{h}^{\prime}\Vert_{0,\omega}^{2}-(\bfu^{h}-\bfu_{0},\bfu_{h}^{\prime\prime})_{\omega}-(\Vert\bfu^{\prime}\Vert_{0,\omega}^{2}-(\bfu-\bfu_{0},\boldsymbol{\bar{u}}^{\prime\prime})_{\omega})\vert\\
&  \leq\vert\Vert\bfu_{h}^{\prime}\Vert_{0,\omega}^{2}-\Vert\bfu^{\prime}\Vert_{0,\omega}^{2}\vert+\vert(\bfu-\bfu_{0},\bfu^{\prime\prime})_{\omega}-(\bfu^{h}-\bfu_{0},\bfu_{h}^{\prime\prime})_{\omega}\vert\\
&  \leq\vert\Vert\bfu_{h}^{\prime}\Vert_{0,\omega}^{2}-\Vert\bfu^{\prime}\Vert_{0,\omega}^{2}\vert+\vert(\bfu-\bfu^{h},\bfu^{\prime\prime})_{\omega}\vert+\vert(\bfu^{h}-\bfu_{0},\bfu_{h}^{\prime\prime})_{\omega}\vert\\
&  \leq Ch\Vert\varphi\Vert_{0,\Omega}^{2}(\Vert\bff\Vert_{0,\Omega}+\Vert\bff\Vert_{0,\Omega}^{2})^{2}%
\end{align*}
Applying Triangle inequality,
\begin{align*}
J_{h}^{\prime\prime}(\gamma)[\varphi,\varphi]  &  =J^{\prime\prime}(\gamma)[\varphi,\varphi]+J_{h}^{\prime\prime}(\gamma)[\varphi,\varphi]-J^{\prime\prime}(\gamma)[\varphi,\varphi]\\
&  \geq\left(  \dfrac{\delta}{2}-Ch(\Vert\bff\Vert_{0,\Omega}+\Vert\bff\Vert_{0,\Omega}^{2})^{2}\right)  \Vert\varphi\Vert_{0,\Omega}^{2}
\end{align*}
Choosing $h_0=\dfrac{\delta}{4(\Vert\bff\Vert_{0,\Omega} +\Vert\bff\Vert_{0,\Omega}^{2})^{2}}$, we have that $\left(\dfrac{\delta}{2}-Ch(\Vert\bff\Vert_{0,\Omega}+\Vert\bff\Vert_{0,\Omega}^{2})^{2}\right)  \geq\dfrac{\delta}{4}$ for any $h\in(0,h_0)$, concluding that $J_{h}^{\prime\prime}(\gamma)[\varphi,\varphi]\geq\dfrac{\delta}{4}\Vert\varphi\Vert_{0,\Omega}^{2}$.
\end{proof}

\begin{theorem}
The semidiscrete optimal control problem
\begin{align}
\text{minimize\quad}  &  J_{h}(\gamma)=\dfrac{1}{2}\Vert\bfu^{h}-\bfu_{0}\Vert_{0,\omega}^{2}+\dfrac{\alpha}{2}\Vert\gamma-\gamma_{0}\Vert_{0,\Omega}^{2} \label{semi-h} \\
\text{subject to\quad}  &  (\forall(\bfw^{h},r^{h})\in V^h)\text{\quad}a(\bfu^{h},\bfw^{h})-b(\bfw^{h},p^{h})+b(\bfu^{h},r^{h})+c(\bfu^{h},\bfu^{h},\bfw^{h})+(\gamma\bfu^{h},\bfv^{h})=(\bff,\bfw^{h})\nonumber\\
&  \gamma\in\mathcal{A}\nonumber
\end{align}
admits at least one global solution $(\boldsymbol{\bar{u}}^{h},\bar{p}^{h},\bar{\gamma})\in H^{h}\times Q^{h}\times\mathcal{A}$.
\end{theorem}
\begin{proof}
The proof is similar to the one for Theorem \ref{existence}. We omit the details.
\end{proof}

\section{Discretization of the optimal control problem}

\begin{definition}
We denote
\begin{align*}
G_{0}^{h}  &  :=\left\{  \gamma^{h}\in L^{2}(\Omega)\mid\left(  \forall T\in\mathcal{T}_{h}\right)  (\exists a_{T}\in\RR)\quad\gamma^{h}\vert_{T}=a_{T}\right\} \\
G_{1}^{h}  &  :=\left\{  \gamma^{h}\in C\left(  \overline{\Omega}\right) \mid\left(  \forall T\in\mathcal{T}_{h}\right)  \quad\gamma^{h}\vert_{T}\in\mathbb{P}_{1}(T)\right\}
\end{align*}

\end{definition}

The spaces $G_{0}^{h}$ and $G_{1}^{h}$ are some possible finite element spaces for the optimal control, corresponding to discontinuous and continuous Galerkin finite element spaces with the lowest polynomial degree. In that case, $\mathcal{A}$ can be changed by $\mathcal{A}^{h}=\mathcal{A}\cap G_{0}^{h}$ or$\mathcal{A}^{h}=\mathcal{A}\cap G_{1}^{h}$ in the semidiscrete optimal control problem obtaining a fully discrete problem. This problem also has at least one global solution $(\boldsymbol{\bar{u}}^{h},\bar{p}^{h},\bar{\gamma}^{h})\in H^{h}\times Q^{h}\times\mathcal{A}^{h}$.

In the final subsection, we also analyze one alternative that consists of a semidiscrete optimal control problem, where only the states and adjoints are discretized. The optimal control must also fulfills the the optimality conditions obtained the previous sections. Then, every local solution $\bar{\gamma}^{h}\in\mathcal{A}$ must verify the identity
\[
\bar{\gamma}^{h}=\Pi_{\lbrack a,b]}\left(  \gamma_{0}+\dfrac{1}{\alpha
}\boldsymbol{\bar{u}}^{h}\cdot\boldsymbol{\bar{v}}^{h}\right)
\]
In the first two subsections, we consider the cases $\mathcal{A}^{h}=\mathcal{A}\cap G_{0}^{h}$ or $\mathcal{A}^{h}=\mathcal{A}\cap G_{1}^{h}$.

\begin{definition}
We define the discrete optimal control problem as
\begin{align}
\text{minimize\quad}  &  J_{h}(\gamma^h)=\dfrac{1}{2}\Vert\bfu^{h}-\bfu_{0}\Vert_{0,\omega}^{2}+\dfrac{\alpha}{2}\Vert\gamma^{h}-\gamma_{0}\Vert_{0,\Omega}^{2} \label{min-h}\\
\text{subject to\quad}  &  (\forall(\bfw^{h},r^{h})\in V^{h})\text{\quad}a(\bfu^{h},\bfw^{h})-b(\bfw^{h},p^{h})+b(\bfu^{h},r^{h})+c(\bfu^{h},\bfu^{h},\bfw^{h})+(\gamma^{h}\bfu^{h},\bfv^{h})=(\bff,\bfw^{h})\nonumber\\
&  \gamma^{h}\in\mathcal{A}^{h}\nonumber
\end{align}
\end{definition}

\begin{definition}
Consider the constants $\varepsilon>0$ and $h>0$. Given $\bar{\gamma} \in\mathcal{A}$ a local solution for \eqref{min}, we denote%
\[
\mathcal{A}_{\varepsilon}^{h}(\bar{\gamma})=\{\gamma^{h}\in\mathcal{A}%
^{h}\text{ }\mid\text{ }\Vert\gamma^{h}-\bar{\gamma}\Vert\leq\varepsilon\}
\]
Then, we define the auxiliary optimal control problem as
\begin{align}
\text{minimize\quad}  &  J_{h}(\gamma^{h})=\dfrac{1}{2}\Vert\bfu^{h}-\bfu_{0}\Vert_{0,\omega}^{2}+\dfrac{\alpha}{2}\Vert\gamma^{h}-\gamma_{0}\Vert_{0,\Omega}^{2} \label{aux-oc}\\
\text{subject to\quad}  &  (\forall(\bfw^{h},r^{h})\in V^{h})\text{\quad}a(\bfu^{h},\bfw^{h})-b(\bfw^{h},p^{h})+b(\bfu^{h},r^{h})+c(\bfu%
^{h},\bfu^{h},\bfw^{h})+(\gamma^{h}\bfu^{h},\bfv^{h})=(\bff,\bfw^{h})_\Omega \nonumber\\
&  \gamma^{h}\in\mathcal{A}_{\varepsilon}^{h}(\bar{\gamma})\nonumber
\end{align}
\end{definition}

Under some hypotheses, this auxiliary control problem has an unique solution.

\begin{definition}
We denote by $\mathcal{P}_{0}^{h}:L^{2}(\Omega)\rightarrow G_{0}^{h}$ and $\mathcal{I}_{1}^{h}:C(\Omega)\rightarrow G_{1}^{h}$ the orthogonal projection to $G_{0}^{h}$ with respect to the $L^{2}(\Omega)$ inner product and the Lagrange interpolator for $G_{1}^{h}$, respectively.
\end{definition}

\begin{proposition}\label{interpolator}
Let $\varphi\in H^{1}(\Omega)$. Then,
\begin{enumerate}
\item $\Vert\varphi-\mathcal{P}_{0}^{h}(\varphi)\Vert_{0,\Omega}\leq Ch\Vert\varphi\Vert_{1,\Omega}$. Furthermore, if $\varphi\in\mathcal{A}$, then $\mathcal{P}_{0}^{h}(\varphi)\in\mathcal{A}\cap G_{0}^{h}$.

\item If $\varphi\in H^{2}(\Omega)$, then $\Vert\varphi-\mathcal{I}_{1}^{h}(\varphi)\Vert_{1,\Omega}\leq Ch^{2}\Vert\varphi\Vert_{2,\Omega}$. Furthermore, if $\varphi\in\mathcal{A}$, then $\mathcal{I}_{1}^{h}(\varphi)\in\mathcal{A}\cap G_{1}^{h}$.
\end{enumerate}
\end{proposition}

\begin{proof}
See Proposition 1.134 and Corollary 1.109 in \cite{EG04}.
\end{proof}

\begin{lemma}
Consider $\mathcal{A}^{h}=\mathcal{A}\cap G_{0}^{h}$ or$\mathcal{A}^{h}=\mathcal{A}\cap G_{1}^{h}$. For all $\varepsilon>0$, there exists $h_0>0$ such that the auxiliary optimal control problem has a solution for all $h\in(0,h_0)$.
\end{lemma}

\begin{proof}
Taking $\bar{\gamma}^{h}=\mathcal{P}_{0}^{h}(\bar{\gamma})$ if $\mathcal{A}^{h}=\mathcal{A}\cap G_{0}^{h}$, or $\bar{\gamma}^{h} =\mathcal{P}_{0}^{h}(\bar{\gamma})$ if $\mathcal{A}^{h}=\mathcal{A}\cap G_{1}^{h}$, there exists $H>0$ such that $\Vert\bar{\gamma}-\bar{\gamma}^{h}\Vert_{0,\Omega}\leq\varepsilon$. Then, $\mathcal{A}_{\varepsilon}^{h}(\bar{\gamma})$ is non empty. We can reply the same techniques as in \cite{ABO23} to prove the existence of a control.
\end{proof}

\begin{lemma}\label{taylor}
Consider $\varepsilon>0$ small enough such that $J_{h}^{\prime\prime}$ is coercive as in Lemma \ref{ellipticity-h} for all $\gamma^{h}\in\mathcal{A}_{\varepsilon}^{h}(\bar{\gamma})$. Then, the auxiliary optimal control problem \eqref{aux-oc} has an unique solution for $h>0$ small enough.
\end{lemma}

\begin{proof}
Let $\gamma_{1}^{h},\gamma_{2}^{h}\in\mathcal{A}_{\varepsilon}^{h} (\bar{\gamma})$ two solutions of \eqref{aux-oc}. Then, by Taylor Theorem, there exists $t\in [0,1]$ such that
\[
J_{h}(\gamma_{1}^{h})=J_{h}(\gamma_{2}^{h})+J_{h}^{\prime}(\gamma_{2}^{h})[\gamma_{1}^{h}-\gamma_{2}^{h}]+\dfrac{1}{2}J^{\prime\prime}(t\gamma_{1}^{h}+(1-t)\gamma_{2}^{h})[\gamma_{1}^{h}-\gamma_{2}^{h},\gamma_{1}^{h}-\gamma_{2}^{h}]
\]
Since $J_{h}(\gamma_{1}^{h})=J_{h}(\gamma_{2}^{h})$ and $J_{h}^{\prime}(\gamma_{2}^{h})[\gamma_{1}^{h}-\gamma_{2}^{h}]\geq0$, we apply Lemma \ref{ellipticity-h}. Thus,%
\[
\dfrac{\delta}{8}\Vert\gamma_{1}^{h}-\gamma_{2}^{h}\Vert_{0,\Omega}^{2}\leq\dfrac{1}{2}J^{\prime\prime}(t\gamma_{1}^{h}+(1-t)\gamma_{2}^{h})[\gamma_{1}^{h}-\gamma_{2}^{h},\gamma_{1}^{h}-\gamma_{2}^{h}]\leq0
\]
proving that $\gamma_{1}^{h}=\gamma_{2}^{h}$.
\end{proof}

\subsection{Case with a Discontinuous Galerkin discrete control}

In this subsection, we consider $\mathcal{A}^{h}=\mathcal{A}\cap G_{0}^{h}$.

\begin{theorem}\label{P0}
Let $\bar{\gamma}\in\mathcal{A}$ a local solution of \eqref{min} that verifies the optimality conditions \eqref{first-order} and \eqref{second-order}. There exists $\varepsilon>0$ and $h>0$ such that the auxiliary optimal control problem \eqref{aux-oc} has an unique solution $\gamma_{\varepsilon}^{h}\in\mathcal{A}_{\varepsilon}^{h}(\bar{\gamma})$. Furthermore, there exists $h_0>0$ such that%
\[
(\forall h\in(0,h_0))\quad\Vert\bar{\gamma}-\gamma_{\varepsilon}^{h}\Vert_{0,\Omega}\leq\dfrac{C}{\delta^{1/2}}h((\Vert\bfu_{0}\Vert_{0,\omega}+\Vert\bff\Vert_{0,\Omega})^{2}+\alpha\Vert\gamma_{0}\Vert_{1,\Omega})
\]
\end{theorem}

\begin{proof}
From Lemma \ref{ellipticity}, there exists a constant $\varepsilon>0$ such that for all $\varphi\in L^{\infty}(\Omega)$ and all $\gamma\in\mathcal{A}_{\varepsilon}(\bar{\gamma})$%
\[
J^{\prime\prime}(\gamma)[\varphi,\varphi]\geq\dfrac{\delta}{2}\Vert\varphi\Vert_{0,\Omega}^{2}%
\]
Analogously, there exists $h_0>0$ such that for all $h\in(0,h_0)$, $\varphi\in L^{\infty}(\Omega)$ and $\gamma^{h}\in\mathcal{A}_{\varepsilon
}^{h}(\bar{\gamma})$%
\[
J_{h}^{\prime\prime}(\gamma^{h})[\varphi,\varphi]\geq\dfrac{\delta}{4}%
\Vert\varphi\Vert_{0,\Omega}^{2}%
\]
If $\hat{\gamma}_{\varepsilon}^{h}\in\mathcal{A}_{\varepsilon}^{h}(\bar {\gamma})$ is the unique solution of the following optimal control problem%
\begin{align*}
\text{minimize\quad}  &  J(\gamma)=\dfrac{1}{2}\Vert\bfu-\bfu_{0}\Vert_{0,\omega}^{2}+\dfrac{\alpha}{2}\Vert\gamma
-\gamma_{0}\Vert_{0,\Omega}^{2}\\
\text{subject to\quad}  &  (\forall(\bfw,r)\in V)\text{\quad}a(\bfu,\bfw)-b(\bfw,p)+b(\bfu,r)+c(\bfu,\bfu,\bfw)+(\gamma^{h}\bfu,\bfv)=(\bff,\bfw)\\
&  \gamma^{h}\in\mathcal{A}_{\varepsilon}^{h}(\bar{\gamma})
\end{align*}
we have
\[
\Vert\bar{\gamma}-\gamma_{\varepsilon}^{h}\Vert_{0,\Omega}\leq\Vert\bar{\gamma}-\hat{\gamma}_{\varepsilon}^{h}\Vert_{0,\Omega}+\Vert\hat{\gamma}_{\varepsilon}^{h}-\gamma_{\varepsilon}^{h}\Vert_{0,\Omega}
\]
Let $e_{1}=\bar{\gamma}-\hat{\gamma}_{\varepsilon}^{h}$. By Taylor Theorem, there exists $t\in\lbrack0,1]$ and $\xi_{t}=t\bar{\gamma}+(1-t)\hat{\gamma }_{\varepsilon}^{h}\in\mathcal{A}_{\varepsilon}(\bar{\gamma})$ such that
\begin{align*}
J^{\prime\prime}(\xi_{t})\left[  e_{1},e_{1}\right]   &  =J^{\prime}(\bar{\gamma})[e_{1}]-J^{\prime}(\hat{\gamma}_{\varepsilon}^{h})[e_{1}]\\ &  =J^{\prime}(\bar{\gamma})[\bar{\gamma}-\hat{\gamma}_{\varepsilon}^{h}]-J^{\prime}(\hat{\gamma}_{\varepsilon}^{h})[\bar{\gamma}-\mathcal{P}_{0}\bar{\gamma}]+J^{\prime}(\hat{\gamma}_{\varepsilon}^{h})[\hat{\gamma}_{\varepsilon}^{h}-\mathcal{P}_{0}\bar{\gamma}]
\end{align*}
Since $\bar{\gamma}$ and $\hat{\gamma}_{\varepsilon}^{h}$ verify the respective optimality conditions, we have $J^{\prime}(\bar{\gamma})[
\bar{\gamma}-\hat{\gamma}_{\varepsilon}^{h}]  \leq0$ and $J^{\prime}(\hat{\gamma}_{\varepsilon}^{h})[\hat{\gamma}_{\varepsilon}^{h}-\mathcal{P}_{0}\bar{\gamma}]\leq0$. Then,
\begin{align*}
\dfrac{\delta}{2}\Vert e_{1}\Vert_{0,\Omega}^{2}  &  \leq J^{\prime\prime}(\xi_{t})\left[  e_{1},e_{1}\right] \\
&  \leq-J^{\prime}(\hat{\gamma}_{\varepsilon}^{h})[\bar{\gamma}-\mathcal{P}_{0}\bar{\gamma}]=-(\alpha(\hat{\gamma}_{\varepsilon}^{h}-\gamma_{0})-\bfu_{\varepsilon}^{h}\cdot\bfv_{\varepsilon}^{h},\bar{\gamma}-\mathcal{P}_{0}\bar{\gamma})\\
&  \leq-\alpha(\hat{\gamma}_{\varepsilon}^{h}-\gamma_{0},\bar{\gamma}-\mathcal{P}_{0}\bar{\gamma})+(\bfu_{\varepsilon}^{h}\cdot\bfv_{\varepsilon}^{h},\bar{\gamma}-\mathcal{P}_{0}\bar{\gamma})
\end{align*}
where $A(\hat{\gamma}_{\varepsilon}^{h})=(\bfu_{\varepsilon},p_{\varepsilon})$ and $B(\hat{\gamma}_{\varepsilon})=(\bfv_{\varepsilon},q_{\varepsilon})$. Since
\[
(\mathcal{P}_{0}(\bfu_{\varepsilon}\cdot\bfv_{\varepsilon}),\bar{\gamma}-\mathcal{P}_{0}\bar{\gamma})=(\mathcal{P}_{0}(\gamma_{0}),\bar{\gamma}-\mathcal{P}_{0}\bar{\gamma})=(\hat{\gamma}_{\varepsilon}^{h},\bar{\gamma}-\mathcal{P}_{0}\bar{\gamma})=0
\]
we have
\begin{align*}
-\alpha(\hat{\gamma}_{\varepsilon}^{h}-\gamma_{0},\bar{\gamma}-\mathcal{P}_{0}(\bar{\gamma}))  &  =\alpha(\mathcal{P}_{0}(\gamma_{0})-\gamma_{0},\bar{\gamma}-\mathcal{P}_{0}\bar{\gamma})\\
&  \leq\dfrac{\alpha^{2}}{2}\Vert\gamma_{0}-\mathcal{P}_{0}(\gamma_{0})\Vert_{0,\Omega}^{2}+\dfrac{1}{2}\Vert\bar{\gamma}-\mathcal{P}_{0}\bar{\gamma}\Vert_{0,\Omega}^{2}
\end{align*}
and
\begin{align*}
(\bfu_{\varepsilon}\cdot\bfv_{\varepsilon},\bar{\gamma}-\mathcal{P}_{0}\bar{\gamma})  &  =(\bfu_{\varepsilon}\cdot\bfv_{\varepsilon}-\mathcal{P}_{0}(\bfu_{\varepsilon}\cdot\bfv_{\varepsilon}),\bar{\gamma}-\mathcal{P}_{0}\bar{\gamma})\\
&  \leq\dfrac{1}{2}\Vert\bfu_{\varepsilon}\cdot\bfv_{\varepsilon}-\mathcal{P}_{0}(\bfu_{\varepsilon}\cdot\bfv_{\varepsilon})\Vert_{0,\Omega}^{2}+\dfrac{1}{2}\Vert\bar{\gamma}-\mathcal{P}_{0}\bar{\gamma}\Vert_{0,\Omega}^{2}%
\end{align*}
Then,%
\begin{align*}
\dfrac{\delta}{2}\Vert e_{1}\Vert_{0,\Omega}^{2}  &  \leq\dfrac{1}{2}\Vert\bfu_{\varepsilon}\cdot\bfv_{\varepsilon}-\mathcal{P}_{0}(\bfu_{\varepsilon}\cdot\bfv_{\varepsilon})\Vert_{0,\Omega}^{2}+\dfrac{1}{2}\Vert\bar{\gamma}-\mathcal{P}_{0}\bar{\gamma}\Vert_{0,\Omega}^{2}+\dfrac{\alpha^{2}}{2}\Vert\gamma_{0}-\mathcal{P}_{0}(\gamma_{0})\Vert_{0,\Omega}^{2}\\
\delta\Vert e_{1}\Vert_{0,\Omega}^{2}  &  \leq Ch^{2}\left(  \Vert\bfu_{\varepsilon}\cdot\bfv_{\varepsilon}\Vert_{1,\Omega}^{2}+\Vert\bar{\gamma}\Vert_{1,\Omega}^{2}+\alpha^{2}\Vert\gamma_{0}\Vert_{1,\Omega}^{2}\right)
\end{align*}
where, applying H\"{o}lder inequality, Lemma \ref{adjoint}, Theorems \ref{f-bound} and \ref{H2}, we have
\begin{align*}
\Vert\bfu_{\varepsilon}\cdot\bfv_{\varepsilon}\Vert_{1,\Omega}  & =\Vert(\nabla\bfu_{\varepsilon})\bfv_{\varepsilon}+(\nabla\bfv_{\varepsilon})\bfu_{\varepsilon}\Vert_{0,\Omega}\\
&  \leq\Vert(\nabla\bfu_{\varepsilon})\bfv_{\varepsilon}\Vert_{0,\Omega}+\Vert(\nabla\bfv_{\varepsilon})\bfu_{\varepsilon}\Vert_{0,\Omega}\\
&  \leq\Vert\bfu_{\varepsilon}\Vert_{1,\Omega}\Vert\bfv_{\varepsilon}\Vert_{0,\infty,\Omega}+\Vert\bfv_{\varepsilon}\Vert_{1,\Omega}\Vert\bfu_{\varepsilon}\Vert_{0,\infty,\Omega}\\
&  \leq C(\Vert\bfu_{0}\Vert_{0,\omega}+\Vert\bff\Vert_{0,\Omega})^{2}%
\end{align*}
and
\begin{align*}
\delta\Vert e_{1}\Vert_{0,\Omega}^{2}  &  \leq Ch^{2}\left(  \Vert\bfu_{\varepsilon}^{h}\cdot\bfv_{\varepsilon}^{h}\Vert_{1,\Omega}^{2}+\Vert\bar{\gamma}\Vert_{1,\Omega}^{2}+\alpha^{2}\Vert\gamma_{0}\Vert_{1,\Omega}^{2}\right) \\
\Vert e_{1}\Vert_{0,\Omega}  &  \leq\dfrac{C}{\delta^{1/2}}h((\Vert\bfu_{0}\Vert_{0,\omega}+\Vert\bff\Vert_{0,\Omega})^{2}+\Vert\bar{\gamma}\Vert_{1,\Omega}+\alpha\Vert\gamma_{0}\Vert_{1,\Omega})\\
&  \leq\dfrac{C}{\delta^{1/2}}h((\Vert\bfu_{0}\Vert_{0,\omega}+\Vert\bff\Vert_{0,\Omega})^{2}+\alpha\Vert\gamma_{0}\Vert_{1,\Omega})
\end{align*}
Let $e_{2}=\gamma_{\varepsilon}^{h}-\hat{\gamma}_{\varepsilon}^{h}$. By Taylor Theorem, there exists $t\in\lbrack0,1]$ and $\xi_{t}=t\gamma_{\varepsilon}^{h}+(1-t)\hat{\gamma}_{\varepsilon}^{h}\in\mathcal{A}_{\varepsilon}^{h}(\bar{\gamma})$ such that%
\[
J_{h}^{\prime\prime}(\xi_{t})\left[  e_{2},e_{2}\right]  =J_{h}^{\prime}(\gamma_{\varepsilon}^{h})[e_{2}]-J_{h}^{\prime}(\hat{\gamma}_{\varepsilon}^{h})[e_{2}]
\]
with
\[
J_{h}^{\prime}(\gamma_{\varepsilon}^{h})[e_{2}]=J_{h}^{\prime}(\gamma_{\varepsilon}^{h})[\gamma_{\varepsilon}^{h}-\hat{\gamma}_{\varepsilon}^{h}]\leq0\leq J^{\prime}(\hat{\gamma}_{\varepsilon}^{h})[\gamma_{\varepsilon}^{h}-\hat{\gamma}_{\varepsilon}^{h}]=J^{\prime}(\hat{\gamma}_{\varepsilon}^{h})[e_{2}]
\]
Then,%
\begin{align*}
\dfrac{\delta}{4}\Vert e_{2}\Vert_{0,\Omega}^{2}  &  \leq J_{h}^{\prime\prime}(\xi_{t})\left[  e_{2},e_{2}\right]  =J_{h}^{\prime}(\gamma_{\varepsilon}^{h})[e_{2}]-J_{h}^{\prime}(\hat{\gamma}_{\varepsilon}^{h})[e_{2}]\\
&  \leq J^{\prime}(\hat{\gamma}_{\varepsilon}^{h})[e_{2}]-J_{h}^{\prime}(\hat{\gamma}_{\varepsilon}^{h})[e_{2}]\\
&  \leq Ch^{2}\Vert e_{2}\Vert_{0,\Omega}(\Vert\bff\Vert_{0,\Omega}+\Vert\bfu_{0}\Vert_{0,\omega})^{2}\\
\Vert e_{2}\Vert_{0,\Omega}  &  \leq\dfrac{C}{\delta}h^{2}(\Vert\bff\Vert_{0,\Omega}+\Vert\bfu_{0}\Vert_{0,\omega})^{2}%
\end{align*}
In conclusion,%
\[
\Vert\bar{\gamma}-\gamma_{\varepsilon}^{h}\Vert_{0,\Omega}\leq\Vert e_{1}\Vert_{0,\Omega}+\Vert e_{2}\Vert_{0,\Omega}\leq\dfrac{C}{\delta^{1/2}}h((\Vert\bfu_{0}\Vert_{0,\omega}+\Vert\bff\Vert_{0,\Omega})^{2}+\alpha\Vert\gamma_{0}\Vert_{1,\Omega})
\]
proving this result.
\end{proof}

\begin{corollary}
Let $\bar{\gamma}\in\mathcal{A}$ a local solution of \eqref{min} that verifies the optimality conditions \eqref{first-order} and \eqref{second-order}. For $h_0>0$ small enough, there exists a sequence $\{\gamma^{h}\}_{0<h<h_0}\ $ of solutions of the discrete optimal control problem \eqref{min-h} such that
\[
\Vert\bar{\gamma}-\gamma^{h}\Vert_{0,\Omega}\leq\dfrac{C}{\delta^{1/2}}h((\Vert\bfu_{0}\Vert_{0,\omega}+\Vert\bff\Vert_{0,\Omega})^{2}+\alpha\Vert\gamma_{0}\Vert_{1,\Omega})
\]
\end{corollary}

\begin{proof}
First, there exists $\varepsilon>0$ small enough such that the solution $\bar{\gamma}_{\varepsilon}^{h}$ of the auxiliary optimal control problem \eqref{aux-oc} is also a local solution for the discrete optimal control problem \eqref{min-h} for $h>0$ small enough. The estimate is a consequence of Theorem \ref{P0}.
\end{proof}

\subsection{Case with a Continuous Lagrange discrete control}

In this subsection, we consider $\mathcal{A}^{h}=\mathcal{A}\cap G_{1}^{h}$. Let $\bar{\gamma}\in\mathcal{A}$ be a local solution of (min), the set $\mathcal{T}_{h}$ can be partitioned as $\mathcal{T}_{h}=\mathcal{T}_{h}^{1}\cup\mathcal{T}_{h}^{2}\cup\mathcal{T}_{h}^{3}$, where
\begin{align*}
\mathcal{T}_{h}^{1}  &  =\left\{  T\in\mathcal{T}_{h}\text{ }\mid\text{ }\bar{\gamma}\vert_{T}=a\text{ or }\bar{\gamma}\vert_{T}=b\right\} \\
\mathcal{T}_{h}^{2}  &  =\left\{  T\in\mathcal{T}_{h}\text{ }\mid\text{}a<\bar{\gamma}<b\text{ on }T\right\} \\
\mathcal{T}_{h}^{3}  &  =\mathcal{T}_{h}^{1}\setminus(\mathcal{T}_{h}^{2}\cup\mathcal{T}_{h}^{3})
\end{align*}
We denote $\Omega_{h,j}=\bigcup\limits_{T\in\mathcal{T}_{h}^{j}}T$ for $j\in\left\{  1,2,3\right\}  $ and we assume that $\Vert\Omega_{h,j}\Vert\leq Ch^{p}$ for some $p\geq1$.

\begin{theorem}
Consider $\bar{\gamma}$ a local solution of \eqref{min} that fulfills the optimality conditions \eqref{first-order} and \eqref{second-order}. For $h_0>0$ small enough, there exists a sequence $\{\gamma^{h}\}_{0<h<h_0}\ $ of solutions of the discrete optimal control problem \eqref{min-h} such that
\[
\Vert\bar{\gamma}-\gamma^{h}\Vert_{0,\Omega}\leq\left(  1+\dfrac{C}{\delta}\right)  \Vert\bar{\gamma}-\mathcal{I}_{1}(\bar{\gamma})\Vert_{0,\Omega}+\dfrac{C}{\delta^{1/2}}\sqrt{J^{\prime}(\bar{\gamma})[\mathcal{I}_{1}(\bar{\gamma})-\bar{\gamma}]}+\dfrac{C}{\delta}h^{2}
\]
\end{theorem}

\begin{proof}
As in the proof of Theorem \ref{P0}, there exists a constant $\varepsilon>0$ such that for all $\varphi\in L^{\infty}(\Omega)$ and all $\gamma\in\mathcal{A}_{\varepsilon}(\bar{\gamma})$
\[
J^{\prime\prime}(\gamma)[\varphi,\varphi]\geq\dfrac{\delta}{2}\Vert\varphi\Vert_{0,\Omega}^{2}
\]
and a constant $h_0>0$ such that for all $h\in(0,h_0)$, $\varphi\in L^{\infty}(\Omega)$ and $\gamma\in\mathcal{A}_{\varepsilon}(\bar{\gamma})$%
\[
J_{h}^{\prime\prime}(\gamma)[\varphi,\varphi]\geq\dfrac{\delta}{4}\Vert\varphi\Vert_{0,\Omega}^{2}
\]
Then, there exists an unique solution $\hat{\gamma}_{\varepsilon}^{h} \in\mathcal{A}_{\varepsilon}^{h}(\bar{\gamma})$ for the auxiliary optimal control problem \eqref{aux-oc}.

Now, we have%
\[
\Vert\bar{\gamma}-\gamma^{h}\Vert_{0,\Omega}\leq\Vert\bar{\gamma}-\mathcal{I}_{1}(\bar{\gamma})\Vert_{0,\Omega}+\Vert\mathcal{I}_{1}(\bar{\gamma})-\gamma^{h}\Vert_{0,\Omega}%
\]
Let $e=\bar{\gamma}-\gamma^{h}$, $e_{1}=\bar{\gamma}-\mathcal{I}_{1}(\bar{\gamma})$ and $e_{2}=\mathcal{I}_{1}(\bar{\gamma})-\gamma^{h}$. We have%
\[
J^{\prime}(\bar{\gamma})[e]=J^{\prime}(\bar{\gamma})[\bar{\gamma}-\gamma^{h}]\leq0\leq J_{h}^{\prime}(\gamma^{h})[\mathcal{I}_{1}(\bar{\gamma})-\gamma^{h}]=J_{h}^{\prime}(\gamma^{h})[e_{2}]
\]
By Taylor Theorem, there exists $t\in\lbrack0,1]$ and $\xi_{t}=t\mathcal{I}_{1}(\bar{\gamma})+(1-t)\gamma^{h}\in\mathcal{A}_{\varepsilon}(\bar{\gamma})$ such that
\begin{align*}
J_{h}^{\prime\prime}(\xi_{t})\left[  e_{2},e_{2}\right]  =  &  J_{h}^{\prime}(\mathcal{I}_{1}(\bar{\gamma}))[e_{2}]-J_{h}^{\prime}(\gamma^{h})[e_{2}]\\
\leq &  J_{h}^{\prime}(\mathcal{I}_{1}(\bar{\gamma}))[e_{2}]-J^{\prime}(\bar{\gamma})[e]\\
\leq &  (J_{h}^{\prime}(\mathcal{I}_{1}(\bar{\gamma}))[e_{2}]-J_{h}^{\prime}(\bar{\gamma})[e_{2}])+(J_{h}^{\prime}(\bar{\gamma})[e_{2}]-J^{\prime}(\bar{\gamma})[e_{2}])-J^{\prime}(\bar{\gamma})[e_{1}]
\end{align*}
where, applying Lemma \ref{aux-1} and Proposition \ref{interpolator},
\begin{align*}
J_{h}^{\prime}(\mathcal{I}_{1}(\bar{\gamma}))[e_{2}]-J_{h}^{\prime}(\bar{\gamma})[e_{2}]  &  \leq C\Vert e_{2}\Vert_{0,\Omega}\Vert\mathcal{I}_{1}(\bar{\gamma})-\bar{\gamma}\Vert_{0,\Omega}=C\Vert e_{2}\Vert_{0,\Omega}\Vert e_{1}\Vert_{0,\Omega}\\
J_{h}^{\prime}(\bar{\gamma})[e_{2}]-J^{\prime}(\bar{\gamma})[e_{2}]  &  \leq Ch^{2}\Vert e_{2}\Vert_{0,\Omega}%
\end{align*}
Then,%
\begin{align*}
\dfrac{\delta}{4}\Vert e_{2}\Vert_{0,\Omega}^{2}  &  \leq J_{h}^{\prime\prime}(\xi_{t})\left[  e_{2},e_{2}\right]  \leq C\Vert e_{2}\Vert_{0,\Omega}(h^{2}+\Vert e_{1}\Vert_{0,\Omega})-J^{\prime}(\bar{\gamma})[e_{1}]\\
&  \leq C\Vert e_{2}\Vert_{0,\Omega}(h^{2}+\Vert e_{1}\Vert_{0,\Omega})+J^{\prime}(\bar{\gamma})[\mathcal{I}_{1}(\bar{\gamma})-\bar{\gamma}]
\end{align*}
proving that%
\[
\Vert e_{2}\Vert_{0,\Omega}\leq\dfrac{C}{\delta}(h^{2}+\Vert e_{1}\Vert_{0,\Omega})+\dfrac{C}{\delta}\sqrt{J^{\prime}(\bar{\gamma})[\mathcal{I}_{1}(\bar{\gamma})-\bar{\gamma}]}
\]
Finally,%
\begin{align*}
\Vert\bar{\gamma}-\gamma^{h}\Vert_{0,\Omega}  &  \leq\Vert\bar{\gamma}-\mathcal{I}_{1}(\bar{\gamma})\Vert_{0,\Omega}+\Vert\mathcal{I}_{1}(\bar{\gamma})-\gamma^{h}\Vert_{0,\Omega}\\
&  \leq\Vert\bar{\gamma}-\mathcal{I}_{1}(\bar{\gamma})\Vert_{0,\Omega}+\dfrac{C}{\delta}(h^{2}+\Vert\bar{\gamma}-\mathcal{I}_{1}(\bar{\gamma})\Vert_{0,\Omega})+\dfrac{C}{\delta}\sqrt{J^{\prime}(\bar{\gamma})[\mathcal{I}_{1}(\bar{\gamma})-\bar{\gamma}]}\\
&  \leq C\left(  1+\dfrac{1}{\delta}\right)  \Vert\bar{\gamma}-\mathcal{I}_{1}(\bar{\gamma})\Vert_{0,\Omega}+\dfrac{C}{\delta^{1/2}}\sqrt{J^{\prime}(\bar{\gamma})[\mathcal{I}_{1}(\bar{\gamma})-\bar{\gamma}]}+\dfrac{C}{\delta}h^{2}%
\end{align*}
proving the theorem.
\end{proof}

\begin{lemma}\label{aux-p1}
Suppose $\gamma_{0}\in H^{2}(\Omega)\cap W^{1,\infty}(\Omega)$ and $\bar{\gamma}\in W^{1,\infty}(\Omega)$. Then, $\bar{\gamma}\in H^{2}(T)$ for each $T\in\mathcal{T}_{h}^{2}$. Furthermore, there exists a constant $C>0$ such that
\[
\Vert\bar{\gamma}-\mathcal{I}_{1}(\bar{\gamma})\Vert_{0,\Omega}^{2}=C(h^{4}\Vert\triangle\bar{\gamma}\Vert_{0,\Omega_{h,2}}^{2}+h^{2+p}\Vert\nabla\bar{\gamma}\Vert_{0,\infty,\Omega}^{2})
\]

\end{lemma}

\begin{proof}
First, if $A(\bar{\gamma})=(\boldsymbol{\bar{u}},\bar{p})$ and $B(\bar{\gamma})=(\boldsymbol{\bar{v}},\bar{q})$, we have $\bar{\gamma}=\gamma_{0}+\dfrac{1}{\alpha}(\boldsymbol{\bar{u}}\cdot\boldsymbol{\bar{v}})$ for all $T\in\mathcal{T}_{h}^{2}$, proving that $\bar{\gamma}\in H^{2}(T)$. Second, we have%
\[
\Vert\bar{\gamma}-\mathcal{I}_{1}(\bar{\gamma})\Vert_{0,\Omega}^{2}=\sum\limits_{T\in\mathcal{T}_{h}^{1}}\Vert\bar{\gamma}-\mathcal{I}_{1}(\bar{\gamma})\Vert_{0,T}^{2}+\sum\limits_{T\in\mathcal{T}_{h}^{2}}\Vert\bar{\gamma}-\mathcal{I}_{1}(\bar{\gamma})\Vert_{0,T}^{2}+\sum\limits_{T\in\mathcal{T}_{h}^{3}}\Vert\bar{\gamma}-\mathcal{I}_{1}(\bar{\gamma})\Vert_{0,T}^{2}
\]
where $\bar{\gamma}=\mathcal{I}_{1}(\bar{\gamma})$ on $T\in\mathcal{T}_{h}^{1}$. Then, applying the local Lagrange interpolator estimates (Theorem 1.103 in \cite{EG04}),
\[
\sum\limits_{T\in\mathcal{T}_{h}^{2}}\Vert\bar{\gamma}-\mathcal{I}_{1}(\bar{\gamma})\Vert_{0,T}^{2}\leq C\sum\limits_{T\in\mathcal{T}_{h}^{2}}h_{T}^{4}\Vert\triangle\bar{\gamma}\Vert_{0,T}^{2}\leq Ch^{4}\Vert\triangle\bar{\gamma}\Vert_{0,\Omega_{h,2}}^{2}%
\]
If $T\in\mathcal{T}_{h}^{3}$, we have that $a=\min\limits_{x\in K}\bar{\gamma}(x)<\max\limits_{x\in K}\bar{\gamma}(x)<b$ or $a<\min\limits_{x\in K}\bar{\gamma}(x)<\max\limits_{x\in K}\bar{\gamma}(x)=b$. In the first case, we have%
\[
\Vert\bar{\gamma}-\mathcal{I}_{1}(\bar{\gamma})\Vert_{0,T}^{2}\leq\Vert T\Vert\Vert\bar{\gamma}-\mathcal{I}_{1}(\bar{\gamma})\Vert_{0,\infty,T}^{2}\leq\Vert T\Vert\Vert\bar{\gamma}-a\Vert_{0,\infty,T}^{2}\leq\Vert T\Vert h_{T}^{2}\Vert\nabla\bar{\gamma}\Vert_{0,\infty,T}^{2}%
\]
as a consequence of the Mean Value Theorem. We obtain a similar estimate for the second case. In consequence,%
\[
\sum\limits_{T\in\mathcal{T}_{h}^{3}}\Vert\bar{\gamma}-\mathcal{I}_{1}(\bar{\gamma})\Vert_{0,T}^{2}\leq\sum\limits_{T\in\mathcal{T}_{h}^{3}}\Vert T\Vert h_{T}^{2}\Vert\nabla\bar{\gamma}\Vert_{0,\infty,T}^{2}\leq Ch^{2+p}\Vert\nabla\bar{\gamma}\Vert_{0,\infty,\Omega_{h,3}}^{2}%
\]
proving the desired estimate.
\end{proof}

\begin{theorem}
Consider $\bar{\gamma}\in W^{1,\infty}(\Omega)$ a local solution of \eqref{min} that fulfills the optimality conditions \eqref{first-order} and \eqref{second-order}. For $h_0>0$ small enough, there exists a sequence $\{\gamma^{h}\}_{0<h<h_0}$ of solutions of the discrete optimal control problem \eqref{min-h} such that
\[
\Vert\bar{\gamma}-\gamma^{h}\Vert_{0,\Omega}\leq\left(  1+\dfrac{C}{\delta}\right)  h^{1+p/2}+\dfrac{C}{\delta}h^{2}%
\]
\end{theorem}

\begin{proof}
Consider $A(\bar{\gamma})=(\boldsymbol{\bar{u}},\bar{p})$ and $B(\bar{\gamma})=(\boldsymbol{\bar{v}},\bar{q})$. Since $\mathcal{I}_{1}(\bar{\gamma})=\bar{\gamma}$ in $\Omega_{h,1}$ and $\bar{\gamma}=\gamma_{0}+\dfrac{1}{\alpha}(\boldsymbol{\bar{u}}\cdot\boldsymbol{\bar{v}})$, we have%
\[
0\leq J^{\prime}(\bar{\gamma})[\mathcal{I}_{1}(\bar{\gamma})-\bar{\gamma}]=\sum\limits_{T\in\mathcal{T}_{h}^{3}}(\mathcal{I}_{1}(\bar{\gamma})-\bar{\gamma},\boldsymbol{\bar{u}}\cdot\boldsymbol{\bar{v}}+\alpha(\bar{\gamma}-\gamma_{0}))_{T}%
\]
Defining $d=\boldsymbol{\bar{u}}\cdot\boldsymbol{\bar{v}}+\alpha(\bar{\gamma}-\gamma_{0})$, we have that for all $T\in\mathcal{T}_{h}^{3}$, there exists $\boldsymbol{x}_{T}\in T$ such that $d(\boldsymbol{x}_{T})=\boldsymbol{0}$. Repeating the same reasoning as in the proof of Lemma \ref{aux-p1}, we obtain
\begin{align*}
(\mathcal{I}_{1}(\bar{\gamma})-\bar{\gamma},\boldsymbol{\bar{u}}\cdot\boldsymbol{\bar{v}}+\alpha(\bar{\gamma}-\gamma_{0}))_{T}  & =(\mathcal{I}_{1}(\bar{\gamma})-\bar{\gamma},d-d(\boldsymbol{x}_{T}))_{T}\\
&  \leq\Vert\mathcal{I}_{1}(\bar{\gamma})-\bar{\gamma}\Vert_{0,T}\Vert d-d(\boldsymbol{x}_{T})\Vert_{0,T}\\
&  \leq\Vert T\Vert h_{T}^{2}\Vert\nabla\bar{\gamma}\Vert_{0,\infty,T}^{2}%
\end{align*}
Then,%
\[
J^{\prime}(\bar{\gamma})[\mathcal{I}_{1}(\bar{\gamma})-\bar{\gamma}]=\sum\limits_{T\in\mathcal{T}_{h}^{3}}(\mathcal{I}_{1}(\bar{\gamma})-\bar{\gamma},\boldsymbol{\bar{u}}\cdot\boldsymbol{\bar{v}}+\alpha(\bar{\gamma}-\gamma_{0}))_{T}\leq\sum\limits_{T\in\mathcal{T}_{h}^{3}}\Vert T\Vert h_{T}^{2}\Vert\nabla\bar{\gamma}\Vert_{0,\infty,T}^{2}\leq Ch^{2+p}\Vert\nabla\bar{\gamma}\Vert_{0,\infty,\Omega_{h,3}}^{2}%
\]
proving that%
\[
\Vert\bar{\gamma}-\gamma^{h}\Vert_{0,\Omega}\leq\left(  1+\dfrac{C}{\delta}\right)  h^{1+p/2}+\dfrac{C}{\delta}h^{2}%
\]
\end{proof}

\subsection{Semidiscrete scheme}
The optimal control problem can also be defined with a discretization of the states and respective adjoint, but with no discretizations for the control $\gamma$. In this case, the local solutions $\bar{\gamma}^{h}\in\mathcal{A}$ must verify the identity $\bar{\gamma}^{h}=\Pi_{\lbrack a,b]}\left(\gamma_{0}+\dfrac{1}{\alpha}\boldsymbol{\bar{u}}^{h}\cdot\boldsymbol{\bar{v}}^{h}\right)  $, where $(\boldsymbol{\bar{u}}^{h},\bar{p}^{h})$ and $(\boldsymbol{\bar{v}}^{h},\bar{q}^{h})$ are the respective optimal states and adjoints. The a priori error estimates are almost a direct consequence of the results obtained in Section 3.

\begin{theorem}\label{semi}
Let $\bar{\gamma}\in\mathcal{A}$ a local solution of \eqref{min} that verifies the optimality conditions \eqref{first-order} and \eqref{second-order}. For $h_{0}>0$ small enough, there exists a sequence $\{\gamma^{h}\}_{0<h<h_{0}}\in\mathcal{A}$ of solutions of the semidiscrete optimal control problem \eqref{semi-h} such that
\[
\Vert\gamma^{h}-\bar{\gamma}\Vert_{0,\Omega}\leq C\Vert\bff%
\Vert_{0,\Omega}(\Vert\bff\Vert_{0,\Omega}+\Vert\bfu%
_{0}\Vert_{0,\omega})
\]

\end{theorem}

\begin{proof}
From Lemma \ref{ellipticity}, there exists $\varepsilon>0$ such that for all $\varphi\in L^{\infty}(\Omega)$ and all $\gamma\in\mathcal{A}_{\varepsilon}(\bar{\gamma})$
\[
J^{\prime\prime}(\gamma)[\varphi,\varphi]\geq\dfrac{\delta}{2}\Vert\varphi\Vert_{0,\Omega}^{2}
\]
Then, the semidiscrete optimal control problem has an unique solution $\gamma^{h}\in\mathcal{A}_{\varepsilon}(\bar{\gamma})$ for $h>0$ small enough. Taking $\varphi=\gamma^{h}-\bar{\gamma}$ and applying Taylor theorem, there exists $t\in\lbrack0,1]$ and $\xi_{t}=t\gamma^{h}+(1-t)\bar{\gamma}\in\mathcal{A}_{\varepsilon}(\bar{\gamma})$ such that
\[
\dfrac{\delta}{2}\Vert\gamma^{h}-\bar{\gamma}\Vert_{0,\Omega}^{2}\leq J^{\prime\prime}(\xi_{t})[\gamma^{h}-\bar{\gamma},\gamma^{h}-\bar{\gamma}]=J^{\prime}(\gamma^{h})[\gamma^{h}-\bar{\gamma}]-J(\bar{\gamma})[\gamma^{h}-\bar{\gamma}]
\]
Since $\bar{\gamma}$ and $\gamma^{h}$ fulfill their respective optimality conditions, we have
\[
J^{\prime}(\bar{\gamma})\left[  \bar{\gamma}-\hat{\gamma}_{\varepsilon}^{h}\right]  \leq0\leq J_{h}^{\prime}(\gamma^{h})[\gamma^{h}-\bar{\gamma}]
\]
Then, applying Lemma \ref{aux-1},
\begin{align*}
\dfrac{\delta}{2}\Vert\gamma^{h}-\bar{\gamma}\Vert_{0,\Omega}^{2}  &  \leq J^{\prime\prime}(\xi_{t})[\gamma^{h}-\bar{\gamma},\gamma^{h}-\bar{\gamma}]=J^{\prime}(\gamma^{h})[\gamma^{h}-\bar{\gamma}]-J(\bar{\gamma})[\gamma^{h}-\bar{\gamma}]\\
&  \leq J^{\prime}(\gamma^{h})[\gamma^{h}-\bar{\gamma}]-J_{h}^{\prime}(\gamma^{h})[\gamma^{h}-\bar{\gamma}]\\
&  \leq Ch^{2}\Vert\gamma^{h}-\bar{\gamma}\Vert_{0,\Omega}\Vert\bff\Vert_{0,\Omega}(\Vert\bff\Vert_{0,\Omega}+\Vert\bfu_{0}\Vert_{0,\omega})
\end{align*}
proving that $\Vert\gamma^{h}-\bar{\gamma}\Vert_{0,\Omega}\leq Ch^{2}\Vert\bff\Vert_{0,\Omega}(\Vert\bff\Vert_{0,\Omega}+\Vert\bfu_{0}\Vert_{0,\omega})$.
\end{proof}

\section{A posteriori error estimate}

\begin{definition}
Let $\bar{\gamma}\in\mathcal{A}$ and $\gamma^{h}\in\mathcal{A}$ local solutions of \eqref{min} and \eqref{min-h}, respectively, such that $A(\bar{\gamma})=(\boldsymbol{\bar{u}},\bar{p})$, $B(\bar{\gamma})=(\boldsymbol{\bar{v}},\bar{q})$, $A_{h}(\gamma^{h})=(\bfu^{h},p^{h})$ and $B(\gamma^{h})=(\bfv^{h},q^{h})$. We define $e(\gamma)=\bar{\gamma}-\gamma^{h}$, $e(\bfu)=\boldsymbol{\bar{u}}-\bfu^{h}$, $e(p)=\bar{p}-p^{h}$,$e(\bfv)=\boldsymbol{\bar{v}}-\bfv^{h}\ $and $e(q)=\bar{q}-q^{h}$. For $T\in\mathcal{T}_{h}$ and $F\in\mathcal{E}_{h}$, we denote
\begin{align*}
\mathcal{R}_{T}  &  =\bff+\nu\triangle\bfu^{h}-(\nabla\bfu^{h})\bfu^{h}-\nabla p^{h}-\gamma^{h}\bfu^{h}\\
\mathcal{R}_{A,T}  &  =\chi_{\omega}(\bfu^{h}-\bfu_{0})+\nu\triangle\bfv^{h}+(\nabla\bfv^{h})\bfu^{h}-(\nabla\bfu^{h})^{T}\bfv^{h}-\nabla q^{h}-\gamma^{h}\bfv^{h}\\
\mathcal{J}_{F}  &  =\jump{(\nabla\bfu^{h}-p^{h}\boldsymbol{I})\boldsymbol{n}}\\
\mathcal{J}_{A,F}  &  =\jump{(\nabla\bfv^{h}+q^{h}\boldsymbol{I})\boldsymbol{n}}
\end{align*}
For $\gamma^{\ast}=\Pi_{\lbrack a,b]}\left(\gamma_{0}+\dfrac{1}{\alpha}(\boldsymbol{\bar{u}}^{h}\cdot\boldsymbol{\bar{u}}^{h})\right)$, we define $(\bfu^{\ast},p^{\ast})=A(\gamma^{\ast})$ and $(\bfv^{\ast},q^{\ast})=B(\gamma^{\ast})$. Finally, we define
\begin{align*}
\eta_{S,T}^{2}  &  =h_{T}^{2}\Vert\mathcal{R}_{T}\Vert_{0,T}^{2}+h_{T}\Vert\mathcal{J}_{F}\Vert_{0,\partial T\setminus\partial\Omega}^{2}+\Vert\operatorname{div}\bfu^{h}\Vert_{0,T}^{2}\\
\eta_{A,T}^{2}  &  =h_{T}^{2}\Vert\mathcal{R}_{A,T}\Vert_{0,T}^{2}+h_{T}\Vert\mathcal{J}_{A,F}\Vert_{0,\partial T\setminus\partial\Omega}^{2}+\Vert\operatorname{div}\bfv^{h}\Vert_{0,T}^{2}\\
\eta_{C,T}^{2}  &  =\Vert\gamma^{h}-\gamma^{\ast}\Vert_{0,T}^{2}\\
\eta_{S,h}^{2}  &  =\sum_{T\in\mathcal{T}_{h}}\eta_{S,T}^{2}\text{\quad}\eta_{A,h}^{2}=\sum_{T\in\mathcal{T}_{h}}\eta_{A,T}^{2}\text{\quad}\eta
_{C,h}^{2}=\sum_{T\in\mathcal{T}_{h}}\eta_{C,T}^{2}\\
\eta_{T}^{2}  &  =\left\{\begin{array}
[c]{ll}%
\eta_{S,T}^{2}+\eta_{A,T}^{2}+\eta_{C,T}^{2} & \text{for the discrete scheme}\\
\eta_{S,T}^{2}+\eta_{A,T}^{2} & \text{for the semidiscrete scheme}
\end{array}
\right. \\
\eta_{h}^{2}  &  =\sum_{T\in\mathcal{T}_{h}}\eta_{T}^{2}=\left\{
\begin{array}
[c]{ll}%
\eta_{S,h}^{2}+\eta_{A,h}^{2}+\eta_{C,h}^{2} & \text{for the discrete
scheme}\\
\eta_{S,h}^{2}+\eta_{A,h}^{2} & \text{for the semidiscrete scheme}%
\end{array}
\right.
\end{align*}
\end{definition}

\begin{lemma}\label{aux-rel}
Under the same hypotheses of Theorem \ref{semi}, for all $\gamma\in\mathcal{A}$, $(\alpha(\gamma^{\ast}-\gamma_{0})+\bfu^{h}\cdot\bfv^{h},\bar{\gamma}-\gamma^{\ast})\geq 0$.
\end{lemma}
\begin{proof}
See Lemma 2.26 in \cite{T10}.
\end{proof}

\subsection{Reliability of the a posterior error estimators}

\begin{lemma}\label{rel-states}
There exists a constant $C>0$ independent on $h$ such that $\Vert A(\gamma^{h})-A_{h}(\gamma^{h})\Vert\leq C\eta_{S,h}$ and $\Vert
B(\gamma^{h})-B_{h}(\gamma^{h})\Vert\leq C\eta_{A,h}$
\end{lemma}
\begin{proof}
See \cite{OWA94} and \cite{V96}.
\end{proof}

\begin{theorem}
$\Vert(e(\bfu),e(p))\Vert+\Vert(e(\bfv),e(q))\Vert+\Vert e(\gamma)\Vert_{0,\Omega}\leq C\eta_{h}$
\end{theorem}

\begin{proof}
First, we present the proof for the discrete scheme. We have%
\[
\Vert e(\gamma)\Vert_{0,\Omega}=\Vert\bar{\gamma}-\gamma^{h}\Vert_{0,\Omega}\leq\Vert\bar{\gamma}-\gamma^{\ast}\Vert_{0,\Omega}+\Vert\gamma^{\ast}-\gamma^{h}\Vert_{0,\Omega}=\Vert\bar{\gamma}-\gamma^{\ast}\Vert_{0,\Omega}+\eta_{C,T}%
\]
and we define $(\boldsymbol{\hat{u}},\hat{p})=A(\gamma^{h})$ and $(\boldsymbol{\hat{v}},\hat{q})=B(\gamma^{h})$. Reasoning as in the proof of Lemma \ref{taylor}, taking $\varphi=\bar{\gamma}-\gamma^{\ast}$ and applying Taylor theorem, there exists $t\in [0,1]$ and $\xi_{t}=t\bar{\gamma}+(1-t)\gamma^{\ast}\in\mathcal{A}_{\varepsilon}(\bar{\gamma})$ such that%
\[
\dfrac{\delta}{2}\Vert\bar{\gamma}-\gamma^{\ast}\Vert_{0,\Omega}^{2}\leq J^{\prime\prime}(\xi_{t})[\bar{\gamma}-\gamma^{\ast},\bar{\gamma}-\gamma^{\ast}]=J^{\prime}(\bar{\gamma})[\bar{\gamma}-\gamma^{\ast}]-J^{\prime}(\gamma^{\ast})[\bar{\gamma}-\gamma^{\ast}]
\]
where $J^{\prime}(\bar{\gamma})[\bar{\gamma}-\gamma^{\ast}]\leq 0$. Applying Lemma \ref{aux-rel},
\begin{align*}
\dfrac{\delta}{2}\Vert\gamma^{h}-\bar{\gamma}\Vert_{0,\Omega}^{2}  & \leq(\alpha(\gamma^{\ast}-\gamma_{0})+\bfu^{h}\cdot\bfv^{h},\bar{\gamma}-\gamma^{\ast})-J^{\prime}(\gamma^{\ast})[\bar{\gamma}-\gamma^{\ast}]\\
&  \leq(\bfu^{h}\cdot\bfv^{h}-\bfu^{\ast}\cdot\bfv^{\ast},\bar{\gamma}-\gamma^{\ast})\\
&  \leq\Vert\gamma^{h}-\bar{\gamma}\Vert_{0,\Omega}\Vert\bfu^{h}\cdot\bfv^{h}-\bfu^{\ast}\cdot\bfv^{\ast}\Vert_{0,\Omega}
\end{align*}
Then, by Holder inequality and Sobolev Embedding Theorem,%
\begin{align*}
\Vert\gamma^{h}-\bar{\gamma}\Vert_{0,\Omega}  &  \leq C\Vert\bfu^{h}\cdot\bfv^{h}-\bfu^{\ast}\cdot\bfv^{\ast}\Vert_{0,\Omega}\\
&  \leq C(\Vert\bfu^{h}\Vert_{0,\Omega}\Vert\bfv^{h}-\bfv^{\ast}\Vert_{0,\Omega}+\Vert\bfu^{h}-\bfu^{\ast}\Vert_{0,\Omega}\Vert\bfv^{\ast}\Vert_{0,\Omega})\\
&  \leq C(\Vert\bfu^{h}\Vert_{0,4,\Omega}\Vert\bfv^{h}-\bfv^{\ast}\Vert_{0,4,\Omega}+\Vert\bfu^{h}-\bfu^{\ast}\Vert_{0,4,\Omega}\Vert\bfv^{\ast}\Vert_{0,4,\Omega})\\
&  \leq C(\vert\bfu^{h}\vert_{1,\Omega}\vert\bfv^{h}-\bfv^{\ast}\vert_{1,\Omega}+\vert\bfu^{h}-\bfu^{\ast}\vert_{1,\Omega}\vert\boldsymbol{\hat{v}}\vert_{1,\Omega})
\end{align*}
where, applying Lemmas \ref{lipschitz} and \ref{rel-states},%
\begin{align*}
\vert\bfu^{h}-\bfu^{\ast}\vert_{1,\Omega}  &  \leq\vert \bfu^{h}-\boldsymbol{\hat{u}}\vert_{1,\Omega}+\vert \boldsymbol{\hat{u}}-\bfu^{\ast}\vert_{1,\Omega}\\
&  \leq\Vert A(\gamma^{h})-A_{h}(\gamma^{h})\Vert+\Vert A(\gamma^{h})-A(\gamma^{\ast})\Vert\\
&  \leq C\eta_{S,h}+C\Vert\gamma^{h}-\gamma^{\ast}\Vert=C(\eta_{S,h}+\eta_{C,h})
\end{align*}
Analogously, $\vert \bfu^{h}-\bfu^{\ast}\vert_{1,\Omega}\leq C(\eta_{A,h}+\eta_{C,h})$. Since $\vert \bfu^{h}\vert_{1,\Omega}\leq C\Vert\bff\Vert_{0,\Omega}$ and $\vert \boldsymbol{\hat{v}}\vert_{1,\Omega}\leq C(\Vert\bff\Vert_{0,\Omega}+\Vert\bfu_{0}\Vert_{0,\omega})$, we obtain
\[
\Vert\gamma^{h}-\bar{\gamma}\Vert_{0,\Omega}\leq C(\eta_{S,h}+\eta_{A,h}+\eta_{C,h})=C\eta_{h}%
\]
proving that%
\[
\Vert e(\gamma)\Vert_{0,\Omega}\leq\Vert\bar{\gamma}-\gamma^{\ast}\Vert_{0,\Omega}+\eta_{C,T}\leq C\eta_{h}%
\]
Following the same reasoning, we have
\begin{align*}
\Vert(e(\bfu),e(p))\Vert &  \leq\Vert A(\bar{\gamma})-A(\gamma^{h})\Vert+\Vert A(\gamma^{h})-A_{h}(\gamma^{h})\Vert\\
&  \leq C\Vert\bar{\gamma}-\gamma^{h}\Vert_{0,\Omega}+C\eta_{S,h}\\
&  \leq C\eta_{h}
\end{align*}
and%
\begin{align*}
\Vert(e(\bfv),e(q))\Vert &  \leq\Vert B(\bar{\gamma})-B(\gamma^{h})\Vert+\Vert B(\gamma^{h})-B_{h}(\gamma^{h})\Vert\\
&  \leq C\Vert\bar{\gamma}-\gamma^{h}\Vert_{0,\Omega}+C\eta_{A,h}\\
&  \leq C\eta_{h}
\end{align*}
In conclusion, $\Vert(e(\bfu),e(p))\Vert+\Vert(e(\bfv),e(q))\Vert+\Vert e(\gamma)\Vert_{0,\Omega}\leq C\eta_{h}$. For the semidiscrete scheme, we have $\gamma^{h}=\gamma^{\ast}$. Then, we repeat the same previous analysis to obtain the estimate. We omit the details for this case.
\end{proof}

\subsection{Efficiency of the a posteriori error estimators}

\begin{lemma}\label{eff-control}
$\eta_{C,h}\leq C\Vert(e(\bfu),e(p))\Vert+\Vert(e(\bfv%
),e(q))\Vert+\Vert e(\gamma)\Vert_{0,\Omega}$
\end{lemma}

\begin{proof}
We have%
\[
\eta_{C,h}=\Vert\gamma^{\ast}-\gamma_{h}\Vert_{0,\Omega}\leq\Vert\gamma^{\ast}-\bar{\gamma}\Vert_{0,\Omega}+\Vert\bar{\gamma}-\gamma^{h}\Vert_{0,\Omega}=\Vert\gamma^{\ast}-\bar{\gamma}\Vert_{0,\Omega}+\Vert e(\gamma)\Vert_{0,\Omega}%
\]
Since $\Pi_{\lbrack a,b]}(\cdot)$ is Lipschitz, applying Holder inequality and Sobolev Embeding Theorem,
\begin{align*}
\Vert\gamma^{\ast}-\bar{\gamma}\Vert_{0,\Omega}  &  =\left\Vert \Pi_{[a,b]}\left(  \gamma_{0}+\dfrac{1}{\alpha}\bfu^{h}\cdot\bfv^{h}\right)  -\Pi_{\lbrack a,b]}\left(  \gamma_{0}+\dfrac{1}{\alpha}\boldsymbol{\bar{u}}\cdot\boldsymbol{\bar{v}}\right)  \right\Vert_{0,\Omega}\\
&  \leq\dfrac{1}{\alpha}\Vert\bfu^{h}\cdot\bfv^{h}-\boldsymbol{\bar{u}}\cdot\boldsymbol{\bar{v}}\Vert_{0,\Omega}\\
&  \leq\dfrac{1}{\alpha}\Vert\bfu^{h}\cdot(\bfv^{h}-\boldsymbol{\bar{v}})+(\bfu^{h}-\boldsymbol{\bar{u}})\cdot\boldsymbol{\bar{v}}\Vert_{0,\Omega}\\
&  \leq\dfrac{1}{\alpha}(\Vert\bfu^{h}\Vert_{0,4,\Omega}\Vert\bfv^{h}-\boldsymbol{\bar{v}}\Vert_{0,4,\Omega}+\Vert\bfu^{h}-\boldsymbol{\bar{u}}\Vert_{0,4,\Omega}\Vert\boldsymbol{\bar{v}}\Vert_{0,4,\Omega})\\
&  \leq\dfrac{C}{\alpha}(\vert \bfu^{h}\vert_{1,\Omega}\vert \bfv^{h}-\boldsymbol{\bar{v}}\vert_{1,\Omega}+\vert \bfu^{h}-\boldsymbol{\bar{u}}\vert_{1,\Omega}\vert \boldsymbol{\bar{v}}\vert_{1,\Omega})\\
&  \leq\dfrac{C}{\alpha}(\Vert\bff\Vert_{0,\Omega}\Vert(e(\bfu),e(p))\Vert+(\Vert\bff\Vert_{0,\Omega}+\Vert\bfu_{0}\Vert_{0,\omega})\Vert(e(\bfv),e(q))\Vert)\\
&  \leq C(\Vert(e(\bfu),e(p))\Vert+\Vert(e(\bfv),e(q))\Vert)
\end{align*}
proving this lemma.
\end{proof}

\begin{definition}
For $T\in\mathcal{T}_{h}$ and $E\in\mathcal{E}_{h}$, we denote $\omega(T)=\bigcup\{T^{\prime}\in\mathcal{T}_{h}$ $\mid$ $\overline{T}\cap\overline{T^{\prime}}\neq\emptyset\}$ and $\omega(E)=\bigcup\{T^{\prime}\in\mathcal{T}_{h}$ $\mid$ $E\cap\overline{T^{\prime}}\neq\emptyset\}$. We denote by $\psi_{T}$ and $\psi_{E}$ the element and edge bubble functions and by $\mathcal{P}:C(E)  \rightarrow C(T)  $ the continuation operator (see Section 3.1 in \cite{V96}).
\end{definition}

\begin{lemma}\label{bubble}
Given $k\in\NN$, there exists a constant $C>0$, depending only on $k$ and the shape-regularity of $\mathcal{T}_{h}$, such that for all $T\in\mathcal{T}_h$, all $E$ edge of $T$, all $q\in\mathbb{P}_{k}\left(  T\right)  $, and all $r\in\mathbb{P}_{k}\left(  E\right)  $ we have
\begin{align*}
\Vert q\Vert _{0,T}^{2}  &  \leq C\Vert \psi_{T}^{1/2}q\Vert _{0,T}^{2}\\
\Vert r\Vert _{0,E}^{2}  &  \leq C\Vert \psi_{E}^{1/2}r\Vert _{0,E}^{2}\\
\Vert \psi_{E}^{1/2} \mathcal{P}(r)\Vert _{0,T}^{2}  &  \leq Ch_{E}\Vert r\Vert _{0,E}^{2}.
\end{align*}
\end{lemma}
\begin{proof}
See Lemma 3.3 in \cite{V96}.
\end{proof}

Let $(\bfw,r)\in V$. Then, applying integration by parts on each $T\in\mathcal{T}_{h}$, we have%
\begin{align*}
&  a(e(\bfu),\bfw)+c(e(\bfu),\boldsymbol{\bar{u}},\bfw)+c(\bfu^{h},e(\bfu),\bfw)-b(\bfw,e(p))+b(e(\bfu),r)+(e(\gamma)\boldsymbol{\bar{u}},\bfw)+(\gamma^{h}e(\bfu),\bfw)\\
&  =\sum\limits_{T\in\mathcal{T}_{h}} [(\mathcal{R}_{T},\bfw)_{0,T}-(r,\operatorname{div}(\bfu^{h}))]  +\sum_{E\in\mathcal{E}_{h}}(\mathcal{J}_{E},\bfw)_{0,E}
\end{align*}
and%
\begin{align*}
&  a(e(\bfv),\bfw)-b(\boldsymbol{r},e(q))+b(e(\bfv),r)+c(\boldsymbol{\bar{u}},\bfw,e(\bfv))+c(e(\bfu),\bfw,\bfv^{h})+(e(\gamma)\boldsymbol{\bar{v}},\bfw)+(\gamma^{h}e(\boldsymbol{\bar{v}}),\bfw)\\
&  =\sum\limits_{T\in\mathcal{T}_{h}} [(\mathcal{R}_{A,T},\bfw)_{0,T}-(r,\operatorname{div}(\bfv^{h}))] +\sum_{E\in\mathcal{E}_{h}}(\mathcal{J}_{A,E},\bfw)_{0,E}%
\end{align*}
Now we proceed by cases. For the discrete scheme, we define
\begin{align*}
\mathcal{R}_{T}^{0}  &  =P(\bff)+\nu\triangle\bfu^{h}-(\nabla\bfu^{h})\bfu^{h}-\nabla p^{h}-\gamma^{h}\bfu^{h}\\
\mathcal{R}_{A,T}^{0}  &  =\chi_{\omega}\bfu^{h}-P(\chi_{\omega}\bfu_{0})+\nu\triangle\bfv^{h}+(\nabla\bfv^{h})\bfu^{h}-(\nabla\bfu^{h})^{T}\bfv^{h}-\nabla q^{h}-\gamma^{h}\bfv^{h}%
\end{align*}
where $\xi_\omega\in L^{\infty} (\Omega)$ is the indicator function of $\omega$. Then,
\begin{align*}
\sum\limits_{T\in\mathcal{T}_{h}}(\mathcal{R}_{T},\bfw)_{0,T}  & =\sum\limits_{T\in\mathcal{T}_{h}}(\mathcal{R}_{T}^{0},\bfw)_{0,T}+\sum\limits_{T\in\mathcal{T}_{h}}(\bff-P(\bff),\bfw)_{0,T}\\
\sum\limits_{T\in\mathcal{T}_{h}}(\mathcal{R}_{A,T},\bfw)_{0,T}  & =\sum\limits_{T\in\mathcal{T}_{h}}(\mathcal{R}_{T}^{0},\bfw)_{0,T}+\sum\limits_{T\in\mathcal{T}_{h}}(\chi_{\omega}\bfu_{0}-P(\chi_{\omega}\bfu_{0}),\bfw)_{0,T}%
\end{align*}
Similarly, we define for the semidiscrete scheme%
\begin{align*}
\mathcal{R}_{T}^{0}  &  =P(\bff)+\nu\triangle\bfu^{h}-(\nabla\bfu^{h})\bfu^{h}-\nabla p^{h}-P(\bar{\gamma}\boldsymbol{\bar{u}})\\
\mathcal{R}_{A,T}^{0}  &  =\chi_{\omega}\bfu^{h}-P(\chi_{\omega}\bfu_{0})+\nu\triangle\bfv^{h}+(\nabla\bfv^{h})\bfu^{h}-(\nabla\bfu^{h})^{T}\bfv^{h}-\nabla q^{h}-P(\bar{\gamma}\boldsymbol{\bar{v}})
\end{align*}
where%
\begin{align*}
\sum\limits_{T\in\mathcal{T}_{h}}(\mathcal{R}_{T},\bfw)_{0,T}  & =\sum\limits_{T\in\mathcal{T}_{h}}\left[  (\mathcal{R}_{T}^{0},\bfw)_{0,T}+(e(\gamma)\bfu^{h},\bfw)_{0,T}+(\bar{\gamma}e(\bfu),\bfw)_{0,T}+(\bff-P(\bff),\bfw)_{0,T}-(\bar{\gamma}\boldsymbol{\bar{u}}-P(\bar{\gamma}\boldsymbol{\bar{u}}),\bfw)_{0,T}\right] \\
\sum\limits_{T\in\mathcal{T}_{h}}(\mathcal{R}_{A,T},\bfw)_{0,T}  & =\sum\limits_{T\in\mathcal{T}_{h}}\left[  (\mathcal{R}_{T}^{0},\bfw)_{0,T}+(e(\gamma)\bfv^{h},\bfw)_{0,T}+(\bar{\gamma}e(\bfv),\bfw)_{0,T}+(\bff-P(\bff),\bfw)_{0,T}-(\bar{\gamma}\boldsymbol{\bar{v}}-P(\bar{\gamma}\boldsymbol{\bar{v}}),\bfw)_{0,T}\right]
\end{align*}

\begin{definition}
Let $\boldsymbol{g}\in\boldsymbol{L}^{2}(\Omega)$ we denote by $\Theta(\bff)$ the oscillation residual term given by%
\[
\Theta(\boldsymbol{g})=\left(  \sum\limits_{T\in\mathcal{T}_{h}}h_{T}^{2}%
\Vert\boldsymbol{g}-P(\boldsymbol{g})\Vert_{0,T}^{2}\right)  ^{1/2}%
\]
\end{definition}

First, we detail the proof of the efficiency in the discrete case.

\begin{lemma}\label{eff-states}
For the discrete case, $\eta_{S,h}\leq C\left(  \Vert(e(\bfu),e(p))\Vert+\Vert e(\gamma)\Vert_{0,\Omega}+\Theta(\bff)\right)  $
\end{lemma}

\begin{proof}
First, consider $T\in\mathcal{T}_{h}$. Taking $(\bfw,r)=(\psi_{T}\mathcal{R}_{T}^{0},0)$, since $\bfw=\boldsymbol{0}$ in $\Omega\setminus T$,
\begin{align*}
\Vert\psi_{T}^{1/2}\mathcal{R}_{T}^{0}\Vert_{0,T}^{2}=  &  \sum\limits_{T\in\mathcal{T}_{h}}(\mathcal{R}_{T}^{0},\bfw)_{0,T}-\sum\limits_{T\in\mathcal{T}_{h}}(\bff-P(\bff),\bfw)_{0,T}\\
=  &  a(e(\bfu),\bfw)+c(e(\bfu),\boldsymbol{\bar{u}},\bfw)+c(\bfu^{h},e(\bfu),\bfw)-b(\bfw,e(p))\\
&  +(e(\gamma)\boldsymbol{\bar{u}},\bfw)+(\gamma^{h}e(\bfu),\bfw)-\sum\limits_{T\in\mathcal{T}_{h}}(\bff-P(\bff),\bfw)_{0,T}
\end{align*}
Applying Lemmas \ref{bounds-1} and \ref{bubble} , Holder inequality and Sobolev Embedding theorem,%
\begin{align*}
\Vert\psi_{T}^{1/2}\mathcal{R}_{T}^{0}\Vert_{0,T}^{2}\leq &  \nu\vert e(\bfu)\vert_{1,T}\vert \bfw\vert_{1,T}+\beta\vert \bfw\vert_{1,T} \vert e(\bfu)\vert_{1,T}(\vert \boldsymbol{\bar{u}}\vert_{1,T}+\vert \bfu^{h}\vert_{1,T})+\sqrt{d}\vert \boldsymbol{w \vert }_{1,T}\Vert e(p)\Vert_{0,T}\\
&  +\Vert e(\gamma)\Vert_{0,T}\Vert\boldsymbol{\bar{u}}\cdot\bfw\Vert_{0,T}+\Vert\gamma^{h}\Vert_{0,T}\Vert e(\bfu)\cdot\bfw\Vert_{0,T}+\Vert\bff-P(\bff)\Vert_{0,T}\Vert\bfw\Vert_{0,T}\\
\leq &  \nu \vert e(\bfu)\vert_{1,T}\vert \bfw\vert_{1,T} +\beta\vert \bfw\vert_{1,T} \vert e(\bfu)\vert_{1,T}(\vert \boldsymbol{\bar{u}}\vert_{1,T}+\vert \bfu^{h}\vert_{1,T})+\sqrt{d}\vert \boldsymbol{w \vert }_{1,T}\Vert e(p)\Vert_{0,T}\\
&  +\Vert e(\gamma)\Vert_{0,T}\Vert\boldsymbol{\bar{u}}\Vert_{0,4,T} \Vert\bfw\Vert_{0,T}+\Vert\gamma^{h}\Vert_{0,T}\Vert e(\bfu)\Vert_{0,4,T}\Vert\bfw\Vert_{0,4,T}+\Vert\bff-P(\bff)\Vert_{0,T}\Vert\bfw\Vert_{0,T}\\
\leq &  \nu \vert e(\bfu)\vert_{1,T}\vert \bfw\vert_{1,T}+\beta\vert \bfw\vert_{1,T} \vert e(\bfu)\vert_{1,T}(\vert \boldsymbol{\bar{u}}\vert_{1,T}+\vert \bfu^{h}\vert_{1,T})+\sqrt{d}\vert \bfw\vert_{1,T}\Vert e(p)\Vert_{0,T}\\
&  +C\vert \bfw\vert_{1,T}(\Vert e(\gamma)\Vert_{0,T}\vert \boldsymbol{\bar{u}}\vert_{1,T}+\Vert\gamma^{h}\Vert_{0,T} \vert e(\bfu)\vert_{1,T})+\Vert\bff-P(\bff)\Vert_{0,T}\Vert\bfw\Vert_{0,T}\\
\leq &  C\vert \bfw\vert_{1,T}( \vert e(\bfu)\vert_{1,T}+\Vert e(p)\Vert_{0,T}+\Vert e(\gamma)\Vert_{0,T})+\Vert\bff-P(\bff)\Vert_{0,T}\Vert\bfw\Vert_{0,T}%
\end{align*}
Since $\bfw$ is polynomial, we have $\vert \bfw\vert_{1,T}\leq Ch_{T}^{-1}\Vert\bfw\Vert_{0,T}\leq Ch_{T}^{-1}\Vert\mathcal{R}_{T}^{0}\Vert_{0,T}$ by applying an inverse inequality (see Lemma 1.138 from \cite{EG04}). Then,
\begin{align*}
C\Vert\mathcal{R}_{T}^{0}\Vert_{0,T}^{2}  &  \leq\Vert\psi_{T}^{1/2}\mathcal{R}_{T}^{0}\Vert_{0,T}^{2}\leq C\vert \bfw\vert_{1,T}(\vert e(\bfu)\vert_{1,T}+\Vert e(p)\Vert_{0,T}+\Vert e(\gamma)\Vert_{0,T})+\Vert\bff-P(\bff)\Vert_{0,T}\Vert\bfw\Vert_{0,T}\\
\Vert\mathcal{R}_{T}^{0}\Vert_{0,T}  &  \leq Ch_{T}^{-1}( \vert e(\bfu)\vert_{1,T}+\Vert e(p)\Vert_{0,T}+\Vert e(\gamma)\Vert_{0,T})+C\Vert\bff-P(\bff)\Vert_{0,T}\\
h_{T}^{2}\Vert\mathcal{R}_{T}^{0}\Vert_{0,T}^{2}  &  \leq C( \vert e(\bfu)\vert_{1,T}^{2}+\Vert e(p)\Vert_{0,T}^{2}+\Vert e(\gamma)\Vert_{0,T}^{2})+Ch_{T}^{2}\Vert\bff-P(\bff)\Vert_{0,T}^{2}
\end{align*}
but $\Vert\mathcal{R}_{T}\Vert_{0,T}\leq\Vert\mathcal{R}_{T}^{0}\Vert _{0,T}+\Vert\bff-P(\bff)\Vert_{0,T}$. Then,
\begin{align*}
h_{T}^{2}\Vert\mathcal{R}_{T}\Vert_{0,T}^{2}  &  \leq C( \vert e(\bfu)\vert_{1,T}^{2}+\Vert e(p)\Vert_{0,T}^{2}+\Vert e(\gamma)\Vert_{0,T}^{2}+h_{T}^{2}\Vert\bff-P(\bff)\Vert_{0,T}^{2})\\
\sum\limits_{T\in\mathcal{T}_{h}}h_{T}^{2}\Vert\mathcal{R}_{T}\Vert_{0,T}^{2} &  \leq C\left(  \Vert e(\bfu),e(p)\Vert+\Vert e(\gamma)\Vert_{0,\Omega}^{2}+\sum\limits_{T\in\mathcal{T}_{h}}h_{T}^{2}\Vert\bff-P(\bff)\Vert_{0,T}^{2}\right)
\end{align*}
Now consider $E\in\mathcal{E}_{h}$ and $(\bfw,r)=(\psi_{E}P_{E}(\mathcal{J}_{E}),0)$. Since $\bfw=\boldsymbol{0}$ in $\Omega\setminus\omega(E)$, reasoning as before, we have
\begin{align*}
\Vert\psi_{E}^{1/2}\mathcal{J}_{E}\Vert_{0,E}^{2}=  &  \sum_{E\in\mathcal{E}_{h}}(\mathcal{J}_{E},\bfw)_{0,E}\\
=  &  a(e(\bfu),\bfw)+c(e(\bfu),\boldsymbol{\bar{u}},\bfw)+c(\bfu^{h},e(\bfu),\bfw)-b(\bfw,e(p))\\
&  +(e(\gamma)\boldsymbol{\bar{u}},\bfw)+(\gamma^{h}e(\bfu),\bfw)-\sum\limits_{T\in\mathcal{T}_{h}}\left[  (\bff-P(\bff),\bfw)_{0,T}+(\mathcal{R}_{T},\bfw)_{0,T}\right] \\
\leq &  C\vert \bfw\vert_{1,\omega(E)}( \vert e(\bfu)\vert_{1,\omega(E)}+\Vert e(p)\Vert_{0,\omega(E)}+\Vert e(\gamma)\Vert_{0,\omega(E)})+(\sum\limits_{T\in\mathcal{T}_{h}}\Vert\bff-P(\bff)\Vert_{0,\omega(E)}+\Vert\mathcal{R}_{T}\Vert_{0,\omega(E)})\Vert\bfw\Vert_{0,\omega(E)}
\end{align*}
where $\vert \bfw\vert_{1,\omega(E)}\leq Ch_{E}^{-1}\Vert\bfw \Vert_{0,\omega(E)}\leq Ch_{E}^{-1/2}\Vert\mathcal{J}_{E}\Vert_{0,E}$ and $\Vert\mathcal{J}_{E}\Vert_{0,E}\leq C\Vert\psi_{E}^{1/2}\mathcal{J}_{E} \Vert_{0,E}$ by applying some inverse inequalities (see Lemma 1.138 in \cite{EG04}). Then,%
\begin{align*}
C\Vert\mathcal{J}_{E}\Vert_{0,E}^{2}  &  \leq\Vert\psi_{E}^{1/2}\mathcal{J}_{E}\Vert_{0,E}^{2}\leq C \vert \bfw\vert_{1,\omega(E)}( \vert e(\bfu)\vert_{1,\omega(E)}+\Vert e(p)\Vert_{0,\omega(E)}+\Vert e(\gamma)\Vert_{0,\omega(E)})+(\Vert\bff-P(\bff)\Vert_{0,\omega(E)}+\Vert\mathcal{R}_{T}\Vert_{0,\omega(E)})\Vert
\bfw\Vert_{0,\omega(E)}\\
h_{E}^{1/2}\Vert\mathcal{J}_{E}\Vert_{0,E}  &  \leq C( \vert e(\bfu)\vert_{1,\omega(E)}+\Vert e(p)\Vert_{0,\omega(E)}+\Vert e(\gamma)\Vert_{0,\omega(E)}+h_{E}(\Vert\bff-P(\bff)\Vert_{0,\omega(E)}+\Vert\mathcal{R}_{T}\Vert_{0,\omega(E)}))\\
h_{E}\Vert\mathcal{J}_{E}\Vert_{0,E}^{2}  &  \leq C( \vert e(\bfu)\vert_{1,\omega(E)}^{2}+\Vert e(p)\Vert_{0,\omega(E)}^{2}+\Vert e(\gamma)\Vert_{0,\omega(E)}^{2}+h_{E}^{2}(\Vert\bff-P(\bff)\Vert_{0,\omega(E)}^{2}+\Vert\mathcal{R}_{T}\Vert_{0,\omega(E)}^{2}))\\
\sum_{E\in\mathcal{E}_{h}}h_{E}\Vert\mathcal{J}_{E}\Vert_{0,E}^{2}  &  \leq C\sum_{E\in\mathcal{E}_{h}}( \vert e(\bfu)\vert_{1,\omega(E)}^{2}+\Vert e(p)\Vert_{0,\omega(E)}^{2}+\Vert e(\gamma)\Vert_{0,\omega(E)}^{2}+h_{E}^{2}(\Vert\bff-P(\bff)\Vert_{0,\omega(E)}^{2}+\Vert\mathcal{R}_{T}\Vert_{0,\omega(E)}^{2}))\\
\sum_{E\in\mathcal{E}_{h}}h_{E}\Vert\mathcal{J}_{E}\Vert_{0,E}^{2}  &  \leq C\left(  \Vert e(\bfu),e(p)\Vert^{2}+\Vert e(\gamma)\Vert_{0,\Omega}^{2}+\sum\limits_{T\in\mathcal{T}_{h}}h_{T}^{2}\Vert\bff-P(\bff)\Vert_{0,T}^{2}\right)
\end{align*}
Finally, it is direct that $\Vert\operatorname{div}\bfu^{h}\Vert_{0,T}^{2}=\Vert\operatorname{div}(\boldsymbol{\bar{u}}-\bfu^{h})\Vert_{0,T}^{2}\leq\sqrt{d} \vert \boldsymbol{\bar{u}}-\bfu^{h}\vert_{1,T}^{2}$. Then, $\sum\limits_{T\in\mathcal{T}_{h}}\Vert\operatorname{div}\bfu^{h}\Vert_{0,T}^{2}\leq\sqrt{d}\Vert(e(\bfu),e(p))\Vert^{2}$. The final estimate is obtained by minor algebraic manipulations.
\end{proof}

\begin{lemma}\label{eff-adj}
If there exists $\mathcal{T}_{\omega}\subseteq\mathcal{T}_{h}$ such that $\overline{\omega}=\bigcup\left\{  T^{\prime}\in\mathcal{T}_{\omega}\right\}$, then $\eta_{A,h}\leq C(\Vert(e(\bfu),e(p))\Vert+\Vert(e(\bfv),e(q))\Vert+\Vert e(\gamma)\Vert_{0,\Omega}+\Theta(\bff)+\Theta(\chi_{\omega_{0}}\bfu_{0}))$ for the discrete case.
\end{lemma}

\begin{proof}
The deduction follows the same scheme as the proof of the previous lemma. We omit the details.
\end{proof}

\begin{theorem}
For the discrete scheme,%
\[
\eta_{h}\leq C(\Vert(e(\bfu),e(p))\Vert+\Vert(e(\bfv),e(q))\Vert+\Vert e(\gamma)\Vert_{0,\Omega}+\Theta(\bff%
)+\Theta(\chi_{\omega_{0}}\bfu_{0}))
\]
\end{theorem}

\begin{proof}
It is direct consequence of Lemmas \ref{eff-control}, \ref{eff-states} and \ref{eff-adj}, and some algebraic manipulations. We omit the details.
\end{proof}

\begin{theorem}
For the semidiscrete scheme,%
\[
\eta_{h}\leq C(\Vert(e(\bfu),e(p))\Vert+\Vert(e(\bfv),e(q))\Vert+\Vert e(\gamma)\Vert_{0,\Omega}+\Theta(\bff)+\Theta(\chi_{\omega_{0}}\bfu_{0})+\Theta(\bar{\gamma}\boldsymbol{\bar{u}})+\Theta(\bar{\gamma}\boldsymbol{\bar{v}}))
\]

\end{theorem}

\begin{proof}
Following the same steps as in the proof of Lemmas \ref{eff-states} and \ref{eff-adj}, we can obtain the inequalities%
\begin{align*}
\eta_{S,h}  &  \leq C\left(  \Vert(e(\bfu),e(p))\Vert+\Vert e(\gamma)\Vert_{0,\Omega}+\Theta(\bff)+\Theta(\bar{\gamma}\boldsymbol{\bar{u}})\right) \\
\eta_{A,h}  &  \leq C(\Vert(e(\bfu),e(p))\Vert+\Vert (e(\bfv),e(q))\Vert+\Vert e(\gamma)\Vert_{0,\Omega}+\Theta(\bff)+\Theta(\chi_{\omega_{0}}\bfu_{0})+\Theta (\bar{\gamma}\boldsymbol{\bar{u}})+\Theta(\bar{\gamma}\boldsymbol{\bar{v}}))
\end{align*}
We omit the details.
\end{proof}

\section*{Acknowledgements}
The author acknowledges Axel Osses and Rodolfo Araya for the fruitful discussions about this article. The author also thanks for the funding of ANID CMM FB210005 Basal.

\bibliographystyle{elsarticle-harv}
\bibliography{ref.bib}
\end{document}
