% SIAM Shared Information Template
% This is information that is shared between the main document and any
% supplement. If no supplement is required, then this information can
% be included directly in the main document.
%\usepackage{caption}
%\usepackage{subcaption}
\usepackage{array}
\newcolumntype{P}[1]{>{\centering\arraybackslash}p{#1}}
\usepackage{multirow}

\usepackage{caption}
\usepackage{subcaption}

% Packages and macros go here
\usepackage{lipsum}
\usepackage{amsfonts}
\usepackage{graphicx}
\usepackage{epstopdf}
\usepackage{amsmath,mathtools}
%\usepackage{algorithmic}
\usepackage{algorithm}
% \usepackage[noend]{algpseudocode}
\usepackage{algpseudocode}
\usepackage{amsmath,amssymb,amsfonts,amsthm,bm}
\usepackage[inline]{enumitem}   
\usepackage{tcolorbox}
\usepackage{mathtools}
\newcommand{\st}{\text{s.t.}}
\newcommand{\dist}{\text{dist}}
\newcommand{\unitVec}{\mathbf{e}}
\newcommand{\barS}{\bar{S}}
\newcommand{\RR}{\mathbb{R}}
\newcommand{\NN}{\mathbb{N}}
\newcommand{\vecOnes}{\mathbf{1}}
\newcommand{\vecZeros}{\mathbf{0}}
\newcommand{\YYY}{\mathcal{Y}}
\newcommand{\NNN}{\mathcal{N}}
\newcommand{\DDD}{\mathcal{D}}
\newcommand{\PPP}{\mathcal{P}}
\newcommand{\SSS}{\mathcal{S}}
\newcommand{\FFF}{\mathcal{F}}
\newcommand{\III}{\mathcal{I}}
\newcommand{\AAA}{\mathcal{A}}
\newcommand{\OOO}{\mathcal{O}}
\newcommand{\ZZZ}{\mathcal{Z}}
\newcommand{\HHH}{\mathcal{H}}
\newcommand{\LLL}{\mathcal{L}}
\newcommand{\BBB}{\mathcal{B}}
\newcommand{\KKK}{\mathcal{K}}
\newcommand{\CCC}{\mathcal{C}}
\newcommand{\MMM}{\mathcal{M}}
\newcommand{\barP}{\bar{\PPP}}
\newcommand{\QQQ}{\mathcal{Q}}
\newcommand{\TTT}{\mathcal{T}}
\newcommand{\XXX}{\mathcal{X}}
\newcommand{\EEE}{\mathcal{E}}

\newcommand{\ftilde}{{\tilde f}}
\newcommand{\ctilde}{{\tilde c}}
\newcommand{\gtilde}{{\tilde g}}
\newcommand{\Jtilde}{{\tilde J}}
\newcommand{\Wtilde}{{\tilde W}}
\newcommand{\Htilde}{{\tilde H}}
\newcommand{\mtilde}{{\tilde m}}
\newcommand{\dtilde}{{\tilde d}}
\newcommand{\Ztilde}{{\tilde Z}}
\newcommand{\phitilde}{{\tilde \phi}}
\newcommand{\Ptilde}{{\tilde P}}
\newcommand{\JJtilde}{{\tilde J_k  \tilde J_k^T}}
\newcommand{\JJbar}{{\bar J_k  \bar J_k^T}}
\newcommand{\ttilde}{{\tilde t}}
\newcommand{\tbar}{{\bar t}}
\newcommand{\that}{{\hat t}}
\newcommand{\Gammatilde}{{\tilde \Gamma}}
\newcommand{\Gammabar}{{\bar \Gamma}}
\newcommand{\Qtilde}{{\tilde Q}}
\newcommand{\Qbar}{{\bar Q}}

\newcommand{\Zbar}{{\bar Z}}
\newcommand{\pbar}{{\bar p}}
\newcommand{\fbar}{{\bar f}}
\newcommand{\cbar}{{\bar c}}
\newcommand{\gbar}{{\bar g}}
\newcommand{\Jbar}{{\bar J}}
\newcommand{\Wbar}{{\bar W}}
\newcommand{\Hbar}{{\bar H}}
\newcommand{\mbar}{{\bar m}}
\newcommand{\dbar}{{\bar d}}
\newcommand{\phibar}{{\bar \phi}}
\newcommand{\taubar}{{\bar \tau}}
\newcommand{\vbar}{{\bar v}}
\newcommand{\ubar}{{\bar u}}
\newcommand{\ybar}{{\bar y}}
\newcommand{\taubarlb}{\bar\tau_{\mathrm{lb}}}
\newcommand{\tautildelb}{\tilde \tau_{\mathrm{lb}}}
\newcommand{\klb}{k_{\mathrm{lb}}}
\newcommand{\Pbar}{{\bar P}}
\newcommand{\vtilde}{{\tilde v}}
\newcommand{\Pibar}{{\bar \Pi}}
\newcommand{\Pitilde}{{\tilde \Pi}}


\newcommand{\zwzbar}{{\bar Z_k^T \tilde W_k \bar Z_k}}
\newcommand{\zwztilde}{{\tilde Z_k^T \tilde W_k \tilde Z_k}}



\newcommand{\dhat}{{\hat d}}
\newcommand{\alphahat}{{\hat \alpha}}
\newcommand{\alphabar}{{\bar \alpha}}


\newcommand{\hatx}{\hat{x}}
\newcommand{\hatp}{\hat{p}}
\newcommand{\ttt}{\theta}
\def\gray{\textcolor{gray}}
\def\cyan{\textcolor{cyan}}
\def\olive{\textcolor{olive}}
\def\violet{\textcolor{violet}}
\def\orange{\textcolor{orange}}
\def\purple{\textcolor{purple}}
\def\brown{\textcolor{brown}}
\def\teal{\textcolor{teal}}
\def\green{\textcolor{green}}
\def\red{\textcolor{red}}
\def\blue{\textcolor{blue}}
\def\magenta{\textcolor{magenta}}
\newcommand\fec[1]{\textcolor{magenta}{#1}}
\newcommand{\tr}{\text{\rm tr}}
\newcommand{\trc}{{\text{\rm tr}^*}}
\newcommand{\apx}{\text{\rm app}}
\newcommand{\mc}{\text{\rm mc}}
\newcommand{\ef}{\text{\rm ef}}
\newcommand{\eg}{\text{\rm eg}}
\newcommand{\sub}{\text{\rm sub}}
\newcommand{\thresh}{\text{\rm thr}}
\newcommand{\fcd}{\text{\rm fcd}}
\newcommand{\icb}{\text{\rm icb}}
\newcommand{\inc}{\text{\rm inc}}
\newcommand{\dec}{\text{\rm dec}}
\newcommand{\bhm}{\text{\rm bhm}}
\newcommand{\enl}{\text{\rm enl}}
\newcommand{\gcp}{\text{\rm gcp}}
\newcommand{\alg}{\text{\rm Alg}}
\newcommand{\curv}{\text{\rm curv}}
\newcommand{\crit}{\text{\rm crit}}
\newcommand{\nbd}{\text{\rm nbd}}
\newcommand{\nulll}{\text{\rm null}}
\newcommand{\epsktau}{\vareps_{\tau,k}}
\newcommand{\epskd}{\varepsilon_{d,k}}
\newcommand{\epsku}{\varepsilon_{u,k}}
\newcommand{\epskv}{\varepsilon_{v,k}}
\newcommand{\Ekmtwo}{E_k^{m, 2}}
\newcommand{\Ekmone}{E_k^{m, 1}}
\newcommand{\Ekm}{E_k^m}
\newcommand{\epskg}{\vareps_{g,k}}
\newcommand{\epskc}{\vareps_{c,k}}
\newcommand{\epskJ}{\vareps_{J,k}}
\newcommand{\epskds}{\vareps_{{d}^s,k}}
\newcommand{\epskdx}{\vareps_{{d}^x,k}}
\newcommand{\Ekphitwo}{E_k^{\phi, 2}}
\newcommand{\Ekphione}{E_k^{\phi, 1}}
\newcommand{\Ekmphi}{E_k^{m, \phi}}
\newcommand{\calEkmone}{\mathcal{E}_k^{m, 1}}
\newcommand{\calEkmtwo}{\mathcal{E}_k^{m, 2}}
\newcommand{\calEkm}{\mathcal{E}_k^m}
\newcommand{\sigmabar}{{\bar \sigma}}
\newcommand{\sigmatilde}{{\tilde \sigma}}

\newcommand{\emOneOne}{e_1^{m, 1}}
\newcommand{\emOneTwo}{e_2^{m, 1}}
\newcommand{\emTwoOne}{c_1^{m, 2}}
\newcommand{\emTwoTwo}{c_2^{m, 2}}

\newcommand{\xizero}[1]{\xi_2(#1)}
\newcommand{\xione}{\xi_{1, k}}
\newcommand{\xitwo}{\xi_5}
\newcommand{\xithree}{\xi_3}
\newcommand{\xifour}{\xi_4}
\newcommand{\xifive}{\xi_6}
\newcommand{\xisix}{\xi_7}


\newcommand{\na}{\nabla}

\newcommand{\sorted}{\textrm{sorted}}
\newcommand{\proj}[2]{\textrm{proj}_{#1}\left(#2\right)} 
\newcommand{\nctrc}{n_x^{(\frac{1}{\trc} - \frac{1}{2})}} %norm constant
\newcommand{\nctr}{n_x^{(\frac{1}{2} - \frac{1}{tr})}} %norm constant
\newcommand{\algorithmicbreak}{\textbf{break}}
\newcommand{\eps}{\epsilon}
\newcommand{\vareps}{\varepsilon}

\newcommand{\BREAK}{\STATE \algorithmicbreak}
\newcommand{\Footnote}[1]{\footnote{#1}}
%\newcommand{\Footnote}[1]{}
\DeclareMathOperator*{\argmax}{arg\,max}
\DeclareMathOperator*{\argmin}{arg\,min}
\DeclareMathOperator{\trial}{trial}
\DeclareMathOperator{\app}{app}
\DeclareMathOperator{\curr}{curr}

%if statements indentation in algorithms
\makeatletter
 \newcommand{\algparbox}[1]{\parbox[t]{\dimexpr\linewidth-\algorithmicindent}{#1\strut}}
 \newcommand{\algparboxtwo}[1]{\parbox[t]{\dimexpr\linewidth-\algorithmicindent-\algorithmicindent}{#1\strut}}
  \newcommand{\algparboxthree}[1]{\parbox[t]{\dimexpr\linewidth-\algorithmicindent-\algorithmicindent-\algorithmicindent}{#1\strut}}
    \newcommand{\algparboxfour}[1]{\parbox[t]{\dimexpr\linewidth-\algorithmicindent-\algorithmicindent-\algorithmicindent-\algorithmicindent}{#1\strut}}
% <=====================================================================
% \algnewcommand{\parState}[1]{\State%
%   \parbox[t]{\dimexpr\linewidth-\algmargin}{\strut #1\strut}}


\newcommand{\zerobf}{\mathbf{0}}
\ifpdf
  \DeclareGraphicsExtensions{.eps,.pdf,.png,.jpg}
\else
  \DeclareGraphicsExtensions{.eps}
\fi

% Add a serial/Oxford comma by default.
\newcommand{\creflastconjunction}{, and~}

% Used for creating new theorem and remark environments
%\newsiamremark{remark}{Remark}
%\newsiamremark{hypothesis}{Hypothesis}
%\crefname{hypothesis}{Hypothesis}{Hypotheses}
%\newsiamthm{claim}{Claim}
%\newsiamthm{assumption}{Assumption}

\newtheorem{theorem}{Theorem}
\newtheorem{assumption}{Assumption}[section]
\newtheorem{corollary}{Corollary}[section]
\newtheorem{definition}{Definition}[section]
\newtheorem{lemma}{Lemma}[section]
\newtheorem{remark}{Remark}[section]

% Sets running headers as well as PDF title and authors
%\headers{Interior-Point for Noisy Constrained Optimization}{Frank E.~Curtis, Shima Dezfulian, Andreas W\"achter}

% Title. If the supplement option is on, then "Supplementary Material"
% is automatically inserted before the title.
%\title{\pagecolor{anthracite}\afterpage{\nopagecolor}


\definecolor{poste_color}{RGB}{222, 184, 65}

\begin{center}
\color{white}
\begin{minipage}{0.99\linewidth}
    \thispagestyle{empty}
    \centering
    \tikz[remember picture,overlay] \node[opacity=0.3,inner sep=0pt] at (current page.center){\includegraphics[width=\paperwidth,height=\paperheight]{assets/topographic2.png}};

    \vspace{-1cm}
    

    \vspace{1cm}
    
    {\Huge {\textbf{Sparks of Explainability}}\\ Recent Advancements in Explaining Large Vision Models\\[0.5em]}

    \vspace{2cm}
    
    {
    Presented by\\
    {\textbf{\Large Thomas Fel}}\\[1em]

    
    Supervised by\\
    \textbf{Prof. Thomas Serre}\\[4em]
    }

    \vspace{1cm}
    
    {
    \raggedright
    Presented and publicly defended on July 25, 2024\\
    }
    \vspace{0em}
    \begin{tabbing}
        \hspace{8cm} \= \kill
        Prof. \textbf{George A. Alvarez} \> Reviewer\\
        \textit{\textcolor{poste_color}{Professor, Harvard University}} \\[0.5em]
        
        Prof. \textbf{Céline Hudelot} \> Reviewer\\
        \textit{\textcolor{poste_color}{Professor, Centrale Paris}} \\[0.5em]

        
        Dr. \textbf{Robert Geirhos} \> Examiner\\
        \textit{\textcolor{poste_color}{Research Scientist, Google}} \\[0.5em]
        
        Prof. \textbf{Ruth Fong} \> Examiner\\
        \textit{\textcolor{poste_color}{Professor, Princeton University}} \\[0.5em]
        
        Prof. \textbf{Rufin Van Rullen} \> Examiner\\
        \textit{\textcolor{poste_color}{Professor, Cerco, CNRS}} \\[0.5em]

        \\
        
        \textbf{Prof. Thomas Serre} \> Thesis Director\\
        \textit{\textcolor{poste_color}{Professor, Brown University \& ANITI}} \\[0.5em]
    \end{tabbing}

    \vspace{1.5cm}
    
    {\Large\textbf{DOCTORAL THESIS}\\[0.5em]}
    {\large Doctoral School of Mathematics, Computer Science, and Telecommunications of Toulouse}\\[2em]
    
\end{minipage}
\end{center}
}
%\titlerunning{An Interior-Point Algorithm with Noisy Function and Derivative Evaluations}

% Authors: full names plus addresses.
%\author{Frank E.~Curtis \and Shima Dezfulian \and Andreas Waechter}

%\institute{%
%  Frank~E.~Curtis \at
%  Department of ISE, Lehigh University, Bethlehem, PA, USA \\
%  \email{frank.e.curtis@lehigh.edu}
%  \and
%  Shima Dezfulian, Andreas Waechter \at
%  Department of IEMS, Northwestern University, Evanston, IL, USA \\
%  \email{dezfulian@u.northwestern.edu}, \email{andreas.waechter@northwestern.edu}
%}

\DeclareMathOperator{\diag}{diag}
\allowdisplaybreaks

\newcommand{\ftrue}{\fbar}
\newcommand{\gtrue}{\gbar}
\newcommand{\ctrue}{\cbar}
\newcommand{\Jtrue}{\Jbar}
\newcommand{\vtrue}{\vbar}
\newcommand{\utrue}{\ubar}
\newcommand{\dtrue}{\dbar}
\newcommand{\ytrue}{\ybar}
\newcommand{\phitrue}{\phibar}
\newcommand{\mtrue}{\mbar}
\newcommand{\tautrue}{\taubar}
\newcommand{\tautruelb}{\taubarlb}
\newcommand{\Wtrue}{\Wbar}
\newcommand{\Htrue}{\Hbar}
\newcommand{\Ztrue}{\Zbar}
\newcommand{\Ptrue}{\Pbar}
\newcommand{\Gammatrue}{\Gammabar}

\newcommand{\JCfrac}{{\tfrac{\|J_k^Tc_k\|}{\|\Jtrue_k^T\ctrue_k\|}}}

%%% Local Variables: 
%%% mode:latex
%%% TeX-master: "ex_article"
%%% End: 
