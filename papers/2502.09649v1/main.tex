\documentclass[conference]{IEEEtran}
\usepackage{times}
\usepackage{graphicx}
\usepackage{tikz}
\usepackage{caption}
\usepackage{algorithm} % For algorithm environment
\usepackage{algorithmic} % For algorithmic environment
\usepackage{tabularx} % For table
\usepackage{makecell}
\usepackage{authblk}  % 引入authblk包

% numbers option provides compact numerical references in the text. 
\usepackage[numbers]{natbib}
\usepackage{multicol}
\usepackage{booktabs}
\usepackage{multirow}
\usepackage{amsmath}
\usepackage{amssymb}
\usepackage[bookmarks=true]{hyperref}

\pdfinfo{
   /Author (Homer Simpson)
   /Title  (Robots: Our new overlords)
   /CreationDate (D:20101201120000)
   /Subject (Robots)
   /Keywords (Robots;Overlords)
}

\begin{document}


% paper title
\title{Imit Diff: Semantics Guided Diffusion Transformer with Dual Resolution Fusion for Imitation Learning}

% You will get a Paper-ID when submitting a pdf file to the conference system
% \author{Author Names Omitted for Anonymous Review. Paper-ID 824}

\author{%
  \textbf{Yuhang Dong}$^1$
  ~~ \textbf{Haizhou Ge}$^2$
  ~~ \textbf{Yupei Zeng}$^1$
  ~~ \textbf{Jiangning Zhang}$^3$
  ~~ \textbf{Beiwen Tian}$^2$
  ~~ \textbf{Guanzhong Tian}$^1$ \\
  ~~ \textbf{Hongrui Zhu}$^1$
  ~~ \textbf{Yufei Jia}$^2$
  ~~ \textbf{Ruixiang Wang}$^4$
  ~~ \textbf{Ran Yi}$^5$ 
  ~~ \textbf{Guyue Zhoui}$^2$
  ~~ \textbf{Longhua Ma}$^1$ \\
  \normalsize $^1$Zhejiang University ~~ $^2$Tsinghua University
  ~~ $^3$Youtu Lab, Tecent \\~~ $^4$Harbin Institute of Technology, Weihai~~ $^5$
Shanghai Jiao Tong University
}


\maketitle

% 插入teaser图
% 我们比较了不同的 visuomotor imitation learning policies。(a) Basic Visuomotor Policy 直接将环境观测和机器人感知 mapping 或 contioning 到 action space 中。(b) High-Level Instruction 额外地通过 VLM 为 policy 进行 high-level 的 text 或运动约束指导。(c) Imit Diff 将 high-level instruction 转化为 pixel-level semantics information 注入到多尺度的视觉特征强化框架中。
\begin{figure*}[htbp]
    \centering
    \includegraphics[width=0.9\textwidth]{Figure/teaser.pdf} 
    \caption{We compare different visuomotor imitation learning policies: (a) \textbf{Basic Visuomotor Policy}, which directly maps or conditions environment observations and robot perception into the action space \citep{zhao2023learning, lee2024behavior}; (b) \textbf{High-Level Instruction Policy}, which incorporates high-level text instruction \citep{liu2024diff, reuss2023multimodal} or motion constraint guidance \citep{huang2023voxposer, pan2025omnimanip, li2024manipllm} for the policy through vision language models (VLM); and (c) \textbf{Imit Diff}, which transforms high-level instructions into pixel-level semantic information and injects it into a multi-scale visual feature enhancement framework.}
    \label{fig:teaser}
\end{figure*}

The escalating challenges of managing vast sensor-generated data, particularly in audio applications, necessitate innovative solutions. Current systems face significant computational and storage demands, especially in real-time applications like gunshot detection systems (GSDS), and the proliferation of edge sensors exacerbates these issues. This paper proposes a groundbreaking approach with a near-sensor model tailored for intelligent audio-sensing frameworks. Utilizing a Fast Fourier Transform (FFT) module, convolutional neural network (CNN) layers, and HyperDimensional Computing (HDC), our model excels in low-energy, rapid inference, and online learning. It is highly adaptable for efficient ASIC design implementation, offering superior energy efficiency compared to conventional embedded CPUs or GPUs, and is compatible with the trend of shrinking microphone sensor sizes. Comprehensive evaluations at both software and hardware levels underscore the model's efficacy. Software assessments through detailed ROC curve analysis revealed a delicate balance between energy conservation and quality loss, achieving up to 82.1\% energy savings with only 1.39\% quality loss. Hardware evaluations highlight the model's commendable energy efficiency when implemented via ASIC design, especially with the Google Edge TPU, showcasing its superiority over prevalent embedded CPUs and GPUs.


%具身智能体在复杂场景下 manipulation 的 performance robustness 和泛化能力始终是一个广受关注的研究方向。其中,visuomotor imitation learning 是具身智能体 Policy 的主流范式之一,它允许 agent 从高维视觉观察和机器人本体感知中 effectively 学习 manipulation skills。
%然而,增加场景的复杂度和 visual distraction,会导致在简单场景下表现良好的决策模型性能下降。实际上,不仅是 simple imitation learning policy,先进的多模态 foundation models such as GPT-4o 或 vision language action models (VLA),也不能很好地关注一张语义丰富的图片中的特定的局部问题。对于 robot control or 多模态大模型,其往往侧重于 action prediction, observation mapping or 多模态 alignment,而缺少直观的视觉感知增强。模型需要隐性地或遵循 high-level text instruction 从相关的视觉区域中获得面向任务语义的定位知识。
%To tackle this challenge problem, we introduce Imit Diff, a diffusion transformer imitation learning framework with dual resolution enhancement guided by fine-grained semantics information。具体来说,our work 有三个关键组成部分。
%1) Semanstic Injection. Imit Diff 通过 vision language models (VLM) 和 vision foundation models 的 pretrain knowledge 将面向任务的语义信息和高层文本指导转化为显式的 pixel-level 视觉定位标签,注入到 environment observation中。
%2) Dual Res Fusion。 我们构建了双分辨率图像观测流,使用双分辨率视觉编码器分别提取全局和细粒度视觉特征。多尺度视觉信息随后在 attention block 中进行融合,在保证计算 effiency 的前提下,为全局视觉观测引入多尺度细粒度信息,提升场景理解能力。
%3) Consistency policy on diffusion transformer。Diffusion based imitation policies 通常受到 denoise times 的困扰。我们建立了基于 consistency policy 的 DiT action head。Policy 的决策层可以通过 single step denoise 实现系统高频响应。额外地,受益于较快的 inference time,我们引入 temperal ensemble 改善预测动作的平滑性。
%我们设计了四个在 manipulation 精细度上具有挑战性的现实世界任务来评估 Imit Diff,并通过增加场景复杂度和 visual distraction 来测试模型的场景理解能力。额外地,我们设计了 visual distraction 和 category generalization 的 zero shot 实验来验证模型是否受益于 dual res enhancement framework and fine-grained semantics injection。实验结果表明,Imit Diff outperforms 现有的 strong baselines。
%In summary, the contributions of our work are three-fold:
%1) We propose Imit Diff, a DiT architecture imitation learning framework with dual res enhancement guied by fine-grained semantics information.
%2) 我们构建了 open-set vision foundation models pipeline 来获得显式视觉遮罩。该方法能够有效处理机器人控制场景的运动模糊、遮挡、物体丢失情况。并将其作为 fine-grained 语义信息引导 policy decision。
%3) 我们在DiT上实现了consistency policy,显著减少了模型推理时间。通过异步控制框架,实现了 open-set vision foundation models 工作流下的实时控制。
%The code will be publicly available soon。

\section{Introduction}


\label{Intro}
The performance robustness and generalization capabilities of embodied agents in complex manipulation scenarios have long been a focus of significant research interest \citep{ju2025robo, yuan2024learning}. Visuomotor imitation learning is one of the mainstream paradigms of robot manipulation policy \citep{chi2023diffusion, shridhar2023perceiver, ze2023gnfactor, florence2022implicit, hansen2022pre}. This approach enables agents to derive state estimation and decision-making capabilities from expert demonstrations that incorporate high-dimensional visual observations and robot proprioception \citep{ze20243d}.

However, as scene complexity and visual distractions increase, the performance of decision models that excel in simpler environments tends to degrade \citep{zheng2024instruction, liurobustness}. Not only do simple imitation learning policies face challenges, but even advanced multimodal foundation models, such as GPT-4o \citep{hurst2024gpt} or vision language action models (VLA) \citep{liu2024rdt, brohan2022rt, brohan2023rt, o2023open, kim2024openvla, wen2024diffusion}, struggle to accurately focus on specific details within semantically complex images. In fact, in robot control and embodied multimodal foundation models, the focus is often on action prediction, observation mapping, or multimodal alignment. Therefore, intuitive visual perception enhancement is typically lacking. Models can only acquire task-oriented semantic localization knowledge from relevant visual regions either implicitly or when guided by high-level text instructions \citep{reuss2023multimodal}.

To tackle this challenge problem, we introduce \textbf{Imit Diff}, a diffusion transformer imitation learning framework with dual resolution enhancement guided by fine-grained semantics information. Specifically, our work has three key components:

\begin{enumerate}

\item \textbf{Semanstic injection.} Imit Diff transforms task-oriented semantic information and high-level textual guidance into explicit pixel-level visual localization labels through the pretrain knowledge of vision language models (VLM) and vision foundation models, and injects them into the policy observation.

\item \textbf{Dual resolution (dual res) fusion.} We develop a dual res image observation stream and employed a dual res vision encoder to extract global and fine-grained visual features. The extracted multi-scale visual information is subsequently fused within an attention block, integrating fine-grained details into the global visual feature. This approach enhances scene understanding while maintaining computational efficiency.

\item \textbf{Consistency policy on diffusion transformer (DiT).} Diffusion-based imitation policies often suffer from inefficiencies due to the required denoising steps. To address this, we design a DiT \citep{peebles2023scalable} action head incorporating a consistency policy \citep{song2023consistency}, enabling the decision layer to achieve high-frequency system responses through single-step denoising. Furthermore, leveraging faster inference times, we introduce temperal ensemble to enhance the smoothness of predicted actions.

\end{enumerate}

We design four real-world tasks with challenging manipulation precision to evaluate Imit Diff and test the model's scene understanding capabilities by introducing increased scene complexity and visual distractions. Additionally, we conducted zero-shot experiments on visual distraction and category generalization to assess the benefits of the dual res enhancement framework and fine-grained semantic injection. Experimental results demonstrate that Imit Diff significantly outperforms existing strong baselines. 

In summary, the contributions of our work are three-fold:

\begin{enumerate}

\item We propose Imit Diff, a DiT architecture imitation learning framework with dual res enhancement guied by fine-grained semantics information.

\item We developed an open-set vision foundation model pipeline to generate explicit visual masks. This approach effectively addresses challenges such as motion blur, occlusion, and object loss in robot control scenarios, leveraging the generated masks as fine-grained semantic information to guide policy decisions.

\item We implemented a consistency policy on DiT, which significantly reduced the model inference time. Through the asynchronous control framework, we achieved real-time control under the workflow of open-set vision foundation models.

\end{enumerate}

The code will be made publicly available soon.


\section{Related Work}
\label{Sec:related_work}

\textbf{Automatic app maturity ratings}: The evaluation of mobile apps often involves various perspectives. In particular, identifying mobile app development is consistent with what is stated in the privacy policy concerning online advertising and tracking ~\cite{nguyen2022freely, nguyen2021measuring}, aiding developers in crafting child-friendly apps concerning both content and privacy aspects~\cite{hu2015protectingcikm, liccardi2014can}. However, fewer studies aimed at mobile app maturity rating. Therefore, there is growing concern regarding inappropriate content and maturity ratings in mobile apps, which are linked to privacy concerns. Early work by Chen et al.~\cite{chen2013isthisapp} proposed Automatic Label of Maturity ratings (ALM), a text-mining-based semi-supervised algorithm that uses app descriptions and user reviews to determine maturity ratings. The authors used the content rating from Apple App Store as the reference standard for a given app. However, this method uses keyword matching while ignoring semantic analysis. Using a similar approach for ground truth establishment, Hu et al.~\cite{hu2015protectingcikm} proposed a text feature-based SVM classifier for content rating prediction with an online training element. The previous two methods solely depend on text features despite having access to other modalities. Liu et al.~\cite{liu2016identifying} and Chenyu et al.~\cite{zhou2022automatic} extended these works by incorporating image and APK features to identify children’s apps. However, features were limited to extracting text using OCR software, colour distribution of the icon and screenshots, and permissions and APIs. More recently, Sun et al.~\cite{sun2023not} identified discrepancies in content ratings of the same app in different geographic regions by defining rating system mappings between geographical regions. However, this research focuses on single modalities or multiple modalities but treats them independently. \\ 
% \vspace{-3mm}

\noindent\textbf{Vision-Language (VL) models}:  Early image-based contrastive representations have made advancements, nearly achieving the performance levels seen in supervised baselines across various downstream tasks such as image classification and retrieval~\cite{chen2020simple, zbontar2021barlow}. Driven by the success of contrastive learning in intra-modal tasks, there has been a growing interest in developing multi-modal objectives (e.g., Vision-Language), enabling the model to comprehend and exploit cross-modal associations.
Pioneering works such as CLIP~\cite{clip} and ALIGN~\cite{align} bridged the gap between the vision and language modalities by learning language and vision encoders jointly with a symmetric cross-entropy loss which is an adaptation of InfoNCE loss~\cite{oord2018representation} for cross-model pairs. CLIP optimises the cosine similarity between text and image embeddings, while ALIGN employs a similar contrastive learning setting with noisy training data. Zhai et al.~\cite{LiT} tuned the text encoder using image-text pairs while keeping the image encoder frozen. The rich embeddings that these methods learn are later adapted to various application domains such as video-text retrieval~\cite{fang2021clip2video, portillo2021straightforward}, image generation~\cite{nichol2021glide}, and visual assistance~\cite{massiceti2023explaining}. 
However, \cite{agarwal2021evaluating, luccioni2024stable} point out the challenges in adapting Large Multi-modal Models (LMMs) for different domains when the downstream task deviates from the originally pre-trained task. To the best of our understanding, ours is the first work to leverage the advances in VL-language models to detect content compliance malpractices specific to mobile apps. 






\section{\methodname{}: Automatic Functionality Annotation Pipeline}
\label{sec: annotation pipeline}
This section introduces \methodname{}, an annotation pipeline (Fig.~\ref{fig: anno pipeline}) that automatically produces contextual element functionality annotations used to enhance VLMs' GUI grounding capabilities.


\begin{table}[t]
\tiny
\centering
\caption{\textbf{Comparing our \methodname{} dataset with existing large-scale UI datasets.} Multi-Res means the samples are collected on devices with various resolutions. Auto Anno. means the samples are collected autonomously. \#Anno. means the number of annotated samples provided by the datasets.}
\label{tab:data comparison}
\begin{tabular}{@{}cccccccc@{}}
\toprule
Dataset & UI Type & \begin{tabular}[c]{@{}c@{}}Multi\\ Res.\end{tabular} & \begin{tabular}[c]{@{}c@{}}Real-world\\ Scenario\end{tabular} & \begin{tabular}[c]{@{}c@{}}Auto\\ Anno. \end{tabular} & \begin{tabular}[c]{@{}c@{}}Contextual\\ Functionality\\ Semantics\end{tabular} & \#Anno. & Task \\ \midrule
WebShop~\citep{yao2022webshop} & Web & \cross & \cross & \cross & \cross & 12k & Web Navigation \\
Mind2Web~\citep{deng2024mind2web} & Web & \cross & \cmark & \cross & \cross & 2.4k & Web Navigation \\
WebArena~\citep{zhou2023webarena} & Web & \cross & \cmark & \cross & \cross & 812 & Web Navigation \\
\midrule
S2W~\citep{Wang2021Screen2WordsAM} & Mobile & \cross & \cmark & \cross & \cross & 112k & Screen Summarization \\
Wid. Cap.~\citep{Li2020WidgetCG} & Mobile & \cross & \cmark & \cross & \cross & 163k & Element Captioning \\
PixelHelp~\citep{Li2020MappingNL} & Mobile & \cross & \cmark & \cross & \cross & 187 & Element Grounding \\
RICOSCA~\citep{Li2020MappingNL} & Mobile & \cross & \cmark & \cross & \cross & 295k & Action Grounding \\
MoTIF~\citep{Burns2022ADF} & Mobile & \cross & \cmark & \cross & \cross & 6k & Mobile Navigation \\
AITW~\citep{rawles2023android} & Mobile & \cross & \cmark & \cross & \cross & 715k & Mobile Navigation \\
RefExp~\citep{Bai2021UIBertLG} & Mobile & \cross & \cmark & \cross & \cross & 20.8k & Element Grounding \\
VWB~\citep{liu2024visualwebbench} & Web & \cross & \cmark & \cross & \cross & 1.5k & Elem. Ground \& Ref. \\
SeeClick Web~\citep{cheng2024seeclick} & Web & \cross & \cmark & \cmark & \cross & 271k & Element Grounding \\
UI REC/REG~\citep{hong2023cogagent} & Web & \cmark & \cmark & \cmark & \cross & 400k & Box2DOM, DOM2Box \\
Ferret-UI~\citep{you2024ferretui} & Mobile & \cmark & \cmark & \cmark & \cross & 250k & Elem. Ground \& Ref. \\
\methodname{} (ours) & Web, Mobile & \cmark & \cmark & \cmark & \cmark & 704k & Functionality Ground \& Ref. \\ \bottomrule
\end{tabular}
\end{table}



\begin{figure}[t]
    \centering
    \includegraphics[width=0.95\linewidth]{figure/AnnoPipeline3.pdf}
    \caption{\textbf{The proposed pipeline for automatic UI functionality annotation.} An LLM is utilized to predict element functionality based on the UI content changes observed during the interaction. LLM-aided rejection and verification are introduced to improve data quality. Finally, the high-quality functionality annotations will be converted to instruction-following data by applying task templates.}
    \label{fig: anno pipeline}
\end{figure}


\subsection{Collecting UI Interaction Trajectories}
Our pipeline initiates by collecting interaction trajectories, which are sequences of UI contents captured by interacting with UI elements. Each trajectory step captures all interactable elements and the accessibility tree (AXTree) that briefly outlines the UI structure, which will be used to generate functionality annotations. To amass these trajectories, we utilize the latest Common Crawl repository as the data source for web UIs and Android Emulator for mobile UIs. Note that illegal websites and Apps are excluded manually from the sources to ensure no pornographic or violent content is included in our dataset. Please refer to Sec.~\ref{sec:supp:record traj detail} for collecting details and data license.

\subsection{Functionality Annotation Based on UI Dynamics}
Subsequently, the pipeline generates functionality annotations for elements in the collected trajectories. Interacting with an element $e$, by clicking or hovering over it, triggers content changes in the UI. In turn, these changes can be used to predict the functionality $f$ of the interacted element. For instance, if clicking an element causes new buttons to appear in a column, we can predict that the element likely functions as a dropdown menu activator (an example in Fig.~\ref{fig: funcpred diff case}).
With this observation, we utilize a capable LLM (i.e., Llama-3-70B~\citep{llama3modelcard}) as a surrogate for humans to summarize an element's functionality based on the UI content changes resulting from interaction. Concretely, we generate compact content differences for AXTrees before ($s_t$) and after ($s_{t+1}$) the interaction using a file-comparing library\footnote{https://docs.python.org/3/library/difflib.html}. Then, we prompt the LLM to thoroughly analyze the UI content changes (addition, deletion, and unchanged lines), present a detailed Chain-of-Thoughts~\citep{wei2022chain} reasoning process explaining how the element affects the UI, and finally summarize the element's functionality.

In cases where element interactions significantly transform the UI and cause lengthy differences—such as navigating to a new screen—we adjust our approach by using UI description changes instead of the AXTree differences. Specifically, we prompt the same LLM to discern the UI hierarchy, describe UI regions, and finally describe the entire UI functionality. After describing the UIs before and after the interaction, the LLM analyzes the description differences, presents reasoning, and summarizes the element's functionality. This annotation process is formulated as:
\begin{equation}
    f = \text{LLM}(p_{\text{anno}}, s_t, s_{t+1})
\end{equation}

where $f$ is the predicted functionality, $p_{\text{anno}}$ is the annotation prompt (Tab.~\ref{tab:supp:funcpred manip prompt} and Tab.~\ref{tab:supp:funcpred nav prompt}). Examples of annotated elements are depicted in Fig.~\ref{fig: our dataset} and more annotation details are explained in Sec.~\ref{sec:supp:anno details}.

\subsection{Removing Invalid Samples via LLM-Aided Rejection}
The collected trajectories may contain invalid samples due to broken UIs, such as incomplete UI loading. These samples are meaningless as they contain corrupted UI content and can mislead the models trained with them.

To filter out these invalid samples, we introduce an LLM-aided rejection approach. Initially, hand-written rules are used to detect obvious broken cases, such as blank UI contents, UIs containing elements indicating content loading, and interaction targets outside of UIs. While these obvious cases constitute a large portion of the invalid samples, there are a few types that are difficult to detect with hand-written rules. For instance, interacting with a “view more” button might unexpectedly redirect the user to a login page instead of the desired information page due to website login restrictions. To identify these challenging samples, we prompt the annotating LLM to also act as a rejector. Specifically, the LLM takes the UI content changes, generated using a file-comparing library, as input, provides detailed reasoning on whether the changes are meaningful for predicting the element's functionality, and finally outputs predictability scores ranging from 0 to 3. This process is formulated as follows:
\begin{equation}
 score = \text{LLM}(p_{\text{reject}}, e, s_t, s_{t+1})
\end{equation}
where $p_{\text{reject}}$ is the rejection prompt (Tab.~\ref{tab:supp:rejection prompt}).

This approach ensures that clear and predictable samples receive higher scores, while those that are ambiguous or unpredictable receive lower scores. For instance, if a button labeled "Show More", upon interaction, clearly adds new content, this sample will considered to provide sufficient changes that can anticipate the content expansion functionality and will get a score of 3. Conversely, if clicking on a "View Profile" link fails to display the profile possibly due to web browser issues, this unpredictable sample will get a score less than 3.

After implementing empirical experiments, we deploy this LLM-based rejector to discard the bottom 30\% of samples based on their scores to strike a balance between the elimination of invalid samples and the preservation of valid ones (More details in Sec.~\ref{suc:supp:reject details}). The samples that pass the hand-written rules and the LLM rejector are subsequently submitted for functionality annotation. Please see representative rejection examples in Fig.~\ref{fig: rejection examples}.

\subsection{Improving Annotation Quality via LLM-Based Verification}
The functionality annotations produced by the LLM probably contain incorrect, ambiguous, and hallucinated samples (See a case in Fig.~\ref{fig: anno pipeline}), which probably misleads the trained VLMs and compromises evaluation accuracy. To improve dataset quality, we prompt LLMs to verify the annotations by checking whether the targeted element $e$ fulfills the intent of the annotated functionality $f$. This process presents the LLMs with the interacted element, its UI context, the UI changes induced by this element, and the functionality generated in the previous annotation process. The LLMs are then tasked with analyzing the UI content changes before predicting whether the interacted element aligns with the given functionality. If the LLMs determine that the interacted element fulfills the functionality given its UI context, the LLMs will grant a full score (An example in Fig.~\ref{fig: verif diff case}). If the interacted element is considered to mismatch the functionality, this functionality can be seen as incorrect as this mismatch indicates that it may not accurately reflect the element's actual role within the UI context.

To mitigate the potential biases in LLMs~\citep{panickssery2024llm, zheng2023judging, bai2024benchmarking}, two different LLMs (i.e., Llama-3-70B~\citep{llama3modelcard} and Mistral-7B-Instruct-v0.2~\citep{mistral}) are employed as verifiers and prompted to output 0-3 scores. The scoring process is formulated as follows:
\begin{equation}
 score = \text{LLM}(p_{\text{verify}}, e, f, s_t, s_{t+1})
\end{equation}
where $p_{\text{verify}}$ denotes the verification prompt (Tab.~\ref{tab:supp:verif prompt}). Only if the two scores are both 3s do we consider the functionality label correct (More details in Sec.~\ref{suc:supp:verif details}). Although this filtering approach seems stringent, we can make up the number of annotations through scaling. 

\begin{figure}[t]
    \centering
    \includegraphics[width=0.9\linewidth]{figure/our_dataset_img.pdf}
    \caption{Element functionality annotations generated by the proposed AutoGUI pipeline for both web and mobile viewpoints.}
    \label{fig: our dataset}
    \vspace{-5mm}
\end{figure}

\subsection{Functionality Grounding and Referring Task Generation}
\vspace{-2mm}
After rejecting, annotating, and verifying, we obtain a high-quality UI functionality dataset containing triplets of \{UI screenshot, Interacted element, Functionality\}. To convert this dataset into an instruction-following dataset for training and evaluation, we generate functionality grounding and referring tasks using diverse prompt templates (see Tab.~\ref{tab:task templates}). To mitigate the difficulty of predicting absolute values for various resolutions, the coordinates of element bounding boxes are all normalized within the range $[0,99]$ (see Fig.~\ref{fig: our dataset} for examples).

\subsection{Explore the \methodname{} Dataset}

\begin{table}[]
\centering
\small
\caption{\textbf{The statistics of the AutoGUI datasets.} The Anno. Tokens and Avg. Words columns show the total number of tokens and the average number of words for the functionality annotations regardless of task templates. The Domains/Apps column shows the number of unique web domains/mobile Apps involved in each split.}
\label{tab:simple data stats}
\begin{tabular}{@{}ccccccc@{}}
\toprule
Split & \#Tasks & Anno. Tokens & Avg. Words & Domains/Apps & Device Ratio   \\                                                                   \midrule
Train & 702k  & 17.9M        & 23.1       & 916     & Web: $54.6\%$, Mobile: $45.4\%$                                              \\ \cmidrule(r){1-6}
Test  & 2k    & 53.4k        & 22.5       & 299     & Web: $50\%$, Mobile: $50\%$                                                                                                               \\ \bottomrule
\end{tabular}
\end{table}

\begin{figure}[t]
    \centering
    \includegraphics[width=1.0\linewidth]{figure/wordcloud_token-dist-comparison.pdf}
    \caption{\textbf{Diversity of the AutoGUI dataset.} \textbf{Left}: The word cloud illustrates the ratios of the verbs representing the main intents in the functionality annotations. \textbf{Right}: Comparing the distributions of the annotation token numbers for our AutoGUI training split, SeeClick Web training data~\citep{cheng2024seeclick}, and Widget Captioning~\citep{Li2020WidgetCG}. The comparison demonstrates that our dataset covers significantly more diverse task lengths.}
    \label{fig: wordcloud and tokdistrib}
\end{figure}
\vspace{-2mm}

The \methodname{} pipeline finally collects 22.4k trajectories, from which we select 2k grounding samples (evenly divided between web and smartphone views) as the test set and remove the trajectories to which these samples belong. Subsequently, 702k samples are randomly selected from the remaining instances to constitute the training set. The statistics of our dataset in Tab.~\ref{tab:simple data stats} and Sec.~\ref{sec:supp:data stats} show that our dataset covers diverse UIs and exhibits variety in lengths and functional semantics of the annotations. Moreover, our dataset presents a unique ensemble of research challenges for developing generalist web agents in real-world settings. As shown in Tab.~\ref{tab:data comparison} and Fig.~\ref{fig: functionality vs others}, our dataset distinguishes itself from existing literature by providing functionality-rich data as well as tasks that require VLMs to discern the contextual functionalities of elements to achieve high grounding accuracy.

\section{Analysis of Data Quality}
This section analyzes the reliability of the proposed annotation pipeline and data quality.

\noindent{\textbf{Comparison with Human Annotation}} To demonstrate the superiority of the proposed automatic annotation pipeline based on open-source LLMs, $N=145$ samples (99 valid and 46 invalid) are randomly selected as a testbed for comparing the annotation correctness of a trained human annotator and the pipeline. Here, correctness is defined as $Correctness = C / (N - R)$, where $C$ and $R$ denote the numbers of correctly annotated and rejected samples, respectively. The denominator subtracts the number of rejected samples as we are more interested in the percentage of correct samples after rejecting the samples considered invalid by the annotator. The authors thoroughly check the annotation results according to the three criteria in Fig.~\ref{fig: check criteria}: 1. Context-specificity. The functionality annotations must include context-specific descriptions to ensure one-to-one mapping between the element and its annotation. 2. Appropriate details. Avoid detailing unnecessary aspects of the UIs to keep the description focused on functionality. 3. No hallucination. The annotations must not include information not grounded in the visual context of the UIs. See more details in Sec.~\ref{sec:supp:humaneval details}.

After experimenting with three runs, Tab.~\ref{tab:ablate autogui} shows that the proposed AutoGUI pipeline achieves high correctness comparable to the trained human annotator (r6 vs. r1). Without rejection and verification (r2), AutoGUI is inferior as it cannot recognize invalid samples. Notably, simply using the rules written by the authors can improve the correctness, which is further enhanced with the LLM-aided rejector (r4 vs. r3). Moreover, utilizing the annotating LLM itself to self-verify its annotations helps AutoGUI surpass the trained annotator (r5 vs. r1). Introducing another LLM verifier (i.e., Mistral-7B-Instruct-v0.2) brings a slight increase which results from Mistral recognizing Llama-3-70B’s incorrect descriptions of how dropdown menu options work. Overall, these results justify the efficacy of the AutoGUI annotation pipeline.

Qualitatively comparing the annotation patterns of the human and AutoGUI (Fig.~\ref{fig: autogui vs human}), we find that AutoGUI employs the strong LLM to generate more detailed and clear annotations which would take significantly more time for the human annotator. This result suggests that the AutoGUI pipeline can lessen the burden of collecting data for training UI-VLMs.

\noindent{\textbf{Impact of LLM Output Uncertainty}} The uncertainty of LLM outputs manifests in annotation, rejection, and verification, possibly impacting the quality of the AutoGUI dataset. To evaluate this impact, we first sample 100 valid samples to test the AutoGUI pipeline for three runs. The consistency rate is 94.5\%, indicating that 94.5\% of the samples possess consistent annotation outcomes (i.e. correct or incorrect) across the runs. We also test the LLM-aided rejector with 46 invalid samples and find that the rejection consistency over three runs is 79.3\%. This indicates that LLM uncertainty impacts this rejection process. Nevertheless, this impact is minor due to the low prevalence of invalid samples (4\% of all samples) that fail the hand-written rules.

In summary, AutoGUI exhibits annotation correctness comparable to that of human annotators and LLM output uncertainty poses a minor impact on the AutoGUI annotation process.



\begin{figure}[t]
    \centering
    \includegraphics[width=0.85\linewidth]{figure/check_criteria_img.pdf}
    \caption{The checking criteria used for comparing AutoGUI pipeline and the human annotator.}
    \label{fig: check criteria}
\end{figure}


\begin{table}[]
\small
\centering
\caption{\textbf{Comparing the AutoGUI and human annotator.} AutoGUI with the proposed rejection and verification achieves annotation correctness comparable to trained human annotators. One LLM means Llama-3-70B and Two LLMs include Mistral-7B-Instruct-v0.2 as well.}
\label{tab:ablate autogui}
\begin{tabular}{@{}ccccc@{}}
\toprule
No. & Annotator  & Rejector   & Verifier              & Correctness \\ \midrule
r1 & Human      & -          & -                     & 95.5\%      \\
r2 & Llama-3-70B & -          & -                     & 64.5\%      \\
r3 & Llama-3-70B & Rules      & -                     & 83.1\%      \\
r4 & Llama-3-70B & Rules+LLM  & -                     & 94.4\%      \\
r5 & Llama-3-70B & Rules+LLM  & One LLM            & 96.0\%      \\
r6 & Llama-3-70B & Rules+LLM & Two LLMs & \textbf{96.7\%}      \\ \bottomrule
\end{tabular}
\end{table}
\vspace{-2mm}





% \section{Experiments}

\section{Analysis}

\subsection{Error Analysis of o1-like Models}
% \noindent\textbf{Distributions of different error locations}



\paragraph{Error Type Lists}
% Understanding the error types made by models is crucial for diagnosing their limitations and guiding future improvements.
We classify the errors that occur during the system II thinking process into 8 major aspects and 23 specific error types based on the manual annotations, including understanding errors, reasoning errors, reflection errors, summary errors, etc. For detailed information about the error categories, see Appendix \ref{app: error_classification}.

\paragraph{What Are the Most Common Errors Across Domains?}

\begin{figure}[t]
    \centering
    \resizebox{1.0\textwidth}{!}
    {\includegraphics{figures/error_type_distribution.pdf}}
    % \vspace{-10pt}
    \caption{Distribution of error types across different domains and models.}
    % \vspace{-3mm}
    \label{fig: error_type}
\end{figure}

To analyze the characteristics of error distribution in different domains, we performed a uniform sampling of the data based on the model, the domain, and the query difficulty. Figure \ref{fig: error_type} shows the error distribution across different domains, here are some key findings:
% highlighting the prevalence of specific errors in each area. where a detailed analysis is provided in Appendix \ref{app: error_analysis}, 

\begin{itemize}[left=1em]
\item \textbf{Math:} The most frequent error type is \textit{Reasoning Error}(25.3\%), followed by \textit{Understanding Error}(15.7\%) and \textit{Calculation Error}(15.4\%). This indicates that while the models often struggle with logical reasoning and problem understanding, low-level computational mistakes also remain a significant issue.

\item \textbf{Programming}: 
\textit{Reasoning Error} (21.5\%) is the most common, followed by \textit{Formal Error} (16.7\%) and \textit{Understanding Error} (12.6\%). The high frequency of \textit{Formal Error} and \textit{Programming Error} (11.8\%) underscores the models' struggles with code-specific details and implementation. 

\item \textbf{PCB}: 
The dominant error types are \textit{Understanding Error} (20.4\%) and \textit{Knowledge Error} (17.3\%), closely followed by \textit{Reasoning Error} (17.3\%). This suggests that the main challenge for current models in the fields of physics, chemistry and biology is to understand field-specific concepts and accurately apply relevant knowledge.

\item \textbf{General Reasoning}: \textit{Reasoning Error} is the most prevalent, accounting for 43\%, followed by comprehension errors, accounting for 19\%, showing that logical reasoning is the primary bottleneck.

\end{itemize}

\paragraph{What Are the Model-Specific Error Patterns?}

% \begin{figure}[t]
%     \centering
%     \includegraphics[width=0.8\textwidth]{figures/error_type_model.pdf}
%     % \vspace{-3mm}
%     \caption{Distribution of Error Types Across Models.}
%     % \vspace{-3mm}
%     \label{fig: error_type_model}
% \end{figure}

We also analyzed errors specific to individual models, providing further insights into model weaknesses, as illustrated in Figure \ref{fig: error_type_model}. The error distributions reveal distinct patterns for each model, highlighting their unique strengths and areas for improvement. Here are some key findings:
%Due to space constraints, we focus here on the key findings from the most commonly used models, with a comprehensive analysis of all models provided in Appendix \ref{app: error_analysis}.

\begin{itemize}[leftmargin=4mm]

\item \textbf{DeepSeek-R1} exhibits its most pronounced weakness in \textit{Reasoning Errors} (22.7\%), indicating challenges in constructing coherent and accurate logical reasoning paths. However, it demonstrates relative strength in handling fundamental tasks, with minimal \textit{Calculation Errors} (3.1\%) and \textit{Programming Errors} (4.4\%).

%achieves strong performance in detail-oriented tasks such as formula computation and code syntax. Its primary limitation lies in reasoning and comprehension capabilities.

\item \textbf{QwQ-32B-Preview} excels at identifying correct problem-solving approaches. However, its effectiveness is significantly hindered by deficiencies in handling finer details, particularly in \textit{Calculation Errors} (17.9\%)

%but its effectiveness is often undermined by deficiencies in handling finer details.

% {QwQ-32B-Preview} demonstrates a relatively balanced performance but is notably weak in \textit{Calculation Errors} (17.9\%), indicating a significant limitation in numerical precision. It also shows a moderate frequency of \textit{Understanding Errors} (17.1\%), suggesting occasional difficulties in problem interpretation. 

\end{itemize}

\begin{tcolorbox}[colback=white!95!gray, colframe=gray!70!black,  title=Key Finding for Error Type]
The primary bottleneck of current models remains reasoning ability. However, detailed errors like calculation and formal mistakes also contribute significantly.
\end{tcolorbox}


\subsection{Reflection Analysis of o1-like Models}


\begin{figure}[t]
    \centering
    \includegraphics[width=0.95\textwidth]{figures/reflection.pdf}
    \caption{Distribution of effective reflection times by models and domains on a sample level. The segments within each pie chart represent how many times effective reflection occurs in one sample, with segment `0' indicating there is no effective reflection.}
    \label{fig: error_type_model}
\end{figure}

\paragraph{Statistics.}
We also conduct a analysis of the total number of reflections and the proportion of effective reflections in the long CoT output of all questions (including questions answered correctly and incorrectly by the model). 
% On average, 
%We observe that the long CoT contains \textit{five} times reflections, indicating that current o1-like models tend to reflect frequently. 

\paragraph{How Effective Are Model Reflections Across Different Models and Domains?}
We classify samples with reflections based on the number of valid reflections to evaluate the ability to produce valid reflections. Specifically, we label samples as \texttt{0} if no valid reflections occur, and \texttt{1}, \texttt{2}, or \texttt{>=3} for samples with one, two, or three and more valid reflections, respectively(all statistical analyses were performed under strictly controlled conditions, ensuring uniform sampling and balanced tasks for a fair comparison). In Figure \ref{fig: error_type_model}, {DeepSeek-R1} exhibits the highest proportion of effective reflections, and the models show a notably higher rate of effective reflections in the {math} domain. However, the overall proportion of valid reflections across all models remains relatively low, ranging between 30\% and 40\%. This suggests that the reflection capabilities of current models require further improvement.
%Detailed statistical data can be found in Appendix D.

\begin{tcolorbox}[colback=white!95!gray, colframe=gray!70!black,  title=Key Finding for Reflection]
Despite frequent reflection attempts, the proportion of effective reflections remains low across models, and  DeepSeek-R1 achieves the highest rate of valid reflections.
\end{tcolorbox}

\subsection{Effective Reasoning of o1-like Models}

\begin{figure}[t]
    \centering
    \includegraphics[width=0.98\textwidth]{figures/effetive_reasoning.pdf}
    \caption{Distribution of effective reasoning ratios.}
    
    \label{fig: effetive_reasoning}
\end{figure}

\paragraph{Statistics.} 
% As previously mentioned, 
Human annotators evaluate the usefulness of the reasoning in each section, enabling us to calculate the proportion of valid reasoning in each response. As illustrated in Figure \ref{fig: effetive_reasoning}, each graph shows the distribution of effective reasoning ratios for a particular model. The red dashed line in each graph indicates the average effective reasoning ratio.

\paragraph{What Proportion of Reasoning in Long CoT Responses is Effective?}
On average, only 73\% of the reasoning in the collected long CoT responses is useful, highlighting significant redundancy issues. Among the models analyzed, \textit{QwQ-32B-Preview} exhibited the lowest proportion of effective reasoning at 70\%, while \textit{DeepSeek-R1} achieved a notably higher proportion compared to the others, demonstrating superior reasoning efficiency.


\begin{tcolorbox}[colback=white!95!gray, colframe=gray!70!black,  title=Key Finding for Reasoning Efficiency]
On average, 27\% of reasoning in long CoT responses we collected is redundant, and DeepSeek-R1 outperforms others in reasoning efficiency.
\end{tcolorbox}
\vspace{-3mm}

\subsection{Reasoning Process Analysis}

Figure ~\ref{fig: action_roles} shows the distribution of each section's action roles in the system II thinking process of the o1-like models. Initially, problem analysis dominates, indicating that the model initially focuses on understanding the requirements and constraints of the problem. As the solution progresses, cognitive activities diversify significantly, with reflection and validation becoming more prominent. In the later part of the reasoning, the distribution of conclusion and summarization gradually increases. 
%As the model progresses from problem analysis, solution implementation and conclusion, it demonstrates the common reasoning template of o1-like models.


\begin{figure}[t]
    \centering
    \includegraphics[width=0.8\textwidth]{figures/action_role.pdf}
    \caption{Distribution of different task types throughout the progress of a long CoT response.}
    \vspace{-3mm}
    
    \label{fig: action_roles}
\end{figure}
\subsection{Results on DeltaBench}

% Please add the following required packages to your document preamble:
% \usepackage{multirow}
\begin{table*}[!t]
\centering
\resizebox{1.0\textwidth}{!}{%
    \begin{tabular}{cccccccccccccccc}
    \toprule
    \multirow{2}{*}{\textbf{Model}} & \multicolumn{3}{c}{\textbf{Overall}} & \textbf{Math} & \textbf{Code} & \textbf{PCB} & \textbf{General} \\
    \cmidrule(lr){2-4} \cmidrule(lr){5-5} \cmidrule(lr){6-6} \cmidrule(lr){7-7} \cmidrule(lr){8-8}
     & \textbf{\textit{Recall}} & \textbf{\textit{Precision}} & \textbf{\textit{F1}} & \textbf{\textit{F1}} & \textbf{\textit{F1}} & \textbf{\textit{F1}} & \textbf{\textit{F1}} \\
    \midrule
    \multicolumn{8}{c}{\textbf{\textit{Process Reward Models (PRMs)}}} \\
    \midrule
    \rowcolor[rgb]{ .988,  .949,  .8} Qwen2.5-Math-PRM-7B & \textbf{30.30} & \textbf{34.96} & \textbf{29.22}  &  \textbf{29.64} & \textbf{23.76} & \underline{31.09} & \underline{34.19}   \\
    \rowcolor[rgb]{ .988,  .949,  .8} Qwen2.5-Math-PRM-72B & \underline{28.16} & \underline{29.37} & \underline{26.38}  & \underline{24.16} & \underline{22.02} & \textbf{31.14} & \textbf{35.83}  \\
    \rowcolor[rgb]{ .988,  .949,  .8} Llama3.1-8B-PRM-Deepseek-Data & 11.7 & 15.59 & 12.02 &  12.28 & 10.95 & 16.76 & 12.59  \\
    \rowcolor[rgb]{ .988,  .949,  .8} Llama3.1-8B-PRM-Mistral-Data & 9.64 & 11.21 & 9.45 & 9.40 & 10.72 & 13.43 & 12.40  \\
    \rowcolor[rgb]{ .988,  .949,  .8} Skywork-o1-Qwen-2.5-1.5B & 3.32 & 3.84 & 3.07 & 1.30 & 6.66 & 5.43 & 7.87  \\
    \rowcolor[rgb]{ .988,  .949,  .8} Skywork-o1-Qwen-2.5-7B & 2.49 & 2.22 & 2.17 & 0.78 & 6.28 & 6.02 & 3.11  \\
    \midrule
     \multicolumn{8}{c}{\textbf{\textit{LLM as Critic Models}}} \\
    \midrule
    \rowcolor[rgb]{ .922,  .89,  .988} GPT-4-turbo-128k & \textbf{57.19} & \textbf{37.35} & \textbf{40.76} & \textbf{37.56} & \textbf{43.06} & \underline{45.54} & \underline{42.17} \\
    \rowcolor[rgb]{ .922,  .89,  .988} GPT-4o-mini & \underline{49.88} & 35.37 & \underline{37.82} & \underline{33.26} & 37.95 & \textbf{45.98} & \textbf{46.39} \\
    \rowcolor[rgb]{ .922,  .89,  .988} Doubao-1.5-Pro & 39.68 & \underline{37.02} & 35.25 & 32.46 & \underline{39.47} & 33.53 & 37.00 \\
    \rowcolor[rgb]{ .922,  .89,  .988} GPT-4o & 36.52 & 32.48 & 30.85 & 28.61 & 28.53 & 39.25 & 36.50 \\
    \rowcolor[rgb]{ .922,  .89,  .988} Qwen2.5-Max & 36.11 & 30.82 & 30.49 & 26.73 & 32.81 & 39.49 & 29.54 \\
    \rowcolor[rgb]{ .922,  .89,  .988} Gemini-1.5-pro & 35.51 & 30.32 & 29.59 & 26.56 & 28.20 & 40.13 & 33.66 \\
    \rowcolor[rgb]{ .922,  .89,  .988} DeepSeek-V3 & 32.33 & 28.13 & 27.33 & 27.04 & 27.73 & 27.35 & 27.45 \\
    \rowcolor[rgb]{ .922,  .89,  .988} Llama-3.1-70B-Instruct & 32.22 & 28.85 & 27.67 & 21.49 & 32.13 & 28.45 & 39.18 \\
    \rowcolor[rgb]{ .922,  .89,  .988} Qwen2.5-32B-Instruct & 30.12 & 28.63 & 26.73 & 22.34 & 31.37 & 33.78 & 24.37 \\
    \rowcolor[rgb]{ .882,  .949,  .89} DeepSeek-R1 & 29.20 & 32.66 & 28.43 & 24.17 & 29.28 & 34.78 & 35.87 \\
    \rowcolor[rgb]{ .882,  .949,  .89} o1-preview & 27.92 & 30.59 & 26.97 & 22.19 & 28.09 & 33.11 & 35.94 \\
    % Gemini-2.0-flash-thinking & 14.02 & 17.36 & 14.56 & 14.79 & 11.97 & 19.34 & 15.26 \\
    \rowcolor[rgb]{ .922,  .89,  .988} Qwen2.5-14B-Instruct & 26.64 & 27.27 & 24.73 & 21.51 & 29.05 & 29.98 & 20.59 \\
    \rowcolor[rgb]{ .922,  .89,  .988} Llama-3.1-8B-Instruct & 25.71 & 28.01 & 24.91 & 18.12 & 32.17 & 27.30 & 29.93 \\
    \rowcolor[rgb]{ .882,  .949,  .89} o1-mini & 22.90 & 22.90 & 19.89 & 16.71 & 21.70 & 20.37 & 26.94 \\
    \rowcolor[rgb]{ .922,  .89,  .988} Qwen2.5-7B-Instruct & 21.99 & 19.61 & 18.63 & 11.61 & 25.92 & 29.85 & 15.18 \\
    \rowcolor[rgb]{ .882,  .949,  .89} DeepSeek-R1-Distill-Qwen-32B & 17.19 & 18.65 & 16.28 & 13.02 & 23.55 & 15.05 & 11.56 \\
    % Gemini-2.0-flash-thinking & 14.02 & 17.36 & 14.56 & 14.79 & 11.97 & 19.34 & 15.26 \\
    \rowcolor[rgb]{ .882,  .949,  .89} DeepSeek-R1-Distill-Qwen-14B & 12.81 & 14.54 & 12.55 & 9.40 & 18.36 & 10.44 & 12.01 \\
    % \rowcolor[rgb]{ .882,  .949,  .89} QwQ-32B-Preview & 10.20 & 10.17 & 9.07 & 7.38 & 8.60 & 14.97 & 10.54 \\
    \bottomrule
    \end{tabular}
}
\caption{Experimental results of PRMs and critic models on DeltaBench. \textbf{Bold} indicates the best results within the same group of models, while \underline{ underline} indicates the second best.}
% \vspace{-4mm}
\label{tab: main}
\end{table*}

% \noindent\textbf{Evaluation Metrics.}
% % To accurately assess the performance of the PRM and critic models on DeltaBench, 
% We employ \textbf{recall}, \textbf{precision}, and \textbf{macro-F1 score} for error sections as evaluation metrics. For the PRMs, we utilize an outlier detection technique based on the Z-Score to make predictions. This method was chosen because threshold-based prediction methods determined from other step-level datasets, such as those used in ProcessBench~\citep{Zheng2024ProcessBenchIP}, may not be reliable due to significant differences in dataset distributions, particularly as DeltaBench focuses on long CoT. Outlier detection helps to avoid this bias. The threshold $t$ for determining the correctness of a section is defined as:
% % \begin{align}
% $t = \mu - \sigma$,
% % \nonumber
% % \label{eq: prm_threshold}
% % \end{align}
% where $\mu$ is the mean of the rewards distribution across the dataset, and $\sigma$ is the standard deviation. Sections falling below $t$ are predicted as error sections. For critic models, all erroneous sections within a long CoT are prompted to be identified. Given that error sections constitute a smaller proportion than correct sections across the dataset, we use macro-F1 to mitigate the potential impact of the imbalance between positive and negative sections. Macro-F1 independently calculates the F1 score for each sample
% % (for our metric, each case) 
% and then takes the average, providing a more balanced evaluation metric when dealing with class imbalance.

\noindent\textbf{Baseline Models.}
% 开源(中英模型,llama3)和闭源模型
% To comprehensively evaluate the performance of current PRMs and critic models, we extensively selected and evaluated a wide range of both open-source and closed-source models on DeltaBench.
% \paragraph{Process Reward Models}
For the \textbf{PRMs}, we select the following models: Qwen2.5-Math-PRM-7B\footnote{\href{https://huggingface.co/Qwen/Qwen2.5-Math-PRM-7B}{Qwen/Qwen2.5-Math-PRM-7B}}, Qwen2.5-Math-PRM-72B\footnote{\href{https://huggingface.co/Qwen/Qwen2.5-Math-PRM-72B}{Qwen/Qwen2.5-Math-PRM-72B}}, Llama3.1-8B-PRM-Deepseek-Data\footnote{\href{https://huggingface.co/RLHFlow/Llama3.1-8B-PRM-Deepseek-Data}{RLHFlow/Llama3.1-8B-PRM-Deepseek-Data}}, Llama3.1 -8B-PRM-Mistral-Data\footnote{\href{https://huggingface.co/RLHFlow/Llama3.1-8B-PRM-Mistral-Data}{RLHFlow/Llama3.1-8B-PRM-Mistral-Data}}, Skywork-o1-Open-PRM- Qwen-2.5-1.5B\footnote{\href{https://huggingface.co/Skywork/Skywork-o1-Open-PRM-Qwen-2.5-1.5B}{Skywork/Skywork-o1-Open-PRM-Qwen-2.5-1.5B}}, and Skywork-o1-Open-PRM-Qwen-2.5-7B\footnote{\href{https://huggingface.co/Skywork/Skywork-o1-Open-PRM-Qwen-2.5-7B}{Skywork/Skywork-o1-Open-PRM-Qwen-2.5-7B}}. 
% These represent some of the best open-source PRMs currently available.
% \paragraph{Critic Models}
We select a group of the most advanced open-source and closed-source LLMs to serve as \textbf{critic models} for evaluation, which includes various GPT-4~\citep{gpt4} variants (such as GPT-4-turbo-128K, GPT-4o-mini, GPT-4o), the Gemini model~\citep{Reid2024Gemini1U}(Gemini-1.5-pro), several Qwen models~\citep{qwen2.5} (such as Qwen2.5-32B-Instruct and Qwen2.5-14B-Instruct), Doubao-1.5-Pro~\citep{doubao2025}
and o1 models~\citep{openai-o1} (o1-preview-0912, o1-mini-0912).
% , and a GPT-3.5 variant (gpt-3.5-16K).



\subsubsection{Main Results}
In Table \ref{tab: main},
we provide the results of different LLMs on DeltaBench. 
For PRMs, we have the following observations: (1). Existing PRMs usually achieve low performance, which indicates that existing PRMs cannot identify the errors in long CoTs effectively and it is necessary to improve the performance of PRMs. (2). Larger PRMs
do not lead to better performance. For example, the Qwen2.5-Math-PRM-72B is inferior to wen2.5-Math-PRM-7B.
For critic models, we have the following findings: (1)
GPT-4-turbo-128k archives the best critique results, which is better than other models (e.g., GPT-4o) a lot in DeltaBench. (2) For o1-like models (e.g., DeepSeek-R1, o1-mini, o1-preview), we observe that the results of these models are not superior to non-o1-like models, with the performance of o1-preview is even lower than Qwen2.5-32B-Instruct.
%Additionally, we observe that the QWQ and DeepSeek-R1-Distill series models exhibit weaknesses in following instructions. 
A detailed analysis of underperforming models is provided in Appendix \ref{app: underperforming}.

% model size
% domains
% o1模型跟普通模型critic能力对比分析


\subsubsection{Further Analysis}

\paragraph{Effect of Long CoT Length.}
\begin{figure}[t]
    \centering
    \includegraphics[width=1.0\textwidth]{figures/4.5.1/length2.pdf}
    \caption{The effect of long CoT length.}
    \label{fig: crtic1}
\end{figure}
In Figure \ref{fig: crtic1}, we compare the average F1-Score performance of critic models and PRMs across varying LongCoT token lengths. 
For critic models, the performance notably declines as token length increases. Initially, models like Deepseek-R1 and GPT-4o exhibit strong performance with shorter sequences (1-3k tokens). However, as token length increases to mid-ranges (4-7k tokens), there is a marked decrease in performance across all models. This trend highlights the growing difficulty for critic models to maintain precision and recall as long CoT response become longer and more complex, likely due to the challenge of evaluating lengthy model outputs. In contrast, PRMs demonstrate greater stability across token lengths, as they evaluate sections sequentially rather than processing the entire output at once. Despite this advantage, PRMs achieve lower overall scores compared to critic models on our evaluation set.

\begin{tcolorbox}[colback=white!95!gray, colframe=gray!70!black, title=Key Finding]
  Critic models exhibit significant performance degradation with longer contexts, while PRMs demonstrate consistent evaluation capability across varying lengths.
\end{tcolorbox}


\paragraph{Performance Analysis Across Different Error Types.}
\begin{figure}[t]
    \centering
    \includegraphics[width=0.9\textwidth]{figures/4.5.2/top_models_per_task.pdf}
    \caption{Results of different LLMs on top-5 errors.}
    \label{fig: top_models_per_task}
\end{figure}
Figure \ref{fig: top_models_per_task} shows the performance of different models on the five most common error types. In terms of error types, most models demonstrate the highest accuracy in recognizing calculation errors. Conversely, the recognition of strategy errors is generally the weakest. In terms of models, there is significant variation in the ability of individual models to recognize different error types. For instance, DeepSeek-V3 achieves an F1 of 36\% on calculation errors but only 23\% on strategy errors. Meanwhile, Llama3.1-8B-PRM-Deepseek performs poorly, with an F1 score of 22\% on calculation errors, and shows a significant decline in performance across the other four error types. This highlights the limited generalization capabilities of most models when recognizing various error types.

\begin{tcolorbox}[colback=white!95!gray, colframe=gray!70!black, title=Key Finding]
  Models exhibit strong performance on calculation errors but struggle with strategy errors, revealing limited generalization across error types.
\end{tcolorbox}

\begin{table}[!ht]
    \centering
    % \scriptsize
    % \footnotesize
    \begin{tabular}{cccc}
    \toprule
        \multirow{2}{*}{Model} & \multicolumn{3}{c}{HitRate@$k$ - Avg(\%)} \\ \cline{2-4}
                           & $k=1$ & $k=3$ & $k=5$ \\ 
                           % \hline
                           \midrule
        Qwen2.5-Math-PRM-7B & \textbf{49.15} & \textbf{69.14} & \textbf{83.14} \\
        Qwen2.5-Math-PRM-72B & \underline{41.13} & \underline{62.70} & \underline{75.73} \\ 
        Llama3.1-8B-PRM-Deepseek-Data & 12.63 & 48.62 & 69.78 \\
        Llama3.1-8B-PRM-Mistral-Data & 8.99 & 42.97 & 65.33 \\
        Skywork-o1-Open-PRM-Qwen-2.5-1.5B & 31.90 & 53.82 & 69.23 \\
        Skywork-o1-Open-PRM-Qwen-2.5-7B & 31.58 & 52.59 & 69.16 \\
        % \hline
        \bottomrule
    \end{tabular}
    \vspace{+3mm}
    \caption{Results of HitRate@$k$. Bold and underlined results indicate the best and the second best.}
    % \vspace{-4mm}
\label{tab: hitrate}
\end{table}

\paragraph{Analysis on HitRate evaluation for PRMs.}

\begin{figure}[t]
    \centering
    \includegraphics[width=\textwidth]{figures/prm_rank.pdf}
    % \vspace{-10pt}
    \caption{Ranking of rewards for the first incorrect section for different PRMs.}
    % \vspace{-3mm}
    \label{fig: prm_rank}
\end{figure}

To better measure the ability of PRMs to identify erroneous sections in long CoTs, we use HitRate@$k$ to evaluate PRMs. Specifically, within a sample, we rank the sections in ascending order based on the rewards given by the PRM, select the smallest $k$ sections, and calculate the recall rate for the erroneous sections among them. Specifically, we define the sorted sections as $S = \{s_1, s_2, \ldots, s_n\}$, with $E$ being the set of erroneous sections. We select the top $k$ sections, denoted as $S_k = \{s_1, s_2, \ldots, s_k\}$. The HitRate@$k$ is  calculated as:
\begin{align}
\text{HitRate@}k = \frac{|S_k \cap E|}{\min(k, |E|)}
% \nonumber
\label{eq: hitrate}
\end{align}
In this formula, $|S_k \cap E|$ indicates the number of erroneous sections identified among the top $k$ sections. This metric reflects the ability of PRMs to effectively identify erroneous sections within the top $k$ candidate sections. In Table \ref{tab: hitrate}, the relative performance rankings among different PRMs are quite similar to the results in Table \ref{tab: main}. Additionally, we observe that for $k=3$ and $k=5$, the performance differences between various PRMs are not particularly significant. However, when $k=1$, the Qwen2.5-Math-PRM-7B shows a clear performance advantage. Figure \ref{fig: prm_rank} illustrates the ranking ability of different PRMs for the first incorrect section within the sample, which is generally consistent with the performance evaluation results of HitRate@k.
% This is because a smaller $k$ value imposes stricter requirements on the PRM's ability to identify errors.

% HitRate@$k$ evaluates the performance of PRMs from the perspective of reward ranking, providing additional evidence for the experimental results and conclusions in Table \ref{tab: main} from a different angle.

\begin{tcolorbox}[colback=white!95!gray, colframe=gray!70!black, title=Key Finding]
  HitRate@k evaluation aligns with the main results, with Qwen2.5-Math-PRM-7B demonstrating superior performance in identifying the first incorrect section.
\end{tcolorbox}


\begin{figure}[t]
    \centering
    \includegraphics[width=0.8\textwidth]{figures/4.5.4/self-critic.pdf}
    % \vspace{-10pt}
    \caption{F1-score comparison of self-critique and cross-model critique abilities for different models.}
    % \vspace{-5mm}
    \label{fig: self-critic}
\end{figure}

\paragraph{Comparative Analysis of Self-Critique Capabilities of LLMs.} We randomly sample queries based on domains and models that generate the long CoT output, followed by a statistical analysis of the model's performance in evaluating its own outputs as well as those of other models. In Figure \ref{fig: self-critic},  Gemini 2.0 Flash Thinking, DeepSeek-R1, and QwQ-32B-Preview show lower self-critique scores compared to their cross-model critique scores, indicating a prevalent deficiency in self-critic abilities. Notably, DeepSeek-R1 exhibits the largest discrepancy, with a 36\% decrease in self-evaluation compared to evaluations of other models. This suggests models' self-critic abilities remain underdeveloped.
% signaling an area that requires improvement.

\begin{tcolorbox}[colback=white!95!gray, colframe=gray!70!black, title=Key Finding]
  LLMs demonstrate weaker self-critique performance compared to cross-model critique, highlighting a fundamental limitation in self-critic capabilities.
\end{tcolorbox}



%%%

% \noindent\textbf{Performance Analysis Across Different Categories}

% \begin{figure}[htbp]
% \centering
% \includegraphics[width=\linewidth]{figures/prm_task_comparison.pdf}
% \caption{Performance of PRMs across different categories (outlier detection).}
% \label{fig: prm_task}
% % \vspace{-0.6cm}
% % \vspace{-4mm}
% \end{figure}


% \noindent\textbf{Performance Variation in Different Lengths of Long CoT}

% \noindent\textbf{Performance Analysis Across Different Error Types}

% \noindent\textbf{Analysis of In-Sample Reward Ranking}


% % \subsection{Evaluation Metrics}

% % \subsection{Main Results}

% % \subsection{Further Analysis}
% \subsection{Analysis on LLM Critics}
%  \textbf{error location}



% \subsubsection{The Performance across different domains}

% \begin{figure}[t]
%     \centering
%     \includegraphics[width=0.5\textwidth]{figures/critic6.pdf}
%     \caption{The score distributions across different domains.}
%     \label{fig: crtic2}
% \end{figure}

% In Figure \ref{fig: crtic2}, we illustrate the F1-score distribution of various large language models (LLMs) across different domains. Analyzing model performance across domains reveals that most models demonstrate stronger critiquing abilities in Physics, Chemistry, Biology, and General Reasoning compared to Mathematics and Programming, indicating higher proficiency in scientific and general reasoning tasks. Meanwhile, the performance of each model varies significantly depending on the domain, reflecting inherent strengths and weaknesses in handling different tasks. For instance, the Gemini-1.5-Pro model achieves an F1-score of 40.1\% in PCB, yet only 26.6\% in Mathematics. These discrepancies underscore challenges in the models' generalization capabilities.









\section{Conclusion}
\label{subsection:conclusion}
In this paper, we introduce \OURS, a novel framework designed to identify high-quality data that aligns well with the LLM’s learned knowledge to reduce hallucination.
% Our proposed method includes Internal Consistency Probing and Semantic Equivalence Identification, which are designed to separately measure the LLM's understanding of the given instruction and target response.
% In this way, we can measure the familiarity of the LLM with the instruction data and prevent the model from being trained on unfamiliar data, thereby reducing hallucinations.
NOVA includes Internal Consistency Probing and Semantic Equivalence Identification, which are designed to separately measure the LLM's familiarity with the given instruction and target response, then prevent the model from being trained on unfamiliar data, thereby reducing hallucinations.
Lastly, we introduce an expert-aligned reward model, considering characteristics beyond just familiarity to enhance data quality.
By considering data quality and avoiding unfamiliar data, we can use the selected data to effectively align LLMs to follow instructions and hallucinate less in the instruction tuning stage.
Experiments and analysis show the effectiveness of \OURS.

\section*{Limitations}
Although empirical experiments have confirmed the effectiveness of the proposed \OURS, two major limitations remain. 
Firstly, our proposed method requires LLMs to generate multiple responses for the given instruction, which introduces additional execution time.
However, it is worth noting that this additional execution time is used to perform offline data filtering, our proposed method does not introduce additional time overhead in the inference phase.
Additionally, \OURS~is primarily used for single-turn instruction data filtering, thus exploring its application in multi-turn scenarios presents an attractive direction for future research.


\bibliographystyle{plainnat}
\bibliography{references}



\end{document}