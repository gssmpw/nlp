\section{Related Work}
\paragraph{Data Augmentation Frame Work}
% Various techniques of generating synthetic data were proposed to further improve the performance. There are augmentation strategies that work fine independent of dataset like Cutout \cite{07devries2017improved}, Mixup \cite{08zhang2017mixup} and Cutmix \cite{09yun2019cutmix}. There are also augmentation strategies that were designed to autoadapt datasets AutoAugment \cite{Cubuk_2019_CVPR}, Fast AutoAugment \cite{NEURIPS2019_6add07cf} and the most recent DADA \cite{li2020dada} and CMDA \cite{tian2021continuous}. In this work, the data augmentation we discuss satisfies some assumptions, as shown in \ref{The augmentation and limitation} according to the CMDA \cite{tian2021continuous}, most types of transformations including geometry, color and filter satisfy our proposed assumptions.
Data augmentation methods play a crucial role in improving the performance of machine learning models in practical applications. These methods encompass a range of techniques, including traditional fixed augmentation methods like Cutout \cite{07devries2017improved}, Mixup \cite{08zhang2017mixup}, and Cutmix  \cite{09yun2019cutmix}. Additionally, there are adaptive augmentation methods such as AutoAugment \cite{Cubuk_2019_CVPR}, Fast AutoAugment \cite{NEURIPS2019_6add07cf}, DADA \cite{li2020dada}, and CMDA \cite{tian2021continuous}, which dynamically design augmentations based on the dataset. Despite the availability of these diverse augmentation methods, there is a dearth of theoretical frameworks for analyzing the population shift phenomenon induced by data augmentation and the associated shifted population risk.

A recent work \cite{chen2020group} provides a theoretical framework that defines the augmentation operator as a group action. However, their framework has certain limitations, as evidenced by several common augmentation operators that are incompatible with the group action framework, as detailed in the Appendix \ref{sec:diff_group}. Our proposed framework can be applied to a wider range of data augmentation operators compared to theirs.

\paragraph{Population Shift}
Population shift is a common concern in machine learning robustness and generalization problems. It refers to a problem in which the population of data changes during some processes, such as a distribution being transformed to other distributions within the same distribution family, and the change of the parameters of a distribution. A common example for population shift in machine learning is the different semantic styles between the training and testing sets, such as PACS \cite{li2017deeper}, Rotated MNIST, Color MNIST \cite{arjovsky2019invariant}, VLCS, and Office-Home \cite{venkateswara2017deep}. However, not all types of population shifts are natural. Style shifts such as PACS are naturally generated distributions, while population shifts such as Rotated MNIST and Color MNIST are artificially generated. It is obvious that all data augmentations will produce an artificial population shift. This work aims to provide a theoretical framework for artificial population shifts and analyze the relationship between the \textbf{shifted population risk} and the \textbf{original population risk}.