\section{Conclusions and Future Work}
% \heng{this section is very boring. it basically repeats what's in intro. it needs to be rewritten to make it more exciting and inspiring}
Text-to-image models have shown great potential in concept generation. In this work, we introduce a framework for novel concept design, which integrates concept ontology construction, data generation, and a T2I model contrastive training pipeline with curriculum learning technique. In addition, we propose a four-dimensional metric that evaluates the quality of generated concept images. Experimental results across three strong T2I models from both automatic and human evaluations demonstrate that our method significantly outperforms the competing baseline methods. Ablation studies also highlight the importance of our affordance sampling and curriculum learning techniques.

\section*{Limitations}
Our work tackles an important yet underexplored problem of retaining functional coherence in AI for design using T2I models. While our model, in comparison to other state-of-the-art models, is able to generate more coherent and faithful images provided a set of affordances, e.g., \texttt{brew}, \texttt{cut} as in Fig.~\ref{fig:close}, our work inherently relies on the human intuition to evaluate the novelty of the generated concepts. Although we try to alleviate the human bias and lack of coverage using LLM-as-a-judge for automatic evaluation, the question may persist. Moreover, although our concept ontology covers many different concept categories, it does not cover every plausible concept category in the real-world. It would be interesting to see follow-up works explore the direction of constructing a more diverse, richer concept ontology, which in turn would contribute to the generation of more novel concept designs.

% \heng{For future work, possibly you can consider multi-agent collaboration, let generator generate some candidates, identifier gives feedback/critique, then the generator takes feedback to keep refining results; also is it possible to collect human designed artifacts in the future as ground truth?}

\section*{Ethical Consideration}
We acknowledge that our work is aligned with the \textit{ACL Code of the Ethics} \footnote{\url{https://www.aclweb.org/portal/content/acl-code-ethics}} and will not raise ethical concerns.
We do not use sensitive datasets/models that may cause any potential issues/risks.


\section*{Acknowledgement}
This research is based upon work supported by U.S. DARPA ECOLE Program No. HR00112390060. The views and conclusions contained herein are those of the authors and should not be interpreted as necessarily representing the official policies, either expressed or implied, of DARPA, or the U.S. Government. The U.S. Government is authorized to reproduce and distribute reprints for governmental purposes notwithstanding any copyright annotation therein.


