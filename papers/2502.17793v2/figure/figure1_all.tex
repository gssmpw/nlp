\begin{figure*}[t]
\centering
\begin{subfigure}[t]{0.41\linewidth}
    \centering
    \includegraphics[width=\linewidth]{figure/images/close_example_final.pdf}
    \caption{Generated concepts with similar affordances.}
    \label{fig:close}
\end{subfigure}
\hspace{0.1in}
\begin{subfigure}[t]{0.41\linewidth}
    \centering
    \includegraphics[width=\linewidth]{figure/images/distant_example_final.pdf}
    \caption{Generated concepts with distant affordances.}
    \label{fig:distant}
\end{subfigure}
\vspace{-0.1in}
\caption{\textbf{Effect of Affordance Sampling on Novel Concept Generation.} Our affordance sampling strategy selects disparate affordance pairs within our ontology, promoting novel functional coherence rather than redundant combinations. Baseline models tend to generate existing concepts for close affordances (Fig~\ref{fig:close}) but struggle with distant pairs, often introducing multiple objects or omitting functions (Fig~\ref{fig:distant}). In contrast, our models consistently generate functionally coherent novel concepts, achieving higher novelty scores for distant affordance pairs.}
% \heng{figure 1: I feel the distant examples in Figure 1 are not distant enough. do you want to add dalle-e results, perhaps pick those our method performs better than dalle?}
% \zhenhailong{(1) can we add some text description for the columns (i.e., erase, rub) to make clear that these are composition of affordances that the modeling being instructed to generate (2) don't forget to change the model name in the figure and mark (ours)?}}
\label{fig:examples}
\vspace{-0.05in}
\end{figure*}
