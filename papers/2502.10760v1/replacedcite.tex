\section{Related work}
%
%
A large set of works propose methods to more reliably find better prompts. Prompt engineering, e.g.\  ____, can improve performance on many tasks by handcrafting prompts using intuitions from system design, search and planning, and empirical trial and error. Prompt optimization, e.g.\ ____, instead employs variants of discrete optimization and search techniques.
%
Another approach for prompt optimization, particularly in agentic or robotic tasks, is to find the best set of demonstrations for in-context learning____.

While much focus has been put on the performance of the resulting prompts, we do not know whether they are actually optimal. It has been challenging to \emph{interpret} the prompts by relating to the desired tasks, and to understand concretely why some prompts work better than others____. These difficulties are exacerbated by the variations across architectures, pretraining mixtures, fine-tuning processes, and nuanced parameters that vary across publicly available LLMs____
%
This has led to the question why some prompts work better than others, despite no obvious difference to humans____, and why some prompts that are intuitively expected to work may disappoint____. The unreliability and unintuitive nature of prompting have caused safety and ethical concerns____.


%
%
%
%
%
%
%
%
%
%
%
%
%
%
%
%
%
%
%
%
%
%
%
%
%
%
%
%
%
%
%


%
\begin{figure*}[ht!]
%
    \centering
    \includegraphics[page=1,width=\textwidth]{figs/OptimalPromptsFigures.pdf}
    \vspace{-1em}
    \caption{Experimental design to obtain optimal prompts under pretraining and task data generators.
    %
    %
    }
    \label{fig:experiment_figure}
    \vspace{-1em}
\end{figure*}

%