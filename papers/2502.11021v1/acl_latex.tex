% This must be in the first 5 lines to tell arXiv to use pdfLaTeX, which is strongly recommended.
\pdfoutput=1
% In particular, the hyperref package requires pdfLaTeX in order to break URLs across lines.

\documentclass[11pt]{article}

% Change "review" to "final" to generate the final (sometimes called camera-ready) version.
% Change to "preprint" to generate a non-anonymous version with page numbers.
\usepackage[preprint]{acl}

% Standard package includes
\usepackage{times}
\usepackage{latexsym}

% For proper rendering and hyphenation of words containing Latin characters (including in bib files)
\usepackage[T1]{fontenc}
% For Vietnamese characters
% \usepackage[T5]{fontenc}
% See https://www.latex-project.org/help/documentation/encguide.pdf for other character sets

% This assumes your files are encoded as UTF8
\usepackage[utf8]{inputenc}

% This is not strictly necessary, and may be commented out,
% but it will improve the layout of the manuscript,
% and will typically save some space.
\usepackage{microtype}

% This is also not strictly necessary, and may be commented out.
% However, it will improve the aesthetics of text in
% the typewriter font.
\usepackage{inconsolata}

%Including images in your LaTeX document requires adding
%additional package(s)
\usepackage{graphicx}
\usepackage{CJKutf8}
\usepackage{enumitem}
\usepackage{multicol}
\usepackage{multirow}
\usepackage{xspace}
\usepackage{amssymb}
\usepackage{epigraph}
\usepackage{mathtools}
\usepackage{changepage}
\usepackage{listings}
\lstset{
basicstyle=\small\ttfamily,
columns=flexible,
breaklines=true
}

\DeclareMathOperator*{\argmin}{arg\,min}

\usepackage{booktabs}

% If the title and author information does not fit in the area allocated, uncomment the following
%
%\setlength\titlebox{<dim>}
%
% and set <dim> to something 5cm or larger.

\title{Leveraging Uncertainty Estimation for Efficient LLM Routing}

% Author information can be set in various styles:
% For several authors from the same institution:
% \author{Author 1 \and ... \and Author n \\
%         Address line \\ ... \\ Address line}
% if the names do not fit well on one line use
%         Author 1 \\ {\bf Author 2} \\ ... \\ {\bf Author n} \\
% For authors from different institutions:
% \author{Author 1 \\ Address line \\  ... \\ Address line
%         \And  ... \And
%         Author n \\ Address line \\ ... \\ Address line}
% To start a separate ``row'' of authors use \AND, as in
% \author{Author 1 \\ Address line \\  ... \\ Address line
%         \AND
%         Author 2 \\ Address line \\ ... \\ Address line \And
%         Author 3 \\ Address line \\ ... \\ Address line}

% \author{First Author \\
%   Affiliation / Address line 1 \\
%   Affiliation / Address line 2 \\
%   Affiliation / Address line 3 \\
%   \texttt{email@domain} \\\And
%   Second Author \\
%   Affiliation / Address line 1 \\
%   Affiliation / Address line 2 \\
%   Affiliation / Address line 3 \\
%   \texttt{email@domain} \\}

\author{Tuo Zhang\thanks{These authors contributed equally to this work.} \\
  University of Southern California \\
  \texttt{tuozhang@usc.edu} \\\And
  Asal Mehradfar\footnotemark[1]\\
  University of Southern California \\
  \texttt{mehradfa@usc.edu} \\\AND
  Dimitrios Dimitriadis \\
  Amazon AI \\
  \texttt{dbdim@amazon.com} \\\And
  Salman Avestimehr \\
  University of Southern California \\
  \texttt{avestime@usc.edu} \\}

%\author{
%  \textbf{First Author\textsuperscript{1}},
%  \textbf{Second Author\textsuperscript{1,2}},
%  \textbf{Third T. Author\textsuperscript{1}},
%  \textbf{Fourth Author\textsuperscript{1}},
%\\
%  \textbf{Fifth Author\textsuperscript{1,2}},
%  \textbf{Sixth Author\textsuperscript{1}},
%  \textbf{Seventh Author\textsuperscript{1}},
%  \textbf{Eighth Author \textsuperscript{1,2,3,4}},
%\\
%  \textbf{Ninth Author\textsuperscript{1}},
%  \textbf{Tenth Author\textsuperscript{1}},
%  \textbf{Eleventh E. Author\textsuperscript{1,2,3,4,5}},
%  \textbf{Twelfth Author\textsuperscript{1}},
%\\
%  \textbf{Thirteenth Author\textsuperscript{3}},
%  \textbf{Fourteenth F. Author\textsuperscript{2,4}},
%  \textbf{Fifteenth Author\textsuperscript{1}},
%  \textbf{Sixteenth Author\textsuperscript{1}},
%\\
%  \textbf{Seventeenth S. Author\textsuperscript{4,5}},
%  \textbf{Eighteenth Author\textsuperscript{3,4}},
%  \textbf{Nineteenth N. Author\textsuperscript{2,5}},
%  \textbf{Twentieth Author\textsuperscript{1}}
%\\
%\\
%  \textsuperscript{1}Affiliation 1,
%  \textsuperscript{2}Affiliation 2,
%  \textsuperscript{3}Affiliation 3,
%  \textsuperscript{4}Affiliation 4,
%  \textsuperscript{5}Affiliation 5
%\\
%  \small{
%    \textbf{Correspondence:} \href{mailto:email@domain}{email@domain}
%  }
%}

\begin{document}
\maketitle
\begin{abstract}
Deploying large language models (LLMs) in edge-cloud environments requires an efficient routing strategy to balance cost and response quality. Traditional approaches prioritize either human-preference data or accuracy metrics from benchmark datasets as routing criteria, but these methods suffer from rigidity and subjectivity. Moreover, existing routing frameworks primarily focus on accuracy and cost, neglecting response quality from a human preference perspective.
%
In this work, we propose the Confidence-Driven LLM Router, a novel framework that leverages uncertainty estimation to optimize routing decisions. 
%
To comprehensively assess routing performance, we evaluate both system cost efficiency and response quality. In particular, we introduce the novel use of LLM-as-a-Judge to simulate human rating preferences, providing the first systematic assessment of response quality across different routing strategies.
%
Extensive experiments on MT-Bench, GSM8K, and MMLU demonstrate that our approach outperforms state-of-the-art routing methods, achieving superior response quality while maintaining cost efficiency.
\end{abstract}




\section{Introduction}
\label{sec:introduction}
The business processes of organizations are experiencing ever-increasing complexity due to the large amount of data, high number of users, and high-tech devices involved \cite{martin2021pmopportunitieschallenges, beerepoot2023biggestbpmproblems}. This complexity may cause business processes to deviate from normal control flow due to unforeseen and disruptive anomalies \cite{adams2023proceddsriftdetection}. These control-flow anomalies manifest as unknown, skipped, and wrongly-ordered activities in the traces of event logs monitored from the execution of business processes \cite{ko2023adsystematicreview}. For the sake of clarity, let us consider an illustrative example of such anomalies. Figure \ref{FP_ANOMALIES} shows a so-called event log footprint, which captures the control flow relations of four activities of a hypothetical event log. In particular, this footprint captures the control-flow relations between activities \texttt{a}, \texttt{b}, \texttt{c} and \texttt{d}. These are the causal ($\rightarrow$) relation, concurrent ($\parallel$) relation, and other ($\#$) relations such as exclusivity or non-local dependency \cite{aalst2022pmhandbook}. In addition, on the right are six traces, of which five exhibit skipped, wrongly-ordered and unknown control-flow anomalies. For example, $\langle$\texttt{a b d}$\rangle$ has a skipped activity, which is \texttt{c}. Because of this skipped activity, the control-flow relation \texttt{b}$\,\#\,$\texttt{d} is violated, since \texttt{d} directly follows \texttt{b} in the anomalous trace.
\begin{figure}[!t]
\centering
\includegraphics[width=0.9\columnwidth]{images/FP_ANOMALIES.png}
\caption{An example event log footprint with six traces, of which five exhibit control-flow anomalies.}
\label{FP_ANOMALIES}
\end{figure}

\subsection{Control-flow anomaly detection}
Control-flow anomaly detection techniques aim to characterize the normal control flow from event logs and verify whether these deviations occur in new event logs \cite{ko2023adsystematicreview}. To develop control-flow anomaly detection techniques, \revision{process mining} has seen widespread adoption owing to process discovery and \revision{conformance checking}. On the one hand, process discovery is a set of algorithms that encode control-flow relations as a set of model elements and constraints according to a given modeling formalism \cite{aalst2022pmhandbook}; hereafter, we refer to the Petri net, a widespread modeling formalism. On the other hand, \revision{conformance checking} is an explainable set of algorithms that allows linking any deviations with the reference Petri net and providing the fitness measure, namely a measure of how much the Petri net fits the new event log \cite{aalst2022pmhandbook}. Many control-flow anomaly detection techniques based on \revision{conformance checking} (hereafter, \revision{conformance checking}-based techniques) use the fitness measure to determine whether an event log is anomalous \cite{bezerra2009pmad, bezerra2013adlogspais, myers2018icsadpm, pecchia2020applicationfailuresanalysispm}. 

The scientific literature also includes many \revision{conformance checking}-independent techniques for control-flow anomaly detection that combine specific types of trace encodings with machine/deep learning \cite{ko2023adsystematicreview, tavares2023pmtraceencoding}. Whereas these techniques are very effective, their explainability is challenging due to both the type of trace encoding employed and the machine/deep learning model used \cite{rawal2022trustworthyaiadvances,li2023explainablead}. Hence, in the following, we focus on the shortcomings of \revision{conformance checking}-based techniques to investigate whether it is possible to support the development of competitive control-flow anomaly detection techniques while maintaining the explainable nature of \revision{conformance checking}.
\begin{figure}[!t]
\centering
\includegraphics[width=\columnwidth]{images/HIGH_LEVEL_VIEW.png}
\caption{A high-level view of the proposed framework for combining \revision{process mining}-based feature extraction with dimensionality reduction for control-flow anomaly detection.}
\label{HIGH_LEVEL_VIEW}
\end{figure}

\subsection{Shortcomings of \revision{conformance checking}-based techniques}
Unfortunately, the detection effectiveness of \revision{conformance checking}-based techniques is affected by noisy data and low-quality Petri nets, which may be due to human errors in the modeling process or representational bias of process discovery algorithms \cite{bezerra2013adlogspais, pecchia2020applicationfailuresanalysispm, aalst2016pm}. Specifically, on the one hand, noisy data may introduce infrequent and deceptive control-flow relations that may result in inconsistent fitness measures, whereas, on the other hand, checking event logs against a low-quality Petri net could lead to an unreliable distribution of fitness measures. Nonetheless, such Petri nets can still be used as references to obtain insightful information for \revision{process mining}-based feature extraction, supporting the development of competitive and explainable \revision{conformance checking}-based techniques for control-flow anomaly detection despite the problems above. For example, a few works outline that token-based \revision{conformance checking} can be used for \revision{process mining}-based feature extraction to build tabular data and develop effective \revision{conformance checking}-based techniques for control-flow anomaly detection \cite{singh2022lapmsh, debenedictis2023dtadiiot}. However, to the best of our knowledge, the scientific literature lacks a structured proposal for \revision{process mining}-based feature extraction using the state-of-the-art \revision{conformance checking} variant, namely alignment-based \revision{conformance checking}.

\subsection{Contributions}
We propose a novel \revision{process mining}-based feature extraction approach with alignment-based \revision{conformance checking}. This variant aligns the deviating control flow with a reference Petri net; the resulting alignment can be inspected to extract additional statistics such as the number of times a given activity caused mismatches \cite{aalst2022pmhandbook}. We integrate this approach into a flexible and explainable framework for developing techniques for control-flow anomaly detection. The framework combines \revision{process mining}-based feature extraction and dimensionality reduction to handle high-dimensional feature sets, achieve detection effectiveness, and support explainability. Notably, in addition to our proposed \revision{process mining}-based feature extraction approach, the framework allows employing other approaches, enabling a fair comparison of multiple \revision{conformance checking}-based and \revision{conformance checking}-independent techniques for control-flow anomaly detection. Figure \ref{HIGH_LEVEL_VIEW} shows a high-level view of the framework. Business processes are monitored, and event logs obtained from the database of information systems. Subsequently, \revision{process mining}-based feature extraction is applied to these event logs and tabular data input to dimensionality reduction to identify control-flow anomalies. We apply several \revision{conformance checking}-based and \revision{conformance checking}-independent framework techniques to publicly available datasets, simulated data of a case study from railways, and real-world data of a case study from healthcare. We show that the framework techniques implementing our approach outperform the baseline \revision{conformance checking}-based techniques while maintaining the explainable nature of \revision{conformance checking}.

In summary, the contributions of this paper are as follows.
\begin{itemize}
    \item{
        A novel \revision{process mining}-based feature extraction approach to support the development of competitive and explainable \revision{conformance checking}-based techniques for control-flow anomaly detection.
    }
    \item{
        A flexible and explainable framework for developing techniques for control-flow anomaly detection using \revision{process mining}-based feature extraction and dimensionality reduction.
    }
    \item{
        Application to synthetic and real-world datasets of several \revision{conformance checking}-based and \revision{conformance checking}-independent framework techniques, evaluating their detection effectiveness and explainability.
    }
\end{itemize}

The rest of the paper is organized as follows.
\begin{itemize}
    \item Section \ref{sec:related_work} reviews the existing techniques for control-flow anomaly detection, categorizing them into \revision{conformance checking}-based and \revision{conformance checking}-independent techniques.
    \item Section \ref{sec:abccfe} provides the preliminaries of \revision{process mining} to establish the notation used throughout the paper, and delves into the details of the proposed \revision{process mining}-based feature extraction approach with alignment-based \revision{conformance checking}.
    \item Section \ref{sec:framework} describes the framework for developing \revision{conformance checking}-based and \revision{conformance checking}-independent techniques for control-flow anomaly detection that combine \revision{process mining}-based feature extraction and dimensionality reduction.
    \item Section \ref{sec:evaluation} presents the experiments conducted with multiple framework and baseline techniques using data from publicly available datasets and case studies.
    \item Section \ref{sec:conclusions} draws the conclusions and presents future work.
\end{itemize}

%PlanGREEN

%GEN-Plan

%G- generate
%R-refine
%E- edit

%% GREEN-plan

%% PURPLE
\begin{figure*}
    \centering
    \Description{PLAID's system architecture diagram. Top part shows the database (a), and bottom part shows the interface (b). The system starts from bottom right as an instructor is interested in a programming domain, then the pipeline described in the text produces reference materials at different levels of granularity, and these are presented in the interface.}
    \includegraphics[width=\textwidth]{img/system-architecture-subgoals.png}
    \caption{PLAID's reference content is generated through an LLM pipeline
    %inspired by the practices of instructors who have successfully identified programming plans. 
    that produces output on three levels.
    First, a wide variety of use cases are generated to create example programs that focus on code's applications. Next, using LLM's explanatory comments that represent subgoals within the code, the examples are segmented into meaningful code snippets. The LLM is queried to generate other plan components for each code snippet. Finally, the code snippets are clustered to identify the most common patterns, representing plan candidates. The full programs are presented in `Programs' views of PLAID interface, whereas snippets are presented in clusters in the `Plan Creation' view.}
    \label{fig:system-pipeline}
\end{figure*}
\section{PLAID: A System for Supporting Plan Identification}
\label{sec:system-design}

Following the design goals devised from the design workshop, we refined our early prototype into PLAID: a
%LLM-powered
tool to assist instructors in their plan identification process.
PLAID synthesizes the capabilities of LLMs in code generation with interactions enabling plan identification practices observed in our studies with instructors.
As we noted in the findings of our design workshop, the LLM-generated candidate plans are not ready to be used as is in instruction, but instructors can readily adapt and correct them (\cref{sec:workshop-findings-condition2}).
PLAID enables collaboration between instructors and LLMs, enhancing the plan identification process by automating its time-intensive information-gathering tasks and facilitating instructors' ability to refine LLM-generated candidate plans based on their knowledge about pedagogy and the programming domain. 



\subsection{Practical Illustration}

To understand how instructors use can PLAID to more easily adopt plan-based pedagogies, we follow Jane, a computer science instructor using PLAID to design programming plans for her course (summarized in \cref{fig:jane-workflow}).

Jane is teaching a programming course for psychology majors and wants to introduce her students to data analysis using Pandas. As her students have limited prior programming experience and use programming for specific goals, she organizes her lecture material around programming plans to emphasize purpose over syntax. 
% that explain practical concepts to students and help them focus on the purpose behind the code they write.
% However, she realizes that all introductory computer science courses offered at her institution only teach basic programming constructs like data structures. After exploring Google Scholar for effective instructional methods to teach application-focused programming to non-computer science majors, she learned about plan-based pedagogies that help them focus on the purpose behind the code they write. In her literature review, she finds out about PLAID, a tool that can help her design domain-specific plans. She reviews the domains supported by the tool (Pandas, Pytorch, Beautifulsoup, and Django) and decides to use Pandas, a popular and powerful data analysis and manipulation library, to create her curriculum. 

She logs in to the PLAID web interface, % and takes time to explore the system's features. 
and asks PLAID to suggest a plan (\cref{fig:jane-workflow}, 1). The first plan recommended to her 
% she sees is a plan to help students learn about
is about reading CSV files. 
She thinks the topic is important and the solution code aligns with her experience; % the solution is promising and represents an important concept that students need to know about.
% She is satisfied with the given solution 
but she finds the generated name and goal to be too generic. She edits (\cref{fig:jane-workflow}, 2) these fields to provide more context that she feels is right for her students.
% She refines those fields and then 
To make this plan more abstract and appropriate for more use cases, %explain how this plan can be used for reading data from different file formats,
she marks the file path as a changeable area (\cref{fig:jane-workflow}, 3), generalizing the plan for reading data from different file formats.

Inspired by the first plan, she decides to create a plan for saving data to disk. She wants to teach the most conventional way of saving data, so she switches to the use case tab (\cref{fig:jane-workflow}, 4) and explores example programs that save data to get a sense of common practices.  %interact with the list of complete programs.
She finds a complete example where a DataFrame is created and and saved to a file. %performs cleaning tasks like deleting NaN values, and exports it.
% She realizes that something she hadn't thought of before: saving new data is almost always necessary after performing data manipulation operations!
She selects the part of the code that exports data to a file and creates a plan from that selection (\cref{fig:jane-workflow}, 5).


For the next plan, she reflects on her own experience with Pandas. She recalls that merging DataFrames was a key concept, but cannot remember the full syntax. 
% Jane reflects on her experience working with Pandas and recalls that merging DataFrames is a key operation when working with big data.
She switches to the full programs tab (\cref{fig:jane-workflow}, 6) that includes complete code examples and searches (\cref{fig:jane-workflow}, 7) for ``\texttt{.merge}'' to find different instances of merging operations. % and tries to use the search bar to find a relevant program that contains ``.merge''. 
After finding a comprehensive example, she selects the relevant section of the code and creates a plan from it.
% She again selects a part of the example, creates plan from the selection, and refines it. She engages with the system iteratively and designs twenty plans for her lecture. 

After designing a set of plans that capture the important topics, she organizes them into groups (\cref{fig:jane-workflow}, 8) 
% also grouped similar plans together
to emphasize sets of plans with similar goals but different implementations. For instance, she takes her plans about \texttt{.merge} and \texttt{.concat} and groups them together to form a category of plans that students can reference when they want to {combine data from different sources}.

% combining data using ``merge'' or ``concat''.

% the the she used plans isn't very good right now
% She exports these plans and starts preparing her lecture slides, using the plans as a way of presenting key concepts to students with minimal programming experience.
She exports these plans to support her students with minimal programming experience by preparing lecture slides that organize the sections around plan goals, using plan solutions as worked examples in class, and providing students with cheat sheets that include relevant plans for their laboratory sessions.
% using the plan goals as titles for different sections of her slides, and using the solutions as references for the examples she creates. Finally, she makes a PDF cheatsheet with all the plans for students to reference during the week's laboratory.
% The next day, she starts preparing her lecture slides and realizes that the names and goals she wrote for her plans represent key concepts in Pandas. She references the plans she created to design annotated examples that she includes on her lecture slides.

%% How does Jane actually use the plans? 
%% > Important to be careful to note that this isn't actually part of the system....
%% > She uses the generated plans to (a) as inspiration for worked examples in teh course, (b) as stems for questions that test how code should be completed
%% > She notices she now has a list of key concepts in the area


\begin{figure*}[h]
        \Description{An annotated screenshot of PLAID's `Programs' view. On the left, a list of use cases such as `Renaming columns in a Frame' and `Plotting a histogram of a column' is shown, with a scrollable list and a search bar. The latter one is selected, and on the right, we see the contents of the program in a monospaced font, with four buttons explained in the caption.}
        \includegraphics[width=\textwidth]{img/system-diagram-1-fixed.png}
        \caption{Plan Identification using PLAID: (a) list of example programs for instructors organized by natural language descriptions, (b) list of full programs of code, (c) search bar enabling easy navigation of given content to find code for specific ideas, (d) button to create a plan using the selected part of the code, (e) button to create a plan using the complete example program, (f) button to view an explanation for a selected code snippet, and (g) button for executing the selected code.}
        \label{fig:system-diagram-1}
\end{figure*}

\subsection{System Design}

At a high level, PLAID\footnote{The code for PLAID can be found at: https://github.com/yosheejain/plaid-interface.} operates on two subsystems: (1) a database of LLM-generated reference materials created through a pipeline that uses \edit{OpenAI's GPT-4o\footnote{https://openai.com/index/hello-gpt-4o/}~\cite{achiam2023gpt}}, inspired by instructors' best practices for identifying programming plans (see ~\cref{fig:system-pipeline})
%LLM for identifying plans in application-focused domains 
and (2) an interface that allows instructors to browse reference materials for relevant code snippets 
% and other plan components to achieve a goal that meets their needs. Then, they refine the candidates to mine plans 
and refine suggested content into programming plans
(see Figures~\ref{fig:system-diagram-1} and~\ref{fig:system-diagram-2}).
% In this section, we describe the implementation of the pipeline generating the reference materials and the key interface features of PLAID.



\subsubsection{Database of Reference Materials for Application-Focused Domains}

PLAID extracts information from reference materials at three levels of granularity to support each instructor's unique workflow: complete programs that address a particular use case, annotated program snippets that include goals and changeable areas, and plan candidates that cluster relevant program snippets together.

\textbf{Generating complete example programs.}
The content at the lowest level of granularity in the PLAID database are the complete programs. 
%These candidate plans were generated using a pipeline to generate \textit{plan-ful examples}, which we define as examples of programming plans in use, with all plan components identified (see Section~\ref{sec:components}). This implementation had three stages: (1) generating in-domain programs, (2) segmenting programs into plan-ful examples, and (3) clustering plan-ful examples into plans. 
\label{sec:llm-pipeline}
% \begin{figure}
% \centering
% % \includegraphics[width=0.5\textwidth]{img/pipeline-new.png}
% \includegraphics[width=\textwidth]{img/new-plan-pipeline.png}
% \caption{The three stage process for generating example programs, segmenting them with plan components, and clustering these plan-ful examples.
% %collecting and processing responses from ChatGPT into plan-ful examples}
% %\caption{The pipeline for LLM plan generation.}
% }
% \label{fig:llm-methods}
% \end{figure}
% \subsubsection{Generating In-Domain Programs}
% Informed by the insights identified in our interview study, we generated programming plans relevant to an application-focused domain: web scraping via BeautifulSoup. We utilized an LLM-based approach to generate these plans with the GPT-4 model from OpenAI using its publicly available API in an iterative workflow. 
% Our participants examined example programs and conducted literature reviews (Section \ref{sec:viewing-programs}) as key parts of their plan identification process. 
As these examples should capture a variance of use cases in the real world, we utilized an LLM trained on a large corpus of computer programs and natural language descriptions~\cite{liu2023isyourcode}.
% Inspired by this, we used Open AI's GPT-4, a state-of-the-art large language model for code generation that is trained on a large corpus of computer programs~\cite{liu2023isyourcode},
% to generate candidate programs along with its respective plan components in the programs.
We prompted\footnote{Full prompts can be found in \cref{sec:appendix-pipeline}.} the model to generate \texttt{specific use cases of <application-focused library>}, defining use case as \texttt{a task you can achieve 
with the given library} (see \cref{sec:use_case_prompt}). Subsequently, we prompted the model to \texttt{write code to do the following: <use case>}, producing a set of 100 example programs with associated tasks (see \cref{sec:code_prompt}). By generating the use cases first and generating the solution later, we avoided the problems with context windows of LLMs where the earlier input might get `forgotten', resulting in the model producing the same output repeatedly. For practical purposes, we generated 100 programs per domain. \edit{To test for potential ``hallucinations'' where the LLM generates plausible yet incorrect code~\cite{Ji_2023_hallucination}, we checked the syntactic validity of the generated programs before developing the rest of our pipeline. No more than one out of 100 generated programs included syntax errors in each of our domains, i.e., Pandas, Django, and PyTorch. Thus, we concluded that hallucinations are not a major threat to the code generation aspect of PLAID.}
%while hallucinations in LLMs are a pressing concern for systems that utilize these models,
% This collection of example programs (which we refer to as 
%dataset 
% $\mathcal{D}$) was used as our primary dataset for further analysis.

\textbf{Generating annotated program snippets.}
% \subsubsection{Segmenting Programs Into Plan-ful Examples}
% We then proceed to compile these examples with each of the plan components generated using ChatGPT. We construct a new dataset with these components, Dataset \((\mathcal{D}^{\textit{Comp}})\).
The second level of granularity in PLAID consists of small program snippets and a goal, with changeable areas annotated. 
We used the generated programs from
% \mathcal{D}$
the prior step as the input to the LLM to add subgoal labels, where we prompted the LLM to annotate subgoals (see \cref{sec:subgoals_prompt}) as comments that describe \texttt{small chunks of code that achieve a task that can be explained in natural language}. These subgoal labels were used to break the full program into shorter snippets. Each snippet was fed back to the model to generate changeable areas (see \cref{sec:ca_prompt}), defined in the prompt as \texttt{parts of the idiom that would change when it is used in different scenarios}. The subgoal label that explained a code snippet corresponded to its goal in the plan view and the list of elements assigned as changeable was used for annotations.
% (see Stage 2 in Figure~\ref{fig:llm-methods}),
%We fragmented these generated programs into smaller code pieces by generating \textit{subgoals} in the program. Then, each goal (Section \ref{sec:goal}) and the accompanying code solution (Section \ref{sec:solution}) were added as a single unit of data in our plan-ful example dataset of components, \(\mathcal{D}^{\textit{Plan-ful}}\). For each of these datapoints, we prompted the model to identify \textit{changeable areas} (Section \ref{sec:changeable}). %The name (Section \ref{sec:name}) was determined later in the pipeline (Stage 2 in Figure \ref{fig:llm-methods}).


% From the results of our qualitative study, we now know about the parts of a programming plan. In order to extract these plans automatically, we used ChatGPT. We accessed it using its publicly available API and we used the GPT-4 model. We selected 3 domains that are interesting for non-majors. This included . 

% For each of these domains, we first asked the LLM to generate 100 use cases. We then re-prompted it with the use cases it generated and asked it to generate code that would be written to accomplish that use case.
% potential for another table?
% add code metrics from stackoverflow github work for chatgpt
% With all these code pieces collected, we then asked ChatGPT to generate each of the plan parts one-by-one.

% \subsubsection*{Extracting Goals and Solutions}Generated programs 
% in \(\mathcal{D}\) 
% typically included a comment before each line, which described that line's functionality. However, these comments did not capture the high-level purpose of the code, as required by a plan goal. To generate more abstract goals for a piece of code, we defined subgoals as \texttt{short descriptions of small pieces of code that do something meaningful} in a prompt and asked the LLM to \texttt{highlight subgoals as comments in the code.} %In our query, we also added the way we define subgoals to provide the relevant context to the model. Specifically, we wrote that 
% The output from this prompt was a modified version of each program
% from \(\mathcal{D}\), 
% where blocks of code are preceded by a comment describing the goal of that block. % of code. % instead of restating functionality. 

% We split each complete program into multiple segments based on these new comments. Thus, the subgoal comments from each complete program I
% n the modified \(\mathcal{D}\) 
% became a plan goal, and the code following that comment became the associated solution. %, collected in \(\mathcal{D}^{\textit{Plan-ful}}\). % After it returned the annotated code piece, we extracted the comment and the following lines of code before the next comment. This pair acted as a subgoal-code piece. We collected all such pairs across all use cases from \(\mathcal{D}\) and added them to \(\mathcal{D}^{\textit{Plan-ful}}\).
% Each goal 
% %(Section \ref{sec:goal}) 
% and solution pair
% %(Section \ref{sec:solution}) 
% was added as a single unit of data in our plan-ful example dataset.
% , \(\mathcal{D}^{\textit{Plan-ful}}\).

% \subsubsection*{Extracting Changeable Areas}To annotate the changeable areas for a plan, we defined changeable areas as \texttt{parts of the plan that would change when it is used in a different context} in our prompt and asked the model to \texttt{return the exact part of the code from the line that would change} for all code pieces from the dataset with plan-ful examples.
% from \(\mathcal{D}^{\textit{Plan-ful}}\). 
% This data was added to \(\mathcal{D}^{\textit{Plan-ful}}\).

% to-do
% \subsubsection{Clustering Plan-ful Examples into Plans}
\textbf{Generating clustered plan candidates.}
\label{sec:clustering}
% We perform k-means clustering on the plans \(\mathcal{D}^{\textit{Plan-ful}}\) to identify clusters of similar code pieces and thus, programming plans.
The highest level of granularity provided in PLAID
%presents users with 
are
plan candidates, in the form of clusters of annotated program snippets. To compare the similarity of program snippets, we used CodeBERT embeddings following prior work~\cite{codebert} and applied Principal Component Analysis (PCA) \cite{PCAanalysis} to reduce the dimensionality of the embedding while preserving 90\% of the variance. The snippets were clustered using the K-means algorithm~\cite{kmeansclustering}, using the mean silhouette coefficient for determining optimal K~\cite{silhouettecoeff}. Each cluster is treated as a plan candidate, with the goal, code, and changeable areas from each program snippet in the cluster presented as a suggested value for the plan attributes.
% We used a clustering algorithm to group similar program snippets 
% plan-ful examples together as a programming plan. For clustering the code pieces, we used the CodeBERT model from Microsoft \cite{codebert} to obtain embeddings for each code piece in our dataset of plan-ful examples
% % in \(\mathcal{D}^{\textit{Plan-ful}}\) 
% and applied Principal Component Analysis (PCA) \cite{PCAanalysis} to reduce the dimensionality of the embedding vectors while preserving 90\% of the variance. These embeddings were clustered using the K-means algorithm~\cite{kmeansclustering}. The optimal number of clusters \(\mathcal{K}\) was determined by assessing all possible \(\mathcal{K}\) values 
% % (where \(\mathcal{K} \in [2, \texttt{length}(\mathcal{D}^{\textit{Plan-ful}})]\))
% using the mean silhouette coefficient \cite{silhouettecoeff}. We assigned each example 
% % in \(\mathcal{D}^{\textit{Plan-ful}}\) 
% to a cluster of similar code pieces. 
% \subsubsection*{Extracting Names}
For each plan candidate, a name (see \cref{sec:name_prompt}) that summarizes all snippets in the cluster was generated by prompting an LLM with the contents of the snippets and stating that it should generate \texttt{a name that reflects the code's purpose} and it should focus on \texttt{what the code is achieving and not the context}. 
% Then, all code snippets from each cluster of examples were provided as input to the LLM along with a prompt asking it to \texttt{devise a name for that cluster of plans}.

% \subsection{Interface for Refining Candidate Plans}

% %nd the back-end server relied on routes written in Flask. The domain-specific candidate plans suggested to the user are queried from the database of candidate plans generated using the LLM. Each participant was required to log in to the web page using their unique credentials, which allowed us to record their activity for analysis. While the complete details of our implementation of the web-based application are out of scope for this paper, we describe its main features in Section~\ref{sec:implementation_of_webinterface}.

% \subsubsection{\edit{Preliminary Technical Evaluation of Generated Content}}

% \edit{syntactic validity and standard code complexity metrics to determine
% their suitability for novices}


\begin{figure*}[h]
    \Description{An annotated screenshot of PLAID's Plan Creation view with three panes, with plans shown as boxes on the left. A plan is highlighted, and we see its components on the middle pane. On the rightmost pane, we see suggested values for the selected component.}
    \includegraphics[width=\textwidth]{img/system-diagram-2-new.png}        
    \caption{Plan Identification using PLAID: (h) button that suggests a domain-specific candidate plan from the system database, (i) pane enabling viewing of similar values for the selected plan component, (j) button to view the solution code as part of a complete program, (k) pane with a structured template for plan design with editable fields to refine plan components, (l) button to copy a selected plan, (m) button to mark snippets of code from the plan solution as changeable areas, and (n) a button to group plans together into a category and add a name.}
    \label{fig:system-diagram-2}
\end{figure*}

% \subsubsection{Key Characteristics}
% PLAID supports the process of plan identification in data processing with Pandas, machine learning with Pytorch, web development using Django, and web scraping using BeautifulSoup. 

\subsubsection{Interface for Designing Programming Plans}
Building on the 
%characteristics addressed in the artifact (Section~\ref{sec:design-artifact}) and 
design goals identified in the design workshop (\cref{sec:design-goals}), PLAID enables a set of key interactions to assist instructors in refining candidates to design plans for their instruction. 



\textbf{Interactions for Initial Plan Identification.}
% Initial Plan Identification with Quick Exploration of Many Authentic Programs
While instructors valued the availability of code examples in the design workshop (Section~\ref{sec:design-workshop-findings}), we observed many opportunities for scaffolding their interaction with the reference material. To this end, PLAID presents example programs in two different views \textbf{(DG1)}. 
% We saw instructors scanning examples, selecting desired code pieces, and copying them over into their plan templates in all conditions in the study. 
The ``Programs (Organized by Use Case)'' (\cref{fig:system-diagram-1}a) tab includes a list of use cases where instructors can click on an item to expand the program for that use case.
The ``Programs (Full Text)''  tab (\cref{fig:system-diagram-1}b) lists all the programs and enables instructors to scroll or search through (\cref{fig:system-diagram-1}c) all the code at once.
% presents the contents of all the programs expanded viewing a list of complete code examples, allowing instructors to look at materials they would typically search for when designing plans.
% equipping instructors with full-code programs organized in a list of short natural language descriptions of common use cases in their domain of expertise. 
Both views support directly creating a plan from the whole example (\cref{fig:system-diagram-1}e), or a selected part of it (\cref{fig:system-diagram-1}d), by copying the solution and the goal of the program into an empty plan template
% < Highlight code in full code and code pane in tab1 and make a plan (D1)
% < Add a button to add full program as a plan too (D1)
further supporting efficient use of the material \textbf{(DG3)}.
% This interaction copies over the selected code and its respective use case into the solution and name fields, respectively. 
% < Code explanation plugin for strange syntax (GPT) (D2)

To facilitate understanding unfamiliar code and syntax, we implemented a ``View Explanation'' button (\textbf{DG2}) that generates a short description of the selected line(s) of code by prompting an LLM (\cref{fig:system-diagram-1}f). 
% In this case, participants hesitated to use the suggested syntax in their plans because its functionality was unclear to them. PLAID supports a button named ``View Explanation'' where the user can select a method, function, or line of code that is unclear and click on it to understand its working \textbf{(D2)}. 
Participants also looked for code execution to validate and understand a program. However, since the code snippets instructors work with are often incomplete in this task, we implemented a ``Run Code'' feature (\textbf{DG2}) that predicts the output of a selected code snippet by prompting an LLM to walk through the code \texttt{step by step}, using Chain-of-Thought prompting~\cite{wei2022chain} (\cref{fig:system-diagram-1}g). Only the predicted output for the code is presented, ignoring other output from the LLM.

% to examine the code behavior and thus mitigate the challenge of being faced with unfamiliar syntax. Thus, using PLAID, instructors are able to run complete programs to view their output \textbf{(D2)}.
% < Search in the use cases (and full progs) (D3)
% Frequently, instructors relied on their expertise and experience to formulate ideas about goals for which they wanted to create plans. While interacting with condition C in the design workshop, interviewees suggested including a mechanism to search for specific keywords within code and  its natural language description. To facilitate the instructor-LLM collaboration, allowing users to find examples implementing their ideas, PLAID includes a search bar that helps users navigate the given use cases, complete programs, and effectively find specific examples they may be looking for \textbf{(D3)}.

\textbf{Interactions for Plan Refinement.}
% Support Plan Refinement with Comparisons of content
% Participants indicated difficulty mining plans from code examples (Section~\ref{sec:challenges_practice}). 
To provide suggestions for code patterns common enough to be potential programming plans,
%To alleviate challenges in identifying content common enough for designing plans, 
we utilize the clustered program snippets from our database. In the ``Plan Creation'' view of PLAID, instructors can ask for suggestions (\cref{fig:system-diagram-2}h) to see a candidate plan to refine (\textbf{DG3}).  \edit{This functionality allows instructors to draw on their experience to recognize common code snippets and decide if they are valuable to teach students.}
% If instructors want to demonstrate their plan as part of a complete code example, they can review these examples reducing the effort that they would need to put in to recall syntax and construct a complete example. 
\edit{This promotes recognition over recall \cite{recognition_over_recall}, thus helping reduce the cognitive effort that instructors may have to put in while designing programming plans traditionally.}
To allow instructors to better understand the context of a plan under refinement, PLAID 
also includes a button for searching for the current solution within the entire set of full programs
%, showing the code snippet in context 
%as part of a complete example
(\textbf{DG3}, \cref{fig:system-diagram-2}j).
% < Keyword search/embedding filter for potential values (D1)

As instructors edit the components of a plan, they are shown similar values from the corresponding component in that cluster (\cref{fig:system-diagram-2}i). By clicking on any suggested value, instructors can replace a plan component with a suggestion that better captures that aspect of the plan \textbf{(DG1)}. \edit{By allowing instructors to view the plan they are working on along with other related code pieces in a split screen view, we promote instructor efficiency by reducing the split-attention effect \cite{tarmizi1988guidance}. In the current plan creation process, even when using LLMs from their chat interface, instructors would have to switch between windows with code examples and their text editor which may increase the load on the instructors' working memory \cite{clark2023learning}. In PLAID, instructors can edit their plans and view similar code pieces at the same time.}

% \edit{By enabling these interactions and thus organizing ``knowledge in the world'' effectively, PLAID reduces the need for instructors to store and retrieve the ``knowledge in their head'' \cite{Norman_DOET}. Thus, PLAID optimizes the plan creation process by allowing efficient search within the ``knowledge in the world'' and reducing the cognitive load while storing and retrieving ``knowledge in the head'', minimizing the total effort required \cite{} by instructors.}
% after searching its code corpus for similar examples using a keyword search \textbf{(DG1)}.
% < Show use case button in solution (add highlighting) (D2)
% To help instructors easily consider the context of a plan as they refine it, PLAID 
% In the design workshop, few instructors emphasized the importance of presenting worked and contextualized examples to students. 

% ‘go to a use case’ button that redirects the user to the tab with full code programs and highlights the plan as part of a complete example \textbf{(D2)}.

\textbf{Interactions for Building Robust and Shareable Plan Descriptions.}
% Support robust/sharable plan descriptions
% From Section~\ref{sec:process_intro_plan_design}, instructors indicated drawing on their experience in the application-specific domain and instructional expertise to think about how to best solve a problem. 
PLAID encourages instructors to design plans in a structured template (\cref{fig:system-diagram-2}k). Moreover, PLAID reinforces the plan template by providing a dedicated method for annotating changeable areas by highlighting any part of the code (\textbf{DG3}, \cref{fig:system-diagram-2}m). Instructors can further explain the changeable areas by adding comments as text.
% \edit{The structured template view of the plan encourages instructors to articulate their mental models of how the plan would generalize to other problems, allowing the transfer of ``knowledge in the head'' to ``knowledge in the world''.}

Our design workshop showed that participants would create a plan and copy it to emphasize alternatives or modifications to the underlying idea. To support this workflow,
% In our design workshop, participants created copies of their plans to display alternative solutions to achieve the same goal, emphasizing that multiple possible solutions in code could accomplish the same goal.
% < Duplicating plans (D3)
% To accelerate this process of teaching a variety of possible solutions, 
PLAID allows users to ``duplicate'' plans on the canvas and further edit them to present alternative solutions for the same plan \textbf{(DG3}, \cref{fig:system-diagram-2}l).
% Highlight text from solution to change it to changeable areas (highlighting code itself) (D4)

% In conditions A and B, instructors highlighted the changeable areas in the code itself.
% To allow participants to emphasize the changeable areas in code in PLAID, we implemented the ``add to changeable areas'' button. After selecting the changeable piece of code, clicking on this button highlights the text in a different color and adds it to the box of changeable areas to complete the templated plan design (\textbf{D4}).
% Grouping plans into categories (D4)
% < Multiple selection of the boxes (D4)
% < Naming groups of boxes (D4)
To encourage instructors to think about organizing plans in ways that they would present them to students, PLAID provides an open canvas view for instructors that allows them to arrange plans as they prefer. In addition, PLAID supports a ``grouping'' feature (\cref{fig:system-diagram-2}n), which allows instructors to combine plans with similar goals together into one category (\textbf{DG4}).

% A handful of users postulated each plan as an example question that can be used on assessments. They intended to create multiple variants of the same question for students. They suggested that being able to visualize the different categories would be helpful. Using PLAID, users can select multiple patterns together, add them to a group, and name the group \textbf{(D4)}.  % :(

\subsubsection{System Architecture}
The pipeline to create reference materials is implemented in Python, using the state-of-the-art large language model GPT-4o (Model Version: 2024-05-13). The interface for PLAID is implemented as a web application in Python as a Flask webserver, with an SQLite database. The user-facing interface is implemented using HTML, CSS, and JavaScript, with the canvas interactions realized with the library `\textit{interact.js}'. 



\section{Experiments}
\label{sec:exp}
Following the settings in Section \ref{sec:existing}, we evaluate \textit{NovelSum}'s correlation with the fine-tuned model performance across 53 IT datasets and compare it with previous diversity metrics. Additionally, we conduct a correlation analysis using Qwen-2.5-7B \cite{yang2024qwen2} as the backbone model, alongside previous LLaMA-3-8B experiments, to further demonstrate the metric's effectiveness across different scenarios. Qwen is used for both instruction tuning and deriving semantic embeddings. Due to resource constraints, we run each strategy on Qwen for two rounds, resulting in 25 datasets. 

\subsection{Main Results}

\begin{table*}[!t]
    \centering
    \resizebox{\linewidth}{!}{
    \begin{tabular}{lcccccccccc}
    \toprule
    \multirow{3}*{\textbf{Diversity Metrics}} & \multicolumn{10}{c}{\textbf{Data Selection Strategies}} \\
    \cmidrule(lr){2-11}
    & \multirow{2}*{\textbf{K-means}} & \multirow{2}*{\vtop{\hbox{\textbf{K-Center}}\vspace{1mm}\hbox{\textbf{-Greedy}}}}  & \multirow{2}*{\textbf{QDIT}} & \multirow{2}*{\vtop{\hbox{\textbf{Repr}}\vspace{1mm}\hbox{\textbf{Filter}}}} & \multicolumn{5}{c}{\textbf{Random}} & \multirow{2}{*}{\textbf{Duplicate}} \\ 
    \cmidrule(lr){6-10}
    & & & & & \textbf{$\mathcal{X}^{all}$} & ShareGPT & WizardLM & Alpaca & Dolly &  \\
    \midrule
    \rowcolor{gray!15} \multicolumn{11}{c}{\textit{LLaMA-3-8B}} \\
    Facility Loc. $_{\times10^5}$ & \cellcolor{BLUE!40} 2.99 & \cellcolor{ORANGE!10} 2.73 & \cellcolor{BLUE!40} 2.99 & \cellcolor{BLUE!20} 2.86 & \cellcolor{BLUE!40} 2.99 & \cellcolor{BLUE!0} 2.83 & \cellcolor{BLUE!30} 2.88 & \cellcolor{BLUE!0} 2.83 & \cellcolor{ORANGE!20} 2.59 & \cellcolor{ORANGE!30} 2.52 \\    
    DistSum$_{cosine}$  & \cellcolor{BLUE!30} 0.648 & \cellcolor{BLUE!60} 0.746 & \cellcolor{BLUE!0} 0.629 & \cellcolor{BLUE!50} 0.703 & \cellcolor{BLUE!10} 0.634 & \cellcolor{BLUE!40} 0.656 & \cellcolor{ORANGE!30} 0.578 & \cellcolor{ORANGE!10} 0.605 & \cellcolor{ORANGE!20} 0.603 & \cellcolor{BLUE!10} 0.634 \\
    Vendi Score $_{\times10^7}$ & \cellcolor{BLUE!30} 1.70 & \cellcolor{BLUE!60} 2.53 & \cellcolor{BLUE!10} 1.59 & \cellcolor{BLUE!50} 2.23 & \cellcolor{BLUE!20} 1.61 & \cellcolor{BLUE!30} 1.70 & \cellcolor{ORANGE!10} 1.44 & \cellcolor{ORANGE!20} 1.32 & \cellcolor{ORANGE!10} 1.44 & \cellcolor{ORANGE!30} 0.05 \\
    \textbf{NovelSum (Ours)} & \cellcolor{BLUE!60} 0.693 & \cellcolor{BLUE!50} 0.687 & \cellcolor{BLUE!30} 0.673 & \cellcolor{BLUE!20} 0.671 & \cellcolor{BLUE!40} 0.675 & \cellcolor{BLUE!10} 0.628 & \cellcolor{BLUE!0} 0.591 & \cellcolor{ORANGE!10} 0.572 & \cellcolor{ORANGE!20} 0.50 & \cellcolor{ORANGE!30} 0.461 \\
    \midrule    
    \textbf{Model Performance} & \cellcolor{BLUE!60}1.32 & \cellcolor{BLUE!50}1.31 & \cellcolor{BLUE!40}1.25 & \cellcolor{BLUE!30}1.05 & \cellcolor{BLUE!20}1.20 & \cellcolor{BLUE!10}0.83 & \cellcolor{BLUE!0}0.72 & \cellcolor{ORANGE!10}0.07 & \cellcolor{ORANGE!20}-0.14 & \cellcolor{ORANGE!30}-1.35 \\
    \midrule
    \midrule
    \rowcolor{gray!15} \multicolumn{11}{c}{\textit{Qwen-2.5-7B}} \\
    Facility Loc. $_{\times10^5}$ & \cellcolor{BLUE!40} 3.54 & \cellcolor{ORANGE!30} 3.42 & \cellcolor{BLUE!40} 3.54 & \cellcolor{ORANGE!20} 3.46 & \cellcolor{BLUE!40} 3.54 & \cellcolor{BLUE!30} 3.51 & \cellcolor{BLUE!10} 3.50 & \cellcolor{BLUE!10} 3.50 & \cellcolor{ORANGE!20} 3.46 & \cellcolor{BLUE!0} 3.48 \\ 
    DistSum$_{cosine}$ & \cellcolor{BLUE!30} 0.260 & \cellcolor{BLUE!60} 0.440 & \cellcolor{BLUE!0} 0.223 & \cellcolor{BLUE!50} 0.421 & \cellcolor{BLUE!10} 0.230 & \cellcolor{BLUE!40} 0.285 & \cellcolor{ORANGE!20} 0.211 & \cellcolor{ORANGE!30} 0.189 & \cellcolor{ORANGE!10} 0.221 & \cellcolor{BLUE!20} 0.243 \\
    Vendi Score $_{\times10^6}$ & \cellcolor{ORANGE!10} 1.60 & \cellcolor{BLUE!40} 3.09 & \cellcolor{BLUE!10} 2.60 & \cellcolor{BLUE!60} 7.15 & \cellcolor{ORANGE!20} 1.41 & \cellcolor{BLUE!50} 3.36 & \cellcolor{BLUE!20} 2.65 & \cellcolor{BLUE!0} 1.89 & \cellcolor{BLUE!30} 3.04 & \cellcolor{ORANGE!30} 0.20 \\
    \textbf{NovelSum (Ours)}  & \cellcolor{BLUE!40} 0.440 & \cellcolor{BLUE!60} 0.505 & \cellcolor{BLUE!20} 0.403 & \cellcolor{BLUE!50} 0.495 & \cellcolor{BLUE!30} 0.408 & \cellcolor{BLUE!10} 0.392 & \cellcolor{BLUE!0} 0.349 & \cellcolor{ORANGE!10} 0.336 & \cellcolor{ORANGE!20} 0.320 & \cellcolor{ORANGE!30} 0.309 \\
    \midrule
    \textbf{Model Performance} & \cellcolor{BLUE!30} 1.06 & \cellcolor{BLUE!60} 1.45 & \cellcolor{BLUE!40} 1.23 & \cellcolor{BLUE!50} 1.35 & \cellcolor{BLUE!20} 0.87 & \cellcolor{BLUE!10} 0.07 & \cellcolor{BLUE!0} -0.08 & \cellcolor{ORANGE!10} -0.38 & \cellcolor{ORANGE!30} -0.49 & \cellcolor{ORANGE!20} -0.43 \\
    \bottomrule
    \end{tabular}
    }
    \caption{Measuring the diversity of datasets selected by different strategies using \textit{NovelSum} and baseline metrics. Fine-tuned model performances (Eq. \ref{eq:perf}), based on MT-bench and AlpacaEval, are also included for cross reference. Darker \colorbox{BLUE!60}{blue} shades indicate higher values for each metric, while darker \colorbox{ORANGE!30}{orange} shades indicate lower values. While data selection strategies vary in performance on LLaMA-3-8B and Qwen-2.5-7B, \textit{NovelSum} consistently shows a stronger correlation with model performance than other metrics. More results are provided in Appendix \ref{app:results}.}
    \label{tbl:main}
    \vspace{-4mm}
\end{table*}


\begin{table}[t!]
\centering
\resizebox{\linewidth}{!}{
\begin{tabular}{lcccc}
\toprule
\multirow{2}*{\textbf{Diversity Metrics}} & \multicolumn{3}{c}{\textbf{LLaMA}} & \textbf{Qwen}\\
\cmidrule(lr){2-4} \cmidrule(lr){5-5} 
& \textbf{Pearson} & \textbf{Spearman} & \textbf{Avg.} & \textbf{Avg.} \\
\midrule
TTR & -0.38 & -0.16 & -0.27 & -0.30 \\
vocd-D & -0.43 & -0.17 & -0.30 & -0.31 \\
\midrule
Facility Loc. & 0.86 & 0.69 & 0.77 & 0.08 \\
Entropy & 0.93 & 0.80 & 0.86 & 0.63 \\
\midrule
LDD & 0.61 & 0.75 & 0.68 & 0.60 \\
KNN Distance & 0.59 & 0.80 & 0.70 & 0.67 \\
DistSum$_{cosine}$ & 0.85 & 0.67 & 0.76 & 0.51 \\
Vendi Score & 0.70 & 0.85 & 0.78 & 0.60 \\
DistSum$_{L2}$ & 0.86 & 0.76 & 0.81 & 0.51 \\
Cluster Inertia & 0.81 & 0.85 & 0.83 & 0.76 \\
Radius & 0.87 & 0.81 & 0.84 & 0.48 \\
\midrule
NovelSum & \textbf{0.98} & \textbf{0.95} & \textbf{0.97} & \textbf{0.90} \\
\bottomrule
\end{tabular}
}
\caption{Correlations between different metrics and model performance on LLaMA-3-8B and Qwen-2.5-7B.  “Avg.” denotes the average correlation (Eq. \ref{eq:cor}).}
\label{tbl:correlations}
\vspace{-2mm}
\end{table}

\paragraph{\textit{NovelSum} consistently achieves state-of-the-art correlation with model performance across various data selection strategies, backbone LLMs, and correlation measures.}
Table \ref{tbl:main} presents diversity measurement results on datasets constructed by mainstream data selection methods (based on $\mathcal{X}^{all}$), random selection from various sources, and duplicated samples (with only $m=100$ unique samples). 
Results from multiple runs are averaged for each strategy.
Although these strategies yield varying performance rankings across base models, \textit{NovelSum} consistently tracks changes in IT performance by accurately measuring dataset diversity. For instance, K-means achieves the best performance on LLaMA with the highest NovelSum score, while K-Center-Greedy excels on Qwen, also correlating with the highest NovelSum. Table \ref{tbl:correlations} shows the correlation coefficients between various metrics and model performance for both LLaMA and Qwen experiments, where \textit{NovelSum} achieves state-of-the-art correlation across different models and measures.

\paragraph{\textit{NovelSum} can provide valuable guidance for data engineering practices.}
As a reliable indicator of data diversity, \textit{NovelSum} can assess diversity at both the dataset and sample levels, directly guiding data selection and construction decisions. For example, Table \ref{tbl:main} shows that the combined data source $\mathcal{X}^{all}$ is a better choice for sampling diverse IT data than other sources. Moreover, \textit{NovelSum} can offer insights through comparative analyses, such as: (1) ShareGPT, which collects data from real internet users, exhibits greater diversity than Dolly, which relies on company employees, suggesting that IT samples from diverse sources enhance dataset diversity \cite{wang2024diversity-logD}; (2) In LLaMA experiments, random selection can outperform some mainstream strategies, aligning with prior work \cite{xia2024rethinking,diddee2024chasing}, highlighting gaps in current data selection methods for optimizing diversity.



\subsection{Ablation Study}


\textit{NovelSum} involves several flexible hyperparameters and variations. In our main experiments, \textit{NovelSum} uses cosine distance to compute $d(x_i, x_j)$ in Eq. \ref{eq:dad}. We set $\alpha = 1$, $\beta = 0.5$, and $K = 10$ nearest neighbors in Eq. \ref{eq:pws} and \ref{eq:dad}. Here, we conduct an ablation study to investigate the impact of these settings based on LLaMA-3-8B.

\begin{table}[ht!]
\centering
\resizebox{\linewidth}{!}{
\begin{tabular}{lccc}
\toprule
\textbf{Variants} & \textbf{Pearson} & \textbf{Spearman} & \textbf{Avg.} \\
\midrule
NovelSum & 0.98 & 0.96 & 0.97 \\
\midrule
\hspace{0.10cm} - Use $L2$ distance & 0.97 & 0.83 & 0.90\textsubscript{↓ 0.08} \\
\hspace{0.10cm} - $K=20$ & 0.98 & 0.96 & 0.97\textsubscript{↓ 0.00} \\
\hspace{0.10cm} - $\alpha=0$ (w/o proximity) & 0.79 & 0.31 & 0.55\textsubscript{↓ 0.42} \\
\hspace{0.10cm} - $\alpha=2$ & 0.73 & 0.88 & 0.81\textsubscript{↓ 0.16} \\
\hspace{0.10cm} - $\beta=0$ (w/o density) & 0.92 & 0.89 & 0.91\textsubscript{↓ 0.07} \\
\hspace{0.10cm} - $\beta=1$ & 0.90 & 0.62 & 0.76\textsubscript{↓ 0.21} \\
\bottomrule
\end{tabular}
}
\caption{Ablation Study for \textit{NovelSum}.}
\label{tbl:ablation}
\vspace{-2mm}
\end{table}

In Table \ref{tbl:ablation}, $\alpha=0$ removes the proximity weights, and $\beta=0$ eliminates the density multiplier. We observe that both $\alpha=0$ and $\beta=0$ significantly weaken the correlation, validating the benefits of the proximity-weighted sum and density-aware distance. Additionally, improper values for $\alpha$ and $\beta$ greatly reduce the metric's reliability, highlighting that \textit{NovelSum} strikes a delicate balance between distances and distribution. Replacing cosine distance with Euclidean distance and using more neighbors for density approximation have minimal impact, particularly on Pearson's correlation, demonstrating \textit{NovelSum}'s robustness to different distance measures.






\section{Conclusion}
In this work, we propose a simple yet effective approach, called SMILE, for graph few-shot learning with fewer tasks. Specifically, we introduce a novel dual-level mixup strategy, including within-task and across-task mixup, for enriching the diversity of nodes within each task and the diversity of tasks. Also, we incorporate the degree-based prior information to learn expressive node embeddings. Theoretically, we prove that SMILE effectively enhances the model's generalization performance. Empirically, we conduct extensive experiments on multiple benchmarks and the results suggest that SMILE significantly outperforms other baselines, including both in-domain and cross-domain few-shot settings.

\newpage
\section*{Limitations and Ethical Considerations}

\noindent\textbf{Limitations.} The primary limitation of our work is that it extends only the dataset provided by MUSE and employs DeepSeek-v3 for question generation. 
To mitigate this generalization risk, we have released our code and the generated audit suite, allowing researchers to utilize our framework to create additional audit datasets and evaluate their quality. Meanwhile, this is also our future work to extend our framework to other benchmarks.

\noindent\textbf{Ethical Considerations.} Machine unlearning can be employed to mitigate risks associated with LLMs in terms of privacy, security, bias, and copyright. Our work is dedicated to providing a comprehensive evaluation framework to help researchers better understand the unlearning effectiveness of LLMs, which we believe will have a positive impact on society.


% Bibliography entries for the entire Anthology, followed by custom entries
%\bibliography{anthology,custom}
% Custom bibliography entries only
\bibliography{custom}

\appendix
\subsection{Lloyd-Max Algorithm}
\label{subsec:Lloyd-Max}
For a given quantization bitwidth $B$ and an operand $\bm{X}$, the Lloyd-Max algorithm finds $2^B$ quantization levels $\{\hat{x}_i\}_{i=1}^{2^B}$ such that quantizing $\bm{X}$ by rounding each scalar in $\bm{X}$ to the nearest quantization level minimizes the quantization MSE. 

The algorithm starts with an initial guess of quantization levels and then iteratively computes quantization thresholds $\{\tau_i\}_{i=1}^{2^B-1}$ and updates quantization levels $\{\hat{x}_i\}_{i=1}^{2^B}$. Specifically, at iteration $n$, thresholds are set to the midpoints of the previous iteration's levels:
\begin{align*}
    \tau_i^{(n)}=\frac{\hat{x}_i^{(n-1)}+\hat{x}_{i+1}^{(n-1)}}2 \text{ for } i=1\ldots 2^B-1
\end{align*}
Subsequently, the quantization levels are re-computed as conditional means of the data regions defined by the new thresholds:
\begin{align*}
    \hat{x}_i^{(n)}=\mathbb{E}\left[ \bm{X} \big| \bm{X}\in [\tau_{i-1}^{(n)},\tau_i^{(n)}] \right] \text{ for } i=1\ldots 2^B
\end{align*}
where to satisfy boundary conditions we have $\tau_0=-\infty$ and $\tau_{2^B}=\infty$. The algorithm iterates the above steps until convergence.

Figure \ref{fig:lm_quant} compares the quantization levels of a $7$-bit floating point (E3M3) quantizer (left) to a $7$-bit Lloyd-Max quantizer (right) when quantizing a layer of weights from the GPT3-126M model at a per-tensor granularity. As shown, the Lloyd-Max quantizer achieves substantially lower quantization MSE. Further, Table \ref{tab:FP7_vs_LM7} shows the superior perplexity achieved by Lloyd-Max quantizers for bitwidths of $7$, $6$ and $5$. The difference between the quantizers is clear at 5 bits, where per-tensor FP quantization incurs a drastic and unacceptable increase in perplexity, while Lloyd-Max quantization incurs a much smaller increase. Nevertheless, we note that even the optimal Lloyd-Max quantizer incurs a notable ($\sim 1.5$) increase in perplexity due to the coarse granularity of quantization. 

\begin{figure}[h]
  \centering
  \includegraphics[width=0.7\linewidth]{sections/figures/LM7_FP7.pdf}
  \caption{\small Quantization levels and the corresponding quantization MSE of Floating Point (left) vs Lloyd-Max (right) Quantizers for a layer of weights in the GPT3-126M model.}
  \label{fig:lm_quant}
\end{figure}

\begin{table}[h]\scriptsize
\begin{center}
\caption{\label{tab:FP7_vs_LM7} \small Comparing perplexity (lower is better) achieved by floating point quantizers and Lloyd-Max quantizers on a GPT3-126M model for the Wikitext-103 dataset.}
\begin{tabular}{c|cc|c}
\hline
 \multirow{2}{*}{\textbf{Bitwidth}} & \multicolumn{2}{|c|}{\textbf{Floating-Point Quantizer}} & \textbf{Lloyd-Max Quantizer} \\
 & Best Format & Wikitext-103 Perplexity & Wikitext-103 Perplexity \\
\hline
7 & E3M3 & 18.32 & 18.27 \\
6 & E3M2 & 19.07 & 18.51 \\
5 & E4M0 & 43.89 & 19.71 \\
\hline
\end{tabular}
\end{center}
\end{table}

\subsection{Proof of Local Optimality of LO-BCQ}
\label{subsec:lobcq_opt_proof}
For a given block $\bm{b}_j$, the quantization MSE during LO-BCQ can be empirically evaluated as $\frac{1}{L_b}\lVert \bm{b}_j- \bm{\hat{b}}_j\rVert^2_2$ where $\bm{\hat{b}}_j$ is computed from equation (\ref{eq:clustered_quantization_definition}) as $C_{f(\bm{b}_j)}(\bm{b}_j)$. Further, for a given block cluster $\mathcal{B}_i$, we compute the quantization MSE as $\frac{1}{|\mathcal{B}_{i}|}\sum_{\bm{b} \in \mathcal{B}_{i}} \frac{1}{L_b}\lVert \bm{b}- C_i^{(n)}(\bm{b})\rVert^2_2$. Therefore, at the end of iteration $n$, we evaluate the overall quantization MSE $J^{(n)}$ for a given operand $\bm{X}$ composed of $N_c$ block clusters as:
\begin{align*}
    \label{eq:mse_iter_n}
    J^{(n)} = \frac{1}{N_c} \sum_{i=1}^{N_c} \frac{1}{|\mathcal{B}_{i}^{(n)}|}\sum_{\bm{v} \in \mathcal{B}_{i}^{(n)}} \frac{1}{L_b}\lVert \bm{b}- B_i^{(n)}(\bm{b})\rVert^2_2
\end{align*}

At the end of iteration $n$, the codebooks are updated from $\mathcal{C}^{(n-1)}$ to $\mathcal{C}^{(n)}$. However, the mapping of a given vector $\bm{b}_j$ to quantizers $\mathcal{C}^{(n)}$ remains as  $f^{(n)}(\bm{b}_j)$. At the next iteration, during the vector clustering step, $f^{(n+1)}(\bm{b}_j)$ finds new mapping of $\bm{b}_j$ to updated codebooks $\mathcal{C}^{(n)}$ such that the quantization MSE over the candidate codebooks is minimized. Therefore, we obtain the following result for $\bm{b}_j$:
\begin{align*}
\frac{1}{L_b}\lVert \bm{b}_j - C_{f^{(n+1)}(\bm{b}_j)}^{(n)}(\bm{b}_j)\rVert^2_2 \le \frac{1}{L_b}\lVert \bm{b}_j - C_{f^{(n)}(\bm{b}_j)}^{(n)}(\bm{b}_j)\rVert^2_2
\end{align*}

That is, quantizing $\bm{b}_j$ at the end of the block clustering step of iteration $n+1$ results in lower quantization MSE compared to quantizing at the end of iteration $n$. Since this is true for all $\bm{b} \in \bm{X}$, we assert the following:
\begin{equation}
\begin{split}
\label{eq:mse_ineq_1}
    \tilde{J}^{(n+1)} &= \frac{1}{N_c} \sum_{i=1}^{N_c} \frac{1}{|\mathcal{B}_{i}^{(n+1)}|}\sum_{\bm{b} \in \mathcal{B}_{i}^{(n+1)}} \frac{1}{L_b}\lVert \bm{b} - C_i^{(n)}(b)\rVert^2_2 \le J^{(n)}
\end{split}
\end{equation}
where $\tilde{J}^{(n+1)}$ is the the quantization MSE after the vector clustering step at iteration $n+1$.

Next, during the codebook update step (\ref{eq:quantizers_update}) at iteration $n+1$, the per-cluster codebooks $\mathcal{C}^{(n)}$ are updated to $\mathcal{C}^{(n+1)}$ by invoking the Lloyd-Max algorithm \citep{Lloyd}. We know that for any given value distribution, the Lloyd-Max algorithm minimizes the quantization MSE. Therefore, for a given vector cluster $\mathcal{B}_i$ we obtain the following result:

\begin{equation}
    \frac{1}{|\mathcal{B}_{i}^{(n+1)}|}\sum_{\bm{b} \in \mathcal{B}_{i}^{(n+1)}} \frac{1}{L_b}\lVert \bm{b}- C_i^{(n+1)}(\bm{b})\rVert^2_2 \le \frac{1}{|\mathcal{B}_{i}^{(n+1)}|}\sum_{\bm{b} \in \mathcal{B}_{i}^{(n+1)}} \frac{1}{L_b}\lVert \bm{b}- C_i^{(n)}(\bm{b})\rVert^2_2
\end{equation}

The above equation states that quantizing the given block cluster $\mathcal{B}_i$ after updating the associated codebook from $C_i^{(n)}$ to $C_i^{(n+1)}$ results in lower quantization MSE. Since this is true for all the block clusters, we derive the following result: 
\begin{equation}
\begin{split}
\label{eq:mse_ineq_2}
     J^{(n+1)} &= \frac{1}{N_c} \sum_{i=1}^{N_c} \frac{1}{|\mathcal{B}_{i}^{(n+1)}|}\sum_{\bm{b} \in \mathcal{B}_{i}^{(n+1)}} \frac{1}{L_b}\lVert \bm{b}- C_i^{(n+1)}(\bm{b})\rVert^2_2  \le \tilde{J}^{(n+1)}   
\end{split}
\end{equation}

Following (\ref{eq:mse_ineq_1}) and (\ref{eq:mse_ineq_2}), we find that the quantization MSE is non-increasing for each iteration, that is, $J^{(1)} \ge J^{(2)} \ge J^{(3)} \ge \ldots \ge J^{(M)}$ where $M$ is the maximum number of iterations. 
%Therefore, we can say that if the algorithm converges, then it must be that it has converged to a local minimum. 
\hfill $\blacksquare$


\begin{figure}
    \begin{center}
    \includegraphics[width=0.5\textwidth]{sections//figures/mse_vs_iter.pdf}
    \end{center}
    \caption{\small NMSE vs iterations during LO-BCQ compared to other block quantization proposals}
    \label{fig:nmse_vs_iter}
\end{figure}

Figure \ref{fig:nmse_vs_iter} shows the empirical convergence of LO-BCQ across several block lengths and number of codebooks. Also, the MSE achieved by LO-BCQ is compared to baselines such as MXFP and VSQ. As shown, LO-BCQ converges to a lower MSE than the baselines. Further, we achieve better convergence for larger number of codebooks ($N_c$) and for a smaller block length ($L_b$), both of which increase the bitwidth of BCQ (see Eq \ref{eq:bitwidth_bcq}).


\subsection{Additional Accuracy Results}
%Table \ref{tab:lobcq_config} lists the various LOBCQ configurations and their corresponding bitwidths.
\begin{table}
\setlength{\tabcolsep}{4.75pt}
\begin{center}
\caption{\label{tab:lobcq_config} Various LO-BCQ configurations and their bitwidths.}
\begin{tabular}{|c||c|c|c|c||c|c||c|} 
\hline
 & \multicolumn{4}{|c||}{$L_b=8$} & \multicolumn{2}{|c||}{$L_b=4$} & $L_b=2$ \\
 \hline
 \backslashbox{$L_A$\kern-1em}{\kern-1em$N_c$} & 2 & 4 & 8 & 16 & 2 & 4 & 2 \\
 \hline
 64 & 4.25 & 4.375 & 4.5 & 4.625 & 4.375 & 4.625 & 4.625\\
 \hline
 32 & 4.375 & 4.5 & 4.625& 4.75 & 4.5 & 4.75 & 4.75 \\
 \hline
 16 & 4.625 & 4.75& 4.875 & 5 & 4.75 & 5 & 5 \\
 \hline
\end{tabular}
\end{center}
\end{table}

%\subsection{Perplexity achieved by various LO-BCQ configurations on Wikitext-103 dataset}

\begin{table} \centering
\begin{tabular}{|c||c|c|c|c||c|c||c|} 
\hline
 $L_b \rightarrow$& \multicolumn{4}{c||}{8} & \multicolumn{2}{c||}{4} & 2\\
 \hline
 \backslashbox{$L_A$\kern-1em}{\kern-1em$N_c$} & 2 & 4 & 8 & 16 & 2 & 4 & 2  \\
 %$N_c \rightarrow$ & 2 & 4 & 8 & 16 & 2 & 4 & 2 \\
 \hline
 \hline
 \multicolumn{8}{c}{GPT3-1.3B (FP32 PPL = 9.98)} \\ 
 \hline
 \hline
 64 & 10.40 & 10.23 & 10.17 & 10.15 &  10.28 & 10.18 & 10.19 \\
 \hline
 32 & 10.25 & 10.20 & 10.15 & 10.12 &  10.23 & 10.17 & 10.17 \\
 \hline
 16 & 10.22 & 10.16 & 10.10 & 10.09 &  10.21 & 10.14 & 10.16 \\
 \hline
  \hline
 \multicolumn{8}{c}{GPT3-8B (FP32 PPL = 7.38)} \\ 
 \hline
 \hline
 64 & 7.61 & 7.52 & 7.48 &  7.47 &  7.55 &  7.49 & 7.50 \\
 \hline
 32 & 7.52 & 7.50 & 7.46 &  7.45 &  7.52 &  7.48 & 7.48  \\
 \hline
 16 & 7.51 & 7.48 & 7.44 &  7.44 &  7.51 &  7.49 & 7.47  \\
 \hline
\end{tabular}
\caption{\label{tab:ppl_gpt3_abalation} Wikitext-103 perplexity across GPT3-1.3B and 8B models.}
\end{table}

\begin{table} \centering
\begin{tabular}{|c||c|c|c|c||} 
\hline
 $L_b \rightarrow$& \multicolumn{4}{c||}{8}\\
 \hline
 \backslashbox{$L_A$\kern-1em}{\kern-1em$N_c$} & 2 & 4 & 8 & 16 \\
 %$N_c \rightarrow$ & 2 & 4 & 8 & 16 & 2 & 4 & 2 \\
 \hline
 \hline
 \multicolumn{5}{|c|}{Llama2-7B (FP32 PPL = 5.06)} \\ 
 \hline
 \hline
 64 & 5.31 & 5.26 & 5.19 & 5.18  \\
 \hline
 32 & 5.23 & 5.25 & 5.18 & 5.15  \\
 \hline
 16 & 5.23 & 5.19 & 5.16 & 5.14  \\
 \hline
 \multicolumn{5}{|c|}{Nemotron4-15B (FP32 PPL = 5.87)} \\ 
 \hline
 \hline
 64  & 6.3 & 6.20 & 6.13 & 6.08  \\
 \hline
 32  & 6.24 & 6.12 & 6.07 & 6.03  \\
 \hline
 16  & 6.12 & 6.14 & 6.04 & 6.02  \\
 \hline
 \multicolumn{5}{|c|}{Nemotron4-340B (FP32 PPL = 3.48)} \\ 
 \hline
 \hline
 64 & 3.67 & 3.62 & 3.60 & 3.59 \\
 \hline
 32 & 3.63 & 3.61 & 3.59 & 3.56 \\
 \hline
 16 & 3.61 & 3.58 & 3.57 & 3.55 \\
 \hline
\end{tabular}
\caption{\label{tab:ppl_llama7B_nemo15B} Wikitext-103 perplexity compared to FP32 baseline in Llama2-7B and Nemotron4-15B, 340B models}
\end{table}

%\subsection{Perplexity achieved by various LO-BCQ configurations on MMLU dataset}


\begin{table} \centering
\begin{tabular}{|c||c|c|c|c||c|c|c|c|} 
\hline
 $L_b \rightarrow$& \multicolumn{4}{c||}{8} & \multicolumn{4}{c||}{8}\\
 \hline
 \backslashbox{$L_A$\kern-1em}{\kern-1em$N_c$} & 2 & 4 & 8 & 16 & 2 & 4 & 8 & 16  \\
 %$N_c \rightarrow$ & 2 & 4 & 8 & 16 & 2 & 4 & 2 \\
 \hline
 \hline
 \multicolumn{5}{|c|}{Llama2-7B (FP32 Accuracy = 45.8\%)} & \multicolumn{4}{|c|}{Llama2-70B (FP32 Accuracy = 69.12\%)} \\ 
 \hline
 \hline
 64 & 43.9 & 43.4 & 43.9 & 44.9 & 68.07 & 68.27 & 68.17 & 68.75 \\
 \hline
 32 & 44.5 & 43.8 & 44.9 & 44.5 & 68.37 & 68.51 & 68.35 & 68.27  \\
 \hline
 16 & 43.9 & 42.7 & 44.9 & 45 & 68.12 & 68.77 & 68.31 & 68.59  \\
 \hline
 \hline
 \multicolumn{5}{|c|}{GPT3-22B (FP32 Accuracy = 38.75\%)} & \multicolumn{4}{|c|}{Nemotron4-15B (FP32 Accuracy = 64.3\%)} \\ 
 \hline
 \hline
 64 & 36.71 & 38.85 & 38.13 & 38.92 & 63.17 & 62.36 & 63.72 & 64.09 \\
 \hline
 32 & 37.95 & 38.69 & 39.45 & 38.34 & 64.05 & 62.30 & 63.8 & 64.33  \\
 \hline
 16 & 38.88 & 38.80 & 38.31 & 38.92 & 63.22 & 63.51 & 63.93 & 64.43  \\
 \hline
\end{tabular}
\caption{\label{tab:mmlu_abalation} Accuracy on MMLU dataset across GPT3-22B, Llama2-7B, 70B and Nemotron4-15B models.}
\end{table}


%\subsection{Perplexity achieved by various LO-BCQ configurations on LM evaluation harness}

\begin{table} \centering
\begin{tabular}{|c||c|c|c|c||c|c|c|c|} 
\hline
 $L_b \rightarrow$& \multicolumn{4}{c||}{8} & \multicolumn{4}{c||}{8}\\
 \hline
 \backslashbox{$L_A$\kern-1em}{\kern-1em$N_c$} & 2 & 4 & 8 & 16 & 2 & 4 & 8 & 16  \\
 %$N_c \rightarrow$ & 2 & 4 & 8 & 16 & 2 & 4 & 2 \\
 \hline
 \hline
 \multicolumn{5}{|c|}{Race (FP32 Accuracy = 37.51\%)} & \multicolumn{4}{|c|}{Boolq (FP32 Accuracy = 64.62\%)} \\ 
 \hline
 \hline
 64 & 36.94 & 37.13 & 36.27 & 37.13 & 63.73 & 62.26 & 63.49 & 63.36 \\
 \hline
 32 & 37.03 & 36.36 & 36.08 & 37.03 & 62.54 & 63.51 & 63.49 & 63.55  \\
 \hline
 16 & 37.03 & 37.03 & 36.46 & 37.03 & 61.1 & 63.79 & 63.58 & 63.33  \\
 \hline
 \hline
 \multicolumn{5}{|c|}{Winogrande (FP32 Accuracy = 58.01\%)} & \multicolumn{4}{|c|}{Piqa (FP32 Accuracy = 74.21\%)} \\ 
 \hline
 \hline
 64 & 58.17 & 57.22 & 57.85 & 58.33 & 73.01 & 73.07 & 73.07 & 72.80 \\
 \hline
 32 & 59.12 & 58.09 & 57.85 & 58.41 & 73.01 & 73.94 & 72.74 & 73.18  \\
 \hline
 16 & 57.93 & 58.88 & 57.93 & 58.56 & 73.94 & 72.80 & 73.01 & 73.94  \\
 \hline
\end{tabular}
\caption{\label{tab:mmlu_abalation} Accuracy on LM evaluation harness tasks on GPT3-1.3B model.}
\end{table}

\begin{table} \centering
\begin{tabular}{|c||c|c|c|c||c|c|c|c|} 
\hline
 $L_b \rightarrow$& \multicolumn{4}{c||}{8} & \multicolumn{4}{c||}{8}\\
 \hline
 \backslashbox{$L_A$\kern-1em}{\kern-1em$N_c$} & 2 & 4 & 8 & 16 & 2 & 4 & 8 & 16  \\
 %$N_c \rightarrow$ & 2 & 4 & 8 & 16 & 2 & 4 & 2 \\
 \hline
 \hline
 \multicolumn{5}{|c|}{Race (FP32 Accuracy = 41.34\%)} & \multicolumn{4}{|c|}{Boolq (FP32 Accuracy = 68.32\%)} \\ 
 \hline
 \hline
 64 & 40.48 & 40.10 & 39.43 & 39.90 & 69.20 & 68.41 & 69.45 & 68.56 \\
 \hline
 32 & 39.52 & 39.52 & 40.77 & 39.62 & 68.32 & 67.43 & 68.17 & 69.30  \\
 \hline
 16 & 39.81 & 39.71 & 39.90 & 40.38 & 68.10 & 66.33 & 69.51 & 69.42  \\
 \hline
 \hline
 \multicolumn{5}{|c|}{Winogrande (FP32 Accuracy = 67.88\%)} & \multicolumn{4}{|c|}{Piqa (FP32 Accuracy = 78.78\%)} \\ 
 \hline
 \hline
 64 & 66.85 & 66.61 & 67.72 & 67.88 & 77.31 & 77.42 & 77.75 & 77.64 \\
 \hline
 32 & 67.25 & 67.72 & 67.72 & 67.00 & 77.31 & 77.04 & 77.80 & 77.37  \\
 \hline
 16 & 68.11 & 68.90 & 67.88 & 67.48 & 77.37 & 78.13 & 78.13 & 77.69  \\
 \hline
\end{tabular}
\caption{\label{tab:mmlu_abalation} Accuracy on LM evaluation harness tasks on GPT3-8B model.}
\end{table}

\begin{table} \centering
\begin{tabular}{|c||c|c|c|c||c|c|c|c|} 
\hline
 $L_b \rightarrow$& \multicolumn{4}{c||}{8} & \multicolumn{4}{c||}{8}\\
 \hline
 \backslashbox{$L_A$\kern-1em}{\kern-1em$N_c$} & 2 & 4 & 8 & 16 & 2 & 4 & 8 & 16  \\
 %$N_c \rightarrow$ & 2 & 4 & 8 & 16 & 2 & 4 & 2 \\
 \hline
 \hline
 \multicolumn{5}{|c|}{Race (FP32 Accuracy = 40.67\%)} & \multicolumn{4}{|c|}{Boolq (FP32 Accuracy = 76.54\%)} \\ 
 \hline
 \hline
 64 & 40.48 & 40.10 & 39.43 & 39.90 & 75.41 & 75.11 & 77.09 & 75.66 \\
 \hline
 32 & 39.52 & 39.52 & 40.77 & 39.62 & 76.02 & 76.02 & 75.96 & 75.35  \\
 \hline
 16 & 39.81 & 39.71 & 39.90 & 40.38 & 75.05 & 73.82 & 75.72 & 76.09  \\
 \hline
 \hline
 \multicolumn{5}{|c|}{Winogrande (FP32 Accuracy = 70.64\%)} & \multicolumn{4}{|c|}{Piqa (FP32 Accuracy = 79.16\%)} \\ 
 \hline
 \hline
 64 & 69.14 & 70.17 & 70.17 & 70.56 & 78.24 & 79.00 & 78.62 & 78.73 \\
 \hline
 32 & 70.96 & 69.69 & 71.27 & 69.30 & 78.56 & 79.49 & 79.16 & 78.89  \\
 \hline
 16 & 71.03 & 69.53 & 69.69 & 70.40 & 78.13 & 79.16 & 79.00 & 79.00  \\
 \hline
\end{tabular}
\caption{\label{tab:mmlu_abalation} Accuracy on LM evaluation harness tasks on GPT3-22B model.}
\end{table}

\begin{table} \centering
\begin{tabular}{|c||c|c|c|c||c|c|c|c|} 
\hline
 $L_b \rightarrow$& \multicolumn{4}{c||}{8} & \multicolumn{4}{c||}{8}\\
 \hline
 \backslashbox{$L_A$\kern-1em}{\kern-1em$N_c$} & 2 & 4 & 8 & 16 & 2 & 4 & 8 & 16  \\
 %$N_c \rightarrow$ & 2 & 4 & 8 & 16 & 2 & 4 & 2 \\
 \hline
 \hline
 \multicolumn{5}{|c|}{Race (FP32 Accuracy = 44.4\%)} & \multicolumn{4}{|c|}{Boolq (FP32 Accuracy = 79.29\%)} \\ 
 \hline
 \hline
 64 & 42.49 & 42.51 & 42.58 & 43.45 & 77.58 & 77.37 & 77.43 & 78.1 \\
 \hline
 32 & 43.35 & 42.49 & 43.64 & 43.73 & 77.86 & 75.32 & 77.28 & 77.86  \\
 \hline
 16 & 44.21 & 44.21 & 43.64 & 42.97 & 78.65 & 77 & 76.94 & 77.98  \\
 \hline
 \hline
 \multicolumn{5}{|c|}{Winogrande (FP32 Accuracy = 69.38\%)} & \multicolumn{4}{|c|}{Piqa (FP32 Accuracy = 78.07\%)} \\ 
 \hline
 \hline
 64 & 68.9 & 68.43 & 69.77 & 68.19 & 77.09 & 76.82 & 77.09 & 77.86 \\
 \hline
 32 & 69.38 & 68.51 & 68.82 & 68.90 & 78.07 & 76.71 & 78.07 & 77.86  \\
 \hline
 16 & 69.53 & 67.09 & 69.38 & 68.90 & 77.37 & 77.8 & 77.91 & 77.69  \\
 \hline
\end{tabular}
\caption{\label{tab:mmlu_abalation} Accuracy on LM evaluation harness tasks on Llama2-7B model.}
\end{table}

\begin{table} \centering
\begin{tabular}{|c||c|c|c|c||c|c|c|c|} 
\hline
 $L_b \rightarrow$& \multicolumn{4}{c||}{8} & \multicolumn{4}{c||}{8}\\
 \hline
 \backslashbox{$L_A$\kern-1em}{\kern-1em$N_c$} & 2 & 4 & 8 & 16 & 2 & 4 & 8 & 16  \\
 %$N_c \rightarrow$ & 2 & 4 & 8 & 16 & 2 & 4 & 2 \\
 \hline
 \hline
 \multicolumn{5}{|c|}{Race (FP32 Accuracy = 48.8\%)} & \multicolumn{4}{|c|}{Boolq (FP32 Accuracy = 85.23\%)} \\ 
 \hline
 \hline
 64 & 49.00 & 49.00 & 49.28 & 48.71 & 82.82 & 84.28 & 84.03 & 84.25 \\
 \hline
 32 & 49.57 & 48.52 & 48.33 & 49.28 & 83.85 & 84.46 & 84.31 & 84.93  \\
 \hline
 16 & 49.85 & 49.09 & 49.28 & 48.99 & 85.11 & 84.46 & 84.61 & 83.94  \\
 \hline
 \hline
 \multicolumn{5}{|c|}{Winogrande (FP32 Accuracy = 79.95\%)} & \multicolumn{4}{|c|}{Piqa (FP32 Accuracy = 81.56\%)} \\ 
 \hline
 \hline
 64 & 78.77 & 78.45 & 78.37 & 79.16 & 81.45 & 80.69 & 81.45 & 81.5 \\
 \hline
 32 & 78.45 & 79.01 & 78.69 & 80.66 & 81.56 & 80.58 & 81.18 & 81.34  \\
 \hline
 16 & 79.95 & 79.56 & 79.79 & 79.72 & 81.28 & 81.66 & 81.28 & 80.96  \\
 \hline
\end{tabular}
\caption{\label{tab:mmlu_abalation} Accuracy on LM evaluation harness tasks on Llama2-70B model.}
\end{table}

%\section{MSE Studies}
%\textcolor{red}{TODO}


\subsection{Number Formats and Quantization Method}
\label{subsec:numFormats_quantMethod}
\subsubsection{Integer Format}
An $n$-bit signed integer (INT) is typically represented with a 2s-complement format \citep{yao2022zeroquant,xiao2023smoothquant,dai2021vsq}, where the most significant bit denotes the sign.

\subsubsection{Floating Point Format}
An $n$-bit signed floating point (FP) number $x$ comprises of a 1-bit sign ($x_{\mathrm{sign}}$), $B_m$-bit mantissa ($x_{\mathrm{mant}}$) and $B_e$-bit exponent ($x_{\mathrm{exp}}$) such that $B_m+B_e=n-1$. The associated constant exponent bias ($E_{\mathrm{bias}}$) is computed as $(2^{{B_e}-1}-1)$. We denote this format as $E_{B_e}M_{B_m}$.  

\subsubsection{Quantization Scheme}
\label{subsec:quant_method}
A quantization scheme dictates how a given unquantized tensor is converted to its quantized representation. We consider FP formats for the purpose of illustration. Given an unquantized tensor $\bm{X}$ and an FP format $E_{B_e}M_{B_m}$, we first, we compute the quantization scale factor $s_X$ that maps the maximum absolute value of $\bm{X}$ to the maximum quantization level of the $E_{B_e}M_{B_m}$ format as follows:
\begin{align}
\label{eq:sf}
    s_X = \frac{\mathrm{max}(|\bm{X}|)}{\mathrm{max}(E_{B_e}M_{B_m})}
\end{align}
In the above equation, $|\cdot|$ denotes the absolute value function.

Next, we scale $\bm{X}$ by $s_X$ and quantize it to $\hat{\bm{X}}$ by rounding it to the nearest quantization level of $E_{B_e}M_{B_m}$ as:

\begin{align}
\label{eq:tensor_quant}
    \hat{\bm{X}} = \text{round-to-nearest}\left(\frac{\bm{X}}{s_X}, E_{B_e}M_{B_m}\right)
\end{align}

We perform dynamic max-scaled quantization \citep{wu2020integer}, where the scale factor $s$ for activations is dynamically computed during runtime.

\subsection{Vector Scaled Quantization}
\begin{wrapfigure}{r}{0.35\linewidth}
  \centering
  \includegraphics[width=\linewidth]{sections/figures/vsquant.jpg}
  \caption{\small Vectorwise decomposition for per-vector scaled quantization (VSQ \citep{dai2021vsq}).}
  \label{fig:vsquant}
\end{wrapfigure}
During VSQ \citep{dai2021vsq}, the operand tensors are decomposed into 1D vectors in a hardware friendly manner as shown in Figure \ref{fig:vsquant}. Since the decomposed tensors are used as operands in matrix multiplications during inference, it is beneficial to perform this decomposition along the reduction dimension of the multiplication. The vectorwise quantization is performed similar to tensorwise quantization described in Equations \ref{eq:sf} and \ref{eq:tensor_quant}, where a scale factor $s_v$ is required for each vector $\bm{v}$ that maps the maximum absolute value of that vector to the maximum quantization level. While smaller vector lengths can lead to larger accuracy gains, the associated memory and computational overheads due to the per-vector scale factors increases. To alleviate these overheads, VSQ \citep{dai2021vsq} proposed a second level quantization of the per-vector scale factors to unsigned integers, while MX \citep{rouhani2023shared} quantizes them to integer powers of 2 (denoted as $2^{INT}$).

\subsubsection{MX Format}
The MX format proposed in \citep{rouhani2023microscaling} introduces the concept of sub-block shifting. For every two scalar elements of $b$-bits each, there is a shared exponent bit. The value of this exponent bit is determined through an empirical analysis that targets minimizing quantization MSE. We note that the FP format $E_{1}M_{b}$ is strictly better than MX from an accuracy perspective since it allocates a dedicated exponent bit to each scalar as opposed to sharing it across two scalars. Therefore, we conservatively bound the accuracy of a $b+2$-bit signed MX format with that of a $E_{1}M_{b}$ format in our comparisons. For instance, we use E1M2 format as a proxy for MX4.

\begin{figure}
    \centering
    \includegraphics[width=1\linewidth]{sections//figures/BlockFormats.pdf}
    \caption{\small Comparing LO-BCQ to MX format.}
    \label{fig:block_formats}
\end{figure}

Figure \ref{fig:block_formats} compares our $4$-bit LO-BCQ block format to MX \citep{rouhani2023microscaling}. As shown, both LO-BCQ and MX decompose a given operand tensor into block arrays and each block array into blocks. Similar to MX, we find that per-block quantization ($L_b < L_A$) leads to better accuracy due to increased flexibility. While MX achieves this through per-block $1$-bit micro-scales, we associate a dedicated codebook to each block through a per-block codebook selector. Further, MX quantizes the per-block array scale-factor to E8M0 format without per-tensor scaling. In contrast during LO-BCQ, we find that per-tensor scaling combined with quantization of per-block array scale-factor to E4M3 format results in superior inference accuracy across models. 


\end{document}
