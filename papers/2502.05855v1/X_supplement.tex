\clearpage
\newpage
%\onecolumn
\begin{appendix}




\subsection{Visual Generalization}
Visual generalization is a critical aspect of robot learning.  A well-trained model should not only perform well on in-domain tasks but also generalize to different objects within the same category and to novel scenes. This section presents our visual generalization tests. Specifically, we evaluate shirt folding on a bimanual AgileX and drink pouring using a Franka Emika robot with a dexterous hand. The former task is evaluated without task-specific adaptation, while the latter is trained with 100 demonstrations of the new embodiment. These tasks were also the focus of the experiments presented in Section~\ref{sec:no_post_training} and Section~\ref{sec:new_embodiment}, respectively. For both tasks, we assess visual generalization across two dimensions: novel objects and novel scenes.

\begin{figure}[t]
    \centering
    \includegraphics[width=0.48\textwidth]{images/visual_generalization.pdf}
    \caption{\textbf{Example of visual generalization.} Here lists some visual generalization settings including unseen objects and unseen scenes.}\label{fig:visual_generalization}
\end{figure}

\begin{table*}[t]
  \centering
  \caption{\textbf{Visual Generalizaton for Dex-VLA}. For each evaluation setting, we report the averaged scores across 3 trials.}
  % soap, hang cup,toothpaste,towel
  \label{tbl:visual_generalization}
  \resizebox{0.8\linewidth}{!}{
      \begin{tabular}{c|c|ccc}
        \toprule
        Task / Generalization & Embodiment &Novel Object & Novel Scene & Novel Object \& Scene \\
        \midrule
         Shirt folding &  Bimanual AgileX & 0.78 & 0.78 & 0.56 \\
         Drink pouring & Dexterous hand& 0.83 & 0.67 & 0.67 \\
        \bottomrule
      \end{tabular}
    }
\end{table*}
For shirt folding, we varied the shirt color (while maintaining size) and altered the background and scene.  For drink pouring, we used unseen cups and bottles, also evaluating the task in different scenes and backgrounds.  The results are presented in Table~\ref{tbl:visual_generalization}.  Our experiments demonstrate that Dex-VLA effectively generalizes to novel visual environments.  As shown in the supplementary video, the model successfully handles even challenging cases, such as folding white shirts. The examples are shown in Figure~\ref{fig:visual_generalization}.




\subsection{Training Cost for Stage 1}
As mentioned in Section~\ref{sec:stage1}, training the entire VLA model from scratch results in failure on nearly all tasks.  Therefore, this section compares the training cost of our Stage 1 (training only the diffusion expert) with that of training the entire VLA model. The test reports the number of training epochs completed per hour. The model is trained on 8 Nvidia H100 GPUs. We deliberately keep the same batch size for fair comparison. 

As shown in Table~\ref{tbl:train_cost}, training only the diffusion expert is 2.78 times faster than training the entire VLA model. This is expected, as the VLA model is three times larger than the diffusion expert alone. This highlights that our training strategy is not only effective but also cost-efficient.

\begin{table}[t]
  \centering
  \caption{\textbf{Comparison of training cost for train only diffusion expert versus train entire VLA.} Training cost is measured by the number of training epochs completed per hour.}
  % soap, hang cup,toothpaste,towel
  \label{tbl:train_cost}
  \resizebox{1\linewidth}{!}{
      \begin{tabular}{c|cc}
        \toprule
        Train Method & Train only Diffusion Expert  & Train Entire VLA \\
        \midrule
        Training Cost & 0.89 epoch/hour& 0.32 epoch/hour \\
        \bottomrule
      \end{tabular}
    }
\end{table}

\subsection{Ablation Study}
\label{sec:ablation}

\begin{table}[t]
  \centering
  \caption{\textbf{Ablation results on action head architecture.} DexVLA (UNet) denotes the variant with a smaller UNet-based action head(93M). We reported the average score on shirt folding task.}
  % soap, hang cup,toothpaste,towel
  \label{tbl:ablate_head}
  \resizebox{1\linewidth}{!}{
      \begin{tabular}{c|ccc}
        \toprule
        Models & UNet(93M) & Diffusion Expert(1B) & Averaged Score \\
        \midrule
        DexVLA (UNet) & \checkmark &  & 0.17 \\
        DexVLA &  & \checkmark & 0.92 \\
        \bottomrule
      \end{tabular}
    }
\end{table}

\begin{table}[t]
  \centering
  \caption{\textbf{Ablation study of sub-step reasoning.} The \checkmark in each stage indicates the use of sub-step reasoning data during that stage. We report the average score on the shirt-folding task.}
  % soap, hang cup,toothpaste,towel
  \label{tbl:ablate_substep}
  \resizebox{0.75\linewidth}{!}{
      \begin{tabular}{c|ccc}
        \toprule
         Stage 1 & Stage 2 & Averaged Score \\
        \midrule
          & \checkmark & 0.07 \\
         \checkmark &  & 0 \\
        \midrule
         \checkmark & \checkmark & 0.92 \\
        \bottomrule
      \end{tabular}
    }
\end{table}


\begin{table}[t]
  \centering
  \caption{\textbf{Ablation study of stage 1 training}. To quantify the impact of stage 1 training, we performed an ablation study, training DexVLA without pre-training the diffusion expert. We compared two variants: Stage 2-Only (DexVLA trained only on single-embodiment data) and Stage 1$+$2 (DexVLA trained on combined cross- and single-embodiment data).}
  \label{tbl:ablate_scratch}
  \resizebox{1\linewidth}{!}{
      \begin{tabular}{c|ccc}
        \toprule
        Models / Data & Stage 1  & Stage 2 & Averaged Score \\
        \midrule
        DexVLA (Stage2-Only) &  & \checkmark & 0 \\
        DexVLA (Stage1$+$2) & \checkmark & \checkmark & 0 \\
        \midrule
        DexVLA &  & \checkmark & 0.92 \\
        \bottomrule
      \end{tabular}
    }
\end{table}

Our key contribution is a novel vision-language-action (VLA) model architecture incorporating a diffusion expert—a significantly larger action expert based on a diffusion transformer. We also introduce a new, effective training approach. Crucially, we leverage substep annotations to train both the diffusion expert and the VLA model within a unified framework. This enables our model to handle extremely complex tasks, such as laundry folding, without relying on a high-level policy model. This section investigates these three aspects, addressing the following questions: 1) Does the larger diffusion transformer architecture of the diffusion expert offer advantages over a smaller UNet model? 2) Is the stage 1 training phase essential for our model's performance? 3) How critical is training with substep reasoning for complex tasks?


\textbf{Does our diffusion expert offer advantages over a smaller model?} Our method uses a diffusion expert based on a one-billion-parameter Transformer architecture. While the scaling law, where larger models lead to greater capacity and improved performance and generalization, has been shown in areas like large language models and image generation, its applicability to robot learning is unclear.  Therefore, we investigate whether increasing the action expert's model size provides benefits compared to smaller models.

To this end, we utilize the standard Diffusion Policy architecture with a UNet model as a baseline.  We chose the UNet architecture because the original paper noted training difficulties with their Transformer-based version. This UNet architecture has 93 million parameters.  Following the same training strategy and using the same amount of data, we trained this network. As shown in Table~\ref{tbl:ablate_head}, the UNet-based action expert performs significantly worse than our method, barely completing the shirt folding task with an average scores of 0.17.  Empirically, we observed greater oscillation in the robot's movements with the UNet model compared to our diffusion expert. We hypothesize that the UNet's fewer parameters contribute to interference between different actions in the parameter space, hindering the model's ability to learn the correct actions.

\textbf{Is diffusion expert pretraining necessary?} Our work employs a three-stage training strategy tailored to our proposed architecture. One might question the necessity of separately training the diffusion expert.  Therefore, we conducted experiments under two alternative training regimes.  The first condenses the three stages into two: we initially pre-train the entire~\methodname~model using all our training data and then fine-tune it on embodied-specific data. The second approach directly trains the model from scratch using only the embodied-specific data.  We evaluated both settings on shirt folding, easy bin picking, and easy bussing table tasks, none of which require the post-training stage. The results, shown in Table~\ref{tbl:ablate_scratch}, demonstrate that neither alternative training strategy successfully completed any of the tasks.  Both models failed to train effectively. We hypothesize that the large number of parameters in the diffusion expert makes optimization challenging.  The stage 1 training in our method not only enables the diffusion expert to learn actions but also ``warms up" its parameters, allowing it to better understand complex visual cues and language instructions.

\textbf{Does sub-step reasoning help?} A key strength of our method is its ability to handle extremely long and complex tasks, such as folding randomly crumpled shirts from a basket.  It also enables the model to complete multi-stage tasks like shirt folding and bin picking without requiring post-training.  Therefore, we now examine the importance of substep reasoning. We conducted an ablation study with two setups: 1) The diffusion expert is trained with direct prompting (each task has only one language instruction), while the VLA-diffusion expert is trained with substep reasoning. 2) Both stage 1 and stage 2 are trained with direct prompting data.  The results are shown in Table~\ref{tbl:ablate_substep}. Training the diffusion expert with direct prompting, even for a relatively simple task like shirt folding, reduces the averaged score from 0.92 to 0.07.  Furthermore, removing substep reasoning from both stages results in a complete failure (0 score). This is a significant observation. It suggests that learning long-horizon tasks within a shared parameter space can sometimes lead to conflicts. We hypothesize that substep reasoning allows the model to learn a more disentangled action space, similar to mapping a continuous action space to a discrete one~\cite{wu2024discrete}. This effectively segments the shared parameter space, allocating a smaller set of parameters to each substep~\cite{wang2023fleet, wang2024poco, wang2024scaling}. This avoids parameter conflicts, leading to improved performance and generalization.



\subsection{Evaluation Protocol}
Each task is evaluated across 10 trials and reported averaged scores. For each task, we list the detailed scoring criterion as follows.
\begin{itemize}
    \item \textbf{Lanudary folding (Bimanual AgileX)}: This task is scored out of 4 and we evaluate 5 shirts in total including 2 middle size and 3 small size. We perform two trials for each item, and the items left to be evaluated starting randomly crumpled in a laundry bin (while previously evaluated items start in a fold). One point is given for picking an item out of the bin and putting it on the table. Another point is given for flattening the shirt or shorts. A third point is granted for folding the shirt or shorts. A final point is given for either placing the item in the corner of the table (if it is the first item evaluated), or stacking it onto an existing stack of folded clothes. This evaluation metric is followed $\pi_{0}$. 
    \item \textbf{Shirt folding (Bimanual AgileX)}: This task is scored out of 3. We perform two trials for each item, and the items are flattened on the table. One point is given for double vertically fold. Another point is granted for a double horizontal fold. A final point is given for pushing the folded shirt to the right blank area.
    \item \textbf{Bussing table (Bimanual AgileX)}: This task is scored out of 3-4 where there are 3-4 objects on the table in both \textbf{easy and hard version}. The main difference is the objects that appeared in the hard version are unseen. A point is given for each correctly sorted object.
    \item \textbf{Dryer unloading (Bimanual AgileX)}: This task is scored out of 2 where there are 2 crumpled shirts in the dryer. A point is given for pick up a shirt and place into the hamper.
    \item \textbf{Sorting (Franka with gripper)}: This task is scored out of 5-8 where there are 5-8 objects on the table. There are four kinds of objects in total, a point is given for each correctly sorted object.
    \item \textbf{Drink pouring (Franka with dexterous hand)}: This task is scored out of 2. A point is given for grab the bottle and pour to the cup. Another point is granted for place down the bottle.
    \item \textbf{Bining picking (Franka with gripper)}: This task is scored out of 4-5 where there are 4-5 objects on the table. The main difference is the objects that appeared in the hard version are unseen. A point is given for each correctly picked and placed object.
    \item \textbf{Packing (Bimanual UR5e)}: This task is scored out of 2 where there are 2 objects on the table. A point is given for each correctly picked and placed object.
\end{itemize}



\subsection{More Implementation Details}
This section gives more details on our model, robot setup, and training method. 

\textbf{Robot setup.} Our setup includes two Franka Emika robots: one equipped with a gripper and the other with a robot hand.  Both Franka robots utilize the same camera configuration, consisting of a ZED 2 camera positioned on both the left and right sides, as well as a ZED Mini wrist camera mounted on the robot itself.  Our bimanual UR5e robot uses a single top-mounted Intel RealSense L515 camera and two Intel RealSense 435i cameras attached to the wrists.  Finally, our mobile AgileX platform has two Intel RealSense 435i wrist cameras and a top-mounted Intel RealSense 457 camera.  Although the mobile AgileX image includes a front camera, it was not used during either training or inference.

\textbf{Sub-step reasoning and data acquisition.}
Training with sub-step reasoning is crucial for Dex-VLA to complete long-horizon tasks without a high-level policy model. We present an ablation study on the importance of sub-step reasoning in Section~\ref{sec:ablation}. Acquiring this data presents two key challenges: obtaining language instructions and segmenting videos with corresponding annotations. We address these challenges with the following strategy.

For object-level tasks (e.g., bin picking, sorting, table bussing), object identification is key. We leverage Grounding-Dino and DINOv2 to annotate object bounding boxes and names, along with the gripper's bounding box.  We then calculate the intersection over union between the gripper and object bounding boxes to determine grasp success. For long-horizon single-object tasks (e.g., fold one shirt), the challenge lies in task segmentation. We created a comprehensive list of potential sub-step reasoning, focusing on major steps lasting at least five seconds each to avoid excessive sub-division. We then used Google Gemini 2.0, providing it with the sub-step list, to segment the videos and select the corresponding reasoning from the list. This proved effective and efficient for labeling.  We only manually checked the Stage 3 training data, as this stage requires higher-quality annotations. This annotation strategy makes our approach feasible.

\textbf{Architectural details.} In this section, we provide a full description of the model architecture. Dex-VLA can be split into two parts, VLM backbone originates from Qwen2-VL~\cite{wang2024qwen2} and diffusion expert. We use Qwen2-VL 2B which is powerful and efficient. Regarding our Diffusion Expert, the total number of parameters for this model is 1 billion parameters. We use 32 layers, with the hidden stage of 1280, and a number of heads of 16. During Stage 1, we only pre-train the diffusion expert with random initialized ResNet-50 to process images and off-the-shelf Distilbert~\cite{sanh2019distilbert} to encode language instructions. Because the original diffusion policy model does not support cross-embodiment training, we adopted a multi-head structure similar to Octo~\cite{octo}. Each embodiment is assigned a unique MLP head. The diffusion expert is trained using the similar settings of our Dex-VLA. In particular, we use the image resolution of 320 $\times$ 240, with three camera views. Each image is processed independently to a ResNet-50. We use the strategy as in RT-1~\cite{brohan2022rt-1} to initialize the FiLM layers. 





\end{appendix}