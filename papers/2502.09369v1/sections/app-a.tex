%%%%%%%%%%%%%%%%%%%%%%%%%%%%%%%%%%%%%%%%%%%%%%%%%%%%%%%%%%%%%%%%%%%%%%%%%%%%%%%
\section{Proof of Proposition 1}\label{app:a}
%%%%%%%%%%%%%%%%%%%%%%%%%%%%%%%%%%%%%%%%%%%%%%%%%%%%%%%%%%%%%%%%%%%%%%%%%%%%%%%

%==============================================================================
\thmequivalence*
%==============================================================================

\textit{Proof}.
%
The first subset in Expression \ref{eqn:subseteq} is immediate from Definitions \ref{eqn:bellman} and \ref{eqn:transitions}.
%
The second subset can be seen as follows:
%
% PVE-PROP-1i
%
Fix $\tilde{\pi}\in\Pie(\pi^{*},\Tau,\mathcal{Q})$,
so for any $\tau\in\Tau$
and $Q\in\mathcal{Q}$
we have the following,
%
\begin{equation}
\mathbb{B}_{\pi^{*},\tau}Q
=
\mathbb{B}_{\tilde{\pi},\tau}Q
\end{equation}
%
But $\mathbb{B}_{\pi^{*},\tau}Q$ and $\mathbb{B}_{\tilde{\pi},\tau}Q$ are also in $\mathcal{Q}$ from the closure assumption, so we can repeatedly apply the Bellman operators $\mathbb{B}_{\pi^{*},\tau}$ and $\mathbb{B}_{\tilde{\pi},\tau}$ on the left and right hand sides for a total of $T$ times, therefore
%
\begin{equation}
\mathbb{B}_{\pi^{*},\tau}^{1}
\hspace{-8pt}\raisebox{1pt}{\scalebox{0.8}{$\circ\dots\circ$}}\hspace{1pt}
\mathbb{B}_{\pi^{*},\tau}^{T}Q
=
\mathbb{B}_{\tilde{\pi},\tau}^{1}
\hspace{-7pt}\raisebox{1pt}{\scalebox{0.8}{$\circ\dots\circ$}}\hspace{1pt}
\mathbb{B}_{\tilde{\pi},\tau}^{T}Q
\end{equation}
%
so $\tilde{\pi}\in\Pie^{T}(\pi^{*},\Tau,\mathcal{Q})$.
%
Next, the equality in Expression \ref{eqn:equality} is obvious.
%
% VE-PROP-2
%
Finally, 
%
% PVE-PROP-3
%
the proper subset in Expression \ref{eqn:subset} can be shown by picking the following model policy $\tilde{\pi}$ as proof (note that $\tilde{\pi}\not\in\Pie(\pi^{*})$):
%
\begin{equation}
\tilde{\pi}((u_{\para},u_{\bot})|x)
:=
\mathbbl{1}_{\{u_{\bot}=\tilde{u}_{\bot}\}}
\pi^{*}(u_{\para}|x)
\end{equation}
%
First, note that the value function $Q_{\pi^{*},\tau}^{1:T}$ for the true policy $\pi^{*}$ profile satisfies the following recursion:
%
\begin{align}
Q_{\pi^{*},\tau}^{t}(x,(&u_{\para},u_{\bot}))
\\
&
=
(\mathbb{B}_{\pi^{*},\tau}^{t}Q_{\pi^{*},\tau}^{t+1})
(x,(u_{\para},u_{\bot}))
\\
&
=
\mathbb{E}_{x'\sim\tau(\cdot|x,(u_{\para},u_{\bot}))}
\mathbb{E}_{(u_{\para}',u_{\bot}')\sim\pi^{*}(\cdot|x')}
Q_{\pi^{*},\tau}^{t+1}(x',(u_{\para}',u_{\bot}'))
\\
&
=
\mathbb{E}_{x'\sim\tau(\cdot|x,(u_{\para},u_{\bot}))}
\textstyle\int_{\mathcal{U}_{\para}}
\textstyle\int_{\mathcal{U}_{\bot}}
\pi^{*}((u_{\para}',u_{\bot}')|x')
Q_{\pi^{*},\tau}^{t+1}(x',(u_{\para}',u_{\bot}'))
du_{\para}'du_{\bot}'
\\
&
=
\mathbb{E}_{x'\sim\tau(\cdot|x,(u_{\para},u_{\bot}))}
\textstyle\int_{\mathcal{U}_{\para}}
\textstyle\int_{\mathcal{U}_{\bot}}
\pi^{*}(u_{\para}'|x')
\pi^{*}(u_{\bot}'|x')
Q_{\pi^{*},\tau}^{t+1}(x',(u_{\para}',u_{\bot}'))
du_{\para}'du_{\bot}'
\\
&
=
\mathbb{E}_{x'\sim\tau(\cdot|x,(u_{\para},u_{\bot}))}
\textstyle\int_{\mathcal{U}_{\para}}
\pi^{*}(u_{\para}'|x')
Q_{\pi^{*},\tau}^{t+1}(x',u_{\para}')
du_{\para}'
\\
&
=
\mathbb{E}_{x'\sim\tau(\cdot|x,(u_{\para},u_{\bot}))}
\mathbb{E}_{u_{\para}'\sim\pi^{*}(\cdot|x')}
Q_{\pi^{*},\tau}^{t+1}(x',u_{\para}')
\end{align}
for $t\in\{1,\dots,T-1\}$, and $Q_{\pi^{*},\tau}^{T}(x,u)=g(x,\theta)$.
%
Likewise, the value function $Q_{\pi^{*},\tau}^{1:T}$ for the model policy profile $\tilde{\pi}$ satisfies the following recursion:
%
\begin{align}\textstyle
Q^{t}(x,(&u_{\para},u_{\bot}))
\\
&
=
(\mathbb{B}_{\tilde{\pi},\tau}^{t}Q^{t+1})
(x,(u_{\para},u_{\bot}))
\\
&
=
\mathbb{E}_{x'\sim\tau(\cdot|x,(u_{\para},u_{\bot}))}
\mathbb{E}_{(u_{\para}',u_{\bot}')\sim\tilde{\pi}(\cdot|x')}
Q^{t+1}(x',(u_{\para}',u_{\bot}'))
\\
&
=
\mathbb{E}_{x'\sim\tau(\cdot|x,(u_{\para},u_{\bot}))}
\textstyle\int_{\mathcal{U}_{\para}}
\textstyle\int_{\mathcal{U}_{\bot}}
\tilde{\pi}((u_{\para}',u_{\bot}')|x')
Q^{t+1}(x',(u_{\para}',u_{\bot}'))
du_{\para}'du_{\bot}'
\\
&
=
\mathbb{E}_{x'\sim\tau(\cdot|x,(u_{\para},u_{\bot}))}
\textstyle\int_{\mathcal{U}_{\para}}
\textstyle\int_{\mathcal{U}_{\bot}}
\mathbbl{1}_{\{u_{\bot}'=\tilde{u}_{\bot}\}}
\pi^{*}(u_{\para}'|x')
Q^{t+1}(x',(u_{\para}',u_{\bot}'))
du_{\para}'du_{\bot}'
\\
&
=
\mathbb{E}_{x'\sim\tau(\cdot|x,(u_{\para},u_{\bot}))}
\textstyle\int_{\mathcal{U}_{\para}}
\pi^{*}(u_{\para}'|x')
Q^{t+1}(x',(u_{\para}',\tilde{u}_{\bot}))
du_{\para}'
\\
&
=
\mathbb{E}_{x'\sim\tau(\cdot|x,(u_{\para},u_{\bot}))}
\mathbb{E}_{u_{\para}'\sim\pi^{*}(\cdot|x')}
Q^{t+1}(x',(u_{\para}',\tilde{u}_{\bot}))
\end{align}
%
for $t\in\{1,\dots,T-1\}$, and $Q^{T}(x,u)=g(x,\theta)$.
%
But $Q_{\pi^{*},\tau}^{1:T}$ is a solution to this recursion, and as solutions to backward recursions are unique, we have that $Q_{\tilde{\pi},\tau}^{1:T}=Q_{\pi^{*},\tau}^{1:T}$,
hence $\tilde{\pi}\in\Pie^{T}(\pi^{*},\Tau,\mathcal{Q})$.

Next, we show $\tilde{\pi}\not\in\Pie(\pi^{*},\Tau)$:
%
Pick $Q(x,(u_{\para},u_{\bot})):=\mathbbl{1}_{\{u_{\bot}\neq\tilde{u}_{\bot}\}}$. Then we have that
%
\begin{align}
(\mathbb{B}_{\pi^{*},\tau}Q)
(x,(u_{\para},u_{\bot}))
&
=
\mathbb{E}_{x'\sim\tau(\cdot|x,(u_{\para},u_{\bot}))}
\mathbb{E}_{(u_{\para}',u_{\bot}')\sim\pi^{*}(\cdot|x')}
Q(x',(u_{\para}',u_{\bot}'))
\\
&
=
\mathbb{E}_{x'\sim\tau(\cdot|x,(u_{\para},u_{\bot}))}
\mathbb{P}(u_{\bot}'\neq\tilde{u}_{\bot}|x',\pi^{*})
\\
&
=
\mathbb{P}(u_{\bot}'\neq\tilde{u}_{\bot}|x,(u_{\para},u_{\bot}),\pi^{*},\tau)
\end{align}

\vspace{-1.5em}
and likewise,
\vspace{-0.75em}

\begin{align}
(\mathbb{B}_{\tilde{\pi},\tau}Q)
(x,(u_{\para},u_{\bot}))
&
=
\mathbb{E}_{x'\sim\tau(\cdot|x,(u_{\para},u_{\bot}))}
\mathbb{E}_{u_{\para}'\sim\pi^{*}(\cdot|x')}
Q(x',(u_{\para}',\tilde{u}_{\bot}))
\\
&
=
\mathbb{E}_{x'\sim\tau(\cdot|x,(u_{\para},u_{\bot}))}
\mathbb{P}(\tilde{u}_{\bot}\neq\tilde{u}_{\bot})
\\
&
=
0
\end{align}
%
Suppose it were true that $\tilde{\pi}\in\Pie(\pi^{*},\Tau,\mathcal{Q})$, so that for all $Q$ we have that
$
(\mathbb{B}_{\pi^{*},\tau}Q)(x,(u_{\para},u_{\bot}))
=
(\mathbb{B}_{\tilde{\pi},\tau}Q)(x,(u_{\para},u_{\bot}))
$,
which means
$
\mathbb{P}(u_{\bot}'=\tilde{u}_{\bot}|x,(u_{\para},u_{\bot}),\pi^{*},\tau)
=
1
$.
%
But this is only possible if $\mathcal{U}_{\bot}$\pix$=$\pix$\{\tilde{u}_{\bot}\}$ hence $\text{card}(\mathcal{U}_{\bot})$\pix$=$\pix$1$, which is a contradiction with the premise that $\text{card}(\mathcal{U}_{\bot})$\pix$>$\pix$1$.
%
\QED