%%%%%%%%%%%%%%%%%%%%%%%%%%%%%%%%%%%%%%%%%%%%%%%%%%%%%%%%%%%%%%%%%%%%%%%%%%%%%%%
\section{Collective Decision-Making}\label{sec:2}
%%%%%%%%%%%%%%%%%%%%%%%%%%%%%%%%%%%%%%%%%%%%%%%%%%%%%%%%%%%%%%%%%%%%%%%%%%%%%%%

Consider a general setting for collective decision-making,
where interactions between a group of individuals are mediated by a \textit{decision mechanism},
yielding a process that produces a final outcome.
%
By way of preliminaries,
we first formalize the notion of a \textit{social choice function},
which is an abstract mapping from preferences to outcomes that is implemented by a mechanism:

%==============================================================================
\begin{redefinition}[restate=defsocial,name=Social Choice Function]\upshape\label{def:social}
%==============================================================================
%
Let $N:=\{1,\dots,n\}$ denote a set of \textit{participants},
who are engaged in the process of making a collective decision.
%
Denote with $\Omega$ the space of \textit{outcomes},
from which the participants are required to make a final selection $\omega\in\Omega$.
%
Denote with $\theta_{i}\in\Theta_{i}$ the \textit{type} characterizing each participant $i\in N$,
which captures their preferences over different outcomes.
%
Moreover, let $\theta:=(\theta_{i})_{i\in N}\in\Theta:=\Theta_{1}\times\dots\times\Theta_{n}$ indicate the \textit{type profile} of all participants.
%
Then a \textit{social choice function} $h$ is a mapping from the space of type profiles into the space of outcomes,
%
\begin{equation}
h:\Theta\rightarrow\Omega
\end{equation}
%
In line with related fields,
``participants'' may also be referred to as ``agents'' and ``players'',
%
and an ``outcome'' may be synonymous with an ``alternative'' and a ``collective decision''.
%
\EOD
\end{redefinition}

We operate in the standard setting for discrete-time Markov decision processes.
%
Let $x\in\mathcal{X}$ denote the \textit{state} variable:
%
While we keep notation simple,
$x$ may play the role of ``contexts'' and ``histories'' that capture all observations prior to the current time step.
%
Let $u_{i}\in\mathcal{U}_{i}$ denote the \textit{action} variable for $i\in N$,
and let $u:=(u_{i})_{i\in N}\in\mathcal{U}:=\mathcal{U}_{1}\times\dots\times\mathcal{U}_{n}$ indicate the \textit{action profile} for all participants.
%
In our setting,
$u$ will largely play the role of ``utterances'' in language space.
%
Denote with $T$ the finite length of each decision episode.
%
The outcome of each episode is precisely the terminal state $\omega:=x^{T}$.

%==============================================================================
\begin{redefinition}[restate=defmechanism,name=Decision Mechanism]\upshape\label{def:mechanism}
%==============================================================================
%
Each participant $i\in N$ is associated with a \textit{behavior} denoted with $\pi_{i}\in\Pie_{i}\subseteq\Delta(\mathcal{U}_{i})^{\mathcal{X}}$,
and let $\pi:=(\pi_{i})_{i\in N}\in\Pie:=\Pie_{1}\times\dots\times\Pie_{n}$ give the \textit{behavior profile} for all participants.
%
A \textit{decision mechanism} $\tau$ is a mapping from states and action profiles to next states,
%
\begin{equation}
\tau\in\Tau\subseteq\Delta(\mathcal{X})^{\mathcal{X}\times\mathcal{U}}
\end{equation}
%
and an \textit{outcome function} $f$ maps behavior profiles and mechanisms into distributions over outcomes.
%
In our Markov decision process setting,
the outcome function is simply the result of ``rolling out'' an episode of the interactive process between $\pi$ and $\tau$ to arrive at (a distribution over) terminal states,
%
\begin{equation}
f:\Pie\times\Tau\rightarrow\Delta(\Omega)
\end{equation}
%
A ``behavior'' may synonymously be referred to as a ``policy'' or ``strategy'',
%
and a ``mechanism'' may be referred to as a ``game'' or ``environment''.
%
Our setting is in general non-stationary: We may use superscripts $t\in\{1,\dots,T\}$ to indicate time steps, but omit them unless explicitly required.
%
\EOD
\end{redefinition}

Finally,
a mechanism $\tau$ is said to \textit{implement} a social choice function $h$ when $h(\theta)=f(\pi, \tau)$
for all $\pi\in\Pi$.
%
Both $\pi$, $\tau$ may depend on $\theta$, though not required.
%
Often,
mechanisms are designed with an \textit{optimization objective} in mind.
%
For instance,
let each participant be associated with a \textit{payoff function},
%
\begin{equation}
g_{i}:\Omega\times\Theta_{i}\rightarrow\mathbb{R}
\end{equation}
%
such that $g_{i}(\omega,\theta_{i})$ gives the payoff that participant $i\in N$ enjoys from the collective decision $\omega$.
%
As before, for convenience, let $g:=(g_{i})_{i\in N}$ and (with some abuse of notation) write $g:\Omega\times\Theta\rightarrow\mathbb{R}^{n}$.
%
Then a ``utilitarian'' social choice function is implemented as
$h(\theta)=f(\pi, \tau^{*})$,
%
by the mechanism:
%
\begin{equation}
\tau^{*}
\in
\underset{\tau\in\Tau}{\text{arg~max}}~
\mathbb{E}_{\omega\sim f(\pi,\tau)}
G(\omega,\theta)
\qquad
\text{where}
\qquad
G(\omega,\theta)
:=
\frac{1}{n}
\sum_{i=1}^{n}g_{i}(\omega,\theta_{i})
\label{eqn:utilitarian}
\end{equation}
%
%
Note that a ``payoff function'' is often interchangeable with a ``reward function'' or ``utility function''.
%
In this work, we use the scenario of \textit{consensus-finding} among individuals as our illustrative example.

\newpage
%==============================================================================
\textbf{Case Study (Consensus-Finding)}~
%==============================================================================
\dayum{
In this setting, a group of human participants shares their thoughts on a debated topic using natural language. Through interaction with a mediation mechanism, they aim to reach a consensus on the matter \cite{bakker2022fine,tessler2023submit}. Specifically, an episode begins with a set of $N$ participants observing a \textit{question} of interest, sampled from a corpus of questions. For instance, this may be: ``Should we adopt a universal basic income (UBI) policy?'' (see Figure \ref{fig:diagram}). Each participant expresses their \textit{opinion} in written text. Next, the mediator mechanism processes these opinions and outputs a \textit{draft consensus} statement. Each participant then provides their own \textit{critique} of the draft consensus in written text. Note that participants do not observe each other's opinions or critiques. Finally, the mediator mechanism processes these critiques to produce a \textit{revised consensus} statement. The episode ends with participants viewing the revised consensus, followed by an individual demographics \textit{questionnaire}.
}

\begin{figure}[H]
\centering
\vspace{-0.25em}
\includegraphics[width=0.99\columnwidth]{figures/DA_Phase1_Diagram.pdf}
\vspace{-2.25em}
\caption{\textit{Consensus-Finding}. Observe that this is an instance of a collective decision-making setting:}
\vspace{-1em}
\label{fig:diagram}
\end{figure}
%
\begin{tcolorbox}[rounded corners, left=8pt, right=7pt, top=14pt, bottom=7pt, boxsep=0pt, grow to left by=2pt, grow to right by=2pt, boxrule=0.75pt, colback=black!02, colframe=black!0]
\begin{itemize}[leftmargin=1em,labelsep=0.45em]
\itemsep-0.75pt
\vspace{-0.75em}
\item $N$ is the set of participants, typically 3--5 for each episode;
\item $\Omega=\mathcal{X}$ is the space of revised consensus statements;
\item $x^{1}\in\mathcal{X}$ is the \textit{question} of interest;
\item $u^{1}_{i}\in\mathcal{U}$ is the \textit{opinion} of participant $i\in N$ on the question;
\item $x^{2}\in\mathcal{X}$ is the \textit{draft consensus} statement taking those into account;
\item $u^{2}_{i}\in\mathcal{U}$ is the \textit{critique} of participant $i\in N$ on the draft consensus;
\item $x^{3}\in\mathcal{X}$ is the \textit{revised consensus} statement taking those into account;
\item $u^{3}_{i}\in\mathcal{U}$ is the dummy action that terminates the episode with a demographics questionnaire;
\item $\theta\in\Theta$ is the participants' preference profile over different consensus opinions;
\item $\pi_{i}(\cdot|x)$ is the utterance behavior of participant $i\in N$ in response to $x\in\mathcal{X}$; and
\item $\tau(\cdot|x,u)$ is the mediation mechanism's transition function at $x\in\mathcal{X}$, $u\in\mathcal{U}$.
\end{itemize}
\end{tcolorbox}
\vspace{0.5em}

\dayum{
%==============================================================================
\textbf{Remark (Mediator Mechanism)}~
%==============================================================================
Two comments deserve brief mention.
%
Firstly, for added context, the mediator mechanism $\tau$ is a 70B-parameter Chinchilla \cite{hoffmann2022training} model, fine-tuned as described in \cite{tessler2023submit} to best assist humans in collectively producing written consensus statements with maximal approval (viz. Equation \ref{eqn:utilitarian}). In particular, these statements are preferred more strongly than several high-quality baselines, including individual human opinions and consensus statements from human mediators.\footnote{The mediator mechanism was not trained to endorse a specific perspective or persuade others of any viewpoint; rather, it was trained purely to produce consensus statements based on the utterances provided by participants.}
%
Secondly, we want to emphasize that for our purposes, we treat the mechanism as a black-box transition function. Our goal is to investigate how \textit{digital representatives} can be trained to operate on behalf of humans in collective decision-making settings, with the consensus-finding scenario as our primary example. In this sense, the inner details of the mechanism are not of importance to us, save for recognizing that $\Tau$ (and $\Pie$, as we shall see) are families of pre-trained and fine-tuned large language models.
}