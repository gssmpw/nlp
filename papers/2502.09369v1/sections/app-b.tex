%%%%%%%%%%%%%%%%%%%%%%%%%%%%%%%%%%%%%%%%%%%%%%%%%%%%%%%%%%%%%%%%%%%%%%%%%%%%%%%
\section{Further Experiment Detail}\label{app:bonus}
%%%%%%%%%%%%%%%%%%%%%%%%%%%%%%%%%%%%%%%%%%%%%%%%%%%%%%%%%%%%%%%%%%%%%%%%%%%%%%%

\textbf{Variants of Digital Representatives}~ As a reminder, a digital representative $\hat{\pi}_{i}$ is a language model trained to produce a critique based on a question, participant $i$'s opinion, a draft consensus (all of which form the \textit{base} information), and optionally any supplemental information about participant $i$. The main text discusses various information sources within the task pipeline, which we outline in Table \ref{tab:prompt-info} along with text examples from the dataset. We experimented with different combinations of conditioning information along various axes of variation to create diverse datasets for fine-tuning. In Figure \ref{fig:dr-variants} we show a handful of variants of digital representatives when conditioning on various data in addition to the base information. Guided by the log-likelihood of the ground truth critiques (in the validation set), we empirically determined the most effective additional information, which included the participant $i$'s opinions and critiques on other debate questions discussed within the same data collection pipeline. An input prompt example from the dataset used to fine-tune the digital representatives as presented in the main text is illustrated in text box \ref{box:prompt-example}.

\vspace{-0.75em}
\begin{table}[H]\small
\newcolumntype{A}{>{\arraybackslash}m{ 0.6cm}}
\newcolumntype{B}{>{\arraybackslash}m{ 5cm}}
\newcolumntype{C}{>{\arraybackslash}m{10.2cm}}
\caption{\textit{Possible prompt information}. A list of sections (with examples) that may be included in the prompt for fine-tuning digital representatives.}
\label{tab:prompt-info}
\begin{center}
\begin{adjustbox}{max width=1.005\textwidth}
\begin{tabular}{A|B|C}
\toprule
\textbf{Alias}
&
\textbf{Description}
&
\textbf{Example section as part of the prompt}
\\
\toprule
\textbf{Base}
&
Current question under debate
&
Current question under debate: Should the government spend more on science and technology research?
\\
\midrule
\textbf{Base}
&
Opinion of participant $i$ about the question under debate
&
Opinion of the participant about the question under debate: I think its really important that the government spends more on Science and Technology. If you think about the problems we have as a society and as a planet - the answers lie with Science and Technology.
\\
\midrule
\textbf{Base}
&
Draft consensus statement
&
Group draft consensus statement (to be critiqued): The government should spend more on science and technology research as this would lead to better technology and medical discoveries to help the world.
\\
\midrule
\textbf{D}
&
Demographic profile of participant $i$ (based on questionnaire)
&
You represent a participant with the following demographic profile. Gender Identity: Male. Age (in decades): 50 to 59. Region: West Midlands. Ethnicity: White. Party voted for in the most recent General Election: Labour. Highest level of education completed: Higher or secondary or further education (A-levels, BTEC, etc.). Religious affiliation: No religion. Approximate household income: £20k to £40k per year. Immigration status in the UK: British Passport holder.
\\
\midrule
\textbf{O}
&
Participant $i$’s opinions about other questions (1 to 3)
&
Question 1 from a past debate: Should the government spend more on science and technology research?

Opinion of the participant about question 1: I believe that the government should be allocating more money for science and technology research.
\\
\midrule
\textbf{C}
&
Participant $i$’s critiques of other questions (1 to 3)
&
Question 1 from a past debate: Should the government spend more on science and technology research?

Critique of the participant about question 1: I totally agree. Science and Technology can help us out of our current problems and make life better for everyone. 
\\
\midrule
\textbf{P}
&
Participant $i$’s position score (as text) about other questions (1 to 3)
&
Question 1 from a past debate: Should the government spend more on science and technology research?

Position of the participant about question 1: agree.

Question 2 from a past debate: Should students be required to pass a literacy and numeracy test before graduating from primary school?

Position of the participant about question 2: strongly disagree.
\\
\bottomrule
\end{tabular}
\end{adjustbox}
\end{center}
\vspace{-1em}
\end{table}


\begin{figure}[H]
\centering
\includegraphics[width=0.96\columnwidth]{figures/variants-loglik.png}
\caption{\textit{Variants of Digital Representatives}. Mean log-likelihood of ground-truth critiques from human participants (from the validation set), evaluated under various digital representatives. All these DRs have 1B parameters and were fine-tuned on datasets conditioned on diverse additional information, as indicated on the x-axis (refer to Table \ref{tab:prompt-info} for details and examples). The optimal variant (\textit{Base+O+C}) incorporates participant $i$'s opinions and critiques to other questions. This variant performs very similarly to its counterpart that additionally includes demographic information (\textit{Base+D+O+C}). However, we opted for the former for simplicity. Note that variants based solely on demographics (\textit{Base+D} or position scoring (\textit{Base+P}) perform much worse. This suggests that integrating participant-specific few-shot information enhances both task- and self-consistency.}
\label{fig:dr-variants}
\end{figure}


\begin{figure}[H]
\centerline{
\includegraphics[width=0.5\columnwidth]{figures/autoeval-other-critique.png}
\includegraphics[width=0.5\columnwidth]{figures/autoeval-predicted-other-critique.png}
}
\centerline{
\includegraphics[width=0.5\columnwidth]{figures/autoeval-irrelevant-critique.png}
\includegraphics[width=0.5\columnwidth]{figures/autoeval-predicted-irrelevant-critique.png}
}
\centerline{
\includegraphics[width=0.5\columnwidth]{figures/autoeval-contrary-critique.png}
\includegraphics[width=0.5\columnwidth]{figures/autoeval-predicted-contrary-critique.png}
}
\centerline{
\includegraphics[width=0.5\columnwidth]{figures/autoeval-own-opinion-critique.png}
\includegraphics[width=0.5\columnwidth]{figures/autoeval-predicted-other-opinion-critique.png}
}
\caption{\textit{Autorater Critique Evaluations: Ablations}. The autorater's win-rate of sampled critiques against a custom baseline, as specified by the grey bar within each plot (included as a sanity check, expected around 50\%). The ground truth (\textit{One's Own Critique}) is also included to set the performance ceiling. The main take-away is that the 30B fine-tuned DR (green) closely matches the performance of ground truth critiques (golden), suggesting its ability to learn both task-specific aspects (e.g., critique-ness, topic) and participant-specific nuances (e.g., individual preferences, style). A similar trend is observed for the 1B fine-tuned DR, albeit to a lesser extent.}
\label{fig:autoeval-critiques}
\end{figure}

~

\begin{story}[ht]
 \framedbox[0.95\textwidth]{
A group of participants is engaged in debates regarding important questions. They have been tasked with coming up with a consensus for a new debate question.

This participant has shared their opinions on other open debate questions, as outlined below.

\textbf{Question 1 from a past debate:} Should students be required to pass a literacy and numeracy test before graduating from primary school?

\textbf{Opinion of the participant about question 1:} I'm against this, children should not be under pressure at such a young age and should concentrate more on being children. Obviously monitor literacy and numeracy but to make them pass a test is unnecessary. Some people develop academically as they get older, so a test at this young age is meaningless. Children will have enough pressure as adults, let them enjoy being children.

\textbf{Critique of the participant about question 1:} Totally agree one hundred per cent with this, pupils should not have to pass a test before graduating but they should be monitored and helped if necessary. Let children be children.
Question 2 from a past debate: Should we require people to pay a fixed tax on any amount of wealth they own above £10 million?

\textbf{Opinion of the participant about question 2:} There is a growing wealth gap in the UK and the world and a tax like this would help narrow that gap, help the country pay its way and help to invest in important infrastructure. However we do not want to set this fixed amount of tax too high that it puts wealthy individuals off from living here, investing here and spending money. I do think though £10 million is the right figure to apply this tax.

\textbf{Critique of the participant about question 2:} Like the group opinion, it aligns well with what my response was. Totally agree with the money helping to fund the public sector and poorest in society. My concern was that the tax should not be set to high and 1\% seems a fair figure. That 1\% would still generate a substanial sum for the country.

Please offer a critique that accurately represents the participant's opinion on a new debate question in relation to the group draft consensus statement.

\textbf{Current question under debate:} Should the government spend more on science and technology research?

\textbf{Opinion of the participant about the question under debate:} I think its really important that the government spends more on Science and Technology. If you think about the problems we have as a society and as a planet - the answers lie with Science and Technology. We have just had a serious pandemic and we were helped by scientists coming up quickly with a vaccination. One of the ways out of the Climate Change crisis is through science and technology coming up with solutions as we as individuals do not want to change pur habits. More money going into Science and Technology will help us counter these crises and future problems. If we don't put the money in, the next pandemic maybe more severe and costly.

\textbf{Group draft consensus statement (to be critiqued):} The government should spend more on science and technology research as this would lead to better technology and medical discoveries to help the world.

\textbf{Critique of the participant:}}
 \caption[]{Prompt example for the digital representative used in the main text (\textit{Base+O+C}).}
 \label{box:prompt-example}
\end{story}

\begin{figure}[H]
\centering
\includegraphics[width=0.96\columnwidth]{figures/autoeval-baseline-candidate.png}
\caption{\textit{Autorater Consensus Evaluations: Ablations}. The autorater's win-rate of generated consensuses (based on single participant substitutions with DRs) against the baseline (based solely on ground truth critiques). For comparison, various baseline consensus statements (as specified on the x-axis) are included from the original dataset. This ablation demonstrates that DRs contribute to generating consensuses that are competitively similar to human-selected consensuses (i.e. red bar).}
\label{fig:consensus-baselines}
\end{figure}