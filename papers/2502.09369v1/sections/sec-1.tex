\section{Introduction}\label{sec:1}
\vspace{-0.25em}

\dayum{
Collective decision-making is a hallmark of intelligent behavior \cite{bose2017collective,mann2018collective},
and is ubiquitous in economic, social, and political spheres of society \cite{narahari2014game,leach2016freedom}.
%
Consider any process in which a group of individuals interactively select a preferred outcome from a universe of alternatives:
%
For instance, consider a diverse group of humans communicating in a mediated environment, seeking to arrive at a consensus opinion on a topic of debate \cite{bakker2022fine,tessler2023submit}.
%
In this context, ``representation'' is the activity of making an (otherwise absent) individual's preferences present in the process through participation by a proxy agent---i.e. their ``representative'' \cite{dovi2006political}.
%
To this end, learned models of human behavior have the potential to fill this role, with practical implications for scenario studies and mechanism design.
%
For instance, it can facilitate personalized and detailed simulations of collective interactions, allowing iterative refinement of mechanisms before real-world deployment.
%
The key question is: \textit{What makes a good representative}?
}

\dayum{
%==============================================================================
\textbf{Simulation for Representation}~
%==============================================================================
In this work, we explore the possibility of training \textit{language agents} to behave in the capacity of representatives of human agents, appropriately expressing the preferences of those individuals whom they stand for.
%
On the one hand,
this has natural connections to existing work in simulating human behavior,
such as
learning from demonstrations
\cite{le2016smooth,yue2018imitation,osa2018imitation,huyuk2021explaining},
generating synthetic data,
\cite{veselovsky2023generating,tang2023does,chan2021medkit, dosovitskiy2017carla},
and creating plausible simulacra of social agents
\cite{park2022social,aher2023using,harding2023ai,argyle2023out,horton2023large,park2023generative}.
%
On the other hand,
unlike such prior work,
our objective in simulating human behavior is specifically for ``representation''. This entails unique requirements:
(a) grounding in the context of \textit{collective interaction},
(b) attending to the granularity of \textit{individual fidelity}, and
(c) operating in the high-dimensional domain of \textit{language space}.
%
Specifically, in this work, we are exploring what representation means, how to measure representativity, and whether language agents can be trained to act as representatives on behalf of humans.
}

\dayum{
%==============================================================================
\textbf{Contributions}~
%==============================================================================
We make three key contributions in this investigation. First, we formalize the setting of \textit{collective decision-making} as the episodic process of interaction between a group of agents and a decision mechanism (Section \ref{sec:2}).
%
Building upon this, our second contribution is to characterize the problem of \textit{digital representation} as the simulation of an agent's behavior so as to yield equivalent outcomes when interacting with the mechanism (Section \ref{sec:3}).
%
Finally, we conduct an empirical case study in the setting of \textit{consensus-finding} in natural language among diverse humans, and demonstrate the feasibility of fine-tuning large language models to act as such digital representatives (Section \ref{sec:4}).
}