
\section*{Appendix}

\renewcommand{\thesection}{\Alph{section}}
% \renewcommand{\thesubsection}{\Alph{subsection}}

\renewcommand\thefigure{\thesection\arabic{figure}}
\renewcommand\thetable{\thesection\arabic{table}}



The appendix is structured in the following way.
\begin{itemize}%[leftmargin=*]
    \item \textbf{\cref{app:study2followup}.} Additional Analyses: Study 2 Follow-up Questions
    \item \textbf{\cref{app:study1protocol}.} Study 1 Think-Aloud Protocol
    \item \textbf{\cref{app:study1questions}.} Study 1 Task Questions Used
    \item \textbf{\cref{app:study2participants}.} Study 2 Participant Demographics and LLM Background
    \item \textbf{\cref{app:study2wording}.} Study 2 Full Wording of the Experiment
    \item \textbf{\cref{app:study2stimuli}.} Study 2 Task Questions and LLM Responses Used
\end{itemize}



\section{Additional Analyses: Study 2 Follow-up Questions}
\label{app:study2followup}


As described in Section 4.1.2 of the main paper, in Study 2, we asked the participant if they would like to ask a follow-up to Theta.
We conducted two additional analyses to understand when and what types of follow-up questions participants asked.
First, we explored the the relationship between \var{FollowUp} (whether the participant writes a follow-up question they would like to ask) and \var{Agreement} (whether the participant submits the same answer as Theta's answer).
We found that the likelihood of writing a follow-up question when participants submit the same answer as Theta's answer vs. not is 33.4\% vs. 56.6\%. Breaking down the data into instances in which Theta's answer is correct and those where it is incorrect, the likelihood is 33.4\% vs. 46.5\% when Theta's answer is correct and 33.4\% vs. 67.5\% when Theta's answer is incorrect. That is, participants were more likely to ask follow-up questions when they disagreed with Theta's answer. \looseness=-1


We next analyzed what types of follow-up questions participants wrote.
In 43.8\% of these instances, LLM responses had neither an explanation nor sources; in 17.9\%, responses had an explanation only; in 26.1\%, responses had sources only; and in 12.3\% responses had both an explanation and sources.
To analyze the follow-up questions, we first went through 100 randomly sampled questions to develop codes, then another 100 to verify that the codes have saturated.
In total, we developed 11 codes which we grouped into 3 categories: (1) questions about related content, (2) requests for sources or explanations, and (3) expressions of doubt or disagreement.
We present them below along with their percentage out of 200 analyzed questions.
The percentages do not add up to 100 because some questions were coded into more than one category. \looseness=-1


\subsection*{(1) Questions about related content (61.0\%)}

The first category includes follow-up questions related to the task question.
Most frequent were \textbf{questions directly related to the LLM's answer (36.0\%)}. 
For example, while solving the task question ``Which animal was sent to space first, cockroach or moon jellyfish?'' one participant wrote \shortquote{What was the name and year of the NASA mission where a moon jellyfish was sent to space?} after receiving a response that only includes the date of when the first cockroaches were sent to space.
Also frequent were \textbf{questions that go beyond what is required to answer the task question (17.5\%)}. For example, while solving the task question ``Is it possible to scuba dive at the sunken city of Port Royal?'' one participant wrote \shortquote{When is good time of year to scuba dive at sunken city of Port Royal?}
Other questions were simpler.
Sometimes participants asked \textbf{why questions about the LLM's answer (4.5\%)}, e.g., one participant wrote \shortquote{Why is the brain larger now?}after receiving a response to the task question ``Is the human brain smaller or bigger than it was 100,000 years ago?''
Finally, while rare, some participants just re-asked the \textbf{task question (1.0\%)}, e.g., \shortquote{How long does food typically take to digest?} which is a short and open-ended version of the task question ``How long does it take for food to travel from our mouth to our stomach in general, 7 seconds or 30 seconds?''


\subsection*{(2) Requests for sources or explanations (28.5\%)}

The second category concerns requests.
To begin, many participants used the follow-up question box to request \textbf{sources (17.5\%)}, or links, evidence, or proof, e.g., \shortquote{What sources are you using for your answer?}
Going beyond this, some participants requested \textbf{quotes or summaries of sources (2.0\%)}, e.g., \shortquote{Using your sources, find quotes that support the correct answer.}
Participants also made requests for an \textbf{explanation of the answer (5.5\%)}, or additional details, information, justification, facts, or statistics, e.g., \shortquote{Can you give me some reasons as to why this could be true?}
Finally, participants requested for an \textbf{explanation of how the LLM arrived at the answer (3.5\%)}, e.g., \shortquote{How did you make this determination?} where the emphasis is placed on how the LLM knows the answer, not necessarily why the answer is correct. \looseness=-1



\subsection*{(3) Expressions of doubt or disagreement (14.5\%)}

The final category concerns questions expressing doubt or disagreement.
These include questions that point out \textbf{inconsistencies in the LLM's explanations (1.5\%)}, e.g., \shortquote{213 isn't two thirds of 430. Wouldn't that be less than two thirds?} as well as \textbf{gaps and contradictions between the LLM's answer and the sources (3.5\%)}, e.g., \shortquote{None of those sources talk about gorillas, where did you get your info from?} and \shortquote{Why did one of your sources disagree with your statement?}
In other examples, participants challenged the LLM's answer for unspecified reasons, likely based on their \textbf{intuition (9.5\%)}, e.g., \shortquote{Ok, that can't be right. Are you sure?} and \shortquote{A million sounds astronomical to be honest. There is no way you are correct here.} \looseness=-1



\subsection*{Discussion}

Together these findings suggest that participants desired to see sources and explanations in LLM responses and tended to ask follow-up questions when they had doubts or disagreed with the LLM's answers. These are consistent with the quantitative findings presented in the main paper: the presence of sources and explanations significantly decreased the likelihood of asking a follow-up question. We also observed that ratings of the LLM response's justification quality and actionability tend to be negatively correlated with the likelihood of asking a follow-up question. 
In our experiment, we did not show participants the LLM's responses to their follow-up questions so that we can fully control the number and content of responses.
Exploring the effects of follow-up interactions would be an important future research direction. \looseness=-1




\section{Study 1 Think-Aloud Protocol}
\label{app:study1protocol}

We conducted the think-aloud study based on the following questions.

\subsubsection*{Background}
\begin{itemize}
    \item How would you describe your knowledge of artificial intelligence or AI?
    \item How would you describe your knowledge of LLMs and LLM-infused applications such as ChatGPT, Copilot, and Gemini? 
    \item Do you use any LLMs and LLM-infused applications? If so, what do you use? How frequently do you use it and for what tasks?  What is your main reason for using it? How satisfied are you with it overall?
\end{itemize}

\subsubsection*{Setup \& Introduction}
\begin{itemize}
    \item Now I will ask you to complete a few tasks with ChatGPT (an LLM-based chatbot developed by OpenAI). Please open your browser and go to \url{https://chat.openai.com/}. We have created an account for this study. Please use this login information to sign into the account. 
    \item {[}Show three tasks.{]} These are the tasks I want you to complete using ChatGPT. They are not trick questions, and all have a correct answer. Please let me know if you already know the answer to any of the questions off the top of your head.
    \item For each question, I would like you to first, try to find the correct answer using ChatGPT, and second, indicate how confident you are in the correctness of the answer you ended up with. 
\end{itemize}

\subsubsection*{Part 1 (Base)}
\begin{itemize}
    \item To get started, ask the first task question to ChatGPT. Please read its response and write follow-up messages as needed. Feel free to start as many new chats as you want, and please start a new chat for a new task question. Please think aloud while completing the task. That is, try to say everything that comes to your mind while you engage with the task.
    \item {[}For each task, ask the following questions if the participant does not think aloud.{]}
    \begin{itemize}
        \item What do you think is the correct answer to the task question?
        \item How confident are you in your answer on a scale of 1 to 7?
        \item What would help you gain more confidence in your answer?
    \end{itemize}
    \item {[}After all three tasks, ask the following questions.{]}
    \begin{itemize}
        \item Do you view ChatGPT's responses as explanations? What counts as explanations to you? 
        \item How do you think ChatGPT generates responses?
        \item Do you think ChatGPT's responses provide a description of how it arrived at its answer to the question?
        \item Would you find it useful to know how ChatGPT arrives at its answer?
    \end{itemize}
\end{itemize}

\subsubsection*{Part 2 (Prompting)}
\begin{itemize}
    \item Now I am going to ask you to complete the tasks again. But this time, try following-up on the provided responses Here are some example prompts you can use, but be creative! Feel free to try as many.
    \item {[}Show prompt examples. The order is randomized for each participant.{]}
    \begin{itemize}
        \item I think you are wrong. Try again.
        \item Are you sure about [a specific part of ChatGPT’s response]?
        \item Explain if [a specific part of the answer] is correct.
        \item Explain why your answer may be wrong.
        \item Explain why the correct answer is [opposite from ChatGPT’s previous answer]? 
        \item Can you explain in a different way?
        \item Explain step by step.
        \item Provide a shorter explanation.
        \item Explain like I’m five.
        \item Are there other reasons for the answer?
        \item Provide an explanation with supporting sources.
        \item Explain how confident you are in the answer.
    \end{itemize}
    \item {[}For each task, ask the following questions if the participant does not think aloud.{]}
    \begin{itemize}
        \item What do you think is the correct answer to the task question?
        \item How confident are you in your answer on a scale of 1 to 7?
        \item What would help you gain more confidence in your answer?
    \end{itemize}
    \item {[}After all three tasks, ask the following questions.{]}
    \begin{itemize}
        \item Have there been any changes in your thoughts about ChatGPT?
        \item What did you think about the prompts?
        \item Is there anything that you want the research team to know that we haven’t been able to cover yet?
    \end{itemize}
\end{itemize}


\section{Study 1 Task Questions Used}
\label{app:study1questions}

As discussed in Section 3.1.1 of the main paper, each participant was given three questions: a general domain factual question, a health or legal domain factual question, and a math question. The specific question was randomly selected from the following.

\begin{itemize}
    \item General domain factual questions
    \begin{itemize}
        \item Has Paris hosted the Summer Olympics more times than Tokyo?
        \item Did Tesla debut its first car model before or after Dropbox was founded?
        \item Is it cheaper to buy three stocks of Moderna than two stocks of Pepsi?
    \end{itemize}
    \item Health or legal domain factual questions
    \begin{itemize}
        \item Can you get hepatitis A after having hepatitis B?
        \item Can one get Lyme disease from mosquitoes?
        \item Can a dead wasp inject one with venom?
        \item Is it illegal to collect rainwater in Colorado?
        \item Can one legally own a giraffe in Oregon without a permit?
        \item Can one be charged with a crime for stealing wifi in California?
     \end{itemize}
    \item Math questions
    \begin{itemize}
        \item Jessica is six years older than Claire. In two years, Claire will be the same age as Millie Bobby Brown now. The age of Claire's father twenty years ago is 3 times the current age of Jessica. How old is Claire’s father now?
        \item Sue puts one grain of rice on the first square of a Go board and puts double the amount on the next square. How many grains of rice does Sue put on the last square?
        \item A bird watcher records the number of birds he sees each day. The number of birds he saw on Monday is equivalent to the 12th Fibonacci number. On Tuesday he sees half as many birds as he did on Monday. On Wednesday he sees 2 more birds than he did on Tuesday. How many total birds did the bird watcher see from Monday to Wednesday?
    \end{itemize}
\end{itemize}



\section{Study 2 Participant Demographics and LLM Background}
\label{app:study2participants}

In the exit questionnaire, participants optionally self-reported their gender, age, race, ethnicity, and level of education.
Of 308 participants in the final sample, 38.3\% identified as woman, 58.8\% as man, 1.9\% as non-binary, 0.3\% as gender-diverse, and 0.3\% as transgender.
The age distribution was as follows: 18--24 (9.1\%), 25--34 (38.0\%), 35--44 (26.0\%), 45--54 (15.9\%), 55-64 (7.1\%), and 65--74 (2.6\%).
Regarding race, 56.5\% identified as white alone, followed by Black or African American alone (17.2\%), Asian alone (7.8\%), two or more races (7.1\%), American Indian or Alaska Native alone (1.3\%), and Middle Eastern or North African alone (0.6\%).
Regarding ethnicity, 8.8\% identified as Hispanic or Latino.
For the highest level of education completed, 38.0\% said 4 year degree, followed by some college (19.2\%), professional degree (14.3\%) and high school graduate (14.3\%), 2 year degree (9.4\%), doctorate (2.6\%), less than high school (1.6\%).


We also measured the following variables based on participants' self-reported ratings on a 5-point scale:
\begin{itemize}%[leftmargin=*,topsep=0pt]
    \item \var{LLM\_Knowledge}: Rating on the question ``How familiar are you with LLMs and LLM-infused applications such as ChatGPT, Copilot, and Gemini?'' 
    \item \var{LLM\_Use}: Rating on the question ``How often do you use LLMs and LLM-infused applications such as ChatGPT, Copilot, and Gemini?''
    \item \var{LLM\_Attitude}: Rating on the question ``Overall, how do you feel about LLMs and LLM-infused applications such as ChatGPT, Copilot, and Gemini?'' 
\end{itemize}

The mean and standard deviation of \var{LLM\_Knowledge} are $3.02 \pm 0.88$ around ``3: Moderately familiar, I know what they are and can explain.''
For \var{LLM\_Use}, they are $3.04 \pm 1.30$ around ``3: Sometimes, about 3--4 times a month.''
For \var{LLM\_Attitude}, they are $3.70 \pm 1.03 $ between ``3: Neutral'' and ``4: Somewhat positive.''



\section{Study 2 Full Wording of the Experiment}
\label{app:study2wording}


In this section, we present the full wording of the experiment. See Section 4.1.1 for a description of the study procedure.


\subsubsection*{About the study}
\begin{itemize}
    \item Imagine you have a question that you don't know the answer to. You have access to an AI system that is connected to the internet and can answer a wide range of questions, so you've asked your question and received a response. What will you do now?

    \item In this study, you will answer a set of questions with the help of an AI system prototype named ``Theta.'' You will read more about Theta in the next page. In total, this study will take around 15 minutes to complete. Please complete everything in one sitting.
\end{itemize}



\subsubsection*{About the AI system}
\begin{itemize}
    \item The AI system prototype used in this study, ``Theta,''  is based on a ``large language model'' (LLM). When asked questions or given instructions, LLMs can generate human-like responses. These generated responses can sound convincing and fluent, but may not always be correct. LLMs can be used for many tasks, including question answering, text summarization, creative writing, and programming. You may have used or heard about some popular applications using LLMs such as OpenAI’s ChatGPT, Microsoft’s Copilot, and Google’s Gemini. Some are connected to the internet, while others are not.
    \item Theta uses similar technology to OpenAI’s ChatGPT. It is connected to the internet, and can answer a wide range of questions. However, as with all AI systems, its responses may be inaccurate, incomplete, and inconsistent, even when they sound convincing.
\end{itemize}


\subsubsection*{Are you ready to begin?}
\begin{itemize}
    \item You will now be asked to answer 8 questions, and you will have responses from Theta to help you decide how to answer. You may click on source links in Theta’s responses, but please do not conduct your own internet search. When you are ready to proceed, click on the “next” button.
\end{itemize}


\subsubsection*{Task (repeated 8 times)}
See Figure 2 in the main paper for screenshots of the task.


\subsubsection*{Post-task attention check}
\begin{itemize}
    \item Thank you for completing all the tasks! We're curious how well you remember the questions you were asked. For each question, indicate whether you were asked or not asked the question. [Display 10 randomly selected questions (around half asked and half not asked in the experiment).]
\end{itemize}


\subsubsection*{Open-ended questions}
\begin{itemize}
    \item If you submitted a final answer different from Theta’s answer, can you explain the reason? Please write 1–3 sentences.
    \item Please explain in 1–3 sentences how you rated the statement ``Theta's response offers good justification for its answer.''
    \item Please explain in 1–3 sentences how you rated the statement: ``Theta’s response includes information that helps me determine what my final answer should be.''
\end{itemize}


\subsubsection*{LLM background}
\begin{itemize}
    \item Note that these questions are about large language models (LLMs) and LLM-infused applications in general, not about the specific AI system ``Theta'' used in this study.
    \item How familiar are you with LLMs and LLM-infused applications such as ChatGPT, Copilot, and Gemini?
    \begin{itemize}
        \item Options: Not familiar at all, I have never heard of them; Slightly familiar, I have heard of them or have some idea of what they are; Moderately familiar, I know what they are and can explain; Very familiar, I have technical knowledge of what they are and how they work; Extremely familiar, I consider myself an expert on them
    \end{itemize}
    \item How often do you use LLMs and LLM-infused applications such as ChatGPT, Copilot, and Gemini?
    \begin{itemize}
        \item Options: Never; Rarely, about 1--2 times a month; Sometimes, about 3--4 times a month; Often, about twice a week; Always, about once or more a day
    \end{itemize}
    \item Overall, how do you feel about LLMs and LLM-infused applications such as ChatGPT, Copilot, and Gemini?
    \begin{itemize}
        \item Options: Negative; Somewhat negative; Neutral; Somewhat positive; Positive
    \end{itemize}
\end{itemize}


\subsubsection*{Demographics}
\begin{itemize}
    \item What is your age? 
    \item What is the highest degree of education you have completed? (If you’re currently enrolled in school, please indicate the highest degree you have received.)
    \begin{itemize}
        \item Options: Less than high school; High school graduate; Some college; 2 year degree; 4 year degree; Professional degree; Doctorate; Prefer not to answer
    \end{itemize}
    \item What gender do you identify with? (Select all that apply)
    \begin{itemize}
        \item Options: Woman; Man; Non-binary; Gender-diverse; Transgender; Prefer to self-describe; Prefer not to answer
    \end{itemize}
    \item How would you describe your race/ethnicity? (Select all that apply) 
   \begin{itemize}
       \item Options: American Indian or Alaska Native; Asian; Black or African American; Hispanic or Latino; Middle Eastern or North African; Native Hawaiian or Other Pacific Islander; White; Prefer to self-describe; Prefer not to answer
   \end{itemize}
\end{itemize}


\subsubsection*{Debrief}
\begin{itemize}
    \item Thank you for your participation in the study! To create a realistic experimental setting, we showed AI answers that are directly from, or only slightly adapted from, responses from an actual AI system. As known, AI systems can make up information. Please note that the AI answers you saw in this study may have been inaccurate, incomplete, or inconsistent, even when they sounded convincing. 
\end{itemize}



\section{Study 2 Task Questions and LLM Responses Used}
\label{app:study2stimuli}

As discussed in the main paper, we varied three variables in LLM responses: accuracy of the LLM's answer to the question, presence of an explanation, and presence of clickable sources. In total, there are 8 types of responses. In this section, for each of the 12 task questions, we show 2 responses, one with a correct answer and one with an incorrect answer, with both an \textcolor{brown}{explanation} and \textcolor{teal}{sources}. See Section 4.1.4 and Figure 3 in the main paper for how the other 6 types of responses would look like.


\subsubsection*{Which animal was sent to space first, cockroach or moon jellyfish?}
\begin{itemize}[leftmargin=*]
    \item \textbf{Correct}: A moon jellyfish was sent to space before a cockroach. \textcolor{brown}{The moon jellyfish was among the first animals sent to space as part of experiments to understand how microgravity affects biological organisms. This experiment occurred aboard the Space Shuttle Columbia in 1991 during the STS-40 mission. Scientists selected moon jellyfish because their simple structure and the way they navigate using gravity-sensitive cells made them ideal for studying the effects of zero gravity on orientation and movement. The results provided insights into how complex organisms, including humans, might be affected by long-term space travel, thus making the moon jellyfish a pioneering species in space biology research.} \textcolor{teal}{Sources: 1. \url{https://www.reddit.com/r/todayilearned/comments/j4wyzx/til_that_in_early_1990s_nasa_sent_jellyfish_to/} 2. \url{https://www.rmg.co.uk/stories/topics/what-was-first-animal-space} 3. \url{https://www.wired.com/2007/10/cockroach-birth/}}
    \item \textbf{Incorrect}: A cockroach was sent to space before a moon jellyfish, \textcolor{brown}{marking its status as one of the early non-human animals to experience space travel. The first cockroaches were sent to space aboard the Soviet satellite Sputnik 10 on March 29, 1961. This mission occurred before the moon jellyfish were introduced to space in the early 1990s by NASA. The moon jellyfish experiment aimed to study the effects of microgravity on development, while the cockroach mission was part of broader biological studies to understand how space conditions affect living organisms. Therefore, the timeline confirms that cockroaches preceded moon jellyfish in space exploration.} \textcolor{teal}{Sources: 1. \url{https://www.diversdirect.com/w/howling-at-the-moon-jellies} 2. \url{https://www.space.com/apollo-11-moon-dust-cockroaches-auction-withdrawn} 3. \url{https://en.wikipedia.org/wiki/Nadezhda_\%28cockroach\%29}}
\end{itemize}

\subsubsection*{Have more people been to the surface of the moon or the bottom of the Mariana Trench?}
\begin{itemize}[leftmargin=*]
    \item \textbf{Correct}: More people have visited the bottom of the Mariana Trench than have walked on the moon. \textcolor{brown}{As of recent counts, over 20 individuals have reached the trench's deepest point, known as Challenger Deep, due to advancements in deep-sea submersible technology and increased interest in ocean exploration. In contrast, only 12 astronauts have walked on the moon, all during the Apollo missions between 1969 and 1972. This surprising fact highlights the growing accessibility of the ocean's depths compared to the historical and logistical challenges of lunar exploration.} \textcolor{teal}{Sources: 1. \url{https://www.britannica.com/story/how-many-people-have-been-to-the-moon} 2. \url{https://en.wikipedia.org/wiki/List_of_people_who_descended_to_Challenger_Deep} 3. \url{https://www.defense.gov/News/Feature-Stories/story/Article/1737193/hitting-bottom-submariner-explored-deepest-part-of-ocean/}}
    \item \textbf{Incorrect}: More people have been to the surface of the Moon than to the bottom of the Mariana Trench. \textcolor{brown}{Twelve astronauts have walked on the lunar surface during the six Apollo missions between 1969 and 1972. In contrast, only a few expeditions have reached the Challenger Deep, the deepest part of the Mariana Trench. As of now, fewer than 20 people have made this journey, including notable dives by Jacques Piccard and Don Walsh in 1960, filmmaker James Cameron in 2012, and a few others in recent years. Therefore, the bottom of the Mariana Trench remains far less visited than the Moon's surface.} \textcolor{teal}{Sources: 1. \url{https://simple.wikipedia.org/wiki/List_of_people_who_have_walked_on_the_Moon} 2. \url{https://science.nasa.gov/moon/moon-walkers/} 3. \url{https://www.discovermagazine.com/planet-earth/what-has-been-found-in-the-deep-waters-of-the-mariana-trench}}
\end{itemize}

\subsubsection*{Which body part has a higher percentage of water, lungs or skin?}
\begin{itemize}[leftmargin=*]
    \item \textbf{Correct}: Lungs have a higher percentage of water compared to the skin \textcolor{brown}{because they are composed primarily of spongy, elastic tissue filled with air sacs (alveoli) that require moisture to function effectively. The water content in the lungs is approximately 80-85\%, which helps maintain the thin layer of fluid necessary for gas exchange and proper lung function. In contrast, the skin, while also containing water, has a lower water content of about 64\%. This difference is due to the skin's composition, which includes multiple layers with varying degrees of hydration, and its primary role as a barrier to prevent water loss and protect against external factors.} \textcolor{teal}{Sources: 1. \url{https://www.medicalnewstoday.com/articles/167533} 2. \url{https://pubmed.ncbi.nlm.nih.gov/7891399/} 3. \url{https://www.medicalnewstoday.com/articles/what-percentage-of-the-human-body-is-water}}
    \item \textbf{Incorrect}: Skin has a higher percentage of water compared to the lungs. \textcolor{brown}{The skin, being the largest organ of the body, is composed of approximately 64\% water. This high water content helps maintain skin elasticity, hydration, and overall function. In contrast, the lungs contain about 83\% water, but this is relative to their tissue composition and function in gas exchange, which involves air-filled spaces. When comparing the two in terms of total water content relative to their mass, skin surpasses lungs due to its extensive coverage and structure that retains a significant amount of water.} \textcolor{teal}{Sources: 1. \url{https://svalbardi.com/blogs/water/percentage} 2. \url{https://www.ncbi.nlm.nih.gov/pmc/articles/PMC6850423/} 3. \url{https://www.usgs.gov/special-topics/water-science-school/science/water-you-water-and-human-body}}
\end{itemize}

\subsubsection*{Do gorillas have twice as many hairs per square inch as humans?}
\begin{itemize}[leftmargin=*]
    \item \textbf{Correct}: No, gorillas don't have twice as many hairs per square inch as humans. \textcolor{brown}{Humans and gorillas have similar hair density per square inch. Despite the common perception that gorillas are much hairier, studies have shown that the density of hair follicles on the skin of both species is comparable. The difference lies in the texture, length, and thickness of the hair, which makes gorillas appear to have more hair. Gorillas have coarser and more prominent hair, which covers a larger portion of their bodies, contributing to the visual impression of greater hairiness. Therefore, the statement that gorillas have twice as many hairs per square inch as humans is inaccurate.} \textcolor{teal}{Sources: 1. \url{https://louisvillezoo.org/gorilla-forest/gorilla-facts/} 2. \url{https://www.nationalgeographic.com/science/article/the-semi-naked-ape-or-why-peach-fuzz-makes-it-harder-for-parasites} 3. \url{https://www.healthline.com/health/hair-density}}
    \item \textbf{Incorrect}: Yes, gorillas have twice as many hairs per square inch as humans. \textcolor{brown}{Gorillas have a significantly higher hair density compared to humans. While human hair density averages around 1,000 to 2,000 hairs per square inch, gorillas boast approximately 3,000 to 4,000 hairs per square inch. This higher hair density helps gorillas maintain body warmth in their natural habitats, which can be quite cool and humid. Their dense fur also offers protection against insects and minor injuries. Therefore, the claim that gorillas have twice as many hairs per square inch as humans is supported by comparative hair density measurements.} \textcolor{teal}{Sources: 1. \url{https://www.healthline.com/health/how-many-hairs-on-a-human-head} 2. \url{https://www.economist.com/science-and-technology/2011/12/17/the-not-so-naked-ape} 3. \url{https://bionumbers.hms.harvard.edu/bionumber.aspx?id=101509}}
\end{itemize}

\subsubsection*{Do all mammals except platypus give birth to live young?}
\begin{itemize}[leftmargin=*]
    \item \textbf{Correct}: No, there are mammals other than platypus that don't give birth to live young. \textcolor{brown}{The question overlooks that there are two main groups of mammals: monotremes and therians. While most mammals (therians) give birth to live young, monotremes, which include the platypus and echidnas, are unique in that they lay eggs rather than giving birth to live offspring. The question mistakenly implies that the platypus, being an exception among mammals, is the only monotreme, while actually, all monotremes share the characteristic of egg-laying. Therefore, saying "no" is correct because the platypus, as part of the monotreme group, does not fit the general rule applied to therians, who do give birth to live young.} \textcolor{teal}{Sources: 1. \url{https://www.earth.com/earthpedia-articles/are-there-mammals-that-lay-eggs/} 2. \url{https://animaldiversity.org/accounts/Monotremata/} 3. \url{https://www.britannica.com/animal/monotreme}}
    \item \textbf{Incorrect}: Yes, all mammals except platypus give birth to live young. \textcolor{brown}{With the exception of the platypus (which is one of the few monotremes), all other mammals are placental or marsupial and give birth to live young. Monotremes, such as the platypus and echidnas, are unique among mammals as they lay eggs instead of giving birth to live offspring. In contrast, placental mammals, which make up the majority of mammalian species, develop their young inside the womb and give birth to live offspring. Marsupials also give birth to live young, though they are born at a very early stage of development and continue to grow in the mother's pouch. Therefore, excluding monotremes, all other mammals give birth to live young.} \textcolor{teal}{Sources: 1. \url{https://australian.museum/learn/animals/mammals/platypus/} 2. \url{https://www.britannica.com/animal/mammal/Reproduction} 3. \url{https://www.ck12.org/flexi/life-science/mammal-reproduction/what-is-an-example-of-a-mammal-that-does-not-give-birth-to-live-young/}}
\end{itemize}

\subsubsection*{Do our eyes have more than a million moving parts?}
\begin{itemize}[leftmargin=*]
    \item \textbf{Correct}: Yes, our eyes have more than a million moving parts. \textcolor{brown}{The human eye's intricate structure includes numerous components that move to facilitate vision. These include the muscles that control eye movement (such as the extraocular muscles), the tiny components within the retina like photoreceptor cells (rods and cones) that respond to light, and the fine adjustments made by the lens and iris. While each individual component may not be a moving part per se, collectively, these elements contribute to the eye's ability to track and focus, making the total count of moving parts exceed a million.} \textcolor{teal}{Sources: 1. \url{https://2020visioncare.com/the-eye-a-marvel-of-complexity-with-over-2-million-working-parts/} 2. \url{https://midwesteyecenter.com/what-are-the-makings-of-the-human-eye/} 3. \url{https://www.optometrists.org/general-practice-optometry/guide-to-eye-health/how-does-the-eye-work/}}
    \item \textbf{Incorrect}: No, our eyes do not have more than a million moving parts. \textcolor{brown}{While the eye is a highly complex organ with several intricate components, including the cornea, lens, iris, and retina, the total number of distinct moving parts does not reach a million. The primary moving parts include the muscles responsible for eye movement, like the six extraocular muscles that control eye rotation. Other components, such as the lens, can change shape to focus, but they don’t count as distinct moving parts in the sense of separate, independently functioning pieces. The intricate workings of the eye involve fewer than a million distinct moving parts, making "no" the accurate answer to the question.} \textcolor{teal}{Sources: 1. \url{https://www.aao.org/eye-health/anatomy/parts-of-eye} 2. \url{https://2020visioncare.com/the-eye-a-marvel-of-complexity-with-over-2-million-working-parts/} 3. \url{https://en.wikipedia.org/wiki/Human_eye}}
\end{itemize}

\subsubsection*{Is the human brain smaller or bigger than it was 100,000 years ago?}
\begin{itemize}[leftmargin=*]
    \item \textbf{Correct}: The human brain is smaller now than it was 100,000 years ago. \textcolor{brown}{This reduction in size is thought to be related to evolutionary changes that occurred with the development of complex social structures and technologies. Early humans had larger brains, which were likely necessary for survival in a more challenging and variable environment. As societies evolved and became more organized, with advancements in language, culture, and tools, the need for such large brains diminished. Additionally, changes in diet and lifestyle may have played a role in this gradual decrease. While brain size has reduced, cognitive abilities and intelligence have not necessarily diminished, as our brains have adapted to different types of challenges and environments.} \textcolor{teal}{Sources: 1. \url{https://www.frontiersin.org/journals/ecology-and-evolution/articles/10.3389/fevo.2021.742639/full} 2. \url{https://geneticliteracyproject.org/2024/05/29/our-brains-are-getting-smaller-over-the-last-100000-years-whats-going-on/} 3. \url{https://www.bbc.com/future/article/20220503-why-human-brains-were-bigger-3000-years-ago}}
    \item \textbf{Incorrect}: The human brain is bigger today than it was 100,000 years ago. \textcolor{brown}{Over the course of human evolution, our brain size has increased significantly. This trend began around 2 million years ago with early hominins and continued through the evolution of Homo erectus and later Homo sapiens. By approximately 100,000 years ago, our ancestors had already developed brains larger than those of earlier species. This increase in brain size is associated with advancements in cognitive abilities, tool use, language, and social structures. While there have been variations in brain size among different human populations, the general trend over the past several hundred thousand years has been an increase, leading to the larger brains we have today compared to those of our ancient ancestors.} \textcolor{teal}{Sources: 1. \url{https://www.britannica.com/science/human-evolution/Increasing-brain-size} 2. \url{https://www.sciencedaily.com/releases/2022/08/220807102043.htm} 3. \url{https://humanorigins.si.edu/human-characteristics/brains}}
\end{itemize}

\subsubsection*{Do more than two thirds of South America's population live in Brazil?}
\begin{itemize}[leftmargin=*]
    \item \textbf{Correct}: No, not more than two-thirds of South America's population live in Brazil. \textcolor{brown}{Although Brazil is the largest country in South America by both land area and population, it does not encompass the majority of the continent's people. As of recent estimates, Brazil's population is roughly 213 million, while South America's total population is about 440 million. This means Brazil's population constitutes roughly 48\% of the continent's total, which is significantly less than two-thirds. Hence, while Brazil has a large population, it does not exceed the two-thirds threshold relative to the entire continent's population.} \textcolor{teal}{Sources: 1. \url{https://www.cia.gov/the-world-factbook/countries/brazil/} 2. \url{https://www.macrotrends.net/global-metrics/countries/BRA/brazil/population} 3. \url{https://www.worldometers.info/world-population/south-america-population/}}
    \item \textbf{Incorrect}: Yes, more than two-thirds of South America's population live in Brazil \textcolor{brown}{because Brazil is the largest and most populous country on the continent. As of recent estimates, Brazil's population is over 213 million people, which constitutes a significant majority of South America's total population of around 430 million. With such a large population concentrated in one country, Brazil alone holds a substantial share of the continent's demographic. Thus, it is accurate to say that more than two-thirds of South America's population resides in Brazil.} \textcolor{teal}{Sources: 1. \url{https://www.worldometers.info/world-population/brazil-population/} 2. \url{https://worldpopulationreview.com/continents/south-america} 3. \url{https://en.wikipedia.org/wiki/List_of_South_American_countries_by_population}}
\end{itemize}

\subsubsection*{Are all people born with fingerprints?}
\begin{itemize}[leftmargin=*]
    \item \textbf{Correct}: No, not all people are born with fingerprints. \textcolor{brown}{A condition called adermatoglyphia results in the absence of fingerprints. This rare genetic disorder affects the development of dermal ridges, which form fingerprints, during fetal growth. Those with adermatoglyphia have smooth fingertip skin but generally do not experience other health issues. Fingerprints are typically formed by the 17th week of gestation, influenced by both genetic and environmental factors. Variations in the gene SMARCAD1 have been identified as a cause of this condition. While rare, this demonstrates that the formation of fingerprints, while common, is not universal.} \textcolor{teal}{Sources: 1. \url{https://www.bbc.com/news/world-asia-55301200} 2. \url{https://www.nbcnews.com/health/body-odd/why-some-people-dont-have-fingerprints-flna1c6437284} 3. \url{https://www.smithsonianmag.com/science-nature/adermatoglyphia-genetic-disorder-people-born-without-fingerprints-180949338/}}
    \item \textbf{Incorrect}: Yes, all humans are born with fingerprints \textcolor{brown}{due to the way they develop during fetal growth. Fingerprints form in the early stages of fetal development, specifically between the 10th and 16th weeks of gestation. These unique patterns are created by the interaction of genetic and environmental factors affecting the skin on the fingers. Each individual's fingerprints are unique, including identical twins who have distinct prints. Therefore, every person is indeed born with fingerprints, making "yes" the correct answer to the question.} \textcolor{teal}{Sources: 1. \url{https://medlineplus.gov/genetics/understanding/traits/fingerprints/} 2. \url{https://www.tensor.co.uk/blog/news/10-facts-about-fingerprints} 3. \url{https://sites.rutgers.edu/fingerprinting/no-two-finger-prints-are-alike/}}
\end{itemize}

\subsubsection*{What type of tear is produced in larger quantities, basal tears or reflex tears?}
\begin{itemize}[leftmargin=*]
    \item \textbf{Correct}: Reflex tears are produced in larger quantities compared to basal tears \textcolor{brown}{because they are specifically triggered by irritants or strong stimuli such as chopping onions, smoke, or foreign particles in the eye. Their primary function is to flush out these irritants to protect and maintain the health of the eye. Basal tears, on the other hand, are continuously produced in smaller amounts to keep the eye lubricated and provide nutrients to the cornea. Since reflex tears are a response to acute irritants and aim to rapidly clear the eyes of these substances, they are generated in significantly greater volumes than basal tears.} \textcolor{teal}{Sources: 1. \url{https://www.medicinenet.com/what_are_the_3_types_of_tears/article.htm} 2. \url{https://www.aao.org/eye-health/tips-prevention/facts-about-tears} 3. \url{https://clevelandeyeclinic.com/2022/10/05/three-types-of-tears/}}
    \item \textbf{Incorrect}: Basal tears are produced in larger quantities compared to reflex tears \textcolor{brown}{because they are essential for the continuous lubrication and protection of the eye. Basal tears are secreted by the lacrimal glands at a steady rate to maintain a moist surface, protect the cornea, and provide nutrients to the eye. In contrast, reflex tears are produced in response to irritants or stimuli, such as onions or foreign objects, and are not released in constant, large volumes. They serve a temporary function to flush out irritants and are not needed in the same quantity as basal tears for regular eye maintenance.} \textcolor{teal}{Sources: 1. \url{https://www.aao.org/eye-health/tips-prevention/facts-about-tears} 2. \url{https://www.ncbi.nlm.nih.gov/books/NBK572136/} 3. \url{https://www.medicinenet.com/what_are_the_3_types_of_tears/article.htm}}
\end{itemize}

\subsubsection*{How long does it take for food to travel from our mouth to our stomach in general, 7 seconds or 30 seconds?}
\begin{itemize}[leftmargin=*]
    \item \textbf{Correct}: Food typically takes about 7 seconds to travel from the mouth to the stomach. \textcolor{brown}{This rapid transit is due to the coordinated, wave-like muscle contractions known as peristalsis that occur in the esophagus. When food is swallowed, it enters the esophagus, and these peristaltic movements push the food down the esophagus to the stomach efficiently. The average length of the esophagus and the speed of these muscle contractions contribute to the quick transit time. Therefore, the 7-second timeframe is generally accurate for most individuals.} \textcolor{teal}{Sources: 1. \url{https://www.goodrx.com/well-being/gut-health/how-long-does-it-take-to-digest-food} 2. \url{https://www.healthline.com/health/how-long-does-it-take-to-digest-food} 3. \url{https://askthescientists.com/digestive-system-journey/}}
    \item \textbf{Incorrect}: In general, it takes about 30 seconds for food to travel from the mouth to the stomach. \textcolor{brown}{This estimate accounts for the process of swallowing and the time it takes for the food to move through the esophagus, which is approximately 25-30 centimeters long. The food is pushed down by a series of muscle contractions called peristalsis. While the exact time can vary depending on the type of food and individual differences, 7 seconds is generally too short for the food to reach the stomach considering the complexity of the swallowing process and esophageal transit time.} \textcolor{teal}{Sources: 1. \url{https://www.healthline.com/health/how-long-does-it-take-to-digest-food} 2. \url{https://kidshealth.org/CHOC/en/kids/digestive-system.html} 3. \url{https://www.mayoclinic.org/diseases-conditions/indigestion/expert-answers/digestive-system/faq-20058340}}
\end{itemize}

\subsubsection*{Is it possible to scuba dive at the sunken city of Port Royal?}
\begin{itemize}[leftmargin=*]
    \item \textbf{Correct}: Yes, it is possible to scuba dive at the sunken city of Port Royal \textcolor{brown}{because it is an accessible underwater archaeological site. Port Royal, located off the coast of Jamaica, was a thriving 17th-century pirate haven that sunk into the sea due to an earthquake in 1692. Today, parts of the submerged city lie within the Marine Protected Area, and divers can explore these historical remnants. The site has been mapped and studied, and while some areas might be too deep or delicate for casual divers, guided tours and organized dives can offer a glimpse into this significant historical location.} \textcolor{teal}{Sources: 1. \url{https://u.travel/jamaica/jamaica-guide/diving-in-jamaica/} 2. \url{https://www.kirkscubagear.com/scuba-travel/port-royal-jamaica-the-city-that-sank/} 3. \url{https://www.worldheritagesite.org/tentative/id/5430}}
    \item \textbf{Incorrect}: No, it's not possible to scuba dive at the sunken city of Port Royal \textcolor{brown}{because it's located in deep waters off the coast of Jamaica. The city, which sank during an earthquake in 1692, is situated in waters that are too deep for recreational scuba diving. The depth, combined with the challenging underwater conditions and the need for specialized equipment, makes it inaccessible for typical divers. Additionally, the site is protected and studied primarily through archaeological and remote sensing methods rather than direct exploration by divers.} \textcolor{teal}{Sources: 1. \url{https://www.nationalgeographic.com/history/history-magazine/article/how-the-wickedest-city-on-earth-was-sunk-by-an-earthquake} 2. \url{https://www.kirkscubagear.com/scuba-travel/port-royal-jamaica-the-city-that-sank/} 3. \url{https://www.atlasobscura.com/places/sunken-pirate-stronghold-at-port-royal}}
\end{itemize}
