%%
%% This is file `sample-sigconf-authordraft.tex',
%% generated with the docstrip utility.
%%
%% The original source files were:
%%
%% samples.dtx  (with options: `all,proceedings,bibtex,authordraft')
%% 
%% IMPORTANT NOTICE:
%% 
%% For the copyright see the source file.
%% 
%% Any modified versions of this file must be renamed
%% with new filenames distinct from sample-sigconf-authordraft.tex.
%% 
%% For distribution of the original source see the terms
%% for copying and modification in the file samples.dtx.
%% 
%% This generated file may be distributed as long as the
%% original source files, as listed above, are part of the
%% same distribution. (The sources need not necessarily be
%% in the same archive or directory.)
%%
%%
%% Commands for TeXCount
%TC:macro \cite [option:text,text]
%TC:macro \citep [option:text,text]
%TC:macro \citet [option:text,text]
%TC:envir table 0 1
%TC:envir table* 0 1
%TC:envir tabular [ignore] word
%TC:envir displaymath 0 word
%TC:envir math 0 word
%TC:envir comment 0 0
%%
%%
%% The first command in your LaTeX source must be the \documentclass
%% command.
%%
%% For submission and review of your manuscript please change the
%% command to \documentclass[manuscript, screen, review]{acmart}.
%%
%% When submitting camera ready or to TAPS, please change the command
%% to \documentclass[sigconf]{acmart} or whichever template is required
%% for your publication.
%%
%%
\PassOptionsToPackage{prologue,dvipsnames}{xcolor}
% \documentclass[manuscript,review,anonymous]{acmart} % Review
% \documentclass[sigconf]{acmart} % Camera-ready
\documentclass[sigconf,authorversion]{acmart} % arXiv

%%
%% \BibTeX command to typeset BibTeX logo in the docs
\AtBeginDocument{%
  \providecommand\BibTeX{{%
    Bib\TeX}}}

%% Rights management information.  This information is sent to you
%% when you complete the rights form.  These commands have SAMPLE
%% values in them; it is your responsibility as an author to replace
%% the commands and values with those provided to you when you
%% complete the rights form.

\copyrightyear{2025}
\acmYear{2025}
% \setcopyright{cc} % ACM
\setcopyright{none} % arXiv
\setcctype{by}
\acmConference[CHI '25]{CHI Conference on Human Factors in Computing Systems}{April 26-May 1, 2025}{Yokohama, Japan}
\acmBooktitle{CHI Conference on Human Factors in Computing Systems (CHI '25), April 26-May 1, 2025, Yokohama, Japan}\acmDOI{10.1145/3706598.3714020}
\acmISBN{979-8-4007-1394-1/25/04}


%%
%% Submission ID.
%% Use this when submitting an article to a sponsored event. You'll
%% receive a unique submission ID from the organizers
%% of the event, and this ID should be used as the parameter to this command.
%%\acmSubmissionID{123-A56-BU3}

%%
%% For managing citations, it is recommended to use bibliography
%% files in BibTeX format.
%%
%% You can then either use BibTeX with the ACM-Reference-Format style,
%% or BibLaTeX with the acmnumeric or acmauthoryear sytles, that include
%% support for advanced citation of software artefact from the
%% biblatex-software package, also separately available on CTAN.
%%
%% Look at the sample-*-biblatex.tex files for templates showcasing
%% the biblatex styles.
%%

%%
%% The majority of ACM publications use numbered citations and
%% references.  The command \citestyle{authoryear} switches to the
%% "author year" style.
%%
%% If you are preparing content for an event
%% sponsored by ACM SIGGRAPH, you must use the "author year" style of
%% citations and references.
%% Uncommenting
%% the next command will enable that style.
%%\citestyle{acmauthoryear}


% Added packages
% \usepackage{enumerate}
\usepackage{enumitem}
\usepackage{multirow}
% \usepackage[htt]{hyphenat}
\usepackage{subcaption}
\usepackage{graphicx}
% \usepackage[export]{adjustbox}
\let\Bbbk\relax % Had to add to get this to compile after someone added the amssymb package
\usepackage{amssymb}% http://ctan.org/pkg/amssymb
\usepackage{pifont}% http://ctan.org/pkg/pifont
\newcommand{\cmark}{\ding{51}}%
\newcommand{\xmark}{\ding{55}}%

% Other comments
\newcommand{\smallsec}[1]{\noindent {\bf #1.}}
\newcommand{\var}[1]{\texttt{#1}}
%\usepackage[dvipsnames]{xcolor}


\usepackage[capitalize]{cleveref}
\crefname{section}{Section}{Sections}
\Crefname{section}{Section}{Sections}
\Crefname{table}{Table}{Tables}
\crefname{table}{Table}{Tables}
\crefname{figure}{Figure}{Figures}

\newcommand{\interviewprefix}{P}
\makeatletter
\newcommand{\interview}[1]{%
  \def\nextitem{\def\nextitem{, }}%ㅇ
  (\@for\el:=#1\do{\nextitem\interviewprefix\el})%
}
\newcommand{\longquote}[2]{\begin{quote}“\textit{#1}” -- P{#2}\end{quote}}%
\newcommand{\shortquote}[1]{“\textit{#1}”}%
\makeatother%


%%
%% end of the preamble, start of the body of the document source.
\begin{document}

%%
%% The "title" command has an optional parameter,
%% allowing the author to define a "short title" to be used in page headers.
\title[Fostering Appropriate Reliance on Large Language Models]{Fostering Appropriate Reliance on Large Language Models:\\The Role of Explanations, Sources, and Inconsistencies}


%%
%% The "author" command and its associated commands are used to define
%% the authors and their affiliations.
%% Of note is the shared affiliation of the first two authors, and the
%% "authornote" and "authornotemark" commands
%% used to denote shared contribution to the research.
\author{Sunnie S. Y. Kim}
\email{sunniesuhyoung@princeton.edu}
\affiliation{%
  \institution{Princeton University}
  \state{New Jersey}
  \country{USA}
}

\author{Jennifer Wortman Vaughan}
\email{jenn@microsoft.com}
\affiliation{%
  \institution{Microsoft Research}
  \state{New York}
  \country{USA}
}

\author{Q. Vera Liao}
\email{veraliao@microsoft.com}
\affiliation{%
  \institution{Microsoft Research}
  \state{Montreal}
  \country{Canada}
}

\author{Tania Lombrozo}
\email{lombrozo@princeton.edu}
\affiliation{%
  \institution{Princeton University}
  \state{New Jersey}
  \country{USA}
}

\author{Olga Russakovsky}
\email{olgarus@princeton.edu}
\affiliation{%
  \institution{Princeton University}
  \state{New Jersey}
  \country{USA}
}


%%
%% By default, the full list of authors will be used in the page
%% headers. Often, this list is too long, and will overlap
%% other information printed in the page headers. This command allows
%% the author to define a more concise list
%% of authors' names for this purpose.
\renewcommand{\shortauthors}{Kim, Vaughan, Liao, Lombrozo, Russakovsky}


%%
%% The abstract is a short summary of the work to be presented in the
%% article.
\begin{abstract} % max 150 words
Large language models (LLMs) can produce erroneous responses that sound fluent and convincing, raising the risk that users will rely on these responses as if they were correct. Mitigating such overreliance is a key challenge. 
Through a think-aloud study in which participants use an LLM-infused application to answer objective questions, we identify several features of LLM responses that shape users' reliance: \textit{explanations} (supporting details for answers), \textit{inconsistencies} in explanations, and \textit{sources}.
Through a large-scale, pre-registered, controlled experiment (N=308), we isolate and study the effects of these features on users' reliance, accuracy, and other measures.
We find that the presence of explanations increases reliance on both correct and incorrect responses. However, we observe less reliance on incorrect responses when sources are provided or when explanations exhibit inconsistencies. We discuss the implications of these findings for fostering appropriate reliance on LLMs.
\end{abstract}




%%
%% The code below is generated by the tool at http://dl.acm.org/ccs.cfm.
%% Please copy and paste the code instead of the example below.
%%
\begin{CCSXML}
<ccs2012>
   <concept>
       <concept_id>10003120.10003121.10011748</concept_id>
       <concept_desc>Human-centered computing~Empirical studies in HCI</concept_desc>
       <concept_significance>500</concept_significance>
       </concept>
   <concept>
       <concept_id>10010147.10010178</concept_id>
       <concept_desc>Computing methodologies~Artificial intelligence</concept_desc>
       <concept_significance>500</concept_significance>
       </concept>
 </ccs2012>
\end{CCSXML}

\ccsdesc[500]{Human-centered computing~Empirical studies in HCI}
\ccsdesc[500]{Computing methodologies~Artificial intelligence}


%%
%% Keywords. The author(s) should pick words that accurately describe
%% the work being presented. Separate the keywords with commas.
\keywords{Large language models, Overreliance, Human-AI interaction, Question answering, Explanations, Sources, Inconsistencies}


%%
%% This command processes the author and affiliation and title
%% information and builds the first part of the formatted document.
\maketitle


\section{Introduction}
\label{sec:intro}

Large language models (LLMs) are powerful tools, capable of a wide range of tasks from text summarization to sentence completion to code generation. Technology companies have leapt at the unprecedented opportunity to build LLM-infused applications that help users with information retrieval and search, learning new things, and performing everyday tasks more efficiently.
Many such applications, such as LLM-infused search engines and chatbots, are predicated on LLMs' ability to provide intricate responses to complex user questions. 
Already millions of people use LLMs to find answers to their questions about health, science, current events, and other domains, and the use of LLMs is widely predicted to grow~\cite{Sharma2024Echochamber,Zhu2023LargeLM,Kapoor2024ICML}. 
However, the responses produced by LLMs are often inaccurate, sometimes in subtle ways~\cite{Ji2023Hallucination,shuster2021eval,santhanam2022rome,Dahl2024Law}.
Inaccurate LLM responses have the potential to mislead users, raising the risk that users will take actions based on the assumption that responses are correct~\cite{Kim2024FAccT,Vasconcelos2023CSCW,nytimes2023lawyer,passi2024appropriate,Weidinger2022Risk}.
While such \emph{overreliance} on AI systems is not a new problem~\cite{zhang2020effect,Bansal2021CHI,Poursabzi-Sangdeh-CHI2021,wang2021explanations,passi2022overreliance}, it may be exacerbated by the introduction of LLMs, since LLM responses are often fluent and convincing even when wrong and public excitement around LLMs is high.


When asked to answer a question, LLMs and systems based on them typically provide a response that contains both an answer to the question and some supporting details or justification for this answer~\cite{Lee2024FAccT,lfqa23}. For example, when asked a math question, an LLM may provide a step-by-step derivation for its answer \cite{Collins2024PNAS,hendrycks2021measuring}. In line with everyday usage and much of the psychology literature~\cite{Lombrozo2012,Lombrozo2006TCS,keil2006explanation}, we refer to such supporting details as an \emph{explanation} of the answer.
(We note that this differs from how the term explanation is often used within the explainable AI community in that we do not make any assumptions about the extent to which it faithfully describes the way that the model arrived at its answer. That is, the explanation describes why the answer is correct, not necessarily why the model output the answer that it did.)
Some authors have argued that such explanations should help users spot incorrect answers, potentially mitigating overreliance~\cite{Gonzalez2021DoEH,Bussone2015,Lai2019FAccT,Vasconcelos2023CSCW}. However, prior work suggests that in many settings, the very presence of an explanation can increase trust and reliance, whether or not it is warranted~\cite{Bansal2021CHI,zhang2020effect,Poursabzi-Sangdeh-CHI2021,wang2021explanations,Fok2024Verifiability,Pafla2024CHI}.
To avoid such unintended negative consequences, it is necessary to understand how users interpret and act upon explanations from LLMs, and how explanations and other features of LLM responses might be adjusted to encourage appropriate reliance.


To explore these questions, we first conduct a think-aloud study with 16 participants with varying knowledge of and experience with LLMs. In this study, participants answer objective questions with the use of the popular LLM-infused application ChatGPT via multi-turn interactions. The goal of this preliminary study is to understand how people perceive LLM responses and which features of a response shape their reliance.
We observe that participants interpret \textit{inconsistencies} in explanations --- that is, sets of statements that cannot be true at the same time~\cite{Hurley2000} --- as a cue of unreliability.
Participants also seek out \textit{sources} to verify supporting details in LLM responses and are less likely to rely on incorrect answers when the sources provide helpful information.


Building on the findings from this study, we next conduct a large-scale, pre-registered, controlled experiment ($N=308$) in which participants answer difficult objective questions with access to LLM responses, i.e., responses from a hypothetical LLM named ``Theta.''\footnote{We note that the line between what we would call an ``LLM'' as opposed to an ``LLM-infused system'' can be blurry, especially when the system takes the form of a chatbot such as Theta or ChatGPT. Throughout the paper, use the term LLM for readability in places where the distinction is not important.}
These responses were created in advance using state-of-the-art LLM-infused applications ChatGPT and Perplexity AI so that we can fully control their features. Specifically, we employ a 2 x 2 x 2 within-subjects design, varying three features of the LLM responses: accuracy of the LLM's answer to the question (correct/incorrect), presence of an explanation (absent/present), and presence of clickable sources (absent/present). Further, we capitalize on the natural inconsistencies that arise in LLM responses to investigate the effects of inconsistencies. We examine the impact of these variables on participants' reliance, accuracy, and other measures, such as confidence, source clicking behavior, time on task, evaluation of LLM responses, and likelihood of asking follow-up questions. \looseness=-1


We find that when either or both an explanation and sources are present, participants report higher confidence in their answer, rate the LLM response higher in terms of the quality of the justification it provides for the answer and the actionability of its response, and are less likely to ask follow-up questions. However, explanations and sources differ in their effects on reliance. Explanations increase reliance on both correct LLM answers and incorrect LLM answers. In contrast, sources increase appropriate reliance on correct LLM answers, although less effectively than explanations, while decreasing overreliance on incorrect LLM answers. Finally, when explanations have inconsistencies, we observe less overreliance on incorrect LLM answers compared to when there are no inconsistencies or when explanations are not provided at all. 
We complement these quantitative findings with qualitative insights and close with a discussion of implications and future research directions for fostering appropriate reliance on LLMs.


Together, our approach and findings offer a number of contributions. (1) Our studies  tackle the timely and critical issue of fostering appropriate reliance on LLMs. Since research on user reliance on LLMs is relatively new, we take a mixed-methods approach, first (via the think-aloud study) identifying features of LLM responses that shape user reliance, and then (via the controlled experiment) isolating and studying the effects of the identified features. (2) Through our two studies, we identify which combinations of features help people achieve appropriate reliance and high task accuracy, providing actionable insights on how to adjust LLM response features. We also contribute a more holistic and nuanced understanding of user reliance on LLMs with insights on people's interpretation of explanations from LLMs, source clicking behavior, and interaction effects between explanations and sources. (3) We provide an in-depth discussion of the implications of our findings, limitations of our work, and future research directions. In particular, we identify providing (accurate and relevant) sources and highlighting inconsistencies and other unreliability cues in LLM responses as promising strategies for fostering appropriate reliance on LLMs. However, such approaches should always be tested with users before deployment. \looseness=-1




\section{Related Work}


\subsection{Appropriate Reliance on AI}

Despite the rapid progress of technology, AI systems still frequently and unexpectedly fail. Without knowing when and how much to rely on a system, a user may experience low-quality interactions or even safety risks in high-stakes settings. 
Prior work has investigated how providing information about an AI system's accuracy \cite{yin2019understanding,He2023Accuracy,Yu2019IUI} and (un)certainty \cite{zhang2020effect,Bansal2021CHI,Green2019,Bucinca2021CSCW,Bussone2015}, explanations of outputs \cite{zhang2020effect,Gonzalez2021DoEH,Bansal2021CHI,Lai2019FAccT,Green2019,Bucinca2021CSCW,Bussone2015}, and onboarding materials \cite{Cai2021,Lai2020Tutorial} impact user reliance, as well as the roles played by human intuition \cite{chen2023understanding}, task complexity \cite{Salimzadeh2023UMAP,Salimzadeh2024CHI}, and other human, AI, and context-related factors \cite{Kim2023Trust}.
However, fostering appropriate reliance on AI remains difficult.
Findings on the effectiveness of proposed methods are mixed, and more research is needed on how reliance is shaped in real-world settings.

While most prior work on AI reliance has been in the context of classical AI models (e.g., specialized classification models), there is a growing body of work looking at reliance on systems based on LLMs or other modern generative AI models \cite{vasconcelos2023generation,spatharioti2023comparing,zhou2024relying,Kim2024FAccT,si2024fact,Lee2024FAccT}. 
For example, several recent studies explored the effect of communicating (un)certainty in LLMs by highlighting uncertain parts of LLM responses \cite{vasconcelos2023generation,spatharioti2023comparing} or inserting natural language expressions of uncertainty \cite{zhou2024relying,Kim2024FAccT}, finding that some but not all types of (un)certainty information help foster appropriate reliance.

Contributing to this line of work, we first take a bottom-up approach to identify the features of LLM responses that impact user reliance in the context of answering objective questions with the assistance of a popular LLM-infused application ChatGPT (\cref{sec:study1}).
In line with findings from prior work~\cite{si2024fact}, we see that reliance is shaped by the content of \textit{explanations} provided by the system, particularly whether or not these explanations contain \textit{inconsistencies}. We also observe that participants seek out \textit{sources} to verify the information provided in responses. We then design a large-scale, pre-registered, controlled experiment to isolate and study the effects of these features (\cref{sec:study2}). We discuss the relevant literature on these features and their impact on AI reliance next.


\subsection{Explanations and Inconsistencies}
\label{sec:llmresponses}


The impact of \emph{explanations} on human understanding and trust of AI systems has been studied extensively within the machine learning and human-computer interaction communities, often under the names explainable AI or interpretable machine learning~\cite{liao2021human,vaughan2021humancentered,arrieta2019explainable,RudinEtAlSurvey2022,Kim2023CHI}. Explanations are often motivated as a way to foster appropriate reliance and trust in AI systems, since in principle they provide clues about whether a system's outputs are reliable. However, empirical studies have shown mixed results, with a large body of work suggesting that providing explanations increases people's tendency to rely on an AI system even when it is incorrect~\cite{zhang2020effect,Bansal2021CHI,Poursabzi-Sangdeh-CHI2021,wang2021explanations}. One potential reason for this is that study participants do not make the effort to deeply engage with the explanations~\cite{kaur2020CHI,buccinca2020proxy,gajos2022people,liao2022designing,Vasconcelos2023CSCW}. That is, instead of encouraging deep, analytical reasoning (System 2 thinking~\cite{Kahneman2003,kahneman2011thinking}), study participants may resort to heuristics, such as the explanation's fluency or superficial cues to expertise~\cite{trout2008}, and defer to the system's response on this basis. People may also be more likely to assume an AI system is trustworthy simply because it provides explanations~\cite{ehsan2021explainable}. Further, some clues of unreliability may be difficult to pick up on without existing domain knowledge~\cite{chen2023understanding}.


Adopting the broad definition of an explanation as an answer to a why question \cite{Lombrozo2012,Wellman2011,bromberger1966why,Fraassen1980}, LLMs often provide explanations by default; when asked a question, LLMs rarely provide the answer alone. For factual questions, they provide details supporting the answer \cite{Lee2024FAccT,lfqa23}, and for math questions, they provide detailed steps to derive the answer \cite{Collins2024PNAS,hendrycks2021measuring}. 
This default behavior is likely due to human preference for verbose responses \cite{chiang2024overreasoning,saito2023verbosity,Zheng2020Verbose}.
Research in psychology has shown that explanations are often sought spontaneously \cite{malle1997behaviors,frazier2009preschoolers}, favored when they are longer, more detailed, or perceived to be more informative \cite{weisberg2015deconstructing,Zemla2017Everyday,bechlivanidis2017concreteness,Aronowitz2020TCS,Liquin2022Satisfaction}, and used to guide subsequent judgments and behaviors \cite{Lombrozo2023Selective,Lombrozo2016}. 
Since LLMs are often fine-tuned on human preference data via approaches such as Reinforcement Learning from Human Feedback (RLHF) \cite{ziegler2019finetuning,Christiano2017RLHF,Ouyang2024RLHF}, such preferences would shape the form of their outputs. We note that the default explanations that LLMs present typically provide evidence to support their answers, but do not necessarily reflect the internal processes by which the LLM arrived at the answer. This distinguishes these explanations from those traditionally studied in the explainable AI literature.


Explanations generated by LLMs are widely known to contain inaccurate information and other flaws \cite{Ji2023Hallucination,shuster2021eval,santhanam2022rome,Dahl2024Law}. We direct readers to recent surveys for comprehensive overviews \cite{huang2023hallucination,wang2023factuality}. In our studies, we found \textit{inconsistencies} in explanations to be an important unreliability cue that shapes participants' reliance. As documented in prior work, inconsistencies can occur within a response; they are sometimes referred to as logical fallacies or self-inconsistency in the NLP community \cite{huang2023reasoning,wang2023selfconsistency}. Inconsistencies can also occur between responses; many studies have demonstrated that LLMs often change their answer to a question when challenged, asked the question in a slightly different way, or re-asked the exact same question \cite{Lee2024FAccT,Elazar2021MeasuringAI,laban2024flipflop}. Such inconsistencies, when noticed, may impact people's evaluation of explanations and reliance on LLMs.



We contribute to this line of work in several ways. We first conduct a qualitative, think-aloud study to understand what features of LLM responses shape people's reliance, and find that reliance is shaped by explanations, inconsistencies in explanations, and sources. We then conduct a larger-scale, pre-registered, controlled experiment to quantitatively examine the effects of these features.
While a previous work by \citet{si2024fact} has studied the effects of LLM-generated explanations and inconsistencies on people's fact-checking performance through a small-scale study (16 participants per condition), our work provides a more holistic picture by studying what (else) might contribute to reliance and how the identified features affect a wider range of variables including people's evaluation of the LLM response's justification quality and actionability and likelihood of asking follow-up questions. 
As for the findings, first, consistent with \citet{si2024fact}, we find that explanations increase people's reliance, including overreliance on incorrect answers, and that inconsistencies in explanations can reduce overreliance. 
Additionally, we find that clickable sources --- which were not studied by \citet{si2024fact} --- increase appropriate reliance on correct answers, while reducing overreliance on incorrect answers, adding empirical knowledge on user reliance on LLMs. 
Lastly, our work also contributes nuanced insights on people's interpretation of LLMs' explanations, source clicking behavior, and interaction effects between explanations and sources.



\subsection{Sources}

The final feature of LLM responses that we study is the presence of \emph{sources}, i.e., clickable links to external material.\footnote{One might consider sources to be a component of an explanation. To simplify the exposition of our results, we treat them as a distinct component of LLM responses throughout this paper.} Sources are increasingly provided by LLM-infused applications, including general-purpose chatbots (e.g., ChatGPT, Gemini) and search engines (e.g., Perplexity AI, Copilot in Bing, SearchGPT). Sources are commonly sought by users, as found in prior work \cite{Kim2024ChatGPT} and supported in our studies. Similar to explanations, however, sources in LLM responses can be flawed in various ways \cite{liu2023evaluating,Alkaissi2023}. For instance, \citet{liu2023evaluating} conducted a human evaluation of popular LLM-infused search engines and found that their responses frequently contain inaccurate sources and unsupported statements. \citet{Alkaissi2023} conducted a case study of ChatGPT in the medical domain and found that it generates fake sources. These issues were observed in our studies as well. Currently there is active research on techniques such as Retrieval Augmented Generation (RAG) \cite{Lewis2020RAG,gao2024rag} to help LLMs provide more accurate information and sources.


It is well known that the presence and quality of sources impact how credible people find given content in other settings \cite{Rieh2007Credibility,Wathen2002Credibility}. However, there has been little work studying how people make use of and rely on sources in the context of LLM-infused applications. On the one hand, the presence of sources might reduce overreliance if people click on the provided links to verify the accuracy of the LLM's response. On the other hand, the presence of sources might increase reliance if people interpret them as signs of credibility and defer to the system without verifying the answers themselves. Indeed, in one study of uncertainty communication in LLM-infused search, participants were found to rarely click on source links~\cite{Kim2024FAccT}. Through a large-scale, pre-registered, controlled experiment (\cref{sec:study2}), we study how the presence of clickable sources impacts people's reliance, task accuracy, and other measures, and how this interacts with the presence of explanations and inconsistencies. In our studies, we use realistic explanations and sources, generated by state-of-the-art LLM-infused applications ChatGPT and Perplexity AI, and provide insights for fostering appropriate reliance on LLMs. \looseness=-1



\begin{figure*}[t!]
\centering
\includegraphics[width=\textwidth]{figures/schematic.png}
\caption{\textbf{Overview of our studies.} In Study 1, participants engaged in multi-turn interactions with ChatGPT to arrive at correct answers to objective questions. Based on a thematic analysis of think-aloud and behavioral data, we identified \textit{explanations}, \textit{inconsistencies}, and \textit{sources} as three features of LLM responses likely to influence user reliance. These three features were then investigated in a controlled experiment (Study 2), with features operationalized as indicated in the schematic illustration. Similar to Study 1, participants solved question-answering tasks. However, this time, they had access to one LLM response whose features we experimentally manipulated.}
\label{fig:schematic}
\end{figure*}



\section{Study 1: Think-Aloud Study}
\label{sec:study1}


Towards the goal of identifying features of LLM responses that can help foster appropriate reliance, we first take a bottom-up approach and conduct a think-aloud study in a relatively natural setting. Specifically, we observe how participants solve question-answering tasks with ChatGPT in multi-turn interactions, and explore how they perceive ChatGPT's responses and what helps them arrive at correct answers despite incorrect answers from ChatGPT.



\subsection{Study 1 Methods}

In this section, we describe our study methods, all of which were reviewed and approved by our Institutional Review Board (IRB) prior to conducting the study. 


\subsubsection{Procedure}


The study session had two parts.
In Part 1 (Base), participants were introduced to the study and asked to complete three question-answering tasks while thinking aloud. Each task involved determining the correct answer to an objective question using ChatGPT\footnote{We created a research account with a Plus subscription. Participants logged into our account and used ChatGPT-4o --- the latest version at the time (June 2024) --- through the web interface with Browsing allowed and Memory disallowed.} and reporting confidence in their final answer on a 1--7 scale. As in natural settings, participants could exchange as many messages with ChatGPT as they wished. Participants could also check the sources provided in ChatGPT's responses, but were asked not to conduct their own internet search.


Each participant was given three questions: a general domain factual question (e.g., ``Has Paris hosted the Summer Olympics more times than Tokyo?''), a health or legal domain factual question (e.g., ``Is it illegal to collect rainwater in Colorado?''), and a math question (e.g., ``Sue puts one grain of rice on the first square of a Go board and puts double the amount on the next square. How many grains of rice does Sue put on the last square?'').
The factual questions were binary questions. The math questions were not binary, but had one correct numerical answer.
The specific question was randomly selected from a set of questions we created in advance based on examples of real user-LLM interactions \cite{ShareGPTanalysis} and prior work \cite{Shi2023ICML,xie2024adv}.
Before beginning the tasks, we also asked each participant if they knew the answer to any of the questions so that we could switch to a different question if they did, but this did not happen.


In Part 2 (Prompting), we asked participants to complete the same three tasks again, but this time while employing follow-up prompts in their engagement with ChatGPT.
We designed Part 2 to explore whether certain prompts can help participants more appropriately rely on ChatGPT and succeed on the tasks.
Since participants had different levels of familiarity with prompting, we provided examples of prompts they could use, such as asking for a different type of explanation (e.g., ``Explain step by step'' and ``Explain like I'm five''), asking for more information (e.g., ``Provide an explanation with supporting sources'' and ``Explain how confident you are in the answer''), and challenging the previous response (e.g., ``Explain why your answer may be wrong'' and ``I think you are wrong. Try again''). Participants could use whichever and as many prompts as they wished.
As in Part 1, participants reported their final answer and confidence in their final answer at the end of each task.


In between Part 1 and Part 2 and before concluding the study, we asked interview questions about participants' perception of and experience with ChatGPT. Details are in the appendix. \looseness=-1




\subsubsection{Participant recruitment and selection}
\label{sec:study1participants}

To recruit participants, we posted a screening survey on Mastodon, X (previously Twitter), and various mailing lists and Slack workspaces within and outside the first author's institution.
The survey included questions about the respondent's knowledge and use of LLMs.
Based on the survey responses, we selectively enrolled participants to maximize the diversity of the study sample's LLM background.
See below for a summary of participants' knowledge and use of LLMs. 
We manually reassigned two participants to different categories than what they selected in their survey when their survey responses did not line up with their described experience (high to low knowledge for one participant and low to high knowledge for another). 
We refer to individual participants by identifier P\#.
\begin{itemize}[noitemsep,topsep=0pt]
    \item \textit{Low-knowledge}: ``Slightly familiar, I have heard of them or have some idea of what they are'' \interview{6,9,13,15} or ``Moderately familiar, I know what they are and can explain'' \interview{2,3,11,14}.
    \item \textit{High-knowledge}: ``Very familiar, I have technical knowledge of what they are and how they work'' \interview{1,4,8,10,16} or ``Extremely familiar, I consider myself an expert on them'' \interview{5,7,12}.
    \item \textit{Low-use}: ``Never'' use LLMs \interview{5,13,15,16} or use LLMs ``Rarely, about 1--2 times a month'' \interview{4} or use LLMs ``Sometimes, about 3-4 times a month'' \interview{3,6,8}.
    \item \textit{High-use}: Use LLMs ``Always, about once or more a day'' \interview{1,2,7,9,10,11,12,14}.
\end{itemize}



\subsubsection{Conducting and analyzing studies} 

We collected data from 16 participants in June 2024, each over a Zoom video call. The study lasted one hour on average, and participants were paid \$20 for their participation.
All sessions were video recorded and transcribed for data analysis.
We used a mix of quantitative and qualitative methods to analyze the study data.
On the quantitative side, we analyzed the accuracy of participants' answers and their self-reported confidence in their answers measured on a 1--7 scale for each task.
Since each participant solved three tasks, once in Part 1 and again in Part 2, there are 6 accuracy and 6 confidence numbers for each participant.
On the qualitative side, we conducted a thematic analysis~\cite{boyatzis1998transforming,BraunClarke2006} of participants' think-aloud data and their responses to interview questions to identify features of LLM responses that shaped participants' reliance.
The first author performed the initial coding, discussed the categories with other authors, and then refined the coding. \looseness=-1





\subsection{Study 1 Results}
\label{sec:study1task}

We first provide some descriptive statistics about participants' accuracy, over- and underreliance, and confidence across the two parts of the study (\cref{sec:study1quantitative}). 
We then discuss which LLM response features participants reported as influences on their reliance (\cref{sec:study1qualitative}). 
We emphasize that this study was not intended to provide statistically significant results, but to identify features that may help foster appropriate reliance. Given the small sample size, we report the quantitative results only to provide context. 



\subsubsection{Accuracy, reliance, and confidence}
\label{sec:study1quantitative}

In Part 1 (Base), we collected data on 48 task instances (16 participants $\times$ 3 tasks). 
For 34 of these instances, ChatGPT gave a correct answer in its first response. (ChatGPT sometimes changed its answer over the course of the interaction, either due to stochasticity or in response to participants' follow-up messages.)
Among these, participants' final answer agreed with ChatGPT's correct answer in 33 instances (average confidence 5.97 on the 1--7 scale) and disagreed in only a single instance (confidence 4.5), indicating that \textbf{underreliance was not prevalent}.
In 13 instances, ChatGPT gave an incorrect answer in its first response. 
Among these, participants' final answer agreed with ChatGPT's incorrect answer in 9 instances (average confidence 6.15) and disagreed in only 4 instances (average confidence 5.61), indicating \textbf{widespread overreliance}.
In a single instance, ChatGPT did not answer the question in its first response, and the participant submitted an incorrect answer with a confidence of 2.


We did not find meaningful differences in participants' accuracy between the 
two parts of the study. That is, \textbf{follow-up prompting did not increase participants' accuracy}, at least based on our small sample of quantitative data.
For 44 out of 47 instances in which the participant completed Part 2 (Prompting) (one participant had to skip a task instance due to lack of time), the participant submitted the same answer in both parts.
In 3 instances, participants submitted an incorrect answer in Part 1 and a correct answer in Part 2.
In 2 of these 3 instances, ChatGPT gave an incorrect answer in Part 1, but gave a correct answer in Part 2. In the other instance, ChatGPT gave incorrect answers in both parts, but the participant arrived at the correct answer in Part 2 after engaging in multiple rounds of interaction with ChatGPT.


Finally, we compared participants' confidence in their answers for the same task between the two parts, finding that it increased in Part 2 in 19 instances, decreased in 8 instances, and stayed the same in 20 instances.
However, \textbf{changes in confidence do not correspond to changes in answers}. As mentioned above, participants changed their answers in only 3 out of 47 instances. In these 3 instances, participants' confidence stayed the same or increased slightly as their answer changed from being incorrect to correct.
Participants' self-described reasons for increased confidence included seeing and checking sources, seeing ChatGPT give the same answer multiple times, and receiving more information in general.
Reasons for decreased confidence included experiencing issues with sources (e.g., links were broken or sources were not reputable) and seeing ChatGPT change answers.



\subsubsection{LLM response features shaping reliance}
\label{sec:study1qualitative}

From a thematic analysis of participants' think-aloud data and responses to interview questions, we found \textbf{explanations}, \textbf{inconsistencies}, and \textbf{sources} to be key features of LLM responses that participants reported as influences on reliance.
First, consistent with our discussion in \cref{sec:intro,sec:llmresponses}, we observed that ChatGPT provided \textbf{explanations} of its answers by default.
Participants found these explanations important for judging the reliability of ChatGPT's answers.
For example, P14 (low-knowledge, high-use) described explanations as \shortquote{very important for having reliability on the answer} and said \shortquote{the more explanation it [ChatGPT] can provide me about the answer [...] the more I would be able to rely on it.}
P11 (high-knowledge, high-use) added that they judge the response by \shortquote{how well ChatGPT explains the answer.}
This participant judged ChatGPT's explanation in one task to be very high quality, noting \shortquote{I would put this on my homework and submit it [...] the quality is very high}.


However, in another task, P11 submitted a different answer from ChatGPT after observing \textbf{inconsistencies}: \shortquote{Since it [ChatGPT] doesn't answer these simple questions consistently, I don't trust it as much.}
Sometimes inconsistencies occurred within a response (e.g., ChatGPT saying Paris hosted the Summer Olympics more times than Tokyo while also saying both have hosted twice). At other times inconsistencies occurred across multiple responses (e.g., ChatGPT changing its answer when asked the same or similar questions, or when challenged).
In either case, \textbf{when participants observed inconsistencies, they often asked follow-up questions and engaged more with the system to resolve the inconsistencies.}
For example, when P8 (high-knowledge, low-use) was considering the question ``Did Tesla debut its first car model before or after Dropbox was founded?'' ChatGPT initially stated that Tesla debuted its first car model in 2008 then later changed the year to 2006. After noticing the inconsistencies, P8 engaged in three more rounds of interaction with ChatGPT to verify individual pieces of information, and arrived at the correct answer.


Finally, participants frequently sought and used \textbf{sources} to determine whether or not to rely on ChatGPT.
More often than not, ChatGPT did not provide sources as part of its responses, even though participants were using the latest version at the time of the study (4o) with browsing capabilities. Participants had to explicitly ask for them using prompts like ``Provide sources for the answer.''
Participants rarely did this in Part 1, and as such, sources were provided in only 17 out of 48 instances. However, in Part 2, participants asked for sources more often after seeing prompt examples and were provided sources in 30 instances.
\textbf{When participants checked sources, they were often able to avoid overreliance on ChatGPT.}
For example, out of 11 instances in which participants submitted correct answers despite incorrect answers from ChatGPT (both parts combined), 7 were instances in which participants checked sources. (In the other 4 instances, sources were not provided, but participants were able to submit correct answers through other strategies, such as repeatedly asking ChatGPT about a piece of information.)
For example, when P2 (low-knowledge, high-use) was solving the question ``Sue puts one grain of rice on the first square of a Go board and puts double the amount on the next square. How many grains of rice does Sue put on the last square?'' ChatGPT built on an incorrect assumption about the size of a Go board and gave an incorrect answer. P2 initially judged it as correct, but after checking sources, realized ChatGPT's error and was able to submit a correct answer.


As discussed in \cref{sec:study1quantitative}, sources also influenced participants' confidence in their answers. \textbf{The presence of sources increased confidence in general, except when there were issues with sources.}
For example, P1 (high-knowledge, high-use) said their confidence increased in Part 2 for one task when they received sources and were able to verify information in ChatGPT's responses. But they said their confidence decreased for another task when some of the source links did not open or did not contain relevant information, highlighting the importance of source \textit{quality} in addition to \textit{presence}.
Finally, we emphasize that \textbf{checking sources did not always eliminate overreliance}.
Out of 30 instances in which participants checked sources (both parts combined), in 4 instances, participants' final answer still agreed with ChatGPT's incorrect answer, which is a sign of overreliance.



\begin{figure*}[t!]
\centering
\includegraphics[width=\textwidth]{figures/task.png}
\caption{\textbf{Screenshots of Study 2's experimental task.} Here the LLM response provides an incorrect answer, includes sources, and includes an explanation (with inconsistencies). See \cref{fig:types} for responses with a correct answer for the same task question.}
\label{fig:task}
\end{figure*}




\section{Study 2: Large-scale, Pre-registered, Controlled Experiment}
\label{sec:study2}


Based on the insights from Study 1, we designed a large-scale, pre-registered, controlled experiment to study the effects of different features of LLM responses on people's reliance, task accuracy, and other measures including confidence, source clicking behavior, time on task, evaluation of LLM responses, and asking of follow-up questions.
The goal of the study was to test whether the findings from Study 1 apply at scale and identify which combinations of features help people achieve appropriate reliance and high task accuracy. \looseness=-1



\subsection{Study 2 Methods}

In this section, we describe our study methods. Before collecting data, we obtained IRB approval and pre-registered our experimental design, analysis plan, and data collection procedures.\footnote{Our pre-registration is viewable at \url{https://aspredicted.org/bg22-yfw7.pdf}.}



\subsubsection{Procedure}

We designed a within-subjects experiment in which participants completed a set of question-answering tasks with LLM responses.
Each task involved determining the correct answer to a binary factual question with access to a response from a hypothetical LLM named ``Theta'' (hereafter we occasionally refer to it as ``the LLM'').
See \cref{fig:task} for an example.
Our experiment had a 2 x 2 x 2 design where we varied three variables in Theta's responses: accuracy of Theta's answer to the question (correct/incorrect), presence of an explanation (absent/present), and presence of clickable sources (absent/present).
In total, there were 8 types of responses. 
Participants completed 8 tasks in the experiment and saw one of each type.
This makes Theta's accuracy 50\%, but participants were not given this information: participants did not receive feedback on whether their answer or Theta's answer was correct after solving a task.
See \cref{fig:types} for examples of different types of responses.


The experiment had three parts.
In the first part, participants were introduced to the study and to Theta.
Theta was described as an LLM-based AI system prototype that uses similar technology to OpenAI's ChatGPT, is connected to the internet, and can answer a wide range of questions.
In the second part, participants answered a total of eight questions.
For each question, participants were provided with a response from Theta and were asked to submit their answer, report their confidence in their answer, and rate Theta's response.
They were told that they could click on source links in Theta's responses, but asked not to conduct their own internet search.
Participants could also optionally write a follow-up question, but they did not see Theta's response to it. We made this choice to fully control the number and content of responses, while being able to collect data on when and what types of follow-up questions participants ask.
We acknowledge that showing one controlled response instead of allowing free-form interaction has limitations (see Section~\ref{sec:limitations}). However, we adopt this method from prior work studying LLMs~\cite{Kim2024FAccT,Lee2024FAccT,si2024fact} as a valid approach for capturing user perceptions and behaviors around LLM responses with the advantage of controlling unwanted noise from free-form interactions (for instance, LLMs making different mistakes across participants in follow-up interactions).


We randomized the order in which questions were presented, as well as the assignment of the 8 response types to the questions.
In the final part, participants filled out an exit questionnaire about their experience with and perception of Theta, their background on LLMs, and basic demographic information.
Lastly, participants were debriefed and reminded that some of the responses they saw may have contained inaccurate information.




\begin{figure*}[t!]
\centering
\includegraphics[width=\textwidth]{figures/types.png}
\caption{\textbf{Types of LLM responses used in Study 2.} We vary three variables in the LLM responses: accuracy of the LLM's answer to the question (correct/incorrect), presence of an explanation (absent/present), and presence of clickable sources (absent/present). In total there are 8 types of responses. Here we show 4 types of responses with a correct answer to the question: ``Do more than two thirds of South America's population live in Brazil?'' See \cref{fig:task} for a response with an incorrect answer.}
\label{fig:types}
\end{figure*}



\subsubsection{Dependent Variables}
\label{sec:dvs}


We formed a set of dependent variables (DVs) using a mix of behavioral and self-reported measures to capture participants' reliance and accuracy, as well as related behaviors and judgments.
First, we measured the \emph{agreement} between a participant's answer and that of Theta; this is a commonly used behavioral measure of reliance \cite{yin2019understanding,zhang2020effect,Lai2019FAccT,Bucinca2021CSCW,Cao2022CSCW,Liu2021CSCW,Lu2021CHI,Mohseni2020}.
Second, we measured the \emph{accuracy} of a participant's answer to assess the task outcome. These are our main two DVs.
To complement them, we also examined participants' \emph{confidence} and \emph{source clicking behavior} as indirect measures of reliance, as well as \emph{time on task}, since efficiency is also an important aspect of task outcome.
These complementary measures have also been commonly studied in prior work~\cite{Poursabzi-Sangdeh-CHI2021,Cao2022CSCW,Kim2022HIVE,CHONG2022,Lu2021CHI,vasconcelos2023generation,Kim2024FAccT}.


Additionally, we had participants evaluate the individual LLM responses.
First, we had participants evaluate the \emph{justification quality} of a response, i.e., whether it offers a good justification for its answer. 
Based on prior work in psychology, we expected this to be correlated with reliance and confidence \cite{Lombrozo2016,douven2018best}, as well as whether participants ask follow-up questions \cite{Liquin2022Satisfaction,frazier2009preschoolers}.
Second, we had participants evaluate the \emph{actionability} of a response, as incorrect responses or responses with low justification quality can still be useful if they are actionable; recall that in Study 1, we observed that participants often treated an LLM response as a starting point for determining what action to take next to arrive at the correct answer. 
Finally, we measured whether participants wrote a follow-up question they would like to ask to Theta. This is in part a proxy for satisfaction: prior work in psychology has found that children are less likely to re-ask a question when they are satisfied with an initial response \cite{Kurkul2018Followup,Frazier2016Satisfying,Mills2017Children}. On the other hand, greater satisfaction with a response can increase curiosity about related content \cite{Liquin2022Satisfaction}.


Formally, we measured the following DVs based on participants' observed behavior:
\begin{itemize}%[topsep=0pt]
    \item \var{Agreement}: TRUE if the participant's final answer is the same as Theta's answer; FALSE otherwise.
    \item \var{Accuracy}: TRUE if the participant's final answer is correct; FALSE otherwise.
    \item \var{SourceClick}: TRUE if the participant clicked on one or more sources; FALSE otherwise.
    \item \var{Time}: Number of minutes from when the participant saw the question to when they clicked next.
\end{itemize}


We additionally measured the following DVs based on participants' self-reported ratings or selections:
\begin{itemize}%[topsep=0pt]
    \item \var{Confidence}: Rating on the question ``How confident are you in your answer?'' on a 7-point scale. 
    \item \var{JustificationQuality}: Rating on the statement ``Theta's response offers good justification for its answer'' on a 7-point scale. 
    \item \var{Actionability}: Rating on the statement ``Theta's response includes information that helps me determine what my final answer should be'' on a 7-point scale. 
    \item \var{Followup}: TRUE if the participant wrote a follow-up question they would like to ask instead of selecting ``I'm satisfied with the current response and would not ask a follow-up question.''
\end{itemize}

All DVs were measured once for each of the 8 tasks.
See \cref{fig:task} for screenshots of an example task.




\subsubsection{Analysis}
\label{sec:study2analysis}

We hypothesized that the three features of LLM responses that we manipulated --- the accuracy of the answer, the presence of sources, and the presence of an explanation --- would affect each of the DVs.
To examine this hypothesis, we used a mixed-effects regression model (logistic or linear depending on the data type), where each participant has a unique ID and each task question has a unique ID.
Specifically, for each DV except \var{SourceClick}, we fit the model 
\texttt{DV $\sim$ AI\_Correct * AI\_Sources * AI\_Explanation + (1|participant) + (1|question)}. 
For \var{SourceClick}, we fit the model \texttt{DV $\sim$ AI\_Correct * AI\_Explanation + (1|participant) + (1|question)} only looking at data points for which participants were provided with sources.
\texttt{AI\_Correct}, \texttt{AI\_Sources}, and \texttt{AI\_Explanation} are binary variables with \texttt{Correct Answer}, \texttt{No Sources}, and \texttt{No Explanation} as the reference levels.


We complemented the main analysis with several additional analyses.
First, we conducted two pre-registered analyses 
exploring how participants reacted to inconsistencies in explanations (\cref{sec:inconsistencies}) and how participants' source clicking behavior relates to other DVs (\cref{sec:sourceclick}).
Analysis details and results are presented in the respective sections.
Second, we conducted a thematic analysis~\cite{boyatzis1998transforming,BraunClarke2006} of participants' free-form answers in the exit questionnaire.
The results are presented in \cref{sec:study2results} alongside the quantitative results from the main analysis.



\subsubsection{Materials}
\label{sec:study2materials}


To simulate a realistic LLM usage scenario of users seeking answers to questions they don't know the answer to, we selected task questions according to the following criteria: (1) most lay people should not know the answer off the top of their head so that they will likely engage with the LLM response and (2) the answer can be objectively and automatically assessed.
To satisfy the criteria, we first created 32 binary factual questions based on facts from the books \textit{Weird But True Human Body}~\cite{WeirdButTrueHumanBody} and \textit{Weird But True World 2024}~\cite{WeirdButTrueWorld2024} by National Geographic Kids. We then ran a short pilot study ($N = 50$) in which we asked participants to answer the 32 questions based on their knowledge and without consulting external sources. This allowed us to assess how commonly known the answers to the questions are in our sample. 
We selected questions with less than 50\% accuracy (i.e., worse than random guessing) as our final set of task questions (12 in total) to satisfy our first selection criterion.
However, we acknowledge that focusing on difficult questions may affect the generalizability of our results. See \cref{fig:task,fig:types} for an example question and the appendix for the full set. \looseness=-1


To create LLM responses that are realistic and reflect the state-of-the-art, we used ChatGPT-4o with a Plus subscription and with Browsing allowed, Memory disallowed, and a new chat for each prompt. 
Initially, we inputted the selected task questions to ChatGPT without any system prompts. Consistent with prior work~\cite{Lee2024FAccT}, we observed that ChatGPT's responses generally follow the same structure: answer to the question (e.g., yes or no) followed by an explanation (supporting details). However, the responses greatly varied in form (e.g., the number of paragraphs and the use of bulleted or numbered lists) and length (ranging from 48 to 213 words). To reduce this variability, we used the system prompt ``Provide a one paragraph response not exceeding 180 words'' following the choices in prior work~\cite{Lee2024FAccT}.


For each task question, we first created a pair of responses with explanations, one with a correct answer and one with an incorrect answer. 
To do so, we used the prompts ``Why is [correct answer/incorrect answer] the correct answer to the question: [Task question]?''
We note that the obtained responses were similar in nature to responses obtained by just asking the task question.
We phrased the prompts this way to reduce any structural differences between responses for correct and incorrect answers.
We made minor edits to improve readability and ensure all responses had the same structure (i.e., answer to the question followed by an explanation). We did not make substantive edits to the content.
To create responses without explanations, we removed the explanation component from these responses.


To create responses with sources, we sent the same prompts to Perplexity AI, one of the most popular LLM-infused search engines, with a Plus subscription and with GPT-4o as the AI model. This is because none of the responses from ChatGPT-4o included sources, even with Browsing allowed.
Perplexity AI's responses included 5 to 10 sources. 
When we analyzed the sources, we found that all sources are real, relevant to the task question, and tended to provide accurate information, although we could not fact-check every single statement in these sources.
To not overwhelm participants, we randomly selected 3 sources and appended them to the responses with/without explanations to create responses with sources.
All responses from ChatGPT-4o and Perplexity AI were retrieved between July 29 and August 1 of 2024 using the latest version at the time. \looseness=-1


After creating different types of LLM responses, we went through the responses again and coded the presence of \textit{inconsistencies}, i.e., sets of statements that cannot be true at the same time~\cite{Hurley2000}, which we found to be an important unreliability cue in Study 1.\footnote{To code inconsistencies, we carefully read each LLM response and reasoned about every pair of statements (whether they can be true at the same time). This was doable because the responses are relatively short (less than 180 words) and do not require specialized knowledge to understand. For the same reasons, we expect most people to be able to notice these inconsistencies. We note that this may not always be the case. The presence of inconsistencies may have little to no effect if they are hard to detect, for example, because the LLM response is long, complex, and requires specialized knowledge to understand that two statements cannot be true at the same time.}
We found that 3 out of 12 responses with explanations for an incorrect answer contained inconsistencies:
(1) For the task question ``Do more than two thirds of South America's population live in Brazil?'' the incorrect response (see \cref{fig:task}) says ``yes'' but later states Brazil's population as around 213 million and South America's total population as around 430 million. (2) For ``Which body part has a higher percentage of water, lungs or skin?'' the incorrect response says ``skin'' but later states skin has 64\% and lungs have 83\% water percentage. (3) For ``Do all mammals except platypus give birth to live young?'' the incorrect response says ``yes'' but later states echidnas as another mammal species that does not give birth to live young.
In contrast, none of the 12 responses with an explanation for a correct answer contained inconsistencies.
While the presence of inconsistencies is not something we control for or manipulate, we coded it to study its effects on reliance and other measures.
See \cref{fig:types} for examples of different types of responses and the appendix for more information on the study materials.


\subsubsection{Participants}

We aimed to collect a minimum of 300 responses post-exclusions.
This number was determined based on a power analysis on pilot data using the simR package in R \cite{simR}.
We conducted data collection using Qualtrics and Prolific in August 2024.
Specifically, we collected responses from 320 U.S.-based adults on Prolific who had completed at least 100 prior tasks with a 95\% or higher approval rating. 
We excluded 12 responses (3.75\%) based on three pre-registered exclusion criteria (3 for response time under 5 minutes, 9 for less than 80\% accuracy on the post-task attention check, and 1 for off-topic free-form response; 1 response was caught on multiple criteria).
Our final sample consists of 308 responses.
Regardless of inclusion or exclusion in the final sample for analysis, we paid all participants \$3.75.
The median study duration was 15.3 minutes, so on average, participants were paid \$14.70 per hour.
See the appendix for more information about participants. \looseness=-1


\subsection{Study 2 Results: Main Analysis}
\label{sec:study2results}


We begin with the main analysis results.
We report the raw data means ($M$) and standard deviations in \cref{tab:main} and the regression results ($\beta, SE, p$) in the text.
We use \textit{significance} to refer to statistical significance at the level of $p < .05$. 
Recall that we fit mixed-effects regression models with three variables and all possible interactions (see \cref{sec:study2analysis} for details).
We did not find a significant three-way interaction for any DVs.
Given our interest in the effects of explanation and sources in LLM responses, we report significant main effects and two-way interactions in the following order: main effects of explanation and interactions with LLM accuracy (\cref{sec:mainexplanation}), main effects of sources and interactions with LLM accuracy (\cref{sec:mainsource}), interactions between explanation and sources (\cref{sec:explsource}), and additional effects of LLM accuracy (\cref{sec:mainaccuracy}).



% This must be in the first 5 lines to tell arXiv to use pdfLaTeX, which is strongly recommended.
\pdfoutput=1
% In particular, the hyperref package requires pdfLaTeX in order to break URLs across lines.

\documentclass[11pt]{article}

% Change "review" to "final" to generate the final (sometimes called camera-ready) version.
% Change to "preprint" to generate a non-anonymous version with page numbers.
\usepackage{acl}

% Standard package includes
\usepackage{times}
\usepackage{latexsym}

% Draw tables
\usepackage{booktabs}
\usepackage{multirow}
\usepackage{xcolor}
\usepackage{colortbl}
\usepackage{array} 
\usepackage{amsmath}

\newcolumntype{C}{>{\centering\arraybackslash}p{0.07\textwidth}}
% For proper rendering and hyphenation of words containing Latin characters (including in bib files)
\usepackage[T1]{fontenc}
% For Vietnamese characters
% \usepackage[T5]{fontenc}
% See https://www.latex-project.org/help/documentation/encguide.pdf for other character sets
% This assumes your files are encoded as UTF8
\usepackage[utf8]{inputenc}

% This is not strictly necessary, and may be commented out,
% but it will improve the layout of the manuscript,
% and will typically save some space.
\usepackage{microtype}
\DeclareMathOperator*{\argmax}{arg\,max}
% This is also not strictly necessary, and may be commented out.
% However, it will improve the aesthetics of text in
% the typewriter font.
\usepackage{inconsolata}

%Including images in your LaTeX document requires adding
%additional package(s)
\usepackage{graphicx}
% If the title and author information does not fit in the area allocated, uncomment the following
%
%\setlength\titlebox{<dim>}
%
% and set <dim> to something 5cm or larger.

\title{Wi-Chat: Large Language Model Powered Wi-Fi Sensing}

% Author information can be set in various styles:
% For several authors from the same institution:
% \author{Author 1 \and ... \and Author n \\
%         Address line \\ ... \\ Address line}
% if the names do not fit well on one line use
%         Author 1 \\ {\bf Author 2} \\ ... \\ {\bf Author n} \\
% For authors from different institutions:
% \author{Author 1 \\ Address line \\  ... \\ Address line
%         \And  ... \And
%         Author n \\ Address line \\ ... \\ Address line}
% To start a separate ``row'' of authors use \AND, as in
% \author{Author 1 \\ Address line \\  ... \\ Address line
%         \AND
%         Author 2 \\ Address line \\ ... \\ Address line \And
%         Author 3 \\ Address line \\ ... \\ Address line}

% \author{First Author \\
%   Affiliation / Address line 1 \\
%   Affiliation / Address line 2 \\
%   Affiliation / Address line 3 \\
%   \texttt{email@domain} \\\And
%   Second Author \\
%   Affiliation / Address line 1 \\
%   Affiliation / Address line 2 \\
%   Affiliation / Address line 3 \\
%   \texttt{email@domain} \\}
% \author{Haohan Yuan \qquad Haopeng Zhang\thanks{corresponding author} \\ 
%   ALOHA Lab, University of Hawaii at Manoa \\
%   % Affiliation / Address line 2 \\
%   % Affiliation / Address line 3 \\
%   \texttt{\{haohany,haopengz\}@hawaii.edu}}
  
\author{
{Haopeng Zhang$\dag$\thanks{These authors contributed equally to this work.}, Yili Ren$\ddagger$\footnotemark[1], Haohan Yuan$\dag$, Jingzhe Zhang$\ddagger$, Yitong Shen$\ddagger$} \\
ALOHA Lab, University of Hawaii at Manoa$\dag$, University of South Florida$\ddagger$ \\
\{haopengz, haohany\}@hawaii.edu\\
\{yiliren, jingzhe, shen202\}@usf.edu\\}



  
%\author{
%  \textbf{First Author\textsuperscript{1}},
%  \textbf{Second Author\textsuperscript{1,2}},
%  \textbf{Third T. Author\textsuperscript{1}},
%  \textbf{Fourth Author\textsuperscript{1}},
%\\
%  \textbf{Fifth Author\textsuperscript{1,2}},
%  \textbf{Sixth Author\textsuperscript{1}},
%  \textbf{Seventh Author\textsuperscript{1}},
%  \textbf{Eighth Author \textsuperscript{1,2,3,4}},
%\\
%  \textbf{Ninth Author\textsuperscript{1}},
%  \textbf{Tenth Author\textsuperscript{1}},
%  \textbf{Eleventh E. Author\textsuperscript{1,2,3,4,5}},
%  \textbf{Twelfth Author\textsuperscript{1}},
%\\
%  \textbf{Thirteenth Author\textsuperscript{3}},
%  \textbf{Fourteenth F. Author\textsuperscript{2,4}},
%  \textbf{Fifteenth Author\textsuperscript{1}},
%  \textbf{Sixteenth Author\textsuperscript{1}},
%\\
%  \textbf{Seventeenth S. Author\textsuperscript{4,5}},
%  \textbf{Eighteenth Author\textsuperscript{3,4}},
%  \textbf{Nineteenth N. Author\textsuperscript{2,5}},
%  \textbf{Twentieth Author\textsuperscript{1}}
%\\
%\\
%  \textsuperscript{1}Affiliation 1,
%  \textsuperscript{2}Affiliation 2,
%  \textsuperscript{3}Affiliation 3,
%  \textsuperscript{4}Affiliation 4,
%  \textsuperscript{5}Affiliation 5
%\\
%  \small{
%    \textbf{Correspondence:} \href{mailto:email@domain}{email@domain}
%  }
%}

\begin{document}
\maketitle
\begin{abstract}
Recent advancements in Large Language Models (LLMs) have demonstrated remarkable capabilities across diverse tasks. However, their potential to integrate physical model knowledge for real-world signal interpretation remains largely unexplored. In this work, we introduce Wi-Chat, the first LLM-powered Wi-Fi-based human activity recognition system. We demonstrate that LLMs can process raw Wi-Fi signals and infer human activities by incorporating Wi-Fi sensing principles into prompts. Our approach leverages physical model insights to guide LLMs in interpreting Channel State Information (CSI) data without traditional signal processing techniques. Through experiments on real-world Wi-Fi datasets, we show that LLMs exhibit strong reasoning capabilities, achieving zero-shot activity recognition. These findings highlight a new paradigm for Wi-Fi sensing, expanding LLM applications beyond conventional language tasks and enhancing the accessibility of wireless sensing for real-world deployments.
\end{abstract}

\section{Introduction}

In today’s rapidly evolving digital landscape, the transformative power of web technologies has redefined not only how services are delivered but also how complex tasks are approached. Web-based systems have become increasingly prevalent in risk control across various domains. This widespread adoption is due their accessibility, scalability, and ability to remotely connect various types of users. For example, these systems are used for process safety management in industry~\cite{kannan2016web}, safety risk early warning in urban construction~\cite{ding2013development}, and safe monitoring of infrastructural systems~\cite{repetto2018web}. Within these web-based risk management systems, the source search problem presents a huge challenge. Source search refers to the task of identifying the origin of a risky event, such as a gas leak and the emission point of toxic substances. This source search capability is crucial for effective risk management and decision-making.

Traditional approaches to implementing source search capabilities into the web systems often rely on solely algorithmic solutions~\cite{ristic2016study}. These methods, while relatively straightforward to implement, often struggle to achieve acceptable performances due to algorithmic local optima and complex unknown environments~\cite{zhao2020searching}. More recently, web crowdsourcing has emerged as a promising alternative for tackling the source search problem by incorporating human efforts in these web systems on-the-fly~\cite{zhao2024user}. This approach outsources the task of addressing issues encountered during the source search process to human workers, leveraging their capabilities to enhance system performance.

These solutions often employ a human-AI collaborative way~\cite{zhao2023leveraging} where algorithms handle exploration-exploitation and report the encountered problems while human workers resolve complex decision-making bottlenecks to help the algorithms getting rid of local deadlocks~\cite{zhao2022crowd}. Although effective, this paradigm suffers from two inherent limitations: increased operational costs from continuous human intervention, and slow response times of human workers due to sequential decision-making. These challenges motivate our investigation into developing autonomous systems that preserve human-like reasoning capabilities while reducing dependency on massive crowdsourced labor.

Furthermore, recent advancements in large language models (LLMs)~\cite{chang2024survey} and multi-modal LLMs (MLLMs)~\cite{huang2023chatgpt} have unveiled promising avenues for addressing these challenges. One clear opportunity involves the seamless integration of visual understanding and linguistic reasoning for robust decision-making in search tasks. However, whether large models-assisted source search is really effective and efficient for improving the current source search algorithms~\cite{ji2022source} remains unknown. \textit{To address the research gap, we are particularly interested in answering the following two research questions in this work:}

\textbf{\textit{RQ1: }}How can source search capabilities be integrated into web-based systems to support decision-making in time-sensitive risk management scenarios? 
% \sq{I mention ``time-sensitive'' here because I feel like we shall say something about the response time -- LLM has to be faster than humans}

\textbf{\textit{RQ2: }}How can MLLMs and LLMs enhance the effectiveness and efficiency of existing source search algorithms? 

% \textit{\textbf{RQ2:}} To what extent does the performance of large models-assisted search align with or approach the effectiveness of human-AI collaborative search? 

To answer the research questions, we propose a novel framework called Auto-\
S$^2$earch (\textbf{Auto}nomous \textbf{S}ource \textbf{Search}) and implement a prototype system that leverages advanced web technologies to simulate real-world conditions for zero-shot source search. Unlike traditional methods that rely on pre-defined heuristics or extensive human intervention, AutoS$^2$earch employs a carefully designed prompt that encapsulates human rationales, thereby guiding the MLLM to generate coherent and accurate scene descriptions from visual inputs about four directional choices. Based on these language-based descriptions, the LLM is enabled to determine the optimal directional choice through chain-of-thought (CoT) reasoning. Comprehensive empirical validation demonstrates that AutoS$^2$-\ 
earch achieves a success rate of 95–98\%, closely approaching the performance of human-AI collaborative search across 20 benchmark scenarios~\cite{zhao2023leveraging}. 

Our work indicates that the role of humans in future web crowdsourcing tasks may evolve from executors to validators or supervisors. Furthermore, incorporating explanations of LLM decisions into web-based system interfaces has the potential to help humans enhance task performance in risk control.






\section{Related Work}
\label{sec:relatedworks}

% \begin{table*}[t]
% \centering 
% \renewcommand\arraystretch{0.98}
% \fontsize{8}{10}\selectfont \setlength{\tabcolsep}{0.4em}
% \begin{tabular}{@{}lc|cc|cc|cc@{}}
% \toprule
% \textbf{Methods}           & \begin{tabular}[c]{@{}c@{}}\textbf{Training}\\ \textbf{Paradigm}\end{tabular} & \begin{tabular}[c]{@{}c@{}}\textbf{$\#$ PT Data}\\ \textbf{(Tokens)}\end{tabular} & \begin{tabular}[c]{@{}c@{}}\textbf{$\#$ IFT Data}\\ \textbf{(Samples)}\end{tabular} & \textbf{Code}  & \begin{tabular}[c]{@{}c@{}}\textbf{Natural}\\ \textbf{Language}\end{tabular} & \begin{tabular}[c]{@{}c@{}}\textbf{Action}\\ \textbf{Trajectories}\end{tabular} & \begin{tabular}[c]{@{}c@{}}\textbf{API}\\ \textbf{Documentation}\end{tabular}\\ \midrule 
% NexusRaven~\citep{srinivasan2023nexusraven} & IFT & - & - & \textcolor{green}{\CheckmarkBold} & \textcolor{green}{\CheckmarkBold} &\textcolor{red}{\XSolidBrush}&\textcolor{red}{\XSolidBrush}\\
% AgentInstruct~\citep{zeng2023agenttuning} & IFT & - & 2k & \textcolor{green}{\CheckmarkBold} & \textcolor{green}{\CheckmarkBold} &\textcolor{red}{\XSolidBrush}&\textcolor{red}{\XSolidBrush} \\
% AgentEvol~\citep{xi2024agentgym} & IFT & - & 14.5k & \textcolor{green}{\CheckmarkBold} & \textcolor{green}{\CheckmarkBold} &\textcolor{green}{\CheckmarkBold}&\textcolor{red}{\XSolidBrush} \\
% Gorilla~\citep{patil2023gorilla}& IFT & - & 16k & \textcolor{green}{\CheckmarkBold} & \textcolor{green}{\CheckmarkBold} &\textcolor{red}{\XSolidBrush}&\textcolor{green}{\CheckmarkBold}\\
% OpenFunctions-v2~\citep{patil2023gorilla} & IFT & - & 65k & \textcolor{green}{\CheckmarkBold} & \textcolor{green}{\CheckmarkBold} &\textcolor{red}{\XSolidBrush}&\textcolor{green}{\CheckmarkBold}\\
% LAM~\citep{zhang2024agentohana} & IFT & - & 42.6k & \textcolor{green}{\CheckmarkBold} & \textcolor{green}{\CheckmarkBold} &\textcolor{green}{\CheckmarkBold}&\textcolor{red}{\XSolidBrush} \\
% xLAM~\citep{liu2024apigen} & IFT & - & 60k & \textcolor{green}{\CheckmarkBold} & \textcolor{green}{\CheckmarkBold} &\textcolor{green}{\CheckmarkBold}&\textcolor{red}{\XSolidBrush} \\\midrule
% LEMUR~\citep{xu2024lemur} & PT & 90B & 300k & \textcolor{green}{\CheckmarkBold} & \textcolor{green}{\CheckmarkBold} &\textcolor{green}{\CheckmarkBold}&\textcolor{red}{\XSolidBrush}\\
% \rowcolor{teal!12} \method & PT & 103B & 95k & \textcolor{green}{\CheckmarkBold} & \textcolor{green}{\CheckmarkBold} & \textcolor{green}{\CheckmarkBold} & \textcolor{green}{\CheckmarkBold} \\
% \bottomrule
% \end{tabular}
% \caption{Summary of existing tuning- and pretraining-based LLM agents with their training sample sizes. "PT" and "IFT" denote "Pre-Training" and "Instruction Fine-Tuning", respectively. }
% \label{tab:related}
% \end{table*}

\begin{table*}[ht]
\begin{threeparttable}
\centering 
\renewcommand\arraystretch{0.98}
\fontsize{7}{9}\selectfont \setlength{\tabcolsep}{0.2em}
\begin{tabular}{@{}l|c|c|ccc|cc|cc|cccc@{}}
\toprule
\textbf{Methods} & \textbf{Datasets}           & \begin{tabular}[c]{@{}c@{}}\textbf{Training}\\ \textbf{Paradigm}\end{tabular} & \begin{tabular}[c]{@{}c@{}}\textbf{\# PT Data}\\ \textbf{(Tokens)}\end{tabular} & \begin{tabular}[c]{@{}c@{}}\textbf{\# IFT Data}\\ \textbf{(Samples)}\end{tabular} & \textbf{\# APIs} & \textbf{Code}  & \begin{tabular}[c]{@{}c@{}}\textbf{Nat.}\\ \textbf{Lang.}\end{tabular} & \begin{tabular}[c]{@{}c@{}}\textbf{Action}\\ \textbf{Traj.}\end{tabular} & \begin{tabular}[c]{@{}c@{}}\textbf{API}\\ \textbf{Doc.}\end{tabular} & \begin{tabular}[c]{@{}c@{}}\textbf{Func.}\\ \textbf{Call}\end{tabular} & \begin{tabular}[c]{@{}c@{}}\textbf{Multi.}\\ \textbf{Step}\end{tabular}  & \begin{tabular}[c]{@{}c@{}}\textbf{Plan}\\ \textbf{Refine}\end{tabular}  & \begin{tabular}[c]{@{}c@{}}\textbf{Multi.}\\ \textbf{Turn}\end{tabular}\\ \midrule 
\multicolumn{13}{l}{\emph{Instruction Finetuning-based LLM Agents for Intrinsic Reasoning}}  \\ \midrule
FireAct~\cite{chen2023fireact} & FireAct & IFT & - & 2.1K & 10 & \textcolor{red}{\XSolidBrush} &\textcolor{green}{\CheckmarkBold} &\textcolor{green}{\CheckmarkBold}  & \textcolor{red}{\XSolidBrush} &\textcolor{green}{\CheckmarkBold} & \textcolor{red}{\XSolidBrush} &\textcolor{green}{\CheckmarkBold} & \textcolor{red}{\XSolidBrush} \\
ToolAlpaca~\cite{tang2023toolalpaca} & ToolAlpaca & IFT & - & 4.0K & 400 & \textcolor{red}{\XSolidBrush} &\textcolor{green}{\CheckmarkBold} &\textcolor{green}{\CheckmarkBold} & \textcolor{red}{\XSolidBrush} &\textcolor{green}{\CheckmarkBold} & \textcolor{red}{\XSolidBrush}  &\textcolor{green}{\CheckmarkBold} & \textcolor{red}{\XSolidBrush}  \\
ToolLLaMA~\cite{qin2023toolllm} & ToolBench & IFT & - & 12.7K & 16,464 & \textcolor{red}{\XSolidBrush} &\textcolor{green}{\CheckmarkBold} &\textcolor{green}{\CheckmarkBold} &\textcolor{red}{\XSolidBrush} &\textcolor{green}{\CheckmarkBold}&\textcolor{green}{\CheckmarkBold}&\textcolor{green}{\CheckmarkBold} &\textcolor{green}{\CheckmarkBold}\\
AgentEvol~\citep{xi2024agentgym} & AgentTraj-L & IFT & - & 14.5K & 24 &\textcolor{red}{\XSolidBrush} & \textcolor{green}{\CheckmarkBold} &\textcolor{green}{\CheckmarkBold}&\textcolor{red}{\XSolidBrush} &\textcolor{green}{\CheckmarkBold}&\textcolor{red}{\XSolidBrush} &\textcolor{red}{\XSolidBrush} &\textcolor{green}{\CheckmarkBold}\\
Lumos~\cite{yin2024agent} & Lumos & IFT  & - & 20.0K & 16 &\textcolor{red}{\XSolidBrush} & \textcolor{green}{\CheckmarkBold} & \textcolor{green}{\CheckmarkBold} &\textcolor{red}{\XSolidBrush} & \textcolor{green}{\CheckmarkBold} & \textcolor{green}{\CheckmarkBold} &\textcolor{red}{\XSolidBrush} & \textcolor{green}{\CheckmarkBold}\\
Agent-FLAN~\cite{chen2024agent} & Agent-FLAN & IFT & - & 24.7K & 20 &\textcolor{red}{\XSolidBrush} & \textcolor{green}{\CheckmarkBold} & \textcolor{green}{\CheckmarkBold} &\textcolor{red}{\XSolidBrush} & \textcolor{green}{\CheckmarkBold}& \textcolor{green}{\CheckmarkBold}&\textcolor{red}{\XSolidBrush} & \textcolor{green}{\CheckmarkBold}\\
AgentTuning~\citep{zeng2023agenttuning} & AgentInstruct & IFT & - & 35.0K & - &\textcolor{red}{\XSolidBrush} & \textcolor{green}{\CheckmarkBold} & \textcolor{green}{\CheckmarkBold} &\textcolor{red}{\XSolidBrush} & \textcolor{green}{\CheckmarkBold} &\textcolor{red}{\XSolidBrush} &\textcolor{red}{\XSolidBrush} & \textcolor{green}{\CheckmarkBold}\\\midrule
\multicolumn{13}{l}{\emph{Instruction Finetuning-based LLM Agents for Function Calling}} \\\midrule
NexusRaven~\citep{srinivasan2023nexusraven} & NexusRaven & IFT & - & - & 116 & \textcolor{green}{\CheckmarkBold} & \textcolor{green}{\CheckmarkBold}  & \textcolor{green}{\CheckmarkBold} &\textcolor{red}{\XSolidBrush} & \textcolor{green}{\CheckmarkBold} &\textcolor{red}{\XSolidBrush} &\textcolor{red}{\XSolidBrush}&\textcolor{red}{\XSolidBrush}\\
Gorilla~\citep{patil2023gorilla} & Gorilla & IFT & - & 16.0K & 1,645 & \textcolor{green}{\CheckmarkBold} &\textcolor{red}{\XSolidBrush} &\textcolor{red}{\XSolidBrush}&\textcolor{green}{\CheckmarkBold} &\textcolor{green}{\CheckmarkBold} &\textcolor{red}{\XSolidBrush} &\textcolor{red}{\XSolidBrush} &\textcolor{red}{\XSolidBrush}\\
OpenFunctions-v2~\citep{patil2023gorilla} & OpenFunctions-v2 & IFT & - & 65.0K & - & \textcolor{green}{\CheckmarkBold} & \textcolor{green}{\CheckmarkBold} &\textcolor{red}{\XSolidBrush} &\textcolor{green}{\CheckmarkBold} &\textcolor{green}{\CheckmarkBold} &\textcolor{red}{\XSolidBrush} &\textcolor{red}{\XSolidBrush} &\textcolor{red}{\XSolidBrush}\\
API Pack~\cite{guo2024api} & API Pack & IFT & - & 1.1M & 11,213 &\textcolor{green}{\CheckmarkBold} &\textcolor{red}{\XSolidBrush} &\textcolor{green}{\CheckmarkBold} &\textcolor{red}{\XSolidBrush} &\textcolor{green}{\CheckmarkBold} &\textcolor{red}{\XSolidBrush}&\textcolor{red}{\XSolidBrush}&\textcolor{red}{\XSolidBrush}\\ 
LAM~\citep{zhang2024agentohana} & AgentOhana & IFT & - & 42.6K & - & \textcolor{green}{\CheckmarkBold} & \textcolor{green}{\CheckmarkBold} &\textcolor{green}{\CheckmarkBold}&\textcolor{red}{\XSolidBrush} &\textcolor{green}{\CheckmarkBold}&\textcolor{red}{\XSolidBrush}&\textcolor{green}{\CheckmarkBold}&\textcolor{green}{\CheckmarkBold}\\
xLAM~\citep{liu2024apigen} & APIGen & IFT & - & 60.0K & 3,673 & \textcolor{green}{\CheckmarkBold} & \textcolor{green}{\CheckmarkBold} &\textcolor{green}{\CheckmarkBold}&\textcolor{red}{\XSolidBrush} &\textcolor{green}{\CheckmarkBold}&\textcolor{red}{\XSolidBrush}&\textcolor{green}{\CheckmarkBold}&\textcolor{green}{\CheckmarkBold}\\\midrule
\multicolumn{13}{l}{\emph{Pretraining-based LLM Agents}}  \\\midrule
% LEMUR~\citep{xu2024lemur} & PT & 90B & 300.0K & - & \textcolor{green}{\CheckmarkBold} & \textcolor{green}{\CheckmarkBold} &\textcolor{green}{\CheckmarkBold}&\textcolor{red}{\XSolidBrush} & \textcolor{red}{\XSolidBrush} &\textcolor{green}{\CheckmarkBold} &\textcolor{red}{\XSolidBrush}&\textcolor{red}{\XSolidBrush}\\
\rowcolor{teal!12} \method & \dataset & PT & 103B & 95.0K  & 76,537  & \textcolor{green}{\CheckmarkBold} & \textcolor{green}{\CheckmarkBold} & \textcolor{green}{\CheckmarkBold} & \textcolor{green}{\CheckmarkBold} & \textcolor{green}{\CheckmarkBold} & \textcolor{green}{\CheckmarkBold} & \textcolor{green}{\CheckmarkBold} & \textcolor{green}{\CheckmarkBold}\\
\bottomrule
\end{tabular}
% \begin{tablenotes}
%     \item $^*$ In addition, the StarCoder-API can offer 4.77M more APIs.
% \end{tablenotes}
\caption{Summary of existing instruction finetuning-based LLM agents for intrinsic reasoning and function calling, along with their training resources and sample sizes. "PT" and "IFT" denote "Pre-Training" and "Instruction Fine-Tuning", respectively.}
\vspace{-2ex}
\label{tab:related}
\end{threeparttable}
\end{table*}

\noindent \textbf{Prompting-based LLM Agents.} Due to the lack of agent-specific pre-training corpus, existing LLM agents rely on either prompt engineering~\cite{hsieh2023tool,lu2024chameleon,yao2022react,wang2023voyager} or instruction fine-tuning~\cite{chen2023fireact,zeng2023agenttuning} to understand human instructions, decompose high-level tasks, generate grounded plans, and execute multi-step actions. 
However, prompting-based methods mainly depend on the capabilities of backbone LLMs (usually commercial LLMs), failing to introduce new knowledge and struggling to generalize to unseen tasks~\cite{sun2024adaplanner,zhuang2023toolchain}. 

\noindent \textbf{Instruction Finetuning-based LLM Agents.} Considering the extensive diversity of APIs and the complexity of multi-tool instructions, tool learning inherently presents greater challenges than natural language tasks, such as text generation~\cite{qin2023toolllm}.
Post-training techniques focus more on instruction following and aligning output with specific formats~\cite{patil2023gorilla,hao2024toolkengpt,qin2023toolllm,schick2024toolformer}, rather than fundamentally improving model knowledge or capabilities. 
Moreover, heavy fine-tuning can hinder generalization or even degrade performance in non-agent use cases, potentially suppressing the original base model capabilities~\cite{ghosh2024a}.

\noindent \textbf{Pretraining-based LLM Agents.} While pre-training serves as an essential alternative, prior works~\cite{nijkamp2023codegen,roziere2023code,xu2024lemur,patil2023gorilla} have primarily focused on improving task-specific capabilities (\eg, code generation) instead of general-domain LLM agents, due to single-source, uni-type, small-scale, and poor-quality pre-training data. 
Existing tool documentation data for agent training either lacks diverse real-world APIs~\cite{patil2023gorilla, tang2023toolalpaca} or is constrained to single-tool or single-round tool execution. 
Furthermore, trajectory data mostly imitate expert behavior or follow function-calling rules with inferior planning and reasoning, failing to fully elicit LLMs' capabilities and handle complex instructions~\cite{qin2023toolllm}. 
Given a wide range of candidate API functions, each comprising various function names and parameters available at every planning step, identifying globally optimal solutions and generalizing across tasks remains highly challenging.



\section{Preliminaries}
\label{Preliminaries}
\begin{figure*}[t]
    \centering
    \includegraphics[width=0.95\linewidth]{fig/HealthGPT_Framework.png}
    \caption{The \ourmethod{} architecture integrates hierarchical visual perception and H-LoRA, employing a task-specific hard router to select visual features and H-LoRA plugins, ultimately generating outputs with an autoregressive manner.}
    \label{fig:architecture}
\end{figure*}
\noindent\textbf{Large Vision-Language Models.} 
The input to a LVLM typically consists of an image $x^{\text{img}}$ and a discrete text sequence $x^{\text{txt}}$. The visual encoder $\mathcal{E}^{\text{img}}$ converts the input image $x^{\text{img}}$ into a sequence of visual tokens $\mathcal{V} = [v_i]_{i=1}^{N_v}$, while the text sequence $x^{\text{txt}}$ is mapped into a sequence of text tokens $\mathcal{T} = [t_i]_{i=1}^{N_t}$ using an embedding function $\mathcal{E}^{\text{txt}}$. The LLM $\mathcal{M_\text{LLM}}(\cdot|\theta)$ models the joint probability of the token sequence $\mathcal{U} = \{\mathcal{V},\mathcal{T}\}$, which is expressed as:
\begin{equation}
    P_\theta(R | \mathcal{U}) = \prod_{i=1}^{N_r} P_\theta(r_i | \{\mathcal{U}, r_{<i}\}),
\end{equation}
where $R = [r_i]_{i=1}^{N_r}$ is the text response sequence. The LVLM iteratively generates the next token $r_i$ based on $r_{<i}$. The optimization objective is to minimize the cross-entropy loss of the response $\mathcal{R}$.
% \begin{equation}
%     \mathcal{L}_{\text{VLM}} = \mathbb{E}_{R|\mathcal{U}}\left[-\log P_\theta(R | \mathcal{U})\right]
% \end{equation}
It is worth noting that most LVLMs adopt a design paradigm based on ViT, alignment adapters, and pre-trained LLMs\cite{liu2023llava,liu2024improved}, enabling quick adaptation to downstream tasks.


\noindent\textbf{VQGAN.}
VQGAN~\cite{esser2021taming} employs latent space compression and indexing mechanisms to effectively learn a complete discrete representation of images. VQGAN first maps the input image $x^{\text{img}}$ to a latent representation $z = \mathcal{E}(x)$ through a encoder $\mathcal{E}$. Then, the latent representation is quantized using a codebook $\mathcal{Z} = \{z_k\}_{k=1}^K$, generating a discrete index sequence $\mathcal{I} = [i_m]_{m=1}^N$, where $i_m \in \mathcal{Z}$ represents the quantized code index:
\begin{equation}
    \mathcal{I} = \text{Quantize}(z|\mathcal{Z}) = \arg\min_{z_k \in \mathcal{Z}} \| z - z_k \|_2.
\end{equation}
In our approach, the discrete index sequence $\mathcal{I}$ serves as a supervisory signal for the generation task, enabling the model to predict the index sequence $\hat{\mathcal{I}}$ from input conditions such as text or other modality signals.  
Finally, the predicted index sequence $\hat{\mathcal{I}}$ is upsampled by the VQGAN decoder $G$, generating the high-quality image $\hat{x}^\text{img} = G(\hat{\mathcal{I}})$.



\noindent\textbf{Low Rank Adaptation.} 
LoRA\cite{hu2021lora} effectively captures the characteristics of downstream tasks by introducing low-rank adapters. The core idea is to decompose the bypass weight matrix $\Delta W\in\mathbb{R}^{d^{\text{in}} \times d^{\text{out}}}$ into two low-rank matrices $ \{A \in \mathbb{R}^{d^{\text{in}} \times r}, B \in \mathbb{R}^{r \times d^{\text{out}}} \}$, where $ r \ll \min\{d^{\text{in}}, d^{\text{out}}\} $, significantly reducing learnable parameters. The output with the LoRA adapter for the input $x$ is then given by:
\begin{equation}
    h = x W_0 + \alpha x \Delta W/r = x W_0 + \alpha xAB/r,
\end{equation}
where matrix $ A $ is initialized with a Gaussian distribution, while the matrix $ B $ is initialized as a zero matrix. The scaling factor $ \alpha/r $ controls the impact of $ \Delta W $ on the model.

\section{HealthGPT}
\label{Method}


\subsection{Unified Autoregressive Generation.}  
% As shown in Figure~\ref{fig:architecture}, 
\ourmethod{} (Figure~\ref{fig:architecture}) utilizes a discrete token representation that covers both text and visual outputs, unifying visual comprehension and generation as an autoregressive task. 
For comprehension, $\mathcal{M}_\text{llm}$ receives the input joint sequence $\mathcal{U}$ and outputs a series of text token $\mathcal{R} = [r_1, r_2, \dots, r_{N_r}]$, where $r_i \in \mathcal{V}_{\text{txt}}$, and $\mathcal{V}_{\text{txt}}$ represents the LLM's vocabulary:
\begin{equation}
    P_\theta(\mathcal{R} \mid \mathcal{U}) = \prod_{i=1}^{N_r} P_\theta(r_i \mid \mathcal{U}, r_{<i}).
\end{equation}
For generation, $\mathcal{M}_\text{llm}$ first receives a special start token $\langle \text{START\_IMG} \rangle$, then generates a series of tokens corresponding to the VQGAN indices $\mathcal{I} = [i_1, i_2, \dots, i_{N_i}]$, where $i_j \in \mathcal{V}_{\text{vq}}$, and $\mathcal{V}_{\text{vq}}$ represents the index range of VQGAN. Upon completion of generation, the LLM outputs an end token $\langle \text{END\_IMG} \rangle$:
\begin{equation}
    P_\theta(\mathcal{I} \mid \mathcal{U}) = \prod_{j=1}^{N_i} P_\theta(i_j \mid \mathcal{U}, i_{<j}).
\end{equation}
Finally, the generated index sequence $\mathcal{I}$ is fed into the decoder $G$, which reconstructs the target image $\hat{x}^{\text{img}} = G(\mathcal{I})$.

\subsection{Hierarchical Visual Perception}  
Given the differences in visual perception between comprehension and generation tasks—where the former focuses on abstract semantics and the latter emphasizes complete semantics—we employ ViT to compress the image into discrete visual tokens at multiple hierarchical levels.
Specifically, the image is converted into a series of features $\{f_1, f_2, \dots, f_L\}$ as it passes through $L$ ViT blocks.

To address the needs of various tasks, the hidden states are divided into two types: (i) \textit{Concrete-grained features} $\mathcal{F}^{\text{Con}} = \{f_1, f_2, \dots, f_k\}, k < L$, derived from the shallower layers of ViT, containing sufficient global features, suitable for generation tasks; 
(ii) \textit{Abstract-grained features} $\mathcal{F}^{\text{Abs}} = \{f_{k+1}, f_{k+2}, \dots, f_L\}$, derived from the deeper layers of ViT, which contain abstract semantic information closer to the text space, suitable for comprehension tasks.

The task type $T$ (comprehension or generation) determines which set of features is selected as the input for the downstream large language model:
\begin{equation}
    \mathcal{F}^{\text{img}}_T =
    \begin{cases}
        \mathcal{F}^{\text{Con}}, & \text{if } T = \text{generation task} \\
        \mathcal{F}^{\text{Abs}}, & \text{if } T = \text{comprehension task}
    \end{cases}
\end{equation}
We integrate the image features $\mathcal{F}^{\text{img}}_T$ and text features $\mathcal{T}$ into a joint sequence through simple concatenation, which is then fed into the LLM $\mathcal{M}_{\text{llm}}$ for autoregressive generation.
% :
% \begin{equation}
%     \mathcal{R} = \mathcal{M}_{\text{llm}}(\mathcal{U}|\theta), \quad \mathcal{U} = [\mathcal{F}^{\text{img}}_T; \mathcal{T}]
% \end{equation}
\subsection{Heterogeneous Knowledge Adaptation}
We devise H-LoRA, which stores heterogeneous knowledge from comprehension and generation tasks in separate modules and dynamically routes to extract task-relevant knowledge from these modules. 
At the task level, for each task type $ T $, we dynamically assign a dedicated H-LoRA submodule $ \theta^T $, which is expressed as:
\begin{equation}
    \mathcal{R} = \mathcal{M}_\text{LLM}(\mathcal{U}|\theta, \theta^T), \quad \theta^T = \{A^T, B^T, \mathcal{R}^T_\text{outer}\}.
\end{equation}
At the feature level for a single task, H-LoRA integrates the idea of Mixture of Experts (MoE)~\cite{masoudnia2014mixture} and designs an efficient matrix merging and routing weight allocation mechanism, thus avoiding the significant computational delay introduced by matrix splitting in existing MoELoRA~\cite{luo2024moelora}. Specifically, we first merge the low-rank matrices (rank = r) of $ k $ LoRA experts into a unified matrix:
\begin{equation}
    \mathbf{A}^{\text{merged}}, \mathbf{B}^{\text{merged}} = \text{Concat}(\{A_i\}_1^k), \text{Concat}(\{B_i\}_1^k),
\end{equation}
where $ \mathbf{A}^{\text{merged}} \in \mathbb{R}^{d^\text{in} \times rk} $ and $ \mathbf{B}^{\text{merged}} \in \mathbb{R}^{rk \times d^\text{out}} $. The $k$-dimension routing layer generates expert weights $ \mathcal{W} \in \mathbb{R}^{\text{token\_num} \times k} $ based on the input hidden state $ x $, and these are expanded to $ \mathbb{R}^{\text{token\_num} \times rk} $ as follows:
\begin{equation}
    \mathcal{W}^\text{expanded} = \alpha k \mathcal{W} / r \otimes \mathbf{1}_r,
\end{equation}
where $ \otimes $ denotes the replication operation.
The overall output of H-LoRA is computed as:
\begin{equation}
    \mathcal{O}^\text{H-LoRA} = (x \mathbf{A}^{\text{merged}} \odot \mathcal{W}^\text{expanded}) \mathbf{B}^{\text{merged}},
\end{equation}
where $ \odot $ represents element-wise multiplication. Finally, the output of H-LoRA is added to the frozen pre-trained weights to produce the final output:
\begin{equation}
    \mathcal{O} = x W_0 + \mathcal{O}^\text{H-LoRA}.
\end{equation}
% In summary, H-LoRA is a task-based dynamic PEFT method that achieves high efficiency in single-task fine-tuning.

\subsection{Training Pipeline}

\begin{figure}[t]
    \centering
    \hspace{-4mm}
    \includegraphics[width=0.94\linewidth]{fig/data.pdf}
    \caption{Data statistics of \texttt{VL-Health}. }
    \label{fig:data}
\end{figure}
\noindent \textbf{1st Stage: Multi-modal Alignment.} 
In the first stage, we design separate visual adapters and H-LoRA submodules for medical unified tasks. For the medical comprehension task, we train abstract-grained visual adapters using high-quality image-text pairs to align visual embeddings with textual embeddings, thereby enabling the model to accurately describe medical visual content. During this process, the pre-trained LLM and its corresponding H-LoRA submodules remain frozen. In contrast, the medical generation task requires training concrete-grained adapters and H-LoRA submodules while keeping the LLM frozen. Meanwhile, we extend the textual vocabulary to include multimodal tokens, enabling the support of additional VQGAN vector quantization indices. The model trains on image-VQ pairs, endowing the pre-trained LLM with the capability for image reconstruction. This design ensures pixel-level consistency of pre- and post-LVLM. The processes establish the initial alignment between the LLM’s outputs and the visual inputs.

\noindent \textbf{2nd Stage: Heterogeneous H-LoRA Plugin Adaptation.}  
The submodules of H-LoRA share the word embedding layer and output head but may encounter issues such as bias and scale inconsistencies during training across different tasks. To ensure that the multiple H-LoRA plugins seamlessly interface with the LLMs and form a unified base, we fine-tune the word embedding layer and output head using a small amount of mixed data to maintain consistency in the model weights. Specifically, during this stage, all H-LoRA submodules for different tasks are kept frozen, with only the word embedding layer and output head being optimized. Through this stage, the model accumulates foundational knowledge for unified tasks by adapting H-LoRA plugins.

\begin{table*}[!t]
\centering
\caption{Comparison of \ourmethod{} with other LVLMs and unified multi-modal models on medical visual comprehension tasks. \textbf{Bold} and \underline{underlined} text indicates the best performance and second-best performance, respectively.}
\resizebox{\textwidth}{!}{
\begin{tabular}{c|lcc|cccccccc|c}
\toprule
\rowcolor[HTML]{E9F3FE} &  &  &  & \multicolumn{2}{c}{\textbf{VQA-RAD \textuparrow}} & \multicolumn{2}{c}{\textbf{SLAKE \textuparrow}} & \multicolumn{2}{c}{\textbf{PathVQA \textuparrow}} &  &  &  \\ 
\cline{5-10}
\rowcolor[HTML]{E9F3FE}\multirow{-2}{*}{\textbf{Type}} & \multirow{-2}{*}{\textbf{Model}} & \multirow{-2}{*}{\textbf{\# Params}} & \multirow{-2}{*}{\makecell{\textbf{Medical} \\ \textbf{LVLM}}} & \textbf{close} & \textbf{all} & \textbf{close} & \textbf{all} & \textbf{close} & \textbf{all} & \multirow{-2}{*}{\makecell{\textbf{MMMU} \\ \textbf{-Med}}\textuparrow} & \multirow{-2}{*}{\textbf{OMVQA}\textuparrow} & \multirow{-2}{*}{\textbf{Avg. \textuparrow}} \\ 
\midrule \midrule
\multirow{9}{*}{\textbf{Comp. Only}} 
& Med-Flamingo & 8.3B & \Large \ding{51} & 58.6 & 43.0 & 47.0 & 25.5 & 61.9 & 31.3 & 28.7 & 34.9 & 41.4 \\
& LLaVA-Med & 7B & \Large \ding{51} & 60.2 & 48.1 & 58.4 & 44.8 & 62.3 & 35.7 & 30.0 & 41.3 & 47.6 \\
& HuatuoGPT-Vision & 7B & \Large \ding{51} & 66.9 & 53.0 & 59.8 & 49.1 & 52.9 & 32.0 & 42.0 & 50.0 & 50.7 \\
& BLIP-2 & 6.7B & \Large \ding{55} & 43.4 & 36.8 & 41.6 & 35.3 & 48.5 & 28.8 & 27.3 & 26.9 & 36.1 \\
& LLaVA-v1.5 & 7B & \Large \ding{55} & 51.8 & 42.8 & 37.1 & 37.7 & 53.5 & 31.4 & 32.7 & 44.7 & 41.5 \\
& InstructBLIP & 7B & \Large \ding{55} & 61.0 & 44.8 & 66.8 & 43.3 & 56.0 & 32.3 & 25.3 & 29.0 & 44.8 \\
& Yi-VL & 6B & \Large \ding{55} & 52.6 & 42.1 & 52.4 & 38.4 & 54.9 & 30.9 & 38.0 & 50.2 & 44.9 \\
& InternVL2 & 8B & \Large \ding{55} & 64.9 & 49.0 & 66.6 & 50.1 & 60.0 & 31.9 & \underline{43.3} & 54.5 & 52.5\\
& Llama-3.2 & 11B & \Large \ding{55} & 68.9 & 45.5 & 72.4 & 52.1 & 62.8 & 33.6 & 39.3 & 63.2 & 54.7 \\
\midrule
\multirow{5}{*}{\textbf{Comp. \& Gen.}} 
& Show-o & 1.3B & \Large \ding{55} & 50.6 & 33.9 & 31.5 & 17.9 & 52.9 & 28.2 & 22.7 & 45.7 & 42.6 \\
& Unified-IO 2 & 7B & \Large \ding{55} & 46.2 & 32.6 & 35.9 & 21.9 & 52.5 & 27.0 & 25.3 & 33.0 & 33.8 \\
& Janus & 1.3B & \Large \ding{55} & 70.9 & 52.8 & 34.7 & 26.9 & 51.9 & 27.9 & 30.0 & 26.8 & 33.5 \\
& \cellcolor[HTML]{DAE0FB}HealthGPT-M3 & \cellcolor[HTML]{DAE0FB}3.8B & \cellcolor[HTML]{DAE0FB}\Large \ding{51} & \cellcolor[HTML]{DAE0FB}\underline{73.7} & \cellcolor[HTML]{DAE0FB}\underline{55.9} & \cellcolor[HTML]{DAE0FB}\underline{74.6} & \cellcolor[HTML]{DAE0FB}\underline{56.4} & \cellcolor[HTML]{DAE0FB}\underline{78.7} & \cellcolor[HTML]{DAE0FB}\underline{39.7} & \cellcolor[HTML]{DAE0FB}\underline{43.3} & \cellcolor[HTML]{DAE0FB}\underline{68.5} & \cellcolor[HTML]{DAE0FB}\underline{61.3} \\
& \cellcolor[HTML]{DAE0FB}HealthGPT-L14 & \cellcolor[HTML]{DAE0FB}14B & \cellcolor[HTML]{DAE0FB}\Large \ding{51} & \cellcolor[HTML]{DAE0FB}\textbf{77.7} & \cellcolor[HTML]{DAE0FB}\textbf{58.3} & \cellcolor[HTML]{DAE0FB}\textbf{76.4} & \cellcolor[HTML]{DAE0FB}\textbf{64.5} & \cellcolor[HTML]{DAE0FB}\textbf{85.9} & \cellcolor[HTML]{DAE0FB}\textbf{44.4} & \cellcolor[HTML]{DAE0FB}\textbf{49.2} & \cellcolor[HTML]{DAE0FB}\textbf{74.4} & \cellcolor[HTML]{DAE0FB}\textbf{66.4} \\
\bottomrule
\end{tabular}
}
\label{tab:results}
\end{table*}
\begin{table*}[ht]
    \centering
    \caption{The experimental results for the four modality conversion tasks.}
    \resizebox{\textwidth}{!}{
    \begin{tabular}{l|ccc|ccc|ccc|ccc}
        \toprule
        \rowcolor[HTML]{E9F3FE} & \multicolumn{3}{c}{\textbf{CT to MRI (Brain)}} & \multicolumn{3}{c}{\textbf{CT to MRI (Pelvis)}} & \multicolumn{3}{c}{\textbf{MRI to CT (Brain)}} & \multicolumn{3}{c}{\textbf{MRI to CT (Pelvis)}} \\
        \cline{2-13}
        \rowcolor[HTML]{E9F3FE}\multirow{-2}{*}{\textbf{Model}}& \textbf{SSIM $\uparrow$} & \textbf{PSNR $\uparrow$} & \textbf{MSE $\downarrow$} & \textbf{SSIM $\uparrow$} & \textbf{PSNR $\uparrow$} & \textbf{MSE $\downarrow$} & \textbf{SSIM $\uparrow$} & \textbf{PSNR $\uparrow$} & \textbf{MSE $\downarrow$} & \textbf{SSIM $\uparrow$} & \textbf{PSNR $\uparrow$} & \textbf{MSE $\downarrow$} \\
        \midrule \midrule
        pix2pix & 71.09 & 32.65 & 36.85 & 59.17 & 31.02 & 51.91 & 78.79 & 33.85 & 28.33 & 72.31 & 32.98 & 36.19 \\
        CycleGAN & 54.76 & 32.23 & 40.56 & 54.54 & 30.77 & 55.00 & 63.75 & 31.02 & 52.78 & 50.54 & 29.89 & 67.78 \\
        BBDM & {71.69} & {32.91} & {34.44} & 57.37 & 31.37 & 48.06 & \textbf{86.40} & 34.12 & 26.61 & {79.26} & 33.15 & 33.60 \\
        Vmanba & 69.54 & 32.67 & 36.42 & {63.01} & {31.47} & {46.99} & 79.63 & 34.12 & 26.49 & 77.45 & 33.53 & 31.85 \\
        DiffMa & 71.47 & 32.74 & 35.77 & 62.56 & 31.43 & 47.38 & 79.00 & {34.13} & {26.45} & 78.53 & {33.68} & {30.51} \\
        \rowcolor[HTML]{DAE0FB}HealthGPT-M3 & \underline{79.38} & \underline{33.03} & \underline{33.48} & \underline{71.81} & \underline{31.83} & \underline{43.45} & {85.06} & \textbf{34.40} & \textbf{25.49} & \underline{84.23} & \textbf{34.29} & \textbf{27.99} \\
        \rowcolor[HTML]{DAE0FB}HealthGPT-L14 & \textbf{79.73} & \textbf{33.10} & \textbf{32.96} & \textbf{71.92} & \textbf{31.87} & \textbf{43.09} & \underline{85.31} & \underline{34.29} & \underline{26.20} & \textbf{84.96} & \underline{34.14} & \underline{28.13} \\
        \bottomrule
    \end{tabular}
    }
    \label{tab:conversion}
\end{table*}

\noindent \textbf{3rd Stage: Visual Instruction Fine-Tuning.}  
In the third stage, we introduce additional task-specific data to further optimize the model and enhance its adaptability to downstream tasks such as medical visual comprehension (e.g., medical QA, medical dialogues, and report generation) or generation tasks (e.g., super-resolution, denoising, and modality conversion). Notably, by this stage, the word embedding layer and output head have been fine-tuned, only the H-LoRA modules and adapter modules need to be trained. This strategy significantly improves the model's adaptability and flexibility across different tasks.


\section{Experiment}
\label{s:experiment}

\subsection{Data Description}
We evaluate our method on FI~\cite{you2016building}, Twitter\_LDL~\cite{yang2017learning} and Artphoto~\cite{machajdik2010affective}.
FI is a public dataset built from Flickr and Instagram, with 23,308 images and eight emotion categories, namely \textit{amusement}, \textit{anger}, \textit{awe},  \textit{contentment}, \textit{disgust}, \textit{excitement},  \textit{fear}, and \textit{sadness}. 
% Since images in FI are all copyrighted by law, some images are corrupted now, so we remove these samples and retain 21,828 images.
% T4SA contains images from Twitter, which are classified into three categories: \textit{positive}, \textit{neutral}, and \textit{negative}. In this paper, we adopt the base version of B-T4SA, which contains 470,586 images and provides text descriptions of the corresponding tweets.
Twitter\_LDL contains 10,045 images from Twitter, with the same eight categories as the FI dataset.
% 。
For these two datasets, they are randomly split into 80\%
training and 20\% testing set.
Artphoto contains 806 artistic photos from the DeviantArt website, which we use to further evaluate the zero-shot capability of our model.
% on the small-scale dataset.
% We construct and publicly release the first image sentiment analysis dataset containing metadata.
% 。

% Based on these datasets, we are the first to construct and publicly release metadata-enhanced image sentiment analysis datasets. These datasets include scenes, tags, descriptions, and corresponding confidence scores, and are available at this link for future research purposes.


% 
\begin{table}[t]
\centering
% \begin{center}
\caption{Overall performance of different models on FI and Twitter\_LDL datasets.}
\label{tab:cap1}
% \resizebox{\linewidth}{!}
{
\begin{tabular}{l|c|c|c|c}
\hline
\multirow{2}{*}{\textbf{Model}} & \multicolumn{2}{c|}{\textbf{FI}}  & \multicolumn{2}{c}{\textbf{Twitter\_LDL}} \\ \cline{2-5} 
  & \textbf{Accuracy} & \textbf{F1} & \textbf{Accuracy} & \textbf{F1}  \\ \hline
% (\rownumber)~AlexNet~\cite{krizhevsky2017imagenet}  & 58.13\% & 56.35\%  & 56.24\%& 55.02\%  \\ 
% (\rownumber)~VGG16~\cite{simonyan2014very}  & 63.75\%& 63.08\%  & 59.34\%& 59.02\%  \\ 
(\rownumber)~ResNet101~\cite{he2016deep} & 66.16\%& 65.56\%  & 62.02\% & 61.34\%  \\ 
(\rownumber)~CDA~\cite{han2023boosting} & 66.71\%& 65.37\%  & 64.14\% & 62.85\%  \\ 
(\rownumber)~CECCN~\cite{ruan2024color} & 67.96\%& 66.74\%  & 64.59\%& 64.72\% \\ 
(\rownumber)~EmoVIT~\cite{xie2024emovit} & 68.09\%& 67.45\%  & 63.12\% & 61.97\%  \\ 
(\rownumber)~ComLDL~\cite{zhang2022compound} & 68.83\%& 67.28\%  & 65.29\% & 63.12\%  \\ 
(\rownumber)~WSDEN~\cite{li2023weakly} & 69.78\%& 69.61\%  & 67.04\% & 65.49\% \\ 
(\rownumber)~ECWA~\cite{deng2021emotion} & 70.87\%& 69.08\%  & 67.81\% & 66.87\%  \\ 
(\rownumber)~EECon~\cite{yang2023exploiting} & 71.13\%& 68.34\%  & 64.27\%& 63.16\%  \\ 
(\rownumber)~MAM~\cite{zhang2024affective} & 71.44\%  & 70.83\% & 67.18\%  & 65.01\%\\ 
(\rownumber)~TGCA-PVT~\cite{chen2024tgca}   & 73.05\%  & 71.46\% & 69.87\%  & 68.32\% \\ 
(\rownumber)~OEAN~\cite{zhang2024object}   & 73.40\%  & 72.63\% & 70.52\%  & 69.47\% \\ \hline
(\rownumber)~\shortname  & \textbf{79.48\%} & \textbf{79.22\%} & \textbf{74.12\%} & \textbf{73.09\%} \\ \hline
\end{tabular}
}
\vspace{-6mm}
% \end{center}
\end{table}
% 

\subsection{Experiment Setting}
% \subsubsection{Model Setting.}
% 
\textbf{Model Setting:}
For feature representation, we set $k=10$ to select object tags, and adopt clip-vit-base-patch32 as the pre-trained model for unified feature representation.
Moreover, we empirically set $(d_e, d_h, d_k, d_s) = (512, 128, 16, 64)$, and set the classification class $L$ to 8.

% 

\textbf{Training Setting:}
To initialize the model, we set all weights such as $\boldsymbol{W}$ following the truncated normal distribution, and use AdamW optimizer with the learning rate of $1 \times 10^{-4}$.
% warmup scheduler of cosine, warmup steps of 2000.
Furthermore, we set the batch size to 32 and the epoch of the training process to 200.
During the implementation, we utilize \textit{PyTorch} to build our entire model.
% , and our project codes are publicly available at https://github.com/zzmyrep/MESN.
% Our project codes as well as data are all publicly available on GitHub\footnote{https://github.com/zzmyrep/KBCEN}.
% Code is available at \href{https://github.com/zzmyrep/KBCEN}{https://github.com/zzmyrep/KBCEN}.

\textbf{Evaluation Metrics:}
Following~\cite{zhang2024affective, chen2024tgca, zhang2024object}, we adopt \textit{accuracy} and \textit{F1} as our evaluation metrics to measure the performance of different methods for image sentiment analysis. 



\subsection{Experiment Result}
% We compare our model against the following baselines: AlexNet~\cite{krizhevsky2017imagenet}, VGG16~\cite{simonyan2014very}, ResNet101~\cite{he2016deep}, CECCN~\cite{ruan2024color}, EmoVIT~\cite{xie2024emovit}, WSCNet~\cite{yang2018weakly}, ECWA~\cite{deng2021emotion}, EECon~\cite{yang2023exploiting}, MAM~\cite{zhang2024affective} and TGCA-PVT~\cite{chen2024tgca}, and the overall results are summarized in Table~\ref{tab:cap1}.
We compare our model against several baselines, and the overall results are summarized in Table~\ref{tab:cap1}.
We observe that our model achieves the best performance in both accuracy and F1 metrics, significantly outperforming the previous models. 
This superior performance is mainly attributed to our effective utilization of metadata to enhance image sentiment analysis, as well as the exceptional capability of the unified sentiment transformer framework we developed. These results strongly demonstrate that our proposed method can bring encouraging performance for image sentiment analysis.

\setcounter{magicrownumbers}{0} 
\begin{table}[t]
\begin{center}
\caption{Ablation study of~\shortname~on FI dataset.} 
% \vspace{1mm}
\label{tab:cap2}
\resizebox{.9\linewidth}{!}
{
\begin{tabular}{lcc}
  \hline
  \textbf{Model} & \textbf{Accuracy} & \textbf{F1} \\
  \hline
  (\rownumber)~Ours (w/o vision) & 65.72\% & 64.54\% \\
  (\rownumber)~Ours (w/o text description) & 74.05\% & 72.58\% \\
  (\rownumber)~Ours (w/o object tag) & 77.45\% & 76.84\% \\
  (\rownumber)~Ours (w/o scene tag) & 78.47\% & 78.21\% \\
  \hline
  (\rownumber)~Ours (w/o unified embedding) & 76.41\% & 76.23\% \\
  (\rownumber)~Ours (w/o adaptive learning) & 76.83\% & 76.56\% \\
  (\rownumber)~Ours (w/o cross-modal fusion) & 76.85\% & 76.49\% \\
  \hline
  (\rownumber)~Ours  & \textbf{79.48\%} & \textbf{79.22\%} \\
  \hline
\end{tabular}
}
\end{center}
\vspace{-5mm}
\end{table}


\begin{figure}[t]
\centering
% \vspace{-2mm}
\includegraphics[width=0.42\textwidth]{fig/2dvisual-linux4-paper2.pdf}
\caption{Visualization of feature distribution on eight categories before (left) and after (right) model processing.}
% 
\label{fig:visualization}
\vspace{-5mm}
\end{figure}

\subsection{Ablation Performance}
In this subsection, we conduct an ablation study to examine which component is really important for performance improvement. The results are reported in Table~\ref{tab:cap2}.

For information utilization, we observe a significant decline in model performance when visual features are removed. Additionally, the performance of \shortname~decreases when different metadata are removed separately, which means that text description, object tag, and scene tag are all critical for image sentiment analysis.
Recalling the model architecture, we separately remove transformer layers of the unified representation module, the adaptive learning module, and the cross-modal fusion module, replacing them with MLPs of the same parameter scale.
In this way, we can observe varying degrees of decline in model performance, indicating that these modules are indispensable for our model to achieve better performance.

\subsection{Visualization}
% 


% % 开始使用minipage进行左右排列
% \begin{minipage}[t]{0.45\textwidth}  % 子图1宽度为45%
%     \centering
%     \includegraphics[width=\textwidth]{2dvisual.pdf}  % 插入图片
%     \captionof{figure}{Visualization of feature distribution.}  % 使用captionof添加图片标题
%     \label{fig:visualization}
% \end{minipage}


% \begin{figure}[t]
% \centering
% \vspace{-2mm}
% \includegraphics[width=0.45\textwidth]{fig/2dvisual.pdf}
% \caption{Visualization of feature distribution.}
% \label{fig:visualization}
% % \vspace{-4mm}
% \end{figure}

% \begin{figure}[t]
% \centering
% \vspace{-2mm}
% \includegraphics[width=0.45\textwidth]{fig/2dvisual-linux3-paper.pdf}
% \caption{Visualization of feature distribution.}
% \label{fig:visualization}
% % \vspace{-4mm}
% \end{figure}



\begin{figure}[tbp]   
\vspace{-4mm}
  \centering            
  \subfloat[Depth of adaptive learning layers]   
  {
    \label{fig:subfig1}\includegraphics[width=0.22\textwidth]{fig/fig_sensitivity-a5}
  }
  \subfloat[Depth of fusion layers]
  {
    % \label{fig:subfig2}\includegraphics[width=0.22\textwidth]{fig/fig_sensitivity-b2}
    \label{fig:subfig2}\includegraphics[width=0.22\textwidth]{fig/fig_sensitivity-b2-num.pdf}
  }
  \caption{Sensitivity study of \shortname~on different depth. }   
  \label{fig:fig_sensitivity}  
\vspace{-2mm}
\end{figure}

% \begin{figure}[htbp]
% \centerline{\includegraphics{2dvisual.pdf}}
% \caption{Visualization of feature distribution.}
% \label{fig:visualization}
% \end{figure}

% In Fig.~\ref{fig:visualization}, we use t-SNE~\cite{van2008visualizing} to reduce the dimension of data features for visualization, Figure in left represents the metadata features before model processing, the features are obtained by embedding through the CLIP model, and figure in right shows the features of the data after model processing, it can be observed that after the model processing, the data with different label categories fall in different regions in the space, therefore, we can conclude that the Therefore, we can conclude that the model can effectively utilize the information contained in the metadata and use it to guide the model for classification.

In Fig.~\ref{fig:visualization}, we use t-SNE~\cite{van2008visualizing} to reduce the dimension of data features for visualization.
The left figure shows metadata features before being processed by our model (\textit{i.e.}, embedded by CLIP), while the right shows the distribution of features after being processed by our model.
We can observe that after the model processing, data with the same label are closer to each other, while others are farther away.
Therefore, it shows that the model can effectively utilize the information contained in the metadata and use it to guide the classification process.

\subsection{Sensitivity Analysis}
% 
In this subsection, we conduct a sensitivity analysis to figure out the effect of different depth settings of adaptive learning layers and fusion layers. 
% In this subsection, we conduct a sensitivity analysis to figure out the effect of different depth settings on the model. 
% Fig.~\ref{fig:fig_sensitivity} presents the effect of different depth settings of adaptive learning layers and fusion layers. 
Taking Fig.~\ref{fig:fig_sensitivity} (a) as an example, the model performance improves with increasing depth, reaching the best performance at a depth of 4.
% Taking Fig.~\ref{fig:fig_sensitivity} (a) as an example, the performance of \shortname~improves with the increase of depth at first, reaching the best performance at a depth of 4.
When the depth continues to increase, the accuracy decreases to varying degrees.
Similar results can be observed in Fig.~\ref{fig:fig_sensitivity} (b).
Therefore, we set their depths to 4 and 6 respectively to achieve the best results.

% Through our experiments, we can observe that the effect of modifying these hyperparameters on the results of the experiments is very weak, and the surface model is not sensitive to the hyperparameters.


\subsection{Zero-shot Capability}
% 

% (1)~GCH~\cite{2010Analyzing} & 21.78\% & (5)~RA-DLNet~\cite{2020A} & 34.01\% \\ \hline
% (2)~WSCNet~\cite{2019WSCNet}  & 30.25\% & (6)~CECCN~\cite{ruan2024color} & 43.83\% \\ \hline
% (3)~PCNN~\cite{2015Robust} & 31.68\%  & (7)~EmoVIT~\cite{xie2024emovit} & 44.90\% \\ \hline
% (4)~AR~\cite{2018Visual} & 32.67\% & (8)~Ours (Zero-shot) & 47.83\% \\ \hline


\begin{table}[t]
\centering
\caption{Zero-shot capability of \shortname.}
\label{tab:cap3}
\resizebox{1\linewidth}{!}
{
\begin{tabular}{lc|lc}
\hline
\textbf{Model} & \textbf{Accuracy} & \textbf{Model} & \textbf{Accuracy} \\ \hline
(1)~WSCNet~\cite{2019WSCNet}  & 30.25\% & (5)~MAM~\cite{zhang2024affective} & 39.56\%  \\ \hline
(2)~AR~\cite{2018Visual} & 32.67\% & (6)~CECCN~\cite{ruan2024color} & 43.83\% \\ \hline
(3)~RA-DLNet~\cite{2020A} & 34.01\%  & (7)~EmoVIT~\cite{xie2024emovit} & 44.90\% \\ \hline
(4)~CDA~\cite{han2023boosting} & 38.64\% & (8)~Ours (Zero-shot) & 47.83\% \\ \hline
\end{tabular}
}
\vspace{-5mm}
\end{table}

% We use the model trained on the FI dataset to test on the artphoto dataset to verify the model's generalization ability as well as robustness to other distributed datasets.
% We can observe that the MESN model shows strong competitiveness in terms of accuracy when compared to other trained models, which suggests that the model has a good generalization ability in the OOD task.

To validate the model's generalization ability and robustness to other distributed datasets, we directly test the model trained on the FI dataset, without training on Artphoto. 
% As observed in Table 3, compared to other models trained on Artphoto, we achieve highly competitive zero-shot performance, indicating that the model has good generalization ability in out-of-distribution tasks.
From Table~\ref{tab:cap3}, we can observe that compared with other models trained on Artphoto, we achieve competitive zero-shot performance, which shows that the model has good generalization ability in out-of-distribution tasks.


%%%%%%%%%%%%
%  E2E     %
%%%%%%%%%%%%


\section{Conclusion}
In this paper, we introduced Wi-Chat, the first LLM-powered Wi-Fi-based human activity recognition system that integrates the reasoning capabilities of large language models with the sensing potential of wireless signals. Our experimental results on a self-collected Wi-Fi CSI dataset demonstrate the promising potential of LLMs in enabling zero-shot Wi-Fi sensing. These findings suggest a new paradigm for human activity recognition that does not rely on extensive labeled data. We hope future research will build upon this direction, further exploring the applications of LLMs in signal processing domains such as IoT, mobile sensing, and radar-based systems.

\section*{Limitations}
While our work represents the first attempt to leverage LLMs for processing Wi-Fi signals, it is a preliminary study focused on a relatively simple task: Wi-Fi-based human activity recognition. This choice allows us to explore the feasibility of LLMs in wireless sensing but also comes with certain limitations.

Our approach primarily evaluates zero-shot performance, which, while promising, may still lag behind traditional supervised learning methods in highly complex or fine-grained recognition tasks. Besides, our study is limited to a controlled environment with a self-collected dataset, and the generalizability of LLMs to diverse real-world scenarios with varying Wi-Fi conditions, environmental interference, and device heterogeneity remains an open question.

Additionally, we have yet to explore the full potential of LLMs in more advanced Wi-Fi sensing applications, such as fine-grained gesture recognition, occupancy detection, and passive health monitoring. Future work should investigate the scalability of LLM-based approaches, their robustness to domain shifts, and their integration with multimodal sensing techniques in broader IoT applications.


% Bibliography entries for the entire Anthology, followed by custom entries
%\bibliography{anthology,custom}
% Custom bibliography entries only
\bibliography{main}
\newpage
\appendix

\section{Experiment prompts}
\label{sec:prompt}
The prompts used in the LLM experiments are shown in the following Table~\ref{tab:prompts}.

\definecolor{titlecolor}{rgb}{0.9, 0.5, 0.1}
\definecolor{anscolor}{rgb}{0.2, 0.5, 0.8}
\definecolor{labelcolor}{HTML}{48a07e}
\begin{table*}[h]
	\centering
	
 % \vspace{-0.2cm}
	
	\begin{center}
		\begin{tikzpicture}[
				chatbox_inner/.style={rectangle, rounded corners, opacity=0, text opacity=1, font=\sffamily\scriptsize, text width=5in, text height=9pt, inner xsep=6pt, inner ysep=6pt},
				chatbox_prompt_inner/.style={chatbox_inner, align=flush left, xshift=0pt, text height=11pt},
				chatbox_user_inner/.style={chatbox_inner, align=flush left, xshift=0pt},
				chatbox_gpt_inner/.style={chatbox_inner, align=flush left, xshift=0pt},
				chatbox/.style={chatbox_inner, draw=black!25, fill=gray!7, opacity=1, text opacity=0},
				chatbox_prompt/.style={chatbox, align=flush left, fill=gray!1.5, draw=black!30, text height=10pt},
				chatbox_user/.style={chatbox, align=flush left},
				chatbox_gpt/.style={chatbox, align=flush left},
				chatbox2/.style={chatbox_gpt, fill=green!25},
				chatbox3/.style={chatbox_gpt, fill=red!20, draw=black!20},
				chatbox4/.style={chatbox_gpt, fill=yellow!30},
				labelbox/.style={rectangle, rounded corners, draw=black!50, font=\sffamily\scriptsize\bfseries, fill=gray!5, inner sep=3pt},
			]
											
			\node[chatbox_user] (q1) {
				\textbf{System prompt}
				\newline
				\newline
				You are a helpful and precise assistant for segmenting and labeling sentences. We would like to request your help on curating a dataset for entity-level hallucination detection.
				\newline \newline
                We will give you a machine generated biography and a list of checked facts about the biography. Each fact consists of a sentence and a label (True/False). Please do the following process. First, breaking down the biography into words. Second, by referring to the provided list of facts, merging some broken down words in the previous step to form meaningful entities. For example, ``strategic thinking'' should be one entity instead of two. Third, according to the labels in the list of facts, labeling each entity as True or False. Specifically, for facts that share a similar sentence structure (\eg, \textit{``He was born on Mach 9, 1941.''} (\texttt{True}) and \textit{``He was born in Ramos Mejia.''} (\texttt{False})), please first assign labels to entities that differ across atomic facts. For example, first labeling ``Mach 9, 1941'' (\texttt{True}) and ``Ramos Mejia'' (\texttt{False}) in the above case. For those entities that are the same across atomic facts (\eg, ``was born'') or are neutral (\eg, ``he,'' ``in,'' and ``on''), please label them as \texttt{True}. For the cases that there is no atomic fact that shares the same sentence structure, please identify the most informative entities in the sentence and label them with the same label as the atomic fact while treating the rest of the entities as \texttt{True}. In the end, output the entities and labels in the following format:
                \begin{itemize}[nosep]
                    \item Entity 1 (Label 1)
                    \item Entity 2 (Label 2)
                    \item ...
                    \item Entity N (Label N)
                \end{itemize}
                % \newline \newline
                Here are two examples:
                \newline\newline
                \textbf{[Example 1]}
                \newline
                [The start of the biography]
                \newline
                \textcolor{titlecolor}{Marianne McAndrew is an American actress and singer, born on November 21, 1942, in Cleveland, Ohio. She began her acting career in the late 1960s, appearing in various television shows and films.}
                \newline
                [The end of the biography]
                \newline \newline
                [The start of the list of checked facts]
                \newline
                \textcolor{anscolor}{[Marianne McAndrew is an American. (False); Marianne McAndrew is an actress. (True); Marianne McAndrew is a singer. (False); Marianne McAndrew was born on November 21, 1942. (False); Marianne McAndrew was born in Cleveland, Ohio. (False); She began her acting career in the late 1960s. (True); She has appeared in various television shows. (True); She has appeared in various films. (True)]}
                \newline
                [The end of the list of checked facts]
                \newline \newline
                [The start of the ideal output]
                \newline
                \textcolor{labelcolor}{[Marianne McAndrew (True); is (True); an (True); American (False); actress (True); and (True); singer (False); , (True); born (True); on (True); November 21, 1942 (False); , (True); in (True); Cleveland, Ohio (False); . (True); She (True); began (True); her (True); acting career (True); in (True); the late 1960s (True); , (True); appearing (True); in (True); various (True); television shows (True); and (True); films (True); . (True)]}
                \newline
                [The end of the ideal output]
				\newline \newline
                \textbf{[Example 2]}
                \newline
                [The start of the biography]
                \newline
                \textcolor{titlecolor}{Doug Sheehan is an American actor who was born on April 27, 1949, in Santa Monica, California. He is best known for his roles in soap operas, including his portrayal of Joe Kelly on ``General Hospital'' and Ben Gibson on ``Knots Landing.''}
                \newline
                [The end of the biography]
                \newline \newline
                [The start of the list of checked facts]
                \newline
                \textcolor{anscolor}{[Doug Sheehan is an American. (True); Doug Sheehan is an actor. (True); Doug Sheehan was born on April 27, 1949. (True); Doug Sheehan was born in Santa Monica, California. (False); He is best known for his roles in soap operas. (True); He portrayed Joe Kelly. (True); Joe Kelly was in General Hospital. (True); General Hospital is a soap opera. (True); He portrayed Ben Gibson. (True); Ben Gibson was in Knots Landing. (True); Knots Landing is a soap opera. (True)]}
                \newline
                [The end of the list of checked facts]
                \newline \newline
                [The start of the ideal output]
                \newline
                \textcolor{labelcolor}{[Doug Sheehan (True); is (True); an (True); American (True); actor (True); who (True); was born (True); on (True); April 27, 1949 (True); in (True); Santa Monica, California (False); . (True); He (True); is (True); best known (True); for (True); his roles in soap operas (True); , (True); including (True); in (True); his portrayal (True); of (True); Joe Kelly (True); on (True); ``General Hospital'' (True); and (True); Ben Gibson (True); on (True); ``Knots Landing.'' (True)]}
                \newline
                [The end of the ideal output]
				\newline \newline
				\textbf{User prompt}
				\newline
				\newline
				[The start of the biography]
				\newline
				\textcolor{magenta}{\texttt{\{BIOGRAPHY\}}}
				\newline
				[The ebd of the biography]
				\newline \newline
				[The start of the list of checked facts]
				\newline
				\textcolor{magenta}{\texttt{\{LIST OF CHECKED FACTS\}}}
				\newline
				[The end of the list of checked facts]
			};
			\node[chatbox_user_inner] (q1_text) at (q1) {
				\textbf{System prompt}
				\newline
				\newline
				You are a helpful and precise assistant for segmenting and labeling sentences. We would like to request your help on curating a dataset for entity-level hallucination detection.
				\newline \newline
                We will give you a machine generated biography and a list of checked facts about the biography. Each fact consists of a sentence and a label (True/False). Please do the following process. First, breaking down the biography into words. Second, by referring to the provided list of facts, merging some broken down words in the previous step to form meaningful entities. For example, ``strategic thinking'' should be one entity instead of two. Third, according to the labels in the list of facts, labeling each entity as True or False. Specifically, for facts that share a similar sentence structure (\eg, \textit{``He was born on Mach 9, 1941.''} (\texttt{True}) and \textit{``He was born in Ramos Mejia.''} (\texttt{False})), please first assign labels to entities that differ across atomic facts. For example, first labeling ``Mach 9, 1941'' (\texttt{True}) and ``Ramos Mejia'' (\texttt{False}) in the above case. For those entities that are the same across atomic facts (\eg, ``was born'') or are neutral (\eg, ``he,'' ``in,'' and ``on''), please label them as \texttt{True}. For the cases that there is no atomic fact that shares the same sentence structure, please identify the most informative entities in the sentence and label them with the same label as the atomic fact while treating the rest of the entities as \texttt{True}. In the end, output the entities and labels in the following format:
                \begin{itemize}[nosep]
                    \item Entity 1 (Label 1)
                    \item Entity 2 (Label 2)
                    \item ...
                    \item Entity N (Label N)
                \end{itemize}
                % \newline \newline
                Here are two examples:
                \newline\newline
                \textbf{[Example 1]}
                \newline
                [The start of the biography]
                \newline
                \textcolor{titlecolor}{Marianne McAndrew is an American actress and singer, born on November 21, 1942, in Cleveland, Ohio. She began her acting career in the late 1960s, appearing in various television shows and films.}
                \newline
                [The end of the biography]
                \newline \newline
                [The start of the list of checked facts]
                \newline
                \textcolor{anscolor}{[Marianne McAndrew is an American. (False); Marianne McAndrew is an actress. (True); Marianne McAndrew is a singer. (False); Marianne McAndrew was born on November 21, 1942. (False); Marianne McAndrew was born in Cleveland, Ohio. (False); She began her acting career in the late 1960s. (True); She has appeared in various television shows. (True); She has appeared in various films. (True)]}
                \newline
                [The end of the list of checked facts]
                \newline \newline
                [The start of the ideal output]
                \newline
                \textcolor{labelcolor}{[Marianne McAndrew (True); is (True); an (True); American (False); actress (True); and (True); singer (False); , (True); born (True); on (True); November 21, 1942 (False); , (True); in (True); Cleveland, Ohio (False); . (True); She (True); began (True); her (True); acting career (True); in (True); the late 1960s (True); , (True); appearing (True); in (True); various (True); television shows (True); and (True); films (True); . (True)]}
                \newline
                [The end of the ideal output]
				\newline \newline
                \textbf{[Example 2]}
                \newline
                [The start of the biography]
                \newline
                \textcolor{titlecolor}{Doug Sheehan is an American actor who was born on April 27, 1949, in Santa Monica, California. He is best known for his roles in soap operas, including his portrayal of Joe Kelly on ``General Hospital'' and Ben Gibson on ``Knots Landing.''}
                \newline
                [The end of the biography]
                \newline \newline
                [The start of the list of checked facts]
                \newline
                \textcolor{anscolor}{[Doug Sheehan is an American. (True); Doug Sheehan is an actor. (True); Doug Sheehan was born on April 27, 1949. (True); Doug Sheehan was born in Santa Monica, California. (False); He is best known for his roles in soap operas. (True); He portrayed Joe Kelly. (True); Joe Kelly was in General Hospital. (True); General Hospital is a soap opera. (True); He portrayed Ben Gibson. (True); Ben Gibson was in Knots Landing. (True); Knots Landing is a soap opera. (True)]}
                \newline
                [The end of the list of checked facts]
                \newline \newline
                [The start of the ideal output]
                \newline
                \textcolor{labelcolor}{[Doug Sheehan (True); is (True); an (True); American (True); actor (True); who (True); was born (True); on (True); April 27, 1949 (True); in (True); Santa Monica, California (False); . (True); He (True); is (True); best known (True); for (True); his roles in soap operas (True); , (True); including (True); in (True); his portrayal (True); of (True); Joe Kelly (True); on (True); ``General Hospital'' (True); and (True); Ben Gibson (True); on (True); ``Knots Landing.'' (True)]}
                \newline
                [The end of the ideal output]
				\newline \newline
				\textbf{User prompt}
				\newline
				\newline
				[The start of the biography]
				\newline
				\textcolor{magenta}{\texttt{\{BIOGRAPHY\}}}
				\newline
				[The ebd of the biography]
				\newline \newline
				[The start of the list of checked facts]
				\newline
				\textcolor{magenta}{\texttt{\{LIST OF CHECKED FACTS\}}}
				\newline
				[The end of the list of checked facts]
			};
		\end{tikzpicture}
        \caption{GPT-4o prompt for labeling hallucinated entities.}\label{tb:gpt-4-prompt}
	\end{center}
\vspace{-0cm}
\end{table*}
% \section{Full Experiment Results}
% \begin{table*}[th]
    \centering
    \small
    \caption{Classification Results}
    \begin{tabular}{lcccc}
        \toprule
        \textbf{Method} & \textbf{Accuracy} & \textbf{Precision} & \textbf{Recall} & \textbf{F1-score} \\
        \midrule
        \multicolumn{5}{c}{\textbf{Zero Shot}} \\
                Zero-shot E-eyes & 0.26 & 0.26 & 0.27 & 0.26 \\
        Zero-shot CARM & 0.24 & 0.24 & 0.24 & 0.24 \\
                Zero-shot SVM & 0.27 & 0.28 & 0.28 & 0.27 \\
        Zero-shot CNN & 0.23 & 0.24 & 0.23 & 0.23 \\
        Zero-shot RNN & 0.26 & 0.26 & 0.26 & 0.26 \\
DeepSeek-0shot & 0.54 & 0.61 & 0.54 & 0.52 \\
DeepSeek-0shot-COT & 0.33 & 0.24 & 0.33 & 0.23 \\
DeepSeek-0shot-Knowledge & 0.45 & 0.46 & 0.45 & 0.44 \\
Gemma2-0shot & 0.35 & 0.22 & 0.38 & 0.27 \\
Gemma2-0shot-COT & 0.36 & 0.22 & 0.36 & 0.27 \\
Gemma2-0shot-Knowledge & 0.32 & 0.18 & 0.34 & 0.20 \\
GPT-4o-mini-0shot & 0.48 & 0.53 & 0.48 & 0.41 \\
GPT-4o-mini-0shot-COT & 0.33 & 0.50 & 0.33 & 0.38 \\
GPT-4o-mini-0shot-Knowledge & 0.49 & 0.31 & 0.49 & 0.36 \\
GPT-4o-0shot & 0.62 & 0.62 & 0.47 & 0.42 \\
GPT-4o-0shot-COT & 0.29 & 0.45 & 0.29 & 0.21 \\
GPT-4o-0shot-Knowledge & 0.44 & 0.52 & 0.44 & 0.39 \\
LLaMA-0shot & 0.32 & 0.25 & 0.32 & 0.24 \\
LLaMA-0shot-COT & 0.12 & 0.25 & 0.12 & 0.09 \\
LLaMA-0shot-Knowledge & 0.32 & 0.25 & 0.32 & 0.28 \\
Mistral-0shot & 0.19 & 0.23 & 0.19 & 0.10 \\
Mistral-0shot-Knowledge & 0.21 & 0.40 & 0.21 & 0.11 \\
        \midrule
        \multicolumn{5}{c}{\textbf{4 Shot}} \\
GPT-4o-mini-4shot & 0.58 & 0.59 & 0.58 & 0.53 \\
GPT-4o-mini-4shot-COT & 0.57 & 0.53 & 0.57 & 0.50 \\
GPT-4o-mini-4shot-Knowledge & 0.56 & 0.51 & 0.56 & 0.47 \\
GPT-4o-4shot & 0.77 & 0.84 & 0.77 & 0.73 \\
GPT-4o-4shot-COT & 0.63 & 0.76 & 0.63 & 0.53 \\
GPT-4o-4shot-Knowledge & 0.72 & 0.82 & 0.71 & 0.66 \\
LLaMA-4shot & 0.29 & 0.24 & 0.29 & 0.21 \\
LLaMA-4shot-COT & 0.20 & 0.30 & 0.20 & 0.13 \\
LLaMA-4shot-Knowledge & 0.15 & 0.23 & 0.13 & 0.13 \\
Mistral-4shot & 0.02 & 0.02 & 0.02 & 0.02 \\
Mistral-4shot-Knowledge & 0.21 & 0.27 & 0.21 & 0.20 \\
        \midrule
        
        \multicolumn{5}{c}{\textbf{Suprevised}} \\
        SVM & 0.94 & 0.92 & 0.91 & 0.91 \\
        CNN & 0.98 & 0.98 & 0.97 & 0.97 \\
        RNN & 0.99 & 0.99 & 0.99 & 0.99 \\
        % \midrule
        % \multicolumn{5}{c}{\textbf{Conventional Wi-Fi-based Human Activity Recognition Systems}} \\
        E-eyes & 1.00 & 1.00 & 1.00 & 1.00 \\
        CARM & 0.98 & 0.98 & 0.98 & 0.98 \\
\midrule
 \multicolumn{5}{c}{\textbf{Vision Models}} \\
           Zero-shot SVM & 0.26 & 0.25 & 0.25 & 0.25 \\
        Zero-shot CNN & 0.26 & 0.25 & 0.26 & 0.26 \\
        Zero-shot RNN & 0.28 & 0.28 & 0.29 & 0.28 \\
        SVM & 0.99 & 0.99 & 0.99 & 0.99 \\
        CNN & 0.98 & 0.99 & 0.98 & 0.98 \\
        RNN & 0.98 & 0.99 & 0.98 & 0.98 \\
GPT-4o-mini-Vision & 0.84 & 0.85 & 0.84 & 0.84 \\
GPT-4o-mini-Vision-COT & 0.90 & 0.91 & 0.90 & 0.90 \\
GPT-4o-Vision & 0.74 & 0.82 & 0.74 & 0.73 \\
GPT-4o-Vision-COT & 0.70 & 0.83 & 0.70 & 0.68 \\
LLaMA-Vision & 0.20 & 0.23 & 0.20 & 0.09 \\
LLaMA-Vision-Knowledge & 0.22 & 0.05 & 0.22 & 0.08 \\

        \bottomrule
    \end{tabular}
    \label{full}
\end{table*}




\end{document}




\subsubsection{Main effects of explanation and interactions with LLM accuracy}
\label{sec:mainexplanation}

We find a significant main effect of explanation on most DVs (all except \var{SourceClick} and \var{Time}).
Specifically, provided that the LLM answer is correct and there are no sources, providing an explanation leads to higher participant agreement with the LLM answer ($M = 78.2\%$ vs. $67.2\%$, $\beta = .60, SE = .19, p = .002$), accuracy ($M = 78.2\%$ vs. $67.2\%$, $\beta = .65, SE = .19, p < .001$), confidence in the final answer ($M = 5.26$ vs. $4.55$, $\beta = .74, SE = .10, p < .001$), rating of the LLM response's justification quality ($M = 5.52$ vs. $2.58$, $\beta = 2.94, SE = .13, p < .001$), and rating of its actionability ($M = 5.14$ vs. $2.56$, $\beta = 2.59, SE = .13, p < .001$).
On the other hand, the likelihood of asking a follow-up question is lower when an explanation is provided ($M = 28.2\%$ vs. $71.4\%$, $\beta = -2.38, SE = .21, p < .001$).


For participants' accuracy, however, we find a significant interaction between the presence of an explanation and the accuracy of the LLM answer ($\beta = -1.00, SE = .28, p < .001$).
In the absence of sources, when the LLM answer is correct, participants' accuracy is higher when an explanation is provided ($M = 78.2\%$ vs. $67.2\%$).
In contrast, when the LLM answer is incorrect, accuracy is lower when an explanation is provided ($M = 17.2\%$ vs. $21.8\%$). 
That is, in both cases, participants submitted the same answer as the LLM's more often when an explanation was provided. 



We find support for these findings in the qualitative data as well.
In their free-form answers in the exit questionnaire, 28 participants wrote that they submitted a different answer from the LLM's answer when there was no explanation.
As put by one participant, \shortquote{One sentence answers felt incomplete and did not explain how Theta arrived at its conclusion.}
Another wrote the absence of explanation \shortquote{made the [LLM's] answer too hard to trust.}


In summary, we find that \textbf{explanations tend to increase reliance, both appropriate reliance on correct answers and overreliance on incorrect answers}.
Explanations also tend to increase participants' confidence in their answer and evaluation of the LLM response, and decrease their likelihood of asking a follow-up question.
Intuitively, this suggests participants viewed LLM responses with explanations as more satisfying and reliable, regardless of their accuracy.
These findings are consistent with prior research \cite{Bansal2021CHI,zhang2020effect,Poursabzi-Sangdeh-CHI2021,wang2021explanations,Fok2024Verifiability,Pafla2024CHI,si2024fact} and suggest explanations from state-of-the-art LLMs can also lead to overreliance and have unintended negative consequences.



\begin{figure*}[t!]
    \centering
    \begin{subfigure}[t]{0.55\textwidth}
        \centering
        \includegraphics[height=1.6in]{figures/main_numbers.png}
        \caption{Effect of explanation, sources, and LLM accuracy}
        \label{fig:accuracy_main}
    \end{subfigure}%
    \hfill
    \begin{subfigure}[t]{0.37\textwidth}
        \centering
        \includegraphics[height=1.6in]{figures/inconsistencies_numbers.png}
        \caption{Effect of inconsistencies}
        \label{fig:accuracy_inconsistent}
    \end{subfigure}
    \caption{\textbf{Summary of participants' accuracy in Study 2.} We plot the raw data means and 95\% confidence intervals for participants' accuracy when provided with different types of LLM responses. When the LLM's answer is correct, participants' accuracy is highest when the LLM response includes an explanation and sources (\cref{fig:accuracy_main} left). When the LLM's answer is incorrect, participants' accuracy is highest when the LLM response includes sources but not an explanation (\cref{fig:accuracy_main} right). When the LLM response includes an explanation for an incorrect answer, participants' accuracy is higher when the explanation is inconsistent (\cref{fig:accuracy_inconsistent}).}
\end{figure*}


\subsubsection{Main effects of sources and interactions with LLM accuracy}
\label{sec:mainsource}

We find a significant main effect of sources on the time spent on the task, as well as on all self-reported variables.
That is, when the LLM answer is correct and there is no explanation, providing sources leads to higher participant time on task ($M = 1.24$ min vs. $1.05$ min, $\beta = .17, SE = .07, p = .027$), confidence in the final answer ($M = 5.50$ vs. $4.55$, $\beta = .96, SE = .10 p < .001$), rating of the LLM response's justification quality ($M = 4.45$ vs. $2.58$, $\beta = 1.88, SE = .13, p < .001$), and rating of its actionability ($M = 4.90$ vs. $2.56$, $\beta = 2.34, SE = .13, p < .001$).
In contrast, the likelihood of asking a follow-up question is lower when sources are provided ($M = 34.4\%$ vs. $71.4\%$, $\beta = -2.04, SE = .20, p < .001$).


However, we find a significant interaction between the presence of sources and LLM accuracy on many DVs.
Beginning with agreement ($\beta = -.83, SE = .27, p = .002$),
provided that there is no explanation, when the LLM answer is correct, agreement is higher when sources are provided ($M = 73.4\%$ vs. $67.2\%$).
But when the LLM answer is incorrect, agreement is lower when sources are provided ($M = 68.2\%$ vs. $78.2\%$).
These results suggest that \textbf{sources tend to increase appropriate reliance on correct answers and reduce overreliance on incorrect answers.}


Significant interactions are also found for all self-reported variables: 
\var{Confidence} ($\beta = -.45, SE = .14, p = .002$), \var{JustificationQuality} ($\beta = -.79, SE =.19, p < .001$), \var{Actionability} ($\beta = -.65, SE = .19, p < .001$), and \var{Followup} ($\beta = 1.05, SE = .27, p < .001$).
Provided that there is no explanation and the LLM answer is correct, providing sources increases \var{Confidence}, \var{JustificationQuality}, and \var{Actionability} while decreasing \var{Followup}.
When the LLM answer is incorrect, these effects of sources are all attenuated.
The fact that sources have different effect sizes for correct and incorrect LLM answers provides further (if indirect) support for the idea that sources can help foster appropriate reliance. \looseness=-1


The final significant interaction between sources and LLM accuracy is found for time on task ($\beta = .33, SE = .11, p = .002$).
Provided that there is no explanation, when the LLM answer is correct, time on task is higher when there are sources ($M = 1.24$ min vs. $1.05$ min). But when the LLM answer is incorrect, this effect of sources on time on task is magnified ($M = 1.39$ min vs. $.89$ min).
A possible reason for this result is that when the LLM answer is incorrect, in some instances participants may have found conflicting information between the LLM response and the sources and spent more time resolving the conflict and completing the task.
For example, 45 participants wrote in their free-form responses that they submitted a different answer from the LLM's answer when it conflicted with the information in the sources, e.g., \shortquote{I trusted the information in the links more than I trusted Theta's answer. Therefore, if the information in the links differed, I submitted a final answer that was different from Theta's.} \looseness=-1


Participants also wrote that the mere presence of sources tended to increase the credibility of the LLM response (e.g., \shortquote{If Theta supplied sources for its answers, I felt the answers were more credible}), while the absence of sources had the opposite effect (e.g., \shortquote{Not having any links provided with [Theta's] answer was a red flag to me to think something is wrong or can't be found}).
29 participants explicitly stated that they submitted a different answer from the LLM's answer when there were no sources in the LLM response.
Additionally, several participants wrote about how they were forced to rely on their intuition when there were no sources, e.g., \shortquote{Without being able to verify info, my gut was my best answer.}
They expressed frustration about this and said they would prefer to have sources since it is \shortquote{easiest to agree or disagree when the AI cited its sources.}




\subsubsection{Interactions between explanation and sources}
\label{sec:explsource}

In addition to the main effects of explanation and sources and their respective interactions with LLM accuracy, we find a significant interaction between explanation and sources for all self-reported variables: \var{Confidence} ($\beta = -.42, SE = .14, p = .004$), \var{JustificationQuality} ($\beta = -1.41, SE = .19, p < .001$), \var{Actionability} ($\beta = -1.36, SE = .19, p < .001$), and \var{Followup} ($\beta = .81, SE = .31, p < .001$).
Provided that the LLM answer is correct, when there are no sources, providing explanations increases \var{Confidence} ($M = 5.26$ vs. $4.55$), \var{JustificationQuality} ($M = 5.52$ vs. $2.58$), and \var{Actionability} ($M = 5.14$ vs. $2.56$), while decreasing \var{Followup} ($M = 28.2\%$ vs. $71.4\%$).
When there are sources, however, providing explanations still increases \var{Confidence} ($M = 5.83$ vs. $5.50$), \var{JustificationQuality} ($M = 5.99$ vs. $4.45$), and \var{Actionability} ($M = 6.13$ vs. $4.90$), while decreasing \var{Followup} ($M = 12.7\%$ vs. $34.4\%$), but all to a lesser extent than when there are no sources.
In short, including both explanation and sources achieves the biggest effects in these measures, though their joint effects are subadditive, i.e., less than the sum of the individual effects.


\begin{figure*}[t!]
\centering
\includegraphics[width=\textwidth]{figures/inconsistencies_all_bracket.png}
\caption{\textbf{Study 2 results on inconsistencies.} We plot the raw data means and 95\% confidence intervals. Brackets indicate statistically significant differences between three types of incorrect LLM responses: No explanation, Consistent explanation, and Inconsistent explanation. Significance is marked as $^\ast$ ($p < .05$), $^{\ast\ast}$ ($p < .01$), and $^{\ast\ast\ast}$ ($p < .001$). See \cref{sec:inconsistencies} for details.}
\label{fig:inconsistencies}
\end{figure*}



\subsubsection{Additional effects of LLM accuracy}
\label{sec:mainaccuracy}

Finally, we find a significant main effect of LLM accuracy on many DVs, in addition to its interactions with explanation and sources reported above.
Provided that there are no sources or explanation, when the LLM answer is incorrect compared to correct, agreement is higher ($M = 78.2\%$ vs. $67.2\%$, $\beta = .60, SE = .19, p = .002$), confidence is higher ($M = 4.92$ vs. $4.55$, $\beta = .37, SE = .10, p < .001$), and \var{Actionability} is higher ($M = 2.91$ vs. $2.56$, $\beta = .35, SE = .13, p = .007$), while accuracy is lower ($M = 21.8\%$ vs. $67.2\%$, $\beta = -2.07, SE = .19, p < .001$) and time on task is lower ($M = .89$ min vs. $1.05$ min, $\beta = -.17, SE = .08, p = .025$).
These results suggest that participants found incorrect answers more plausible than correct answers for the task questions used in the experiment.
This is not surprising as we deliberately selected challenging questions, i.e., questions with less than 50\% human accuracy in our pilot study.
As such, this is likely a feature of our stimulus materials rather than a generalizable finding.




\subsection{Study 2 Results: Additional Analyses}
\label{sec:study2additional}


Finally, we report results from our additional pre-registered analyses on the effects of inconsistencies in explanations (\cref{sec:inconsistencies}) and the relationship between participants' source clicking behavior and other DVs (\cref{sec:sourceclick}).


\subsubsection{Inconsistencies in explanations}
\label{sec:inconsistencies}


In Study 1, we found inconsistencies in explanations to be an important unreliability cue that participants often noticed.
While the presence of inconsistencies is not something we control for or manipulate, we explore whether and how the natural inconsistencies that arose in LLM responses have effects on the DVs with a pre-registered analysis. 
Specifically, we used analysis of variance (ANOVA) to compare the means of DVs across three types of incorrect LLM responses: No explanation ($N = 616$), consistent explanation ($N = 461$), and inconsistent explanation ($N = 155$), where
$N$ indicates the number of instances for which participants received a given response type.
If there was a significant difference across response types, we conducted pairwise comparisons with post-hoc Tukey tests.
We only analyze responses with an incorrect answer, as none of the responses with a correct answer contained inconsistencies (as described in \cref{sec:study2materials}).
We present the results in \cref{fig:inconsistencies}.



For most DVs (all except \var{SourceClick} and \var{Time}) we find a significant difference across response types.
Overall, overreliance on incorrect answers is most prevalent when participants receive consistent explanations, as evidenced by the highest agreement with the LLM answer, confidence in their final answer, and ratings of justification quality and actionability, as well as the lowest accuracy and likelihood of asking follow-up questions.
In comparison, when participants receive inconsistent explanations, agreement is significantly lower ($M = 69.7\%$ vs. $83.3\%$ $p = .002$), rating of the LLM response's justification quality is significantly lower ($M = 5.13$ vs. $5.59$, $p = .028$), while accuracy is significantly higher ($M = 30.3\%$ vs. $16.7\%$, $p = .002$).
While our study materials did not allow us to investigate the effect of inconsistencies when an LLM answer is correct, which may happen less naturally based on our observations, these results suggest that \textbf{inconsistencies can help reduce overreliance on incorrect answers induced by explanations.}


Consistent with the quantitative results, 19 participants stated in their free-form answers that they disagreed with Theta %the LLM (``Theta'') 
when \shortquote{Theta's responses were contradictory.}
For example, several participants wrote about how Theta provided a logically inconsistent response to the question ``Do more than two thirds of South America's population live in Brazil?'' (See \cref{fig:task} for the response.)
As one participant elaborated, \shortquote{The Brazilian and South American population answer contradicted itself. Two-thirds would imply 66\% but given the number of Brazilians compared to the total population of South America given in the answer, the actual percentage is closer to 50\%.} \looseness=-1




\subsubsection{Source clicking behavior}
\label{sec:sourceclick}

From our main analysis (\cref{sec:study2results}), we did not find any significant effect on when participants chose to click on the provided source links.
We only found a marginally significant main effect of explanation such that participants' source clicking likelihood is lower when there is an explanation than not ($M = 25.0\%$ vs. $28.2\%$, $\beta = -.62, SE = .36, p = .086$).
However, there is high variance across individuals.
According to our tracking, 189 out of 308 participants never clicked on sources, 33 participants clicked on sources in one task, 18 participants in two tasks, 23 participants in three tasks, and 45 participants in all four tasks for which sources were provided.


To better understand participants' source clicking behavior, we conducted a pre-registered analysis to examine its relationship with other DVs.
Specifically, we used ANOVA to compare the means of DVs between instances in which participants were provided sources but did not click on any ($N = 914$) and instances in which participants were provided sources and clicked on one or more ($N = 318$).
Among the latter, 164 are instances in which the LLM answer is correct and 154 are instances in which the LLM answer is incorrect. \looseness=-1


We find that when participants click on sources, accuracy is higher ($M = 60.1\%$ vs. $49.2\%$, $p < .001$) and time on task is higher ($M = 2.11$ min vs. $1.08$ min, $p < .001$), while rating of the LLM response's justification quality is lower ($4.58$ vs. $5.08$, $p < .001$).
The accuracy and time on task results are intuitive.
The sources in our study stimuli tended to provide accurate and relevant information (see \cref{sec:study2materials} for details), so when participants clicked on sources, they likely found correct answers at the expense of spending more time on task.
Indeed, we see that source clicking was helpful when the LLM gave an incorrect answer.
The increase in accuracy is bigger when the LLM answer is incorrect ($M = 37.0\%$ vs. $24.2\%$) than when the LLM answer is correct ($M = 81.7\%$ vs. $74.8\%$).
For reference, when LLM responses do not include sources, participants' answer accuracy is $M = 19.5\%$ when the LLM answer is incorrect and $M = 72.7\%$ when the LLM answer is correct.


There are multiple possible factors that might influence the finding that the rating of justification quality is lower when participants have clicked on the provided sources. Participants may have clicked on sources because they found the LLM response's justification quality to be low, or their rating may have decreased after examining the sources.
Again breaking down the data into instances in which the LLM answer is correct and those where it is incorrect, participants' rating of the response's justification quality when they clicked on sources vs. not is $M = 5.04$ vs. $5.29$ when the LLM answer is correct and $M = 4.08$ vs. $4.87$ when the LLM answer is incorrect.


Together, these results suggest engaging with the content of (accurate and relevant) sources can be an effective way of improving decision outcomes.
However, the presence of explanation may reduce users' natural tendency to examine sources, especially when they find the explanation to be of high quality.
It could be helpful to nudge users to pay more attention to sources by highlighting sources or placing sources above explanations.




\section{Discussion}
\label{sec:discussion}


\subsection{Implications of Findings}


\subsubsection*{Explanations}
In our studies, we found that explanations play an important role in shaping users' reliance.
In Study 1, we gained qualitative insights on how participants interpreted and used explanations to judge the reliability of LLM answers.
In Study 2, we examined the effects of the presence of an explanation, as well as its interaction with other variables, and found that explanations increase reliance on both correct and incorrect responses.
This is consistent with prior findings in HCI that explanations can increase overreliance \cite{Bansal2021CHI,zhang2020effect,Poursabzi-Sangdeh-CHI2021,wang2021explanations,Fok2024Verifiability}, including explanations generated by LLMs~\cite{si2024fact,Pafla2024CHI}.
It is also consistent with prior work in psychology, which finds that explanations are often found compelling even when they contain little content \cite{langer1978mindlessness,Giffin2017} or content that experts judge irrelevant \cite{hopkins2016seductive}, and that effects of superficial cues on explanation quality are more severe when time and prior knowledge are limited \cite{hopkins2019,kelemen2013professional}. 
In the absence of effort and expertise, users will inevitably rely on superficial cues to explanation quality, such as fluency \cite{trout2008}, a characteristic that LLM explanations typically possess in spades.
This suggests a potential tension in providing LLM explanations to lay users: the properties that make such explanations intelligible and compelling may be precisely those that lead to overreliance.
As such, we encourage LLM explanations to be evaluated and optimized for appropriate reliance, in addition to other qualities such as fluency, justification quality, and satisfaction.



\subsubsection*{Sources}
Our results offer some basis for optimism, however: sources helped reduce overreliance on incorrect answers and increase appropriate reliance on correct answers.
One possibility is that sources encouraged participants to engage in slow and careful System 2 thinking, instead of quick and automatic System 1 thinking \cite{Kahneman2003, kahneman2011thinking}.
In our study, participants spent significantly more time on task when provided with sources, especially when the LLM's answer was incorrect.
The qualitative data also supports this. Many participants wrote that they checked sources. Many also wrote that they submitted a different answer from the LLM's answer when it conflicted with the information in the sources.
We emphasize, however, that the sources provided in Study 2 were all real and tended to provide accurate and relevant information.
This is not always the case. Recent work has found that popular LLM-infused applications frequently generate statements that are not supported by sources \cite{liu2023evaluating} and sometimes even generate fake sources \cite{Alkaissi2023}.
If the provided sources are junk or just broken links, then presumably they will not help foster appropriate reliance. They could potentially even hurt by making the LLM response look more trustworthy, similar to how flawed and meaningless explanations have been found to increase people's trust and reliance \cite{Eiband2019Placebic,Schemmer2022AIES,kaur2020CHI}.
In addition to improving the quality of sources in LLM responses, future work should explore different issues with sources (e.g., fake, unreliable, conflicting sources and inaccurate summaries of sources), design choices (e.g., location of sources and amount of preview), and their effects on people's perceptions and behaviors.


When it comes to choosing between providing sources only and providing sources and an explanation, there are benefits and drawbacks to each.
When the LLM answer is incorrect, participants' accuracy is highest on responses with sources only ($M = 31.8\%$), followed by responses with explanation and sources ($M = 23.1\%$), neither ($M = 21.8\%$), and explanation only ($M = 17.2\%$) --- suggesting that providing sources only is most effective at reducing overreliance on incorrect answers.
However, it is not as effective at improving appropriate reliance when the LLM answer is correct. Here, participants' accuracy is highest on responses with explanation and sources ($M = 79.9\%$), followed by responses with explanation only ($M = 78.2\%$), sources only ($M = 73.4\%$), and neither ($M = 67.2\%$).
In contexts where LLMs have much higher accuracy than users, providing sources only can lead to lower overall accuracy than providing sources and an explanation.
Further, participants rated responses with sources only lower in terms of justification quality and actionability, compared to responses with sources and explanation, suggesting that people prefer the latter.


\subsubsection*{Inconsistencies and other unreliability cues}
Finally, we found that LLM responses contain new forms of unreliability cues.
Prior research, in particular the work by \citet{chen2023understanding}, found that people identify AI models' biases, inability to consider contexts or multiple features, and lower performance on rare instances as cues of unreliability.
In our studies, we identified other cues such as inconsistencies in explanations and lack of explanation or sources --- all of which are related to the particular characteristics of LLMs.
For example, some inconsistencies occur due to the stochastic nature of LLMs: LLMs can generate different responses for the same input, unlike deterministic AI models.
Even within a single response, inconsistencies occur because LLMs are not trained to generate only logically consistent statements.
The other cues are connected to LLMs' natural language modality and ability to handle a wide variety of tasks, which lead to responses with much more diverse features and forms compared to classical AI models with fixed output spaces. \looseness=-1


Intriguingly, we found positive effects of such unreliability cues when it comes to reducing overreliance.
In Study 1, participants who noticed unreliability cues engaged with the LLM responses more thoroughly.
In Study 2, participants relied less on incorrect LLM responses when they were provided with explanations containing inconsistencies than those without.
These findings, along with prior findings on other unreliability cues (e.g., inconsistencies between multiple responses \cite{si2024fact,Lee2024FAccT}), suggest that guiding people's attention to these cues can be an effective approach to reducing overreliance.
For example, we could apply computational methods to automatically detect inconsistencies (e.g., ~\cite{tacl2022,contradiction2008}) then use highlighting to draw people's attention to the detected inconsistencies. Other interventions (e.g., expressing uncertainty, displaying disclaimers, and encouraging source checking) could be applied jointly for cases in which inconsistency detection is difficult or where LLM responses are consistently inaccurate.
We suggest future research to explore more thoroughly what unreliability cues exist in LLM responses and how to design interventions that help people notice and reason about these cues.





\subsection{Explanation of the Answer vs. Explanation of How the LLM Arrived at the Answer}


Throughout the work, we have used the term \textit{explanation} to refer to supporting details in LLM responses that justify the LLM's answer to the input question.
This is different from how the term is often used within the explainable AI community in that we do not make any assumptions about the extent to which it faithfully describes the way that the model arrived at its answer. 
We emphasize that faithfulness is extremely difficult for users --- or even model developers --- to evaluate, especially without access to the model's internals.
Evaluating the faithfulness of model explanations is an active area of research \cite{weijie2024,Zhao2024,atanasova2023faithfulness,jacovi2020faithfulness,agarwal2024faithfulness}.


Nevertheless, many participants in Study 1 interpreted ChatGPT's responses as including somewhat faithful explanations of how the system arrived at its answer, especially when the responses had certain characteristics \interview{3,6,7,8,10,11,14,15,16}.
For some, the critical characteristic was the presence of sources \interview{6,16}. As P16 (high-knowledge, low-use) described, \shortquote{I would think of the citation itself as an explanation because it kind of implies `I'm giving you this information because it came from this source' and then me as a human can evaluate that source.}
For others, it was the step-by-step form of responses, which are common for math questions \interview{3,7,8,14}. After seeing them, P14 (low-knowledge, high-use) said, \shortquote{I think it's very clear how did it [ChatGPT] provides me the answer.} \looseness=-1


In contrast, three participants, all with high knowledge of LLMs, were strongly opposed to the idea that ChatGPT could provide explanations of how it arrived at its answers \interview{5,12,13}.
P5 (high-knowledge, low-use) stated that \shortquote{it's provably false that ChatGPT's responses provide a description of how it arrives at its answers,} emphasizing that ChatGPT's responses are \shortquote{definitely and empirically not explanations because there's no reflection in the model.}
Similarly, P12 (high-knowledge, high-use) said they don't think of ChatGPT as explaining anything to them, and that ChatGPT was just \shortquote{trained to provide answers that look like an explanation because that's what we would find most useful.}
P4 (high-knowledge, low-use) shared this view and emphasized that \shortquote{there's no way to interpret how the answer came from.}
They noted that the explanations ChatGPT offers describe \shortquote{how a normal person would reach the answer,} and are not explanations of how ChatGPT arrives at its answers. \looseness=-1


In sum, while there was considerable variability between individuals, we found that many participants, especially those without much knowledge of LLMs, viewed ChatGPT's responses as including somewhat faithful explanations for how the system arrived at its answer.
This raises a concern because first, again, there is no reliable way for users or anyone to evaluate their faithfulness without access to the system's internals, and second, recent work has found explanations from LLMs are often not faithful to their process \cite{turpin2023language,Zhao2024,wiegreffe2022reframing,Marasovi2021FewShotSW,lyu2023faithful} and can easily be manipulated, e.g., to rationalize incorrect information \cite{pan2023on,buchanan2021truth,Zellers2019Fakenews}.
Such assumptions can be strengthened by the increasing anthropomorphization of LLMs and lead to inappropriate reliance \cite{Weidinger2022Risk,Shanahan2024Anthro,Cohn2024CHI}.
We suggest future research to explore strategies for improving people's understanding of LLMs \cite{long2020literacy,Annapureddy2024literacy} and study how they are connected to reliance behaviors. \looseness=-1





\subsection{Limitations}
\label{sec:limitations}


There are several limitations of our work that are worth reflecting on.
First and foremost, our studies were conducted in the context of objective question-answering and may not generalize to other contexts of LLM use (e.g., writing, idea generation, and task automation).
We encourage the community to conduct more empirical studies on how user reliance is shaped in various contexts.

Each of our studies has a different set of strengths and limitations.
Study 1 was a think-aloud study that offered descriptive examples of how users interpret and act upon different LLM response features in a relatively natural setting.
However, prior work has pointed out that the set-up of a think-aloud study can cause people to behave differently than they would otherwise \cite{Hertzum2009ScrutinizingUE,Boren2000,fox2011procedures}.
For example, we saw a much higher rate of source clicking in Study 1 (M = 63.8\%) than in Study 2 (M = 25.8\%) which was an online experiment. 
We also emphasize that the LLM response features identified in Study 1 are not comprehensive. We suggest future work to explore what other features influence users' reliance and can help them succeed in tasks despite inaccuracies from LLMs. 


In Study 2, we employed a different research method (a controlled experiment), prioritizing the generalizability of findings by controlling as many other variables as possible.
For example, in the experiment, participants saw exactly one response from Theta, created in advance using the state-of-the-art LLM-infused applications ChatGPT and Perplexity AI, instead of interacting with a real system in multiple rounds.
While participants referred to Theta as ``AI'' or ``LLM'' in the exit questionnaire (e.g., \shortquote{I just trusted the AI when I didn't know the answer already}), we did not measure participants' general perceptions of Theta or inquire about their experience. 
Hence, it is more accurate to view Study 2 as a study of people's perceptions and behaviors around specific LLM responses rather than a study of people's interactions with LLMs.
While showing one controlled LLM response is a commonly used method (e.g.,~\cite{Kim2024FAccT,si2024fact,Lee2024FAccT}), people's perceptions and behaviors may change over time, meriting further studies in more interactive settings. \looseness=-1


Additionally, we set Theta's accuracy to be 50\% which is significantly worse than the state-of-the-art. While this choice allowed us to compare the effects of LLM response features on relying on correct vs. incorrect answers in a balanced fashion, future work should explore whether there are interaction effects between these features and the LLM's accuracy.
There are also implications of our experimental task, which was answering difficult factual questions (that less than 50\% of pilot study participants knew the answer to). 
We chose this task to simulate realistic scenarios of people seeking answers to questions they don't know the answer to. However, it is possible that our findings may not generalize to tasks where people have sufficient prior knowledge and can more deeply engage with the content of the LLM responses.
Finally, there are many LLM response features that we did not study or control for (e.g., simplicity of explanations \cite{lombrozo2007simplicity}, quality of sources \cite{Rieh2007Credibility,Wathen2002Credibility}, and presence of jargon \cite{cruz2024effect}).
We encourage future work to explore different features and methods to understand user interactions with LLMs, an emerging research area whose importance will only grow with time. \looseness=-1




\section{Conclusion}

We conducted two empirical studies to understand how different features of LLM responses shape users' reliance.
We found that the presence of explanations increases reliance on both correct and incorrect responses. 
However, we observed less reliance on incorrect responses when sources are provided or when explanations exhibit inconsistencies.
Our findings highlight the importance of evaluating LLM response features with users before deployment.
Our findings also suggest that providing (accurate and relevant) sources and designing interventions that help users notice and reason about inconsistencies and other unreliability cues in explanations can be promising directions for fostering appropriate reliance on LLMs.



%%
%% The acknowledgments section is defined using the "acks" environment
%% (and NOT an unnumbered section). This ensures the proper
%% identification of the section in the article metadata, and the
%% consistent spelling of the heading.
\begin{acks}
We foremost thank the participants for sharing their time and experiences. We also thank the members of the Princeton Visual AI Lab, the Princeton HCI Lab, and the Princeton Concepts \& Cognition Lab, as well as the anonymous reviewers for thoughtful feedback and discussion. We acknowledge support from the NSF Graduate Research Fellowship Program (SK) and the Princeton SEAS Howard B. Wentz, Jr. Junior Faculty Award (OR).
\end{acks}

%%
%% The next two lines define the bibliography style to be used, and
%% the bibliography file.
\bibliographystyle{ACM-Reference-Format}
\bibliography{references}


%%
%% If your work has an appendix, this is the place to put it.
\appendix
\subsection{Lloyd-Max Algorithm}
\label{subsec:Lloyd-Max}
For a given quantization bitwidth $B$ and an operand $\bm{X}$, the Lloyd-Max algorithm finds $2^B$ quantization levels $\{\hat{x}_i\}_{i=1}^{2^B}$ such that quantizing $\bm{X}$ by rounding each scalar in $\bm{X}$ to the nearest quantization level minimizes the quantization MSE. 

The algorithm starts with an initial guess of quantization levels and then iteratively computes quantization thresholds $\{\tau_i\}_{i=1}^{2^B-1}$ and updates quantization levels $\{\hat{x}_i\}_{i=1}^{2^B}$. Specifically, at iteration $n$, thresholds are set to the midpoints of the previous iteration's levels:
\begin{align*}
    \tau_i^{(n)}=\frac{\hat{x}_i^{(n-1)}+\hat{x}_{i+1}^{(n-1)}}2 \text{ for } i=1\ldots 2^B-1
\end{align*}
Subsequently, the quantization levels are re-computed as conditional means of the data regions defined by the new thresholds:
\begin{align*}
    \hat{x}_i^{(n)}=\mathbb{E}\left[ \bm{X} \big| \bm{X}\in [\tau_{i-1}^{(n)},\tau_i^{(n)}] \right] \text{ for } i=1\ldots 2^B
\end{align*}
where to satisfy boundary conditions we have $\tau_0=-\infty$ and $\tau_{2^B}=\infty$. The algorithm iterates the above steps until convergence.

Figure \ref{fig:lm_quant} compares the quantization levels of a $7$-bit floating point (E3M3) quantizer (left) to a $7$-bit Lloyd-Max quantizer (right) when quantizing a layer of weights from the GPT3-126M model at a per-tensor granularity. As shown, the Lloyd-Max quantizer achieves substantially lower quantization MSE. Further, Table \ref{tab:FP7_vs_LM7} shows the superior perplexity achieved by Lloyd-Max quantizers for bitwidths of $7$, $6$ and $5$. The difference between the quantizers is clear at 5 bits, where per-tensor FP quantization incurs a drastic and unacceptable increase in perplexity, while Lloyd-Max quantization incurs a much smaller increase. Nevertheless, we note that even the optimal Lloyd-Max quantizer incurs a notable ($\sim 1.5$) increase in perplexity due to the coarse granularity of quantization. 

\begin{figure}[h]
  \centering
  \includegraphics[width=0.7\linewidth]{sections/figures/LM7_FP7.pdf}
  \caption{\small Quantization levels and the corresponding quantization MSE of Floating Point (left) vs Lloyd-Max (right) Quantizers for a layer of weights in the GPT3-126M model.}
  \label{fig:lm_quant}
\end{figure}

\begin{table}[h]\scriptsize
\begin{center}
\caption{\label{tab:FP7_vs_LM7} \small Comparing perplexity (lower is better) achieved by floating point quantizers and Lloyd-Max quantizers on a GPT3-126M model for the Wikitext-103 dataset.}
\begin{tabular}{c|cc|c}
\hline
 \multirow{2}{*}{\textbf{Bitwidth}} & \multicolumn{2}{|c|}{\textbf{Floating-Point Quantizer}} & \textbf{Lloyd-Max Quantizer} \\
 & Best Format & Wikitext-103 Perplexity & Wikitext-103 Perplexity \\
\hline
7 & E3M3 & 18.32 & 18.27 \\
6 & E3M2 & 19.07 & 18.51 \\
5 & E4M0 & 43.89 & 19.71 \\
\hline
\end{tabular}
\end{center}
\end{table}

\subsection{Proof of Local Optimality of LO-BCQ}
\label{subsec:lobcq_opt_proof}
For a given block $\bm{b}_j$, the quantization MSE during LO-BCQ can be empirically evaluated as $\frac{1}{L_b}\lVert \bm{b}_j- \bm{\hat{b}}_j\rVert^2_2$ where $\bm{\hat{b}}_j$ is computed from equation (\ref{eq:clustered_quantization_definition}) as $C_{f(\bm{b}_j)}(\bm{b}_j)$. Further, for a given block cluster $\mathcal{B}_i$, we compute the quantization MSE as $\frac{1}{|\mathcal{B}_{i}|}\sum_{\bm{b} \in \mathcal{B}_{i}} \frac{1}{L_b}\lVert \bm{b}- C_i^{(n)}(\bm{b})\rVert^2_2$. Therefore, at the end of iteration $n$, we evaluate the overall quantization MSE $J^{(n)}$ for a given operand $\bm{X}$ composed of $N_c$ block clusters as:
\begin{align*}
    \label{eq:mse_iter_n}
    J^{(n)} = \frac{1}{N_c} \sum_{i=1}^{N_c} \frac{1}{|\mathcal{B}_{i}^{(n)}|}\sum_{\bm{v} \in \mathcal{B}_{i}^{(n)}} \frac{1}{L_b}\lVert \bm{b}- B_i^{(n)}(\bm{b})\rVert^2_2
\end{align*}

At the end of iteration $n$, the codebooks are updated from $\mathcal{C}^{(n-1)}$ to $\mathcal{C}^{(n)}$. However, the mapping of a given vector $\bm{b}_j$ to quantizers $\mathcal{C}^{(n)}$ remains as  $f^{(n)}(\bm{b}_j)$. At the next iteration, during the vector clustering step, $f^{(n+1)}(\bm{b}_j)$ finds new mapping of $\bm{b}_j$ to updated codebooks $\mathcal{C}^{(n)}$ such that the quantization MSE over the candidate codebooks is minimized. Therefore, we obtain the following result for $\bm{b}_j$:
\begin{align*}
\frac{1}{L_b}\lVert \bm{b}_j - C_{f^{(n+1)}(\bm{b}_j)}^{(n)}(\bm{b}_j)\rVert^2_2 \le \frac{1}{L_b}\lVert \bm{b}_j - C_{f^{(n)}(\bm{b}_j)}^{(n)}(\bm{b}_j)\rVert^2_2
\end{align*}

That is, quantizing $\bm{b}_j$ at the end of the block clustering step of iteration $n+1$ results in lower quantization MSE compared to quantizing at the end of iteration $n$. Since this is true for all $\bm{b} \in \bm{X}$, we assert the following:
\begin{equation}
\begin{split}
\label{eq:mse_ineq_1}
    \tilde{J}^{(n+1)} &= \frac{1}{N_c} \sum_{i=1}^{N_c} \frac{1}{|\mathcal{B}_{i}^{(n+1)}|}\sum_{\bm{b} \in \mathcal{B}_{i}^{(n+1)}} \frac{1}{L_b}\lVert \bm{b} - C_i^{(n)}(b)\rVert^2_2 \le J^{(n)}
\end{split}
\end{equation}
where $\tilde{J}^{(n+1)}$ is the the quantization MSE after the vector clustering step at iteration $n+1$.

Next, during the codebook update step (\ref{eq:quantizers_update}) at iteration $n+1$, the per-cluster codebooks $\mathcal{C}^{(n)}$ are updated to $\mathcal{C}^{(n+1)}$ by invoking the Lloyd-Max algorithm \citep{Lloyd}. We know that for any given value distribution, the Lloyd-Max algorithm minimizes the quantization MSE. Therefore, for a given vector cluster $\mathcal{B}_i$ we obtain the following result:

\begin{equation}
    \frac{1}{|\mathcal{B}_{i}^{(n+1)}|}\sum_{\bm{b} \in \mathcal{B}_{i}^{(n+1)}} \frac{1}{L_b}\lVert \bm{b}- C_i^{(n+1)}(\bm{b})\rVert^2_2 \le \frac{1}{|\mathcal{B}_{i}^{(n+1)}|}\sum_{\bm{b} \in \mathcal{B}_{i}^{(n+1)}} \frac{1}{L_b}\lVert \bm{b}- C_i^{(n)}(\bm{b})\rVert^2_2
\end{equation}

The above equation states that quantizing the given block cluster $\mathcal{B}_i$ after updating the associated codebook from $C_i^{(n)}$ to $C_i^{(n+1)}$ results in lower quantization MSE. Since this is true for all the block clusters, we derive the following result: 
\begin{equation}
\begin{split}
\label{eq:mse_ineq_2}
     J^{(n+1)} &= \frac{1}{N_c} \sum_{i=1}^{N_c} \frac{1}{|\mathcal{B}_{i}^{(n+1)}|}\sum_{\bm{b} \in \mathcal{B}_{i}^{(n+1)}} \frac{1}{L_b}\lVert \bm{b}- C_i^{(n+1)}(\bm{b})\rVert^2_2  \le \tilde{J}^{(n+1)}   
\end{split}
\end{equation}

Following (\ref{eq:mse_ineq_1}) and (\ref{eq:mse_ineq_2}), we find that the quantization MSE is non-increasing for each iteration, that is, $J^{(1)} \ge J^{(2)} \ge J^{(3)} \ge \ldots \ge J^{(M)}$ where $M$ is the maximum number of iterations. 
%Therefore, we can say that if the algorithm converges, then it must be that it has converged to a local minimum. 
\hfill $\blacksquare$


\begin{figure}
    \begin{center}
    \includegraphics[width=0.5\textwidth]{sections//figures/mse_vs_iter.pdf}
    \end{center}
    \caption{\small NMSE vs iterations during LO-BCQ compared to other block quantization proposals}
    \label{fig:nmse_vs_iter}
\end{figure}

Figure \ref{fig:nmse_vs_iter} shows the empirical convergence of LO-BCQ across several block lengths and number of codebooks. Also, the MSE achieved by LO-BCQ is compared to baselines such as MXFP and VSQ. As shown, LO-BCQ converges to a lower MSE than the baselines. Further, we achieve better convergence for larger number of codebooks ($N_c$) and for a smaller block length ($L_b$), both of which increase the bitwidth of BCQ (see Eq \ref{eq:bitwidth_bcq}).


\subsection{Additional Accuracy Results}
%Table \ref{tab:lobcq_config} lists the various LOBCQ configurations and their corresponding bitwidths.
\begin{table}
\setlength{\tabcolsep}{4.75pt}
\begin{center}
\caption{\label{tab:lobcq_config} Various LO-BCQ configurations and their bitwidths.}
\begin{tabular}{|c||c|c|c|c||c|c||c|} 
\hline
 & \multicolumn{4}{|c||}{$L_b=8$} & \multicolumn{2}{|c||}{$L_b=4$} & $L_b=2$ \\
 \hline
 \backslashbox{$L_A$\kern-1em}{\kern-1em$N_c$} & 2 & 4 & 8 & 16 & 2 & 4 & 2 \\
 \hline
 64 & 4.25 & 4.375 & 4.5 & 4.625 & 4.375 & 4.625 & 4.625\\
 \hline
 32 & 4.375 & 4.5 & 4.625& 4.75 & 4.5 & 4.75 & 4.75 \\
 \hline
 16 & 4.625 & 4.75& 4.875 & 5 & 4.75 & 5 & 5 \\
 \hline
\end{tabular}
\end{center}
\end{table}

%\subsection{Perplexity achieved by various LO-BCQ configurations on Wikitext-103 dataset}

\begin{table} \centering
\begin{tabular}{|c||c|c|c|c||c|c||c|} 
\hline
 $L_b \rightarrow$& \multicolumn{4}{c||}{8} & \multicolumn{2}{c||}{4} & 2\\
 \hline
 \backslashbox{$L_A$\kern-1em}{\kern-1em$N_c$} & 2 & 4 & 8 & 16 & 2 & 4 & 2  \\
 %$N_c \rightarrow$ & 2 & 4 & 8 & 16 & 2 & 4 & 2 \\
 \hline
 \hline
 \multicolumn{8}{c}{GPT3-1.3B (FP32 PPL = 9.98)} \\ 
 \hline
 \hline
 64 & 10.40 & 10.23 & 10.17 & 10.15 &  10.28 & 10.18 & 10.19 \\
 \hline
 32 & 10.25 & 10.20 & 10.15 & 10.12 &  10.23 & 10.17 & 10.17 \\
 \hline
 16 & 10.22 & 10.16 & 10.10 & 10.09 &  10.21 & 10.14 & 10.16 \\
 \hline
  \hline
 \multicolumn{8}{c}{GPT3-8B (FP32 PPL = 7.38)} \\ 
 \hline
 \hline
 64 & 7.61 & 7.52 & 7.48 &  7.47 &  7.55 &  7.49 & 7.50 \\
 \hline
 32 & 7.52 & 7.50 & 7.46 &  7.45 &  7.52 &  7.48 & 7.48  \\
 \hline
 16 & 7.51 & 7.48 & 7.44 &  7.44 &  7.51 &  7.49 & 7.47  \\
 \hline
\end{tabular}
\caption{\label{tab:ppl_gpt3_abalation} Wikitext-103 perplexity across GPT3-1.3B and 8B models.}
\end{table}

\begin{table} \centering
\begin{tabular}{|c||c|c|c|c||} 
\hline
 $L_b \rightarrow$& \multicolumn{4}{c||}{8}\\
 \hline
 \backslashbox{$L_A$\kern-1em}{\kern-1em$N_c$} & 2 & 4 & 8 & 16 \\
 %$N_c \rightarrow$ & 2 & 4 & 8 & 16 & 2 & 4 & 2 \\
 \hline
 \hline
 \multicolumn{5}{|c|}{Llama2-7B (FP32 PPL = 5.06)} \\ 
 \hline
 \hline
 64 & 5.31 & 5.26 & 5.19 & 5.18  \\
 \hline
 32 & 5.23 & 5.25 & 5.18 & 5.15  \\
 \hline
 16 & 5.23 & 5.19 & 5.16 & 5.14  \\
 \hline
 \multicolumn{5}{|c|}{Nemotron4-15B (FP32 PPL = 5.87)} \\ 
 \hline
 \hline
 64  & 6.3 & 6.20 & 6.13 & 6.08  \\
 \hline
 32  & 6.24 & 6.12 & 6.07 & 6.03  \\
 \hline
 16  & 6.12 & 6.14 & 6.04 & 6.02  \\
 \hline
 \multicolumn{5}{|c|}{Nemotron4-340B (FP32 PPL = 3.48)} \\ 
 \hline
 \hline
 64 & 3.67 & 3.62 & 3.60 & 3.59 \\
 \hline
 32 & 3.63 & 3.61 & 3.59 & 3.56 \\
 \hline
 16 & 3.61 & 3.58 & 3.57 & 3.55 \\
 \hline
\end{tabular}
\caption{\label{tab:ppl_llama7B_nemo15B} Wikitext-103 perplexity compared to FP32 baseline in Llama2-7B and Nemotron4-15B, 340B models}
\end{table}

%\subsection{Perplexity achieved by various LO-BCQ configurations on MMLU dataset}


\begin{table} \centering
\begin{tabular}{|c||c|c|c|c||c|c|c|c|} 
\hline
 $L_b \rightarrow$& \multicolumn{4}{c||}{8} & \multicolumn{4}{c||}{8}\\
 \hline
 \backslashbox{$L_A$\kern-1em}{\kern-1em$N_c$} & 2 & 4 & 8 & 16 & 2 & 4 & 8 & 16  \\
 %$N_c \rightarrow$ & 2 & 4 & 8 & 16 & 2 & 4 & 2 \\
 \hline
 \hline
 \multicolumn{5}{|c|}{Llama2-7B (FP32 Accuracy = 45.8\%)} & \multicolumn{4}{|c|}{Llama2-70B (FP32 Accuracy = 69.12\%)} \\ 
 \hline
 \hline
 64 & 43.9 & 43.4 & 43.9 & 44.9 & 68.07 & 68.27 & 68.17 & 68.75 \\
 \hline
 32 & 44.5 & 43.8 & 44.9 & 44.5 & 68.37 & 68.51 & 68.35 & 68.27  \\
 \hline
 16 & 43.9 & 42.7 & 44.9 & 45 & 68.12 & 68.77 & 68.31 & 68.59  \\
 \hline
 \hline
 \multicolumn{5}{|c|}{GPT3-22B (FP32 Accuracy = 38.75\%)} & \multicolumn{4}{|c|}{Nemotron4-15B (FP32 Accuracy = 64.3\%)} \\ 
 \hline
 \hline
 64 & 36.71 & 38.85 & 38.13 & 38.92 & 63.17 & 62.36 & 63.72 & 64.09 \\
 \hline
 32 & 37.95 & 38.69 & 39.45 & 38.34 & 64.05 & 62.30 & 63.8 & 64.33  \\
 \hline
 16 & 38.88 & 38.80 & 38.31 & 38.92 & 63.22 & 63.51 & 63.93 & 64.43  \\
 \hline
\end{tabular}
\caption{\label{tab:mmlu_abalation} Accuracy on MMLU dataset across GPT3-22B, Llama2-7B, 70B and Nemotron4-15B models.}
\end{table}


%\subsection{Perplexity achieved by various LO-BCQ configurations on LM evaluation harness}

\begin{table} \centering
\begin{tabular}{|c||c|c|c|c||c|c|c|c|} 
\hline
 $L_b \rightarrow$& \multicolumn{4}{c||}{8} & \multicolumn{4}{c||}{8}\\
 \hline
 \backslashbox{$L_A$\kern-1em}{\kern-1em$N_c$} & 2 & 4 & 8 & 16 & 2 & 4 & 8 & 16  \\
 %$N_c \rightarrow$ & 2 & 4 & 8 & 16 & 2 & 4 & 2 \\
 \hline
 \hline
 \multicolumn{5}{|c|}{Race (FP32 Accuracy = 37.51\%)} & \multicolumn{4}{|c|}{Boolq (FP32 Accuracy = 64.62\%)} \\ 
 \hline
 \hline
 64 & 36.94 & 37.13 & 36.27 & 37.13 & 63.73 & 62.26 & 63.49 & 63.36 \\
 \hline
 32 & 37.03 & 36.36 & 36.08 & 37.03 & 62.54 & 63.51 & 63.49 & 63.55  \\
 \hline
 16 & 37.03 & 37.03 & 36.46 & 37.03 & 61.1 & 63.79 & 63.58 & 63.33  \\
 \hline
 \hline
 \multicolumn{5}{|c|}{Winogrande (FP32 Accuracy = 58.01\%)} & \multicolumn{4}{|c|}{Piqa (FP32 Accuracy = 74.21\%)} \\ 
 \hline
 \hline
 64 & 58.17 & 57.22 & 57.85 & 58.33 & 73.01 & 73.07 & 73.07 & 72.80 \\
 \hline
 32 & 59.12 & 58.09 & 57.85 & 58.41 & 73.01 & 73.94 & 72.74 & 73.18  \\
 \hline
 16 & 57.93 & 58.88 & 57.93 & 58.56 & 73.94 & 72.80 & 73.01 & 73.94  \\
 \hline
\end{tabular}
\caption{\label{tab:mmlu_abalation} Accuracy on LM evaluation harness tasks on GPT3-1.3B model.}
\end{table}

\begin{table} \centering
\begin{tabular}{|c||c|c|c|c||c|c|c|c|} 
\hline
 $L_b \rightarrow$& \multicolumn{4}{c||}{8} & \multicolumn{4}{c||}{8}\\
 \hline
 \backslashbox{$L_A$\kern-1em}{\kern-1em$N_c$} & 2 & 4 & 8 & 16 & 2 & 4 & 8 & 16  \\
 %$N_c \rightarrow$ & 2 & 4 & 8 & 16 & 2 & 4 & 2 \\
 \hline
 \hline
 \multicolumn{5}{|c|}{Race (FP32 Accuracy = 41.34\%)} & \multicolumn{4}{|c|}{Boolq (FP32 Accuracy = 68.32\%)} \\ 
 \hline
 \hline
 64 & 40.48 & 40.10 & 39.43 & 39.90 & 69.20 & 68.41 & 69.45 & 68.56 \\
 \hline
 32 & 39.52 & 39.52 & 40.77 & 39.62 & 68.32 & 67.43 & 68.17 & 69.30  \\
 \hline
 16 & 39.81 & 39.71 & 39.90 & 40.38 & 68.10 & 66.33 & 69.51 & 69.42  \\
 \hline
 \hline
 \multicolumn{5}{|c|}{Winogrande (FP32 Accuracy = 67.88\%)} & \multicolumn{4}{|c|}{Piqa (FP32 Accuracy = 78.78\%)} \\ 
 \hline
 \hline
 64 & 66.85 & 66.61 & 67.72 & 67.88 & 77.31 & 77.42 & 77.75 & 77.64 \\
 \hline
 32 & 67.25 & 67.72 & 67.72 & 67.00 & 77.31 & 77.04 & 77.80 & 77.37  \\
 \hline
 16 & 68.11 & 68.90 & 67.88 & 67.48 & 77.37 & 78.13 & 78.13 & 77.69  \\
 \hline
\end{tabular}
\caption{\label{tab:mmlu_abalation} Accuracy on LM evaluation harness tasks on GPT3-8B model.}
\end{table}

\begin{table} \centering
\begin{tabular}{|c||c|c|c|c||c|c|c|c|} 
\hline
 $L_b \rightarrow$& \multicolumn{4}{c||}{8} & \multicolumn{4}{c||}{8}\\
 \hline
 \backslashbox{$L_A$\kern-1em}{\kern-1em$N_c$} & 2 & 4 & 8 & 16 & 2 & 4 & 8 & 16  \\
 %$N_c \rightarrow$ & 2 & 4 & 8 & 16 & 2 & 4 & 2 \\
 \hline
 \hline
 \multicolumn{5}{|c|}{Race (FP32 Accuracy = 40.67\%)} & \multicolumn{4}{|c|}{Boolq (FP32 Accuracy = 76.54\%)} \\ 
 \hline
 \hline
 64 & 40.48 & 40.10 & 39.43 & 39.90 & 75.41 & 75.11 & 77.09 & 75.66 \\
 \hline
 32 & 39.52 & 39.52 & 40.77 & 39.62 & 76.02 & 76.02 & 75.96 & 75.35  \\
 \hline
 16 & 39.81 & 39.71 & 39.90 & 40.38 & 75.05 & 73.82 & 75.72 & 76.09  \\
 \hline
 \hline
 \multicolumn{5}{|c|}{Winogrande (FP32 Accuracy = 70.64\%)} & \multicolumn{4}{|c|}{Piqa (FP32 Accuracy = 79.16\%)} \\ 
 \hline
 \hline
 64 & 69.14 & 70.17 & 70.17 & 70.56 & 78.24 & 79.00 & 78.62 & 78.73 \\
 \hline
 32 & 70.96 & 69.69 & 71.27 & 69.30 & 78.56 & 79.49 & 79.16 & 78.89  \\
 \hline
 16 & 71.03 & 69.53 & 69.69 & 70.40 & 78.13 & 79.16 & 79.00 & 79.00  \\
 \hline
\end{tabular}
\caption{\label{tab:mmlu_abalation} Accuracy on LM evaluation harness tasks on GPT3-22B model.}
\end{table}

\begin{table} \centering
\begin{tabular}{|c||c|c|c|c||c|c|c|c|} 
\hline
 $L_b \rightarrow$& \multicolumn{4}{c||}{8} & \multicolumn{4}{c||}{8}\\
 \hline
 \backslashbox{$L_A$\kern-1em}{\kern-1em$N_c$} & 2 & 4 & 8 & 16 & 2 & 4 & 8 & 16  \\
 %$N_c \rightarrow$ & 2 & 4 & 8 & 16 & 2 & 4 & 2 \\
 \hline
 \hline
 \multicolumn{5}{|c|}{Race (FP32 Accuracy = 44.4\%)} & \multicolumn{4}{|c|}{Boolq (FP32 Accuracy = 79.29\%)} \\ 
 \hline
 \hline
 64 & 42.49 & 42.51 & 42.58 & 43.45 & 77.58 & 77.37 & 77.43 & 78.1 \\
 \hline
 32 & 43.35 & 42.49 & 43.64 & 43.73 & 77.86 & 75.32 & 77.28 & 77.86  \\
 \hline
 16 & 44.21 & 44.21 & 43.64 & 42.97 & 78.65 & 77 & 76.94 & 77.98  \\
 \hline
 \hline
 \multicolumn{5}{|c|}{Winogrande (FP32 Accuracy = 69.38\%)} & \multicolumn{4}{|c|}{Piqa (FP32 Accuracy = 78.07\%)} \\ 
 \hline
 \hline
 64 & 68.9 & 68.43 & 69.77 & 68.19 & 77.09 & 76.82 & 77.09 & 77.86 \\
 \hline
 32 & 69.38 & 68.51 & 68.82 & 68.90 & 78.07 & 76.71 & 78.07 & 77.86  \\
 \hline
 16 & 69.53 & 67.09 & 69.38 & 68.90 & 77.37 & 77.8 & 77.91 & 77.69  \\
 \hline
\end{tabular}
\caption{\label{tab:mmlu_abalation} Accuracy on LM evaluation harness tasks on Llama2-7B model.}
\end{table}

\begin{table} \centering
\begin{tabular}{|c||c|c|c|c||c|c|c|c|} 
\hline
 $L_b \rightarrow$& \multicolumn{4}{c||}{8} & \multicolumn{4}{c||}{8}\\
 \hline
 \backslashbox{$L_A$\kern-1em}{\kern-1em$N_c$} & 2 & 4 & 8 & 16 & 2 & 4 & 8 & 16  \\
 %$N_c \rightarrow$ & 2 & 4 & 8 & 16 & 2 & 4 & 2 \\
 \hline
 \hline
 \multicolumn{5}{|c|}{Race (FP32 Accuracy = 48.8\%)} & \multicolumn{4}{|c|}{Boolq (FP32 Accuracy = 85.23\%)} \\ 
 \hline
 \hline
 64 & 49.00 & 49.00 & 49.28 & 48.71 & 82.82 & 84.28 & 84.03 & 84.25 \\
 \hline
 32 & 49.57 & 48.52 & 48.33 & 49.28 & 83.85 & 84.46 & 84.31 & 84.93  \\
 \hline
 16 & 49.85 & 49.09 & 49.28 & 48.99 & 85.11 & 84.46 & 84.61 & 83.94  \\
 \hline
 \hline
 \multicolumn{5}{|c|}{Winogrande (FP32 Accuracy = 79.95\%)} & \multicolumn{4}{|c|}{Piqa (FP32 Accuracy = 81.56\%)} \\ 
 \hline
 \hline
 64 & 78.77 & 78.45 & 78.37 & 79.16 & 81.45 & 80.69 & 81.45 & 81.5 \\
 \hline
 32 & 78.45 & 79.01 & 78.69 & 80.66 & 81.56 & 80.58 & 81.18 & 81.34  \\
 \hline
 16 & 79.95 & 79.56 & 79.79 & 79.72 & 81.28 & 81.66 & 81.28 & 80.96  \\
 \hline
\end{tabular}
\caption{\label{tab:mmlu_abalation} Accuracy on LM evaluation harness tasks on Llama2-70B model.}
\end{table}

%\section{MSE Studies}
%\textcolor{red}{TODO}


\subsection{Number Formats and Quantization Method}
\label{subsec:numFormats_quantMethod}
\subsubsection{Integer Format}
An $n$-bit signed integer (INT) is typically represented with a 2s-complement format \citep{yao2022zeroquant,xiao2023smoothquant,dai2021vsq}, where the most significant bit denotes the sign.

\subsubsection{Floating Point Format}
An $n$-bit signed floating point (FP) number $x$ comprises of a 1-bit sign ($x_{\mathrm{sign}}$), $B_m$-bit mantissa ($x_{\mathrm{mant}}$) and $B_e$-bit exponent ($x_{\mathrm{exp}}$) such that $B_m+B_e=n-1$. The associated constant exponent bias ($E_{\mathrm{bias}}$) is computed as $(2^{{B_e}-1}-1)$. We denote this format as $E_{B_e}M_{B_m}$.  

\subsubsection{Quantization Scheme}
\label{subsec:quant_method}
A quantization scheme dictates how a given unquantized tensor is converted to its quantized representation. We consider FP formats for the purpose of illustration. Given an unquantized tensor $\bm{X}$ and an FP format $E_{B_e}M_{B_m}$, we first, we compute the quantization scale factor $s_X$ that maps the maximum absolute value of $\bm{X}$ to the maximum quantization level of the $E_{B_e}M_{B_m}$ format as follows:
\begin{align}
\label{eq:sf}
    s_X = \frac{\mathrm{max}(|\bm{X}|)}{\mathrm{max}(E_{B_e}M_{B_m})}
\end{align}
In the above equation, $|\cdot|$ denotes the absolute value function.

Next, we scale $\bm{X}$ by $s_X$ and quantize it to $\hat{\bm{X}}$ by rounding it to the nearest quantization level of $E_{B_e}M_{B_m}$ as:

\begin{align}
\label{eq:tensor_quant}
    \hat{\bm{X}} = \text{round-to-nearest}\left(\frac{\bm{X}}{s_X}, E_{B_e}M_{B_m}\right)
\end{align}

We perform dynamic max-scaled quantization \citep{wu2020integer}, where the scale factor $s$ for activations is dynamically computed during runtime.

\subsection{Vector Scaled Quantization}
\begin{wrapfigure}{r}{0.35\linewidth}
  \centering
  \includegraphics[width=\linewidth]{sections/figures/vsquant.jpg}
  \caption{\small Vectorwise decomposition for per-vector scaled quantization (VSQ \citep{dai2021vsq}).}
  \label{fig:vsquant}
\end{wrapfigure}
During VSQ \citep{dai2021vsq}, the operand tensors are decomposed into 1D vectors in a hardware friendly manner as shown in Figure \ref{fig:vsquant}. Since the decomposed tensors are used as operands in matrix multiplications during inference, it is beneficial to perform this decomposition along the reduction dimension of the multiplication. The vectorwise quantization is performed similar to tensorwise quantization described in Equations \ref{eq:sf} and \ref{eq:tensor_quant}, where a scale factor $s_v$ is required for each vector $\bm{v}$ that maps the maximum absolute value of that vector to the maximum quantization level. While smaller vector lengths can lead to larger accuracy gains, the associated memory and computational overheads due to the per-vector scale factors increases. To alleviate these overheads, VSQ \citep{dai2021vsq} proposed a second level quantization of the per-vector scale factors to unsigned integers, while MX \citep{rouhani2023shared} quantizes them to integer powers of 2 (denoted as $2^{INT}$).

\subsubsection{MX Format}
The MX format proposed in \citep{rouhani2023microscaling} introduces the concept of sub-block shifting. For every two scalar elements of $b$-bits each, there is a shared exponent bit. The value of this exponent bit is determined through an empirical analysis that targets minimizing quantization MSE. We note that the FP format $E_{1}M_{b}$ is strictly better than MX from an accuracy perspective since it allocates a dedicated exponent bit to each scalar as opposed to sharing it across two scalars. Therefore, we conservatively bound the accuracy of a $b+2$-bit signed MX format with that of a $E_{1}M_{b}$ format in our comparisons. For instance, we use E1M2 format as a proxy for MX4.

\begin{figure}
    \centering
    \includegraphics[width=1\linewidth]{sections//figures/BlockFormats.pdf}
    \caption{\small Comparing LO-BCQ to MX format.}
    \label{fig:block_formats}
\end{figure}

Figure \ref{fig:block_formats} compares our $4$-bit LO-BCQ block format to MX \citep{rouhani2023microscaling}. As shown, both LO-BCQ and MX decompose a given operand tensor into block arrays and each block array into blocks. Similar to MX, we find that per-block quantization ($L_b < L_A$) leads to better accuracy due to increased flexibility. While MX achieves this through per-block $1$-bit micro-scales, we associate a dedicated codebook to each block through a per-block codebook selector. Further, MX quantizes the per-block array scale-factor to E8M0 format without per-tensor scaling. In contrast during LO-BCQ, we find that per-tensor scaling combined with quantization of per-block array scale-factor to E4M3 format results in superior inference accuracy across models. 


\end{document}
\endinput
%%
%% End of file `sample-sigconf-authordraft.tex'.
