\section{Conclusion}

This study introduced a comprehensive framework for evaluating tokenization strategies, highlighting the importance of balancing linguistic integrity and computational efficiency in morphologically rich and low-resource languages. By focusing on metrics such as token purity, Turkish Token Percentage (TR \%), processing time, and vocabulary size, the evaluation revealed the significant role of tokenization in shaping downstream model performance. Using the TR-MMLU dataset, which provides a rigorous benchmark for large language models, the analysis demonstrated that linguistic alignment is critical for achieving robust results in morphologically complex languages like Turkish.

The findings revealed that parameter size alone is not a definitive predictor of performance. For example, \texttt{gemma-2} (27.2 billion parameters) outperformed the larger \texttt{llama-3.1} (70.6 billion parameters) in TR-MMLU benchmarks, achieving higher scores for both TR \% and Pure \%, underscoring its superior ability to capture Turkish-specific morphological structures. Conversely, smaller models like \texttt{Qwen2.5}, while computationally efficient, exhibited lower linguistic fidelity due to their limited alignment with Turkish vocabulary and grammar. These results highlight the necessity of language-specific tokenization strategies to optimize performance for morphologically rich languages.

In addition to these findings, the study evaluated several tokenizers under development, such as \texttt{AhmetSemih/tr\_tokenizer} and \texttt{aliarda/turkish\_tokenizer}, which achieved remarkable initial results. These tokenizers demonstrated high TR \% and Pure \%, reflecting their ability to generate tokens aligned with Turkish linguistic structures. However, these initial developments represent only the foundational stage, with substantial potential for further improvements through the integration of advanced morphological analysis, unsupervised learning techniques, and domain-specific adaptations.

The implications of these findings extend beyond Turkish NLP. Linguistically informed tokenization strategies can enhance performance across a wide array of applications, particularly in low-resource settings where linguistic integrity is paramount. For example, in machine translation and sentiment analysis, preserving morphological and syntactic structures can significantly improve accuracy. Similarly, in specialized domains such as healthcare or legal contexts, domain-specific tokenizers can align with specialized terminologies, boosting the precision of text classification and information retrieval tasks.

Future research will focus on iterative refinements of tokenization strategies, including dynamic token generation tailored to downstream tasks and domain-specific requirements. Expanding the framework to evaluate tokenization performance across other morphologically rich languages, such as Finnish or Hungarian, and conducting cross-linguistic comparisons will provide deeper insights into the universal and language-specific aspects of tokenization.

In summary, this study establishes a new standard for evaluating tokenization strategies by combining linguistic fidelity and computational efficiency metrics. It demonstrates that tailored tokenization strategies can enable even smaller or less-optimized models to excel in morphologically complex settings. By advancing these strategies, this research aims to foster the development of robust, linguistically informed tokenizers that enhance the quality and applicability of large language models across multilingual and domain-specific NLP tasks.