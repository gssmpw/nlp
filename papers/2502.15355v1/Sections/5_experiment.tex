\section{EXPERIMENTS}


    %  We conduct extensive experiments on three real-world datasets (\gw{including two public accessible datasets and one private industrial dataset}) to answer the following research questions:
    % \begin{itemize}
    %     % \item \textbf{(RQ1)} 我们的模型与sota的CTR模型与量化推荐模型相比效果如何?(对应我们的overall performance部分)
    %     \item \textbf{(RQ1)} How does MemE-CTR perform compared to state-of-the-art CTR models and quantized recommendation models?
    
    %     % \item \textbf{(RQ2)} 我们的模型的不同模块如何带来提升?(对应我们的ablation study部分)
    %     \item \textbf{(RQ2)} How do the different modules of MemE-CTR contribute to performance improvement?
    
    %     % \item \textbf{(RQ3)} 我们的模型对实验设置的变化是否敏感,是否会明显受到超参数或预训练模型改变的影响?(对应我的Hyper-Parameter Analysis和Pretraining method analysis)
    %     \item \textbf{(RQ3)} Is our model responsive to variations in the experimental setup, and is it significantly influenced by alterations in hyperparameters or the pre-trained model?
    %     % \gw{Maybe put the industrial datasets experiments on a single section?} \kf{OK.}
        
    %     % \item \textbf{(RQ4)} 我们的模型在节约内存和提升速度方面能够带来什么样的提升?(对应我们的Time\&Memory Efficiency部分)
    %     \item \textbf{(RQ4)} What improvements can MemE-CTR bring in terms of memory efficiency and speed enhancement?

    %     % \item \textbf{(RQ5)} 为什么MemE-CTR要使用PQ作为它的量化方法,而不使用RQ等方法?
    %     \item \textbf{(RQ5)} Why does MemE-CTR use PQ as its quantization method instead of other methods?

    %     % RQ6 我们的模型能否在大规模工业数据集上同样取得优秀的表现?
    %     \item \textbf{(RQ6)} Can our model achieve high performance on large-scale industrial datasets?
    % \end{itemize}

   
    
    \section{Dataset Generation}
\label{sec:dataset}
\revise{
To train the proposed GNN, we constructed a dataset of building structures and a subset of these structures were subjected to fire simulations using FEA. The dataset generation process is illustrated in \figref{fig:dataset_generation_procedure}. Initially, a total of 33,000 building structures with geometrical details, material properties, and gravity loads were created. Due to randomness in generating these structures, a filter is applied to remove unreasonable data after gravity load simulation, which included 15,377 structures. A trade-off between computational feasibility and model performance is made among the remaining 17,623 structures. As further labeling structures with MIDR requires resource-intensive fire simulations via OpenSeesRT, a large proportion of 16,050 structures is selected as unlabeled dataset. On the other hand, each of the other 1,573 structures was further subjected to 30 different fire simulations, forming the labeled dataset containing $1,573\times 30 = 47,190$ fire cases.} This section details the step-by-step process for generating the dataset, including geometry creation, material property assignment, and simulations due to gravity loads and fire scenarios. 
% To train the proposed neural network, we constructed a dataset comprising building structure data and a subset of fire scenario data. The dataset generation process is illustrated in \figref{fig:dataset_generation_procedure}. 
% A total of 33,000 building structures with geometric details, material properties, and gravity loads were initially created. Out of these, 3,000 structures were selected as labeled data, and the remaining 30,000 were designated as unlabeled data. Further, about half of them filtered out due to instability under gravity loads only. 
\begin{figure*}[h!]
    \centering
    \includegraphics[width=0.8\linewidth]{figures/dataset_filter_procedure.pdf}
    \caption{Workflow for dataset generation (geometry, material property, gravity loads, and fire scenarios).}
    \label{fig:dataset_generation_procedure}
\end{figure*}

\subsection{Geometry Generation}
\label{subsec:geometry_generation}
The geometry of the building structures forms the foundation of the dataset. Regular 
\revise{3D structures} resembling multi-story parking structures or shopping malls were generated, with parameters such as building floor dimensions and story heights selected randomly. Each building structure is composed of multiple rooms, which serve as the basic unit in this study. A room herein is a cuboid space defined by specific length, width, and height. Within a structure, rooms of the same dimensions are uniformly arranged along the length, width, and height, corresponding to the $x$-, $y$-, and $z$-axes, respectively. Structures vary in room size and number of rooms along each axis. Specifically, the room length, width, and height are independently sampled from a uniform distribution within the interval $[2, 5]$ meters along the three directions of the structure. Similarly, the room number along each axis is uniformly sampled independently as an integer within the interval $[2, 7]$, i.e., the maximum number of stories of the buildings simulated in this study is 7.

To introduce variability and simulate real-world scenarios, approximately $8\%$ of structural elements (beams or columns) are randomly removed after initial geometry creation. 
\revise{Such removal is not fire-induced damage, but reflects functional diversity often observed in real buildings, such as open spaces designed for activities in shopping malls, e.g., ice skating rinks. Examples of the generated geometries are illustrated in \figref{fig:example_generated_geometry}, showcasing the diversity and realism of the dataset. This element removal does not affect the definition of room's geometry in the structure and nor does it affect the number of considered fire scenarios.} 

\revise{A range of coefficient of variation values ($3.3\%$ to $17.5\%$) was derived from prior studies that investigated the statistics of geometrical and material properties of structural components of buildings (e.g., \cite{mirza1979variations, lee2004probabilistic}). These studies provide empirical data on the natural variability in parameters such as Young's modulus, yield strength, and dimensions of structural elements due to manufacturing tolerances and material inconsistencies. By selecting $8\%$ for the removal of structural elements in our database, we aimed to maintain a level of variability that is representative of real-world uncertainties while ensuring computational feasibility. This choice ensures that the database captures realistic deviations without introducing extreme cases that may not be commonly encountered in practice.}

\begin{figure*}[h!]
    \centering
    \includegraphics[width=\linewidth]{figures/example_generated_geometry.pdf}
    \caption{Examples of generated structural geometry of different sizes (all dimensions in meters).}
    \label{fig:example_generated_geometry} 
\end{figure*}

{\blockRevise

In this study, we opted for a deterministic square, dimension of $0.1$ m, solid cross-sectional steel elements due to their simplicity in modeling and analysis. Square sections exhibit uniform geometrical properties in all directions, simplifying the computation of structural responses and avoiding complications associated with more complex shapes, such as wide-flange sections, facilitating the computational efficiency and scalability to generate a large dataset. This choice also helps to mitigate issues related to stress concentrations and facilitates a more straightforward representation of structural behavior under thermal loads. 

\textit{Remark:} The selected cross-section provides a comparable flexural rigidity to a $W 130 \times 130 \times 28.1$ wide-flange section (metric units), albeit with significantly higher axial rigidity. This cross-section is acceptable for gravity-load-designed frames under service loading conditions where the models assume fully rigid, moment-resisting beam-column connections for the evaluation of the IDR under thermal loading. This assumption is reasonable in this computational study where the primary interest is to understand the global deformation response of frames under fire conditions. The selection of uniform square cross-sections for both beams and columns, rather than adherence to standard capacity design principles, was made here primarily for computational efficiency and to reduce design parameters in the database generation process. This choice allows for simplified and scalable approach to analyze the fire-induced response of generic steel frames without the need for large section variations, where this study mainly focuses on the fire vulnerability assessment using ML-based predictions. However, if additional loading conditions, e.g., seismic or wind loads, were to be considered, larger sections, strong-column/weak-beam principle, and ductile detailing would be required in the generated buildings for realistic structural behavior under combined loading conditions. Future studies may also consider investigating the influence of variable cross-sectional dimensions and semi-rigid connections on the structural performance under fire conditions. 
} % blockRevise

\subsection{Material Properties}
Steel is chosen as the material for the structures. To reflect real-world variations, we randomly assign one of five slightly different steel material types to each structural element. \revise{
The ranges of material properties are provided in \tabref{tab:material_property_ranges} and the properties are sampled from uniform distributions of the corresponding ranges. These variations simulate differences arising from manufacturing batches or regional material properties. That these properties are at ambient temperature and change when the temperature rises due to a fire. The selection of materials with varying properties is aimed at increasing the diversity of the data. Our goal is to represent as wide a range of data as possible with a limited amount of building structure data, thereby enhancing the generalization ability of the GNN. Our assumed material property ranges are expected to be wider than the real-world conditions based on findings in \cite{mirza1979variations, lee2004probabilistic}. Therefore, we are essentially tackling a more challenging and general task. If we can solve this problem, we are confident that our method will perform equally well or even better in real-world scenarios.
}
\begin{table}[h!]
    \centering
    \caption{Material properties ranges for considered steel structures.}
    \begin{tabular}{lc}
        \toprule
        Property & Range \\
        \midrule
        Young's modulus & [168, 252] GPa \\
        Yield strength & [220, 330] MPa \\
        Strain-hardening ratio & [0.8, 1.2] \% \\
        \bottomrule
    \end{tabular}
    \label{tab:material_property_ranges}
\end{table}

\subsection{Gravity Loads}
Gravity loads are applied to columns and beams based on their \revise{influence (tributary) areas as typically conducted in structural analysis. The considered ``service'' load conditions include the column self-weight and the additional loads directly supported on the beams from their self-weight and weights of the reinforced concrete slabs, people as live load, and building content. An edge beam typically carries approximately half the gravity load supported by a parallel interior beam}. The ranges of gravity loads are listed in \tabref{tab:gravity_load_ranges}. \revise{The loads are sampled from uniform distributions of the corresponding ranges.} Structures that failed to meet an MIDR threshold of $1\%$ under gravity loads were deemed unacceptable designs and filtered out, as such configurations of randomly chosen geometry, material, and gravity load combinations were considered unrealistic from a regulatory and practicality points of view.
\begin{table}[h!]
    \centering
    \caption{Gravity load ranges for considered beams and columns.}
    \begin{tabular}{lc}
        \toprule
        Element & Range (kN/m)  \\
        \midrule
        Column & [0.5, 1.0]  \\
        Edge beam & [1.5, 4.5]  \\
        Interior beam & [3.0, 7.5]  \\
        \bottomrule
    \end{tabular}
    \label{tab:gravity_load_ranges}
\end{table} 

\subsection{Rule-based Thermal Load Generation}
\label{subsec:thermal_load_generation}
To evaluate a building's structural response during a fire event, we employed a simplified rule-based approach for thermal load generation. 
% Previous studies \cite{nan_structuralfire_2023} have demonstrated that steel structures rapidly equilibrate with surrounding gases temperatures due to efficient heat exchange. Consequently, gas temperatures can be directly used as inputs for FEA tools, e.g., OpenSees, simplifying the process of modeling thermal loads. 
% Accurately simulating temperature fields in fire scenarios poses significant challenges. Advanced thermodynamic simulations, such as those performed using Fire Dynamics Simulator (FDS) \cite{mcgrattan_fire_2000}, provide precise temperature predictions. However, these methods are hindered by high computational costs, prolonging execution times, and limited scalability, making them impractical for generating large datasets. Additionally, real-world fire loads often display substantial spatial variability across different rooms \cite{dundar_fire_2023}, resulting in scenario-specific temperature fields with limited generalizability. For example, studies on bridge fires \cite{he_study_2024} have demonstrated that environmental factors, such as wind speeds, can significantly influence temperature distributions. Furthermore, even within identical scenarios, variations in fire modeling methodologies can produce distinctly different temperature fields \cite{zhang_temperature_2020, du_new_2012}. These challenges emphasize the need for efficient and adaptable methods to generate fire temperature data.
% To address these issues, we adopted a rule-based approach to model temperature variations. 
According to \cite{spearpoint_fire_2008}, a typical fire development follows a predictable pattern. During the {\em{growth stage}}, the temperature rises slowly and approximately linearly after ignition. This is followed by the {\em{flashover stage}}, where temperatures increase rapidly to peak values. After reaching the peak, the temperature either stabilizes or continues to rise slowly until the {\em{decay stage}} begins. Inspired by this fire development pattern, we describe the temperature evolution in time, $t$, prior to the decay stage in two distinct stages:
\begin{enumerate}
    \item {\bf{Initial linear increase stage}}: For $t \in [0, t_1)$, temperature increases gradually and linearly as the fire spreads through the building. This stage represents the time before the fire directly affects a structural element.  
    \item {\bf{ISO 834 fire curve stage}}: For $t \in [t_1, t_{\thre}]$, temperature rises rapidly following the ISO 834 curve \cite{ISO834}, modeling the direct impact of the fire on the structural element. 
\end{enumerate}
The slope of the linear temperature increase, $c$, and the transition time, $t_1$, are influenced by the spatial relationship between the fire source and the structural element. For the second stage of temperature evolution, we utilize the ISO 834 curve, a widely accepted standard for fire resistance testing. This standardized fire curve describes the temperature rise over time, enabling rapid and consistent thermal fields across various scenarios. The duration of fire simulation in this study is set to $t_{\thre}=60$ minutes. This value represents the upper limit for the temperature evolution of each structural element, providing a consistent basis for analyzing the structural response to fire.

Let $(x, y, z)$ represents the midpoint of a structural element and $(x_{\subfire}, y_{\subfire}, z_{\subfire})$ the fire source point. \revise{Integer parameters $h$ and $h_{\subfire}$ correspond to the respective floor levels of the element and the fire source}. The temperature evolution for each element is expressed as follows:
\begin{enumerate}
    \item Linear increase stage ($0 < t < t_1$):
    \begin{equation}
    T(t) = c \cdot t,
    \end{equation}
    where $c$, the rate of temperature increase ($^\circ\mathrm{C}/\mathrm{min}$), depends on the height difference between the element, $h$, and the fire source, $h_{\subfire}$:
    \begin{equation}
        c = 
        \begin{cases} 
        5\left/\left(h - h_{\subfire} + 1\right)\right., & h \geq h_{\subfire}, \\
        2\left/\left(h_{\subfire} - h\right)\right., & h < h_{\subfire}.
        \end{cases}
    \end{equation}
     \item ISO 834 stage ($t \geq t_1$):
\begin{equation}
    T(t) = c \cdot t_1 + 345 \log_{10} \left(8 \left(t - t_1\right) + 1\right).
\end{equation}
\end{enumerate}

The transition (arrival) time $t_1$, marking the end of the linear stage, depends on the spatial distance between the fire source and the element. We define the following two Euclidean distances $L_p$ in the $xy$ plane and $L_s$ in the $xyz$ space:
\begin{eqnarray}
L_p & \triangleq & \sqrt{(x - x_{\subfire})^2 + (y - y_{\subfire})^2}, \\
\label{eq:Lp}
L_s & \triangleq & \sqrt{(x - x_{\subfire})^2 + (y - y_{\subfire})^2 + (z - z_{\subfire})^2}.
\label{eq:Ls}
\end{eqnarray}
Accordingly, the transition time, $t_1$, is expressed as follows:
\begin{equation}
    t_1 = 
    \begin{cases}
    \beta_{1} \cdot \left(1 - \exp\left\{- L_s\left/\alpha_{1}\right.\right\}\right), & h > h_{\subfire}, \\
    \beta_{2} \cdot \left(1 - \exp\left\{- L_p\left/\alpha_{2}\right.\right\}\right), & h = h_{\subfire}, \\
    \beta_{3} \cdot \left(1 - \exp\left\{- L_s\left/\alpha_{3}\right.\right\}\right), & h < h_{\subfire} .
    \end{cases}
    \label{eq:t1}
\end{equation}
The parameters $\beta_i$ and $\alpha_i$ for determining $t_1$ are summarized in Table~\ref{tab:fire_spread_parameters}. In this study, we take $r_{\mathrm{up}}=0.95$ and $r_{\mathrm{down}}=0.97$.
\begin{table}[ht]
    \centering
    \caption{Fire spread parameters for $t_1$ calculations.}
    \begin{tabular}{lcc}
        \toprule
        Case  & $\beta_i$ & $\alpha_i$  \\
        \midrule
        $i=1$, Upward spread & $16 \left.\left(1-r_{\mathrm{up}}^{\left|h-h_{\subfire}\right|}\right)\right/\left(1-r_{\mathrm{up}}\right)$ & $10$  \\
        $i=2$, Horizontal spread & $18$ & $18$  \\
        $i=3$, Downward spread & $30 \left.\left(1-r_{\mathrm{down}}^{\left|h-h_{\subfire}\right|}\right)\right/\left(1-r_{\mathrm{down}}\right)$ & $5$  \\
        \bottomrule
    \end{tabular}
    \label{tab:fire_spread_parameters}
\end{table}

\figref{fig:t1_curve} illustrates the $t_1$ curves for various fire scenarios: (1) fire originating on the lower floor, $h-h_{\subfire}=1$ with rapid upward spread, (2) fire on the same floor, $h=h_{\subfire}$ with the fastest spread, and (3) fire on the upper floor, $h_{\subfire}-h=1$ with slow downward spread. The exponential decay in $t_1$ reflects the accelerating fire propagation speed as the distance increases. \figref{fig:t1_curve} also indicates that the employed simplified model is consistent with the Markov chain-based dynamic model given by \cite{cheng_dynamic_2011}, where the rooms at the same floor of the fire point start flashover slightly before the corresponding upper floors. Additionally, $\beta_{1}$ and $\beta_{3}$ are the summation of a geometric sequence, where story level $h$ is the index. The common ratios $r_{\mathrm{up}}<1$ in $\beta_{1}$ and $r_{\mathrm{down}}<1$ in $\beta_{3}$ indicate that the fire speeds up to spread through the next story, which is consistent with the real-world fire spread mechanism given in \cite{hokugo_mechanism_2000}. The temperature profile within the range $t \in [0, t_{\thre}]$ is subsequently used as the thermal load in OpenSeesRT simulations to compute displacements at each structural node at time $t_{\thre}$.
\begin{figure}[h!]
    \centering
    \includegraphics[width=0.8\linewidth]{figures/m204_t1_curve.pdf}
    \caption{Three examples for the $t_1$ curve.}
    \label{fig:t1_curve}
\end{figure}

\revise{
\textit{Remark:} The effects of structural elements, such as concrete floor slabs and partitions, are not explicitly modeled in our approach. Instead, their influence is implicitly captured through the careful selection of the parameters $ \alpha, \beta, r_\mathrm{up} $, and $ r_\mathrm{down} $. This parameterization provides a unified framework for generating temperature fields. Indeed, fire propagation is governed by a multitude of factors and remains an open research question. For instance, if the fire resistance of a floor slab is enhanced by fire protective coating, the corresponding model can account for this by decreasing $\alpha_1$ \& $\alpha_3$, increasing $\beta_1$ \& $\beta_3$, and adopting larger values for $r_\mathrm{up}$ \& $r_\mathrm{down}$, which collectively slow down the vertical spread of fire. Conversely, scenarios involving higher amounts of combustible materials would warrant the opposite adjustments. This flexible and integrated approach avoids the need to design separate models for different fire propagation scenarios while still capturing the essential effects.
}

\revise{
In conclusion, our rule-based approach is a computationally efficient method for approximating fire temperature fields, enabling large-scale dataset generation to train predictive models. By combining ISO 834 fire curves with spatial considerations and embedding structural effects through parameter calibration, the method achieves a balanced trade-off between accuracy and scalability, making it a practical solution for thermal load modeling in fire scenarios. After generating the temperature of each beam or column according to the middle point, the temperature is applied as uniform thermal load to the elements of the structure in question using OpenSeesRT. 
}

% In conclusion, this rule-based approach is a computationally efficient method to approximate fire temperature fields, enabling large-scale dataset generation to train predictive models. By combining ISO 834 fire curves with spatial considerations, the method balances accuracy and scalability, making it a practical solution for thermal load modeling in fire scenarios.

% \subsection{Interstory Drift Ratio}
\subsection{OpenSeesRT Simulation}
\label{subsec:opensees_simulation}

The thermal and mechanical responses of 3D frame structures under combined fire and gravity loads are simulated using OpenSeesRT \cite{perez2024openseesrt}. \revise{In the simulation, the IDR of each node at $t_{\thre}$ is computed using the computed nodal displacements. Each structural model features six degrees of freedom per node (3 translational  and 3 rotational), with linear geometrical transformations (\texttt{geomTransf: Linear}) defining how the element local coordinate systems are mapped to the global coordinate system and assuming small displacements and rotations. Although OpenSeesRT allows a variety of options for modeling finite deformations, in the present simulations and mainly for simplicity, we did not consider large deformations. All bottom nodes (nodes on the ground) are fully constrained in all six degrees of freedom, while degrees of freedom os all other nodes are free.} Material behavior is temperature-dependent and modeled with \texttt{Steel01Thermal}, while fiber-based sections (\texttt{FiberThermal}) capture nonlinear interactions between thermal and mechanical responses at the cross-section level. \revise{Structural elements are represented as displacement-based Euler-Bernoulli beam-columns (\texttt{dispBeamColumnThermal}). This element  formulation accounts for thermal strains (temperature gradients) in the section, which is discretized into fibers. Numerical integration is used along the length of each element using three integration (Gauss) points, one at each end and the third in the middle of the element.}

{\revise{Thermal expansion of steel members plays a crucial role in IDR development. In reality, reinforced concrete floor slabs heat at a different rate than steel members due to their higher thermal mass and lower thermal conductivity. This differential heating can lead to restrained thermal expansion, introducing axial compression in beams and affecting the overall structural response. In this study, explicit {\em{composite action}} between steel members and concrete slabs is not modeled. Instead, our approach focuses on isolating the response of the steel structural frame, which is often the critical load-bearing component in fire scenarios. This assumption aligns with prior studies \cite{Possidente_2024} demonstrating that steel structures reach thermal equilibrium with surrounding gases quickly, allowing the use of uniform thermal loading in fire analysis. Future work could enhance this framework by incorporating slab-beam interaction effects, through a refined FEA for an extended dataset where constraints imposed by floor slabs are explicitly considered.}

The analysis begins with the application of gravity loads, followed by incremental thermal loads simulating the fire exposure. A static nonlinear solver using  \texttt{ExpressNewton} algorithm ensures convergence, while the \texttt{NormDispIncr} test maintains accuracy. An incremental \texttt{LoadControl} scheme with small step sizes is employed to guarantee numerical stability, using 10\% for gravity loads and 1\% for thermal loads. 

\revise{
In the thermal load analysis, uniform thermal load is applied to each beam or column, i.e., the temperature of each element is set to be that at the middle point, according to \secref{subsec:thermal_load_generation}. The \texttt{Steel01Thermal} material allows the properties (e.g., Young's modulus and yield strength) to be adjusted at increasing temperatures according to \cite{EN1993} using its Table 3.1: Reduction factors for the stress-strain relationship of carbon steel at elevated temperatures. For example, if the Young’s modulus at ambient temperature is $E_0$, then as the temperature ($T$) increases, the modulus changes as $E(T) = \eta (T) \times E_0$. \cite{EN1993} directly provides the values of $\eta(T) \in \left[0,1\right] $ at every $100 ^\circ\mathrm{C}$ interval and recommends using linear interpolation to obtain $\eta(T)$ for intermediate values of $T$.
} OpenSeesRT documentation \cite{OpenSeesThermalExamples} provides several examples of thermal analyses.

This modeling framework accommodates variations in material properties, cross-sectional geometries, and temperature profiles, providing robust simulations of structural behavior under fire conditions. The primary settings and configurations for the OpenSeesRT simulations are summarized in \tabref{tab:ops_detail}.
\begin{table}[h!]
    \centering
        \caption{Key settings of OpenSeesRT simulations.}
    \begin{tabular}{l|>{\raggedright\arraybackslash}p{0.6\linewidth}} %
    \toprule
    Modeling Aspect     & Details \\
    \midrule
    Geometry            & 3D models; 6 degrees of freedom per node \\
    Transformation      & geomTransf: Linear \\ 
    Material            & Steel01Thermal \\
    Section             & FiberThermal; Cross-section: $0.1$ m $\times$ $0.1$ m \\ 
    Element type        & {dispBeamColumnThermal} \\ 
    Loading             & Gravity loads: {beamUniform}; Thermal loads: {beamThermal} \\
    Integration scheme  & Incremental {LoadControl}; Step size: $10\%$ (gravity analysis), $1\%$ (thermal analysis) \\
    Nonlinear solver    & {ExpressNewton} algorithm; {UmfPack} solver; Convergence test: {NormDispIncr} tolerance: $10^{-8}$; Maximum \# iterations per step: $1000$. \\ 
    \bottomrule
    \end{tabular}
    \label{tab:ops_detail}
\end{table}

For each structure in the labeled dataset, 30 fire points are selected using a dual-granularity approach, \revise{i.e., two-stage sampling strategy,} to ensure they are well-distributed. Specifically, rooms are sequentially selected, with one fire point randomly chosen within each selected room. If a building is large and contains more than 30 rooms, we randomly select 30 rooms without replacement, i.e., ensuring that no more than one fire point is located in the same room. Conversely, if the building is small and has fewer than 30 rooms, all rooms are initially selected, with one fire point randomly assigned to each room. Additionally, rooms are then selected with replacement until a total of 30 fire points are assigned. \revise{The room-level sampling prioritizes selecting distinct rooms to avoid spatial clustering of fire points, while the point-level sampling ensures intra-room variability. This approach aligns with stratified sampling principles commonly used for efficient spatial representation, where multi-stage sampling strategies optimize coverage and variability, e.g., \cite{arunachalam_generalized_2023}, and enables a more comprehensive characterizing of how the structures respond under fire conditions.}
% This selection method prevents fire points from clustering too closely while maintaining an element of randomness. By distributing fire points in this manner, the 30 fire scenarios are effectively utilized, enabling a more comprehensive characterizing of how the structures respond under fire conditions.

\subsection{Summary of the Dataset Generation}
As discussed in this section and related to  \figref{fig:dataset_generation_procedure}, three key steps were considered in the development of the dataset: 
\begin{enumerate}
    \item {\bf{Filtering process}}: Structures with MIDR exceeding $1\%$ under gravity loads were excluded,  resulting in $1,573$ labeled structures retained for fire simulation and $16,050$ unlabeled structures for training the MFSP predictor.
    \item {\bf{Fire simulations}}: For each retained labeled structure, 30 fire scenarios were simulated using OpenSeesRT, yielding $47,190$ fire cases.
    \item {\bf{Data distribution check}}: MIDR distributions for labeled and unlabeled data under gravity loads were highly similar, because both datasets were generated using the same method. Under fire conditions, the MIDR distribution shifted, reflecting significant structural deformation with values reaching a maximum of about 6\%, an average of 1.70\%, and a standard deviation of 1.12\%. This step ensured a diverse and comprehensive dataset for the proposed predictive framework.
\end{enumerate}
The statistical distribution histograms for MIDR (after applying the $1\%$ filtering threshold \revise{for gravity load responses}) under different loading conditions are plotted in \figref{fig:histogram_mdr}. Figures \ref{fig:histogram_mdr}(a) and \ref{fig:histogram_mdr}(b) show the MIDR distributions of the labeled and unlabeled data, respectively, under gravity loads only. \figref{fig:histogram_mdr}(c) shows the MIDR distribution of the labeled data under the combined effects of gravity and fire loads. Fire load causes the structures to significantly deform, leading to a noticeably \revise{right-skewed} MIDR distribution.

\begin{figure*}[h!]
    \centering
    \includegraphics[width=\linewidth]{figures/histogram_mdr.pdf}
    \caption{Histograms of MIDR for labeled and unlabeled structures with gravity loads and fire cases.}
    \label{fig:histogram_mdr}
\end{figure*}

\revise{
This dataset provides the basis for training and testing the performance of the GNN-based framework. Although we employed a simplified rule-based thermal load generation method compared with conventional CFD-based simulations, the temperature field, the changes of the material properties, and the response of the structures, are all still highly nonlinear and complex. Therefore, it is still a challenging task for the NN to predict the MIDRs based on this dataset.
}
    % \subsection{Experiment Setup}
    %     \subsubsection{Datasets.}
    %         To evaluate the performance of our proposed models, we utilized three different datasets, including two public accessible datasets and one private industrial dataset. 
    %         \begin{itemize}[leftmargin=*,align=left]
    %             \item Criteo: An industry-standard dataset available on Kaggle\footnote{https://www.kaggle.com/c/criteo-display-ad-challenge/}, which consists of user click data collected over one week to predict ad click-through rates (CTR). This comprehensive dataset contains 45 million samples and 39 features, including 13 continuous features and 26 categorical features, providing valuable insights for model evaluation and performance benchmarking.
                
    %             \item Avazu: A benchmark dataset available on Kaggle\footnote{https://www.kaggle.com/c/avazu-ctr-prediction/}, that is wildly used for CTR prediction and contains detailed user click data over 11 consecutive days. The dataset consists of approximately 40 million samples and includes 24 features, offering a robust foundation for algorithm testing and model improvement.

    %             \item Industrial: A private industrial dataset drawn from an online ad platform that has impressions of more than 400 million users. 
    %             The dataset contains hundreds of features, including categorical features, numerical features, and sequential features. 
    %             Among these features, there are 32 features with more than 100,000 vocab sizes, and the largest vocab size can reach six million.
    %         \end{itemize}
    %         We followed the preprocessing steps outlined in AFN \cite{AFN} to split
    %         Criteo and Avazu into training, validation, and test sets in a 7:2:1 ratio following the time order.
    %         For the Industrial dataset, we split them into train/val/test sets with a 6:1:1 proportion.
    %         Table \ref{tab:dataset} provides detailed statistics for these two public datasets. 



\subsection{Experiment Setup}
    \subsubsection{Datasets.}
        We evaluated our model using three datasets: two public and one private industrial dataset. 
        \begin{itemize}[leftmargin=*,align=left]
            \item Criteo: A standard dataset from Kaggle\footnote{https://www.kaggle.com/c/criteo-display-ad-challenge/} with one week of user click data for CTR prediction. It includes 45 million samples and 39 features (13 continuous, 26 categorical), useful for model evaluation.

            \item Avazu: Another Kaggle dataset\footnote{https://www.kaggle.com/c/avazu-ctr-prediction/}, used for CTR prediction with 11 days of user click data. It contains about 40 million samples and 24 features, serving as a solid base for testing algorithms.

            \item Industrial: A private dataset from the Huawei ad platform with over 400 million user impressions. It includes hundreds of features, both categorical and numerical, with 32 features having vocab sizes over 100,000, and the largest reaching six million.
        \end{itemize}
        We followed AFN \cite{AFN} preprocessing, splitting Criteo and Avazu datasets into training, validation, and test sets in a 7:2:1 ratio by time order. The Industrial dataset was split into train/val/test sets in a 6:1:1 ratio. Table \ref{tab:dataset} provides detailed statistics for the public datasets.





            
    \subsubsection{Baseline Models}
    To demonstrate the effectiveness of the proposed model, we select some representative CTR models for comparison.
    We also compare our proposed model with some hashing and quantization-based methods to validate the superiority of our model. 
    The details are listed as follows:
         
\textbf{Representative CTR Models:}
    \begin{itemize}[leftmargin=*,align=left]
        \item \textbf{LR} \cite{LR}: Logistic Regression is a linear model using a logistic function for binary variables. It is often a baseline in recommendation systems due to its simplicity and interpretability.
    % \end{itemize}     
    % \textbf{Factorization-based Models}
    % \begin{itemize}[leftmargin=*,align=left]
        \item \textbf{FM} \cite{FM}: Factorization Machine models pairwise interactions between features efficiently, making it suitable for sparse data.
        \item \textbf{AFM} \cite{AFM}: Attentional Factorization Machine enhances FM by using attention to model feature interaction importance.
        \item \textbf{DeepFM} \cite{DeepFM}: Combines FM for low-order and DNN for high-order interactions, improving recommendation accuracy.
    % \end{itemize}          
    % \textbf{Neural Network Models}
    %     \begin{itemize}[leftmargin=*,align=left]
            \item \textbf{PNN} \cite{PNN}: Product-based Neural Network uses product operations between feature embeddings to model feature interactions.
            \item \textbf{DCNv2} \cite{DCNv2}: Deep \& Cross Network v2 captures explicit and implicit feature interactions by stacking cross and deep layers.
            \item \textbf{AutoInt}\cite{AutoInt}: Utilizes self-attention mechanisms to automatically learn feature interactions without manual feature engineering.
            \item \textbf{AFN} \cite{AFN}: Adaptive Factorization Network dynamically learns the importance of feature interactions using a neural network.
            \item \textbf{SAM} \cite{SAM}: Self-Attentive Model leverages self-attention to capture complex feature interactions effectively.
            \item \textbf{GDCN} \cite{GDCN}: Gate-based Deep Cross Network uses gating mechanisms to model the complex relationships between users and items, enhancing collaborative filtering signals.
        \end{itemize}
              
    \textbf{Hashing and Quantization Models:}
        \begin{itemize}[leftmargin=*,align=left]
            \item \textbf{DHE} \cite{DHE}: Deep Hash Embedding uses hashing techniques for dimensionality reduction, improving memory and time efficiency.
            \item \textbf{xLightFM} \cite{xLightFM}: An extremely memory-efficient Factorization Machine that uses codebooks for embedding composition and adapts codebook size with neural architecture search.
        \end{itemize}
                % \gw{Where is AFM, AFN, SAM, DHE, and LightFM?}
    % \subsubsection{Evaluation Metrics.}
        \begin{table*}[t]
\centering
\footnotesize

\begin{subtable}{\textwidth}
\centering
\begin{adjustbox}{width=\textwidth, center}
\begin{tabular}{llccccccccccc}
\toprule
\multirow{2}{*}{Model} & \multirow{2}{*}{Modality} & \multicolumn{11}{c}{mAP (\%)} \\
\cmidrule(lr){3-13}
 & & 0.20s & 0.40s & 0.60s & 0.80s & 1.00s & 1.20s & 1.40s & 1.60s & 1.80s & 2.00s & Avg \\
\midrule
\multirow{4}{*}{\centering Transformer} 
 & A & 72.2 & 65.2 & 60.3 & 56.8 & 54.4 & 53.1 & 52.4 & 52.0 & 51.6 & 51.4  & 56.9 $\pm$ 0.05 
 \\
 & V & 52.0 & 51.7 & 51.6 & 51.3 & 51.1 & 50.9 & 50.8 & 50.5 & 50.3 & 50.1  & 51.0 $\pm$ 0.08 
 \\
 & A+V & 73.8 & 66.9 & 62.1 & 58.5 & 56.3 & 55.0 & 54.1 & 53.7 & 53.3 & 53.0 & 58.7 $\pm$ 0.13
 \\
 & A+V\textsuperscript{P} & 73.4 & 66.8 & 61.8 & 58.3 & 56.1 & 54.8 & 54.1 & 53.5 & 53.2 & 52.7  & 58.5 $\pm$ 0.26 
 \\
\midrule
\multirow{4}{*}{\centering GRU} 
 & A & 71.5 & 65.0 & 60.1 & 57.0 & 55.0 & 53.8 & 52.9 & 52.2 & 51.5 & 50.9  & 57.0 $\pm$ 0.30 
 \\
 & V & 53.0 & 52.7 & 52.4 & 52.0 & 51.7 & 51.6 & 51.2 & 51.1 & 50.8 & 50.6 & 51.7 $\pm$ 0.29 
 \\
 & A+V & 73.5 & 68.1 & 63.7 & 60.7 & 59.1 & 58.1 & 57.2 & 56.3 & 55.4 & 54.4 & 60.6 $\pm$ 0.17 
 \\
 & A+V\textsuperscript{P} & 70.8 & 64.9 & 60.1 & 56.9 & 55.0 & 53.8 & 53.0 & 52.4 & 51.8 & 51.4 & 57.0 $\pm$ 0.29 
 \\
\midrule
\multirow{4}{*}{\centering Mamba} 
 & A & 67.5 & 62.2 & 58.4 & 55.7 & 54.0 & 52.9 & 52.0 & 51.1 & 50.2 & 49.6 & 55.4 $\pm$ 0.62 
 \\
 & V & 52.2 & 51.8 & 51.5 & 51.1 & 50.9 & 50.7 & 50.5 & 50.4 & 50.0 & 49.7 & 50.9 $\pm$ 0.21 
 \\
 & A+V & 71.8 & 65.4 & 60.5 & 57.1 & 55.0 & 53.9 & 53.5 & 53.1 & 52.3 & 51.8 & 57.4 $\pm$ 0.26 
 \\
 & A+V\textsuperscript{P} & 68.9 & 63.2 & 59.1 & 56.0 & 54.0 & 52.7 & 51.8 & 51.4 & 50.7 & 50.1 & 55.8 $\pm$ 0.43 
 \\
\bottomrule
\end{tabular}
\end{adjustbox}
\label{tab:main_result_a}
\caption*{(a) Results on EasyCom} %
\end{subtable}


\begin{subtable}{\textwidth}
\centering
\begin{adjustbox}{width=\textwidth, center}
\begin{tabular}{llccccccccccc}
\toprule
\multirow{2}{*}{Model} & \multirow{2}{*}{Modality} & \multicolumn{11}{c}{mAP (\%)} \\
\cmidrule(lr){3-13}
 & & 0.20s & 0.40s & 0.60s & 0.80s & 1.00s & 1.20s & 1.40s & 1.60s & 1.80s & 2.00s & Avg \\
\midrule
\multirow{4}{*}{\centering Transformer} 
 & A & 78.8 & 74.9 & 71.8 & 69.7 & 68.1 & 67.0 & 66.3 & 65.7 & 65.1 & 64.7 & 69.2 $\pm$ 0.03
 \\
 & V & 58.7 & 58.5 & 58.4 & 58.2 & 58.1 & 58.0 & 57.9 & 57.8 & 57.7 & 57.7 & 58.0 $\pm$ 0.27
 \\
 & A+V & 78.1 & 74.3 & 71.5 & 69.4 & 68.0 & 67.0 & 66.3 & 65.7 & 65.3 & 64.9 & 69.0 $\pm$ 0.24
 \\
 & A+V\textsuperscript{P} & 78.4 & 74.5 & 71.5 & 69.4 & 67.9 & 66.7 & 65.9 & 65.4 & 65.0 & 64.5 & 68.9 $\pm$ 0.18
\\
\midrule
\multirow{4}{*}{\centering GRU} 
 & A & 78.6 & 74.8 & 71.8 & 69.6 & 68.1 & 66.9 & 66.2 & 65.6 & 65.2 & 64.8  & 69.2 $\pm$ 0.25
 \\
 & V & 58.6 & 58.3 & 58.1 & 57.9 & 57.8 & 57.8 & 57.7 & 57.6 & 57.5 & 57.5  & 57.9 $\pm$ 0.61
 \\
 & A+V & 76.4 & 73.0 & 70.4 & 68.5 & 67.1 & 66.3 & 65.6 & 65.2 & 64.7 & 64.4  & 68.2 $\pm$ 0.42
 \\
 & A+V\textsuperscript{P} & 76.9 & 73.4 & 70.6 & 68.6 & 67.3 & 66.3 & 65.6 & 65.1 & 64.7 & 64.4  & 68.3 $\pm$ 0.18
 \\
\midrule
\multirow{4}{*}{\centering Mamba} 
 & A & 77.4 & 73.6 & 70.5 & 68.5 & 66.9 & 65.8 & 65.0 & 64.3 & 63.9 & 63.5  & 67.9 $\pm$ 0.37
 \\
 & V & 58.2 & 58.1 & 57.9 & 57.8 & 57.6 & 57.5 & 57.5 & 57.4 & 57.4 & 57.3 & 57.7 $\pm$ 0.28
 \\
 & A+V & 76.0 & 72.5 & 69.8 & 67.9 & 66.6 & 65.6 & 64.8 & 64.2 & 63.8 & 63.5  & 67.5 $\pm$ 0.18
 \\
 & A+V\textsuperscript{P} & 74.1 & 70.8 & 68.1 & 66.2 & 64.8 & 63.9 & 63.2 & 62.7 & 62.3 & 62.0 & 65.8 $\pm$ 0.23
 \\
\bottomrule
\end{tabular}
\end{adjustbox}
\label{tab:main_result_b}
\caption*{(b) Results on Ego4D} 
\end{subtable}

\caption{Mean average precision (mAP) scores on (a)~EasyCom and (b)~Ego4D at time steps 0.20\,s to 2.00\,s for Transformer, GRU, and Mamba architectures under Audio~(A), Visual~(V), and Audio+Visual~(A+V) modalities. Models with \textsuperscript{P} are pretrained on YT-Conversation. Per-timestep values come from a single random seed, while ``Avg'' shows mean $\pm$ SE over five seeds (see \Cref{app:stat_results} for full multi-seed results). Using both A and V yields the best performance overall.}




\label{tab:main_result}
\end{table*}












 
    %         We evaluate the algorithms using two popular metrics: AUC \cite{AUC} and Logloss \cite{LogicR}. The AUC (Area Under the ROC Curve) metric measures the ability of the model to rank positive items (i.e., samples with label 1) higher than negative ones. A higher AUC indicates better recommendation performance. Logloss measures the distance between the predicted probabilities and the ground-truth labels.
    %         A smaller Logloss value indicates better prediction performance. 
    %         For our task, it is important to notice that our approach saves up to over 90\% of memory.  
    %         Therefore, even achieving performance on par with the baselines can still bring significant business value in practical recommendation scenarios.

    % \subsubsection{Parameter Settings.}
    %     We implemented all models utilizing FuxiCTR, an open-source CTR prediction library \footnote{https://fuxictr.github.io/}. 
    %     To ensure a fair comparison, we standardized the embedding dimension across all models to 40 and set the batch size to 10,000. 
    %     The learning rate was tuned from the set {1e-1, 1e-2, 1e-3, 1e-4}, while $L_2$ regularization was adjusted within the range {0, 1e-1, 1e-2, 1e-3, 1e-4, 1e-5}, and the dropout ratio varied from 0 to 0.9. 
    %     All baseline models were trained using the Adam \cite{adam} optimizer. 
    %     We also varied the codebook size among {256, 512, 1024, 2048} and the number of code layers for PQ layers among {2, 4, 8}. 
    %     To ensure the robustness and reliability of our findings, each experiment was conducted five times, with the average performance being reported. 
    %     We measured the memory usage of each model based on the complete model size. 
    %     Moreover, we present the experimental results using the optimal parameter configurations. 
    %     Moreover, we employed the pre-trained embedding layer of DeepFM \cite{DeepFM} as the initial embedding before quantization.



\subsubsection{Evaluation Metrics}
% We assess the algorithms using AUC and Logloss metrics. AUC (Area Under the ROC Curve) evaluates the model's ability to rank positive instances higher than negatives, with higher values indicating better performance. Logloss measures the discrepancy between predicted probabilities and actual labels, where lower values indicate better accuracy. Our method achieves over 90\% memory savings, maintaining competitive performance with baselines and offering significant practical value.
We evaluate the algorithms using AUC and Logloss metrics to ensure a comprehensive performance assessment. AUC (Area Under the ROC Curve) evaluates the model's capability to rank positive instances above negatives, with higher values indicating superior discrimination ability. Logloss, on the other hand, measures the accuracy of predicted probabilities in relation to actual labels, with lower values indicating better model calibration and precision. Our method achieves over 90\% memory savings while maintaining performance on par with baseline models. This efficiency, coupled with competitive accuracy, underscores the method's practical value, especially in environments where memory constraints are a concern.



\subsubsection{Parameter Settings}
All models were implemented using the FuxiCTR library\footnote{https://fuxictr.github.io/}. We standardized the embedding dimension to 40 and batch size to 10,000. The learning rate was chosen from \{1e-1, 1e-2, 1e-3, 1e-4\}, with $L_2$ regularization from \{0, 1e-1, 1e-2, 1e-3, 1e-4, 1e-5\}, and dropout ratios from 0 to 0.9. Models were trained using the Adam optimizer. Codebook sizes were \{256, 512, 1024, 2048\} and PQ layers \{2, 4, 8\}. Experiments were repeated five times to ensure reliability, reporting average results. Memory usage was based on full model size, and results are presented with optimal parameters. We used the pre-trained embedding layer of DeepFM as initialization before quantization.






    \subsection{Performance Comparison}
        % \subsubsection{Public Dataset Performance}
        % \label{subsec:public_performance}
        % In this section, we evaluate the performance of our proposed framework, MemE-CTR, on two widely-used CTR prediction datasets: Criteo and Avazu. 
        % In this section, we compare MemE-CTR with baseline models, and the results are presented in Table \ref{tab:MainTable}.
        % In this section, we compare MEC with baseline models in terms of both CTR performance and memory usage, and the results are presented in Table \ref{tab:MainTable}. We also conduct Wilcoxon signed rank tests \cite{p_test} to evaluate the statistical significance of MEC with the base model. We have the following observations:
        % The results in Table \ref{tab:MainTable} highlight several key findings:

        % \textbf{Finding 1:} Meme-CTR achieves performance that is on par with or superior to traditional CTR models. \gw{which model? on par with or superior? give a definite conclusion} 

        % \textbf{(1) The embedding tables play a great part in model parameters.} 
        % The parameters of traditional CTR models are mainly concentrated in the embedding layer. 
        % Despite the model structure and complexity varying a lot from each other, the overall parameters are much of the same size. 
        % As shown in Table \ref{tab:MainTable}, optimizing the embedding tables can reduce model parameters by over 90\%, highlighting the significant potential for quantization and compression techniques to improve efficiency and performance.
        
        \begin{figure}[t]
            \centering
            \setlength{\abovecaptionskip}{-5pt}   % 图注上方的距离
            \setlength{\belowcaptionskip}{0pt}   % 图注下方的距离
            \includegraphics[width=0.75\textwidth]{Sections/Figures/bars.pdf}
            \caption{Time efficiency performance}
        \label{fig:time_eff}
        % \vspace{-14pt}
        \end{figure}
        
        % \textbf{(2) Hashing and quantization models bring improvements but still have limitations. } 
        % An examination of rows 9 to 16 in Table \ref{tab:MainTable} reveals that existing hashing and quantization techniques, such as DHE \cite{DHE} and xLightFM \cite{xLightFM} have achieved a significant reduction of parameters but lead to a marked deterioration in performance. 
        % The primary reason for this decline is that both DHE and xLightFM prioritize optimal compression of the embedding table during quantization while overlooking the imbalances in data distribution and the unevenness of embedding caused by traditional quantization methods.
        % Apart from that, the existing quantization methods must be trained end-to-end and cannot be migrated to other CTR models directly.
        % In contrast, Our approach addresses these challenges by using a portable method that rebalances the distribution of quantized data, thus achieving superior performance compared to alternative compression models.


        % \textbf{(3) Embedding lookup time is closely tied to the size of the embedding tables.}
        % As shown in Figure \ref{fig:time_eff}, we compared the total time taken by each model to perform embedding lookup 100 times under identical experimental conditions.
        % With quantization algorithms, we achieved about a 50\% reduction in time, thanks to the smaller vocabulary size. 
        % This shows a strong correlation between inference time and memory size. 
        % Moreover, we also analyzed the training latency, and our experiments demonstrated that the modifications to the PQ quantization algorithm introduced negligible latency compared to the training process itself, underscoring the method's practicality. 
        % Full experimental results can be found in Appendix C.
        % Therefore, with MEC's significant parameter compression and high-quality quantization, we improved both inference efficiency and model performance.
        
        % \textbf{(4) MEC achieves superior performance across all datasets.} 
        % As shown in Table 2, when GDCN is used as the base model, $\text{MEC}_{GDCN}$ performs better than GDCN by 0.4\% on the Avazu dataset and 0.07\% on the Criteo dataset, while using fewer parameters. 
        % It is important to note that, as a pluggable model, MEC achieves improvements of up to 0.33\% and 0.41\% over both PNN and GDCN, with the most significant enhancements seen in the results from the Avazu dataset.
        % The more significant improvements in the Avazu dataset can be attributed to its pronounced long-tailed distribution, particularly in certain feature fields with a large number of features. 
        % This long-tail effect typically hampers the embedding capability of conventional models. 
        % Our method effectively addresses this issue by employing popularity-based regularization, which enhances the overall quality of feature embeddings.




        In this section, we compare MEC with baseline models in terms of both CTR performance and memory usage, and the results are presented in Table \ref{tab:MainTable}. We also conduct Wilcoxon signed rank tests \cite{p_test} to evaluate the statistical significance of MEC with the base model. We have the following observations:
        
\textbf{(1) Embedding tables significantly impact model parameters.} 
In traditional CTR models, most parameters are concentrated in the embedding layer. Despite variations in model structure and complexity, the overall parameter size remains similar. Table \ref{tab:MainTable} shows that optimizing embedding tables can reduce model parameters by over 90\%, highlighting the potential for quantization and compression to enhance efficiency and performance.

\textbf{(2) Hashing and quantization models have benefits but limitations.} 
Rows 9 to 16 in Table \ref{tab:MainTable} indicate that methods like DHE \cite{DHE} and xLightFM \cite{xLightFM} reduce parameters significantly but degrade performance. This is due to their focus on compression without addressing data distribution imbalances caused by traditional quantization. These methods also require end-to-end training and lack portability. Our approach overcomes these issues by rebalancing quantized data distribution, achieving better performance.

\textbf{(3) Embedding lookup time correlates with embedding table size.}
Fig. \ref{fig:time_eff} shows that embedding lookup time is reduced by about 50\% with quantization due to smaller vocabulary sizes, indicating a strong link between inference time and memory size. Training latency analysis reveals that PQ quantization modifications add negligible latency, confirming the method's practicality. Full results are in Section 5.4. MEC's parameter compression and high-quality quantization enhance both inference efficiency and model performance.

\textbf{(4) MEC excels across all datasets.} 
Table 2 shows that $\text{MEC}_{GDCN}$ outperforms GDCN by 0.4\% on Avazu and 0.07\% on Criteo, with fewer parameters. As a pluggable model, MEC improves performance by up to 0.33\% and 0.41\% over PNN and GDCN, especially on Avazu due to its long-tailed distribution. This distribution often limits conventional models, but our method's popularity-based regularization enhances feature embedding quality.











    % \subsection{Industrial Dataset Performance}
    %     In Table \ref{tab:industry}, we present a comprehensive analysis of the performance and memory usage of MEC applied to an industrial dataset. 
    %     The comparison demonstrates that MEC, by utilizing Product Quantization's robust quantization capabilities reduces more than 99.7\% memory consumption of the embedding layer. 
    %     Additionally, our implementation of popularity-weighted regularization and contrastive learning effectively addresses common challenges encountered in quantized models, such as code allocation imbalance and uneven distribution of quantization centers. 
    %     Consequently, even with substantial parameter quantization, our method achieves performance that is comparable to, or even superior to, that of the baseline model.

% \subsection{Industrial Dataset Performance}

% Table \ref{tab:industry} shows that MEC reduces embedding layer memory usage by over 99.7\% using Product Quantization. The use of popularity-weighted regularization and contrastive learning addresses issues like code imbalance and uneven quantization distribution. Despite heavy quantization, our method matches or exceeds baseline model performance.

\subsection{Industrial Dataset Performance}

Table \ref{tab:industry} clearly demonstrates that MEC achieves over 99.7\% reduction in embedding layer memory usage through efficient Product Quantization. The integration of popularity-weighted regularization and contrastive learning effectively addresses typical challenges like code imbalance and uneven quantization distribution. Despite significant quantization, our method consistently maintains or even surpasses the performance of the baseline model.




    
    % \subsection{Time \& Memory Efficiency}
    
    %     \begin{table}[t]
\setlength{\abovecaptionskip}{0.05cm}
\setlength{\belowcaptionskip}{0.2cm}
\caption{Time \& memory efficiency result}
\setlength{\tabcolsep}{3mm}{
\resizebox{0.45\textwidth}{!}{
\begin{tabular}{c|c|c|c|c}
\toprule
 &
    \multicolumn{2}{c|}{\multirow{-1}{*}{\textbf{Criteo}}} &
    \multicolumn{2}{c}{\multirow{-1}{*}{\textbf{Avazu}}} \\ \midrule
    \multicolumn{1}{c|}{\textbf{Variants}} &
    \multicolumn{1}{c|}{\textbf{Time Costs}} &
    \multicolumn{1}{c|}{\textbf{Params}} &
    \multicolumn{1}{c|}{\textbf{Time Costs}} &
    \multicolumn{1}{c}{\textbf{Params}} \\   \midrule\midrule
    \textbf{PNN} & 280ms & 166.9MB & 148ms & 103.9MB \\
    \textbf{DHE} & 154ms & 1.60MB & 73ms & 0.45MB \\
    \textbf{xLightFM} & 158ms & 1.92MB & 76ms & 0.64MB \\
    \textbf{MemE-PNN} & \textbf{146ms} & 2.51MB & \textbf{75ms} & 0.66MB \\
    \textbf{}
\bottomrule
\end{tabular}
}}
\label{tab:Time_memory}
\end{table}
    %     \kf{Considering changing this table to a chart, being drawn by Lvhang}
    %     This study compares the memory and time consumption during the embedding step across various models, including the baseline PNN \cite{PNN}, DHE \cite{DHE}, xLightFM \cite{xLightFM}, and our proposed MemE-PNN. We utilize PNN as the baseline model, focusing exclusively on measuring the inference time and parameter count of the embedding layer. Table \ref{tab:Time_memory} displays the total time required to infer 100 batches.
        
    %     From Table \ref{tab:Time_memory}, it is evident that the baseline PNN \cite{PNN} exhibits the highest memory consumption and processing time, attributable to its large embedding table. In practical recommendation scenarios, the use of extensive embedding tables can lead to challenges associated with exceeding GPU memory limitations, thereby causing notable delays in data retrieval and processing.
        
    %     MemE-CTR, DHE \cite{DHE}, and xLightFM \cite{xLightFM} exhibit similar compression capabilities in terms of Time Costs and Params. 
    %     With the same parameters, all three methods achieve approximately 99\% compression in Params. However, as shown in Section 2, MemE-CTR significantly outperforms DHE and xLightFM in recommendation performance.
        
    %     Specifically, MemE-CTR achieves an AUC improvement of up to 0.003 over DHE and xLightFM. Given that a 0.001 improvement in AUC is considered highly significant for CTR tasks, this level of enhancement is crucial for real-world recommendation scenarios. 
    %     This performance advantage arises because DHE and xLightFM focus too much on maximizing memory savings, which compromises the quality of embeddings. 
    %     In contrast, MemE-CTR employs Product Quantization (PQ) to maintain high quantization efficiency while using popularity-weighted regularization and contrastive learning to allocate better codes for features. 
    %     This approach not only enhances the quality of quantized embeddings but also significantly improves the model's recommendation performance, making it more effective in real-world industrial applications.


    \begin{figure}[t]
        \centering
        \includegraphics[width=\textwidth]{Sections/Figures/line_v5.pdf}
        \vspace{-2em}
        \caption{Hyper-Parameter Performance}
    \label{fig:line_chart}
    \end{figure}
% \vspace{-3em}
    \subsection{In-depth Analysis}
        \begin{table}[t]
\centering
\setlength{\abovecaptionskip}{0.05cm}
\setlength{\belowcaptionskip}{0.2cm}
\caption{Performance on industrial dataset}
\setlength{\tabcolsep}{2mm}{
\resizebox{0.75\textwidth}{!}{
\begin{tabular}{c|c|c|c|c|c}
\toprule
    \textbf{Metric} & \textbf{PRAUC}$\uparrow$ & \textbf{PCOC}$\uparrow$ & \textbf{AUC}$\uparrow$ & \textbf{LogLoss}$\downarrow$ & \textbf{Params}$\downarrow$ \\ \midrule
    \textbf{baseline} & 0.91913 & 0.96561 & 0.84141 & 0.40721 & 3142.3MB \\
    \textbf{MEC} & \textbf{0.91918} & \textbf{0.96673} & \textbf{0.84143} & \textbf{0.40683} & \textbf{7.014MB} \\
\bottomrule
\end{tabular}
}}
\label{tab:industry}
% \vspace{-10pt}
\end{table}

        

        % \subsubsection{\textbf{Ablation Study of MEC}}
        %     \begin{table*}
  [t]
  \centering
  \resizebox{\textwidth}{!}{%
  \begin{tabular}{cccccccccccc}
    \toprule \multicolumn{2}{c}{Components}                                                             & \multicolumn{5}{c}{Re-executability Rate (\%)} & \multicolumn{5}{c}{Readability (\#)} \\
    \cmidrule(lr){1-2} \cmidrule(lr){3-7} \cmidrule(lr){8-12}        \hspace{8pt}\labelemoji\hspace{8pt}                                                                & \hspace{8pt}\toolemoji\hspace{8pt}                                      & O0                                 & O1             & O2             & O3             & AVG            & O0             & O1             & O2             & O3             & AVG            \\
    \hline
    \rowcolor[rgb]{0.93,0.93,0.93}\multicolumn{12}{c}{\textbf{Initialize with LLM4Decompile-End-6.7B~\citep{llm4decompile}}}   \\
    \xmark                                                                                              & \xmark                                    & 69.51                              & 46.95          & 50.61          & 46.34          & 53.35          & 3.98 & 3.41 & 3.44 & 3.38 & 3.55 \\
    \cmark                                                                                              & \xmark                                    & 75.61                              & 50.61          & 50.00          & 50.00          & 56.55          & 4.01 & 3.44 & 3.39 & \textbf{3.49} & 3.58 \\
    \xmark                                                                                              & \cmark                                    & 83.54                     & \textbf{56.10}          & 51.22          & 50.61 & 60.37 & 4.05 & 3.51 & 3.51 & 3.42 & 3.62 \\
    \cmark                                                                                              & \cmark                                    & \textbf{85.37}                            & \textbf{56.10}                     & \textbf{51.83} & \textbf{52.43}          & \textbf{61.43} & \textbf{4.13} & \textbf{3.60} & \textbf{3.54} & \textbf{3.49} & \textbf{3.69} \\

    \rowcolor[rgb]{0.93,0.93,0.93}\multicolumn{12}{c}{\textbf{Initialize with Deepseek-Coder-6.7B-base~\citep{deepseekcoder}}} \\
    \xmark                                                                                              & \xmark                                    & 59.15                              & 35.98          & 39.02          & 37.80          & 42.99          & 3.71 & 3.05 & 3.16 & 3.05 & 3.24 \\
    \cmark                                                                                              & \xmark                                    & 66.46                              & 41.46          & 38.41          & 36.59          & 45.73          & 3.76 & 3.17 & \textbf{3.21} & 3.08 & 3.31 \\
    \xmark                                                                                              & \cmark                                    & 70.73                              & 39.63          & 39.02          & 40.24          & 47.41          & 3.90 & 3.17 & 3.08 & 3.11 & 3.31 \\
    \cmark                                                                                              & \cmark                                    & \textbf{79.88}                     & \textbf{45.73} & \textbf{43.90} & \textbf{42.68} & \textbf{53.05} & \textbf{3.96} & \textbf{3.21} & 3.18 & \textbf{3.19} & \textbf{3.38} \\
    \bottomrule
  \end{tabular}%
  }
  \caption{The ablation study of different methods across four optimization levels
  (O0, O1, O2, O3), as well as their average scores (AVG). The results in bold represent the optimal performance. The ~\labelemoji~ and ~\toolemoji~ means Relabedling and Function Call. \textbf{Bold} denotes the best performance.}
  \label{tab:ablation}
\end{table*}
        %     To investigate the effect of each part of MEC, we investigated the impact of regularization and contrastive learning on the performance of MEC by evaluating three variants on PNN \cite{PNN}: (1) w/o cons., which excludes contrastive learning; (2) w/o reg., which excludes popularity-weighted regularization; (3) basic PQ, which employs standard Product Quantization (PQ) \cite{PQ}; and (4) freq. PQ, which integrates frequency information before applying PQ quantization.
        %     The results, presented in Table \ref{tab:Ablation}, are based on experiments conducted with the Criteo and Avazu datasets. Several key insights can be derived from these findings. 
            
        %     Direct application of PQ for quantization results in a notable performance drop. 
        %     While PQ effectively alleviates memory constraints by directly quantizing the feature embedding table, the suboptimal distribution of codes during quantization diminishes the uniqueness of features, thereby weakening the expressive power of embeddings and subsequently degrading CTR prediction performance. 
            
        %     Furthermore, incorporating frequency information leads to an even greater performance decline. 
        %     The results for freq. PQ demonstrates that directly adding frequency information significantly harms performance due to the highly imbalanced nature of data distribution. 
        %     The quantization centers become biased towards high-frequency features, resulting in poor representation of low-frequency features and a marked decrease in CTR prediction accuracy. 
            
        %     On the other hand, the use of regularization significantly improves quantization outcomes, maintaining performance levels close to the base model. 
        %     Regularization mitigates the dominance of high-frequency features in the quantization centers, ensuring that low-frequency features do not share quantized representations with high-frequency features. 
        %     This enhancement improves the model's ability to represent low-frequency features, counteracting the negative impact caused by frequency-based enhancements. 
        %     Contrastive learning enhances quantization by increasing separation between representations, ensuring balanced code distribution and performance. The combination of regularization and contrastive learning achieves optimal results by balancing high- and low-frequency features and homogenizing quantized representations.


\subsubsection{Analysis of Hyper-Parameter.}
As shown in Fig. \ref{fig:line_chart}, we analyzed key hyper-parameters in our MEC framework: the regularization loss coefficient ($\alpha$), contrastive loss coefficient ($\beta$), embedding dimension ($d$), and number of embedding layers ($m$). 
Optimal values were $\alpha = 0.001$, $\beta = 0.01$, $d = 2048$, and $m = 4$. Deviations from these values caused issues like underfitting, overfitting, information loss, or suboptimal relationship modeling.


\subsubsection{\textbf{Ablation Study of MEC}}
\begin{table*}
  [t]
  \centering
  \resizebox{\textwidth}{!}{%
  \begin{tabular}{cccccccccccc}
    \toprule \multicolumn{2}{c}{Components}                                                             & \multicolumn{5}{c}{Re-executability Rate (\%)} & \multicolumn{5}{c}{Readability (\#)} \\
    \cmidrule(lr){1-2} \cmidrule(lr){3-7} \cmidrule(lr){8-12}        \hspace{8pt}\labelemoji\hspace{8pt}                                                                & \hspace{8pt}\toolemoji\hspace{8pt}                                      & O0                                 & O1             & O2             & O3             & AVG            & O0             & O1             & O2             & O3             & AVG            \\
    \hline
    \rowcolor[rgb]{0.93,0.93,0.93}\multicolumn{12}{c}{\textbf{Initialize with LLM4Decompile-End-6.7B~\citep{llm4decompile}}}   \\
    \xmark                                                                                              & \xmark                                    & 69.51                              & 46.95          & 50.61          & 46.34          & 53.35          & 3.98 & 3.41 & 3.44 & 3.38 & 3.55 \\
    \cmark                                                                                              & \xmark                                    & 75.61                              & 50.61          & 50.00          & 50.00          & 56.55          & 4.01 & 3.44 & 3.39 & \textbf{3.49} & 3.58 \\
    \xmark                                                                                              & \cmark                                    & 83.54                     & \textbf{56.10}          & 51.22          & 50.61 & 60.37 & 4.05 & 3.51 & 3.51 & 3.42 & 3.62 \\
    \cmark                                                                                              & \cmark                                    & \textbf{85.37}                            & \textbf{56.10}                     & \textbf{51.83} & \textbf{52.43}          & \textbf{61.43} & \textbf{4.13} & \textbf{3.60} & \textbf{3.54} & \textbf{3.49} & \textbf{3.69} \\

    \rowcolor[rgb]{0.93,0.93,0.93}\multicolumn{12}{c}{\textbf{Initialize with Deepseek-Coder-6.7B-base~\citep{deepseekcoder}}} \\
    \xmark                                                                                              & \xmark                                    & 59.15                              & 35.98          & 39.02          & 37.80          & 42.99          & 3.71 & 3.05 & 3.16 & 3.05 & 3.24 \\
    \cmark                                                                                              & \xmark                                    & 66.46                              & 41.46          & 38.41          & 36.59          & 45.73          & 3.76 & 3.17 & \textbf{3.21} & 3.08 & 3.31 \\
    \xmark                                                                                              & \cmark                                    & 70.73                              & 39.63          & 39.02          & 40.24          & 47.41          & 3.90 & 3.17 & 3.08 & 3.11 & 3.31 \\
    \cmark                                                                                              & \cmark                                    & \textbf{79.88}                     & \textbf{45.73} & \textbf{43.90} & \textbf{42.68} & \textbf{53.05} & \textbf{3.96} & \textbf{3.21} & 3.18 & \textbf{3.19} & \textbf{3.38} \\
    \bottomrule
  \end{tabular}%
  }
  \caption{The ablation study of different methods across four optimization levels
  (O0, O1, O2, O3), as well as their average scores (AVG). The results in bold represent the optimal performance. The ~\labelemoji~ and ~\toolemoji~ means Relabedling and Function Call. \textbf{Bold} denotes the best performance.}
  \label{tab:ablation}
\end{table*}

To thoroughly assess the contributions of different components in MEC, we evaluated three distinct variants on PNN \cite{PNN}: (1) without contrastive learning (w/o cons.); (2) without popularity-weighted regularization (w/o reg.); (3) using basic Product Quantization (PQ); and (4) frequency-based PQ (freq. PQ). The results, as shown in Table \ref{tab:Ablation}, were obtained using the Criteo and Avazu datasets.


% Applying PQ directly causes a significant performance drop. Although PQ reduces memory usage by quantizing the embedding table, it leads to suboptimal feature representation, weakening model expressiveness and CTR prediction accuracy.
% Including frequency information further decreases performance due to data imbalance. High-frequency features dominate the quantization centers, poorly representing low-frequency features and reducing accuracy.
% Regularization improves outcomes by preventing high-frequency features from dominating quantization, thus better representing low-frequency features. This counters the negative effects of frequency imbalance.
% Contrastive learning enhances separation between representations, promoting balanced code distribution. The combination of regularization and contrastive learning provides the best results by effectively balancing feature frequencies and improving quantized representation quality.

Directly applying PQ reduces performance by weakening feature representation and CTR prediction accuracy. Simply including frequency information exacerbates this by allowing high-frequency features to dominate, poorly representing low-frequency features. Regularization mitigates this imbalance by preventing domination and better representing low-frequency features. Contrastive learning enhances representation separation and code distribution. Together, regularization and contrastive learning balance feature frequencies and improve quantized representation quality.


\begin{table}[t]
\centering
\setlength{\abovecaptionskip}{0.05cm}
\setlength{\belowcaptionskip}{0.2cm}
\caption{Latency of training stage}
\setlength{\tabcolsep}{3mm}{
\resizebox{0.45\textwidth}{!}{
\begin{tabular}{c|c|c}
\toprule
    \textbf{Metric} & \textbf{Criteo} & \textbf{Avazu} \\   \midrule\midrule
    \textbf{w/o PQ} & \textbf{24.07s} & \textbf{13.96s} \\
    \textbf{basic PQ} & 24.92s & 14.67s \\
    \textbf{w/o cons.} & 24.95s & 14.68s \\
    \textbf{w/o reg.} & 25.50s & 15.29s \\
    \textbf{$\text{MEC}_{PNN}$} & 25.56s & 15.31s \\
\bottomrule
\end{tabular}
}}
\label{tab:Latency}
% \vspace{-10pt}
\end{table}

\subsubsection{Analysis of Training Latency}
\label{subsub:latency}
In this section, we analyze the training latency, given the computational constraints in real-world scenarios where training time needs careful consideration. 
Table \ref{tab:Latency} presents the results based on the Criteo and Avazu datasets, averaged over 10 runs. The PQ component adds negligible training time compared to the main CTR model. Popularity-weighted regularization does not increase time complexity, and while contrastive learning slightly increases training time, it remains manageable. Overall, our approach improves quantized embedding quality without significantly impacting training latency.


        % \subsubsection{\textbf{Analysis of Quantization Methods.}}
        %     \begin{table}[t]
\centering
\setlength{\abovecaptionskip}{0.05cm}
\setlength{\belowcaptionskip}{0.2cm}
\caption{Comparison of quantization methods}
\setlength{\tabcolsep}{3mm}{
\resizebox{0.7\textwidth}{!}{
\begin{tabular}{c|c|c|c|c}
\toprule
 &
    \multicolumn{2}{c|}{\multirow{-1}{*}{\textbf{Criteo}}} &
    \multicolumn{2}{c}{\multirow{-1}{*}{\textbf{Avazu}}} \\ \midrule
    \multicolumn{1}{c|}{\textbf{Pretrain model}} &
    \multicolumn{1}{c|}{\textbf{AUC}} &
    \multicolumn{1}{c|}{\textbf{Logloss}} &
    \multicolumn{1}{c|}{\textbf{AUC}} &
    \multicolumn{1}{c}{\textbf{Logloss}} \\   \midrule\midrule
    \textbf{AQ+GDCN} & 0.8025 & 0.4507 & 0.7407 & 0.3871 \\
    \textbf{RQ+GDCN} & 0.8094 & 0.4421 & 0.7513 & 0.3752 \\
    \textbf{PQ+GDCN} & \textbf{0.8102} & \textbf{0.4415} & \textbf{0.7538} & \textbf{0.3728} \\  \midrule
    \textbf{AQ+PNN} & 0.8038 & 0.4480 & 0.7451 & 0.3813 \\
    \textbf{RQ+PNN} & 0.8099 & 0.4418 & 0.7528 & 0.3749 \\
    \textbf{PQ+PNN} & \textbf{0.8105} & \textbf{0.4412} & \textbf{0.7556} & \textbf{0.3737} \\
\bottomrule

\end{tabular}
}}
\label{tab:rq_result}
% \vspace{-14pt}
\end{table}
        %     In this study, we evaluated various quantization methods for our MEC framework to balance memory efficiency and model performance. We compared Product Quantization (PQ) \cite{PQ} with Additive Quantization (AQ) \cite{AQ} and Residual Quantization (RQ) \cite{RQVAE}. Our experiments showed that PQ was the most effective among the three quantization methods.
            
        %     As shown in Table \ref{tab:rq_result}, RQ's performance lagged behind PQ due to high correlation among generated codes, causing instability in contrastive learning and unreliable negative samples. PQ, however, manages high-dimensional embedding tables by generating multiple sub-codebooks, preserving expressive power while reducing memory usage. The independence of these sub-codebooks ensures stability in contrastive learning, making PQ a robust choice for MEC.
        %     Similarly, AQ performed less effectively than PQ. AQ's additive process introduces redundancy and noise, reducing memory efficiency and stability. PQ avoids these issues with independent sub-codebooks, minimizing redundancy and optimizing memory use, thus enhancing contrastive learning stability.


\subsubsection{\textbf{Analysis of Quantization Methods}}

\begin{table}[t]
\centering
\setlength{\abovecaptionskip}{0.05cm}
\setlength{\belowcaptionskip}{0.2cm}
\caption{Comparison of quantization methods}
\setlength{\tabcolsep}{3mm}{
\resizebox{0.7\textwidth}{!}{
\begin{tabular}{c|c|c|c|c}
\toprule
 &
    \multicolumn{2}{c|}{\multirow{-1}{*}{\textbf{Criteo}}} &
    \multicolumn{2}{c}{\multirow{-1}{*}{\textbf{Avazu}}} \\ \midrule
    \multicolumn{1}{c|}{\textbf{Pretrain model}} &
    \multicolumn{1}{c|}{\textbf{AUC}} &
    \multicolumn{1}{c|}{\textbf{Logloss}} &
    \multicolumn{1}{c|}{\textbf{AUC}} &
    \multicolumn{1}{c}{\textbf{Logloss}} \\   \midrule\midrule
    \textbf{AQ+GDCN} & 0.8025 & 0.4507 & 0.7407 & 0.3871 \\
    \textbf{RQ+GDCN} & 0.8094 & 0.4421 & 0.7513 & 0.3752 \\
    \textbf{PQ+GDCN} & \textbf{0.8102} & \textbf{0.4415} & \textbf{0.7538} & \textbf{0.3728} \\  \midrule
    \textbf{AQ+PNN} & 0.8038 & 0.4480 & 0.7451 & 0.3813 \\
    \textbf{RQ+PNN} & 0.8099 & 0.4418 & 0.7528 & 0.3749 \\
    \textbf{PQ+PNN} & \textbf{0.8105} & \textbf{0.4412} & \textbf{0.7556} & \textbf{0.3737} \\
\bottomrule

\end{tabular}
}}
\label{tab:rq_result}
% \vspace{-14pt}
\end{table}

We evaluated quantization methods for MEC, comparing Product Quantization (PQ) with Additive Quantization (AQ) and Residual Quantization (RQ). Table \ref{tab:rq_result} shows that PQ outperforms the others. RQ suffers from code correlation, destabilizing contrastive learning, while AQ introduces redundancy and noise. PQ's independent sub-codebooks maintain stability and memory efficiency, making it the best choice for MEC.

\subsubsection{\textbf{Analysis of Pre-train Models}}
\begin{table}[t]
\centering
\setlength{\abovecaptionskip}{0.05cm}
\setlength{\belowcaptionskip}{0.2cm}
\caption{Performance on multiple pretrain models}
\setlength{\tabcolsep}{3mm}{
\resizebox{0.72\textwidth}{!}{
\begin{tabular}{c|c|c|c|c}
\toprule
 &
    \multicolumn{2}{c|}{\multirow{-1}{*}{\textbf{Criteo}}} &
    \multicolumn{2}{c}{\multirow{-1}{*}{\textbf{Avazu}}} \\ \midrule
    \multicolumn{1}{c|}{\textbf{Pretrain model}} &
    \multicolumn{1}{c|}{\textbf{AUC}} &
    \multicolumn{1}{c|}{\textbf{Logloss}} &
    \multicolumn{1}{c|}{\textbf{AUC}} &
    \multicolumn{1}{c}{\textbf{Logloss}} \\   \midrule\midrule
    \textbf{FM} & 0.8104 & 0.4413 & 0.7551 & 0.3741 \\
    \textbf{DeepFM} & \textbf{0.8105} & \textbf{0.4412} & \textbf{0.7556} & 0.3737 \\
    \textbf{DCNv2} & 0.8104 & 0.4412 & 0.7542 & \textbf{0.3736} \\
\bottomrule
\end{tabular}
}}
\label{tab:pretrain_model_analysis}
% \vspace{-13pt}
\end{table}

We evaluated the impact of different pre-trained models (FM \cite{FM}, DeepFM \cite{DeepFM}, DCNv2 \cite{DCNv2}) on our framework for embedding generation and quantization, using PNN for downstream CTR prediction. Table \ref{tab:pretrain_model_analysis} shows minor performance variations across models, highlighting our framework's robust generalizability. This enables the use of simpler models in industrial applications without significant performance loss, allowing efficient deployment in various scenarios.




        % \subsubsection{\textbf{Additional Analyses}}
        %     We also conducted further analysis, including hyperparameter analysis in Appendix \ref{app:hyper} and analysis of pre-train models in Appendix \ref{app:pretrain_models}.To provide readers with an intuitive understanding of the uniformity in quantized embedding distributions, we present a visualization of code distribution in Appendix ~\ref{app:visual}. Additionally, the analysis of training latency is in Appendix \ref{app:latency}, providing deeper insights into the framework's time efficiency.

        




