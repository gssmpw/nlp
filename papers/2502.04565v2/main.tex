\documentclass{article}

% Language setting
% Replace `english' with e.g. `spanish' to change the document language
\usepackage[english]{babel}

% Set page size and margins
% Replace `letterpaper' with`a4paper' for UK/EU standard size
\usepackage[letterpaper,top=2cm,bottom=2cm,left=3cm,right=3cm,marginparwidth=1.75cm]{geometry}

% Useful packages
\usepackage{amsmath}
\usepackage{graphicx}
\usepackage[colorlinks=true, allcolors=blue]{hyperref}

\title{Private Federated Learning In Real World Application -- A Case Study}
\author{An Ji, Bortik Bandyopadhyay, Congzheng Song\\
Natarajan Krishnaswami, Prabal Vashisht, Rigel Smiroldo, Isabel Litton\\
Sayantan Mahinder, Mona Chitnis, Andrew W Hill\\
Apple}

\begin{document}
\maketitle

\begin{abstract}
This paper presents an implementation of machine learning model training using private federated learning (PFL) on edge devices. We introduce a novel framework that uses PFL to address the challenge of training a model using users' private data. The framework ensures that user data remain on individual devices, with only essential model updates transmitted to a central server for aggregation with privacy guarantees. We detail the architecture of our app selection model, which incorporates a neural network with attention mechanisms and ambiguity handling through uncertainty management. Experiments conducted through off-line simulations and on device training demonstrate the feasibility of our approach in real-world scenarios. Our results show the potential of PFL to improve the accuracy of an app selection model by adapting to changes in user behavior over time, while adhering to privacy standards. The insights gained from this study are important for industries looking to implement PFL, offering a robust strategy for training a predictive model directly on edge devices while ensuring user data privacy. \end{abstract}

\section{Introduction}
\label{sec:introduction}
The business processes of organizations are experiencing ever-increasing complexity due to the large amount of data, high number of users, and high-tech devices involved \cite{martin2021pmopportunitieschallenges, beerepoot2023biggestbpmproblems}. This complexity may cause business processes to deviate from normal control flow due to unforeseen and disruptive anomalies \cite{adams2023proceddsriftdetection}. These control-flow anomalies manifest as unknown, skipped, and wrongly-ordered activities in the traces of event logs monitored from the execution of business processes \cite{ko2023adsystematicreview}. For the sake of clarity, let us consider an illustrative example of such anomalies. Figure \ref{FP_ANOMALIES} shows a so-called event log footprint, which captures the control flow relations of four activities of a hypothetical event log. In particular, this footprint captures the control-flow relations between activities \texttt{a}, \texttt{b}, \texttt{c} and \texttt{d}. These are the causal ($\rightarrow$) relation, concurrent ($\parallel$) relation, and other ($\#$) relations such as exclusivity or non-local dependency \cite{aalst2022pmhandbook}. In addition, on the right are six traces, of which five exhibit skipped, wrongly-ordered and unknown control-flow anomalies. For example, $\langle$\texttt{a b d}$\rangle$ has a skipped activity, which is \texttt{c}. Because of this skipped activity, the control-flow relation \texttt{b}$\,\#\,$\texttt{d} is violated, since \texttt{d} directly follows \texttt{b} in the anomalous trace.
\begin{figure}[!t]
\centering
\includegraphics[width=0.9\columnwidth]{images/FP_ANOMALIES.png}
\caption{An example event log footprint with six traces, of which five exhibit control-flow anomalies.}
\label{FP_ANOMALIES}
\end{figure}

\subsection{Control-flow anomaly detection}
Control-flow anomaly detection techniques aim to characterize the normal control flow from event logs and verify whether these deviations occur in new event logs \cite{ko2023adsystematicreview}. To develop control-flow anomaly detection techniques, \revision{process mining} has seen widespread adoption owing to process discovery and \revision{conformance checking}. On the one hand, process discovery is a set of algorithms that encode control-flow relations as a set of model elements and constraints according to a given modeling formalism \cite{aalst2022pmhandbook}; hereafter, we refer to the Petri net, a widespread modeling formalism. On the other hand, \revision{conformance checking} is an explainable set of algorithms that allows linking any deviations with the reference Petri net and providing the fitness measure, namely a measure of how much the Petri net fits the new event log \cite{aalst2022pmhandbook}. Many control-flow anomaly detection techniques based on \revision{conformance checking} (hereafter, \revision{conformance checking}-based techniques) use the fitness measure to determine whether an event log is anomalous \cite{bezerra2009pmad, bezerra2013adlogspais, myers2018icsadpm, pecchia2020applicationfailuresanalysispm}. 

The scientific literature also includes many \revision{conformance checking}-independent techniques for control-flow anomaly detection that combine specific types of trace encodings with machine/deep learning \cite{ko2023adsystematicreview, tavares2023pmtraceencoding}. Whereas these techniques are very effective, their explainability is challenging due to both the type of trace encoding employed and the machine/deep learning model used \cite{rawal2022trustworthyaiadvances,li2023explainablead}. Hence, in the following, we focus on the shortcomings of \revision{conformance checking}-based techniques to investigate whether it is possible to support the development of competitive control-flow anomaly detection techniques while maintaining the explainable nature of \revision{conformance checking}.
\begin{figure}[!t]
\centering
\includegraphics[width=\columnwidth]{images/HIGH_LEVEL_VIEW.png}
\caption{A high-level view of the proposed framework for combining \revision{process mining}-based feature extraction with dimensionality reduction for control-flow anomaly detection.}
\label{HIGH_LEVEL_VIEW}
\end{figure}

\subsection{Shortcomings of \revision{conformance checking}-based techniques}
Unfortunately, the detection effectiveness of \revision{conformance checking}-based techniques is affected by noisy data and low-quality Petri nets, which may be due to human errors in the modeling process or representational bias of process discovery algorithms \cite{bezerra2013adlogspais, pecchia2020applicationfailuresanalysispm, aalst2016pm}. Specifically, on the one hand, noisy data may introduce infrequent and deceptive control-flow relations that may result in inconsistent fitness measures, whereas, on the other hand, checking event logs against a low-quality Petri net could lead to an unreliable distribution of fitness measures. Nonetheless, such Petri nets can still be used as references to obtain insightful information for \revision{process mining}-based feature extraction, supporting the development of competitive and explainable \revision{conformance checking}-based techniques for control-flow anomaly detection despite the problems above. For example, a few works outline that token-based \revision{conformance checking} can be used for \revision{process mining}-based feature extraction to build tabular data and develop effective \revision{conformance checking}-based techniques for control-flow anomaly detection \cite{singh2022lapmsh, debenedictis2023dtadiiot}. However, to the best of our knowledge, the scientific literature lacks a structured proposal for \revision{process mining}-based feature extraction using the state-of-the-art \revision{conformance checking} variant, namely alignment-based \revision{conformance checking}.

\subsection{Contributions}
We propose a novel \revision{process mining}-based feature extraction approach with alignment-based \revision{conformance checking}. This variant aligns the deviating control flow with a reference Petri net; the resulting alignment can be inspected to extract additional statistics such as the number of times a given activity caused mismatches \cite{aalst2022pmhandbook}. We integrate this approach into a flexible and explainable framework for developing techniques for control-flow anomaly detection. The framework combines \revision{process mining}-based feature extraction and dimensionality reduction to handle high-dimensional feature sets, achieve detection effectiveness, and support explainability. Notably, in addition to our proposed \revision{process mining}-based feature extraction approach, the framework allows employing other approaches, enabling a fair comparison of multiple \revision{conformance checking}-based and \revision{conformance checking}-independent techniques for control-flow anomaly detection. Figure \ref{HIGH_LEVEL_VIEW} shows a high-level view of the framework. Business processes are monitored, and event logs obtained from the database of information systems. Subsequently, \revision{process mining}-based feature extraction is applied to these event logs and tabular data input to dimensionality reduction to identify control-flow anomalies. We apply several \revision{conformance checking}-based and \revision{conformance checking}-independent framework techniques to publicly available datasets, simulated data of a case study from railways, and real-world data of a case study from healthcare. We show that the framework techniques implementing our approach outperform the baseline \revision{conformance checking}-based techniques while maintaining the explainable nature of \revision{conformance checking}.

In summary, the contributions of this paper are as follows.
\begin{itemize}
    \item{
        A novel \revision{process mining}-based feature extraction approach to support the development of competitive and explainable \revision{conformance checking}-based techniques for control-flow anomaly detection.
    }
    \item{
        A flexible and explainable framework for developing techniques for control-flow anomaly detection using \revision{process mining}-based feature extraction and dimensionality reduction.
    }
    \item{
        Application to synthetic and real-world datasets of several \revision{conformance checking}-based and \revision{conformance checking}-independent framework techniques, evaluating their detection effectiveness and explainability.
    }
\end{itemize}

The rest of the paper is organized as follows.
\begin{itemize}
    \item Section \ref{sec:related_work} reviews the existing techniques for control-flow anomaly detection, categorizing them into \revision{conformance checking}-based and \revision{conformance checking}-independent techniques.
    \item Section \ref{sec:abccfe} provides the preliminaries of \revision{process mining} to establish the notation used throughout the paper, and delves into the details of the proposed \revision{process mining}-based feature extraction approach with alignment-based \revision{conformance checking}.
    \item Section \ref{sec:framework} describes the framework for developing \revision{conformance checking}-based and \revision{conformance checking}-independent techniques for control-flow anomaly detection that combine \revision{process mining}-based feature extraction and dimensionality reduction.
    \item Section \ref{sec:evaluation} presents the experiments conducted with multiple framework and baseline techniques using data from publicly available datasets and case studies.
    \item Section \ref{sec:conclusions} draws the conclusions and presents future work.
\end{itemize}
\section{Background}\label{sec:backgrnd}

\subsection{Cold Start Latency and Mitigation Techniques}

Traditional FaaS platforms mitigate cold starts through snapshotting, lightweight virtualization, and warm-state management. Snapshot-based methods like \textbf{REAP} and \textbf{Catalyzer} reduce initialization time by preloading or restoring container states but require significant memory and I/O resources, limiting scalability~\cite{dong_catalyzer_2020, ustiugov_benchmarking_2021}. Lightweight virtualization solutions, such as \textbf{Firecracker} microVMs, achieve fast startup times with strong isolation but depend on robust infrastructure, making them less adaptable to fluctuating workloads~\cite{agache_firecracker_2020}. Warm-state management techniques like \textbf{Faa\$T}~\cite{romero_faa_2021} and \textbf{Kraken}~\cite{vivek_kraken_2021} keep frequently invoked containers ready, balancing readiness and cost efficiency under predictable workloads but incurring overhead when demand is erratic~\cite{romero_faa_2021, vivek_kraken_2021}. While these methods perform well in resource-rich cloud environments, their resource intensity challenges applicability in edge settings.

\subsubsection{Edge FaaS Perspective}

In edge environments, cold start mitigation emphasizes lightweight designs, resource sharing, and hybrid task distribution. Lightweight execution environments like unikernels~\cite{edward_sock_2018} and \textbf{Firecracker}~\cite{agache_firecracker_2020}, as used by \textbf{TinyFaaS}~\cite{pfandzelter_tinyfaas_2020}, minimize resource usage and initialization delays but require careful orchestration to avoid resource contention. Function co-location, demonstrated by \textbf{Photons}~\cite{v_dukic_photons_2020}, reduces redundant initializations by sharing runtime resources among related functions, though this complicates isolation in multi-tenant setups~\cite{v_dukic_photons_2020}. Hybrid offloading frameworks like \textbf{GeoFaaS}~\cite{malekabbasi_geofaas_2024} balance edge-cloud workloads by offloading latency-tolerant tasks to the cloud and reserving edge resources for real-time operations, requiring reliable connectivity and efficient task management. These edge-specific strategies address cold starts effectively but introduce challenges in scalability and orchestration.

\subsection{Predictive Scaling and Caching Techniques}

Efficient resource allocation is vital for maintaining low latency and high availability in serverless platforms. Predictive scaling and caching techniques dynamically provision resources and reduce cold start latency by leveraging workload prediction and state retention.
Traditional FaaS platforms use predictive scaling and caching to optimize resources, employing techniques (OFC, FaasCache) to reduce cold starts. However, these methods rely on centralized orchestration and workload predictability, limiting their effectiveness in dynamic, resource-constrained edge environments.



\subsubsection{Edge FaaS Perspective}

Edge FaaS platforms adapt predictive scaling and caching techniques to constrain resources and heterogeneous environments. \textbf{EDGE-Cache}~\cite{kim_delay-aware_2022} uses traffic profiling to selectively retain high-priority functions, reducing memory overhead while maintaining readiness for frequent requests. Hybrid frameworks like \textbf{GeoFaaS}~\cite{malekabbasi_geofaas_2024} implement distributed caching to balance resources between edge and cloud nodes, enabling low-latency processing for critical tasks while offloading less critical workloads. Machine learning methods, such as clustering-based workload predictors~\cite{gao_machine_2020} and GRU-based models~\cite{guo_applying_2018}, enhance resource provisioning in edge systems by efficiently forecasting workload spikes. These innovations effectively address cold start challenges in edge environments, though their dependency on accurate predictions and robust orchestration poses scalability challenges.

\subsection{Decentralized Orchestration, Function Placement, and Scheduling}

Efficient orchestration in serverless platforms involves workload distribution, resource optimization, and performance assurance. While traditional FaaS platforms rely on centralized control, edge environments require decentralized and adaptive strategies to address unique challenges such as resource constraints and heterogeneous hardware.



\subsubsection{Edge FaaS Perspective}

Edge FaaS platforms adopt decentralized and adaptive orchestration frameworks to meet the demands of resource-constrained environments. Systems like \textbf{Wukong} distribute scheduling across edge nodes, enhancing data locality and scalability while reducing network latency. Lightweight frameworks such as \textbf{OpenWhisk Lite}~\cite{kravchenko_kpavelopenwhisk-light_2024} optimize resource allocation by decentralizing scheduling policies, minimizing cold starts and latency in edge setups~\cite{benjamin_wukong_2020}. Hybrid solutions like \textbf{OpenFaaS}~\cite{noauthor_openfaasfaas_2024} and \textbf{EdgeMatrix}~\cite{shen_edgematrix_2023} combine edge-cloud orchestration to balance resource utilization, retaining latency-sensitive functions at the edge while offloading non-critical workloads to the cloud. While these approaches improve flexibility, they face challenges in maintaining coordination and ensuring consistent performance across distributed nodes.


\section{Feasibility Study using Offline PFL simulations}

Before we can implement Private Federated Learning (PFL) on devices, it is essential to begin the process with comprehensive offline simulations. We use Apple's public framework \texttt{pfl-research}\footnote{https://github.com/apple/pfl-research}, which provides a platform for running offline PFL simulations with any available central data, without requiring actual user devices. This simulation environment replicates the behavior of PFL during real-time training with user devices.  Such offline simulations offer critical insights into the potential efficacy of PFL for the intended application, allowing practitioners to assess the balance between privacy protection and utility even before PFL deployments. Moreover, the use of simulations facilitates extensive hyperparameter tuning on a large scale. This capability is essential for determining the optimal settings for both learning and privacy-specific hyperparameters, ensuring that the model performs effectively while adhering to required privacy constraints. This preparatory step is pivotal for ensuring that the adoption of PFL is both practical and aligned with the specific needs and constraints of the application at hand.

The App Selection model described in Section~\ref{sec:model_desc} consists of a Deep Neural Network model, whose weights can be learned or updated using Private Federated Learning~\cite{reddi2020adaptive} . We have evaluated offline PFL simulations on the DNN part of the model in two different ways.:
\begin{itemize}
    \item \textbf{Training from scratch} : In this setting, all weights of the neural network are initialized randomly and then learned as part of the back-propagation step. Thus the model is trained from scratch, which makes this a more complex and time consuming task. It is a relatively higher data and compute hungry setup, but can be used to support breaking changes in the model architecture or adding new features to the modeling task. 
    \item \textbf{Fine-tuning from an existing checkpoint} : In this setting, we start from an already trained model checkpoint (i.e., the weights have been learned previously) and then only a subset of weights of the neural network are unfrozen i.e., updated as part of back-propagation step, while the remaining weights are frozen. This setting is particularly useful when we already have a fixed architecture, fixed set of features \& a solid baseline model to which we will simply add data to continuously update the model to adapt to distribution shift (need citation).
\end{itemize}


Unless otherwise mentioned, we have used AdamW~\cite{loshchilov2017decoupled} as the server side optimizer and SGD as the client side optimizer for PFL simulations using \texttt{pfl-research}. 
The PFL model’s performance is evaluated using the accuracy metric defined in Section~\ref{sec:eval_metrics} computed on the fixed validation set.
Note that the training and evaluation data required for each of the two above setups are different, and hence are elaborated in the subsequent sections.

\subsection{Training from scratch}
\label{sec:scratch_train}

In this setup, we want to simulate the PFL-based training from scratch with Differential Privacy (DP) enabled. 
We start the training with a randomly initialized set of model parameters and train the entire model from scratch using a relatively large offline data set consisting of $\sim$788K data points. 
As the baseline, we have trained the same architecture with the entire training data from scratch without any PFL, which we refer to as \textit{Cymba} for simplicity and ease of reading.
We have a dedicated validation set consisting of $\sim$178K data points, which we use to compare the performance of the PFL trained model against the non-PFL baseline (i.e., Cymba).

Unless otherwise specified, we have used Gaussian Moments Accountant (Needs citation) implemented in \texttt{pfl-research} as the central privacy mechanism, with parameters as : $\epsilon = 2.0$, $\delta = 1e-6$, and $\text{Clipping Bound} = 0.1$. 
Additionally, for simplicity of offline simulation, we have fixed the \textit{mean data points per user} $= 1$ and \textit{local epochs} $= 3$, after some initial hyper-parameter exploration.
In all our experiments, we have set a higher Local Learning rate (LLR) for the on-device SGD step due to sparsity of training data per device, and a lower Central Learning rate (CLR) for the server side Adam optimizer step. This configuration has generally helped us achieve a good distributed training set up using PFL, while reducing the risk of training divergence due to sub-optimal hyper-parameter choice.

We design the offline simulations in such a way that the insights from these simulations can help plan the duration of on device training too.
Note that for the app selection model, there are two very important parameters that determine the overall time and resources needed for on device training viz. the \textit{number of devices per Central iteration} and the \textit{number of Central iterations}. 
We chose two sets of values to represent the High and Low resource Budget settings to determine the performance bounds of the PFL training :
\begin{itemize}
    \item \textbf{High Resource Budget} : In this setting, we choose 10K devices per central iteration, and 500 number of central iterations. Thus we take approximately $(10\text{K} * 500)/788\text{K} = 6.3$ passes on the entire training data, which is smaller than the total number of epochs used to train the non-PFL Cymba baseline.
    \item \textbf{Low Resource Budget} : In this setting, we choose 1K devices per central iteration, and number of central iterations = 500. Thus we take approximately $(1\text{K} * 500)/788{\text{K}} = 0.63$ passes on the entire training data, which is smaller than the total number of epochs used to train the non-PFL Cymba baseline.
\end{itemize}
We observe that in the high resource budget setting, the PFL trained model is almost similar in performance compared to Cymba, with a very minor regression that is acceptable, given that Differential Privacy will have a negative impact on the model's learning capacity.
As expected, the low resource budget setting appears to have significantly more regression compared to Cymba, likely because of the significantly smaller number of passes on the training data, coupled with the impact of Differential Privacy. 
Note that in this setting, it is difficult to achieve an improvement over an existing non-PFL trained baseline using the exact same training data.
However, given the PFL trained model's performance under the High Resource Budget setting, we consider that we may be successful in training a PFL model from scratch during on device training, which may have a competitive performance (i.e., within acceptable limits of performance regression) compared to the non-PFL trained baseline.

\begin{table}
\centering
\begin{tabular}{l|l|l|r}
Model & CLR & LLR & Accuracy\\\hline
Cymba & - & - & 0.856\\
PFL with Low Resource Budget & 0.001 & 0.01 & 0.83 \\
PFL with High Resource Budget & 0.0005 & 0.01 & 0.852
\end{tabular}
\caption{Performance of Training from Scratch with v/s without PFL}
\label{tab:scratch_train} 
\end{table}


Additionally, as part of offline simulations, we did multiple rounds of hyper-parameter tuning by varying the number of devices per central iteration, number of Central iterations, Central learning rate, Local number of epochs, Local learning rate and Privacy Clipping Bound before selecting the above configuration.
Some key observations are presented below:
\begin{itemize}
    \item We fixed the Local Learning Rate to a relatively higher value (0.01) as it is a bit difficult to modify/control this parameter during on device training. However, we varied the Central Learning Rate and observed that a value between [0.1, 0.9] often causes the training to diverge, while a value of 0.0005 tend to yield consistently good results on the offline data. Additionally, we tested various learning rate scheduling strategies (like Polynomial, Cyclic etc.) for the Central Learning rate, but did not observe any major gains over a fixed Central Learning rate.
    \item We noticed that the Local Number of Epochs  $\leq3$ tends to give better results. Increasing the number of epochs any further causes training divergence, even with small values of Local Learning Rate.
    \item We varied the number of Central Iterations from 500 to 25,000 but the benefits in terms of gains gradually decreases. Hence we fix the number of Central iterations to 500 (which is much smaller than 2K), as it will reduce the on device training time significantly when compared to the time taken for 2K iterations, given that the difference in performance is very small.
    \item Assuming one data sample per device, we varied the number of Devices between 1K and 150K. The general observation is that 5K to 10K devices tended to yield good results. If we choose 5K for on device training, then it will speed up the training time as well as reduce the network communication load with the backend PFL server.
\end{itemize}




\subsection{Fine-tuning from an existing checkpoint}
\label{sec:finetune_chkp}

In this setup, we start from an already trained model checkpoint (i.e., the weights have been learned previously using a different data set) and then train a subset of weights of the neural network using random sampled data. 
In this case, Cymba is the base model from which we start the training, and use a random sampled batch of $\sim$814K data points as training data, and another random sampled batch of $\sim$176K data points as Validation set, for performance comparison.
We test two variations to establish the performance bounds: a)~training all layers of the pre-existing model and b)~training only the top layer of the pre-existing model, while freezing the remaining layers.
We explore these 2 variants with the goal to reduce the number of PFL trained parameters in the model, as it is easier to train a model with fewer parameters since it reduces the on-device training complexity, lowers the network communication cost between the device and server as well as helps with Privacy.

The Accuracy metric of the PFL fine-tuned models are presented in the Y-axis in Figure~\ref{fig:fine_tune_cymba}, while the X-axis refers to the number of PFL Central Iterations for each of the different training config.
The performance of the Cymba model on this random sampled Validation set is also reported for tracking purposes.
Unless otherwise specified, we have used Gaussian Moments Accountant (Needs citation) implemented in PFL-Research(needs citation) as the central privacy mechanism, with parameters as : $Epsilon = 2.0$, $Delta = 1e-6$ and $Clipping Bound = 0.1$. 
For fine-tuning from existing checkpoint, we have set the \textit{mean data points per user} $= 1$, \textit{Local Number of Epochs} $= 1$, \textit{number of devices per Central iteration} $= 5000$ and the \textit{number of Central iterations} $= 500$. 
For fair comparison, we have trained a dedicated model from scratch (labeled as \textit{Train from scratch} in Figure~\ref{fig:fine_tune_cymba}) using the random sampled $\sim$814K training data points, while reducing the \textit{number of devices per Central iteration} $= 5000$.
We have done a hyper-parameter search for the learning rate and chose the best configuration for each setting when reporting the above metrics.


\begin{figure}
\centering
\includegraphics[width=\textwidth]{figures/PFL_Paper_Figure_2.png}
\caption{Fine-tuning Cymba from an existing checkpoint}
\label{fig:fine_tune_cymba}
\end{figure}


In Figure~\ref{fig:fine_tune_cymba}, we observe that fine-tuning with freshly random sampled training data improves the performance of the PFL trained models on Validation set, compared to the static Cymba baseline. 
This indicates that the app selection model can adjust to the user's behavioral shift over time (as represented within the random sampled Training and Validation sets) using PFL, which is very useful for continuous model maintenance/upgrade activity.
Note that fine-tuning only the top layer of Cymba appears to be performing better than fine-tuning all layers,which reduces the amount of communication bandwidth spent to transfer the PFL weights from device to the servers, thereby making PFL less network bandwidth-intensive operation. Also fewer learnable parameters is better for DP thereby making this training paradigm a more suitable one for this use-case.
Finally, using PFL to train a model from scratch using random sampled data still under-performs all the settings, but the gap in performance is smaller if hyper-parameter tuning can be done appropriately.



\begin{figure}
\centering
\includegraphics[width=\textwidth]{figures/PFL_Paper_Figure_3.png}
\caption{Using simulations to predict the performance of the fine-tuning of Top Layers of the Cymba model as a function of Number of devices (cohort size) and corresponding total number of data points.}
\label{fig:data_req_for_tuning}
\end{figure}

In Figure~\ref{fig:data_req_for_tuning}, we plot the performance of fine-tuning the Top Layers of the Cymba model by varying the \textit{number of devices per Central iteration} as (1K, 2K, 5K). Assuming \textit{mean data points per user = 1}, in the X-axis we plot the number of data points that have been used for fine-tuning the top layers of Cymba for the corresponding Cohort size. The Y-axis shows the relative difference in Accuracy of the corresponding model checkpoint compared to the performance of the Cymba model on a fixed Validation set. This plot gives us an approximate insight into the amount of data points required to obtain a certain percentage of relative accuracy improvement, which will help us to select the corresponding parameters and the duration of the on device training. We observed that with a Cohort size of 2K while we achieve an improvement with fewer data points, model performance plateaus quickly, even though the performance can be further improved with more data points using a Cohort size of 5K. For example, with 504K data points, we can achieve close to 2\% relative improvement in Accuracy with a Cohort size of 5K, while the relative improvement is lesser with a Cohort size of 2K. This shows the importance of selecting the Cohort size appropriately in conjunction with the duration of the on device training (i.e., total number of data points to use for fine-tuning) to achieve best improvement.


%\subsection{Ablation study}
%\textbf{Training data retention period estimation experiments}


\section{PFL on device training}

\subsection{On device training data}

Training data are generated during inference time through user's explicit feedback.  These data are stored on device. Each training record includes feature values, ground truth labels and metadata. On device data storage system provides a mechanism whereby a particular task can filter out records which satisfy certain matching criteria specified by PFL server. For example, it is possible to match on device OS versions, or target a specific set of data produce by a specific on-device asset through this mechanism.  

\subsection{Federate statistics}

We use Federated Statistics, \cite{CorriganGibbs2017PrioPR}, which is Apple’s end-to-end platform for learning histogram queries from sensitive data on-device, to run histogram queries from on-device data to gain training data insights, such as how much data are available to participate in the PFL training. Before we launch PFL training, as part of the feasibility study, using the FedStats query, we found that ABSOLUTE NUMBER of devices have at least 1 valid sample to participate in PFL training thereby satisfying our data requirements. This shows us that we can complete PFL training iterations with reasonable latency and achieve model convergence. 

\subsection{On device plug-in design}

To enable real devices to process local data and contribute to a Personalized Federated Learning (PFL) task, an on-device plugin was developed. The primary objectives of the plugin are to process local data stored on the device using parameters defined in the PFL task description and attachments, compute a model update or generate training statistics and metrics, and then send these results to a central server for aggregation.
The plugin also includes an on-device differential privacy component. It ensures user privacy and security by applying differential privacy (DP) techniques and encrypting the model updates or training statistics before transmission, protecting sensitive data throughout the process.

The training workflow shown in Figure~\ref{fig 4} involves several key components to ensure on-device data processing and training while maintaining privacy and security:

\begin{enumerate}
    \item \textbf{Inference Framework and On-device Data Store:} The inference framework collects training data and stores it in the On-device Data Store, which is essential for PFL tasks.
    \item \textbf{Data Utilization by FedStats Server:} The FedStats Server utilizes the data stored on the device to aggregate statistics or provide insights into data distribution without directly accessing raw data.
    \item \textbf{PFL Plugin and On-device Orchestration:} The PFL plugin processes local data using task descriptions and parameters provided by the On device orchestration, which communicates with the PFL Server to receive these descriptions and attachments. This runs in a secure and isolated environment as sandboxed process.
    \item \textbf{Differential Privacy and Encryption:} After processing the data and computing model updates or training statistics, these outputs are passed through the Differential Privacy component. It applies DP techniques to anonymize and secure the data, adding noise to protect user privacy. The aggregated and encrypted updates are then sent back to the PFL Server.
    \item \textbf{Model updates at PFL Server:} The PFL Server aggregates updates from multiple devices, contributing to overall model training without compromising individual data privacy.
\end{enumerate}

\begin{figure}
\centering
\includegraphics[width=\textwidth]{figures/live_training_diagram.png}
\caption{PFL on device training workflow}
\label{fig 4}
\end{figure}

\subsection{On device evaluation}

To process local records, the plug-in will use the task sent from the server to devices, as well as model files and any additional files required by the plug-in. For example, in our case, app selection model can be trained on-device and the difference between the model parameter values before and after on-device training represent the result to be aggregated on a server for PFL training. 

We use custom-built tool to configure the necessary parameters for training a model or computing statistics on-device for a particular PFL task. Metrics will be computed at this stage in the plug-in, which will also be sent to a server for aggregation. When training ML models using PFL, metrics include training loss and training/evaluation accuracy. We also compute additional user facing metrics in Section 2.2. Metrics will be sent in the metadata field to the server from devices, along with encrypted results. 

\subsection{PFL training results}

We conducted several training cycles on different sizes of traffic. The results are summarized in Table~\ref{Result},

\begin{table}
\centering
\begin{tabular}{c|cc}
Model&CDER&Disambiguation Rate\\\hline
Baseline&89.18\%&1.99\%\\
PFL trained model&89.86\%&1.99\%\end{tabular}
\caption{\label{Result}Model Evaluation Results}
\end{table}
The PFL trained model showed about 0.6\% of absolute gain in CDER while keeping the same Disambiguation rate over our baseline model. The model’s gain is mainly due to users' change in behavior over time. The old (baseline) model trained on older server side data has drifted away from more recent data. The PFL model was trained on more recent data which captured this distribution change in user behavior.

We have also A/B tested this PFL trained model. Our A/B experiment was conducted for 2 weeks on about 15M devices. The PFL trained model achieved a 0.07\% gain in the top-line metric of system task completion rate and a 15.6\% decrease in the Disambiguation rate. This indicates our PFL trained model improved user experience by correctly predicting users' intended apps.




We present RiskHarvester, a risk-based tool to compute a security risk score based on the value of the asset and ease of attack on a database. We calculated the value of asset by identifying the sensitive data categories present in a database from the database keywords. We utilized data flow analysis, SQL, and Object Relational Mapper (ORM) parsing to identify the database keywords. To calculate the ease of attack, we utilized passive network analysis to retrieve the database host information. To evaluate RiskHarvester, we curated RiskBench, a benchmark of 1,791 database secret-asset pairs with sensitive data categories and host information manually retrieved from 188 GitHub repositories. RiskHarvester demonstrates precision of (95\%) and recall (90\%) in detecting database keywords for the value of asset and precision of (96\%) and recall (94\%) in detecting valid hosts for ease of attack. Finally, we conducted an online survey to understand whether developers prioritize secret removal based on security risk score. We found that 86\% of the developers prioritized the secrets for removal with descending security risk scores.








\bibliographystyle{abbrv}
\bibliography{main}
    

\end{document}