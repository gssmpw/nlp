\documentclass{article}

% Language setting
% Replace `english' with e.g. `spanish' to change the document language
\usepackage[english]{babel}

% Set page size and margins
% Replace `letterpaper' with`a4paper' for UK/EU standard size
\usepackage[letterpaper,top=2cm,bottom=2cm,left=3cm,right=3cm,marginparwidth=1.75cm]{geometry}

% Useful packages
\usepackage{amsmath}
\usepackage{graphicx}
\usepackage[colorlinks=true, allcolors=blue]{hyperref}

\title{Private Federated Learning In Real World Application -- A Case Study}
\author{An Ji, Bortik Bandyopadhyay, Congzheng Song\\
Natarajan Krishnaswami, Prabal Vashisht, Rigel Smiroldo, Isabel Litton\\
Sayantan Mahinder, Mona Chitnis, Andrew W Hill\\
Apple}

\begin{document}
\maketitle

\begin{abstract}
This paper presents an implementation of machine learning model training using private federated learning (PFL) on edge devices. We introduce a novel framework that uses PFL to address the challenge of training a model using users' private data. The framework ensures that user data remain on individual devices, with only essential model updates transmitted to a central server for aggregation with privacy guarantees. We detail the architecture of our app selection model, which incorporates a neural network with attention mechanisms and ambiguity handling through uncertainty management. Experiments conducted through off-line simulations and on device training demonstrate the feasibility of our approach in real-world scenarios. Our results show the potential of PFL to improve the accuracy of an app selection model by adapting to changes in user behavior over time, while adhering to privacy standards. The insights gained from this study are important for industries looking to implement PFL, offering a robust strategy for training a predictive model directly on edge devices while ensuring user data privacy. \end{abstract}

\documentclass[../main.tex]{subfiles}
\graphicspath{{../images/}}
\makeatletter
\def\input@path{{../images/}}
\makeatother
\begin{document}
\section{Introduction}
\begin{figure}
\centering
\begin{tikzpicture}
\node[inner sep=0pt] (ws) at (0, 0) {
\includegraphics[height=.4\textwidth, trim={10cm 0 10cm 0},clip]{world_space.png}};
\node[inner sep=0pt] (cs) at (6,0) {\includegraphics[height=.4\textwidth, trim={10cm 1cm 10cm 4cm},clip]{conf_space.png}};
\end{tikzpicture}
\vspace{-5pt}
\label{fig:pbrm_intro}
\caption{\textbf{Left}: Shows world space obstacles as grey spheres. Robots start and goal configuration is colored red and green, respectively. Configurations along the computed path are colored transparent blue. \textbf{Right:} Mapped world space scenario to configuration space. Obstacle region is the grey mesh. Red spheres are collision-free regions computed by the neural SCDF. The optimized shortest path in the convex corridor is the blue curve.}
\vspace{-25pt}
\end{figure}
Motion planning is the problem of finding a collision-free trajectory that connects a given start and goal configuration. The planning takes place in the configuration space of the robot. For single body robots, like mobile robots or drones, the configuration space and the world space are usually the same. This simplifies the planning, since explicit obstacle representations are available which enables geometrical tools like separating hyperplanes, smallest distance to obstacles etc., to be used when designing motion planning algorithms. For multi-body robots like manipulators, the situation is completely different. The world space obstacles are usually mapped to non-convex regions, and to make the problem even harder, the mapping is usually not known. Forming explicit representations of the obstacle region in the configuration space is usually too expensive or intractable. Despite all of this, sampling based planners are used with great success, which mainly is due to their use of implicit representations of the obstacle region. The basic idea is to construct a graph in the configuration space that covers and connects the collision-free region. From this graph, a path can be extracted that connects a given start and goal configuration. The approach is computationally expensive, since the graph is constructed with the smallest geometrical building block available, points, which represents a collision-check. Furthermore, the extracted paths from the graph are non-smooth and jagged due to the stochastic nature of the approach. This adds an additional post-processing step to the process, where the paths are shortcutted and smoothened, before the path can be used for tracking. Clearly a lot of time is invested to form this graph and produce smooth paths. Thus, if the obstacles start to move, then all of this work is done in no use, since all points that make up this graph need to be re-verified, which is simply too time consuming to be done in real time.
\\\\
In this work, we want to address the existing drawbacks of the sampling based planners. Our main contribution is an improved motion planner where each vertex in the graph covers a collision-free region in the form of a sphere instead of a point and where the edges are formed with neighboring intersecting spheres. This representation has the advantage of instead of returning piecewise linear paths, returning a sequence of overlapping spheres, i.e. a convex corridor, that connects a given start and goal configuration, illustrated in Figure \ref{fig:pbrm_intro}. This convex corridor allows us to use convex optimization to produce smooth trajectories, instead of computationally expensive post-processing methods. The representation further allows us to estimate the coverage of the collision-free space, which gives us awareness and feedback in the offline roadmap construction phase. Finally, our representation is simple to adapt to moving obstacles, simply requery for the new radii and recheck for intersections. 
\\\\
The spherical collision-free regions are formed using a signed distance function (SDF), which is a function that returns the smallest distance from an arbitrary point to the boundary of an obstacle. As the name implies, the distance is signed, thus if the point is inside the obstacle it is negative otherwise positive. If the distance is positive, a sphere with radius equal to the distance is guaranteed to cover a collision-free region. Using an SDF in motion planning is not new, but what is novel about our approach is that we express the distance in the configuration space instead of the world space and by doing so allows us to form these convex collision-free regions. We refer to the resulting SDF as a signed configuration distance function (SCDF). Computing an SCDF analytically is non-trivial, our approach is therefore to parameterize the SCDF with a deep neural network and learn the mapping by supervised learning. Our resulting neural SCDF can compute distances for different parameter values of obstacle shapes and we also show how multiple distances can be combined, thus making our approach flexible.
\section{Related work}
Motion planning algorithms can roughly be divided into three families, grid-based, sampling based and optimization based methods. Grid-based methods (GBM) discretize the planning space from which a graph is then compiled. A standard search method is A$^\star$ \citep{a_star}, which is classified as an \textit{informed} search method, since it employs a heuristic function to speed up the search. A$^\star$ guarantees to return an optimal path at the level of discretization used. GBMs usually discretize the planning space by a regular lattice and this limits the GBMs to problems with low dimensionality due to the curse of dimensionality. Thus, GBMs are usually limited to single-body robots where the degrees of freedom (DOF) are low. To overcome the inherent scaling problem with the GBMs, stochastic methods are usually used for multi-body robots. These methods are termed as sampling-based methods (SBM) and core members within this family are the rapidly-exploring random trees (RRT) \citep{rrt} and the probabilistic roadmap (PRM) \citep{prm}. RRT grows a tree from the start configuration and explores the collision-free region in a rapid way until it is able to connect to the goal region. RRT is usually improved by bi-directional planning \citep{rrt_connect}, i.e. an additional tree is grown from the goal configuration and the trees are tested for connection after any tree has been expanded. RRT is a single-query method, thus it searches for a path from scratch each time it is queried. Contrary to this, PRM is a multi-query method, which solves for multiple queries without starting from scratch. PRM does this by creating a roadmap (graph) that covers the collision-free space as an offline step. The graph is then used to solve for multiple queries. PRMs are used in cases where the environment does not change since the extra offline step is too computationally costly and needs to be re-done if the environment is changed. In our work, we address this inherent issue by using a different roadmap representation. Our vertices in the graph cover a collision-free region in the form of spheres and we form the edges by checking for intersecting spheres. If something in the environment changes, we recompute the spheres radii and recheck the intersections, without relying on collision detection. We use a trained neural network to compute the sphere radius, therefore querying for the radius can be done fast, hence our representation enables the PRM for dynamic environments.
\\\\
In the recent decades, optimization based methods (OBM) \citep{chomp, schulman, itomp, stomp} have been introduced as an alternative to SBM for multi-body robots. Like the SBM, the OBMs scale well to higher dimensional problems and produce smoother motion. It is common to use a SDF in the optimization since it is a smooth function, thus enabling gradient-based methods. However, the standard way of expressing the SDF is in world space. The distance therefore needs to be mapped to the configuration space by the forward kinematics. This mapping makes the optimization problem a non-linear program (NLP), which is computationally expensive to solve. Recently, a different approach has been proposed. In \cite{mp_gcs} motion planning is formulated as a convex optimization problem by using the graph of convex sets framework \citep{gcs}. The underlying idea is to decompose the collision-free space into intersecting convex sets from which a convex optimization problem is formulated. In cases where an explicit representation of the obstacles in the configuration space exists, like for single-body robots, creating collision-free convex regions can be done fast \citep{iris}. For multi-body robots, this is non-trivial. Existing work does this successfully \citep{iris_nlp, iris_c} by an optimization based approach, but the methods are still too time consuming to be used in the presence of moving obstacles. Our approach is instead to use deep learning to learn an SDF expressed in the configuration space. With this, we can query for shortest distances to the collision boundary, which allows us to expand spherical regions which are collision-free. Our approach is fast and therefore enables our suggested roadmap planner to be used in dynamic environments.
\\\\
Recent research has focused on learning collision detection \citep{fk_kernel_distance, diffco, graphdistnet} by predicting the signed distance between the robot links and the surrounding obstacles in the world space. The learned SDF is used in trajectory optimization but since the distance is expressed in the world space, the problem becomes an NLP and therefore takes a long time to solve. We take a novel approach and suggest to instead express the signed distance in the configuration space. This allows us to improve the PRM at the same time as it enables convex optimization for trajectory optimization, which runs faster and is more reliable than NLP solvers. In \cite{cspf} a learned signed distance function in the configuration space is proposed similar to our approach. However, their approach is restricted to point cloud representations, while we propose to represent the obstacles as parameterized geometric shapes, e.g. spheres. Furthermore, we also show how to use our learned SCDF to improve an existing roadmap planner.
\section{Problem formulation}
A robot is located in the world space, $\W \subset \R^3 $. The unique location of the robot is given by its configuration $\q \in \C$, where $\C$ is the configuration space. The set of points covered by the robots bodies at a certain configuration is expressed as $\B(\q) \subset \W$. The robot is surrounded by $\NrObst$ obstacles $\O = \bigcup_{i=1}^{\NrObst} \O_i$, where  $\O_i \subset \W$. The representation of the obstacle in the configuration space is the set $\C\O_i = \{\q \in \C \: |\: \B(\q) \cap \O_i \neq \emptyset \}$. The obstacle space is formed as $\Co = \bigcup_{i=1}^{\NrObst} \C \O_i$. The complement is referred to as the free space, $\Cf = \C \setminus \Co$. The path planning problem is a tuple, ($\Cf$, $\qStart$, $\qGoal$), where we want to connect a query pair, consisting of a start, $\qStart$, and goal configuration, $\qGoal$, with a geometric path, $\q(s): [0, 1] \mapsto \Cf$, such that $\q(0)=\qStart$ and $\q(1)=\qGoal$, or report correctly when such a path does not exist.
\end{document}

\section{Basic Background: Supervised Learning and the PAC Model}
\label{sec:background}

At this point almost everyone has heard of machine learning (ML). Anyone likely to stumble upon this article will have also heard of its most influential special case, supervised learning, and those theoretically inclined will also be familiar with the PAC model. Nonetheless, I will set the stage by  recapping the basics.

\subsection{Basics of Supervised Learning}%Let's set the stage in any case

\emph{Supervised Learning} is the task of ``coming up'' with a function $f: \X \to \Y$ to ``explain'' or ``fit'' a sequence of input/output examples   $(x_1,y_1), \ldots, (x_n,y_n)$, with $x_i \in \X$ and $y_i \in \Y$.  Here $\X$ is a \emph{data domain} consisting of \emph{datapoints} $x \in \X$, $\Y$ is a \emph{label set} consisting of \emph{labels} $y \in \Y$, and the sequence $(x_1,y_1),\ldots,(x_n,y_n)$ is the \emph{training data} consisting of \emph{labeled examples (a.k.a. samples)}~$(x_i,y_i)$.  I~will refer to the chosen function $f$ as a \emph{predictor}, and to $n$ as the \emph{sample size}. A \emph{learning algorithm} takes as input training data, and outputs (some representation of) a predictor $f \in \Y^\X$.\footnote{Note that this describes the usual \emph{batch}, a.k.a.~\emph{offline}, setting of supervised learning. I do not discuss other paradigms such as online or active learning in this article.} 



Success in supervised learning is defined as \emph{generalization} to  future examples: For a typical \emph{test example}  $(x_{\tst},y_{\tst})$, the predicted label $y'_{\tst}=f(x_{\tst})$ should ``equal'' $y_{\tst}$, perhaps approximately. We usually assume the test example is drawn from the same  ``source'' as the training data  --- commonly, i.i.d.~from the same distribution. The quality of the prediction is quantified by $\ell(y'_{\tst},y_{\tst})$, where $\ell:~\Y~\times~\Y \to \RR_{\geq 0}$ is a \emph{loss function} chosen as part of the problem definition. Common loss functions include the 0-1 loss $\ell_{0-1}(y',y) = [y' \neq y]$ for \emph{classification} problems,\footnote{The notation $[P]$ denotes $1$ when predicate $P$ is true, and denotes $0$ when $P$ is false.} as well as the absolute loss $|y'-y|$ or squared loss $(y'-y)^2$ for \emph{regression problems} featuring $\Y  \sse \RR$.

Nontrivial generalization properties are typically only possible if one assumes something about the data.\footnote{The need for such an assumption is formalized by the  \emph{no free lunch theorems} of supervised learning \cite{wolpert_connection_1992,wolpert_lack_1996,schaffer_conservation_1994}.} The Bayesian approach to  machine learning, common in many applications, assumes some parametric form for the distribution generating the data, and postulates a prior on the parameters. This is not the approach I will take in this article. Instead, I will focus on the frequentist --- and some would say ``worst-case'' or ``adversarial'' ---  approach that is common in the computational learning theory community, embodied by the PAC model. Here we assume that the (training and test) data can be explained, perhaps approximately, by a function in some ``simple enough to learn'' class of functions $\H \sse \Y^\X$, often called the \emph{hypotheses}. Equivalently, we  seek a predictor which explains the unseen data roughly  as well as the best hypothesis $h^* \in \H$, whether or not we assume that $h^*$ itself provides a perfect explanation.



 \paragraph{Common Algorithmic Templates.} Perhaps the best known general-purpose supervised learning algorithm is \emph{empirical risk minimization (ERM)}, which chooses as its predictor a hypothesis $f \in \H$ minimizing $\frac{1}{n} \sum_{i=1}^n \ell(f(x_i),y_i)$ --- a quantity called the \emph{training error}, \emph{empirical error}, or \emph{empirical risk} of $f$. %\footnote{When multiple hypotheses minimize the empirical risk, we assume ERM breaks ties arbitrarily.}
A common template for generalizing ERM involves adding a \emph{regularization term} $\psi(f)$ to the  objective function, typically chosen to measure some notion of ``hypothesis complexity.'' An algorithm instantiating this template is known as a \emph{structural risk minimizer (SRM)}, and chooses as its predictor the hypothesis $f \in \H$ minimizing the \emph{structural risk} $\frac{1}{n} \sum_{i=1}^n \ell(f(x_i),y_i) + \psi(f)$. Other well-known algorithms, such as gradient descent and its variations,  can frequently be interpreted as approximate implementations of ERM or SRM.


\paragraph{Proper vs Improper Learning.} A learning algorithm is said to be \emph{proper} if its predictor $f$ is always chosen from the hypothesis class, i.e., $f \in \H$, otherwise it is said to be \emph{improper}. ERM  is an example of a proper learning algorithm, as are SRM algorithms of the form described above.  In the \emph{proper regime} of learning, algorithms are required to be proper. This article will be concerned with the more flexible \emph{improper regime} (a.k.a \emph{representation-independent learning}), where no such constraint is placed on the learner. In other words, all we care about is predictive power at test time, rather than any insights derived from the functional form or representation of the predictor~itself.


\subsection{The PAC Model}
A standard mathematical setup for evaluation of supervised learning algorithms, at least in the theoretical computer science community, is Valiant's \emph{Probably Approximately Correct (PAC) model} of learning (see e.g.~\cite{kearns_introduction_1994,mohri_foundations_2018}). Here, we assume there is an unknown distribution $\D$ on $\X \times \Y$ from which training and test data are  drawn.  Specifically, the labeled datapoints of the training set  $(x_1,y_1), \ldots, (x_n,y_n)$, as well as the test data  $(x_\tst,y_\tst)$, are i.i.d.~from $\D$. Often it is assumed that $\D$ lies in some class of distributions of interest. The \emph{true expected loss}, or simply \emph{loss}, of a predictor $f: \X \to \Y$ is the expected loss it incurs on draws from $\D$, written $L_\D(f) = \Ex_{(x,y) \sim \D} \ell(f(x),y)$.


There are two main ``settings'' in PAC learning. The  \emph{realizable setting} only requires that the data be perfectly explained by some hypothesis in $\H$. More generally, the \emph{agnostic setting} makes no assumption relating the data to the hypotheses, but shifts the goalposts as necessary to allow nontrivial guarantees: the expected loss at test time is evaluated only ``relative'' to that of the best hypothesis $h^* \in \H$. There are other settings which make more nuanced assumptions, such as $\D$ being of a particular parametric form or its support living in some (unknown) lower-dimensional space, etc. I will mostly discuss the realizable and agnostic settings in this article, those being the simplest and most studied from a theoretical perspective. %TODO:We will briefly discuss other settings in Section ??

The PAC model demands high probability guarantees of learners, in the worst case over distributions of interest. Consider first the realizable setting, where $\D$ is such that $\min_{h \in \H} L_{\D}(h) = 0$. A PAC learner has \emph{error} $\epsilon=\epsilon(n)$ and \emph{confidence} $\delta=\delta(n)$ if, when training data consists of $n$ i.i.d~samples from a realizable distribution $\D$, it produces a predictor $f$  satisfying $L_\D(f) \leq \epsilon$ with probability at least $1-\delta$. In the agnostic setting, where $\D$ can be arbitrary, we require $L_\D(f) - \min_{h \in \H} L_\D(h) \leq \epsilon$ with probability $1-\delta$.

In both the realizable and agnostic settings, we look for PAC learners with small $\epsilon$ and $\delta$ as a function of the sample size $n$. An equivalent perspective looks at the sample complexity $m(\epsilon,\delta)$, which is the minimum sample size which guarantees error  at most $\epsilon$ with probability at least $1-\delta$. We say a problem is \emph{PAC learnable} if its PAC sample complexity is finite whenever $\epsilon,\delta > 0$.

For most PAC learning problems, learnability and sample complexity are characterized in terms of a  ``dimension'' of the hypothesis class. Most prominently this is the \emph{VC dimension} for binary classification, the \emph{fat shattering dimension} for agnostic regression, and the \emph{DS dimension} for multiclass classification (see \cite{anthony_neural_1999,daniely_optimal_2014,brukhim_characterization_2022}). Treatment of these is beyond the scope of this article. The unfamiliar reader need not worry, however,  as dimensions will feature only tangentially in our~discussion.




%\paragraph{Learning settings: Realizable, Agnostic, etc.} In learning theory, evaluating a supervised learning algorithm requires specifying a data model and an objective. We will leave the details of the data model flexible for now, to allow for both the PAC model and the adversarial transductive model. Nonetheless we will describe two variations, which we call ``settings'', which cut across different models. The  \emph{realizable setting}  requires only that the data be perfectly explained by some hypothesis $h \in \H$ --- i.e., there exists a hypothesis which is guaranteed to suffer a loss of $0$ on training and test data. The performance of the learning algorithm is its expected loss at test time for some ``worst case'' realizable instance. More generally, the \emph{agnostic setting} makes no assumption relating the data to the hypotheses, but shifts the goalposts as necessary to allow nontrivial guarantees: the expected loss at test time is evaluated only ``relative'' to that of the best hypothesis $h^* \in \H$, again for some ``worst case'' instance. There are other settings which make more nuanced assumptions about the data, such as it is drawn from a distribution of a particular parametric form, or that it lives in some (unknown) lower-dimensional space, etc. We will mostly discuss the realizable and agnostic settings, those being the simplest and most studied from a theoretical perspective.




%%% Local Variables:
%%% mode: latex
%%% TeX-master: "learning_matching"
%%% End:

\section{Feasibility Study using Offline PFL simulations}

Before we can implement Private Federated Learning (PFL) on devices, it is essential to begin the process with comprehensive offline simulations. We use Apple's public framework \texttt{pfl-research}\footnote{https://github.com/apple/pfl-research}, which provides a platform for running offline PFL simulations with any available central data, without requiring actual user devices. This simulation environment replicates the behavior of PFL during real-time training with user devices.  Such offline simulations offer critical insights into the potential efficacy of PFL for the intended application, allowing practitioners to assess the balance between privacy protection and utility even before PFL deployments. Moreover, the use of simulations facilitates extensive hyperparameter tuning on a large scale. This capability is essential for determining the optimal settings for both learning and privacy-specific hyperparameters, ensuring that the model performs effectively while adhering to required privacy constraints. This preparatory step is pivotal for ensuring that the adoption of PFL is both practical and aligned with the specific needs and constraints of the application at hand.

The App Selection model described in Section~\ref{sec:model_desc} consists of a Deep Neural Network model, whose weights can be learned or updated using Private Federated Learning~\cite{reddi2020adaptive} . We have evaluated offline PFL simulations on the DNN part of the model in two different ways.:
\begin{itemize}
    \item \textbf{Training from scratch} : In this setting, all weights of the neural network are initialized randomly and then learned as part of the back-propagation step. Thus the model is trained from scratch, which makes this a more complex and time consuming task. It is a relatively higher data and compute hungry setup, but can be used to support breaking changes in the model architecture or adding new features to the modeling task. 
    \item \textbf{Fine-tuning from an existing checkpoint} : In this setting, we start from an already trained model checkpoint (i.e., the weights have been learned previously) and then only a subset of weights of the neural network are unfrozen i.e., updated as part of back-propagation step, while the remaining weights are frozen. This setting is particularly useful when we already have a fixed architecture, fixed set of features \& a solid baseline model to which we will simply add data to continuously update the model to adapt to distribution shift (need citation).
\end{itemize}


Unless otherwise mentioned, we have used AdamW~\cite{loshchilov2017decoupled} as the server side optimizer and SGD as the client side optimizer for PFL simulations using \texttt{pfl-research}. 
The PFL model’s performance is evaluated using the accuracy metric defined in Section~\ref{sec:eval_metrics} computed on the fixed validation set.
Note that the training and evaluation data required for each of the two above setups are different, and hence are elaborated in the subsequent sections.

\subsection{Training from scratch}
\label{sec:scratch_train}

In this setup, we want to simulate the PFL-based training from scratch with Differential Privacy (DP) enabled. 
We start the training with a randomly initialized set of model parameters and train the entire model from scratch using a relatively large offline data set consisting of $\sim$788K data points. 
As the baseline, we have trained the same architecture with the entire training data from scratch without any PFL, which we refer to as \textit{Cymba} for simplicity and ease of reading.
We have a dedicated validation set consisting of $\sim$178K data points, which we use to compare the performance of the PFL trained model against the non-PFL baseline (i.e., Cymba).

Unless otherwise specified, we have used Gaussian Moments Accountant (Needs citation) implemented in \texttt{pfl-research} as the central privacy mechanism, with parameters as : $\epsilon = 2.0$, $\delta = 1e-6$, and $\text{Clipping Bound} = 0.1$. 
Additionally, for simplicity of offline simulation, we have fixed the \textit{mean data points per user} $= 1$ and \textit{local epochs} $= 3$, after some initial hyper-parameter exploration.
In all our experiments, we have set a higher Local Learning rate (LLR) for the on-device SGD step due to sparsity of training data per device, and a lower Central Learning rate (CLR) for the server side Adam optimizer step. This configuration has generally helped us achieve a good distributed training set up using PFL, while reducing the risk of training divergence due to sub-optimal hyper-parameter choice.

We design the offline simulations in such a way that the insights from these simulations can help plan the duration of on device training too.
Note that for the app selection model, there are two very important parameters that determine the overall time and resources needed for on device training viz. the \textit{number of devices per Central iteration} and the \textit{number of Central iterations}. 
We chose two sets of values to represent the High and Low resource Budget settings to determine the performance bounds of the PFL training :
\begin{itemize}
    \item \textbf{High Resource Budget} : In this setting, we choose 10K devices per central iteration, and 500 number of central iterations. Thus we take approximately $(10\text{K} * 500)/788\text{K} = 6.3$ passes on the entire training data, which is smaller than the total number of epochs used to train the non-PFL Cymba baseline.
    \item \textbf{Low Resource Budget} : In this setting, we choose 1K devices per central iteration, and number of central iterations = 500. Thus we take approximately $(1\text{K} * 500)/788{\text{K}} = 0.63$ passes on the entire training data, which is smaller than the total number of epochs used to train the non-PFL Cymba baseline.
\end{itemize}
We observe that in the high resource budget setting, the PFL trained model is almost similar in performance compared to Cymba, with a very minor regression that is acceptable, given that Differential Privacy will have a negative impact on the model's learning capacity.
As expected, the low resource budget setting appears to have significantly more regression compared to Cymba, likely because of the significantly smaller number of passes on the training data, coupled with the impact of Differential Privacy. 
Note that in this setting, it is difficult to achieve an improvement over an existing non-PFL trained baseline using the exact same training data.
However, given the PFL trained model's performance under the High Resource Budget setting, we consider that we may be successful in training a PFL model from scratch during on device training, which may have a competitive performance (i.e., within acceptable limits of performance regression) compared to the non-PFL trained baseline.

\begin{table}
\centering
\begin{tabular}{l|l|l|r}
Model & CLR & LLR & Accuracy\\\hline
Cymba & - & - & 0.856\\
PFL with Low Resource Budget & 0.001 & 0.01 & 0.83 \\
PFL with High Resource Budget & 0.0005 & 0.01 & 0.852
\end{tabular}
\caption{Performance of Training from Scratch with v/s without PFL}
\label{tab:scratch_train} 
\end{table}


Additionally, as part of offline simulations, we did multiple rounds of hyper-parameter tuning by varying the number of devices per central iteration, number of Central iterations, Central learning rate, Local number of epochs, Local learning rate and Privacy Clipping Bound before selecting the above configuration.
Some key observations are presented below:
\begin{itemize}
    \item We fixed the Local Learning Rate to a relatively higher value (0.01) as it is a bit difficult to modify/control this parameter during on device training. However, we varied the Central Learning Rate and observed that a value between [0.1, 0.9] often causes the training to diverge, while a value of 0.0005 tend to yield consistently good results on the offline data. Additionally, we tested various learning rate scheduling strategies (like Polynomial, Cyclic etc.) for the Central Learning rate, but did not observe any major gains over a fixed Central Learning rate.
    \item We noticed that the Local Number of Epochs  $\leq3$ tends to give better results. Increasing the number of epochs any further causes training divergence, even with small values of Local Learning Rate.
    \item We varied the number of Central Iterations from 500 to 25,000 but the benefits in terms of gains gradually decreases. Hence we fix the number of Central iterations to 500 (which is much smaller than 2K), as it will reduce the on device training time significantly when compared to the time taken for 2K iterations, given that the difference in performance is very small.
    \item Assuming one data sample per device, we varied the number of Devices between 1K and 150K. The general observation is that 5K to 10K devices tended to yield good results. If we choose 5K for on device training, then it will speed up the training time as well as reduce the network communication load with the backend PFL server.
\end{itemize}




\subsection{Fine-tuning from an existing checkpoint}
\label{sec:finetune_chkp}

In this setup, we start from an already trained model checkpoint (i.e., the weights have been learned previously using a different data set) and then train a subset of weights of the neural network using random sampled data. 
In this case, Cymba is the base model from which we start the training, and use a random sampled batch of $\sim$814K data points as training data, and another random sampled batch of $\sim$176K data points as Validation set, for performance comparison.
We test two variations to establish the performance bounds: a)~training all layers of the pre-existing model and b)~training only the top layer of the pre-existing model, while freezing the remaining layers.
We explore these 2 variants with the goal to reduce the number of PFL trained parameters in the model, as it is easier to train a model with fewer parameters since it reduces the on-device training complexity, lowers the network communication cost between the device and server as well as helps with Privacy.

The Accuracy metric of the PFL fine-tuned models are presented in the Y-axis in Figure~\ref{fig:fine_tune_cymba}, while the X-axis refers to the number of PFL Central Iterations for each of the different training config.
The performance of the Cymba model on this random sampled Validation set is also reported for tracking purposes.
Unless otherwise specified, we have used Gaussian Moments Accountant (Needs citation) implemented in PFL-Research(needs citation) as the central privacy mechanism, with parameters as : $Epsilon = 2.0$, $Delta = 1e-6$ and $Clipping Bound = 0.1$. 
For fine-tuning from existing checkpoint, we have set the \textit{mean data points per user} $= 1$, \textit{Local Number of Epochs} $= 1$, \textit{number of devices per Central iteration} $= 5000$ and the \textit{number of Central iterations} $= 500$. 
For fair comparison, we have trained a dedicated model from scratch (labeled as \textit{Train from scratch} in Figure~\ref{fig:fine_tune_cymba}) using the random sampled $\sim$814K training data points, while reducing the \textit{number of devices per Central iteration} $= 5000$.
We have done a hyper-parameter search for the learning rate and chose the best configuration for each setting when reporting the above metrics.


\begin{figure}
\centering
\includegraphics[width=\textwidth]{figures/PFL_Paper_Figure_2.png}
\caption{Fine-tuning Cymba from an existing checkpoint}
\label{fig:fine_tune_cymba}
\end{figure}


In Figure~\ref{fig:fine_tune_cymba}, we observe that fine-tuning with freshly random sampled training data improves the performance of the PFL trained models on Validation set, compared to the static Cymba baseline. 
This indicates that the app selection model can adjust to the user's behavioral shift over time (as represented within the random sampled Training and Validation sets) using PFL, which is very useful for continuous model maintenance/upgrade activity.
Note that fine-tuning only the top layer of Cymba appears to be performing better than fine-tuning all layers,which reduces the amount of communication bandwidth spent to transfer the PFL weights from device to the servers, thereby making PFL less network bandwidth-intensive operation. Also fewer learnable parameters is better for DP thereby making this training paradigm a more suitable one for this use-case.
Finally, using PFL to train a model from scratch using random sampled data still under-performs all the settings, but the gap in performance is smaller if hyper-parameter tuning can be done appropriately.



\begin{figure}
\centering
\includegraphics[width=\textwidth]{figures/PFL_Paper_Figure_3.png}
\caption{Using simulations to predict the performance of the fine-tuning of Top Layers of the Cymba model as a function of Number of devices (cohort size) and corresponding total number of data points.}
\label{fig:data_req_for_tuning}
\end{figure}

In Figure~\ref{fig:data_req_for_tuning}, we plot the performance of fine-tuning the Top Layers of the Cymba model by varying the \textit{number of devices per Central iteration} as (1K, 2K, 5K). Assuming \textit{mean data points per user = 1}, in the X-axis we plot the number of data points that have been used for fine-tuning the top layers of Cymba for the corresponding Cohort size. The Y-axis shows the relative difference in Accuracy of the corresponding model checkpoint compared to the performance of the Cymba model on a fixed Validation set. This plot gives us an approximate insight into the amount of data points required to obtain a certain percentage of relative accuracy improvement, which will help us to select the corresponding parameters and the duration of the on device training. We observed that with a Cohort size of 2K while we achieve an improvement with fewer data points, model performance plateaus quickly, even though the performance can be further improved with more data points using a Cohort size of 5K. For example, with 504K data points, we can achieve close to 2\% relative improvement in Accuracy with a Cohort size of 5K, while the relative improvement is lesser with a Cohort size of 2K. This shows the importance of selecting the Cohort size appropriately in conjunction with the duration of the on device training (i.e., total number of data points to use for fine-tuning) to achieve best improvement.


%\subsection{Ablation study}
%\textbf{Training data retention period estimation experiments}


\section{PFL on device training}

\subsection{On device training data}

Training data are generated during inference time through user's explicit feedback.  These data are stored on device. Each training record includes feature values, ground truth labels and metadata. On device data storage system provides a mechanism whereby a particular task can filter out records which satisfy certain matching criteria specified by PFL server. For example, it is possible to match on device OS versions, or target a specific set of data produce by a specific on-device asset through this mechanism.  

\subsection{Federate statistics}

We use Federated Statistics, \cite{CorriganGibbs2017PrioPR}, which is Apple’s end-to-end platform for learning histogram queries from sensitive data on-device, to run histogram queries from on-device data to gain training data insights, such as how much data are available to participate in the PFL training. Before we launch PFL training, as part of the feasibility study, using the FedStats query, we found that ABSOLUTE NUMBER of devices have at least 1 valid sample to participate in PFL training thereby satisfying our data requirements. This shows us that we can complete PFL training iterations with reasonable latency and achieve model convergence. 

\subsection{On device plug-in design}

To enable real devices to process local data and contribute to a Personalized Federated Learning (PFL) task, an on-device plugin was developed. The primary objectives of the plugin are to process local data stored on the device using parameters defined in the PFL task description and attachments, compute a model update or generate training statistics and metrics, and then send these results to a central server for aggregation.
The plugin also includes an on-device differential privacy component. It ensures user privacy and security by applying differential privacy (DP) techniques and encrypting the model updates or training statistics before transmission, protecting sensitive data throughout the process.

The training workflow shown in Figure~\ref{fig 4} involves several key components to ensure on-device data processing and training while maintaining privacy and security:

\begin{enumerate}
    \item \textbf{Inference Framework and On-device Data Store:} The inference framework collects training data and stores it in the On-device Data Store, which is essential for PFL tasks.
    \item \textbf{Data Utilization by FedStats Server:} The FedStats Server utilizes the data stored on the device to aggregate statistics or provide insights into data distribution without directly accessing raw data.
    \item \textbf{PFL Plugin and On-device Orchestration:} The PFL plugin processes local data using task descriptions and parameters provided by the On device orchestration, which communicates with the PFL Server to receive these descriptions and attachments. This runs in a secure and isolated environment as sandboxed process.
    \item \textbf{Differential Privacy and Encryption:} After processing the data and computing model updates or training statistics, these outputs are passed through the Differential Privacy component. It applies DP techniques to anonymize and secure the data, adding noise to protect user privacy. The aggregated and encrypted updates are then sent back to the PFL Server.
    \item \textbf{Model updates at PFL Server:} The PFL Server aggregates updates from multiple devices, contributing to overall model training without compromising individual data privacy.
\end{enumerate}

\begin{figure}
\centering
\includegraphics[width=\textwidth]{figures/live_training_diagram.png}
\caption{PFL on device training workflow}
\label{fig 4}
\end{figure}

\subsection{On device evaluation}

To process local records, the plug-in will use the task sent from the server to devices, as well as model files and any additional files required by the plug-in. For example, in our case, app selection model can be trained on-device and the difference between the model parameter values before and after on-device training represent the result to be aggregated on a server for PFL training. 

We use custom-built tool to configure the necessary parameters for training a model or computing statistics on-device for a particular PFL task. Metrics will be computed at this stage in the plug-in, which will also be sent to a server for aggregation. When training ML models using PFL, metrics include training loss and training/evaluation accuracy. We also compute additional user facing metrics in Section 2.2. Metrics will be sent in the metadata field to the server from devices, along with encrypted results. 

\subsection{PFL training results}

We conducted several training cycles on different sizes of traffic. The results are summarized in Table~\ref{Result},

\begin{table}
\centering
\begin{tabular}{c|cc}
Model&CDER&Disambiguation Rate\\\hline
Baseline&89.18\%&1.99\%\\
PFL trained model&89.86\%&1.99\%\end{tabular}
\caption{\label{Result}Model Evaluation Results}
\end{table}
The PFL trained model showed about 0.6\% of absolute gain in CDER while keeping the same Disambiguation rate over our baseline model. The model’s gain is mainly due to users' change in behavior over time. The old (baseline) model trained on older server side data has drifted away from more recent data. The PFL model was trained on more recent data which captured this distribution change in user behavior.

We have also A/B tested this PFL trained model. Our A/B experiment was conducted for 2 weeks on about 15M devices. The PFL trained model achieved a 0.07\% gain in the top-line metric of system task completion rate and a 15.6\% decrease in the Disambiguation rate. This indicates our PFL trained model improved user experience by correctly predicting users' intended apps.




Software development is increasingly conceived as a collaboration activity between developers and AIs. Indeed, IDEs already implement features to enable interactive development, with AI suggesting implementations that are reused by developers.

Although multiple studies show this interaction can be successful, there is still limited understanding of how the models must be configured and used in the context of code generation tasks. This study addresses this gap, systematically investigating the impact of several key parameters, including the repeated submission of a prompt to accommodate for the non-deterministic nature of the models.

Our study reveals several key findings about the usage of ChatGPT. In particular, we discovered how creativity, although up to a limited extent, is useful to increase the range of methods whose code can be generated correctly. A major role is played by parameter top-p, which is commonly underrated, and instead has a major impact on the correctness of the results, with lower values producing better results. Finally, prompts should be submitted multiple times, with $5$ repetitions combined with a temperature of $1.2$ resulting in an effective configuration in our experiments.  

Future work concerns two main research directions. One is about replicating this experiment with other AI assistants, to validate our findings in multiple contexts. The second research direction concerns finding strategies to deal with the need to submit the same prompt multiple times to obtain a useful result, and thus developing approaches able to select or merge multiple responses automatically. 








\bibliographystyle{abbrv}
\bibliography{main}
    

\end{document}