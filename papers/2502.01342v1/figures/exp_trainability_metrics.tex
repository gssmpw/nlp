

\begin{figure*}[h]
    \centering
    \includegraphics[width=0.30\textwidth]{figures/sources/trainability_randomlabelMNIST_title.pdf}\\
    \includegraphics[width=0.32\textwidth]{figures/sources/trainability_randomlabelMNIST_dormant_neuron.pdf}
    \includegraphics[width=0.32\textwidth]{figures/sources/trainability_randomlabelMNIST_average_unit_sign_entropy.pdf}
    \includegraphics[width=0.32\textwidth]{figures/sources/trainability_randomlabelMNIST_effective_rank.pdf}
    \includegraphics[width=0.30\textwidth]{figures/sources/trainability_permutedMNIST_title.pdf}\\
    \includegraphics[width=0.32\textwidth]{figures/sources/trainability_permutedMNIST_dormant_neuron.pdf}
    \includegraphics[width=0.32\textwidth]{figures/sources/trainability_permutedMNIST_average_unit_sign_entropy.pdf}
    \includegraphics[width=0.32\textwidth]{figures/sources/trainability_permutedMNIST_effective_rank.pdf}

\caption{\textbf{Metrics for Measuring Plasticity Loss.} This figure presents the Dormant Neuron Ratio, Average Sign Entropy, and Effective Rank across tasks, comparing Vanilla, Dropout, and AID on random label MNIST (\textbf{Top}) and permuted MNIST (\textbf{Bottom}). Notably, AID maintains key metrics where Dropout fails, demonstrating its ability to mitigate plasticity loss and sustain network adaptability.}
    
    \label{exp_trainability_metrics}
\end{figure*}


