

\section{Details of Property \ref{property:He_init}}
\label{app:He_init}

    At the forward pass case, the property of the ReLU function used in the derivation of equations in \citet{he2015delving}'s work as follows:
    \begin{align*}
        \mathbb E [(r(y))^2] &= \frac{1}{2} \mathbb E[y^2] \\
        &= \frac{1}{2} (\mathbb E[y^2]-\mathbb E[y]^2) \\
        &= \frac{1}{2} Var(y)
    \end{align*}

    where $r(\cdot)$ is ReLU activation, $y$ represents pre-activation value which is a random variable from a zero-centered symmetric distribution. We show that substituting ReLU with AID in this equation yields the same result:

    \begin{align*}
        \mathbb E [(AID_p(y))^2] &=  p\mathbb E[(r(y))^2] + (1-p) \mathbb E[(\bar r(y))^2]\\
        &= p\times \frac{1}{2}Var(y) + (1-p)\times \frac{1}{2}Var(y) \\
        &= \frac{1}{2}Var(y).
    \end{align*}

    Additionally, during the backward pass ,  \citet{he2015delving} utilize that derivative of ReLU function, $r'(y)$, is zero or one with the same probabilities. We have same condition that $y$ is a random variable from a zero-centered symmetric distribution.
    Applying this to AID, we can confirm that it yields the same result:

    \begin{align*}
        P(AID_p'(y) = 1) &= P(AID_p'(y) = 1 | y\geq0)P(y\geq0) + P(AID_p'(y) = 1 | y<0)P(y<0) \\
        &= p\times\frac{1}{2} + (1-p)\times\frac{1}{2} \\
        &= \frac{1}{2} \\
        P(AID_p'(y) = 0) &= P(AID_p'(y) = 0 | y\geq0)P(y\geq0) + P(AID_p'(y) = 0 | y<0)P(y<0) \\
        &= (1-p)\times\frac{1}{2} + p\times\frac{1}{2} \\
        &= \frac{1}{2}.
    \end{align*}

    Therefore, under the given conditions, replacing the ReLU function with AID during the derivation process yields identical results. Consequently, it can be concluded that Kaiming He initialization is also well-suited for AID.