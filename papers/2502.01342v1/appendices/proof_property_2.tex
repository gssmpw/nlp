

\section{Proof of Property \ref{property:2}}
\label{app:proof_property_2}
\begin{proof}
Consider an input vector $\mathbf{x} = (x_1, x_2, \dots, x_n) \in \mathbb{R}^n$.
After applying $AID_p$ to $\mathbf{x}$, the $i$-th element is expressed as:
\begin{align*}
AID_p(\mathbf{x})_i 
&= 
\begin{cases}
    p_{i} x_i & x_i \geq 0, \\
    (1-p_{i}) x_i & x_i < 0,
\end{cases}
\end{align*}
where $p_{i}$ is a random variable sampled from $Ber(p)$.

To enable vectorized computation, we introduce a mask matrix $\mathbf{M} \in \mathbb{R}^{n \times n}$, satisfying $AID_p(\mathbf{x}) = \mathbf{M} \mathbf{x}$.
Specifically, the diagonal elements $m_{(i,i)}$ of $\mathbf{M}$ are sampled as follows:
\[
m_{(i,i)} \sim Ber((2p-1)H(x_i) + (1-p)),
\]
where $H(\cdot)$ is the step function, mapping positive values to 1 and negative values to 0. Then, the matrix $\mathbf{M}$ can be written as:
\begin{align}
    \mathbf{M} = (2\mathbf{P} - \mathbf{I}) H(\mathbf{x}) + (\mathbf{I} - \mathbf{P}) 
    \label{eq:M}
\end{align}
where $\mathbf{I}$ is the identity matrix of size $n \times n$ and $\mathbf{P} \doteq \text{diag}((p_1, p_2, \dots, p_n))$, where $\text{diag}(\cdot)$ is the function that creates a matrix whose diagonal elements are given by the corresponding vector.


Using equation \ref{eq:M}, the AID operation can be expanded as:
\begin{align}
AID_p(\mathbf{x}) &= \mathbf{M} \mathbf{x} \nonumber\\
&= [(2\mathbf{P} - \mathbf{I}) H(\mathbf{x}) + (\mathbf{I} - \mathbf{P})] \mathbf{x} \nonumber\\
&= (2\mathbf{P} - \mathbf{I}) r(\mathbf{x}) + (\mathbf{I} - \mathbf{P}) \mathbf{x} \label{eq:AID_expansion} \\
&= (2\mathbf{P} - \mathbf{I}) r(\mathbf{x}) + (\mathbf{I} - \mathbf{P}) (\mathbf{x} - r(\mathbf{x}) + r(\mathbf{x})) \nonumber\\
&= \mathbf{P} r(\mathbf{x}) + (\mathbf{I} - \mathbf{P}) (\mathbf{x} - r(\mathbf{x})) \nonumber\\
&= \mathbf{P} r(\mathbf{x}) + (\mathbf{I} - \mathbf{P}) \bar{r}(\mathbf{x}) \nonumber\\
&= [\mathbf{P} r + (\mathbf{I} - \mathbf{P}) \bar{r}] \mathbf{x}.\nonumber
\end{align}

Thus, applying AID is equivalent to applying ReLU with probability $p$, and negative ReLU with probability $1-p$.
\end{proof}

