\section{Applications} \label{sec:applications}
\algoname enables various downstream tasks; we demonstrate two.

\myparagraph{Temporal Segmentation}Temporal segmentation is the task of partitioning a temporal sequence into disjoint groups, where frames sharing similar characteristics are grouped. 
For a clean sample $X_{0}$, either generated or given, and skeleton $\mss$, we extract DIFT features at diffusion step $t_{seg}\!=\!\!3$ and layer $l_{seg}\!=\!\!1$. The features are averaged along the joint dimension to produce a single feature vector per frame. We apply PCA for dimensionality reduction and then use K-means to cluster the frames into $k\!\!=\!\!3$ categories. Our results are visualized in \cref{fig:temoral_seg} and in the supp. video.
This application reinforces \cref{sec:analysis}, showing that \algoname's latent features are effective frame descriptors.
However, in \cref{sec:analysis}, frames are grouped by similarity to $X^{ref}$, while here they are grouped by similarity to each other.

\myparagraph{Editing} 

We demonstrate our method's versatility through two motion editing applications: \emph{in-betweening} for temporal manipulation and \emph{body-part editing}  for spatial modifications, both leveraging the same underlying approach.
For \emph{in-betweening}, the prefix and suffix of the motion are fixed, allowing the model to generate the middle.
For \emph{body-part editing}, we fix some of the joints and let the model generate the rest.
Given a fixed subset (temporal or spatial) of the motion sequence tokens, we override the denoised $\hat{x}_{0}$ at each sampling iteration with the fixed motion part. 
This approach ensures fidelity to the fixed input while synthesizing the missing elements of the motion.
Our results, in \cref{fig:inpainting} and the supp. video, show a smooth and natural transition between the given and the synthesized parts, and demonstrate that our model successfully generalizes techniques previously limited to human skeletons \cite{tevet2023human} to accommodate diverse skeletal structures.


\begin{figure}
    

    \centering
    
    \includegraphics[width=\columnwidth]{Images/cross_gen.pdf}
    
    \caption{
        \textbf{\Crossgen.} 
        Right: a generated motion featuring an action not in the performing skeleton’s ground truth.
        Left: notably, the nearest ground truth originates from a different character.
    }
    \label{fig:cross_gen}
    \Description[]{}  %
\end{figure}


