\section{Analysis}\label{sec:analysis}

\subsection{Latent Space Analysis}
In this section, we examine \algoname's latent space and show that it features a unified manifold for joints across all skeletons.
We use DIFT \cite{tang2023emergent}, a framework designed for detecting correspondence in the latent space of models undergoing diffusion.
DIFT features are intermediate activations from layer $l_{corr}$, extracted during a single denoising pass on a sample that has been noised directly to diffusion step $t_{corr}$. These features serve as effective semantic descriptors for predicting correspondence.
Note that the values we choose for $l_{corr}$ and $t_{corr}$ align with those used in the original DIFT work.
Let $X^{ref}$ denote a reference motion, and let $X^{tgt}$ denote a motion in which we search for corresponding parts.
Let $S^{ref}, S^{tgt}$ denote their skeletons, respectively.

Our spatial and temporal correspondence results are illustrated in \cref{fig:spatial_cor,fig:temoral_cor} respectively, and in the supplementary video.

\myparagraph{Spatial Correspondence}
\begin{figure}
    

    \centering
    
    \includegraphics[width=0.8\columnwidth]{Images/joint_cor.pdf}
    
    \caption{
        \textbf{Spatial Correspondence.} Monkey (top left) depicts the reference skeleton, while the fox, scorpion, and bird depict different target skeletons. 
        Target skeleton joints are color-coded to match their corresponding joints in the reference. 
        For better visualization, we color the bones to match their adjacent joints. 
        Note the correspondence in limbs, spine, and tail.
       }
    \label{fig:spatial_cor}
    \Description[]{}  %
\end{figure}

We show that manifold features of semantically similar skeletal joints across different characters are close to each other.
Our objective is to find the most similar \emph{joint} in $\mss^{ref}$ for each joint in $\mss^{tgt}$.
To achieve this, we extract DIFT features for both motions $X^{ref}, X^{tgt}$ at diffusion step $t_{corr} = 2$ and layer $l_{corr} = 0$, average them along the temporal axis, and obtain a single feature vector per joint.
Using cosine similarity, we detect the closest counterpart for each joint in $S_{tgt}$. 


\myparagraph{Temporal Correspondence}
\begin{figure}
    

    \centering
    
    \includegraphics[width=\columnwidth]{Images/temporal_cor.pdf}
    
    \caption{
    \textbf{Temporal Correspondence.} Monkey (top row) features the reference motion, while the Crab and Lynx represent two target motions. The frames of the targets are color-coded to align with their corresponding reference frames.
    Note the correspondence: aggressive motion segments are pink, idle frames blue, and transitional frames green.
    }
    \label{fig:temoral_cor}
    \Description[]{}  %
\end{figure}

We show that \algoname can recognize pose-level similarities and identify analogous actions across different skeletons.
This time, our objective is to find the most similar \emph{frame} in $X^{ref}$ for each frame in $X^{tgt}$.
To accomplish this goal, we extract DIFT features at diffusion step $t_{corr}=3$ and layer $l_{corr}=1$, and average them along the skeletal axis, resulting in a single feature vector per frame.
We use cosine similarity to detect the closest counterpart for each frame in $X_{tgt}$. 

\ifarxiv
\begin{figure}
    

    \centering
    
    \includegraphics[width=\columnwidth]{Images/intra_chicken.pdf}
    
    \caption{
    \textbf{\Ingen.} The top row depicts two ground truth chicken motions: pecking (left) and walking (right). The bottom row presents synthesized motions of an adapted SinMDM (left) and \algoname (right). The emphasized frames in \algoname demonstrate spatial composition of walking and pecking, introducing novel poses not present in the ground truth. SinMDM embeds entire poses, hence cannot spatially-compose joints. 
    }
    \label{fig:in_gen}
    \Description[]{}  %
\end{figure}

\fi
\subsection{Generalization Forms}



We identify three forms of generalization in our generated motions.

\myparagraph{\Ingen} dubbed \ingen, refers to generalization within a specific skeleton, featured as both \emph{temporal composition} -- combining motion segments from dataset instances, and \emph{spatial composition} -- introducing novel poses by combining skeletal parts of ground truth poses. Notably, \emph{spatial composition} is enabled by our per-joint encoding, which provides the flexibility required for such diversity. In \cref{fig:in_gen} and in our supp. video, we showcase \algoname's \ingen and highlight how other methods, which embed the entire pose, fail to achieve a comparable variety.

\myparagraph{\Crossgen} dubbed \crossgen, 
is expressed through shared motion motifs across different characters. This form of generalization enables the adaptation of motion behaviors originally performed by other skeletons, as shown in \cref{fig:cross_gen} and our supp. video. 
When motions must strictly align with typical behaviors, the training dataset can be restricted accordingly.

\myparagraph{\Unseengen} 
extends to skeletons not encountered during training, and illustrated in \cref{fig:unseen} and the video.

