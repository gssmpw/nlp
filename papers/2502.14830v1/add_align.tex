% This must be in the first 5 lines to tell arXiv to use pdfLaTeX, which is strongly recommended.
\pdfoutput=1
% In particular, the hyperref package requires pdfLaTeX in order to break URLs across lines.

\documentclass[11pt]{article}

% Change "review" to "final" to generate the final (sometimes called camera-ready) version.
% Change to "preprint" to generate a non-anonymous version with page numbers.
\usepackage[]{acl}

% Standard package includes
\usepackage{times}
\usepackage{latexsym}

% For proper rendering and hyphenation of words containing Latin characters (including in bib files)
\usepackage[T1]{fontenc}
% For Vietnamese characters
% \usepackage[T5]{fontenc}
% See https://www.latex-project.org/help/documentation/encguide.pdf for other character sets

% This assumes your files are encoded as UTF8
\usepackage[utf8]{inputenc}

% This is not strictly necessary, and may be commented out,
% but it will improve the layout of the manuscript,
% and will typically save some space.
\usepackage{microtype}

% This is also not strictly necessary, and may be commented out.
% However, it will improve the aesthetics of text in
% the typewriter font.
\usepackage{inconsolata}

%Including images in your LaTeX document requires adding
%additional package(s)
\usepackage{graphicx}

\usepackage{lipsum} % For dummy text
\usepackage{stfloats} % For better float placement
% Change "review" to "final" to generate the final (sometimes called camera-ready) version.
% Change to "preprint" to generate a non-anonymous version with page numbers.
%\usepackage{acl}
\usepackage[]{acl}
% Standard package includes
\usepackage{times}
\usepackage{latexsym}
\usepackage{hyperref}
\usepackage{amsmath}
\usepackage{amssymb}
\usepackage{array}
\usepackage{makecell}
% For proper rendering and hyphenation of words containing Latin characters (including in bib files)
\usepackage[T1]{fontenc}
\usepackage{inputenc}
% For Vietnamese characters
% \usepackage[T5]{fontenc}
% See https://www.latex-project.org/help/documentation/encguide.pdf for other character sets

% This assumes your files are encoded as UTF8
\usepackage[utf8]{inputenc}

% This is not strictly necessary, and may be commented out,
% but it will improve the layout of the manuscript,
% and will typically save some space.
\usepackage{microtype}

% This is also not strictly necessary, and may be commented out.
% However, it will improve the aesthetics of text in
% the typewriter font.
\usepackage{inconsolata}

%Including images in your LaTeX document requires adding
%additional package(s)
\usepackage{graphicx}
\usepackage{booktabs}
\usepackage{hyperref}
\usepackage{tabularx}

\usepackage{fontawesome5}


% nicer tables
\usepackage{booktabs, colortbl} 
\usepackage{multirow,makecell} 
\usepackage{multicol}
% nicer equations
\usepackage{amsmath}
\usepackage{amssymb}
% subfigure
\usepackage{caption}
\usepackage{subcaption}
% paragraph sign
\usepackage{titlesec}
% tikz
\usepackage{pgfplots}
% strikethrough
\usepackage[normalem]{ulem} 
% parskip for list
\usepackage{enumitem}
% \rotatebox macro
\usepackage{graphicx} 
\usepackage{hyperref}
%
\usepackage{svg}

\usepackage{enumitem}
\usepackage[splitrule]{footmisc}
\setlist[itemize]{noitemsep, nolistsep}
\usepackage{color, colortbl}
	
\definecolor{Gray}{gray}{0.9}

% If the title and author information does not fit in the area allocated, uncomment the following
%
%\setlength\titlebox{<dim>}
%
% and set <dim> to something 5cm or larger.

\title{Middle-Layer Representation Alignment for Cross-Lingual Transfer in Fine-Tuned LLMs}
% Multilingual Alignment

% Author information can be set in various styles:
% For several authors from the same institution:
% \author{Author 1 \and ... \and Author n \\
%         Address line \\ ... \\ Address line}
% if the names do not fit well on one line use
%         Author 1 \\ {\bf Author 2} \\ ... \\ {\bf Author n} \\
% For authors from different institutions:
% \author{Author 1 \\ Address line \\  ... \\ Address line
%         \And  ... \And
%         Author n \\ Address line \\ ... \\ Address line}
% To start a separate ``row'' of authors use \AND, as in
% \author{Author 1 \\ Address line \\  ... \\ Address line
%         \AND
%         Author 2 \\ Address line \\ ... \\ Address line \And
%         Author 3 \\ Address line \\ ... \\ Address line}

\author{Danni Liu \phantom{\and} Jan Niehues \\
        Karlsruhe Institute of Technology, Germany \\
        \texttt{\{danni.liu, jan.niehues\}@kit.edu}} 

%\author{
%  \textbf{First Author\textsuperscript{1}},
%  \textbf{Second Author\textsuperscript{1,2}},
%  \textbf{Third T. Author\textsuperscript{1}},
%  \textbf{Fourth Author\textsuperscript{1}},
%\\
%  \textbf{Fifth Author\textsuperscript{1,2}},
%  \textbf{Sixth Author\textsuperscript{1}},
%  \textbf{Seventh Author\textsuperscript{1}},
%  \textbf{Eighth Author \textsuperscript{1,2,3,4}},
%\\
%  \textbf{Ninth Author\textsuperscript{1}},
%  \textbf{Tenth Author\textsuperscript{1}},
%  \textbf{Eleventh E. Author\textsuperscript{1,2,3,4,5}},
%  \textbf{Twelfth Author\textsuperscript{1}},
%\\
%  \textbf{Thirteenth Author\textsuperscript{3}},
%  \textbf{Fourteenth F. Author\textsuperscript{2,4}},
%  \textbf{Fifteenth Author\textsuperscript{1}},
%  \textbf{Sixteenth Author\textsuperscript{1}},
%\\
%  \textbf{Seventeenth S. Author\textsuperscript{4,5}},
%  \textbf{Eighteenth Author\textsuperscript{3,4}},
%  \textbf{Nineteenth N. Author\textsuperscript{2,5}},
%  \textbf{Twentieth Author\textsuperscript{1}}
%\\
%\\
%  \textsuperscript{1}Affiliation 1,
%  \textsuperscript{2}Affiliation 2,
%  \textsuperscript{3}Affiliation 3,
%  \textsuperscript{4}Affiliation 4,
%  \textsuperscript{5}Affiliation 5
%\\
%  \small{
%    \textbf{Correspondence:} \href{mailto:email@domain}{email@domain}
%  }
%}

\begin{document}
\maketitle

\begin{abstract}
While large language models demonstrate remarkable capabilities at task-specific applications through fine-tuning, 
extending these benefits across diverse languages is essential for broad accessibility.
However, effective cross-lingual transfer is hindered by LLM performance gaps across languages and the scarcity of fine-tuning data in many languages.
Through analysis of LLM internal representations from over 1,000+ language pairs,
we discover that middle layers exhibit the strongest potential for cross-lingual alignment.
Building on this finding, 
we propose a middle-layer alignment objective integrated into task-specific training.
Our experiments on slot filling, machine translation, and structured text generation show consistent improvements in cross-lingual transfer, especially to lower-resource languages.
The method is robust to the choice of alignment languages and generalizes to languages unseen during alignment.
Furthermore, 
we show that separately trained alignment modules can be merged with existing task-specific modules, improving cross-lingual capabilities without full re-training.
Our code is publicly available\footnote{\url{https://github.com/dannigt/mid-align}}.%provided in the supplementary materials.

\end{abstract}



\section{Introduction}
Decoder-only large language models (LLMs) have emerged as the dominant paradigm in NLP.
While these models exhibit promising zero-shot capabilities \cite{DBLP:conf/iclr/WeiBZGYLDDL22,palm2}, 
further task-specific fine-tuning remains crucial for optimal performance in many applications \cite{DBLP:conf/icml/ShenTHAF24,DBLP:conf/iclr/Xu0SA24,DBLP:journals/corr/abs-2402-17733}.
During fine-tuning,
a practical challenge is that the available training data rarely covers all languages supported by LLMs.
This highlights the importance of cross-lingual transfer to extend task-specific performance gains across languages.

While cross-lingual transfer has been extensively studied \cite{DBLP:conf/apsipa/WangZ15,ruder-etal-2019-transfer,artetxe-schwenk-2019-massively},
achieving it on generative tasks with variable-length outputs
remains challenging \cite{vu-etal-2022-overcoming,li-murray-2023-zero} compared to classification tasks.
This challenge is especially relevant for LLMs, 
which formulate all tasks as next-token prediction problems. 

The theoretical foundation of cross-lingual transfer lies in the analogous relationships between concepts across languages. 
This intuition was first demonstrated in cross-lingual word embeddings \cite{DBLP:journals/corr/MikolovLS13,DBLP:conf/iclr/LampleCRDJ18,xu2021crosslingualbertcontextualembedding}, 
where these vector representations exhibit isometric relationships, 
i.e., the geometric structure of semantically equivalent items is preserved across different languages.
This isometry property has proven crucial for transferring learned models across languages \cite{schuster-etal-2019-cross,wang2024bridginglanguagegapslarge}.
Subsequent encoder-decoder models \cite{ha2016multilingualneuralmachinetranslation} and decoder-only models \cite{wu-etal-2024-representational} also exhibit similar properties in their internal representations.

While pretrained multilingual models naturally develop some degree of unified multilingual representations  \cite{pires-etal-2019-multilingual, conneau-etal-2020-unsupervised,muller-etal-2021-first}, 
explicitly strengthening the relationships between semantically equivalent content has shown benefits in various downstream tasks:
cross-lingual 
retrieval \cite{yu-etal-2018-multilingual},
parallel text mining \cite{schwenk-etal-2021-wikimatrix}, 
zero-shot classification \cite{hu-etal-2021-explicit,gritta-iacobacci-2021-xeroalign} and translation \cite{arivazhagan2019missingingredientzeroshotneural,pham-etal-2019-improving,duquenne-etal-2022-modules}.
Despite different approaches,
these works share a common objective: 
\textit{aligning} representations of semantically equivalent content across languages while preserving overall expressiveness.

Cross-lingual alignment approaches have been successfully applied to models preceding LLMs.
For \textit{encoder-only} models, 
outputs can be aligned by e.g., minimizing distances between parallel sentence representations \cite{feng-etal-2022-language} or
cross-lingual masked language modeling objectives  \cite{DBLP:conf/nips/ConneauL19}.
These techniques are largely applicable to \textit{encoder-decoder} models, 
where alignment is typically enforced to the encoder outputs \cite{DBLP:journals/corr/abs-2308-11466}.
In contrast, \textit{decoder-only} models lack such clear separation between input processing and output generation.
This makes it less obvious where and how to optimize for cross-lingual alignment, 
as also highlighted in the survey by \citet{hammerl-etal-2024-understanding}.


In this work,
we start by quantifying the degree of cross-lingual alignment present in two prominent LLMs, 
Llama 3 \cite{grattafiori2024llama3herdmodels} and Qwen 2.5 \cite{qwen2025qwen25technicalreport}. 
We then apply these insights to improve cross-lingual transfer in task-specific fine-tuning.
By alternatively training on alignment and task-specific data, we aim to improve the cross-lingual generalization to languages without fine-tuning data.
We demonstrate transfer improvements across diverse tasks: slot filling, machine translation, and structured text generation.
Our main findings include:
\begin{itemize}[nolistsep,leftmargin=*]
    \item Applying alignment objectives to middle layers during LLM task-specific fine-tuning improves cross-lingual transfer (\S\ref{subsec:gains_on_transfer})
    and enhances alignment across all network depths (\S\ref{subsec:loss_placement}).
    \item The transfer improvements extend beyond those languages seen in alignment (\S\ref{subsec:gains_on_transfer}).
    \item Our approach is robust to the choice of languages used for alignment training (\S\ref{subsec:res_level}, \ref{subsec:domain_lang_generalization}).
    \item Task-specific and alignment modules trained separately can be combined post-hoc to improve transfer performance (\S\ref{subsec:merging}).
\end{itemize}



\begin{figure}[t]
    \centering
    % trim left, bottom, right, top
    \begin{subfigure}[b]{1.01\linewidth}
         \centering
         \includegraphics[width=\linewidth,clip,trim={0 0.6cm 0 0}]{figures/retrieval_accuracy_by_layer.pdf}
         \caption{Cross-lingual semantic alignment (measured by average retrieval accuracy over 35 languages and 1190 language directions) varies by layer, with the middle layer showing the highest score. Lower-resource languages are poorly aligned.}
         \label{fig:retrieval_by_layer}
     \end{subfigure}
     
    \begin{subfigure}[b]{\linewidth}
         \centering
         \includegraphics[width=1.01\linewidth,clip,trim={0 0.1cm 0 0}]{figures/retrieval_error_vs_transfer_performance.pdf}
         \caption{Positive correlation between base model cross-lingual semantic alignment and downstream transfer performance.}
         \label{fig:retrieval_vs_transfer_correlation}
     \end{subfigure}
    
    \caption{Two observations (\S\ref{sec:analysis}) motivating our approach of aligning multilingual representations (\S\ref{sec:approach}).}
    \label{fig:correlation}
\end{figure}

\section{Analyzing Cross-Lingual Alignment} \label{sec:analysis}

To understand how well LLM representations capture semantic equivalence across languages,
we use translation retrieval as a diagnostic task. 
We choose this retrieval task over other metrics like cosine similarity or SVCCA score \cite{DBLP:conf/nips/RaghuGYS17}
because it better captures \textit{relative} semantic relationships. 
That is,
if a model's representations enable us to identify a sentence's translation from a set of candidates,
the exact numerical distance between the query and the retrieved translation 
is less important than the ability to rank translations as the most semantically similar.

Specifically, 
we first extract model activations at each network layer for all language variants of the input text. 
To handle variable-length sequences, 
we create fixed-size sentence embeddings by mean-pooling the activations over the sequence length dimension. 
For translation retrieval, 
given a query sentence in one language, 
we compare its embedding to the embeddings of candidate sentences in the target language using ratio-based margin similarity \cite{artetxe-schwenk-2019-margin}\footnote{shown to outperform cosine similarity for cross-lingual retrieval tasks \cite{artetxe-schwenk-2019-margin}}. 
For $N$ languages, 
we evaluate retrieval accuracy across all $N(N-1)$ possible language pairs.
We use the \textsc{FLoRes-200} dataset \cite{DBLP:journals/nature/Team24}, 
which provides high-quality multiway parallel texts across diverse languages (detailed setup in \S\ref{subsec:setup_evaluation}).

Our investigation of LLama 3 and and Qwen 2.5 models\footnote{specifically the \texttt{8B-Instruct} and \texttt{7B-Instruct} variants} reveals three key findings:

\paragraph{Overall weak semantic alignment, with peak in middle layers:}
As shown in \autoref{fig:retrieval_by_layer}, 
the average translation retrieval accuracy across 1,190 language pairs remains below 50\%, 
with Llama~3 outperforming Qwen 2.5. 
Low-resource languages\footnote{resource levels as defined by \citet{DBLP:journals/nature/Team24}} show especially weak alignment, achieving less than half of the overall average accuracy. 
In particular, 
the \textit{middle} layers of both models demonstrate the strongest retrieval performance. 
This suggests stronger potential for cross-lingual transfer at these intermediate representations.

\paragraph{Strong correlation between base LLM semantic alignment and downstream task transfer:}
To what extent can the semantic alignment present in the base LLM predict cross-lingual transfer performance after supervised fine-tuning?
Using multilingual slot filling as a case study, 
we train models on 5 high-resource languages jointly and evaluate transfer performance on 25 additional languages 
(detailed setup in \S\ref{subsec:exp_setup_data}). 
As shown in \autoref{fig:retrieval_vs_transfer_correlation}, 
for both Llama 3 and Qwen 2.5, 
we observe strong positive correlations ($p<0.01$) between middle-layer retrieval accuracy and downstream task performance.
This correlation suggests that increasing cross-lingual alignment in LLM intermediate representations may improve cross-lingual transfer.

\paragraph{Task-specific fine-tuning preserves but does not enhance semantic alignment:}
After analyzing the base LLMs, 
we examine how supervised fine-tuning affects the models' internal semantic alignment.
Using the same multilingual slot filling task as before, 
we study both English-only and multilingual fine-tuning. 
Despite multilingual fine-tuning being an established method for improving cross-lingual transfer \cite{li-murray-2023-zero,chirkova-nikoulina-2024-key}, 
we observe that neither training configuration alters the models' cross-lingual semantic alignment (\autoref{fig:FT_impact_on_retrieval}). 
This preservation of baseline alignment patterns,
even under multilingual training,
indicates that pure fine-tuning does not sufficiently strengthen cross-lingual alignment. 
This further motivates us towards explicit cross-lingual alignment during fine-tuning.

\begin{figure*}[ht!]
    \centering
    % trim left, bottom, right, top
    \includegraphics[width=1\linewidth,clip,trim={4.5cm 3cm 2cm 0.65cm}]{figures/overview.pdf}
    \caption{Illustration of our approach, alternating training between task-specific (left) and alignment (right) objectives. 
    The alignment objective operates on middle-layer representations.}
    \label{fig:overall_approach}
\end{figure*}


\begin{figure}[t]
    \centering
    % trim left, bottom, right, top
         \centering
         \includegraphics[width=\linewidth,clip,trim={0 0.6cm 0 0}]{figures/FT_retrieval_accuracy_by_layer.pdf}
         \caption{Task-specific fine-tuning shows minimal impact on semantic alignment.}
    \label{fig:FT_impact_on_retrieval}
\end{figure}


% \end{itemize}
\section{Explicit Alignment in fine-tuning} \label{sec:approach}
We propose an alternating training strategy to encourage cross-lingual alignment while maintaining task performance.
As illustrated in \autoref{fig:overall_approach},
we optimize either the task-specific objective or the alignment objective in each training step.

\noindent
\textbf{Task Objective:} 
We follow standard causal language modeling, 
using a cross-entropy loss over the predicted text conditioned on the input prefix.

\noindent
\textbf{Alignment Objective:} 
We use a contrastive loss motivated by its successful applications in sentence embedding \cite{feng-etal-2022-language},
dense retrieval \cite{karpukhin-etal-2020-dense}
and modality alignment \cite{ye-etal-2022-cross,DBLP:conf/cvpr/GirdharELSAJM23}.
The loss maximizes the similarity between translations while minimizing similarity between non-translations.
Given a batch $\mathcal{B}$ of $n$ pairs of parallel sentences, the alignment loss for a sentence pair $(s,t)$ is:
\begin{equation}
\mathcal{L}_{\text{align}} = -\log \frac{\exp(\text{sim}(\mathbf{h}_s^i, \mathbf{h}_t^i))}{\sum_{v\in\mathcal{B}} \exp(\text{sim}(\mathbf{h}_s^i, \mathbf{h}_v^i))}
\end{equation}
where $\mathbf{h}_s^i$ is the mean-pooled\footnote{Initial experiments with attention pooling degraded performance.
We also tried a stop-gradient operator on English representations to  align non-English representations towards English, but it did not give consistent gains.} hidden states at the $i^{th}$ LLM layer for input $s$ and 
$\text{sim}(\cdot,\cdot)$ is a similarity function.
Motivated our finding that middle layers have the strongest cross-lingual alignment potential, 
we select $i$ as the middle layer
and compare its performance to other layer positions.
We use cosine similarity following prior works \cite{gao-etal-2021-simcse,ye-etal-2022-cross}.
The similarity score is optionally scaled by a temperature parameter $\tau$,
which controls the peakiness of the softmax distribution and in turn determines the relative importance of non-translation pairs.
This temperature parameter is tuned on the development sets.

\noindent
\textbf{Activating Individual Objectives:} 
Note that the task and alignment losses can be activated separately. 
Deactivating the alignment loss degenerates to standard task-only training. 
Conversely, deactivating the task loss trains the model only for alignment.
This modularity enables us to subsequently combine separately-trained task and alignment models.

\section{Experimental Setup} \label{sec:exp_setup}

\subsection{Data} \label{subsec:exp_setup_data}
\begin{table}[h!]
\centering
\caption{Dataset properties. Number of samples ($\bm n$), dimensions ($\bm d$), number of ground truth clusters ($\bm k$), number of noise points (\#noise), DISCO score for the ground truth labels (DISCO) including noise, and the source.}\label{tbl:dataset_overview}
\resizebox{0.65\linewidth}{!}{
\begin{tabular}{cclrrrrr}
\toprule
\toprule
&& Dataset & $n$ & $d$ & $k$ & \#noise & Source \\
\midrule
\parbox[t]{2mm}{\multirow{19}{*}{\rotatebox[origin=c]{90}{\textbf{Density-based 2D-Data}}}}
&& Boxes & 21,600 & 2 & 12 & 0 & \cite{yoon2023clustering}\\
\cmidrule{2-8}
&\parbox[t]{2mm}{\multirow{18}{*}{\rotatebox[origin=c]{90}{Tomas Barton Benchmark}}}
& D31 & 3100 & 2 & 31 & 0 & \cite{deric}\\
&& 3-spiral & 312 & 2 & 3 & 0 & \cite{deric}\\
&& aggregation & 788 & 2 & 7 & 0 & \cite{deric}\\
&& chainlink & 1,000 & 3 & 2 & 0 & \cite{deric}\\
&& cluto-t4-8k & 8,000 & 2 & 6 & 764 & \cite{deric}\\
&& cluto-t5-8k & 8,000 & 2 & 7 & 1,153 & \cite{deric}\\
&& cluto-t7-10k & 10,000 & 2 & 9 & 792 & \cite{deric}\\
&& cluto-t8-8k & 8,000 & 2 & 8 & 323 & \cite{deric}\\
&& complex8 & 2,551 & 2 & 8 & 0 & \cite{deric}\\
&& complex9 & 3,031 & 2 & 9 & 0 & \cite{deric}\\
&& compound & 399 & 2 & 6 & 0 & \cite{deric}\\
&& dartboard1 & 1,000 & 2 & 4 & 0 & \cite{deric}\\
&& diamond9 & 3,000 & 2 & 9 & 0 & \cite{deric}\\
&& jain & 3,373 & 2 & 2 & 0 & \cite{deric}\\
&& pathbased & 299 & 2 & 3 & 0 & \cite{deric}\\
&& smile1 & 1,000 & 2 & 4 & 0 & \cite{deric}\\
\cmidrule{2-8}
&& Synth\_low & 5,000 & 100 & 10 & 500 & \cite{densiredPaper}\\
&& Synth\_high & 5,000 & 100 & 10 & 500 & \cite{densiredPaper}\\
\midrule
\parbox[t]{2mm}{\multirow{13}{*}{\rotatebox[origin=c]{90}{\textbf{Real-World Data}}}}
&\parbox[t]{2mm}{\multirow{7}{25mm}{\rotatebox[origin=c]{90}{Tabular Data}}}
& Mice & 1,077 & 68 & 8 & 0 & \cite{uciRepo}\\
&& adipose & 14,947 & 2 & 12 & 0 & \cite{broad_singlecell}
%\tablefootnote{\url{https://singlecell.broadinstitute.org/single_cell/study/SCP1376/a-single-cell-atlas-of-human-and-mouse-white-adipose-tissue}}
\\
&& airway & 14,163 & 2 & 10 & 0 & \cite{broad_singlecell}
%\tablefootnote{\url{https://singlecell.broadinstitute.org/single_cell/study/SCP64/a-single-cell-atlas-of-the-airway-epithelium-reveals-the-cftr-rich-pulmonary-ionocyte}}
\\
&& lactate & 39,825 & 2 & 6 & 0 & \cite{broad_singlecell}
%\tablefootnote{\url{https://singlecell.broadinstitute.org/single_cell/study/SCP1671/cellular-and-transcriptional-diversity-over-the-course-of-human-lactation}}
\\
&& HAR & 10,299 & 561 & 6 & 0 & \cite{uciRepo}\\
&& letterrec. & 20,000 & 16 & 26 & 0 & \cite{uciRepo}\\
&& Pendigits & 10,992 & 16 & 10 & 0 & \cite{uciRepo}\\
\cmidrule{2-8}
&\parbox[t]{2mm}{\multirow{6}{25mm}{\rotatebox[origin=c]{90}{Image Data}}}
& COIL20 & 1,440 & 16,384 & 20 & 0 & \cite{source_coil20}\\
&& COIL100 & 7,200 & 49,152 & 100 & 0 & \cite{source_coil100}\\
&& cmu\_faces & 624 & 960 & 20 & 0 & \cite{uciRepo}\\
&& Optdigits & 5,620 & 64 & 10 & 0 & \cite{uciRepo}\\
&& USPS & 9,298 & 256 & 10 & 0 & \cite{usps}\\
&& MNIST & 70,000 & 784 & 10 & 0 & \cite{mnist}\\
% && htru2 & 17,898 & 8 & 2 & 0 & \cite{uciRepo}\\
% && KMNIST & 70,000 & 784 & 10 & 0 & \cite{kmnist}\\
% && FMNIST & 70,000 & 784 & 10 & 0 & \cite{fmnist}\\
% && Weizmann & 5,701 & 77,760 & 90 & 0 & \cite{weizmannData}\\
% && Keck & 12,922 & 30,000 & 56 & 0 & \cite{keckData}\\
\bottomrule
\bottomrule
\end{tabular}}
\end{table}


In general, we fine-tune on several 
high-resource languages and then evaluate transfer performance on additional languages.
We do not focus on English-only fine-tuning, since our initial experiments demonstrated that multilingual fine-tuning substantially outperforms English-only fine-tuning\footnote{These English-only FT results are in Appendix~\ref{sec:appendix_english_only}.}, 
thus establishing it as a stronger baseline.
\autoref{tab:data_overview} presents a dataset overview.
Descriptions of the language codes are in \autoref{sec:appendix_language_list}.

\noindent
\textbf{Main Task Data:} 
We evaluate our approach on slot filling and machine translation, 
both modeled as generative tasks with templates shown in Appendix~\ref{sec:appendix_prompt_format}.
For slot filling, 
we use the \textsc{MASSIVE} dataset \cite{fitzgerald-etal-2023-massive}.
We train on 5 high-resource languages, 
and evaluate transfer performance on 15 additional diverse languages, 5 of which have non-Latin writing systems.
This task presents a challenge due to the 60 possible slots,
requiring strictly following the output format for correct parsing.
For machine translation, we use ALMA \cite{DBLP:conf/iclr/Xu0SA24}'s training and test data, and additionally test on 6 zero-shot directions from \textsc{WMT 23} \cite{kocmi-etal-2023-findings}.

\noindent
\textbf{Challenge Task Data:}
To assess performance on long-sequence processing and structured text generation,
we include JSON generation as a challenge task.
We use the UNER dataset \cite{mayhew-etal-2024-universal} from the Aya collection \cite{singh-etal-2024-aya}, 
which requires following example instructions and extracting named entities into JSON format.
A challenge not present in the previous tasks is the longer inputs, with an average input length exceeding 150 tokens in English. 
For this task,  we train on 3 high-resource languages (en, pt, zh) and transfer to the 5 remaining languages.

\noindent
\textbf{Alignment Data:}
For alignment, we mainly use parallel data to English from Tatoeba \cite{tiedemann-2020-tatoeba}, %\footnote{\url{huggingface.co/datasets/Helsinki-NLP/tatoeba}} 
except for machine translation, where the training sentences are inherently parallel.
For slot filling, our main experiments align the five languages with the weakest baseline\footnote{their baseline is an XLM-R model trained on English} transfer performance (cy, jv, jp, sw, tl) reported by the dataset creators \cite{fitzgerald-etal-2023-massive}.
We choose them because their weak baseline performance suggests a lack of effective transfer, providing a strong testbed for evaluating the potential benefits of our alignment approach.
For ablation, we alter the following factors of the alignment data:
\begin{itemize}
    \item Resource level (low, medium, high-resource)
    \item Language coverage 
    \item Domain (oracle data, different, very distant)
\end{itemize}
For machine translation, given the inherent semantic equivalence of translation pairs, we directly leverage the translation data for alignment.
For JSON generation, we align the two lowest-resourced in UNER (da and sv)\footnote{While 
Serbian (sr) is also low-resourced in UNER, 
we exclude it from alignment due to data quality.
Running language identification reveals that many sentences in the Serbian alignment data are not actually in Serbian.} to English.
For lower-resource languages, the alignment data are a few hundreds as detailed in \autoref{sec:appendix_dataset_details}.

\subsection{Evaluation} \label{subsec:setup_evaluation}
\noindent
\textbf{Semantic Alignment Evaluation:}
As described in \S\ref{sec:analysis}, we evaluate cross-lingual semantic alignment by retrieval accuracy. 
Given $N$ languages, we perform many-to-many retrieval and average the accuracy over the $N(N-1)$ language pairs.
For the initial analyses (\S\ref{sec:analysis}), the 35 languages are listed in Appendix \ref{sec:appendix_language_list}.
We use the FLoRes-200 \cite{DBLP:journals/nature/Team24} development set with 997 parallel sentences. 
While FLoRes partially overlaps with ALMA's training data, 
it remains the only reliable massively multilingual multiway corpus to the best of our knowledge. 
Alternative such as Tatoeba have been advised against due to data imbalance and noise \cite{heffernan-etal-2022-bitext,janeiro2024mexmatokenlevelobjectivesimprove}. 
We also demonstrate that this overlap does not result in memorization effects (\S\ref{subsec:domain_lang_generalization}).
When reporting an aggregated retrieval accuracy for a model, 
we average over all language pairs at even-numbered layers' retrieval accuracy, 
excluding the input embedding layer.

\noindent
\textbf{Task Performance Evaluation:}
For slot filling, we report F$_1$ scores using the original evaluation script by \citet{fitzgerald-etal-2023-massive}.
For machine translation, we report BLEU\footnote{nrefs:1|case:mixed|eff:no|tok:13a|smooth:exp|version:2.4.2
sacreBLEU \cite{post-2018-call} signature, 
with "tok:ja-mecab-0.996-IPA" for Japanese and "tok:zh" for Chinese.} \cite{papineni-etal-2002-bleu} and COMET-22 \cite{rei-etal-2022-comet} scores.
For JSON generation, 
we parse the generated outputs back to named entity tuples and then evaluate F$_1$ scores.

\subsection{Model, Training, and Inference} 
We build upon Llama \cite{grattafiori2024llama3herdmodels} and Qwen \cite{qwen2025qwen25technicalreport}, specifically 
\texttt{Meta-Llama-3-8B-Instruct}\footnote{chosen over more recent versions to limit test set contamination, as its knowledge cutoff (March 2023) predates our translation test set (WMT 23).}
and \texttt{Qwen2.5-7B-Instruct}. 
We use LoRA \cite{DBLP:conf/iclr/HuSWALWWC22} adapters with a rank of 8 for all attention components and linear projections.
The effective batch size is 128 for both objectives, 
with mini-batches of 32 examples considered for the contrastive objective\footnote{While contrastive learning typically benefits from larger batch sizes  \cite{DBLP:conf/nips/ChenZXCD0TZC22}, our initial experiments with increased batch sizes did not give consistent improvements.}. 
Alignment data from different languages are re-sampled to an approximately uniform distribution. 
More details are in \autoref{sec:appendix_training_details}.

\section{Main Results} \label{sec:main_res}
% This must be in the first 5 lines to tell arXiv to use pdfLaTeX, which is strongly recommended.
\pdfoutput=1
% In particular, the hyperref package requires pdfLaTeX in order to break URLs across lines.

\documentclass[11pt]{article}

% Change "review" to "final" to generate the final (sometimes called camera-ready) version.
% Change to "preprint" to generate a non-anonymous version with page numbers.
\usepackage{acl}

% Standard package includes
\usepackage{times}
\usepackage{latexsym}

% Draw tables
\usepackage{booktabs}
\usepackage{multirow}
\usepackage{xcolor}
\usepackage{colortbl}
\usepackage{array} 
\usepackage{amsmath}

\newcolumntype{C}{>{\centering\arraybackslash}p{0.07\textwidth}}
% For proper rendering and hyphenation of words containing Latin characters (including in bib files)
\usepackage[T1]{fontenc}
% For Vietnamese characters
% \usepackage[T5]{fontenc}
% See https://www.latex-project.org/help/documentation/encguide.pdf for other character sets
% This assumes your files are encoded as UTF8
\usepackage[utf8]{inputenc}

% This is not strictly necessary, and may be commented out,
% but it will improve the layout of the manuscript,
% and will typically save some space.
\usepackage{microtype}
\DeclareMathOperator*{\argmax}{arg\,max}
% This is also not strictly necessary, and may be commented out.
% However, it will improve the aesthetics of text in
% the typewriter font.
\usepackage{inconsolata}

%Including images in your LaTeX document requires adding
%additional package(s)
\usepackage{graphicx}
% If the title and author information does not fit in the area allocated, uncomment the following
%
%\setlength\titlebox{<dim>}
%
% and set <dim> to something 5cm or larger.

\title{Wi-Chat: Large Language Model Powered Wi-Fi Sensing}

% Author information can be set in various styles:
% For several authors from the same institution:
% \author{Author 1 \and ... \and Author n \\
%         Address line \\ ... \\ Address line}
% if the names do not fit well on one line use
%         Author 1 \\ {\bf Author 2} \\ ... \\ {\bf Author n} \\
% For authors from different institutions:
% \author{Author 1 \\ Address line \\  ... \\ Address line
%         \And  ... \And
%         Author n \\ Address line \\ ... \\ Address line}
% To start a separate ``row'' of authors use \AND, as in
% \author{Author 1 \\ Address line \\  ... \\ Address line
%         \AND
%         Author 2 \\ Address line \\ ... \\ Address line \And
%         Author 3 \\ Address line \\ ... \\ Address line}

% \author{First Author \\
%   Affiliation / Address line 1 \\
%   Affiliation / Address line 2 \\
%   Affiliation / Address line 3 \\
%   \texttt{email@domain} \\\And
%   Second Author \\
%   Affiliation / Address line 1 \\
%   Affiliation / Address line 2 \\
%   Affiliation / Address line 3 \\
%   \texttt{email@domain} \\}
% \author{Haohan Yuan \qquad Haopeng Zhang\thanks{corresponding author} \\ 
%   ALOHA Lab, University of Hawaii at Manoa \\
%   % Affiliation / Address line 2 \\
%   % Affiliation / Address line 3 \\
%   \texttt{\{haohany,haopengz\}@hawaii.edu}}
  
\author{
{Haopeng Zhang$\dag$\thanks{These authors contributed equally to this work.}, Yili Ren$\ddagger$\footnotemark[1], Haohan Yuan$\dag$, Jingzhe Zhang$\ddagger$, Yitong Shen$\ddagger$} \\
ALOHA Lab, University of Hawaii at Manoa$\dag$, University of South Florida$\ddagger$ \\
\{haopengz, haohany\}@hawaii.edu\\
\{yiliren, jingzhe, shen202\}@usf.edu\\}



  
%\author{
%  \textbf{First Author\textsuperscript{1}},
%  \textbf{Second Author\textsuperscript{1,2}},
%  \textbf{Third T. Author\textsuperscript{1}},
%  \textbf{Fourth Author\textsuperscript{1}},
%\\
%  \textbf{Fifth Author\textsuperscript{1,2}},
%  \textbf{Sixth Author\textsuperscript{1}},
%  \textbf{Seventh Author\textsuperscript{1}},
%  \textbf{Eighth Author \textsuperscript{1,2,3,4}},
%\\
%  \textbf{Ninth Author\textsuperscript{1}},
%  \textbf{Tenth Author\textsuperscript{1}},
%  \textbf{Eleventh E. Author\textsuperscript{1,2,3,4,5}},
%  \textbf{Twelfth Author\textsuperscript{1}},
%\\
%  \textbf{Thirteenth Author\textsuperscript{3}},
%  \textbf{Fourteenth F. Author\textsuperscript{2,4}},
%  \textbf{Fifteenth Author\textsuperscript{1}},
%  \textbf{Sixteenth Author\textsuperscript{1}},
%\\
%  \textbf{Seventeenth S. Author\textsuperscript{4,5}},
%  \textbf{Eighteenth Author\textsuperscript{3,4}},
%  \textbf{Nineteenth N. Author\textsuperscript{2,5}},
%  \textbf{Twentieth Author\textsuperscript{1}}
%\\
%\\
%  \textsuperscript{1}Affiliation 1,
%  \textsuperscript{2}Affiliation 2,
%  \textsuperscript{3}Affiliation 3,
%  \textsuperscript{4}Affiliation 4,
%  \textsuperscript{5}Affiliation 5
%\\
%  \small{
%    \textbf{Correspondence:} \href{mailto:email@domain}{email@domain}
%  }
%}

\begin{document}
\maketitle
\begin{abstract}
Recent advancements in Large Language Models (LLMs) have demonstrated remarkable capabilities across diverse tasks. However, their potential to integrate physical model knowledge for real-world signal interpretation remains largely unexplored. In this work, we introduce Wi-Chat, the first LLM-powered Wi-Fi-based human activity recognition system. We demonstrate that LLMs can process raw Wi-Fi signals and infer human activities by incorporating Wi-Fi sensing principles into prompts. Our approach leverages physical model insights to guide LLMs in interpreting Channel State Information (CSI) data without traditional signal processing techniques. Through experiments on real-world Wi-Fi datasets, we show that LLMs exhibit strong reasoning capabilities, achieving zero-shot activity recognition. These findings highlight a new paradigm for Wi-Fi sensing, expanding LLM applications beyond conventional language tasks and enhancing the accessibility of wireless sensing for real-world deployments.
\end{abstract}

\section{Introduction}

In today’s rapidly evolving digital landscape, the transformative power of web technologies has redefined not only how services are delivered but also how complex tasks are approached. Web-based systems have become increasingly prevalent in risk control across various domains. This widespread adoption is due their accessibility, scalability, and ability to remotely connect various types of users. For example, these systems are used for process safety management in industry~\cite{kannan2016web}, safety risk early warning in urban construction~\cite{ding2013development}, and safe monitoring of infrastructural systems~\cite{repetto2018web}. Within these web-based risk management systems, the source search problem presents a huge challenge. Source search refers to the task of identifying the origin of a risky event, such as a gas leak and the emission point of toxic substances. This source search capability is crucial for effective risk management and decision-making.

Traditional approaches to implementing source search capabilities into the web systems often rely on solely algorithmic solutions~\cite{ristic2016study}. These methods, while relatively straightforward to implement, often struggle to achieve acceptable performances due to algorithmic local optima and complex unknown environments~\cite{zhao2020searching}. More recently, web crowdsourcing has emerged as a promising alternative for tackling the source search problem by incorporating human efforts in these web systems on-the-fly~\cite{zhao2024user}. This approach outsources the task of addressing issues encountered during the source search process to human workers, leveraging their capabilities to enhance system performance.

These solutions often employ a human-AI collaborative way~\cite{zhao2023leveraging} where algorithms handle exploration-exploitation and report the encountered problems while human workers resolve complex decision-making bottlenecks to help the algorithms getting rid of local deadlocks~\cite{zhao2022crowd}. Although effective, this paradigm suffers from two inherent limitations: increased operational costs from continuous human intervention, and slow response times of human workers due to sequential decision-making. These challenges motivate our investigation into developing autonomous systems that preserve human-like reasoning capabilities while reducing dependency on massive crowdsourced labor.

Furthermore, recent advancements in large language models (LLMs)~\cite{chang2024survey} and multi-modal LLMs (MLLMs)~\cite{huang2023chatgpt} have unveiled promising avenues for addressing these challenges. One clear opportunity involves the seamless integration of visual understanding and linguistic reasoning for robust decision-making in search tasks. However, whether large models-assisted source search is really effective and efficient for improving the current source search algorithms~\cite{ji2022source} remains unknown. \textit{To address the research gap, we are particularly interested in answering the following two research questions in this work:}

\textbf{\textit{RQ1: }}How can source search capabilities be integrated into web-based systems to support decision-making in time-sensitive risk management scenarios? 
% \sq{I mention ``time-sensitive'' here because I feel like we shall say something about the response time -- LLM has to be faster than humans}

\textbf{\textit{RQ2: }}How can MLLMs and LLMs enhance the effectiveness and efficiency of existing source search algorithms? 

% \textit{\textbf{RQ2:}} To what extent does the performance of large models-assisted search align with or approach the effectiveness of human-AI collaborative search? 

To answer the research questions, we propose a novel framework called Auto-\
S$^2$earch (\textbf{Auto}nomous \textbf{S}ource \textbf{Search}) and implement a prototype system that leverages advanced web technologies to simulate real-world conditions for zero-shot source search. Unlike traditional methods that rely on pre-defined heuristics or extensive human intervention, AutoS$^2$earch employs a carefully designed prompt that encapsulates human rationales, thereby guiding the MLLM to generate coherent and accurate scene descriptions from visual inputs about four directional choices. Based on these language-based descriptions, the LLM is enabled to determine the optimal directional choice through chain-of-thought (CoT) reasoning. Comprehensive empirical validation demonstrates that AutoS$^2$-\ 
earch achieves a success rate of 95–98\%, closely approaching the performance of human-AI collaborative search across 20 benchmark scenarios~\cite{zhao2023leveraging}. 

Our work indicates that the role of humans in future web crowdsourcing tasks may evolve from executors to validators or supervisors. Furthermore, incorporating explanations of LLM decisions into web-based system interfaces has the potential to help humans enhance task performance in risk control.






\section{Related Work}
\label{sec:relatedworks}

% \begin{table*}[t]
% \centering 
% \renewcommand\arraystretch{0.98}
% \fontsize{8}{10}\selectfont \setlength{\tabcolsep}{0.4em}
% \begin{tabular}{@{}lc|cc|cc|cc@{}}
% \toprule
% \textbf{Methods}           & \begin{tabular}[c]{@{}c@{}}\textbf{Training}\\ \textbf{Paradigm}\end{tabular} & \begin{tabular}[c]{@{}c@{}}\textbf{$\#$ PT Data}\\ \textbf{(Tokens)}\end{tabular} & \begin{tabular}[c]{@{}c@{}}\textbf{$\#$ IFT Data}\\ \textbf{(Samples)}\end{tabular} & \textbf{Code}  & \begin{tabular}[c]{@{}c@{}}\textbf{Natural}\\ \textbf{Language}\end{tabular} & \begin{tabular}[c]{@{}c@{}}\textbf{Action}\\ \textbf{Trajectories}\end{tabular} & \begin{tabular}[c]{@{}c@{}}\textbf{API}\\ \textbf{Documentation}\end{tabular}\\ \midrule 
% NexusRaven~\citep{srinivasan2023nexusraven} & IFT & - & - & \textcolor{green}{\CheckmarkBold} & \textcolor{green}{\CheckmarkBold} &\textcolor{red}{\XSolidBrush}&\textcolor{red}{\XSolidBrush}\\
% AgentInstruct~\citep{zeng2023agenttuning} & IFT & - & 2k & \textcolor{green}{\CheckmarkBold} & \textcolor{green}{\CheckmarkBold} &\textcolor{red}{\XSolidBrush}&\textcolor{red}{\XSolidBrush} \\
% AgentEvol~\citep{xi2024agentgym} & IFT & - & 14.5k & \textcolor{green}{\CheckmarkBold} & \textcolor{green}{\CheckmarkBold} &\textcolor{green}{\CheckmarkBold}&\textcolor{red}{\XSolidBrush} \\
% Gorilla~\citep{patil2023gorilla}& IFT & - & 16k & \textcolor{green}{\CheckmarkBold} & \textcolor{green}{\CheckmarkBold} &\textcolor{red}{\XSolidBrush}&\textcolor{green}{\CheckmarkBold}\\
% OpenFunctions-v2~\citep{patil2023gorilla} & IFT & - & 65k & \textcolor{green}{\CheckmarkBold} & \textcolor{green}{\CheckmarkBold} &\textcolor{red}{\XSolidBrush}&\textcolor{green}{\CheckmarkBold}\\
% LAM~\citep{zhang2024agentohana} & IFT & - & 42.6k & \textcolor{green}{\CheckmarkBold} & \textcolor{green}{\CheckmarkBold} &\textcolor{green}{\CheckmarkBold}&\textcolor{red}{\XSolidBrush} \\
% xLAM~\citep{liu2024apigen} & IFT & - & 60k & \textcolor{green}{\CheckmarkBold} & \textcolor{green}{\CheckmarkBold} &\textcolor{green}{\CheckmarkBold}&\textcolor{red}{\XSolidBrush} \\\midrule
% LEMUR~\citep{xu2024lemur} & PT & 90B & 300k & \textcolor{green}{\CheckmarkBold} & \textcolor{green}{\CheckmarkBold} &\textcolor{green}{\CheckmarkBold}&\textcolor{red}{\XSolidBrush}\\
% \rowcolor{teal!12} \method & PT & 103B & 95k & \textcolor{green}{\CheckmarkBold} & \textcolor{green}{\CheckmarkBold} & \textcolor{green}{\CheckmarkBold} & \textcolor{green}{\CheckmarkBold} \\
% \bottomrule
% \end{tabular}
% \caption{Summary of existing tuning- and pretraining-based LLM agents with their training sample sizes. "PT" and "IFT" denote "Pre-Training" and "Instruction Fine-Tuning", respectively. }
% \label{tab:related}
% \end{table*}

\begin{table*}[ht]
\begin{threeparttable}
\centering 
\renewcommand\arraystretch{0.98}
\fontsize{7}{9}\selectfont \setlength{\tabcolsep}{0.2em}
\begin{tabular}{@{}l|c|c|ccc|cc|cc|cccc@{}}
\toprule
\textbf{Methods} & \textbf{Datasets}           & \begin{tabular}[c]{@{}c@{}}\textbf{Training}\\ \textbf{Paradigm}\end{tabular} & \begin{tabular}[c]{@{}c@{}}\textbf{\# PT Data}\\ \textbf{(Tokens)}\end{tabular} & \begin{tabular}[c]{@{}c@{}}\textbf{\# IFT Data}\\ \textbf{(Samples)}\end{tabular} & \textbf{\# APIs} & \textbf{Code}  & \begin{tabular}[c]{@{}c@{}}\textbf{Nat.}\\ \textbf{Lang.}\end{tabular} & \begin{tabular}[c]{@{}c@{}}\textbf{Action}\\ \textbf{Traj.}\end{tabular} & \begin{tabular}[c]{@{}c@{}}\textbf{API}\\ \textbf{Doc.}\end{tabular} & \begin{tabular}[c]{@{}c@{}}\textbf{Func.}\\ \textbf{Call}\end{tabular} & \begin{tabular}[c]{@{}c@{}}\textbf{Multi.}\\ \textbf{Step}\end{tabular}  & \begin{tabular}[c]{@{}c@{}}\textbf{Plan}\\ \textbf{Refine}\end{tabular}  & \begin{tabular}[c]{@{}c@{}}\textbf{Multi.}\\ \textbf{Turn}\end{tabular}\\ \midrule 
\multicolumn{13}{l}{\emph{Instruction Finetuning-based LLM Agents for Intrinsic Reasoning}}  \\ \midrule
FireAct~\cite{chen2023fireact} & FireAct & IFT & - & 2.1K & 10 & \textcolor{red}{\XSolidBrush} &\textcolor{green}{\CheckmarkBold} &\textcolor{green}{\CheckmarkBold}  & \textcolor{red}{\XSolidBrush} &\textcolor{green}{\CheckmarkBold} & \textcolor{red}{\XSolidBrush} &\textcolor{green}{\CheckmarkBold} & \textcolor{red}{\XSolidBrush} \\
ToolAlpaca~\cite{tang2023toolalpaca} & ToolAlpaca & IFT & - & 4.0K & 400 & \textcolor{red}{\XSolidBrush} &\textcolor{green}{\CheckmarkBold} &\textcolor{green}{\CheckmarkBold} & \textcolor{red}{\XSolidBrush} &\textcolor{green}{\CheckmarkBold} & \textcolor{red}{\XSolidBrush}  &\textcolor{green}{\CheckmarkBold} & \textcolor{red}{\XSolidBrush}  \\
ToolLLaMA~\cite{qin2023toolllm} & ToolBench & IFT & - & 12.7K & 16,464 & \textcolor{red}{\XSolidBrush} &\textcolor{green}{\CheckmarkBold} &\textcolor{green}{\CheckmarkBold} &\textcolor{red}{\XSolidBrush} &\textcolor{green}{\CheckmarkBold}&\textcolor{green}{\CheckmarkBold}&\textcolor{green}{\CheckmarkBold} &\textcolor{green}{\CheckmarkBold}\\
AgentEvol~\citep{xi2024agentgym} & AgentTraj-L & IFT & - & 14.5K & 24 &\textcolor{red}{\XSolidBrush} & \textcolor{green}{\CheckmarkBold} &\textcolor{green}{\CheckmarkBold}&\textcolor{red}{\XSolidBrush} &\textcolor{green}{\CheckmarkBold}&\textcolor{red}{\XSolidBrush} &\textcolor{red}{\XSolidBrush} &\textcolor{green}{\CheckmarkBold}\\
Lumos~\cite{yin2024agent} & Lumos & IFT  & - & 20.0K & 16 &\textcolor{red}{\XSolidBrush} & \textcolor{green}{\CheckmarkBold} & \textcolor{green}{\CheckmarkBold} &\textcolor{red}{\XSolidBrush} & \textcolor{green}{\CheckmarkBold} & \textcolor{green}{\CheckmarkBold} &\textcolor{red}{\XSolidBrush} & \textcolor{green}{\CheckmarkBold}\\
Agent-FLAN~\cite{chen2024agent} & Agent-FLAN & IFT & - & 24.7K & 20 &\textcolor{red}{\XSolidBrush} & \textcolor{green}{\CheckmarkBold} & \textcolor{green}{\CheckmarkBold} &\textcolor{red}{\XSolidBrush} & \textcolor{green}{\CheckmarkBold}& \textcolor{green}{\CheckmarkBold}&\textcolor{red}{\XSolidBrush} & \textcolor{green}{\CheckmarkBold}\\
AgentTuning~\citep{zeng2023agenttuning} & AgentInstruct & IFT & - & 35.0K & - &\textcolor{red}{\XSolidBrush} & \textcolor{green}{\CheckmarkBold} & \textcolor{green}{\CheckmarkBold} &\textcolor{red}{\XSolidBrush} & \textcolor{green}{\CheckmarkBold} &\textcolor{red}{\XSolidBrush} &\textcolor{red}{\XSolidBrush} & \textcolor{green}{\CheckmarkBold}\\\midrule
\multicolumn{13}{l}{\emph{Instruction Finetuning-based LLM Agents for Function Calling}} \\\midrule
NexusRaven~\citep{srinivasan2023nexusraven} & NexusRaven & IFT & - & - & 116 & \textcolor{green}{\CheckmarkBold} & \textcolor{green}{\CheckmarkBold}  & \textcolor{green}{\CheckmarkBold} &\textcolor{red}{\XSolidBrush} & \textcolor{green}{\CheckmarkBold} &\textcolor{red}{\XSolidBrush} &\textcolor{red}{\XSolidBrush}&\textcolor{red}{\XSolidBrush}\\
Gorilla~\citep{patil2023gorilla} & Gorilla & IFT & - & 16.0K & 1,645 & \textcolor{green}{\CheckmarkBold} &\textcolor{red}{\XSolidBrush} &\textcolor{red}{\XSolidBrush}&\textcolor{green}{\CheckmarkBold} &\textcolor{green}{\CheckmarkBold} &\textcolor{red}{\XSolidBrush} &\textcolor{red}{\XSolidBrush} &\textcolor{red}{\XSolidBrush}\\
OpenFunctions-v2~\citep{patil2023gorilla} & OpenFunctions-v2 & IFT & - & 65.0K & - & \textcolor{green}{\CheckmarkBold} & \textcolor{green}{\CheckmarkBold} &\textcolor{red}{\XSolidBrush} &\textcolor{green}{\CheckmarkBold} &\textcolor{green}{\CheckmarkBold} &\textcolor{red}{\XSolidBrush} &\textcolor{red}{\XSolidBrush} &\textcolor{red}{\XSolidBrush}\\
API Pack~\cite{guo2024api} & API Pack & IFT & - & 1.1M & 11,213 &\textcolor{green}{\CheckmarkBold} &\textcolor{red}{\XSolidBrush} &\textcolor{green}{\CheckmarkBold} &\textcolor{red}{\XSolidBrush} &\textcolor{green}{\CheckmarkBold} &\textcolor{red}{\XSolidBrush}&\textcolor{red}{\XSolidBrush}&\textcolor{red}{\XSolidBrush}\\ 
LAM~\citep{zhang2024agentohana} & AgentOhana & IFT & - & 42.6K & - & \textcolor{green}{\CheckmarkBold} & \textcolor{green}{\CheckmarkBold} &\textcolor{green}{\CheckmarkBold}&\textcolor{red}{\XSolidBrush} &\textcolor{green}{\CheckmarkBold}&\textcolor{red}{\XSolidBrush}&\textcolor{green}{\CheckmarkBold}&\textcolor{green}{\CheckmarkBold}\\
xLAM~\citep{liu2024apigen} & APIGen & IFT & - & 60.0K & 3,673 & \textcolor{green}{\CheckmarkBold} & \textcolor{green}{\CheckmarkBold} &\textcolor{green}{\CheckmarkBold}&\textcolor{red}{\XSolidBrush} &\textcolor{green}{\CheckmarkBold}&\textcolor{red}{\XSolidBrush}&\textcolor{green}{\CheckmarkBold}&\textcolor{green}{\CheckmarkBold}\\\midrule
\multicolumn{13}{l}{\emph{Pretraining-based LLM Agents}}  \\\midrule
% LEMUR~\citep{xu2024lemur} & PT & 90B & 300.0K & - & \textcolor{green}{\CheckmarkBold} & \textcolor{green}{\CheckmarkBold} &\textcolor{green}{\CheckmarkBold}&\textcolor{red}{\XSolidBrush} & \textcolor{red}{\XSolidBrush} &\textcolor{green}{\CheckmarkBold} &\textcolor{red}{\XSolidBrush}&\textcolor{red}{\XSolidBrush}\\
\rowcolor{teal!12} \method & \dataset & PT & 103B & 95.0K  & 76,537  & \textcolor{green}{\CheckmarkBold} & \textcolor{green}{\CheckmarkBold} & \textcolor{green}{\CheckmarkBold} & \textcolor{green}{\CheckmarkBold} & \textcolor{green}{\CheckmarkBold} & \textcolor{green}{\CheckmarkBold} & \textcolor{green}{\CheckmarkBold} & \textcolor{green}{\CheckmarkBold}\\
\bottomrule
\end{tabular}
% \begin{tablenotes}
%     \item $^*$ In addition, the StarCoder-API can offer 4.77M more APIs.
% \end{tablenotes}
\caption{Summary of existing instruction finetuning-based LLM agents for intrinsic reasoning and function calling, along with their training resources and sample sizes. "PT" and "IFT" denote "Pre-Training" and "Instruction Fine-Tuning", respectively.}
\vspace{-2ex}
\label{tab:related}
\end{threeparttable}
\end{table*}

\noindent \textbf{Prompting-based LLM Agents.} Due to the lack of agent-specific pre-training corpus, existing LLM agents rely on either prompt engineering~\cite{hsieh2023tool,lu2024chameleon,yao2022react,wang2023voyager} or instruction fine-tuning~\cite{chen2023fireact,zeng2023agenttuning} to understand human instructions, decompose high-level tasks, generate grounded plans, and execute multi-step actions. 
However, prompting-based methods mainly depend on the capabilities of backbone LLMs (usually commercial LLMs), failing to introduce new knowledge and struggling to generalize to unseen tasks~\cite{sun2024adaplanner,zhuang2023toolchain}. 

\noindent \textbf{Instruction Finetuning-based LLM Agents.} Considering the extensive diversity of APIs and the complexity of multi-tool instructions, tool learning inherently presents greater challenges than natural language tasks, such as text generation~\cite{qin2023toolllm}.
Post-training techniques focus more on instruction following and aligning output with specific formats~\cite{patil2023gorilla,hao2024toolkengpt,qin2023toolllm,schick2024toolformer}, rather than fundamentally improving model knowledge or capabilities. 
Moreover, heavy fine-tuning can hinder generalization or even degrade performance in non-agent use cases, potentially suppressing the original base model capabilities~\cite{ghosh2024a}.

\noindent \textbf{Pretraining-based LLM Agents.} While pre-training serves as an essential alternative, prior works~\cite{nijkamp2023codegen,roziere2023code,xu2024lemur,patil2023gorilla} have primarily focused on improving task-specific capabilities (\eg, code generation) instead of general-domain LLM agents, due to single-source, uni-type, small-scale, and poor-quality pre-training data. 
Existing tool documentation data for agent training either lacks diverse real-world APIs~\cite{patil2023gorilla, tang2023toolalpaca} or is constrained to single-tool or single-round tool execution. 
Furthermore, trajectory data mostly imitate expert behavior or follow function-calling rules with inferior planning and reasoning, failing to fully elicit LLMs' capabilities and handle complex instructions~\cite{qin2023toolllm}. 
Given a wide range of candidate API functions, each comprising various function names and parameters available at every planning step, identifying globally optimal solutions and generalizing across tasks remains highly challenging.



\section{Preliminaries}
\label{Preliminaries}
\begin{figure*}[t]
    \centering
    \includegraphics[width=0.95\linewidth]{fig/HealthGPT_Framework.png}
    \caption{The \ourmethod{} architecture integrates hierarchical visual perception and H-LoRA, employing a task-specific hard router to select visual features and H-LoRA plugins, ultimately generating outputs with an autoregressive manner.}
    \label{fig:architecture}
\end{figure*}
\noindent\textbf{Large Vision-Language Models.} 
The input to a LVLM typically consists of an image $x^{\text{img}}$ and a discrete text sequence $x^{\text{txt}}$. The visual encoder $\mathcal{E}^{\text{img}}$ converts the input image $x^{\text{img}}$ into a sequence of visual tokens $\mathcal{V} = [v_i]_{i=1}^{N_v}$, while the text sequence $x^{\text{txt}}$ is mapped into a sequence of text tokens $\mathcal{T} = [t_i]_{i=1}^{N_t}$ using an embedding function $\mathcal{E}^{\text{txt}}$. The LLM $\mathcal{M_\text{LLM}}(\cdot|\theta)$ models the joint probability of the token sequence $\mathcal{U} = \{\mathcal{V},\mathcal{T}\}$, which is expressed as:
\begin{equation}
    P_\theta(R | \mathcal{U}) = \prod_{i=1}^{N_r} P_\theta(r_i | \{\mathcal{U}, r_{<i}\}),
\end{equation}
where $R = [r_i]_{i=1}^{N_r}$ is the text response sequence. The LVLM iteratively generates the next token $r_i$ based on $r_{<i}$. The optimization objective is to minimize the cross-entropy loss of the response $\mathcal{R}$.
% \begin{equation}
%     \mathcal{L}_{\text{VLM}} = \mathbb{E}_{R|\mathcal{U}}\left[-\log P_\theta(R | \mathcal{U})\right]
% \end{equation}
It is worth noting that most LVLMs adopt a design paradigm based on ViT, alignment adapters, and pre-trained LLMs\cite{liu2023llava,liu2024improved}, enabling quick adaptation to downstream tasks.


\noindent\textbf{VQGAN.}
VQGAN~\cite{esser2021taming} employs latent space compression and indexing mechanisms to effectively learn a complete discrete representation of images. VQGAN first maps the input image $x^{\text{img}}$ to a latent representation $z = \mathcal{E}(x)$ through a encoder $\mathcal{E}$. Then, the latent representation is quantized using a codebook $\mathcal{Z} = \{z_k\}_{k=1}^K$, generating a discrete index sequence $\mathcal{I} = [i_m]_{m=1}^N$, where $i_m \in \mathcal{Z}$ represents the quantized code index:
\begin{equation}
    \mathcal{I} = \text{Quantize}(z|\mathcal{Z}) = \arg\min_{z_k \in \mathcal{Z}} \| z - z_k \|_2.
\end{equation}
In our approach, the discrete index sequence $\mathcal{I}$ serves as a supervisory signal for the generation task, enabling the model to predict the index sequence $\hat{\mathcal{I}}$ from input conditions such as text or other modality signals.  
Finally, the predicted index sequence $\hat{\mathcal{I}}$ is upsampled by the VQGAN decoder $G$, generating the high-quality image $\hat{x}^\text{img} = G(\hat{\mathcal{I}})$.



\noindent\textbf{Low Rank Adaptation.} 
LoRA\cite{hu2021lora} effectively captures the characteristics of downstream tasks by introducing low-rank adapters. The core idea is to decompose the bypass weight matrix $\Delta W\in\mathbb{R}^{d^{\text{in}} \times d^{\text{out}}}$ into two low-rank matrices $ \{A \in \mathbb{R}^{d^{\text{in}} \times r}, B \in \mathbb{R}^{r \times d^{\text{out}}} \}$, where $ r \ll \min\{d^{\text{in}}, d^{\text{out}}\} $, significantly reducing learnable parameters. The output with the LoRA adapter for the input $x$ is then given by:
\begin{equation}
    h = x W_0 + \alpha x \Delta W/r = x W_0 + \alpha xAB/r,
\end{equation}
where matrix $ A $ is initialized with a Gaussian distribution, while the matrix $ B $ is initialized as a zero matrix. The scaling factor $ \alpha/r $ controls the impact of $ \Delta W $ on the model.

\section{HealthGPT}
\label{Method}


\subsection{Unified Autoregressive Generation.}  
% As shown in Figure~\ref{fig:architecture}, 
\ourmethod{} (Figure~\ref{fig:architecture}) utilizes a discrete token representation that covers both text and visual outputs, unifying visual comprehension and generation as an autoregressive task. 
For comprehension, $\mathcal{M}_\text{llm}$ receives the input joint sequence $\mathcal{U}$ and outputs a series of text token $\mathcal{R} = [r_1, r_2, \dots, r_{N_r}]$, where $r_i \in \mathcal{V}_{\text{txt}}$, and $\mathcal{V}_{\text{txt}}$ represents the LLM's vocabulary:
\begin{equation}
    P_\theta(\mathcal{R} \mid \mathcal{U}) = \prod_{i=1}^{N_r} P_\theta(r_i \mid \mathcal{U}, r_{<i}).
\end{equation}
For generation, $\mathcal{M}_\text{llm}$ first receives a special start token $\langle \text{START\_IMG} \rangle$, then generates a series of tokens corresponding to the VQGAN indices $\mathcal{I} = [i_1, i_2, \dots, i_{N_i}]$, where $i_j \in \mathcal{V}_{\text{vq}}$, and $\mathcal{V}_{\text{vq}}$ represents the index range of VQGAN. Upon completion of generation, the LLM outputs an end token $\langle \text{END\_IMG} \rangle$:
\begin{equation}
    P_\theta(\mathcal{I} \mid \mathcal{U}) = \prod_{j=1}^{N_i} P_\theta(i_j \mid \mathcal{U}, i_{<j}).
\end{equation}
Finally, the generated index sequence $\mathcal{I}$ is fed into the decoder $G$, which reconstructs the target image $\hat{x}^{\text{img}} = G(\mathcal{I})$.

\subsection{Hierarchical Visual Perception}  
Given the differences in visual perception between comprehension and generation tasks—where the former focuses on abstract semantics and the latter emphasizes complete semantics—we employ ViT to compress the image into discrete visual tokens at multiple hierarchical levels.
Specifically, the image is converted into a series of features $\{f_1, f_2, \dots, f_L\}$ as it passes through $L$ ViT blocks.

To address the needs of various tasks, the hidden states are divided into two types: (i) \textit{Concrete-grained features} $\mathcal{F}^{\text{Con}} = \{f_1, f_2, \dots, f_k\}, k < L$, derived from the shallower layers of ViT, containing sufficient global features, suitable for generation tasks; 
(ii) \textit{Abstract-grained features} $\mathcal{F}^{\text{Abs}} = \{f_{k+1}, f_{k+2}, \dots, f_L\}$, derived from the deeper layers of ViT, which contain abstract semantic information closer to the text space, suitable for comprehension tasks.

The task type $T$ (comprehension or generation) determines which set of features is selected as the input for the downstream large language model:
\begin{equation}
    \mathcal{F}^{\text{img}}_T =
    \begin{cases}
        \mathcal{F}^{\text{Con}}, & \text{if } T = \text{generation task} \\
        \mathcal{F}^{\text{Abs}}, & \text{if } T = \text{comprehension task}
    \end{cases}
\end{equation}
We integrate the image features $\mathcal{F}^{\text{img}}_T$ and text features $\mathcal{T}$ into a joint sequence through simple concatenation, which is then fed into the LLM $\mathcal{M}_{\text{llm}}$ for autoregressive generation.
% :
% \begin{equation}
%     \mathcal{R} = \mathcal{M}_{\text{llm}}(\mathcal{U}|\theta), \quad \mathcal{U} = [\mathcal{F}^{\text{img}}_T; \mathcal{T}]
% \end{equation}
\subsection{Heterogeneous Knowledge Adaptation}
We devise H-LoRA, which stores heterogeneous knowledge from comprehension and generation tasks in separate modules and dynamically routes to extract task-relevant knowledge from these modules. 
At the task level, for each task type $ T $, we dynamically assign a dedicated H-LoRA submodule $ \theta^T $, which is expressed as:
\begin{equation}
    \mathcal{R} = \mathcal{M}_\text{LLM}(\mathcal{U}|\theta, \theta^T), \quad \theta^T = \{A^T, B^T, \mathcal{R}^T_\text{outer}\}.
\end{equation}
At the feature level for a single task, H-LoRA integrates the idea of Mixture of Experts (MoE)~\cite{masoudnia2014mixture} and designs an efficient matrix merging and routing weight allocation mechanism, thus avoiding the significant computational delay introduced by matrix splitting in existing MoELoRA~\cite{luo2024moelora}. Specifically, we first merge the low-rank matrices (rank = r) of $ k $ LoRA experts into a unified matrix:
\begin{equation}
    \mathbf{A}^{\text{merged}}, \mathbf{B}^{\text{merged}} = \text{Concat}(\{A_i\}_1^k), \text{Concat}(\{B_i\}_1^k),
\end{equation}
where $ \mathbf{A}^{\text{merged}} \in \mathbb{R}^{d^\text{in} \times rk} $ and $ \mathbf{B}^{\text{merged}} \in \mathbb{R}^{rk \times d^\text{out}} $. The $k$-dimension routing layer generates expert weights $ \mathcal{W} \in \mathbb{R}^{\text{token\_num} \times k} $ based on the input hidden state $ x $, and these are expanded to $ \mathbb{R}^{\text{token\_num} \times rk} $ as follows:
\begin{equation}
    \mathcal{W}^\text{expanded} = \alpha k \mathcal{W} / r \otimes \mathbf{1}_r,
\end{equation}
where $ \otimes $ denotes the replication operation.
The overall output of H-LoRA is computed as:
\begin{equation}
    \mathcal{O}^\text{H-LoRA} = (x \mathbf{A}^{\text{merged}} \odot \mathcal{W}^\text{expanded}) \mathbf{B}^{\text{merged}},
\end{equation}
where $ \odot $ represents element-wise multiplication. Finally, the output of H-LoRA is added to the frozen pre-trained weights to produce the final output:
\begin{equation}
    \mathcal{O} = x W_0 + \mathcal{O}^\text{H-LoRA}.
\end{equation}
% In summary, H-LoRA is a task-based dynamic PEFT method that achieves high efficiency in single-task fine-tuning.

\subsection{Training Pipeline}

\begin{figure}[t]
    \centering
    \hspace{-4mm}
    \includegraphics[width=0.94\linewidth]{fig/data.pdf}
    \caption{Data statistics of \texttt{VL-Health}. }
    \label{fig:data}
\end{figure}
\noindent \textbf{1st Stage: Multi-modal Alignment.} 
In the first stage, we design separate visual adapters and H-LoRA submodules for medical unified tasks. For the medical comprehension task, we train abstract-grained visual adapters using high-quality image-text pairs to align visual embeddings with textual embeddings, thereby enabling the model to accurately describe medical visual content. During this process, the pre-trained LLM and its corresponding H-LoRA submodules remain frozen. In contrast, the medical generation task requires training concrete-grained adapters and H-LoRA submodules while keeping the LLM frozen. Meanwhile, we extend the textual vocabulary to include multimodal tokens, enabling the support of additional VQGAN vector quantization indices. The model trains on image-VQ pairs, endowing the pre-trained LLM with the capability for image reconstruction. This design ensures pixel-level consistency of pre- and post-LVLM. The processes establish the initial alignment between the LLM’s outputs and the visual inputs.

\noindent \textbf{2nd Stage: Heterogeneous H-LoRA Plugin Adaptation.}  
The submodules of H-LoRA share the word embedding layer and output head but may encounter issues such as bias and scale inconsistencies during training across different tasks. To ensure that the multiple H-LoRA plugins seamlessly interface with the LLMs and form a unified base, we fine-tune the word embedding layer and output head using a small amount of mixed data to maintain consistency in the model weights. Specifically, during this stage, all H-LoRA submodules for different tasks are kept frozen, with only the word embedding layer and output head being optimized. Through this stage, the model accumulates foundational knowledge for unified tasks by adapting H-LoRA plugins.

\begin{table*}[!t]
\centering
\caption{Comparison of \ourmethod{} with other LVLMs and unified multi-modal models on medical visual comprehension tasks. \textbf{Bold} and \underline{underlined} text indicates the best performance and second-best performance, respectively.}
\resizebox{\textwidth}{!}{
\begin{tabular}{c|lcc|cccccccc|c}
\toprule
\rowcolor[HTML]{E9F3FE} &  &  &  & \multicolumn{2}{c}{\textbf{VQA-RAD \textuparrow}} & \multicolumn{2}{c}{\textbf{SLAKE \textuparrow}} & \multicolumn{2}{c}{\textbf{PathVQA \textuparrow}} &  &  &  \\ 
\cline{5-10}
\rowcolor[HTML]{E9F3FE}\multirow{-2}{*}{\textbf{Type}} & \multirow{-2}{*}{\textbf{Model}} & \multirow{-2}{*}{\textbf{\# Params}} & \multirow{-2}{*}{\makecell{\textbf{Medical} \\ \textbf{LVLM}}} & \textbf{close} & \textbf{all} & \textbf{close} & \textbf{all} & \textbf{close} & \textbf{all} & \multirow{-2}{*}{\makecell{\textbf{MMMU} \\ \textbf{-Med}}\textuparrow} & \multirow{-2}{*}{\textbf{OMVQA}\textuparrow} & \multirow{-2}{*}{\textbf{Avg. \textuparrow}} \\ 
\midrule \midrule
\multirow{9}{*}{\textbf{Comp. Only}} 
& Med-Flamingo & 8.3B & \Large \ding{51} & 58.6 & 43.0 & 47.0 & 25.5 & 61.9 & 31.3 & 28.7 & 34.9 & 41.4 \\
& LLaVA-Med & 7B & \Large \ding{51} & 60.2 & 48.1 & 58.4 & 44.8 & 62.3 & 35.7 & 30.0 & 41.3 & 47.6 \\
& HuatuoGPT-Vision & 7B & \Large \ding{51} & 66.9 & 53.0 & 59.8 & 49.1 & 52.9 & 32.0 & 42.0 & 50.0 & 50.7 \\
& BLIP-2 & 6.7B & \Large \ding{55} & 43.4 & 36.8 & 41.6 & 35.3 & 48.5 & 28.8 & 27.3 & 26.9 & 36.1 \\
& LLaVA-v1.5 & 7B & \Large \ding{55} & 51.8 & 42.8 & 37.1 & 37.7 & 53.5 & 31.4 & 32.7 & 44.7 & 41.5 \\
& InstructBLIP & 7B & \Large \ding{55} & 61.0 & 44.8 & 66.8 & 43.3 & 56.0 & 32.3 & 25.3 & 29.0 & 44.8 \\
& Yi-VL & 6B & \Large \ding{55} & 52.6 & 42.1 & 52.4 & 38.4 & 54.9 & 30.9 & 38.0 & 50.2 & 44.9 \\
& InternVL2 & 8B & \Large \ding{55} & 64.9 & 49.0 & 66.6 & 50.1 & 60.0 & 31.9 & \underline{43.3} & 54.5 & 52.5\\
& Llama-3.2 & 11B & \Large \ding{55} & 68.9 & 45.5 & 72.4 & 52.1 & 62.8 & 33.6 & 39.3 & 63.2 & 54.7 \\
\midrule
\multirow{5}{*}{\textbf{Comp. \& Gen.}} 
& Show-o & 1.3B & \Large \ding{55} & 50.6 & 33.9 & 31.5 & 17.9 & 52.9 & 28.2 & 22.7 & 45.7 & 42.6 \\
& Unified-IO 2 & 7B & \Large \ding{55} & 46.2 & 32.6 & 35.9 & 21.9 & 52.5 & 27.0 & 25.3 & 33.0 & 33.8 \\
& Janus & 1.3B & \Large \ding{55} & 70.9 & 52.8 & 34.7 & 26.9 & 51.9 & 27.9 & 30.0 & 26.8 & 33.5 \\
& \cellcolor[HTML]{DAE0FB}HealthGPT-M3 & \cellcolor[HTML]{DAE0FB}3.8B & \cellcolor[HTML]{DAE0FB}\Large \ding{51} & \cellcolor[HTML]{DAE0FB}\underline{73.7} & \cellcolor[HTML]{DAE0FB}\underline{55.9} & \cellcolor[HTML]{DAE0FB}\underline{74.6} & \cellcolor[HTML]{DAE0FB}\underline{56.4} & \cellcolor[HTML]{DAE0FB}\underline{78.7} & \cellcolor[HTML]{DAE0FB}\underline{39.7} & \cellcolor[HTML]{DAE0FB}\underline{43.3} & \cellcolor[HTML]{DAE0FB}\underline{68.5} & \cellcolor[HTML]{DAE0FB}\underline{61.3} \\
& \cellcolor[HTML]{DAE0FB}HealthGPT-L14 & \cellcolor[HTML]{DAE0FB}14B & \cellcolor[HTML]{DAE0FB}\Large \ding{51} & \cellcolor[HTML]{DAE0FB}\textbf{77.7} & \cellcolor[HTML]{DAE0FB}\textbf{58.3} & \cellcolor[HTML]{DAE0FB}\textbf{76.4} & \cellcolor[HTML]{DAE0FB}\textbf{64.5} & \cellcolor[HTML]{DAE0FB}\textbf{85.9} & \cellcolor[HTML]{DAE0FB}\textbf{44.4} & \cellcolor[HTML]{DAE0FB}\textbf{49.2} & \cellcolor[HTML]{DAE0FB}\textbf{74.4} & \cellcolor[HTML]{DAE0FB}\textbf{66.4} \\
\bottomrule
\end{tabular}
}
\label{tab:results}
\end{table*}
\begin{table*}[ht]
    \centering
    \caption{The experimental results for the four modality conversion tasks.}
    \resizebox{\textwidth}{!}{
    \begin{tabular}{l|ccc|ccc|ccc|ccc}
        \toprule
        \rowcolor[HTML]{E9F3FE} & \multicolumn{3}{c}{\textbf{CT to MRI (Brain)}} & \multicolumn{3}{c}{\textbf{CT to MRI (Pelvis)}} & \multicolumn{3}{c}{\textbf{MRI to CT (Brain)}} & \multicolumn{3}{c}{\textbf{MRI to CT (Pelvis)}} \\
        \cline{2-13}
        \rowcolor[HTML]{E9F3FE}\multirow{-2}{*}{\textbf{Model}}& \textbf{SSIM $\uparrow$} & \textbf{PSNR $\uparrow$} & \textbf{MSE $\downarrow$} & \textbf{SSIM $\uparrow$} & \textbf{PSNR $\uparrow$} & \textbf{MSE $\downarrow$} & \textbf{SSIM $\uparrow$} & \textbf{PSNR $\uparrow$} & \textbf{MSE $\downarrow$} & \textbf{SSIM $\uparrow$} & \textbf{PSNR $\uparrow$} & \textbf{MSE $\downarrow$} \\
        \midrule \midrule
        pix2pix & 71.09 & 32.65 & 36.85 & 59.17 & 31.02 & 51.91 & 78.79 & 33.85 & 28.33 & 72.31 & 32.98 & 36.19 \\
        CycleGAN & 54.76 & 32.23 & 40.56 & 54.54 & 30.77 & 55.00 & 63.75 & 31.02 & 52.78 & 50.54 & 29.89 & 67.78 \\
        BBDM & {71.69} & {32.91} & {34.44} & 57.37 & 31.37 & 48.06 & \textbf{86.40} & 34.12 & 26.61 & {79.26} & 33.15 & 33.60 \\
        Vmanba & 69.54 & 32.67 & 36.42 & {63.01} & {31.47} & {46.99} & 79.63 & 34.12 & 26.49 & 77.45 & 33.53 & 31.85 \\
        DiffMa & 71.47 & 32.74 & 35.77 & 62.56 & 31.43 & 47.38 & 79.00 & {34.13} & {26.45} & 78.53 & {33.68} & {30.51} \\
        \rowcolor[HTML]{DAE0FB}HealthGPT-M3 & \underline{79.38} & \underline{33.03} & \underline{33.48} & \underline{71.81} & \underline{31.83} & \underline{43.45} & {85.06} & \textbf{34.40} & \textbf{25.49} & \underline{84.23} & \textbf{34.29} & \textbf{27.99} \\
        \rowcolor[HTML]{DAE0FB}HealthGPT-L14 & \textbf{79.73} & \textbf{33.10} & \textbf{32.96} & \textbf{71.92} & \textbf{31.87} & \textbf{43.09} & \underline{85.31} & \underline{34.29} & \underline{26.20} & \textbf{84.96} & \underline{34.14} & \underline{28.13} \\
        \bottomrule
    \end{tabular}
    }
    \label{tab:conversion}
\end{table*}

\noindent \textbf{3rd Stage: Visual Instruction Fine-Tuning.}  
In the third stage, we introduce additional task-specific data to further optimize the model and enhance its adaptability to downstream tasks such as medical visual comprehension (e.g., medical QA, medical dialogues, and report generation) or generation tasks (e.g., super-resolution, denoising, and modality conversion). Notably, by this stage, the word embedding layer and output head have been fine-tuned, only the H-LoRA modules and adapter modules need to be trained. This strategy significantly improves the model's adaptability and flexibility across different tasks.


\section{Experiment}
\label{s:experiment}

\subsection{Data Description}
We evaluate our method on FI~\cite{you2016building}, Twitter\_LDL~\cite{yang2017learning} and Artphoto~\cite{machajdik2010affective}.
FI is a public dataset built from Flickr and Instagram, with 23,308 images and eight emotion categories, namely \textit{amusement}, \textit{anger}, \textit{awe},  \textit{contentment}, \textit{disgust}, \textit{excitement},  \textit{fear}, and \textit{sadness}. 
% Since images in FI are all copyrighted by law, some images are corrupted now, so we remove these samples and retain 21,828 images.
% T4SA contains images from Twitter, which are classified into three categories: \textit{positive}, \textit{neutral}, and \textit{negative}. In this paper, we adopt the base version of B-T4SA, which contains 470,586 images and provides text descriptions of the corresponding tweets.
Twitter\_LDL contains 10,045 images from Twitter, with the same eight categories as the FI dataset.
% 。
For these two datasets, they are randomly split into 80\%
training and 20\% testing set.
Artphoto contains 806 artistic photos from the DeviantArt website, which we use to further evaluate the zero-shot capability of our model.
% on the small-scale dataset.
% We construct and publicly release the first image sentiment analysis dataset containing metadata.
% 。

% Based on these datasets, we are the first to construct and publicly release metadata-enhanced image sentiment analysis datasets. These datasets include scenes, tags, descriptions, and corresponding confidence scores, and are available at this link for future research purposes.


% 
\begin{table}[t]
\centering
% \begin{center}
\caption{Overall performance of different models on FI and Twitter\_LDL datasets.}
\label{tab:cap1}
% \resizebox{\linewidth}{!}
{
\begin{tabular}{l|c|c|c|c}
\hline
\multirow{2}{*}{\textbf{Model}} & \multicolumn{2}{c|}{\textbf{FI}}  & \multicolumn{2}{c}{\textbf{Twitter\_LDL}} \\ \cline{2-5} 
  & \textbf{Accuracy} & \textbf{F1} & \textbf{Accuracy} & \textbf{F1}  \\ \hline
% (\rownumber)~AlexNet~\cite{krizhevsky2017imagenet}  & 58.13\% & 56.35\%  & 56.24\%& 55.02\%  \\ 
% (\rownumber)~VGG16~\cite{simonyan2014very}  & 63.75\%& 63.08\%  & 59.34\%& 59.02\%  \\ 
(\rownumber)~ResNet101~\cite{he2016deep} & 66.16\%& 65.56\%  & 62.02\% & 61.34\%  \\ 
(\rownumber)~CDA~\cite{han2023boosting} & 66.71\%& 65.37\%  & 64.14\% & 62.85\%  \\ 
(\rownumber)~CECCN~\cite{ruan2024color} & 67.96\%& 66.74\%  & 64.59\%& 64.72\% \\ 
(\rownumber)~EmoVIT~\cite{xie2024emovit} & 68.09\%& 67.45\%  & 63.12\% & 61.97\%  \\ 
(\rownumber)~ComLDL~\cite{zhang2022compound} & 68.83\%& 67.28\%  & 65.29\% & 63.12\%  \\ 
(\rownumber)~WSDEN~\cite{li2023weakly} & 69.78\%& 69.61\%  & 67.04\% & 65.49\% \\ 
(\rownumber)~ECWA~\cite{deng2021emotion} & 70.87\%& 69.08\%  & 67.81\% & 66.87\%  \\ 
(\rownumber)~EECon~\cite{yang2023exploiting} & 71.13\%& 68.34\%  & 64.27\%& 63.16\%  \\ 
(\rownumber)~MAM~\cite{zhang2024affective} & 71.44\%  & 70.83\% & 67.18\%  & 65.01\%\\ 
(\rownumber)~TGCA-PVT~\cite{chen2024tgca}   & 73.05\%  & 71.46\% & 69.87\%  & 68.32\% \\ 
(\rownumber)~OEAN~\cite{zhang2024object}   & 73.40\%  & 72.63\% & 70.52\%  & 69.47\% \\ \hline
(\rownumber)~\shortname  & \textbf{79.48\%} & \textbf{79.22\%} & \textbf{74.12\%} & \textbf{73.09\%} \\ \hline
\end{tabular}
}
\vspace{-6mm}
% \end{center}
\end{table}
% 

\subsection{Experiment Setting}
% \subsubsection{Model Setting.}
% 
\textbf{Model Setting:}
For feature representation, we set $k=10$ to select object tags, and adopt clip-vit-base-patch32 as the pre-trained model for unified feature representation.
Moreover, we empirically set $(d_e, d_h, d_k, d_s) = (512, 128, 16, 64)$, and set the classification class $L$ to 8.

% 

\textbf{Training Setting:}
To initialize the model, we set all weights such as $\boldsymbol{W}$ following the truncated normal distribution, and use AdamW optimizer with the learning rate of $1 \times 10^{-4}$.
% warmup scheduler of cosine, warmup steps of 2000.
Furthermore, we set the batch size to 32 and the epoch of the training process to 200.
During the implementation, we utilize \textit{PyTorch} to build our entire model.
% , and our project codes are publicly available at https://github.com/zzmyrep/MESN.
% Our project codes as well as data are all publicly available on GitHub\footnote{https://github.com/zzmyrep/KBCEN}.
% Code is available at \href{https://github.com/zzmyrep/KBCEN}{https://github.com/zzmyrep/KBCEN}.

\textbf{Evaluation Metrics:}
Following~\cite{zhang2024affective, chen2024tgca, zhang2024object}, we adopt \textit{accuracy} and \textit{F1} as our evaluation metrics to measure the performance of different methods for image sentiment analysis. 



\subsection{Experiment Result}
% We compare our model against the following baselines: AlexNet~\cite{krizhevsky2017imagenet}, VGG16~\cite{simonyan2014very}, ResNet101~\cite{he2016deep}, CECCN~\cite{ruan2024color}, EmoVIT~\cite{xie2024emovit}, WSCNet~\cite{yang2018weakly}, ECWA~\cite{deng2021emotion}, EECon~\cite{yang2023exploiting}, MAM~\cite{zhang2024affective} and TGCA-PVT~\cite{chen2024tgca}, and the overall results are summarized in Table~\ref{tab:cap1}.
We compare our model against several baselines, and the overall results are summarized in Table~\ref{tab:cap1}.
We observe that our model achieves the best performance in both accuracy and F1 metrics, significantly outperforming the previous models. 
This superior performance is mainly attributed to our effective utilization of metadata to enhance image sentiment analysis, as well as the exceptional capability of the unified sentiment transformer framework we developed. These results strongly demonstrate that our proposed method can bring encouraging performance for image sentiment analysis.

\setcounter{magicrownumbers}{0} 
\begin{table}[t]
\begin{center}
\caption{Ablation study of~\shortname~on FI dataset.} 
% \vspace{1mm}
\label{tab:cap2}
\resizebox{.9\linewidth}{!}
{
\begin{tabular}{lcc}
  \hline
  \textbf{Model} & \textbf{Accuracy} & \textbf{F1} \\
  \hline
  (\rownumber)~Ours (w/o vision) & 65.72\% & 64.54\% \\
  (\rownumber)~Ours (w/o text description) & 74.05\% & 72.58\% \\
  (\rownumber)~Ours (w/o object tag) & 77.45\% & 76.84\% \\
  (\rownumber)~Ours (w/o scene tag) & 78.47\% & 78.21\% \\
  \hline
  (\rownumber)~Ours (w/o unified embedding) & 76.41\% & 76.23\% \\
  (\rownumber)~Ours (w/o adaptive learning) & 76.83\% & 76.56\% \\
  (\rownumber)~Ours (w/o cross-modal fusion) & 76.85\% & 76.49\% \\
  \hline
  (\rownumber)~Ours  & \textbf{79.48\%} & \textbf{79.22\%} \\
  \hline
\end{tabular}
}
\end{center}
\vspace{-5mm}
\end{table}


\begin{figure}[t]
\centering
% \vspace{-2mm}
\includegraphics[width=0.42\textwidth]{fig/2dvisual-linux4-paper2.pdf}
\caption{Visualization of feature distribution on eight categories before (left) and after (right) model processing.}
% 
\label{fig:visualization}
\vspace{-5mm}
\end{figure}

\subsection{Ablation Performance}
In this subsection, we conduct an ablation study to examine which component is really important for performance improvement. The results are reported in Table~\ref{tab:cap2}.

For information utilization, we observe a significant decline in model performance when visual features are removed. Additionally, the performance of \shortname~decreases when different metadata are removed separately, which means that text description, object tag, and scene tag are all critical for image sentiment analysis.
Recalling the model architecture, we separately remove transformer layers of the unified representation module, the adaptive learning module, and the cross-modal fusion module, replacing them with MLPs of the same parameter scale.
In this way, we can observe varying degrees of decline in model performance, indicating that these modules are indispensable for our model to achieve better performance.

\subsection{Visualization}
% 


% % 开始使用minipage进行左右排列
% \begin{minipage}[t]{0.45\textwidth}  % 子图1宽度为45%
%     \centering
%     \includegraphics[width=\textwidth]{2dvisual.pdf}  % 插入图片
%     \captionof{figure}{Visualization of feature distribution.}  % 使用captionof添加图片标题
%     \label{fig:visualization}
% \end{minipage}


% \begin{figure}[t]
% \centering
% \vspace{-2mm}
% \includegraphics[width=0.45\textwidth]{fig/2dvisual.pdf}
% \caption{Visualization of feature distribution.}
% \label{fig:visualization}
% % \vspace{-4mm}
% \end{figure}

% \begin{figure}[t]
% \centering
% \vspace{-2mm}
% \includegraphics[width=0.45\textwidth]{fig/2dvisual-linux3-paper.pdf}
% \caption{Visualization of feature distribution.}
% \label{fig:visualization}
% % \vspace{-4mm}
% \end{figure}



\begin{figure}[tbp]   
\vspace{-4mm}
  \centering            
  \subfloat[Depth of adaptive learning layers]   
  {
    \label{fig:subfig1}\includegraphics[width=0.22\textwidth]{fig/fig_sensitivity-a5}
  }
  \subfloat[Depth of fusion layers]
  {
    % \label{fig:subfig2}\includegraphics[width=0.22\textwidth]{fig/fig_sensitivity-b2}
    \label{fig:subfig2}\includegraphics[width=0.22\textwidth]{fig/fig_sensitivity-b2-num.pdf}
  }
  \caption{Sensitivity study of \shortname~on different depth. }   
  \label{fig:fig_sensitivity}  
\vspace{-2mm}
\end{figure}

% \begin{figure}[htbp]
% \centerline{\includegraphics{2dvisual.pdf}}
% \caption{Visualization of feature distribution.}
% \label{fig:visualization}
% \end{figure}

% In Fig.~\ref{fig:visualization}, we use t-SNE~\cite{van2008visualizing} to reduce the dimension of data features for visualization, Figure in left represents the metadata features before model processing, the features are obtained by embedding through the CLIP model, and figure in right shows the features of the data after model processing, it can be observed that after the model processing, the data with different label categories fall in different regions in the space, therefore, we can conclude that the Therefore, we can conclude that the model can effectively utilize the information contained in the metadata and use it to guide the model for classification.

In Fig.~\ref{fig:visualization}, we use t-SNE~\cite{van2008visualizing} to reduce the dimension of data features for visualization.
The left figure shows metadata features before being processed by our model (\textit{i.e.}, embedded by CLIP), while the right shows the distribution of features after being processed by our model.
We can observe that after the model processing, data with the same label are closer to each other, while others are farther away.
Therefore, it shows that the model can effectively utilize the information contained in the metadata and use it to guide the classification process.

\subsection{Sensitivity Analysis}
% 
In this subsection, we conduct a sensitivity analysis to figure out the effect of different depth settings of adaptive learning layers and fusion layers. 
% In this subsection, we conduct a sensitivity analysis to figure out the effect of different depth settings on the model. 
% Fig.~\ref{fig:fig_sensitivity} presents the effect of different depth settings of adaptive learning layers and fusion layers. 
Taking Fig.~\ref{fig:fig_sensitivity} (a) as an example, the model performance improves with increasing depth, reaching the best performance at a depth of 4.
% Taking Fig.~\ref{fig:fig_sensitivity} (a) as an example, the performance of \shortname~improves with the increase of depth at first, reaching the best performance at a depth of 4.
When the depth continues to increase, the accuracy decreases to varying degrees.
Similar results can be observed in Fig.~\ref{fig:fig_sensitivity} (b).
Therefore, we set their depths to 4 and 6 respectively to achieve the best results.

% Through our experiments, we can observe that the effect of modifying these hyperparameters on the results of the experiments is very weak, and the surface model is not sensitive to the hyperparameters.


\subsection{Zero-shot Capability}
% 

% (1)~GCH~\cite{2010Analyzing} & 21.78\% & (5)~RA-DLNet~\cite{2020A} & 34.01\% \\ \hline
% (2)~WSCNet~\cite{2019WSCNet}  & 30.25\% & (6)~CECCN~\cite{ruan2024color} & 43.83\% \\ \hline
% (3)~PCNN~\cite{2015Robust} & 31.68\%  & (7)~EmoVIT~\cite{xie2024emovit} & 44.90\% \\ \hline
% (4)~AR~\cite{2018Visual} & 32.67\% & (8)~Ours (Zero-shot) & 47.83\% \\ \hline


\begin{table}[t]
\centering
\caption{Zero-shot capability of \shortname.}
\label{tab:cap3}
\resizebox{1\linewidth}{!}
{
\begin{tabular}{lc|lc}
\hline
\textbf{Model} & \textbf{Accuracy} & \textbf{Model} & \textbf{Accuracy} \\ \hline
(1)~WSCNet~\cite{2019WSCNet}  & 30.25\% & (5)~MAM~\cite{zhang2024affective} & 39.56\%  \\ \hline
(2)~AR~\cite{2018Visual} & 32.67\% & (6)~CECCN~\cite{ruan2024color} & 43.83\% \\ \hline
(3)~RA-DLNet~\cite{2020A} & 34.01\%  & (7)~EmoVIT~\cite{xie2024emovit} & 44.90\% \\ \hline
(4)~CDA~\cite{han2023boosting} & 38.64\% & (8)~Ours (Zero-shot) & 47.83\% \\ \hline
\end{tabular}
}
\vspace{-5mm}
\end{table}

% We use the model trained on the FI dataset to test on the artphoto dataset to verify the model's generalization ability as well as robustness to other distributed datasets.
% We can observe that the MESN model shows strong competitiveness in terms of accuracy when compared to other trained models, which suggests that the model has a good generalization ability in the OOD task.

To validate the model's generalization ability and robustness to other distributed datasets, we directly test the model trained on the FI dataset, without training on Artphoto. 
% As observed in Table 3, compared to other models trained on Artphoto, we achieve highly competitive zero-shot performance, indicating that the model has good generalization ability in out-of-distribution tasks.
From Table~\ref{tab:cap3}, we can observe that compared with other models trained on Artphoto, we achieve competitive zero-shot performance, which shows that the model has good generalization ability in out-of-distribution tasks.


%%%%%%%%%%%%
%  E2E     %
%%%%%%%%%%%%


\section{Conclusion}
In this paper, we introduced Wi-Chat, the first LLM-powered Wi-Fi-based human activity recognition system that integrates the reasoning capabilities of large language models with the sensing potential of wireless signals. Our experimental results on a self-collected Wi-Fi CSI dataset demonstrate the promising potential of LLMs in enabling zero-shot Wi-Fi sensing. These findings suggest a new paradigm for human activity recognition that does not rely on extensive labeled data. We hope future research will build upon this direction, further exploring the applications of LLMs in signal processing domains such as IoT, mobile sensing, and radar-based systems.

\section*{Limitations}
While our work represents the first attempt to leverage LLMs for processing Wi-Fi signals, it is a preliminary study focused on a relatively simple task: Wi-Fi-based human activity recognition. This choice allows us to explore the feasibility of LLMs in wireless sensing but also comes with certain limitations.

Our approach primarily evaluates zero-shot performance, which, while promising, may still lag behind traditional supervised learning methods in highly complex or fine-grained recognition tasks. Besides, our study is limited to a controlled environment with a self-collected dataset, and the generalizability of LLMs to diverse real-world scenarios with varying Wi-Fi conditions, environmental interference, and device heterogeneity remains an open question.

Additionally, we have yet to explore the full potential of LLMs in more advanced Wi-Fi sensing applications, such as fine-grained gesture recognition, occupancy detection, and passive health monitoring. Future work should investigate the scalability of LLM-based approaches, their robustness to domain shifts, and their integration with multimodal sensing techniques in broader IoT applications.


% Bibliography entries for the entire Anthology, followed by custom entries
%\bibliography{anthology,custom}
% Custom bibliography entries only
\bibliography{main}
\newpage
\appendix

\section{Experiment prompts}
\label{sec:prompt}
The prompts used in the LLM experiments are shown in the following Table~\ref{tab:prompts}.

\definecolor{titlecolor}{rgb}{0.9, 0.5, 0.1}
\definecolor{anscolor}{rgb}{0.2, 0.5, 0.8}
\definecolor{labelcolor}{HTML}{48a07e}
\begin{table*}[h]
	\centering
	
 % \vspace{-0.2cm}
	
	\begin{center}
		\begin{tikzpicture}[
				chatbox_inner/.style={rectangle, rounded corners, opacity=0, text opacity=1, font=\sffamily\scriptsize, text width=5in, text height=9pt, inner xsep=6pt, inner ysep=6pt},
				chatbox_prompt_inner/.style={chatbox_inner, align=flush left, xshift=0pt, text height=11pt},
				chatbox_user_inner/.style={chatbox_inner, align=flush left, xshift=0pt},
				chatbox_gpt_inner/.style={chatbox_inner, align=flush left, xshift=0pt},
				chatbox/.style={chatbox_inner, draw=black!25, fill=gray!7, opacity=1, text opacity=0},
				chatbox_prompt/.style={chatbox, align=flush left, fill=gray!1.5, draw=black!30, text height=10pt},
				chatbox_user/.style={chatbox, align=flush left},
				chatbox_gpt/.style={chatbox, align=flush left},
				chatbox2/.style={chatbox_gpt, fill=green!25},
				chatbox3/.style={chatbox_gpt, fill=red!20, draw=black!20},
				chatbox4/.style={chatbox_gpt, fill=yellow!30},
				labelbox/.style={rectangle, rounded corners, draw=black!50, font=\sffamily\scriptsize\bfseries, fill=gray!5, inner sep=3pt},
			]
											
			\node[chatbox_user] (q1) {
				\textbf{System prompt}
				\newline
				\newline
				You are a helpful and precise assistant for segmenting and labeling sentences. We would like to request your help on curating a dataset for entity-level hallucination detection.
				\newline \newline
                We will give you a machine generated biography and a list of checked facts about the biography. Each fact consists of a sentence and a label (True/False). Please do the following process. First, breaking down the biography into words. Second, by referring to the provided list of facts, merging some broken down words in the previous step to form meaningful entities. For example, ``strategic thinking'' should be one entity instead of two. Third, according to the labels in the list of facts, labeling each entity as True or False. Specifically, for facts that share a similar sentence structure (\eg, \textit{``He was born on Mach 9, 1941.''} (\texttt{True}) and \textit{``He was born in Ramos Mejia.''} (\texttt{False})), please first assign labels to entities that differ across atomic facts. For example, first labeling ``Mach 9, 1941'' (\texttt{True}) and ``Ramos Mejia'' (\texttt{False}) in the above case. For those entities that are the same across atomic facts (\eg, ``was born'') or are neutral (\eg, ``he,'' ``in,'' and ``on''), please label them as \texttt{True}. For the cases that there is no atomic fact that shares the same sentence structure, please identify the most informative entities in the sentence and label them with the same label as the atomic fact while treating the rest of the entities as \texttt{True}. In the end, output the entities and labels in the following format:
                \begin{itemize}[nosep]
                    \item Entity 1 (Label 1)
                    \item Entity 2 (Label 2)
                    \item ...
                    \item Entity N (Label N)
                \end{itemize}
                % \newline \newline
                Here are two examples:
                \newline\newline
                \textbf{[Example 1]}
                \newline
                [The start of the biography]
                \newline
                \textcolor{titlecolor}{Marianne McAndrew is an American actress and singer, born on November 21, 1942, in Cleveland, Ohio. She began her acting career in the late 1960s, appearing in various television shows and films.}
                \newline
                [The end of the biography]
                \newline \newline
                [The start of the list of checked facts]
                \newline
                \textcolor{anscolor}{[Marianne McAndrew is an American. (False); Marianne McAndrew is an actress. (True); Marianne McAndrew is a singer. (False); Marianne McAndrew was born on November 21, 1942. (False); Marianne McAndrew was born in Cleveland, Ohio. (False); She began her acting career in the late 1960s. (True); She has appeared in various television shows. (True); She has appeared in various films. (True)]}
                \newline
                [The end of the list of checked facts]
                \newline \newline
                [The start of the ideal output]
                \newline
                \textcolor{labelcolor}{[Marianne McAndrew (True); is (True); an (True); American (False); actress (True); and (True); singer (False); , (True); born (True); on (True); November 21, 1942 (False); , (True); in (True); Cleveland, Ohio (False); . (True); She (True); began (True); her (True); acting career (True); in (True); the late 1960s (True); , (True); appearing (True); in (True); various (True); television shows (True); and (True); films (True); . (True)]}
                \newline
                [The end of the ideal output]
				\newline \newline
                \textbf{[Example 2]}
                \newline
                [The start of the biography]
                \newline
                \textcolor{titlecolor}{Doug Sheehan is an American actor who was born on April 27, 1949, in Santa Monica, California. He is best known for his roles in soap operas, including his portrayal of Joe Kelly on ``General Hospital'' and Ben Gibson on ``Knots Landing.''}
                \newline
                [The end of the biography]
                \newline \newline
                [The start of the list of checked facts]
                \newline
                \textcolor{anscolor}{[Doug Sheehan is an American. (True); Doug Sheehan is an actor. (True); Doug Sheehan was born on April 27, 1949. (True); Doug Sheehan was born in Santa Monica, California. (False); He is best known for his roles in soap operas. (True); He portrayed Joe Kelly. (True); Joe Kelly was in General Hospital. (True); General Hospital is a soap opera. (True); He portrayed Ben Gibson. (True); Ben Gibson was in Knots Landing. (True); Knots Landing is a soap opera. (True)]}
                \newline
                [The end of the list of checked facts]
                \newline \newline
                [The start of the ideal output]
                \newline
                \textcolor{labelcolor}{[Doug Sheehan (True); is (True); an (True); American (True); actor (True); who (True); was born (True); on (True); April 27, 1949 (True); in (True); Santa Monica, California (False); . (True); He (True); is (True); best known (True); for (True); his roles in soap operas (True); , (True); including (True); in (True); his portrayal (True); of (True); Joe Kelly (True); on (True); ``General Hospital'' (True); and (True); Ben Gibson (True); on (True); ``Knots Landing.'' (True)]}
                \newline
                [The end of the ideal output]
				\newline \newline
				\textbf{User prompt}
				\newline
				\newline
				[The start of the biography]
				\newline
				\textcolor{magenta}{\texttt{\{BIOGRAPHY\}}}
				\newline
				[The ebd of the biography]
				\newline \newline
				[The start of the list of checked facts]
				\newline
				\textcolor{magenta}{\texttt{\{LIST OF CHECKED FACTS\}}}
				\newline
				[The end of the list of checked facts]
			};
			\node[chatbox_user_inner] (q1_text) at (q1) {
				\textbf{System prompt}
				\newline
				\newline
				You are a helpful and precise assistant for segmenting and labeling sentences. We would like to request your help on curating a dataset for entity-level hallucination detection.
				\newline \newline
                We will give you a machine generated biography and a list of checked facts about the biography. Each fact consists of a sentence and a label (True/False). Please do the following process. First, breaking down the biography into words. Second, by referring to the provided list of facts, merging some broken down words in the previous step to form meaningful entities. For example, ``strategic thinking'' should be one entity instead of two. Third, according to the labels in the list of facts, labeling each entity as True or False. Specifically, for facts that share a similar sentence structure (\eg, \textit{``He was born on Mach 9, 1941.''} (\texttt{True}) and \textit{``He was born in Ramos Mejia.''} (\texttt{False})), please first assign labels to entities that differ across atomic facts. For example, first labeling ``Mach 9, 1941'' (\texttt{True}) and ``Ramos Mejia'' (\texttt{False}) in the above case. For those entities that are the same across atomic facts (\eg, ``was born'') or are neutral (\eg, ``he,'' ``in,'' and ``on''), please label them as \texttt{True}. For the cases that there is no atomic fact that shares the same sentence structure, please identify the most informative entities in the sentence and label them with the same label as the atomic fact while treating the rest of the entities as \texttt{True}. In the end, output the entities and labels in the following format:
                \begin{itemize}[nosep]
                    \item Entity 1 (Label 1)
                    \item Entity 2 (Label 2)
                    \item ...
                    \item Entity N (Label N)
                \end{itemize}
                % \newline \newline
                Here are two examples:
                \newline\newline
                \textbf{[Example 1]}
                \newline
                [The start of the biography]
                \newline
                \textcolor{titlecolor}{Marianne McAndrew is an American actress and singer, born on November 21, 1942, in Cleveland, Ohio. She began her acting career in the late 1960s, appearing in various television shows and films.}
                \newline
                [The end of the biography]
                \newline \newline
                [The start of the list of checked facts]
                \newline
                \textcolor{anscolor}{[Marianne McAndrew is an American. (False); Marianne McAndrew is an actress. (True); Marianne McAndrew is a singer. (False); Marianne McAndrew was born on November 21, 1942. (False); Marianne McAndrew was born in Cleveland, Ohio. (False); She began her acting career in the late 1960s. (True); She has appeared in various television shows. (True); She has appeared in various films. (True)]}
                \newline
                [The end of the list of checked facts]
                \newline \newline
                [The start of the ideal output]
                \newline
                \textcolor{labelcolor}{[Marianne McAndrew (True); is (True); an (True); American (False); actress (True); and (True); singer (False); , (True); born (True); on (True); November 21, 1942 (False); , (True); in (True); Cleveland, Ohio (False); . (True); She (True); began (True); her (True); acting career (True); in (True); the late 1960s (True); , (True); appearing (True); in (True); various (True); television shows (True); and (True); films (True); . (True)]}
                \newline
                [The end of the ideal output]
				\newline \newline
                \textbf{[Example 2]}
                \newline
                [The start of the biography]
                \newline
                \textcolor{titlecolor}{Doug Sheehan is an American actor who was born on April 27, 1949, in Santa Monica, California. He is best known for his roles in soap operas, including his portrayal of Joe Kelly on ``General Hospital'' and Ben Gibson on ``Knots Landing.''}
                \newline
                [The end of the biography]
                \newline \newline
                [The start of the list of checked facts]
                \newline
                \textcolor{anscolor}{[Doug Sheehan is an American. (True); Doug Sheehan is an actor. (True); Doug Sheehan was born on April 27, 1949. (True); Doug Sheehan was born in Santa Monica, California. (False); He is best known for his roles in soap operas. (True); He portrayed Joe Kelly. (True); Joe Kelly was in General Hospital. (True); General Hospital is a soap opera. (True); He portrayed Ben Gibson. (True); Ben Gibson was in Knots Landing. (True); Knots Landing is a soap opera. (True)]}
                \newline
                [The end of the list of checked facts]
                \newline \newline
                [The start of the ideal output]
                \newline
                \textcolor{labelcolor}{[Doug Sheehan (True); is (True); an (True); American (True); actor (True); who (True); was born (True); on (True); April 27, 1949 (True); in (True); Santa Monica, California (False); . (True); He (True); is (True); best known (True); for (True); his roles in soap operas (True); , (True); including (True); in (True); his portrayal (True); of (True); Joe Kelly (True); on (True); ``General Hospital'' (True); and (True); Ben Gibson (True); on (True); ``Knots Landing.'' (True)]}
                \newline
                [The end of the ideal output]
				\newline \newline
				\textbf{User prompt}
				\newline
				\newline
				[The start of the biography]
				\newline
				\textcolor{magenta}{\texttt{\{BIOGRAPHY\}}}
				\newline
				[The ebd of the biography]
				\newline \newline
				[The start of the list of checked facts]
				\newline
				\textcolor{magenta}{\texttt{\{LIST OF CHECKED FACTS\}}}
				\newline
				[The end of the list of checked facts]
			};
		\end{tikzpicture}
        \caption{GPT-4o prompt for labeling hallucinated entities.}\label{tb:gpt-4-prompt}
	\end{center}
\vspace{-0cm}
\end{table*}
% \section{Full Experiment Results}
% \begin{table*}[th]
    \centering
    \small
    \caption{Classification Results}
    \begin{tabular}{lcccc}
        \toprule
        \textbf{Method} & \textbf{Accuracy} & \textbf{Precision} & \textbf{Recall} & \textbf{F1-score} \\
        \midrule
        \multicolumn{5}{c}{\textbf{Zero Shot}} \\
                Zero-shot E-eyes & 0.26 & 0.26 & 0.27 & 0.26 \\
        Zero-shot CARM & 0.24 & 0.24 & 0.24 & 0.24 \\
                Zero-shot SVM & 0.27 & 0.28 & 0.28 & 0.27 \\
        Zero-shot CNN & 0.23 & 0.24 & 0.23 & 0.23 \\
        Zero-shot RNN & 0.26 & 0.26 & 0.26 & 0.26 \\
DeepSeek-0shot & 0.54 & 0.61 & 0.54 & 0.52 \\
DeepSeek-0shot-COT & 0.33 & 0.24 & 0.33 & 0.23 \\
DeepSeek-0shot-Knowledge & 0.45 & 0.46 & 0.45 & 0.44 \\
Gemma2-0shot & 0.35 & 0.22 & 0.38 & 0.27 \\
Gemma2-0shot-COT & 0.36 & 0.22 & 0.36 & 0.27 \\
Gemma2-0shot-Knowledge & 0.32 & 0.18 & 0.34 & 0.20 \\
GPT-4o-mini-0shot & 0.48 & 0.53 & 0.48 & 0.41 \\
GPT-4o-mini-0shot-COT & 0.33 & 0.50 & 0.33 & 0.38 \\
GPT-4o-mini-0shot-Knowledge & 0.49 & 0.31 & 0.49 & 0.36 \\
GPT-4o-0shot & 0.62 & 0.62 & 0.47 & 0.42 \\
GPT-4o-0shot-COT & 0.29 & 0.45 & 0.29 & 0.21 \\
GPT-4o-0shot-Knowledge & 0.44 & 0.52 & 0.44 & 0.39 \\
LLaMA-0shot & 0.32 & 0.25 & 0.32 & 0.24 \\
LLaMA-0shot-COT & 0.12 & 0.25 & 0.12 & 0.09 \\
LLaMA-0shot-Knowledge & 0.32 & 0.25 & 0.32 & 0.28 \\
Mistral-0shot & 0.19 & 0.23 & 0.19 & 0.10 \\
Mistral-0shot-Knowledge & 0.21 & 0.40 & 0.21 & 0.11 \\
        \midrule
        \multicolumn{5}{c}{\textbf{4 Shot}} \\
GPT-4o-mini-4shot & 0.58 & 0.59 & 0.58 & 0.53 \\
GPT-4o-mini-4shot-COT & 0.57 & 0.53 & 0.57 & 0.50 \\
GPT-4o-mini-4shot-Knowledge & 0.56 & 0.51 & 0.56 & 0.47 \\
GPT-4o-4shot & 0.77 & 0.84 & 0.77 & 0.73 \\
GPT-4o-4shot-COT & 0.63 & 0.76 & 0.63 & 0.53 \\
GPT-4o-4shot-Knowledge & 0.72 & 0.82 & 0.71 & 0.66 \\
LLaMA-4shot & 0.29 & 0.24 & 0.29 & 0.21 \\
LLaMA-4shot-COT & 0.20 & 0.30 & 0.20 & 0.13 \\
LLaMA-4shot-Knowledge & 0.15 & 0.23 & 0.13 & 0.13 \\
Mistral-4shot & 0.02 & 0.02 & 0.02 & 0.02 \\
Mistral-4shot-Knowledge & 0.21 & 0.27 & 0.21 & 0.20 \\
        \midrule
        
        \multicolumn{5}{c}{\textbf{Suprevised}} \\
        SVM & 0.94 & 0.92 & 0.91 & 0.91 \\
        CNN & 0.98 & 0.98 & 0.97 & 0.97 \\
        RNN & 0.99 & 0.99 & 0.99 & 0.99 \\
        % \midrule
        % \multicolumn{5}{c}{\textbf{Conventional Wi-Fi-based Human Activity Recognition Systems}} \\
        E-eyes & 1.00 & 1.00 & 1.00 & 1.00 \\
        CARM & 0.98 & 0.98 & 0.98 & 0.98 \\
\midrule
 \multicolumn{5}{c}{\textbf{Vision Models}} \\
           Zero-shot SVM & 0.26 & 0.25 & 0.25 & 0.25 \\
        Zero-shot CNN & 0.26 & 0.25 & 0.26 & 0.26 \\
        Zero-shot RNN & 0.28 & 0.28 & 0.29 & 0.28 \\
        SVM & 0.99 & 0.99 & 0.99 & 0.99 \\
        CNN & 0.98 & 0.99 & 0.98 & 0.98 \\
        RNN & 0.98 & 0.99 & 0.98 & 0.98 \\
GPT-4o-mini-Vision & 0.84 & 0.85 & 0.84 & 0.84 \\
GPT-4o-mini-Vision-COT & 0.90 & 0.91 & 0.90 & 0.90 \\
GPT-4o-Vision & 0.74 & 0.82 & 0.74 & 0.73 \\
GPT-4o-Vision-COT & 0.70 & 0.83 & 0.70 & 0.68 \\
LLaMA-Vision & 0.20 & 0.23 & 0.20 & 0.09 \\
LLaMA-Vision-Knowledge & 0.22 & 0.05 & 0.22 & 0.08 \\

        \bottomrule
    \end{tabular}
    \label{full}
\end{table*}




\end{document}


The main results are summarized in \autoref{tab:overall_results}. 
Before assessing our proposed approach, 
we first establish the necessity of supervised FT by comparing it with zero-shot usage of the LLMs (rows $(2,5)$ vs. $(1,4)$).
On slot filling, the zero-shot performance of Llama 3 is very poor, 
achieving only 6.6\% F$_1$ on English due to difficulties in adhering to task-specific formats.
We therefore do not evaluate its zero-shot performance on all languages.
In machine translation, supervised fine-tuning shows substantial gains of 4-6 COMET over zero-shot.

\subsection{Overall Performance Comparison} \label{subsec:gains_on_transfer}

\noindent
\textbf{Gains in cross-lingual transfer with supervised performance preserved:}
Our approach improves cross-lingual transfer across different tasks and models. 
For slot filling, 
we observe gains in both supervised and transfer (F$_1$ $+$0.4 and $+$1.5 respectively) settings on Llama fine-tuning, 
with similar improvements on Qwen (F$_1$ $+$0.7 supervised, $+$1.8 transfer). 
In machine translation with Llama in row $(3)$, 
our approach brings substantial gains when transferring to out-of-English directions ($+$1.5 BLEU, $+$1.1 COMET).
For into-English directions, 
there is a modest improvement in $+$0.5 BLEU and $+$0.2 COMET.
The larger gains on out-of-English directions suggest the approach is more beneficial for non-English generation in this case.
For Qwen in row $(6)$, 
our approach shows minor gains in into-English translation ($+$1.1 BLEU but no change in COMET), and does not influence out-of-English scores.
It also leads to a degradation ($-$0.8 COMET) on supervised directions.
This is potentially due to Qwen's non-English-centric pretraining combined with our English-centric alignment data.
With this exception, 
our approach maintains or improves supervised performance while enhancing transfer.

\noindent
\textbf{Aligned languages improve the most, but gains extend to other languages:}
The diverse language coverage in the slot filling dataset allows us to 
compare how the alignment objective benefits transfer to both aligned and non-aligned languages.
While aligned languages show the strongest improvements (F$_1$ $+$4.2 and $+$4.9 for Llama and Qwen respectively),
the benefits extend to other languages. 
Over the remaining 10 non-aligned languages, 
there is an average F$_1$ improvement of 0.4 (per-language results in Appendix~\ref{sec:appendix_individual_languages}).
This suggest that the alignment step enhances the model's general cross-lingual transfer capabilities rather than optimizing for specific language pairs. 

\noindent
\textbf{Smaller gains on non-Latin script languages:}
Beyond overall performance improvements,
we observe smaller gains on typologically diverse languages.
Specifically, 
for the non-Latin script transfer languages in the slot filling task (Greek, Hindi, Japanese, Thai, Urdu), 
the average improvement is only 0.5 F$_1$ in contrast to the overall average gain of 1.5.
This reduced gain is likely related to suboptimal tokenization for these languages in multilingual models \cite{rust-etal-2021-good,DBLP:conf/nips/PetrovMTB23,hong-etal-2024-accelerating}.
When tokens poorly align with linguistic units, 
the mean-pooled sentence representations may poorly capture semantics, 
thereby impacting our alignment objective. 

\subsection{Alignment Loss Placement} \label{subsec:loss_placement}
To validate our choice of middle-layer alignment motivated by the analysis in \S\ref{sec:analysis},
we compare performance when applying the alignment loss at different network depths: bottom (8\textsuperscript{th}), middle (16\textsuperscript{th}), and top (32\textsuperscript{nd}) layers of Llama. 

\noindent
\textbf{Middle layer placement achieves more balanced improvements in transfer languages:}
As shown in \autoref{tab:loss_position},
compared to the "middle" configuration, 
the "bottom" configuration clearly leads to poor overall performance in both supervised and transfer settings, 
with a particularly strong degradation on the slot filling task.
While top-layer alignment maintains overall strong performance, 
it shows more unbalanced gains across transfer languages, 
as evidenced by the higher standard deviation of performance gains on transfer languages. 

\noindent
\textbf{Middle layer placement achieves better alignment across network depths:}
To better understand the effects of different loss placements, 
we run the translation retrieval task over model activations at from different intermediate layers.
As shown in \autoref{fig:ablation_loss_layer},
When the alignment loss is applied at the middle (16\textsuperscript{th}) layer, 
semantic alignment is enhanced not only at that layer but also in multiple preceding layers. 
In contrast, 
top-layer alignment primarily affects only the final layer, 
and bottom-layer alignment shows limited improvement in alignment quality across all layers. 
This is likely because the lower layers are occupied with processing more fundamental text features \cite{belinkov-etal-2017-neural,peters-etal-2018-deep} rather than abstract semantic meanings.

\begin{table}[t!]
    \small
    \centering
    \setlength\tabcolsep{2.2pt}
    \begin{tabular}{l c c c c c c ccc}
    \toprule
    %\textbf{}
    &\textbf{Supervised{\tiny $\uparrow$}} 
    %& \makecell{\textbf{Avg. utt.} \\ \textbf{duration (s)}} 
    & \textbf{Transfer{\tiny $\uparrow$}}
    & \textbf{Transfer SD{\tiny $\downarrow$}} 
    % & \textbf{Align.}
    %& \textbf{Retrieval {\tiny $\uparrow$}} 
    \\
    \midrule
    \multicolumn{4}{l}{
    \textbf{Slot filling} (\textsc{MASSIVE}): F$_1$}  
    \\
    Middle (layer 16) &
    77.0 & 
    61.7 &
    2.6 %&
    %73.2
    \\
    Top (layer 32) &
    76.6 & 
    62.0 &
    3.3 %&
    %61.6 
    \\
    Bottom (layer 8) &
    76.8 & 
    58.0 &
    2.9 %&
    %54.0
    \\
    \midrule
    \multicolumn{4}{l}{
    \textbf{Machine translation} (\textsc{WMT}23): COMET
    } 
    \\
    Middle (layer 16) &
    81.5 & 
    80.7 &
    3.7 %&
    %(84.5)
    \\
    Top (layer 32) &
    82.0 & 
    80.2 &
    4.2
    \\
    Bottom (layer 8) &
    81.2 & 
    80.1 &
    5.6
    \\
    \bottomrule
    \end{tabular}
    \caption{Impact of alignment loss placement on supervised and transfer performance. 
    "Top" leads to more uneven gains across languages,
    while "bottom" degrades both supervised and transfer performance.\label{tab:loss_position}}
\end{table}

\begin{figure}[t!]
    \centering
    % trim left, bottom, right, top
    \includegraphics[width=\linewidth,clip,trim={0 0.2cm 0 0}]{figures/ablation_retrieval_accuracy_by_loss_layer.pdf}
    \caption{Retrieval accuracy over model depths when adding alignment loss on different layers. Middle layer placement (layer 16) results in overall better alignment.}
    \label{fig:ablation_loss_layer}
\end{figure}


\subsection{Impact on Representation Retrieval}
To assess the impact of the alignment loss on the learned model representations, 
we also report the retrieval accuracy for all languages involved in each task (20 for slot filling and 9 for machine translation) after fine-tuning in \autoref{tab:overall_results}.
For Llama on the slot filling task, the alignment loss substantially improves retrieval accuracy from 39.4\% to 73.2\%.
{For Qwen, the alignment loss does not improve retrieval among the 20 slot filling languages,
possibly due to the lower accuracy of the base model with many low-resource languages with 0\% accuracy,
making improvement more challenging.}
For machine translation, 
as noted earlier \S\ref{subsec:setup_evaluation}, the retrieval test data overlaps with part of the task training data, 
potentially inflating accuracy (marked in brackets in \autoref{tab:overall_results}).
However, we verify that this overlap does \textit{not} lead to perfect retrieval accuracy: 
Specifically, at the 16\textsuperscript{th} layer where the alignment loss is applied, 
English-source retrieval accuracies for supervised languages show varying accuracy:
cs (98.1\%), de (96.5\%), is (66.9\%), ru (90.6\%), and zh (94.8\%).
This suggests that the overlap does not make the retrieval diagnostic task trivial.

\section{Analyses}

\subsection{Resource Level of Alignment Languages} \label{subsec:res_level}
\begin{table}[t!]
    \small
    \centering
    \setlength\tabcolsep{1.1pt}
    \begin{tabular}{l c c c c}
    \toprule
    \textbf{Resource}
    &\textbf{Super.} 
    %& \makecell{\textbf{Avg. utt.} \\ \textbf{duration (s)}} 
    & \textbf{Transfer}
    & \textbf{Gain on Aligned}
    % & \textbf{Retrieval} 
    \\
    & 
    (5 lang.) &
    (15 lang.) &
    (4/5 lang.)
    \\
    \midrule
    SFT (row $(2)$ \autoref{tab:overall_results}) &
    76.6 &
    60.2 &
    --
    \\
    Low (row $(3)$ \autoref{tab:overall_results}) &
    77.0 & 
    61.7 & 
    $+$3.8 %&
    \\
    Medium  &
    77.8 & 
    61.4 & 
    $+$1.1 %&
    \\
    High  &
    77.6 & 
    60.4 & 
    $+$0.7 %&
    \\
    \bottomrule
    \end{tabular}
    \caption{Ablation of using alignment languages from different resource levels on slot filling with Llama.
    \label{tab:resource_level}}
\end{table}


In our main experiments, we selected the 5 languages with the weakest performance from the \textsc{MASSIVE} baseline \cite{fitzgerald-etal-2023-massive} for alignment. 
We now vary the resource level of the alignment languages
using a medium-resource group with \{el, hi, th, tr\}$-$en
and a high-resource group with \{ar, es, ru, zh\}$-$en, 
which also have supervised task training data. 
As shown in \autoref{tab:resource_level}, 
all three configurations improve F$_1$ scores for the languages involved in alignment. 
However, the low-resource group exhibit the largest gains ($+$3.8 F$_1$), 
indicating that our approach is most beneficial to languages with weaker initial performance.
Moreover, 
overall transfer gains relative to the SFT baseline diminish when using high-resource languages for alignment, 
likely because these languages already have well-aligned representations and aligning them provides little benefit to lower-resource languages in the transfer set.
Overall, 
the results show that our approach is robust to the choice of alignment languages,
but selecting initially poorly aligned languages could provide broader benefits across different languages.

\subsection{Generalization of Learned Alignment}
\label{subsec:domain_lang_generalization}
\autoref{tab:generalization} examines the language and domain generalization of our alignment component.
To isolate the effects of task-specific joint training, 
we train the models using only the alignment loss,
following the same setup as our previous experiments but without optimizing on task-specific data.
We then evaluate retrieval accuracy as described in \S\ref{subsec:setup_evaluation}.

\noindent
\textbf{Language Generalization:}
While our main experiments align multiple language pairs, 
we now use single languages for alignment. 
As shown in \autoref{tab:generalization} (upper portion),
that single-language alignment training leads to diminished performance compared to multilingual training. 
Interestingly, 
we see comparable accuracy drops regardless of which individual language is used for alignment,
suggesting that the gains of multilingual alignment come from the diversity of the training data rather than characteristics of individual languages.

\noindent
\textbf{Domain Generalization:}
To isolate the effects of multilinguality,
we focus on alignment between a single language pair (English-German). 
In \autoref{tab:generalization} (lower portion),
we first establish an oracle setup using models trained on FLoRes data (Wikipedia domain, 
overlapping with retrieval data).
We then compare to two setups where the alignment data come from other domains: 
Tatoeba (short sentences for language learning; different) and IWSLT 2017 (public speaking transcriptions; very distant).
While we observe a decrease in retrieval accuracy compared to the oracle setup, 
the results suggest that, to enforce alignment into the model, it is not strictly necessary to source alignment data from the same domain as the task-specific data.

\begin{table}[t]
    \small
    \centering
    \setlength\tabcolsep{15pt}
    \begin{tabular}{l c c c c}
    \toprule
    \textbf{Alignment Data}
    % &\textbf{Supervised} 
    %& \makecell{\textbf{Avg. utt.} \\ \textbf{duration (s)}} 
    % & \textbf{Zero-shot}
    & \textbf{Overall} (20 lang.)
    \\
    \midrule
    Multi \{ar,es,ru,zh,{sw}\}-en &
    % &
    % &
    80.2
    \\
    \phantom{00}Only de-en  &
    % &
    % &
    71.9
    \\
    \phantom{00}Only es-en &
    % &
    % &
    72.9
    \\
    \phantom{00}Only zh-en &
    % &
    % &
    72.7
    \\
    \midrule
    de-en FLoRes (oracle) & 
    % &
    % &
    77.7
    \\
    \phantom{00}Tatoeba (different) &
    % &
    % &
    71.9
    \\
    \phantom{00}IWSLT (very distant) &
    % &
    % &
    68.5
    \\
    \bottomrule
    \end{tabular}
    \caption{Retrieval accuracy when alignment data come from different languages and domains on Llama. \label{tab:generalization}}
\end{table}

\begin{table}[t!]
    \small
    \centering
    \setlength\tabcolsep{6.5pt}
    \begin{tabular}{l c c c c}
    \toprule
    \textbf{Resource}
    &\textbf{Supervised} 
    %& \makecell{\textbf{Avg. utt.} \\ \textbf{duration (s)}} 
    & \textbf{Transfer}
    % & \textbf{Retrieval} 
    \\
    \midrule
    \multicolumn{4}{l}{
    \textbf{Slot filling} (\textsc{MASSIVE}): F$_1$}  
    \\
    % Baseline &
    % 76.6 &
    % 60.2 &
    % --
    % \\
    SFT (row $(2)$ \autoref{tab:overall_results}) &
    76.6 \phantom{($+$0.0)}&  
    60.2 \phantom{($+$0.0)}& 
    \\
    Joint (row $(3)$ \autoref{tab:overall_results}) &
    77.0 ($+$0.4) & 
    61.7 ($+$1.5)
    \\
    Merge &
    76.9 ($+$0.3) & 
    61.3 ($+$1.1) %& 
    \\
    \midrule
    \multicolumn{4}{l}{
    \textbf{Machine translation} (\textsc{WMT}23): COMET 
    } 
    \\
    SFT (row $(4)$ \autoref{tab:overall_results}) &
    81.5 \phantom{($+$0.0)}&  
    79.6 \phantom{($+$0.0)}& 
    \\
    Joint (row $(5)$ \autoref{tab:overall_results}) &
    81.5 ($+$0.0) & 
    80.7 ($+$1.1)
    \\
    Merge &
    82.0 ($+$0.5) & 
    80.2 ($+$0.6) %& 
    \\
    \bottomrule
    \end{tabular}
    \caption{Result of merging separately-trained task and alignment modules on Llama.\label{tab:merging}} %(En$\rightarrow$X)
\end{table}


\subsection{Merging Alignment and Task Modules} \label{subsec:merging}
Our previous experiments focused on models jointly trained on both task and alignment objectives.
However, in practice, 
it may be necessary to enhance existing task-specific models with cross-lingual capabilities, 
where joint re-training is infeasible due to computational constraints or unavailability of the original task training data.
Inspired by recent advances in model merging
\cite{DBLP:conf/nips/MatenaR22,DBLP:conf/iclr/IlharcoRWSHF23}, 
we explore the feasibility of combining separately-trained task and alignment modules.
We merge two sets of trained LoRA adapters by averaging their weights\footnote{We use a weighted average tuned on the development set (details in Appendix \ref{appendix:inference_details})}: 
the alignment module trained in isolation (\S\ref{subsec:domain_lang_generalization}), 
and task-specific modules (rows (2) and (5) in \autoref{tab:overall_results}).

\autoref{tab:merging} shows that this post-hoc merging brings comparable improvements comparable to joint training.
Moreover, the improvements are more evenly distributed across languages compared to the larger gains observed on languages used directly in alignment.
These results demonstrate that our alignment approach is modular and can be combined with existing task-specific models.

\begin{table}[t!]
    \small
    \centering
    \setlength\tabcolsep{8pt}
    \begin{tabular}{l c c c c}
    \toprule
    %\textbf{Resource}
    &\textbf{Supervised} 
    %& \makecell{\textbf{Avg. utt.} \\ \textbf{duration (s)}} 
    & \textbf{Transfer}
    & \textbf{Transfer}
    % & \textbf{Retrieval} 
    \\
    & 
    (en, pt, zh) &
    (da, sv) &
    (5 lang.)
    \\
    \midrule
    Llama SFT  &
    83.4 & 
    82.1 & 
    79.3
    \\
    $+$ alignment  &
    82.4 & 
    83.1 & 
    79.8
    \\
    \bottomrule
    \end{tabular}
    \caption{Results on JSON generation evaluated with F$_1$ after parsing the output. \label{tab:uner}}
\end{table}

\subsection{Long Sequence Processing}
\label{subsec:uner}
We investigate a more challenge task requiring longer input and output generation using UNER (\S\ref{subsec:exp_setup_data}).
As shown in \autoref{tab:uner}, 
while aligned languages still show improvements, 
the gains are more modest compared to previous experiments, 
with an F$_1$ increase of 1.0 on aligned languages and 0.5 across all transfer languages. 
Moreover, 
there is an average degradation of 1.0 F$_1$ on supervised languages, 
mainly due to the decline in Chinese ($-$2.2 F$_1$).  
A potential reason is the mismatch between our sentence-level alignment objective and the requirements of processing longer sequences.

\section{Related Works}
\textbf{Multilingual Capabilities of LLMs:}
LLM performance varies across languages due to imbalanced pre-training data volume.  
However, even predominantly English-centric models \cite{DBLP:journals/corr/abs-2307-09288} exhibit some degree of multilingual capability \cite{aycock-bawden-2024-topic,yuan-etal-2024-vocabulary},
potentially due to the unintentional ingestion of multilingual data during pretraining \cite{briakou-etal-2023-searching}.
Meanwhile, many recent LLMs have expanded their language coverage \cite{grattafiori2024llama3herdmodels,qwen2025qwen25technicalreport}.
Despite these inherent multilingual capabilities, 
extending them to downstream tasks in low-resource settings \cite{adelani-etal-2024-comparing,iyer-etal-2024-exploring} remains challenging. 

\noindent
\textbf{Multilingual Representation Alignment:}
Enhancing meaningful cross-lingual relationships between model representations has been a well-studied area in the context of many tasks,
including intermediate tasks such as 
bilingual lexicon induction \cite{zhang-etal-2017-adversarial} and
sentence embeddings \cite{feng-etal-2022-language,li-etal-2023-dual},
as well as more direct applications like 
information retrieval \cite{DBLP:journals/tmlr/IzacardCHRBJG22} and translation \cite{pham-etal-2019-improving,pan-etal-2021-contrastive}.
In the context of LLMs, 
\citet{wang2024bridginglanguagegapslarge} 
use linear projections learned offline to align non-English representations with English ones during decoding.
Our work differs in that our alignment objective is parameterized by the same weights as task-specific fine-tuning, 
and is directly applicable to multilingual fine-tuning.
\citet{wu-etal-2024-representational} align LLM top-layer representations specifically for the task of semantic textual similarity (STS).
Different from this work, they do not consider cross-lingual transfer in downstream tasks or explore intermediate LLM layers for alignment.

\noindent
\textbf{LLM Representation Analysis:}
Several recent works have analyzed LLM internal representations with
geometric analysis of representation spaces \cite{razzhigaev-etal-2024-shape,lee2024multimodalfoundationmodelsencode},
probing classifiers \cite{wang-etal-2024-probing-emergence,li2025exploringmultilingualprobinglarge},
or logit lens analysis \cite{wu2024semantichubhypothesislanguage}.
In particular,
\citet{wu2024semantichubhypothesislanguage} identify “semantic hubs” in LLM middle layers, which integrate information from various data types.
Our findings are orthogonal to their work on multi-modality.

\section{Conclusion}
We presented a simple yet effective approach for enhancing cross-lingual transfer in LLMs through middle-layer representation alignment during fine-tuning. 
Our experimental results lead to several practical recommendations:
1) Aligning a few weakly-performing languages yields broad transfer benefits. 
A few hundreds of parallel sentences as alignment data are sufficient.
2) Alignment data can be sourced from different domains as the task.
3) Existing task-specific models can be enhanced with our approach via parameter merging
without the need of full re-training.

\newpage
\section*{Limitations}
\paragraph{Typologically diverse languages:}
As discussed in \S\ref{subsec:gains_on_transfer}, 
our approach shows smaller gains on languages non-Latin scripts. 
This limitation is likely related to fundamental tokenization challenges, where suboptimal token segmentation negatively impacts the quality of mean-pooled representations.
While our initial experiments on attention pooling did not lead to improvements, 
exploring more sophisticated pooling mechanisms could potentially address this challenge in future work.

\paragraph{Computational overhead during training:}
The alternating optimization between task and alignment objectives doubles the computational cost during training compared to standard fine-tuning. 
In computationally constrained settings, our merging approach, 
which separates task-specific and alignment training, 
should be prioritized.  
Given that alignment can be effectively performed using only a small number of parallel sentences (a few hundred per language), 
this modular approach can significantly reduce the overall computational cost.

\paragraph{Trade-offs between supervised and transfer performance in challenging scenarios:}
While our approach generally maintains or improves supervised task performance while improving transfer,
we observe degradation in supervised performance in two specific scenarios.
First, in structured text generation (\S\ref{subsec:uner}), the method shows reduced effectiveness and can impair supervised performance ($-$1.0 F$_1$), suggesting that our sentence-level alignment may interfere with the processing of longer, structured sequences.
Second, when applying the method to models with weak initial cross-lingual alignment (\S\ref{subsec:gains_on_transfer}), there could be a trade-off between improved transfer and supervised performance.

% Bibliography entries for the entire Anthology, followed by custom entries
%\bibliography{anthology,custom}
% Custom bibliography entries only
\bibliography{custom,anthology}

\appendix



\section{English-Only Fine-Tuning Results}
\label{sec:appendix_english_only}
\begin{table*}[h]
    \small
    \centering
    \setlength\tabcolsep{1.7pt}
    \begin{tabular}{l c c c c c  c c c c c  ccccccccccccccccccccccccccc}
    \toprule
    % & \multicolumn{5}{c}{\textbf{Supervised}}
    % & \multicolumn{5}{c}{\textbf{Transfer (aligned)}}
    % & \multicolumn{10}{c}{\textbf{Transfer (other)}}
    % \\
    & 
    ar &
    en &
    es &
    ru &
    zh &
    cy &
    ja &
    jv &
    sw &
    tl &
    af &
    az &
    de &
    el &
    fr &
    hi &
    is &
    th &
    tr &
    ur 
    \\
    \midrule
    English-only 
    &
59.8 &
82.5 &
82.4 &
65.8 &
61.6 &
60.3 &
39.7 &
37.8 &
39.8 &
57.5 &
60.3 &
39.6 &
71.1 &
64.8 &
68.2 &
62.1 &
39.2 &
75.3 &
52.9 &
49.9 &
    \\
    Multilingual 
    &
    75.5 &
81.7 &
74.5 &
77.6 &
73.8 &
44.0 &
65.8 &
41.0 &
42.8 &
65.0 &
66.0 &
49.0 &
75.0 &
69.4 &
71.9 &
70.0 &
45.0 &
79.9 &
60.4 &
57.1
    \\
    \bottomrule
    \end{tabular}
    \caption{Per-languages F$_1$ results on slot filling of English-only finetuning compared to multilingual fine-tuning on \{ar, en, es, ru, zh\}. 
    Multilingual fine-tuning shows stronger transfer performance. \label{tab:full_results_enonly_vs_multilingual}}
\end{table*}



\begin{table*}[h!]
    \small
    \centering
    %\setlength\tabcolsep{1pt}
    \begin{tabular}{c c c c cc c}
    \toprule
    \textbf{Code} 
    & \textbf{FLoRes Code} 
    & \textbf{Full Name}
    & \textbf{Slot Filling}
    & \textbf{Machine Translation}
    & \textbf{JSON Generation}
    \\
    \midrule
    af &
    afr\_Latn &
    Afrikaans  &
    $\checkmark$ 
    \\
    \rowcolor{Gray}
    az &
    azj\_Latn &
    North Azerbaijani &
    $\checkmark$ &&
    \\
    ar &
    arb\_Arab &
    Modern Standard Arabic &
    $\checkmark$ 
    \\\rowcolor{Gray}
    cs &
    ces\_Latn &
    Czech &&
    $\checkmark$ &
    \\
    cy &
    cym\_Latn &
    Welsh&
    $\checkmark$ 
    \\
    \rowcolor{Gray}
    da &
    dan\_Latn &
    Danish &&&    $\checkmark$ 
    \\
    de &
    deu\_Latn & 
    German &
    $\checkmark$ &
    $\checkmark$ 
    \\
    \rowcolor{Gray}
    el &
    ell\_Grek &
    Greek &
    $\checkmark$ &&
    \\
    en &
    eng\_Latn &
    English &
    $\checkmark$ &
    $\checkmark$ &
    $\checkmark$ 
    \\
    \rowcolor{Gray}
    es &
    spa\_Latn &
    Spanish &
    $\checkmark$ &&
    \\
    fr &
    fra\_Latn &
    French &
    $\checkmark$ 
    \\
    \rowcolor{Gray}
    he &
    heb\_Hebr &
    Hebrew &&
    $\checkmark$ &
    \\
    hi &
    hin\_Deva &
    Hindi &
    $\checkmark$ 
    \\
    \rowcolor{Gray}
    hr &
    hrv\_Latn &
    Croatian &&&
    $\checkmark$ 
    \\
    is &
    isl\_Latn & 
    Icelandic &
    $\checkmark$ &
    $\checkmark$ 
    \\
    \rowcolor{Gray}
    ja &
    jpn\_Jpan&
    Japanese &
    $\checkmark$ &
    $\checkmark$ &
    \\
    jv &
    jav\_Latn &
    Javanese &
    $\checkmark$ 
    \\
    \rowcolor{Gray}
    pt &
    por\_Latn &
    Portuguese &&&
    $\checkmark$ 
    \\
    ru &
    rus\_Cyrl &
    Russian &
    $\checkmark$ &
    $\checkmark$ 
    \\
    \rowcolor{Gray}
    sk &
    slk\_Latn &
    Slovak &&&
    $\checkmark$ 
    \\
    sr &
    srp\_Cyrl &
    Serbian &&&
    $\checkmark$ 
    \\
    \rowcolor{Gray}
    sv &
    swe\_Latn &
    Swedish &&&
    $\checkmark$ 
    \\
    sw &
    swh\_Latn &
    Swahili &
    $\checkmark$ 
    \\
    \rowcolor{Gray}
    th &
    tha\_Thai &
    Thai &
    $\checkmark$ &&
    \\
    tl &
    tgl\_Latn &
    Tagalog &
    $\checkmark$ 
    \\
    \rowcolor{Gray}
    tr &
    tur\_Latn &
    Turkish &
    $\checkmark$ &&
    \\
    uk &
    ukr\_Cyrl &
    Ukrainian &&
    $\checkmark$ 
    \\
    \rowcolor{Gray}
    ur &
    urd\_Arab &
    Urdu &
    $\checkmark$ &&
    \\
    zh &
    zho\_Hans &
    Chinese (Simplified) &
    $\checkmark$ &
    $\checkmark$ &
    $\checkmark$ 
    \\
    \bottomrule
    \end{tabular}
    \caption{List of languages evaluated on different downstream tasks. \label{tab:full_language_list}}
\end{table*}

\autoref{tab:full_results_enonly_vs_multilingual} compares English-only and multilingual fine-tuning on \textsc{MASSIVE}.
Multilingual fine-tuning substantially outperforms English-only in cross-lingual transfer performance.


\section{Dataset Details}
\label{sec:appendix_dataset_details}
All our task training data are retrieved from HuggingFace\footnote{
MASSIVE: \url{https://huggingface.co/datasets/AmazonScience/massive}\\
ALMA: \url{https://huggingface.co/datasets/haoranxu/ALMA-Human-Parallel}\\
UNER: \url{https://huggingface.co/datasets/CohereForAI/aya_collection/viewer/templated_uner_llm}}.
The translation test sets are hosted by WMT\footnote{ \url{https://github.com/wmt-conference/wmt23-news-systems/tree/master/txt}}.
The alignment data are sourced from Tatoeba\footnote{\url{https://huggingface.co/datasets/Helsinki-NLP/tatoeba}}
with its default version of
\texttt{v2021-07-22} at the time of writing.
We filter out translations that are empty or include multiple sentences.
The lowest-resource alignment languages have a few hundred parallel sentences: 
Javanese (264), Swahili (371), Welsh (823).
The ablation de-en alignment data is from IWSLT 2017\footnote{\url{https://huggingface.co/datasets/IWSLT/iwslt2017}} \cite{cettolo-etal-2017-overview}.


\section{List of Languages} \label{sec:appendix_language_list}
The languages involved in our downstream tasks are listed in \autoref{tab:full_language_list}.
The 35 languages in the initial analyses in \S\ref{sec:analysis}
include all languages in slot fill and machine translation.
They additionally include the following languages: 
am (Amharic), bn (Bengali), it (Italian), hu (Hungrian), hy (Armenian),
id (Indonesian), kn (Kannada), ka (Georgian ), mn (Mongolian), km (Khmer), 
ko (Korean), and lv (Latvian).



\section{Training and Inference Details}
\label{sec:appendix_training_details}

\subsection{Training Hyperparameters}
Fine-tuning is performed using LoRA \cite{DBLP:conf/iclr/HuSWALWWC22} adapters with a rank of 8 for all attention components and linear projections (query, key, value, output, gate, up, down).
We set LoRA's $\alpha$ parameter to 16 and dropout to 0.1.
The number of trainable parameter is 20,971,520 on Llama 3, and 20,185,088 on Qwen 2.5.
We train at most 5 epochs on the task data. 
Training on all our tasks converged before reaching the max number of epochs.
The learning rate is set to 5e-4 with inverse square root schedule and warmup up ratio 0.03.
We save checkpoints and evaluate every 200 optimization steps,
and early stop if the development loss does not improve for 5 consecutive evaluations.
For the temperature parameter $\tau$ in the contrastive loss, 
we searched among \{0.1, 1.0, 1.5, 2.0\} based on development loss on machine translation.
For Llama we 0.1, for Qwen we use 1.5.

\subsection{Prompt Format}
\label{sec:appendix_prompt_format}
\paragraph{Slot Filling}
The system prompt is shortened from \citet{DBLP:conf/interspeech/0001G23}.
\begin{itemize}[nolistsep,leftmargin=*]
\item \textbf{System}: Given a command from the user, a voice assistant will extract entities essential for carry out the command. 
Your task is to extract the entities as words from the command if they fall under a predefined list of entity types.
\item \textbf{User}: wake me up at five am this week
\item \textbf{Assistant}: time: five am; date: this week
\item \textbf{User} (de): wecke mich in dieser woche um fünf uhr auf
\item \textbf{Assistant} (de): date: dieser woche; time: fünf uhr
\end{itemize}

For \textbf{zero-shot slot filling} experiments, we need to specify more requirements in the system prompt with the template also following \citet{DBLP:conf/interspeech/0001G23}:

{
Given a command from the user, a voice assistant like Siri or Olly will extract entities from the command that are essential for carry out the the command. For example, for a command about playing a specific song, the name of the song mentioned by the user would be an entity, falling under the type of “song name”.

Your task is to extract the entities as words from the command if they fall under any of the types given below according to the following description:

transport\_descriptor
house\_place
music\_album
sport\_type
playlist\_name
movie\_name
song\_name
place\_name
radio\_name
cooking\_type
weather\_descriptor
person
email\_folder
business\_type
audiobook\_author
transport\_type
general\_frequency
meal\_type
game\_name
device\_type
transport\_name
time\_zone
joke\_type
drink\_type
email\_address
food\_type
date
relation
currency\_name
ingredient
player\_setting
movie\_type
definition\_word
game\_type
list\_name
artist\_name
personal\_info
audiobook\_name
timeofday
transport\_agency
media\_type
podcast\_name
coffee\_type
business\_name
news\_topic
app\_name
podcast\_descriptor
color\_type
music\_genre
event\_name
time
change\_amount
alarm\_type
order\_type
music\_descriptor

Please give answers like:

1. person: john; contact\_field: phone number

2. transport\_app: uber; time\_of\_day: tonight; time: ten pm

3. None

4. music\_genre: jazz

etc., each taking a single line. The entity type must be one of the types given above, and the entity must be copied verbatim from the command. There could be zero, one, or multiple entities in a command.
}


\begin{table*}[t!]
    \small
    \centering
    \setlength\tabcolsep{2pt}
    \begin{tabular}{l c c c c c | c c c c c | ccccccccccccccccccccccccccc}
    \toprule
    & \multicolumn{5}{c}{\textbf{Supervised}}
    & \multicolumn{5}{c}{\textbf{Transfer (aligned)}}
    & \multicolumn{10}{c}{\textbf{Transfer (other)}}
    \\
    & 
    ar &
    en &
    es &
    ru &
    zh &
    cy &
    ja &
    jv &
    sw &
    tl &
    af &
    az &
    de &
    el &
    fr &
    hi &
    is &
    th &
    tr &
    ur 
    \\
    \midrule
    Llama  3 SFT 
    &
75.5 &
81.7 &
74.5 &
77.6 &
73.8 &
44.0 &
65.8 &
41.0 &
42.8 &
65.0 &
66.0 &
49.0 &
75.0 &
69.4 &
71.9 &
70.0 &
45.0 &
79.9 &
60.4 &
57.1
    \\
    $+$ align
    &
75.1 &
82.0 &
74.9 &
78.0 &
74.9 &
49.4 &
66.5 &
48.2 &
47.7 &
65.5 &
66.2 &
47.9 &
74.7 &
72.4 &
72.1 &
69.6 &
48.0 &
79.1 &
62.2 &
56.1 &
    \\
    Qwen 2.5 SFT &
74.7 &
81.1 &
74.0 &
77.5 &
74.1 &
27.0 &
67.3 &
32.9 &
23.5 &
57.4 &
58.9 &
45.9 &
74.6 &
63.3 &
70.8 &
60.0 &
34.4 &
79.9 &
59.9 &
46.5 &
    \\
    $+$ align &
    74.9 &
82.5 &
74.8 &
78.0 &
75.1 &
36.5 &
68.3 &
39.6 &
30.4 &
57.8 &
63.1 &
42.5 &
74.6 &
63.3 &
70.9 &
61.3 &
35.8 &
80.2 &
58.1 &
47.2 &
    \\
    \bottomrule
    \end{tabular}
    \caption{Per-languages F$_1$ results on slot filling. \label{tab:full_results_slot_filling}}
\end{table*}

\begin{table*}[h!]
    \small
    \centering
    \setlength\tabcolsep{2pt}
    \begin{tabular}{l c c c c c | c c c c c | ccc | cccccccccccccccccccccccc}
    \toprule
    & \multicolumn{5}{c}{\textbf{Supervised X$\rightarrow$En}}
    & \multicolumn{5}{c}{\textbf{Supervised En$\rightarrow$X}}
    & \multicolumn{3}{c}{\textbf{Transfer X$\rightarrow$En}}
    & \multicolumn{3}{c}{\textbf{Transfer En$\rightarrow$X}}
    \\
    & 
    cs &
    de &
    is &
    ru &
    zh &
    cs &
    de &
    is &
    ru &
    zh &
    he &
    ja &
    uk &
    he &
    ja &
    uk 
    \\
    \midrule
    &
    \multicolumn{16}{c}{BLEU} \\
    Llama  3 SFT 
    &
37.8 &
43.0 &
28.3 &
32.0 &
22.5 &
25.9 &
35.5 &
10.6 &
25.2 &
38.9 &
39.3 &
17.5 &
38.7 &
14.5 &
14.2 &
17.7 
    \\
    $+$ align &
38.4 &
43.1 &
29.1 &
32.4 &
23.0 &
24.7 &
34.7 &
10.9 &
24.4 &
38.1 &
39.8 &
18.8 &
38.4 &
16.0 &
15.6 &
19.5 
    \\
    Qwen 2.5 SFT &
36.1 &
40.8 &
20.5 &
30.6 &
23.2 &
21.5 &
33.7 &
6.8 &
25.3 &
45.3 &
34.6 &
18.9 &
35.6 &
13.3 &
17.6 &
13.0 
    \\
    $+$ align &
36.6 &
41.4 &
21.2 &
30.9 &
24.0 &
20.5 &
32.7 &
4.8 &
25.0 &
45.3 &
36.3 &
19.4 &
36.8 &
12.7 &
17.8 &
13.5 
    \\
&
\multicolumn{16}{c}{COMET} \\
Llama 3 SFT &
85.2 &
84.9 &
81.0 &
82.4 &
79.7 &
84.3 &
81.8 &
68.7 &
83.3 &
84.2 &
83.6 &
79.8 &
85.1 &
75.7 &
83.5 &
79.7 
\\
 $+$ align &
 85.5 &
84.9 &
81.1 &
82.4 &
79.8 &
83.8 &
81.6 &
69.0 &
83.3 &
84.0 &
83.6 &
80.1 &
85.2 &
77.1 &
84.2 &
80.8 
 \\
Qwen 2.5 SFT &
84.8 &
84.7 &
74.1 &
82.6 &
80.2 &
80.8 &
80.6 &
52.0 &
83.3 &
86.1 &
82.3 &
81.3 &
84.5 &
70.7 &
85.5 &
74.6 
\\
$+$ align &
85.1 &
84.7 &
74.4 &
82.6 &
80.4 &
79.5 &
80.1 &
46.5 &
83.1 &
85.8 &
82.2 &
81.4 &
84.6 &
70.7 &
85.7 &
74.4 
\\
    \bottomrule
    \end{tabular}
    \caption{Per-languages BLEU and COMET results on machine translation. \label{tab:full_results_machine_translation}}
\end{table*}

\paragraph{Machine Translation}
\begin{itemize}[nolistsep,leftmargin=*]
    \item \textbf{System}: Translate the following sentences from English to German.
    \item \textbf{User}: Police arrest 15 after violent protest outside UK refugee hotel.
    \item \textbf{Assistant}: Polizei verhaftet 15 Menschen nach gewalttätigen Protesten vor einer Flüchtlingsunterkunft in Großbritannien
\end{itemize}

\paragraph{JSON Generation} 
\begin{itemize}[nolistsep,leftmargin=*]
    \item \textbf{User}: Please identify all the named entities mentioned in the input sentence provided below. Use only the categories: PER - person, ORG - organization, and LOC - location. Remember, nationalities are neither locations nor organizations, and organizations can represent other groups of people. Pay attention to the provided example. You should only output the results in JSON format, following a similar structure to the example result provided. Example sentence and results: Where in the world is Iguazu? "Results": [ { "TypeName": "LOC", "Text": "Iguazu", "Start": 22, "End": 28 } ] Considering the input sentence below, what is the output result? Widely considered to be one of the most spectacular waterfalls in the world, the Iguazu Falls on the border of Argentina and Brazil, are a certainly must see attraction in the area.
    \item \textbf{Assistant}: "Results": [ { "TypeName": "LOC", "Text": "Iguazu Falls", "Start": 81, "End": 93 }, { "TypeName": "LOC", "Text": "Argentina", "Start": 111, "End": 120 }, { "TypeName": "LOC", "Text": "Brazil", "Start": 125, "End": 131 } ]
\end{itemize}


\subsection{Inference Details} \label{appendix:inference_details}
We use greedy decoding in all experiments for easily reproducible results.
For the model merging experiments, 
we searched among weights \{0.5, 0.7, 0.9\} for the task-specific LoRA modules on the \textsc{MASSIVE} development set and chose 0.9 for our experiments.

\subsection{Details for Retrieval}
To evaluate cross-lingual retrieval performance,
we adapt the implementation from LASER\footnote{\url{https://github.com/facebookresearch/LASER/tree/main/tasks/xsim}} \cite{schwenk-etal-2021-wikimatrix}
to process representations extracted offline.

\section{Results for Individual Languages}
\label{sec:appendix_individual_languages}
The detailed results for \autoref{tab:overall_results} are in \autoref{tab:full_results_slot_filling} (slot filling) and \autoref{tab:full_results_machine_translation} (machine translation).



\end{document}
