\section{Introduction}

Intelligent tutoring systems are revolutionizing education by democratizing access to personalized and effective learning experiences \cite{vanlehn2011relative}. However, most educational tutoring systems are text-based, providing feedback only through natural language and limited visual support \cite{wang2023examining}. On the other hand, humans have long relied on visual aids like sketches and diagrams to support reasoning and problem-solving. Whether drawing auxiliary lines in geometry, diagrams for graph theory, or visualizing abstract concepts in math, visual tools are essential for breaking down and understanding complex ideas~\cite{bobek2016creating,raiyn2016role,shabiralyani2015impact}. Studies show that students who engage with visual aids grasp ideas more effectively than those who rely solely on textual explanations \cite{shabiralyani2015impact}. Language-only systems are therefore unintuitive and often fall short when addressing tasks that require visual and spatial reasoning.

Today's large multimodal models (LMMs) provide a new opportunity to generate diagrams and sketches for tutoring systems automatically \cite{liang2024foundations,hu2024visual,hurst2024gpt,zhou2024transfusion}. Our paper prototypes such a system, called \name, to enable multimodal input and output interaction for problem-solving (see Figure~\ref{fig:teaser}). \name\ is built upon a pre-trained LMM and fine-tuned to provide step-by-step hints and explanations in both language and visual diagrams to tutor students. \name\ creates diagrams accurately and robustly by generating Python programs to output diagrams when executed. The diagram, along with corresponding textual hints, is sent to the student on their `sketchpad' to be freely annotated or sketched on. This multimodal human-AI interaction framework enables the student to collaborate back and forth with the LMM on a shared interactive whiteboard, creating a natural and seamless interaction loop where both the human and the LMM can share text, images, and annotations.

We conduct several case studies with university students on a range of math tasks (including geometry and calculus), and find that \name\ leads to improved task comprehension, problem-solving accuracy, and engagement levels among students. Unlike traditional systems, this approach emphasizes a hands-on, learner-centered methodology that adapts to individual needs while leveraging the benefits of vision and language reasoning \cite{shabiralyani2015impact}. In addition to addressing the immediate gaps in educational technology, this research also contributes to the broader field of AI by exploring how multimodal systems can enhance human-computer interaction, leading to better AI systems that complement human expertise.