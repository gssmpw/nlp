\section{Limitations and Future Works}

\textbf{Visualization accuracy.} Visualization through code generation can sometimes fail to produce an accurate diagram. Table \ref{table:inaccurateresults} shows examples of correct and incorrect visualizations. The correct visualization correctly calculates the radius as the distance between the center (R) and a point on the circle (S). The incorrect one hard-codes the radius to 100, causing the points to lie within rather than on the circle. Incorrect diagrams may confuse both the user and the LMM, leading to an incorrect answer. Future work could extend previous text-based verifiers \cite{cobbe2021training, lightman2024lets} to include a vision modality that can check the quality of generated diagrams and regenerate the diagram if necessary.

\begin{table*}[t]
\centering
\vspace{2mm}
\begin{tabular}{m{0.6\linewidth}|m{0.4\linewidth}}
\hline
\textbf{Generated Code} & \textbf{Visualization} \\
\hline
\includegraphics[width=\linewidth]{Supporting_files/figures/radiuscode.png} 
\includegraphics[width=\linewidth]{Supporting_files/figures/rightcode1.png} 
\includegraphics[width=\linewidth]{Supporting_files/figures/wrongcode2.png}  & 
\includegraphics[width=0.45\linewidth]{Supporting_files/figures/geometry_labelled_angles.png}  \hspace{3mm} 
\includegraphics[width=0.15\linewidth]{Supporting_files/figures/greencheckemoji.png}\\
\includegraphics[width=\linewidth]{Supporting_files/figures/wrongcode1.png} 
\includegraphics[width=\linewidth]{Supporting_files/figures/rightcode2.png}  & 
\includegraphics[width=0.45\linewidth]{Supporting_files/figures/geometry_fail_diagram.png} \hspace{3mm} \includegraphics[width=0.15\linewidth]{Supporting_files/figures/redxemoji.png}\\
\hline
\end{tabular}
\vspace{1mm}
\caption{Row 1: Accurate Visualization. The radius is set as a variable, and annotations are relative to the given points, leading to an accurate portrayal of the circle diagram (in green). Row 2: Inaccurate visualization. The radius is set to 100, and annotations are hard-coded, leading to the points being misaligned on the circle diagram (in red).}
\vspace{-4mm}
\label{table:inaccurateresults}
\end{table*}
\raggedbottom

\textbf{Expansion of user studies.} While the system has been primarily tested on mathematical problems, its potential extends to other STEM disciplines. Future work will evaluate its effectiveness in subjects like physics, where visualizations of force diagrams and motion trajectories could enhance learning. Expanding testing to a larger pool of diverse users and academic topics will help assess the system’s adaptability. Additionally, benchmarking against existing educational tools will provide quantitative and qualitative insights into its comparative performance and areas for refinement.

\textbf{Long-term impact assessment.} Current evaluations focus on short-term metrics such as accuracy and user feedback. To measure long-term effectiveness, future studies should incorporate delayed testing sessions to assess knowledge retention. Tracking student performance over time will provide insights into sustained learning benefits.