\section{Related Work}
\vspace{-0.5ex}
Since 2017, 3GPP has been continuously releasing technical specifications and reports regarding the integration of satellites and 5G~\cite{38811,38821,23737,28841,36763,38863}, sparking widespread discussion in the academic community about mobile satellite networks~\cite{discussion1,discussion2,discussion3}. Due to recent breakthroughs in technology and cost control for LEO satellites, as well as their inherent advantages over higher orbit satellites, many efforts have been made to integrate LEO satellites with mobile networks.

% on the base of standard 5G network
One mainstream approach is to modify the network structure to accommodate the movement of LEO satellites~\cite{spaceMobileNet, UPFandGNB, CoreDesign, anotherApp, satelessCoreDesign}. This includes discussing the deployment locations of network functions and introducing new network functions. However, most of these works do not focus on user plane issues or fail to provide sufficient improvements in reducing latency.
A recent work~\cite{spaceMobileNet} aims to shorten control plane latency by deploying part of the core network functions on satellites. However, it overlooks user plane issues, resulting in users still experiencing long end-to-end latency. 

Another approach attempts to overcome satellite mobility from a higher perspective by introducing new anchor management mechanisms without modifying the mobile networks themselves~\cite{skycastle, MM2, MM3}. These efforts often focus only on the latency variations caused by satellite mobility, rather than considering the entire end-to-end path. Work~\cite{skycastle} proposes a global mobility management mechanism, which provides low-latency global internet service to users through an anchor manager and distributed satellite anchor points. However, this mechanism focuses on latency changes within the satellite network and can only allocate anchor points that meet latency requirements rather than the optimal anchor point.

Our proposed architecture involves deploying the network function (i.e., S-UPF) on satellites. However, by redesigning the PDU session establishment process, we do not introduce additional overhead on the control plane. On the other hand, by expanding the available anchor points and comprehensively considering multiple end-to-end paths, we achieve a significant reduction in end-to-end latency, surpassing existing schemes. We consider both the user plane and the control plane and conduct comprehensive system-level experiments on a real data-driven platform, ensuring that the experimental results closely reflect real-world scenarios.