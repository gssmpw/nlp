\documentclass[sigconf]{acmart}
% \pdfoutput=1
%\usepackage[UTF8]{ctex}
%\usepackage[letterpaper]{geometry}
% \settopmatter{printacmref=false} % Removes citation information below abstract
% \renewcommand\footnotetextcopyrightpermission[1]{} % removes footnote with conference information in first column
% \pagestyle{plain} % removes running headers
\usepackage{graphicx}
\usepackage{amsmath}
\usepackage{booktabs}
\usepackage{algorithm}
\usepackage{algorithmic}
\usepackage{amsfonts}
\usepackage{multirow}
\usepackage{makecell}
\usepackage{subfigure}
\usepackage{color}
\usepackage{bm}
\usepackage{epstopdf}
\usepackage{url}
%\usepackage{flushend}
\usepackage[cal=cm]{mathalfa}
\usepackage{balance}
\usepackage{threeparttable}
\usepackage{lipsum}
\usepackage{enumitem}
 \usepackage{pifont}
 \usepackage{tabularx}
 \usepackage{makecell}
 \usepackage{float}
 % \usepackage{authblk}


\newcommand{\argmin}{\operatornamewithlimits{argmin}}
\newcommand{\argmax}{\operatornamewithlimits{argmax}}


\setlength{\paperheight}{11in}
\setlength{\paperwidth}{8.5in}


\renewcommand{\algorithmicrequire}{ \textbf{Input:}}     %Use Input in the format of Algorithm
\renewcommand{\algorithmicensure}{ \textbf{Output:}}    %UseOutput in the format of Algorithm
%%
%% \BibTeX command to typeset BibTeX logo in the docs
\AtBeginDocument{%
  \providecommand\BibTeX{{%
    \normalfont B\kern-0.5em{\scshape i\kern-0.25em b}\kern-0.8em\TeX}}}


\settopmatter{printacmref=false} 
\renewcommand\footnotetextcopyrightpermission[1]{}

\begin{document}
\title{ChorusCVR: Chorus Supervision for Entire Space Post-Click Conversion Rate Modeling}



\author{Wei Cheng$^\S$}
\affiliation{
  \institution{Kuaishou Technology}
  \country{chengwei07@kuaishou.com}
 }
 
\author{Yucheng Lu$^\S$}
\affiliation{
  \institution{Kuaishou Technology}
  \country{luyucheng@kuaishou.com}
 }

\author{Boyang Xia$^\S$}
\affiliation{
  \institution{Kuaishou Technology}
  \country{xiaboyang@kuaishou.com}
 }
 
\author{Jiangxia Cao$^{\star}$}
\thanks{$^\S$Equal contributions to this work}
\thanks{$^\star$Corresponding author.}
\affiliation{
  \institution{Kuaishou Technology}
  \country{caojiangxia@kuaishou.com}
 }

 \author{Kuan Xu}
\affiliation{
  \institution{Kuaishou Technology}
  \country{xukuan@kuaishou.com}
 }


 \author{Mingxing Wen}
\affiliation{
  \institution{Kuaishou Technology}
  \country{wenmingxing@kuaishou.com}
 }

\author{Wei Jiang}
\affiliation{
  \institution{Kuaishou Technology}
  \country{jiangwei@kuaishou.com}
 }


 \author{Jiaming Zhang}
\affiliation{
  \institution{Kuaishou Technology}
  \country{zhangjiaming07@kuaishou.com}
 }
 
  \author{Zhaojie Liu}
\affiliation{
  \institution{Kuaishou Technology}
  \country{zhaotianxing@kuaishou.com}
 }

 \author{Liyin Hong}
\affiliation{
  \institution{Kuaishou Technology}
  \country{hongliyin@kuaishou.com}
 }


 \author{Kun Gai}
\affiliation{
  \institution{Unaffiliated}
  \country{gai.kun@qq.com}
 }

 \author{Guorui Zhou}
\affiliation{
  \institution{Kuaishou Technology}
  \country{zhouguorui@kuaishou.com}
 }



 
\renewcommand{\shorttitle}{ChorusNet}


\begin{abstract}
\begin{abstract}

%Large Language Models (LLMs) trained with single-turn reward often respond passively to 
%under-specified
%open-ended or ambiguous
%user requests, failing to guide users in discovering their ultimate intents. This results in ineffective and inefficient human-LLM collaborations, with users needing multiple iterations to clarify their goals. 
%To address this limitation, 
%
Large Language Models %(LLMs) 
are typically trained with next-turn rewards, limiting their ability to optimize for long-term interaction. As a result, 
they often respond passively to ambiguous or open-ended user requests, 
%failing to guide users toward discovering their ultimate intents and leading to ineffecive conversations. 
failing to help users reach their ultimate intents and leading to inefficient conversations.
To address these limitations,
%
we introduce \mbox{\name{}}, a novel and general training framework that enhances multiturn human-LLM collaboration.
%The key innovation of \name{} 
Its key innovation
is a collaborative simulation that estimates the long-term contribution of responses using  \textit{Multiturn-aware Rewards}.
%of model responses. 
By reinforcement fine-tuning these rewards, \name{} goes beyond responding to user requests, and actively uncovers user intent and offers insightful suggestions---a key step towards more human-centered AI.
% \name{} is reword-model training-free, scalable, and plug-and-play. 
% which can be integrated with prevailing reinforcement fine-tuning frameworks.
%We conduct a large-scale user study with \numturker{} judges, where \name{} increases user satisfaction by \realsatisfyimprov and reduces user time by \realtimeimprov{}.
%Additionally, 
We also devise a multiturn interaction benchmark 
%in simulated environments 
with three challenging tasks such as document creation. 
%Compared to our strongest baselines, 
\name{} significantly outperforms our 
%strongest 
baselines with averages of \taskimprov higher task performance and \itrimprov improved interactivity by LLM judges.
%% Not sure where to put this, but we could bring it back:
%Overall, \name{} advances human-centered AI by enabling more effective and efficient human-LLM collaboration.
Finally, we conduct a large user study with \numturker{} judges, where \name{} increases user satisfaction by \realsatisfyimprov and reduces user time by \realtimeimprov{}.

\end{abstract}

\begin{figure*}[t]
    \centering
    \includegraphics[width=1.0\linewidth]{figures/examples_v4}
    \vspace{-20pt}
    \caption{Real examples from \name{} and non-collaborative LLM fine-tuing. (a) Non-collaborative LLM fine-tuing relies single-turn rewards on immediate responses, which exhibits passive behaviors that follow the user's requests, leading to user frustration, less efficient process, and less satisfactory results. (b) \name{} incorporates Multiturn-aware Rewards from collaborative simulation, enabling forward-looking strategies. This results in more high-performing, efficient, and interactive conversations that anticipate future needs, propose timely clarification, and provide insightful suggestions. 
    }
    \label{fig:examples}
\end{figure*}

\end{abstract}




\begin{CCSXML}
<ccs2012>
<concept>
<concept_id>10002951.10003317.10003347.10003350</concept_id>
<concept_desc>Information systems~Recommender systems</concept_desc>
<concept_significance>500</concept_significance>
</concept>
% <concept>
% <concept_id>10010147.10010257.10010293.10010294</concept_id>
% <concept_desc>Computing methodologies~Neural networks</concept_desc>
% <concept_significance>500</concept_significance>
% </concept>
</ccs2012>
\end{CCSXML}

\ccsdesc[500]{Information systems~Recommender systems}
% \ccsdesc[500]{Computing methodologies~Neural networks}

\keywords{Ranking Model; Post-Click Conversion Rate Estimation;}

\maketitle



% 冲!
%\vspace{-4mm}
\section{Introduction}

In machine learning, the availability of vast amounts of unlabeled data has created an opportunity to learn meaningful representations without relying on costly labeled datasets \cite{jaiswal2020survey,shurrab2022self,jing2020self}. Self-supervised learning has emerged as a powerful solution to this problem, allowing models to leverage the inherent structure in data to build useful representations. Among self-supervised methods, contrastive learning (CL) is widely adopted for its ability to create robust representations by distinguishing between similar (positive) and dissimilar (negative) data pairs. With success in fields like image and language processing \cite{chen2020simple,radford2021learning}, contrastive learning now also shows promise in domains where cross-modal, noisy, or structurally complex data make labeling especially challenging \cite{liu2021drop,vishnubhotla2024towards,chen2023instance}.


Traditional contrastive learning methods primarily aim to bring positive pairs---often augmentations of the same sample---closer together in representation space. While effective, this approach often struggles with real-world challenges such as noise in views, variations in data quality, or shifts introduced by complex transformations, where positive pairs may not perfectly align. Additionally, in tasks requiring domain generalization, aligning representations across diverse domains (e.g., variations in style or sensor type) is critical but difficult to achieve with standard contrastive learning, which typically lacks mechanisms for incorporating domain-specific relationships. These limitations highlight the need for a more flexible approach that can adapt alignment strategies based on the data structure, allowing for finer control over similarity and dissimilarity among samples. 


To address this challenge, we introduce a novel \emph{generalized contrastive alignment} (GCA) framework, which reinterprets contrastive learning as a distributional alignment problem. Our method allows flexible control over the alignment of samples by defining a target transport plan, \(\mathbf{P}_{tgt}\), that serves as a customizable alignment guide. For example, setting \(\mathbf{P}_{tgt}\) to resemble a diagonal matrix encourages each positive to align primarily with itself or its augmentations, thereby reducing the effect of noise between views. Alternatively, we can incorporate more complex constraints, such as weighting alignments based on view quality or enforcing partial alignment structures where noise or data heterogeneity is prevalent. This flexibility enables GCA to adapt effectively to a wide range of tasks, from simple twin view alignments to scenarios with noisy or variably aligned data.

Our approach also bridges connections between GCA and established methods, such as InfoNCE (INCE) \cite{oord2018representation}, Robust InfoNCE (RINCE) \cite{chuang2022robust}, and BYOL \cite{grill2020bootstrap}, demonstrating that these can be viewed as iterative alignment objectives with Bregman projections \cite{cai2022developments,grathwohl2019your}. This perspective allows us to systematically analyze and improve uniformity within the latent space, a property that enhances representation quality and ultimately boosts downstream classification performance.

We validate our method through extensive experiments on both image classification and noisy data tasks, demonstrating that GCA’s unbalanced OT (UOT) formulations improve classification performance by relaxing our constraints on alignment. Our results show that \ours~offers a robust and versatile framework for contrastive learning, providing flexibility and performance gains over existing methods and presenting a promising approach to addressing different sources of variability in self-supervised learning.


The contributions of this work include:
\begin{itemize}
    \item A new framework called \emph{generalized contrastive alignment} (GCA), which reinterprets standard contrastive learning as a distributional alignment problem, using optimal transport to provide flexible control over alignment objectives. This approach allows us to derive a novel class of contrastive losses and algorithms that adapt effectively to varied data structures and build  customizable transport plans.

    \item We present GCA-UOT, a contrastive learning method that achieves strong performance on standard augmentation regimes and excels in scenarios with more extreme augmentations or data corrupted by transformations. GCA-UOT leverages unbalanced transport to adaptively weight positive alignments, enhancing robustness against view noise and cross-domain variations.

    \item We provide theoretical guarantees for the convergence of our GCA-based methods and show that our alignment objectives improve representation quality by enhancing the uniformity of negatives and strengthening alignment within positive pairs. This leads to more discriminative and resilient representations, even in challenging data conditions.

    \item Empirically, we demonstrate the effectiveness of \ours~in both image classification and domain generalization tasks. Through flexible, unbalanced OT-based losses, \ours~achieves superior classification performance and adapts alignment to include domain-specific information where relevant, without compromising classification accuracy in domain generalization.
\end{itemize}



%
%
\section{Methodology}


\begin{figure*}[t]
  \centering
\includegraphics[width=0.70\textwidth]{main_figure_v2.png}
  \vspace{-1em}
  \caption{Systematic overview of our Chorus CVR model.}
  \label{choruscvr}
  \vspace{-1em}
\end{figure*}


\subsection{Preliminary}
In the ranking stage of industrial recommendation system, all \textit{exposure} user-item pairs will be collected and formed as a data-streaming for model training, i.e., $\mathcal{D}$.
Specifically, each user-item sample in $\mathcal{D}$ could represent as $(u, i, \{\mathbf{x}_u, \mathbf{x}_i,$ $\mathbf{x}_{ui}\}, o_{ui}, r_{ui}) \in \mathcal{D}$, where $u$/$i$ denotes the user-item pair, $\mathbf{x}_u\in\mathbb{R}^{d_u}, \mathbf{x}_i\in\mathbb{R}^{d_i}, \mathbf{x}_{ui}\in\mathbb{R}^{d_{ui}}$ are the user-side features (e.g., user ID), item-side features (e.g., item ID), and item-aware cross features (e.g., SIM \cite{sim}).
%
The $o_{ui}\in\{0,1\}$ and $r_{ui}\in\{0,1\}$ are user-item ground-truth interacted labels, where $o_{ui}$ denotes whether user $u$ clicked item $i$ and $r_{ui}$ denotes whether user $u$ converted item $i$. 
%
According to the entire \textbf{\textit{exposure} space} $\mathcal{D}$, we could further obtain several subset spaces:
%
\begin{itemize}[leftmargin=*,align=left]
\item \textbf{\textit{Click} space} $\mathcal{O}\in\mathcal{D}$, if click label $o_{ui} = 1$.
\item \textbf{\textit{un-Click} space} $\mathcal{N}=\mathcal{D} - \mathcal{O}$, if click label $o_{ui} = 0$.
\item \textbf{\textit{Conversion} space} $\mathcal{R}\in\mathcal{O}$: if label $o_{ui}=1$ and $r_{ui}=1$.
\item \textbf{\textit{un-Conversion} space} $\mathcal{M}=\mathcal{O}-\mathcal{R}$: if label $o_{ui}=1$ and $r_{ui}=0$.
\end{itemize}
%
Based on them, a simple ranking model can be formed as:
\begin{equation}
% \small
\begin{split}
&\hat{y}^{ctr}_{ui} = \texttt{MLP}^{ctr}(\mathbf{x}_{ui}),\quad \hat{y}^{cvr}_{ui} = \texttt{MLP}^{cvr}(\mathbf{x}_{ui}),\\
&\mathbf{x}_{ui} = \texttt{Multi-Task-Encoder}(\mathbf{x}_u\oplus \mathbf{x}_i\oplus \mathbf{x}_{ui}),\\
\end{split}
\label{base}
\end{equation}
where the $\oplus$ denotes the concatenate operator, $\mathbf{x}\in\mathbb{R}^d$ is the encoded hidden states, and $\texttt{MLP}(\cdot)$ denotes a stacked neural-network. We use a share-bottom based multi-task paradigm to predict CTR and CVR scores, $\hat{y}^{ctr},\hat{y}^{cvr}$.
%
Next, we directly minimize the cross-entropy binary classification loss to train CTR tower and CVR towers with corresponding space samples:
%
%
%
\begin{equation}
% \small
% \footnotesize
\begin{split}
&\mathcal{L}^{ctr} = - \frac{1}{|\mathcal{D}|}\big(\sum_{(u,i)\in\mathcal{D}}\delta(\hat{y}^{ctr}_{ui}, o_{ui})\big),\\
&\mathcal{L}^{cvr} = - \frac{1}{|\mathcal{O}|}\big(\sum_{(u,i)\in\mathcal{O}}\delta(\hat{y}^{cvr}_{ui}, r_{ui})\big).
\end{split}
\label{crossentropy}
\end{equation}
%
% 
%
In inference, given the hundreds item candidates in a certain user request, we could obtain predicted CTCVR by  $\hat{y}^{ctcvr}_{ui} =\hat{y}^{ctr}_{ui} \cdot \hat{y}^{cvr}_{ui}$ for each item, which is used for final ranking. Then top K highest items will be returned and shown to user. The CVR are learned in click space during training but be predicted in an assumed explore space during inference, which brings up the question of sample selection bias problem.
% 





To alleviate sample selection bias,  ESMM \cite{essm} expand the click-space CVR learning task to exposure-space CTCVR learning task, to directly solve the inconsistency between training and inference:
\begin{equation}
% \small
\begin{split}
\mathcal{L}^{ctcvr} = &- \frac{1}{|\mathcal{D}|}\Big(\sum_{(u,i)\in\mathcal{D}}\delta(\hat{y}^{ctr}_{ui}\cdot\hat{y}^{cvr}_{ui}, o_{ui}\cdot r_{ui})\Big)
\end{split}
\label{ctxcvr}
\end{equation}
which treats all un-clicked samples as negative samples of CTCVR task. However those un-clicked samples that would be converted if clicked, which are falsely negative samples,  still leads to missing not at random (MNAR) problem  \cite{multiipw}. To mitigate this problem, inverse propensity weighting (IPW) \cite{multiipw,escm2} based method inversely weight the CVR loss in click space by propensity score of observing  $(u,i)$ in click space $\mathcal{O}$, to eliminate the influence of click event to CVR estimation in entire space $D$
\begin{equation}
\small
\begin{split}
\mathcal{L}^{cvr}_{IPW} =& - \frac{1}{|\mathcal{O}|}\Big(\sum_{(u,i)\in\mathcal{O}}\frac{\delta(\hat{y}^{cvr}_{ui}, r_{ui})}{\hat{y}^{ctr}_{ui}}\Big),
\end{split}
\label{cvripw}
\end{equation}
% 
%
Our method is based on above ESMM with IPW  framework. Although alleviating SSB and MNAR problem, IPW-based methods still lack reasonable labels for \textit{un-clicked} samples, which we solve by generating discriminative and robust soft labels.






\subsection{ChorusCVR}
In this section, we dive into ChorusCVR and explain how we realize entire-space debiased CVR learning by generating discriminative and robust soft CVR labels (as shown in Figure~\ref{choruscvr}).

\subsubsection{Negative sample Discrimination Module (NDM)}

As mentioned before, the soft labels introduced by previous works are suboptimal for lack either discriminability or robustness. As shown in Figure~\ref{intro} (d), an ideal discrimination surface should separate the factual negative samples (clicked but un-converted) from positive samples (clicked \& converted), and factual negative samples from ambiguous negative samples (un-clicked). With this in mind, we find the ideal discrimination surface implies a new task, CTunCVR prediction. We formulate CTunCVR labels as:
\begin{equation}
y^{ctuncvr} = o_{ui} * (1-r_{ui}) = 
\begin{cases} 
1 & o_{ui}=1~\&~r_{ui} = 0, \\
0 &  o_{ui}=0, \\
0 & o_{ui}=1~\&~r_{ui} = 1,
\end{cases}
\end{equation}
where only \textit{clicked but un-converted} samples are positive samples, both \textit{clicked \& converted} and \textit{un-clicked} samples are negative samples. Instead of directly predicting CTunCVR score in exposure space, we follow a typical two-stage prediction paradigm to obtain CTunCVR to 
reduce cumulative error. We firstly introduce an additional unCVR tower to predict unCVR score $\hat{y}^{uncvr}$, then combine it with $\hat{y}^{ctr}$ to form CTunCVR score:
% 
\begin{equation}
% \small
\begin{split}
\hat{y}^{uncvr}_{ui} = \hat{y}^{ctr}_{ui}\cdot\hat{y}^{cvr}_{ui}\quad \quad
\hat{y}^{ctuncvr}_{ui} =\hat{y}^{ctr}_{ui}\cdot\hat{y}^{uncvr}_{ui}
\end{split}
\label{uncvr}
\end{equation}
Then we can naturally optimize CTunCVR objective in exposure space by cross entropy loss: 
\begin{equation}
% \small
\begin{split}
\mathcal{L}^{ctuncvr} = - \frac{1}{|\mathcal{D}|}\Big(\sum_{(u,i)\in\mathcal{D}}\delta(\hat{y}^{ctuncvr}_{ui}, o_{ui} * (1-r_{ui})\Big).
\end{split}
\label{uncvr}
\end{equation}
% With the help of this formulation, we can narrow down the problem to the accurate estimation of unCVR. 
With the help of $\mathcal{L}^{ctuncvr}$ and an extra \textbf{unCVR prediction result} $\hat{y}^{uncvr}_{ui}$, we can narrow down the aforementioned problem to consider \textbf{R1. Discriminability} and \textbf{R2. Robustness} problem at same time.
%
For the $\hat{y}^{uncvr}_{ui}$ generation, we add an mirror unCVR tower which similar with the Eq.(\ref{base}) and (\ref{crossentropy}):
%
%
%
% 
%
%
% $
% 
\begin{equation}
% \smalluncvr
\begin{split}
\hat{y}^{uncvr}_{ui} &= \texttt{MLP}^{uncvr}(\mathbf{x}_{ui}), \\
\mathcal{L}^{uncvr} = - \frac{1}{|\mathcal{O}|}\Big(&\sum_{(u,i)\in\mathcal{O}}\delta\big(\hat{y}^{uncvr}_{ui}, 1-r_{ui})\big)\Big)
\end{split}
\label{uncvr}
\end{equation}
% 
Next, analogously with the Eq.(\ref{cvripw}), we then adopt the predicted click $\hat{y}^{ctr}_{ui}$ to inversely weight the unCVR error, to $\mathcal{L}^{uncvr}$ as:
% 
\begin{equation}
% \small
\begin{split}
\mathcal{L}^{uncvr}_{IPW} = - \frac{1}
{|\mathcal{O}|}\Big(\sum_{(u,i)\in\mathcal{O}}\frac{\delta(\hat{y}^{uncvr}_{ui}, 1-r_{ui})}{\hat{y}^{ctr}_{ui}}\Big)
\end{split}
\label{uncvr}
\end{equation}
In this way, the \textit{click} space tendency can be alleviated that higher/lower $pCTR$ sample will declined/enhanced for a fair training. So far we obtain debiased unCVR soft labels, which we will utilize to help the CTunCVR training and CVR component supervision.



\subsubsection{Soft Alignment Module (SAM)}
Up to now, we fulfill the initial goal of obtain high-quality soft labels in un-clicked space. In this section we present the solution to utilize the unCVR score as soft labels to supervise CVR learning, which we call \emph{soft alignment mechanism}. We first use $1-unCVR$ manner as soft labels for entropy-based CVR learning. In the same time, we also use $1-CVR$ manner to generate soft labels for unCVR learning, in a mutual supervision fashion to align unCVR predictions to CVR. All these objectives are inversely weighted by predicted CTR in a IPW paradigm (see $\mathcal{L}^{align1}_{IPW}$ and $\mathcal{L}^{align2}_{IPW}$ in Figure.~\ref{choruscvr}). To further alleviate SSB for un-click space, we also propose a \textit{un-click space IPW} approach, to inversely weight the un-click samples with $1-pCTR$ for CTR and unCVR alignment objectives (see $\mathcal{L}^{align3}_{IPW}$ and $\mathcal{L}^{align4}_{IPW}$ in Figure.~\ref{choruscvr}). Overall, all alignment objectives are as follows:
\begin{equation}
% \small
\begin{split}
\mathcal{L}^{align}_{IPW} = &- \frac{1}{|\mathcal{O}|}\big(\frac{\delta(\hat{y}^{cvr}_{ui}, 1-\texttt{sg}(\hat{y}^{uncvr}_{ui}))}{\hat{y}^{ctr}_{ui}}\big)
- \frac{1}{|\mathcal{N}|}\big(\frac{\delta(\hat{y}^{cvr}_{ui}, 1-\texttt{sg}(\hat{y}^{uncvr}_{ui}))}{1-\hat{y}^{ctr}_{ui}}\big)\\
&- \frac{1}{|\mathcal{O}|}\big(\frac{\delta(\hat{y}^{uncvr}_{ui}, 1-\texttt{sg}(\hat{y}^{cvr}_{ui}))}{\hat{y}^{ctr}_{ui}}\big)
- \frac{1}{|\mathcal{N}|}\big(\frac{\delta(\hat{y}^{uncvr}_{ui}, 1-\texttt{sg}(\hat{y}^{cvr}_{ui}))}{1-\hat{y}^{ctr}_{ui}}\big)
\end{split}
\label{soft}
\end{equation}

where the $\texttt{sg}(\cdot)$ means the stop gradient function, the $\hat{y}^{ctr}_{ui}, (1 - \hat{y}^{ctr}_{ui})$ denote the click propensity in the \textit{click} and \textit{un-click} space, respectively.
% 
%
All losses of our ChorusCVR are as follows:
\begin{equation}
% \small
\begin{split}
\mathcal{L} = \mathcal{L}^{ctcvr} + \mathcal{L}^{cvr}_{IPW} + \mathcal{L}^{ctuncvr} + \mathcal{L}^{uncvr}_{IPW} + \mathcal{L}^{align}_{IPW}
\end{split}
\label{soft}
\end{equation}
In this way, our ChorusCVR  make CVR and unCVR supervise each other during training, which results in an equilibrium. 





\begin{table}[t!]
\centering
\caption{Offline results(\%) in terms of CTR-AUC, CTCVR-AUC and logloss at Ali-CCP and Kuaishou.}
\vspace{-1em}
\setlength{\tabcolsep}{3pt}{
% \resizebox{\linewidth}{!}{
\begin{tabular}{l|cc|cc}
\toprule
\multirow{4}{*}{\makecell{Models}} 
& \multicolumn{2}{c|}{Ali-CCP} & \multicolumn{2}{c}{Kuaishou}   \\ 
\cmidrule(r){2-5} & \multicolumn{2}{c|}{AUC}  & \multicolumn{2}{c}{AUC}   \\ 
\cmidrule(r){2-3} \cmidrule(r){4-5} & CVR  & CTCVR &  CVR & CTCVR  \\
\hline
ESMM \cite{essm} & 0.5963 & 0.5802 & 0.8609 & 0.9276\\ 
ESCM$^2$-DR \cite{escm2} & 0.6354 & 0.6203& 0.8617 & 0.9280\\
ESCM$^2$-IPW \cite{escm2} & 0.6385 & 0.6126& 0.8619 & 0.9283\\
UKD \cite{ukd} & 0.6451 & 0.6282& 0.8615 & 0.9279\\
DCMT \cite{dcmt} & 0.6447 & \underline{0.6375} & \underline{0.8628} & 0.9281\\
DDPO \cite{ddpo} & \underline{0.6496} & 0.6326& 0.8623 & \underline{0.9289}\\
NISE \cite{nise} & 0.6418 & 0.6291 & 0.8614 & 0.9274\\
ChorusCVR & \textbf{0.6589} & \textbf{0.6401}& \textbf{0.8639} & \textbf{0.9304}\\
\hline
Rela.Impr. & +1.43\% & +0.40\% & +0.13\% & +0.16\% \\
\hline
ChorusCVR w/o NDM & 0.6498 & 0.6347& 0.8625 & 0.9284\\
ChorusCVR w/o SAM & 0.6407 & 0.6139 & 0.8622 & 0.9281\\
% DCMT+Ctuncvr loss & 0.6492 & 0.6387& 0.8626 & 0.9293\\
\bottomrule
\end{tabular}
}
\vspace{-2em}
% }
\label{mainoffline}
\end{table}




\section{Experiments}
In this section, we conduct extensive online and offline experiments to verify the efficacy of our model following 3 research questions: 
\begin{itemize}[leftmargin=*,align=left]
\item \textbf{RQ1:} How does our model perform on industrial recommendation datasets?
\item \textbf{RQ2:} Can our model bring improvements of e-commerce A/B test metrics on online product environment?
\item \textbf{RQ3:} Can our model alleviate the bias on un-clicked samples?
\end{itemize}
\textbf{Datasets.} To evaluate the performance of our method and comparison baselines, we conduct comprehensive experiments on both public and industrial datasets.



• Public dataset: The Ali-CCP (Alibaba Click and Conversion Prediction) dataset \cite{essm} is a benchmark dataset for CVR and CTR prediction, which collected from traffic logs in Taobao e-commerce platform. Ali-CCP contains 33 features and 84M samples. The training set contains 42M exposure samples, 1.6M click samples and 9k conversion samples and the test set contains 42M exposure samples, 1.7M click samples and 9.4k conversion samples.

• Industrial dataset: The industrial dataset is colloected from Kuaishou e-commerce live-streaming platform. Kuaishou e-commerce live-streaming is a popular content interest e-commerce platform with \textbf{tens of millions} daily active users. The dataset contains more than 1000 features including user profiles (gender, age, etc), user actions (click, buy, etc), seller profiles (industry, sales, etc) and goods information. 

\textbf{Compared Methods.} ESMM \cite{essm} learns CVR task through a CTR task and a CTCVR task to alleviate ssb and data sparsity issues. ESCM$^2$-IPW \cite{escm2} incorporates the inverse propensity weighting (IPW) \cite{ipw} method to regularize ESMM’s CVR estimation. ESCM$^2$-DR \cite{escm2} augments ESCM$^2$-IPW with an auxiliary imputation loss to models the CVR with the Doubly Robust(DR) method. UKD \cite{ukd} introduces a transfer adversarial learning approach to generate soft conversion pseudo-labels in the unclicked space. DCMT \cite{dcmt} proposed a counterfactual mechanism to directly debias CVR in the entire space. DDPO \cite{ddpo} employs a conversion propensity prediction network to generate soft conversion pseudo-labels in the un-clicked space. NISE \cite{nise} follows a semi-supervised learning paradigm and use predicted CVR as CVR pseudo labels for un-clicked samples.

\textbf{Offline Results (RQ1):} 
We evaluate the efficacy of our model by the Area Under ROC (AUC) of CVR and CTCVR prediction tasks. The experimental results on two datasets are shown in Table 1. On Ali-CCP dataset, our model outperforms DDPO by 1.4\% (\textbf{0.6589} v.s. 0.6496) for CVR-AUC and DCMT by 0.4\% (\textbf{0.6401} v.s. 0.6375) for CTCVR-AUC, respectively. On Kuaishou dataset, our model outperforms DCMT by 0.1\% (\textbf{0.8639} v.s. 0.8628) for CVR-AUC and DDPO by 0.16\% (\textbf{0.9304} v.s. 0.9289) for CTCVR-AUC, respectively. In a nutshell, our model consistently outperforms the best-performing baselines on both two datasets in a large margin in terms of two tasks. 
\begin{figure}[t]
  \centering
  \includegraphics[width=5cm]{pcoc.png}
  \vspace{-1em}
  \caption{The PCOC analysis.}
  \label{pcoc}
  \vspace{-2em}
\end{figure}
\textit{Ablation Study.} To verify the efficacy of each parts of ChorusCVR, we evaluate the performance of variants without NDM and SAM. It is noteworthy that the unCVR objective is optimized by 1 - $y^{CVR}$ label in click space in `ChorusCVR w/o NDM'. We find that without NDM, our model degrades 1.3\% (\textbf{0.6589} v.s. 0.6498) on CVR AUC on Ali-CCP dataset, for lack of discrimination between negative samples of different levels. Meanwhile, we can observe that without SAM, the `ChorusCVR w/o SAM' performs even poorer than `ChorusCVR w/o NDM', presents 4\% lower CTCVR AUC on Ali-CCP dataset, for lack of reasonable pesudo supervisions on un-click samples. 

\textbf{Online Results (RQ2):}
We deploy our method in production environment of Kuaishou e-commerce live-streaming platform to conduct online A/B testing for 8 days. %As shown in Tab. 2,
Compared with the base model (DCMT), our model presents large improvements on CVR(+0.12\%), orders (+0.851\%) and DAC (+0.705\%) with 95\% confidence intervals. 




\textbf{Analysis (RQ3):}
To investigate whether our method can address the bias in un-clicked samples, we try to compare the  accuracy of cvr predicitons between our method and baselines on un-clicked samples. However, the conversion labels of un-clicked samples are inaccessible in the real world. Inspired by the inverse propensity weighting based methods, we propose a compromised comparison approach, to use samples with low pCTR as a substitute for un-clicked samples and observe the model's pCVR on these low CTR samples. We show the estimated pCVR and actual cvr for samples with different pCVR the deviations between two curves signify the prediction bias of the model.

As shown in Fig. \ref{pcoc}, we can observe that DCMT represents obvious bias on low pCTR samples, which severely overestimates the CVR on low pCtr samples. However, our model accurately predicts the CVR on those low pCtr samples for reasonable CVR supervisions on un-clicked samples. 










\section{Conclusion}
In this paper, to alleviate sample selection bias in CVR prediction task, we propose an effective method ChorusCVR. To generate discriminative and robust soft labels, we propose Negative sample Discrimination Module to obtain soft CTunCVR labels which can separate negative samples of different levels. Then we design a 
Soft Alignment Module for debiased CVR learning in un-click space with soft labels. We demonstrated the superior performance of the proposed
ChorusCVR in offline experiments. In addition, we conduct online A/B testing, obtaining +0.851\% improvements on orders of industrial e-commerce living stream, which demonstrates the effectiveness and universality of ChorusCVR in online
systems. 
% Moreover, ChorusCVR has been deployed on ranking system in KUAISHOU e-commerce living stream.



% \newpage
\balance
\bibliographystyle{ACM-Reference-Format}
\bibliography{sample-base-extend.bib}
\end{document}
\endinput