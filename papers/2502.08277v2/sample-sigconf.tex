\documentclass[sigconf]{acmart}
% \pdfoutput=1
%\usepackage[UTF8]{ctex}
%\usepackage[letterpaper]{geometry}
% \settopmatter{printacmref=false} % Removes citation information below abstract
% \renewcommand\footnotetextcopyrightpermission[1]{} % removes footnote with conference information in first column
% \pagestyle{plain} % removes running headers
\usepackage{graphicx}
\usepackage{amsmath}
\usepackage{booktabs}
\usepackage{algorithm}
\usepackage{algorithmic}
\usepackage{amsfonts}
\usepackage{multirow}
\usepackage{makecell}
\usepackage{subfigure}
\usepackage{color}
\usepackage{bm}
\usepackage{epstopdf}
\usepackage{url}
%\usepackage{flushend}
\usepackage[cal=cm]{mathalfa}
\usepackage{balance}
\usepackage{threeparttable}
\usepackage{lipsum}
\usepackage{enumitem}
 \usepackage{pifont}
 \usepackage{tabularx}
 \usepackage{makecell}
 \usepackage{float}
 % \usepackage{authblk}


\newcommand{\argmin}{\operatornamewithlimits{argmin}}
\newcommand{\argmax}{\operatornamewithlimits{argmax}}


\setlength{\paperheight}{11in}
\setlength{\paperwidth}{8.5in}


\renewcommand{\algorithmicrequire}{ \textbf{Input:}}     %Use Input in the format of Algorithm
\renewcommand{\algorithmicensure}{ \textbf{Output:}}    %UseOutput in the format of Algorithm
%%
%% \BibTeX command to typeset BibTeX logo in the docs
\AtBeginDocument{%
  \providecommand\BibTeX{{%
    \normalfont B\kern-0.5em{\scshape i\kern-0.25em b}\kern-0.8em\TeX}}}


\settopmatter{printacmref=false} 
\renewcommand\footnotetextcopyrightpermission[1]{}

\begin{document}
\title{ChorusCVR: Chorus Supervision for Entire Space Post-Click Conversion Rate Modeling}



\author{Wei Cheng$^\S$}
\affiliation{
  \institution{Kuaishou Technology}
  \country{chengwei07@kuaishou.com}
 }
 
\author{Yucheng Lu$^\S$}
\affiliation{
  \institution{Kuaishou Technology}
  \country{luyucheng@kuaishou.com}
 }

\author{Boyang Xia$^\S$}
\affiliation{
  \institution{Kuaishou Technology}
  \country{xiaboyang@kuaishou.com}
 }
 
\author{Jiangxia Cao$^{\star}$}
\thanks{$^\S$Equal contributions to this work}
\thanks{$^\star$Corresponding author.}
\affiliation{
  \institution{Kuaishou Technology}
  \country{caojiangxia@kuaishou.com}
 }

 \author{Kuan Xu}
\affiliation{
  \institution{Kuaishou Technology}
  \country{xukuan@kuaishou.com}
 }


 \author{Mingxing Wen}
\affiliation{
  \institution{Kuaishou Technology}
  \country{wenmingxing@kuaishou.com}
 }

\author{Wei Jiang}
\affiliation{
  \institution{Kuaishou Technology}
  \country{jiangwei@kuaishou.com}
 }


 \author{Jiaming Zhang}
\affiliation{
  \institution{Kuaishou Technology}
  \country{zhangjiaming07@kuaishou.com}
 }
 
  \author{Zhaojie Liu}
\affiliation{
  \institution{Kuaishou Technology}
  \country{zhaotianxing@kuaishou.com}
 }

 \author{Liyin Hong}
\affiliation{
  \institution{Kuaishou Technology}
  \country{hongliyin@kuaishou.com}
 }


 \author{Kun Gai}
\affiliation{
  \institution{Unaffiliated}
  \country{gai.kun@qq.com}
 }

 \author{Guorui Zhou}
\affiliation{
  \institution{Kuaishou Technology}
  \country{zhouguorui@kuaishou.com}
 }



 
\renewcommand{\shorttitle}{ChorusNet}


\begin{abstract}


Large language models (LLMs) have demonstrated impressive capabilities in disease diagnosis. However, their effectiveness in identifying rarer diseases, which are inherently more challenging to diagnose, remains an open question. Rare disease performance is critical with the increasing use of LLMs in healthcare settings.  This is especially true if a primary care physician needs to make a rarer prognosis from only a patient conversation so that they can take the appropriate next step. To that end, several clinical decision support systems are designed to support providers in rare disease identification. 
Yet their utility is limited due to their lack of knowledge of common disorders and difficulty of use.  

In this paper, we propose \methodname to combine the knowledge LLMs with expert systems.  We use jointly use an expert system and LLM to simulate rare disease chats.  This data is used to train a rare disease candidate predictor model.  Candidates from this smaller model are then used as additional inputs to black-box LLM to make the final differential diagnosis. Thus, \methodname allows for a balance between rare and common diagnoses.  We present results on over 575 rare diseases, beginning with Abdominal Actinomycosis and ending with Wilson's Disease.  Our approach significantly improves the baseline performance of black-box LLMs by over 17\% in Top-5 accuracy. We also find that our candidate generation performance is high (\textit{e.g.} 88.8\% on gpt-4o generated chats).
\end{abstract}




\begin{CCSXML}
<ccs2012>
<concept>
<concept_id>10002951.10003317.10003347.10003350</concept_id>
<concept_desc>Information systems~Recommender systems</concept_desc>
<concept_significance>500</concept_significance>
</concept>
% <concept>
% <concept_id>10010147.10010257.10010293.10010294</concept_id>
% <concept_desc>Computing methodologies~Neural networks</concept_desc>
% <concept_significance>500</concept_significance>
% </concept>
</ccs2012>
\end{CCSXML}

\ccsdesc[500]{Information systems~Recommender systems}
% \ccsdesc[500]{Computing methodologies~Neural networks}

\keywords{Ranking Model; Post-Click Conversion Rate Estimation;}

\maketitle



% 冲!
\section{Introduction}


\begin{figure}[t]
  \centering
  \includegraphics[width=7cm]{head_figure.pdf}
  \vspace{-1em}
  \caption{A conceptual comparison between our proposed ChorusCVR and existing CVR models on the perspective of the discrimination spaces of soft labels.}
  \label{intro}
  \vspace{-2em}
\end{figure}
Recommender systems are crafted to provide users with personalized content (videos, products and ads, \emph{etc.,}) that match their preferences \cite{youtubenet,sim,mmoe,din}. Generally, industrial RecSys typically divided into two major stages. 1) Retrieval stage, which aims to search thousands of related candidates from massive item pool. 2) Ranking stage, which aims to estimate interaction probability, \emph{e.g.,} click-through rate (CTR) and post-click conversion rate (CVR), for each user-item pair for retrieved candidates, and select a set of best items for users. In this paper, we focus on the post-click conversion rate (CVR) estimation task during ranking stage.

\textit{Problem statement.} Typically, a positive CVR sample follows the following data funnel: 
%
\textit{exposure} $\mathcal{D}$$\to$\textit{click} $\mathcal{O}$$\to$\textit{conversion} $\mathcal{R}$, where the \textit{click} space $\mathcal{O}$ is around about $4\sim6\%$ of \textit{exposure} space $\mathcal{D}$ and the \textit{conversion} space $\mathcal{R}$ takes up $2\sim4\%$ of \textit{click} space $\mathcal{O}$.
%
Different with CTR which is learned using exposure space samples, CVR is typically learned using only click space samples because we are unaware the un-clicked samples would be converted or not. 
% 
%
However, during online inference, the CTR and CVR scores are utilized in the same assumed exposure space, which leads to a well-known mismatch sample selected bias (SSB) issue \cite{ssb1,ssb2,ssb3,dcmt}, that CVR learning module is trained in \textit{click} space $\mathcal{O}$ but is used for inference at  \textit{exposure} space $\mathcal{D}$.
%









\textit{Motivation.} To alleviate the SSB problem, previous wisdom introduce several techniques to extend CVR task to \textit{exposure} space.
%
Specifically, ESMM \cite{essm} propose a click-through \& conversion rate (CTCVR) task to merge two CVR and CTR scores as one score to supervised it in \textit{exposure} space, which successfully extend the CVR to entire space to solve space inconsistency between training and inference.
%
Unfortunately, the CTCVR loss made a strong assumption that \textbf{\textbf{un-clicked} training samples are hard negative samples in CTCVR training}. This assumption overlooks some ambiguous negative samples which may be easy for users to buy after he/she clicked, but without chance to be clicked yet \cite{multiipw,dcmt}.
%
To alleviate this false negative sample issue, the recent works are dedicated to find reasonable pseudo soft labels to for \textbf{un-clicked} sample learning.
%
Specifically, the DCMT \cite{dcmt} propose to regularize the CVR objectives by a complementary constraint with a novel counterfactual CVR objective. 
%
For counterfactual CVR learning, DCMT first assumes all un-clicked items as positive samples while all converted items are negative samples, and then apply a $CVR = 1 - counterfactualCVR$ constraints for CVR learning module, as shown in Figure~\ref{intro}(b). 
%
Besides, the NISE \cite{nise} and DDPO \cite{ddpo} first utilize the outputs of an additional CVR tower learned in click space to act as pseudo soft label, and then employ a cross-entropy constraints $CVR\thickapprox extraCVR$ in un-click space, as shown in Figure~\ref{intro}(c).
% 

It has come to our attention that the quality of soft CVR labels is the key to mitigating SSB issues.  So we ask, \textit{what requirements should an ideal soft CVR label satisfies}? Our key insight is, an ideal soft label should at least satisfy two requirements: \textbf{R1. Discriminability}: for a clicked sample, the label can discriminate it would be converted or not; \textbf{R2. Robustness}: for un-converted sample, the label can separate the factually un-converted sample in click space from those ambiguous un-converted sample in un-click space. With this in mind, we  present the discrimination surface of soft labels in existing methods (DCMT, NISE and DDPO) in Figure~\ref{intro}. We find their discrimination surface are fully overlapped with certain part of the surfaces of  ESMM (CTCVR task w/o soft labels), which miss either \textbf{R1} or \textbf{R2}. Thus, few of existing methods can meet all these requirements.

To fill this gap, we present a novel entire-space dual multi-task learning model, namly \textbf{ChorusCVR}, to realize effective CVR learning in un-click space that  fulfills both \textbf{R1} and \textbf{R2} (see Figure~\ref{intro}(d)).  The ChorusCVR consists of two modules, \emph{i.e.,} \textbf{N}egative sample \textbf{D}iscrimination \textbf{M}odule (NDM) and \textbf{S}oft \textbf{A}lignment \textbf{M}odule (SAM). In NDM, we introduce a novel CTunCVR auxiliary task, to provide robust soft CVR labels with the ability to discriminate factual CVR negative samples (clicked but un-converted) and ambiguous CVR negative samples (un-clicked).  In SAM, we utilize generated CTunCVR soft outputs to supervise  CVR learning with several alignment objectives, to realize debiased CVR learning in entire-space. Our contributions can be summarized as follows:
\begin{itemize}[leftmargin=*,align=left]
    \item We introduce a novel CTunCVR auxiliary task to provide soft CVR labels with both discriminability and robustness in entire space.
    \item We propose a novel ChorusCVR model with effective alignment objectives for debiased CVR modelling in entire space.
    \item We conduct extensive experiments on both public and production environment datasets and online A/B testing to verify the efficacy of our method, which show that our ChorusCVR achieves superior performance over all existing state-of-the-art methods.  
\end{itemize}
%
%
\section{Methodology}


\begin{figure*}[t]
  \centering
\includegraphics[width=0.70\textwidth]{main_figure_v2.png}
  \vspace{-1em}
  \caption{Systematic overview of our Chorus CVR model.}
  \label{choruscvr}
  \vspace{-1em}
\end{figure*}


\subsection{Preliminary}
In the ranking stage of industrial recommendation system, all \textit{exposure} user-item pairs will be collected and formed as a data-streaming for model training, i.e., $\mathcal{D}$.
Specifically, each user-item sample in $\mathcal{D}$ could represent as $(u, i, \{\mathbf{x}_u, \mathbf{x}_i,$ $\mathbf{x}_{ui}\}, o_{ui}, r_{ui}) \in \mathcal{D}$, where $u$/$i$ denotes the user-item pair, $\mathbf{x}_u\in\mathbb{R}^{d_u}, \mathbf{x}_i\in\mathbb{R}^{d_i}, \mathbf{x}_{ui}\in\mathbb{R}^{d_{ui}}$ are the user-side features (e.g., user ID), item-side features (e.g., item ID), and item-aware cross features (e.g., SIM \cite{sim}).
%
The $o_{ui}\in\{0,1\}$ and $r_{ui}\in\{0,1\}$ are user-item ground-truth interacted labels, where $o_{ui}$ denotes whether user $u$ clicked item $i$ and $r_{ui}$ denotes whether user $u$ converted item $i$. 
%
According to the entire \textbf{\textit{exposure} space} $\mathcal{D}$, we could further obtain several subset spaces:
%
\begin{itemize}[leftmargin=*,align=left]
\item \textbf{\textit{Click} space} $\mathcal{O}\in\mathcal{D}$, if click label $o_{ui} = 1$.
\item \textbf{\textit{un-Click} space} $\mathcal{N}=\mathcal{D} - \mathcal{O}$, if click label $o_{ui} = 0$.
\item \textbf{\textit{Conversion} space} $\mathcal{R}\in\mathcal{O}$: if label $o_{ui}=1$ and $r_{ui}=1$.
\item \textbf{\textit{un-Conversion} space} $\mathcal{M}=\mathcal{O}-\mathcal{R}$: if label $o_{ui}=1$ and $r_{ui}=0$.
\end{itemize}
%
Based on them, a simple ranking model can be formed as:
\begin{equation}
% \small
\begin{split}
&\hat{y}^{ctr}_{ui} = \texttt{MLP}^{ctr}(\mathbf{x}_{ui}),\quad \hat{y}^{cvr}_{ui} = \texttt{MLP}^{cvr}(\mathbf{x}_{ui}),\\
&\mathbf{x}_{ui} = \texttt{Multi-Task-Encoder}(\mathbf{x}_u\oplus \mathbf{x}_i\oplus \mathbf{x}_{ui}),\\
\end{split}
\label{base}
\end{equation}
where the $\oplus$ denotes the concatenate operator, $\mathbf{x}\in\mathbb{R}^d$ is the encoded hidden states, and $\texttt{MLP}(\cdot)$ denotes a stacked neural-network. We use a share-bottom based multi-task paradigm to predict CTR and CVR scores, $\hat{y}^{ctr},\hat{y}^{cvr}$.
%
Next, we directly minimize the cross-entropy binary classification loss to train CTR tower and CVR towers with corresponding space samples:
%
%
%
\begin{equation}
% \small
% \footnotesize
\begin{split}
&\mathcal{L}^{ctr} = - \frac{1}{|\mathcal{D}|}\big(\sum_{(u,i)\in\mathcal{D}}\delta(\hat{y}^{ctr}_{ui}, o_{ui})\big),\\
&\mathcal{L}^{cvr} = - \frac{1}{|\mathcal{O}|}\big(\sum_{(u,i)\in\mathcal{O}}\delta(\hat{y}^{cvr}_{ui}, r_{ui})\big).
\end{split}
\label{crossentropy}
\end{equation}
%
% 
%
In inference, given the hundreds item candidates in a certain user request, we could obtain predicted CTCVR by  $\hat{y}^{ctcvr}_{ui} =\hat{y}^{ctr}_{ui} \cdot \hat{y}^{cvr}_{ui}$ for each item, which is used for final ranking. Then top K highest items will be returned and shown to user. The CVR are learned in click space during training but be predicted in an assumed explore space during inference, which brings up the question of sample selection bias problem.
% 





To alleviate sample selection bias,  ESMM \cite{essm} expand the click-space CVR learning task to exposure-space CTCVR learning task, to directly solve the inconsistency between training and inference:
\begin{equation}
% \small
\begin{split}
\mathcal{L}^{ctcvr} = &- \frac{1}{|\mathcal{D}|}\Big(\sum_{(u,i)\in\mathcal{D}}\delta(\hat{y}^{ctr}_{ui}\cdot\hat{y}^{cvr}_{ui}, o_{ui}\cdot r_{ui})\Big)
\end{split}
\label{ctxcvr}
\end{equation}
which treats all un-clicked samples as negative samples of CTCVR task. However those un-clicked samples that would be converted if clicked, which are falsely negative samples,  still leads to missing not at random (MNAR) problem  \cite{multiipw}. To mitigate this problem, inverse propensity weighting (IPW) \cite{multiipw,escm2} based method inversely weight the CVR loss in click space by propensity score of observing  $(u,i)$ in click space $\mathcal{O}$, to eliminate the influence of click event to CVR estimation in entire space $D$
\begin{equation}
\small
\begin{split}
\mathcal{L}^{cvr}_{IPW} =& - \frac{1}{|\mathcal{O}|}\Big(\sum_{(u,i)\in\mathcal{O}}\frac{\delta(\hat{y}^{cvr}_{ui}, r_{ui})}{\hat{y}^{ctr}_{ui}}\Big),
\end{split}
\label{cvripw}
\end{equation}
% 
%
Our method is based on above ESMM with IPW  framework. Although alleviating SSB and MNAR problem, IPW-based methods still lack reasonable labels for \textit{un-clicked} samples, which we solve by generating discriminative and robust soft labels.






\subsection{ChorusCVR}
In this section, we dive into ChorusCVR and explain how we realize entire-space debiased CVR learning by generating discriminative and robust soft CVR labels (as shown in Figure~\ref{choruscvr}).

\subsubsection{Negative sample Discrimination Module (NDM)}

As mentioned before, the soft labels introduced by previous works are suboptimal for lack either discriminability or robustness. As shown in Figure~\ref{intro} (d), an ideal discrimination surface should separate the factual negative samples (clicked but un-converted) from positive samples (clicked \& converted), and factual negative samples from ambiguous negative samples (un-clicked). With this in mind, we find the ideal discrimination surface implies a new task, CTunCVR prediction. We formulate CTunCVR labels as:
\begin{equation}
y^{ctuncvr} = o_{ui} * (1-r_{ui}) = 
\begin{cases} 
1 & o_{ui}=1~\&~r_{ui} = 0, \\
0 &  o_{ui}=0, \\
0 & o_{ui}=1~\&~r_{ui} = 1,
\end{cases}
\end{equation}
where only \textit{clicked but un-converted} samples are positive samples, both \textit{clicked \& converted} and \textit{un-clicked} samples are negative samples. Instead of directly predicting CTunCVR score in exposure space, we follow a typical two-stage prediction paradigm to obtain CTunCVR to 
reduce cumulative error. We firstly introduce an additional unCVR tower to predict unCVR score $\hat{y}^{uncvr}$, then combine it with $\hat{y}^{ctr}$ to form CTunCVR score:
% 
\begin{equation}
% \small
\begin{split}
\hat{y}^{uncvr}_{ui} = \hat{y}^{ctr}_{ui}\cdot\hat{y}^{cvr}_{ui}\quad \quad
\hat{y}^{ctuncvr}_{ui} =\hat{y}^{ctr}_{ui}\cdot\hat{y}^{uncvr}_{ui}
\end{split}
\label{uncvr}
\end{equation}
Then we can naturally optimize CTunCVR objective in exposure space by cross entropy loss: 
\begin{equation}
% \small
\begin{split}
\mathcal{L}^{ctuncvr} = - \frac{1}{|\mathcal{D}|}\Big(\sum_{(u,i)\in\mathcal{D}}\delta(\hat{y}^{ctuncvr}_{ui}, o_{ui} * (1-r_{ui})\Big).
\end{split}
\label{uncvr}
\end{equation}
% With the help of this formulation, we can narrow down the problem to the accurate estimation of unCVR. 
With the help of $\mathcal{L}^{ctuncvr}$ and an extra \textbf{unCVR prediction result} $\hat{y}^{uncvr}_{ui}$, we can narrow down the aforementioned problem to consider \textbf{R1. Discriminability} and \textbf{R2. Robustness} problem at same time.
%
For the $\hat{y}^{uncvr}_{ui}$ generation, we add an mirror unCVR tower which similar with the Eq.(\ref{base}) and (\ref{crossentropy}):
%
%
%
% 
%
%
% $
% 
\begin{equation}
% \smalluncvr
\begin{split}
\hat{y}^{uncvr}_{ui} &= \texttt{MLP}^{uncvr}(\mathbf{x}_{ui}), \\
\mathcal{L}^{uncvr} = - \frac{1}{|\mathcal{O}|}\Big(&\sum_{(u,i)\in\mathcal{O}}\delta\big(\hat{y}^{uncvr}_{ui}, 1-r_{ui})\big)\Big)
\end{split}
\label{uncvr}
\end{equation}
% 
Next, analogously with the Eq.(\ref{cvripw}), we then adopt the predicted click $\hat{y}^{ctr}_{ui}$ to inversely weight the unCVR error, to $\mathcal{L}^{uncvr}$ as:
% 
\begin{equation}
% \small
\begin{split}
\mathcal{L}^{uncvr}_{IPW} = - \frac{1}
{|\mathcal{O}|}\Big(\sum_{(u,i)\in\mathcal{O}}\frac{\delta(\hat{y}^{uncvr}_{ui}, 1-r_{ui})}{\hat{y}^{ctr}_{ui}}\Big)
\end{split}
\label{uncvr}
\end{equation}
In this way, the \textit{click} space tendency can be alleviated that higher/lower $pCTR$ sample will declined/enhanced for a fair training. So far we obtain debiased unCVR soft labels, which we will utilize to help the CTunCVR training and CVR component supervision.



\subsubsection{Soft Alignment Module (SAM)}
Up to now, we fulfill the initial goal of obtain high-quality soft labels in un-clicked space. In this section we present the solution to utilize the unCVR score as soft labels to supervise CVR learning, which we call \emph{soft alignment mechanism}. We first use $1-unCVR$ manner as soft labels for entropy-based CVR learning. In the same time, we also use $1-CVR$ manner to generate soft labels for unCVR learning, in a mutual supervision fashion to align unCVR predictions to CVR. All these objectives are inversely weighted by predicted CTR in a IPW paradigm (see $\mathcal{L}^{align1}_{IPW}$ and $\mathcal{L}^{align2}_{IPW}$ in Figure.~\ref{choruscvr}). To further alleviate SSB for un-click space, we also propose a \textit{un-click space IPW} approach, to inversely weight the un-click samples with $1-pCTR$ for CTR and unCVR alignment objectives (see $\mathcal{L}^{align3}_{IPW}$ and $\mathcal{L}^{align4}_{IPW}$ in Figure.~\ref{choruscvr}). Overall, all alignment objectives are as follows:
\begin{equation}
% \small
\begin{split}
\mathcal{L}^{align}_{IPW} = &- \frac{1}{|\mathcal{O}|}\big(\frac{\delta(\hat{y}^{cvr}_{ui}, 1-\texttt{sg}(\hat{y}^{uncvr}_{ui}))}{\hat{y}^{ctr}_{ui}}\big)
- \frac{1}{|\mathcal{N}|}\big(\frac{\delta(\hat{y}^{cvr}_{ui}, 1-\texttt{sg}(\hat{y}^{uncvr}_{ui}))}{1-\hat{y}^{ctr}_{ui}}\big)\\
&- \frac{1}{|\mathcal{O}|}\big(\frac{\delta(\hat{y}^{uncvr}_{ui}, 1-\texttt{sg}(\hat{y}^{cvr}_{ui}))}{\hat{y}^{ctr}_{ui}}\big)
- \frac{1}{|\mathcal{N}|}\big(\frac{\delta(\hat{y}^{uncvr}_{ui}, 1-\texttt{sg}(\hat{y}^{cvr}_{ui}))}{1-\hat{y}^{ctr}_{ui}}\big)
\end{split}
\label{soft}
\end{equation}

where the $\texttt{sg}(\cdot)$ means the stop gradient function, the $\hat{y}^{ctr}_{ui}, (1 - \hat{y}^{ctr}_{ui})$ denote the click propensity in the \textit{click} and \textit{un-click} space, respectively.
% 
%
All losses of our ChorusCVR are as follows:
\begin{equation}
% \small
\begin{split}
\mathcal{L} = \mathcal{L}^{ctcvr} + \mathcal{L}^{cvr}_{IPW} + \mathcal{L}^{ctuncvr} + \mathcal{L}^{uncvr}_{IPW} + \mathcal{L}^{align}_{IPW}
\end{split}
\label{soft}
\end{equation}
In this way, our ChorusCVR  make CVR and unCVR supervise each other during training, which results in an equilibrium. 





\begin{table}[t!]
\centering
\caption{Offline results(\%) in terms of CTR-AUC, CTCVR-AUC and logloss at Ali-CCP and Kuaishou.}
\vspace{-1em}
\setlength{\tabcolsep}{3pt}{
% \resizebox{\linewidth}{!}{
\begin{tabular}{l|cc|cc}
\toprule
\multirow{4}{*}{\makecell{Models}} 
& \multicolumn{2}{c|}{Ali-CCP} & \multicolumn{2}{c}{Kuaishou}   \\ 
\cmidrule(r){2-5} & \multicolumn{2}{c|}{AUC}  & \multicolumn{2}{c}{AUC}   \\ 
\cmidrule(r){2-3} \cmidrule(r){4-5} & CVR  & CTCVR &  CVR & CTCVR  \\
\hline
ESMM \cite{essm} & 0.5963 & 0.5802 & 0.8609 & 0.9276\\ 
ESCM$^2$-DR \cite{escm2} & 0.6354 & 0.6203& 0.8617 & 0.9280\\
ESCM$^2$-IPW \cite{escm2} & 0.6385 & 0.6126& 0.8619 & 0.9283\\
UKD \cite{ukd} & 0.6451 & 0.6282& 0.8615 & 0.9279\\
DCMT \cite{dcmt} & 0.6447 & \underline{0.6375} & \underline{0.8628} & 0.9281\\
DDPO \cite{ddpo} & \underline{0.6496} & 0.6326& 0.8623 & \underline{0.9289}\\
NISE \cite{nise} & 0.6418 & 0.6291 & 0.8614 & 0.9274\\
ChorusCVR & \textbf{0.6589} & \textbf{0.6401}& \textbf{0.8639} & \textbf{0.9304}\\
\hline
Rela.Impr. & +1.43\% & +0.40\% & +0.13\% & +0.16\% \\
\hline
ChorusCVR w/o NDM & 0.6498 & 0.6347& 0.8625 & 0.9284\\
ChorusCVR w/o SAM & 0.6407 & 0.6139 & 0.8622 & 0.9281\\
% DCMT+Ctuncvr loss & 0.6492 & 0.6387& 0.8626 & 0.9293\\
\bottomrule
\end{tabular}
}
\vspace{-2em}
% }
\label{mainoffline}
\end{table}




\section{Experiments}
In this section, we conduct extensive online and offline experiments to verify the efficacy of our model following 3 research questions: 
\begin{itemize}[leftmargin=*,align=left]
\item \textbf{RQ1:} How does our model perform on industrial recommendation datasets?
\item \textbf{RQ2:} Can our model bring improvements of e-commerce A/B test metrics on online product environment?
\item \textbf{RQ3:} Can our model alleviate the bias on un-clicked samples?
\end{itemize}
\textbf{Datasets.} To evaluate the performance of our method and comparison baselines, we conduct comprehensive experiments on both public and industrial datasets.



• Public dataset: The Ali-CCP (Alibaba Click and Conversion Prediction) dataset \cite{essm} is a benchmark dataset for CVR and CTR prediction, which collected from traffic logs in Taobao e-commerce platform. Ali-CCP contains 33 features and 84M samples. The training set contains 42M exposure samples, 1.6M click samples and 9k conversion samples and the test set contains 42M exposure samples, 1.7M click samples and 9.4k conversion samples.

• Industrial dataset: The industrial dataset is colloected from Kuaishou e-commerce live-streaming platform. Kuaishou e-commerce live-streaming is a popular content interest e-commerce platform with \textbf{tens of millions} daily active users. The dataset contains more than 1000 features including user profiles (gender, age, etc), user actions (click, buy, etc), seller profiles (industry, sales, etc) and goods information. 

\textbf{Compared Methods.} ESMM \cite{essm} learns CVR task through a CTR task and a CTCVR task to alleviate ssb and data sparsity issues. ESCM$^2$-IPW \cite{escm2} incorporates the inverse propensity weighting (IPW) \cite{ipw} method to regularize ESMM’s CVR estimation. ESCM$^2$-DR \cite{escm2} augments ESCM$^2$-IPW with an auxiliary imputation loss to models the CVR with the Doubly Robust(DR) method. UKD \cite{ukd} introduces a transfer adversarial learning approach to generate soft conversion pseudo-labels in the unclicked space. DCMT \cite{dcmt} proposed a counterfactual mechanism to directly debias CVR in the entire space. DDPO \cite{ddpo} employs a conversion propensity prediction network to generate soft conversion pseudo-labels in the un-clicked space. NISE \cite{nise} follows a semi-supervised learning paradigm and use predicted CVR as CVR pseudo labels for un-clicked samples.

\textbf{Offline Results (RQ1):} 
We evaluate the efficacy of our model by the Area Under ROC (AUC) of CVR and CTCVR prediction tasks. The experimental results on two datasets are shown in Table 1. On Ali-CCP dataset, our model outperforms DDPO by 1.4\% (\textbf{0.6589} v.s. 0.6496) for CVR-AUC and DCMT by 0.4\% (\textbf{0.6401} v.s. 0.6375) for CTCVR-AUC, respectively. On Kuaishou dataset, our model outperforms DCMT by 0.1\% (\textbf{0.8639} v.s. 0.8628) for CVR-AUC and DDPO by 0.16\% (\textbf{0.9304} v.s. 0.9289) for CTCVR-AUC, respectively. In a nutshell, our model consistently outperforms the best-performing baselines on both two datasets in a large margin in terms of two tasks. 
\begin{figure}[t]
  \centering
  \includegraphics[width=5cm]{pcoc.png}
  \vspace{-1em}
  \caption{The PCOC analysis.}
  \label{pcoc}
  \vspace{-2em}
\end{figure}
\textit{Ablation Study.} To verify the efficacy of each parts of ChorusCVR, we evaluate the performance of variants without NDM and SAM. It is noteworthy that the unCVR objective is optimized by 1 - $y^{CVR}$ label in click space in `ChorusCVR w/o NDM'. We find that without NDM, our model degrades 1.3\% (\textbf{0.6589} v.s. 0.6498) on CVR AUC on Ali-CCP dataset, for lack of discrimination between negative samples of different levels. Meanwhile, we can observe that without SAM, the `ChorusCVR w/o SAM' performs even poorer than `ChorusCVR w/o NDM', presents 4\% lower CTCVR AUC on Ali-CCP dataset, for lack of reasonable pesudo supervisions on un-click samples. 

\textbf{Online Results (RQ2):}
We deploy our method in production environment of Kuaishou e-commerce live-streaming platform to conduct online A/B testing for 8 days. %As shown in Tab. 2,
Compared with the base model (DCMT), our model presents large improvements on CVR(+0.12\%), orders (+0.851\%) and DAC (+0.705\%) with 95\% confidence intervals. 




\textbf{Analysis (RQ3):}
To investigate whether our method can address the bias in un-clicked samples, we try to compare the  accuracy of cvr predicitons between our method and baselines on un-clicked samples. However, the conversion labels of un-clicked samples are inaccessible in the real world. Inspired by the inverse propensity weighting based methods, we propose a compromised comparison approach, to use samples with low pCTR as a substitute for un-clicked samples and observe the model's pCVR on these low CTR samples. We show the estimated pCVR and actual cvr for samples with different pCVR the deviations between two curves signify the prediction bias of the model.

As shown in Fig. \ref{pcoc}, we can observe that DCMT represents obvious bias on low pCTR samples, which severely overestimates the CVR on low pCtr samples. However, our model accurately predicts the CVR on those low pCtr samples for reasonable CVR supervisions on un-clicked samples. 










\section{Conclusion}
In this paper, to alleviate sample selection bias in CVR prediction task, we propose an effective method ChorusCVR. To generate discriminative and robust soft labels, we propose Negative sample Discrimination Module to obtain soft CTunCVR labels which can separate negative samples of different levels. Then we design a 
Soft Alignment Module for debiased CVR learning in un-click space with soft labels. We demonstrated the superior performance of the proposed
ChorusCVR in offline experiments. In addition, we conduct online A/B testing, obtaining +0.851\% improvements on orders of industrial e-commerce living stream, which demonstrates the effectiveness and universality of ChorusCVR in online
systems. 
% Moreover, ChorusCVR has been deployed on ranking system in KUAISHOU e-commerce living stream.



% \newpage
\balance
\bibliographystyle{ACM-Reference-Format}
\bibliography{sample-base-extend.bib}
\end{document}
\endinput