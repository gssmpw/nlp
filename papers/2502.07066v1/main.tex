% TEMPLATE for Usenix papers, specifically to meet requirements of
%  USENIX '05
% originally a templ ate for producing IEEE-format articles using LaTeX.
%   written by Matthew Ward, CS Department, Worcester Polytechnic Institute.
% adapted by David Beazley for his excellent SWIG paper in Proceedings,
%   Tcl 96
% turned into a smartass generic template by De Clarke, with thanks to
%   both the above pioneers
% use at your own risk.  Complaints to /dev/null.
% make it two column with no page numbering, default is 10 point

% Munged by Fred Douglis <douglis@research.att.com> 10/97 to separate
% the .sty file from the LaTeX source template, so that people can
% more easily include the .sty file into an existing document.  Also
% changed to more closely follow the style guidelines as represented
% by the Word sample file. 

% Note that since 2010, USENIX does not require endnotes. If you want
% foot of page notes, don't include the endnotes package in the 
% usepackage command, below.

\documentclass[letterpaper,twocolumn,10pt]{article}
\usepackage{usenix}

\def\method{\text MixMin~}
\def\methodnospace{\text MixMin}
\def\genmethod{$\mathbb{R}$\text Min~}
\def\genmethodnospace{ $\mathbb{R}$\text Min}

\usepackage{usenix,epsfig,endnotes,amsmath,amssymb}
\usepackage{mdframed}
\usepackage{booktabs} % For professional-looking tables
\usepackage[letterpaper,margin=1in]{geometry}
\usepackage{longtable} % For tables spanning multiple pages (optional)
\usepackage[normalem]{ulem}

\begin{document}


%don't want date printed
\date{}

%make title bold and 14 pt font (Latex default is non-bold, 16 pt)
\title{\Large General-Purpose $f$-DP Estimation and Auditing in a Black-Box Setting \bf }

\author{
    {\rm Önder Askin}$^{1}$, 
    {\rm Holger Dette}$^{1}$, 
    {\rm Martin Dunsche}$^{1,}$\thanks{Corresponding author: \texttt{martin.dunsche@rub.de}}, 
    {\rm Tim Kutta}$^{2}$, 
    {\rm Yun Lu}$^{3}$, 
    {\rm Yu Wei}$^{4}$, 
    {\rm Vassilis Zikas}$^{4}$\thanks{Authors are listed in alphabetical order.} \\
    \\
    $^1$Ruhr-University Bochum\\
    $^2$Aarhus University \\
    $^3$University of Victoria \\
    $^4$Georgia Institute of Technology
}



\maketitle

% Use the following at camera-ready time to suppress page numbers.
% Comment it out when you first submit the paper for review.
\thispagestyle{empty}


During the early stages of interface design, designers need to produce multiple sketches to explore a design space.  Design tools often fail to support this critical stage, because they insist on specifying more details than necessary. Although recent advances in generative AI have raised hopes of solving this issue, in practice they fail because expressing loose ideas in a prompt is impractical. In this paper, we propose a diffusion-based approach to the low-effort generation of interface sketches. It breaks new ground by allowing flexible control of the generation process via three types of inputs: A) prompts, B) wireframes, and C) visual flows. The designer can provide any combination of these as input at any level of detail, and will get a diverse gallery of low-fidelity solutions in response. The unique benefit is that large design spaces can be explored rapidly with very little effort in input-specification. We present qualitative results for various combinations of input specifications. Additionally, we demonstrate that our model aligns more accurately with these specifications than other models. 

% OLD ABSTRACT
%When sketching Graphical User Interfaces (GUIs), designers need to explore several aspects of visual design simultaneously, such as how to guide the user’s attention to the right aspects of the design while making the intended functionality visible. Although current Large Language Models (LLMs) can generate GUIs, they do not offer the finer level of control necessary for this kind of exploration. To address this, we propose a diffusion-based model with multi-modal conditional generation. In practice, our model optionally takes semantic segmentation, prompt guidance, and flow direction to generate multiple GUIs that are aligned with the input design specifications. It produces multiple examples. We demonstrate that our approach outperforms baseline methods in producing desirable GUIs and meets the desired visual flow.

% Designing visually engaging Graphical User Interfaces (GUIs) is a challenge in HCI research. Effective GUI design must balance visual properties, like color and positioning, with user behaviors to ensure GUIs easy to comprehend and guide attention to critical elements. Modern GUIs, with their complex combinations of text, images, and interactive components, make it difficult to maintain a coherent visual flow during design.
% Although current Large Language Models (LLMs) can generate GUIs, they often lack the fine control necessary for ensuring a coherent visual flow. To address this, we propose a diffusion-based model that effectively handles multi-modal conditional generation. Our model takes semantic segmentation, optional prompt guidance, and ordered viewing elements to generate high-fidelity GUIs that are aligned with the input design specifications.
% We demonstrate that our approach outperforms baseline methods in producing desirable GUIs and meets the desired visual flow. Moreover, a user study involving XX designers indicates that our model enhances the efficiency of the GUI design ideation process and provides designers with greater control compared to existing methods.    



% %%%%%%%%%%%%%%%%%%%%%%%%%%%%%%%%%%%%%%%%%%%%%%%%%%%%%%
% % Writing Clinic Comments:
% %%%%%%%%%%%%%%%%%%%%%%%%%%%%%%%%%%%%%%%%%%%%%%%%%%%%%%
% % Define: Effective UI design
% % Motivate GANs and write in full form.
% % LLMs vs ControlNet vs GANs
% % Say something about the Figma plugin?
% % Write the work is novel or what has been done before
% % What is desirable UI and how to evalutate that?
% % Visual Flow - main theme (center around it)
% % Re-Title: use word Flow!
% % Use ControlNet++ & SPADE for abstract.
% % Write about input/output. 
% % Why better than previous work?
% %%%%%%%%%%%%%%%%%%%%%%%%%%%%%%%%%%%%%%%%%%%%%%%%%%%%%

% % v2:
% % \noindent \textcolor{red}{\textbf{NEW Abstract!} (Post Writing Clinic 1 - 25-Jun)}

% % \noindent \textcolor{red}{----------------------------------------------------------------------}

% % \noindent Designing user interfaces (UIs) is a time-consuming process, particularly for novice designers. 
% % Creating UI designs that are effective in market funneling or any other designer defined goal requires a good understanding of the visual flow to guide users' attention to UI elements in the desired order. 
% % While current Large Language Models (LLMs) can generate UIs from just prompts, they often lack finer pixel-precise control and fail to consider visual flow. 
% % In this work, we present a UI synthesis method that incorporates visual flow alongside prompts and semantic layouts. 
% % Our efficient approach uses a carefully designed Generative Adversarial Network (GAN) optimized for scenarios with limited data, making it more suitable than diffusion-based and large vision-language models.
% % We demonstrate that our method produces more "desirable" UIs according to the well-known contrast, repetition, alignment, and proximity principles of design. 
% % We further validate our method through comprehensive automatic non-reference, human-preference aligned network scoring and subjective human evaluations.
% % Finally, an evaluation with xx non-expert designers using our contributed Figma plugin shows that <method-name> improves the time-efficiency as well as the overall quality of the UI design development cycle.

% % \noindent \textcolor{red}{----------------------------------------------------------------------}


% \noindent \textcolor{blue}{\textbf{NEW Abstract!} (Pre Writing Clinic 9-July)}

% \noindent \textcolor{blue}{----------------------------------------------------------------------}

% \noindent Exploring different graphical user interface (GUI) design ideas is time-consuming, particularly for novice designers. 
% Given the segmentation masks, design requirement as prompt, and/or preferred visual flow, we aim to facilitate creative exploration for GUI design and generate different UI designs for inspiration.
% While current Vision Language Models (VLMs) can generate GUIs from just prompts, they often lack control over visual concepts and flow that are difficult to convey through language during the generation process. 
% In this work, we present FlowGenUI, a semantic map-guided GUI synthesis method that optionally incorporates visual flow information based on the user's choice alongside language prompts. 
% We demonstrate that our model not only creates more realistic GUIs but also creates "predictable" (how users pay attention to and order of looking at GUI elements) GUIs.
% Our approach uses Stable Diffusion (SD), a large paired image-text pretrained diffusion model with a rich latent space that we steer toward realistic GUIs using a trainable copy of SD's encoder for every condition (segmentation masks, prompts, and visual flow). 
% We further provide a semantic typography feature to create custom text-fonts and styles while also alleviating SD's inherent limitations in drawing coherent, meaningful and correct aspect-ratio text. 
% Finally, a subjective evaluation study of XX non-expert and expert designers demonstrates the efficiency and fidelity of our method.


% This process encourages creativity and prevents designers from falling into habitual patterns.


% ------------------------------------------------------------------
% Joongi Why is it important to create realistic GUI?
% I do not see how the Visual Flow given on the left hand side is reflected in the results on the right hand side. 
% I’d avoid making unsubstantiated claims about designers (falling into habitual patterns).
% The UIs you generate do not “align with users’ attention patterns” but rather try to control it (that’s what visual flow means)
% ------------------------------------------------------------------
% Comments - Writing Clinic - 9th July:
% Improve title. More names: FlowGen
% Figure 1: Use an inference time hand-drawn mask
% Figure 1: Show both workflows. Add a designer --> Input.
% Figure 1: Make them more diverse
% ------------------------------------------------------------------
% Designing graphical user interfaces (GUIs) requires human creativity and time. Designers often fall into habitual patterns, which can limit the exploration of new ideas. 
% To address this, we introduce FlowGenUI, a method that facilitates creative exploration and generates diverse GUI designs for inspiration. By using segmentation masks, design requirements as prompts, and/or selected visual flows, our approach enhances control over the visual concepts and flows during the generation process, which current Vision Language Models (VLMs) often lack.
% FlowGenUI uses Stable Diffusion (SD), a largely pretrained text-to-image diffusion model, and guides it to create realistic GUIs. 
% We achieve this by using a trainable copy of SD's encoder for each condition (segmentation masks, prompts, and visual flow). 
% This method enables the creation of more realistic and predictable GUIs that align with users' attention patterns and their preferred order of viewing elements.
% We also offer a semantic typography feature that creates custom text fonts and styles while addressing SD's limitations in generating coherent, meaningful, and correctly aspect-ratio text.
% Our approach's efficiency and fidelity are evaluated through a subjective user study involving XX designers. 
% The results demonstrate the effectiveness of FlowGenUI in generating high-quality GUI designs that meet user requirements and visual expectations.

% ---------------------------------------


%A critical and general issue remains while using such deep generative priors: creating coherent, meaningful and correct aspect-ratio text. 
%We tackle this issue within our framework and additionally provide a semantic typography feature to create custom text-fonts and styles. 


% %Creating UI designs that are effective in market funneling or any other designer-defined goal requires a good understanding of the visual flow to guide users' attention to UI elements in the desired order. 
% %While current largely pre-trained Vision Language Models (VLMs) can generate GUIs from just prompts, they often lack finer or pixel-precise control which can be crucial for many easy-to-understand visual concepts but difficult to convey through language. 
% % However, obtaining such pixe-level labels is an extremely expensive so we
% % For example - overlaying text on images with certain aspect ratios and two equally separated buttons 
% Additionally, all prior GUI generation work fails to consider visual flow information during the generation process. 
% We demonstrate that visual flow-informed generation not only creates more realistic and human-friendly GUIs but also creates "predictable" (how users pay attention to and order of looking at GUI elements) UIs that could be beneficial for designers for tasks like creating effective market funnels.
% In this work, we present a semantic map-guided GUI synthesis method that optionally incorporates visual flow information based on the user's choice alongside language prompts. 
% Our approach uses Stable Diffusion, a large (billions) paired image-text pretrained diffusion model with a rich latent space that we steer toward realistic GUIs using an ensemble of ControlNets. 
% % TODO: Mention it in 1 sentence:
% A critical and general issue remains while using such deep generative priors: creating coherent, meaningful and correct aspect-ratio text. 
% We tackle this issue within our framework and additionally provide a semantic typography feature to create custom text-fonts and styles. 
% To evaluate our method, we demonstrate that our method produces more "desirable" UIs according to the well-known contrast, repetition, alignment, and proximity principles of design. 
% % We further validate our method through comprehensive automatic non-reference and human-preference aligned scores. (TODO: Maybe Unskip if we get UIClip from Jason!)
% % TODO: Re-word this and only keep ideation cycles and time-efficiency.
% Finally, a subjective evaluation study of XX non-expert and expert designers demonstrates the efficiency and fidelity of our method.
% % improves the time-efficiency by quick iterations of the UI design ideation process.
% %Finally, an evaluation with xx non-expert designers using our contributed <method-name> improves the time-efficiency by quick iterations of the UI design ideation cycle.

%\noindent \textcolor{blue}{----------------------------------------------------------------------}


%In an evaluation with xx designers, we found that GenerativeLayout: 1) enhances designers' exploration by expanding the coverage of the design space, 2) reduces the time required for exploration, and 3) maintains a perceived level of control similar to that of manual exploration.



% Present-day graphical user interfaces (GUIs) exhibit diverse arrangements of text, graphics, and interactive elements such as buttons and menus, but representations of GUIs have not kept up. They do not encapsulate both semantic and visuo-spatial relationships among elements. %\color{red} 
% To seize machine learning's potential for GUIs more efficiently, \papername~ exploits graph neural networks to capture individual elements' properties and their semantic—visuo-spatial constraints in a layout. The learned representation demonstrated its effectiveness in multiple tasks, especially generating designs in a challenging GUI autocompletion task, which involved predicting the positions of remaining unplaced elements in a partially completed GUI. The new model's suggestions showed alignment and visual appeal superior to the baseline method and received higher subjective ratings for preference. 
% Furthermore, we demonstrate the practical benefits and efficiency advantages designers perceive when utilizing our model as an autocompletion plug-in.


% Overall pipeline: Maybe drop semantic typography / visual flow?

\section{Introduction}

Tutoring has long been recognized as one of the most effective methods for enhancing human learning outcomes and addressing educational disparities~\citep{hill2005effects}. 
By providing personalized guidance to students, intelligent tutoring systems (ITS) have proven to be nearly as effective as human tutors in fostering deep understanding and skill acquisition, with research showing comparable learning gains~\citep{vanlehn2011relative,rus2013recent}.
More recently, the advancement of large language models (LLMs) has offered unprecedented opportunities to replicate these benefits in tutoring agents~\citep{dan2023educhat,jin2024teach,chen2024empowering}, unlocking the enormous potential to solve knowledge-intensive tasks such as answering complex questions or clarifying concepts.


\begin{figure}[t!]
\centering
\includegraphics[width=1.0\linewidth]{Figs/Fig.intro.pdf}
\caption{An illustration of coding tutoring, where a tutor aims to proactively guide students toward completing a target coding task while adapting to students' varying levels of background knowledge. \vspace{-5pt}}
\label{fig:example}
\end{figure}

\begin{figure}[t!]
\centering
\includegraphics[width=1.0\linewidth]{Figs/Fig.scaling.pdf}
\caption{\textsc{Traver} with the trained verifier shows inference-time scaling for coding tutoring (detailed in \S\ref{sec:scaling_analysis}). \textbf{Left}: Performance vs. sampled candidate utterances per turn. \textbf{Right}: Performance vs. total tokens consumed per tutoring session. \vspace{-15pt}}
\label{fig:scale}
\end{figure}


Previous research has extensively explored tutoring in educational fields, including language learning~\cite{swartz2012intelligent,stasaski-etal-2020-cima}, math reasoning~\cite{demszky-hill-2023-ncte,macina-etal-2023-mathdial}, and scientific concept education~\cite{yuan-etal-2024-boosting,yang2024leveraging}. 
Most aim to enhance students' understanding of target knowledge by employing pedagogical strategies such as recommending exercises~\cite{deng2023towards} or selecting teaching examples~\cite{ross-andreas-2024-toward}. 
However, these approaches fall short in broader situations requiring both understanding and practical application of specific pieces of knowledge to solve real-world, goal-driven problems. 
Such scenarios demand tutors to proactively guide people toward completing targeted tasks (e.g., coding).
Furthermore, the tutoring outcomes are challenging to assess since targeted tasks can often be completed by open-ended solutions.



To bridge this gap, we introduce \textbf{coding tutoring}, a promising yet underexplored task for LLM agents.
As illustrated in Figure~\ref{fig:example}, the tutor is provided with a target coding task and task-specific knowledge (e.g., cross-file dependencies and reference solutions), while the student is given only the coding task. The tutor does not know the student's prior knowledge about the task.
Coding tutoring requires the tutor to proactively guide the student toward completing the target task through dialogue.
This is inherently a goal-oriented process where tutors guide students using task-specific knowledge to achieve predefined objectives. 
Effective tutoring requires personalization, as tutors must adapt their guidance and communication style to students with varying levels of prior knowledge. 


Developing effective tutoring agents is challenging because off-the-shelf LLMs lack grounding to task-specific knowledge and interaction context.
Specifically, tutoring requires \textit{epistemic grounding}~\citep{tsai2016concept}, where domain expertise and assessment can vary significantly, and \textit{communicative grounding}~\citep{chai2018language}, necessary for proactively adapting communications to students' current knowledge.
To address these challenges, we propose the \textbf{Tra}ce-and-\textbf{Ver}ify (\textbf{\model}) agent workflow for building effective LLM-powered coding tutors. 
Leveraging knowledge tracing (KT)~\citep{corbett1994knowledge,scarlatos2024exploring}, \model explicitly estimates a student's knowledge state at each turn, which drives the tutor agents to adapt their language to fill the gaps in task-specific knowledge during utterance generation. 
Drawing inspiration from value-guided search mechanisms~\citep{lightman2023let,wang2024math,zhang2024rest}, \model incorporates a turn-by-turn reward model as a verifier to rank candidate utterances. 
By sampling more candidate tutor utterances during inference (see Figure~\ref{fig:scale}), \model ensures the selection of optimal utterances that prioritize goal-driven guidance and advance the tutoring progression effectively. 
Furthermore, we present \textbf{Di}alogue for \textbf{C}oding \textbf{T}utoring (\textbf{\eval}), an automatic protocol designed to assess the performance of tutoring agents. 
\eval employs code generation tests and simulated students with varying levels of programming expertise for evaluation. While human evaluation remains the gold standard for assessing tutoring agents, its reliance on time-intensive and costly processes often hinders rapid iteration during development. 
By leveraging simulated students, \eval serves as an efficient and scalable proxy, enabling reproducible assessments and accelerated agent improvement prior to final human validation. 



Through extensive experiments, we show that agents developed by \model consistently demonstrate higher success rates in guiding students to complete target coding tasks compared to baseline methods. We present detailed ablation studies, human evaluations, and an inference time scaling analysis, highlighting the transferability and scalability of our tutoring agent workflow.


%%%Note, old intro is commented out, in the file old_intro below} %%%




\section{Preliminaries}\label{sec:prelim}

In this section, we provide details on hypothesis testing, differential privacy and tools from statistics and machine learning that our methods rely on.

\subsection{Hypothesis testing} \label{sec:hyp}

We provide a brief introduction into the key concepts of hypothesis testing. We confine ourselves to the special case of sample size $1$, most relevant to $f$-DP. 
For a general introduction we refer to \cite{Bickel2001}.
Consider two probability distributions $P, Q$ on the Euclidean space $\mathbb{R}^d$ and a random variable $X$. It is unknown from which of the two distributions $X$ is drawn and the task is to decide between the two competing hypotheses
\begin{align} \label{e:hyp2}
H_0: X \sim P\quad \textnormal{vs.} \quad H_1: X \sim Q.
\end{align}
The problem is similar to a classification task (see Section \ref{sec:class} below). The key difference to classification is that in hypothesis testing, there exists a default belief $H_0$ that is preferred over $H_1$. The user switches from $H_0$ to $H_1$ only if the data ($X$) suggests it strongly enough. In this context, a hypothesis test is a binary, potentially randomized function $g: \mathbb{R}^d \to \{0,1\}$, where $g(X)=0$ implies to stay with $H_0$, while $g(X)=1$ implies that the user should switch to $H_1$ ($H_0$ is "rejected"). Just as in classification, the decision to reject/fail to reject can be erroneous and the error rates of these decisions are called $\alpha$, the "type-I error", and $\beta$, the "type-II error". Their formal definitions are
\[
\alpha^{(g)}:= \Pr_{X \sim P}[g(X)= 1], \quad \beta^{(g)}:= \Pr_{X \sim Q}[g(X)= 0].
\]
One test $g$ is better than another $g'$, if simultaneously
\[
\alpha^{(g)} \le \alpha^{(g')} \quad \textnormal{and} \quad \beta^{(g)} \le \beta^{(g')}.
\]
This comparison of statistical tests naturally leads to the issue of optimal tests, 
%, considered in the next section.
%\subsection{Optimal tests} \label{sec:optimal}
% We consider the setup from the previous section and 
and we define the optimal level-$\alpha$-test as the argmin of
 \[
 \{\beta^{(g)}: g \,\,\textnormal{is a  test}\,\, \textnormal{with } \,\, \alpha^{(g)}\le \alpha\}.
 \]
 The minimum is achieved and the corresponding optimal test is provided by the {\em likelihood ratio (LR) test} in the Neyman-Pearson lemma, a fundamental result in statistics. In the following, we assume the two probability measures $P,Q$ in hypotheses \eqref{e:hyp2} have some probability densities $p,q$.

\begin{theo}[Neyman-Pearson Lemma~\cite{neyman1933ix}]
\label{theo: NP lemma}
For any $\alpha \in [0,1]$, the smallest  type-II error $\beta(\alpha)$ among all level-$\alpha$-tests is achieved by the \textbf{{\em likelihood ratio (LR) test}}, characterized by two constants $\eta \ge 0$ and $\lambda\in [0,1]$ and has the following rejection rule:
\begin{itemize}
    \item[1)] Reject $H_0$ if $q(X)/p(X) >\eta$.
    \item[2)] If $q(X)/p(X)=\eta$, flip an unfair coin with probability $\lambda$ of heads. If the outcome is heads, reject $H_0$.
\end{itemize} 
The constants $(\eta, \lambda)$ are chosen such that the type-I error is exactly $\alpha$.
%\begin{align*}
%    \beta(\alpha) = 1 - \int q \cdot \mathbb{I}\{p / q  \leq \eta\}
%\end{align*}
%where $\eta \in [0, +\infty]$ such that $\alpha = \int p \cdot \mathbb{I}\{p / q  \leq \eta\}$.
\end{theo}

\noindent \textbf{Notations.}  Neyman-Pearson  motivates the use of the following notations. First, for any type-I-error $\alpha$ there is a corresponding (optimal) $\beta$ implied by the Lemma. These constants are achieved by a pair $(\eta, \lambda)$ and we can thus write $\alpha(\eta, \lambda), \beta(\eta, \lambda)$ for them. When we are only interested in the result of the non-randomized test with $\lambda=0$, we will just write   $\alpha(\eta), \beta(\eta)$.

%While the Neyman-Pearson lemma is theoretically important as an optimality statement, its practical application is often limited. This is because the true densities, $p$ and  $q$, are typically unknown. For instance, in our black-box setting we do not know the algorithmic output distributions in advance. Accordingly, a main challenge in our subsequent procedure will be the approximation of the optimal test decision, without perfect knowledge of the densities.


%Additionally, the lemma requires knowledge of the level sets, where the ratio of the two densities equals a certain threshold, which is challenging to estimate. The focus of our work is to develop a test that approaches the optimality of the Neyman-Pearson test using estimated densities, without the need to estimate these level sets.

%\textcolor{red}{Since Corollary~\ref{corollary: NP lemma} doesn't include the random test case currently, there might need a smooth transfer to introduce it as a corollary of Theorem~\ref{theo: NP lemma}. @Tim, can you help to polish it out?}

%Define the set $\cS_{\eta} = \{x: p(x)/q(x) \leq \eta \}$ and we have 
%\begin{align*}
%    \prdis{\rX \sim P}{\rX \in \cS_{\eta}} = \alpha(\eta), \prdis{\rX \sim Q}{\rX \in \cS_{\eta}} = 1 - \beta(\eta).
%\end{align*}
%Intuitively, $\cS_{\eta}$ is the set of outcomes where we reject the hypothesis $H = 0$, where $\eta$ is picked such that the best possible (randomized) test has a type-I error exactly equal to $\alpha.$ By introducing $\cS_{\eta}$, we rewrite Theorem~\ref{theo: NP lemma} to the following corollary.

%\begin{cor}
 %   \label{corollary: NP lemma}
%    For $\alpha \in [0,1]$, if there exists $\eta$ such that $\prdis{\rX \sim P}{\rX \in \cS_{\eta}} = \alpha$, then it holds that
 %   \begin{align*}
 %   \beta(\alpha) = 1 - \prdis{\rX \sim Q}{\rX \in \cS_{\eta}}.
%    \end{align*}
%where $\cS_{\eta}$ is chosen such that $\prdis{\rX \sim P}{\rX \in \cS_{\eta}} = \alpha$.
%\end{cor}

\subsection{($f$-)Differential Privacy (DP)} \label{sec:fdp}
%\textcolor{blue}{
DP requires that the output of mechanism $\Mech$ is similar on all {\em neighboring} datasets $\DB, \DB'$ that differ in exactly one data point (we also call $\DB, \DB'$ {\em neighbors}). 
%that person's data. 
%We call datasets $\DB,\DB'$ neighbors/neighboring if they differ in exactly one data point.



\begin{definition}[DP~\cite{Dwork2006}]
A mechanism $\Mech$ is $(\varepsilon,\delta)$-DP if for all neighboring datasets $\DB, \DB'$ and any set $\cS$, 
  \begin{equation*}    
\Pr(\Mech(\DB) \in \cS) \leq e^\varepsilon \, \Pr(\Mech(\DB') \in \cS) + \delta~.
\end{equation*}
\end{definition}



Informally, if $\Mech$ is $(\varepsilon, \delta)$-DP, an adversary's ability to decide whether $\Mech$ was run on $\DB$ or $\DB'$ is bounded by $\delta$ and $e^{\varepsilon}$. 
For instance, any statistical level-$\alpha$-test $g$ that aims at deciding this problem must incur a type-II-error of at least $1 - e^{\varepsilon} \, \alpha - \delta$. The notion of $f$-DP was introduced to make this observation more rigorous.
%After introducing the basics of (optimal) tests, we can now formally describe $f$-DP. 
Given a pair of neighbors $\DB$ and $\DB'$ and a sample $\rX$, consider the hypotheses:
% \begin{align*}
%     H_0: \rX \sim P,& \text{ where } P = \Mech(\DB)\\   
%     H_1: \rX \sim Q,& \text{ where } Q =\Mech(\DB').
% \end{align*}
\begin{align*}
    &H_0: \rX \sim P\\   
    &H_1: \rX \sim Q,
\end{align*}
where $\Mech(\DB)$ and $\Mech(\DB')$ are distributed to $P, Q$, respectively.  
Roughly speaking, good privacy requires these two hypotheses to be hard to distinguish. That is, for any hypothesis test with type-I error $\alpha$, its type-II error $\beta$ should be large. This is captured by the trade-off function $T$ between $P$ and $Q$. 
\begin{definition}[Trade-off function~\cite{Dong2022}]
    For any two 
distributions $P$ and $Q$ on the same space, the 
trade-off function $T$ is: $$T(\alpha) := \inf \{\beta^{(g)}: g \,\,\textnormal{ test }\,\, \textnormal{ with } \,\, \alpha^{(g)}\le \alpha\}$$
\end{definition}

$\Mech$ is $f$-DP if its privacy is at least as good (its trade-off function is at least as large) as $f$, when considering all neighboring datasets.
\begin{definition}[$f$-DP~\cite{Dong2022}]
    A mechanism $\Mech$ is $f$-DP if for all neighboring datasets $\DB, \DB'$ it holds that
    $T \geq f$. Here, $T$ is the trade-off function implied by $\Mech(\DB) \sim P$ and $\Mech(\DB') \sim Q$. 
\end{definition}

%As discussed in the introduction, in line with previous black-box methods, we will confine our discussion on $f$-DP to a {\em single} pair of neighboring datasets $D,D'$.

We say $f$ is the {\em optimal/true} privacy parameter if it is the largest $f$ such that $\Mech$ is $f$-DP---such optimality is necessary to define for meaningful $f$-DP estimation, as any $\Mech$ is trivially $f$-DP for $f = 0$ (since the type-II error in hypothesis testing is always $\geq 0$).
% - a formulation suitable for our black-box methods. \todo{TO DISCUSS: I think more words are needed here to say *why* black box implies we only consider a single pair of neighbours. This discussion probably needs to go in Sec 3/4 rather than here}

%A more general overview is given in \cite{Dong2022}. So, let us consider a DP algorithm $M$ with output distributions $M(D) \sim P$ and $M(D') \sim Q$. We study the distinguishability of $P$ and $Q$ through the lens of hypothesis testing. Roughly speaking, high privacy equals high indistinguishability of $P,Q$ equals high type-I and type-II errors even for the best statistical tests. More precisely, for all type-I errors $\alpha \in [0,1]$ collectively, we study the smallest possible type-II errors, afforded by the Neyman Pearson test. We can regard this as a function
%\[
%\alpha \mapsto T(\alpha) := \inf \{\beta^{(g)}: g \,\,\textnormal{ test }\,\, \textnormal{ with } \,\, \alpha^{(g)}\le \alpha\}.
%\]
%We use the notation "$T$" for this curve, because it illustrates the optimal trade-off between type-I and type-II errors. We refer to $T$ as the \textit{optimal trade-off function}. 
%As shown in \cite{Dong2022} any trade-off function is continuous, non-increasing and convex. Higher-values of $T$ correspond to higher errors and by our above reasoning to higher indistinguishability and thus higher privacy. The highest privacy guarantee is given in the case where $P=Q$ and the analyst can only randomly guess between $H_0-H_1$. In this case, $T$ equals the diagonal $T(\alpha)=1-\alpha$.
%For a (potentially different) trade-off curve $f$ we say that the pair $P,Q$ satisfies $f$-DP if 
%\[
%T(\alpha) \ge f(\alpha), \qquad \forall \alpha \in [0,1].
%\]
%After the discussion of $f$-DP, we now turn to the technical tools of our black-box methods, namely kernel density estimation and machine learning classifiers.

\subsection{Kernel Density Estimation} \label{sec:kde}
Kernel density estimation (KDE) is a well-studied tool from non-parametric statistics to approximate an unknown density $p$ by an estimator $\hat{p}$.
% and $\hat{q}$. 
More concretely, in the presence of sample data $\rX_1, \dots, \rX_n \sim p$ with $\rX_i \in \mathbb{R}^d$, the KDE for $p$ is given by 
\begin{align*}
    \hat{p}(t) := \frac{1}{n b^d} \sum_{i=1}^n K\Big( \frac{t-\rX_i}{b}\Big).
\end{align*}
One can think of the KDE as a smoothed histogram where the bandwidth parameter $b >0$ corresponds to the bin size for histograms. The kernel function $K$ indicates the weight we assign each observation $X_i$ and is oftentimes taken to be the Gaussian kernel with
\begin{align*}
   K(t) = \frac{1}{(2 \pi)^{d/2}} \; \exp \left( -\frac{\vert t \vert^2}{2} \right).
\end{align*}
The appropriate choice of $b$ and $K$ can ensure the uniform convergence of $\hat p$ to the true, underlying density $p$ (as in Assumption \ref{ass1}). Higher smoothness of the density $p$ is generally associated with faster convergence rates and we refer to \cite{Jiang2017}  and \cite{Scott2015} for a rigorous definition of KDE and associated convergence results.

%\todo{TODO: @Onder Merge with Overview of Techniques}
%\textbf{Density estimation in DP} Density estimation in general and KDE in particular is 
%an important tool in the black box assessment of DP. For some examples, we refer to \cite{??}, \cite{??} and \cite{..}\todo{please add the references you had in mind}. The reason is that DP can typically be expressed as some transformation of the density ratio $p/q$ - this is true for standard DP (a supremum), Rényi DP (an integral) and, as we exploit in this paper, $f$-DP via the Neyman-Pearson test.
%, i.e., 
%\begin{align*}
%    \mathbb{P} \left(\sup_{t} |\hat{p}(t)-p(t)|>a_n \right)=o(1)
%\end{align*}
%for a sequence $(a_n)_n$ with $a_n >0$ that converges to $0$. 
%We point the interested reader to \cite{Jiang2017}  and \cite{Scott2015} for a suitable choice of $K$ and $b$ and the construction of KDEs in general.



\subsection{Machine Learning Classifiers} \label{sec:class}
\textbf{Binary classifiers} is the final addition to our technical toolbox. We begin with some notations: We denote a generic classifier on the Euclidean space $\mathbb{R}^d$ by $\phi$. Formally, a {\em classifier} is not different from a statistical test: It is a (potentially random) binary function $\phi: \mathbb{R}^d \to \{0,1\}$. However, its interpretation is different from hypothesis testing, because we do not have a default belief in a label $0$ or $1$. 
%both binary labels $0$ and $1$ are equal and we do not have a default belief. 
Let us now consider a probability distribution $\mathcal{P}$ on the combined space of inputs and outputs $\mathbb{R}^d \times \{0,1\}$. A classification error has occurred for a pair $(x,y) \in \mathbb{R}^d \times \{0,1\}$, whenever $\phi(x) \neq y$. If $(x,y)$ are randomly drawn from $\mathcal{P}$, we define the risk of the classifier $\phi$ w.r.t. to $\mathcal{P}$ as
    \begin{align*}
        R(h) = \Pr\limits_{(x,y)\sim \Dist}[\phi(x) \neq y].
    \end{align*}
\smallskip
\noindent \textbf{Bayes Classification Problem.} The Bayes classification problem refers to a setup to generate the distribution $\mathcal{P}$, where a Bernoulli random variable $\rY \in \{0,1\}$ is drawn and then, a second variable $X$ with
\begin{align*}
    (\rX|\rY=0) \sim P, \qquad (\rX|\rY=1) \sim Q.
\end{align*}
% and the task is to minimize the risk. 
In our work, we specifically consider the case where $\rY$ is drawn from a fair coin flip (i.e., $\Pr[\rY=0] = \Pr[\rY=1] = \frac{1}{2}$), and we denote this setup by $\bbcP{P}{Q}$.

\smallskip
% \noindent \textbf{Optimal classifiers} The \textit{Bayes optimal classifier} $\phi^*$ is the one that has minimal risk in the Bayes classification problem. In practice, $\phi^*$ is typically unknown, because it depends on the (unknown) $P,Q$. However, it is possible to approximate $\phi^*$ using the feasible nearest neighborhood classifier. More concretely, one train a kNN classifier $\kNNclassifier{n}$ with $n$ samples by simply storing $n$ independent samples from distribution $\mathcal{P}$. To predict the label of an observation $o \in \cO,$ $\kNNclassifier{n}$ returns the label taking a majority vote of the class labels of its $k$ nearest neighbors (Euclidean distance in our context) in the stored training points. 

\noindent \textbf{Bayes (Optimal) classifiers.} $\phi^*$ minimizes the risk in the Bayes classification problem. However, $\phi^*$ is usually unknown in practice because it depends on the (unknown) $P$ and $Q$. To approximate $\phi^*$, one can use a feasible nearest-neighbor classifier~\cite{altman1992introduction}. Specifically, a $k$-nearest neighbors ($k$-NN) classifier, denoted as $\kNNclassifier{n}$, assigns a label to an observation $o \in \cO$ by identifying its $k$ closest neighbors\footnote{In our context, closeness is measured using Euclidean distance} from the size $n$ training set. The label is then determined by a majority vote among these $k$ neighbors.


% Specifically, a $k$-nearest neighbors ($k$-NN) classifier, denoted as $\kNNclassifier{n}$, can be trained using $n$ independent samples drawn from the distribution $\mathcal{P}$. To predict the label of an observation $o \in \cO$, $\kNNclassifier{n}$ returns a label based on a majority vote of the class labels of the $k$ nearest neighbors (measured using Euclidean distance in our context) among the stored training samples.

% \textcolor{orange}{\textbf{Yu:} Write (very very shortly) what the kNN classifier is - it is not defined. Do not use additional notations, besides $\kNNclassifier{n}$. Then state the below theorem with the new, \underline{pruned notations} from the above paragraph! Adjust notations in 5.1 accoridngly. We need to reduce notations!!}

The following convergence result for $k$-NN gauges how close the true risk $R(\kNNclassifier{n})$ of the $k$-NN classifier $\kNNclassifier{n}$ is to the risk of the optimal classifier, $R(\phi^{*})$. 


\begin{theo} [\textbf{Convergence of $k$-NN Classifier}~\cite{Books:DGL96}]
\label{thm:covergence of kNN}
Let $\Dist$ be a joint distribution with  support  $\cO \times \mathcal{Y}.$ If the conditional distribution $\Dist|\mathcal{Y}$ has a density, $\cO \subseteq \mathbb{R}^d,$ and $k = \sqrt{n},$ then for every $\epsilon >0$ there is an $n_0$ such that for $n>n_0,$ 
{\small
    \begin{align*}
        \Pr[|R(\kNNclassifier{n}) - R(\phi^{*})| > \epsilon] \leq 2e^{-n\epsilon^2/(72c^2_d)},
    \end{align*}
}
where $c_d$\footnote{By Lemma 5.5 of~\cite{Books:DGL96}, $c_d$ satisfies $c_d \leq (1+{2}/{\sqrt{2-\sqrt{3}}})^d - 1$.} is the minimal number of cones centered at the origin of angle $\pi/6$ that cover $\mathbb{R}^d.$ Note that if the number of dimensions $d$ is constant, then $c_d$ is also a constant.
\end{theo}


%\todo{TODO: @Onder Merge with Overview of Techniques}
%\textbf{Classifiers in DP} The recent Eureka estimator~\cite{Lu2024} related Bayes classifiers to DP. Through the established connection between DP and $f$-DP~\cite{Dong2022}, their results imply an indirect link between Bayes classifiers and $f$-DP already. However, this connection is somewhat obscure, as fully recovering the $f$ function requires knowledge of every parameter tuple $(\varepsilon, \delta)$ for $\varepsilon \geq 0$ in approximate DP, which appears computationally infeasible. In this paper, we study (optimal) classifiers directly within the framework of $f$-DP, forging a new and explicit link between the two.

% \textbf{Classifiers in DP} The recent Eureka estimator~\cite{Lu2024}  related Bayes classifiers to  DP. Through the established connection between  DP and $f$-DP~\cite{Dong2022}, their results imply an indirect link between Bayes classifiers and $f$-DP already. However, this connection is rather obscure, since the relation of $f$-DP and approximate DP itself is non-trivial \MD{I dont understand why the relation is non-trivial?
% We might want to say sth like: the relation between $f$-DP and approximate DP is local in a sense that any pair $(\varepsilon,\delta)$ only leads one point $(\alpha,\beta)$?! Therefore to make use of the connection, one would have to compute many $\varepsilon, \delta$ pairs}. In this paper, we study (optimal) classifiers directly within the framework of $f$-DP, forging a new and explicit link between the two.

% \textbf{Classifiers in DP}  Recently, optimal classifiers have been related to approximate DP in \cite{Lu2024} and given the connection of approximate DP to $f$-DP, they are indirectly connected to $f$-DP already. This connection however is rather obscure, since the relation of $f$-DP and approximate DP itself is non-trivial. In this paper, we study (optimal) classifiers in the context of $f$-DP and forge a new, direct link between the two topics.





\section{Overview of Techniques}\label{sec:overview_techniques}
Our goal is to provide an estimation and auditing procedure for the optimal privacy curve $f$ of a mechanism $\Mech$. This task can be broken down into two parts: (1) Selecting datasets $\DB,\DB'$ that cause the largest difference in $\Mech$'s output distributions and (2) Developing an estimator/auditor for the trade-off curve given that choice of $\DB, \DB'$. %for the trade-off function $T = T(M(D),M(D'))$. 
In line with previous works on black-box estimation/auditing, we focus on task (2). The selection of $\DB,\DB'$ has been studied in the black-box setting and can typically be guided by simple heuristics \cite{StatDP, DP-Sniper, Lokna2023}.% We will therefore focus on part (2) and rely on output samples gathered from $M$ on $D,D'$ for this.


Our proposed estimator of a trade-off curve relies on KDEs. Density estimation in general and KDE in particular is 
an important tool in the black box assessment of DP. For some examples, we refer to \cite{Liu2019}, \cite{Dette2022} and \cite{Kutta2024}. The reason is that DP can typically be expressed as some transformation of the density ratio $p/q$ -- this is true for standard DP (a supremum), Rényi DP (an integral) and, as we exploit in this paper, $f$-DP via the Neyman-Pearson test. A feature of our new approach is that we do not simply plug in our estimators in the definition of $f$-DP, but rather use them to make a novel, approximately optimal test. This test is not only easier to analyze than the standard likelihood ratio (LR) test but also retains similar properties (see the next section for details).
% , that is easier to study than the standard LR test, but has similar properties (see the next section).\\
%We will work with a carefully randomized approximation of the optimal Neyman-Pearson test in order avoid estimating level sets where $p = q$\todo{This might be a bit specific for reader? e.g., suggest `to avoid issues in Neyman-Pearson when $p = q$'.} We then obtain an estimate for entire trade-off curve $f$ by plugging our KDEs into this randomized test and evaluating it over over a grid in $[0,1]$. Our estimator converges uniformly to the optimal parameter $f$.
%\todo{Overview of techniques was moved here because it seems you might need some of the preliminaries to understand it. }
%Intuitively, our techniques combine the well-known LR (likelihood ratio) test in statistics (Prelim.~\ref{sec:optimal}), with techniques from binary classification (Prelim.~\ref{sec:class}), the latter of which extends the results from a recent DP estimator~\cite{Lu2024}. 
%\todo{Why we focus on a pair of neighbors for fDP} 
%In the stronger black-box setting, it is generally unknown for which datasets a mechanism may fail to satisfy privacy---similar to DP, $f$-DP requires the privacy parameter $f$ to hold for all pairs of neighbors $D, D'$. As such, in our knowledge all previous black-box (and even some white-box, e.g.~\cite{Nasr2023}) works focus on these tasks under one (or a limited number of) pair of neighboring datasets. 
%We discuss the estimator first: given a mechanism $M$ and a pair of neighboring datasets $D, D'$, the goal is to estimate the true trade-off curve between errors $(\alpha, \beta)$ for any test for hypotheses $H_0, H_1$, corresponding to $M(D), M(D')$ (Prelim.~\ref{sec:fdp}). This is done by introducing a {\em new} technique, which we term the {\em perturbed} LR test. This test approximates the well-known, possibly random, optimal LR test, but circumvents the knowledge of the distributions of $M(D), M(D')$ by construction. Under natural assumptions (Assumption~\ref{ass1}) on the densities  of these two distributions, 
%we construct an algorithm which converges to the optimal $f$-DP curve for these two hypotheses. That is, after estimating the curve, we can, given any $\alpha$, output the corresponding optimal $\beta(\alpha)$. \todo{<---Please correct the statements in this  paragraph if they are incorrect.}
%\MD{Do we need this here? I think we already capture the above in the Contributions, which we might want to extend instead of writing this here.}


Our second goal (Sec.~\ref{sec:audit}) is to audit whether a mechanism~$\Mech$ satisfies a claimed trade-off $f$, given datasets $\DB$ and $\DB'$. At a high level, we address this task by identifying and studying the \emph{most vulnerable point} on the trade-off curve $T$ of $\Mech$ --- the point most likely to violate $f$-DP. We begin by using our $f$-DP estimator to compute a value $\eta$ (from the Neyman-Pearson framework in Sec.~\ref{sec:hyp}), which defines a point $\bigl(\alpha(\eta), \beta(\eta)\bigr)$ on the true privacy curve $T$ of the mechanism~$\Mech$. $\eta$ is chosen such that $\bigl(\alpha(\eta), \beta(\eta)\bigr)$ has the largest distance from the claimed trade-off curve~$f$ asymptotically, which we prove in Prop.~\ref{prop1}.
%We prove that it converges to the point in the curve with the largest difference when subtracting from the claimed trade-off curve $f$ (Prop.~\ref{prop1}). Intuitively, this is the {\em most vulnerable point} that is  most likely to violate $f$-DP. 
Next, by extending a technique proposed in~\cite{Lu2024}, we express $\bigl(\alpha(\eta), \beta(\eta)\bigr)$ in terms of the Bayes risk of a carefully constructed Bayesian classification problem, and approximate that Bayes risk using a feasible binary classifier (e.g., $k$-nearest neighbors). By deploying the $k$-NN classifier we obtain a confidence interval that contains our vulnerable point $(\alpha,\beta)$ with high probability.
% We then examine this $\eta$ value of interest further, by extending a technique used in~\cite{Lu2024}. Specifically, our auditor employs a Bayes (optimal) binary classifier (which can be approximated by, e.g., $k$-nearest neighbor) to approximate the corresponding true $(\alpha, \beta)$ value. 
Finally, our auditor decides whether to reject (or fail to reject) the claimed $f$ curve by checking whether the corresponding point $(\alpha, \beta')$ on $f$ with $f(\alpha) = \beta'$ is contained in this interval or not.
% We then compare our estimate of $(\alpha, \beta)$ to the corresponding point $(\alpha, \beta')$ on the given claimed trade-off curve $f$---note that we will compare using the same $\alpha$. %on the $x$-axis.  In other words, we check whether $f$-DP is satisfied, by checking this 'worst' point. 
Leveraging the convergence properties of $k$-NN, our auditor provides a provable and tuneable confidence region that depends on sample size. We also note that the connection between Bayes classifiers and $f$-DP that underpins our auditor may be of independent interest, as it offers a new interpretation of $f$-DP by framing it in terms of Bayesian classification problems.
% In addition to this practical perspective, our newly-forged connection between (Bayes) classifiers and $f$-DP is a highly non-trivial theoretical contribution. It extends the recently-shown~\cite{Lu2024} connection between classifiers and standard DP; the combination of these two results may be of independent interest.


%Our second goal (Sec.~\ref{sec:audit}) is to audit whether, given $D, D'$ and a claimed trade-off $f$, the mechanism $M$ satisfies $f$-DP. This problem is relevant, since typically a DP algorithm is accompanied with a corresponding privacy claim. To audit, we first use our $f$-DP curve estimator from above to identify a pair $(\alpha, \beta)$ with high vulnerability. On a technical level, we search for a threshold $\eta$, where the Neyman-Pearson test produces particularly small errors. ${}$ We then examine the pair $(\alpha(\eta), \beta(\eta))$ extending a technique used in~\cite{Lu2024}. Specifically, we reformulate $f$-DP as an optimal classification problem, which is a contribution of independent interest. We then employ an approximation of the Bayes optimal classifier to solve this problem. The result is a small (square) range of values $\square \subset [0,1^2]$ and we know with high confidence that $(\alpha(\eta), \beta(\eta)) \in \square$. If values in $\square$ are incompatible with the claimed curve, the auditor detects a privacy violation. More precisely, the curve in $f$-DP identifies a region where $(\alpha(\eta), \beta(\eta))$ should be and if this region and the confidence range $\square$ are non-overlapping we have a contradiction (since both should include $(\alpha(\eta), \beta(\eta))$). If this outcome occurs, the auditor reject the privacy claim $f$.
%approximate the corresponding true $(\alpha, \beta)$ value for the audited mechanism. We then compare our estimate of $(\alpha, \beta)$ to the corresponding point $(\alpha, \beta')$ on the given claimed trade-off curve $f$---note that we must compare using the same $\alpha$ on the $x$-axis. In other words, we check whether $f$-DP is satisfied, by checking this 'worst' point. The use of $k$-NN in our auditor gives us a provable and tuneable confidence region that depends on sample size. 
%\todo{added this part about classifiers connection w/dp}
%In addition to this practical perspective, 
%our newly-forged connection between (Bayes) classifiers and $f$-DP is a highly non-trivial theoretical contribution. It  extends the recently-shown~\cite{Lu2024} connection between classifiers and standard DP; the combination of these two results may be of independent interest.


%\textbf{Classifiers in DP} The recent Eureka estimator~\cite{Lu2024} related Bayes classifiers to DP. Through the established connection between DP and $f$-DP~\cite{Dong2022}, their results imply an indirect link between Bayes classifiers and $f$-DP already. However, this connection is somewhat obscure, as fully recovering the $f$ function requires knowledge of every parameter tuple $(\varepsilon, \delta)$ for $\varepsilon \geq 0$ in approximate DP, which appears computationally infeasible. In this paper, we study (optimal) classifiers directly within the framework of $f$-DP, forging a new and explicit link between the two.


%\todo{TODO: Here, say something about the optimal test for the hypothesis pair being LR test and cite relevent paper(s)}.

%\todo{TODO: Say here how to get the worst $\eta$}

%\todo{TODO: estimating w/confidence bound using Bayes estimator}

%\todo{TODO: some words here on application/experiments}




% \paragraph{Paper Organisation} 
% Preliminaries are introduced in Sec.~\ref{sec:prelim}. We propose our $f$-DP curve estimator in Sec.~\ref{sec:4} and auditor in Sec.~\ref{sec:goal2}. We validate and benchmark our constructions in Sec.~\ref{sec6} using the standard Laplace/Gaussian mechanisms, as well as privacy amplification by subsampling and  DP-SGD. We delve into more detail on related work in Sec.~\ref{sec:relatedwork} 

\section{Goal 1: $f$-DP Estimation} \label{sec:4}

In this section, we develop a new method for the approximation of the entire optimal trade-off curve. The trade-off curve results from a study of the Neyman-Pearson test where any type-I error $\alpha$ is associated with the smallest possible type-II error $\beta$ (see our introduction for details). Understood as a function in $\alpha$  we denote the type-II error by $T:[0,1] \to [0,1]$ and call it a trade-off curve. We note that any trade-off curve is continuous, non-increasing and convex (see \cite{Dong2022}).

\subsection{Estimation of the $f$-DP curve}

 
 Our approach is based on the perturbed likelihood ratio (LR) test which mimics the properties of the optimal Neyman-Pearson test, but requires less knowledge about the distributions involved. In the following, we denote by $P,Q$ the output distributions of $M(D), M(D')$ respectively. The corresponding probability densities are denoted by $p,q$.\\
\textbf{The perturbed LR test.} The optimal test for the hypotheses pair 
\[
H_0: X \sim p\quad \textnormal{vs.} \quad H_1: X \sim q
\]
is the Neyman-Pearson test described this test in Section \ref{sec:hyp}.  It is also called a \textit{likelihood ratio} (LR) test, because it rejects $H_0$ if the density ratio satisfies $q(X)/p(X)>\eta$ for some threshold $\eta$. If $q(X)/p(X)=\eta$ the test rejects randomly with probability $\lambda.$
In a black-box scenario this process is difficult to mimic, even if two good estimators, say $\hat p, \hat q$ of $p,q$ are available. Even if $\hat p \approx p$ and $\hat q \approx q$ it will usually be the case that
\[
q(x)/p(x) = \eta  \quad \textnormal{does not imply} \quad \hat q/ \hat p =\eta
\]
(it may hold that $\hat p/ \hat q \approx \eta $, but typically not exact equality). In principle, one could cope with this problem by modifying the condition $\hat q/ \hat p =\eta$ to $\approx \eta$ to mimic the optimal test. Yet, the implementation of this approach turns out to be difficult. In particular, it would involve two tuneable arguments $(\eta, \lambda)$, as well as further parameters (to specify "$\approx$") making approximations costly and unstable. A simpler and more robust approach is to focus on a different test rather than the optimal one - a test that is close to optimal but does not require the knowledge of when $q/p$ is constant. For this purpose, we introduce here the novel \textit{perturbed LR test}. 
We define it as follows: Let $U \in [-1/2, 1/2]$ be uniformly distributed and $h>0$ a (small) number. Then we make the decision
\begin{align} \label{e:PLR}
"\textnormal{reject $\,\,H_0\,\,$ if } \quad q(X)/p(X)>\eta + h U".
\end{align}
Just as the Neyman-Pearson test, the perturbed LR test is randomized. Instead of flipping a coin when $q/p=\eta$, the threshold $\eta$ is perturbed with a small, random noise. Obviously the perturbed LR test does not require knowledge of the level sets $\{q/p=\eta\}$, making it more practical for our purposes.
To formulate a theoretical result for this test, we impose two natural assumptions.
\begin{ass} \label{ass1} $ $
    \begin{itemize}
        \item[i)] The densities $p,q$ are continuous.
        \item[ii)] There exists only a finite number of values $\eta \ge 0$ where the set $\{q/p=\eta\}$ has positive mass.
       % \item[iii)] The optimal trade-off curve $T$ implied by $p,q$ is continuous.
    \end{itemize}
\end{ass}
The second assumption is met for all density models that the authors are aware of and in particular for all mechanisms commonly used in DP. %The third assumption facilitates the presentation of our below results, as is guarantees convergence of the estimated trade-off curve to $T$ uniformly; i.e. simultaneously for all arguments $\alpha$. This condition can be dropped at the expense of a more technical formulation, where uniform convergence is replaced by convergence in the Skorohod metric (see \cite{billingsley:1999}). We do not pursue such generalizations here, since they are not practically relevant.\\
Let us denote the $f$-DP curve of the perturbed LR test by $T_h$. The next Lemma shows that for small values of $h$ the perturbed LR test performs as the optimal LR test.
\begin{lem} \label{lem1}
Under Assumption \ref{ass1} it holds that
\[
\lim_{h \downarrow 0} \sup_{\alpha \in [0,1]}|T(\alpha)-T_h(\alpha)|=0.
\]
\end{lem}
 \textbf{Approximating $T_h$.} The Lemma shows that, to create an estimator of the optimal trade-off curve $T$, it is sufficient to approximate the curve $T_h$ of the perturbed LR test for some small $h$. This is an easier task, since we do not need to know the level sets $\{q/p=\eta\}$ for all $\eta$. Indeed, suppose we have two estimators $\hat p, \hat q$, we can run a perturbed LR test with them, just as in equation \eqref{e:PLR}. A short theoretical derivation (found in the appendix) then shows that running the perturbed LR test for $\hat p, \hat q$ and some threshold $\eta$, yields the following type-I and type-II errors:
\begin{align}
\hat \alpha_h(\eta) := &\int_{x \in [-h/2,h/2]} \frac{1}{h}\int_{\hat q /\hat p  > \eta +x} \hat p , \\\hat \beta_h(\eta) :=& \int_{x \in [-h/2,h/2]} \frac{1}{h}  \int_{\hat q /\hat p  > \eta +x} \hat q .
\end{align}
The entire trade-off-curve for the perturbed LR test with $(\hat p, \hat q)$ is then given by $\hat T_h$ with
\begin{align} \label{e:def:Th}
\hat T_h(\alpha) = \hat \beta_h(\eta) \quad \Leftrightarrow \quad \alpha = \hat \alpha_h(\eta).
\end{align}
For the curve estimate $\hat T_h$ to be close to $T_h$ (and thus $T$), the involved density estimators need to be adequately precise. We hence impose the following regularity condition on them. In the condition, $n$ is the sample size used to make the estimators.
\begin{ass} \label{ass2}
    The density estimators $\hat p, \hat q$ are themselves continuous probability densities and decaying to $0$ at $\pm \infty$ (see eq. \eqref{e:decay} for a precise definition). For a null-sequence of non-negative numbers $(a_n)_{n \in \mathbb{N}}$ they satisfy
    \begin{align*}
     & \Pr[\sup_{x } |\hat p(x)-p(x)|>a_n]=o(1)\\
    and \quad &\Pr[\sup_{x } |\hat q(x)-q(x)|>a_n]=o(1). 
    \end{align*} 
\end{ass}
The above assumption is in particular satisfied by KDE (see Section \ref{sec:kde}), where the convergence speed $a_n$ depends on the smoothness of the underlying densities. However, in principle other estimation techniques than KDE could be used, as long as they produce continuous estimators. The next result formally proves the consistency of $\hat T_h$. The notation of "$o_P(1)$" refers to a sequence of random variables converging to $0$ in probability.

\begin{theo} \label{theo:1}
    Suppose that Assumptions \ref{ass1} and \ref{ass2} hold, and that $h=h_n$ is a positive number depending on $n$ with $h_n \to 0$ and $h_n/a_n \to \infty$. Then, as $n \to \infty$ it follows that
    \[
    \sup_{\alpha \in [0,1]}|\hat T_h(\alpha)-T(\alpha)|=o_P(1).
    \]
\end{theo}
The above result proves that simultaneously for all $\alpha$, the curve $\hat T_h$ approximates the optimal trade-off function $T$. Thus, we have achieved the first goal of this work. The (very favorable) empirical properties of $\hat T_h$ will be studied in Section \ref{sec6}. We have also incorporated Algorithm \ref{alg:pointwise_KDE_estimator} for an overview of the procedure. 
% \begin{algorithm}[h]
% \footnotesize
% \algorithmicrequire \; \parbox[t]{\dimexpr0.9\linewidth-\algorithmicindent}{Black-box access to $M$; Threshold vector $\eta > 0$; Sample size $n$.}\\[0.1cm]
% \algorithmicensure \, An estimate $(\hat{\alpha}(\eta), \hat{\beta}(\eta))$ of $(\alpha(\eta), \beta(\eta))$ for tuple $(P, Q)$.
% \begin{algorithmic}[1]
%     \State Set parameter $h$. 
%     \State Set the density estimation algorithm $\cA$. By default, use the KDE algorithm.
%     \Function{\textnormal{PTLR Estimatior} $\ptlr{h}{\cA}(M, \eta,n)$}{}
%     \State Compute the estimated densities $\hat{p}, \hat{q}$ based on outputs of $M$ by running $\cA$ with a sample size of $n$.
%     \State Compute $\hat{\alpha}(\eta_i) \leftarrow \int_{x \in [-h/2,h/2]} \frac{1}{h}\int_{\hat q /\hat p  > \eta_i +x} \hat p$ for all $\eta_i\in \eta$
%     \State Compute $\hat{\beta}(\eta_i) \leftarrow \int_{x \in [-h/2,h/2]} \frac{1}{h}  \int_{\hat q /\hat p  > \eta_i +x} \hat q$ for all $\eta_i\in \eta$ 
%     \State Return vector $(\hat{\alpha}(\eta), \hat{\beta}(\eta))$
%     \EndFunction
% \end{algorithmic}
% \caption{PTLR: A Perturbed Likelihood Ratio Test Algorithm for $f$-DP Estimation}
% \label{alg:pointwise_KDE_estimator}
% \end{algorithm}


\subsection{Finding maximum vulnerabilities} We conclude this section by some preparations for the second goal - auditing $f$-DP. The precise problem of auditing is described in Section \ref{sec:audit}. Here, we only mention that the task of auditing is to check (in some sense) whether $f$-DP holds for a claimed trade-off curve, say $f=T^{(0)}$.
As an initial step to check $T^{(0)}$-DP  we create the estimator $\hat T_h$ for the optimal curve $T$. If $T^{(0)}$-DP holds, this means that
\begin{align} \label{e:H0fDP}
     T(\alpha)\ge  T^{(0)}(\alpha)\quad \forall \alpha \in [0,1].
\end{align}
A priori, we cannot say whether this is true or not. However, by comparing our estimator $\hat T_h$ with  $T^{(0)}$ we can gather some evidence. For example, if $\hat T_h(\alpha)$ is much smaller than $ T^{(0)}(\alpha)$ for some $\alpha$, then it seems that the claim \eqref{e:H0fDP} is probably false. We will develop a rigorous criterion for what "much smaller" means in the next section. For now, we will confine ourselves to identifying a point, where privacy seems most likely to be broken. We therefore define 
\begin{align} \label{e:def:eta}
\hat \eta^* \in \textnormal{argmax} \big\{T^{(0)}(\hat \alpha_h(\eta))-\hat T_h(\hat \alpha_h(\eta)): \eta\ge 0\big\} 
\end{align}
and the next result shows that the discrepancy between $T^{(0)}$ and $T$ is indeed maximized in $\hat \eta^* $ for large $n$.\\
%\todo{Question: Does this not also have the issue to directly using the estimator, that we don't know the rate of convergence to the true most-vulnerable point?} \textcolor{orange}{Yes.}\todo{To discuss: how to describe our auditor's accuracy (esp in the intro, given that we try to differentiate our estimator/auditor results)}
\begin{prop} \label{prop1}
    Suppose that the assumptions of Theorem \ref{theo:1} hold. Then, it follows that 
    \begin{align*}
    &T^{(0)}(\hat \alpha_h(\hat \eta^*)) - T(\hat \alpha_h(\hat \eta^*)) \\
    =&\sup_{\alpha \in [0,1]}\big[ T^{(0)}(\alpha)-T(\alpha)\big]+o_P(1).
    \end{align*}
\end{prop}
The threshold $\hat \eta^*$ demarcates the greatest weakness of the $T^{(0)}$-privacy claim and it is therefore ideally suited as a starting point for our auditing approach in Section \ref{sec:audit}.


%\begin{itemize}
%    \item Theorem/lemmas for Goal 1 using KDE
%    \item Lemma where KDE can also give you the $\eta$ with the biggest violation and point to next section on how to use that for Goal 2.
%\end{itemize}
\section{Resource Aware Type System and Inference} \label{sec:inference}

In this section, we present the resource-aware type system based on RABC introduced in \cref{sec:calculus} and a type-inference algorithm based on the AARA methodology.
%
\cref{sec:inference:types} introduces resource-enriched types, which augment the types of RABC with resource annotations.
%
% Rich type $\tau$ is plain type $t$ enriched with potential annotation $\alpha$, e.g. $\kwd{list}(\alpha)$ for $\kwd{list}$. Then, we introduce the typing context and its read/write operation used for type checking, followed by the definition of function signatures. 
%
\cref{sec:inference:subtyping} formulates a subtyping relation among resource-enriched types and uses the relation to construct a lattice of types sketched in \cref{sec:overview:Lattice}.
%
% With subtyping, we can tell what rich types are well formed. The subtyping relation over rich types actually forms a lattice with meet and join operations. We define the lattice operations and extend it to the typing context.
%
\cref{sec:inference:eval} and \cref{sec:inference:exec} present the resource-aware typing rules for expressions and statements, respectively.
%
\cref{sec:inference:infer} discusses a type-inference algorithm for the resource-aware type system.

\subsection{Rich Types, Contexts, and Signatures} \label{sec:inference:types}
\begin{figure}[t]
\small
    \begin{align*}
    \tag{undefined} \textbf{RichType}~ \tau &::= \bot \\
    \tag{atom types} &|~ \kwd{i32} ~|~ \kwd{bool} \\
    \tag{list} &|~ \kwd{list}(\alpha)\\
    \tag{box} &|~ \kwd{box}(\kwd{list}(\alpha)) \\
    \tag{shared borrow} &|~ \&^\kwd{s}(\tau) \\
    \tag{mutable borrow} &|~ \&^\kwd{m}(\tau_\text{c}, \tau_\text{p})
    \end{align*}
    \caption{Rich Types}
    \label{fig:rich-type}
\end{figure}
\begin{figure}[t]
\small
    \judgement{Enrich (Selected)}{$\textit{enrich}~ t ~\textit{as}~ \tau$}
    \begin{mathpar}
    \inferrule[Enrich-List]
    {\alpha~\text{fresh}}
    {\textit{enrich}~ \kwd{list} ~\textit{as}~  \kwd{list}(\alpha)}
    \and
    \inferrule[Enrich-Shared]
    {\textit{enrich}~ t ~\textit{as}~ \tau}
    {\textit{enrich}~ \&^\kwd{s}(t) ~\textit{as}~ \&^\kwd{s}(\tau)}
    \and
    \inferrule[Enrich-Mutable]
    {\textit{enrich}~ t ~\textit{as}~ \tau_\text{c}
    \\ \textit{enrich}~ t ~\textit{as}~ \tau_\text{p}
    }
    {\textit{enrich}~ \&^\kwd{m}(t) ~\textit{as}~ \&^\kwd{m}(\tau_\text{c}, \tau_\text{p})}
    \end{mathpar}
    \caption{Enrichment}
    \label{fig:enrich}
\end{figure}

\begin{figure}[t]
\small
    \judgement{Context Reading}{$\Gamma\vdash p \hookrightarrow \tau$}
    \begin{mathpar}
    \inferrule[$\Gamma$-Rd-Var]
    {\Gamma(x)=\tau}
    {\Gamma\vdash x \hookrightarrow \tau}
    \and
    \inferrule[$\Gamma$-Rd-Box]
    {\Gamma\vdash p \hookrightarrow \kwd{box}(\tau)}
    {\Gamma\vdash * p\hookrightarrow \tau}
    \and
    \inferrule[$\Gamma$-Rd-Shared]
    {\Gamma\vdash p \hookrightarrow \&^\kwd{s}(\tau)}
    {\Gamma\vdash *p \hookrightarrow \tau}
    \and
    \inferrule[$\Gamma$-Rd-Mutable]
    {\Gamma\vdash p \hookrightarrow \&^\kwd{m}(\tau_\text{c},\tau_\text{p})}
    {\Gamma\vdash * p\hookrightarrow \tau_\text{c}}    
    \end{mathpar}
    
    \judgement{Context Writing}{$\GWt{\Gamma}{p}{\tau}{\Gamma'}$}
    \begin{mathpar}
    \inferrule[$\Gamma$-Wt-Var]
    {\forall y\not=x, \Gamma'(y)=\Gamma(y) 
    \\ \Gamma'(x) = \tau}
    {\GWt{\Gamma}{x}{\tau}{\Gamma'}}
    \and
    \inferrule[$\Gamma$-Wt-Box]
    {\Gamma\vdash p\hookrightarrow \kwd{box}(\_)
    \\ \GWt{\Gamma}{p}{\kwd{box}(\kwd{list}(\alpha))}{\Gamma'}
    }
    {\GWt{\Gamma}{*p}{\kwd{list}(\alpha)}{\Gamma'}}
    \\
    \inferrule*[Right=\rulename{$\Gamma$-Wt-Shared}]
    {\Gamma\vdash p\hookrightarrow \&^\kwd{s}(\_)
    \\ \GWt{\Gamma}{p}{\&^\kwd{s}(\tau)}{\Gamma'}
    }
    {\GWt{\Gamma}{*p}{\tau}{\Gamma'}}
    \\
    \inferrule*[Right=\rulename{$\Gamma$-Wt-Mutable}]
    {\Gamma\vdash p\hookrightarrow \&^\kwd{m}(\tau_\text{c}, \tau_\text{p})
    \\ \vdash \tau_\text{c}
    \\ \GWt{\Gamma}{p}{\&^\kwd{m}(\tau, \tau_\text{p})}{\Gamma'}
    }
    {\GWt{\Gamma}{*p}{\tau}{\Gamma'}}
    \end{mathpar}
    \caption{Context Reading and Writing}
    \label{fig:sta-rw}
\end{figure}

\begin{figure}[t]
\small
    \judgement{Signatures}{$\vdash f \Rightarrow (\Gamma_f, \delta_f)$}
    \begin{mathpar}
    \inferrule
    {\text{fn}~ f ~(\vec{x}_\text{param}:\vec{t}_\text{param}, \vec{x}_\text{local}:\vec{t}_\text{local}, x_\text{ret}:t_\text{ret}) \{~ s ~\}
    \\\\ \textit{enrich}~ \vec{t}_\text{param} ~\textit{as}~ \vec{\tau}_\text{param}
    \\ \textit{enrich}~ \vec{t}_\text{local} ~\textit{as}~ \vec{\tau}_\text{local}
    \\ \textit{enrich}~ t_\text{ret} ~\textit{as}~ \tau_\text{ret}
    \\\\ \GWt{\emptyset}{\vec{x}_\text{param}}{\vec{\tau}_\text{param}}{\Gamma_1} 
    \\ \GWt{\Gamma_1}{\vec{x}_\text{local}}{\vec{\tau}_\text{local}}{\Gamma_2}
    \\ \GWt{\Gamma_2}{x_\text{ret}}{\tau_\text{ret}}{\Gamma_f}
    \\ \delta_f ~\text{fresh} }
    {\vdash f \Rightarrow (\Gamma_f, \delta_f)}
    \end{mathpar}
    \caption{Function Signatures}
    \label{fig:fun-sig}
\end{figure}

\textbf{Rich types} are types enriched with potential annotation $\alpha$ as in \cref{fig:rich-type} and \cref{fig:enrich}. 
%
The rich type $\bot$ denotes zero potential as the minimum among all rich types. 
%
The rich type $\kwd{list}(\alpha)$, represents the potential function $\alpha\cdot n$ for list $l$ with length $n$.
%
In shared borrows $\&^\kwd{s}(\tau)$, $\tau$ represents the potential function of borrowed value. 
% 
Mutable borrows $\&^\kwd{m}(\tau_\text{c}, \tau_\text{p})$ contains 2 components. $\tau_\text{c}$ is the \textbf{c}urrent type, which denotes the current potential of mutable borrow. $\tau_\text{p}$ is the \textbf{p}rophecy type, which denotes the prophecy potential when the mutable borrow ends.  

Typing \textbf{context} $\Gamma : \mathbf{Variable}\to\mathbf{RichType}$ is a partial map, where unused variables can be mapped to $\bot$. Similarly, in \cref{fig:sta-rw}, we extend the reading and writing operation on typing context from variable $x$ to place $p$. It is worth noting that rules \rulename{$\Gamma$-Rd-Mutable} and \rulename{$\Gamma$-Wt-Mutable} indicate to read and write the mutable borrow on its current component $\tau_\text{c}$. We explicitly point out that $\vdash \tau_\text{c}$ in the premise of rule \rulename{$\Gamma$-Wt-Mutable} is \textbf{dropping condition} for soundness, detailed in \cref{sec:inference:subtyping}. Because when we update $\tau_\text{c}$, the old $\tau_\text{c}$ should be restored if it is a mutable borrow.

\textbf{Signature} $(\Sigma_f, \delta_f)$ of a function $f$ compose a typing context $\Sigma_f$ and a resource unknown variable $\delta_f \in \ZZ$. As shown in \cref{fig:fun-sig}, context $\Gamma_f$ contains rich types for parameters, local variables, and the return variable. $\delta_f$ indicates the resource consumption irrelevant to parameters.

\subsection{Subtyping, Well-formedness, and Merging} \label{sec:inference:subtyping}
\begin{figure}[t]
\small
    \judgement{Subtyping}{$\tau_1 \preceq \tau_2$}
    \begin{mathpar}
    \inferrule*[Right=\rulename{S-Bot}]
    {~}
    {\bot\preceq\tau}
    \and
    \inferrule*[Right=\rulename{S-Int}]
    {~}
    {\kwd{i32}\preceq\kwd{i32}}
    \and
    \inferrule*[Right=\rulename{S-Bool}]
    {~}
    {\kwd{bool}\preceq\kwd{bool}}
    \and
    \inferrule*[Right=\rulename{S-List}]
    {\alpha_1 \leq \alpha_2}
    {\kwd{list}(\alpha_1)\preceq\kwd{list}(\alpha_2)}
    \\
    \inferrule[S-Box]
    {\alpha_1 \leq \alpha_2}
    {\kwd{box}(\kwd{list}(\alpha_1))\preceq\kwd{box}(\kwd{list}(\alpha_2))}
    \and
    \inferrule[S-Shared]
    {\tau_1 \preceq \tau_2}
    {\&^\kwd{s}(\tau_1)\preceq\&^\kwd{s}(\tau_2)}
    \and
    \inferrule[S-Mutable]
    {\tau_{\text{c}, 1}\preceq \tau_{\text{c}, 2}
    \\ \tau_{\text{p}, 2}\preceq \tau_{\text{p}, 1} }
    {\&^\kwd{m}(\tau_{\text{c}, 1}, \tau_{\text{p}, 1})\preceq\&^\kwd{m}(\tau_{\text{c}, 2}, \tau_{\text{p}, 2})}
    \end{mathpar}
    \caption{Rich Subtyping}
    \label{fig:rich-subtyping}
\end{figure}
\begin{figure}[t]
\small
    \judgement{Well-formedness}{$\vdash \tau$}
    \begin{mathpar}

    \inferrule[WF-Bot]
    {~}
    {\vdash \bot}
    \and
    \inferrule[WF-Int]
    {~}
    {\vdash \kwd{i32}}
    \and
    \inferrule[WF-Bool]
    {~}
    {\vdash \kwd{bool}}
    \and
    \inferrule[WF-List]
    {\alpha \geq 0}
    {\vdash \kwd{list}(\alpha)}
    \and
    \inferrule[WF-Box]
    {\alpha \geq 0}
    {\vdash \kwd{box}(\kwd{list}(\alpha))}
    \\
    \inferrule*[Right=\rulename{WF-Shared}]
    {\vdash \tau
    }
    {\vdash \&^\kwd{s}(\tau)}
    \and
    \inferrule*[Right=\rulename{WF-Mutable}]
    {\tau_\text{p} \preceq \tau_\text{c}
    \\ \vdash \tau_\text{c}
    \\ \vdash \tau_\text{p}
    }
    {\vdash \&^\kwd{m}(\tau_\text{c}, \tau_\text{p})}
    \end{mathpar}
    \caption{Well-formedness}
    \label{fig:rich-type-wf}
\end{figure}
\begin{figure}[t]
\small
    \judgement{Context Merging}{$\Gamma_1 \sqcap \Gamma_2 = \{ x \hookrightarrow \Gamma_1(x)\cap\Gamma_2(x) : x \in \mathbf{dom}(\Gamma_1)=\mathbf{dom}(\Gamma_2)\}$}
    \judgement{Meet/Join (Selected)}{$\tau_1\cap\tau_2 / \tau_1\cup\tau_2$}
    \begin{mathpar}        
    \inferrule*[Right=Meet-List]
    {\min(\alpha_1, \alpha_2)=\alpha}
    {\kwd{list}(\alpha_1)\cap\kwd{list}(\alpha_2)=\kwd{list}(\alpha)}
    \and
    \inferrule*[Right=Join-List]
    {\max(\alpha_1, \alpha_2)=\alpha}
    {\kwd{list}(\alpha_1)\cup\kwd{list}(\alpha_2)=\kwd{list}(\alpha)}
    \\
    
    \inferrule*[Right=Meet-Shared]
    {\tau_1 \cap \tau_2=\tau}
    {\&^\kwd{s}(\tau_1)\cap\&^\kwd{s}(\tau_2)=\&^\kwd{s}(\tau)}
    \and
    \inferrule*[Right=Join-Shared]
    {\tau_1 \cup \tau_2=\tau}
    {\&^\kwd{s}(\tau_1)\cup\&^\kwd{s}(\tau_2)=\&^\kwd{s}(\tau)}
    \\

    \inferrule*[Right=Meet-Mutable]
    {\tau_{\text{c}, 1} \cap \tau_{\text{c}, 2}=\tau_\text{c}
    \\ \tau_{\text{p}, 1} \cup \tau_{\text{p}, 2}=\tau_\text{p}
    \\ \tau_{\text{p}, 1} \preceq \tau_{\text{c}, 1}
    \\ \tau_{\text{p}, 2} \preceq \tau_{\text{c}, 2}
    }
    {\&^\kwd{m}(\tau_{\text{c}, 1}, \tau_{\text{p}, 1})\cap\&^\kwd{m}(\tau_{\text{c}, 2}, \tau_{\text{p}, 2})=\&^\kwd{m}(\tau_\text{c}, \tau_\text{p})}
    \\
    \inferrule*[Right=Join-Mutable]
    {\tau_{\text{c}, 1} \cup \tau_{\text{c}, 2}=\tau_\text{c}
    \\ \tau_{\text{p}, 1} \cap \tau_{\text{p}, 2}=\tau_\text{p}
    \\ \tau_{\text{p}, 1} \preceq \tau_{\text{c}, 1}
    \\ \tau_{\text{p}, 2} \preceq \tau_{\text{c}, 2}
    }
    {\&^\kwd{m}(\tau_{\text{c}, 1}, \tau_{\text{p}, 1})\cup\&^\kwd{m}(\tau_{\text{c}, 2}, \tau_{\text{p}, 2})=\&^\kwd{m}(\tau_\text{c}, \tau_\text{p})}
    \end{mathpar}
    \caption{Merging}
    \label{fig:sta-merging}
\end{figure}

The order relation $\leq$ on resources derives another order relation on rich types, the \emph{subtyping} relation in \cref{fig:rich-subtyping}. The interpretation of subtyping $\tau_1 \preceq \tau_2$ is that the value $v$ typed with $\tau_1$ has \textbf{less} resource than the value $v$ typed with $\tau_2$. 
%
The rich type $\bot$ is a subtype of any type because $\bot$ denotes zero potential.
%
It is worth noting that in \rulename{S-Mutable}, $\tau_\text{p}$ is contravariant because prophecy type $\tau_\text{p}$ denotes the prophecy potential to return. 
%
The reflexive rule and the transitive rule are derivable.

A well-formed rich type always denotes a non-negative potential function. Our type system can drop well-formed types without sacrificing soundness.
%
\rulename{WF-List} and \rulename{WF-Box} request $\alpha \geq 0$, which makes $\alpha \cdot n \geq 0$ for list $l$ with length $n\geq0$. 
%
\rulename{WF-Shared} is a structural rule. For example, if $\kwd{list}(\alpha)$ is well-formed, so is $\&^\kwd{s}(\kwd{list}(\alpha))$. Rust borrow checker ensures that $\tau$ in $\&^\kwd{s}(\tau)$ satisfies $\tau\not=\&^\kwd{m}(\_, \_)$. Our type system supports nested borrows.
%
Besides structural premises $\vdash \tau_\text{c}$ and $\vdash \tau_\text{p}$, \rulename{WF-Mutable} demands \textbf{dropping condition} $\tau_\text{p} \preceq \tau_\text{c}$ in $\&^\kwd{m}(\tau_\text{c}, \tau_\text{p})$. The condition is called dropping condition because it works as dropping mutable borrows in \cref{fig:ex-prophecy}. The dropping condition makes sure that mutable borrow types denote non-negative potentials, as illustrated in \cref{sec:soundness}. 


\textbf{Merging} is a conservative approximation of resource potentials after conditional branching. Under typing context $\Gamma$, the type system checks statements $s_1$ and $s_2$ in different branches and gets remainder contexts $\Gamma_1$ and $\Gamma_2$.
The type system should merge them to check continuation. 
%
As illustrated in \cref{fig:sta-merging}, to merge typing contexts is to merge rich types at each $x$ in the domain of two contexts. 
% 
Because the prophecy type $\tau_\text{p}$ in mutable borrow is contravariant, we need to define not only the meet of types but also the join of types. Our purpose is to construct a \emph{lattice} with the property that $\tau_1\cap\tau_2\preceq \tau_i \preceq\tau_1\cup\tau_2, \forall i=1, 2$. Hence, merging over contexts is non-increasing on resources to conservatively fulfill soundness. 
%
The lattice operations of $\kwd{list}(\alpha_1)$ and $\kwd{list}(\alpha_1)$ are inherited from the resource's $\min$ and $\max$, so it is readily comprehensible. 
%
Notice that dropping conditions appear in rules \rulename{Meet-Mutable} and \rulename{Join-Mutable}. They are to fulfill soundness for weak updates, which is mentioned in \cref{sec:overview:Lattice}. Recall that dropping borrows without these conditions may increase resources in both original places indicated by $\tau_{\text{p}, 1}$ and $\tau_{\text{p}, 2}$, to break soundness. 

% In practical implementation, the well-formedness here $\vdash\&^\kwd{m}(\tau_{\text{c}, i}, \tau_{\text{p}, i})$ can be exchanged to $\tau_{\text{p}, i}\preceq\tau_{\text{c}, i}$, due to redundant well-formedness for substructures, $\vdash\tau_{\text{c}, i}$ and $\vdash\tau_{\text{p}, i}$ already covered by induction hypothesis on $\tau_{\text{c}, 1}\cap/\cup\tau_{\text{c}, 2}$ and $\tau_{\text{c}, 1}\cup/\cap\tau_{\text{p}, 2}$. 

It is worth noting that we support nested borrows like $\&^\kwd{s}(\&^\kwd{s}(\tau))$, $\&^\kwd{m}(\&^\kwd{s}(\tau_\text{c}), \&^\kwd{s}(\tau_\text{p}))$ and $\&^\kwd{m}(\&^\kwd{m}(\tau_\text{cc}, \tau_\text{cp}), \&^\kwd{m}(\tau_\text{cc}, \tau_\text{pp}))$. Rust's borrow mechanisms exclude nested borrows like shared borrows of mutable borrows $\&^\kwd{s}(\&^\kwd{m}(\tau_\text{c}, \tau_\text{p}))$, because they violate the property that at most one mutable borrow from the same piece of data is live at the same time.

\subsection{Typing Expressions} \label{sec:inference:eval}\
\begin{figure}[t]
\small
\judgement{Typing Expressions (Selected)}{$\Gamma\vdash e \hookrightarrow \tau\dashv\Gamma'$}
    \begin{mathpar}
    \inferrule*[Right=\rulename{$\Gamma$-Ev-Nil}]
    {\alpha ~\text{fresh}}
    {\Gamma\vdash \kwd{nil} \hookrightarrow \kwd{list}(\alpha)\vdash\Gamma}
    \and
    \inferrule*[Right=\rulename{$\Gamma$-Ev-Move}]
    {\Gamma\vdash p \hookrightarrow \tau
    \\ \GWt{\Gamma}{p}{\bot}{\Gamma'}
    }
    {\Gamma\vdash \kwd{move}~p \hookrightarrow \tau\dashv\Gamma'}
    \\

    \inferrule*[Right=\rulename{$\Gamma$-Ev-Shared}]
    {\Gamma\vdash p \hookrightarrow \tau
    \\ \textit{share}~ \tau ~\textit{as}~\tau_1, \tau_2
    \\ \GWt{\Gamma}{p}{\tau_1}{\Gamma'}
    }
    {\Gamma\vdash \&^\kwd{s}~p \hookrightarrow \&^\kwd{s}(\tau_2)\dashv\Gamma'}
    \\
    
    \inferrule*[Right=\rulename{$\Gamma$-Ev-Mutable}]
    {\Gamma\vdash p \hookrightarrow \tau
    \\ \textit{prophesy}~ \tau ~\textit{as}~ \tau_\text{p} 
    \\ \GWt{\Gamma}{p}{\tau_\text{p}}{\Gamma'}
    }
    {\Gamma\vdash \&^\kwd{m}~p \hookrightarrow \&^\kwd{m}(\tau, \tau_\text{p})\dashv\Gamma'}
    \end{mathpar}
    \caption{Typing Expressions}
    \label{fig:sta-eval}
\end{figure}

\begin{figure}[t]
\small
    \judgement{Sharing (Selected)}{$\textit{share}~ \tau ~\textit{as}~\tau_1, \tau_2$}
    \begin{mathpar}
    \inferrule*[Right=\rulename{Share-List}]
    {\alpha_1, \alpha_2 ~\text{fresh}
    \\\alpha = \alpha_1 + \alpha_2}
    {\textit{share}~ \kwd{list}(\alpha) ~\textit{as}~\kwd{list}(\alpha_1), \kwd{list}(\alpha_2)}
    \end{mathpar}
    \judgement{Prophesying (Selected)}{$\textit{prophesy}~ \tau_\text{c} ~\textit{as}~\tau_\text{p}$}
    \begin{mathpar}
    \inferrule*[Right=\rulename{Prophesy-List}]
    {\alpha_\text{p}~\text{fresh}}
    {\textit{prophesy}~ \kwd{list}(\alpha) ~\textit{as}~ \kwd{list}(\alpha_\text{p})}
    \end{mathpar}
    \caption{Sharing and Prophesying}
    \label{fig:sta-sharing-prophesying}
\end{figure}

\cref{fig:sta-eval} presents how to type check expressions via judgement $\Gamma\vdash e\hookrightarrow \tau\dashv\Gamma'$. Unlike the dynamic evaluation $V\vdash e \rightsquigarrow v$, checking expressions may modify $\Gamma$ to the remainder context $\Gamma'$. 
%
Rule \rulename{$\Gamma$-Ev-Nil} introduces a fresh unknown potential annotation $\alpha$ for $\kwd{nil}$. 
%
Rule \rulename{$\Gamma$-Ev-Move} explicitly moves the type $\tau$ out from place $p$, making $\GWt{\Gamma}{p}{\bot}{\Gamma'}$. 
%

Shared and mutable borrows modify typing context, as illustrated in rule \rulename{$\Gamma$-Ev-Shared} and \rulename{$\Gamma$-Ev-Mutable} with sharing and prophesying. \cref{fig:sta-sharing-prophesying} selects essential rules of $\textit{share}~ \tau ~\textit{as}~ \tau_1, \tau_2$ and $\textit{prophesy}~ \tau ~\textit{as}~ \tau_\text{p}$ for borrows. 

\textbf{Shared borrows} are handled with sharing $\textit{share}~ \tau ~\textit{as}~ \tau_1, \tau_2$. Recall the example in \cref{fig:ex-sharing}. We select the rule \rulename{Share-List} to reveal the essence of sharing. Sharing is splitting resource annotation $\alpha$ into $\alpha_1$ and $\alpha_2$ with linear constraint $\alpha = \alpha_1 + \alpha_2$. In rule \rulename{$\Gamma$-Ev-Shared}, we write $\tau_1$ back to original place $p$, with $\tau_2$ lent out. There is no sharing of mutable borrows as $\textit{share}~ \&^\kwd{m}(\_, \_) ~\textit{as}~ \tau_1, \tau_2$, because a well-checked program will never incur shared borrows of mutable borrows $\&^\kwd{s}(\&^\kwd{m}(\tau_\text{c}, \tau_\text{p}))$.

\textbf{Mutable borrows} are handled with prophesying $\textit{prophesy}~ \tau ~\textit{as}~ \tau_\text{p}$. Recall the example in \cref{fig:ex-prophecy}. The selected rule \rulename{Prophesy-List} prophesy $\alpha_\text{p}$ as the prophecy potential when the mutable borrow ends. In rule \rulename{$\Gamma$-Ev-Mutable}, we write prophecy type $\tau_\text{p}$ to the place $p$. Once the borrow ends, the dropping condition $\vdash \&^\kwd{m}(\tau, \tau_\text{p})$ ensures that the prophecy type $\tau_\text{p}$ is bounded by current type $\tau$.

\subsection{Typing Statements} \label{sec:inference:exec}
\begin{figure}[t]
\small
    \judgement{Typing Statements (Selected)}{$\Gamma\vdash s \hookrightarrow^\delta \dashv\Gamma'$}
    \begin{mathpar}
    \inferrule*[Right=\rulename{$\Gamma$-Ex-Tick}]
    {~}
    {\Gamma\vdash\kwd{tick}(\delta)\hookrightarrow^\delta\vdash\Gamma}
    \and
    \inferrule*[Right=\rulename{$\Gamma$-Ex-Drop}]
    {\Gamma\vdash p\hookrightarrow \tau
    \\ \vdash \tau
    \\ \GWt{\Gamma}{p}{\bot}{Gamma'}
    }
    {\Gamma\vdash \kwd{drop}~p \hookrightarrow^0\dashv \Gamma'}
    \\

    \inferrule*[Right=\rulename{$\Gamma$-Ex-Cons}]
    {\Gamma\vdash e_1\hookrightarrow \kwd{i32} \dashv \Gamma_1
    \\ \Gamma_1\vdash e_2\hookrightarrow \kwd{box}(\kwd{list}(\alpha'))\dashv\Gamma_2
    \\ \GWt{\Gamma_2}{p}{\kwd{list}(\alpha')}{\Gamma'}}
    {\Gamma\vdash p\from \kwd{cons}(e_1, e_2)\hookrightarrow^{\alpha'}\dashv\Gamma'}
    \\

    \inferrule*[Right=\rulename{$\Gamma$-Ex-If}]
    {\Gamma\vdash p\hookrightarrow \kwd{bool}
    \\ \Gamma\vdash s_1\hookrightarrow^{\delta_1}\dashv\Gamma_1
    \\ \Gamma\vdash s_2\hookrightarrow^{\delta_2}\dashv\Gamma_2
    \\ \max(\delta_1, \delta_2)=\delta
    \\ \Gamma_1\sqcap\Gamma_2=\Gamma' }
    {\Gamma\vdash \kwd{if}~ p ~\kwd{then}~ s_1 ~\kwd{else}~ s_2 ~\kwd{end} \hookrightarrow^\delta \dashv\Gamma'}
    \\
    \inferrule*[Right=\rulename{$\Gamma$-Ex-Mat}]
    {\Gamma\vdash p\hookrightarrow \kwd{list}(\alpha)
    \\ \Gamma\vdash s_1\hookrightarrow^{\delta_1}\dashv\Gamma_1
    \\\\ \GWt{\Gamma}{p}{\bot}{\Gamma_{\text{b}, 1}}
    \\ \GWt{\Gamma_{\text{b}, 1}}{x_\text{hd}}{\kwd{i32}}{\Gamma_{\text{b}, 2}}
    \\ \GWt{\Gamma_{\text{b}, 2}}{x_\text{tl}}{\kwd{box}(\kwd{list}(\alpha))}{\Gamma_\text{b}}
    \\ \Gamma_\text{b}\vdash s_2\hookrightarrow^{\delta_2}\dashv\Gamma'_\text{b}
    \\\\ \Gamma'_\text{b}\vdash x_\text{tl}\hookrightarrow \kwd{list}(\beta)
    \\ \GWt{\Gamma'_\text{b}}{x_\text{hd}}{\bot}{\Gamma'_{\text{b}, 1}}
    \\ \GWt{\Gamma'_{\text{b}, 1}}{x_\text{tl}}{\bot}{\Gamma'_{\text{b}, 2}}
    \\ \GWt{\Gamma'_{\text{b}, 2}}{p}{\kwd{list}(\beta)}{\Gamma_2}
    \\\\ \max(\delta_1, \delta_2-(\alpha-\beta))=\delta
    \\ \Gamma_1\sqcap\Gamma_2=\Gamma'}
    {\Gamma\vdash \kwd{match}~ p ~ \{\kwd{nil}\mapsto s_1, \kwd{cons}(x_\text{hd}, x_\text{tl})\mapsto s_2\} \hookrightarrow^\delta \dashv\Gamma'}
    \\

    \inferrule*[Right=\rulename{$\Gamma$-Ex-App}]
    {\text{fn}~ f ~(\vec{x}_\text{param}:\vec{t}_\text{param}, \vec{x}_\text{local}:\vec{t}_\text{local}, x_\text{ret}:t_\text{ret}) \{~ s ~\}
    \\\\ \vdash f \Leftarrow (\Gamma_f, \delta_f)
    \\ \Gamma_f\vdash x_\text{ret} \hookrightarrow \tau_\text{ret}, (\forall x_i\in\vec{x}_\text{param}, i=1, ..., n) \Gamma_f \vdash x_i \hookrightarrow \tau_{\text{param}, i}
    \\\\ \Gamma_0=\Gamma, (\forall e_i\in \vec{e}, i=1, ..., n) \Gamma_{i-1}\vdash e_i\hookrightarrow\tau_{\text{arg}, i}\dashv\Gamma_i
    \\ (\forall i=1,..,n)~ \tau_{\text{param}, i} = \tau_{\text{arg}, i}
    \\ \Gamma_n \vdash p \hookrightarrow \tau
    \\ \vdash \tau
    \\ \GWt{\Gamma_n}{p}{\tau_\text{ret}}{\Gamma'}
    }
    {\Gamma\vdash p\from f(\vec{e})\hookrightarrow^{\delta_f}\dashv\Gamma'}
    \end{mathpar}
    \caption{Typing Statements}
    \label{fig:sta-exec}
\end{figure}

\begin{figure}[t]
\small
    \judgement{Function Analysis}{$\vdash f \Leftarrow (\Gamma_f, \delta_f)$}
    \begin{mathpar}
    \inferrule
    {\text{fn}~ f ~(\vec{x}_\text{param}:\vec{t}_\text{param}, \vec{x}_\text{local}:\vec{t}_\text{local}, x_\text{ret}:t_\text{ret}) \{~ s ~\}
    \\  \vdash f \Rightarrow (\Gamma_f, \delta_f)
    \\ \Gamma_f\vdash s\hookrightarrow^\delta\dashv\Gamma'_f
    \\\\ \forall x \in \textbf{dom}(\Gamma'_f), \vdash \Gamma'_f(x)
    \\ \Gamma'_f \vdash x_\text{ret} \hookrightarrow \tau'_\text{ret}
    \\ \Gamma_f \vdash x_\text{ret} \hookrightarrow \tau_\text{ret}
    \\ \tau'_\text{ret} = \tau_\text{ret}
    \\ \delta = \delta_f}
    {\vdash f \Leftarrow (\Gamma_f, \delta_f)}
    \end{mathpar}
    \caption{Function Analysis}
    \label{fig:fun-anal}
\end{figure}

\cref{fig:sta-exec} presents how to type check statements as judgement $\Gamma\vdash s \hookrightarrow^\delta \dashv\Gamma'$. Under context $\Gamma$, the statement $s$ is checked with resource consumption $\delta$, and context becomes $\Gamma'$. 

Rule \rulename{$\Gamma$-Ex-Tick} indicates $\kwd{tick}(\delta)$ consumes $\delta$ unit of resource. Rule \rulename{$\Gamma$-Ex-Drop} drops the type $\tau$ with well-formedness $\vdash\tau$ as the dropping condition. Rule \rulename{$\Gamma$-Ex-Cons} indicates that $\kwd{cons}$ will consume $\alpha$ unit of resource for continuation payment, when the tail $e_2$ is typed with $\kwd{box}(\kwd{list}(\alpha))$. 

\textbf{Branching statements} require context merging, as in \rulename{$\Gamma$-Ex-If} and \rulename{$\Gamma$-Ex-Mat}. Rule \rulename{$\Gamma$-Ex-If} is simpler to merge contexts with the consumption as the maximum of those branches. Rule \rulename{$\Gamma$-Ex-Mat} is more intricate, due to resource potential stored in \kwd{cons}. The $\kwd{cons}$ branch will obtain $\alpha-\beta$ units of potential, therefore the net consumption is $\delta_2-(\alpha-\beta)$. Given $\Gamma\vdash p \hookrightarrow \kwd{list}(\alpha)$, The potential is not $\alpha$ but $\alpha-\beta$. $\beta$ is the remainder potential, indicated by $\Gamma'_\text{b} \vdash x_\text{tl} \hookrightarrow \kwd{list}(\beta)$. The subscript $\text{b}$ of $\Gamma_\text{b}$ means \textbf{b}inding, similar to rule \rulename{V-Ex-Mat}.

\textbf{Function application} is intractable because of recursive functions. Rule \rulename{$\Gamma$-Ex-App} assumes the function $f$ has a well-checked signature $(\Gamma_f, \delta_f)$, with judgement $\vdash f \Leftarrow (\Gamma_f, \delta_f)$ in \cref{fig:fun-anal}, different from $\vdash f \Rightarrow (\Gamma_f, \delta_f)$. Other premises are to ensure that the resources of actual arguments are equal to those of formal parameters. 

\subsection{Type Inference} \label{sec:inference:infer}
To this point, our type system has been primarily declarative because the well-checked signature in rule \rulename{$\Gamma$-Ex-App} is assumed to be pre-existent. Same as other AARA systems (such as Resource-aware ML~\cite{RaML}), we use linear programming to convert the declarative type system to an algorithmic version. The type system creates symbolic variables to denote unknown annotations in rich types and signatures. The type system then collects linear constraints among those symbolic variables and finally solves them via linear programming solvers. 

Readers might have perceived that a recursive function requires a well-checked signature during checking and that a function can exhibit multiple signatures at different call sites. To automatically analyze functions, we need to preprocess the call graph. First, we group recursive functions as strongly connected components. Second, we topologically sort groups to determine an order to analyze. For each group, we predefine signatures of functions in the group via the judgement $\textit{enrich}~ t ~\textit{as}~ \tau$ in \cref{fig:enrich}. During function analysis, the signature $(\Gamma_f, \delta_f)$ in \rulename{$\Gamma$-Ex-App} should be replaced with the predefined one if $f$ is in the group. Otherwise, $f$ is in the previously analyzed group, so we should clone that group's signature and linear constraints. It is necessary to clone instead of copy them because annotations in signatures and constraints are sensitive to actual arguments of function calls.

With linear constraints collected during function analysis and a heuristic objective, we can employ a linear programming solver to find instances of annotations that satisfy those constraints automatically. The inferred annotations in signatures will characterize functions' resource consumption. 

% The $\vdash f \Rightarrow (\Gamma_f, \delta_f)$ in premises of $\vdash f \Leftarrow (\Gamma_f, \delta_f)$, and function application in body $s$, require a topological order of functions to analyze. And (mutual) recursive functions require a strongly connected grouping of functions. In our implementation, all these requirements turn into strongly connected component analysis and topological sorting.



\section{Experiments} \label{sec6}
We investigate the empirical performance of our new procedures in various experiments to demonstrate their effectiveness.
%To demonstrate the effectiveness of our new procedures, we investigate their empirical performance in the following experiments. 
Recall that our procedures are developed for two distinct goals, namely estimation of the optimal trade-off curve $T$ (see Section \ref{sec:4}) and auditing a privacy claim $T^{(0)}$ (see Section \ref{sec:goal2}). We will run experiments for both of these objectives. \\
%These goals correspond to Sections \ref{sec:4} and \ref{sec:goal2} respectively. \\
%This section aims to validate the theoretical results presented in Section~\todo{cite section} and Section~\todo{cite section}. \\
\textbf{Experiment Setting:} 
%We have outlined two distinct objectives along with their corresponding methodologies:
%\begin{description}
 %   \item[\textbf{Goal 1: Uniform Estimation of the Privacy Curve $T$}]  
 %   The first objective is to uniformly estimate an unknown privacy curve $T$, as stated in Theorem~\ref{theo:1}. To validate not only the theoretical correctness but also the practical effectiveness of this estimation approach, we conducted a simulation study on all four mechanisms. The results of this study are presented in \todo{Table~\ref{tab:estimation_f_curves} and Figure~\ref{fig:todo}.}
 %   \item[\textbf{Goal 2: Detection of Privacy Violations}] 
 %   The second objective is inferential in nature. As formulated in Theorem~\ref{theo:auditor}, the goal is to detect privacy violations for a predefined false positive rate. To demonstrate the effectiveness of this methodology, we constructed faulty algorithms and analyzed their behavior. The results of this analysis are depicted in Figure~\ref{fig:todo}.
%\end{description}
Throughout the experiments, we consider databases $\DB,\DB' \in [0,1]^r$, where the participant number is always $r=10$. As discussed in Section \ref{sec:overview_techniques}, we first choose a pair of neighboring datasets such that there is a large difference in the output distributions of $\Mech(D)$ and $\Mech(D')$. We can achieve this by simply choosing $D$ and $D'$ to be as far apart as possible (while still remaining neighbors) and we settle on the choice 
%As typical in the privacy validation literature, we consider two neighboring databases that are far apart. On the $r$-dimensional cube $[0,1]^r$ we make the natural choice of
\begin{equation}\label{eq_databases}
    \DB=(0,\hdots, 0)\quad \textnormal{and} \quad \DB'=(1,0,\hdots, 0)
\end{equation}
for all our experiments.
%and notice that similar results as the ones below hold for other pairs of databases. %Our methods do however work just as well for other data bases $D$ and $D'$.
%Additionally, for data lying in the unit cube, this choice is natural, as these two databases are far apart on the unit cube.

\subsection{Mechanisms}\label{sec:algorithms}
In this section, we test our methods on two frequently encountered mechanisms from the auditing literature: the Gaussian mechanism and differentially private Stochastic Gradient Descent (DP-SGD). We study two other prominent DP algorithms -- the Laplace and Subsampling mechanism -- in Appendix \ref{AppB}. \\
%We apply our methods to four mechanisms frequently encountered in the privacy literature: the Gaussian mechanism, the Laplace mechanism, the Subsampling mechanism, and, most notably, the Noisy Stochastic Gradient Descent (DP-SGD) mechanism. These algorithms are quite  heterogeneous and hence collectively form a good benchmark to evaluate our methods. We quickly review these mechanisms and specify parameter settings. \\

\noindent \textbf{Gaussian mechanism.}
We consider the summary statistic $S(x)= \sum_{i=1}^{10} x_i$ and the mechanism
\begin{equation*}
    M(x):= S(x)+Y~,
\end{equation*}
where $Y\sim \mathcal N (0, \sigma^2)$. The statistic $S(x)$ is privatized by the random noise $Y$ if the variance $\sigma^2$ of the Normal distribution is appropriately scaled. We choose $\sigma = 1$ for our experiments and note that - in our setting - the optimal trade-off curve is given by 
\begin{align*}
     T_{Gauss}(\alpha)= \Phi(\Phi^{-1}(1-\alpha)- 1).
\end{align*}
We point the reader to \cite{Dong2022} for more details. \\








%\textbf{Additive Noise Mechanisms}
%The most prominent DP algorithms are the Laplace and the Gaussian Mechanism. If we consider the summary statistic $S(x)= \sum_{i=1}^{10} x_i$, the output can be described by
%\begin{equation*}
%    M(x):= S(x)+Y~,
%\end{equation*}
%where $Y\sim Lap(0,b)$ or $Y\sim \mathcal N (0, \sigma^2)$, respectively. Here $b>0$ denotes the scale parameter in the Laplace distribution and $\sigma^2$ the variance for the normal distribution. Given the structure of $M$, these mechanisms are generically referred to as "additive noise mechanisms". Appropriately scaled, both mechanisms fulfill $f$-DP. For the Gaussian mechanism we set $\sigma=1$, for which \cite{Dong2022} derived the trade-off curve
%\begin{equation*}
%    T_{Gauss}(\alpha)= \Phi(\Phi^{-1}(1-\alpha)-\mu)
%\end{equation*}
%as an explicit expression of the optimal trade-off function between $P = \mathcal N(0,1)$ and $Q = \mathcal N(\mu,1)$. For the Laplace mechanism, we set $b=1$, and note again that \cite{Dong2022} derived an explicit formula given by
%\begin{equation*}
%    T_{Lap}(\alpha)=\begin{cases}
%        1-\exp(\mu)\alpha,  &\alpha<\exp(-\mu)/2~,\\
%        \exp(-\mu)/4 \alpha,  &\exp(-\mu)/2\leq \alpha\leq 1/2~,\\
%        \exp(-\mu)(1-\alpha), &\alpha>1/2
 %   \end{cases}
%\end{equation*}
%for the optimal trade-off function between $P = Lap(0,1)$ and $Q = Lap(\mu,1)$.\\
%After the simpler additive noise mechanisms, we turn to the more advanced mechanisms of subsampling and the  DP-SGD, both of which play a role in the context of private machine learning.\\
%\textbf{Subsampling Mechanism:} Random subsampling provides an effective way to enhance the privacy of a DP mechanism $M$. We only provide a rough outline here and refer for details to \cite{Dong2022}.
%In simple words, we choose an integer $m$ with  $1\leq m< r$, where $r$ is the size of the databases $D$. In subsampling, we extract random subsamples of size $m$ of these databases, giving us the smaller databases $\bar D$. The mechanism $M$ is then applied to  $\bar D$ instead of $D$, providing an additional layer of protection to users. If a user is not part of $\bar D$, their privacy cannot be compromised. The amplifying effect of subsampling is visible in the optimal trade-off curve: If $M$ implies the curve $T$, it turns out that $M(\bar D)$ implies the curve
%\begin{equation*}
%    \bar T(\alpha)=  \frac{m}{r}T(\alpha)+\frac{r-m}{r}(1-\alpha),
%\end{equation*}
%which is strictly more private than $T$ for any $m<r$. A minor technical peculiarity of subsampling is that the resulting curve $\bar T$ is generally not symmetrical, even if $T$ is (see \cite{Dong2022} for details on the symmetry of trade-off functions). Trade-off curves are usually considered to be symmetrical and one can symmetrize $\bar T$ by applying a symmetrizing operator $\mathbf{C}$ (again, see \cite{Dong2022}). In our simulations we will adhere to this procedure and depict the symmetrized version $\mathbf{C}[\bar T]$ together with a symmetrized estimator. Further details on the symmetrization can be found in Appendix \ref{AppB}.
%At first glance, this result may seem complicated, but it is in fact quite natural. Suppose we select $m$ entries randomly for the two neighboring databases $\DB,\DB'$ with each of size $k$. All entries in  $\DB,\DB'$ except one are identical - so the chance of selecting the entry where they differ (in $m$ draws) is $p=m/k$. 
%Conversely the probability of not selecting named entry is $(1-p)$.  
%This characterizes the construction of $T_p$, and the factor $(1-\alpha)$ corresponds to perfect privacy, as in that case $\alpha+\beta=\alpha+1-\alpha=1$. With that in hand, $C_p(T)=\min\{T_p,T_p^{-1}\}^{**}$ (again see in \cite{Dong2022}) is simply a symmetrization of $T_p$ in a sense that it is the greatest convex minorant of the minimum $T_p$ and $T_p^{-1}$ (\todo{maybe cite sth}). Here, for a function $T$, $T^{**}$ denotes the twice convex conjugate of $T$.\\  
%For the following experiments involving subsampling, we use the Gaussian mechanism as $M$ (with $\sigma=1$) and obtain the subsampled version $M'$, by fixing the parameter $m=5$ (recall that $r=10$). \\
%Observing only independent outputs of $M(x)$ will only yield an estimator for $\hat T_p$. As an additional contribution, we approximate $T_{sub}(\alpha)$ by incorporating a numerical approximation based on $\hat T_p$. \todo{should we expand here?}\\

\noindent \textbf{DP-SGD.} The DP-SGD mechanism is designed to (privately) approximate a solution for the empirical risk minimization problem
\begin{equation*}
\theta^*=argmin_{\theta\in \Theta} \mathcal L_x(\theta) \quad \text{with} \quad \mathcal L_x(\theta)=\frac{1}{r}\sum_{i=1}^{r} \ell(\theta, x_i)~.
\end{equation*}
Here, $\ell$ denotes a loss function, $\Theta$ a closed convex set and $\theta^*\in \Theta$ the unique optimizer. For sake of brevity, we provide a description of DP-SGD in the appendix (see Algorithm \ref{alg:noisy_sgd}). In our setting, we consider the loss function $\ell(\theta, x_i)=\frac{1}{2} (\theta-x_i)^2$, initial model $\theta_0=0$ and $\Theta=\mathbb{R}$. The remaining parameters are fixed as $\sigma=0.2, \rho = 0.2, \tau = 10, m=5$. In order to have a theoretical benchmark for our subsequent empirical findings, we also derive the theoretical trade-off curve $T_{SGD}$ analytically for our setting and choice of databases (see Appendix \ref{AppB} for details). Our calculations yield
%For the choice of databases as in equation \eqref{eq_databases}, one can compute the trade-off curve $T_{SGD}$ analytically: 
\begin{equation*}
    T_{SGD}(\alpha)=\sum_{I\subset \{1,\hdots, \tau \}} \frac{1}{2^{\tau}}\Phi\Big(\Phi^{-1} (1-\alpha)-\frac{\mu_I}{\bar\sigma}\Big)~.
\end{equation*}
where $\mu_I$ is chosen as in \eqref{mu_I} and $\bar{\sigma}$ as in \eqref{sigma_bar}.

\subsection{Simulations}
We begin by outlining the parameter settings of our KDE and $k$-NN methods for our simulations. We then discuss the metrics employed to validate our theoretical findings and, in a last step, present and analyze our simulation results.\\
\textbf{Parameter settings:}
%For the subsequent simulations we always use the same parameters across all algorithms, acknowledging the black-box setting. 
For the KDEs, we consider different sample sizes of $n_1=10^2,10^3,10^4,10^5,10^6$ and we fix the perturbation parameter at $h=0.1$. For the bandwidth parameter $b$ (see Sec. \ref{sec:kde}), we use the method of \cite{bandwidth}. To approximate the optimal trade-off curve, we use $1000$ equidistant values for $\eta$ between $0$ and $15$ (see Algorithm \ref{alg:pointwise_KDE_estimator} for details on the procedure). For the $k$-NN, we set the training sample size to $n_2=10^6,10^7,10^8$ and testing sample size to $10^6$. \\
%\todo{Yu: are you sure that this is sufficient?}\\

\noindent \textbf{Estimation}
The first goal of this work is estimation of the optimal trade-off curve $T$. In our experiments, we want to illustrate the uniform convergence of the estimator $\hat T_h$ to the optimal curve $T$, derived in Theorem \ref{theo:1}. Therefore, we consider increasing sample sizes $n_1$ to study the decreasing error. The distance of $\hat T_h$ and $T$ in each simulation run is measured by the  uniform distance\footnote{Of course, one cannot practically maximize over all (infinitely many) arguments $\alpha \in [0,1]$. The estimator $\hat T_h$ is made for a grid of values for $\eta$ (see our parameter settings above) and we maximize over all gridpoints.} %maximum distance on a grid $G$ 
%We repeatedly estimate the respective trade-off curves of the four mechanism introduced in Section \ref{sec:algorithms} and computed 
\[
    Error_T:=\sup_{\alpha \in [0,1]}|\hat T_h(\alpha)-T(\alpha)|.
\]
%on a grid $G$ of $[0,1]$. 
%In our setting, we defined $G$ as the grid given by the KDE. Since, we choose $1000$ $\eta$ equidistant, we will get $1000$ $\alpha$ values. However, they do not have to be equidistant nor unique. 
To study not only the distance in one simulation run, but across many, we calculate $Error_T$ in $1000$ independent runs and take the (empirical) mean squared error
\begin{equation}\label{eq:mse}
    MSE(Error_T):= \Ex{Error_T^2}
    %\mathbb{E}\mathbb Var(Error_G)+\mathbb E[Error_G]^2~.
\end{equation}
The results are depicted in Figure \ref{fig:estimation_mse} for the DP algorithms described in this section and the appendix. On top of that, we also construct figures that upper and lower bound the worst case errors for the Gaussian mechanism and DP-SGD over the $1000$ simulation runs. These plots visually show how the error of the estimator $\hat T_h$ shrinks as $n_1$ grows. 
% for the sample size $n_1=1000$. 
%For that, we computed the worst estimation point wise on an equidistant discretization of $[0,1]$ and interpolated the curves linearly. 
The results are summarized in Figures \ref{fig:gaussian}-\ref{fig:sgd}.\\
\begin{figure}
\centering\includegraphics[width=0.75\linewidth]{Figures/plot_table.png}
    \caption{\centering
    Empirical MSE defined in \eqref{eq:mse} to empirically validate Theorem \ref{theo:1} for varying sample sizes $n_1$ and over $1000$ simulation runs each.}\label{fig:estimation_mse}
\end{figure}
\begin{figure*}[h!]
    \centering
    \subfloat[$n_1=10^3$]{\includegraphics[width=0.3\textwidth]{Figures/Gaussian_shade_1000.png}}
    \hfill
    \subfloat[$n_1=10^4$]{\includegraphics[width=0.3\textwidth]{Figures/Gaussian_shade_10000.png}}
    \hfill
    \vspace{-0.2cm}
    \subfloat[$n_1=10^5$]{\includegraphics[width=0.3\textwidth]{Figures/Gaussian_shade_100000.png}}
    \caption{Estimation of the Gaussian Trade-off curve $T_{Gauss}$ for varying sample sizes and $\mu=1$. Min- and Max Curve lower- and upper bound the worst point-wise deviation from the true curve $T_{Gauss}$ over $1000$ simulations.}
    \label{fig:gaussian}
\vspace{-0.1cm}
\centering
    \subfloat[$n_1=10^3$]{\includegraphics[width=0.3\textwidth]{Figures/SGD_shade_1000.png}}
    \hfill
    \subfloat[$n_1=10^4$]{\includegraphics[width=0.3\textwidth]{Figures/SGD_shade_10000.png}}
    \hfill
    \vspace{-0.2cm}
    \subfloat[$n_1=10^5$]{\includegraphics[width=0.3\textwidth]{Figures/SGD_shade_100000.png}}
    \caption{Estimation of the DP-SGD Trade-off curve $T_{SGD}$ for varying sample sizes. Min- and Max Curve lower- and upper bound the worst point-wise deviation from the true curve $T_{SGD}$ over $1000$ simulations.}
    \label{fig:sgd}
\end{figure*}


\noindent {\textbf{Inference}\label{Inference}}
Next, we turn to the second goal of this work: Auditing a $T^{(0)}$-DP claim for a postulated trade-off curve $T^{(0)}$. 
The theoretical foundations of our auditor can be found in Theorem \ref{theo:auditor}. The theorem makes two guarantees: First, that for a mechanism $M$ satisfying $T^{(0)}$-DP the auditor will (correctly) not detect a violation, except with low, user-determined probability $\gamma$. Second, if $M$ violates  $T^{(0)}$-DP, the auditor will (correctly) detect the violation for sufficiently large sample sizes $n_1,n_2$. Together, these results mean that if a violation of $T^{(0)}$-DP is detected by the auditor, the user can have high confidence that $M$ does indeed not satisfy $T^{(0)}$-DP. 
%To begin, we examine the first result, which ensures, informally speaking that the auditor will not generate more than $\gamma>0$ false positives, when auditing a mechanism $M$. The second result guarantees that if the claimed privacy does not hold, the auditor will eventually identify this with probability $1$ as the sample size increases.
For the first part, we consider a scenario, where the claimed trade-off curve $T^{(0)}$ is the correct one $T^{(0)}=T$ ($M$ does not violate $T^{(0)}$-DP). For the second part, we choose a function $T^{(0)}$ above the true curve $T$ ($M$ violates $T^{(0)}$-DP). We will consider both scenarios for the Gaussian mechanism and DP-SGD.
%We will use two of the four mechanism for illustration: First, the standard Gaussian mechanism, as an example of an additive noise mechanism and second the DP-SGD mechanism, as an example of a machine learning mechanism.
%We start with auditing correctly claimed curves $T^{(0)}$. For that purpose, 
We run our auditor (Algorithm \ref{alg:auditor}) with parameters $n_1=10^4$ and $\gamma=0.05$ fixed. The choice of $\gamma=0.05$ is standard for confidence regions in statistics and we further explore the impact of $n_1$ and $\gamma$ in additional experiments in Appendix \ref{AppB}. Here, we focus on the most impactful parameter, the sample size $n_2$ and study values of  $n_2 = 10^6,10^7,10^8$. \\
Technically, the auditor only outputs a binary response that indicates whether a violation is detected or not. However, in our below experiments, we depict the inner workings of the auditor and geometrically illustrate how a decision is reached. More precisely, in Figure \ref{fig:not_faulty_sgd_gauss} we depict the claimed trade-off curve $T^{(0)}$ as a blue line. The auditor makes an estimate for the true trade-of curve $T$, namely $\hat T_h$ depicted as the orange line. The location, where the orange line (estimated DP) and the blue line (claimed DP) are the furthest apart is indicated by the vertical, dashed green line. This position is associated with the threshold $\hat \eta^*$ in Algorithm \ref{alg:pointwise_KDE_estimator}. As a second step, $\hat \eta^*$ is used in the $k$NN method to make a confidence region, depicted as a purple square (this is $\square_\gamma$ from \eqref{e:defsq}). If the square is fully below the claimed curve $T^{(0)}$, a violation is detected (Figure \ref{fig:faulty_sgd_gauss}) and if not, then no violation is detected (Figures \ref{fig:gaussian} and \ref{fig:sgd}). As we can see, detecting violations requires $n_2$ to be large enough, especially when $T^{(0)}$ and $T$ are close to each other. \\
For the incorrect $T^{(0)}$-DP claims, we have done the following: For the Gaussian case (Figure \ref{fig:faulty_sgd_gauss}), we have used a trade-off curve with parameter $\mu=0.5$ instead of the true $\mu=1$. For DP-SGD, we have used the trade-off curve corresponding to $\tau = 5$ instead of the true $\tau =10$ iterations (Figure \ref{fig:faulty_sgd_gauss}). 

%the trade-off curve of DP-SGD with a correctly claimed privacy curve and false claim. The correctly claimed can be found in Figure \ref{fig:not_faulty_sgd_gauss}. For the incorrectly claimed curve, it was stated that DP-SGD ran for only $t_{-}=5$ iterations, accessing the data just five times and thereby reinforcing the privacy guarantee. However, in reality it ran $t_{-}=10$ times, leading to a privacy breach. The results are depicted in Figure \ref{fig:faulty_sgd_gauss}.
%In this algorithm, we Algorithm \ref{alg:KDE_estimator} as a subroutine to derive an estimate $\hat \eta^*$ and we use the sample size $n_1=10^4$. Second, we run the $k$-NN with different sample sizes $n_2=10^6,10^7,10^8$. For the confidence level, we set $\gamma=0.05$, which yields confidence squares induced by $w(\gamma)$ defined in equation \eqref{e:wgamma}. 
%For the first audit in displayed in Figure \ref{fig:not_faulty_sgd_gauss}, we have audited a correctly claimed trade-off curve $T_0$. We detect a faulty mechanism, whenever the purple square $\square_\gamma=\square_{0.05}$ is disjoint from the claimed curve $T_0$. For a faulty implementation, we have use $\mu=0.5$ for the claimed curve $T_0$, while in reality the true curve only fulfills $\mu=1$. The results are displayed in Figure \ref{fig:faulty_sgd_gauss}. To complement these results with the DP-SGD, we also considered DP-SGD with a correctly claimed privacy curve and false claim. The correctly claimed can be found in Figure \ref{fig:not_faulty_sgd_gauss}. For the incorrectly claimed curve, it was stated that DP-SGD ran for only $t_{-}=5$ iterations, accessing the data just five times and thereby reinforcing the privacy guarantee. However, in reality it ran $t_{-}=10$ times, leading to a privacy breach. The results are depicted in Figure \ref{fig:faulty_sgd_gauss}.
\begin{figure*}[h!]
    \centering
    \subfloat[\centering $n_2=10^6$,\textbf{Ground Truth:} No Violation; \newline \textbf{Decision:} \textcolor{green}{"No Violation"}{\textcolor{green}{\scalebox{1.5}{\ding{51}}}}]{\includegraphics[width=0.3\textwidth]{Figures/gauss_100.png}}
    \hfill
    \subfloat[\centering $n_2=10^7$,\textbf{ Ground truth:} No Violation; \newline \textbf{Decision:} \textcolor{green}{"No Violation"}{\textcolor{green}{\scalebox{1.5}{\ding{51}}}}]{\includegraphics[width=0.3\textwidth]{Figures/gauss_100_7.png}}
    \hfill
    \subfloat[\centering $n_2=10^8$, \textbf{ Ground truth:}No Violation; \newline \textbf{Decision:} \textcolor{green}{"No Violation"}{\textcolor{green}{\scalebox{1.5}{\ding{51}}}}]{\includegraphics[width=0.3\textwidth]{Figures/gauss_100_8.png}}
   \vspace{-1em}
    \subfloat[\centering $n_2=10^6$, \textbf{Ground truth:} No Violation; \newline \textbf{Decision:} \textcolor{green}{"No Violation"}{\textcolor{green}{\scalebox{1.5}{\ding{51}}}}]{\includegraphics[width=0.3\textwidth]{Figures/sgd_100.png}}
    \hfill
    \subfloat[\centering $n_2=10^7$, \textbf{Ground truth:} No Violation; \newline \textbf{Decision:} \textcolor{green}{"No Violation"}{\textcolor{green}{\scalebox{1.5}{\ding{51}}}}]{\includegraphics[width=0.3\textwidth]{Figures/sgd_100_7.png}}
    \hfill
    \subfloat[\centering $n_2=10^8$, \textbf{Ground truth:} No Violation; \newline \textbf{Decision:} \textcolor{green}{"No Violation"}{\textcolor{green}{\scalebox{1.5}{\ding{51}}}}]{\includegraphics[width=0.3\textwidth]{Figures/sgd_100_8.png}} \caption{\textbf{Auditing a correct Mechanism:} Claimed curve $\textcolor{blue}{T^{(0)}} = T_{Gauss}$ (a,b,c) and $\textcolor{blue}{T^{(0)}} = T_{SGD}$ (d,e,f). Obtain critical vertical line with step 3 in Algorithm \ref{alg:auditor} with intercept $(\hat\alpha(\hat\eta^*), \hat \beta(\hat \eta^*))$, $k$-NN point estimator \small{\textcolor{purple}{\ding{108}}} $(\tilde\alpha(\hat\eta^*), \tilde \beta(\hat\eta^*))$ and confidence region $\textcolor{purple}{\square}$. The sample size for the KDE is $n_1=10^4$ and the confidence parameter is $\gamma=0.05$.}
    \label{fig:not_faulty_sgd_gauss}
\end{figure*}
\begin{figure*}[h!]
    \centering
    \subfloat[\centering $n_2=10^6$, \textbf{Ground truth:} Violation; \newline \textbf{Decision:} \textcolor{red}{"No Violation"}{\textcolor{red}{\scalebox{1.5}{\ding{55}}}}]{\includegraphics[width=0.3\textwidth]{Figures/gauss_faulty_100.png}}
    \hfill
    \subfloat[\centering $n_2=10^7$, \textbf{Ground truth:} Violation; \newline \textbf{Decision:} \textcolor{green}{"Violation"}{\textcolor{green}{\scalebox{1.5}{\ding{51}}}}]{\includegraphics[width=0.3\textwidth]{Figures/gauss_faulty_100_7.png}}
    \hfill
    \subfloat[\centering $n_2=10^8$, \textbf{Ground truth:} Violation; \newline \textbf{Decision:} \textcolor{green}{"Violation"}{\textcolor{green}{\scalebox{1.5}{\ding{51}}}}]{\includegraphics[width=0.3\textwidth]{Figures/gauss_faulty_100_8.png}}
    \vspace{-1em}
    \subfloat[\centering $n_2=10^6$, \textbf{Ground truth:} Violation; \newline \textbf{Decision:} \textcolor{red}{"No Violation"}{\textcolor{red}{\scalebox{1.5}{\ding{55}}}}]
    {\includegraphics[width=0.3\textwidth]{Figures/sgd_faulty_100.png}}
    \hfill
    \subfloat[\centering $n_2=10^7$, \textbf{Ground truth:} Violation; \newline \textbf{Decision:} \textcolor{red}{"No Violation"}{\textcolor{red}{\scalebox{1.5}{\ding{55}}}}]{\includegraphics[width=0.3\textwidth]{Figures/sgd_faulty_100_7.png}}
    \hfill
    \subfloat[\centering $n_2=10^8$, \textbf{Ground truth:} Violation; \newline \textbf{Decision:} \textcolor{green}{"Violation"}{\textcolor{green}{\scalebox{1.5}{\ding{51}}}}]{\includegraphics[width=0.3\textwidth]{Figures/sgd_faulty_100_8.png}}
     \caption{\textbf{Auditing a faulty Mechanism:} Claimed Curve $\textcolor{blue}{T^{(0)}} = T_{Gauss}$ (a,b,c) with $\mu=0.5$ and $\textcolor{blue}{T^{(0)}} = T_{SGD}$ (d,e,f) with $t_{-}=5$. Both mechanisms assume stronger privacy ($\mu=0.5<1$ and $t_{-}=5<10$). Critical vertical line derived by KDEs using step 3 in Algorithm \ref{alg:auditor} with intercept $(\hat\alpha(\hat\eta^*), \hat \beta(\hat \eta^*))$, $k$-NN point estimator {\textcolor{purple}{\ding{108}}} $(\tilde\alpha(\hat\eta^*), \tilde \beta(\hat\eta^*))$ and confidence region $\textcolor{purple}{\square}$. The sample size for KDE is $n_1=10^4$ and the confidence parameter is $\gamma=0.05$.}
    \label{fig:faulty_sgd_gauss}
\end{figure*}

\noindent\textbf{Implementation Details} The implementation is done using python and R. \footnote{\scriptsize{\url{https://github.com/stoneboat/fdp-estimation}}}. For the simulations, we have used a local device and a server. All runtimes were collected on a local device with an Intel Core i5-1135G7 processor (2.40 GHz), 16 GB of memory, and running Ubuntu 22.04.5, averaged over $10$ simulations. Thus, we demonstrate fast runtimes even on a standard personal computer.
Additionally, we used a server with four AMD EPYC 7763 64-Core (3.5 GHz) processors and 2 TB of memory and running Ubuntu 22.04.4 was used for repetitive simulations. For python, we have used Python 3.10.12 and the libraries "numpy" \cite{2020NumPy-Array}, "scikit-learn" \cite{pedregosa2011scikit} and "scipy" \cite{2020SciPy-NMeth}. For R, we used R version 4.3.1 and the libraries "fdrtool" \cite{fdrtool} and "Kernsmooth" \cite{Kernsmooth}. 
\begin{table}[h!]
\centering
\begin{tabular}{|l|c|}
\hline
\textbf{Algorithm}                           & \textbf{Runtime in seconds} \\ \hline
Gaussian mechanism              &  26.3                                                    \\ \hline
Laplace mechanism            &    30.51                                                     \\ \hline
Subsampling mechanism         &   27.82                                                      \\ \hline
DP-SGD           &              61.1                                         \\ \hline
 
\end{tabular}
\caption{Average runtimes of Algorithm \ref{alg:pointwise_KDE_estimator} for $n_1=10^5$ over $10$ runs to obtain the full trade-off curve $T$.}
\label{tab:running_times_KDE}
\end{table}
\begin{table}[h!]
\centering
\begin{tabular}{|l|c|}
\hline
\textbf{Algorithm}                           & \textbf{Runtime in seconds} \\ \hline
Gaussian mechanism              &    62.63                                                  \\ \hline
Laplace mechanism            &        67.04                                                 \\ \hline
Subsampling mechanism         &     66.98                                                  \\ \hline
DP-SGD           &    114.86                                                 \\ \hline
 
\end{tabular}
\caption{Average runtimes of Algorithm \ref{alg: general BayBox estimator} for $n_2=10^6$ over $5$ runs to obtain one point of the trade-off curve $T$ with confidence region.} %\\[-8ex]} 
\label{tab:running_times_kNN}
\end{table}

\subsection{Interpretation of the results}
Our experiments empirically showcase details of our methods' concrete performance. 
%refine our understanding of certain details of our methods. 
For Goal 1 (estimation), we see in Figure \ref{fig:estimation_mse} the fast decay of the estimation error of $\hat T_h$ for the optimal trade-off curve. The estimation error decays fast in $n_1$, regardless of whether there are plateau values in the sense of Assumption \ref{ass1} (e.g. Laplace Mechanism) or not (e.g. Gaussian Mechanism).
These quantitative results are supplemented by the visualizations in  
Figures~\ref{fig:gaussian}--\ref{fig:sgd}, where we depict the largest distance of $\hat T_h$ and $T$ in $1000$ simulation runs (captured by the red band). Even for the modest sample size of $n_1 = 10^3$, this band is fairly tight and for $n_1 = 10^5$ the estimation error is almost too minute to plot. We find this convergence astonishingly fast. It may be partly explained by the estimator $\hat T_h$ being structurally similar to $T$ -  after all $\hat T_h$ is also designed to be a trade-off curve for an almost optimal LR test.
The approximation over the entire unit interval corresponds to the uniform convergence guarantee in Theorem~\ref{theo:1}. 
%demonstrate that even with relatively small sample sizes, such as $n_1 = 10^3$, the worst global error across 1000 simulations remains notably small. For that observe that the combined worst deviation from the true curve $T_0$ across $1000$ simulations is already for $n_1=10^4$ almost negligible. As the sample size increases even further, the error converges to zero for all $\alpha\in[0,1]$ as visible in Figures~\ref{fig:gaussian}--\ref{fig:sgd} (c). This aligns with the uniform convergence established in Theorem~\ref{theo:1}. In addition to that, throughout all mechanisms, Figure \ref{fig:estimation_mse} illustrates the convergence for the same set of parameters, highlighting the robustness and adaptability necessary in a black-box setting. Here, we also emphasize that the MSEs are computed on an equidistant grid evaluated on the KDEs. This distinction is important because, in principle, for an equidistant $\eta$, the distribution of the $\alpha$ values could theoretically concentrate on a few points. This behavior is especially observable whenever the quotient of the densities is constant for a non-negligible subset of $[0,1]$. An example of that would be the Laplace mechanism.
%To address this potential issue, we have evaluated the error on both the grid implied by the KDE and an equidistant grid. Through this comparison, we have observed that such concentration does not impact the estimation. If this phenomenon arises, a linear interpolation is sufficient, as the trade-off curve is also linear on that subset, and if it does not arise, then the $\alpha$'s are also evenly distributed. \todo{@Tim: do you agree with me? I think it is important to make this remark, as someone could be a bit confused when observing this clustered results for the Laplacian algorithm}
% \\

For Goal 2 (inference), we recall that a  $T^{(0)}$-DP violation is detected if the box $\square_\gamma$ (purple) lies completely below the postulated curve $T^{(0)}$ (blue). In Figure \ref{fig:not_faulty_sgd_gauss} we consider the case of no violation where $T=T^{(0)}$, and we expect not to detect a violation. This is indeed what happens, since $\square_\gamma$ intersects with the curve $T^{(0)}$ in all considered cases. Interestingly, we observe that $\square_\gamma$ has a center close to $\alpha=0$ in the cases where no violation occurs (such a behavior might give additional visual evidence to users that no violation occurs).
%In principal, one would reject the privacy curve, whenever the purple square $\textcolor{purple}{\square}$ is disjoint from the blue \textcolor{blue}{curve}, i.e.
%\begin{equation*}
%    \textcolor{purple}{\square} \cap \textcolor{blue}{\textnormal{curve}}=\emptyset~.
%\end{equation*}
%In Figure \ref{fig:not_faulty_sgd_gauss} and \ref{fig:not_faulty_sgd_gauss} we have displayed the case where the claimed privacy indeed holds, so we expect to not detect a violation. For both mechanism, we can observe that for any sample size, we correctly do not reject that claim, as the $\textcolor{purple}{\square}$ and the $\textcolor{blue}{curve}$ are not disjoint. 
In Figure \ref{fig:faulty_sgd_gauss}, we display the case of faulty claims, where the privacy breach is caused by a smaller variance for both mechanisms under investigation. In accordance with Theorem \ref{theo:auditor}, we expect a detection of a violation if $n_2$ is large enough. This is indeed what happens, at a sample size of $n_2=10^7$ for the Gaussian mechanism and at  $n_2=10^8$ for DP-SGD. As we can see, larger samples $n_2$ are needed to expose claims $T^{(0)}$ that are closer to the truth $T$ (as for DP-SGD in our example). For larger $n_2$ the square $\square_\gamma$ shrinks (see eq. \eqref{e:defsq}) leading to a higher resolution of the auditor. 
%While for both mechanisms, we do not detect a statistical significant deviation for $n_2=10^6$ (since the $\textcolor{purple}{\square}$ is not disjoint from the blue \textcolor{blue}{curve}), already for $n_2=10^7$, the auditor detects the privacy violation for the Gaussian mechanism. Regarding the DP-SGD case, we have to increase the sample size to $n_2=10^8$ to detect the violation. The clear message here is that smaller privacy violations (curves are closer to each other), the larger the sample size $n_2$ has to be, to obtain small confidence regions. This is a classical pattern in statistics and aligns with Theorem \ref{theo:auditor} part (2). The main reason here is that higher $n_2$ significantly shrinks the confidence square (recall Theorem \label{thm: accuracy stat of kNN BayBox estimator}
%), which eventually will be fully below the \textcolor{blue}{curve}, indicating a statistically significant difference. Here, we explicitly want to point out that even though e.g. Figure \ref{fig:faulty_sgd_gauss} (a) was not significant, one can already take the deviation from the claimed curve as a first indication and just increase the sample size for a stronger evidence. Consequently, we strongly encourage users to also consider the point estimator derived from the $k$-NN algorithm as a potential indicator of faulty implementations. In fact, our empirical evidence indicates that the confidence interval for our $k$-NN based point estimator is significantly narrower than the theoretical bound we derived. This discrepancy arises because the theoretical convergence rate of the $k$-NN algorithm is generally not tight; in practice, the performance of the $k$-NN algorithm converges more rapidly than the theoretical rate suggests. To put this into more practical observations, the difference of the estimators derived for $n_2=10^6,10^7,10^8$ were for all mechanism negligible.

\section{Rethinking Sparse Attention Methods}
\label{sec:critique}

Modern sparse attention methods have made significant strides in reducing the theoretical computational complexity of transformer models. However, most approaches predominantly apply sparsity during inference while retaining a pretrained Full Attention backbone, potentially introducing architectural bias that limits their ability to fully exploit sparse attention's advantages. Before introducing our native sparse architecture, we systematically analyze these limitations through two critical lenses.


\begin{figure*}[t] 
\centering 
\includegraphics[width=1\textwidth]{figures/fig2.pdf} 
\caption{Overview of \method{}'s architecture. Left: The framework processes input sequences through three parallel attention branches: For a given query, preceding keys and values are processed into compressed attention for coarse-grained patterns, selected attention for important token blocks, and sliding attention for local context. Right: Visualization of different attention patterns produced by each branch. Green areas indicate regions where attention scores need to be computed, while white areas represent regions that can be skipped.}
\label{fig:framework}
\end{figure*}


\subsection{The Illusion of Efficient Inference}

Despite achieving sparsity in attention computation, many methods fail to achieve corresponding reductions in inference latency, primarily due to two challenges:

\textbf{Phase-Restricted Sparsity.}
Methods such as H2O \citep{h2o} apply sparsity during autoregressive decoding while requiring computationally intensive pre-processing (e.g. attention map calculation, index building) during prefilling. In contrast, approaches like MInference \citep{minference} focus solely on prefilling sparsity. 
These methods fail to achieve acceleration across all inference stages, as at least one phase remains computational costs comparable to Full Attention.
The phase specialization reduces the speedup ability of these methods in prefilling-dominated workloads like book summarization and code completion, or decoding-dominated workloads like long chain-of-thought~\citep{cot} reasoning.

\textbf{Incompatibility with Advanced Attention Architecture.}
Some sparse attention methods fail to adapt to modern decoding efficient architectures like Mulitiple-Query Attention~(MQA) \citep{mqa} and Grouped-Query Attention~(GQA) \citep{gqa}, which significantly reduced the memory access bottleneck during decoding by sharing KV across multiple query heads. For instance, in approaches like Quest \citep{quest}, each attention head independently selects its KV-cache subset. Although it demonstrates consistent computation sparsity and memory access sparsity in Multi-Head Attention (MHA) models, it presents a different scenario in models based on architectures like GQA, where the memory access volume of KV-cache corresponds to the union of selections from all query heads within the same GQA group. This architectural characteristic means that while these methods can reduce computation operations, the required KV-cache memory access remains relatively high.
This limitation forces a critical choice: while some sparse attention methods reduce computation, their scattered memory access pattern conflicts with efficient memory access design from advanced architectures.

These limitations arise because many existing sparse attention methods focus on KV-cache reduction or theoretical computation reduction, but struggle to achieve significant latency reduction in advanced frameworks or backends.
This motivates us to develop algorithms that combine both advanced architectural and hardware-efficient implementation to fully leverage sparsity for improving model efficiency.


\subsection{The Myth of Trainable Sparsity}
Our pursuit of native trainable sparse attention is motivated by two key insights from analyzing inference-only approaches:
(1) \textbf{\textit{Performance Degradation}}: Applying sparsity post-hoc forces models to deviate from their pretrained optimization trajectory. As demonstrated by \citet{magicpig}, top 20\% attention can only cover 70\% of the total attention scores, rendering structures like retrieval heads in pretrained models vulnerable to pruning during inference.
(2)~\textbf{\textit{Training Efficiency Demands}}: 
Efficient handling of  long-sequence training is crucial for modern LLM development. This includes both pretraining on longer documents to enhance model capacity, and subsequent adaptation phases such as long-context fine-tuning and reinforcement learning. However, existing sparse attention methods primarily target inference, leaving the computational challenges in training largely unaddressed. This limitation hinders the development of more capable long-context models through efficient training. Additionally, efforts to adapt existing sparse attention for training also expose challenges:



\textbf{Non-Trainable Components.} Discrete operations in methods like ClusterKV~\citep{clusterkv} 
(includes k-means clustering) and MagicPIG~\citep{magicpig} (includes SimHash-based selecting) create discontinuities in the computational graph. These non-trainable components prevent gradient flow through the token selection process, limiting the model's ability to learn optimal sparse patterns. 

\textbf{Inefficient Back-propagation.} Some theoretically trainable sparse attention methods suffer from practical training inefficiencies. Token-granular selection strategy used in approaches like HashAttention~\citep{desai2024hashattention} leads to the need to load a large number of individual tokens from the KV cache during attention computation. 
This non-contiguous memory access prevents efficient adaptation of fast attention techniques like FlashAttention, which rely on contiguous memory access and blockwise computation to achieve high throughput.
As a result, implementations are forced to fall back to low hardware utilization, significantly degrading training efficiency.



\subsection{Native Sparsity as an Imperative}

These limitations in inference efficiency and training viability motivate our fundamental redesign of sparse attention mechanisms.
We propose \method{}, a natively sparse attention framework that addresses both computational efficiency and training requirements.
In the following sections, we detail the algorithmic design and operator implementation of \method{}.



\section{Conclusion}\label{sec:summary_discussion}

In our work we construct the first general-purpose $f$-DP estimator and auditor in a black-box setting, by extending techniques from statistics and classification theory. Our constructions enjoy not only formal guarantees---convergence of our estimator and a tuneable confidence region for our auditor---but also an impressive concrete performance. We demonstrate  our methods on well-used mechanisms such as subsampling and DP-SGD, showing their accuracy and efficiency on both a server and personal computer setup.

%on new techniques that extend beyond the current literature on DP estimation/auditing. 

%As a more in-depth discussion, we would like to note the distinction of contribution (1) and (2). Our estimator estimates the optimal trade-off curve based on outputs of a given mechanism $M$. No prior belief about the trade-off curve is required and the user obtains a \textit{close approximation of the entire curve}. In auditing, the starting point is different. Here the mechanism is associated with a privacy claim in the form of a hypothesized tradeoff curve $f_0$. This situation is typical in scenarios, where a privatizing mechanism is provided by a third party (that claims to provide at least $f_0$-DP). The auditor allows a user to check if the mechanism is $f_0$-DP. Thus, instead of considering the entire curve, the auditor \textit{identifies the point of highest vulnerability and checks in this point whether the privacy claim $f_0$ is breached.} The auditor cannot do the job of the estimator because it is pointwise. The estimator cannot be used for auditing, because it does not deliver confidence regions at any point. Both tasks occur in practice which is why we address both of them. We also notice that our solutions are interlocking, since (1) can be used as a subroutine for (2) (for pre-selection of vulnerable points). 
%\todo{TODO: Somewhere, should discuss how to tune the confidence region/bound of auditor}
\section{Acknowledgments}
\noindent This work was funded by the Deutsche Forschungsgemein-
schaft (DFG, German Research Foundation) under Germany’s
Excellence Strategy - EXC 2092 CASA - 390781972. Tim Kutta was partially funded by the AUFF Nova Grant 47222. Yun Lu is supported by NSERC (RGPIN-03642-2023). Vassilis Zikas and Yu Wei are supported in part by Sunday Group, Incorporated. The work of Holger Dette has been partially supported by the DFG Research unit 5381 {\it Mathematical Statistics in the Information Age}, project number 460867398. 

% \section*{Ethics Considerations}

% We attest that the algorithms our privacy auditor/estimator identify as having privacy vulnerabilities are already recognized for (or specifically created with) these issues. Thus, this work does not expose new vulnerabilities. Additionally, we do not make use of private datasets in our experiments. Thus, our experiment results do not introduce privacy leaks. We attest that our auditor/estimator do not have any biases stemming from a conflict of interest. While our auditor can be used on other existing mechanisms, our implementation mitigates risks by ensuring our output is limited to alerts to potential privacy violation, and does not leak any additional information about the dataset(s) tested.

% \section*{Open Science}

% To comply with the Open Science Policy, we will make all artifacts publicly available. In our experiments section, we  ensure the transparency and reproducibility of our methodology by describing the dataset used, the specification of our machines, and citing all algorithms tested.

 
{\footnotesize \bibliographystyle{acm}
\bibliography{main}}
\newpage 

$ $\newpage
\appendix

\section{Appendix}
The appendix is dedicated to proofs and technical details of our results. Throughout our proofs we will use the notation $R= o_P(1)$ for a remainder $R$ that satisfies $R \overset{P}{\to}0$ (convergence in probability).

\begin{table}[ht]
\centering
\caption{Overview of Notation Used in the Paper}
\label{tab:notation}
\begin{tabular}{ll}
\toprule
\textbf{Notation} & \textbf{Description} \\
\midrule
\( D,D'\) & Pair of adjacent databases \\
\( M \) & ($f$-)DP Mechanism \\
\( \Pr{}[ ], \Ex{ } \) & Probability, Expectation\\
\( P,Q \) & Output distributions of $M(D), M(D')$ \\
\(\MixtureD{P}{\eta}\) & Mixture distribution with parameter $\eta$\\
\( p,q\) & Probability densities of $P,Q$  \\
\( \alpha, \beta \) & type-I \& type-II errors \\
& (typically of the Neyman-Pearson test) \\
\(\hat \alpha_h, \hat \beta_h\) & Estimated errors using KDE \\
\(\tilde \alpha, \tilde \beta\) & Estimated errors using $k$-NN \\
& (typically of the Neyman-Pearson test) \\
\( T\) & optimal trade-off curve for $p,q$\\
\( T^{(0)}\) & trade-off curve that is audited\\
\( T_h\) & trade-off curve of perturbed LR test \\
\( \hat T_h\) & estimated trade-off curve using KDE \\
\(  \eta\) & threshold in LR tests \\
& vulnerability \\
\( \hat \eta^*\) & estimated threshold of maximum \\
& vulnerability \\
\(  \lambda\) & randomization parameter in \\
 & Neyman-Pearson test \\
 \(  h\) & randomization parameter in \\
 & perturbed LR test \\
\( \phi, \kNNclassifier{n} \) & generic classifier, $k$-NN classifier \\
\( \phi^* \) & Bayes optimal classifier \\
\( \gamma, w(\gamma) \) & confidence level \& 
margin of error\\
$\square_\gamma$ & confidence region for \\
 & type-I-type-II errors\\
 $n, n_1, n_2$ & sample size parameters \\
\bottomrule
\end{tabular}
\end{table}
\subsection{Proofs for Goal 1 (Estimation)}

\textbf{Consequences of Theorem \ref{theo:1}} The main result in Section \ref{sec:4} is Theorem \ref{theo:1}. Lemma \ref{lem1} can be seen as a special case, putting $\hat p=p, \hat q=q$ . Then, Assumption \ref{ass2} is met for the constant sequence $a_n=0$. It follows by this construction that $\hat T_h =T_h$, is non-random and only depends on $h$. Any choice of $h\downarrow 0$ is permissible and  Lemma \ref{lem1} follows from the Theorem. Proposition \ref{prop1} too is a direct consequence of Theorem \ref{theo:1}. To see this, we notice that
\begin{align*}
    & T_0(\hat \alpha_h(\hat \eta^*)) - T(\hat \alpha_h(\hat \eta^*)) \\
    =& T_0(\hat \alpha_h(\hat \eta^*)) - \hat T_h(\hat \alpha_h(\hat \eta^*))+o_P(1)\\
    = & \sup_{\alpha \in [0,1]}\{T_0(\alpha) - \hat T_h(\alpha)\}+o_P(1)\\
    =& \sup_{\alpha \in [0,1]}\{T_0(\alpha) -  T(\alpha)\}+o_P(1).
\end{align*}
In the first and last step, we have used the uniform convergence of Theorem \ref{theo:1}, which allows us to replace $T$ by $\hat T_h$ while only incurring an $o_P(1)$ error. In the second step, we have used the definition of $\hat \alpha_h(\hat \eta^*)$ as the maximizer of the difference between $T_0$ and $\hat T_h$. Thus Proposition \ref{prop1} follows. We now turn to the proof of the theorem. The proof is presented for densities on the real line. Extensions to $\mathbb{R}^d$ are straightforward and therefore not discussed. \\
\textbf{Preliminaries} Recall that a complete separable metric space is Polish. The real numbers, equipped with the absolute value distance is a Polish space. The continuous functions $\mathcal{C}_0$ on the real line that vanish at $\pm \infty$, i.e. that satisfy
\begin{align} \label{e:decay}
\lim_{x \to \infty} f(x) = \lim_{x \to \infty} f(-x)=0 
\end{align}
is a Polish space if equipped with the supremum norm
\[
\|f\|:= \sup_{x \in \mathbb{R}}|f(x)|.
\] 
The product of complete, separable metric spaces is complete and separable if equipped with the maximum metric, i.e. the space $\mathcal{C}_0 \times \mathcal{C}_0\times \mathbb{R}\times \mathbb{R}$ is Polish. 
Now, the vector 
\[
(\hat p, \hat q, \|\hat p-p\|_\infty/a_n, \|\hat q-q\|_\infty/a_n)
\]
lives on this space (for each $n$) and convergences to the limit $(p,q,0,0)$ in probability (see Assumption \ref{ass2}). Accordingly we can use Skorohod's theorem to find a probability space, where this convergence is a.s. 
\[
(\hat p, \hat q, \|\hat p-p\|_\infty/a_n, \|\hat q-q\|_\infty/a_n) \to (p,q,0,0) \quad a.s.
\]
It is a direct consequence that on this space it holds a.s.
\[
\|\hat p-p\| =o(a_n), \quad \|\hat q-q\| =o(a_n).
\]
In the following, we will work on this modified probability space and exploit the a.s. convergence. We will fix the outcome and regard $\hat p, \hat q$ as sequences of deterministic functions, converging to their respective limits at a rate $o(a_n)$.\\
    Next, it suffices to show the desired result pointwise for any $\alpha$. This reduction is well-known. For a sequence of  continuous, monotonically decreasing functions $(f_n)_n$ living on the unit interval $[0,1]$, pointwise convergence  to a continuous, monotonically decreasing limit $f$ on $[0,1]$ implies uniform convergence. The same argument lies at the heart of the proof of the famous Glivenko-Cantelli Theorem (see \cite{vaart:wellner:1996}). We now want to demonstrate the convergence $|\hat T(\alpha) -T(\alpha)|=o(1)$ pointwise. More precisely, we will demonstrate that for the pair $(\alpha, T(\alpha))$, there exist values of $\eta$ such that $\hat \alpha_h(\eta) \to \alpha$ and $\hat\beta_h(\eta) \to T(\alpha)$. Since the proofs of both convergence results work exactly in the same way, we restrict ourselves in this proof to show that $\hat \alpha_h(\eta) \to \alpha$. So let us consider a fixed but arbitrary value of $\alpha \in [0,1]$ and begin the proof.\\
    \textbf{Case 1:} We first consider the case where $\eta\ge 0$ (the threshold in the optimal LR test) is such that the set $\{q/p=\eta\}$ has $0$ mass. In this case, the coin toss with probability $\lambda$ can be ignored (it happens with probability $0$) and  we can define the type-I-error  $\alpha$ of the Neyman-Pearson test as 
    \[
\alpha = \int p \cdot \mathbb{I}\{q/p  > \eta\}.
\]
In this case, we want to show that 
\begin{align*}
& \int_{x \in [-h/2,h/2]} \frac{1}{h}\int_{\hat q /\hat p  > \eta +x} \hat p \\
=& \int  \int_{x \in [-h/2,h/2]}  \hat p \frac{1}{h}\,\,\mathbb{I}\{ \hat q /\hat p  > \eta +x\}=:\int \hat g \\
\to &  \int_{q/p  > \eta}  p  = \int p \cdot \mathbb{I}\{q/p  > \eta\} =:\int g .
\end{align*}
Here we have defined the functions $ g, \hat g$ in the obvious way. We will now show $\hat g$ converges pointwise to $g$. For this purpose consider the interval $[-K,K]$ for a large enough $K$, such that
\[
\int_{[-K,K]^c} p<\zeta \qquad \textnormal{and} \qquad\int_{[-K,K]^c} q<\zeta
\]
for a number $\zeta$ that we can make arbitrarily small. Given the uniform convergence of the density estimators on the interval $[-K,K]$ it holds for all $n$ sufficiently large that also
\[
\int_{[-K,K]^c} \hat p<\zeta \qquad \textnormal{and} \qquad\int_{[-K,K]^c} \hat q<\zeta.
\]
Accordingly we have 
 \[
 \bigg| \int \hat g - g\bigg| \le 2 \zeta + \bigg| \int_{[-K,K]} \hat g -g\bigg|.
 \]
 We then focus on the second term on the right and fix some argument $y \in [-K,K]$. It holds that either $ q(y)/ p(y)$ is bigger or smaller than $\eta$ (equality occurs only on a null-set and can therefore be neglected). Let us focus on the case where $q(y)/ p(y)> \eta$. If this is so, then it follows that in a small environment, say for $y' \in [y-\zeta', y+\zeta']$ we also have $q(y')/ p(y')>\eta$. For all large enough $n$ it follows that $h/2<\zeta'$. Then, it is easy to see that also $\hat q(y')/\hat p(y') >\eta$ for all $y' \in [y-\zeta', y+\zeta']$ simultaneously, for all sufficiently large $n$. If this is the case, the indicators in the definition of $\hat g, g$ become $1$ and $\hat g=\hat p$, $g=p$. 
 So, we have pointwise $\hat g(y)=\hat p(y)  \to p(y) =g(y)$. Since $\hat g$ is also bounded for all sufficiently large $n$ (since the integral over the indicator is bounded and the sequence $\hat p$ is uniformly convergent to the bounded function $p$) we obtain by the theorem of dominated convergence that 
 \[
 \bigg| \int_{[-K,K]} \hat g -g\bigg|\to 0.
 \]
 This shows that 
 \[
 \limsup_n|\hat \alpha_h(\eta) - \alpha|=\mathcal{O}(\zeta).
 \]
 Finally, letting $\zeta \downarrow 0$ in a second limit shows the desired approximation in this case.\\
 \textbf{Case 2:} Next, we consider the case where the set $\{q/p=\eta\}$ has positive mass for some $\eta>0$.\footnote{We omit the simpler case where $\eta=0$ and $L=0$ anyways.}
 This means that the coin-flip in the definition of the optimal LR test plays a role and we set the probability $\lambda $ to some value in $[0,1]$.
  We then consider as estimator the value $\hat \alpha(\eta-b h)$ for a value $b$ that we will determine below. Let us, for ease of notation, define the probability 
 \[
 L := \int_{q/p=\eta} p
 \]
and appreciate that then
\begin{align} \label{e:dec}
\alpha = \alpha'+\mathcal{O}(\zeta) + \lambda L.
\end{align}
We explain the decomposition: In equation \eqref{e:dec}, $\alpha'$ is the rejection probability of the LR test defined by the decision to reject whenever $q(y)/p(y)>\eta+\zeta''$ for some small number $\zeta''$. For all small enough values of $\zeta''$ the threshold $\eta+\zeta''$ is not a plateau value (there are only finitely many of them; see Assumption \ref{ass1}). It follows that 
\[
\alpha' = \int p \cdot \mathbb{I}\{q/p  > \eta+\zeta''\}.
\]
Next, for any fixed constant $\zeta>0$ we can choose $\zeta''$ small enough such that
\begin{align} \label{e:int1}
\int p \cdot \mathbb{I}\{\eta <q/p  \le \eta+\zeta''\} < \zeta.
\end{align}
This explains the second term on the right of equation \eqref{e:dec}. The third term corresponds to the probability of rejecting whenever $q/p=\eta$ (this probability is $L$) times the probability that the coin shows heads (reject) with probability $\lambda$.\\
Now, using these definitions, we decompose the set
\begin{align*}
&\{ \hat q /\hat p  > \eta-b h +x\}\\
=& \{ \eta +\zeta'' \ge \hat q /\hat p  > \eta-b h +x\} \cup  \{  \hat q /\hat p  > \eta +\zeta''\}. 
\end{align*}
This yields the decomposition
\begin{align} \label{e:alcon}
& \hat \alpha_h(\eta-b h)
= \hat \alpha_h(\eta+\zeta'')\\
&+ \int  \int_{x \in [-h/2,h/2]}  \hat p \,\, \frac{1}{h}\,\,\mathbb{I}\{ \eta +\zeta'' \ge \hat q /\hat p  > \eta-b h +x\} .\nonumber 
\end{align}
Now, by part 1 of this proof we have  
\[
|\hat \alpha_h(\eta+\zeta'') - \alpha'|=o(1).
\]
 Next, we study the integral on the right side of eq. \eqref{e:alcon} and for this purpose define the objects
\begin{align*}
    \tilde g := & \int_{x \in [-h/2,h/2]}  \hat p \,\,\frac{1}{h}\,\,\mathbb{I}\{ A_1\}, \\
    \tilde f := & \int_{x \in [-h/2,h/2]}  \hat p \,\,\frac{1}{h}\,\,\mathbb{I}\{ A_2\}.\\
    A_1:= &\{\eta +\zeta'' \ge \hat q /\hat p  > \eta-b h +x,q/p=\eta\}, \\
    A_2:=& \{\eta +\zeta'' \ge \hat q /\hat p  > \eta-b h +x,q/p \neq \eta\}.
\end{align*}
Now, let us consider a value $y$ where $q(y)/p(y) \neq \eta$ and for sake of argument let us focus on the (more difficult) case $q(y)/p(y) >\eta$. If $q(y)/p(y) > \eta+\zeta''$, it follows that eventually $\hat p(y)/\hat q(y) > \eta+\zeta''$ and hence $\tilde f(y)=0$. The case where $q(y)/p(y) = \eta+\zeta''$ is a null-set and hence negligible (it is not a plateau value). The case where  $q(y)/p(y) \in (\eta, \eta+\zeta'')$ implies that eventually $\hat p(y)/\hat q(y) \in (\eta, \eta+\eta'')$ and thus eventually $\tilde f(y) = \hat p(y)$ which converges pointwise to $p$. Thus, we have by dominated convergence that 
\[
\int \tilde f \to \int p \cdot \mathbb{I}\{\eta <q/p  \le \eta+\zeta''\} <\zeta.
\]
The fact that the integral is bounded by $\zeta$ was established in eq. \eqref{e:int1}. This means that for all $n$ large enough we have 
\[
\int \tilde f < \zeta.
\]
Now, let us focus on a value of $y$ where $q(y)/p(y)=\eta$. In this case it follows that $q(y), p(y)>0$ and we have 
\[
\frac{\hat q(y)}{\hat p(y)} = \frac{q(y)}{p(y)} +o(a_n) = \eta +o(a_n).
\]
Notice that we can rewrite $\tilde g$ as
\begin{align*}
    \int_{x \in [-1/2,1/2]}  \hat p \,\,\mathbb{I}\{\eta +\zeta'' \ge \hat q /\hat p  > \eta-b h +hx,q/p=\eta\}.
\end{align*}
Now, for any $x>b$ it follows that the indicator will eventually be $0$, because 
\[
\hat q /\hat p =\eta+o(a_n) << \eta + h(x-b)
\]
(because $a_n=o(h)$ by assumption in the Theorem). By similar reasoning the indicator is $1$ if $x<b$. This means that $\tilde g$ converges for any fixed $y$ with $q(y)/p(y)=\eta$ to $p(y) \cdot (1/2+b)$ and using majorized convergence yields
\[
\int \tilde g \to (1/2+b) \int_{q/p=\eta} p =  (1/2+b)L.
\]
Now, we can choose  $b=\lambda-1/2$ to get that the right side is equal to $\lambda L$. Putting these considerations together, we have shown that
\[
\limsup_n |\alpha -\hat \alpha_h(\eta-[\lambda-1/2] h)| = \mathcal{O}(\zeta). 
\]
Taking the limit $\zeta \downarrow 0$ afterwards yields the desired result. 
\subsection{Proofs for Goal 2 (Auditing)}
Before we proceed to the proofs, we state a simple but useful consequence of the Neyman-Pearson Lemma.
\begin{cor}
   \label{corollary: NP lemma}
    Let set $\cS_{\eta} = \{x: p(x)/q(x) \leq \eta \}$.
    For $\alpha \in [0,1]$, if there exists $\eta$ such that $\prdis{\rX \sim P}{\rX \in \cS_{\eta}} = \alpha$, then it holds that
    \begin{align*}
    \beta(\alpha) = 1 - \prdis{\rX \sim Q}{\rX \in \cS_{\eta}}.
    \end{align*}
% where $\cS_{\eta}$ is chosen such that $\prdis{\rX \sim P}{\rX \in \cS_{\eta}} = \alpha$.
\end{cor}

\begin{proof}[\textbf{Proof of Lemma~\ref{lemma: accuracy stat of general BayBox estimator}}]
    \label{proof: for lemma: accuracy stat of general BayBox estimator}
    We prove the statement that $\abs{\tilde{\alpha}(\eta) - \alpha(\eta)} \leq \sqrt{\frac{1}{2n}\ln{\frac{2}{\gamma}}}$ if $\eta \geq 1$. The proof of the second statement follows a similar approach. We begin with a few definitions.
    Let the observation set be defined as  
    \begin{align*}
        \cO := \Supp{P} \cup \Supp{Q} \cup \{\bot\},
    \end{align*}
    i.e. the range of observation. Define the indicator function $\mathbb{I}_{\cS_{\eta}} : \cO \mapsto \bits$, which takes as input an observation $x$ from the observation set $\cO$, outputting 1 if $x \in \cS_\eta$ and $0$ otherwise. Also, recall the definition of the set $\cS_{\eta} = \{x: p(x)/q(x) \leq \eta \}$ 
    % \todo{is it maybe $<$ instead of $\leq$? maybe doesn't matter} 
    as the set of all observation $x \in \cO$ where $p(x)$ is less than or equal to $\eta q(x)$ (as before $p, q$ are the densities of distributions $P, Q$).
    
    We first show that $\mathbb{I}_{\cS_{\eta}}$ is exactly the Bayes classifier $\phi^{*}$ for the Bayesian binary classification problem $\bbcP{\MixtureD{P}{\eta}}{Q}$. We prove this by showing for every $x \in \cO$, $\phi^{*}(x) = \mathbb{I}_{\cS_{\eta}}(x)$. Therefore, consider the tuple of random variable $(\rX, \rY) \sim \bbcP{\MixtureD{P}{\eta}}{Q}$. Then, for every observation $x \in \cO \setminus \{\bot\}$, we have  
    \begin{align*}
       \phi^{*}(x) ={}& \argmax_{\bits} \{\pr{\rY = 0 | \rX = x}, \pr{\rY = 1 | \rX = x}\} \tag{by Bayes classifier $\phi^{*}$'s construction}\\
        ={}& \argmax_{\bits} \{\pr{\rY = 0, \rX = x}, \pr{\rY = 1, \rX = x}\} \tag{by Bayes Theorem}\\
        ={}& \argmax_{\bits} \{\frac{1}{\eta}p(x), q(x)\} \\
        ={}& \mathbb{I}_{\cS_{\eta}}(x) \tag{by $\mathbb{I}_{\cS_{\eta}}$'s definition}.
    \end{align*}
    For an observation $x = \bot$, it is easy to check $\phi^{*}(x) = \mathbb{I}_{\cS_{\eta}}(x) = 0,$ as $q(x) = 0.$

    Then, we also observe that  
    \begin{align*}
        \alpha(\eta) ={}& \prdis{\rX \sim P}{\rX \in \cS_{\eta}} \tag{By Corollary~\ref{corollary: NP lemma}}\\
        ={}& \prdis{\rX \sim P}{\mathbb{I}_{\cS_{\eta}}(\rX) = 1} \\
        ={}& \prdis{\rX \sim P}{\phi^*(\rX) = 1} \tag{$\phi^* = \mathbb{I}_{\cS_{\eta}}$}\\
        ={}& \Exf{\rX \sim P}{\phi^*(\rX)}
    \end{align*}

    Recall that in algorithm~\ref{alg: general BayBox estimator}, BayBox estimatior $\bbe{\phi^*}$ computes the empirical mean of $\phi^*(\rX)$, i.e., $\tilde{\alpha}(\eta)$, as the estimate of $\alpha(\eta)$. By Hoeffding's Inequality, we finally conclude that
    \begin{align*}
           &\pr{\abs{\tilde{\alpha}(\eta) - \alpha(\eta)} > \sqrt{\frac{1}{2n}\ln{\frac{2}{\gamma}}}} \\
        ={}&\pr{\abs{\frac{1}{n}\rSum{i}{1}{n}\rZ_i - \Ex{\frac{1}{n}\rSum{i}{1}{n}\rZ_i}} > \sqrt{\frac{1}{2n}\ln{\frac{2}{\gamma}}}} \tag{$\rZ_i \defin \phi^*(\rX_i), \rX_i \overset{\text{i.i.d.}}{\sim} P$}\\
        \leq{}& \gamma.
    \end{align*}
\end{proof}

\begin{proof}[\textbf{Proof of Theorem~\ref{thm: accuracy stat of kNN BayBox estimator}}]
    \label{proof: for thm: accuracy stat of kNN BayBox estimator}
    We prove the first statement $\abs{\tilde{\alpha}(\eta) - \alpha(\eta)} \leq \sqrt{\frac{1}{2n}\ln{\frac{4}{\gamma}}} + \sqrt{\frac{144c_d^2}{n}\ln{\frac{4}{\gamma}}}$, and the second statement follows by a similar approach. 
    
    With probability at least $1 - \gamma$, we have 
    \begin{align*}
           & \abs{\tilde{\alpha}(\eta) - \alpha(\eta)}\\
        ={}& \abs{\frac{1}{n}\rSum{i}{1}{n}\kNNclassifier{n}(\rX_i) - \Ex{\frac{1}{n}\rSum{i}{1}{n}\phi^*(\rX_i)}} \tag{$\rX_i \overset{\text{i.i.d.}}{\sim} P$}\\
        ={}& \abs{\frac{1}{n}\rSum{i}{1}{n}\kNNclassifier{n}(\rX_i) - \Ex{\phi^*(\rX)}} \tag{$\rX \sim P$}\\
        \leq{}& \abs{\frac{1}{n}\rSum{i}{1}{n}\kNNclassifier{n}(\rX_i) - \Ex{\kNNclassifier{n}(\rX)}} + \abs{\Ex{\kNNclassifier{n}(\rX)} - \Ex{\phi^*(\rX)}} \\
        \leq{}& \sqrt{\frac{1}{2n}\ln{\frac{4}{\gamma}}} + \abs{\Ex{\kNNclassifier{n}(\rX)} - \Ex{\phi^*(\rX)}} \tag{by Hoeffding's Inequality}\\
        ={}& \sqrt{\frac{1}{2n}\ln{\frac{4}{\gamma}}} + \abs{ \pr{\kNNclassifier{n}(\rX) = 1} - \pr{\phi^*(\rX) = 1} }\\
        ={}& \sqrt{\frac{1}{2n}\ln{\frac{4}{\gamma}}} + \abs{ \pr{\kNNclassifier{n}(\rX) \neq 0} - \pr{\phi^*(\rX) \neq 0} }\\
        \leq{}& \sqrt{\frac{1}{2n}\ln{\frac{4}{\gamma}}} + 2|R(\kNNclassifier{n}) - R(\phi^{*})|\\
        \leq{}& \sqrt{\frac{1}{2n}\ln{\frac{4}{\gamma}}} + 12\sqrt{\frac{2c_d^2}{n}\ln{\frac{4}{\gamma}}}\tag{by Theorem~\ref{thm:covergence of kNN}}.
    \end{align*}

    We note that the first equality follows the idea in the proof of Lemma~\ref{lemma: accuracy stat of general BayBox estimator}, by just replacing the Bayes classifier with the concrete $k$-NN classifier.
\end{proof}

\begin{proof}[\textbf{Proof of Theorem~\ref{theo:auditor}}] Recall the notation of Section \ref{sec:hyp}. In this proof, we will additionally assume that $T^{(0)}$ is strictly decaying, to make the presentation of our arguments slightly more easy to understand.\\
Now, consider  $\hat \eta^* \ge 0$ and the corresponding pair $(\alpha(\hat \eta^* ), \beta(\hat \eta^*))$ on the optimal trade-off curve.\footnote{Formally, we condition on $\hat \eta^* $, which is generated by the first part of the algorithm using KDEs. Since the coins from the KDE and the $k$-NN part of the algorithm are independent, we can simply treat  $\hat \eta^* $ as fixed.} According to Theorem \ref{thm: accuracy stat of kNN BayBox estimator} the probability that 
\begin{align} \label{e:gcon}
|\alpha(\hat \eta^* ) - \tilde \alpha(\hat \eta^* )|, |\beta(\hat \eta^* )-\tilde \beta(\hat \eta^* )| \le w(\gamma) 
\end{align}
is eventually (as $n_2\to \infty$) $\ge 1- \gamma$. Let us now condition on the event where \eqref{e:gcon} holds. The algorithm detects a violation, if
\[
i^* > \tilde{\alpha}(\hat{\eta}^*) + w(\gamma),
\]
where $i^*$ solves the equation $T^{(0)}(i^*) = \tilde{\beta}(\hat{\eta}^*) + w(\gamma)$.
We apply $T^{(0)}$ on both sides, which gives us the detection condition
\begin{align} \label{e:condition}
\tilde{\beta}(\hat{\eta}^*) + w(\gamma) < T^{(0)}(\tilde{\alpha}(\hat{\eta}^*) + w(\gamma)).
\end{align}
On the event characterized by \eqref{e:gcon} we have
\[
\tilde{\beta}(\hat{\eta}^*) + w(\gamma) \ge \beta(\hat{\eta}^*) 
\]
and, using the strict monotonicity of the trade-off curve $T^{(0)}$ 
\begin{align*}
    T^{(0)}(\tilde{\alpha}(\hat{\eta}^*) + w(\gamma)) \le T^{(0)}(\alpha(\hat{\eta}^*)).
\end{align*}
Now, in part 1) of the Theorem, we assume that $T^{(0)}(\alpha) \le T(\alpha)$ and hence
\[
T^{(0)}(\alpha(\hat{\eta}^*)) \le T(\alpha(\hat{\eta}^*)) = \beta(\hat{\eta}^*).
\]
But this means that the condition \eqref{e:condition} can only be met if $\beta(\hat{\eta}^*)>\beta(\hat{\eta}^*)$, which is false. Hence, conditional on \eqref{e:gcon}, which asymptotically holds with probability $\ge 1-\gamma$ the algorithm does not (falsely) detect a privacy violation and 
\[
\liminf_{n_2 \to \infty} \,\,\Pr\Big[ A = "\textnormal{No Violation}"\Big]=1-\gamma_{n_1}\ge 1-\gamma.
\]
It follows directly that
\[
\liminf_{n_1 \to \infty} 1-\gamma_{n_1} \ge  1-\gamma
\]
showing the first part of the theorem.\\
Now, in part 2), suppose that there exists a privacy violation. The trade-off function is strictly convex and it is not hard to see that this implies that it equals the set $\{(\alpha(\eta), \beta(\eta): \eta \ge 0\}$ in this case (the constant $\lambda$ in the Neyman-Pearson test can be set to $0$ everywhere). We also define the maximum violation
\[
v^* = \sup_{\alpha \in [0,1]}\big[ T^{(0)}(\alpha)-T(\alpha)\big]
\]
and the set of thresholds
\[
\Psi:= \big\{\eta \ge 0:T^{(0)}(\alpha(\eta))-T(\alpha(\eta)) \ge  v^*/2\big\}.
\]
It holds by the proof of Theorem \ref{theo:1} case 1) that 
\[
\sup_\eta |\hat \alpha_h(\eta) - \alpha(\eta)|\overset{P}{\to} 0, \quad as \,\,\, n_1 \to \infty.
\]
In particular, it follows that
\[
\Pr[\hat \eta^* \in \Psi]= 1-r_{n_1},
\]
where $r_{n_1} \to 0$ as $n_1 \to \infty$.
If the above statement were false, it would follow on an event with asymptotically positive probability that 
\[
T^{(0)}(\alpha(\hat \eta^*))-T(\alpha(\hat \eta^*)) \le (1/2) v^*
\]
leading to a contradiction with Proposition \ref{prop1}. Now, we condition on the event $\{\hat \eta^* \in \Psi\}$ and pass the parameter to the BayBox estimator, which returns the estimator pair $(\tilde{\alpha}(\hat{\eta}^*), \tilde{\beta}(\hat{\eta}^*))$. Now, keeping $n_1$ fixed and letting $n_2 \to \infty$ it follows that
\begin{align*}
   & \tilde{\alpha}(\hat{\eta}^*) + w(\gamma) \overset{P}{\to} \alpha(\hat{\eta}^*), \quad  \tilde{\beta}(\hat{\eta}^*) + w(\gamma) \to \beta(\hat{\eta}^*).
\end{align*}
Given the continuity of the function $T^{(0)}$ (every trade-off function is continuous) it follows that conditionally on $\Psi$
\begin{align*}
&T^{(0)}(\tilde{\alpha}(\hat{\eta}^*) + w(\gamma)) \to T^{(0)}(\alpha(\hat{\eta}^*)) \ge T(\alpha(\hat{\eta}^*) +v^*/2\\
= &\beta(\hat{\eta}^*) +\nu^*/2>  \beta(\hat{\eta}^*)
\end{align*}
and the detection condition in \eqref{e:condition} is asymptotically fulfilled as $n_2 \to \infty$. Thus, we have 
\[
\lim_{n_2 \to \infty}\Pr[A = \textnormal{"Violation"}| \{\hat \eta^* \in \Psi\}]  =1
\] 
and hence
\[
\liminf_{n_2 \to \infty}\Pr[A = \textnormal{"Violation"}]\ge 1-r_{n_1}.
\]
Taking the limit $n_1 \to \infty$ we have $r_{n_1} \to 0$ and the result follows.
\end{proof}

\section{Additional Experiments and Details} \label{AppB}

In this section, we provide some additional details on our experiments and implementations.

\subsection{Implementation details}

Algorithm \ref{alg:pointwise_KDE_estimator} gives a pseudo-code of our trade-off curve estimator $\hat T_h$, presented in Section \ref{sec:4}. 

\begin{algorithm}[h]
\footnotesize
\algorithmicrequire \; \parbox[t]{\dimexpr0.9\linewidth-\algorithmicindent}{Black-box access to $\Mech$; Threshold $\eta > 0$; Sample size $n$, databases $\DB, \DB'$.}\\[0.1cm]
\algorithmicensure \, An estimate $(\hat{\alpha}(\eta), \hat{\beta}(\eta))$ of $(\alpha(\eta), \beta(\eta))$ for tuple $(P, Q)$, where $\Mech(\DB)$ and $\Mech(\DB')$ are distributed to $P, Q$, respectively.
\begin{algorithmic}[1]
    \State Choose perturbation parameter $h$. 
    \State Set the density estimation algorithm $\cA$. By default, use the KDE algorithm.
    \Function{\textnormal{PTLR Estimatior} $\ptlr{h}{\cA}(M, \DB, \DB', \eta,n)$}{}
    \State Compute the estimated densities $\hat{p}$ and $\hat{q}$ by running $\mathcal{A}$ on $n$ independent copys of $\Mech(\DB)$ and $\Mech(\DB')$, respectively.
    % based on outputs of $M$ by running $\cA$ with a sample size of $n$.
    \State Compute $\hat{\alpha}(\eta) \leftarrow \int_{x \in [-h/2,h/2]} \frac{1}{h}\int_{\hat q /\hat p  > \eta +x} \hat p$ 
    \State Compute $\hat{\beta}(\eta) \leftarrow \int_{x \in [-h/2,h/2]} \frac{1}{h}  \int_{\hat q /\hat p  > \eta +x} \hat q$ 
    \State Return $(\hat{\alpha}(\eta), \hat{\beta}(\eta))$
    \EndFunction
\end{algorithmic}
\caption{PTLR: A Perturbed Likelihood Ratio Test Algorithm for $f$-DP Estimation}
\label{alg:pointwise_KDE_estimator}
\end{algorithm}

Next, we turn to the DP-SGD algorithm from our Experiments section. The pseudocode for that algorithm can be found in Algorithm \ref{alg:noisy_sgd} below. Note that we add Gaussian noise $Z_t \sim \mathcal{N}(0, \sigma^2)$ to the parameter $\theta_t$ at each iteration of DP-SGD. The operator $\Pi_{\Theta}$ projects the averaged and perturbed gradient onto the space $\Theta$ and is thus similar to clipping that gradient. We can derive the exact trade-off function of this algorithm for our choice of databases in \eqref{eq_databases} and our specifications from Section \ref{sec:algorithms}. More concretely, we first consider the distribution of DP-SGD on $D = (0, \dots, 0)$ and note that 
\begin{align*}
    \theta_{t+1} = \theta_t - \rho \, (\theta_t + Z_{t+1}) 
\end{align*}
for each $t \in \{0, \dots, \tau\}$. Some calculations then yield that $\Theta_{\tau} \sim \mathcal{N}(0, \bar{\sigma}^2)$ with 
\begin{align} \label{sigma_bar}
    \bar{\sigma}^2 = \rho^2 \, \sigma^2 \, \frac{1 - (1 - \rho)^{2 \tau}}{1 - (1 - \rho)^{2}}.
\end{align}
Similarly, we have for $D' = (1, 0, \dots, 0)$ that 
\begin{align*}
     \theta_{t+1} = (1 - \rho) \, \theta_t + \rho \,  Z_{t+1} 
\end{align*}
for each $t \in \{0, \dots, \tau\}$. Here, $Z_t$ is a Gaussian mixture with 
\begin{align*}
    Z_t \sim \frac{1}{2} \, \mathcal{N}\left(0,\sigma^2\right) + \frac{1}{2} \, \mathcal{N}\left(\frac{1}{m},\sigma^2\right).
\end{align*}
We can then see that $ \theta_{\tau} = \tilde{Z}_1 + \dots + \tilde{Z}_{\tau} $
where the $\tilde{Z}_t$ are independent Gaussian mixtures with
\begin{align*}
    \tilde{Z}_t & \sim \frac{1}{2} \, \mathcal{N}\left(0, \rho^2 \, (1- \rho)^{2 (\tau - t)} \, \sigma^2\right) \\ & + \frac{1}{2} \, \mathcal{N}\left(\frac{\rho (1 - \rho)^{\tau - t}}{m}, \rho^2 \, (1- \rho)^{2 (\tau - t)} \, \sigma^2\right).
\end{align*}
By defining 
\begin{align} \label{mu_I}
    \mu_I := \sum\limits_{t \in I} \frac{\rho (1 - \rho)^{\tau - t}}{m}
\end{align}
and choosing $\bar{\sigma}$ as in \eqref{sigma_bar}, we get that
\begin{align*}
    \theta_{\tau} \sim \sum\limits_{t \subset \{1, \dots, \tau\}} \frac{1}{2^{\tau}} \mathcal{N}(\mu_I, \bar{\sigma}^2).
\end{align*}
Having derived the distribution of $M(D)$ and $M(D')$, we take a look at the corresponding LR-test $g$ and note that it can be expressed as
\begin{align*}
    g(x) = \begin{cases}
       1  & x > c \\
       0  & x \leq c \\
    \end{cases}
\end{align*}
for some threshold $c$. A few calculations then yield the trade-off curve
\begin{align*}
    T_{SGD}(\alpha)=\sum_{I\subset \{1,\hdots, \tau\}} \frac{1}{2^{\tau}}\Phi\Big(\Phi^{-1} (1-\alpha)-\frac{\mu_I}{\bar\sigma}\Big)~.
\end{align*}

\begin{algorithm}[h]
\footnotesize
\algorithmicrequire \; \parbox[t]{\dimexpr0.9\linewidth-\algorithmicindent}{Dataset $x = (x_1, \ldots, x_r)$, loss function $\ell(\theta, x)$,\\ Parameters: initial state $\theta_0$, learning rate $\rho$, batch size $m$, time horizon $\tau$, noise scale $\sigma$, closed and convex space $\Theta$.}\\[0.1cm]
\algorithmicensure \, Final parameter $\theta_{\tau}$.
\begin{algorithmic}[1]
    \For{$t = 1, \ldots, \tau$}
        \State \textbf{Subsampling:} Take a uniformly random subsample $I_t \subseteq \{1, \ldots, r\}$ with batch size $m$.
        \For{$i \in I_t$}
            \State \textbf{Compute gradient:} $v_t^{(i)} \leftarrow \nabla_\theta \ell(\theta_t, x_i)$
           % \State \textbf{Clip gradient:} $v_t^{(i)} \leftarrow v_t^{(i)} / \max\{1, \|v_t^{(i)}\|_2 / C\}$
        \EndFor
        \State \textbf{Average, perturb, and descend:}
        \[
        \theta_{t+1} \leftarrow \theta_t - \rho \; \Pi_{\Theta} \left( \frac{1}{m} \sum_{i \in I_t} v_t^{(i)} + Z_t \right)
        \]
    \EndFor
    \State \textbf{Output:} $\theta_{\tau}$
\end{algorithmic}
\caption{DP-SGD Algorithm}
\label{alg:noisy_sgd}
\end{algorithm}



\begin{figure*}[h!]
    \centering
    \subfloat[$n_1=10^3$]{\includegraphics[width=0.3\textwidth]{Figures/Laplace_shade_1000.png}}
    \hfill
    \subfloat[$n_1=10^4$]{\includegraphics[width=0.3\textwidth]{Figures/Laplace_shade_10000.png}}
    \hfill
    \vspace{-0.2cm}
    \subfloat[$n_1=10^5$]{\includegraphics[width=0.3\textwidth]{Figures/Laplace_shade_100000.png}}
  \caption{Estimation of the Laplace Trade-off curve $T_{Lap}$ for varying sample sizes and $\mu=1$. Min- and Max Curve lower- and upper bound the worst point-wise deviation from the true curve $T_{Lap}$ over $1000$ simulations.}
    \label{fig:laplace}
\vspace{-0.1cm}
    \centering
    \subfloat[$n_1=10^3$]{\includegraphics[width=0.3\textwidth]{Figures/Subsampling_shade_1000.png}}
    \hfill
    \subfloat[$n_1=10^4$]{\includegraphics[width=0.3\textwidth]{Figures/Subsampling_shade_10000.png}}
    \hfill
    \vspace{-0.2cm}
    \subfloat[$n_1=10^5$]{\includegraphics[width=0.3\textwidth]{Figures/Subsampling_shade_100000.png}}
     \caption{Estimation of the Subsampling Trade-off curve $T_{Sub}$ with the Gaussian mechanism for $\mu=1$ and varying sample sizes and $\mu=1$. Min- and Max Curve lower- and upper bound the worst point-wise deviation from the true curve $T_{Sub}$ over $1000$ simulations.}
    \label{fig:subsampling}
\end{figure*}





\subsection{Additional Algorithms}
We test our estimation procedure on the Laplace and Subsampling algorithm, which often serve as building blocks in more sophisticated privacy mechanisms. We select the same setting for our experiments as in Section \ref{sec6} and choose $D$ and $D'$ as in \eqref{eq_databases}. \\

\noindent \textbf{Laplace mechanism.} We consider the summary statistic $S(x)= \sum_{i=1}^{10} x_i$ and the mechanism
\begin{equation*}
    M(x):= S(x)+Y~,
\end{equation*}
where $Y\sim \mathcal Lap(0, \sigma)$. The statistic $S(x)$ is privatized by the random noise $Y$ if the scale parameter $\sigma > 0$ of the Laplace distribution is chosen appropriately. We choose $\sigma = 1$ for our experiments and observe that the optimal trade-off curve is given by 
\begin{align*}
    T_{Lap}(\alpha)=\begin{cases}
        1- e \, \alpha,  &\alpha<e^{-1}/2~,\\
        e^{-1}/4 \alpha,  &e^{-1}/2\leq \alpha\leq 1/2~,\\
        e^{-1}(1-\alpha), &\alpha>1/2.
    \end{cases}
\end{align*}
We point the interested reader to \cite{Dong2022} for more details on how to derive $T_{Lap}$. \\

\noindent \textbf{Subsampling algorithm.} Random subsampling provides an effective way to enhance the privacy of a DP mechanism $M$. We only provide a rough outline here and refer for details to \cite{Dong2022}.
In simple words, we choose an integer $m$ with  $1\leq m< r$, where $r$ is the size of the database $D$. We then draw a random subsample of size $m$ from $D$, giving us the smaller database $\bar D$ of size $m$. The mechanism $M$ is then applied to $\bar D$ instead of $D$, providing users with an additional layer of privacy (if a user is not part of $\bar D$, their privacy cannot be compromised). The amplifying effect that subsampling has on privacy is visible in the optimal trade-off curve: If $M$ has the trade-off curve $T$, then $M(\bar D)$ has the trade-off curve
\begin{equation*}
    \bar T(\alpha)=  \frac{m}{r}T(\alpha)+\frac{r-m}{r}(1-\alpha),
\end{equation*}
which is strictly more private than $T$ for any $m<r$. A minor technical peculiarity of subsampling is that the resulting curve $\bar T$ is not necessarily symmetric, even if $T$ is (see \cite{Dong2022} for details on the symmetry of trade-off functions). Trade-off curves are usually considered to be symmetric and one can symmetrize $\bar T$ by applying a symmetrizing operator $\mathbf{C}$ with 
\begin{equation*}
    \mathbf{C}[T](x)=\begin{cases}
         T(x), \quad &x\in [0,x^*]\\
        x^*+ T (x^*)-x, \quad &x\in [x^*, T(x^*)]\\
         T^{-1}(x), \quad &x\in [ T(x^*),1],
    \end{cases}
\end{equation*}
where $x^*$ is the unique fix-point of $T$ with $T(x^*)=x^*$ (for more details we refer to \cite{Dong2022}). Another mathematical representation of $\mathbf{C}$ that we use in our code is 
$\mathbf{C}(T)=\min\{T,T^{-1}\}^{**}$, where the index $**$ signifies double convex conjugation. We incorporate this operation into our estimation procedure by simply applying $\mathbf{C}$ to our estimate of the trade-off function $T$. For our experiments involving subsampling, we use the Gaussian mechanism for $M$ (with $\sigma=1$) and obtain the subsampled version $M'$ by fixing the parameter $m=5$ (recall that $r=10$). \\




\noindent Similar to the experiments section, we construct figures that upper and lower bound the worst case errors for the Laplace mechanism and the Subsampling algorithm over $1000$ simulation runs. We can see again that the error of the estimator $\hat T_h$ shrinks significantly, as $n_1$ grows. 




\subsection{Additional simulations}

We present some results that complement the main findings in our experiment section. We use the same setup as described in our experiments and investigate a faulty implementation of the Gaussian mechanism. We study two things: First, the impact of the parameter $\gamma$, where we vary $\gamma$ between very small and relatively large values. As we can see, smaller values of $\gamma$ lead to larger boxes $\square_\gamma$ which make it harder for the auditor to detect violations. Secondly, we consider the impact of the sample size $n_1$ ranging from the very modest value of $10^2$ up to $10^4$. We see that the sample size has very little impact on the performance of the procedure and it already works well for fairly small samples $n_1$ ($n_2$ has a greater impact, as we have seen in our experiments).
\begin{figure*}[t!]
    \centering
    \subfloat[\centering $\gamma=0.001$, \textbf{Ground truth:} Violation; \textbf{Decision:} \textcolor{red}{"No Violation"}{\textcolor{red}{\scalebox{1.5}{\ding{55}}}}]{\includegraphics[width=0.3\textwidth]{Additional_Figures/gauss_faulty_gamma_0.001_mu_0.2.png}}
    \hfill
    \subfloat[\centering $\gamma=0.01$, \textbf{Ground truth:} Violation; \textbf{Decision:} \textcolor{red}{"No Violation"}{\textcolor{red}{\scalebox{1.5}{\ding{55}}}}]{\includegraphics[width=0.3\textwidth]{Additional_Figures/gauss_faulty_gamma_0.01_mu_0.2.png}}
    \hfill
    \subfloat[\centering $\gamma=0.1$, \textbf{Ground truth:} Violation; \textbf{Decision:} \textcolor{green}{"Violation"}{\textcolor{green}{\scalebox{1.5}{\ding{51}}}}]{\includegraphics[width=0.3\textwidth]{Additional_Figures/gauss_faulty_gamma_0.1_mu_0.2.png}}
    \hfill
    \subfloat[\centering $n_2=10^2$,
  \textbf{Ground truth:} Violation; \textbf{Decision:} \textcolor{green}{"Violation"}{\textcolor{green}{\scalebox{1.5}{\ding{51}}}}]
  {\includegraphics[width=0.3\textwidth]{Additional_Figures/gauss_faulty_gamma_0.05_mu_0.2_n_2.png}}
    \hfill
    \subfloat[\centering $n_1=10^3$, \textbf{Ground truth:} Violation; \textbf{Decision:} \textcolor{green}{"Violation"}{\textcolor{green}{\scalebox{1.5}{\ding{51}}}}]{\includegraphics[width=0.3\textwidth]{Additional_Figures/gauss_faulty_gamma_0.05_mu_0.2_n_3.png}}
    \hfill
    \subfloat[\centering $n_1=10^4$, \textbf{Ground truth:} Violation; \textbf{Decision:} \textcolor{green}{"Violation"}{\textcolor{green}{\scalebox{1.5}{\ding{51}}}}]{\includegraphics[width=0.3\textwidth]{Additional_Figures/gauss_faulty_gamma_0.05_mu_0.2_n_4.png}}
        \caption{\textbf{Auditing a faulty Mechanism:} Claimed Curve $\textcolor{blue}{T^{(0)}} = \textcolor{blue}{T_{Gauss}}$ with $\mu=0.2$, but in reality $\mu=1$. For (a),(b),(c) we consider $n_1=10^4$, and for (d),(e),(f) we have considered various sample sizes for the KDEs, respectively. Throughout the simulations we keep $n_2=10^6$ fix.}\label{fig:additional_experiments}
       % \caption{\textbf{Auditing a faulty Mechanism:} Claimed Curve $\textcolor{blue}{T_0} = \textcolor{blue}{T_{SGD}}$ with $t_{-}=5$ iteration, but in reality data is accessed $t_{-}=10$ times. Estimate $\textcolor{blue}{T_0}$ by $\textcolor{orange}{\hat T_0}$.  With critical vertical line with intercept $(\hat\alpha(\hat\eta^*), \hat \beta(\hat \eta^*))$, $k$-NN point estimator {\textcolor{purple}{\ding{108}}} $(\tilde\alpha(\hat\eta^*), \tilde \beta(\hat\eta^*))$ and confidence region $\textcolor{purple}{\square}$. The sample size for KDE is $n_1=10^2$ and the confidence parameter is $\gamma=0.05$.}
\end{figure*}
%The derivation of this result is quite involved. 

$ $\\[1000ex] $ $ \newpage

%the subsampling mechanism uniformly samples a subset of size $m$ out of the $k$ individuals and performs the statistical analysis based on that subset instead of the whole database. For an $f-$DP mechanism $M$ acting on a database with participant number $k$, \cite{Dong2022} characterize the privacy guarantee by an operator $C_p$ acting on trade-off function. In particular, we have for $p=m/k$ that
%\begin{equation*}
 %   T_{Sub}(\alpha)= C_p[T]%(\alpha)~,
%\end{equation*}
%where 
%\begin{equation*}
%    C_p[T](\alpha)=\begin{cases}
  %      T_p(x), \quad &x\in [0,x^*]\\
  %      x^*+T_p(x^*)-x, \quad &x\in [x^*,T_p(x^*)]\\
   %     T_p^{-1}(x), \quad &x\in [T_p(x^*),1]
  %  \end{cases}
%\end{equation*}
%Roughly speaking $p =m/k$ corresponds to the probability of drawing a specific individual from a database of size $k$ in $m$ draws (and $(1-p)$ is the probability of not-drawing that individual). This motivates the specific form of $T_p$. With that in hand, the formula of $C_p$ follows from the (more general) expression  $C_p(T)=\min\{T_p,T_p^{-1}\}^{**}$ in \cite{Dong2022}. Here for a curve $f$, the function $f^*$ is the convex conjugate. is simply a symmetrization of $T_p$ in a sense that it is the greatest convex minorant of the minimum $T_p$ and $T_p^{-1}$ (\todo{maybe cite sth}). Here, for a function $T$, $T^{**}$ denotes the twice convex conjugate of $T$.


%\theendnotes

\end{document}







