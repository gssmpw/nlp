\subsection{Proofs for Goal 1 (Estimation)}

\textbf{Consequences of Theorem \ref{theo:1}} The main result in Section \ref{sec:4} is Theorem \ref{theo:1}. Lemma \ref{lem1} can be seen as a special case, putting $\hat p=p, \hat q=q$ . Then, Assumption \ref{ass2} is met for the constant sequence $a_n=0$. It follows by this construction that $\hat T_h =T_h$, is non-random and only depends on $h$. Any choice of $h\downarrow 0$ is permissible and  Lemma \ref{lem1} follows from the Theorem. Proposition \ref{prop1} too is a direct consequence of Theorem \ref{theo:1}. To see this, we notice that
\begin{align*}
    & T_0(\hat \alpha_h(\hat \eta^*)) - T(\hat \alpha_h(\hat \eta^*)) \\
    =& T_0(\hat \alpha_h(\hat \eta^*)) - \hat T_h(\hat \alpha_h(\hat \eta^*))+o_P(1)\\
    = & \sup_{\alpha \in [0,1]}\{T_0(\alpha) - \hat T_h(\alpha)\}+o_P(1)\\
    =& \sup_{\alpha \in [0,1]}\{T_0(\alpha) -  T(\alpha)\}+o_P(1).
\end{align*}
In the first and last step, we have used the uniform convergence of Theorem \ref{theo:1}, which allows us to replace $T$ by $\hat T_h$ while only incurring an $o_P(1)$ error. In the second step, we have used the definition of $\hat \alpha_h(\hat \eta^*)$ as the maximizer of the difference between $T_0$ and $\hat T_h$. Thus Proposition \ref{prop1} follows. We now turn to the proof of the theorem. The proof is presented for densities on the real line. Extensions to $\mathbb{R}^d$ are straightforward and therefore not discussed. \\
\textbf{Preliminaries} Recall that a complete separable metric space is Polish. The real numbers, equipped with the absolute value distance is a Polish space. The continuous functions $\mathcal{C}_0$ on the real line that vanish at $\pm \infty$, i.e. that satisfy
\begin{align} \label{e:decay}
\lim_{x \to \infty} f(x) = \lim_{x \to \infty} f(-x)=0 
\end{align}
is a Polish space if equipped with the supremum norm
\[
\|f\|:= \sup_{x \in \mathbb{R}}|f(x)|.
\] 
The product of complete, separable metric spaces is complete and separable if equipped with the maximum metric, i.e. the space $\mathcal{C}_0 \times \mathcal{C}_0\times \mathbb{R}\times \mathbb{R}$ is Polish. 
Now, the vector 
\[
(\hat p, \hat q, \|\hat p-p\|_\infty/a_n, \|\hat q-q\|_\infty/a_n)
\]
lives on this space (for each $n$) and convergences to the limit $(p,q,0,0)$ in probability (see Assumption \ref{ass2}). Accordingly we can use Skorohod's theorem to find a probability space, where this convergence is a.s. 
\[
(\hat p, \hat q, \|\hat p-p\|_\infty/a_n, \|\hat q-q\|_\infty/a_n) \to (p,q,0,0) \quad a.s.
\]
It is a direct consequence that on this space it holds a.s.
\[
\|\hat p-p\| =o(a_n), \quad \|\hat q-q\| =o(a_n).
\]
In the following, we will work on this modified probability space and exploit the a.s. convergence. We will fix the outcome and regard $\hat p, \hat q$ as sequences of deterministic functions, converging to their respective limits at a rate $o(a_n)$.\\
    Next, it suffices to show the desired result pointwise for any $\alpha$. This reduction is well-known. For a sequence of  continuous, monotonically decreasing functions $(f_n)_n$ living on the unit interval $[0,1]$, pointwise convergence  to a continuous, monotonically decreasing limit $f$ on $[0,1]$ implies uniform convergence. The same argument lies at the heart of the proof of the famous Glivenko-Cantelli Theorem (see \cite{vaart:wellner:1996}). We now want to demonstrate the convergence $|\hat T(\alpha) -T(\alpha)|=o(1)$ pointwise. More precisely, we will demonstrate that for the pair $(\alpha, T(\alpha))$, there exist values of $\eta$ such that $\hat \alpha_h(\eta) \to \alpha$ and $\hat\beta_h(\eta) \to T(\alpha)$. Since the proofs of both convergence results work exactly in the same way, we restrict ourselves in this proof to show that $\hat \alpha_h(\eta) \to \alpha$. So let us consider a fixed but arbitrary value of $\alpha \in [0,1]$ and begin the proof.\\
    \textbf{Case 1:} We first consider the case where $\eta\ge 0$ (the threshold in the optimal LR test) is such that the set $\{q/p=\eta\}$ has $0$ mass. In this case, the coin toss with probability $\lambda$ can be ignored (it happens with probability $0$) and  we can define the type-I-error  $\alpha$ of the Neyman-Pearson test as 
    \[
\alpha = \int p \cdot \mathbb{I}\{q/p  > \eta\}.
\]
In this case, we want to show that 
\begin{align*}
& \int_{x \in [-h/2,h/2]} \frac{1}{h}\int_{\hat q /\hat p  > \eta +x} \hat p \\
=& \int  \int_{x \in [-h/2,h/2]}  \hat p \frac{1}{h}\,\,\mathbb{I}\{ \hat q /\hat p  > \eta +x\}=:\int \hat g \\
\to &  \int_{q/p  > \eta}  p  = \int p \cdot \mathbb{I}\{q/p  > \eta\} =:\int g .
\end{align*}
Here we have defined the functions $ g, \hat g$ in the obvious way. We will now show $\hat g$ converges pointwise to $g$. For this purpose consider the interval $[-K,K]$ for a large enough $K$, such that
\[
\int_{[-K,K]^c} p<\zeta \qquad \textnormal{and} \qquad\int_{[-K,K]^c} q<\zeta
\]
for a number $\zeta$ that we can make arbitrarily small. Given the uniform convergence of the density estimators on the interval $[-K,K]$ it holds for all $n$ sufficiently large that also
\[
\int_{[-K,K]^c} \hat p<\zeta \qquad \textnormal{and} \qquad\int_{[-K,K]^c} \hat q<\zeta.
\]
Accordingly we have 
 \[
 \bigg| \int \hat g - g\bigg| \le 2 \zeta + \bigg| \int_{[-K,K]} \hat g -g\bigg|.
 \]
 We then focus on the second term on the right and fix some argument $y \in [-K,K]$. It holds that either $ q(y)/ p(y)$ is bigger or smaller than $\eta$ (equality occurs only on a null-set and can therefore be neglected). Let us focus on the case where $q(y)/ p(y)> \eta$. If this is so, then it follows that in a small environment, say for $y' \in [y-\zeta', y+\zeta']$ we also have $q(y')/ p(y')>\eta$. For all large enough $n$ it follows that $h/2<\zeta'$. Then, it is easy to see that also $\hat q(y')/\hat p(y') >\eta$ for all $y' \in [y-\zeta', y+\zeta']$ simultaneously, for all sufficiently large $n$. If this is the case, the indicators in the definition of $\hat g, g$ become $1$ and $\hat g=\hat p$, $g=p$. 
 So, we have pointwise $\hat g(y)=\hat p(y)  \to p(y) =g(y)$. Since $\hat g$ is also bounded for all sufficiently large $n$ (since the integral over the indicator is bounded and the sequence $\hat p$ is uniformly convergent to the bounded function $p$) we obtain by the theorem of dominated convergence that 
 \[
 \bigg| \int_{[-K,K]} \hat g -g\bigg|\to 0.
 \]
 This shows that 
 \[
 \limsup_n|\hat \alpha_h(\eta) - \alpha|=\mathcal{O}(\zeta).
 \]
 Finally, letting $\zeta \downarrow 0$ in a second limit shows the desired approximation in this case.\\
 \textbf{Case 2:} Next, we consider the case where the set $\{q/p=\eta\}$ has positive mass for some $\eta>0$.\footnote{We omit the simpler case where $\eta=0$ and $L=0$ anyways.}
 This means that the coin-flip in the definition of the optimal LR test plays a role and we set the probability $\lambda $ to some value in $[0,1]$.
  We then consider as estimator the value $\hat \alpha(\eta-b h)$ for a value $b$ that we will determine below. Let us, for ease of notation, define the probability 
 \[
 L := \int_{q/p=\eta} p
 \]
and appreciate that then
\begin{align} \label{e:dec}
\alpha = \alpha'+\mathcal{O}(\zeta) + \lambda L.
\end{align}
We explain the decomposition: In equation \eqref{e:dec}, $\alpha'$ is the rejection probability of the LR test defined by the decision to reject whenever $q(y)/p(y)>\eta+\zeta''$ for some small number $\zeta''$. For all small enough values of $\zeta''$ the threshold $\eta+\zeta''$ is not a plateau value (there are only finitely many of them; see Assumption \ref{ass1}). It follows that 
\[
\alpha' = \int p \cdot \mathbb{I}\{q/p  > \eta+\zeta''\}.
\]
Next, for any fixed constant $\zeta>0$ we can choose $\zeta''$ small enough such that
\begin{align} \label{e:int1}
\int p \cdot \mathbb{I}\{\eta <q/p  \le \eta+\zeta''\} < \zeta.
\end{align}
This explains the second term on the right of equation \eqref{e:dec}. The third term corresponds to the probability of rejecting whenever $q/p=\eta$ (this probability is $L$) times the probability that the coin shows heads (reject) with probability $\lambda$.\\
Now, using these definitions, we decompose the set
\begin{align*}
&\{ \hat q /\hat p  > \eta-b h +x\}\\
=& \{ \eta +\zeta'' \ge \hat q /\hat p  > \eta-b h +x\} \cup  \{  \hat q /\hat p  > \eta +\zeta''\}. 
\end{align*}
This yields the decomposition
\begin{align} \label{e:alcon}
& \hat \alpha_h(\eta-b h)
= \hat \alpha_h(\eta+\zeta'')\\
&+ \int  \int_{x \in [-h/2,h/2]}  \hat p \,\, \frac{1}{h}\,\,\mathbb{I}\{ \eta +\zeta'' \ge \hat q /\hat p  > \eta-b h +x\} .\nonumber 
\end{align}
Now, by part 1 of this proof we have  
\[
|\hat \alpha_h(\eta+\zeta'') - \alpha'|=o(1).
\]
 Next, we study the integral on the right side of eq. \eqref{e:alcon} and for this purpose define the objects
\begin{align*}
    \tilde g := & \int_{x \in [-h/2,h/2]}  \hat p \,\,\frac{1}{h}\,\,\mathbb{I}\{ A_1\}, \\
    \tilde f := & \int_{x \in [-h/2,h/2]}  \hat p \,\,\frac{1}{h}\,\,\mathbb{I}\{ A_2\}.\\
    A_1:= &\{\eta +\zeta'' \ge \hat q /\hat p  > \eta-b h +x,q/p=\eta\}, \\
    A_2:=& \{\eta +\zeta'' \ge \hat q /\hat p  > \eta-b h +x,q/p \neq \eta\}.
\end{align*}
Now, let us consider a value $y$ where $q(y)/p(y) \neq \eta$ and for sake of argument let us focus on the (more difficult) case $q(y)/p(y) >\eta$. If $q(y)/p(y) > \eta+\zeta''$, it follows that eventually $\hat p(y)/\hat q(y) > \eta+\zeta''$ and hence $\tilde f(y)=0$. The case where $q(y)/p(y) = \eta+\zeta''$ is a null-set and hence negligible (it is not a plateau value). The case where  $q(y)/p(y) \in (\eta, \eta+\zeta'')$ implies that eventually $\hat p(y)/\hat q(y) \in (\eta, \eta+\eta'')$ and thus eventually $\tilde f(y) = \hat p(y)$ which converges pointwise to $p$. Thus, we have by dominated convergence that 
\[
\int \tilde f \to \int p \cdot \mathbb{I}\{\eta <q/p  \le \eta+\zeta''\} <\zeta.
\]
The fact that the integral is bounded by $\zeta$ was established in eq. \eqref{e:int1}. This means that for all $n$ large enough we have 
\[
\int \tilde f < \zeta.
\]
Now, let us focus on a value of $y$ where $q(y)/p(y)=\eta$. In this case it follows that $q(y), p(y)>0$ and we have 
\[
\frac{\hat q(y)}{\hat p(y)} = \frac{q(y)}{p(y)} +o(a_n) = \eta +o(a_n).
\]
Notice that we can rewrite $\tilde g$ as
\begin{align*}
    \int_{x \in [-1/2,1/2]}  \hat p \,\,\mathbb{I}\{\eta +\zeta'' \ge \hat q /\hat p  > \eta-b h +hx,q/p=\eta\}.
\end{align*}
Now, for any $x>b$ it follows that the indicator will eventually be $0$, because 
\[
\hat q /\hat p =\eta+o(a_n) << \eta + h(x-b)
\]
(because $a_n=o(h)$ by assumption in the Theorem). By similar reasoning the indicator is $1$ if $x<b$. This means that $\tilde g$ converges for any fixed $y$ with $q(y)/p(y)=\eta$ to $p(y) \cdot (1/2+b)$ and using majorized convergence yields
\[
\int \tilde g \to (1/2+b) \int_{q/p=\eta} p =  (1/2+b)L.
\]
Now, we can choose  $b=\lambda-1/2$ to get that the right side is equal to $\lambda L$. Putting these considerations together, we have shown that
\[
\limsup_n |\alpha -\hat \alpha_h(\eta-[\lambda-1/2] h)| = \mathcal{O}(\zeta). 
\]
Taking the limit $\zeta \downarrow 0$ afterwards yields the desired result. 