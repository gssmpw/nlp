\section{Related Work}
\subsection{Stance Detection}
Stance detection is a subtask within opinion mining that automatically classifies text stances concerning a specified target. Early studies concentrated on political debates and online forums, while contemporary research has shifted towards online social platforms like Twitter and Weibo. Classifying stances from social media data presents a challenge ____, primarily due to the diverse and informal characteristics like 
brevity, Abbreviations, and emoji. Recent research has attempted to detect the stance of discourses on social media. SemEval-2016 ____ introduces a shared task focused on the detection of stance in tweets. The dataset in SemEval-2016 is widely utilised for subsequent research. Siddiqua et al. ____ proposed a novel method for detecting stance in microblog posts by leveraging syntactic tree structures. The authors argue that syntactic tree representations effectively capture contextual and structural information in text, which enhances the performance of stance detection models. Chang Xu et al. ____ investigate the generalisation of classifiers across various targets through a neural model that utilises acquired knowledge from a source target to inform a destination target. Ji et al. ____ propose a many-to-one CTSD model utilising meta-learning techniques. They also enhance meta-learning by implementing a balanced and easy-to-hard learning pattern. The Stance Reasoner framework leverages language models with explicit reasoning paths to perform zero-shot stance detection on social media ____. The approach enhances interpretability by generating reasoning explanations for stance predictions and achieves competitive results across multiple datasets without task-specific fine-tuning. 


\subsection{Stance Detection With Sentiment Information}
Affective information, including sentiments and emotions, is crucial. Such information help capture underlying feelings that influence stances, improving classification accuracy ____. The study by Mohammad et al. ____ explores methods for detecting stance in tweets and examines how stance interacts with sentiment in online discourse. The authors highlight that while stance and sentiment are related, they are distinct, and understanding both can provide deeper insights into opinions expressed on social media platforms. Smith and Lee ____ investigate linguistic features pertinent to stance detection, focussing on the contributions of lexical, syntactic, and pragmatic cues in discerning a user's stance within text. The research indicates that the integration of various linguistic features, including modality, sentiment, and discourse markers, enhances the precision of stance classification across a range of topics and contexts. Sun et al. ____ introduced a joint neural network model designed to simultaneously predict the stance and sentiment of a post. The model first learns a deep shared representation between stance and sentiment information, and then sentimental information is utilized for stance detection through a neural stacking model. Huang and Yang ____ proposed a multi-stance detection model incorporating entiment features to improve the accuracy of stance classification. The model uses an LDA-based five-category stance indicator system, extracts sentiment features via a lexicon, and applies a hybrid neural network for precise stance detection. This study utilises sentiment as the supervisory signal for disentangled representation learning, distinguishing it from prior research, owing to the architecture of the variational autoencoder. An auxiliary task for sentiment prediction is incorporated into the proposed model. PoliStance-VAE effectively integrates sentiment information with contextual information for stance detection through disentangled variables.