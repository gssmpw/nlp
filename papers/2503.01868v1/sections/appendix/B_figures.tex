\section{Appendix: Supplementary Figures}
%

\begin{figure}[H]
    \centering
    \includegraphics[width=0.5\textwidth]{figures/overlapping_p2p_conv.pdf}
    \caption{Diagram of computation in our overlapped {\tt point-to-point convolution} scheme.}
    \label{sup-fig:overlapping_comm}
\end{figure}

\begin{figure}[H]
    \includegraphics[width=0.99\textwidth]{figures/recall.pdf}
    \vspace{-0.1cm}
    \caption{\textbf{[Left]:} Validation perplexity on {\tt OpenGenome2} after midtraining extension with different techniques, at model scales of 7B and 40B. The extensions are performed on the base Evo 2 7B and 40B models. We also provide a linear fit at each scale. \textbf{[Right]:} Recall performance of 40B 1M measured via the needle-in-the-haystack task described in \citep{brixievo2}}
    \label{fig:niah_extension}
\end{figure}

\begin{figure}[H]
    \includegraphics[width=0.45\textwidth]{figures/tflops_40b.pdf} \hfill
    \includegraphics[width=0.45\textwidth]{figures/mfu_40b.pdf}
    \caption{MFU and TFLOPS / s / GPU of 40B models with same distributed configuration and different architectures. For StripedHyena 2, we achieve peak MFU of 34\% at 16K context. See Table \ref{tab:appendix_scaling} for details on the measurement protocol.}
    \label{fig:all_scaling2}
\end{figure}

\begin{figure}[H]
    \centering
    \includegraphics[width=0.8\textwidth]{figures/operator_profiles_tflops.pdf}
    \caption{Forward TFLOPs / second of \textsf{Hyena-SE}, \textsf{Hyena-MR} and other common operators in architecture design.}
    \label{fig:tflop-app}
\end{figure}