\usepackage[T1]{fontenc}  
\usepackage[bookmarks=true, 
            colorlinks=true,
            linkcolor=bluegray,
            citecolor=junglegreen]{hyperref}
\usepackage{natbib}
% \usepackage[backref=true, 
%             sorting=none]{biblatex}
%\addbibresource{./main.bib}
%\usepackage[letterpaper, margin=.9in]
\usepackage[a4paper, margin=.9in]{geometry}
\usepackage{wrapfig}
\usepackage{booktabs}  
\usepackage[bottom]{footmisc}
% \usepackage{longtable}
% \usepackage{ducksay}
% \usepackage{etaremune}
% \usepackage{afterpage}
% \usepackage{capt-of}
% \usepackage[table, x11names]{xcolor}
% \usepackage{decorule}
% \usepackage{scalerel,xparse}
\usepackage{enumitem} % customization of `list` environments
%%%% Graphics %%%%
%%%%%%%%%%%%%%%%%%

% count figures by section
\usepackage{chngcntr}
\counterwithin{figure}{section} 
\counterwithin{table}{section}

% colors and graphics
\usepackage{graphicx}
\usepackage{tikz}
\usepackage{tikz-cd}
\usepackage{hf-tikz}
\usepackage{pgfplots} 
\pgfplotsset{compat=1.17} 
\pgfplotsset{
        table/search path={figures/drawings},
    }
\usetikzlibrary{fadings}
\usetikzlibrary{shapes, arrows, fit, backgrounds, arrows.meta}

\usetikzlibrary{matrix}
\usetikzlibrary{shadows.blur}
\usetikzlibrary{patterns, tikzmark}
\usetikzlibrary{decorations.pathreplacing, calc, decorations.markings,}
\usetikzlibrary{positioning}

\usepgfplotslibrary{groupplots}
\usepgfplotslibrary{patchplots}
\definecolor{bluegray}{rgb}{0.4, 0.6, 0.8}
\definecolor{bluebell}{rgb}{0.64, 0.64, 0.82}
\definecolor{etonblue}{rgb}{0.59, 0.78, 0.64}
\definecolor{junglegreen}{rgb}{0.16, 0.67, 0.53}
\definecolor{bg}{gray}{0.97}
\definecolor{olive}{rgb}{0.6, 0.6, 0.2}
\definecolor{sand}{rgb}{0.8666666666666667, 0.8, 0.4666666666666667}
\definecolor{wine}{rgb}{0.5333333333333333, 0.13333333333333333, 0.3333333333333333}
\definecolor{deblue}{RGB}{11,132,147}
\definecolor{ocra}{RGB}{204, 119, 34}

\def\ocra{\color{ocra}}
\def\orange{\color{orange!70!white}}
\def\red{\color{red!70!white}}
\def\lblue{\color{blue!70!white}}
\def\lgreen{\color{green!70!black}}
\def\blue{\color{blue!70!white}}
\def\brown{\color{red!70!green!70!white}}

\newcommand{\fcircle}[2][red,fill=red]{\tikz[baseline=-0.5ex]\draw[#1,radius=#2] (0,0.03) circle ;}

\pgfplotsset{%
            mesh line legend/.style={legend image code/.code=\meshlinelegend#1},%
}
\makeatletter
\long\def\meshlinelegend#1{%
    \scope[%
        #1,
        /pgfplots/mesh/rows=1,
        /pgfplots/mesh/cols=4,
        /pgfplots/mesh/num points=,
        /tikz/x={(0.44237cm,0cm)}, 
        /tikz/y={(0cm,0.23932cm)},
        /tikz/z={(0.0cm,0cm)},
        scale=0.4,
    ]
    \let\pgfplots@metamax=\pgfutil@empty
    \pgfplots@curplot@threedimtrue

    \pgfplotsplothandlermesh
    \pgfplotstreamstart

    \def\simplecoordinate(##1,##2,##3){%
        \pgfmathparse{1000*(##3)}%
        \pgfmathfloatparsenumber\pgfmathresult
        \let\pgfplots@current@point@meta=\pgfmathresult
        \pgfplotstreampoint{\pgfqpointxyz@orig{##1}{##2}{##3}}%
    }%

    \pgfplotsforeachungrouped \x in {0,...,\pgfkeysvalueof{/pgfplots/samples}}{
        \pgfmathsetmacro\y{\x/\pgfkeysvalueof{/pgfplots/samples}}
        \pgfmathsetmacro\x{\x/\pgfkeysvalueof{/pgfplots/samples}*3}
        \simplecoordinate(\x,0,\y)
    }

    \pgfplotstreamend
    \pgfusepath{stroke}
    \endscope
}%
\makeatother

%%%% Fonts %%%%%%%
% Japanese Characters
% Lettrine
\usepackage{lettrine}
\usepackage{Typocaps}
\renewcommand{\LettrineFontHook}{\color{VioletRed4}\Typocapsfamily{}}
\LettrineTextFont{\itshape}
\setcounter{DefaultLines}{3}%

%%%% Theorems %%%%
%%%%%%%%%%%%%%%%%%
\usepackage{amsthm}
\newtheorem{corollary}{Corollary}[section]
\newtheorem{definition}{Definition}[section]
\newtheorem{lemma}{Lemma}[section]
\newtheorem{proposition}{Proposition}[section]
\newtheorem{remark}{Remark}[section]
\newtheorem{theorem}{Theorem}[section]
%%%% Tables %%%%
%%%%%%%%%%%%%%%%
\newcommand{\specialcell}[2][c]{%
    \begin{tabular}[#1]{@{}l@{}}#2\end{tabular}}



%%%% Table of Contents %%%%
%%%%%%%%%%%%%%%%%%%%%%%%%%%
\usepackage{minitoc}
\renewcommand \thepart{}
\renewcommand \partname{}

%%%% Custom cross-ref %%%%
%%%%%%%%%%%%%%%%%%%%%%%%%%
\newcommand{\chapref}[1]{\hyperref[#1]{Chapter \ref{#1}}}
\newcommand{\secref}[1]{\hyperref[#1]{Section \ref{#1}}}


%%%% Margin notes %%%%
%%%%%%%%%%%%%%%%%%%%%
\usepackage{marginnote}
\renewcommand\raggedrightmarginnote{\sloppy}
\renewcommand\raggedleftmarginnote{\sloppy}

%%%% Boxes %%%%
%%%%%%%%%%%%%%%
\usepackage[many]{tcolorbox}
\tcbuselibrary{breakable,xparse,skins}
\usepackage{bigstrut}
% add math symbols as unicode for the minted box
\usepackage{newunicodechar}
\newunicodechar{λ}{{$\mathtt\lambda$}}
\newunicodechar{μ}{{$\mathtt\mu$}}
\newtcolorbox{note}[1]{breakable, enhanced, sharp corners, frame hidden, #1}

% add math symbols as unicode for the minted box
\usepackage{newunicodechar}
\newunicodechar{λ}{{$\mathtt\lambda$}}
\newunicodechar{μ}{{$\mathtt\mu$}}

% emphasized text
\usepackage{xparse}
\DeclareTColorBox{emphbox}{O{black}O{0cm}}{
    enhanced jigsaw,
    breakable,
    outer arc=0pt,
    arc=0pt,
    colback=white,
    rightrule=0pt,
    toprule=0pt,
    top=0pt,
    right=0pt,
    bottom=0pt,
    bottomrule=0pt,
    colframe=#1,
    enlarge left by=#2,
    width=\linewidth-#2,
}

\DeclareTColorBox{emphbox}{O{black}O{0cm}}{
    empty,
    breakable=true,
    outer arc=0pt,
    arc=0pt,
    rightrule=0pt,
    leftrule=2pt,
    borderline west={2pt}{0pt}{#1},
    toprule=0pt,
    top=0pt,
    right=-3pt,
    bottom=0pt,
    bottomrule=0pt,
    colframe=#1,
    enlarge left by=#2,
    width=\linewidth-#2,
}
\usepackage{listings}
\makeatletter
\newcommand\BeraMonottfamily{%
  \def\fvm@Scale{0.85}% scales the font down
  \fontfamily{fvm}\selectfont% selects the Bera Mono font
}
\makeatother

\lstset{
  basicstyle=\BeraMonottfamily, 
  frame=single,
}
\newenvironment{codeblock}[4][]
 {\VerbatimEnvironment
  \begin{minted}[
   mathescape,
   numbersep=5pt,
   gobble=2,
   bgcolor=gray!20,
   fontsize=\scriptsize,
   framesep=2mm,
    #1]{#2}}
 {\end{minted} 
 }
\usepackage[parfill]{parskip}
\usepackage{xcolor,colortbl}
\usepackage{subcaption}
\usepackage{authblk}
\usepackage{algorithm}     % For the algorithm environment
\usepackage{algpseudocode} % For the algorithmic environment
\usepackage{amsmath}       % For mathematical symbols like \mathbb{R}
\usepackage{float}   
\usepackage{multicol}      % For the [H] placement specifier
\usepackage[frozencache,cachedir=.,outputdir=.]{minted}
% left margin in itemize
\usepackage{enumitem}
\setlist[itemize]{leftmargin=2em}
\usemintedstyle{emacs}
\usepackage{tikz}
\usetikzlibrary{matrix,shapes,decorations.pathreplacing,fit,backgrounds}
%
