\PassOptionsToPackage{dvipsnames,table}{xcolor}

\documentclass[sigconf]{acmart}
\usepackage[dvipsnames,table]{xcolor}
\usepackage{booktabs} % For formal tables
\usepackage{tabularx}

\usepackage{multirow}
\usepackage{multicol}
\usepackage{makecell}
\usepackage{balance}
\usepackage[utf8]{inputenc}
\usepackage{kotex}
\usepackage{enumitem}
\usepackage[ruled,lined, linesnumbered]{algorithm2e}
\usepackage{algpseudocode}
\usepackage{threeparttable}
\usepackage{array}
\usepackage{tikz}
\usetikzlibrary{calc}
\usepackage{ragged2e}
\usepackage{svg}

\DeclareMathOperator*{\argmin}{arg\,min}
\DeclareMathOperator*{\argmax}{arg\,max}

\sloppy

\newcommand{\red}[1]{\cellcolor{Maroon!8}\textcolor{BrickRed!30!black}{#1}}
\newcommand{\blue}[1]{\cellcolor{SeaGreen!8}\textcolor{RoyalBlue!30!black}{#1}}
\newcommand{\purple}[1]{\cellcolor{Orchid!8}\textcolor{Plum!30!black}{#1}}

\setlength{\fboxsep}{0.1em}
\newcommand{\textred}[1]{\colorbox{Maroon!8}{\textcolor{BrickRed!30!black}{#1}}}
\newcommand{\textblue}[1]{\colorbox{SeaGreen!8}{\textcolor{RoyalBlue!30!black}{#1}}}
\newcommand{\textpurple}[1]{\colorbox{Orchid!8}{\textcolor{Plum!30!black}{#1}}}

\newlength\mylen
\newcommand\myinput[1]{%
  \settowidth\mylen{\KwIn{}}%
  \setlength\hangindent{\mylen}%
  \hspace*{0.5\mylen}#1\\}
\usepackage{tabto}

\newcommand{\itab}[1]{\hspace{0em}\rlap{#1}}
\newcommand{\ttab}[1]{\hspace{.07\textwidth}\rlap{#1}}
\let\oldnl\nl
\newcommand{\nonl}{\renewcommand{\nl}{\let\nl\oldnl}}
\newcolumntype{P}[1]{>{\centering\arraybackslash}p{#1}}

% Copyright
%\setcopyright{none}
\setcopyright{acmcopyright}
%\setcopyright{acmlicensed}
%\setcopyright{rightsretained}
%\setcopyright{usgov}
%\setcopyright{usgovmixed}
%\setcopyright{cagov}
%\setcopyright{cagovmixed}


% DOI
\acmDOI{10.1145/3672608.3707957}

% ISBN
\acmISBN{978-1-4503-8713-2/22/04}

%Conference
\acmConference[SAC'25]{ACM SAC Conference}{March 31 - April 4, 2025}{Sicily, Italy}  
\acmYear{2025}
\copyrightyear{2025}

%\acmArticle{4}
%\acmPrice{15.00}



\begin{document}
\title[]{UKTA: Unified Korean Text Analyzer}





\author{Seokho Ahn}
\authornote{ Equal contribution.}
\orcid{0000-0002-5715-4057}
\affiliation{%
  \institution{Inha University}
  \city{Incheon} 
  \state{South Korea} 
}
\email{sokho0514@inha.edu}

\author{Junhyung Park}
\authornotemark[1]
\orcid{0009-0000-6714-9684}
\affiliation{%
  \institution{Inha University}
  \city{Incheon} 
  \state{South Korea} 
}
\email{quixote1103@inha.edu}

\author{Ganghee Go}
\authornotemark[1]
\orcid{0009-0006-1027-105X}
\affiliation{%
  \institution{Inha University}
  \city{Incheon} 
  \state{South Korea} 
}
\email{khko99@inha.edu}

\author{Chulhui Kim}
\orcid{0000-0003-4161-2882}
\affiliation{%
  \institution{Inha University}
  \city{Incheon} 
  \state{South Korea} 
}
\email{clearfe@inha.ac.kr}

\author{Jiho Jung}
\orcid{0009-0000-5711-1834}
\affiliation{%
  \institution{Inha University}
  \city{Incheon} 
  \state{South Korea} 
}
\email{jacob\_jjh@inha.ac.kr}

\author{Myung Sun Shin}
\orcid{0009-0003-1832-8987}
\affiliation{%
  \institution{Inha University}
  \city{Incheon} 
  \state{South Korea} 
}
\email{rescript@inha.ac.kr}

\author{Do-Guk Kim}
\authornotemark[2]
\orcid{0000-0002-3564-7002}
\affiliation{%
  \institution{Inha University}
  \city{Incheon} 
  \state{South Korea} 
}
\email{dgkim@inha.ac.kr}


\author{Young-Duk Seo}
\authornote{ Co-corresponding authors.}
\orcid{0000-0001-8542-2058}
\affiliation{%
  \institution{Inha University}
  \city{Incheon} 
  \state{South Korea} 
}
\email{mysid88@inha.ac.kr}


\renewcommand{\shortauthors}{Ahn et al.}


\begin{abstract}
Evaluating writing quality is complex and time-consuming often delaying feedback to learners. While automated writing evaluation tools are effective for English, Korean automated writing evaluation tools face challenges due to their inability to address multi-view analysis, error propagation, and evaluation explainability. To overcome these challenges, we introduce \textsf{UKTA} (\textbf{U}nified \textbf{K}orean \textbf{T}ext \textbf{A}nalyzer), a comprehensive Korea text analysis and writing evaluation system. \textsf{UKTA} provides accurate low-level morpheme analysis, key lexical features for mid-level explainability, and transparent high-level rubric-based writing scores. Our approach enhances accuracy and quadratic weighted kappa over existing baseline, positioning \textsf{UKTA} as a leading multi-perspective tool for Korean text analysis and writing evaluation.
\end{abstract}

%
% The code below should be generated by the tool at
% http://dl.acm.org/ccs.cfm
% Please copy and paste the code instead of the example below. 
%
\begin{CCSXML}
<ccs2012>
   <concept>
       <concept_id>10010405.10010489.10010491</concept_id>
       <concept_desc>Applied computing~Interactive learning environments</concept_desc>
       <concept_significance>500</concept_significance>
       </concept>
   <concept>
       <concept_id>10010147.10010178</concept_id>
       <concept_desc>Computing methodologies~Artificial intelligence</concept_desc>
       <concept_significance>300</concept_significance>
       </concept>
   <concept>
       <concept_id>10010147.10010178.10010179</concept_id>
       <concept_desc>Computing methodologies~Natural language processing</concept_desc>
       <concept_significance>100</concept_significance>
       </concept>
 </ccs2012>
\end{CCSXML}

\ccsdesc[500]{Applied computing~Interactive learning environments}
\ccsdesc[300]{Computing methodologies~Artificial intelligence}
\ccsdesc[100]{Computing methodologies~Natural language processing}


\keywords{Automated writing evaluation, Text analyzer, Morpheme analysis, Lexical feature analysis}

%-----------------------------------------------------
\begin{teaserfigure}
  %\includegraphics[width=1\textwidth]{Fig/teaser.png}
  \includegraphics[width=0.95\textwidth]{teaser.pdf}
  % \includesvg[inkscapelatex=false, width=0.95\textwidth]{Fig/teaser.svg}
  \centering
  \caption{\textsf{UKTA} is a comprehensive Korean text analyzer that provides morpheme analysis, lexical feature analysis, and explainable writing evaluation: \normalfont{\textbf{(A)} Users can input Korean text as a file or paragraph, \textbf{(B)} Display multi-perspective results such as morphemes and lexical features, and \textbf{(C)} Provide explainable, visualized writing evaluation results in the form of rubric scores, along with the top features that contributed to the scores. Users can download these results in various formats, including JSON, TXT, and CSV files.}}
  \label{fig:teaser}
  \vspace{\baselineskip}
\end{teaserfigure}
%-----------------------------------------------------

\maketitle



\begin{figure}[ht]
    \centering
    \includegraphics[width=0.8\linewidth]{graphs/greater_than_naive.pdf}
    \vspace{0.5cm}
    \includegraphics[width=0.8\linewidth]{graphs/p1_bottom.png}
    \vspace{-5pt}
    \caption{\textcolor{positional}{Positional} vs.\ \textcolor{nonpositional}{non-positional} circuits. In a \textcolor{nonpositional}{non-positional} circuit, the same edges must be included at all positions. A \textcolor{positional}{positional} circuit can distinguish between the same edge at different positions. This specificity yields better trade-offs between circuit size and faithfulness. It can also increase both precision and recall.}
    \label{fig:p1}
    \vspace{-5pt}
\end{figure}

\section{Introduction}

\looseness=-1
A primary goal of interpretability research is to characterize the internal mechanisms in language models (LMs) and other NLP models. 
A core approach in this area is \textbf{circuit discovery}---identifying the minimal subgraph within the model's computation graph that performs a specific task \citep{olah2021framework,olah-mech}.
Typically, the nodes of a circuit represent model components (e.g., attention heads, neurons, or layers).
While manual circuit discovery methods can yield position-specific insights \citep{wanginterpretability,goldowskydill2023localizingmodelbehaviorpath}, \emph{automatic methods often overlook positional information}, treating components as uniformly relevant across all input token positions \citep{conmytowards,syed2023attribution}. 
For instance, if an attention head is included in a circuit, it is assumed to contribute equally to the computation for every position in the input sequence.
The assumption that circuits are position-invariant ignores the fact that different positions often require distinct computations.
By ignoring positions, current methods limit their ability to capture mechanisms that operate across positions, such as interactions between attention heads across positions.

In this study, we start by demonstrating that positional agnosticism is a significant limitation (\S\ref{sec:motivating}). Then, to address these limitations, we introduce a new approach: position-aware edge attribution patching (PEAP; \S\ref{sec:full_circ_discovery}; Figure~\ref{fig:p1}). Current approaches  assume that if an edge is in a circuit, then the same edge will be in the circuit at all positions, thus leading to low precision. It is also assumed that an edge's importance should be aggregated across positions before deciding whether it should be included in the circuit; this can lead to cancellation effects, and thus low recall. PEAP instead allows us to compute the importance of cross-positional edges, and separately evaluates edge importance at each position. We show that this leads to smaller and more accurate circuits; see Figure~\ref{fig:p1}.

Incorporating positional information into circuit discovery is straightforward when inputs have the same length and structure across examples.

However, realistic datasets are not nearly this templatic.
How, then, can we incorporate positional information into automatic circuit discovery?
To address this challenge, we propose \textbf{schemas} (\S\ref{sec:schema}). 
Schemas assign semantic labels to spans of tokens, enabling information aggregation across examples even when the spans differ in length.

For example, in the input ``The \textcolor{positional}{war} lasted from 1453 to 14\underline{\hspace{1em}},'' the span ``\textcolor{positional}{war}'' could be labeled as ``\emph{Subject}''.
This enables handling spans with varying lengths: the phrase ``\textcolor{positional}{Black Plague}'' in another example can be treated as a single positional span with the same role as ``\textcolor{positional}{war}''.
In experiments with two LMs and three tasks, we find that circuits discovered using schemas achieve a better trade-off between circuit size and faithfulness to the model's behavior than position-agnostic circuits.
Importantly, position-aware circuits offer a more precise representation of the underlying mechanisms, providing a more concise foundation for mechanistic explanations.

We also present a fully automated pipeline for schema generation and application (\S\ref{sec:schema-generation}) using large language models (LLMs). 
We evaluate the quality of the generated schemas and their utility in discovering position-aware circuits (\S\ref{sec:schema-eval}).
Notably, circuits derived using automatically generated and applied schemas achieve comparable faithfulness scores to circuits discovered with human-designed and manually applied schemas.

We summarize our contributions as follows:
\begin{itemize}[noitemsep,leftmargin=*,topsep=1pt,parsep=1pt]
    \item Introduce a position-aware circuit discovery method, which obtains better faithfulness than position-agnostic discovery.  
    \item Introduce dataset schemas,  facilitating positional circuit discovery in more naturalistic settings. 
    \item Develop an automated schema generation and application pipeline with LLMs, yielding schemas that are comparable to manually-annotated ones.
\end{itemize}

\section{Related Work}

\subsection{Penetration Depth Computation}

The computation of penetration depth often utilizes the Minkowski sum, a well-regarded algorithm documented in Dobkin et al.'s work~\cite{dobkin1993computing}.
This method shows high efficacy for convex shapes, where the simplicity of the objects allows for accurate and computationally efficient penetration depth calculations~\cite{dobkin1993computing,varadhan2004accurate,hachenberger2009exact}.
However, applying this algorithm to concave shapes significantly increases computational complexity.  
As a result, research has focused on developing methods to approximate penetration depth more efficiently for these shapes~\cite{cameron1997enhancing,bergen1999fast,lien2010simple,je2012polydepth}.  

Beyond the Minkowski sum, other methods have been explored, including techniques such as utilizing distance fields or the Hausdorff distance for penetration depth calculations~\cite{fisher2001fast,sud2006fast,SIG09HIST}.

Tang et al.\cite{SIG09HIST} devised an efficient algorithm for calculating the Hausdorff distance between two objects within a given error bound.
They also demonstrated that the proposed algorithm can accelerate penetration depth computation by focusing on the Hausdorff distance in overlapping regions of objects.
Building upon Tang et al.'s method, Zheng et al.\cite{zheng2022economic} improved performance using a BVH-based framework with a four-point strategy.
This method has achieved a performance improvement of up to 20 times compared to Tang et al.'s technique~\cite{SIG09HIST}.
\revision{A common feature of these works, known as the culling-based method, is computing bounds for the Hausdorff distance and reducing the search space.}

\revision{Although culling-based methods have demonstrated significant performance gains, they face challenges in leveraging parallel hardware.  
Updating and sharing bounds require synchronization, which is not well-suited for massively parallel processing architectures such as GPUs.}

\revision{In this work, we propose a GPU-based penetration depth algorithm that specifically accelerates two key processes using RT core technology:  
(1) detecting the overlapping volume and (2) calculating the Hausdorff distance.  
To highlight the effectiveness of our approach, we also implemented a CPU-based penetration depth algorithm based on Tang et al.~\cite{SIG09HIST} and Zheng et al.~\cite{zheng2022economic} for performance comparison.}

%In this work, we propose a GPU-based penetration depth algorithm, specifically accelerating two key processes with RT core technology:  
%first, detecting the overlapping volume; and second, calculating the Hausdorff distance.  
%To highlight our method's effectiveness, we also implemented a CPU-based penetration depth algorithm based on Tang et al.~\cite{SIG09HIST} and Zheng et al.~\cite{zheng2022economic} for performance comparison.  

%utilize a Hausdorff distance-based method for penetration depth calculation, accelerating two key processes with RT core technology: 

%A notable development in this area is the work of Tang et al., who devised algorithms for the rapid calculation of the Hausdorff distance between two objects~\cite{SIG09HIST}.
%Their approach is geared towards efficient penetration depth calculation by focusing on the Hausdorff distance in overlapping object regions.


%One of the algorithms for calculating penetration depth is the Minkowski sum.\cite{dobkin1993computing} The Minkowski sum is useful to compute penetration depth between two convex objects because they have a simple shape so the Minkowski sum can calculate accurate penetration depth with low computational complexity~\cite{dobkin1993computing,varadhan2004accurate,hachenberger2009exact}.
%However, applying the Minkowski sum in cases involving concave objects is challenging due to higher computational complexity. As a result, prior research has focused on quickly computing an approximate penetration depth in these scenarios~\cite{cameron1997enhancing,bergen1999fast,lien2010simple,je2012polydepth}.

%Instead of the Minkowski sum method, there have also been attempts to calculate the penetration depth based on the distance field or the vertices that make up the objects~\cite{fisher2001fast,sud2006fast,SIG09HIST}. Tang et al.~\cite{SIG09HIST} proposed the algorithms that compute the Hausdorff distance between two objects quickly and showed that can be computed penetration depth to fast by calculating the Hausdorff distance for the overlapping area of two objects.

%In this paper, the proposed method is based on Tang's methods~\cite{SIG09HIST}, and then partially divided into steps detecting overlapping volume step and the Hausdorff distance step. These two steps accelerated with RT core.

\subsection{Ray-Tracing Core-Based Acceleration}

\revision{Recent advancements in GPU technology have led to the integration of dedicated ray-tracing cores (RT cores), enabling hardware-accelerated ray tracing.
These cores optimize intersection checks between rays and objects, allowing for efficient ray-bounding box and ray-triangle intersection tests.
To utilize RT cores, various frameworks such as DXR, OptiX~\cite{parker2010optix}, and Vulkan have been developed.
RT cores primarily accelerate ray intersection tasks by efficiently traversing acceleration hierarchies.}

%The Ray-Tracing Core (RT-core) is NVIDIA’s specialized hardware for accelerating ray tracing.
%Integrated into RTX GPUs like the GeForce RTX series

%\revision{Notably, OptiX~\cite{parker2010optix} is an NVIDIA-supported SDK.
%The ray-tracing core primarily facilitates two tasks: building an acceleration hierarchy and executing ray intersection tasks with traversal.}
%OptiX operates by launching a CUDA kernel and invoking a ray generation ($ray_{gen}$) shader.
%Each CUDA core thread makes requests to the ray-tracing core, which then executes appropriate shaders like intersection ($IS$), miss($miss_{hit}$), closest hit($closest$), and any hit($any_{hit}$).
%Consequently, OptiX enables access to the results of ray-primitive intersection tests.

While the core purpose of ray-tracing cores is to expedite ray tracing, recent studies have explored their application beyond this traditional scope~\cite{wald2019rtx,zhu2022rtnn,thoman2022multi,nagarajan2023rt,meneses2023accelerating,morrical2023attribute}.
Wald et al.~\cite{wald2019rtx} addressed the problem of locating points within tetrahedra using ray-tracing cores.
Zhu et al.~\cite{zhu2022rtnn} introduced a K-Nearest Neighbor (K-NN) algorithm utilizing ray-tracing cores, achieving performance improvements of 2.2 to 65.0 times compared to previous GPU-based nearest neighbor search algorithms.
Thoman et al.~\cite{thoman2022multi} employed RT cores for Room Impulse Response (RIR) simulation.
Nagarajan et al.~\cite{nagarajan2023rt} implemented RT core-based DBSCAN clustering, reporting up to 4 times higher performance enhancement.
Meneses et al.~\cite{meneses2023accelerating} proposed RT core-based Range Minimum Query (RMQ) algorithms, yielding performance up to 2.3 times faster than existing RMQ methods.

\revision{
For collision detection between objects, one of the fundamental proximity queries, researchers have explored ray-tracing approaches even before the introduction of RT-core technology.
Hermann et al.\cite{hermann2008ray} proposed ray-tracing-based collision detection methods for deformable bodies.
Youngjun et al.\cite{kim2010mesh} applied Hermann's idea to medical simulation.
Lehericey et al.\cite{lehericey2015gpu} introduced GPU ray-traced collision detection algorithms for cloth simulation.
Recently, these approaches have been extended to utilize RT cores, as demonstrated by Sui et al.\cite{sui2024hardware}, who proposed discrete and continuous collision detection algorithms using ray-tracing cores.
Unlike these works, which focus on determining when and where collisions occur, our work focuses on calculating penetration depth.
}

In line with these advancements, this study uniquely applies RT-core technology to compute penetration depth, diverging from traditional ray-tracing applications and thereby contributing a novel approach to this field.

%\subsection{Collision detection with Ray-tracing}

%\YW{There have been attempts to apply the ray tracing approaches for collision detection~\cite{hermann2008ray, kim2010mesh, lehericey2015gpu}. Hermann et al~\cite{hermann2008ray} proposed ray tracing collision detection methods for deformable bodies. Youngjun et al~\cite{kim2010mesh} apply Hermann's idea for Medical simulation. Lehericey et al~\cite{lehericey2015gpu} introduced GPU ray-traced collision detection algorithms for cloth simulation.
%However, these methods proposed deformable objects, not solid- or discrete- objects, and there is no report about the result using ray tracing core yet. Therefore, our research implements the penetration depth algorithm with ray tracing methods and reports the benefit of ray tracing core.}

%\YW{Sui et al~\cite{sui2024hardware} proposed the method for discrete and continuous collision detection with ray tracing core. They generate the ray candidate as much as the edge of the source mesh and investigate the intersections to solve discrete collision detection. And also, to solve continuous collision detection, they build sphere-swept volumes with OptiX B-Spline curves using continuous trajectory points that are pre-computed and trace the ray samely. However, their implementation only considers non-penetrating collision, and because of that reason, there need for other approaches to compute penetration cases.}

%\YW{To address this issue, our approacthe has propose the methods to find penetration surface with RT core (that called RT-PPE). Not only that, our methods report the penetration depth as computing the Hausdorff distance between the penetration surface.}

%Recently, modern GPU embedded ray tracing core for hardware accelerated ray tracing.

%To access the ray tracing core, we can use DXR, OptiX~\cite{parker2010optix}, and Vulkan.
%Above all, OptiX~\cite{parker2010optix} is NVIDIA NVIDIA-supported SDK. The ray tracing core actually works about two tasks. One is a built acceleration hierarchy, and another is ray intersection task with traversal. Therefore OptiX launches one CUDA kernel and called $ray\_gen$ shader. Each CUDA core thread requests to ray tracing core, and then ray tracing core executes a suitable shader such as $IS$, $miss\_hit$, $closest$, $any\_hit$ shader.
%Finally, we can access ray-primitive intersection test results using OptiX shader.

%While the ray tracing core is designed for accelerating ray tracing, recent research tried using the ray tracing core for other purposes~\cite{wald2019rtx,zhu2022rtnn,thoman2022multi,nagarajan2023rt,meneses2023accelerating,morrical2023attribute}.
%%Beyond ray tracing
%Wald et al~\cite{wald2019rtx} solved the point in location of tetrahedron problem using ray tracing cores.
%%RTNN
%Zhu et al~\cite{zhu2022rtnn} proposed K-NN(K-Nearest Neighbor) algorithms using ray tracing cores. They achieved a performance of 2.2-65.0 times faster than prior GPU-based nearest neighbor search algorithms.
%%RIR Simulation
%Thoman et al~\cite{thoman2022multi} utilized the RT core to RIR(Room impulse response) simulation,
%%RT-DBSCAN
%Nagarajan et al~\cite{nagarajan2023rt} implemented DBSCAN clustering with RT core and achieved performance up to 4x times.
%%RTX-RMQ
%Meneses et al~\cite{meneses2023accelerating} proposed RT core-based RMQ(Range minimum query) algorithms, and they got performance up to 2.3x than state-of-the-art RMQ algorithms.

%%
%Similar to prior research, this study is distinguished by utilizing RT-core for computing penetration depth, as opposed to conventional ray tracing problems.



\section{Unified Korean Text Analyzer}
\label{sec:04_model}
%-----------------------------------------------------
\begin{figure}[t]
\includegraphics[width=1\columnwidth]{model.pdf}
\centering

\caption{Illustrative overview of \textsf{UKTA}.} 
\label{fig:approach}
\end{figure}
%-----------------------------------------------------


This section introduces \textsf{UKTA} (\textbf{U}nified \textbf{K}orean \textbf{T}ext \textbf{A}nalyzer), which sequentially performs low-level morpheme analysis (in Section \ref{sec:04_model_morpheme}), mid-level lexical feature analysis (in Section \ref{sec:04_model_feature}), and high-level automatic writing evaluation (in Section \ref{sec:04_model_scoring}).
The overall process for our tool is illustrated in Figure \ref{fig:approach}.



%-----------------------------------------------------
\begin{figure*}[t]
  \includegraphics[width=1\textwidth]{functionality.pdf}
  \centering
  \caption{\textsf{UKTA} functionality. \normalfont{\textbf{(A) Functionality in morpheme analysis results}: Providing both table (A-1) and list (A-1) format, with an interactive and intuitive intuitive interface; results can be downloaded in JSON and TXT formats (A-3). \textbf{(B) Functionality in lexical feature analysis results}: Provided as categorized lexical features (B-1) with a list format (B-2); results can be downloaded in TXT and CSV format with selected features (B-3).}}
  \label{fig:functions}
  \vspace{\baselineskip}
\end{figure*}
%-----------------------------------------------------


\subsection{Morpheme Analysis\label{sec:04_model_morpheme}}

This section describes the importance of Korean morpheme analysis and outlines the methods.
Morpheme analysis is the first step, low-level analysis before conducting Korean writing evaluation and lexical feature analysis. 
However, due to the nature of Korean, accurately segmenting wordpieces into morpheme units is challenging, as their forms can change due to different suffixes \cite{matteson2018rich}. 
Such errors can propagate to subsequent steps, including lexical feature analysis and writing evaluation. 

For example, morpheme analysis results should differ, `나\tiny{/NP}\normalsize{+는}\tiny{/JX}\normalsize{ (\textit{Na-Neun})' and `날}\tiny{/VV}\normalsize{+는}\tiny{/ETM}\normalsize{ (\textit{Nal-Neun})'}, even for the same type of wordpiece `나는 (\textit{Naneun})'. 
These errors can distort the values of some feature, reducing the reliability of feature analysis and writing evaluation. 
To minimize error propagation, we conducted morpheme analysis based on Bareun\footnote{https://bareun.ai/} analyzer, known for its highest accuracy, ensuring reliable results for subsequent lexical feature analysis and writing evaluation.

Our system provides morpheme analysis results in a clear, intuitive and user-friendly format as illustrated in Figure \ref{fig:functions}(A). It ensures easy interpretation and efficient analysis.


\subsection{Lexical Feature Analysis \label{sec:04_model_feature}} 



This section introduces the process of mid-level feature analysis and describes various features.
After morpheme analysis, diverse features are numerically evaluated based on the morphemes. 
We provide numerical results for 294 features, broadly categorized into three groups: \textit{\textsf{basic lexical features}}, \textit{\textsf{lexical diversity}}, and \textit{\textsf{cohesion}}.
These features not only provide numerical information but also offer explainable insights for subsequent writing evaluation results.
Detailed descriptions of each group are provided below.



\subsubsection*{Basic Features} 
The basic lexical features of a text represent its fundamental linguistic composition. 
These features include measurements such as count, density, and length of morphemes or words. 
Accurate tagging and categorization of morphemes are essential for ensuring the precision of these metrics. 
Additionally, a list of sentences containing each morpheme is provided to clarify their contextual use Figure \ref{fig:teaser}(B-3). 
This detailed examination of basic features offers fundamental insights into the structural and linguistic properties of the text, serving as a basis for calculating more complex features.

\subsubsection*{Lexical Diversity}
Lexical diversity \cite{hout2007richness} 
 is measured based on the degree of connectivity between sentences or paragraphs, reflecting vocabulary depth and linguistic diversity \cite{ha2019lexical, kim2024korcat}.
This measure is evaluated through the calculation of lexical features such as Type-Token Ratio (TTR) and other diversity features. 
We provide each lexical diversity feature for all tokens, as well as for each specific morpheme. An example lexical diversity output is provided in Figure \ref{fig:teaser}(B-3).
A detailed description of the key feature for measuring lexical diversity included in our system is provided as follows.

\begin{itemize}[leftmargin=1.1em]
\item \textsf{\textit{Type-Token Ratios}}: TTR, RTTR (Root TTR), and CTTR (Corrected TTR) are fundamental features for calculating lexical diversity. 
These features tend to decrease as text length increases, making them suitable for comparisons between texts of similar length \cite{joan2013ttr}.
Formally, these features are calculated as:
\begin{equation}
    \text{TTR} = \frac{t}{w}, \quad
    \text{RTTR} = \frac{t}{\sqrt{w}}, \quad
    \text{CTTR} = \frac{t}{\sqrt{2w}}
\end{equation}
where \(t\) and \(w\) are the number of unique morphemes and the total number of morphemes in a given text, respectively. 



\item \textsf{\textit{Equal segmented Type-Token Ratios}}: MSTTR (Mean Segmental TTR) and MATTR (Moving Average TTR) are both extensions of the traditional TTR, designed to address its sensitivity to text length. 
MSTTR calculates the average TTR over equal-length, and non-overlapping segments of a text, which standardizes the measure across different text lengths:
\begin{equation}
    \text{MSTTR} = \frac{1}{k} \sum_{i=1}^{k} \frac{t_i}{w_i} 
\end{equation}
where \(k = \left\lceil \frac{N}{n} \right\rceil\) is the number of non-overlapping segments, with \(N\) as the total number of tokens in the text and \(n\) as the window size. 
Here, \(t_i\) and \(w_i\) denote the number of unique tokens and the total number of morphemes in the \(i\)-th segment, respectively.
MATTR, in contrast, uses a moving window to compute TTR over overlapping segments, providing a more stable and consistent evaluation of lexical diversity across varying text lengths:
\begin{equation}
    \text{MATTR} = \frac{1}{k'} \sum_{i=1}^{k'} \frac{t_i}{w_i} = \frac{1}{k'n} \sum_{i=1}^{k'} t_i
\end{equation}
where \(k' = N-n+1\) is the number of overlapping segments.


\item \textsf{\textit{Textual lexical diversity}}:  
MTLD (Measure of Textual Lexical Diversity) addresses the sensitivity of TTR to text length by measuring how long a sequence of words must be to reach a fixed threshold of TTR decline. 
This approach provides a more stable and length-independent measure of lexical diversity.
In other words, MTLD is determined as the mean length of non-overlapping segments of varying length that satisfies the following:
\begin{equation}
    \text{MTLD} = \frac{N}{K}, \text{ where }\frac{t_i}{w_i} \leq \theta_\textrm{TTR}\text{ for all }1\leq i\leq K 
\end{equation}
\textit{i.e.}, \(K\) is the largest number of segments where the TTR value of each segment is below a predetermined threshold \(\theta_\textrm{TTR}\).

\item \textsf{\textit{Probabilistic lexical diversity}}: HD-D (Hypergeometric Distribution of Diversity) calculates the probability of encountering different morpheme types within randomly sampled subsets of the text, accounting for both morpheme occurrence and text length. 
Formally, HD-D calculates the average probability that each unique morpheme token \(t\) will appear at least once within a random sample of size \(S\):
\begin{equation}
    \text{HDD} = \frac{1}{S} \sum_{t} \left[ 1 - \frac{\binom{N - f_t}{S}}{\binom{N}{S}} \right]
\end{equation}
where \(f_t\) is the number of occurrences of token type \(t\) in the text. 

\item \textsf{\textit{Model-based lexical diversity}}:  
Voc-D estimates the relationship between tokens and types across various sample sizes, adjusting for the text length to provide a robust measure of vocabulary richness. This method addresses the limitations of raw TTR by accounting for variations in text length and complexity:
\begin{equation}
    \text{VOCD} = \argmin_{D} \sum_{n=35}^{50} \left( \overline{\text{TTR}}_n - \frac{D}{D+n} \right)^2
\end{equation}
where $\overline{\text{TTR}}_n$ is the empirical mean TTR of 100 random subsamples for size $n$, $D$ is the VOCD score that minimizes the difference between the empirical TTR values and the theoretical curve \(\frac{D}{D+n}\).
\end{itemize} 




\subsubsection*{Cohesion}

Cohesion assesses topic consistency within a paragraph and the similarity of meanings between sentences \cite{kim2024korcat}.  
First, the topic sentence is identified by comparing the extracted keyword with each sentence in the paragraph (topic consistency), followed by calculating the similarity between the topic sentence and the remaining sentences (sentence similarity). 
\textsf{UKTA} uses KeyBert \cite{maarten2023keybert} for extracting key topics (keywords) and SBERT \cite{nils2019sbert} for measuring sentence similarity. 
Additionally, lexical overlap is used as a measure of cohesion, assessing the shared morphemes between adjacent sentences or paragraphs \cite{crossley2019taaco, kim2024korcat}. 
Two main types of lexical overlap are used: adjacent overlap, which counts the number of overlapping morphemes, and binary adjacent overlap, which only checks for their presence.
A higher lexical overlap indicates a stronger structural similarity in the text.



\subsection{Automatic Writing Evaluation\label{sec:04_model_scoring}} 

This section introduces a high-level automatic writing evaluation process that integrates the previously suggested low- and mid-level lexical features with the existing Korean automated writing evaluation model. 
The automatic writing evaluation task can be formally described as follows: 
Given an essay \( X = \{x_i\}_{i=1}^{N} \), consisting of \( N \) sentences, the objective is to predict 10 evaluation scores \( y_{i=1}^{10} \) corresponding to distinct evaluation criteria (commonly referred to as \textit{rubric}) such as grammar, vocabulary, and consistency. 

The architecture of our automatic writing evaluation model is shown in Figure \ref{fig:approach}(c). 
Previous Korean automated writing evaluation models have primarily focused on raw text, without considering the overall characteristics of the essay \cite{lee2022argument, lee2023pasta}. 
However, our model is capable of training on both the raw essay features (\textit{i.e.}, sentence-level features) and the overall characteristics (\textit{i.e.}, essay-level features), including basic lexical features, lexical diversity, and cohesion.
These essay-level features provide a comprehensive perspective of the essay’s quality. 
These features are derived from the Korean morpheme analyzer, enabling the model to perform accurate essay-level analysis during training. 
This process improves the overall accuracy of the writing evaluation.
Finally, we use the attention weights from the attention layer to emphasize the importance of different essay-level features for each sample, highlighting which features contributed to the model's predictions. 
Unlike previous approaches, we provide multiple-view analysis results using these attention scores, enhancing both the reliability and explainability of the writing evaluation results.

In summary, we utilize both i) sentence-level and ii) essay-level representations, then iii) combining them for a reliable and explainable writing evaluation.
The detailed description of the proposed model is as follows:

\begin{enumerate}[label={{\roman*)}}, leftmargin=1.1em]
\item {\textsf{\textit{Extracting sentence-level representations.} } 
We extract sentence-level representations of the essay using a pre-trained language model and a bidirectional Gated Recurrent Unit (BiGRU), as illustrated in the bottom of Figure \ref{fig:approach}(c).
The process begins by dividing the essay into $N$ sentences, and each sentence is tokenized.
The tokenized sentences are input into KoBERT\footnote{https://github.com/SKTBrain/KoBERT}, a BERT model pre-trained on a large-scale Korean corpus.
This step produces $N$ embedding vectors, denoted as $\mathbf{e}_1, \mathbf{e}_2, \dots, \mathbf{e}_N$.
The resulting sentence-level embedding vectors are subsequently fed into a BiGRU. 
These embedding vectors are then passed through a BiGRU, which computes the final sentence-level representation of the essay using its last hidden state vector, $\mathbf{h} = [\overrightarrow{\mathbf{h}}; \overleftarrow{\mathbf{h}}]$.
}
%
\item {\textsf{\textit{Extracting essay-level representations.} } 
We first extract the lexical features $\mathbf{f} \in \mathbb{R}^{294}$, which consist of 294 values from the raw text. 
These features are normalized using a standard scaler:
\begin{equation}
\mathbf{f'} = \frac{\mathbf{f} - \mu}{\sigma}
\end{equation}
where $\mu$ and $\sigma$ represent the mean and standard deviation calculated from the feature, respectively. 
This normalization process ensures that each feature is on the same scale and comparable. 
The normalized feature vector $\mathbf{f'}$ is then passed through an attention layer, which assigns different importance weights $\mathbf{A}$ to each feature. 
The attention-weighted vector is computed through an element-wise multiplication of the attention weights and the normalized features:
\begin{equation}
\mathbf{f_A} = \mathbf{A} \odot \mathbf{f'}
\end{equation}
where $\odot$ denotes element-wise multiplication. 
This operation emphasizes the most relevant features for the task. 
The attention-weighted vector $\mathbf{f_A}$ is then passed through a dense layer along with $\mathbf{f'}$, which outputs the final essay-level representation $\mathbf{v_e}$.
}
%
\item {\textsf{\textit{Combining sentence- and essay-level representations.} } 
Finally, the generated sentence-level representation vector $\mathbf{h}$ is concatenated with the essay-level representation vector $\mathbf{v_e}$. 
This combined vector is then passed through a linear layer followed by a sigmoid activation function to predict essay scores for 10 evaluation rubric criteria. 
During the training, the mean squared error (MSE) loss function is used.
}

\end{enumerate} 


\section{Experiments}
\label{sec:experiment}

\subsection{Experimental Setup}
\label{sec:exp_setup}
The experiments are mainly conducted on SD1.5 \cite{sd1} and SDXL \cite{sdxl} without refiner. The LRM is first trained on Pick-a-Pic and then used to fine-tune diffusion models through LPO. Unless otherwise specified, we employ \textit{homogeneous optimization}.

\textbf{LRM Training.} We denote the LRM based on SD1.5 and SDXL as LRM-1.5 and LRM-XL, respectively. They are trained on the filtered Pick-a-Pic v1 \cite{pickscore} as clarified in Sec.\;\ref{sec:lrm_train}. The $gs$ in the VFE module is set to 7.5. 
More details are in \cref{sec:experimental_detail}.

\textbf{LPO Training.} The same 4k prompts in SPO are used for the LPO training, randomly sampled from the training set of Pick-a-Pic v1. The DDIM scheduler \cite{ddim} with 20 inference steps is employed. We use all steps for sampling and training, \ie $t\in[0,50,...,900,950]$. The dynamic threshold range $[th_{min}, th_{max}]$ is set to $[0.35, 0.5]$ for SD1.5 and $[0.45, 0.6]$ for SDXL. The $\beta$ in Eqn.\;(\ref{eq:spo_loss}) is set to 500 and the $K$ in the sampling process is set to 4. Further details can be found in \cref{sec:experimental_detail}.

\begin{table}[t]
    \centering
    \vspace{-2.5mm}
    \caption{General and aesthetic preference scores on Pick-a-Pic validation unique set. $^*$ denotes the metrics are copied from \cite{spo}. Others are evaluated using the official model.}
    \vskip 0.05in
    \label{tab:preferenece_eval}
    \scriptsize
    \setlength{\tabcolsep}{1.0mm}{
    \scalebox{1.1}{
    \begin{tabular}{l c c c c c}
         \toprule
         Method & PickScore & ImageReward & HPSv2 & HPSv2.1 & Aesthetic \\
         \midrule
         \textcolor{gray}{SD1.5} & & & & & \\
         \hspace{1pt} Original & 20.56 & 0.0076 & 26.46 & 24.05 & 5.468 \\
         \hspace{1pt} $^*$DDPO & 21.06 & 0.0817 & - & 24.91 & 5.591 \\
         \hspace{1pt} $^*$D3PO & 20.76 & -0.1235 & - & 23.97 & 5.527 \\
         \hspace{1pt} Diff.-DPO & 20.99 & 0.3020 & 27.03 & 25.54 & 5.595 \\
         \hspace{1pt} SPO & 21.22 & 0.1678 & 26.73 & 25.83 & 5.927 \\
         \rowcolor{cyan!15}\hspace{1pt} LPO & \textbf{21.69} & \textbf{0.6588} & \textbf{27.64} & \textbf{27.86} & \textbf{5.945} \\
         \midrule
         \textcolor{gray}{SDXL} & & & & & \\
         \hspace{1pt} Original & 21.65 & 0.4780 & 27.06 & 26.05 & 5.920 \\
         \hspace{1pt} Diff.-DPO & 22.22 & 0.8527 & 28.10 & 28.47 & 5.939 \\
         \hspace{1pt} MaPO & 21.89 & 0.7660 & 27.61 & 27.44 & 6.095 \\
         \hspace{1pt} SPO & 22.70 & 0.9951 & 28.42 & 31.15 & 6.343 \\
         \rowcolor{cyan!15}\hspace{1pt} LPO & \textbf{22.86} & \textbf{1.2166} & \textbf{28.96} & \textbf{31.89} & \textbf{6.360} \\
         \bottomrule
    \end{tabular}}}
    % \vspace{-3mm}
    \vskip -0.15in
\end{table}


\begin{table*}[t]
    \vspace{-2.5mm}
    \caption{Quantitative results on T2I-CompBench++ \cite{t2i_compbench}.}
    \vskip 0.05in
    \label{tab:t2i_eval}
    \centering
    \scriptsize
    \setlength{\tabcolsep}{1.8mm}{
    \scalebox{1.1}{
    \begin{tabular}{c l c c c c c c c c}
         \toprule
         Model & Method & Color & Shape & Texture & 2D-Spatial & 3D-Spatial & Numeracy & Non-Spatial & Complex \\
         \midrule
         \multirow{4}{*}{SD1.5} & Original \cite{sd1} & 0.3783 & 0.3616 & 0.4172 & 0.1230 & 0.2967 & 0.4485 & 0.3104 & 0.2999 \\
         & Diff.-DPO \cite{diffusion_dpo} & 0.4090 & 0.3664 & 0.4253 & 0.1336 & 0.3124 & 0.4543 & \textbf{0.3115} & 0.3042 \\
         & SPO \cite{spo} & 0.4112 & 0.4019 & 0.4044 & 0.1301 & 0.2909 & 0.4372 & 0.3008 & 0.2988 \\
         & \cellcolor{cyan!15}LPO & 
         \cellcolor{cyan!15}\textbf{0.5042} &
         \cellcolor{cyan!15}\textbf{0.4522} & 
         \cellcolor{cyan!15}\textbf{0.5259} & 
         \cellcolor{cyan!15}\textbf{0.1928} & 
         \cellcolor{cyan!15}\textbf{0.3562} & 
         \cellcolor{cyan!15}\textbf{0.4845} & 
         \cellcolor{cyan!15}0.3110 &
         \cellcolor{cyan!15}\textbf{0.3308}\\
         \midrule
         \multirow{5}{*}{SDXL} & Original \cite{sdxl} & 0.5833 & 0.4782 & 0.5211 & 0.1936 & 0.3319 & 0.4874 & 0.3137 & 0.3327 \\
         & Diff.-DPO \cite{diffusion_dpo} & 0.6941 & 0.5311 & 0.6127 & 0.2153 & 0.3686 & 0.5304 & \textbf{0.3178} & 0.3525 \\
         & MaPO \cite{mapo} & 0.6090 & 0.5043 & 0.5485 & 0.1964 & 0.3473 & 0.5015 & 0.3154 & 0.3229 \\
         & SPO \cite{spo} & 0.6410 & 0.4999 & 0.5551 & 0.2096 & 0.3629 & 0.4931 & 0.3098 & 0.3467 \\
         & \cellcolor{cyan!15}LPO & 
         \cellcolor{cyan!15}\textbf{0.7351} & 
         \cellcolor{cyan!15}\textbf{0.5463} & \cellcolor{cyan!15}\textbf{0.6606} &
         \cellcolor{cyan!15}\textbf{0.2414} &
         \cellcolor{cyan!15}\textbf{0.4075} &
         \cellcolor{cyan!15}\textbf{0.5493} &
         \cellcolor{cyan!15}0.3152 &
         \cellcolor{cyan!15}\textbf{0.3801}\\
         \bottomrule
    \end{tabular}}}
    \vspace{-2mm}
    % \vskip -0.1in
\end{table*}


\begin{table*}[t]
    \begin{minipage}{0.63\linewidth}
        \vspace{-2mm}
        \caption{Quantitative results on GenEval \cite{geneval}.}
        \vskip 0.05in
        \label{tab:geneval}
        \centering
        \scriptsize
        \setlength{\tabcolsep}{1.1mm}{
        \scalebox{1.1}{
        \begin{tabular}{l l c c c c c c c}
             \toprule
             Model & Method & \makecell[c]{Single \\ Object} & \makecell[c]{Two \\ Object} & Counting & Colors & Position & \makecell[c]{Color \\ Attribution} & Overall \\
             \midrule
             \multirow{4}{*}{SD1.5} & Original & 97.50 & 37.12 & 34.69 & 75.53 & 3.75 & 6.75 & 42.56 \\
             & Diff.-DPO & \textbf{98.44} & 38.38 & 36.25 & 77.93 & 4.50 & 7.25 & 43.79 \\
             & SPO & 95.00 & 33.84 & 32.50 & 69.95 & 4.25 & 7.25 & 40.46 \\
             & \cellcolor{cyan!15}LPO & \cellcolor{cyan!15}97.81 &
             \cellcolor{cyan!15}\textbf{54.80}&
             \cellcolor{cyan!15}\textbf{40.94}&
             \cellcolor{cyan!15}\textbf{79.52}&
             \cellcolor{cyan!15}\textbf{7.00}& 
             \cellcolor{cyan!15}\textbf{10.25}&
             \cellcolor{cyan!15}\textbf{48.39}\\
             \midrule
             \multirow{5}{*}{SDXL} & Original & 93.75 & 63.38 & 30.94 & 80.05 & 9.25 & 19.00 & 49.40  \\
             & Diff.-DPO & 99.06 & 76.52 & \textbf{45.00} & 88.83 & 11.50 & 25.75 & 57.78 \\
             & MaPO & 95.63 & 68.94 & 32.19 & 83.51 & 11.50 & 17.75 & 51.59 \\
             & SPO & 94.38 & 69.44 & 31.88 & 81.65 & 10.25 & 15.50 & 50.52  \\
             & \cellcolor{cyan!15}LPO & \cellcolor{cyan!15}\textbf{99.69} &
             \cellcolor{cyan!15}\textbf{81.57} &
             \cellcolor{cyan!15}43.75 &
             \cellcolor{cyan!15}\textbf{89.10} &
             \cellcolor{cyan!15}\textbf{14.00} &
             \cellcolor{cyan!15}\textbf{27.50} & 
             \cellcolor{cyan!15}\textbf{59.27}\\
             \bottomrule
        \end{tabular}}}
        \vskip -0.1in
    \end{minipage}
    \hfill
    \begin{minipage}{0.35\linewidth}
        \vspace{-2mm}
        \caption{Comparisons of training speed.}
        \vskip 0.05in
        \label{tab:speed}
        \centering
        % \footnotesize
        \scriptsize
        \setlength{\tabcolsep}{1.1mm}{
        \scalebox{1.1}{
        \begin{tabular}{l c c c}
             \toprule
             Method & \makecell[c]{Reward \\ Modeling} & \makecell[c]{Preference \\ Optimization} & \makecell[c]{Total $\downarrow$ \\ (A100 h)} \\
             \midrule
             \textcolor{gray}{SD1.5} \\
             \hspace{1pt} Diff.-DPO & 0 & 240 & 240 \\
             \hspace{1pt} SPO & 32 & 48 & 80 \\
             \hspace{1pt} \cellcolor{cyan!15}LPO & \cellcolor{cyan!15}\textbf{15} & \cellcolor{cyan!15}\textbf{8} & \cellcolor{cyan!15}\textbf{23} \\
             \midrule
             \textcolor{gray}{SDXL} \\
             \hspace{1pt} Diff.-DPO & 0 & 2,560 & 2,560 \\
             \hspace{1pt} SPO & 116 & 118 & 234 \\
             \hspace{1pt} \cellcolor{cyan!15}LPO & \cellcolor{cyan!15}\textbf{52} & \cellcolor{cyan!15}\textbf{40} & \cellcolor{cyan!15}\textbf{92} \\
             \bottomrule
        \end{tabular}}}
        \vskip -0.1in
    \end{minipage}
    \vspace{-0.8mm}
\end{table*}

\begin{table}[t]
    \centering
    \vspace{-2mm}
    \caption{Heterogeneous optimization based on LRM-SD1.5. P-S and I-R denote the PickScore and ImageReward metrics.}
    \vskip 0.05in
    \label{tab:sd15_for_sd21}
    \scriptsize
    \setlength{\tabcolsep}{1.0mm}{
    \scalebox{1.0}{
    \begin{tabular}{c c c c c c c c}
         \toprule
         Model & Method & Aesthetic & GenEval & P-S & I-R & HPSv2 & HPSv2.1\\
         \midrule
         SD2.1 & Original & 5.673 & 48.59 & 20.92 & 0.3063 & 27.05 & 25.49 \\
         \tiny(Same VAE) & \cellcolor{cyan!15}LPO & \cellcolor{cyan!15}\textbf{5.969} & \cellcolor{cyan!15}\textbf{56.01}  & \cellcolor{cyan!15}\textbf{21.76} & \cellcolor{cyan!15}\textbf{0.7978} & \cellcolor{cyan!15}\textbf{28.05} & \cellcolor{cyan!15}\textbf{28.61} \\
         \midrule
         SDXL & Original & 5.920 & \textbf{49.40} & \textbf{21.65} & \textbf{0.4780} & 27.06 & 26.05\\
         \tiny(Diff. VAE) & \cellcolor{cyan!15}LPO & \cellcolor{cyan!15}\textbf{5.953} & \cellcolor{cyan!15}40.85 & \cellcolor{cyan!15}20.82 & \cellcolor{cyan!15}0.3919 & \cellcolor{cyan!15}\textbf{27.10} & \cellcolor{cyan!15}\textbf{26.69} \\
         \bottomrule
    \end{tabular}}}
    % \vspace{-2mm}
    \vskip -0.15in
\end{table}


\textbf{Baseline Methods.} We compare LPO with DDPO \cite{ddpo}, D3PO \cite{d3po}, Diffusion-DPO \cite{diffusion_dpo}, MaPO \cite{mapo}, and SPO \cite{spo}. These methods are trained on similar datasets, such as Pick-a-Pic v1 and v2, to ensure a fair comparison. Details are provided in \cref{sec:experimental_detail}.


\textbf{Evaluation Protocol.} We evaluate various diffusion models across three dimensions: general preference, aesthetic preference, and text-image alignment. The PickScore \cite{pickscore}, HPSv2 \cite{hpsv2}, HPSv2.1 \cite{hpsv2}, and ImageReward \cite{imagereward} are utilized to assess the general preference. The aesthetic preference is evaluated using the Aesthetic Score \cite{aesthetic}. Consistent with \cite{spo}, both general and aesthetic preferences are assessed on the validation unique split of Pick-a-Pic v1, which has 500 different prompts. For text-image alignment, we employ the GenEval \cite{geneval} and T2I-CompBench++ \cite{t2i_compbench} metrics. All images are generated using the DDIM scheduler with 20 steps. Additionally, to assess the correlations between the LRM and aesthetics as well as text-image alignment, we propose two corresponding metrics. Specifically, we calculate the score gaps $G_*,*\in\{A,C,L\}$ between winning and losing images, where $A$, $C$, $L$ represent Aesthetic, CLIP, and LRM. For LRM, the score is taken at $t=0$. Then the Pearson Correlation Coefficient \cite{pearson} between $G_L$ and $G_A$ is referred to as \textit{Aes-Corr} while that between $G_L$ and $G_C$ is termed \textit{CLIP-Corr}. They are evaluated on the validation unique and test unique splits of Pick-a-Pic v1.

\subsection{Main Results}


\textbf{Quantitative Comparison.} As indicated in Tab.\;\ref{tab:preferenece_eval}, Tab.\;\ref{tab:t2i_eval}, and Tab.\;\ref{tab:geneval}, Diffusion-DPO excels in enhancing the text-image alignment, while SPO focuses more on aesthetics. LPO outperforms both methods across three dimensions, achieving higher Aesthetic Scores and superior performance on T2I-CompBench++ and GenEval metrics, leading to improved general preference scores. The user study results indicate similar findings, as discussed in \cref{sec:add_exp}. Notably, the LPO-optimized SD1.5 even exhibits performance comparable to the original SDXL model across various metrics.  We further validate the effectiveness of \textit{heterogeneous optimization} in Tab.\;\ref{tab:sd15_for_sd21}. SD1.5 and SD2.1 \cite{sd1} share the same VAE encoder, but SD1.5 has a smaller text encoder. Remarkably, fine-tuning SD2.1 using LRM-1.5 still yields significant improvements across various aspects, demonstrating that a smaller and inferior diffusion model can effectively fine-tune a larger and more advanced model as long as they share the same VAE encoder. In contrast, applying LRM-1.5 for the LPO of SDXL is ineffective due to the distribution mismatch in their latent spaces, which arises from differences in their VAE encoders.

\textbf{Qualitative Comparison.} The qualitative comparisons of various methods are illustrated in Fig.\;\ref{fig:main_comparison} and Fig.\;\ref{fig:vis_15_1}-Fig.\;\ref{fig:vis_xl_4}. The images generated by Diffusion-DPO exhibit deficiencies in color and detail, whereas those produced by SPO demonstrate lower semantic relevance. Additionally, SPO's excessive focus on aesthetics may lead to an overabundance of details in some images, making them appear cluttered. In contrast, the images produced by LPO achieve a strong balance between text-image alignment and aesthetic quality, delivering a higher overall image quality.


\textbf{Training Efficiency Comparison.} LPO achieves significantly faster training speed. As shown in Tab.\;\ref{tab:speed}, considering the time required for both reward modeling and preference optimization, LPO requires only 23 A100 hours for SD1.5---just 1/10 of the training time needed for Diffusion-DPO and 1/3.5 of that for SPO. For SDXL, LPO's training time is reduced to 1/28 and 1/2.5 of that for Diffusion-DPO and SPO, respectively. This efficiency is primarily due to LPO performing reward modeling and preference optimization directly in the latent space, avoiding the additional computational overhead of converting to pixel space.

\subsection{Ablation Studies}
\label{sec:ablation_study}
If not specified, ablation experiments are conducted on SD1.5. Due to space limitations, we only use PickScore to reflect general preference in Tab.\;\ref{tab:ablation_data} and Tab.\;\ref{tab:ablation_lrm}.


\textbf{MPCF.} As shown in Tab.\;\ref{tab:ablation_data}, MPCF plays a critical role in LRM training. As discussed in Sec.\;\ref{sec:lrm_train}, the inconsistent preference issue makes training on the full dataset (wo MPCF) ineffective, since it hinders the LRM from adequately focusing on aesthetics or text-image alignment, resulting in inferior LPO performance. On the other hand, different filtering strategies can profoundly impact the preference patterns of both the LRM and LPO-optimized models. The first filtering strategy strictly requires that winning images score higher than losing images across all aspects. However, since the diffusion model lacks explicit text-image alignment pre-training like CLIP, it is prone to overfitting to the visual features of the images, as indicated by a higher Aes-Corr. This overfitting results in reduced attention to alignment, as reflected by lower CLIP-Corr and GenEval scores. The second and third strategies relax the aesthetic constraints to varying degrees. However, excessively lenient constraints (the 3rd strategy) may cause LRM to focus solely on text-image alignment while neglecting image quality, resulting in a negative Aes-Corr. In contrast, the second strategy balances these two aspects better, leading to the highest general preference scores.


\begin{table}[t]
    \centering
    \vspace{-2.5mm}
    \caption{Ablation results on MPCF of LRM's training data. The second strategy balances aesthetics and alignment better.}
    \vskip 0.05in
    \label{tab:ablation_data}
    \scriptsize
    \setlength{\tabcolsep}{1.0mm}{
    \scalebox{1.1}{
    \begin{tabular}{c c c c c c}
         \toprule
         \multirow{2}{*}{Strategy} & \multicolumn{2}{c}{LRM} & \multicolumn{3}{c}{LPO} \\
         \cmidrule(lr){2-3} \cmidrule(lr){4-6}
          & Aes-Corr & CLIP-Corr & Aesthetic & GenEval & PickScore \\
         \midrule
         wo MPCF & 0.1342 & 0.2274 & 5.772 & 45.66 & 21.49 \\
         1 & \textbf{0.4860} & 0.1011 & \textbf{6.390} & 45.77 & \underline{21.61} \\
         \rowcolor{cyan!15}2 & 0.1136 & 0.3588 & \underline{5.945} & \underline{48.39} & \textbf{21.69} \\
         3 & -0.1152 & \textbf{0.4480} & 5.750 & \textbf{48.62} & 21.47 \\
         \bottomrule
    \end{tabular}}}
    % \vspace{-2mm}
    \vskip -0.1in
\end{table}


\begin{table}[t]
    \centering
    \vspace{-2mm}
    \caption{Ablation results on the VFE module of LRM. Introducing VFE leads to better alignment and general preferences.}
    \vskip 0.05in
    \label{tab:ablation_lrm}
    \scriptsize
    \setlength{\tabcolsep}{1.0mm}{
    \scalebox{1.1}{
    \begin{tabular}{c c c c c c c }
         \toprule
         \multirow{2}{*}{VFE} & \multirow{2}{*}{$gs$} & \multicolumn{2}{c}{LRM} & \multicolumn{3}{c}{LPO} \\
         \cmidrule(lr){3-4} \cmidrule(lr){5-7}
          &  & Aes-Corr & CLIP-Corr & Aesthetic & GenEval & PickScore\\
         \midrule
         \xmark & 1.0 & \textbf{0.1712} & 0.3211 & \textbf{6.053} & 46.60 & 21.51  \\
         \cmark & 3.0 & 0.1233 & 0.3441 & 5.923 & 47.35 & 21.53 \\
         \rowcolor{cyan!15}\cmark & 7.5 & 0.1136 & 0.3588 & \underline{5.945} & \textbf{48.39} & \textbf{21.69}\\
         \cmark & 10.0 & 0.1063 & \textbf{0.3592} & 5.937 & \underline{48.13} & \underline{21.56}\\
         \bottomrule
    \end{tabular}}}
    % \vspace{-2mm}
    \vskip -0.1in
\end{table}


\begin{table}[t]
    \centering
    \vspace{-2.5mm}
    \caption{Ablation results on optimization timestep ranges in LPO.}
    \vskip 0.05in
    \label{tab:ablation_timestep}
    \scriptsize
    \setlength{\tabcolsep}{1.0mm}{
    \scalebox{1.1}{
    \begin{tabular}{c c c c c c c}
         \toprule
         Range of $t$ & Aesthetic & GenEval & P-S & I-R & HPSv2 & HPSv2.1 \\
         \midrule
         \texttt{[}0, 200\texttt{]} & 5.434 & 40.11 & 20.46 & -0.0987 & 26.25 & 23.61 \\
         \texttt{[}250, 450\texttt{]} & 5.527 & 43.00 & 20.76 & 0.1430 & 26.90 & 25.37 \\
         \texttt{[}500, 700\texttt{]} & 5.742 & 44.44 & 20.95 & 0.1591 & 26.71 & 25.16\\
         \texttt{[}750, 950\texttt{]} & \underline{5.853} & \underline{48.28} & 
         \underline{21.54} & \underline{0.6337} & \underline{27.47} & \underline{27.64} \\
         \midrule
         \texttt{[}0, 450\texttt{]} & 5.573 & 42.71 & 20.63 & 0.0204 & 26.69 & 24.88 \\
         \texttt{[}0, 700\texttt{]} & 5.765 & 44.93 & 21.02 & 0.3087 &  27.10 & 26.25\\
         \rowcolor{cyan!15}\texttt{[}0, 950\texttt{]} & \textbf{5.945} & \textbf{48.39} & \textbf{21.69} & \textbf{0.6588} & \textbf{27.64} & \textbf{27.86} \\
         \bottomrule
    \end{tabular}}}
    % \vspace{-2mm}
    \vskip -0.1in
\end{table}


\begin{table}[t]
    \centering
    \vspace{-2mm}
    \caption{Ablation results on different threshold strategies.}
    \vskip 0.05in
    \label{tab:ablation_threshold}
    \scriptsize
    \setlength{\tabcolsep}{1.0mm}{
    \scalebox{1.1}{
    \begin{tabular}{c c c c c c c }
         \toprule
          Threshold & Aesthetic & GenEval & P-S & I-R & HPSv2 & HPSv2.1\\
         \midrule
         0.3 & 5.853 & 46.75 & 21.22 & 0.5112  & 27.30 & 27.12 \\ 
         0.4 & 5.832 & 48.32 & 21.32 & 0.4789 & 27.08 & 26.37 \\
         0.5 & 5.900 & 48.39 & 21.57 & 0.6088 & 27.54 & \underline{27.42} \\
         0.6 & 5.877 & 47.97 & 21.35 & 0.5510 & 27.25 & 26.73 \\
         \midrule
         \texttt{[}0.3, 0.45\texttt{]} & \underline{5.916} & \textbf{49.43} & \underline{21.58} & \underline{0.6405} & \underline{27.55} & 27.33\\
         \rowcolor{cyan!15}\texttt{[}0.35, 0.5\texttt{]} & \textbf{5.945} & 48.39 & \textbf{21.69} & \textbf{0.6588} & \textbf{27.64} & \textbf{27.86} \\
         \texttt{[}0.4, 0.55\texttt{]} & 5.882 & \underline{48.77} & 21.48 & 0.4791 & 27.30 & 27.13\\
         \bottomrule
    \end{tabular}}}
    % \vspace{-2mm}
    \vskip -0.1in
\end{table}


\textbf{Structure of LRM.} As illustrated in Tab.\;\ref{tab:ablation_lrm}, the introduction of VFE ($gs>1$) leads to lower Aes-Corr values but higher CLIP-Corr values, indicating an enhanced emphasis on text-image alignment. This results in improvements in both the GenEval score and PickScore, with only a minor decline in the Aesthetic Score. As $gs$ increases, the LRM's correlation with alignment steadily improves, while its correlation with aesthetics decreases. When $gs$ is set to 7.5, the model achieves the best overall performance.

\textbf{Optimization Timesteps.} Tab.\;\ref{tab:ablation_timestep} ablates different optimization timestep ranges, indicating that larger timesteps lead to better performance. The results achieved within the range of $[750, 950]$ are nearly comparable to those achieved through optimization across the entire denoising process, \ie $[0,950]$. We suggest this is because diffusion models focus on low-frequency information, such as image layout and style, during larger timesteps, while emphasizing high-frequency texture details during smaller timesteps. The low-frequency components formed in higher timesteps play a decisive role in determining the overall quality of the generated images. This observation also demonstrates the effectiveness of LRM, even in very large timesteps. The qualitative comparison of different ranges is shown in Fig.\;\ref{fig:vis_timestep}.

\textbf{Dynamic Sampling Threshold.} The standard deviation $\sigma_t$ of samples at smaller timesteps is relatively small according to the DDPM scheduling \cite{ddpm}, making the constant threshold insufficient to accommodate all timesteps. As indicated in Tab.\;\ref{tab:ablation_threshold}, the dynamic threshold strategy generally outperforms the constant threshold across different intervals, effectively alleviating this problem. We further explore other dynamic strategies in \cref{sec:add_exp}.


\section{Conclusion and future work}
In this study, we examined the ability of LLMs to produce self-generated counterfactual explanations (SCEs).
We design a prompt-based setup for evaluating the efficacy of \SCEs.
Our results show that LLMs consistently struggle with generating valid \SCEs. In many cases model prediction on a \SCE does not yield the same target prediction for which the model crafted the \SCE.
Surprisingly, we find that LLMs put significant emphasis on the context---the prediction on \SCE is significantly impacted by the presence of original prediction and instructions for generating the \SCE.
Based on this empirical evidence, we argue that LLMs are still far from being able to explain their own predictions counterfactually.
Our findings add to similar insights from recent studies on other forms of self-explanations~\cite{lanham2023measuring,tanneru2024quantifying}.



Our work opens several avenues for future work. Inspired by counterfactual data augmentation~\cite{sachdeva2023catfood}, one could include the counterfactual explanation capabilities a part of the LLM training process. This inclusion may enhance the counterfactual reasoning capabilities of the LLM. Follow ups should also explore the effect of prompt tuning, specifically, model-tailored prompts for generating \SCEs. These approaches might lead to better quality \SCEs.


We limited our investigation to open source models of upto 70B parameters. Extending our analysis to larger and more recent models, \eg, DeepSeek R1 671B, and closed source models like OpenAI o3 would be an interesting avenue for future work.

Finally, our experiments were limited to relatively simple tasks: classification and mathematics problems where the solution is an integer. This limitation was mainly due to the fact that it is difficult to automatically judge validity of answers for more open-ended language generation tasks like search and information retrieval. Scaling our analysis to such tasks would require significant human-annotation resources, and is an important direction for future investigations.





\section*{Acknowledgements}
This work was supported in part by the National Research Foundation of Korea (NRF) grant funded by the Korea government (MSIT) (No. NRF-2022R1C1C1012408), in part by Institute of Information \& communications Technology Planning \& Evaluation (IITP) grants funded by the Korea government (MSIT) (No. 2022-0-00448/RS-2022-II220448, Deep Total Recall: Continual Learning for Human-Like Recall of Artificial Neural Networks, and No. RS-2022-00155915, Artificial Intelligence Convergence Innovation Human Resources Development (Inha University)), and in part by the Ministry of Education of the Republic of Korea and the National Research Foundation of Korea (No. NRF-2023S1A5A2A21085373). 


\balance
\bibliographystyle{ACM-Reference-Format}
\bibliography{main.bib} 


\end{document}
