\section{Conclusion}
\label{sec:06_conclusion}

In this work, we propose \textsf{UKTA}, a comprehensive Korean text analysis and writing evaluation system designed for practical use. Unlike existing Korean writing evaluation tools, \textsf{UKTA} provides an automated writing evaluation score along with analyses such as morpheme analysis, lexical diversity features, and cohesion, enhancing the evaluation explainability and reliability. Additionally, our proposed evaluation model based on lexical features outperforms baselines relying solely on raw text data. By analyzing the importance of each feature, \textsf{UKTA} demonstrates its ability to identify the factors influencing Korean writing evaluation.

\textsf{UKTA} opens new possibilities for Korean writing evaluation, offering educators transparent scoring metrics and researchers deeper insights into language features. Future studies could explore its application in diverse educational settings or further refine its feature set to enhance accuracy and adaptability. With its comprehensive approach, \textsf{UKTA} is poised to become a reliable tool for both academic research and practical writing assessment.
