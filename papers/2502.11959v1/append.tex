\newtcolorbox{casebox}{
    colframe=black!50,    % Set border color
    colback=white,       % Background color inside the box
    coltitle=black,      % Title color (if you add a title)
    boxrule=0.4mm,       % Thicker border for a cleaner look
    arc=1mm,             % Slightly smaller rounded corners
    top=1mm,             % Add a little more space at the top
    bottom=1mm,          % Add a little more space at the bottom
    left=1.6mm,            % Adjusting side padding
    right=1.6mm,           % Adjusting side padding
    before skip=2mm,     % Add some vertical space before the box
    after skip=2mm,      % Add space after the box
    % fontupper=\small
}



\section{Appendix}

\subsection{Prompts for Baseline Models}
\label{sec:appendix_prompt_baseline}
We incorporate various prompt-based methods in our experiments to ensure a fair comparison. To maintain consistency, we keep most of the prompt content similar to that used for \themodel in Section~\ref{sec:ourprompt}. Below, we list the prompts used for the baseline models.

For the zero-shot and LoRA fine-tuning experiments, we use the following prompt:
\begin{mybox}
\textit{Based on the evidence, determine if the claim is supported by the evidence or refuted by it.\\
Claim: \texttt{[claim text $c$]}\\
Evidence: \texttt{(1)[evidence text $e_1$](2)...}}\\
\textit{Please respond with only whether the claim is ``Supported'' or ``Refuted.''}
\end{mybox}

For the zero-shot + CoT and STaR* experiments, the prompt is as follows:
\begin{mybox}
\textit{Based on the evidence, determine if the claim is supported by the evidence or refuted by it.\\
Claim: \texttt{[claim text $c$]}\\
Evidence: \texttt{(1)[evidence text $e_1$](2)...}}\\
\textit{Think step by step, output your response in the following format:\\
Chain: \texttt{[your reasoning chain]}\\
Answer:\texttt{[the claim is supported or the claim is refuted]}}
\end{mybox}

For the Few-shot experiment, the prompt is similar to the zero-shot prompt but includes examples:
\begin{mybox}
\textit{Based on the evidence, determine if the claim is supported by the evidence or refuted by it.\\
Please respond with only whether the claim is ``Supported'' or ``Refuted.'' Here are some examples:\\ 
Claim: Simon Grundel-Helmfelt is most ...\\
Evidence: (1) Baron Simon Grundel ... (2)...
Output: Refuted\\
\\(...more examples...)\\\\
Follow the above examples:\\
Claim: \texttt{[claim text $c$]}\\
Evidence: \texttt{(1)[evidence text $e_1$](2)...}\\
Output:
}
\end{mybox}

For the few-shot + structured CoT, we use the same prompt as \themodel, but with examples of structured reasoning:
\begin{mybox}
\textit{Based on the evidence, determine if the claim is supported by the evidence or refuted by it. Output the reasoning chain. Here are some examples:\\ 
Claim: Simon Grundel-Helmfelt is most ...\\
Evidence: (1) Baron Simon Grundel ... (2)...
Chain: C1: Simon Grundel ... \\
\\(...more examples...)\\\\
Follow the above examples:\\
Claim: \texttt{[claim text $c$]}\\
Evidence: \texttt{(1)[evidence text $e_1$](2)...}\\
Chain:
}
\end{mybox}

\subsection{Prompts with Hint}\label{sec:prompthint}
In \themodel, we add hint and regenerate reasoning chains for the ones that falsely predict the label of the claim. If the truth label is $p=\textit{Supported}$, we use prompt:
\begin{mybox}
\textit{Based on the evidence, determine if the claim is supported by the evidence or refuted by it. Output the reasoning chain.\\
Claim: \texttt{[claim text $c$]}\\
Evidence: \texttt{(1)[evidence text $e_1$](2)...}}\\
\textit{Hint: Every detail in this claim is supported.}
\end{mybox}
If the truth label is $p=\textit{Refuted}$, we use prompt:
\begin{mybox}
\textit{Based on the evidence, determine if the claim is supported by the evidence or refuted by it. Output the reasoning chain.\\
Claim: \texttt{[claim text $c$]}\\
Evidence: \texttt{(1)[evidence text $e_1$](2)...}}\\
\textit{Hint: The claim should be refuted, locate the error in the reasoning chain.}
\end{mybox}

\subsection{Dataset Statistics}\label{sec:datasetinfo}
The following table presents the information of the dataset (validation set) that we have tested on. ``HV'' represents HOVER, while ``FS'' stands for FEVEROUS.
\begin{table}[h]  
  \centering  
  \resizebox{\columnwidth}{!}{  
  \setlength{\tabcolsep}{3pt}
    \begin{tabular}{ccccc}  
    \toprule
    \textbf{Dataset} & \textbf{HV-2} & \textbf{HV-3} & \textbf{HV-4} & \textbf{FS-S} \\
    \midrule  
    Total  & 1,126 & 1,835 & 1,039 & 2,962 \\
    Supported Claims & 521 & 968 & 511 & 1411 \\
    Refuted Claims & 605 & 867 & 528 & 1551\\
    Avg. Words in Claim & 19.6 & 24.1 & 32.2 & 27.6 \\
    Avg. Evidence Pieces & 2.0 & 3.0 & 4.0 & 2.1 \\
    Avg. Words in Evidence & 137.3 & 211.1 & 278.2 & 50.3 \\
    \bottomrule 
    \end{tabular}%  
    }  
  \label{tab:datasets}  
\end{table}

\subsection{\themodel-Generated Reasoning Chains}\label{sec:cases}
In this section, we present several reasoning chains generated by \themodel. The evidence is omitted due to its length, which makes it unsuitable for display. Red coloring is used to highlight where the reasoning chains become incorrect.
\small{\begin{casebox}
\textbf{Claim}: The artist whose work was displayed in 1974 at Debel Gallery, was closely associated with the Viennese Actionism group, while Howard Zieff was an American television commercial director.\\
\textbf{Label}: Supported\\\\
\textbf{Model Output}:\\
C1: The artist whose work was displayed in 1974 at Debel Gallery was closely associated with the Viennese Actionism group.\\Entity Resolution: Artist -> Rudolf Schwarzkogler (from E2)\\Resolution Verification: E1 confirms Rudolf Schwarzkogler's work was displayed in 1974 at Debel Gallery.\\Verification: E2 supports that Rudolf Schwarzkogler was closely associated with the Viennese Actionism group.\\Status: Supported.\vspace{1mm}\\C2: Howard Zieff was an American television commercial director.\\Entity Resolution: Howard Zieff -> Howard B. Zieff (from E3)\\Resolution Verification: E3 confirms Howard B. Zieff was an American director, television commercial director, and advertising photographer.\\Verification: E3 supports that Howard Zieff was an American television commercial director.\\Status: Supported.
\end{casebox}}

\small{\begin{casebox}
\textbf{Claim}: Ben Karlin wrote the 2013 episode of the TV show, Netflix, that was directed by the actor who played Kevin Arnold in ``The Wonder Years''\\
\textbf{Label}: Refuted\\\\
\textbf{Model Output}:\\
C1: Ben Karlin wrote the 2013 episode of the TV show, Netflix.\\Entity Resolution: TV show -> Modern Family (from E1)\\Resolution Verification: E1 confirms 'ClosetCon '13' aired in 2013, and E1 also confirms Ben Karlin wrote it.\\Verification: E1 supports that Ben Karlin wrote the 2013 episode, but it's clear that the TV show is Modern Family, not Netflix.\\Status: Refuted.\vspace{1mm}\\C2: The actor who played Kevin Arnold in \"The Wonder Years\" directed the 2013 episode.\\Entity Resolution: Actor -> Fred Savage (from E2)\\Resolution Verification: E2 confirms Fred Savage played Kevin Arnold in ``The Wonder Years''.\\Verification: E2 supports that Fred Savage played Kevin Arnold, but it's clear that he directed the episode in 2013, not that he played Kevin Arnold in 2013.\\Status: Refuted.\vspace{1mm}\\C3: The TV show was Netflix.\\Entity Resolution: TV show -> Modern Family (from E1)\\Resolution Verification: E1 confirms 'ClosetCon '13' aired on Modern Family, not Netflix.\\Verification: E1 refutes that the TV show was Netflix.\\Status: Refuted.
\end{casebox}}

\small{\begin{casebox}
\textbf{Claim}: The builder of Shadow Creek Golf Course is an Jewish real estate developer. The builder also owns the Encore hotel and casino in Las Vegas.\\
\textbf{Label}: Refuted\\\\
\textbf{Model Output}:\\
C1: The builder of Shadow Creek Golf Course is an Jewish real estate developer.\\Entity Resolution: Builder of Shadow Creek Golf Course -> Steve Wynn (from E3)\\Resolution Verification: E3 confirms Steve Wynn built Shadow Creek Golf Course.\\Verification: E3 shows Steve Wynn built Shadow Creek Golf Course, but it doesn't provide any information about his religion. There's no evidence to support the claim that Steve Wynn is Jewish.\\Status: Refuted.\vspace{1mm}\\C2: The builder also owns the Encore hotel and casino in Las Vegas.\\Entity Resolution: Builder -> Steve Wynn (from E3)\\Resolution Verification: E3 confirms Steve Wynn built Shadow Creek Golf Course, and E2 confirms Wynn Resorts owns Encore.\\Verification: E3 shows Steve Wynn built Shadow Creek Golf Course, and E2 confirms Wynn Resorts owns Encore.\\Status: Supported.
\end{casebox}}

\small{\begin{casebox}
\textbf{Claim}: Carnegie Mellon University, not the institution which is home to the Lyme Academy of Fine Arts, is a university in Pennsylvania.\\
\textbf{Label}: Supported\\\\
\textbf{Model Output}:\\
C1: Carnegie Mellon University, not the institution which is home to the Lyme Academy of Fine Arts, is a university in Pennsylvania.\\Entity Resolution: institution which is home to the Lyme Academy of Fine Arts -> University of New Haven (from E1)\\Resolution Verification: E1 confirms Lyme Academy College of Fine Arts is part of University of New Haven, not CMU. E3 confirms CMU is in Pennsylvania.\\Verification: E1 refutes \textcolor{red}{the institution which is home to the Lyme Academy of Fine Arts being CMU}. E3 supports CMU being in Pennsylvania.\\Status: Refuted.
\end{casebox}}
(The model misunderstood the meaning of the claim.)

\small{\begin{casebox}
\textbf{Claim}: Orange is the main flavor of both \textcolor{red}{zero-calorie} soft drinks, Solo from Norway, and the drink advertised with Krupa (song).\\
\textbf{Label}: Refuted\\\\
\textbf{Model Output}:\\
C1: Orange is the main flavor of Solo from Norway.\\Verification: E1 confirms Solo is an orange-flavored soft drink from Norway.\\Status: Supported.\vspace{1mm}\\C2: Orange is the main flavor of the drink advertised with Krupa (song).\\Entity Resolution: Drink advertised with Krupa -> Sunkist\\Resolution Verification: E3 confirms Krupa is from an Sunkist advert.\\Verification: E3 confirms Krupa is from an Sunkist advert and E2 confirms Sunkist is primarily orange flavored.\\Status: Supported.
\end{casebox}}
(The model lost the information of "zero-calorie" in the claim during the Claim Decomposition process, leading to an error.)

\small{\begin{casebox}
\textbf{Claim}: The star of Forever Strong played a character that is based on the Hanna-Barbera show ``Birdman and the Galaxy Trio''.\\
\textbf{Label}: Supported\\\\
\textbf{Model Output}:\\
C1: The star of Forever Strong played a character that is based on the Hanna-Barbara show ``Birdman and the Galaxy Trio''.\\Entity Resolution: Star of Forever Strong -> Gary Cole (from E1)\\Resolution Verification: E1 confirms Gary Cole was a star of Forever Strong.\\Entity Resolution: \textcolor{red}{Character played by Gary Cole -> ? (from E2)}\\Resolution Verification: E2 confirms Gary Cole voiced Harvey Birdman, but it doesn't mention him playing \textcolor{red}{a character in Forever Strong}.\\Verification: E1 shows Gary Cole was a star of Forever Strong, but E2 shows \textcolor{red}{he voiced Harvey Birdman, not a character in Forever Strong}. Therefore, the claim is refuted.\\Status: Refuted.
\end{casebox}}
(The model incorrectly associated the character in the claim with ``Forever Strong,'' leading to an error in Entity Resolution.)