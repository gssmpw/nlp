% This is a modified version of Springer's LNCS template suitable for anonymized MICCAI 2025 main conference submissions. 
% Original file: samplepaper.tex, a sample chapter demonstrating the LLNCS macro package for Springer Computer Science proceedings; Version 2.21 of 2022/01/12

\documentclass[runningheads]{llncs}
%
\usepackage[T1]{fontenc}
% T1 fonts will be used to generate the final print and online PDFs,
% so please use T1 fonts in your manuscript whenever possible.
% Other font encodings may result in incorrect characters.
%
\usepackage{graphicx,verbatim}
\usepackage{amstext}
\usepackage{amsmath}
\usepackage{xcolor}
\usepackage{bbm}
% \usepackage{subfigure}
\usepackage{booktabs}
\usepackage{float}
\usepackage{hyperref}
\usepackage{url}
\usepackage{algorithm}
\usepackage{algorithmicx}
\usepackage{algpseudocode}
\usepackage{soul}
\usepackage{subcaption}
% \usepackage{subfig}

% \usepackage{natbib}
% Used for displaying a sample figure. If possible, figure files should
% be included in EPS format.
%
% If you use the hyperref package, please uncomment the following two lines
% to display URLs in blue roman font according to Springer's eBook style:
%\usepackage{color}
%\renewcommand\UrlFont{\color{blue}\rmfamily}
%\urlstyle{rm}
%
\newcommand{\quotes}[1]{``#1''}
\begin{document}
%
% \title{The Devil is in the Prompts: Identifying and Mitigating Privacy Risks in Synthetic Chest X-Ray Generation}
\title{The Devil is in the Prompts: De-Identification Traces Enhance Memorization Risks in Synthetic Chest X-Ray Generation}
%
% \begin{comment}  %% Removed for anonymized MICCAI 2025 submission
% \author{Raman Dutt\inst{1} \and
% \authorrunning{F. Author et al.}
% % First names are abbreviated in the running head.
% % If there are more than two authors, 'et al.' is used.
% %
% \institute{University of Edinburgh, Edinburgh, United Kingdom 
% \email{raman.dutt@ed.ac.uk}\\
% \url{http://www.springer.com/gp/computer-science/lncs} \and
% ABC Institute, Rupert-Karls-University Heidelberg, Heidelberg, Germany\\
% \email{\{abc,lncs\}@uni-heidelberg.de}}

% % \end{comment}

% \author{Anonymized Authors}  %% Added for anonymized MICCAI 2025 submission
% \authorrunning{Anonymized Author et al.}
% \institute{Anonymized Affiliations \\
%     \email{email@anonymized.com}}

\author{Raman Dutt} 
\authorrunning{Raman Dutt et al.}
\institute{University of Edinburgh \\
    \email{raman.dutt@ed.ac.uk}}

    
\maketitle              
\begin{abstract}
Generative models, particularly text-to-image (T2I) diffusion models, play a crucial role in medical image analysis. However, these models are prone to training data memorization, posing significant risks to patient privacy. Synthetic chest X-ray generation is one of the most common applications in medical image analysis with the MIMIC-CXR dataset serving as the primary data repository for this task. This study presents the first systematic attempt to identify prompts and text tokens in MIMIC-CXR that contribute the most to training data memorization. Our analysis reveals two unexpected findings: \textbf{(1)} \emph{prompts containing traces of de-identification procedures (markers introduced to hide Protected Health Information) are the most memorized,} and \textbf{(2)} \emph{among all tokens, de-identification markers contribute the most towards memorization}. This highlights a broader issue with the standard anonymization practices and T2I synthesis with MIMIC-CXR. To exacerbate, existing inference-time memorization mitigation strategies are ineffective and fail to sufficiently reduce the model's reliance on memorized text tokens. On this front, we propose actionable strategies for different stakeholders to enhance privacy and improve the reliability of generative models in medical imaging. Finally, our results provide a foundation for future work on developing and benchmarking memorization mitigation techniques for synthetic chest X-ray generation using the MIMIC-CXR dataset. The anonymized code is available \href{https://anonymous.4open.science/r/diffusion_memorization-8011/README.md}{here}.
% This raises concerns about the structure of textual annotations in MIMIC-CXR. Based on our findings, we propose actionable strategies to enhance privacy and improve the reliability of generative models in medical imaging. Additionally, our results provide a foundation for future work on developing and benchmarking memorization mitigation techniques for synthetic chest X-ray generation using the MIMIC-CXR dataset. 


\keywords{Memorization  \and Diffusion Models \and Synthetic Image Generation.}
\end{abstract}

\section{Introduction}
% High-quality data is often regarded as the \quotes{new gold}\footnote{https://www.forbes.com/councils/forbestechcouncil/2023/03/27/how-to-make-use-of-the-new-gold-data/}, a notion that holds particular significance in medical image analysis, where access to large-scale datasets remains a major challenge in developing clinically deployable AI models \cite{dutt2023parameter}. Generative modelling, particularly through diffusion models \cite{sohl2015deep,song2021denoising,ho2020denoising}, has emerged as a powerful tool for generating high-quality, novel data across various modalities \cite{dhariwal2021diffusion,kim2022guided,xu2022geodiff}. In medical imaging, diffusion models offer a promising solution to data scarcity while also addressing privacy, ethical, and legal concerns associated with data sharing across institutions \cite{jordon2020synthetic,yoon2020anonymization,murtaza2023synthetic}. Their utility has been demonstrated in synthesizing high-quality radiographs \cite{chambon2022roentgen}, augmenting datasets \cite{saragih2024using,wang2024majority}, and improving downstream fairness \cite{ktena2024generative}. Continued advancements in diffusion modelling are expected to further enhance their effectiveness in these areas.  

High-quality data, often regarded as the "\textit{new gold}"\footnote{https://www.forbes.com/councils/forbestechcouncil/2023/03/27/how-to-make-use-of-the-new-gold-data/}, is vital in medical image analysis where large-scale datasets are scarce, hindering clinically viable AI development \cite{dutt2023parameter}. Diffusion models \cite{song2021denoising,ho2020denoising} have proven effective in producing novel, high-fidelity data.
% across modalities \cite{dhariwal2021diffusion,kim2022guided,xu2022geodiff}. 
In medical imaging, they address data scarcity while mitigating privacy, ethical, and legal challenges in data sharing \cite{yoon2020anonymization,murtaza2023synthetic}. Their efficacy is demonstrated in synthesizing radiographs \cite{chambon2022roentgen}, augmenting datasets \cite{saragih2024using,wang2024majority}, and enhancing downstream fairness \cite{ktena2024generative}, with ongoing advances promising further impact.

Generative models, despite their benefits, are prone to memorizing training data \cite{somepalli2023diffusion,somepalli2023understanding,wen2024detecting,dutt2024memcontrol,dutt2024capacity}, which threatens patient privacy. They may produce near-identical copies of training images, exposing sensitive details and enabling re-identification attacks that link synthetic outputs to real patients \cite{fernandez2023privacy}.
% Addressing this risk is crucial for the safe and ethical deployment of generative models in medical imaging. 

\textbf{The Unique Case of MIMIC-CXR:} Previous studies have linked memorization in diffusion models to the lexical structure of text prompts \cite{wen2024detecting}. Highly specific captions often act as keys into the model's memory, allowing the model to retrieve and replicate particular samples \cite{somepalli2023diffusion}, exhibiting memorization. MIMIC-CXR presents a distinct case as its text captions follow a structured phrase pattern, and multiple images often share identical captions due to similarities in clinical findings. For instance, in a filtered subset of 110K samples, 2337 instances share the caption \quotes{\textit{No acute cardiopulmonary abnormality.}}, indicating a normal finding. Furthermore, the publicly released version contains numerous traces of a specific marker (\quotes{\textcolor{red}{\textunderscore\textunderscore\textunderscore}}) used to de-identify the Protected Health Information (PHI) \footnote{https://www.hhs.gov/hipaa/for-professionals/privacy/index.html} which can further enhance caption specificity.

Given MIMIC-CXR's central role in developing T2I models for chest X-ray synthesis \cite{chambon2022roentgen,perez2024radedit,dutt2023parameter,dutt2024memcontrol}, it is crucial to investigate memorization at both the prompt and token levels to identify elements contributing most significantly to training data memorization. Similar analyses in natural image datasets \cite{webster2023duplication,schuhmann2022laion} have shaped benchmarks for detecting and mitigating memorization, underscoring the importance of conducting such a study for the medical imaging domain.     
 
To summarize, our core contributions are as follows:  
\textbf{(1)} We conduct the first systematic analysis to identify specific text prompts and tokens in MIMIC-CXR that contribute the most to memorization.  
\textbf{(2)} Our prompt-level (Sec \ref{sec:prompt_contribution}) and token-level (Sec. \ref{sec:token_contribution}) analysis uncovers a surprising yet concerning finding: \textcolor{red!60}{tokens introduced through standard de-identification procedures contribute the most to memorization.}  
\textbf{(3)} We release a comprehensive list of memorized prompts to facilitate future research on developing and benchmarking memorization mitigation techniques for synthetic chest X-ray generation using the MIMIC-CXR dataset.  


\section{Related Work}

\textbf{Memorization in Generative Models: }Deep generative models have been shown to exhibit various forms of memorization, including training data extraction \cite{carlini2023extracting}, content replication \cite{somepalli2023diffusion}, and data copying \cite{somepalli2023understanding}. In the medical domain, \cite{akbar2023beware} found that diffusion models tend to memorize significantly more than GANs \cite{goodfellow}. Additionally, \cite{dar2023investigating} emphasized the need for robust mitigation strategies, highlighting the notable memorization in 3D Latent Diffusion Models (LDMs). \\\textbf{Mitigation Mechanisms: } Several mechanisms have been developed to mitigate memorization. \cite{somepalli2023understanding} introduced training and inference-time approaches, such as augmenting caption diversity. \cite{ren2024unveiling} presented a method that identifies memorized tokens by analyzing cross-attention scores, while \cite{wen2024detecting} devised an efficient procedure that leverages text-conditional noise for detection and mitigation. In medical image analysis, \cite{fernandez2023privacy} proposed a framework to remove samples that elevate memorization risk. Additionally, \cite{dutt2024memcontrol,dutt2024capacity} demonstrated that managing model capacity through Parameter-Efficient Fine-Tuning (PEFT) \cite{dutt2023parameter} can significantly reduce memorization. \\
Unlike prior studies that concentrate on mitigating memorization, our work underscores a fundamental flaw in data de-identification and employs established frameworks \cite{wen2024detecting,somepalli2023understanding} to demonstrate its connection to memorization.


\section{Preliminaries}

\subsection{Diffusion Models}

% Diffusion models operate through a two-phase process: \textbf{forward} and \textbf{reverse} diffusion. The forward diffusion phase follows a fixed Markov chain over $T$ steps, where Gaussian noise is systematically introduced at each step. This noise is incrementally added to a data sample $x_0$, which is drawn from the real data distribution \(q(x_0)\)
% The process is mathematically defined as:

% \begin{align} \label{eqn:forward}
%     q\left(x_t \mid x_{t-1}\right)=\mathcal{N}(x_t ; \sqrt{1-\beta_t} x_{t-1}, \beta_t \mathbf{I}),
% \end{align}

% where \(\beta_t\) denotes the scheduled variance at timestep \(t\). The closed-form expression for this process is given by: 

% \begin{equation} \label{eqn:step2}
%     x_t = \sqrt{\overline{\alpha}_t}x_0 + \sqrt{1-\overline{\alpha_t}}\epsilon,
%     % \ \text{where} \ \overline{\alpha}_{t}=\prod_{i=1}^{t}(1-\beta_{t}),
% \end{equation}
% %
% where $\overline{\alpha}_{t}=\prod_{i=1}^{t}(1-\beta_{t})$. \\

% During the reverse diffusion process, a Gaussian vector \(x_T\) is drawn from \(\mathcal{N}(0, 1)\) and gradually denoised to reconstruct an image \(x_0\) that belongs to the data distribution \(q(x)\). At each denoising step, a noise estimator \(\epsilon_\theta\) estimates and removes the noise component \(\epsilon_\theta(x_t)\) that was introduced to \(x_0\). The formulation for computing \(x_{t-1}\) is given by:

% \begin{equation}
%     x_{t-1} = \sqrt{\overline{\alpha}_{t-1}}\hat{x}_0^t + \sqrt{1 - \overline{\alpha}_{t-1}}\epsilon_{\theta}(x_t),
% \end{equation}
% %
% where
% %
% \begin{equation}
%     \hat{x}_0^t = \frac{x_t - \sqrt{1 - \overline{\alpha}_{t-1}}\epsilon_{\theta}(x_t)}{\sqrt{\overline{\alpha}_t}}.
% \end{equation}

Diffusion models consist of two phases: forward and reverse diffusion. In the forward process, a data sample is gradually corrupted over $T$ steps by adding Gaussian noise according to a fixed Markov chain. At each step, the noise is injected as:
\begin{align} \label{eqn:forward}
    q\left(x_t \mid x_{t-1}\right)=\mathcal{N}(x_t ; \sqrt{1-\beta_t} x_{t-1}, \beta_t \mathbf{I}),
\end{align}
which leads to the closed-form expression
\begin{equation*} \label{eqn:step2}
    x_t = \sqrt{\overline{\alpha}_t}x_0 + \sqrt{1-\overline{\alpha_t}}\epsilon,
    % \ \text{where} \ \overline{\alpha}_{t}=\prod_{i=1}^{t}(1-\beta_{t}),
\end{equation*}

where $\overline{\alpha}_{t}=\prod_{i=1}^{t}(1-\beta_{t})$ \\
In the reverse process, one begins with a sample $x_T \sim \mathcal{N}(0,1)$ and iteratively denoises it to recover $x_0$. At each step, a learned noise estimator $\epsilon_{\theta}(x_t)$ predicts and subtracts the noise, updating the state as
\begin{equation*}
    x_{t-1} = \sqrt{\overline{\alpha}_{t-1}}\hat{x}_0^t + \sqrt{1 - \overline{\alpha}_{t-1}}\epsilon_{\theta}(x_t),
\end{equation*}
where $\hat{x}_0^t$ represents the intermediate estimate of $x_0$.



\subsection{Efficient Memorization Detection via Text-Conditional Noise} \label{sec:detection_framework}

% Here, we introduce the notion of \textbf{text-conditional noise} that serves as an efficient indicator of memorization \cite{wen2024detecting}.
A standard T2I stable-diffusion pipeline consists of a text encoder $T_E$, a variational autoencoder (VAE) $V_E$, and a noise predictor (U-Net). As noted in \cite{wen2024detecting}, for non-memorized prompts, the generated images are primarily influenced by the initial noise. In such cases, the model follows a denoising track influenced by both the initial noise and text-conditioning. However, for memorized prompts, the model overfits to a fixed denoising track, making the generated image largely independent of the initial noise. In this scenario, the model’s predictions become predominantly reliant on text-conditioning. 

This phenomenon is demonstrated in Fig \ref{fig:multiple_generations}. For a prompt that has been identified as \quotes{\textit{memorized}}, the generations across multiple seeds show a striking resemblance to one another, indicating independence on the initial noise (controlled by the generation seed). On the contrary, multiple generations for a \quotes{\textit{non-memorized}} prompt, show differences with change in generation seed. 

Leveraging this insight, tracking the \emph{text-conditional noise} at each timestep emerges as a robust metric for detecting memorization \cite{wen2024detecting}. Given the noise predictor $\epsilon_{\theta}$ and $T$ timesteps, a prompt $p$ and an empty string $\emptyset$ with corresponding embeddings $e_p$ and $e_{\emptyset}$, the memorization detection metric $d_{mem}$ can be defined as:
\begin{equation*} 
    \boxed{ d_{mem} = \frac{1}{T} \sum_{t=1}^{T} \|\epsilon_{\theta}(x_t, e_p) - \epsilon_{\theta}(x_t, e_{\emptyset})\|_2. }
\end{equation*}

% \begin{equation*} \label{eqn:mem_metric}
%     d_{mem} = \frac{1}{T} \sum_{t=1}^{T} \|\epsilon_{\theta}(x_t, e_p) - \epsilon_{\theta}(x_t, e_{\emptyset})\|_2.
%     % \label{equation:detection_metric}
% \end{equation*}

A higher value of $d_{mem}$ signifies a stronger memorization. This framework offers greater efficiency by providing a reliable memorization signal from the very first sampling step \cite{wen2024detecting}, making it well-suited for examining large datasets such as MIMIC-CXR. 

% Unlike other detection methods that (1) perform a brute-force comparison between the real and generated images \cite{somepalli2023diffusion} or (2) analyse multiple generations per prompt \cite{carlini2023extracting}, this framework offers greater efficiency by providing a reliable memorization signal from the very first sampling step \cite{wen2024detecting}. This efficiency makes it well-suited for deployment on large datasets such as MIMIC-CXR.

\begin{figure}[htbp]
    \centering
    
    \subfloat[\textbf{ \textcolor{red}{(Memorized)} Prompt:} \textit{AP chest compared to \textcolor{black}{\textunderscore\textunderscore\textunderscore}: Previous mild pulmonary edema has resolved. There is no pneumonia ...
}
    ]{
        \includegraphics[width=0.8\textwidth]{figures/mem_grid.png}
    }
    
    \subfloat[\textbf{ \textcolor{cyan}{(Non-Memorized)} Prompt:} \textit{The right-sided chest tube, right-sided PICC line, and feeding tube are unchanged in position ... 
}
    ]{
        \includegraphics[width=0.8\textwidth]{figures/non_mem_grid2.png}
    }
    
    % \caption{Figure illustrating multiple generations for the same prompt across different initialization seeds. The top row corresponds to a \textit{memorized} prompt, while the bottom row represents a \textit{non-memorized} prompt. \\
    % For \quotes{\textit{memorized}} prompts, generated images remain nearly identical across different seeds, indicating independence from the initial noise. \\
    % In contrast, \textit{non-memorized} prompts exhibit diversity in generated samples, demonstrating their dependence on the initial noise.
    % }  
    \caption{Multiple generations for a single prompt across various initialization seeds. The top row shows a \textit{memorized} prompt, where images remain nearly identical regardless of the seed, indicating independence from initial noise. In contrast, the bottom row displays a \textit{non-memorized} prompt, with diverse outputs reflecting sensitivity to the initial noise, indicating no memorization.}
    \label{fig:multiple_generations}
\end{figure}


\section{Experiments}

\textbf{Experimental Setup.} A reliable memorization signal necessitates an in-domain latent diffusion model capable of generating high-quality chest X-rays. For this task, we employ the off-the-shelf \textit{RadEdit} model \cite{perez2024radedit}, which integrates a biomedical text encoder \cite{bannur2023learning} and the VAE from SDXL \cite{podell2023sdxl}. This model is particularly well-suited to our setup as it includes the MIMIC-CXR dataset in its training corpus. For detecting memorization in prompts, we employ the framework from \cite{wen2024detecting} (Sec \ref{sec:detection_framework}) due to its reliability and efficiency.

\subsection{Detecting Memorized Prompts in MIMIC-CXR} \label{sec:prompt_contribution}

\begin{figure}[htb] 
    \centering
    \includegraphics[width=0.9\textwidth]{figures/Prompts_text_norms.png} 
    \caption{Visualizing the distribution of text-conditional norms for unique prompts in the MIMIC-CXR dataset (largest to smallest). Prompts in the top 1 percentile, exhibiting the highest norms, are highlighted in red. Prompts exhibiting high norms indicate they are potentially memorized.}
    \label{fig:memorization_framework}
\end{figure}

\boxed{\textbf{Setup:}} To identify all memorized prompts in the MIMIC-CXR dataset, we begin by extracting the subset of all unique prompts. Using a text-to-image pipeline comprising a pre-trained denoising U-Net ($\epsilon$), a text-encoder ($T_E$) and a VAE ($V_E$), we track and store the text-conditional noise for each unique prompt at every denoising timestep. Finally, we compute the average text-conditional noise across all timesteps to quantify memorization. This gives us a memorization score ($d_{mem}$) for each unique prompt in the dataset.
% The full procedure is detailed in Algorithm \ref{alg:memorization_detection}. 
\\ \boxed{\textbf{Results:}} Figure \ref{fig:memorization_framework} illustrates the distribution of the memorization scores for all unique prompts, sorted in descending order for visual clarity. The distribution follows a \textit{heavy-tailed} pattern, with a small subset of prompts (on the left) exhibiting significantly higher norms, indicating a stronger contribution to memorization. The prompts corresponding to the top 1 percentile of norm values, highlighted in red and referred to as \quotes{\textcolor{red}{\textit{memorized prompts}}} hereafter, represent the most extreme cases indicating the highest contribution towards memorization. The gradual decline in norm values across the remaining prompts suggests a varying degree of influence on memorization, with the majority exhibiting relatively lower norms. This variability underscores the need for further investigation into prompts with the highest memorization scores, as they may reveal underlying patterns that contribute to memorization risks. We conduct further analysis in section \ref{sec:token_contribution}.

\subsection{Examining Individual Token Contribution: Traces of De-Identification Enhance Memorization} \label{sec:token_contribution}

\boxed{\textbf{Token-Level Analysis:}} Building on the \textit{prompt-level} analysis in Section \ref{sec:prompt_contribution}, we extend our investigation to the \textit{token-level}. Specifically, we focus on the set of \textit{memorized prompts} and analyze the contribution of individual tokens toward memorization. \\
\boxed{\textbf{Results:}} Our findings consistently show that within memorized prompts, the de-identification marker is the token contributing most significantly to memorization, as illustrated in Figures \ref{fig:token_contribution}. We hypothesize two key reasons for this phenomenon: \textbf{(1)} The de-identification marker is a distinct and unique token, differing from all other tokens in the MIMIC-CXR text corpus. \textbf{(2)} It appears frequently across the dataset, occurring in 21,373 unique prompts. This high frequency allows the model to learn spurious correlations, leading to the memorization of specific samples. 
This finding is particularly concerning as de-identification is a standard practice before publicly releasing medical datasets. Our results highlight the need to reassess current de-identification methodologies to prevent unintended memorization in generative models.

\begin{figure}[htb]
    \centering
    
    \subfloat[\textbf{Prompt:} \textit{AP chest compared to \textcolor{red}{\textunderscore\textunderscore\textunderscore}: Previous mild pulmonary edema has resolved.  There is no pneumonia. Several small lung nodules and the large right paratracheal mediastinal mass are manifestations of lung cancer. Heart size normal.  No appreciable pleural effusion.}
    ]{
        % \includegraphics[width=0.9\textwidth]{figures/prompt_with_dash_21201.png}
        \includegraphics[width=0.9\textwidth]{figures/prompt_with_dash_37.png}
    }
    \caption{Figure illustrating the text-conditional norm for each token in a memorized prompt. We only plot the tokens with the top 25 norm values for visual clarity. Amongst all tokens, the PHI de-identification token (\quotes{\textcolor{red}{\textunderscore\textunderscore\textunderscore}}) holds the most significant contribution towards memorization. This behaviour is replicated across all memorized prompts.}
    \label{fig:token_contribution}
\end{figure}

% \textcolor{red}{What about the memorized prompts that do not contain de-identification markers? What are the most significant tokens in that?}

% \textbf{\textcolor{red}{Write about the data stats here. How many samples were there in the memorized prompts? How many contain underscore and how many do not.}} \\

% Additionally, other entities such as geographical locations and clinical provider names are also classified as PHI and masked accordingly. As a result, 21,373 unique text prompts in the dataset contain traces of de-identification.

\subsection{Existing Intervention Methods are Ineffective}

\begin{figure}[htb] 
    \centering
    \includegraphics[width=0.9\textwidth]{figures/mitigation_strategies.png} 
    \caption{Figure depicting multiple generations for the same prompt and different mitigation strategies. The visual similarity across different generations and mitigation methods indicates their ineffectiveness.}
    \label{fig:mitigation_strategies}
\end{figure}

% \boxed{\textbf{Setting:}} 
In this section, we investigate whether applying memorization mitigation strategies to de-identification traces can effectively reduce memorization. Specifically, we evaluate different inference-time mitigation techniques \cite{somepalli2023understanding}: \textbf{(1)} \textit{\textbf{Random Word Addition (RWA)}}, where de-identification markers are replaced with random words; \textbf{(2)} \textit{\textbf{Random Number Addition (RNA)}}, where markers are substituted with random numbers; and \textbf{(3)} the \textbf{complete removal} of de-identification markers from the prompt. \\ \boxed{\textbf{Results:}} We assess memorization by analyzing multiple generations across different initialization seeds for the same prompt. Memorization is qualitatively indicated by the similarity among generated images. For a quantitative evaluation, we compute the mean L2 distance between 50 generated samples using the same prompt, where a lower L2 distance signifies stronger memorization.
% The qualitative results are presented in Fig. \ref{fig:mitigation_strategies}. 
Across all mitigation strategies, we observe that the model continues to generate visually similar images. Simply replacing de-identification markers with a random word or number, or even removing them entirely, remains ineffective. Quantitative analysis reinforces this observation. The average L2 distance over 50 generations remains nearly unchanged after applying mitigation strategies: 0.38 for the original prompt versus 0.45, 0.43, and 0.42 with mitigation strategies applied. These findings indicate a deeper underlying issue that must be addressed at the training level.
% As shown in Fig. \ref{fig:mitigation_strategies_with_bar_plots}, the mean L2 distance across 50 generations remains largely unchanged, regardless of the applied mitigation strategy. These findings indicate a deeper underlying issue that must be addressed at the training level.


\section{Discussion and Conclusion}
This section examines potential factors through which de-identification practices may inadvertently heighten the risks of memorization and compromise privacy preservation. We also offer recommendations for medical AI researchers involved in dataset curation, pre-processing, and the training of T2I models with a focus on mitigating memorization. \\ \boxed{\textbf{Why Do de-identification Markers Lead to Memorization?}} The text corpus in MIMIC-CXR exhibits a distinct lexical structure, notably marked by the frequent occurrence of the de-identification token (\quotes{\textcolor{red}{\textunderscore\textunderscore\textunderscore}}). Introduced during the de-identification process, this token offers no substantive information for text-to-image generation. Instead, it creates a spurious correlation with the corresponding images. As a result, such highly specific tokens can serve as retrieval keys, allowing for the extraction of particular data points that appear as repeated, replicated generations, indicating memorization, as shown in \cite{somepalli2023understanding}. \\ \boxed{\textbf{Recommendations for Enhancing Privacy Preservation: }}  
We propose several actionable strategies for different stakeholders. \\\textbf{Dataset curators} should refrain from using a uniform de-identification marker across the entire dataset. By employing a rule-based de-identification approach as in \cite{johnson2016mimic}, curators can randomize the marker symbols. This method not only enhances the diversity of captions that can mitigate memorization \cite{somepalli2023understanding} but also helps to minimize the risk of establishing spurious correlations between specific tokens and images. \\
\textbf{Model developers} tasked with training T2I models should invest additional effort in pre-processing dataset captions. For example, recaptioning datasets to eliminate redundant tokens can enhance both the quality and diversity of the captions. Additionally, employing an in-domain vision-language model (VLM) \cite{llavamed} can refine the language and augment the information density of the captions. This strategy is expected to improve caption diversity and boost generative performance \cite{segalis2023pictureworththousandwords}. \\\\ In summary, our work tackles the challenges of memorization and privacy preservation. By focusing on MIMIC-CXR, the most widely used dataset for T2I generation of chest X-rays, we reveal a critical flaw in the conventional de-identification procedure employed in medical datasets, establishing a clear connection to memorization. Moreover, we demonstrate that removing memorization from trained models is a complex task, with standard mitigation techniques falling short. To address this issue at its source, we offer targeted recommendations for various stakeholders. Finally, we release a list of memorized prompts to support future benchmarking and the development of more effective mitigation strategies.

\newpage
\clearpage

\bibliographystyle{splncs04}
\bibliography{mybibliography}

% \subsection{Lloyd-Max Algorithm}
\label{subsec:Lloyd-Max}
For a given quantization bitwidth $B$ and an operand $\bm{X}$, the Lloyd-Max algorithm finds $2^B$ quantization levels $\{\hat{x}_i\}_{i=1}^{2^B}$ such that quantizing $\bm{X}$ by rounding each scalar in $\bm{X}$ to the nearest quantization level minimizes the quantization MSE. 

The algorithm starts with an initial guess of quantization levels and then iteratively computes quantization thresholds $\{\tau_i\}_{i=1}^{2^B-1}$ and updates quantization levels $\{\hat{x}_i\}_{i=1}^{2^B}$. Specifically, at iteration $n$, thresholds are set to the midpoints of the previous iteration's levels:
\begin{align*}
    \tau_i^{(n)}=\frac{\hat{x}_i^{(n-1)}+\hat{x}_{i+1}^{(n-1)}}2 \text{ for } i=1\ldots 2^B-1
\end{align*}
Subsequently, the quantization levels are re-computed as conditional means of the data regions defined by the new thresholds:
\begin{align*}
    \hat{x}_i^{(n)}=\mathbb{E}\left[ \bm{X} \big| \bm{X}\in [\tau_{i-1}^{(n)},\tau_i^{(n)}] \right] \text{ for } i=1\ldots 2^B
\end{align*}
where to satisfy boundary conditions we have $\tau_0=-\infty$ and $\tau_{2^B}=\infty$. The algorithm iterates the above steps until convergence.

Figure \ref{fig:lm_quant} compares the quantization levels of a $7$-bit floating point (E3M3) quantizer (left) to a $7$-bit Lloyd-Max quantizer (right) when quantizing a layer of weights from the GPT3-126M model at a per-tensor granularity. As shown, the Lloyd-Max quantizer achieves substantially lower quantization MSE. Further, Table \ref{tab:FP7_vs_LM7} shows the superior perplexity achieved by Lloyd-Max quantizers for bitwidths of $7$, $6$ and $5$. The difference between the quantizers is clear at 5 bits, where per-tensor FP quantization incurs a drastic and unacceptable increase in perplexity, while Lloyd-Max quantization incurs a much smaller increase. Nevertheless, we note that even the optimal Lloyd-Max quantizer incurs a notable ($\sim 1.5$) increase in perplexity due to the coarse granularity of quantization. 

\begin{figure}[h]
  \centering
  \includegraphics[width=0.7\linewidth]{sections/figures/LM7_FP7.pdf}
  \caption{\small Quantization levels and the corresponding quantization MSE of Floating Point (left) vs Lloyd-Max (right) Quantizers for a layer of weights in the GPT3-126M model.}
  \label{fig:lm_quant}
\end{figure}

\begin{table}[h]\scriptsize
\begin{center}
\caption{\label{tab:FP7_vs_LM7} \small Comparing perplexity (lower is better) achieved by floating point quantizers and Lloyd-Max quantizers on a GPT3-126M model for the Wikitext-103 dataset.}
\begin{tabular}{c|cc|c}
\hline
 \multirow{2}{*}{\textbf{Bitwidth}} & \multicolumn{2}{|c|}{\textbf{Floating-Point Quantizer}} & \textbf{Lloyd-Max Quantizer} \\
 & Best Format & Wikitext-103 Perplexity & Wikitext-103 Perplexity \\
\hline
7 & E3M3 & 18.32 & 18.27 \\
6 & E3M2 & 19.07 & 18.51 \\
5 & E4M0 & 43.89 & 19.71 \\
\hline
\end{tabular}
\end{center}
\end{table}

\subsection{Proof of Local Optimality of LO-BCQ}
\label{subsec:lobcq_opt_proof}
For a given block $\bm{b}_j$, the quantization MSE during LO-BCQ can be empirically evaluated as $\frac{1}{L_b}\lVert \bm{b}_j- \bm{\hat{b}}_j\rVert^2_2$ where $\bm{\hat{b}}_j$ is computed from equation (\ref{eq:clustered_quantization_definition}) as $C_{f(\bm{b}_j)}(\bm{b}_j)$. Further, for a given block cluster $\mathcal{B}_i$, we compute the quantization MSE as $\frac{1}{|\mathcal{B}_{i}|}\sum_{\bm{b} \in \mathcal{B}_{i}} \frac{1}{L_b}\lVert \bm{b}- C_i^{(n)}(\bm{b})\rVert^2_2$. Therefore, at the end of iteration $n$, we evaluate the overall quantization MSE $J^{(n)}$ for a given operand $\bm{X}$ composed of $N_c$ block clusters as:
\begin{align*}
    \label{eq:mse_iter_n}
    J^{(n)} = \frac{1}{N_c} \sum_{i=1}^{N_c} \frac{1}{|\mathcal{B}_{i}^{(n)}|}\sum_{\bm{v} \in \mathcal{B}_{i}^{(n)}} \frac{1}{L_b}\lVert \bm{b}- B_i^{(n)}(\bm{b})\rVert^2_2
\end{align*}

At the end of iteration $n$, the codebooks are updated from $\mathcal{C}^{(n-1)}$ to $\mathcal{C}^{(n)}$. However, the mapping of a given vector $\bm{b}_j$ to quantizers $\mathcal{C}^{(n)}$ remains as  $f^{(n)}(\bm{b}_j)$. At the next iteration, during the vector clustering step, $f^{(n+1)}(\bm{b}_j)$ finds new mapping of $\bm{b}_j$ to updated codebooks $\mathcal{C}^{(n)}$ such that the quantization MSE over the candidate codebooks is minimized. Therefore, we obtain the following result for $\bm{b}_j$:
\begin{align*}
\frac{1}{L_b}\lVert \bm{b}_j - C_{f^{(n+1)}(\bm{b}_j)}^{(n)}(\bm{b}_j)\rVert^2_2 \le \frac{1}{L_b}\lVert \bm{b}_j - C_{f^{(n)}(\bm{b}_j)}^{(n)}(\bm{b}_j)\rVert^2_2
\end{align*}

That is, quantizing $\bm{b}_j$ at the end of the block clustering step of iteration $n+1$ results in lower quantization MSE compared to quantizing at the end of iteration $n$. Since this is true for all $\bm{b} \in \bm{X}$, we assert the following:
\begin{equation}
\begin{split}
\label{eq:mse_ineq_1}
    \tilde{J}^{(n+1)} &= \frac{1}{N_c} \sum_{i=1}^{N_c} \frac{1}{|\mathcal{B}_{i}^{(n+1)}|}\sum_{\bm{b} \in \mathcal{B}_{i}^{(n+1)}} \frac{1}{L_b}\lVert \bm{b} - C_i^{(n)}(b)\rVert^2_2 \le J^{(n)}
\end{split}
\end{equation}
where $\tilde{J}^{(n+1)}$ is the the quantization MSE after the vector clustering step at iteration $n+1$.

Next, during the codebook update step (\ref{eq:quantizers_update}) at iteration $n+1$, the per-cluster codebooks $\mathcal{C}^{(n)}$ are updated to $\mathcal{C}^{(n+1)}$ by invoking the Lloyd-Max algorithm \citep{Lloyd}. We know that for any given value distribution, the Lloyd-Max algorithm minimizes the quantization MSE. Therefore, for a given vector cluster $\mathcal{B}_i$ we obtain the following result:

\begin{equation}
    \frac{1}{|\mathcal{B}_{i}^{(n+1)}|}\sum_{\bm{b} \in \mathcal{B}_{i}^{(n+1)}} \frac{1}{L_b}\lVert \bm{b}- C_i^{(n+1)}(\bm{b})\rVert^2_2 \le \frac{1}{|\mathcal{B}_{i}^{(n+1)}|}\sum_{\bm{b} \in \mathcal{B}_{i}^{(n+1)}} \frac{1}{L_b}\lVert \bm{b}- C_i^{(n)}(\bm{b})\rVert^2_2
\end{equation}

The above equation states that quantizing the given block cluster $\mathcal{B}_i$ after updating the associated codebook from $C_i^{(n)}$ to $C_i^{(n+1)}$ results in lower quantization MSE. Since this is true for all the block clusters, we derive the following result: 
\begin{equation}
\begin{split}
\label{eq:mse_ineq_2}
     J^{(n+1)} &= \frac{1}{N_c} \sum_{i=1}^{N_c} \frac{1}{|\mathcal{B}_{i}^{(n+1)}|}\sum_{\bm{b} \in \mathcal{B}_{i}^{(n+1)}} \frac{1}{L_b}\lVert \bm{b}- C_i^{(n+1)}(\bm{b})\rVert^2_2  \le \tilde{J}^{(n+1)}   
\end{split}
\end{equation}

Following (\ref{eq:mse_ineq_1}) and (\ref{eq:mse_ineq_2}), we find that the quantization MSE is non-increasing for each iteration, that is, $J^{(1)} \ge J^{(2)} \ge J^{(3)} \ge \ldots \ge J^{(M)}$ where $M$ is the maximum number of iterations. 
%Therefore, we can say that if the algorithm converges, then it must be that it has converged to a local minimum. 
\hfill $\blacksquare$


\begin{figure}
    \begin{center}
    \includegraphics[width=0.5\textwidth]{sections//figures/mse_vs_iter.pdf}
    \end{center}
    \caption{\small NMSE vs iterations during LO-BCQ compared to other block quantization proposals}
    \label{fig:nmse_vs_iter}
\end{figure}

Figure \ref{fig:nmse_vs_iter} shows the empirical convergence of LO-BCQ across several block lengths and number of codebooks. Also, the MSE achieved by LO-BCQ is compared to baselines such as MXFP and VSQ. As shown, LO-BCQ converges to a lower MSE than the baselines. Further, we achieve better convergence for larger number of codebooks ($N_c$) and for a smaller block length ($L_b$), both of which increase the bitwidth of BCQ (see Eq \ref{eq:bitwidth_bcq}).


\subsection{Additional Accuracy Results}
%Table \ref{tab:lobcq_config} lists the various LOBCQ configurations and their corresponding bitwidths.
\begin{table}
\setlength{\tabcolsep}{4.75pt}
\begin{center}
\caption{\label{tab:lobcq_config} Various LO-BCQ configurations and their bitwidths.}
\begin{tabular}{|c||c|c|c|c||c|c||c|} 
\hline
 & \multicolumn{4}{|c||}{$L_b=8$} & \multicolumn{2}{|c||}{$L_b=4$} & $L_b=2$ \\
 \hline
 \backslashbox{$L_A$\kern-1em}{\kern-1em$N_c$} & 2 & 4 & 8 & 16 & 2 & 4 & 2 \\
 \hline
 64 & 4.25 & 4.375 & 4.5 & 4.625 & 4.375 & 4.625 & 4.625\\
 \hline
 32 & 4.375 & 4.5 & 4.625& 4.75 & 4.5 & 4.75 & 4.75 \\
 \hline
 16 & 4.625 & 4.75& 4.875 & 5 & 4.75 & 5 & 5 \\
 \hline
\end{tabular}
\end{center}
\end{table}

%\subsection{Perplexity achieved by various LO-BCQ configurations on Wikitext-103 dataset}

\begin{table} \centering
\begin{tabular}{|c||c|c|c|c||c|c||c|} 
\hline
 $L_b \rightarrow$& \multicolumn{4}{c||}{8} & \multicolumn{2}{c||}{4} & 2\\
 \hline
 \backslashbox{$L_A$\kern-1em}{\kern-1em$N_c$} & 2 & 4 & 8 & 16 & 2 & 4 & 2  \\
 %$N_c \rightarrow$ & 2 & 4 & 8 & 16 & 2 & 4 & 2 \\
 \hline
 \hline
 \multicolumn{8}{c}{GPT3-1.3B (FP32 PPL = 9.98)} \\ 
 \hline
 \hline
 64 & 10.40 & 10.23 & 10.17 & 10.15 &  10.28 & 10.18 & 10.19 \\
 \hline
 32 & 10.25 & 10.20 & 10.15 & 10.12 &  10.23 & 10.17 & 10.17 \\
 \hline
 16 & 10.22 & 10.16 & 10.10 & 10.09 &  10.21 & 10.14 & 10.16 \\
 \hline
  \hline
 \multicolumn{8}{c}{GPT3-8B (FP32 PPL = 7.38)} \\ 
 \hline
 \hline
 64 & 7.61 & 7.52 & 7.48 &  7.47 &  7.55 &  7.49 & 7.50 \\
 \hline
 32 & 7.52 & 7.50 & 7.46 &  7.45 &  7.52 &  7.48 & 7.48  \\
 \hline
 16 & 7.51 & 7.48 & 7.44 &  7.44 &  7.51 &  7.49 & 7.47  \\
 \hline
\end{tabular}
\caption{\label{tab:ppl_gpt3_abalation} Wikitext-103 perplexity across GPT3-1.3B and 8B models.}
\end{table}

\begin{table} \centering
\begin{tabular}{|c||c|c|c|c||} 
\hline
 $L_b \rightarrow$& \multicolumn{4}{c||}{8}\\
 \hline
 \backslashbox{$L_A$\kern-1em}{\kern-1em$N_c$} & 2 & 4 & 8 & 16 \\
 %$N_c \rightarrow$ & 2 & 4 & 8 & 16 & 2 & 4 & 2 \\
 \hline
 \hline
 \multicolumn{5}{|c|}{Llama2-7B (FP32 PPL = 5.06)} \\ 
 \hline
 \hline
 64 & 5.31 & 5.26 & 5.19 & 5.18  \\
 \hline
 32 & 5.23 & 5.25 & 5.18 & 5.15  \\
 \hline
 16 & 5.23 & 5.19 & 5.16 & 5.14  \\
 \hline
 \multicolumn{5}{|c|}{Nemotron4-15B (FP32 PPL = 5.87)} \\ 
 \hline
 \hline
 64  & 6.3 & 6.20 & 6.13 & 6.08  \\
 \hline
 32  & 6.24 & 6.12 & 6.07 & 6.03  \\
 \hline
 16  & 6.12 & 6.14 & 6.04 & 6.02  \\
 \hline
 \multicolumn{5}{|c|}{Nemotron4-340B (FP32 PPL = 3.48)} \\ 
 \hline
 \hline
 64 & 3.67 & 3.62 & 3.60 & 3.59 \\
 \hline
 32 & 3.63 & 3.61 & 3.59 & 3.56 \\
 \hline
 16 & 3.61 & 3.58 & 3.57 & 3.55 \\
 \hline
\end{tabular}
\caption{\label{tab:ppl_llama7B_nemo15B} Wikitext-103 perplexity compared to FP32 baseline in Llama2-7B and Nemotron4-15B, 340B models}
\end{table}

%\subsection{Perplexity achieved by various LO-BCQ configurations on MMLU dataset}


\begin{table} \centering
\begin{tabular}{|c||c|c|c|c||c|c|c|c|} 
\hline
 $L_b \rightarrow$& \multicolumn{4}{c||}{8} & \multicolumn{4}{c||}{8}\\
 \hline
 \backslashbox{$L_A$\kern-1em}{\kern-1em$N_c$} & 2 & 4 & 8 & 16 & 2 & 4 & 8 & 16  \\
 %$N_c \rightarrow$ & 2 & 4 & 8 & 16 & 2 & 4 & 2 \\
 \hline
 \hline
 \multicolumn{5}{|c|}{Llama2-7B (FP32 Accuracy = 45.8\%)} & \multicolumn{4}{|c|}{Llama2-70B (FP32 Accuracy = 69.12\%)} \\ 
 \hline
 \hline
 64 & 43.9 & 43.4 & 43.9 & 44.9 & 68.07 & 68.27 & 68.17 & 68.75 \\
 \hline
 32 & 44.5 & 43.8 & 44.9 & 44.5 & 68.37 & 68.51 & 68.35 & 68.27  \\
 \hline
 16 & 43.9 & 42.7 & 44.9 & 45 & 68.12 & 68.77 & 68.31 & 68.59  \\
 \hline
 \hline
 \multicolumn{5}{|c|}{GPT3-22B (FP32 Accuracy = 38.75\%)} & \multicolumn{4}{|c|}{Nemotron4-15B (FP32 Accuracy = 64.3\%)} \\ 
 \hline
 \hline
 64 & 36.71 & 38.85 & 38.13 & 38.92 & 63.17 & 62.36 & 63.72 & 64.09 \\
 \hline
 32 & 37.95 & 38.69 & 39.45 & 38.34 & 64.05 & 62.30 & 63.8 & 64.33  \\
 \hline
 16 & 38.88 & 38.80 & 38.31 & 38.92 & 63.22 & 63.51 & 63.93 & 64.43  \\
 \hline
\end{tabular}
\caption{\label{tab:mmlu_abalation} Accuracy on MMLU dataset across GPT3-22B, Llama2-7B, 70B and Nemotron4-15B models.}
\end{table}


%\subsection{Perplexity achieved by various LO-BCQ configurations on LM evaluation harness}

\begin{table} \centering
\begin{tabular}{|c||c|c|c|c||c|c|c|c|} 
\hline
 $L_b \rightarrow$& \multicolumn{4}{c||}{8} & \multicolumn{4}{c||}{8}\\
 \hline
 \backslashbox{$L_A$\kern-1em}{\kern-1em$N_c$} & 2 & 4 & 8 & 16 & 2 & 4 & 8 & 16  \\
 %$N_c \rightarrow$ & 2 & 4 & 8 & 16 & 2 & 4 & 2 \\
 \hline
 \hline
 \multicolumn{5}{|c|}{Race (FP32 Accuracy = 37.51\%)} & \multicolumn{4}{|c|}{Boolq (FP32 Accuracy = 64.62\%)} \\ 
 \hline
 \hline
 64 & 36.94 & 37.13 & 36.27 & 37.13 & 63.73 & 62.26 & 63.49 & 63.36 \\
 \hline
 32 & 37.03 & 36.36 & 36.08 & 37.03 & 62.54 & 63.51 & 63.49 & 63.55  \\
 \hline
 16 & 37.03 & 37.03 & 36.46 & 37.03 & 61.1 & 63.79 & 63.58 & 63.33  \\
 \hline
 \hline
 \multicolumn{5}{|c|}{Winogrande (FP32 Accuracy = 58.01\%)} & \multicolumn{4}{|c|}{Piqa (FP32 Accuracy = 74.21\%)} \\ 
 \hline
 \hline
 64 & 58.17 & 57.22 & 57.85 & 58.33 & 73.01 & 73.07 & 73.07 & 72.80 \\
 \hline
 32 & 59.12 & 58.09 & 57.85 & 58.41 & 73.01 & 73.94 & 72.74 & 73.18  \\
 \hline
 16 & 57.93 & 58.88 & 57.93 & 58.56 & 73.94 & 72.80 & 73.01 & 73.94  \\
 \hline
\end{tabular}
\caption{\label{tab:mmlu_abalation} Accuracy on LM evaluation harness tasks on GPT3-1.3B model.}
\end{table}

\begin{table} \centering
\begin{tabular}{|c||c|c|c|c||c|c|c|c|} 
\hline
 $L_b \rightarrow$& \multicolumn{4}{c||}{8} & \multicolumn{4}{c||}{8}\\
 \hline
 \backslashbox{$L_A$\kern-1em}{\kern-1em$N_c$} & 2 & 4 & 8 & 16 & 2 & 4 & 8 & 16  \\
 %$N_c \rightarrow$ & 2 & 4 & 8 & 16 & 2 & 4 & 2 \\
 \hline
 \hline
 \multicolumn{5}{|c|}{Race (FP32 Accuracy = 41.34\%)} & \multicolumn{4}{|c|}{Boolq (FP32 Accuracy = 68.32\%)} \\ 
 \hline
 \hline
 64 & 40.48 & 40.10 & 39.43 & 39.90 & 69.20 & 68.41 & 69.45 & 68.56 \\
 \hline
 32 & 39.52 & 39.52 & 40.77 & 39.62 & 68.32 & 67.43 & 68.17 & 69.30  \\
 \hline
 16 & 39.81 & 39.71 & 39.90 & 40.38 & 68.10 & 66.33 & 69.51 & 69.42  \\
 \hline
 \hline
 \multicolumn{5}{|c|}{Winogrande (FP32 Accuracy = 67.88\%)} & \multicolumn{4}{|c|}{Piqa (FP32 Accuracy = 78.78\%)} \\ 
 \hline
 \hline
 64 & 66.85 & 66.61 & 67.72 & 67.88 & 77.31 & 77.42 & 77.75 & 77.64 \\
 \hline
 32 & 67.25 & 67.72 & 67.72 & 67.00 & 77.31 & 77.04 & 77.80 & 77.37  \\
 \hline
 16 & 68.11 & 68.90 & 67.88 & 67.48 & 77.37 & 78.13 & 78.13 & 77.69  \\
 \hline
\end{tabular}
\caption{\label{tab:mmlu_abalation} Accuracy on LM evaluation harness tasks on GPT3-8B model.}
\end{table}

\begin{table} \centering
\begin{tabular}{|c||c|c|c|c||c|c|c|c|} 
\hline
 $L_b \rightarrow$& \multicolumn{4}{c||}{8} & \multicolumn{4}{c||}{8}\\
 \hline
 \backslashbox{$L_A$\kern-1em}{\kern-1em$N_c$} & 2 & 4 & 8 & 16 & 2 & 4 & 8 & 16  \\
 %$N_c \rightarrow$ & 2 & 4 & 8 & 16 & 2 & 4 & 2 \\
 \hline
 \hline
 \multicolumn{5}{|c|}{Race (FP32 Accuracy = 40.67\%)} & \multicolumn{4}{|c|}{Boolq (FP32 Accuracy = 76.54\%)} \\ 
 \hline
 \hline
 64 & 40.48 & 40.10 & 39.43 & 39.90 & 75.41 & 75.11 & 77.09 & 75.66 \\
 \hline
 32 & 39.52 & 39.52 & 40.77 & 39.62 & 76.02 & 76.02 & 75.96 & 75.35  \\
 \hline
 16 & 39.81 & 39.71 & 39.90 & 40.38 & 75.05 & 73.82 & 75.72 & 76.09  \\
 \hline
 \hline
 \multicolumn{5}{|c|}{Winogrande (FP32 Accuracy = 70.64\%)} & \multicolumn{4}{|c|}{Piqa (FP32 Accuracy = 79.16\%)} \\ 
 \hline
 \hline
 64 & 69.14 & 70.17 & 70.17 & 70.56 & 78.24 & 79.00 & 78.62 & 78.73 \\
 \hline
 32 & 70.96 & 69.69 & 71.27 & 69.30 & 78.56 & 79.49 & 79.16 & 78.89  \\
 \hline
 16 & 71.03 & 69.53 & 69.69 & 70.40 & 78.13 & 79.16 & 79.00 & 79.00  \\
 \hline
\end{tabular}
\caption{\label{tab:mmlu_abalation} Accuracy on LM evaluation harness tasks on GPT3-22B model.}
\end{table}

\begin{table} \centering
\begin{tabular}{|c||c|c|c|c||c|c|c|c|} 
\hline
 $L_b \rightarrow$& \multicolumn{4}{c||}{8} & \multicolumn{4}{c||}{8}\\
 \hline
 \backslashbox{$L_A$\kern-1em}{\kern-1em$N_c$} & 2 & 4 & 8 & 16 & 2 & 4 & 8 & 16  \\
 %$N_c \rightarrow$ & 2 & 4 & 8 & 16 & 2 & 4 & 2 \\
 \hline
 \hline
 \multicolumn{5}{|c|}{Race (FP32 Accuracy = 44.4\%)} & \multicolumn{4}{|c|}{Boolq (FP32 Accuracy = 79.29\%)} \\ 
 \hline
 \hline
 64 & 42.49 & 42.51 & 42.58 & 43.45 & 77.58 & 77.37 & 77.43 & 78.1 \\
 \hline
 32 & 43.35 & 42.49 & 43.64 & 43.73 & 77.86 & 75.32 & 77.28 & 77.86  \\
 \hline
 16 & 44.21 & 44.21 & 43.64 & 42.97 & 78.65 & 77 & 76.94 & 77.98  \\
 \hline
 \hline
 \multicolumn{5}{|c|}{Winogrande (FP32 Accuracy = 69.38\%)} & \multicolumn{4}{|c|}{Piqa (FP32 Accuracy = 78.07\%)} \\ 
 \hline
 \hline
 64 & 68.9 & 68.43 & 69.77 & 68.19 & 77.09 & 76.82 & 77.09 & 77.86 \\
 \hline
 32 & 69.38 & 68.51 & 68.82 & 68.90 & 78.07 & 76.71 & 78.07 & 77.86  \\
 \hline
 16 & 69.53 & 67.09 & 69.38 & 68.90 & 77.37 & 77.8 & 77.91 & 77.69  \\
 \hline
\end{tabular}
\caption{\label{tab:mmlu_abalation} Accuracy on LM evaluation harness tasks on Llama2-7B model.}
\end{table}

\begin{table} \centering
\begin{tabular}{|c||c|c|c|c||c|c|c|c|} 
\hline
 $L_b \rightarrow$& \multicolumn{4}{c||}{8} & \multicolumn{4}{c||}{8}\\
 \hline
 \backslashbox{$L_A$\kern-1em}{\kern-1em$N_c$} & 2 & 4 & 8 & 16 & 2 & 4 & 8 & 16  \\
 %$N_c \rightarrow$ & 2 & 4 & 8 & 16 & 2 & 4 & 2 \\
 \hline
 \hline
 \multicolumn{5}{|c|}{Race (FP32 Accuracy = 48.8\%)} & \multicolumn{4}{|c|}{Boolq (FP32 Accuracy = 85.23\%)} \\ 
 \hline
 \hline
 64 & 49.00 & 49.00 & 49.28 & 48.71 & 82.82 & 84.28 & 84.03 & 84.25 \\
 \hline
 32 & 49.57 & 48.52 & 48.33 & 49.28 & 83.85 & 84.46 & 84.31 & 84.93  \\
 \hline
 16 & 49.85 & 49.09 & 49.28 & 48.99 & 85.11 & 84.46 & 84.61 & 83.94  \\
 \hline
 \hline
 \multicolumn{5}{|c|}{Winogrande (FP32 Accuracy = 79.95\%)} & \multicolumn{4}{|c|}{Piqa (FP32 Accuracy = 81.56\%)} \\ 
 \hline
 \hline
 64 & 78.77 & 78.45 & 78.37 & 79.16 & 81.45 & 80.69 & 81.45 & 81.5 \\
 \hline
 32 & 78.45 & 79.01 & 78.69 & 80.66 & 81.56 & 80.58 & 81.18 & 81.34  \\
 \hline
 16 & 79.95 & 79.56 & 79.79 & 79.72 & 81.28 & 81.66 & 81.28 & 80.96  \\
 \hline
\end{tabular}
\caption{\label{tab:mmlu_abalation} Accuracy on LM evaluation harness tasks on Llama2-70B model.}
\end{table}

%\section{MSE Studies}
%\textcolor{red}{TODO}


\subsection{Number Formats and Quantization Method}
\label{subsec:numFormats_quantMethod}
\subsubsection{Integer Format}
An $n$-bit signed integer (INT) is typically represented with a 2s-complement format \citep{yao2022zeroquant,xiao2023smoothquant,dai2021vsq}, where the most significant bit denotes the sign.

\subsubsection{Floating Point Format}
An $n$-bit signed floating point (FP) number $x$ comprises of a 1-bit sign ($x_{\mathrm{sign}}$), $B_m$-bit mantissa ($x_{\mathrm{mant}}$) and $B_e$-bit exponent ($x_{\mathrm{exp}}$) such that $B_m+B_e=n-1$. The associated constant exponent bias ($E_{\mathrm{bias}}$) is computed as $(2^{{B_e}-1}-1)$. We denote this format as $E_{B_e}M_{B_m}$.  

\subsubsection{Quantization Scheme}
\label{subsec:quant_method}
A quantization scheme dictates how a given unquantized tensor is converted to its quantized representation. We consider FP formats for the purpose of illustration. Given an unquantized tensor $\bm{X}$ and an FP format $E_{B_e}M_{B_m}$, we first, we compute the quantization scale factor $s_X$ that maps the maximum absolute value of $\bm{X}$ to the maximum quantization level of the $E_{B_e}M_{B_m}$ format as follows:
\begin{align}
\label{eq:sf}
    s_X = \frac{\mathrm{max}(|\bm{X}|)}{\mathrm{max}(E_{B_e}M_{B_m})}
\end{align}
In the above equation, $|\cdot|$ denotes the absolute value function.

Next, we scale $\bm{X}$ by $s_X$ and quantize it to $\hat{\bm{X}}$ by rounding it to the nearest quantization level of $E_{B_e}M_{B_m}$ as:

\begin{align}
\label{eq:tensor_quant}
    \hat{\bm{X}} = \text{round-to-nearest}\left(\frac{\bm{X}}{s_X}, E_{B_e}M_{B_m}\right)
\end{align}

We perform dynamic max-scaled quantization \citep{wu2020integer}, where the scale factor $s$ for activations is dynamically computed during runtime.

\subsection{Vector Scaled Quantization}
\begin{wrapfigure}{r}{0.35\linewidth}
  \centering
  \includegraphics[width=\linewidth]{sections/figures/vsquant.jpg}
  \caption{\small Vectorwise decomposition for per-vector scaled quantization (VSQ \citep{dai2021vsq}).}
  \label{fig:vsquant}
\end{wrapfigure}
During VSQ \citep{dai2021vsq}, the operand tensors are decomposed into 1D vectors in a hardware friendly manner as shown in Figure \ref{fig:vsquant}. Since the decomposed tensors are used as operands in matrix multiplications during inference, it is beneficial to perform this decomposition along the reduction dimension of the multiplication. The vectorwise quantization is performed similar to tensorwise quantization described in Equations \ref{eq:sf} and \ref{eq:tensor_quant}, where a scale factor $s_v$ is required for each vector $\bm{v}$ that maps the maximum absolute value of that vector to the maximum quantization level. While smaller vector lengths can lead to larger accuracy gains, the associated memory and computational overheads due to the per-vector scale factors increases. To alleviate these overheads, VSQ \citep{dai2021vsq} proposed a second level quantization of the per-vector scale factors to unsigned integers, while MX \citep{rouhani2023shared} quantizes them to integer powers of 2 (denoted as $2^{INT}$).

\subsubsection{MX Format}
The MX format proposed in \citep{rouhani2023microscaling} introduces the concept of sub-block shifting. For every two scalar elements of $b$-bits each, there is a shared exponent bit. The value of this exponent bit is determined through an empirical analysis that targets minimizing quantization MSE. We note that the FP format $E_{1}M_{b}$ is strictly better than MX from an accuracy perspective since it allocates a dedicated exponent bit to each scalar as opposed to sharing it across two scalars. Therefore, we conservatively bound the accuracy of a $b+2$-bit signed MX format with that of a $E_{1}M_{b}$ format in our comparisons. For instance, we use E1M2 format as a proxy for MX4.

\begin{figure}
    \centering
    \includegraphics[width=1\linewidth]{sections//figures/BlockFormats.pdf}
    \caption{\small Comparing LO-BCQ to MX format.}
    \label{fig:block_formats}
\end{figure}

Figure \ref{fig:block_formats} compares our $4$-bit LO-BCQ block format to MX \citep{rouhani2023microscaling}. As shown, both LO-BCQ and MX decompose a given operand tensor into block arrays and each block array into blocks. Similar to MX, we find that per-block quantization ($L_b < L_A$) leads to better accuracy due to increased flexibility. While MX achieves this through per-block $1$-bit micro-scales, we associate a dedicated codebook to each block through a per-block codebook selector. Further, MX quantizes the per-block array scale-factor to E8M0 format without per-tensor scaling. In contrast during LO-BCQ, we find that per-tensor scaling combined with quantization of per-block array scale-factor to E4M3 format results in superior inference accuracy across models. 


\end{document}
