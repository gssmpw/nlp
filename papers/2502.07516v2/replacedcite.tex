\section{Related Work}
\textbf{Memorization in Generative Models: }Deep generative models have been shown to exhibit various forms of memorization, including training data extraction ____, content replication ____, and data copying ____. In the medical domain, ____ found that diffusion models tend to memorize significantly more than GANs ____. Additionally, ____ emphasized the need for robust mitigation strategies, highlighting the notable memorization in 3D Latent Diffusion Models (LDMs). \\\textbf{Mitigation Mechanisms: } Several mechanisms have been developed to mitigate memorization. ____ introduced training and inference-time approaches, such as augmenting caption diversity. ____ presented a method that identifies memorized tokens by analyzing cross-attention scores, while ____ devised an efficient procedure that leverages text-conditional noise for detection and mitigation. In medical image analysis, ____ proposed a framework to remove samples that elevate memorization risk. Additionally, ____ demonstrated that managing model capacity through Parameter-Efficient Fine-Tuning (PEFT) ____ can significantly reduce memorization. \\
Unlike prior studies that concentrate on mitigating memorization, our work underscores a fundamental flaw in data de-identification and employs established frameworks ____ to demonstrate its connection to memorization.