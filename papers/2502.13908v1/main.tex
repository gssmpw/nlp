 %%
%% This is file `sample-sigconf.tex',
%% generated with the docstrip utility.
%%
%% The original source files were:
%%
%% samples.dtx  (with options: `all,proceedings,bibtex,sigconf')
%% 
%% IMPORTANT NOTICE:
%% 
%% For the copyright see the source file.
%% 
%% Any modified versions of this file must be renamed
%% with new filenames distinct from sample-sigconf.tex.
%% 
%% For distribution of the original source see the terms
%% for copying and modification in the file samples.dtx.
%% 
%% This generated file may be distributed as long as the
%% original source files, as listed above, are part of the
%% same distribution. (The sources need not necessarily be
%% in the same archive or directory.)
%%
%%
%% Commands for TeXCount
%TC:macro \cite [option:text,text]
%TC:macro \citep [option:text,text]
%TC:macro \citet [option:text,text]
%TC:envir table 0 1
%TC:envir table* 0 1
%TC:envir tabular [ignore] word
%TC:envir displaymath 0 word
%TC:envir math 0 word
%TC:envir comment 0 0
%%
%%
%% The first command in your LaTeX source must be the \documentclass
%% command.
%%
%% For submission and review of your manuscript please change the
%% command to \documentclass[manuscript, screen, review]{acmart}.
%%
%% When submitting camera ready or to TAPS, please change the command
%% to \documentclass[sigconf]{acmart} or whichever template is required
%% for your publication.
%%
%%
% \documentclass[sigconf]{acmart}
% \documentclass[sigconf,natbib=true,anonymous=true]{acmart}
\documentclass[sigconf,natbib=true]{acmart}

\usepackage{acronym}
\newacronym{rl}{RL}{Reinforcement Learning}
\newacronym{drl}{DRL}{Deep Reinforcement Learning}
\newacronym{mdp}{MDP}{Markov Decision Process}
\newacronym{ppo}{PPO}{Proximal Policy Optimization}
\newacronym{sac}{SAC}{Soft Actor-Critic}
\newacronym{epvf}{EPVF}{Explicit Policy-conditioned Value Function}
\newacronym{unf}{UNF}{Universal Neural Functional}
\newcommand{\todo}[1]{\textcolor{red}{TODO: #1}}
\newcommand{\missingcit}[1]{\textcolor{orange}{CIT: #1}}
\newcommand{\note}[1]{\textcolor{blue}{{}{#1}{}}}


%%
%% \BibTeX command to typeset BibTeX logo in the docs
\AtBeginDocument{%
  \providecommand\BibTeX{{%
    Bib\TeX}}}

%% Rights management information.  This information is sent to you
%% when you complete the rights form.  These commands have SAMPLE
%% values in them; it is your responsibility as an author to replace
%% the commands and values with those provided to you when you
%% complete the rights form.
\setcopyright{acmlicensed}
\copyrightyear{2018}
\acmYear{2018}
\acmDOI{XXXXXXX.XXXXXXX}

%% These commands are for a PROCEEDINGS abstract or paper.
\acmConference[SIGIR '25]{Make sure to enter the correct
  conference title from your rights confirmation email}{June 03--05,
  2018}{Woodstock, NY}
%%
%%  Uncomment \acmBooktitle if the title of the proceedings is different
%%  from ``Proceedings of ...''!
%%
%%\acmBooktitle{Woodstock '18: ACM Symposium on Neural Gaze Detection,
%%  June 03--05, 2018, Woodstock, NY}
\acmISBN{978-1-4503-XXXX-X/18/06}


%%
%% Submission ID.
%% Use this when submitting an article to a sponsored event. You'll
%% receive a unique submission ID from the organizers
%% of the event, and this ID should be used as the parameter to this command.
%%\acmSubmissionID{123-A56-BU3}

%%
%% For managing citations, it is recommended to use bibliography
%% files in BibTeX format.
%%
%% You can then either use BibTeX with the ACM-Reference-Format style,
%% or BibLaTeX with the acmnumeric or acmauthoryear sytles, that include
%% support for advanced citation of software artefact from the
%% biblatex-software package, also separately available on CTAN.
%%
%% Look at the sample-*-biblatex.tex files for templates showcasing
%% the biblatex styles.
%%

%%
%% The majority of ACM publications use numbered citations and
%% references.  The command \citestyle{authoryear} switches to the
%% "author year" style.
%%
%% If you are preparing content for an event
%% sponsored by ACM SIGGRAPH, you must use the "author year" style of
%% citations and references.
%% Uncommenting
%% the next command will enable that style.
%%\citestyle{acmauthoryear}


\usepackage{algorithmic}
\usepackage{adjustbox}
\usepackage{subfig}

\newcommand{\partitle}[1]{\vspace{1mm}\noindent\textbf{#1.}}

%%
%% end of the preamble, start of the body of the document source.
\begin{document}

%%
%% The "title" command has an optional parameter,
%% allowing the author to define a "short title" to be used in page headers.
% \title{LLM-as-a-Rel: Benchmarking Automatic Relevance Judgments}
% \title{LLM-as-a-Rel: A Collection of LLM-Generated Relevance Judgements to Benchmark Automatic Annotations}
\title{Judging the Judges: \\ A Collection of LLM-Generated Relevance Judgements}

%%
%% The "author" command and its associated commands are used to define
%% the authors and their affiliations.
%% Of note is the shared affiliation of the first two authors, and the
%% "authornote" and "authornotemark" commands
%% used to denote shared contribution to the research.

\author{
Hossein A.~Rahmani
}
\orcid{0000-0002-2779-4942} 
\affiliation{%
        \institution{University College London}
        \city{London}
        \country{UK}
}
\email{hossein.rahmani.22@ucl.ac.uk}

\author{Clemencia Siro}
\orcid{0000-0001-5301-4244} 
\affiliation{%
        \institution{University of Amsterdam}
        \city{Amsterdam}
        \country{The Netherlands}
}
\email{c.n.siro@uva.nl}


\author{
Mohammad Aliannejadi
}
\orcid{0000-0002-9447-4172} 
\affiliation{%
        \institution{University of Amsterdam}
        \city{Amsterdam}
        \country{The Netherlands}
}
\email{m.aliannejadi@uva.nl}

\author{Nick Craswell}
\orcid{0000-0002-9351-8137} 
\affiliation{%
        \institution{Microsoft}
        \city{Bellevue}
        \country{US}
}
\email{nickcr@microsoft.com}

\author{Charles L.~A.~Clarke}
\orcid{0000-0001-8178-9194}
\affiliation{%
  \institution{University of Waterloo}
  \city{Waterloo, Ontario}
  \country{Canada}
}
\email{claclark@gmail.com}

\author{Guglielmo Faggioli}
\orcid{0000-0002-5070-2049} 
\affiliation{%
        \institution{University of Padua}
        \city{Padua}
        \country{Italy}
}
\email{faggioli@dei.unipd.it}


\author{Bhaskar Mitra}
\orcid{0000-0002-5270-5550} 
\affiliation{%
        \institution{Microsoft}
        \city{Montréal}
        \country{Canada}
}
\email{bmitra@microsoft.com}

\author{Paul Thomas}
\orcid{0000-0003-2425-3136} 
\affiliation{%
        \institution{Microsoft}
        \city{Adelaide}
        \country{Australia}
}
\email{pathom@microsoft.com}

\author{Emine Yilmaz}
\orcid{0000-0003-4734-4532} 
\affiliation{%
        \institution{University College London \& Amazon}
        \city{London}
        \country{UK}
}
\email{emine.yilmaz@ucl.ac.uk}
%%
%% By default, the full list of authors will be used in the page
%% headers. Often, this list is too long, and will overlap
%% other information printed in the page headers. This command allows
%% the author to define a more concise list
%% of authors' names for this purpose.
\renewcommand{\shortauthors}{Rahmani et al.}

%%
%% The abstract is a short summary of the work to be presented in the
%% article.
\begin{abstract}
Using Large Language Models (LLMs) for relevance assessments offers promising opportunities to improve Information Retrieval (IR), Natural Language Processing (NLP), and related fields. Indeed, LLMs hold the promise of allowing IR experimenters to build evaluation collections with a fraction of the manual human labor currently required. This could help with fresh topics on which there is still limited knowledge and could mitigate the challenges of evaluating ranking systems in low-resource scenarios, where it is challenging to find human annotators. Given the fast-paced recent developments in the domain, many questions concerning LLMs as assessors are yet to be answered. Among the aspects that require further investigation, we can list the impact of various components in a relevance judgment generation pipeline, such as the prompt used or the LLM chosen.

This paper benchmarks and reports on the results of a large-scale automatic relevance judgment evaluation, the \textit{LLMJudge challenge} at SIGIR 2024, where different relevance assessment approaches were proposed. In detail, we release and benchmark 42 LLM-generated labels of the TREC 2023 Deep Learning track relevance judgments produced by eight international teams who participated in the challenge. Given their diverse nature, these automatically generated relevance judgments can help the community not only investigate systematic biases caused by LLMs but also explore the effectiveness of ensemble models, analyze the trade-offs between different models and human assessors, and advance methodologies for improving automated evaluation techniques.
The released resource is available at the following link: \url{https://llm4eval.github.io/LLMJudge-benchmark/}. 
\end{abstract}

%%
%% The code below is generated by the tool at http://dl.acm.org/ccs.cfm.
%% Please copy and paste the code instead of the example below.
%%
% \begin{CCSXML}
% <ccs2012>
%    <concept>
%        <concept_id>10002951.10003317</concept_id>
%        <concept_desc>Information systems~Information retrieval</concept_desc>
%        <concept_significance>500</concept_significance>
%        </concept>
%  </ccs2012>
% \end{CCSXML}

% \ccsdesc[500]{Information systems~Information retrieval}

%%
%% Keywords. The author(s) should pick words that accurately describe
%% the work being presented. Separate the keywords with commas.
% \keywords{LLM-as-a-Rel, LLM4Eval, Relevance Judgment, Benchmark}
%% A "teaser" image appears between the author and affiliation
%% information and the body of the document, and typically spans the
%% page.

% \received{20 February 2007}
% \received[revised]{12 March 2009}
% \received[accepted]{5 June 2009}

%%
%% This command processes the author and affiliation and title
%% information and builds the first part of the formatted document.
\maketitle


% humans are sensitive to the way information is presented.

% introduce framing as the way we address framing. say something about political views and how information is represented.

% in this paper we explore if models show similar sensitivity.

% why is it important/interesting.



% thought - it would be interesting to test it on real world data, but it would be hard to test humans because they come already biased about real world stuff, so we tested artificial.


% LLMs have recently been shown to mimic cognitive biases, typically associated with human behavior~\citep{ malberg2024comprehensive, itzhak-etal-2024-instructed}. This resemblance has significant implications for how we perceive these models and what we can expect from them in real-world interactions and decisionmaking~\citep{eigner2024determinants, echterhoff-etal-2024-cognitive}.

The \textit{framing effect} is a well-known cognitive phenomenon, where different presentations of the same underlying facts affect human perception towards them~\citep{tversky1981framing}.
For example, presenting an economic policy as only creating 50,000 new jobs, versus also reporting that it would cost 2B USD, can dramatically shift public opinion~\cite{sniderman2004structure}. 
%%%%%%%% 图1:  %%%%%%%%%%%%%%%%
\begin{figure}[t]
    \centering
    \includegraphics[width=\columnwidth]{Figs/01.pdf}
    \caption{Performance comparison (Top-1 Acc (\%)) under various open-vocabulary evaluation settings where the video learners except for CLIP are tuned on Kinetics-400~\cite{k400} with frozen text encoders. The satisfying in-context generalizability on UCF101~\cite{UCF101} (a) can be severely affected by static bias when evaluating on out-of-context SCUBA-UCF101~\cite{li2023mitigating} (b) by replacing the video background with other images.}
    \label{fig:teaser}
\end{figure}


Previous research has shown that LLMs exhibit various cognitive biases, including the framing effect~\cite{lore2024strategic,shaikh2024cbeval,malberg2024comprehensive,echterhoff-etal-2024-cognitive}. However, these either rely on synthetic datasets or evaluate LLMs on different data from what humans were tested on. In addition, comparisons between models and humans typically treat human performance as a baseline rather than comparing patterns in human behavior. 
% \gabis{looks good! what do we mean by ``most studies'' or ``rarely'' can we remove those? or we want to say that we don't know of previous work doing both at the same time?}\gili{yeah the main point is that some work has done each separated, but not all of it together. how about now?}

In this work, we evaluate LLMs on real-world data. Rather than measuring model performance in terms of accuracy, we analyze how closely their responses align with human annotations. Furthermore, while previous studies have examined the effect of framing on decision making, we extend this analysis to sentiment analysis, as sentiment perception plays a key explanatory role in decision-making \cite{lerner2015emotion}. 
%Based on this, we argue that examining sentiment shifts in response to reframing can provide deeper insights into the framing effect. \gabis{I don't understand this last claim. Maybe remove and just say we extend to sentiment analysis?}

% Understanding how LLMs respond to framing is crucial, as they are increasingly integrated into real-world applications~\citep{gan2024application, hurlin2024fairness}.
% In some applications, e.g., in virtual companions, framing can be harnessed to produce human-like behavior leading to better engagement.
% In contrast, in other applications, such as financial or legal advice, mitigating the effect of framing can lead to less biased decisions.
% In both cases, a better understanding of the framing effect on LLMs can help develop strategies to mitigate its negative impacts,
% while utilizing its positive aspects. \gabis{$\leftarrow$ reading this again, maybe this isn't the right place for this paragraph. Consider putting in the conclusion? I think that after we said that people have worked on it, we don't necessarily need this here and will shorten the long intro}


% If framing can influence their outputs, this could have significant societal effects,
% from spreading biases in automated decision-making~\citep{ghasemaghaei2024understanding} to reducing public trust in AI-generated content~\citep{afroogh2024trust}. 
% However, framing is not inherently negative -- understanding how it affects LLM outputs can offer valuable insights into both human and machine cognition.
% By systematically investigating the framing effect,


%It is therefore crucial to systematically investigate the framing effect, to better understand and mitigate its impact. \gabis{This paragraph is important - I think that right now it's saying that we don't want models to be influenced by framing (since we want to mitigate its impact, right?) When we talked I think we had a more nuanced position?}




To better understand the framing effect in LLMs in comparison to human behavior,
we introduce the \name{} dataset (Section~\ref{sec:data}), comprising 1,000 statements, constructed through a three-step process, as shown in Figure~\ref{fig:fig1}.
First, we collect a set of real-world statements that express a clear negative or positive sentiment (e.g., ``I won the highest prize'').
%as exemplified in Figure~\ref{fig:fig1} -- ``I won the highest prize'' positive base statement. (2) next,
Second, we \emph{reframe} the text by adding a prefix or suffix with an opposite sentiment (e.g., ``I won the highest prize, \emph{although I lost all my friends on the way}'').
Finally, we collect human annotations by asking different participants
if they consider the reframed statement to be overall positive or negative.
% \gabist{This allows us to quantify the extent of \textit{sentiment shifts}, which is defined as labeling the sentiment aligning with the opposite framing, rather then the base sentiment -- e.g., voting ``negative'' for the statement ``I won the highest prize, although I lost all my friends on the way'', as it aligns with the opposite framing sentiment.}
We choose to annotate Amazon reviews, where sentiment is more robust, compared to e.g., the news domain which introduces confounding variables such as prior political leaning~\cite{druckman2004political}.


%While the implications of framing on sensitive and controversial topics like politics or economics are highly relevant to real-world applications, testing these subjects in a controlled setting is challenging. Such topics can introduce confounding variables, as annotators might rely on their personal beliefs or emotions rather than focusing solely on the framing, particularly when the content is emotionally charged~\cite{druckman2004political}. To balance real-world relevance with experimental reliability, we chose to focus on statements derived from Amazon reviews. These are naturally occurring, sentiment-rich texts that are less likely to trigger strong preexisting biases or emotional reactions. For instance, a review like ``The book was engaging'' can be framed negatively without invoking specific cultural or political associations. 

 In Section~\ref{sec:results}, we evaluate eight state-of-the-art LLMs
 % including \gpt{}~\cite{openai2024gpt4osystemcard}, \llama{}~\cite{dubey2024llama}, \mistral{}~\cite{jiang2023mistral}, \mixtral{}~\cite{mistral2023mixtral}, and \gemma{}~\cite{team2024gemma}, 
on the \name{} dataset and compare them against human annotations. We find  that LLMs are influenced by framing, somewhat similar to human behavior. All models show a \emph{strong} correlation ($r>0.57$) with human behavior.
%All models show a correlation with human responses of more than $0.55$ in Pearson's $r$ \gabis{@Gili check how people report this?}.
Moreover, we find that both humans and LLMs are more influenced by positive reframing rather than negative reframing. We also find that larger models tend to be more correlated with human behavior. Interestingly, \gpt{} shows the lowest correlation with human behavior. This raises questions about how architectural or training differences might influence susceptibility to framing. 
%\gabis{this last finding about \gpt{} stands in opposition to the start of the statement, right? Even though it's probably one of the largest models, it doesn't correlate with humans? If so, better to state this explicitly}

This work contributes to understanding the parallels between LLM and human cognition, offering insights into how cognitive mechanisms such as the framing effect emerge in LLMs.\footnote{\name{} data available at \url{https://huggingface.co/datasets/gililior/WildFrame}\\Code: ~\url{https://github.com/SLAB-NLP/WildFrame-Eval}}

%\gabist{It also raises fundamental philosophical and practical questions -- should LLMs aim to emulate human-like behavior, even when such behavior is susceptible to harmful cognitive biases? or should they strive to deviate from human tendencies to avoid reproducing these pitfalls?}\gabis{$\leftarrow$ also following Itay's comment, maybe this is better in the dicsussion, since we don't address these questions in the paper.} %\gabis{This last statement brings the nuance back, so I think it contradicts the previous parapgraph where we talked about ``mitigating'' the effect of framing. Also, I think it would be nice to discuss this a bit more in depth, maybe in the discussion section.}






\section{Related Work}
%%%%%%%%%%%%%%
% Know-Item Retrieval and Query Simulation
%%%%%%%%%%%%%%
\subsection{Query Simulation and Know-Item Retrieval}

Query simulation methods have been used for various purposes, including document expansion \cite{nogueira2019docT5query} and synthetic test collection generation \cite{Rahmani24synthetic}. In the context of known-item retrieval, these methods have been explored to improve retrieval strategies \cite{OgilvieCallan03combining} and evaluation frameworks \cite{Azzopardi06testbeds, hagen2015corpus}.



%% Query Simulation
\textit{Simulating} the known-item queries has long been an active research area \cite{balog2006overviewWebclef, Azzopardi07SimulatedQueries, Kim09desktop, Elsweiler2011Seeding}.
Early work \cite{Azzopardi07SimulatedQueries} generated synthetic queries using term-based likelihood models, selecting query terms based on their likelihood within a randomly chosen document. Later studies adapted this approach for desktop search \cite{Kim09desktop} and email re-finding \cite{Elsweiler2011Seeding}, demonstrating its effectiveness for simulated evaluations of know-item retrieval models.
%
The \textit{validation} of these query simulators has also been a key focus.
System ranking correlation \cite{balog2006overviewWebclef}, retrieval score distribution comparisons \cite{Azzopardi07SimulatedQueries}, and synthetic versus human query resemblance \cite{Kim09desktop} have been used to assess their reliability.


While valuable, known-item search queries differ significantly from TOT queries, which are longer and more complex. Despite progress in simulating known-item queries, TOT retrieval remains unexplored. This paper bridges that gap by introducing novel TOT query elicitation methods and adapting established validation techniques \cite{zeigler2000theory} to ensure alignment with real-world queries, enabling scalable and accurate simulated evaluations.






%%%%%%%%%%%%%%
% TOT Datasets
%%%%%%%%%%%%%%
\subsection{TOT Datasets}
Several datasets have been developed to support research on TOT retrieval, primarily collected from online CQA platforms and focused on specific domains. MS-TOT \cite{arguello-movie-identification} was constructed from the \textit{IRememberThisMovie} website and human-annotated with tags in the Movie domain. It also includes qualitative coding of TOT queries and demonstrates significant room for improvement in current retrieval technologies for such information needs. Similarly, \citet{gameTOT} collected TOT queries from Reddit's \textit{/r/tipofmyjoystick} subreddit in the Game domain, providing coded tag information. Other datasets include Reddit-TOMT \cite{Bhargav-2022-wsdm}, focused on movies and books from Reddit's \textit{/r/tipofmytongue} subreddit; TOT-Music \cite{Bhargav23MusicTOT}, targeting the Music domain from the same subreddit; and Whatsthatbook \cite{lin-etal-2023-whatsthatbook}, sourced from \textit{GoodReads}, focused on the Book domain.



In response to the domain specificity of these datasets, recent efforts have aimed to expand TOT datasets across multiple areas. \citet{Meier21-complex-reddit} expanded to general casual leisure domains using data from six Reddit subreddits, including games, books, and music, although other identified domains, such as videos and people, remain underrepresented. Similarly, TOMT-KIS \cite{frobe2023-performance-pred} extended the collection from \textit{/r/tipofmytongue} by adapting \citet{Bhargav-2022-wsdm}'s approach with fewer filtering restrictions, resulting in 1.28 million TOT queries. However, only 47\% of these queries have identified answers, and the dataset continues to exhibit severe domain skewness toward a few topics. 


In this work, we develop and validate TOT query elicitation methods using the Movie domain for robust evaluation, then expand to Landmark and Person to assess applicability across underrepresented domains.



\section{LLMJudge Resource}
\label{sec:llmjudge_resource}
This section details how we designed the \texttt{LLMJudge} challenge task, the data construction process, and the evaluation metrics.

\begin{figure}
    \centering
    \subfloat[Dev set\label{fig:sample-dev}]
    {
        {
            \includegraphics[scale=0.26]{figs/data/Full-Human_Sample-Human_black_dev.pdf}
        }
    }%
    % \quad
    \subfloat[Test set\label{fig:sample-test}]
    {
        {
            \includegraphics[scale=0.26]{figs/data/Full-Human_Sample-Human_black_test.pdf}
        }
    }%
    \caption{Samples correlation with TREC 2023 DL full qrel}%
    \label{fig:data-samples}%
\end{figure}

\subsection{LLMJudge Task}
The task of the LLMJudge challenge is, given the query and passage as input, how they are relevant. Similar to TREC 2023 Deep Learning track \cite{craswell2024overview}, we use \textit{four-point scale} judgments to evaluate the relevance of the query to the passage as follows:

\begin{itemize}
    \item\textbf{[3] Perfectly relevant}: The passage is dedicated to the query and contains the exact answer. 
    \item\textbf{[2] Highly relevant}: The passage has some answers for the query, but the answer may be a bit unclear, or hidden amongst extraneous information. 
    \item\textbf{[1] Related}: The passage seems related to the query but does not answer it. 
    \item\textbf{[0] Irrelevant}: The passage has nothing to do with the query. 
\end{itemize}

More specifically, the LLMJudge challenge is, by providing the datasets that include queries, passages, and query-passage files to participants, to ask LLMs to generate a score [0, 1, 2, 3] indicating the relevance of the query to the passage.

\subsection{LLMJudge Data}
The \texttt{LLMJudge} challenge dataset is built upon the passage retrieval task dataset of the TREC 2023 Deep Learning track\footnote{\url{https://microsoft.github.io/msmarco/TREC-Deep-Learning.html}} \cite{craswell2024overview}. The TREC 2023 Deep Learning track qrel consists of 82 queries, including 51 real queries and 31 synthetic queries (13 generated by T5 and 18 generated by GPT-4). To create a dev and test set similar to the TREC 2023 Deep Learning track full qrel, we randomly sampled 15 queries from 51 real queries, 5 queries from T5 queries, and 5 queries from GPT-4 queries for each set. Figure \ref{fig:data-samples} shows Kendall's $\tau$ correlation of the TREC 2023 Deep Learning track run submission on LLMJudge sampled dev and test sets with the TREC 2023 Deep Learning track full qrel. Table \ref{tbl:llmjudge-dataset} shows the statistics of the \texttt{LLMJudge} challenge datasets. The test set is used for the generation of judgment by participants, while the development set could be used for few-shot or fine-tuning purposes.

\begin{table}
    % \centering
    \caption{Statistics of LLMJudge Dataset}
    \label{tbl:llmjudge-dataset}
        \begin{tabular}{lcc}
            \toprule
            \textbf{} & \textbf{Dev} & \textbf{Test} \\
            \midrule
             \# queries  & 25    & 25 \\
             \# passage & 7,224 & 4,414 \\
             \# qrels    & 7,263 & 4,423 \\
             \midrule
             \# irrelevant (0)         & 4,538 & 2,005 \\
             \# related (1)            & 1,403 & 1,233 \\
             \# highly relevant (2)    & 625 & 808 \\
             \# perfectly relevant (3) & 697 & 377 \\
            \bottomrule
        \end{tabular}
\end{table}

\subsection{Evaluation}
We evaluate submission results on two different levels, the correlation of the judgments and the ranking correlation of systems evaluated using judgment submissions:

\begin{itemize}
    \item \textbf{Label Correlation.} We use the automated evaluation metrics Cohen's Kappa ($\kappa$) and Krippendorff's Alpha ($\alpha$) on human judgments and the judgments submitted by participants;
    \item \textbf{System Ranking Correlation.} We use Kendall's Tau ($\tau$) and Spearman's rank ($\rho$) correlation to evaluate the system ordering of TREC 2023 Deep Learning Track \cite{craswell2024overview} submitted systems on human judgments and participants' LLM-based judgments.
\end{itemize}

We use \texttt{scikit-learn}\footnote{\url{https://scikit-learn.org/stable/index.html}} to compute Cohen's $\kappa$, Kendall's $\tau$, Spearman's $\rho$. Krippendorff's $\alpha$ is also calculated using the Fast Krippendorff\footnote{\url{https://github.com/pln-fing-udelar/fast-krippendorff}} Python package. 

\subsection{Publicly Available Resources}
To facilitate research in the area we have made the LLMJudge dataset, sample prompt, quick starter for automatic judgment, submitted runs, prompts, codes for quick starting the evaluation, and more detailed results publicly available on the \texttt{LLMJudge} webpage at: \url{https://llm4eval.github.io/LLMJudge-benchmark/}.
\section{Submitted Runs}
\label{sec:intro-methods}
We provide all submitted runs as a resource for future research and comparison. The submissions include 9 \emph{baseline approaches} developed by the organizers and 33 \emph{methods from participating teams}. Analysis of these submissions reveals several methodological directions in \ac{LLM}-based relevance assessment, focusing on \emph{prompting techniques, model adaptation, multi-phase evaluation, aggregation strategies, and classification-based refinement}.  

\begin{table}
    \centering
    \caption{LLMJudge challenge submissions details. Ensemble (Ens.) indicates if submissions combine multiple judges or use them as features to train a classifier for judgment. LR: Logistic Regression, ET: ExtraTrees, GaussianNB: Gaussian Naive Bayes are classifiers. If a submission used multiple prompts, we consider the more advanced one (CoT > Zero-Shot) in this table. FT: Fine-Tuning, N: Numerical, S: Semantic.}
    \adjustbox{max width=\columnwidth}{%
    \begin{tabular}{llccccc}
    \toprule
    \textbf{Submission ID} & \textbf{Model} & \textbf{Size} & \textbf{FT} & \textbf{Prompt} & \textbf{Label} & \textbf{Ens.} \\
    \midrule
    NISTRetrieval-instruct0 & Llama-3-Instruct & 8B & - & Zero-shot & N & - \\
    NISTRetrieval-instruct1 & Llama-3-Instruct & 8B & - & Zero-shot & N & - \\
    NISTRetrieval-instruct2 & Llama-3-Instruct & 8B & - & Zero-shot & N & - \\
    NISTRetrieval-reason0 & Llama-3-Instruct & 8B & - & CoT & N & - \\
    NISTRetrieval-reason1 & Llama-3-Instruct & 8B & - & CoT & N & - \\
    NISTRetrieval-reason2 & Llama-3-Instruct & 8B & - & CoT & N & - \\
    Olz-exp & GPT-4o & - & - & Zero-Shot & S & - \\
    Olz-gpt4o & GPT-4o & - & - & CoT & S & -  \\
    Olz-halfbin & Llama-3-Instruct & 8B & - & CoT & S + N & LR \\
    Olz-somebin & Llama-3-Instruct & 8B  & - & CoT & S + N & LR \\
    Olz-multiprompt & Llama-3-Instruct & 8B & - & CoT & S + N & \checkmark \\
    RMITIR-GPT4o & GPT-4o & - & - & Zero-Shot & N & \checkmark \\
    RMITIR-llama38b & Llama-3-Instruct & 8B & - & Zero-Shot & N & \checkmark \\
    RMITIR-llama70B & Llama-3-Instruct & 70B & - & Zero-Shot & N & \checkmark \\
    TREMA-4prompts & Llama-3-Instruct & 8B & - & Zero-Shot & N & - \\
    TREMA-CoT & Llama-3-Instruct & 8B & - & CoT & N & - \\
    TREMA-all & ChatGPT-3.5/FlanT5-Large & 783M & - & Few-Shot & N & ET \\
    TREMA-direct & ChatGPT-3.5/FlanT5-Large & 783M & - & Few-Shot & N & ET \\
    TREMA-naiveBdecompose & ChatGPT-3.5/FlanT5-Large & 783M & - & Zero-Shot & N & GNB \\
    TREMA-nuggets & ChatGPT-3.5/FlanT5-Large & 783M & - & Zero-Shot & N & ET \\
    TREMA-other & ChatGPT-3.5/FlanT5-Large & 783M & - & Zero-Shot & N & - \\
    TREMA-questions & ChatGPT-3.5/FlanT5-Large & 783M & - & Zero-Shot & N & ET \\
    TREMA-rubric0 & ChatGPT-3.5/FlanT5-Large & 783M & - & Zero-Shot & N & - \\
    TREMA-sumdecompose & Llama-3-Instruct & 8B & - & Zero-Shot & N & - \\
    h2oloo-fewself & GPT-4o & - & - & Few-Shot & N & - \\
    h2oloo-zeroshot1 & Llama-3-Instruct & 8B & \checkmark & Zero-Shot & N & - \\
    h2oloo-zeroshot2 & Llama-3-Instruct & 8B & \checkmark & Zero-Shot & N & - \\
    llmjudge-cot1 & GPT-3.5-turbo & - & - & CoT & N & - \\
    llmjudge-cot2 & GPT-3.5-turbo-16k & - & - & CoT & N & - \\
    llmjudge-cot3 & GPT-4-32k & - & - & CoT & N & - \\
    llmjudge-simple1 & GPT-3.5-turbo & - & - & Zero-Shot & N & - \\
    llmjudge-simple2 & GPT-3.5-turbo-16k & - & - & Zero-Shot & N & - \\
    llmjudge-simple3 & GPT-4-32k & - & - & Zero-shot & N & - \\
    llmjudge-thomas1 & GPT-3.5-turbo & - & - & Zero-Shot & N & - \\
    llmjudge-thomas2 & GPT-3.5-turbo-16k & - & - & Zero-Shot & N & - \\
    llmjudge-thomas3 & GPT-4-32k & - & - & Zero-Shot & N & - \\
    prophet-setting1 & Llama-3-Instruct & 8B & \checkmark & Zero-Shot & S & - \\
    prophet-setting2 & Llama-3-Instruct & 8B & \checkmark & Zero-Shot & S & - \\
    prophet-setting4 & Llama-3-Instruct & 8B & \checkmark & Zero-Shot & S & - \\
    willia-umbrela1 & GPT-4o & - & - & Zero-Shot & N & - \\
    willia-umbrela2 & GPT-4o & - & - & Zero-Shot & S & - \\
    willia-umbrela3 & GPT-4o & - & - & Zero-Shot & S + N & \checkmark \\
    \bottomrule
    \end{tabular}
    }
    \label{tab:infos}
\end{table}

Most submissions implement either \emph{direct prompting} or \emph{criteria decomposition pipelines}. \emph{Direct prompting methods} range from simple \emph{relevance scoring instructions} to \emph{chain-of-thought reasoning}, where \acp{LLM} justify their judgments before assigning a score. Some approaches explore \emph{zero-shot prompting}, while others incorporate \emph{semantic label assignments, linguistic alignment, or multi-prompt aggregation} to improve consistency and reduce overestimation biases.  
Beyond prompting, some teams \emph{fine-tune LLMs on relevance datasets}, including \emph{TREC Deep Learning track qrels} and the \emph{LLMJudge development set}, testing different \emph{model sizes (8B vs.~70B)} to assess the impact of adaptation on evaluation performance.  
A subset of submissions structures evaluation into \emph{multi-phase pipelines}, applying \emph{binary filtering before graded scoring}, \emph{question-based reasoning}, or \emph{decision trees} to refine assessments. Other approaches decompose relevance into \emph{specific dimensions} such as \emph{exactness, coverage, topicality, and contextual fit}, or employ \emph{nugget-based assessments} for more granular judgments.  
To enhance robustness, several methods \emph{combine outputs from multiple prompts or models} using \emph{multi-prompt averaging, binary-to-graded conversions, or conservative ensembling} to stabilize scores. Others treat \emph{relevance assessment as a classification task}, extracting features from LLM outputs and training \emph{Machine Learning classifiers} to refine final scores and improve alignment with human judgments.  

Below we detail the baselines and a summary of the submitted runs. We also summarize the submission details in Table \ref{tab:infos}.

\partitle{LLMJudge Baseline}
The baseline judges provided by the LLMJudge challenge organizers serve as reference methods for evaluation. Three distinct approaches are proposed as baselines: \texttt{llmjudge\\-simple}, \texttt{llmjudge-cot}, and \texttt{llmjudge-thomas}. The \texttt{llmjudge-\\simple} method employs a straightforward prompt, instructing the model to directly provide a relevance judgment based on the query and passage. In contrast, \texttt{llmjudge-cot} adopts a chain-of-thought (CoT) approach, prompting the model to articulate its reasoning process before delivering a judgment. Lastly, \texttt{llmjudge-thomas} incorporates the prompt design introduced by \cite{thomas2023large}, offering an alternative strategy for evaluation.

\partitle{NISTRetrieval-instruct}
This is a submission from NIST which has three different variants, namely, \texttt{NISTRetrieval-instruct0}, \texttt{NISTRetrieval-instruct1}, and \texttt{NISTRetrieval-instruct2} that aims to investigate the reproducibility of the method proposed by \citet{thomas2023large} and the reproducibility capabilities of LLMs when we used them for automatic relevance judgment.

\partitle{NISTRetrieval-reason}
Similar to \texttt{NISTRetrieval-instruct}, this NIST submission includes three related methods -- \texttt{NISTRetrieval-\\reason0}, \texttt{NISTRetrieval-reason1}, and \texttt{NISTRetrieval-reason2}. The team observed that prompting LLMs to provide reasoning across various tasks could improve response quality. To examine whether this approach could also enhance relevance judgment, they modified the prompt from \citet{thomas2023large} to allow the LLM to generate reasoning. These three runs were included to assess the reproducibility capabilities of LLMs when used for evaluation.

\partitle{Prophet-setting}
This method builds on the idea of fine-tuning an LLM with different available datasets for automatic relevance judgment, as described in \cite{meng2024query}\footnote{Code is available at \url{https://github.com/ChuanMeng/QPP-GenRE}}. Specifically, they fine-tuned \texttt{Llama-3-8B} under three different settings, training the model for five epochs in each. These settings include: \texttt{Prophet-setting1}, fine-tuned on the LLMJudge development set; \texttt{Prophet-setting2}, fine-tuned on the qrels of TREC-DL 2019, 2020, and 2021; and \texttt{Prophet-setting4}, which combines fine-tuning on the qrels of TREC-DL 2019, 2020, and 2021 with the LLMJudge development set.

\partitle{William-umbrela1}
This approach is zero-shot prompting the LLM to produce relevance assessments. They used UMBRELA \cite{upadhyay2024umbrela} to generate relevance judgments using the prompting technique suggested by \citet{thomas2023large}. The team mentioned that ``\textit{I tried many different approaches, but I did not manage to find anything that really seemed to consistently improve on zero-shot. It seemed like this dataset may have been harder and/or noisier than others referenced in the literature -- on my development set it was hard to get $> 0.3$ Cohen's $\kappa$, whereas the literature mentions values of $0.4$ up to $0.6$ even.}''.

\partitle{William-umbrela2}
The main idea of this method is to take the approach from the UMBRELA \cite{upadhyay2024umbrela} -- zero-shot prompting technique from \citet{thomas2023large}, but to see if the performance would be improved by asking the model to output semantic labels (i.e., \textit{Irrelevant}, \textit{Related}, \textit{Highly relevant}, \textit{Perfectly relevant}), rather than a numerical score (i.e., 0, 1, 2, 3).

\partitle{William-umbrela3}
This method is an ensemble of~ \texttt{William-\\umbrela1} and \texttt{William-umbrela2} approaches by taking the \textit{min}. The team mentioned that ``\textit{The logic behind using min as an aggregator is that in this dataset, it pays to be conservative in the rating. They also said that on a subset of the training data that they held out for testing, this ensembling approach outperformed either of the two other approaches (i.e., William-umbrela1 and William-umbrela2)''}.

\partitle{H2oloo-fewself}
This method uses the best prompt proposed by Thomas et al.~\cite{thomas2023large} to instruct GPT-4o. It incorporates few-shot examples to guide the model in distinguishing between relevant labels effectively.

\partitle{H2oloo-zeroshot1}
This method fined-tuned a Llama-8B using the TREC DL 2019 to 2022 qrels for relevance judgment prediction.

\partitle{H2oloo-zeroshot2}
This method fined-tuned a Llama-8B using the TREC DL 2019 to 2022 qrels and the LLMJudge dev set qrel for relevance judgment prediction.

\partitle{Olz-gpt4o}
This method uses a simple prompt where they just ask for the relevance judgment without any special techniques. The idea is to see how models can solve relevance judgment tasks without considering any particular prompting or fine-tuning techniques. The primary goal is to assess whether a low-effort prompt could reliably derive relevance labels from LLMs that are practically usable.

\partitle{Olz-exp}
This method is similar to \texttt{Olz-gpt4o} but they also asked LLM to reason its judgment as part of the evaluation.

\partitle{Olz-halfbin}
This method leverages Llama-3 models with $8B$ and $70B$ parameters to assess document relevance using nine distinct prompts. These prompts are divided into two categories: four \emph{graded relevance prompts}, which instruct the model to assign a score from 0 to 3 with slight instruction variations, and five \emph{binary relevance prompts}, which require binary judgments with different definitions of relevance.  
Both model variants generate outputs for all nine prompts. These outputs serve as features for training a logistic regression classifier, which produces the final graded labels. Training is conducted using labels generated by GPT-4o (via the \texttt{Olz-gpt4o} method) rather than the standard development set annotations, based on the assumption that the development and test set labels may have been derived using different methods. Analyzing these discrepancies, the team found GPT-4o’s judgments more aligned with their expectations, leading to its adoption as the primary reference for training.

\partitle{Olz-somebin}
The procedure of this method is identical to the \texttt{Olz-halfbin} method, except the logistic regression classifier was trained on the provided development set labels instead of those generated by GPT-4o (using \texttt{Olz-gpt4o} method).

\partitle{Olz-multiprompt}
This method, instead of using a classifier like \texttt{Olz-halfbin} and \texttt{Olz-somebin}, directly aggregated the relevance judgments by averaging. The binary labels were first scaled by multiplying them by three (to convert them into $0$ or $3$). Then a simple average was calculated across the nine prompts and rounded on a scale of 0 to 3, and the resulting value served as the final graded label.

\partitle{RMIT-IR}
This submission introduces three relevance assessors, \texttt{RMITIR-GPT4o}, \texttt{RMITIR-llama38b}, and \texttt{RMITIR-llama70B}. The proposed approach begins by having the LLM provide a binary relevance judgment to filter out irrelevant queries and improve irrelevance filtering. Next, three scores are generated, and averaged, and the result is rounded to produce the final score. The method was tested using three different LLMs: GPT-4o (\texttt{RMITIR-GPT4o}), Llama3-8B (\texttt{RMITIR-llama38b}), and Llama3-70B (\texttt{RMITIR-llama70B}). The team noted that ``\textit{GPT-4o appears to be the best-performing model based on our experiences}.''

\partitle{TREMA-4prompt}   
This method evaluates passage relevance by decomposing it into four specific criteria: exactness (how precisely the passage answers the query), coverage (proportion of content discussing the query), topicality (subject alignment between passage and query), and contextual fit (presence of relevant background). The evaluation follows a two-phase process where each criterion is first assessed independently and then combined through a final prompt to determine the overall relevance label. Full details of the criteria and rationale are provided in \cite{farzi2024best}.

\partitle{TREMA-CoT}
This method implements a chain-of-thought evaluation process inspired by \citet{Sun2023IsCG}. The approach consists of three phases: First, the \ac{LLM} makes a binary relevance judgment (yes/no) of the passage. Based on this judgment, different relevance criteria are evaluated in the second phase - for relevant passages, exactness and coverage are assessed, while non-relevant passages are evaluated on contextual fit and topicality (all scored 0-3). In the final phase, these scores determine the overall relevance label: relevant passages receive labels 2-3 based on exactness and coverage scores, while non-relevant passages receive labels 0-1 based on contextual fit and topicality assessment.

\partitle{TREMA-other}
This approach investigates whether aligning the linguistic styles of queries and passages can enhance relevance judgments. In the first phase, the LLM generates a query-like representation for each passage, designed to match the query's linguistic style and length. This generated query serves as a summary of the passage's content, formatted in a way that aligns with typical query phrasing. In the second phase, the LLM evaluates the similarity between the original query and the generated query on a scale from 0 to 3, corresponding to the relevance labeling system. Higher similarity scores indicate a stronger alignment between the passage's content and the query's intent. This method integrates linguistic style alignment with content relevance to improve relevance labeling.

\partitle{TREMA-sumdecompose}
This method consists of two phases. Phase one is identical to the \texttt{TREMA-4prompt} method, where the ``relevance'' is decomposed into four criteria, leading to four criteria-specific grades.  In Phase Two, the individual grades from Phase One are summed to produce a total grade. Based on this total, a final relevance label between 0 and 3 is assigned to each query-passage pair: a total grade of 10-12 yields a relevance label of 3, 7-9 yields a relevance label of 2, 5-6 yields a relevance label of 1, and 0-4 yields a relevance label of 0.

\partitle{TREMA-naiveBdecompose}
This method consists of two phases. Phase one is identical to the \texttt{TREMA-4prompt} method, where the ``relevance'' is decomposed into four criteria, leading to four criteria-specific grades. In phase two, these decomposed grades are aggregated into a final relevance label using a Gaussian Naive Bayes model, implemented with Scikit-learn's GaussianNB() classifier. The model is trained on the decomposed feature grades and then predicts the relevance label for each passage.

\partitle{TREMA-rubric0}
This method is based on the RUBRIC Autograder Workbench \cite{dietz2024workbench}. This method defines the relevance of the query via 10 open-ended questions. The questions are generated using the ChatGPT 3.5 model. Each passage is scanned whether it is possible to answer each of the questions (and how well), which is captured as a grade. They use the FLAN-T5-large LLM from Huggingface to grade the answerability from 0 (worst) to 5 (best). Details and prompts are available in the Workbench benchmark \cite{dietz2024workbench}. The grades are mapped to relevance labels by a heuristic mapping on the second-highest grade achieved on any of the questions. Grade 5 is mapped to relevance label 3, grade 4 is mapped to label 1 and all other grades are mapped to label 0. This was the best manual mapping on the dev set \cite{farzi2024rubric}.

\partitle{TREMA-questions}
Same question and grading as in \texttt{TREMA\\-rubric0}, but uses a more elaborate calibration for converting grades to relevance labels, based on \texttt{scikit-learn}'s ExtraTrees classifier. The classifier is based on features that include ranked grades for each question (sorted in descending order), ranked question difficulty (based on average grades across the pool), and counts of correct answers at various grade thresholds (e.g., number of answers graded 5, 4 or better, etc.). Each of these features is encoded using both one-hot and numerical representations to capture detailed information about question-based relevance. The classifier is trained on the dev set.

\begin{table}[]
    \centering
    \caption{Judgment and system ranking correlation of LLMJudge submissions. $\kappa$: Cohen's Kappa, $\alpha$: Krippendorff's alpha, $\tau$: Kendall's Tau, $\rho$: Spearman’s rank correlation. The best results per column are denoted in bold and the second best results are denoted in \textit{italic}.}
    \adjustbox{max width=\columnwidth}{%
    \begin{tabular}{lcccc}
    \toprule
    \textbf{Submission ID} & \textbf{$\kappa$} & \textbf{$\alpha$} & \textbf{$\tau$} & \textbf{$\rho$} \\
    \midrule
    NISTRetrieval-instruct0 & 0.1877 & 0.3819 & 0.9440 & 0.9907 \\
    NISTRetrieval-instruct1 & 0.1874 & 0.3812 & 0.9440 & 0.9907 \\
    NISTRetrieval-instruct2 & 0.1880 & 0.3821 & 0.9440 & 0.9907 \\
    NISTRetrieval-reason0   & 0.1844 & 0.3874 & 0.9052 & 0.9810 \\
    NISTRetrieval-reason1   & 0.1845 & 0.3872 & 0.9009 & 0.9802 \\
    NISTRetrieval-reason2   & 0.1838 & 0.3874 & 0.9052 & 0.9810 \\
    Olz-exp                 & 0.2519 & 0.4701 & 0.9009 & 0.9819 \\
    Olz-gpt4o               & 0.2625 & 0.5020 & 0.8793 & 0.9758 \\
    Olz-halfbin             & 0.2064 & 0.4536 & 0.9085 & 0.9830 \\
    Olz-multiprompt         & 0.2445 & 0.4551 & 0.9267 & 0.9867 \\
    Olz-somebin             & 0.2109 & 0.4471 & 0.9042 & 0.9822 \\
    RMITIR-GPT4o            & 0.2388 & 0.4108 & 0.8966 & 0.9798 \\
    RMITIR-llama38b         & 0.2006 & 0.3873 & 0.8879 & 0.9758 \\
    RMITIR-llama70B         & 0.2654 & 0.4873 & 0.9353 & 0.9883 \\
    TREMA-4prompts          & 0.1829 & 0.2888 & \textit{0.9483} & \textbf{0.9919} \\
    TREMA-CoT               & 0.1961 & 0.3852 & 0.8956 & 0.9799 \\
    TREMA-all               & 0.1471 & 0.3855 & 0.9138 & 0.9863 \\
    TREMA-direct            & 0.1742 & 0.3729 & 0.9009 & 0.9819 \\
    TREMA-naiveBdecompose   & 0.1741 & 0.3579 & 0.9128 & 0.9838 \\
    TREMA-nuggets           & 0.0604 & 0.1691 & 0.8664 & 0.9718 \\
    TREMA-other             & 0.1408 & 0.2712 & 0.8276 & 0.9447 \\
    TREMA-questions         & 0.1137 & 0.3148 & 0.9095 & 0.9839 \\
    TREMA-rubric0           & 0.0779 & 0.1036 & 0.8276 & 0.9544 \\
    TREMA-sumdecompose      & 0.2088 & 0.3926 & 0.9300 & 0.9870 \\
    h2oloo-fewself          & 0.2774 & \textbf{0.4958} & 0.9085 & 0.9822 \\
    h2oloo-zeroshot1        & \textit{0.2817} & 0.4812 & 0.9181 & 0.9827 \\
    h2oloo-zeroshot2        & 0.2589 & 0.3898 & 0.8353 & 0.9604 \\
    llmjudge-cot1           & 0.1284 & 0.3218 & 0.9267 & 0.9871 \\
    llmjudge-cot2           & 0.1560 & 0.3263 & 0.9267 & 0.9875 \\
    llmjudge-cot3           & 0.2271 & 0.4870 & 0.9267 & 0.9851 \\
    llmjudge-simple1        & 0.0754 & 0.2808 & 0.9181 & 0.9863 \\
    llmjudge-simple2        & 0.1327 & 0.3672 & 0.8966 & 0.9790 \\
    llmjudge-simple3        & 0.2110 & 0.4642 & 0.9052 & 0.9810 \\
    llmjudge-thomas1        & 0.1236 & 0.3207 & 0.8664 & 0.9689 \\
    llmjudge-thomas2        & 0.1723 & 0.3853 & 0.8793 & 0.9750 \\
    llmjudge-thomas3        & 0.2293 & 0.4877 & 0.9181 & 0.9867 \\
    prophet-setting1        & 0.1823 & 0.4069 & 0.9042 & 0.9826 \\
    prophet-setting2        & 0.1757 & 0.3144 & \textbf{0.9516} & \textit{0.9914} \\
    prophet-setting4        & 0.1471 & 0.1623 & 0.8568 & 0.9608 \\
    willia-umbrela1         & \textbf{0.2863} & \textit{0.4918} & 0.9009 & 0.9806 \\
    willia-umbrela2         & 0.2688 & 0.4556 & 0.8870 & 0.9769 \\
    willia-umbrela3         & 0.2741 & 0.4535 & 0.8707 & 0.9730 \\
    \bottomrule
    \end{tabular}
    }
    
    \label{tab:main-results}
\end{table}

\partitle{TREMA-nuggets}
Same approach as \texttt{TREMA-questions}, but uses 10 open-ended key fact nuggets instead of questions, along with an adapted prompt that assesses whether key facts are mentioned in the passage. The same ExtraTrees classifier with the same features is used for converting grades into relevance labels.

\partitle{TREMA-direct}
This approach focuses exclusively on features of direct relevance
labeling methods, which instruct an LLM to judge whether a passage is relevant for a query, using a variety of prompts from \citet{Sun2023IsCG}, \citet{faggioli2023perspectives}, and HELM \cite{liang2022holistic}. The model excludes question-based and nugget-based features, simplifying its input to focus solely on the predictive power of direct labeling. The relevance labels are obtained with an ExtraTrees classifier trained on the dev set. Features include binary or multi-class predictions from labeling approaches. Each label is encoded using both one-hot and numerical encodings to capture both categorical and ordinal aspects of the predictions. This approach is computationally lighter than TREMA-all and serves as a baseline to evaluate how well direct relevance labels alone can predict passage relevance.

\partitle{TREMA-all}
This approach incorporates all features from \texttt{TREMA-\\questions}, \texttt{TREMA-nuggets}, and \texttt{TREMA-direct} approaches via a single ExtraTrees classifier that is trained on the dev set.
\section{Experiments}
\subsection{Implementation Details}
\label{sec:implemenet}
\paragraph{Datasets} We use the Multiview Rendering Dataset \cite{qiu2023richdreamer,zuo2024sparse3d} based on Objaverse \cite{objaverse} for training. The dataset includes 260K objects, with 38 views rendered for each object, with a resolution of $512\times512$. To obtain the surface point clouds, we transform the 3D models according to the rendering settings, filter out those that are not aligned with the rendered images, and use Poisson sampling method\cite{poisson} to sample the surface.  We randomly split the final processed data into training and testing sets, with the training dataset consisting of 200K objects.
% We take our in-domain evaluation using the test set from Objaverse, including 2000 objects. To evaluate our model's cross-domain 能力, We 在Google Scanned Objects(GSO) dataset进行评估,which 包含1030个真实的扫描3D model, and we take 32 views renderd for each model in 球面. 
We conduct our in-domain evaluation using the test set from Objaverse, which includes 2,000 objects. To assess our model's cross-domain capabilities, we evaluate it on the Google Scanned Objects (GSO) \cite{downs2022google}dataset, which contains 1,030 real-world scanned 3D models, with 32 views rendered for each model on a spherical surface.

% 我们使用单图作为输入,以所有available的views作为noval views 来评测我们和所有比较方法的单图生成质量. And take the peak-signal-to-noise ratio (PSNR),
% perceptual quality measure LPIPS, structural similarity index (SSIM) 作为evaluation metrics, which is same to previous work\cite{zou2024triplane, chen2025lara}


% 在我们的实现中,anchor latents的fix length是2048,维度是8. the model dim of Anchor-GS VAE是512, with 两个transformer block in encoder and eight transformer block in decoder. For training the Anchor-GS VAE, we set the weights of the losses with $\lambda_s=1, \lambda_l=1, \lambda_c=1, \lambda_e=1 $ and $\lambda_{KL}=0.001$.
%
% For the Seed-Anchor Mapping Module, we use 24 transformer bolck to implement with a model dim 512, 其中 4 bolcks for downsample and 4 blocks for upsample. For the Seed Points Generation, we use 24 transfomrmer blocks to implement with model dim 512. And thanks to the seed points 的sparse 特性, we can directly learn the distribution of seed points 而不需要去train 一个 VAE. 
%
% We train the Anchor-GS VAE use only a subset of our collected datasets, with a batchsize of 128 on 8 40G A100 with 24K steps. For trainging the Seed-Anchor Mapping Module, we use our full collected datasets, with a batchsize 0f 1280 on 64 32G V100 with 20K steps. For training the Seed Points Generation Module, We training on 48 32G V100 with 54K steps.
% In our implementation, the anchor latents have a fixed length of 2048 and a dimension of 8, and the model dimension in our transformer blocks is 512, each transformer block has two attention layers and a feed-forward layer, similar to \cite{zou2024triplane}. The Anchor-GS VAE consists of two transformer blocks in the encoder and eight transformer blocks in the decoder. For training the Anchor-GS VAE, we random select 8 views,  one view as input and all 8 views as the ground truth images for supervision, and we set the loss weights as follows: \(\lambda_s = 1\), \(\lambda_l = 1\), \(\lambda_c = 1\), \(\lambda_e = 1\), and \(\lambda_{KL} = 0.001\).  


\paragraph{Network}
In our implementation, the anchor latents have a fixed length of 2048 and a dimension of 8. The model dimension in our transformer blocks is 512, with each transformer block comprising two attention layers and a feed-forward layer, following the design in \cite{zou2024triplane}. The Anchor-GS VAE consists of two transformer blocks in the encoder and eight transformer blocks in the decoder. 
%
The Seed-Anchor Mapping Module is implemented using 24 transformer blocks, with 4 blocks for downsampling and 4 blocks for upsampling. Similarly, the Seed Points Generation Module is implemented with 24 transformer blocks. Leveraging the sparsity of seed points, we directly learn their distribution without requiring a VAE. The image conditioning in our model is extracted using DINOv2\cite{oquab2023dinov2}.


\paragraph{Training Details}
For training the Anchor-GS VAE, we randomly select 8 views per object, using one view as the input and all 8 views as ground truth images for supervision. The loss weights are set as \(\lambda_s = 1\), \(\lambda_l = 1\), \(\lambda_c = 1\), \(\lambda_e = 1\), and \(\lambda_{KL} = 0.001\). We train the Anchor-GS VAE on a subset of our collected dataset containing approximately 40K objects, using a batch size of 128 on 8 A100 GPUs (40GB) for 24K steps. The Seed-Anchor Mapping Module is trained on the full dataset with a batch size of 1280 on 64 V100 GPUs(32GB) for 20K steps. The Seed Points Generation Module is trained on 48 V100 GPUs (32GB) for 54K steps.  We use the AdamW optimizer with an initial learning rate of \(4 \times 10^{-4}\), which is gradually reduced to zero using cosine annealing during training. The sampling steps for both the Seed-Anchor Mapping Module and the Seed Points Generation Module are set to 50 during inference.
% For training the Anchor-GS VAE, we randomly select 8 views per object, using one view as the input and all 8 views as ground truth images for supervision. The loss weights are set as follows: \(\lambda_s = 1\), \(\lambda_l = 1\), \(\lambda_c = 1\), \(\lambda_e = 1\), and \(\lambda_{KL} = 0.001\).
% We train the Anchor-GS VAE using a subset of our collected dataset around 40K objects with a batch size of 128 on 8 A100 GPUs (40GB) for 24K steps. The Seed-Anchor Mapping Module is trained on the full dataset with a batch size of 1280 on 64 V100 GPUs (32GB) for 20K steps. For the Seed Points Generation Module, we train on 48 V100 GPUs (32GB) for 54K steps. We use the AdamW optimizer with a learning rate of 4e-4, and the learning rate is cosine anneled to zero during training.


\paragraph{Baseline}
% We compared our methods with 之前的SOTA的3D生成模型 in one image input setting. LGM and LaRa use one image as input, then use multi-view diffusion models to get four views of the object, then get corresponding 3DGS from the multiview images in a feed-forard mamner. TriplaneGS first get dense point clouds from the single input image, then use a triplane to 聚合特征 then get the corresponding attributes of 3DGS, achieving SOTA performances.
We compared our method with previous SOTA 3DGS generation models in the single-image input setting. LGM and LaRA rely on 2D multi-view diffusion priors to obtain multi-view images, which are then used to generate the output 3DGS in a feed-forward manner, as described in ~\ref{sec:related-2d-diffusion}. TriplaneGS~\cite{zou2024triplane} does not require a 2D diffusion prior, directly generating 3DGS from a single input image, as outlined in ~\ref{sec:related_3d}. Both of them achieving SOTA performance. For each compared method, we use the official models and provided weights and ensure careful alignment of the camera parameters.

% Both  LGM~\cite{tang2025lgm} and LaRa~\cite{chen2025lara} take one image as input and then use multi-view diffusion models\cite{shi2023mvdream} to generate four views of the object. These multi-view images are subsequently converted into corresponding 3D Gaussian Splatting (3DGS) representations in a feed-forward manner. TriplaneGS~\cite{zou2024triplane}, on the other hand, first generates dense point clouds from the single input image and then aggregates features using a triplane representation to infer the corresponding attributes of 3DGS, also achieving SOTA performance. For each compared method, we use the official models and provided weights and ensure careful alignment of the camera parameters.\todo{remove duplicate description, mention LaRA is designed for four views input}
% compared methods






\subsection{Results of VAE Reconstruction}
In Fig. \ref{fig:vae}, we present the results of our Anchor-GS VAE. Given point clouds and a single image, our Anchor-GS VAE achieves high-quality reconstructions with detailed geometry and textures.



\subsection{Results of 3D Generation }
\label{sec:comparison}
% 1. 首先讲在哪些数据上进行评估。然后逐个分析结果的值,最后说我们的效果达到了SOTA的效果
% 2. 展示可视化的结果,再逐个分析。表明我们的方法相比于没有用diffusion的方法能更好的学习三维物体的分布。
% Table 1展示了我们的方法和previous SOTA methods在Objaverse和GSO上的评测结果。As described in \ref{sec:implemenet},评测在一个dense viewpoint settings下进行,the results are average use all available objects and viewpoints in the testing datasets. The 多视角不一致 in the multiview diffusion model used by LGM 和 LARA 会导致生成几何不一致的3DGS,特别是导致在新视角下的伪影,So 他们会产生相对较低的值在我们的dense viewpoint评估settings下。相比于他们,我们的方法不借助于2D diffusion先验,可以直接从单张图像中得到理想的3D表示。TriplaneGS使用通过tansformer block一次forward facing得到point clouds,但是这种方式往往不能准确学习到3D点云的分布,always failed in 输入图像中没有的区域。与之相比,我们的方法使用diffusion-based methods, 首先学习一个sparse 的seed points, which is easy to learn and can学习到3D的distribution,then mapping from seed points to anchor latents.SO get more 鲁邦的结果。
\paragraph{Metrics} 
Following previous works \cite{zou2024triplane, chen2025lara}, we use peak signal-to-noise ratio (PSNR), perceptual quality measure LPIPS, and structural similarity index (SSIM) as evaluation metrics to assess different aspects of image similarity between the predicted and ground truth. Additionally, we report the time required to infer a single 3DGS. We use a single image as input and evaluate the 3D generation quality using all available views as testing views to compare our method with others, all renderings are performed at a resolution of 512.

Tab. \ref{tab:quantitative comparison} presents the quantitative evaluation results of our method compared to previous SOTA methods on the Objaverse and GSO datasets, along with qualitative results shown in Fig. \ref{fig:image-3d}. The multi-view diffusion model used in LGM tend to produce more diverse but uncontrollable results, and lacks precise camera pose control. As a result, it fails in our dense viewpoints evaluation, achieving PSNR scores of 12.76 and 13.81 on the Objaverse and GSO test sets, respectively.

As shown in Tab. \ref{tab:quantitative comparison}, LGM and LaRa, influenced by the multi-view inconsistency of 2D diffusion models, achieve relatively lower scores in our dense viewpoint evaluation. In contrast, our method achieves the best results across both datasets, with only a slight overhead in inference time.

Fig. \ref{fig:image-3d} presents the first six rows from the Objaverse dataset and the last three rows from the GSO dataset. All methods are compared using the same camera viewpoints. For the Objaverse dataset, the rendering viewpoints are the left and rear views relative to the input viewpoint, while for the GSO dataset, the views are selected to showcase the object as completely as possible. Compared to methods using 2D diffusion priors, such as LGM and LaRa, our method demonstrates better multi-view geometric consistency, while the former tends to generate artifacts or unrealistic results in our displayed views. Compared to TGS, our method learns the 3D object distribution more effectively, resulting in more geometrically consistent multi-view results, such as the sharp feature in the left view in the first knife case.
% Compared to these methods, 我们的方法能取得更multiview geometry consistent的结果. 例如所有的方法都在第一个case中不能正确的表示输入图像中的sharp feature,导致在所有视图中都是近似的结果.

% As described in Sec. \ref{sec:implemenet}, the evaluations are conducted under a dense viewpoint setting, with the results averaged over all available objects and viewpoints in the testing datasets. The multiview diffusion models used by LGM \todo{add lgm fail reason: Image dream can't control viewpoint}and LaRa exhibit inconsistencies across viewpoints, resulting in geometrically inconsistent 3D Gaussian Splatting (3DGS) representations. This inconsistency particularly manifests as artifacts when rendering from novel viewpoints, leading to relatively lower performance under our dense viewpoint evaluation setting. In contrast, our method does not rely on 2D diffusion priors and directly generates a high-quality 3D representation from a single input image. 

% TriplaneGS employs a Transformer-based approach to predict dense point clouds in a single forward pass. However, this approach can face challenges in accurately capturing the 3D distribution of points, particularly in regions not visible in the input image, which may lead to less optimal performance in some cases. In comparison, our method adopts a diffusion-based strategy, first learning a sparse set of seed points. This approach simplifies the learning process, allowing the model to better capture the underlying 3D distribution. The seed points are then mapped to anchor latents, resulting in more robust and consistent outcomes.

\subsection{Editing Results Based on Drag}
% The drag results are presented in Fig. \ref{fig:edit}.
As shown in Fig. \ref{fig:edit}, our method enables Seed-Points-Driven Deformation. Starting with generated seed points from the input image, the sparse nature of the seed points allows for easy editing using 3D tools (e.g., Blender\cite{blender}) with a few drag operations. The edited 3DGS can then be obtained within 2 seconds.
% 如Fig. \ref{fig:edit}.所示,我们的方法可以进行Seed-Points-Driven Deformation. 对于一个generated seed points from input image,
% 由于seed points稀疏的特性,我们可以很方便借助3D编辑工具(blender)使用有限的几次Drag操作对seed points进行编辑,and以2s时间得到编辑后的3DGS.
\subsection{Ablation Study}
% \paragraph{coarse2fine in vae}
% \paragraph{two-stage generation}
% \todo{not finish here}
\paragraph{Seed Points Generation}
% We use the Recitified flow model to learn the generation of seed points with the conditon of single input image. Due to the sparse of seed points, the flow model is easy to learn and 可以很好的学到seed points' distribution. We also 实现this module using transformer block 使用一次feed forward的方法从learnable embeddings 中得到point cloud,like \cite{zou2024triplane}. As shown in Fig. \ref{fig:ablation-seed-gen}, the Feed-forward method failed to learn the distribution of the seed points, 在图片上不可见的区域无法生成理想的结果。
We employ a Rectified Flow model to generate seed points conditioned on a single input image. Owing to the sparsity of the seed points, the flow model is easier to train and effectively learns the distribution of the seed points. However, we also explored an alternative implementation using a transformer-based feed-forward approach, where point clouds are generated from learnable embeddings in a single pass, as in \cite{zou2024triplane}. As demonstrated in Fig. \ref{fig:ablation-seed-gen}, the feed-forward approach struggles to capture the true distribution of seed points and fails to produce satisfactory results in regions not visible in the input image.


\paragraph{Dimension Alignment}
% 为了让Seed-Anchor Mapping Module 的起点和target具有相同的维度,我们将Seed points 通过VAE的encoder based on Eq. \ref{eq:encode_seed_latents}. 这可以保证Mapping的起点的分布和终点的分布更接近,从而降低了Seed-Anchor Mapping Module的学习难度并且避免了Mapping时对image condition的过度依赖. And the alignment bewteen points and image achieved by projection in encoder is vivtal in the Seed-Points-Driven Deformation, as we can change the position of draged seed points while preserve it's correponding projed feats.
To match the dimension of the starting and target points in the Seed-Anchor Mapping Module, we encode the seed points using the Anchor-GS VAE encoder (Eq. \ref{eq:encode_seed_latents}). This process brings their distributions closer, reducing learning difficulty and reliance on image conditions. 
% Additionally, the projection-based alignment between points and the image in the encoder is critical for Seed-Points-Driven Deformation, enabling position adjustments of dragged seed points while preserving their projected features, as shown in Eq.\ref{eq:edit_seed_encode}. 
To validate this method, we conducted experiments by replacing the encoding approach with positional encoding .  When using positional encoding, the Seed-Anchor Mapping overly relied on the image condition, neglecting the contribution of the seed points and failing to enable seed-driven 3DGS deformation, as shown in Fig. \ref{fig:ablation-seed-enc}. 

% When positional encoding is used, the Seed-Anchor Mapping overly relies on the image condition, neglecting the true geometric state of the seed points and failing to achieve seed-driven 3DGS deformation.
% When using positional encoding, the Seed-Anchor Mapping overly relied on the image condition, 忽略了seed points真实的几何状态 and failing to enable seed-driven 3DGS deformation. 
% To validate this method, we conducted experiments by independently testing two variations: replacing the encoding approach with positional encoding and removing the projection of seed points onto the input image during encoding (Fig. \ref{fig:ablation-seed-enc}).  When using positional encoding, the Seed-Anchor Mapping overly relied on the image condition, neglecting the contribution of the seed points and failing to enable seed-driven 3DGS deformation. Separately, without projection-based alignment, the Mapping Module failed when the seed points and the input image were misaligned under the given viewpoint.
% 为了验证这个方法的有效性,我们测试了将基于encoder of anchor-GS VAE的方法换成positional encoding 和 在encode时不采用seed points与输入图像的投影。结果如图所示。在采用positional encoding时,the Seed-Anchor Mapping 将会过度依赖于image condition,导致忽略了从start points本身出发,从而无法实现基于seed points 的3DGS deformation. And lack of the projection-based alignment, the Mapping Module 会在seed points与input image 在给定视角下不一致的情况下fail.


\paragraph{Token Alignment}
We ensure token alignment in Flow Matching by organizing tokens around seed points, followed by  cluster-based partition and repetition. To evaluate its effectiveness, we conducted two ablation experiments, as shown in Tab. \ref{tab:ablation-tokenalign}. In the \textit{No-cluster+No-repetition} setting, we omitted the clustering step, aligning only the corresponding seed and anchor latents while filling unmatched portions with noise. This also prevented cluster-based downsampling in the Flow Model, resulting in doubled memory consumption. In the \textit{No-cluster} setting, we repeated the seed latents to match the number of anchor latents but left them unordered, leading to disorganized token matching. As shown in Tab. \ref{tab:ablation-tokenalign}, on a 40K subset with the same number of training steps, the absence of token alignment significantly degraded Flow Matching performance, resulting in inaccurate correspondences.
% We ensure token alignment in Flow Matching by organizing tokens around seed points, followed by repetition and cluster-based rearrangement. To validate its effectiveness, we conducted two ablation experiments. In the first, we repeated seed latents to match the number of anchor latents but left them unordered, leading to disordered token matching. In the second, we aligned only the corresponding seed and anchor latents, filling the unmatched portions with noise. Without cluster-based rearrangement, downsampling in the Rectified Flow Model became impossible, doubling memory consumption. Tab. \ref{tab:ablation-tokenalign} shows that on a 40K subset, with the same number of training steps, flow matching performance is significantly degraded without token alignment, failing to produce accurate correspondences.

% We ensure token alignment in Flow Matching by organizing tokens around seed points, followed by repetition and cluster-based rearrangement. To validate its effectiveness, we conducted two ablation experiments, as shown in Tab. \ref{tab:ablation-tokenalign}. In the No-cluster, 我们不再进行分cluster,aligned only the corresponding seed and anchor latents, filling the unmatched portions with noise. And we can't do cluster-based downsample in the Flow Model, 这会导致 memory consumption double. In the No-rearrange, we repeated seed latents to match the number of anchor latents but left them unordered, leading to disordered token matching. Tab. \ref{tab:ablation-tokenalign} shows that on a 40K subset, with the same number of training steps, flow matching performance is significantly degraded without token alignment, failing to produce accurate correspondences.


% we repeated seed latents to match the number of anchor latents but left them unordered, leading to disordered token matching. In the second, we aligned only the corresponding seed and anchor latents, filling the unmatched portions with noise. Without cluster-based rearrangement, downsampling in the Rectified Flow Model became impossible, doubling memory consumption. Tab. \ref{tab:ablation-tokenalign} shows that on a 40K subset, with the same number of training steps, flow matching performance is significantly degraded without token alignment, failing to produce accurate correspondences.
% Token alignment 保证了Flow matching 中start point和end point中token数目和对应位置上的语义是匹配的,使用以Seed points为center的方式来组织token,并进行repeat和rearrange。我们设计了消融实验来验证这个模块的必要性和有效性,第一个对照实验是在对seed latets repeat到和anchor latents相同的个数后,我们不对anchor latents进行重排序, so the token之间的匹配关系是杂乱的。第二个对照实验是我们仅仅将对应的seed latents和anchor latents进行对齐,其余无法对齐的部分使用noise进行填充。并且由于没有进行cluster based rearrange,我们无法在Rectified flow Modle中进行downsaple, which 增加了两倍的计算时的显存消耗。在一个40K subset上 经过同样的training steps后,结果如表所示。Without the token alignment, the Flow matching的效果大打折扣,无法得到理想的对应的结果。
% We ensure token alignment in Flow Matching by organizing the tokens around seed points and performing repeating and rearranging operations. This guarantees that the number of tokens and their semantic correspondence between the start and end points are aligned. To validate the necessity and effectiveness of this module, we designed ablation experiments. The first experiment involved repeating the seed latents to match the number of anchor latents but without reordering the anchor latents, resulting in a disordered matching relationship between tokens. The second experiment aligned only the corresponding seed latents and anchor latents, filling the unmatched portions with noise. Without the cluster-based rearrangement, we were unable to downsample in the Rectified Flow Model, which increased the memory consumption during computation by a factor of two. After training on a 40K subset for the same number of steps, the results, shown in the table, indicate that without token alignment, the performance of flow matching is significantly degraded, failing to achieve the desired correspondence.


% Table
\begin{table}%
\caption{ Quantitative evaluation of our method compared to previous work. $\dagger$ achieves relatively lower PSNR values in the evaluation, so we display the results in Sec. \ref{sec:comparison}.}
\label{tab:quantitative comparison}
% \begin{minipage}{\columnwidth}
\resizebox{0.5\textwidth}{!}{
% \begin{center}
\begin{tabular}{llllllll}
  \toprule
  \multirow{2}{*}{Method}  & \multicolumn{3}{c}{Objaverse\cite{objaverse}}   & \multicolumn{3}{c}{GSO\cite{downs2022google}}& \multirow{2}{*}{Time(s)}\\
% \cline{2-4}   \cline{5-7} \cline{8-10} \cline{11-13}
\cmidrule(r){2-4}  \cmidrule(r){5-7} 
   & PSNR$\uparrow$& SSIM$\uparrow$& LPIPS$\downarrow$ & PSNR$\uparrow$& SSIM$\uparrow$& LPIPS $\downarrow$
   \\ \midrule
  LGM$\dagger$\cite{tang2025lgm}     & -&0.836&0.211&-&0.833&0.21&4.82\\
  LaRa\cite{chen2025lara}  & 16.57&0.860&0.174&15.98&9.869&0.162&9.50\\
  TriplaneGS\cite{zou2024triplane}  &18.80 &0.883&0.143&19.84&0.900&0.120&0.70\\
  Ours &20.92&0.896&0.120&20.52&0.904&0.1122&4.71\\
  \bottomrule
\end{tabular}
% \end{center}
}
\end{table}%


\begin{table}%
\caption{Ablation about token alignment}
\label{tab:ablation-tokenalign}
\begin{minipage}{\columnwidth}
\begin{center}
\begin{tabular}{llll}
  \toprule
   & PSNR$\uparrow$& SSIM$\uparrow$& LPIPS$\downarrow$ 
   \\ \midrule
  No-cluster+No-repetition  & 18.84&0.877&0.141\\
  No-cluster     & 19.20 &0.876&0.142\\
  ours-full  &19.94 &0.881&0.134\\
  \bottomrule
\end{tabular}
\end{center}
\bigskip\centering


\end{minipage}
\end{table}%

%figure
\begin{figure}
  \includegraphics[width=\linewidth]{figs/ablation_seed.pdf}
  \caption{Ablation study about different seed points geneartion methods: (a) using our method, (b) using Transformer.}
  \label{fig:ablation-seed-gen}
\end{figure}

\begin{figure}
  \includegraphics[width=\linewidth]{figs/ablation_enc.pdf}
  \caption{Without Dimension Alignment, seed-points-driven deformation fails}
  \label{fig:ablation-seed-enc}
\end{figure}



\section{Conclusion and Suggestions}

Our work, including the creation of \texttt{ScholarLens} and the proposal of \texttt{LLMetrica}, provides methods for assessing LLM penetration in scholarly writing and peer review. By incorporating diverse data types and a range of evaluation techniques, we consistently observe the growing influence of LLMs across various scholarly processes, raising concerns about the credibility of academic research. As LLMs become more integrated into scholarly workflows, it is crucial to establish strategies that ensure their responsible and ethical use, addressing both content creation and the peer review process. 

Despite existing guidelines restricting LLM-generated content in scholarly writing and peer review,\footnote{\href{https://aclrollingreview.org/acguidelines\#-task-3-checking-review-quality-and-chasing-missing-reviewers}{Area Chair} \&  \href{https://aclrollingreview.org/reviewerguidelines\#q-can-i-use-generative-ai}{Reviewer} \& \href{https://www.aclweb.org/adminwiki/index.php/ACL_Policy_on_Publication_Ethics\#Guidelines_for_Generative_Assistance_in_Authorship}{Author} guidelines.} challenges still remain. 
To address these, we propose the following based on our work and findings: 
(i) \textbf{Increase transparency in LLM usage within scholarly processes} by incorporating LLM assistance into review checklists, encouraging explicit acknowledgment of LLM support in paper acknowledgments, and 
reporting LLM usage patterns across diverse demographic groups;
% reporting LLM penetration based on social demographic features;
(ii) \textbf{Adopt policies to prevent irresponsible LLM reviewers} by establishing feedback channels for authors on LLM-generated reviews and developing fine-grained LLM detection models~\cite{abassy-etal-2024-llm, cheng2024beyond, artemova2025beemobenchmarkexperteditedmachinegenerated} to distinguish acceptable LLM roles (e.g., language improvement vs. content creation);
(iii) \textbf{Promote data-driven research in scholarly processes} by supporting the collection of review data for further robust analysis~\cite{dycke-etal-2022-yes}.\footnote{\url{https://arr-data.aclweb.org/}}

% make LLM usage transparent in scholarly processes: such as incorporating LLM usage into review checklists, encouraging explicit acknowledgment of LLM assistance in paper acknowledgments, and reporting LLM penetration based on social demographic features; (ii) Adopt policies to prevent irresponsible LLM reviewers: such as providing authors feedback on LLM-assisted reviews, and developing fine-grained LLM detection models~\cite{cheng2024beyond} to distinguish acceptable LLM roles (e.g., language improvement vs. content creation); (iii) Encourage data-driven research in scholarly processes: such as supporting review data collection for further research.

 



% \clearpage
%%
%% The acknowledgments section is defined using the "acks" environment
%% (and NOT an unnumbered section). This ensures the proper
%% identification of the section in the article metadata, and the
%% consistent spelling of the heading.
\begin{acks}
The challenge is organized as a joint effort by the University College London, Microsoft, the University of Amsterdam, the University of Waterloo, and the University of Padua. The views expressed in the content are solely those of the authors and do not necessarily reflect the views or endorsements of their employers and/or sponsors. This research is supported by the Engineering and Physical Sciences Research Council [EP/S021566/1], CAMEO, PRIN 2022 n.~2022ZLL7MW and by the Dreams Lab, a collaboration between Huawei Finland, the University of Amsterdam, and the Vrije Universiteit Amsterdam.
\end{acks}

% \begin{table}[ht]
    \centering
    \caption{The number of labels assigned by human judges and LLMJudge challenge submissions to each judgment level. Bold indicates the closest prediction to the number of labels assigned by humans.}
    \adjustbox{max width=\columnwidth}{%
    \begin{tabular}{lcccc}
    \toprule
    \textbf{Submission ID} & \textbf{0} & \textbf{1} & \textbf{2} & \textbf{3} \\
    \midrule
    human & 2005 & 1233 & 808 & 377 \\
    \midrule
    NISTRetrieval-instruct0 & 1115 & 2092 & 1216 & 0 \\
    NISTRetrieval-instruct1 & 1115 & 2092 & 1216 & 0 \\
    NISTRetrieval-instruct2 & 1117 & 2088 & 1218 & 0 \\
    NISTRetrieval-reason0 & 1159 & 1922 & 1340 & 2 \\
    NISTRetrieval-reason1 & 1158 & 1924 & 1339 & 2 \\
    NISTRetrieval-reason2 & 1159 & 1921 & 1341 & 2 \\
    Olz-exp & 2435 & 1210 & 456 & 322 \\
    Olz-gpt4o & 2258 & 1274 & 504 & \textbf{387} \\
    Olz-halfbin & 2100 & 1458 & 277 & 588 \\
    Olz-multiprompt & 1713 & 976 & 1244 & 490 \\
    Olz-somebin & 2102 & 595 & 1004 & 722 \\
    RMITIR-GPT4o & 3056 & 349 & 730 & 288 \\
    RMITIR-llama38b & 2576 & 614 & 1058 & 175 \\
    RMITIR-llama70B & 2154 & 243 & 1581 & 443 \\
    TREMA-4prompts & 1027 & 751 & 2213 & 432 \\
    TREMA-CoT & 1835 & 1122 & 906 & 560 \\
    TREMA-all & 2399 & 616 & 734 & 674 \\
    TREMA-direct & 2404 & 87 & 342 & 1590 \\
    TREMA-naiveBdecompose & 2627 & 645 & 1117 & 34 \\
    TREMA-nuggets & 2127 & 893 & 1044 & 359 \\
    TREMA-other & 1305 & 809 & 2021 & 288 \\
    TREMA-questions & 2450 & 253 & \textbf{767} & 953 \\
    TREMA-rubric0 & 3122 & 1211 & 0 & 90 \\
    TREMA-sumdecompose & 2518 & 254 & 1049 & 602 \\
    h2oloo-fewself & 2470 & 732 & 557 & 664 \\
    h2oloo-zeroshot1 & 2353 & 1225 & 597 & 248 \\
    h2oloo-zeroshot2 & 2920 & 771 & 476 & 255 \\
    llmjudge-cot1 & 991 & 1921 & 1383 & 128 \\
    llmjudge-cot2 & 1111 & 955 & 2149 & 208 \\
    llmjudge-cot3 & 1902 & 1321 & 486 & 714 \\
    llmjudge-simple1 & 777 & 2228 & 1217 & 200 \\
    llmjudge-simple2 & 1720 & 905 & 1375 & 423 \\
    llmjudge-simple3 & 1834 & 1318 & 485 & 786 \\
    llmjudge-thomas1 & 1049 & 1803 & 1402 & 169 \\
    llmjudge-thomas2 & 1886 & 733 & 1585 & 219 \\
    llmjudge-thomas3 & \textbf{1934} & 1184 & 559 & 746 \\
    prophet-setting1 & 2506 & 885 & 629 & 403 \\
    prophet-setting2 & 2903 & 852 & 651 & 17 \\
    prophet-setting4 & 3359 & 763 & 281 & 20 \\
    willia-umbrela1 & 2335 & \textbf{1231} & 608 & 249 \\
    willia-umbrela2 & 2705 & 1029 & 350 & 339 \\
    willia-umbrela3 & 2710 & 1041 & 446 & 226 \\
    \bottomrule
    \end{tabular}
    }
    \label{tab:labels}
\end{table}

%%
%% The next two lines define the bibliography style to be used, and
%% the bibliography file.
\bibliographystyle{ACM-Reference-Format}
\bibliography{references}

\clearpage

%%
%% If your work has an appendix, this is the place to put it.
\appendix

% \section{Appendix}

\begin{table}[ht]
    \centering
    \caption{The number of labels assigned by human judges and LLMJudge challenge submissions to each judgment level. Bold indicates the closest prediction to the number of labels assigned by humans.}
    \adjustbox{max width=\columnwidth}{%
    \begin{tabular}{lcccc}
    \toprule
    \textbf{Submission ID} & \textbf{0} & \textbf{1} & \textbf{2} & \textbf{3} \\
    \midrule
    human & 2005 & 1233 & 808 & 377 \\
    \midrule
    NISTRetrieval-instruct0 & 1115 & 2092 & 1216 & 0 \\
    NISTRetrieval-instruct1 & 1115 & 2092 & 1216 & 0 \\
    NISTRetrieval-instruct2 & 1117 & 2088 & 1218 & 0 \\
    NISTRetrieval-reason0 & 1159 & 1922 & 1340 & 2 \\
    NISTRetrieval-reason1 & 1158 & 1924 & 1339 & 2 \\
    NISTRetrieval-reason2 & 1159 & 1921 & 1341 & 2 \\
    Olz-exp & 2435 & 1210 & 456 & 322 \\
    Olz-gpt4o & 2258 & 1274 & 504 & \textbf{387} \\
    Olz-halfbin & 2100 & 1458 & 277 & 588 \\
    Olz-multiprompt & 1713 & 976 & 1244 & 490 \\
    Olz-somebin & 2102 & 595 & 1004 & 722 \\
    RMITIR-GPT4o & 3056 & 349 & 730 & 288 \\
    RMITIR-llama38b & 2576 & 614 & 1058 & 175 \\
    RMITIR-llama70B & 2154 & 243 & 1581 & 443 \\
    TREMA-4prompts & 1027 & 751 & 2213 & 432 \\
    TREMA-CoT & 1835 & 1122 & 906 & 560 \\
    TREMA-all & 2399 & 616 & 734 & 674 \\
    TREMA-direct & 2404 & 87 & 342 & 1590 \\
    TREMA-naiveBdecompose & 2627 & 645 & 1117 & 34 \\
    TREMA-nuggets & 2127 & 893 & 1044 & 359 \\
    TREMA-other & 1305 & 809 & 2021 & 288 \\
    TREMA-questions & 2450 & 253 & \textbf{767} & 953 \\
    TREMA-rubric0 & 3122 & 1211 & 0 & 90 \\
    TREMA-sumdecompose & 2518 & 254 & 1049 & 602 \\
    h2oloo-fewself & 2470 & 732 & 557 & 664 \\
    h2oloo-zeroshot1 & 2353 & 1225 & 597 & 248 \\
    h2oloo-zeroshot2 & 2920 & 771 & 476 & 255 \\
    llmjudge-cot1 & 991 & 1921 & 1383 & 128 \\
    llmjudge-cot2 & 1111 & 955 & 2149 & 208 \\
    llmjudge-cot3 & 1902 & 1321 & 486 & 714 \\
    llmjudge-simple1 & 777 & 2228 & 1217 & 200 \\
    llmjudge-simple2 & 1720 & 905 & 1375 & 423 \\
    llmjudge-simple3 & 1834 & 1318 & 485 & 786 \\
    llmjudge-thomas1 & 1049 & 1803 & 1402 & 169 \\
    llmjudge-thomas2 & 1886 & 733 & 1585 & 219 \\
    llmjudge-thomas3 & \textbf{1934} & 1184 & 559 & 746 \\
    prophet-setting1 & 2506 & 885 & 629 & 403 \\
    prophet-setting2 & 2903 & 852 & 651 & 17 \\
    prophet-setting4 & 3359 & 763 & 281 & 20 \\
    willia-umbrela1 & 2335 & \textbf{1231} & 608 & 249 \\
    willia-umbrela2 & 2705 & 1029 & 350 & 339 \\
    willia-umbrela3 & 2710 & 1041 & 446 & 226 \\
    \bottomrule
    \end{tabular}
    }
    \label{tab:labels}
\end{table}

\begin{table}[ht]
    \centering
    \caption{Krippendorff's $\alpha$ correlation in 4-point scale agreement and difference binarize the judgment scale. Judgment levels to the left of the pipe are considered irrelevant, while those to the right are considered relevant.}
    \adjustbox{max width=\columnwidth}{%
    \begin{tabular}{lcccc}
    \toprule
    \textbf{Submission ID} & \textbf{4-point} & \textbf{0|123} & \textbf{01|23} & \textbf{012|3} \\
    \midrule
    NISTRetrieval-instruct0 & 0.3819 & 0.2811 & 0.3021 & -0.0444 \\
    NISTRetrieval-instruct1 & 0.3812 & 0.2801 & 0.3021 & -0.0444 \\
    NISTRetrieval-instruct2 & 0.3821 & 0.2823 & 0.3013 & -0.0444 \\
    NISTRetrieval-reason0 & 0.3874 & 0.263 & 0.3381 & -0.0336 \\
    NISTRetrieval-reason1 & 0.3872 & 0.2624 & 0.3385 & -0.0336 \\
    NISTRetrieval-reason2 & 0.3874 & 0.262 & 0.3388 & -0.0336 \\
    Olz-exp & 0.4701 & 0.3941 & 0.3499 & 0.2933 \\
    Olz-gpt4o & 0.502 & 0.421 & 0.3619 & 0.3067 \\
    Olz-halfbin & 0.4536 & 0.4005 & 0.2534 & 0.2405 \\
    Olz-multiprompt & 0.4551 & 0.3737 & 0.3829 & 0.2137 \\
    Olz-somebin & 0.4471 & 0.3851 & 0.378 & 0.1014 \\
    RMITIR-GPT4o & 0.4108 & 0.3125 & 0.395 & 0.257 \\
    RMITIR-llama38b & 0.3873 & 0.3169 & 0.3194 & 0.1268 \\
    RMITIR-llama70B & 0.4873 & 0.416 & 0.3679 & 0.2839 \\
    TREMA-4prompts & 0.2888 & 0.2644 & 0.1888 & 0.1661 \\
    TREMA-CoT & 0.3852 & 0.3172 & 0.3176 & 0.18 \\
    TREMA-all & 0.3855 & 0.3191 & 0.2957 & 0.0618 \\
    TREMA-direct & 0.3729 & 0.315 & 0.3259 & 0.0868 \\
    TREMA-naiveBdecompose & 0.3579 & 0.2949 & 0.2916 & -0.018 \\
    TREMA-nuggets & 0.1691 & 0.1499 & 0.0967 & -0.0076 \\
    TREMA-other & 0.2712 & 0.2547 & 0.1477 & 0.1399 \\
    TREMA-questions & 0.3148 & 0.2562 & 0.2758 & 0.0125 \\
    TREMA-rubric0 & 0.1036 & 0.1172 & -0.0895 & 0.0167 \\
    TREMA-sumdecompose & 0.3926 & 0.3138 & 0.343 & 0.1995 \\
    h2oloo-fewself & 0.4958 & 0.4108 & 0.428 & 0.2978 \\
    h2oloo-zeroshot1 & 0.4812 & 0.4058 & 0.385 & 0.3063 \\
    h2oloo-zeroshot2 & 0.3898 & 0.3418 & 0.3175 & 0.2769 \\
    llmjudge-cot1 & 0.3218 & 0.1764 & 0.2788 & 0.116 \\
    llmjudge-cot2 & 0.3263 & 0.2173 & 0.2429 & 0.2002 \\
    llmjudge-cot3 & 0.487 & 0.3853 & 0.3979 & 0.2233 \\
    llmjudge-simple1 & 0.2808 & 0.05 & 0.2857 & 0.1528 \\
    llmjudge-simple2 & 0.3672 & 0.2317 & 0.3097 & 0.2003 \\
    llmjudge-simple3 & 0.4642 & 0.3581 & 0.397 & 0.2012 \\
    llmjudge-thomas1 & 0.3207 & 0.1679 & 0.278 & 0.1725 \\
    llmjudge-thomas2 & 0.3853 & 0.294 & 0.2891 & 0.2229 \\
    llmjudge-thomas3 & 0.4877 & 0.3909 & 0.3942 & 0.2321 \\
    prophet-setting1 & 0.4069 & 0.3419 & 0.2892 & 0.1677 \\
    prophet-setting2 & 0.3144 & 0.2815 & 0.2225 & -0.0093 \\
    prophet-setting4 & 0.1623 & 0.1627 & 0.0797 & 0.0006 \\
    willia-umbrela1 & 0.4918 & 0.4129 & 0.3939 & 0.3124 \\
    willia-umbrela2 & 0.4556 & 0.3961 & 0.3298 & 0.3193 \\
    willia-umbrela3 & 0.4535 & 0.3965 & 0.3314 & 0.3185 \\
    \bottomrule
    \end{tabular}
    }
    \label{tab:alpha}
\end{table}

\end{document}
\endinput
%%
%% End of file `sample-sigconf.tex'.
