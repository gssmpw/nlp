 %%
%% This is file `sample-sigconf.tex',
%% generated with the docstrip utility.
%%
%% The original source files were:
%%
%% samples.dtx  (with options: `all,proceedings,bibtex,sigconf')
%% 
%% IMPORTANT NOTICE:
%% 
%% For the copyright see the source file.
%% 
%% Any modified versions of this file must be renamed
%% with new filenames distinct from sample-sigconf.tex.
%% 
%% For distribution of the original source see the terms
%% for copying and modification in the file samples.dtx.
%% 
%% This generated file may be distributed as long as the
%% original source files, as listed above, are part of the
%% same distribution. (The sources need not necessarily be
%% in the same archive or directory.)
%%
%%
%% Commands for TeXCount
%TC:macro \cite [option:text,text]
%TC:macro \citep [option:text,text]
%TC:macro \citet [option:text,text]
%TC:envir table 0 1
%TC:envir table* 0 1
%TC:envir tabular [ignore] word
%TC:envir displaymath 0 word
%TC:envir math 0 word
%TC:envir comment 0 0
%%
%%
%% The first command in your LaTeX source must be the \documentclass
%% command.
%%
%% For submission and review of your manuscript please change the
%% command to \documentclass[manuscript, screen, review]{acmart}.
%%
%% When submitting camera ready or to TAPS, please change the command
%% to \documentclass[sigconf]{acmart} or whichever template is required
%% for your publication.
%%
%%
% \documentclass[sigconf]{acmart}
% \documentclass[sigconf,natbib=true,anonymous=true]{acmart}
\documentclass[sigconf,natbib=true]{acmart}

\usepackage{acronym}
\begin{acronym}
\acro{gan}[GANs]{Generative Adversarial Networks}
\acro{rl}[RL]{Reinforcement Learning}
\acro{pae}[PAE]{Periodic Autoencoder}
\acro{fld}[FLD]{Fourier Latent Dynamics}
\acro{ppo}[PPO]{Proximal Policy Optimization}
\acro{fft}[FFT]{Fast Fourier Transform}
\acro{pca}[PCA]{Principal Component Analysis}
\acro{dfm}[DFM]{Deep Fourier Mimic}
\acro{dof}[DoF]{Degrees of Freedom}
\acro{mlp}[MLPs]{Multi-Layer Perceptrons}
\end{acronym}


\newcommand{\todo}[1]{\textcolor{red}{TODO: #1}}
\newcommand{\missingcit}[1]{\textcolor{orange}{CIT: #1}}
\newcommand{\note}[1]{\textcolor{blue}{{}{#1}{}}}


%%
%% \BibTeX command to typeset BibTeX logo in the docs
\AtBeginDocument{%
  \providecommand\BibTeX{{%
    Bib\TeX}}}

%% Rights management information.  This information is sent to you
%% when you complete the rights form.  These commands have SAMPLE
%% values in them; it is your responsibility as an author to replace
%% the commands and values with those provided to you when you
%% complete the rights form.
\setcopyright{acmlicensed}
\copyrightyear{2018}
\acmYear{2018}
\acmDOI{XXXXXXX.XXXXXXX}

%% These commands are for a PROCEEDINGS abstract or paper.
\acmConference[SIGIR '25]{Make sure to enter the correct
  conference title from your rights confirmation email}{June 03--05,
  2018}{Woodstock, NY}
%%
%%  Uncomment \acmBooktitle if the title of the proceedings is different
%%  from ``Proceedings of ...''!
%%
%%\acmBooktitle{Woodstock '18: ACM Symposium on Neural Gaze Detection,
%%  June 03--05, 2018, Woodstock, NY}
\acmISBN{978-1-4503-XXXX-X/18/06}


%%
%% Submission ID.
%% Use this when submitting an article to a sponsored event. You'll
%% receive a unique submission ID from the organizers
%% of the event, and this ID should be used as the parameter to this command.
%%\acmSubmissionID{123-A56-BU3}

%%
%% For managing citations, it is recommended to use bibliography
%% files in BibTeX format.
%%
%% You can then either use BibTeX with the ACM-Reference-Format style,
%% or BibLaTeX with the acmnumeric or acmauthoryear sytles, that include
%% support for advanced citation of software artefact from the
%% biblatex-software package, also separately available on CTAN.
%%
%% Look at the sample-*-biblatex.tex files for templates showcasing
%% the biblatex styles.
%%

%%
%% The majority of ACM publications use numbered citations and
%% references.  The command \citestyle{authoryear} switches to the
%% "author year" style.
%%
%% If you are preparing content for an event
%% sponsored by ACM SIGGRAPH, you must use the "author year" style of
%% citations and references.
%% Uncommenting
%% the next command will enable that style.
%%\citestyle{acmauthoryear}


\usepackage{algorithmic}
\usepackage{adjustbox}
\usepackage{subfig}

\newcommand{\partitle}[1]{\vspace{1mm}\noindent\textbf{#1.}}

%%
%% end of the preamble, start of the body of the document source.
\begin{document}

%%
%% The "title" command has an optional parameter,
%% allowing the author to define a "short title" to be used in page headers.
% \title{LLM-as-a-Rel: Benchmarking Automatic Relevance Judgments}
% \title{LLM-as-a-Rel: A Collection of LLM-Generated Relevance Judgements to Benchmark Automatic Annotations}
\title{Judging the Judges: \\ A Collection of LLM-Generated Relevance Judgements}

%%
%% The "author" command and its associated commands are used to define
%% the authors and their affiliations.
%% Of note is the shared affiliation of the first two authors, and the
%% "authornote" and "authornotemark" commands
%% used to denote shared contribution to the research.

\author{
Hossein A.~Rahmani
}
\orcid{0000-0002-2779-4942} 
\affiliation{%
        \institution{University College London}
        \city{London}
        \country{UK}
}
\email{hossein.rahmani.22@ucl.ac.uk}

\author{Clemencia Siro}
\orcid{0000-0001-5301-4244} 
\affiliation{%
        \institution{University of Amsterdam}
        \city{Amsterdam}
        \country{The Netherlands}
}
\email{c.n.siro@uva.nl}


\author{
Mohammad Aliannejadi
}
\orcid{0000-0002-9447-4172} 
\affiliation{%
        \institution{University of Amsterdam}
        \city{Amsterdam}
        \country{The Netherlands}
}
\email{m.aliannejadi@uva.nl}

\author{Nick Craswell}
\orcid{0000-0002-9351-8137} 
\affiliation{%
        \institution{Microsoft}
        \city{Bellevue}
        \country{US}
}
\email{nickcr@microsoft.com}

\author{Charles L.~A.~Clarke}
\orcid{0000-0001-8178-9194}
\affiliation{%
  \institution{University of Waterloo}
  \city{Waterloo, Ontario}
  \country{Canada}
}
\email{claclark@gmail.com}

\author{Guglielmo Faggioli}
\orcid{0000-0002-5070-2049} 
\affiliation{%
        \institution{University of Padua}
        \city{Padua}
        \country{Italy}
}
\email{faggioli@dei.unipd.it}


\author{Bhaskar Mitra}
\orcid{0000-0002-5270-5550} 
\affiliation{%
        \institution{Microsoft}
        \city{Montréal}
        \country{Canada}
}
\email{bmitra@microsoft.com}

\author{Paul Thomas}
\orcid{0000-0003-2425-3136} 
\affiliation{%
        \institution{Microsoft}
        \city{Adelaide}
        \country{Australia}
}
\email{pathom@microsoft.com}

\author{Emine Yilmaz}
\orcid{0000-0003-4734-4532} 
\affiliation{%
        \institution{University College London \& Amazon}
        \city{London}
        \country{UK}
}
\email{emine.yilmaz@ucl.ac.uk}
%%
%% By default, the full list of authors will be used in the page
%% headers. Often, this list is too long, and will overlap
%% other information printed in the page headers. This command allows
%% the author to define a more concise list
%% of authors' names for this purpose.
\renewcommand{\shortauthors}{Rahmani et al.}

%%
%% The abstract is a short summary of the work to be presented in the
%% article.
\begin{abstract}
Using Large Language Models (LLMs) for relevance assessments offers promising opportunities to improve Information Retrieval (IR), Natural Language Processing (NLP), and related fields. Indeed, LLMs hold the promise of allowing IR experimenters to build evaluation collections with a fraction of the manual human labor currently required. This could help with fresh topics on which there is still limited knowledge and could mitigate the challenges of evaluating ranking systems in low-resource scenarios, where it is challenging to find human annotators. Given the fast-paced recent developments in the domain, many questions concerning LLMs as assessors are yet to be answered. Among the aspects that require further investigation, we can list the impact of various components in a relevance judgment generation pipeline, such as the prompt used or the LLM chosen.

This paper benchmarks and reports on the results of a large-scale automatic relevance judgment evaluation, the \textit{LLMJudge challenge} at SIGIR 2024, where different relevance assessment approaches were proposed. In detail, we release and benchmark 42 LLM-generated labels of the TREC 2023 Deep Learning track relevance judgments produced by eight international teams who participated in the challenge. Given their diverse nature, these automatically generated relevance judgments can help the community not only investigate systematic biases caused by LLMs but also explore the effectiveness of ensemble models, analyze the trade-offs between different models and human assessors, and advance methodologies for improving automated evaluation techniques.
The released resource is available at the following link: \url{https://llm4eval.github.io/LLMJudge-benchmark/}. 
\end{abstract}

%%
%% The code below is generated by the tool at http://dl.acm.org/ccs.cfm.
%% Please copy and paste the code instead of the example below.
%%
% \begin{CCSXML}
% <ccs2012>
%    <concept>
%        <concept_id>10002951.10003317</concept_id>
%        <concept_desc>Information systems~Information retrieval</concept_desc>
%        <concept_significance>500</concept_significance>
%        </concept>
%  </ccs2012>
% \end{CCSXML}

% \ccsdesc[500]{Information systems~Information retrieval}

%%
%% Keywords. The author(s) should pick words that accurately describe
%% the work being presented. Separate the keywords with commas.
% \keywords{LLM-as-a-Rel, LLM4Eval, Relevance Judgment, Benchmark}
%% A "teaser" image appears between the author and affiliation
%% information and the body of the document, and typically spans the
%% page.

% \received{20 February 2007}
% \received[revised]{12 March 2009}
% \received[accepted]{5 June 2009}

%%
%% This command processes the author and affiliation and title
%% information and builds the first part of the formatted document.
\maketitle


The increasing reliance on LLMs for multimodal tasks across far-reaching sectors such as healthcare, finance, and manufacturing underscores the need to assess the accuracy and reliability of the information they generate. Vision-Language Models (VLM) have achieved state-of-the-art (SoTA) performance on Visual Question-Answering (VQA) benchmarks, and these models often utilize Retrieval-Augmented Generation (RAG) to maintain factual accuracy and relevance in a dynamic information environment. However, this has led to uncertainty in the information the LLM bases its answer on, as it may choose between parametric memory and retrieved sources. When models rely on memorized information instead of dynamically retrieving information, they may inadvertently propagate outdated or incorrect information, causing serious legal and ethical risks and undermining trust and reliability in AI systems \citep{huang2023survey}.
% The ability to strike a balance between generalization and specialization in AI systems is therefore crucial for ensuring the safe, reliable use of these technologies in real-world applications.

Despite these concerns, the way that Vision-Language models (VLMs) memorize and retrieve information, particularly in complex multimodal tasks, remains under-explored. Current research often focuses on either the general capabilities of large language models (LLMs) or the specialized retrieval mechanisms in retrieval augmented generation systems (RAG) \citep{incontext_rag,chen_murag_2022,liu_universal_2023}. Particularly in the context of multimodal retrieval and multihop reasoning, few studies analyze the tradeoff between finetuning for specialized tasks and zero-shot prompting for general-purpose vision-language capabilities. A lack of consensus on how to approach this tradeoff motivates the development of measures to quantify reliance on parametric memory, as well as metrics for quantifying the potential performance impact of extending LLMs with RAG systems.

To address this gap, we investigate how multimodal QA models balance accuracy with memorization on the WebQA benchmark. We compare finetuned multimodal systems against zero-shot VLMs, analyzing how retrieval performance influences QA accuracy. In particular, we focus on cases where retrieval fails, allowing us to measure reliance on parametric memory through two proposed metrics---the \ppr (\PPR) which quantifies how much model accuracy is influenced by retrieval quality, contrasting performance in best-case versus worst-case retrieval scenarios, and the \ucr (\UCR) which measures how often correct QA responses are generated when the retriever fails, providing a proxy for memorization.

To enable this analysis, we make several methodological contributions. For the finetuned QA models, we investigate Vision-Transformer (ViT) architectures, which allow for multihop reasoning over multiple sources. To investigate the impact of retrieval performance on trained LMs, we propose a variable-input Fusion-in-Decoder (FiD) model \cite{tanaka_slidevqa_2023, nlvr2}, building upon the VoLTA architecture \citep{pramanick_volta_2023}. For the zero-shot case, we build upon previous research on In-Context Retrieval \citep{incontext_rag} by demonstrating that LLMs such as GPT-4o are capable of performing the final ranking step of the retrieval process. In doing so, we find that GPT-4o, a general-purpose LLM, achieves SoTA performance on the WebQA task, outperforming existing finetuned RAG models by a significant margin (7\% higher accuracy). 

Crucially, our results reveal that while retrieval-augmented models reduce memorization, the training paradigm plays an important role. Finetuned models exhibit higher reliance on parametric memory, whereas zero-shot RAG approaches have lower memorization scores at the cost of accuracy. This suggests that while retrieval modules may mitigate the risks associated with outdated or incorrect information, SoTA performance requires that they be coupled with specialized QA models. Our memorization measures contribute to the development of transparent and reliable AI systems, particularly in applications where the sourcing of up-to-date, factual information is critical.



% We investigate the impact of question complexity on the ability of these models to integrate multiple data sources—such as images, text, and external retrievers—and produce coherent and accurate answers. We also explore whether in-context retrieval can be a viable alternative to traditional retrieval-augmented systems, offering a more streamlined approach to multimodal QA.

% To achieve this, we first compare zero-shot prompting multimodal LLMs with finetuned multimodal systems. We evaluate both types of models on the WebQA benchmark, a dataset designed for complex question answering that requires reasoning across both image and text sources. For the finetuned models, we use a Fusion-in-Decoder (FiD) architecture, which allows for multihop reasoning over multiple sources. Additionally, we introduce the concept of In-Context Retrieval Language Modeling (RLM), where the LLM itself performs retrieval tasks without the need for external retrievers. This method builds upon existing research in in-context learning  and aims to explore the viability of LLMs retrieving relevant sources and generating accurate answers directly from their context window.

% In order to investigate source utilization in finetuned multimodal models and LLMs, three lines of inquiry are established; 
% \begin{itemize}
%     \item Study 1: retrieval vs QA performance on webQA (motivating example, does QA answer correctly even with incorrect sources?)
%     \item Study 2: performance on adversarial examples where parametric knowledge would be incorrect by design
%     \item Study 3: improving performance on adversarial examples by fine-tuning (i.e model robustness)
% \end{itemize}

% Note, there is one weakness in this plan which is tying in the work we've already done. 
% If we added something from adversarial generation to the retrieval experiment (like a combination of study 1 + 3) it would be complete. So for instance we could try fine-tuning the retriever with adversarial examples (and not just the QA model)

% \begin{figure}
%     \centering
%     \includegraphics[width=0.95\linewidth]{figures/segmentation/webqa_segment_infill.png}
%     \caption{Example of the segmentation substitution pipeline from the WebQA task.}
%     % d5c76d760dba11ecb1e81171463288e9
%     \label{fig:seg_sub_pipeline}
% \end{figure}



% Retrieval augmented generation (RAG) with zero-shot prompting and fine-tuning Large Language Models (LLMs) have become the go-to methods for tasks relying on information retrieval and text generation. In many cases the LLMs parametric memory can sufficiently generalize to answer questions without being provided with retrieval mechanisms for out-of-domain knowledge. However, LLMs often hallucinate and provide wrong information in certain scenarios. This problem is amplified even further on open-domain Question Answering (QA) tasks involving multiple modalities. Grounded text generation using retrieved sources \citep{lewis2021retrievalaugmented} has been extensively studied for text-to-text QA tasks, but its application in multimodal settings has not been studied as much.


% Multimodal reasoning and question answering have gained prominence in recent research endeavors, with an increasing emphasis on handling various forms of data, particularly text and images. In this study, we address a specific gap in the existing literature by focusing on the development of a versatile multihop model capable of accommodating varying numbers of input images.

% Our motivation for this research lies in the growing complexity of answering questions using information on the web, where the challenge of navigating the open-domain setting is further complicated by the presence of multiple modalities and sometimes requires reasoning over multiple sources. WebQA is an ideal dataset on which to compare performance of finetuned RAG systems against general purpose LLMs; it is multimodal, with correct answers requiring reasoning over image and text sources. It is multihop, requiring a complex reasoning process over multiple sources. Finally, WebQA questions from different categories can be broken down into subdomains to analyze performance over domains of varying cardinality.

% Motivated by the real-world challenges of building retrieval and question answering (QA) systems, we design and finetune a closed domain, multimodal, multihop QA model, that is capable of reasoning over a varying number of sources taken as input from an external retriever module. This research contributes to the relatively underexplored domain of multihop reasoning across various input sources and modalities. Our goal is to explore the challenges posed by these scenarios and develop strategies that enable QA models to retrieve relevant information, conduct logical or numerical reasoning across diverse modalities, and generate coherent responses in natural language. To our knowledge, this is the first application of the Fusion-in-Decoder (FiD) architecture \cite{tanaka_slidevqa_2023, nlvr2} that is shown to work with a variable number of inputs, enabling multi-hop reasoning over sources.

% In-Context Learning refers to the ability of LLMs to perform any task by simply providing examples in the input prompt \citep{dong2022survey,min2022rethinking}. Inspired by this research, we propose a method to use the LLM itself as a multimodal retriever, potentially eschewing the requirement of a distinct retrieval module, thereby allowing the design of simpler retrieval-augmented QA systems. We dub this method In-Context Retrieval Language Modeling (RLM). To the best of the authors knowledge, In-Content RLM is disparate from other retrieval augmented approaches which utilize external retrieval modules \citep{incontext_rag,chen_murag_2022,liu_universal_2023}. Despite being a natural extension of In-Context learning, In-Context RLM has not yet been studied empirically.

% To expand on our contribution of In-Context Retrieval, this stems from the well-researched in-context learning of LLMs. In-context learning is the ability of a model to perform any task given a sufficient context window \citep{dong2022survey,min2022rethinking}. Such tasks could include retrieval and ranking, but typically, the go-to solution for tasks requiring retrieval has been RAG. To the best of the authors knowledge, In-Context Retrieval is distinct from In-Context Retrieval Augmented Language Modelling (RALM), and despite being a natural extension of In-Context learning, In-Context Retrieval has not yet been shown empirically.

% Finally, we explore the tradeoff between using zero-shot prompting LLMs and the fine-tuning approach. While we find that, overall, GPT-4o obtains SoTA performance on the WebQA task, outperforming the accuracy of existing finetuned RAG approaches by 7\%, finetuned approaches still perform better on more restricted subdomains\footnote{``In-Context RLM" @ \url{https://eval.ai/web/challenges/challenge-page/1255/leaderboard/3168}}. Finally, we validate that GPT-4o is relying on retrieval abilities to solve the task; we find that GPT-4o is capable of retrieving relevant sources in the presence of distractors and furthermore, when GPT-4o fails to retrieve correct sources, it answers incorrectly 75\% of the time, meaning that it is not relying on parametric memory for this task.

% \paragraph{Contributions}
% Based on our experimentation and analysis on the WebQA benchmark, we make the following contributions:
% \begin{itemize}
%     \item Propose a new architecture for multimodal multihop QA that takes variable number of input sources inspired by the Fusion-in-Decoder method.
%     \item Comparison of general purpose LLMs vs specialized models on the WebQA benchmark.
%     \item Observation of In-Context Multimodal Retrieval abilities of GPT-4o and that it does not rely on parametric memory for multimodal QA.
%     \item Analysis of relationship between retrieval and QA task performance.
%     \item Analysis of task and query complexity on the performance of retrieval and QA tasks.
% \end{itemize}
















% Throughout this paper, we will present our methodology, experiments, and findings, emphasizing our approach to multihop reasoning over varying numbers of input images. We believe that our work contributes to a deeper understanding of multimodal reasoning and has the potential to enhance the capabilities of question-answering systems in the intricate, multimodal landscape of web-based information.
\section{Related Work}
\label{sec:relatedwork}
Traditionally, an experimental \ac{IR} collection includes three elements, a corpus, a set of topics, and the relevance judgments, defining which documents are relevant in response to the topics.
Over the last 30 years, since the first TREC campaign~\cite{DBLP:conf/trec/1992}, the most common strategy to obtain such relevance judgments has involved expert annotators, capable of providing the most accurate labels. 
The cost of this process can be partially reduced with pooling~\cite{croft2009search}, but the monetary and temporal costs of building an \ac{IR} experimental collection following this paradigm remain extremely high.

Automatic relevance judgment has recently received significant attention in the IR community. In earlier studies, ~\citet{faggioli2023perspectives} studied different levels of human and LLMs collaboration for automatic relevance judgment. They suggested the need for humans to support and collaborate with LLMs for a human-machine collaboration judgment. ~\citet{thomas2023large} leverage LLMs capabilities in judgment at scale, in Microsoft Bing. They used real searcher feedback to build an LLM and prompt in a way that matches the small sample of searcher preferences. Their experiments show that LLMs can be as good as human annotators in indicating the best systems. They also comprehensively investigated various prompts and prompt features for the task and revealed that LLM performance on judgments can vary with simple paraphrases of prompts. Recently, \citet{rahmani2024synthetic} have studied fully synthetic test collection using LLMs. In their study, they generated synthetic queries and synthetic judgment to build a full synthetic test collation for retrieval evaluation. They have shown that LLMs can generate a synthetic test collection that results in system ordering performance similar to evaluation results obtained using the real test collection.

On a different line, \citet{DBLP:conf/sigir/Dietz24} defines a LLM-based ``autograding'' approach. This evaluation strategy targets generated content that cannot be evaluated in a purely offline scenario and it consists of using a question bank as the evaluation test-bed. An \ac{LLM} measures the effectiveness of the generative model in answering the questions, possibly with the supervision of a human. The autograding approach proposed by \citet{DBLP:conf/sigir/Dietz24} includes an automatic passage evaluation whose task aligns with the one evaluated in \texttt{LLMJudge}.

\subsection{Criticisms and Open Challenges}
The use of \acp{LLM} as assessors comes with major bias risks and challenges that should not be neglected, especially considering the impact they might have in the development of \ac{IR} evaluation.

\partitle{Bias}
First and most importantly, \acp{LLM} are affected by bias~\cite{DBLP:conf/fat/BenderGMS21}. Their internal representation of the concepts is, by construction, conditioned on the context such concepts appear in~\cite{DBLP:conf/nips/VaswaniSPUJGKP17}. Thus, depending on the underlying data, the \ac{LLM} might form a biased notion of relevance that might reflect upon the relevance judgments generated by it. Quantifying the bias, identifying its source, and mitigating its consequences are still open issues that need to be addressed. We hope that the release of this collection will help the research community with the needed data to study how to deal with the bias in \ac{LLM}-generated relevance judgments.

\partitle{Circularity}
A second source of concern when it comes to using \acp{LLM} as assessors relates to the risk of \textit{circular evaluation}~\cite{faggioli2023perspectives,DBLP:journals/corr/abs-2409-15133}. For example, the same \ac{LLM} might be used to generate relevance judgments and as a document ranker. This would induce a strong bias on the validity and generalizability of the relevance judgments.

\partitle{Environmental Impact}
An often hidden cost of the \acp{LLM} concerns their environmental impact in terms of energy utilization, carbon emissions~\cite{DBLP:journals/corr/abs-2408-09713,DBLP:conf/sigir/ScellsZZ22}, and water consumption~\cite{DBLP:conf/ictir/ZucconSZ23}.
While \acp{LLM} might allow building collections at a fraction of the monetary and temporal cost, we should account for the environmental impact of such a process, limiting our reliance on ``disposable'' relevance judgments.

\partitle{Vulnerability to Attacks and Adversarial Misuse}
\citet{DBLP:conf/ecir/ParryFMPH24} and \citet{DBLP:conf/sigir-ap/Alaofi0SS24} illustrate the vulnerability of the \acp{LLM} to mischievous manipulations of the corpus. For example,~\citet{DBLP:conf/ecir/ParryFMPH24} show that, by introducing keywords such as the term ``relevant'' in a document, it will more likely considered relevant by an \ac{LLM}. Similar behavior is observed also by \citet{DBLP:conf/sigir-ap/Alaofi0SS24}, who notice that by introducing the query on the document, more probably an \ac{LLM} will consider the document relevant to such a query --- even if the rest of the document is composed by random terms.
More recently, \citet{DBLP:journals/corr/abs-2412-17156} show how, by properly crafting an adversarial run, it is possible to cheat an \ac{LLM} used as an assessor. \citet{DBLP:journals/corr/abs-2412-17156} crafted a run following the same approach used by~\citet{upadhyay2024umbrela} to pool the documents and build the \ac{LLM}-generated relevance judgments used for TREC 2024 RAG. Such a run achieved consistently higher effectiveness under the fully automatic evaluation paradigm compared to its performance based on manual relevance judgments. 

By releasing this collection of \ac{LLM}-generated relevance judgments we want to foster the analysis and study of possible sources of biases and systematic errors, to mitigate them and allow for the development of more effective and robust future solutions that involve \acp{LLM} as tools to support the annotation process.
\section{LLMJudge Resource}
\label{sec:llmjudge_resource}
This section details how we designed the \texttt{LLMJudge} challenge task, the data construction process, and the evaluation metrics.

\begin{figure}
    \centering
    \subfloat[Dev set\label{fig:sample-dev}]
    {
        {
            \includegraphics[scale=0.26]{figs/data/Full-Human_Sample-Human_black_dev.pdf}
        }
    }%
    % \quad
    \subfloat[Test set\label{fig:sample-test}]
    {
        {
            \includegraphics[scale=0.26]{figs/data/Full-Human_Sample-Human_black_test.pdf}
        }
    }%
    \caption{Samples correlation with TREC 2023 DL full qrel}%
    \label{fig:data-samples}%
\end{figure}

\subsection{LLMJudge Task}
The task of the LLMJudge challenge is, given the query and passage as input, how they are relevant. Similar to TREC 2023 Deep Learning track \cite{craswell2024overview}, we use \textit{four-point scale} judgments to evaluate the relevance of the query to the passage as follows:

\begin{itemize}
    \item\textbf{[3] Perfectly relevant}: The passage is dedicated to the query and contains the exact answer. 
    \item\textbf{[2] Highly relevant}: The passage has some answers for the query, but the answer may be a bit unclear, or hidden amongst extraneous information. 
    \item\textbf{[1] Related}: The passage seems related to the query but does not answer it. 
    \item\textbf{[0] Irrelevant}: The passage has nothing to do with the query. 
\end{itemize}

More specifically, the LLMJudge challenge is, by providing the datasets that include queries, passages, and query-passage files to participants, to ask LLMs to generate a score [0, 1, 2, 3] indicating the relevance of the query to the passage.

\subsection{LLMJudge Data}
The \texttt{LLMJudge} challenge dataset is built upon the passage retrieval task dataset of the TREC 2023 Deep Learning track\footnote{\url{https://microsoft.github.io/msmarco/TREC-Deep-Learning.html}} \cite{craswell2024overview}. The TREC 2023 Deep Learning track qrel consists of 82 queries, including 51 real queries and 31 synthetic queries (13 generated by T5 and 18 generated by GPT-4). To create a dev and test set similar to the TREC 2023 Deep Learning track full qrel, we randomly sampled 15 queries from 51 real queries, 5 queries from T5 queries, and 5 queries from GPT-4 queries for each set. Figure \ref{fig:data-samples} shows Kendall's $\tau$ correlation of the TREC 2023 Deep Learning track run submission on LLMJudge sampled dev and test sets with the TREC 2023 Deep Learning track full qrel. Table \ref{tbl:llmjudge-dataset} shows the statistics of the \texttt{LLMJudge} challenge datasets. The test set is used for the generation of judgment by participants, while the development set could be used for few-shot or fine-tuning purposes.

\begin{table}
    % \centering
    \caption{Statistics of LLMJudge Dataset}
    \label{tbl:llmjudge-dataset}
        \begin{tabular}{lcc}
            \toprule
            \textbf{} & \textbf{Dev} & \textbf{Test} \\
            \midrule
             \# queries  & 25    & 25 \\
             \# passage & 7,224 & 4,414 \\
             \# qrels    & 7,263 & 4,423 \\
             \midrule
             \# irrelevant (0)         & 4,538 & 2,005 \\
             \# related (1)            & 1,403 & 1,233 \\
             \# highly relevant (2)    & 625 & 808 \\
             \# perfectly relevant (3) & 697 & 377 \\
            \bottomrule
        \end{tabular}
\end{table}

\subsection{Evaluation}
We evaluate submission results on two different levels, the correlation of the judgments and the ranking correlation of systems evaluated using judgment submissions:

\begin{itemize}
    \item \textbf{Label Correlation.} We use the automated evaluation metrics Cohen's Kappa ($\kappa$) and Krippendorff's Alpha ($\alpha$) on human judgments and the judgments submitted by participants;
    \item \textbf{System Ranking Correlation.} We use Kendall's Tau ($\tau$) and Spearman's rank ($\rho$) correlation to evaluate the system ordering of TREC 2023 Deep Learning Track \cite{craswell2024overview} submitted systems on human judgments and participants' LLM-based judgments.
\end{itemize}

We use \texttt{scikit-learn}\footnote{\url{https://scikit-learn.org/stable/index.html}} to compute Cohen's $\kappa$, Kendall's $\tau$, Spearman's $\rho$. Krippendorff's $\alpha$ is also calculated using the Fast Krippendorff\footnote{\url{https://github.com/pln-fing-udelar/fast-krippendorff}} Python package. 

\subsection{Publicly Available Resources}
To facilitate research in the area we have made the LLMJudge dataset, sample prompt, quick starter for automatic judgment, submitted runs, prompts, codes for quick starting the evaluation, and more detailed results publicly available on the \texttt{LLMJudge} webpage at: \url{https://llm4eval.github.io/LLMJudge-benchmark/}.
\section{Submitted Runs}
\label{sec:intro-methods}
We provide all submitted runs as a resource for future research and comparison. The submissions include 9 \emph{baseline approaches} developed by the organizers and 33 \emph{methods from participating teams}. Analysis of these submissions reveals several methodological directions in \ac{LLM}-based relevance assessment, focusing on \emph{prompting techniques, model adaptation, multi-phase evaluation, aggregation strategies, and classification-based refinement}.  

\begin{table}
    \centering
    \caption{LLMJudge challenge submissions details. Ensemble (Ens.) indicates if submissions combine multiple judges or use them as features to train a classifier for judgment. LR: Logistic Regression, ET: ExtraTrees, GaussianNB: Gaussian Naive Bayes are classifiers. If a submission used multiple prompts, we consider the more advanced one (CoT > Zero-Shot) in this table. FT: Fine-Tuning, N: Numerical, S: Semantic.}
    \adjustbox{max width=\columnwidth}{%
    \begin{tabular}{llccccc}
    \toprule
    \textbf{Submission ID} & \textbf{Model} & \textbf{Size} & \textbf{FT} & \textbf{Prompt} & \textbf{Label} & \textbf{Ens.} \\
    \midrule
    NISTRetrieval-instruct0 & Llama-3-Instruct & 8B & - & Zero-shot & N & - \\
    NISTRetrieval-instruct1 & Llama-3-Instruct & 8B & - & Zero-shot & N & - \\
    NISTRetrieval-instruct2 & Llama-3-Instruct & 8B & - & Zero-shot & N & - \\
    NISTRetrieval-reason0 & Llama-3-Instruct & 8B & - & CoT & N & - \\
    NISTRetrieval-reason1 & Llama-3-Instruct & 8B & - & CoT & N & - \\
    NISTRetrieval-reason2 & Llama-3-Instruct & 8B & - & CoT & N & - \\
    Olz-exp & GPT-4o & - & - & Zero-Shot & S & - \\
    Olz-gpt4o & GPT-4o & - & - & CoT & S & -  \\
    Olz-halfbin & Llama-3-Instruct & 8B & - & CoT & S + N & LR \\
    Olz-somebin & Llama-3-Instruct & 8B  & - & CoT & S + N & LR \\
    Olz-multiprompt & Llama-3-Instruct & 8B & - & CoT & S + N & \checkmark \\
    RMITIR-GPT4o & GPT-4o & - & - & Zero-Shot & N & \checkmark \\
    RMITIR-llama38b & Llama-3-Instruct & 8B & - & Zero-Shot & N & \checkmark \\
    RMITIR-llama70B & Llama-3-Instruct & 70B & - & Zero-Shot & N & \checkmark \\
    TREMA-4prompts & Llama-3-Instruct & 8B & - & Zero-Shot & N & - \\
    TREMA-CoT & Llama-3-Instruct & 8B & - & CoT & N & - \\
    TREMA-all & ChatGPT-3.5/FlanT5-Large & 783M & - & Few-Shot & N & ET \\
    TREMA-direct & ChatGPT-3.5/FlanT5-Large & 783M & - & Few-Shot & N & ET \\
    TREMA-naiveBdecompose & ChatGPT-3.5/FlanT5-Large & 783M & - & Zero-Shot & N & GNB \\
    TREMA-nuggets & ChatGPT-3.5/FlanT5-Large & 783M & - & Zero-Shot & N & ET \\
    TREMA-other & ChatGPT-3.5/FlanT5-Large & 783M & - & Zero-Shot & N & - \\
    TREMA-questions & ChatGPT-3.5/FlanT5-Large & 783M & - & Zero-Shot & N & ET \\
    TREMA-rubric0 & ChatGPT-3.5/FlanT5-Large & 783M & - & Zero-Shot & N & - \\
    TREMA-sumdecompose & Llama-3-Instruct & 8B & - & Zero-Shot & N & - \\
    h2oloo-fewself & GPT-4o & - & - & Few-Shot & N & - \\
    h2oloo-zeroshot1 & Llama-3-Instruct & 8B & \checkmark & Zero-Shot & N & - \\
    h2oloo-zeroshot2 & Llama-3-Instruct & 8B & \checkmark & Zero-Shot & N & - \\
    llmjudge-cot1 & GPT-3.5-turbo & - & - & CoT & N & - \\
    llmjudge-cot2 & GPT-3.5-turbo-16k & - & - & CoT & N & - \\
    llmjudge-cot3 & GPT-4-32k & - & - & CoT & N & - \\
    llmjudge-simple1 & GPT-3.5-turbo & - & - & Zero-Shot & N & - \\
    llmjudge-simple2 & GPT-3.5-turbo-16k & - & - & Zero-Shot & N & - \\
    llmjudge-simple3 & GPT-4-32k & - & - & Zero-shot & N & - \\
    llmjudge-thomas1 & GPT-3.5-turbo & - & - & Zero-Shot & N & - \\
    llmjudge-thomas2 & GPT-3.5-turbo-16k & - & - & Zero-Shot & N & - \\
    llmjudge-thomas3 & GPT-4-32k & - & - & Zero-Shot & N & - \\
    prophet-setting1 & Llama-3-Instruct & 8B & \checkmark & Zero-Shot & S & - \\
    prophet-setting2 & Llama-3-Instruct & 8B & \checkmark & Zero-Shot & S & - \\
    prophet-setting4 & Llama-3-Instruct & 8B & \checkmark & Zero-Shot & S & - \\
    willia-umbrela1 & GPT-4o & - & - & Zero-Shot & N & - \\
    willia-umbrela2 & GPT-4o & - & - & Zero-Shot & S & - \\
    willia-umbrela3 & GPT-4o & - & - & Zero-Shot & S + N & \checkmark \\
    \bottomrule
    \end{tabular}
    }
    \label{tab:infos}
\end{table}

Most submissions implement either \emph{direct prompting} or \emph{criteria decomposition pipelines}. \emph{Direct prompting methods} range from simple \emph{relevance scoring instructions} to \emph{chain-of-thought reasoning}, where \acp{LLM} justify their judgments before assigning a score. Some approaches explore \emph{zero-shot prompting}, while others incorporate \emph{semantic label assignments, linguistic alignment, or multi-prompt aggregation} to improve consistency and reduce overestimation biases.  
Beyond prompting, some teams \emph{fine-tune LLMs on relevance datasets}, including \emph{TREC Deep Learning track qrels} and the \emph{LLMJudge development set}, testing different \emph{model sizes (8B vs.~70B)} to assess the impact of adaptation on evaluation performance.  
A subset of submissions structures evaluation into \emph{multi-phase pipelines}, applying \emph{binary filtering before graded scoring}, \emph{question-based reasoning}, or \emph{decision trees} to refine assessments. Other approaches decompose relevance into \emph{specific dimensions} such as \emph{exactness, coverage, topicality, and contextual fit}, or employ \emph{nugget-based assessments} for more granular judgments.  
To enhance robustness, several methods \emph{combine outputs from multiple prompts or models} using \emph{multi-prompt averaging, binary-to-graded conversions, or conservative ensembling} to stabilize scores. Others treat \emph{relevance assessment as a classification task}, extracting features from LLM outputs and training \emph{Machine Learning classifiers} to refine final scores and improve alignment with human judgments.  

Below we detail the baselines and a summary of the submitted runs. We also summarize the submission details in Table \ref{tab:infos}.

\partitle{LLMJudge Baseline}
The baseline judges provided by the LLMJudge challenge organizers serve as reference methods for evaluation. Three distinct approaches are proposed as baselines: \texttt{llmjudge\\-simple}, \texttt{llmjudge-cot}, and \texttt{llmjudge-thomas}. The \texttt{llmjudge-\\simple} method employs a straightforward prompt, instructing the model to directly provide a relevance judgment based on the query and passage. In contrast, \texttt{llmjudge-cot} adopts a chain-of-thought (CoT) approach, prompting the model to articulate its reasoning process before delivering a judgment. Lastly, \texttt{llmjudge-thomas} incorporates the prompt design introduced by \cite{thomas2023large}, offering an alternative strategy for evaluation.

\partitle{NISTRetrieval-instruct}
This is a submission from NIST which has three different variants, namely, \texttt{NISTRetrieval-instruct0}, \texttt{NISTRetrieval-instruct1}, and \texttt{NISTRetrieval-instruct2} that aims to investigate the reproducibility of the method proposed by \citet{thomas2023large} and the reproducibility capabilities of LLMs when we used them for automatic relevance judgment.

\partitle{NISTRetrieval-reason}
Similar to \texttt{NISTRetrieval-instruct}, this NIST submission includes three related methods -- \texttt{NISTRetrieval-\\reason0}, \texttt{NISTRetrieval-reason1}, and \texttt{NISTRetrieval-reason2}. The team observed that prompting LLMs to provide reasoning across various tasks could improve response quality. To examine whether this approach could also enhance relevance judgment, they modified the prompt from \citet{thomas2023large} to allow the LLM to generate reasoning. These three runs were included to assess the reproducibility capabilities of LLMs when used for evaluation.

\partitle{Prophet-setting}
This method builds on the idea of fine-tuning an LLM with different available datasets for automatic relevance judgment, as described in \cite{meng2024query}\footnote{Code is available at \url{https://github.com/ChuanMeng/QPP-GenRE}}. Specifically, they fine-tuned \texttt{Llama-3-8B} under three different settings, training the model for five epochs in each. These settings include: \texttt{Prophet-setting1}, fine-tuned on the LLMJudge development set; \texttt{Prophet-setting2}, fine-tuned on the qrels of TREC-DL 2019, 2020, and 2021; and \texttt{Prophet-setting4}, which combines fine-tuning on the qrels of TREC-DL 2019, 2020, and 2021 with the LLMJudge development set.

\partitle{William-umbrela1}
This approach is zero-shot prompting the LLM to produce relevance assessments. They used UMBRELA \cite{upadhyay2024umbrela} to generate relevance judgments using the prompting technique suggested by \citet{thomas2023large}. The team mentioned that ``\textit{I tried many different approaches, but I did not manage to find anything that really seemed to consistently improve on zero-shot. It seemed like this dataset may have been harder and/or noisier than others referenced in the literature -- on my development set it was hard to get $> 0.3$ Cohen's $\kappa$, whereas the literature mentions values of $0.4$ up to $0.6$ even.}''.

\partitle{William-umbrela2}
The main idea of this method is to take the approach from the UMBRELA \cite{upadhyay2024umbrela} -- zero-shot prompting technique from \citet{thomas2023large}, but to see if the performance would be improved by asking the model to output semantic labels (i.e., \textit{Irrelevant}, \textit{Related}, \textit{Highly relevant}, \textit{Perfectly relevant}), rather than a numerical score (i.e., 0, 1, 2, 3).

\partitle{William-umbrela3}
This method is an ensemble of~ \texttt{William-\\umbrela1} and \texttt{William-umbrela2} approaches by taking the \textit{min}. The team mentioned that ``\textit{The logic behind using min as an aggregator is that in this dataset, it pays to be conservative in the rating. They also said that on a subset of the training data that they held out for testing, this ensembling approach outperformed either of the two other approaches (i.e., William-umbrela1 and William-umbrela2)''}.

\partitle{H2oloo-fewself}
This method uses the best prompt proposed by Thomas et al.~\cite{thomas2023large} to instruct GPT-4o. It incorporates few-shot examples to guide the model in distinguishing between relevant labels effectively.

\partitle{H2oloo-zeroshot1}
This method fined-tuned a Llama-8B using the TREC DL 2019 to 2022 qrels for relevance judgment prediction.

\partitle{H2oloo-zeroshot2}
This method fined-tuned a Llama-8B using the TREC DL 2019 to 2022 qrels and the LLMJudge dev set qrel for relevance judgment prediction.

\partitle{Olz-gpt4o}
This method uses a simple prompt where they just ask for the relevance judgment without any special techniques. The idea is to see how models can solve relevance judgment tasks without considering any particular prompting or fine-tuning techniques. The primary goal is to assess whether a low-effort prompt could reliably derive relevance labels from LLMs that are practically usable.

\partitle{Olz-exp}
This method is similar to \texttt{Olz-gpt4o} but they also asked LLM to reason its judgment as part of the evaluation.

\partitle{Olz-halfbin}
This method leverages Llama-3 models with $8B$ and $70B$ parameters to assess document relevance using nine distinct prompts. These prompts are divided into two categories: four \emph{graded relevance prompts}, which instruct the model to assign a score from 0 to 3 with slight instruction variations, and five \emph{binary relevance prompts}, which require binary judgments with different definitions of relevance.  
Both model variants generate outputs for all nine prompts. These outputs serve as features for training a logistic regression classifier, which produces the final graded labels. Training is conducted using labels generated by GPT-4o (via the \texttt{Olz-gpt4o} method) rather than the standard development set annotations, based on the assumption that the development and test set labels may have been derived using different methods. Analyzing these discrepancies, the team found GPT-4o’s judgments more aligned with their expectations, leading to its adoption as the primary reference for training.

\partitle{Olz-somebin}
The procedure of this method is identical to the \texttt{Olz-halfbin} method, except the logistic regression classifier was trained on the provided development set labels instead of those generated by GPT-4o (using \texttt{Olz-gpt4o} method).

\partitle{Olz-multiprompt}
This method, instead of using a classifier like \texttt{Olz-halfbin} and \texttt{Olz-somebin}, directly aggregated the relevance judgments by averaging. The binary labels were first scaled by multiplying them by three (to convert them into $0$ or $3$). Then a simple average was calculated across the nine prompts and rounded on a scale of 0 to 3, and the resulting value served as the final graded label.

\partitle{RMIT-IR}
This submission introduces three relevance assessors, \texttt{RMITIR-GPT4o}, \texttt{RMITIR-llama38b}, and \texttt{RMITIR-llama70B}. The proposed approach begins by having the LLM provide a binary relevance judgment to filter out irrelevant queries and improve irrelevance filtering. Next, three scores are generated, and averaged, and the result is rounded to produce the final score. The method was tested using three different LLMs: GPT-4o (\texttt{RMITIR-GPT4o}), Llama3-8B (\texttt{RMITIR-llama38b}), and Llama3-70B (\texttt{RMITIR-llama70B}). The team noted that ``\textit{GPT-4o appears to be the best-performing model based on our experiences}.''

\partitle{TREMA-4prompt}   
This method evaluates passage relevance by decomposing it into four specific criteria: exactness (how precisely the passage answers the query), coverage (proportion of content discussing the query), topicality (subject alignment between passage and query), and contextual fit (presence of relevant background). The evaluation follows a two-phase process where each criterion is first assessed independently and then combined through a final prompt to determine the overall relevance label. Full details of the criteria and rationale are provided in \cite{farzi2024best}.

\partitle{TREMA-CoT}
This method implements a chain-of-thought evaluation process inspired by \citet{Sun2023IsCG}. The approach consists of three phases: First, the \ac{LLM} makes a binary relevance judgment (yes/no) of the passage. Based on this judgment, different relevance criteria are evaluated in the second phase - for relevant passages, exactness and coverage are assessed, while non-relevant passages are evaluated on contextual fit and topicality (all scored 0-3). In the final phase, these scores determine the overall relevance label: relevant passages receive labels 2-3 based on exactness and coverage scores, while non-relevant passages receive labels 0-1 based on contextual fit and topicality assessment.

\partitle{TREMA-other}
This approach investigates whether aligning the linguistic styles of queries and passages can enhance relevance judgments. In the first phase, the LLM generates a query-like representation for each passage, designed to match the query's linguistic style and length. This generated query serves as a summary of the passage's content, formatted in a way that aligns with typical query phrasing. In the second phase, the LLM evaluates the similarity between the original query and the generated query on a scale from 0 to 3, corresponding to the relevance labeling system. Higher similarity scores indicate a stronger alignment between the passage's content and the query's intent. This method integrates linguistic style alignment with content relevance to improve relevance labeling.

\partitle{TREMA-sumdecompose}
This method consists of two phases. Phase one is identical to the \texttt{TREMA-4prompt} method, where the ``relevance'' is decomposed into four criteria, leading to four criteria-specific grades.  In Phase Two, the individual grades from Phase One are summed to produce a total grade. Based on this total, a final relevance label between 0 and 3 is assigned to each query-passage pair: a total grade of 10-12 yields a relevance label of 3, 7-9 yields a relevance label of 2, 5-6 yields a relevance label of 1, and 0-4 yields a relevance label of 0.

\partitle{TREMA-naiveBdecompose}
This method consists of two phases. Phase one is identical to the \texttt{TREMA-4prompt} method, where the ``relevance'' is decomposed into four criteria, leading to four criteria-specific grades. In phase two, these decomposed grades are aggregated into a final relevance label using a Gaussian Naive Bayes model, implemented with Scikit-learn's GaussianNB() classifier. The model is trained on the decomposed feature grades and then predicts the relevance label for each passage.

\partitle{TREMA-rubric0}
This method is based on the RUBRIC Autograder Workbench \cite{dietz2024workbench}. This method defines the relevance of the query via 10 open-ended questions. The questions are generated using the ChatGPT 3.5 model. Each passage is scanned whether it is possible to answer each of the questions (and how well), which is captured as a grade. They use the FLAN-T5-large LLM from Huggingface to grade the answerability from 0 (worst) to 5 (best). Details and prompts are available in the Workbench benchmark \cite{dietz2024workbench}. The grades are mapped to relevance labels by a heuristic mapping on the second-highest grade achieved on any of the questions. Grade 5 is mapped to relevance label 3, grade 4 is mapped to label 1 and all other grades are mapped to label 0. This was the best manual mapping on the dev set \cite{farzi2024rubric}.

\partitle{TREMA-questions}
Same question and grading as in \texttt{TREMA\\-rubric0}, but uses a more elaborate calibration for converting grades to relevance labels, based on \texttt{scikit-learn}'s ExtraTrees classifier. The classifier is based on features that include ranked grades for each question (sorted in descending order), ranked question difficulty (based on average grades across the pool), and counts of correct answers at various grade thresholds (e.g., number of answers graded 5, 4 or better, etc.). Each of these features is encoded using both one-hot and numerical representations to capture detailed information about question-based relevance. The classifier is trained on the dev set.

\begin{table}[]
    \centering
    \caption{Judgment and system ranking correlation of LLMJudge submissions. $\kappa$: Cohen's Kappa, $\alpha$: Krippendorff's alpha, $\tau$: Kendall's Tau, $\rho$: Spearman’s rank correlation. The best results per column are denoted in bold and the second best results are denoted in \textit{italic}.}
    \adjustbox{max width=\columnwidth}{%
    \begin{tabular}{lcccc}
    \toprule
    \textbf{Submission ID} & \textbf{$\kappa$} & \textbf{$\alpha$} & \textbf{$\tau$} & \textbf{$\rho$} \\
    \midrule
    NISTRetrieval-instruct0 & 0.1877 & 0.3819 & 0.9440 & 0.9907 \\
    NISTRetrieval-instruct1 & 0.1874 & 0.3812 & 0.9440 & 0.9907 \\
    NISTRetrieval-instruct2 & 0.1880 & 0.3821 & 0.9440 & 0.9907 \\
    NISTRetrieval-reason0   & 0.1844 & 0.3874 & 0.9052 & 0.9810 \\
    NISTRetrieval-reason1   & 0.1845 & 0.3872 & 0.9009 & 0.9802 \\
    NISTRetrieval-reason2   & 0.1838 & 0.3874 & 0.9052 & 0.9810 \\
    Olz-exp                 & 0.2519 & 0.4701 & 0.9009 & 0.9819 \\
    Olz-gpt4o               & 0.2625 & 0.5020 & 0.8793 & 0.9758 \\
    Olz-halfbin             & 0.2064 & 0.4536 & 0.9085 & 0.9830 \\
    Olz-multiprompt         & 0.2445 & 0.4551 & 0.9267 & 0.9867 \\
    Olz-somebin             & 0.2109 & 0.4471 & 0.9042 & 0.9822 \\
    RMITIR-GPT4o            & 0.2388 & 0.4108 & 0.8966 & 0.9798 \\
    RMITIR-llama38b         & 0.2006 & 0.3873 & 0.8879 & 0.9758 \\
    RMITIR-llama70B         & 0.2654 & 0.4873 & 0.9353 & 0.9883 \\
    TREMA-4prompts          & 0.1829 & 0.2888 & \textit{0.9483} & \textbf{0.9919} \\
    TREMA-CoT               & 0.1961 & 0.3852 & 0.8956 & 0.9799 \\
    TREMA-all               & 0.1471 & 0.3855 & 0.9138 & 0.9863 \\
    TREMA-direct            & 0.1742 & 0.3729 & 0.9009 & 0.9819 \\
    TREMA-naiveBdecompose   & 0.1741 & 0.3579 & 0.9128 & 0.9838 \\
    TREMA-nuggets           & 0.0604 & 0.1691 & 0.8664 & 0.9718 \\
    TREMA-other             & 0.1408 & 0.2712 & 0.8276 & 0.9447 \\
    TREMA-questions         & 0.1137 & 0.3148 & 0.9095 & 0.9839 \\
    TREMA-rubric0           & 0.0779 & 0.1036 & 0.8276 & 0.9544 \\
    TREMA-sumdecompose      & 0.2088 & 0.3926 & 0.9300 & 0.9870 \\
    h2oloo-fewself          & 0.2774 & \textbf{0.4958} & 0.9085 & 0.9822 \\
    h2oloo-zeroshot1        & \textit{0.2817} & 0.4812 & 0.9181 & 0.9827 \\
    h2oloo-zeroshot2        & 0.2589 & 0.3898 & 0.8353 & 0.9604 \\
    llmjudge-cot1           & 0.1284 & 0.3218 & 0.9267 & 0.9871 \\
    llmjudge-cot2           & 0.1560 & 0.3263 & 0.9267 & 0.9875 \\
    llmjudge-cot3           & 0.2271 & 0.4870 & 0.9267 & 0.9851 \\
    llmjudge-simple1        & 0.0754 & 0.2808 & 0.9181 & 0.9863 \\
    llmjudge-simple2        & 0.1327 & 0.3672 & 0.8966 & 0.9790 \\
    llmjudge-simple3        & 0.2110 & 0.4642 & 0.9052 & 0.9810 \\
    llmjudge-thomas1        & 0.1236 & 0.3207 & 0.8664 & 0.9689 \\
    llmjudge-thomas2        & 0.1723 & 0.3853 & 0.8793 & 0.9750 \\
    llmjudge-thomas3        & 0.2293 & 0.4877 & 0.9181 & 0.9867 \\
    prophet-setting1        & 0.1823 & 0.4069 & 0.9042 & 0.9826 \\
    prophet-setting2        & 0.1757 & 0.3144 & \textbf{0.9516} & \textit{0.9914} \\
    prophet-setting4        & 0.1471 & 0.1623 & 0.8568 & 0.9608 \\
    willia-umbrela1         & \textbf{0.2863} & \textit{0.4918} & 0.9009 & 0.9806 \\
    willia-umbrela2         & 0.2688 & 0.4556 & 0.8870 & 0.9769 \\
    willia-umbrela3         & 0.2741 & 0.4535 & 0.8707 & 0.9730 \\
    \bottomrule
    \end{tabular}
    }
    
    \label{tab:main-results}
\end{table}

\partitle{TREMA-nuggets}
Same approach as \texttt{TREMA-questions}, but uses 10 open-ended key fact nuggets instead of questions, along with an adapted prompt that assesses whether key facts are mentioned in the passage. The same ExtraTrees classifier with the same features is used for converting grades into relevance labels.

\partitle{TREMA-direct}
This approach focuses exclusively on features of direct relevance
labeling methods, which instruct an LLM to judge whether a passage is relevant for a query, using a variety of prompts from \citet{Sun2023IsCG}, \citet{faggioli2023perspectives}, and HELM \cite{liang2022holistic}. The model excludes question-based and nugget-based features, simplifying its input to focus solely on the predictive power of direct labeling. The relevance labels are obtained with an ExtraTrees classifier trained on the dev set. Features include binary or multi-class predictions from labeling approaches. Each label is encoded using both one-hot and numerical encodings to capture both categorical and ordinal aspects of the predictions. This approach is computationally lighter than TREMA-all and serves as a baseline to evaluate how well direct relevance labels alone can predict passage relevance.

\partitle{TREMA-all}
This approach incorporates all features from \texttt{TREMA-\\questions}, \texttt{TREMA-nuggets}, and \texttt{TREMA-direct} approaches via a single ExtraTrees classifier that is trained on the dev set.
\section{Results}
\label{sec:results}
% In this section, we answer the research questions formulated in Section \ref{experiments}.

\subsection{Effectiveness of indexing succinct facts to improve information retrieval efficiency} To answer \textbf{RQ1}, we measure the memory and computational costs of fact-checking using full-Wikipedia compared to the pruned version proposed in this work. We first measure the index size on disk, measuring the raw JSON file size containing the article titles and texts, for each experiment setting. In \autoref{fig:disk_size}, we observe a significant reduction in disk space usage with HoVer having a reduction ranging from \textbf{44-55\%}, and WiCE from \textbf{44-57\%}. Additionally, the number of sentences stored in the index also decreases, with HoVer showing a reduction from 52-61\% and WiCE 52-59\%, indicating that at least half of the sentences are not helpful in claim verification.

\begin{table}[htb!]
\small
\centering
\footnotesize
\begin{tabular}{c c c c c}
\toprule
\small
\textbf{Method} & \textbf{Disk Size} & \textbf{Size reduction} & \textbf{\#Sentences} & \textbf{\% decrease}\\
\hline \hline
\multicolumn{1}{l}{\colorg\textbf{HoVer}} & \colorg& \colorg & \colorg & \colorg  \\
Full-Wiki  &  11.28 GiB & - & 94,914,378 & -   \\
Fact Extraction & 6.19 GiB & \down{45}& 45,894,704 & \down{52} \\
Citation Extraction & \textbf{5.07 GiB} & \down{\textbf{55} } & \textbf{36,886,889} & \down{\textbf{61}} \\
Fusion & 5.45 GiB & \down{52} & 39,842,574 & \down{{58}} \\

\hline
\multicolumn{1}{l}{\colorg \textbf{WiCE}} & \colorg &  \colorg & \colorg & \colorg \\
Full-Wiki & 15.28 GiB & - &  126,533,841 & -  \\
Fact Extraction & 8.56 GiB & \down{44} & 61,040,380 & \down{52}\\
Citation Extraction & \textbf{6.56 GiB} & \down{\textbf{57}} & \textbf{51,735,961} & \down{\textbf{59}}  \\
Fusion & 6.85 GiB & \down{55}& 54,070,295 & \down{57} \\

\hline
\end{tabular}
\caption{Comparison sizes for the corpora per experiment setting, consisting of English Wikipedia articles 2017 (HoVer) and 2024 (WiCE). Reduction is measured compared to the  Full-Wiki data setting. \down{} denotes a reduction in corpus size and number of sentence compared to Full-Wiki setting.}
\label{fig:disk_size}
\end{table}
\vspace{-2em}

 Following the reduction in disk size, a notable improvement in retrieval latency is evident, as demonstrated in \autoref{tab:bm25_latency}.
Regarding document retrieval latency (which encompasses both column values), there's an observed speedup ranging from approximately 1.5x (334 ms) to 1.6x (316 ms) compared to the original experimental setting for HoVer (495 ms). Similarly, in WiCE experiments, we witness a comparable speedup rate ranging from 1.4x (446 ms) to 1.6x (399 ms) compared to the original experimental setting (636 ms). This observation suggests that while the reduced text size contributes to efficient retrieval, it could further be improved.

\begin{table}[htb!]
\centering
\small
\footnotesize
% \vspace{-1cm}
\begin{tabular}{l c c c c}
\multirow{2}{*}{\makecell{\textbf{Methods}}} & \multirow{2}{*}{\textbf{Retrieval}} & \multirow{2}{*}{\makecell{\textbf{Total Latency}}} & \multirow{2}{*}{\makecell{\textbf{Speedup}}} \\
& \\
\hline
\multicolumn{1}{l}{\colorg\textit{HoVer}} & \colorg & \colorg & \colorg \\
 Full-Wiki & 426  ms & 659 ms & - \\
Fact Extraction & 257 ms  & 338  ms & 1.9x \\
Citation Extraction & 246 ms  &  327  ms & \speedup{2.0x}  \\
Fusion & 265 ms  & 345  ms & 1.9x  \\

\multicolumn{1}{l}{\colorg\textit{WiCE}} & \colorg & \colorg & \colorg \\
  Full-Wiki &  559 ms  &  831  ms & - \\
Fact Extraction & 372 ms  & 468   ms & 1.8x \\
Citation Extraction & 330 ms   & 419  ms & \speedup{2.0x} \\
Fusion & 347 ms & 436  ms & 1.9x \\
\hline
\end{tabular}
\caption{Retrieval and total latency for Sparse retrieval with Re-ranking. Speedup is compared with respect to the total latency of the Full-Wiki setup.}
\label{tab:bm25_latency}
\vspace{-2em}
\end{table}

% \begin{table}[htpb!]
\centering
\footnotesize
\begin{tabular}{l c c c c c c c c}
\multirow{2}{*}{} & \multicolumn{2}{c}{\makecell{Document retrieval}} & \multirow{2}{*}{\makecell{Sentence \\ Retrieval}} & \multirow{2}{*}{\makecell{Claim \\ Verification}} & \multicolumn{2}{c}{Total Latency} & \multicolumn{2}{c}{Speedup} \\
\cline{2-3}\cline{6-7}\cline{8-9}
& CPU & GPU &  & &  CPU & GPU & CPU & GPU \\ 
\hline
\multicolumn{1}{l}{\textit{HoVer}} &  \\
 \textbf{Full-Wiki (S+R) } &  \multicolumn{2}{c}{\textbf{491  ms}} & \textbf{157  ms} & \textbf{7 ms} &  \multicolumn{2}{c}{\textbf{659 ms}} & - & - \\

Full-Wiki & 515 ms & 31 ms & 153 ms & 8 ms & 676 ms & 192  ms & 1.0x & 3.4x \\
% Original  &  523 ms & 30 ms  & - & 9 ms & 532 ms & 39 ms & 1.2x & 16.9x \\
Claim detection & 513 ms & 23 ms & - & 8 ms & 521 ms & 31 ms & 1.3x & 21.3x  \\
Citation Extraction & 479 ms & 23 ms & - & 9 ms &  488 ms & 32 ms & 1.4x & 20.6x \\
Fusion & 500 ms  & 23 ms & - & 9 ms & 509 ms & 32  ms & 1.3x & 20.6x \\

\multicolumn{1}{l}{\textit{WiCE}} & \\
 \textbf{Full-Wiki (S+R)} & \multicolumn{2}{c}{\textbf{636 ms}} & \textbf{186 ms} & \textbf{9  ms}  & \multicolumn{2}{c}{\textbf{831  ms}} & - & - \\
Full-Wiki & 685 ms & 34 ms & 184 ms & 9 ms & 878 ms & 227  ms & 1.0x & 3.7x \\
% Original  & 610  ms & 34 ms  & - & 9 ms & 619  ms & 43 ms & 1.3x & 19.3x \\
Claim detection &  622 ms & 31 ms  & - & 9 ms & 631 ms & 40 ms & 1.3x & 20.8x \\
Citation Extraction &  610 ms & 31 ms  & - & 9 ms & 619 ms & 40 ms & 1.3x & 20.8x  \\
Fusion & 619  ms & 31 ms  & - & 9 ms & 628 ms & 40 ms & 1.3x & 20.8x  \\[5mm]
\hline
\end{tabular}
\caption{Retrieval and inference latency for Dense retrieval setup on data settings. Speedup is compared with respect to the total latency of the Sparse Retrieval setup with original data setting (bold font).}
\label{tab:faiss_latency}
\end{table}




% However, sparse retrieval does not capture semantic information and requires a costly re-ranking stage. While transitioning from Sparse to Dense retrieval may help improve the performance dense retrieval introduces additional computational costs as seen in Table \ref{tab:faiss_latency}. This is due to our use of FAISS utilising fixed-dimensionality vectors where despite varying article text lengths, the constant number of text embeddings minimizes impact on retrieval speed. However, dense retrieval libraries offer GPU support, which can yield substantial speedups compared to the CPU-based retrieval of BM25 and FAISS. GPU retrieval shows substantial speedups: 16.6-22.3x for HoVer and 17.9-20.2x for WiCE compared to CPU retrieval. Compared to BM25, GPU retrieval offers 16.0-21.5x speedup for HoVer and 18.7-20.5x speedup for WiCE. Thus, making Dense Retrieval a highly efficient and viable option over standard Sparse Retrieval with re-ranking.
%\vspace{-0.5em}

\noindent \textbf{Insight 1}: \textit{
Extraction of succinct facts reduces storage requirements and improves latency for Sparse retrieval while only leading to a minor loss in task performance.}
\vspace{-2em}
% %%%%%%%%%%%%%%%%%%%%%%%%%%%%%%%%%%%%%%%%%%%%%%%%%%%%%%%%%%%%%%%%%%%%%%%%%%%%%%%%%%%

\subsection{Effectiveness of pruned knowledge sources on overall pipeline efficiency and downstream fact-checking performance?}
To answer \textbf{RQ2}, we now shift focus to analyzing the inference time throughout the entire pipeline instead of solely the retrieval part. Extraction of just supporting facts not only has a improvement in the retrieval stage but also on the overall inference latency across the pipeline. For HoVer this being a 1.9-2.0x speedup and 1.8-2.0x for WiCE experiments. This improvement can be attributed to not only faster retrieval times but also the elimination of the Sentence Retrieval stage, which previously imposed significant latency overhead. 

%\begin{table}[htb!]
\small
\begin{tabular}{c c c c c c c c}
\multirow{2}{*}{Experiment setting} & \multirow{2}{*}{Accuracy} & \multicolumn{2}{c}{F1} & \multicolumn{2}{c}{Precision} & \multicolumn{2}{c}{Recall}  \\ 
\cline{3-8}
  & &  Weighted  & Macro & Weighted & Macro & Weighted & Macro      \\
\hline
\multicolumn{1}{l}{\textit{Sparse + Re-ranking}} & & & & \\
Full-Wiki & \textbf{67.79} & \textbf{67.59} & \textbf{67.63} & \textbf{68.45} & \textbf{68.39} & \textbf{67.79}  & \textbf{67.93}\\
Claim detection & \underline{62.33} & 62.02 & 62.08 & \underline{62.98} & \underline{62.92} & \underline{62.33} & \underline{62.50} \\
Citation Extraction & 60.91 & 60.61 & 60.66 & 61.47 & 61.42 & 60.91 & 61.07 \\
Fusion & 62.28 & \underline{62.15} & \underline{62.18} & 62.60 & 62.56 & 62.28 & 62.39  \\[5mm]

\hline
\multicolumn{1}{l}{\textit{Dense Retrieval}} & & & & \\
Full-Wiki & 64.60 & 64.45 & 64.45 & 64.86 & 64.86 & 64.60 & 64.60 \\
% Original & 62.90 & 62.72 & 62.76 & 63.33 & 63.28 & 62.90 & 63.02 \\
Claim detection & \underline{61.50} & \underline{60.94} & \underline{60.94} & \underline{62.20} & \underline{62.20} & \underline{61.50} & \underline{61.50} \\
Citation Extraction & 59.67 & 59.40 & 59.46 & 60.13 & 60.09 & 59.67 & 59.82 \\
Fusion & 59.51 & 59.32 & 59.37 & 59.85 & 59.81 & 59.51 & 59.64  \\[5mm]

\hline
\multicolumn{1}{l}{\textit{Index Compression}} & & & & & & &  \\
Full-Wiki & 63.30 & 62.54 & 62.54 & 64.48 & 64.48 & 63.30 & 63.30 \\
% Original & 63.02 & 62.08 & 62.08 & 64.46 & 64.46 & 63.02 & 63.02  \\
Claim detection & \underline{61.92} & \underline{61.71} & \underline{61.71} & \underline{62.19} & \underline{62.19} & \underline{61.92} & \underline{61.93}  \\
Citation Extraction & 59.98 & 59.12 & 59.12 & 60.89 & 60.89 & 59.98 & 59.98   \\
Fusion & 61.58 & 61.43 & 61.43 & 61.75 & 61.75 & 61.58 & 61.58   \\[5mm]

\hline
\end{tabular}
\caption{Performance experiments on HoVer data and adjustments using full document text of English Wikipedia. The underlined-styled values represent the second best  within each retrieval setup.}
\label{tab:hover_performance_metrics}
\end{table}
% Full document text 
% 
% \begin{figure}
%     \centering
%     \includegraphics[width=0.82\linewidth]{figs/graphs/hover_perf.png}
%     \caption{HoVer performance comparison}
%     \label{fig:enter-label}
% \end{figure}
% %\begin{table}[htb!]
\centering
\footnotesize
\begin{tabular}{c c c c c c c c}
\multirow{2}{*}{Experiment setting} & \multirow{2}{*}{Accuracy} & \multicolumn{2}{c}{F1} & \multicolumn{2}{c}{Precision} & \multicolumn{2}{c}{Recall}  \\ 
\cline{3-8}
  & &  Weighted  & Macro & Weighted & Macro & Weighted & Macro      \\
\hline
\multicolumn{1}{l}{\textit{Sparse + Re-ranking}} & & & & \\
Full-Wiki & \textbf{63.69} &  \textbf{61.84} &  \textbf{55.24} &  \textbf{61.12} &  \textbf{56.54} &  \textbf{63.69} &  \textbf{55.32 } \\
Claim detection & 61.90 & 60.12 & \underline{53.33} & 59.27 & 54.26 & 61.90 & \underline{53.53} \\
Citation Extraction & 61.01 & 59.56 & 52.96 & 58.75 & 53.59 & 61.01 & 53.09 \\
Fusion & \underline{63.39} & \underline{60.21} & 52.48 & \underline{59.46} & \underline{54.69} & \underline{63.39} & 53.27  \\[5mm]

\hline
\multicolumn{1}{l}{\textit{Dense Retrieval}} & & & & \\
Full-Wiki &  61.61 & 60.95 & 55.21 & 60.47 & 55.49 & 61.61 & 55.13 \\
% Full-Wiki  & 60.42 & 58.80 & 51.96 & 57.90 & 52.58 & 60.42 & 52.19 \\
Claim detection & 61.01 & 58.94 & 51.78 & 57.96 & 52.70 & 61.01 & 52.17 \\
Citation Extraction & 58.63 & 58.48 & \underline{52.92} & 58.35 & 52.95 & 58.63 & \underline{52.91} \\
Fusion & \underline{61.31} & \underline{59.34} & 52.30 & 58.40 & \underline{53.23} & \underline{61.31} & 52.62  \\[5mm]

\hline
\multicolumn{1}{l}{\textit{Index Compression}} & & & & & & &  \\
Full-Wiki & 62.46 & 61.38 & 55.27 & 60.74 & 55.84 & 62.46 & 55.20  \\
% Original  & 60.46 & 60.63 & 55.64 & 60.81 & 55.60 & 60.46 & 55.70  \\
Claim detection & 59.31 & 59.02 & \underline{53.32} & 58.77 & 53.39 & 59.31 & \underline{53.30}  \\
Citation Extraction & 60.74 & 59.21 & 52.42 & 58.34 & 53.04 & 60.74 & 52.60  \\
Fusion & \underline{63.04} & \underline{59.79} & 51.89 & 58.94 & \underline{53.97} & \underline{63.04} & 52.76 \\[5mm]

\hline
\end{tabular}
\caption{Performance experiments on WiCE data and adjustments using full document text of English Wikipedia. The bold-styled values represent the baseline while the underlined-styled values represent the highest scores of the re-ranked data within a retrieval setup category.}
\label{tab:wice_performance_metrics}
\end{table}
% Full document text 
% 

% \begin{figure}[hbt!]
%     \centering
%     \includegraphics[width=0.85\linewidth]{figs/graphs/wice_perf.png}
%     \caption{WiCe performance comparison}
%     \label{fig:enter-label}
% \end{figure}

\begin{figure*}[hbt!]
    \begin{subfigure}{.5\textwidth}
        

\begin{tikzpicture}
\edef\mylst{"67.59","64.45","62.54"}
\edef\explora{"62.15","59.32","61.43"}

    \begin{axis}[
            ybar=1.5pt,
            width=6.7cm,
            bar width=0.35,
            every axis plot/.append style={fill},
            grid=major,
            xtick={1, 4, 8,9,11},
            xticklabels={Sparse + re-rank, Dense, IC},
            ylabel style = {font=\tiny},
        yticklabel style = {font=\boldmath \tiny,xshift=0.05ex},
        xticklabel style ={font=\tiny,yshift=0.5ex},
            ylabel={Performance (F1 weighted)},
            enlarge x limits=0.15,
            ymin=0,
            ymax=86,
            legend style ={font=\tiny,yshift=0.05ex},
            area legend,
            nodes near coords style={font=\tiny,align=center,text width=1em},
            legend entries={FW, FE, CE, Fu},
            legend cell align={left},
            legend pos=north west,
            legend columns=-1,
            legend style={/tikz/every even column/.append style={column sep=0.06cm}},
        ]
        \addplot+[
            ybar,
            plotColor1*,
            postaction={
                    pattern=north east lines
                },
                    nodes near coords=\pgfmathsetmacro{\mystring}{{\mylst}[\coordindex]}\textbf{\mystring},
        ] plot coordinates {
                (1,67.59)
                (4,64.45)
                (8,62.54)
            };
        \addplot+[
            ybar,
            plotColor2*,
        ] plot coordinates {
                (1,62.02)
                (4,60.94)
                (8,61.71)
                (9,0)
            };

                    \addplot+[
            ybar,
            plotColor3*,
            draw=black,
    nodes near coords align={vertical},
            postaction={
                    pattern=north west lines
                },
        ] plot coordinates {
                (1,60.61)
                (4,59.40)
                (8,59.12)
                (9,0)
            };
             \addplot+[
            ybar,
            plotColor4*,
            draw=black,
            postaction={
                    pattern=north east lines
                },
            nodes near coords=\pgfmathsetmacro{\mystring}{{\explora}[\coordindex]}\textbf{\mystring},
        ] plot coordinates {
                (1,62.15)
                (4,59.32)
                (8,61.43)
            };
    \end{axis}
\end{tikzpicture}

\subcaption{HoVer}
    \end{subfigure}
        \begin{subfigure}{.5\textwidth}
    

\begin{tikzpicture}
\edef\mylst{"61.84","60.95","61.38"}
\edef\explora{"60.21","59.34","59.79"}

    \begin{axis}[
            ybar=1.5pt,
            width=6.7cm,
            bar width=0.35,
            every axis plot/.append style={fill},
            grid=major,
            xtick={1, 4, 8,9,11},
            xticklabels={Sparse + re-rank, Dense, IC},
            ylabel style = {font=\tiny},
        yticklabel style = {font=\boldmath \tiny,xshift=0.05ex},
        xticklabel style ={font=\tiny,yshift=0.5ex},
            ylabel={Performance (F1 weighted)},
            enlarge x limits=0.15,
            ymin=0,
            ymax=86,
            legend style ={font=\tiny,yshift=0.05ex},
            area legend,
            nodes near coords style={font=\tiny,align=center,text width=1em},
            legend entries={FW, FE, CE, Fu},
            legend cell align={left},
            legend pos=north west,
            legend columns=-1,
            legend style={/tikz/every even column/.append style={column sep=0.06cm}},
        ]
        \addplot+[
            ybar,
            plotColor1*,
            postaction={
                    pattern=north east lines
                },
                    nodes near coords=\pgfmathsetmacro{\mystring}{{\mylst}[\coordindex]}\textbf{\mystring},
        ] plot coordinates {
                (1,61.84)
                (4,60.95)
                (8,61.38)
            };
        \addplot+[
            ybar,
            plotColor2*,
        ] plot coordinates {
                (1,60.12)
                (4,58.94)
                (8,59.02)
                (9,0)
            };

                    \addplot+[
            ybar,
            plotColor3*,
            draw=black,
    nodes near coords align={vertical},
            postaction={
                    pattern=north west lines
                },
        ] plot coordinates {
                (1,59.56)
                (4,58.48)
                (8,59.21)
                (9,0)
            };
             \addplot+[
            ybar,
            plotColor4*,
            draw=black,
            postaction={
                    pattern=north east lines
                },
            nodes near coords=\pgfmathsetmacro{\mystring}{{\explora}[\coordindex]}\textbf{\mystring},
        ] plot coordinates {
                (1,60.21)
                (4,59.34)
                (8,59.79)
            };
    \end{axis}
\end{tikzpicture}

    \subcaption{Wice}

    \end{subfigure}
    \caption{HoVer and WiCe task performance (FW- Full-Wiki, FE - Fact Extraction, IC- Index Compression, CE - Citation Extraction, Fu - Fusion)}
    \label{fig:performance_plot}
    \end{figure*}

\begin{figure*}[hbt!]
    \begin{subfigure}{.5\textwidth}
        

\begin{tikzpicture}
\edef\mylst{"67.59","64.45","62.54"}
\edef\explora{"62.15","59.32","61.43"}

    \begin{axis}[
            ybar=1.5pt,
            width=6.2cm,
            bar width=0.35,
            every axis plot/.append style={fill},
            grid=major,
            xtick={1, 4, 8,9,11},
            xticklabels={Sparse + re-rank, Dense, IC},
            ylabel style = {font=\tiny},
        yticklabel style = {font=\boldmath \tiny,xshift=0.05ex},
        xticklabel style ={font=\tiny,yshift=0.5ex},
            ylabel={Recall@10},
            enlarge x limits=0.15,
            ymin=0,
            ymax=0.5,
            legend style ={font=\tiny,yshift=0.05ex},
            area legend,
            nodes near coords style={font=\tiny,align=center,text width=1em},
            legend entries={FW, FE, CE, Fu},
            legend cell align={left},
            legend pos=north west,
            legend columns=-1,
            legend style={/tikz/every even column/.append style={column sep=0.06cm}},
        ]
        \addplot+[
            ybar,
            plotColor1*,
            postaction={
                    pattern=north east lines
                },
        ] plot coordinates {
                (1,0.136)
                (4,0.123)
                (8,0.098)
            };
        \addplot+[
            ybar,
            plotColor2*,
        ] plot coordinates {
                (1,0.105)
                (4,0.094)
                (8,0.098)
                (9,0)
            };

                    \addplot+[
            ybar,
            plotColor3*,
            draw=black,
    nodes near coords align={vertical},
            postaction={
                    pattern=north west lines
                },
        ] plot coordinates {
                (1,0.126)
                (4,0.143)
                (8,0.096)
                (9,0)
            };
             \addplot+[
            ybar,
            plotColor4*,
            draw=black,
        ] plot coordinates {
                (1,0.124)
                (4,0.141)
                (8,0.097)
            };
    \end{axis}
\end{tikzpicture}

\subcaption{WiCE (nDCG@10)}
    \end{subfigure}
        \begin{subfigure}{.5\textwidth}
    

\begin{tikzpicture}
\edef\mylst{"67.59","64.45","62.54"}
\edef\explora{"62.15","59.32","61.43"}

    \begin{axis}[
            ybar=1.5pt,
            width=6.4cm,
            bar width=0.35,
            every axis plot/.append style={fill},
            grid=major,
            xtick={1, 4, 8,9,11},
            xticklabels={Sparse + re-rank, Dense, IC},
            ylabel style = {font=\tiny},
        yticklabel style = {font=\boldmath \tiny,xshift=0.05ex},
        xticklabel style ={font=\tiny,yshift=0.5ex},
            ylabel={Recall@10},
            enlarge x limits=0.15,
            ymin=0,
            ymax=0.5,
            legend style ={font=\tiny,yshift=0.05ex},
            area legend,
            nodes near coords style={font=\tiny,align=center,text width=1em},
            legend entries={FW, FE, CE, Fu},
            legend cell align={left},
            legend pos=north west,
            legend columns=-1,
            legend style={/tikz/every even column/.append style={column sep=0.06cm}},
        ]
        \addplot+[
            ybar,
            plotColor1*,
        ] plot coordinates {
                (1,0.309)
                (4,0.195)
                (8,0.160)
            };
        \addplot+[
            ybar,
            plotColor2*,
        ] plot coordinates {
                (1,0.241)
                (4,0.166)
                (8,0.163)
                (9,0)
            };

                    \addplot+[
            ybar,
            plotColor3*,
            draw=black,
    nodes near coords align={vertical},
            postaction={
                    pattern=north west lines
                },
        ] plot coordinates {
                (1,0.295)
                (4,0.226)
                (8,0.174)
                (9,0)
            };
             \addplot+[
            ybar,
            plotColor4*,
            draw=black,
        ] plot coordinates {
                (1,0.286)
                (4,0.223)
                (8,0.160)
            };
    \end{axis}
\end{tikzpicture}

    \subcaption{WiCE (Recall@10)}

    \end{subfigure}
    \caption{Retrieval performance comparison}
    \label{fig:retrieval_perf}
    \end{figure*}
The evaluation of Sparse and Dense Retrieval setups in HoVer and WiCE experiments reveals that Sparse Retrieval, particularly fact extraction (FE) and Fusion approaches, maintains performance closest to the Full-Wiki setup as measured by weighted F1 in Figure \ref{fig:performance_plot}, while citation extraction has a larger drop in performance. Most notably, the Fusion method compared to the other methods has relatively high scores, underscoring the importance of combining supporting facts extraction methods for optimal results. We also report retrieval performance for WiCE Figure \ref{fig:retrieval_perf} using measures nDCG@10 and Recall@10 using annotated documents provided for WiCE. We observe trends similar to overall task performance demonstrating that efficient retrieval approaches explored do not negatively impact task performance to a significant extent.

\noindent\textbf{Insight 2}: \textit{We find that extracting supporting facts improves efficiency across the entire pipeline, with Sparse setups achieving up to 2.0x speedups with only a minimal performance decline.}
\vspace{-1em}

\begin{table}[htb!]
\centering
\footnotesize
\begin{tabular}{l  c c c c }
\hline
\multirow{2}{*}{Method}  & \multicolumn{2}{c}{Total Latency}  & \multicolumn{2}{c}{Speedup} \\
\cline{2-5}
& CPU & GPU  &  CPU & GPU  \\ 
\hline \hline
\multicolumn{1}{l}{\colorg \textit{HoVer}} & \colorg & \colorg & \colorg & \colorg \\
 Full-Wiki (S+R) &  \multicolumn{2}{c}{659 ms} & - & - \\
Full-Wiki & 214  ms & 174 ms & 3.1x &  3.8x \\
% Original  &  55 ms & 13 ms  & - & 12 ms & 67 ms & 25 ms & 9.8x &  26.4x \\
Fact Extraction  & 60 ms & 21 ms & 11.0x & 31.4x  \\
Citation Extraction  & \textbf{51 ms} & \textbf{20 ms} & \textbf{\speedup{12.9x}} & \textbf{\speedup{33.0x}} \\
Fusion  & 63 ms & 24 ms &  10.5x & 27.5x \\
\hline
\multicolumn{1}{l}{\colorg\textit{WiCE}} & \colorg & \colorg & \colorg & \colorg   \\
 Full-Wiki (S+R) & \multicolumn{2}{c}{831  ms} & - & - \\
Full-Wiki &  292 ms & 238 ms & 2.8x  &  3.5x \\
% Original  &  95 ms & 43  ms  & - & 11 ms & 106 ms & 54 ms & 7.8x &  15.4x \\
Fact Extraction  & 103 ms & 48 ms & 8.1x & 17.3x \\
Citation Extraction  & 98 ms & 46 ms & 8.5x & 18.1x \\
Fusion  &\textbf{98 ms} & \textbf{46  ms} & \speedup{8.5x} &  \speedup{18.1x} \\
\hline
\end{tabular}
\caption{Latency and speedup measurements for Index compression setup. Speedup is compared with respect to the total latency of Sparse-retrieval + Re-rank (S+R) pipeline with the Full-Wiki setup. (S+R) runs on both CPU and GPU, sparse retrieval running on CPU and rest of components running on GPU}
%\vspace{-1cm}
\label{tab:jpq_latency}
\end{table}

%\begin{table}[htb!]
\centering
\footnotesize
\begin{tabular}{l c c c c c c c c}
\hline
\multirow{2}{*}{Method} & \multicolumn{2}{c}{\makecell{Term-based \\ document retrieval}} & \multirow{2}{*}{\makecell{Sentence \\ Retrieval}} & \multirow{2}{*}{\makecell{Claim \\ Verification}} & \multicolumn{2}{c}{Total Latency}  & \multicolumn{2}{c}{Speedup} \\
\cline{2-3}\cline{6-7}\cline{8-9}
& CPU & GPU &  & &  CPU & GPU & CPU & GPU \\ 
\hline \hline
 \multicolumn{1}{l}{\colorg\textit{HoVer}} & \colorg & \colorg & \colorg & \colorg & \colorg & \colorg & \colorg & \colorg \\
 \textbf{Full-Wiki (S+R)} &  \multicolumn{2}{c}{\textbf{491  ms}} & \textbf{157  ms} & \textbf{7 ms} &  \multicolumn{2}{c}{\textbf{659 ms}} & - & - \\
Full-Wiki &  53 ms & 13 ms & 153 ms & 8 ms & 214  ms & 174 ms & 3.1x &  3.8x \\
% Original  &  55 ms & 13 ms  & - & 12 ms & 67 ms & 25 ms & 9.8x &  26.4x \\
Claim detection  &  51 ms & 12  ms  & - &  9 ms & 60 ms & 21 ms & 11.0x & 31.4x  \\
Citation Extraction  & 46  ms & 11 ms  & - & 9 ms & 51 ms & 20 ms & 12.9x & 33.0x \\
Fusion  & 51  ms & 12 ms  & - & 12 ms & 63 ms & 24 ms &  10.5x & 27.5x \\
\hline
\multicolumn{1}{l}{\colorg\textit{WiCE}} & \colorg & \colorg & \colorg & \colorg & \colorg & \colorg & \colorg & \colorg  \\
 \textbf{Full-Wiki (S+R)} & \multicolumn{2}{c}{\textbf{636 ms}} & \textbf{186 ms} & \textbf{9  ms}  & \multicolumn{2}{c}{\textbf{831  ms}} & - & - \\
Full-Wiki & 97 ms & 43 ms & 186 ms & 9 ms & 292 ms & 238 ms & 2.8x  &  3.5x \\
% Original  &  95 ms & 43  ms  & - & 11 ms & 106 ms & 54 ms & 7.8x &  15.4x \\
Claim detection  &  92 ms & 37 ms  & - & 11 ms & 103 ms & 48 ms & 8.1x & 17.3x \\
Citation Extraction  & 89  ms & 37 ms  & - & 9 ms & 98 ms & 46 ms & 8.5x & 18.1x \\
Fusion  & 89  ms & 37  ms  & - &  9 ms & 98 ms & 46  ms & 8.5x &  18.1x \\
\hline
\end{tabular}
\caption{Retrieval and inference latency for Index compression setup. Speedup is compared to the total latency of (S+R) pipeline with Full-Wiki setup.}
\label{tab:jpq_latency}
\vspace{-2em}
\end{table}

% %%%%%%%%%%%%%%%%%%%%%%%%%%%%%%%%%%%%%%%%%%%%%%%%%%%%%%%%%%%%%%%%%%%%%%%%%%%%%%%%%%%
\vspace{-2em}
\subsection{Effectiveness of index compression on enhancing the efficiency of dense retrieval and fact-checking systems?}

To answer \textbf{RQ3}, we make use of index compression to further improve upon Dense Retrieval setups, not only with respect to memory requirements but also enhancing total inference latency compared to the sparse retrieve + re-rank setups in classical pipelines.  The index sizes of Wikipedia collection for standard dense retrieval are 7.51 GiB for HoVer and 9.70 GiB for WiCE. Using the JPQ index compression model with M=96 subvectors, we significantly reduced the storage space for vector embeddings from 1.5 KiB to 104.12 B. This reduced the HoVer index size to 544.89 MiB and the WiCE index to \textbf{672.95 MiB}, achieving a \textbf{93\% reduction (14.4:1 compression ratio)}. Further reducing subvectors could decrease the index size but may impact performance.


The utilization of JPQ index compression leads to significant reductions in retrieval latency compared to dense Retrieval and sparse retrieval, as demonstrated in \autoref{tab:jpq_latency}. CPU retrieval gains notable speedups of approximately 10.0x for HoVer experiments and 7.0x for WiCE experiments, while GPU retrieval shows about 2.0x and 0.8x speedups, respectively. These improvements are attributed to learned compression in JPQ, enhancing computational efficiency. 
Significant improvements are also observed when examining the inference latency of the whole pipeline. The CPU-based approaches shows impressive speedups (upto \textbf{12.9x} for HoVer and \textbf{8.5x} for WiCE), and GPU-based approaches achieve even higher gains (\textbf{33.0x} for HoVer and \textbf{18.1x} for WiCE). 

Surprisingly, in our experiments we observe that JPQ yields better results than standard Dense Retrieval as shown in Figure \ref{fig:performance_plot}. This is particularly due to joint training of the query encoder and index compression. In addition, JPQ employs end-end negative sampling, which further improves retrieval performance despite significant compression of embeddings.

\mpara{Insight 3}: \textit{We find that index compression reduces index size by \textbf{93\%} resulting in speedups for CPU-based setups up to 10x and GPU-based setups more than 20x compared to classical fact-checking pipeline.}

\subsection{Discussion of Live Fact-checking results}
We employ the pruned index (2024 Wiki collection) using our Fusion approach followed by compression of the index for live Fact-checking of 2024 presidential debate. The pipeline comprises a dense retrieval using compressed index followed by claim verification. We use the query encoder and NLI models trained on HoVer for this application. We compare this approach to also the classical sparse-retrieval+re-rank fact-checking pipeline over the Full-Wiki collection (without pruning). The task performance is shown in Figure \ref{fig:livefc} and the corresponding pipeline efficiency is shown in Table \ref{tab:livefc}. We observe that the pruned collection coupled with retrieval using index compression leads to impressive speedups (\textbf{3.4x}) over classical pipeline over the full collection without significant drop in task performance (Figure \ref{fig:livefc}). The results demonstrate that efficient retrieval is critical for live fact-checking. Our experiments demonstrate that our approach for efficient retrieval provides significant speedups on CPUs further making the technology accessible even in low-resource scenarios which has significant implications in aiding detection of misinformation and disinformation at scale.
\begin{figure}
\centering
 \hspace{6em}     \begin{subfigure}{.8\textwidth}
        

\begin{tikzpicture}
\edef\mylst{"56.95","56.66","56.94",""}
\edef\explora{"55.92","57.82","52.93",""}

    \begin{axis}[
            ybar=14pt,
            width=6cm,
            bar width=0.35,
            every axis plot/.append style={fill},
            grid=major,
            xtick={1, 4, 8,9,11},
            xticklabels={Sparse + re-rank, Dense, IC},
            ylabel style = {font=\small},
        yticklabel style = {font=\boldmath \tiny,xshift=0.05ex},
        xticklabel style ={font=\tiny,yshift=0.5ex},
            ylabel={Performance (F1 weighted)},
            enlarge x limits=0.15,
            ymin=0,
            ymax=86,
            legend style ={font=\tiny,yshift=0.05ex},
            area legend,
            nodes near coords style={font=\tiny,align=center,text width=1em},
            legend entries={Full-Wiki, Fusion},
            legend cell align={left},
            legend pos=north west,
            legend columns=-1,
            legend style={/tikz/every even column/.append style={column sep=0.06cm}},
        ]
        \addplot+[
            ybar,
            plotColor1*,
            postaction={
                    pattern=north east lines
                },
                    nodes near coords=\pgfmathsetmacro{\mystring}{{\mylst}[\coordindex]}\textbf{\mystring},
        ] plot coordinates {
                (1,56.95)
                (4,56.66)
                (8,56.94)
            };
        \addplot+[
            ybar,
            plotColor2*,
            postaction={
                    pattern=north east lines
                },
            nodes near coords=\pgfmathsetmacro{\mystring}{{\explora}[\coordindex]}\textbf{\mystring},
        ] plot coordinates {
                (1,55.92)
                (4,57.82)
                (8,52.83)
                (9,0)
            };

    \end{axis}
\end{tikzpicture}


    \end{subfigure}
    \caption{Live fact-checking performance across different corpus setups}
    \label{fig:livefc}
\end{figure}
\vspace{-2em}
\begin{table}[htb!]
\centering
\footnotesize % Reduced font size
\setlength{\tabcolsep}{3pt} % Reduce space between columns
\renewcommand{\arraystretch}{0.9} % Reduce space between rows
\begin{tabular}{l c c c c c c}
\multirow{2}{*} & \multicolumn{2}{c}{\makecell{ Retrieval}}   & \multicolumn{2}{c}{Total Latency} & \multicolumn{2}{c}{Speedup} \\
\cline{2-3}\cline{4-5}\cline{5-7} \\[-1mm]
& CPU & GPU & CPU& GPU& CPU & GPU \\
\hline \\

% Term-based Document Retrieval
\colorg\textit{Sparse + Re-ranking} & \colorg & \colorg & \colorg  & \colorg & \colorg & \colorg\\ 
Full-Wiki & 463  & -   & 695  & - & \multicolumn{2}{c}{-} \\
Fusion & 274  & -   & 479   & - & \multicolumn{2}{c}{1.5x} \\
\hline \\
% Dense Retrieval setup
\colorg\textit{Dense Retrieval} & \colorg & \colorg & \colorg & \colorg & \colorg  & \colorg\\ 
Full-Wiki &  433   & 32  & 553   &  152  & 1.3x & 4.6x \\
Fusion & 412   & 32 & 511  & 131  & 1.4x & 5.3x \\
\hline \\
% Index Compression setup
\colorg\textit{Index Compression} & \colorg & \colorg & \colorg & \colorg & \colorg & \colorg\\  
Full-Wiki & 100  & 50   & 228  & 178  & 3.0x & 3.9x \\
\textbf{Fusion (ours)} & \textbf{89 }   & \textbf{43}  & \textbf{203}  & 157  & \speedup{3.4x} & 4.4x \\

\hline
\end{tabular}
\caption{Latency Comparisons for Live Fact-checking (in milliseconds (ms))}

\label{tab:livefc}
\end{table}


% \section{RQ 1: How does indexing supporting facts improve information retrieval efficiency?}
% In this section, we investigate the impact of indexing supporting facts on information retrieval efficiency by comparing the disk space utilization and retrieval latency across different experiment settings. Here we aim to discern the benefits of storing only supporting facts in the index as opposed to the entire corpus. 

% \subsection{Corpus Size}
% \begin{table}[htb!]
\small
\centering
\footnotesize
\begin{tabular}{c c c c c}
\toprule
\small
\textbf{Method} & \textbf{Disk Size} & \textbf{Size reduction} & \textbf{\#Sentences} & \textbf{\% decrease}\\
\hline \hline
\multicolumn{1}{l}{\colorg\textbf{HoVer}} & \colorg& \colorg & \colorg & \colorg  \\
Full-Wiki  &  11.28 GiB & - & 94,914,378 & -   \\
Fact Extraction & 6.19 GiB & \down{45}& 45,894,704 & \down{52} \\
Citation Extraction & \textbf{5.07 GiB} & \down{\textbf{55} } & \textbf{36,886,889} & \down{\textbf{61}} \\
Fusion & 5.45 GiB & \down{52} & 39,842,574 & \down{{58}} \\

\hline
\multicolumn{1}{l}{\colorg \textbf{WiCE}} & \colorg &  \colorg & \colorg & \colorg \\
Full-Wiki & 15.28 GiB & - &  126,533,841 & -  \\
Fact Extraction & 8.56 GiB & \down{44} & 61,040,380 & \down{52}\\
Citation Extraction & \textbf{6.56 GiB} & \down{\textbf{57}} & \textbf{51,735,961} & \down{\textbf{59}}  \\
Fusion & 6.85 GiB & \down{55}& 54,070,295 & \down{57} \\

\hline
\end{tabular}
\caption{Comparison sizes for the corpora per experiment setting, consisting of English Wikipedia articles 2017 (HoVer) and 2024 (WiCE). Reduction is measured compared to the  Full-Wiki data setting. \down{} denotes a reduction in corpus size and number of sentence compared to Full-Wiki setting.}
\label{fig:disk_size}
\end{table}
\vspace{-2em}
% To get an idea of how storing just the supporting facts data in the index improves efficiency compared to storing the entire corpus, a comparison can be made on how much these different settings occupy disk space. As mentioned in \autoref{sec:metrics}, to get an accurate estimate, only the dictionaries containing the article's title and document text are saved to raw JSON files. Across all experiment settings as seen in \autoref{fig:disk_size},  a notable reduction in disk space usage is observed compared to the original Wikipedia document corpus. This reduction ranges from approximately 45\% (claim detection) to 55\% depending on the setting for the HoVer corpus data. Likewise, for the WiCE corpus data, we can observe approximately 44\% to 57\% reduction. Moreover, in correlation with the reduced disk size, it is evident that the number of sentences stored in the index also decreases across each experiment setting compared to the original corpus data. For HoVer this ranges from 52\%  (claim detection) to 61\% (citation extraction) and WiCE ranges from 52\% to 59\%. This indicates that at least half of the sentences are considered as not claim-worthy across the different re-ranking methods.

% \subsection{Retrieval Latency}\label{ssec:retrieval_latency}
% \paragraph{Sparse retrieval} Following the reduction in disk size, a notable enhancement in retrieval latency is evident, as demonstrated in both the Term-based and Neural-based document retrieval columns of \autoref{tab:bm25_latency}. To avoid any ambiguity, it's crucial to clarify that the speedup listed in the table pertains to the total latency, which is relevant for addressing RQ2, rather than solely focusing on document retrieval.
% Regarding document retrieval latency (which encompasses both column values), there's an observed speedup ranging from approximately 1.5x (334 ms) to 1.6x (316 ms) compared to the original experimental setting for HoVer (495 ms). Similarly, in WiCE experiments, we witness a comparable speedup rate ranging from 1.4x (446 ms) to 1.6x (399 ms) compared to the original experimental setting (636 ms). This observation suggests that while the reduced text size contributes to expedited retrieval, the enhancement is only somewhat proportional.

% \begin{table}[htb!]
\centering
\small
\footnotesize
% \vspace{-1cm}
\begin{tabular}{l c c c c}
\multirow{2}{*}{\makecell{\textbf{Methods}}} & \multirow{2}{*}{\textbf{Retrieval}} & \multirow{2}{*}{\makecell{\textbf{Total Latency}}} & \multirow{2}{*}{\makecell{\textbf{Speedup}}} \\
& \\
\hline
\multicolumn{1}{l}{\colorg\textit{HoVer}} & \colorg & \colorg & \colorg \\
 Full-Wiki & 426  ms & 659 ms & - \\
Fact Extraction & 257 ms  & 338  ms & 1.9x \\
Citation Extraction & 246 ms  &  327  ms & \speedup{2.0x}  \\
Fusion & 265 ms  & 345  ms & 1.9x  \\

\multicolumn{1}{l}{\colorg\textit{WiCE}} & \colorg & \colorg & \colorg \\
  Full-Wiki &  559 ms  &  831  ms & - \\
Fact Extraction & 372 ms  & 468   ms & 1.8x \\
Citation Extraction & 330 ms   & 419  ms & \speedup{2.0x} \\
Fusion & 347 ms & 436  ms & 1.9x \\
\hline
\end{tabular}
\caption{Retrieval and total latency for Sparse retrieval with Re-ranking. Speedup is compared with respect to the total latency of the Full-Wiki setup.}
\label{tab:bm25_latency}
\vspace{-2em}
\end{table}
% \paragraph{CPU-based Dense Retrieval} One might typically anticipate a more pronounced disparity between the original data and the reranked data in the document retrieval phase. However when transitioning from the Sparse retrieval setup to the Dense retrieval setup, as depicted in the first column of \autoref{tab:faiss_latency}, only negligible differences between the different settings are observed. This is attributed to FAISS utilizing vectors instead of computing the relevance ranking of documents to the query, as is the case with BM25. Despite variations in the length of each article across settings, the number of text embeddings (with fixed dimensionality size) created remains constant, corresponding to the number of encoded text spans, which is consistent across settings. Thus minimizing the impact of extracting supporting facts on document retrieval latency when using Dense Retrieval. Comparing the Dense document retrieval (CPU) column in \autoref{tab:faiss_latency} to the baselines listed in \autoref{tab:bm25_latency}, it is observed to be of a similar latency or even slightly slower. For HoVer, we can observe a 0.9x (523 ms) to 1.0x (479 ms) compared to the baseline (495 ms). Likewise, for WiCE we can observe a similar latency speedup of 0.9x (685 ms) to 1.0x (610 ms) speedup compared to its baseline (636 ms). This suggests that the indexing of supporting facts would not significantly impact information retrieval efficiency in such scenarios. 

% \begin{table}[htpb!]
\centering
\footnotesize
\begin{tabular}{l c c c c c c c c}
\multirow{2}{*}{} & \multicolumn{2}{c}{\makecell{Document retrieval}} & \multirow{2}{*}{\makecell{Sentence \\ Retrieval}} & \multirow{2}{*}{\makecell{Claim \\ Verification}} & \multicolumn{2}{c}{Total Latency} & \multicolumn{2}{c}{Speedup} \\
\cline{2-3}\cline{6-7}\cline{8-9}
& CPU & GPU &  & &  CPU & GPU & CPU & GPU \\ 
\hline
\multicolumn{1}{l}{\textit{HoVer}} &  \\
 \textbf{Full-Wiki (S+R) } &  \multicolumn{2}{c}{\textbf{491  ms}} & \textbf{157  ms} & \textbf{7 ms} &  \multicolumn{2}{c}{\textbf{659 ms}} & - & - \\

Full-Wiki & 515 ms & 31 ms & 153 ms & 8 ms & 676 ms & 192  ms & 1.0x & 3.4x \\
% Original  &  523 ms & 30 ms  & - & 9 ms & 532 ms & 39 ms & 1.2x & 16.9x \\
Claim detection & 513 ms & 23 ms & - & 8 ms & 521 ms & 31 ms & 1.3x & 21.3x  \\
Citation Extraction & 479 ms & 23 ms & - & 9 ms &  488 ms & 32 ms & 1.4x & 20.6x \\
Fusion & 500 ms  & 23 ms & - & 9 ms & 509 ms & 32  ms & 1.3x & 20.6x \\

\multicolumn{1}{l}{\textit{WiCE}} & \\
 \textbf{Full-Wiki (S+R)} & \multicolumn{2}{c}{\textbf{636 ms}} & \textbf{186 ms} & \textbf{9  ms}  & \multicolumn{2}{c}{\textbf{831  ms}} & - & - \\
Full-Wiki & 685 ms & 34 ms & 184 ms & 9 ms & 878 ms & 227  ms & 1.0x & 3.7x \\
% Original  & 610  ms & 34 ms  & - & 9 ms & 619  ms & 43 ms & 1.3x & 19.3x \\
Claim detection &  622 ms & 31 ms  & - & 9 ms & 631 ms & 40 ms & 1.3x & 20.8x \\
Citation Extraction &  610 ms & 31 ms  & - & 9 ms & 619 ms & 40 ms & 1.3x & 20.8x  \\
Fusion & 619  ms & 31 ms  & - & 9 ms & 628 ms & 40 ms & 1.3x & 20.8x  \\[5mm]
\hline
\end{tabular}
\caption{Retrieval and inference latency for Dense retrieval setup on data settings. Speedup is compared with respect to the total latency of the Sparse Retrieval setup with original data setting (bold font).}
\label{tab:faiss_latency}
\end{table}




% \paragraph{GPU-based Dense Retrieval} However, it is worth noting that Dense retrieval can still be faster, particularly with dense retrieval libraries such as FAISS offering GPU support, which can yield substantial speedups compared to both CPU retrieval of BM25 and FAISS. This advantage is evident in the data, showcasing notable speedups ranging from 16.6x to 22.3x speedup for HoVer GPU retrieval over CPU retrieval, and 17.9x to 20.2x speedup for WiCE. Furthermore, when comparing FAISS GPU retrieval to the BM25 retrieval, we can see an approximate 16.0x (31 ms) to 21.5x (23 ms) speedup for HoVer and 18.7x (34 ms) to 20.5x (31 ms) speedup for WiCE. Therefore the GPU-based approach makes Dense Retrieval a viable option, unlike the CPU-based variant. 

% \subsection{Key Takeaways} 
% Extracting supporting facts from the data corpus can lead to only requiring to store at least half of the data. Although this has a positive effect on the latency for Sparse retrieval, with Dense document retrieval this is not the case due to how the vector embeddings are constructed (being per article rather than per sentence). Furthermore, while CPU-based Dense retrieval may not necessarily outperform Sparse retrieval methods in terms of latency, thereby presenting less immediate appeal, the incorporation of GPU support leads to significant speed enhancements. Thus, although extracting supporting facts does not help much in Dense document retrieval unlike Sparse retrieval in terms of retrieval latency, the incorporation of the GPU-based Dense retrieval renders it a much more compelling option for achieving efficiency. 

% %%%%%%%%%%%%%%%%%%%%%%%%%%%%%%%%%%%%%%%%%%%%%%%%%%%%%%%%%%%%%%%%%%%%%%%%%%%%%%%%%%%

% \section{RQ 2: How does indexing supporting facts affect overall pipeline efficiency and downstream fact-checking performance?}
% In continuation of the previous research inquiry concerning retrieval latency and disk size, this section delves into an analysis of the overall inference time across the entire pipeline. Additionally, recognizing that faster processing times do not necessarily equate to better performance a further analysis will be done on the performance metrics.

% \subsection{Inference Latency}
% \paragraph{Sparse Retrieval Setup:} The enhancement in retrieval latency, as evidenced in \autoref{tab:bm25_latency}, mirrors a noticeable improvement in the overall inference latency across the pipeline. This improvement spans approximately 1.9x to 2.0x for the HoVer experiments and 1.8x to 2.0x for WiCE experiments. However, the reduction in total latency cannot be solely ascribed to faster retrieval times. It also arises from the elimination of the Sentence Retrieval stage, which previously imposed significant latency overhead. Upon closer inspection of \autoref{tab:bm25_latency}, it becomes apparent that the absence of the Sentence Retrieval stage impacts the Claim Verification stage. Notably, experiments conducted on the original corpus data exhibit much lower inference latency compared to supporting facts data. Nevertheless, the variance between these experiment settings is minimal, and the impact on total latency results is insignificant. This overall trend indicates that indexing supporting facts for the BM25 retrieval setup predominantly benefits inference times for the Rule-based document retrieval and Sentence Retrieval stages. Furthermore, it reveals that the Claim Verification stage is slightly, yet negligibly, affected when considering the entire pipeline inference.

% \paragraph{Dense Retrieval Setup:} In a similar vein as the document retrieval comparisons of RQ1 (see \autoref{ssec:retrieval_latency}), the total inference of the Dense retrieval setup presents notable differences in results between CPU- and GPU-based Dense retrieval compared to Sparse retrieval. This divergence is evident in \autoref{tab:faiss_latency}, where for HoVer experiments, the CPU-based approach exhibits a 1.2x to 1.4x speedup, while the GPU-based approach demonstrates a 16.9x to 21.3x speedup compared to the baseline. Similarly, WiCE experiments show approximately a 1.3x speedup for the CPU-based approach and 19.3x to 20.8x speedup for the GPU-based approach. The key distinction lies in the influence of omitting the Sentence Retrieval stage for the original corpus data. Its omission introduces significant overhead to the total latency. For the CPU-based approach, this translates to a 1.3x speedup for HoVer (676 ms vs. 532 ms) and a 1.4x speedup for WiCE (878 ms vs. 619 ms). Conversely, the GPU-based approach experiences a 4.9x speedup for HoVer (192 ms vs. 39 ms) and a 5.3x speedup for WiCE (227 ms vs. 43 ms). Overall, this underscores that including Sentence Retrieval adds substantial overhead, especially for GPU-based approaches operating with lower latency magnitudes. Therefore, the supporting facts data for Dense Retrieval, while not significantly impacting document retrieval, offers significant speedup for total inference latency, allowing for the effective omission of the Sentence Retrieval stage and its associated latency overhead.

% \subsection{Performance Metrics Evaluation}

% \begin{table}[htb!]
\small
\begin{tabular}{c c c c c c c c}
\multirow{2}{*}{Experiment setting} & \multirow{2}{*}{Accuracy} & \multicolumn{2}{c}{F1} & \multicolumn{2}{c}{Precision} & \multicolumn{2}{c}{Recall}  \\ 
\cline{3-8}
  & &  Weighted  & Macro & Weighted & Macro & Weighted & Macro      \\
\hline
\multicolumn{1}{l}{\textit{Sparse + Re-ranking}} & & & & \\
Full-Wiki & \textbf{67.79} & \textbf{67.59} & \textbf{67.63} & \textbf{68.45} & \textbf{68.39} & \textbf{67.79}  & \textbf{67.93}\\
Claim detection & \underline{62.33} & 62.02 & 62.08 & \underline{62.98} & \underline{62.92} & \underline{62.33} & \underline{62.50} \\
Citation Extraction & 60.91 & 60.61 & 60.66 & 61.47 & 61.42 & 60.91 & 61.07 \\
Fusion & 62.28 & \underline{62.15} & \underline{62.18} & 62.60 & 62.56 & 62.28 & 62.39  \\[5mm]

\hline
\multicolumn{1}{l}{\textit{Dense Retrieval}} & & & & \\
Full-Wiki & 64.60 & 64.45 & 64.45 & 64.86 & 64.86 & 64.60 & 64.60 \\
% Original & 62.90 & 62.72 & 62.76 & 63.33 & 63.28 & 62.90 & 63.02 \\
Claim detection & \underline{61.50} & \underline{60.94} & \underline{60.94} & \underline{62.20} & \underline{62.20} & \underline{61.50} & \underline{61.50} \\
Citation Extraction & 59.67 & 59.40 & 59.46 & 60.13 & 60.09 & 59.67 & 59.82 \\
Fusion & 59.51 & 59.32 & 59.37 & 59.85 & 59.81 & 59.51 & 59.64  \\[5mm]

\hline
\multicolumn{1}{l}{\textit{Index Compression}} & & & & & & &  \\
Full-Wiki & 63.30 & 62.54 & 62.54 & 64.48 & 64.48 & 63.30 & 63.30 \\
% Original & 63.02 & 62.08 & 62.08 & 64.46 & 64.46 & 63.02 & 63.02  \\
Claim detection & \underline{61.92} & \underline{61.71} & \underline{61.71} & \underline{62.19} & \underline{62.19} & \underline{61.92} & \underline{61.93}  \\
Citation Extraction & 59.98 & 59.12 & 59.12 & 60.89 & 60.89 & 59.98 & 59.98   \\
Fusion & 61.58 & 61.43 & 61.43 & 61.75 & 61.75 & 61.58 & 61.58   \\[5mm]

\hline
\end{tabular}
\caption{Performance experiments on HoVer data and adjustments using full document text of English Wikipedia. The underlined-styled values represent the second best  within each retrieval setup.}
\label{tab:hover_performance_metrics}
\end{table}
% Full document text 
% 

% \begin{table}[htb!]
\centering
\footnotesize
\begin{tabular}{c c c c c c c c}
\multirow{2}{*}{Experiment setting} & \multirow{2}{*}{Accuracy} & \multicolumn{2}{c}{F1} & \multicolumn{2}{c}{Precision} & \multicolumn{2}{c}{Recall}  \\ 
\cline{3-8}
  & &  Weighted  & Macro & Weighted & Macro & Weighted & Macro      \\
\hline
\multicolumn{1}{l}{\textit{Sparse + Re-ranking}} & & & & \\
Full-Wiki & \textbf{63.69} &  \textbf{61.84} &  \textbf{55.24} &  \textbf{61.12} &  \textbf{56.54} &  \textbf{63.69} &  \textbf{55.32 } \\
Claim detection & 61.90 & 60.12 & \underline{53.33} & 59.27 & 54.26 & 61.90 & \underline{53.53} \\
Citation Extraction & 61.01 & 59.56 & 52.96 & 58.75 & 53.59 & 61.01 & 53.09 \\
Fusion & \underline{63.39} & \underline{60.21} & 52.48 & \underline{59.46} & \underline{54.69} & \underline{63.39} & 53.27  \\[5mm]

\hline
\multicolumn{1}{l}{\textit{Dense Retrieval}} & & & & \\
Full-Wiki &  61.61 & 60.95 & 55.21 & 60.47 & 55.49 & 61.61 & 55.13 \\
% Full-Wiki  & 60.42 & 58.80 & 51.96 & 57.90 & 52.58 & 60.42 & 52.19 \\
Claim detection & 61.01 & 58.94 & 51.78 & 57.96 & 52.70 & 61.01 & 52.17 \\
Citation Extraction & 58.63 & 58.48 & \underline{52.92} & 58.35 & 52.95 & 58.63 & \underline{52.91} \\
Fusion & \underline{61.31} & \underline{59.34} & 52.30 & 58.40 & \underline{53.23} & \underline{61.31} & 52.62  \\[5mm]

\hline
\multicolumn{1}{l}{\textit{Index Compression}} & & & & & & &  \\
Full-Wiki & 62.46 & 61.38 & 55.27 & 60.74 & 55.84 & 62.46 & 55.20  \\
% Original  & 60.46 & 60.63 & 55.64 & 60.81 & 55.60 & 60.46 & 55.70  \\
Claim detection & 59.31 & 59.02 & \underline{53.32} & 58.77 & 53.39 & 59.31 & \underline{53.30}  \\
Citation Extraction & 60.74 & 59.21 & 52.42 & 58.34 & 53.04 & 60.74 & 52.60  \\
Fusion & \underline{63.04} & \underline{59.79} & 51.89 & 58.94 & \underline{53.97} & \underline{63.04} & 52.76 \\[5mm]

\hline
\end{tabular}
\caption{Performance experiments on WiCE data and adjustments using full document text of English Wikipedia. The bold-styled values represent the baseline while the underlined-styled values represent the highest scores of the re-ranked data within a retrieval setup category.}
\label{tab:wice_performance_metrics}
\end{table}
% Full document text 
% 

% \paragraph{Sparse Retrieval performance} Utilising the metrics laid out in \autoref{sec:metrics}, the pipeline results have been evaluated for the different settings and laid out in \autoref{tab:hover_performance_metrics} for the HoVer experiments and \autoref{tab:wice_performance_metrics} for WiCE experiments. When comparing the different HoVer experiment settings within the Sparse Retrieval setup, Claim detection comes the closest to the baseline with close to 5.5 points difference across the metrics for the HoVer experiments. Important to note is that Fusion follows close with less than a point difference. For the WiCE Sparse retrieval setup, the opposite occurred with the Fusion data being the closest with a marginal 0.3 point difference followed by Claim detection with a 1.5 points difference. In both datasets, the Citation extraction takes the biggest loss in accuracy that being 6.9 points for HoVer and 2.7 points difference for WiCE. We can reason the fact that citation extraction takes the biggest performance degradation to the fact that not all claim-worthy sentences contain citations, therefore missing out on crucial evidence sentences. Unlike the other settings which consider the complete text instead of only the cited sentences and determine claim-worthiness on what the claim-detection model selects. Overall, relating to the inference time, we can see that for HoVer with a speedup of at least 1.5x to 1.6x, we only lose 6.9 to 5.5 points in performance across various metrics for the best re-ranking setup. Likewise, for WiCE, with a speedup of 1.4x to 1.6x we only lose 2.7 to 0.3 points. This positively demonstrates that indexing just the supporting facts does show meaningful results in terms of overall pipeline efficiency, while maintaining roughly the same performance. Additionally, this also indicates we can achieve good results by using a combination of citation extraction together with another supporting facts extraction method such as Claim detection.

% \paragraph{Dense Retrieval performance} When examining the performance results of Dense retrieval compared to Sparse Retrieval, it becomes evident that there is a slight decline across all experiments. For HoVer, this decline ranges from a modest 0.8 point difference in Claim detection to a more substantial 2.9 points in Fusion data. Similarly, WiCE experiences a loss ranging from a 0.9 difference in accuracy between Claim detection settings to approximately 2.4 points in Citation extraction. Crucially, it is to assess how these performances compare against the baselines. In HoVer, the accuracy loss ranges from 8.3 points for Fusion data to 6.3 points for Claim detection. WiCE experiences a loss ranging from 5.1 points in Citation extraction to 2.4 points in Fusion. These findings suggest that while transitioning from Sparse retrieval to just a Dense retrieval component incurs some loss, it's not substantial across various experiments involving supporting facts data. Moreover, the performance is notably strong in claim detection, while citation extraction lags behind by only a few points. Interestingly, while Fusion performs as well as Citation extraction in HoVer experiments, Fusion data outperforms Claim detection in WiCE. This highlights the significance of combining citation extraction with another supporting facts extraction method to achieve optimal results, similar to the Sparse retrieval setup.

% \paragraph{Sentence Retrieval stage ablation} Comparing experiments on the original data between the two retrieval methods reveals a more significant decline for HoVer, with a loss of 3.2 points with Sentence Selection and 4.9 points without it. For WiCE, the difference is 2.1 points with Sentence Selection and 3.3 points without it. When assessing these losses against the baselines, it becomes evident that both methods generally outperform the supporting facts data experiments by a few points. This suggests that the contribution of the Sentence Retrieval stage in the pipeline to performance improvement is marginal. With the supporting facts extraction thus becomes quite effective in achieving nearly the same performance. Consequently, to enhance efficiency, eliminating this Sentence Retrieval stage would result in only a loss of less than a few points.

% \subsection{Key Takeaways} 
% Incorporating supporting facts into both Sparse and Dense retrieval setups yields noteworthy enhancements in overall pipeline efficiency. Sparse retrieval setups demonstrate speedups ranging from up to around 1.5x, while Dense retrieval setups exhibit even more substantial improvements, achieving up to approximately 20.0x with GPU-based approaches. These notable speedups are primarily attributed to the removal of the Sentence Retrieval stage, which incurs considerable latency overhead. Further evaluation indicates a minor decline in performance when transitioning from Sparse to Dense retrieval, though the loss is not substantial. Specifically, claim detection remains robust, while citation extraction may lag behind by a few points. However, Fusion data yields promising results, often comparable to or outperforming other extraction methods, emphasizing the significance of amalgamating various extraction techniques for supporting facts. Moreover, ablation experiments on the Sentence Retrieval stage reveal its marginal contribution to performance improvement. Comparisons between original data and supporting facts data show only a slight decline in performance, showcasing that utilising only the supporting facts only incurs a modest loss in performance (around 6 points for HoVer and 3 points for WiCE). This suggests that although supporting facts do not affect document retrieval latency in the Dense Retrieval setup, it does help with overall pipeline latency due to avoiding the latency overhead of Sentence Selection. In conclusion, these results underscore the meaningful impact of indexing supporting facts on the overall pipeline efficiency, with only minimal losses in downstream fact-checking performance.

% %%%%%%%%%%%%%%%%%%%%%%%%%%%%%%%%%%%%%%%%%%%%%%%%%%%%%%%%%%%%%%%%%%%%%%%%%%%%%%%%%%%

% \section{RQ 3: In what ways does index compression enhance the efficiency of dense retrieval and fact-checking systems?}

% In this final research inquiry concerning the addition of index compression, this section explores how index compression improves upon Dense Retrieval in not only the constructed index size, but also document retrieval and total inference latency. Additionally, a final comparison will be made on the overall performance against Sparse retrieval and standard Dense Retrieval.

% \subsection{Compressed Index Size}
% In our FAISS experiments, we consistently observe an index size of approximately 7.51 GiB across all HoVer settings and 9.70 GiB across all WiCE settings. While one might anticipate that re-ranking would influence the amount of text utilized for generating vector embeddings, it's crucial to note that the index size remains unchanged. This is due to the fact that we generate vector embeddings on a per-article basis with only the text itself being altered. To address this issue, we employed JPQ, an index compression model. Despite using a relatively high number of subvectors for the JPQ model (M=96), we observed a significant reduction in the total index size. Specifically, the individual vector embeddings now occupy only 104.12 B in storage space, down from 1.5 KiB previously. This reduction is remarkable. For the HoVer experiments, the index size decreased from 7.51 GiB to 544.89 MiB, and for the WiCE experiments, we observed a decrease from 9.70 GiB to 672.95 MiB. Overall, this constitutes an impressive reduction of nearly 93\% or a compression ratio of 14.4:1 in index size for both experiment setups. It's worth noting that employing fewer sub-vectors could potentially lead to an even more substantial reduction in index size; however, this would come at the cost of decreased performance.

% \begin{table}[htb!]
\centering
\footnotesize
\begin{tabular}{l c c c c c c c c}
\hline
\multirow{2}{*}{Method} & \multicolumn{2}{c}{\makecell{Term-based \\ document retrieval}} & \multirow{2}{*}{\makecell{Sentence \\ Retrieval}} & \multirow{2}{*}{\makecell{Claim \\ Verification}} & \multicolumn{2}{c}{Total Latency}  & \multicolumn{2}{c}{Speedup} \\
\cline{2-3}\cline{6-7}\cline{8-9}
& CPU & GPU &  & &  CPU & GPU & CPU & GPU \\ 
\hline \hline
 \multicolumn{1}{l}{\colorg\textit{HoVer}} & \colorg & \colorg & \colorg & \colorg & \colorg & \colorg & \colorg & \colorg \\
 \textbf{Full-Wiki (S+R)} &  \multicolumn{2}{c}{\textbf{491  ms}} & \textbf{157  ms} & \textbf{7 ms} &  \multicolumn{2}{c}{\textbf{659 ms}} & - & - \\
Full-Wiki &  53 ms & 13 ms & 153 ms & 8 ms & 214  ms & 174 ms & 3.1x &  3.8x \\
% Original  &  55 ms & 13 ms  & - & 12 ms & 67 ms & 25 ms & 9.8x &  26.4x \\
Claim detection  &  51 ms & 12  ms  & - &  9 ms & 60 ms & 21 ms & 11.0x & 31.4x  \\
Citation Extraction  & 46  ms & 11 ms  & - & 9 ms & 51 ms & 20 ms & 12.9x & 33.0x \\
Fusion  & 51  ms & 12 ms  & - & 12 ms & 63 ms & 24 ms &  10.5x & 27.5x \\
\hline
\multicolumn{1}{l}{\colorg\textit{WiCE}} & \colorg & \colorg & \colorg & \colorg & \colorg & \colorg & \colorg & \colorg  \\
 \textbf{Full-Wiki (S+R)} & \multicolumn{2}{c}{\textbf{636 ms}} & \textbf{186 ms} & \textbf{9  ms}  & \multicolumn{2}{c}{\textbf{831  ms}} & - & - \\
Full-Wiki & 97 ms & 43 ms & 186 ms & 9 ms & 292 ms & 238 ms & 2.8x  &  3.5x \\
% Original  &  95 ms & 43  ms  & - & 11 ms & 106 ms & 54 ms & 7.8x &  15.4x \\
Claim detection  &  92 ms & 37 ms  & - & 11 ms & 103 ms & 48 ms & 8.1x & 17.3x \\
Citation Extraction  & 89  ms & 37 ms  & - & 9 ms & 98 ms & 46 ms & 8.5x & 18.1x \\
Fusion  & 89  ms & 37  ms  & - &  9 ms & 98 ms & 46  ms & 8.5x &  18.1x \\
\hline
\end{tabular}
\caption{Retrieval and inference latency for Index compression setup. Speedup is compared to the total latency of (S+R) pipeline with Full-Wiki setup.}
\label{tab:jpq_latency}
\vspace{-2em}
\end{table}



% \subsection{Pipeline Efficiency}
%  \paragraph{Document Retrieval Latency} When examining the retrieval latency outlined in \autoref{tab:jpq_latency}, a notable observation can be made towards the Dense document retrieval compared to the Dense Retrieval results outlined in \autoref{tab:faiss_latency}. This significant enhancement can be primarily attributed to the utilization of the index compression model, which effectively reduces the index size. As a result, retrieval latency experiences a considerable improvement due to the smaller vector embeddings, facilitating faster similarity computation. Here one can observe a substantial speedup achieved in CPU retrieval of approximately 10.0x across the HoVer experiment settings and 7.0x for WiCE experiments. Similarly, GPU retrieval exhibits a speedup of approximately 2.0x for HoVer experiments and 0.8x for WiCE experiments. This is generally in line with the reported results in the original JPQ paper \cite{zhan2021jointly}. Although the measurements for HoVer fall in line with these reported results, one may notice that the WiCE retrieval speedup is lower than that of HoVer. This is even worse for the GPU-based retrieval latency instead of being better than the standard GPU-based Dense retrieval. We reason this to the fact that the WiCE claim dataset is a lot more complex. In the original WiCE paper, the results that were reported already indicate a not so particularly high performance being achieved. This coupled with the use of a different model for creating the embeddings results in marginally worse performance instead of a speedup such as the case with HoVer. 
 
%  \paragraph{Pipeline Inference Latency} In examining the total inference latency, as further detailed in \autoref{tab:jpq_latency}, the utilization of compressed indexing and the ensuing document retrieval speed enhancements result in a notable boost across the board. The advancements brought about by JPQ, which further build upon the foundations of Dense Retrieval, are particularly significant. Notably, CPU latency has seen a substantial improvement compared to previous benchmarks on the supporting fact data, exhibiting a noteworthy speedup ranging from 10.5x to 12.9x for HoVer experiments, and 8.1x to 8.5x for WiCE experiments relative to their respective baselines. Meanwhile, the GPU-based approach, especially in the case of HoVer experiments, has yielded even more impressive results, achieving speedups ranging from 27.5x to 33.0x. While WiCE experiments on the GPU may not experience such dramatic speedups, they still showcase marked enhancements over their original baselines that range from 17.3x to 18.1x speedups. When assessing the impact of the Sentence Selection stage on the original corpus data settings, the findings reinforce the observations made in the standard Dense Retrieval setup. Furthermore, the disparity in the reported speedups between the tables underscores the significance of incorporating index compression. 
  
% \subsection{Performance Metrics Evaluation}
% When comparing the performance of JPQ in the HoVer experiments (as shown in \autoref{tab:hover_performance_metrics}) as well as the performance of the WiCE experiments (presented in \autoref{tab:wice_performance_metrics}), a notable trend emerges. The index compression brought by JPQ generally yields higher scores compared to the standard Dense retrieval experiments. This improvement is particularly striking as the gap between the JPQ experiments and the baseline performances is further narrowed. In the HoVer experiments, this enhancement ranges from marginal increases of less than a point in Claim detection and Citation extraction to a significant 2-point boost in the Fusion data. Conversely, in the WiCE experiments, while Claim detection experiences a slight decline of almost 2 points, Citation extraction and Fusion demonstrate the opposite trend.
% Typically, one might expect index compression techniques to yield inferior results compared to the standard Dense retrieval setup due to the lossy nature of compressing embeddings. However, a straightforward explanation for this unexpected improvement lies in the utilization of different pre-trained models for generating the embeddings. In the standard Dense retrieval, we rely on the all-MiniLM-L6-v2 model, which maps sentences and paragraphs to a 384-dimensional dense vector space. In contrast, the JPQ model employed for index compression initially generates embeddings of size 768 and subsequently reduces the embedding size using PQ centroids to achieve smaller vector sizes. Furthermore, it's worth noting that JPQ learns the index for the query vectors, unlike the approach in standard Dense retrieval where the index is kept separate. The latter essentially operates in a zero-shot inference manner, as we do not fine-tune the encoders on specific datasets but instead store and retrieve the created embeddings directly in our FAISS setup.


% \subsection{Key Takeaways} 
% Enhancing Dense retrieval through the use of index compression via the JPQ model has remarkably reduced the index size for Dense retrieval by a substantial 93\%. Further analysis indicates significant speedups of up to 10.0x for the CPU-based approach, while the GPU-based approach achieves a modest speedup of up to 2.0x in the HoVer experiments. However, it experiences a slight slowdown in the WiCE experiments. A huge emphasis on achieving efficiency is particularly pertinent in the context of CPU-based Dense Retrieval with index compression. Here the latency times of the CPU-based approach come in close to the GPU-based approach. These findings not only signify efficiency gains concerning resource utilization for index storage, but also pave the way for experiments on lower-end machines especially ones lacking GPU capabilities. Thereby maximizing the benefits of CPU-based methodologies. Regarding performance, experiments involving index compression generally outperform standard Dense retrieval. This superiority can be attributed to the utilization of different pre-trained models and learned index techniques, resulting in slightly enhanced outcomes.

\section{Conclusion}\label{sec:con}

Our work contributes empirical insights on the photorealism of AI-generated images and a taxonomy of artifacts commonly found in AI-generated images, organized into five categories: anatomical implausibilities, stylistic artifacts, functional implausibilities, violations of physics, and sociocultural implausibilities. We find that the photorealism of AI-generated images depends on the scene complexity of the image, the kind of artifacts and implausibilities, if any, detectable in an image, the duration of visual attention to an image, and the quality of human effort to select appropriate prompts and curate images. A question such as ``How photorealistic are state-of-the-art diffusion models'' may sound simple, but the answer is more complex and depends on many details, including what images are generated and selected, how photorealism is measured, what real images are included in the experiment, and how much time, skill, and effort a human participant has and willing to offer. This paper offers an initial exploration into how we can address this question and develops a practical taxonomy that offers scaffolding for building AI--literacy interventions to help people navigate the capabilities and limitations of diffusion models and whether an image is AI-generated or not. 

\begin{acks}

This material is based upon work supported by Robert Pozen, and in part with funding from the Department of Defense (DoD). Any opinions, findings, conclusions, or recommendations expressed in this material are those of the authors and do not necessarily reflect the views of the DoD or any agency or entity of the United States Government. We thank Will Thompson from Kellogg Research Support for performing a replication check.
\end{acks}

% \clearpage
%%
%% The acknowledgments section is defined using the "acks" environment
%% (and NOT an unnumbered section). This ensures the proper
%% identification of the section in the article metadata, and the
%% consistent spelling of the heading.
\begin{acks}
The challenge is organized as a joint effort by the University College London, Microsoft, the University of Amsterdam, the University of Waterloo, and the University of Padua. The views expressed in the content are solely those of the authors and do not necessarily reflect the views or endorsements of their employers and/or sponsors. This research is supported by the Engineering and Physical Sciences Research Council [EP/S021566/1], CAMEO, PRIN 2022 n.~2022ZLL7MW and by the Dreams Lab, a collaboration between Huawei Finland, the University of Amsterdam, and the Vrije Universiteit Amsterdam.
\end{acks}

% \begin{table}[ht]
    \centering
    \caption{The number of labels assigned by human judges and LLMJudge challenge submissions to each judgment level. Bold indicates the closest prediction to the number of labels assigned by humans.}
    \adjustbox{max width=\columnwidth}{%
    \begin{tabular}{lcccc}
    \toprule
    \textbf{Submission ID} & \textbf{0} & \textbf{1} & \textbf{2} & \textbf{3} \\
    \midrule
    human & 2005 & 1233 & 808 & 377 \\
    \midrule
    NISTRetrieval-instruct0 & 1115 & 2092 & 1216 & 0 \\
    NISTRetrieval-instruct1 & 1115 & 2092 & 1216 & 0 \\
    NISTRetrieval-instruct2 & 1117 & 2088 & 1218 & 0 \\
    NISTRetrieval-reason0 & 1159 & 1922 & 1340 & 2 \\
    NISTRetrieval-reason1 & 1158 & 1924 & 1339 & 2 \\
    NISTRetrieval-reason2 & 1159 & 1921 & 1341 & 2 \\
    Olz-exp & 2435 & 1210 & 456 & 322 \\
    Olz-gpt4o & 2258 & 1274 & 504 & \textbf{387} \\
    Olz-halfbin & 2100 & 1458 & 277 & 588 \\
    Olz-multiprompt & 1713 & 976 & 1244 & 490 \\
    Olz-somebin & 2102 & 595 & 1004 & 722 \\
    RMITIR-GPT4o & 3056 & 349 & 730 & 288 \\
    RMITIR-llama38b & 2576 & 614 & 1058 & 175 \\
    RMITIR-llama70B & 2154 & 243 & 1581 & 443 \\
    TREMA-4prompts & 1027 & 751 & 2213 & 432 \\
    TREMA-CoT & 1835 & 1122 & 906 & 560 \\
    TREMA-all & 2399 & 616 & 734 & 674 \\
    TREMA-direct & 2404 & 87 & 342 & 1590 \\
    TREMA-naiveBdecompose & 2627 & 645 & 1117 & 34 \\
    TREMA-nuggets & 2127 & 893 & 1044 & 359 \\
    TREMA-other & 1305 & 809 & 2021 & 288 \\
    TREMA-questions & 2450 & 253 & \textbf{767} & 953 \\
    TREMA-rubric0 & 3122 & 1211 & 0 & 90 \\
    TREMA-sumdecompose & 2518 & 254 & 1049 & 602 \\
    h2oloo-fewself & 2470 & 732 & 557 & 664 \\
    h2oloo-zeroshot1 & 2353 & 1225 & 597 & 248 \\
    h2oloo-zeroshot2 & 2920 & 771 & 476 & 255 \\
    llmjudge-cot1 & 991 & 1921 & 1383 & 128 \\
    llmjudge-cot2 & 1111 & 955 & 2149 & 208 \\
    llmjudge-cot3 & 1902 & 1321 & 486 & 714 \\
    llmjudge-simple1 & 777 & 2228 & 1217 & 200 \\
    llmjudge-simple2 & 1720 & 905 & 1375 & 423 \\
    llmjudge-simple3 & 1834 & 1318 & 485 & 786 \\
    llmjudge-thomas1 & 1049 & 1803 & 1402 & 169 \\
    llmjudge-thomas2 & 1886 & 733 & 1585 & 219 \\
    llmjudge-thomas3 & \textbf{1934} & 1184 & 559 & 746 \\
    prophet-setting1 & 2506 & 885 & 629 & 403 \\
    prophet-setting2 & 2903 & 852 & 651 & 17 \\
    prophet-setting4 & 3359 & 763 & 281 & 20 \\
    willia-umbrela1 & 2335 & \textbf{1231} & 608 & 249 \\
    willia-umbrela2 & 2705 & 1029 & 350 & 339 \\
    willia-umbrela3 & 2710 & 1041 & 446 & 226 \\
    \bottomrule
    \end{tabular}
    }
    \label{tab:labels}
\end{table}

%%
%% The next two lines define the bibliography style to be used, and
%% the bibliography file.
\bibliographystyle{ACM-Reference-Format}
\bibliography{references}

\clearpage

%%
%% If your work has an appendix, this is the place to put it.
\appendix

% \section{Appendix}

\begin{table}[ht]
    \centering
    \caption{The number of labels assigned by human judges and LLMJudge challenge submissions to each judgment level. Bold indicates the closest prediction to the number of labels assigned by humans.}
    \adjustbox{max width=\columnwidth}{%
    \begin{tabular}{lcccc}
    \toprule
    \textbf{Submission ID} & \textbf{0} & \textbf{1} & \textbf{2} & \textbf{3} \\
    \midrule
    human & 2005 & 1233 & 808 & 377 \\
    \midrule
    NISTRetrieval-instruct0 & 1115 & 2092 & 1216 & 0 \\
    NISTRetrieval-instruct1 & 1115 & 2092 & 1216 & 0 \\
    NISTRetrieval-instruct2 & 1117 & 2088 & 1218 & 0 \\
    NISTRetrieval-reason0 & 1159 & 1922 & 1340 & 2 \\
    NISTRetrieval-reason1 & 1158 & 1924 & 1339 & 2 \\
    NISTRetrieval-reason2 & 1159 & 1921 & 1341 & 2 \\
    Olz-exp & 2435 & 1210 & 456 & 322 \\
    Olz-gpt4o & 2258 & 1274 & 504 & \textbf{387} \\
    Olz-halfbin & 2100 & 1458 & 277 & 588 \\
    Olz-multiprompt & 1713 & 976 & 1244 & 490 \\
    Olz-somebin & 2102 & 595 & 1004 & 722 \\
    RMITIR-GPT4o & 3056 & 349 & 730 & 288 \\
    RMITIR-llama38b & 2576 & 614 & 1058 & 175 \\
    RMITIR-llama70B & 2154 & 243 & 1581 & 443 \\
    TREMA-4prompts & 1027 & 751 & 2213 & 432 \\
    TREMA-CoT & 1835 & 1122 & 906 & 560 \\
    TREMA-all & 2399 & 616 & 734 & 674 \\
    TREMA-direct & 2404 & 87 & 342 & 1590 \\
    TREMA-naiveBdecompose & 2627 & 645 & 1117 & 34 \\
    TREMA-nuggets & 2127 & 893 & 1044 & 359 \\
    TREMA-other & 1305 & 809 & 2021 & 288 \\
    TREMA-questions & 2450 & 253 & \textbf{767} & 953 \\
    TREMA-rubric0 & 3122 & 1211 & 0 & 90 \\
    TREMA-sumdecompose & 2518 & 254 & 1049 & 602 \\
    h2oloo-fewself & 2470 & 732 & 557 & 664 \\
    h2oloo-zeroshot1 & 2353 & 1225 & 597 & 248 \\
    h2oloo-zeroshot2 & 2920 & 771 & 476 & 255 \\
    llmjudge-cot1 & 991 & 1921 & 1383 & 128 \\
    llmjudge-cot2 & 1111 & 955 & 2149 & 208 \\
    llmjudge-cot3 & 1902 & 1321 & 486 & 714 \\
    llmjudge-simple1 & 777 & 2228 & 1217 & 200 \\
    llmjudge-simple2 & 1720 & 905 & 1375 & 423 \\
    llmjudge-simple3 & 1834 & 1318 & 485 & 786 \\
    llmjudge-thomas1 & 1049 & 1803 & 1402 & 169 \\
    llmjudge-thomas2 & 1886 & 733 & 1585 & 219 \\
    llmjudge-thomas3 & \textbf{1934} & 1184 & 559 & 746 \\
    prophet-setting1 & 2506 & 885 & 629 & 403 \\
    prophet-setting2 & 2903 & 852 & 651 & 17 \\
    prophet-setting4 & 3359 & 763 & 281 & 20 \\
    willia-umbrela1 & 2335 & \textbf{1231} & 608 & 249 \\
    willia-umbrela2 & 2705 & 1029 & 350 & 339 \\
    willia-umbrela3 & 2710 & 1041 & 446 & 226 \\
    \bottomrule
    \end{tabular}
    }
    \label{tab:labels}
\end{table}

\begin{table}[ht]
    \centering
    \caption{Krippendorff's $\alpha$ correlation in 4-point scale agreement and difference binarize the judgment scale. Judgment levels to the left of the pipe are considered irrelevant, while those to the right are considered relevant.}
    \adjustbox{max width=\columnwidth}{%
    \begin{tabular}{lcccc}
    \toprule
    \textbf{Submission ID} & \textbf{4-point} & \textbf{0|123} & \textbf{01|23} & \textbf{012|3} \\
    \midrule
    NISTRetrieval-instruct0 & 0.3819 & 0.2811 & 0.3021 & -0.0444 \\
    NISTRetrieval-instruct1 & 0.3812 & 0.2801 & 0.3021 & -0.0444 \\
    NISTRetrieval-instruct2 & 0.3821 & 0.2823 & 0.3013 & -0.0444 \\
    NISTRetrieval-reason0 & 0.3874 & 0.263 & 0.3381 & -0.0336 \\
    NISTRetrieval-reason1 & 0.3872 & 0.2624 & 0.3385 & -0.0336 \\
    NISTRetrieval-reason2 & 0.3874 & 0.262 & 0.3388 & -0.0336 \\
    Olz-exp & 0.4701 & 0.3941 & 0.3499 & 0.2933 \\
    Olz-gpt4o & 0.502 & 0.421 & 0.3619 & 0.3067 \\
    Olz-halfbin & 0.4536 & 0.4005 & 0.2534 & 0.2405 \\
    Olz-multiprompt & 0.4551 & 0.3737 & 0.3829 & 0.2137 \\
    Olz-somebin & 0.4471 & 0.3851 & 0.378 & 0.1014 \\
    RMITIR-GPT4o & 0.4108 & 0.3125 & 0.395 & 0.257 \\
    RMITIR-llama38b & 0.3873 & 0.3169 & 0.3194 & 0.1268 \\
    RMITIR-llama70B & 0.4873 & 0.416 & 0.3679 & 0.2839 \\
    TREMA-4prompts & 0.2888 & 0.2644 & 0.1888 & 0.1661 \\
    TREMA-CoT & 0.3852 & 0.3172 & 0.3176 & 0.18 \\
    TREMA-all & 0.3855 & 0.3191 & 0.2957 & 0.0618 \\
    TREMA-direct & 0.3729 & 0.315 & 0.3259 & 0.0868 \\
    TREMA-naiveBdecompose & 0.3579 & 0.2949 & 0.2916 & -0.018 \\
    TREMA-nuggets & 0.1691 & 0.1499 & 0.0967 & -0.0076 \\
    TREMA-other & 0.2712 & 0.2547 & 0.1477 & 0.1399 \\
    TREMA-questions & 0.3148 & 0.2562 & 0.2758 & 0.0125 \\
    TREMA-rubric0 & 0.1036 & 0.1172 & -0.0895 & 0.0167 \\
    TREMA-sumdecompose & 0.3926 & 0.3138 & 0.343 & 0.1995 \\
    h2oloo-fewself & 0.4958 & 0.4108 & 0.428 & 0.2978 \\
    h2oloo-zeroshot1 & 0.4812 & 0.4058 & 0.385 & 0.3063 \\
    h2oloo-zeroshot2 & 0.3898 & 0.3418 & 0.3175 & 0.2769 \\
    llmjudge-cot1 & 0.3218 & 0.1764 & 0.2788 & 0.116 \\
    llmjudge-cot2 & 0.3263 & 0.2173 & 0.2429 & 0.2002 \\
    llmjudge-cot3 & 0.487 & 0.3853 & 0.3979 & 0.2233 \\
    llmjudge-simple1 & 0.2808 & 0.05 & 0.2857 & 0.1528 \\
    llmjudge-simple2 & 0.3672 & 0.2317 & 0.3097 & 0.2003 \\
    llmjudge-simple3 & 0.4642 & 0.3581 & 0.397 & 0.2012 \\
    llmjudge-thomas1 & 0.3207 & 0.1679 & 0.278 & 0.1725 \\
    llmjudge-thomas2 & 0.3853 & 0.294 & 0.2891 & 0.2229 \\
    llmjudge-thomas3 & 0.4877 & 0.3909 & 0.3942 & 0.2321 \\
    prophet-setting1 & 0.4069 & 0.3419 & 0.2892 & 0.1677 \\
    prophet-setting2 & 0.3144 & 0.2815 & 0.2225 & -0.0093 \\
    prophet-setting4 & 0.1623 & 0.1627 & 0.0797 & 0.0006 \\
    willia-umbrela1 & 0.4918 & 0.4129 & 0.3939 & 0.3124 \\
    willia-umbrela2 & 0.4556 & 0.3961 & 0.3298 & 0.3193 \\
    willia-umbrela3 & 0.4535 & 0.3965 & 0.3314 & 0.3185 \\
    \bottomrule
    \end{tabular}
    }
    \label{tab:alpha}
\end{table}

\end{document}
\endinput
%%
%% End of file `sample-sigconf.tex'.
