%%%%%%%%%%%%%%%%%%%%%%% ICML 2025 LATEX SUBMISSION FILE %%%%%%%%%%%%%%%%%%%%%%%

\documentclass{article}
\usepackage{microtype}
\usepackage{graphicx}
\usepackage{subfigure}
\usepackage{booktabs} % for professional tables
\usepackage{multirow}
\usepackage{stfloats}
\usepackage{enumitem}

% hyperref makes hyperlinks in the resulting PDF.
% If your build breaks (sometimes temporarily if a hyperlink spans a page)
% please comment out the following usepackage line and replace
% \usepackage{icml2025} with \usepackage[nohyperref]{icml2025} above.
\usepackage{hyperref}

% Attempt to make hyperref and algorithmic work together better:
\newcommand{\theHalgorithm}{\arabic{algorithm}}

% Use the following line for the initial blind version submitted for review:
% \usepackage{icml2025}

% If accepted, instead use the following line for the camera-ready submission:
\usepackage[accepted]{icml2025}

% For theorems and such
\usepackage{amsmath}
\usepackage{amssymb}
\usepackage{mathtools}
\usepackage{amsthm}
\usepackage[american]{babel}

% if you use cleveref
\usepackage[capitalize,noabbrev]{cleveref}

%%%%%%%%%%%%%%%%%%%%%%%%%%%%%%%%
% THEOREMS
%%%%%%%%%%%%%%%%%%%%%%%%%%%%%%%%
\theoremstyle{plain}
\newtheorem{theorem}{Theorem}[section]
\newtheorem{proposition}[theorem]{Proposition}
\newtheorem{lemma}[theorem]{Lemma}
\newtheorem{corollary}[theorem]{Corollary}
\theoremstyle{definition}
\newtheorem{definition}[theorem]{Definition}
\newtheorem{assumption}[theorem]{Assumption}
\theoremstyle{remark}
\newtheorem{remark}[theorem]{Remark}

% Todonotes is useful during development; simply uncomment the next line
% and comment out the line below the next line to turn off comments
% \usepackage[disable,textsize=tiny]{todonotes}
\usepackage[textsize=tiny]{todonotes}


% The \icmltitle you define below is probably too long as a header.
% Therefore, a short form for the running title is supplied here:
\icmltitlerunning{Understanding the Limits of Deep Tabular Methods with Temporal Shift}

\begin{document}

\twocolumn[
\icmltitle{Understanding the Limits of Deep Tabular Methods with Temporal Shift}

% It is OKAY to include author information, even for blind
% submissions: the style file will automatically remove it for you
% unless you've provided the [accepted] option to the icml2025
% package.

% List of affiliations: The first argument should be a (short)
% identifier you will use later to specify author affiliations
% Academic affiliations should list Department, University, City, Region, Country
% Industry affiliations should list Company, City, Region, Country

% You can specify symbols, otherwise they are numbered in order.
% Ideally, you should not use this facility. Affiliations will be numbered
% in order of appearance and this is the preferred way.
% \icmlsetsymbol{equal}{*}

\begin{icmlauthorlist}
\icmlauthor{Hao-Run Cai}{nju,lab}
\icmlauthor{Han-Jia Ye}{nju,lab}
\end{icmlauthorlist}

\icmlaffiliation{nju}{School of Artificial Intelligence, Nanjing University, China}
\icmlaffiliation{lab}{National Key Laboratory for Novel Software Technology, Nanjing University, China}

\icmlcorrespondingauthor{Han-Jia Ye}{yehj@lamda.nju.edu.cn}

% You may provide any keywords that you
% find helpful for describing your paper; these are used to populate
% the "keywords" metadata in the PDF but will not be shown in the document
\icmlkeywords{Tabular Data, Distribution Shift, Temporal Shift, Open Environment, Machine Learning, ICML}

\vskip 0.3in
]

% this must go after the closing bracket ] following \twocolumn[ ...

% This command actually creates the footnote in the first column
% listing the affiliations and the copyright notice.
% The command takes one argument, which is text to display at the start of the footnote.
% The \icmlEqualContribution command is standard text for equal contribution.
% Remove it (just {}) if you do not need this facility.

\printAffiliationsAndNotice{}  % leave blank if no need to mention equal contribution
% \printAffiliationsAndNotice{\icmlEqualContribution} % otherwise use the standard text.

\begin{abstract}  
Test time scaling is currently one of the most active research areas that shows promise after training time scaling has reached its limits.
Deep-thinking (DT) models are a class of recurrent models that can perform easy-to-hard generalization by assigning more compute to harder test samples.
However, due to their inability to determine the complexity of a test sample, DT models have to use a large amount of computation for both easy and hard test samples.
Excessive test time computation is wasteful and can cause the ``overthinking'' problem where more test time computation leads to worse results.
In this paper, we introduce a test time training method for determining the optimal amount of computation needed for each sample during test time.
We also propose Conv-LiGRU, a novel recurrent architecture for efficient and robust visual reasoning. 
Extensive experiments demonstrate that Conv-LiGRU is more stable than DT, effectively mitigates the ``overthinking'' phenomenon, and achieves superior accuracy.
\end{abstract}  
\section{Introduction}
\label{sec:introduction}
The business processes of organizations are experiencing ever-increasing complexity due to the large amount of data, high number of users, and high-tech devices involved \cite{martin2021pmopportunitieschallenges, beerepoot2023biggestbpmproblems}. This complexity may cause business processes to deviate from normal control flow due to unforeseen and disruptive anomalies \cite{adams2023proceddsriftdetection}. These control-flow anomalies manifest as unknown, skipped, and wrongly-ordered activities in the traces of event logs monitored from the execution of business processes \cite{ko2023adsystematicreview}. For the sake of clarity, let us consider an illustrative example of such anomalies. Figure \ref{FP_ANOMALIES} shows a so-called event log footprint, which captures the control flow relations of four activities of a hypothetical event log. In particular, this footprint captures the control-flow relations between activities \texttt{a}, \texttt{b}, \texttt{c} and \texttt{d}. These are the causal ($\rightarrow$) relation, concurrent ($\parallel$) relation, and other ($\#$) relations such as exclusivity or non-local dependency \cite{aalst2022pmhandbook}. In addition, on the right are six traces, of which five exhibit skipped, wrongly-ordered and unknown control-flow anomalies. For example, $\langle$\texttt{a b d}$\rangle$ has a skipped activity, which is \texttt{c}. Because of this skipped activity, the control-flow relation \texttt{b}$\,\#\,$\texttt{d} is violated, since \texttt{d} directly follows \texttt{b} in the anomalous trace.
\begin{figure}[!t]
\centering
\includegraphics[width=0.9\columnwidth]{images/FP_ANOMALIES.png}
\caption{An example event log footprint with six traces, of which five exhibit control-flow anomalies.}
\label{FP_ANOMALIES}
\end{figure}

\subsection{Control-flow anomaly detection}
Control-flow anomaly detection techniques aim to characterize the normal control flow from event logs and verify whether these deviations occur in new event logs \cite{ko2023adsystematicreview}. To develop control-flow anomaly detection techniques, \revision{process mining} has seen widespread adoption owing to process discovery and \revision{conformance checking}. On the one hand, process discovery is a set of algorithms that encode control-flow relations as a set of model elements and constraints according to a given modeling formalism \cite{aalst2022pmhandbook}; hereafter, we refer to the Petri net, a widespread modeling formalism. On the other hand, \revision{conformance checking} is an explainable set of algorithms that allows linking any deviations with the reference Petri net and providing the fitness measure, namely a measure of how much the Petri net fits the new event log \cite{aalst2022pmhandbook}. Many control-flow anomaly detection techniques based on \revision{conformance checking} (hereafter, \revision{conformance checking}-based techniques) use the fitness measure to determine whether an event log is anomalous \cite{bezerra2009pmad, bezerra2013adlogspais, myers2018icsadpm, pecchia2020applicationfailuresanalysispm}. 

The scientific literature also includes many \revision{conformance checking}-independent techniques for control-flow anomaly detection that combine specific types of trace encodings with machine/deep learning \cite{ko2023adsystematicreview, tavares2023pmtraceencoding}. Whereas these techniques are very effective, their explainability is challenging due to both the type of trace encoding employed and the machine/deep learning model used \cite{rawal2022trustworthyaiadvances,li2023explainablead}. Hence, in the following, we focus on the shortcomings of \revision{conformance checking}-based techniques to investigate whether it is possible to support the development of competitive control-flow anomaly detection techniques while maintaining the explainable nature of \revision{conformance checking}.
\begin{figure}[!t]
\centering
\includegraphics[width=\columnwidth]{images/HIGH_LEVEL_VIEW.png}
\caption{A high-level view of the proposed framework for combining \revision{process mining}-based feature extraction with dimensionality reduction for control-flow anomaly detection.}
\label{HIGH_LEVEL_VIEW}
\end{figure}

\subsection{Shortcomings of \revision{conformance checking}-based techniques}
Unfortunately, the detection effectiveness of \revision{conformance checking}-based techniques is affected by noisy data and low-quality Petri nets, which may be due to human errors in the modeling process or representational bias of process discovery algorithms \cite{bezerra2013adlogspais, pecchia2020applicationfailuresanalysispm, aalst2016pm}. Specifically, on the one hand, noisy data may introduce infrequent and deceptive control-flow relations that may result in inconsistent fitness measures, whereas, on the other hand, checking event logs against a low-quality Petri net could lead to an unreliable distribution of fitness measures. Nonetheless, such Petri nets can still be used as references to obtain insightful information for \revision{process mining}-based feature extraction, supporting the development of competitive and explainable \revision{conformance checking}-based techniques for control-flow anomaly detection despite the problems above. For example, a few works outline that token-based \revision{conformance checking} can be used for \revision{process mining}-based feature extraction to build tabular data and develop effective \revision{conformance checking}-based techniques for control-flow anomaly detection \cite{singh2022lapmsh, debenedictis2023dtadiiot}. However, to the best of our knowledge, the scientific literature lacks a structured proposal for \revision{process mining}-based feature extraction using the state-of-the-art \revision{conformance checking} variant, namely alignment-based \revision{conformance checking}.

\subsection{Contributions}
We propose a novel \revision{process mining}-based feature extraction approach with alignment-based \revision{conformance checking}. This variant aligns the deviating control flow with a reference Petri net; the resulting alignment can be inspected to extract additional statistics such as the number of times a given activity caused mismatches \cite{aalst2022pmhandbook}. We integrate this approach into a flexible and explainable framework for developing techniques for control-flow anomaly detection. The framework combines \revision{process mining}-based feature extraction and dimensionality reduction to handle high-dimensional feature sets, achieve detection effectiveness, and support explainability. Notably, in addition to our proposed \revision{process mining}-based feature extraction approach, the framework allows employing other approaches, enabling a fair comparison of multiple \revision{conformance checking}-based and \revision{conformance checking}-independent techniques for control-flow anomaly detection. Figure \ref{HIGH_LEVEL_VIEW} shows a high-level view of the framework. Business processes are monitored, and event logs obtained from the database of information systems. Subsequently, \revision{process mining}-based feature extraction is applied to these event logs and tabular data input to dimensionality reduction to identify control-flow anomalies. We apply several \revision{conformance checking}-based and \revision{conformance checking}-independent framework techniques to publicly available datasets, simulated data of a case study from railways, and real-world data of a case study from healthcare. We show that the framework techniques implementing our approach outperform the baseline \revision{conformance checking}-based techniques while maintaining the explainable nature of \revision{conformance checking}.

In summary, the contributions of this paper are as follows.
\begin{itemize}
    \item{
        A novel \revision{process mining}-based feature extraction approach to support the development of competitive and explainable \revision{conformance checking}-based techniques for control-flow anomaly detection.
    }
    \item{
        A flexible and explainable framework for developing techniques for control-flow anomaly detection using \revision{process mining}-based feature extraction and dimensionality reduction.
    }
    \item{
        Application to synthetic and real-world datasets of several \revision{conformance checking}-based and \revision{conformance checking}-independent framework techniques, evaluating their detection effectiveness and explainability.
    }
\end{itemize}

The rest of the paper is organized as follows.
\begin{itemize}
    \item Section \ref{sec:related_work} reviews the existing techniques for control-flow anomaly detection, categorizing them into \revision{conformance checking}-based and \revision{conformance checking}-independent techniques.
    \item Section \ref{sec:abccfe} provides the preliminaries of \revision{process mining} to establish the notation used throughout the paper, and delves into the details of the proposed \revision{process mining}-based feature extraction approach with alignment-based \revision{conformance checking}.
    \item Section \ref{sec:framework} describes the framework for developing \revision{conformance checking}-based and \revision{conformance checking}-independent techniques for control-flow anomaly detection that combine \revision{process mining}-based feature extraction and dimensionality reduction.
    \item Section \ref{sec:evaluation} presents the experiments conducted with multiple framework and baseline techniques using data from publicly available datasets and case studies.
    \item Section \ref{sec:conclusions} draws the conclusions and presents future work.
\end{itemize}
\section{RELATED WORK}
\label{sec:relatedwork}
In this section, we describe the previous works related to our proposal, which are divided into two parts. In Section~\ref{sec:relatedwork_exoplanet}, we present a review of approaches based on machine learning techniques for the detection of planetary transit signals. Section~\ref{sec:relatedwork_attention} provides an account of the approaches based on attention mechanisms applied in Astronomy.\par

\subsection{Exoplanet detection}
\label{sec:relatedwork_exoplanet}
Machine learning methods have achieved great performance for the automatic selection of exoplanet transit signals. One of the earliest applications of machine learning is a model named Autovetter \citep{MCcauliff}, which is a random forest (RF) model based on characteristics derived from Kepler pipeline statistics to classify exoplanet and false positive signals. Then, other studies emerged that also used supervised learning. \cite{mislis2016sidra} also used a RF, but unlike the work by \citet{MCcauliff}, they used simulated light curves and a box least square \citep[BLS;][]{kovacs2002box}-based periodogram to search for transiting exoplanets. \citet{thompson2015machine} proposed a k-nearest neighbors model for Kepler data to determine if a given signal has similarity to known transits. Unsupervised learning techniques were also applied, such as self-organizing maps (SOM), proposed \citet{armstrong2016transit}; which implements an architecture to segment similar light curves. In the same way, \citet{armstrong2018automatic} developed a combination of supervised and unsupervised learning, including RF and SOM models. In general, these approaches require a previous phase of feature engineering for each light curve. \par

%DL is a modern data-driven technology that automatically extracts characteristics, and that has been successful in classification problems from a variety of application domains. The architecture relies on several layers of NNs of simple interconnected units and uses layers to build increasingly complex and useful features by means of linear and non-linear transformation. This family of models is capable of generating increasingly high-level representations \citep{lecun2015deep}.

The application of DL for exoplanetary signal detection has evolved rapidly in recent years and has become very popular in planetary science.  \citet{pearson2018} and \citet{zucker2018shallow} developed CNN-based algorithms that learn from synthetic data to search for exoplanets. Perhaps one of the most successful applications of the DL models in transit detection was that of \citet{Shallue_2018}; who, in collaboration with Google, proposed a CNN named AstroNet that recognizes exoplanet signals in real data from Kepler. AstroNet uses the training set of labelled TCEs from the Autovetter planet candidate catalog of Q1–Q17 data release 24 (DR24) of the Kepler mission \citep{catanzarite2015autovetter}. AstroNet analyses the data in two views: a ``global view'', and ``local view'' \citep{Shallue_2018}. \par


% The global view shows the characteristics of the light curve over an orbital period, and a local view shows the moment at occurring the transit in detail

%different = space-based

Based on AstroNet, researchers have modified the original AstroNet model to rank candidates from different surveys, specifically for Kepler and TESS missions. \citet{ansdell2018scientific} developed a CNN trained on Kepler data, and included for the first time the information on the centroids, showing that the model improves performance considerably. Then, \citet{osborn2020rapid} and \citet{yu2019identifying} also included the centroids information, but in addition, \citet{osborn2020rapid} included information of the stellar and transit parameters. Finally, \citet{rao2021nigraha} proposed a pipeline that includes a new ``half-phase'' view of the transit signal. This half-phase view represents a transit view with a different time and phase. The purpose of this view is to recover any possible secondary eclipse (the object hiding behind the disk of the primary star).


%last pipeline applies a procedure after the prediction of the model to obtain new candidates, this process is carried out through a series of steps that include the evaluation with Discovery and Validation of Exoplanets (DAVE) \citet{kostov2019discovery} that was adapted for the TESS telescope.\par
%



\subsection{Attention mechanisms in astronomy}
\label{sec:relatedwork_attention}
Despite the remarkable success of attention mechanisms in sequential data, few papers have exploited their advantages in astronomy. In particular, there are no models based on attention mechanisms for detecting planets. Below we present a summary of the main applications of this modeling approach to astronomy, based on two points of view; performance and interpretability of the model.\par
%Attention mechanisms have not yet been explored in all sub-areas of astronomy. However, recent works show a successful application of the mechanism.
%performance

The application of attention mechanisms has shown improvements in the performance of some regression and classification tasks compared to previous approaches. One of the first implementations of the attention mechanism was to find gravitational lenses proposed by \citet{thuruthipilly2021finding}. They designed 21 self-attention-based encoder models, where each model was trained separately with 18,000 simulated images, demonstrating that the model based on the Transformer has a better performance and uses fewer trainable parameters compared to CNN. A novel application was proposed by \citet{lin2021galaxy} for the morphological classification of galaxies, who used an architecture derived from the Transformer, named Vision Transformer (VIT) \citep{dosovitskiy2020image}. \citet{lin2021galaxy} demonstrated competitive results compared to CNNs. Another application with successful results was proposed by \citet{zerveas2021transformer}; which first proposed a transformer-based framework for learning unsupervised representations of multivariate time series. Their methodology takes advantage of unlabeled data to train an encoder and extract dense vector representations of time series. Subsequently, they evaluate the model for regression and classification tasks, demonstrating better performance than other state-of-the-art supervised methods, even with data sets with limited samples.

%interpretation
Regarding the interpretability of the model, a recent contribution that analyses the attention maps was presented by \citet{bowles20212}, which explored the use of group-equivariant self-attention for radio astronomy classification. Compared to other approaches, this model analysed the attention maps of the predictions and showed that the mechanism extracts the brightest spots and jets of the radio source more clearly. This indicates that attention maps for prediction interpretation could help experts see patterns that the human eye often misses. \par

In the field of variable stars, \citet{allam2021paying} employed the mechanism for classifying multivariate time series in variable stars. And additionally, \citet{allam2021paying} showed that the activation weights are accommodated according to the variation in brightness of the star, achieving a more interpretable model. And finally, related to the TESS telescope, \citet{morvan2022don} proposed a model that removes the noise from the light curves through the distribution of attention weights. \citet{morvan2022don} showed that the use of the attention mechanism is excellent for removing noise and outliers in time series datasets compared with other approaches. In addition, the use of attention maps allowed them to show the representations learned from the model. \par

Recent attention mechanism approaches in astronomy demonstrate comparable results with earlier approaches, such as CNNs. At the same time, they offer interpretability of their results, which allows a post-prediction analysis. \par


% Consider a lasso optimization procedure with potentially distinct regularization penalties:
% \begin{align}
%     \hat{\beta} = \arg\min_{\beta}\{\|y-X\beta\|^2_2+\sum_{i=1}^{N}\lambda_i|\beta_i|\}.
% \end{align}
\subsection{Supervised Data-Driven Learning}\label{subsec:supervised}
We consider a generic data-driven supervised learning procedure. Given a dataset \( \mathcal{D} \) consisting of \( n \) data points \( (x_i, y_i) \in \mathcal{X} \times \mathcal{Y} \) drawn from an underlying distribution \( p(\cdot|\theta) \), our goal is to estimate parameters \( \theta \in \Theta \) through a learning procedure, defined as \( f: (\mathcal{X} \times \mathcal{Y})^n \rightarrow \Theta \) 
that minimizes the predictive error on observed data. 
Specifically, the learning objective is defined as follows:
\begin{align}
\hat{\theta}_f := f(\mathcal{D}) = \arg\min_{\theta} \mathcal{L}(\theta, \mathcal{D}),
\end{align}
where \( \mathcal{L}(\cdot,\mathcal{D}) := \sum_{i=1}^{n} \mathcal{L}(\cdot, (x_i, y_i))\), and $\mathcal{L}$ is a loss function quantifying the error between predictions and true outcomes. 
Here, $\hat{\theta}_f$ is the parameter that best explains the observed data pairs \( (x_i, y_i) \) according to the chosen loss function \( \mathcal{L} (\cdot) \).

\paragraph{Feature Selection.}
Feature selection aims to improve model \( f \)'s predictive performance while minimizing redundancy. 
%Formally, given data \( X \), response \( y \), feature set \( \mathcal{F} \), loss function \( \mathcal{L}(\cdot) \), and a feature limit \( k \), the objective is:
% \begin{align}
% \mathcal{S}^* = \arg \min_{\mathcal{S} \subseteq \mathcal{F}, |\mathcal{S}| \leq k} \mathcal{L}(y, f(X_\mathcal{S})) + \lambda R(\mathcal{S}),
% \end{align}
% where \( X_\mathcal{S} \) is the submatrix of \( X \) for selected features \( \mathcal{S} \), \( \lambda \) is a regularization parameter, and \( R(\mathcal{S}) \) penalizes feature redundancy.
 State-of-the-art techniques fall into four categories: (i) filter methods, which rank features based on statistical properties like Fisher score \citep{duda2001pattern,song2012feature}; (ii) wrapper methods, which evaluate model performance on different feature subsets \citep{kohavi1997wrappers}; (iii) embedded methods, which integrate feature selection into the learning process using techniques like regularization \citep{tibshirani1996LASSO,lemhadri2021lassonet}; and (iv) hybrid methods, which combine elements of (i)-(iii) \citep{SINGH2021104396,li2022micq}. This paper focuses on embedded methods via Lasso, benchmarking against approaches from (i)-(iii).

\subsection{Language Modeling}
% The objective of language modeling is to learn a probability distribution \( p_{LM}(x) \) over sequences of text \( x = (X_1, \ldots, X_{|x|}) \), such that \( p_{LM}(x) \approx p_{text}(x) \), where \( p_{text}(x) \) represents the true distribution of natural language. This process involves estimating the likelihood of token sequences across variable lengths and diverse linguistic structures.
% Modern large language models (LLMs) are trained on vast datasets spanning encyclopedias, news, social media, books, and scientific papers \cite{gao2020pile}. This broad training enables them to generalize across domains, learn contextual knowledge, and perform zero-shot learning—tackling new tasks using only task descriptions without fine-tuning \cite{brown2020gpt3}.
Language modeling aims to approximate the true distribution of natural language \( p_{\text{text}}(x) \) by learning \( p_{\text{LM}}(x) \), a probability distribution over text sequences \( x = (X_1, \ldots, X_{|x|}) \). Modern large language models, trained on diverse datasets \citep{gao2020pile}, exhibit strong generalization across domains, acquire contextual knowledge, and perform zero-shot learning—solving new tasks using only task descriptions—or few-shot learning by leveraging a small number of demonstrations \citep{brown2020gpt3}.
\paragraph{Retrieval-Augmented Generation (RAG).} Retrieval-Augmented Generation (RAG) enhances the performance of generative language models by  integrating a domain-specific information retrieval process  \citep{lewis2020retrieval}. The RAG framework comprises two main components: \textit{retrieval}, which extracts relevant information from external knowledge sources, and \textit{generation}, where an LLM generates context-aware responses using the prompt combined with the retrieved context. Documents are indexed through various databases, such as relational, graph, or vector databases \citep{khattab2020colbert, douze2024faiss, peng2024graphretrievalaugmentedgenerationsurvey}, enabling efficient organization and retrieval via algorithms like semantic similarity search to match the prompt with relevant documents in the knowledge base. RAG has gained much traction recently due to its demonstrated ability to reduce incidence of hallucinations and boost LLMs' reliability as well as performance \citep{huang2023hallucination, zhang2023merging}. 
 
% image source: https://medium.com/@bindurani_22/retrieval-augmented-generation-815c1ae438d8
\begin{figure}
    \centering
\includegraphics[width=1.03\linewidth]{fig/fig1.pdf}
\vspace{-0.6cm}
\scriptsize 
    \caption{Retrieval Augmented Generation (RAG) based $\ell_1$-norm weights (penalty factors) for Lasso. Only feature names---no training data--- are included in LLM prompt.} 
    \label{fig:rag}
\end{figure}
% However, for the RAG model to be effective given the input token constraints of the LLM model used, we need to effectively process the retrieval documents through a procedure known as \textit{chunking}.

\subsection{Task-Specific Data-Driven Learning}
LLM-Lasso aims to bridge the gap between data-driven supervised learning and the predictive capabilities of LLMs trained on rich metadata. This fusion not only enhances traditional data-driven methods by incorporating key task-relevant contextual information often overlooked by such models, but can also be especially valuable in low-data regimes, where the learning algorithm $f:\mathcal{D}\rightarrow\Theta$ (seen as a map from datasets $\mathcal{D}$ to the space of decisions $\Theta$) is susceptible to overfitting.

The task-specific data-driven learning model $\tilde{f}:\mathcal{D}\times\mathcal{D}_\text{meta}\rightarrow\Theta$ can be described as a metadata-augmented version of $f$, where a link function $h(\cdot)$ integrates metadata (i.e. $\mathcal{D}_\text{meta}$) to refine the original learning process. This can be expressed as:
\[
\tilde{f}(\mathcal{D}, \mathcal{D}_\text{meta}) := \mathcal{T}(f(\mathcal{D}),  h(\mathcal{D}_{\text{meta}})),
\]
where the functional $\mathcal{T}$ takes the original learning algorithm $f(\mathcal{D})$ and transforms it into a task-specific learning algorithm $\tilde{f}(\mathcal{D}, \mathcal{D}_\text{meta})$ by incorporating the metadata $\mathcal{D}_\text{meta}$. 
% In particular, the link function $h(\mathcal{D}_{\text{meta}})$ provides a structured mechanism summarizing the contextual knowledge.

There are multiple approaches to formulate $\mathcal{T}$ and $h$.
%to ``inform" the data-driven model $f$ of %meta knowledge. 
For instance, LMPriors \citep{choi2022lmpriorspretrainedlanguagemodels} designed $h$ and $\mathcal{T}$ such that $h(\mathcal{D}_{\text{meta}})$ first specifies which features to retain (based on a probabilistic prior framework), and then $\mathcal{T}$ keeps the selected features and removes all the others from the original learning objective of $f$. 
Note that this approach inherently is restricted as it selects important features solely based on $\mathcal{D}_\text{meta}$ without seeing $\mathcal{D}$.

In contrast, we directly embed task-specific knowledge into the optimization landscape through regularization by introducing a structured inductive bias. This bias guides the learning process toward solutions that are consistent with metadata-informed insights, without relying on explicit probabilistic modeling. Abstractly, this can be expressed as:
\begin{align}
    \!\!\!\!\!\hat{\theta}_{\tilde{f}} := \tilde{f}(\mathcal{D},\mathcal{D}
    _\text{meta})= \arg\min_{\theta} \mathcal{L}(\theta, \mathcal{D}) + \lambda R(\theta, \mathcal{D}_{\text{meta}}),
\end{align}
where \( \lambda \) is a regularization parameter, \( R(\cdot) \) is a regularizer, and $\theta$ is the prediction parameter.
%We explain our framework with more details in the following section.


% Our research diverges from both aforementioned approaches by positioning the LLM not as a standalone feature selector but as an enhancement to data-driven models through an embedded feature selection method, L-LASSO. L-LASSO incorporates domain expertise—auxiliary natural language metadata about the task—via the LLM-informed LASSO penalty, which is then used in statistical models to enhance predictive performance. This method integrates the rich, context-sensitive insights of LLMs with the rigor and transparency of statistical modeling, bridging the gap between data-driven and knowledge-driven feature selection approaches. To approach this task, we need to tackle two key components: (i). train an LLM that is expert in the task-specific knowledge; (ii). inform data-driven feature selector LASSO with LLM knowledge.

% In practice, this involves combining techniques like prompt engineering and data engineering to develop an effective framework for integrating metadata into existing data-driven models. We will go through this in detail in Section \ref{mthd} and \ref{experiment}.


\section{Why Temporal Split is Ineffective for Tabular Data?}
\label{sec:why_bad}

\begin{figure*}[t]
  \centering
  % \vspace{-15pt}
  \begin{minipage}{0.39\linewidth}
      \centering
      \includegraphics[width=\linewidth]{image/cooking-time-mlp_plr-performance.pdf}
  \end{minipage}
  \hfill
  \begin{minipage}{0.59\linewidth}
      \centering
      \includegraphics[width=0.24\linewidth]{image/mmd/homesite-insurance_time_linear.png}
      \includegraphics[width=0.24\linewidth]{image/mmd/ecom-offers_time_linear.png}
      \includegraphics[width=0.24\linewidth]{image/mmd/homecredit-default_time_linear.png}
      \includegraphics[width=0.24\linewidth]{image/mmd/sberbank-housing_time_linear.png} \\
      \includegraphics[width=0.24\linewidth]{image/mmd/cooking-time_time_linear.png}
      \includegraphics[width=0.24\linewidth]{image/mmd/delivery-eta_time_linear.png}
      \includegraphics[width=0.24\linewidth]{image/mmd/maps-routing_time_linear.png}
      \includegraphics[width=0.24\linewidth]{image/mmd/weather_time_linear.png}
  \end{minipage}
  \vspace{-10pt}
  \caption{Left: The loss distribution shows how the model's performance distributed across time slices under different validation splitting strategies. The vertical axis represents the loss, where lower is better. Non-lagged splits (b) and (d) achieve better performance around \(T_\text{train}\) compared to lagged splits (a) and (c), while the higher-biased split (c) performs better on training-available data but fails to generalize compared to the lower-biased split (a). The loss distribution is smoothed by a Gaussian filter for better visualization. Right: The MMD heatmap visualizing the distribution distance between different time slices using linear kernel. The time slices are divided by date.}
  \label{fig:mmd}
  \vspace{-10pt}
\end{figure*}

The fundamental challenge in addressing temporal shifts stems from the inherent difficulty in characterizing evolving data distributions, which requires models to maintain robustness against unknown future variations. 
Temporal splits are frequently used in forecasting-related tasks to uphold the sequential dependencies inherent in the data \cite{bergmeir2012use, zeng2023transformers, han2024revisiting}, yet they generally show inferior performance compared to random splits in tabular data contexts, as evidenced in \cref{sec:introduction} and \cref{fig:random}.
This counterintuitive phenomenon warrants systematic analysis through multiple perspectives:
We first start with the most intuitive distinction between random split and temporal split, and verify the hypothesis of how training lag (\cref{subsec:lag}) and validation bias (\cref{subsec:bias}) impact the model performance.
Then we discover the equivalence of the validation set in different temporal directions (\cref{subsec:equivalence}). Building on this, we propose an enhanced splitting strategy (\cref{subsec:citerion}) for temporally shifted tabular data, which achieves performance comparable to random splitting while offering improved stability.

\subsection{Training Lag}
\label{subsec:lag}

From the perspective of the training set, temporal split adopted in \citet{rubachev2024tabred} introduces a \textit{temporal lag} between training and test set, while random split provides more instances closer to the test time for training. An intuitive hypothesis is that instances closer to the test time are more reliable, as the distribution near the test time tends to be more similar to the actual test-time distribution, and using them for training could have a greater effect than for validation. 
Hence we adapt a pair of splitting strategies, shown in \cref{fig:split} left (a) and (b), where identical validation and test sets are maintained while varying the lag between training and test set, thereby isolating and quantifying the impact of training lag on model performance.

We selected MLP-PLR, ModernNCA, and TabM as representative methods for MLP architecture, retrieval-based, and ensemble-based methods, respectively, and conducted experiments of splitting strategy on these three method.
Experimental results shown in \cref{fig:split} right infers that this hypothesis holds for all three methods, with a total average improvement of 1.62\%. Among them, the retrieval-based methods ModernNCA shows the highest improvement of 2.19\%, indicating the importance of no-lag candidates for retrieval-based methods in the presence of temporal shifts.

\subsection{Validation Bias}
\label{subsec:bias}

Current deep tabular methods rely on information from the validation set for model selection, as deep learning models are optimized epoch-wise during training. In the context of temporal shifts, this reliance becomes particularly problematic, since there is a
lack of accurate validation of the test data in the training
stage \cite{ganin2015unsupervised, blanchard2021domain}. The time gap between the training and test sets is larger than the gap between the training and validation sets in the previous temporal split, leading to a more significant distribution shift at test time. This makes it more challenging to accurately predict test-time instances compared to validation, thereby introducing \textit{bias into the validation process}.
Therefore we further design the splitting strategy shown in \cref{fig:split} left (c), which shares the training and test sets with (a), but the latter has a more considerable validation bias since the time interval difference between train-val and train-test is much larger.

The performance comparison of split (a) and (c), as shown in \cref{fig:split} right, confirms that validation bias also has a notable impact on performance, especially for ensemble-based methods. The total average improvement is 0.59\%, while the ensemble-based TabM show a more significant improvement of 0.83\%. This is explainable since the ensemble-based methods are robust to the training data quality by reducing the variance, but sensitive to the bias of validation.

\begin{table*}[t]
    \centering
    {\footnotesize{
    \setlength{\tabcolsep}{3pt}
    \begin{tabular}{lcccccccccccc}
    \toprule
    \textbf{Splits} & {MLP} & {PLR} & {FT-T} & {SNN} & {DCNv2} & {TabR} & {MNCA} & {TabM} & {XGBoost} & {CatBoost} & {LGBM} & \textbf{Avg. Imp.} \\
    \midrule
    \multicolumn{13}{l}{\textbf{Mean Performance $\uparrow$}} \\
    Random & $+$4.30\% & $+$0.73\% & $+$3.76\% & $+$1.38\% & $+$2.11\% & $+$2.00\% & $+$2.53\% & $+$1.51\% & $+$1.79\% & $+$2.09\% & $+$1.73\% & $+$2.17\% \\
    Ours & $+$3.50\% & $+$0.75\% & $+$2.78\% & $+$1.38\% & $+$3.01\% & $+$2.20\% & $+$2.49\% & $+$1.25\% & $+$2.06\% & $+$2.37\% & $+$2.14\% & \textbf{$+$2.18\%} \\
    \midrule
    \multicolumn{13}{l}{\textbf{Standard Deviation $\downarrow$}} \\
    Random & $+$1.81\% & $+$126\% & $+$0.15\% & $+$44.0\% & $+$0.06\% & $+$224\% & $+$74.9\% & $+$44.3\% & $+$456\% & $+$105\% & $+$616\% & $+$154\% \\
    Ours & $+$29.7\% & $+$50.9\% & $+$5.59\% & $+$16.8\% & $+$8.86\% & $+$82.2\% & $-$10.8\% & $+$37.7\% & $-$15.2\% & $-$30.6\% & $+$8.59\% & \textbf{$+$16.7\%} \\
    \midrule
    \multicolumn{13}{l}{\textbf{Performance Rankings $\downarrow$}} \\
    Random & 8.250 & 5.625 & 5.625 & 10.250 & 9.625 & 8.000 & \underline{4.750} & \textbf{2.750} & \underline{3.125} & \underline{3.125} & 4.875 & -- \\
    Ours & 8.000 & 5.750 & 7.500 & 9.500 & 8.375 & 8.125 & 4.875 & \underline{4.000} & \underline{3.375} & \textbf{2.125} & \underline{4.375} & -- \\
    \bottomrule
    \end{tabular}}}
    \vspace{-10pt}
    \caption{Comparison of performance and stability between the random split in \cref{fig:random} and our proposed temporal split in \cref{subsec:citerion}, measured by the \textit{average percentage change} on the TabReD benchmark, along with the performance ranking of each method. 
    ``PLR,'' ``MNCA,'' and ``LGBM'' denote ``MLP-PLR,'' ``ModernNCA,'' and ``LightGBM,'' respectively. 
    The percentage change represents the difference in the mean (higher is better) or the standard deviation (lower is better, indicating stability) of performance, relative to the baseline temporal split in \citet{rubachev2024tabred}, for each method.
    The results show that our temporal splitting strategy achieves performance comparable to the random split, while offering significantly better stability. The ranking change is minimal, with token-based methods favoring the random split and tree-based methods performing better with our temporal split. The comparison of methods under random and temporal splits is also plotted in \cref{fig:random} and \cref{fig:temporal}. Detailed results are provided in \cref{sec:appendix_result}, with extended rankings.}
    \label{tab:stability}
    \vspace{-12pt}
\end{table*}

\subsection{Bridging the Past and Future in Temporal Split}
\label{subsec:equivalence}

Through the above exploration, we have confirmed the improvements in model performance achieved by reducing the training lag and validation bias. The main question now becomes how to utilize the above insights.

To illustrate how reducing the training lag and validation bias improves model performance, we visualize the loss distribution of the model across different time slices under different validation splitting strategies. \cref{fig:mmd} left shows the loss distribution for the Cooking Time dataset using MLP-PLR model, it clearly shows how reducing training lag and validation bias improves the model performance. Each splits achieve the best performance among the training-available time slices, which allows the model to perform better on instances closer to the test time in the non-lagged split (b) compared to the lagged splits (a) and (c). Furthermore, the higher-biased split (c) works better on training splits but struggles on the test splits, while the lower-biased split (a) achieves a more balanced performance across training and test splits. 
This indicates that a lower-biased validation set brings a more precise direction of the test-time distribution, enabling the model to generalize better.
The above observation again enhances the insights of training lag and validation bias. On a deeper level, reducing the training lag concentrates the model's performance around a time point closer to the test time, while precise validation ensures that the model's performance can effectively generalize from this concentrated time point to the test period.

In addition, another interesting observation in \cref{fig:mmd} left is that the model in split (b) never meet the most former data splits during training and validation, but achieves a relatively well performance on it compared to the test splits. Building on the insight that the validation set is mainly for maintaining generalization performance, this finding inspired us to explore whether data from the opposite temporal direction could be an effective validation set, which is rely on the assumption that the temporal shifts distributed uniformly across each time slices. 
We further use a heatmap to visualize the distribution distance between different time slices using Maximum Mean Discrepancy (MMD) \cite{gretton2006kernel}, as shown in \cref{fig:mmd} right. The MMD heatmap reveals that all datasets exhibit temporal shifts characterized by trends and multiple periodic components. The regular diagonal stripes both indicate the presence of periodicity, and suggest that the sample distributions at identical time intervals are similar, which confirms the empirical uniformity of temporal shifts across time slices.

Finally we design the splitting strategy shown in \cref{fig:split} left (d),  where the validation set is in the opposite temporal direction compared to split (b), meanwhile maintaining the training lag and validation bias. The results in \cref{fig:split} right indicate a performance drop of 0.91\% compared to split (b). However, it is important to highlight that, this performance degradation is noticeably smaller than the improvement observed in (b) relative to (a). This indicates that adopting this alternative splitting strategy to minimize the training gap is always a desirable approach.

In detail, the model performance on this splitting strategy is highly dependent on the dataset, which is primarily due to the nonuniform temporal sampling in some datasets, where maintaining the same amount of validation data compromises the consistency of the time intervals. Further explanation can be found in \cref{sec:appendix_dataset}.

\begin{figure*}[t!]
  \centering
  \includegraphics[width=\linewidth]{image/MMD_Explaination_new_new.pdf}
  \vspace{-20pt}
  \caption{Left: Detailed MMD heatmap visualization of the Homesite Insurance dataset, showing the \textit{trend} (lighter colors farther from the diagonal) and yearly/weekly \textit{periodicity} (stripes at different scales) in the data. Mid and Right: The MMD heatmap of the representation learned by MLP \textit{before and after adopting our temporal embedding} on this dataset. Without temporal embedding, the model only learns two patterns (weekdays and weekends) with weak discrimination. After applying temporal embedding, the model's representation aligns with the original data, capturing phase-specific knowledge for each temporal stage (day of the week) and achieving clear distinction.}
  \label{fig:periodic}
  \vspace{-10pt}
\end{figure*}

\subsection{Our Temporal Split}
\label{subsec:citerion}

Based on the above findings, we introduce the following training protocol for temporal tabular data:
\begin{enumerate}[noitemsep,topsep=0pt,leftmargin=*]
  \item The lag between training and test set should be minimized since instances near the test time are more valuable for training, rather than for model selection.
  \item The validation bias should be minimized, which can be achieved by reducing the time interval difference between train-val and train-test.
  \item The equivalence property of the validation set in different temporal direction is maintained for most tasks. 
  An effective validation is available in the opposite temporal direction by aligning the degree of shift in the validation set with the actual shift between training and testing data.
\end{enumerate}
We further propose a more effective temporal splitting strategy that fully leverages this protocol, shown in the bottom of \cref{fig:split} left. 
In this strategy, the training lag is minimized to zero, and the validation set is also aligned with the test set, as it has a similar time interval relative to the training set thus exhibits a similar degree of distribution shift.

The performance and stability comparison between the random split and our newly proposed temporal split is presented in \cref{tab:stability}. The results demonstrate that our splitting strategy achieves a performance improvement comparable to the random split (2.18\% vs. 2.17\% on average) relative to the baseline temporal split in \citet{rubachev2024tabred}. 
Additionally, while our temporal split shows a modest increase in the standard deviation of performance scores (AUC for classification and RMSE for regression) by 16.69\%, the random split results in a much larger increase of 153.81\%. This indicates that, while both methods achieve similar performance gains, our temporal split significantly outperforms the random split in terms of stability. A comparison of stability using the robustness score is provided in \cref{sec:appendix_result}.
\section{What is Lost in Temporal Training?}
\label{sec:what_is_lost}

Looking back to the MMD heatmap in \cref{fig:mmd} right, we observe that the original data distribution offers a rich source of temporal information, including the periodicity and the trend. \cref{fig:periodic} left presents a detailed MMD heatmap visualization of the Homesite Insurance dataset, revealing both yearly and weekly periodicity, as well as the underlying trend.
We further investigate the impact of temporal shifts on deep tabular methods from the perspective of feature representations.

Unexpectedly, by comparing the MMD heatmaps of the learned representations of MLP in our training protocol (shown in \cref{fig:periodic} mid), we observe that the periodicity and the trend are lost in the model representations. Instead of the diagonal stripes observed in the original data distribution, the learned representations exhibit only shallow grids and a more uniform distribution, suggesting that temporal information has not been effectively preserved. This indicates that the model has captured the distinction between weekdays and weekends but failed to capture the long-term periodicity and finer details of short-term cycles.

This phenomenon may account for the suboptimal performance of datasets with clear periodic patterns, such as Cooking Time, Delivery ETA, and Maps Routing datasets, which are socially related and exhibit distinct weekly cycles, while the Weather dataset, being influenced by natural cycles, shows a clear yearly pattern. We would expect models using temporal splits to capture long-term periodicity and trends, enabling them to learn extrapolative knowledge. However, this critical knowledge does not seem to be effectively learned. In contrast, random splits appear more proficient at capturing local patterns, particularly those associated with short-term periodicity. This could explain why temporal splitting does not consistently outperform random splitting in these cases. Furthermore, it also explains the reasons why existing methods encounter challenges in dealing with temporal shifts.
\section{Temporal Embeddings}
\label{sec:embedding}

\begin{figure}[t]
  \centering 
  \includegraphics[width=\linewidth]{image/adaptive.pdf}
  \vspace{-22pt}
  \caption{Illustration of adaptive approaches for temporal shift mitigation. Our proposed temporal embedding method offers a lightweight alternative that implicitly enables adaptation by embedding temporal information. This allows the model to learn phase-specific knowledge and adjust the mapping \( f_t \) accordingly, thereby achieving temporal generalization. }
  \label{fig:adaptive}
  \vspace{-3pt}
\end{figure}
\begin{table}[t]
  \centering
  \setlength{\tabcolsep}{4pt}
\begin{tabular}{lcccc}
    \toprule
    \textbf{Emb.} & MLP & MLP-PLR & ModernNCA & \textbf{Avg. Imp.} \\
    \midrule
    Num & $-$0.04\% & $-$0.06\% & $-$0.04\% & $-$0.05\% \\
    Time & $-$0.70\% & $-$0.15\% & $-$0.32\% & $-$0.39\% \\
    PLR & $+$0.70\% & \textbf{$+$0.01\%} & $+$0.02\% & $+$0.25\% \\
    Ours & \textbf{$+$1.31\%} & \textbf{$+$0.01\%} & \textbf{$+$0.30\%} & \textbf{$+$0.54\%} \\
    \bottomrule
    \end{tabular}
  \vspace{-10pt}
  \caption{Performance improvement after adopting different temporal embeddings on TabReD benchmark. The embedding is applied only to timestamps, which are then treated as numerical features for the model. Non-learnable embeddings are generally ineffective, while the learnable PLR embedding provides a slight improvement only when applied to MLP. Our temporal embedding outperforms all other embeddings. Detailed results are provided in \cref{sec:appendix_result}.}
  \label{tab:ablation}
  \vspace{-15pt}
\end{table}

Building on the above analysis, it is essential to incorporate temporal information into the model in a manner that effectively captures the underlying temporal dependencies. To address this issue, we propose a \textit{lightweight, plug-and-play temporal embedding method} specifically designed for timestamps, aiming to investigate whether providing explicit temporal information through embedding can lead to performance improvements. Empirically, timestamps are often treated as noise and discarded. However, in the context of temporal shifts and temporal splitting, timestamps likely contain crucial temporal information that enables the model to align with the periodicity and trends inherent in the temporal sequence. Existing works have also emphasized that, in certain contexts, timestamps serve as a valuable feature \cite{wang2024rethinking, zeng2024howmuch}.

Our temporal embedding also serves as an \textit{adaptive model-based approach}, as the model can learn temporal stage-specific knowledge by leveraging timestamp information, thereby adaptively adjusting its mapping at different temporal stages after deployment. 
Formally, the model’s objective function can be written as \( f = g \circ h\), where \( g \) maps the input features to the same representation space. 
When the data exhibits temporal shifts, this implies that there exists a specific mapping \( f_t = g_t \circ h_t\) at time step \( t \). The model can now learn phase-specific knowledge and adjust the mapping \( f_t \) accordingly, as illustrated in \cref{fig:adaptive}.

In our embedding method, We fit multi-scale periodicity using Fourier series expansion \cite{tancik2020fourier, li2021learnable}. The Fourier series can be expressed as:
\begin{align*}
    f(t) = \sum_{k=1}^{\infty} a_k \cos\left( \frac{2\pi k t}{T} \right) + b_k \sin\left( \frac{2\pi k t}{T} \right).\notag
\end{align*}
By increasing the order of the expansion, we can approximate any continuous periodic function $f$ with period $T$ according to the approximation properties of Fourier series.
Our temporal embedding can be described as follows:
\begin{align*}
    \text{Temporal}(t) = \left[ \text{ReLU}\left( \text{Linear}\left( \text{Periodic}(t) \right) \right), \text{Trend}(t) \right]\notag
\end{align*}
where $t$ is the timestamp, and the two components of the embedding each capture the periodicity and trend of the timestamp, respectively. Those are further defined as
\begin{align*}
    \text{Periodic}(t) = \left[ \text{Fourier}(t, T_1), \dots, \text{Fourier}(t, T_m) \right],\notag
\end{align*}
where $T_i$, $i \in [m]$ are \( m \) given periodicity priors for Fourier-based embedding. In our experiments, we set $T_i$ for yearly, monthly, weekly, and daily periodicity. These periodicity are chosen based on the common temporal patterns observed in the datasets, which effectively capture natural patterns in datasets with inherent temporal correlations, such as Weather, by modeling yearly and daily periodicity, and societal-related temporal patterns, such as monthly and weekly periodicity, in datasets like Homesite Insurance and Cooking Time.
Each Fourier embedding is defined as
\begin{align*}
    \text{Fourier}(t, T) = \left[ \sin\left( \frac{2\pi t k}{T} \right), \cos\left( \frac{2\pi t k}{T} \right) \right],\notag
\end{align*}
for $k \in \{1, 2, \dots, \text{order}\}$, which represents the process of Fourier series expansion. By decomposing multiple periods into the components of a Fourier series and passing them through a learnable linear layer, the process of solving and aggregating parameters $a$ and $b$ in the Fourier series can be simulated, resulting in a set of periodic functions that incorporate multiple cycles. The ReLU activation further facilitates the selection process, ensuring the quality of the learned periodic information.

\begin{figure*}[t]
  \centering 
  \vspace{-5pt}
  \includegraphics[width=\linewidth]{image/performance_comparison_temporal_embedding.pdf}
  \vspace{-28pt}
  \caption{Performance comparison before and after adopting our proposed temporal embedding into our training protocol on the TabReD benchmark. This figure follows the same setup as \cref{fig:random}, allowing for direct comparison. Detailed results are provided in \cref{sec:appendix_result}.}
  \label{fig:temporal}
  \vspace{-10pt}
\end{figure*}

In addition to the periodic component, we also provide an optional trend term for the temporal embedding. When the trend is enabled, the final embedding is augmented with a standardized timestamp, which captures the linear temporal shift beyond the periodic components, represented by
\begin{align*}
    \text{Trend}(t) = \text{z-score}(t).\notag
\end{align*}
To thoroughly evaluate the effectiveness of our embedding method, we designed a series of comparative experiments, including the following configurations:
\begin{itemize}[noitemsep,topsep=0pt,leftmargin=*]
    \item None: Timestamps are not utilized as the baseline.
    \item Num: Treating the timestamp as a single numerical feature and directly inputting it into the model.
    \item Time: Decomposing the timestamp into six numerical features: year, month, day, hour, minute, and second, which introduces partial periodicity information while maintaining a human-readable representation of the timestamp.
    \item PLR: Applying a learnable one-dimension PLR  embedding \cite{gorishniy2022embeddings} to the timestamp, enabling the model to capture temporal patterns adaptively.
\end{itemize}

We choose MLP, MLP-PLR and ModernNCA for comparison, the results are presented in \cref{tab:ablation}. The results indicate that while directly using non-learnable embedding methods can be effective in certain cases, it generally leads to a performance degradation, which is consistent with the common practice of treating timestamps as noise. 
The performance of PLR embedding is not stable, likely because it discards linear trends and struggles to accurately capture periodic patterns. Results from the MLP-PLR method show no significant improvement, which may be attributed to its incompatibility with the existing numerical embedding. 

The models’ performance improvement after adopting our temporal embedding are presented in \cref{fig:temporal}. Most methods demonstrate improvements, highlighting the importance of leveraging temporal information in addressing temporal shift tasks. Notably, MLP, DCNv2, and FT-T show significant improvements, while as previously mentioned, MLP-PLR, TabR, TabM, and ModernNCA exhibit limited gains, likely due to their incompatibility with numerical embeddings. This suggests that temporal features may require dedicated embedding strategies rather than relying on existing numerical embedding approaches. After applying temporal embedding, both ModernNCA and TabR demonstrate strong performance, indicating that with an appropriate training protocol and temporal embedding, even retrieval-based methods, which are typically most affected by distributional shift, can regain their practical utility.

The MMD heatmap of the model representation after adopting temporal embedding is shown in \cref{fig:periodic} right. The patterns are closer to the original data, reflecting that it captures rich and correct temporal information, thus effectively alleviating the loss of temporal information during training. The reappearance of diagonal stripes indicates that the model has learned independent representations for each temporal phase within the period, thereby confirming the adaptive role of temporal embedding.
\section{Conclusion}
In this work, we propose a simple yet effective approach, called SMILE, for graph few-shot learning with fewer tasks. Specifically, we introduce a novel dual-level mixup strategy, including within-task and across-task mixup, for enriching the diversity of nodes within each task and the diversity of tasks. Also, we incorporate the degree-based prior information to learn expressive node embeddings. Theoretically, we prove that SMILE effectively enhances the model's generalization performance. Empirically, we conduct extensive experiments on multiple benchmarks and the results suggest that SMILE significantly outperforms other baselines, including both in-domain and cross-domain few-shot settings.

% \section*{Impact statement}

This paper proposes that machine learning can and should be used to maximize social welfare. In principle, and by construction, the impact of our proposed framework on society aims to be positive. But our paper also points to the inherent difficulties of identifying, and making formal, what `good for society' is. We lean on the field of welfare economics, which has for decades contended with this challenge, for ideas on how the learning community can begin to approach this daunting task.
However, even if these ideas are conceptually appealing,
the path to practical welfare improvement presents many challenges---%
some expected, others unforseen.
% and will likely include many ups and downs.
For example, we may specify incorrect social welfare functions;
or we may specify them correctly but be unable to optimize them appropriately;
or we may be able to optimize but find that 
our assumptions are wrong, that theory differs from practice,
or that there were other considerations and complexities that we did not take into account.
For this we can look to other related fields---%
such as fairness, privacy, and alignment in machine learning---%
which have taken (and are still taking) similar journeys,
and learn from both their success and mistakes.
% and hope that ours will be similar.

Any discipline that seeks to affect policy should do so with much deliberation and care. Whereas welfare economics was designed with the explicit purpose of supporting (and influencing) policymakers,
machine learning has found itself in a similar position, but likely without any planned intent.
On the one hand, adjusting machine learning to support notions, such as social welfare,
that it was not designed to support initially can prove challenging.
However, and as we argue throughout, we believe that building on top of existing machinery is a more practical approach than to begin from scratch.
The necessity of confronting with welfare consideration can also
be an opportunity---as we can leverage these novel challenges
to make machine learning practice more informed, transparent, responsible, and socially aware.


% At the same time, the novelty of the challenges that welfare considerations present to the field make this an opportunity---%
% for chaning the role of machine learning in society for the better in a manner that is informed, transparent, and aware.



% In the unusual situation where you want a paper to appear in the
% references without citing it in the main text, use \nocite
\nocite{rubachev2024tabred}

\bibliography{ref}
\bibliographystyle{icml2025}

%%%%%%%%%%%%%%%%%%%%%%%%%%%%%%%%%%%%%%%%%%%%%%%%%%%%%%%%%%%%%%%%%%%%%%%%%%%%%%%
% APPENDIX

\newpage
\appendix
\onecolumn

\section{Dataset Explanation}
\label{sec:appendix_dataset}

The detailed information about the dataset is provided in \cref{tab:dataset}. We used the dataset from TabReD \cite{rubachev2024tabred} in its entirety and applied the same preprocessing method.

\begin{table}[ht]
\vspace{-5pt}
\centering
\setlength{\tabcolsep}{5pt}
\begin{tabular}{clrrcl}
\toprule
Abbr. & Dataset        &  Samples &  Features & Task Type & Task Description \\ \midrule
HI & Homesite Insurance       & 260753      & 296         & Classification          & Insurance plan acceptance prediction \\ 
EO & Ecom Offers              & 160057       & 119         & Classification          & Predict whether a user will redeem an offer \\ 
HD & HomeCredit Default       & 381664 & 696        & Classification          & Loan default prediction \\ 
SH & Sberbank Housing         & 28321      & 387         & Regression          & Real estate price prediction \\ 
CT & Cooking Time             & 319986 & 195       & Regression             & Restaurant order cooking time estimation \\
DE & Delivery ETA             & 350516  & 225       & Regression             & Grocery delivery courier ETA prediction \\ 
MR & Maps Routing             & 279945 & 1026      & Regression             & Navigation app ETA from live road-graph features \\
WE & Weather                  & 423795  & 98        & Regression             & Weather prediction (temperature) \\ \bottomrule
\end{tabular}
\vspace{-10pt}
\caption{Overview of Datasets. Task descriptions from \citet{rubachev2024tabred}.}
\label{tab:dataset}
\vspace{-5pt}
\end{table}

In \cref{fig:split}, \cref{tab:stability}, and \cref{tab:ablation}, we compare the average performance improvement for each method under different strategies ({\it e.g.} splitting or temporal embedding), specifically the percentage increase in AUC on classification datasets and the percentage decrease in RMSE on regression datasets. In \cref{fig:random} and \cref{fig:temporal}, we present a comparison of performance across different training protocols and methods, where all results are reported as performance improvements relative to the MLP performance under the original split in \cite{rubachev2024tabred}. Since performance improvements between methods can often be influenced by outliers, we apply a robust average, excluding the maximum and minimum performance improvements across the eight datasets before calculating the mean.

The only distinction in the implementation is that TabReD employs different embedding methods for numerical and categorical features for each method-dataset pair. Specifically, it uses identity or noisy-quantile encoding for numerical features and one-hot or ordinal encoding for categorical features.
To ensure a fair comparison, we reproduced the experiment from TabReD by removing numerical embeddings and fixing categorical embeddings to one-hot encoding where necessary. This adjustment is essential for accurately evaluating the performance of the methods and for advancing future research on temporal embeddings.

It is important to note that the HomeCredit Default dataset suffers from severe class imbalance, which makes it challenging for methods with limited feature extraction capabilities, such as MLP, SNN \cite{klambauer2017self}, and DCNv2 \cite{wang2021dcn}, to perform well without additional numerical feature embeddings.
While the AUC of the naive MLP (as well as SNN and DCNv2 methods) on this dataset drops to around 0.55, rendering the comparison with other methods (which typically achieve an AUC greater than 0.80) meaningless, it is noteworthy that the insights on training protocols and temporal embeddings explored in this work still lead to significant performance improvements on this dataset, as shown in \cref{sec:appendix_result}.

\begin{figure*}[b]
  \vspace{-10pt}
  \centering
  \includegraphics[width=0.23\linewidth]{image/sample/1.png}
  \includegraphics[width=0.23\linewidth]{image/sample/2.png}
  \includegraphics[width=0.23\linewidth]{image/sample/3.png}
  \includegraphics[width=0.23\linewidth]{image/sample/4.png} \\
  \includegraphics[width=0.23\linewidth]{image/sample/5.png}
  \includegraphics[width=0.23\linewidth]{image/sample/6.png}
  \includegraphics[width=0.23\linewidth]{image/sample/7.png}
  \includegraphics[width=0.23\linewidth]{image/sample/8.png}
  \vspace{-10pt}
  \caption{The temporal sampling distribution for TabReD datasets. The time slices are divided into half days ({\it i.e.}, 12 hours).}
  \label{fig:sample}
\end{figure*}

At the same time, our experiment on the equivalence (\cref{fig:split}) of the validation set shows that the datasets whose experimental results seriously fail to meet the equivalence assumption, such as Sberbank Housing and Ecom Offers, are confirmed to have the most serious nonuniformity in sampling. The temporal sampling distribution for each dataset is shown in \cref{fig:sample}. 
Instead of focusing on the time span between the training and validation sets, we concentrate on ensuring that the number of instances in both sets is equal across different splits, thereby enabling a fair comparison of the splitting strategies. When the temporal sampling distribution becomes highly non-uniform, the observation in \cref{fig:mmd} no longer holds, as it relies on a constant time span for the validation set. Consequently, our insights regarding the equivalence of the validation sets are significantly affected in such situations.

This indicates that the conclusion regarding the equivalence of the validation set on different temporal direction is actually stronger, especially for datasets with uniform sampling over time.
\section{Experimental Setup}
\label{sec:appendix_setup}

We adopt training, evaluation and tuning setup from \citet{ye2024closer} and \citet{liu2024talent}. We tune hyper-parameters using Optuna \cite{akiba2019optuna}, performing 100 trials for most methods to identify the best configuration. The hyper-parameter search space follows exactly the settings in \citet{rubachev2024tabred}. Using these optimal hyper-parameters, each method is trained with 15 random seeds, and the average performance across seeds is reported. We label the attribute type (numerical or categorical) for gradient boosting methods such as CatBoost. For all deep learning methods, we use a batch size of 1024 and
AdamW \cite{loshchilov2019decoupled} as the optimizer.

We followed \citet{rubachev2024tabred} and performed only 25 hyper-parameter tuning runs for FT-T and TabR, as these methods exhibit lower efficiency on datasets with large feature dimensions and sample sizes. For our temporal embedding, we conducted separate hyper-parameter searches for the periodic order and linear trend. However, since 25 tuning trials

For classification tasks, we evaluate models using AUC (higher is better) as the primary metric and use RMSE (lower is better) for regression tasks to select the best-performing model during training on the validation set.

To ensure the validity of random splitting, each group of random split experiments was tested on three distinct random splits, with 15 random seeds run on each split. The mean performance across these runs is reported as the final result. The variance of the random split is calculated based on all 45 results (3 splits × 15 seeds), as the random split is subject to variance from both the split selection and the running seeds during the training phase. This approach better reflects the overall stability of the standard procedure.
\begin{table*}[th]
    \caption{
    Coding accuracy across various tasks and backbones using automatic framework before the agent discussion.}
    \centering
    \resizebox{\textwidth}{!}{
    \begin{tabular}{l|ccccccc}
        \toprule
        \textbf{Backbone (w/o intervention)} & \textbf{BCD-PT} & \textbf{BCD-D} & \textbf{CES} & \textbf{CN-NES} & \textbf{CN-NP} & \textbf{FWPE} & \textbf{PIS} \\ 
        \midrule\midrule
        GPT-4O                     & \textbf{0.41$^\star$} & 0.36 & 0.58 & \textbf{0.79} & 0.58 & 0.91 & 0.87 \\ 
        GPT-4O w/ COT              & 0.24 & 0.32 & 0.55 & 0.71 & 0.23 & 0.91 & 0.82 \\ 
        GPT-4O w/ TOT              & \textbf{0.41} & \textbf{0.38} & 0.50 & 0.72 & 0.35 & 0.91 & 0.87 \\ 
        GPT-4O w/ self-consistency & \textbf{0.41} & 0.38 & \textbf{0.63} & 0.78 & \textbf{0.60} & \textbf{0.92} & \textbf{0.90} \\ 
        \midrule
        GPT-4O-mini                & 0.27 & 0.37 & 0.53 & 0.73 & 0.32 & 0.87 & 0.81 \\ 
        GPT-4O-mini w/ COT         & 0.11 & 0.36 & 0.53 & 0.69 & 0.42 & 0.85 & 0.66 \\ 
        GPT-4O-mini w/ TOT         & 0.24 & 0.32 & \textbf{0.55} & \textbf{0.78} & \textbf{0.47} & \textbf{0.89} & 0.81 \\ 
        GPT-4O-mini w/ self-consistency
                                   & \textbf{0.32} & \textbf{0.39} & \textbf{0.55} & 0.76 & 0.37 & 0.88 & \textbf{0.84} \\ 
        \bottomrule
    \end{tabular}
    }\\
    \vspace{1mm}
    {\footnotesize \raggedright $^\star$\, Bold values indicate the best performance in each model category.}
    \vspace{-2mm}
    \label{table:appendix_result}
\end{table*}

\end{document}