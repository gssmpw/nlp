\section{Related Work}
% 总起讲这一节回顾了什么
This section reviews previous work on lane-changing prediction and SNN. We first discuss the development of lane-changing prediction methods, comparing kinematic or kinetic models and behavioral models with learning-based approaches, and highlight their limitations. Next, we introduce SNN, emphasizing its advantages and applications, particularly in traffic and autonomous vehicle domains.
\subsection{Lane-Changing Intention Prediction}
% 换道预测是指在交通场景中,预测其他车辆是否会进行换道操作以及何时进行换道的任务。
Lane-changing prediction refers to predicting whether and when other vehicles will execute a lane change in a traffic environment.
% 早期的换道预测方法主要基于运动学或动力学模型,通过数学公式进行建模, 例如
% XX使用极坐标多项式对车辆轨迹进行拟合并做出预测,但由于多项式无法充分捕捉复杂的非线性动态,导致无法完全拟合车辆的真实轨迹。
% 非线性动力学模型和模型预测控制的应用确保了纵向安全和横向操控稳定,但这种方法未考虑交通参与者的动态预测,限制了其在复杂交通环境中的实际应用。
Early lane-changing prediction approaches were primarily based on kinematic or kinetic models, typically using mathematical formulas for modeling. For example, \cite{nelson1989continuous} used polar coordinate polynomials to fit vehicle trajectories and predict their motion, demonstrating an effective mathematical approach for trajectory prediction. Nonlinear dynamic models and model predictive control were utilized to ensure longitudinal safety and lateral stability in lane-changing scenarios, showcasing their potential in maintaining vehicle control under various conditions \cite{liu2018dynamic}.
% 此外, 行为模型在换道预测中也得到了广泛验证, 例如  研究重点模拟了驾驶员的行为决策过程,通过明确选择目标车道并评估换道安全性的方法提高了换道预测的准确性,但在复杂的交通场景下的适用性不足。
Besides, behavioral modeling methods were also widely utilized. For example, \cite{treiber2009modeling} modeled the driver’s decision-making process and improved lane-changing prediction accuracy by explicitly selecting target lanes and evaluating lane-change safety.
% 这些模型严重依赖模型的复杂度, 高复杂度模型虽然实现了高预测准确率但在部署实时性上面临严重不足
These models achieve high prediction accuracy by relying heavily on their complexity, however, the computational demands make them unsuitable for resource-constrained onboard deployment in real-time applications.
% 近年来,基于学习的方法逐渐成为换道预测的主流。例如
In recent years, learning-based methods have gradually become the mainstream in lane-changing prediction.
% xx使用了多层感知机 MLP学习驾驶轨迹并进行换道预测, 实现了较高的准确率, 但是无法预测完整路径的换道预测
% 结合双向 RNN和 LSTM 对时间序列驾驶数据进行学习也能够对高速公路换道意图进行准确的预测, 然而模型的复杂使得其部署效率无法满足实时需求
% XX 使用了回声状态网络(Echo State Networks, ESNs)和主成分分析(PCA)对自然驾驶数据进行学习,实现了对换道行为的高精度预测。然而,这种方法依赖于驾驶员相关数据,在预测周围车辆(SV)的换道意图时存在局限性。
% 针对驾驶数据不足的问题,xxx 提出了一种基于在线迁移学习的换道意图预测方法,在数据量有限的情况下成功实现了目标车辆换道意图的检测,但在实际应用中,尤其是右侧换道行为的预测准确性仍有待提升。
For example, \cite{tomar2010prediction} employed a multilayer perceptron to learn driving trajectories and predict lane changes, demonstrating the potential of neural networks for capturing lane-changing patterns. Similarly, the combination of bidirectional RNN and LSTM proved effective for predicting highway lane-changing intentions using time-series driving data, showcasing the strength of temporal models in handling sequential information \cite{xing2020ensemble}.
\cite{griesbach2020prediction} advanced lane-changing prediction by applying ESN and Principal Component Analysis to naturalistic driving data, leveraging driver-specific data for precision. Additionally, \cite{zhang2021target} introduced an online transfer learning approach that efficiently predicted lane-changing intentions even with limited data, expanding the applicability of learning-based methods in diverse scenarios.
% overall, 这些基于学习的方法无法实现高效的训练和部署与高预测准确率之间的有效平衡
Overall, these learning-based methods fail to achieve both efficient training and deployment and high prediction accuracy at the same time.
Therefore, a deployment-friendly approach for efficient lane-changing intention prediction is needed.

\begin{figure*}[t]
    \centering
    \includegraphics[width=0.85\textwidth]{framework.png}
    \caption{Schematic of the SNN model for lane-changing intention prediction. The model processes a time-series input of vehicle state features through four main components: (1) Event Data Construction to generate the input matrix, (2) Feature Extraction with a Linear layer to expand feature dimensions, (3) Temporal Modeling using a LIF layer to capture temporal dependencies, and (4) Classification where a Linear layer maps the features to lane-change intention categories, followed by a Softmax activation for prediction.}
    \label{framework}
\end{figure*}

\subsection{Spiking Neural Networks}
% 脉冲神经网络(SNN)是一种高效、低功耗且具有生物可解释性的神经网络,特别适合处理时间序列数据和资源受限的实时应用。
SNN are efficient, low-power, and biologically interpretable neural networks that are particularly suited for processing time-series data and real-time applications in resource-constrained environments.
% 近年来 SNN 在很多领域得到了广泛的应用
% Fang等人在时间序列分类任务中使用脉冲神经网络(SNN),通过设计稀疏时空脉冲编码方案和训练算法,在低功耗的前提下实现了与深度神经网络相当的分类性能。
% Kim等人在动态视觉传感(DVS)任务中使用脉冲神经网络(SNN),实现了深层SNN在多个基准数据集上的最先进优化性能。
% []等人通过使用SNN完成视觉识别任务,在SNN导向的数据集上表现出更优的性能,并显著降低了能耗。
% []在复杂视觉识别任务中提出了一种新的 ANN-SNN 转换机制,在确保识别准确率的同时显著提升了训练效率。
In recent years, SNN have been widely applied in many fields. For example, \cite{fang2020multivariate} utilized SNN for time series classification and designed a sparse spatiotemporal spike encoding scheme and training algorithm, achieving classification performance comparable to deep neural networks under low-power conditions. \cite{kim2021optimizing} utilized SNN for dynamic vision sensing (DVS) optimization tasks, achieving optimal performance for deep SNN across multiple benchmark datasets. \cite{deng2020rethinking} used SNN for visual recognition tasks and demonstrated superior performance on SNN-specific datasets with a significant reduction in energy consumption. \cite{sengupta2019going} proposed a new ANN-SNN conversion mechanism for complex visual recognition tasks that ensured recognition accuracy and significantly improved training efficiency.
% 脉冲神经网络在交通领域的应用
In the field of transportation, SNN have also garnered significant attention.
% 在其他交通参与者的分类任务中,SNN与事件摄像头的结合实现了低功耗和低延迟的高效分类性能。
% []在自动驾驶车辆的车道保持任务中应用SNN,并结合STDP实现了快速学习,展现出更高的定位精度,同时显著降低了计算能耗。
In the classification tasks of other traffic participants, the combination of SNN and event-based cameras demonstrated efficient classification capabilities with low power consumption and low latency \cite{viale2021carsnn}.
\cite{lopez2021spiking} proposed novel SNN for AV radar signal processing, which achieved efficient processing performance and maintained low power consumption in simulated driving scenarios.
\cite{bing2020indirect} applied SNN to lane-keeping tasks to achieve fast learning, which improved positioning accuracy and significantly reduced computational energy consumption.
% 总而言之 xx
In summary, SNN has shown great promise in transportation applications, meeting the critical demands of autonomous driving for fast, energy-efficient deployment.