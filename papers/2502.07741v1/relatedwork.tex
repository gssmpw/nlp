\section{Related Work}
Accurate modeling of the Greenland Ice Sheet requires understanding the interplay of highly interconnected climate variables. Traditional univariate anomaly detection methods, while effective for specific analysis, fail to capture critical insights arising from interactions between variables \cite{flach2017multivariate}. Multivariate analysis addresses this limitation by identifying anomalies resulting from the interaction of multiple variables.

Since no real-world scenario happens in isolation, coupled with the recent advancements in computational techniques, the multivariate analysis of extreme climate events has become significant to increase our understanding. The Greenland Ice Sheet exhibits strong interaction among multiple variables. For instance, the significant mass loss and melting in 2019 were partially due to low snowfall during the preceding winter \cite{nsidc2019} and an unprecedented rainfall at the summit of Greenland ice sheet in 2021 triggered a three-day melt episode \cite{overland2022arctic}, demonstrating how interconnected variables drive surface snow melt. Multivariate analysis offers further insights into these dynamics. For example, Kao et al. (2009) \cite{kao2009motivating} showcased the utility of copula-based methods in modeling complex dependencies between variables in hydrometeorological data, enabling the detection of compound anomalies often missed by traditional univariate methods. Additionally, Flach et al. (2017) \cite{flach2017multivariate} identifies workflow for detecting anomalous patterns in multivariate Earth observation by highlighting the importance of feature extraction steps.  

G Pang et al. (2021) \cite{pang2021deep} classify anomaly detection using deep learning into feature extraction, representation learning, and end-to-end detection method. Representation learning techniques, such as Variational Autoencoders (VAEs), model complex data distributions in an unsupervised approach, making it suitable for anomaly detection in multivariate climate datasets to capture intricate relationships between variables.
Deep learning methods have significantly advanced multivariate anomaly detection. For example, variational autoencoders (VAEs) \cite{an2015variational, sinha2020variational}, transformers \cite{tuli2022tranad, wang2022variational}, long short-term memory (LSTM) networks \cite{lindemann2021survey, ergen2019unsupervised}, and convolutional neural networks (CNNs) \cite{thill2021temporal, he2019temporal} have all demonstrated the ability model complex dependencies across variables. A review of deep learning techniques by Li and Jung (2023) \cite{li2023deep} categorizes anomalies into specific time points, intervals, or entire series. They also highlighted the need for methods that detect anomalies and explain the underlying causes.

Explainability is a critical component of anomaly detection. Understanding the drivers of anomalies is often as important as their detection. Attention mechanisms focus on the most relevant features in sequential data \cite{wang2022variational, kotipalli2024role}, providing interpretable insights into anomaly drivers through visualizations such as heatmaps. This visualization enables our understanding of the model's process. Post-hoc explainability methods, such as SHAP values \cite{lundberg2020local} and LIME \cite{garreau2020explaining}, can be integrated into anomaly detection pipelines to rank feature importance, particularly useful in high-dimensional data. Trifunov et al. (2021) \cite{trifunov2021anomaly} implemented a counterfactual reasoning approach leveraging Maximally Divergent Interval (MDI) \cite{barz2018detecting}, this method simulates "what-if" scenarios that were used for attributing anomalies to specific variables.

Multivariate climate datasets, characterized by strong temporal dependencies, require sophisticated methods to isolate the contribution of individual variables. Integrating deep learning models with domain knowledge enhances the interpretability and accuracy of anomaly attribution in climate systems \cite{jiang2024interpretable, yang2024interpretable}. Building on these advances, our research extends the application of anomaly detection and attribution to the Greenland Ice Sheet by combining counterfactual reasoning method with a Cluster-LSTM-VAE framework \cite{ale2024harnessing}. This approach identifies anomalies and attributes them to specific climate features. It provides actionable insights into surface melt dynamics at the atmosphere interface and validates the method against established techniques such as SHAP values and Random Forest-based feature importance.