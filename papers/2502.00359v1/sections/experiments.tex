\section{Experiment}

Through extensive experiments, we aim to validate the following questions:

\begin{compactitem}
  \item Does the latent space of our VAE possess richer semantics and a more structured arrangement compared to traditional VAE spaces?
  \item Is the representation aligned latent space beneficial for generation?
  \item Can the diffusion model trained on the ReaLS effectively perform downstream tasks?
\end{compactitem}

\subsection{Experimental Setup}

\paragraph{Implementation Details.} Our model training is divided into two phases. The first phase is VAE training, followed by latent diffusion training in the second phase. In the first phase, we load SD-VAE~\cite{rombach2022high} which is widely used in LDMs as the pre-trained parameters and then train it on ImageNet~\cite{deng2009imagenet}. We employ DINOv2-large-reg~\cite{oquab2023dinov2} as our semantic extraction model, and utilize a two-layer MLP with GeLU activation functions as the alignment network. In the second phase, we strictly adhere to the training methods of DiT and SiT to ensure a fair comparison. \cref{hyperpara} presents the hyperparameters used in both training phases.


\begin{table}[ht]
\vskip -0.15in
\centering
% \renewcommand{\arraystretch}{0.9}
\setlength{\tabcolsep}{5pt}
\small % \scriptsize % \scriptsize %\footnotesize % \tiny
\caption{\textbf{Training Hyperparameter.}}
% \begin{center}
% \begin{small}
% \begin{sc}
% \resizebox{7cm}{!}{
\begin{tabular}{cccccc}
\toprule
Optimizer & lr   & Schedule & min\_lr & Batch Size & Epoch \\ \midrule
\multicolumn{6}{c}{VAE Training}                           \\ 
AdamW     & 5e-5 & Cosine      & 0.0     & 64         & 10    \\ \midrule
\multicolumn{6}{c}{DiT/SiT Training}                       \\ 
AdamW     & 1e-4 & -        & -       & 256        & -     \\ 
\bottomrule
\end{tabular}
% }
% \end{sc}
% \end{small}
% \end{center}
\label{hyperpara}
\vskip -0.1in
\end{table}




\paragraph{Evaluation.} To validate the semantic capability of the VAE, we designed a new metric based on the latent similarity after different augmentations, denoted as semantic consistency (SC). Its calculation is shown in~\cref{alg:sc}.

\begin{algorithm}[h]
\caption{Semantic Consistency (SC)}
\label{alg:sc}
\begin{algorithmic}
\STATE $x_1 \gets \text{RandomAugmentation}(x)$
\STATE $x_2 \gets \text{RandomAugmentation}(x)$
\STATE $z_1 \gets \text{VAE.encode}(x_1)$
\STATE $z_2 \gets \text{VAE.encode}(x_2)$
\STATE $\text{SC} \gets \text{CosineSimilarity}(z_1, z_2)$
\end{algorithmic}
\end{algorithm}

For traditional VAEs, since they merely compress images, the differences in pixel values after applying two different data augmentations lead to different latent representations for the same image, resulting in a lower SC value. In contrast, our VAE incorporates semantic information, so although the images undergo different data augmentations, their semantics do not change significantly. Therefore, $z_1$ and $z_2$ are closer together in the latent space.

For the generative model, we evaluate its quality with Fréchet Inception Distance (FID)~\cite{fid}, sFID, Inception Score (IS), precision (Pre.), and recall (Rec.), with all metrics assessed on the generated 50,000 samples.

\paragraph{Sampler.} We use the SDE Euler sampler with 250 steps for SiT and set the last step size to 0.04.

\paragraph{Baseline.} We use DiT~\cite{dit} and SiT~\cite{sit} as baseline models. Specifically, we trained four models, that is DiT-B/2, SiT-B/2, SiT-L/2, and SiT-XL/2, on our VAE. These models did not undergo any modifications to their network architecture or hyperparameters.

\subsection{Representation Aligned Latent Space}

We provide evidence from three experiments that our VAE latent space contains richer semantic information. 

% 第一,vae latents的聚类可视化
First, we randomly select 10 categories from ImageNet, with 128 images per category, and obtain their latent representations through VAE encoding, which are then reduced in dimensionality using t-SNE. The visualization in Figure~\ref{fig:cluster} clearly shows that, compared to traditional VAEs, our VAE exhibits significant clustering of categories in the latent representations. This indicates that our latent space has better structural properties, with images from the same category being closer together in the space.

% 第二,attention map可视化
Second, we visualize the attention map between one token $z_{ij}$ and all tokens from the VAE latents. The visualization results in Figure~\ref{fig:cluster} show that Our VAE preserves more semantic information in latent space, with tokens from the same object exhibiting higher similarity.

% 第三,用刚才设计的“语义一致性”指标做个定量分析
Third, we conduct a quantitative analysis of the semantic invariance of our VAE compared to traditional VAEs using the SC metric. The calculation of the SC metric is shown in Algorithm~\ref{alg:sc}, where a higher SC value indicates better semantic consistency in the latents. Table~\ref{tab: Semantic consistency} demonstrate that our VAE can extract the similar semantics between two different variants of the same image.

\begin{table}[ht]
\vskip -0.1in
\centering
% \renewcommand{\arraystretch}{0.9}
\setlength{\tabcolsep}{5pt}
\small % \scriptsize % \scriptsize %\footnotesize % \tiny
\caption{\textbf{Semantic Consistency.} A higher SC indicates that more semantic information is retained in the latent.}
% \begin{center}
% \begin{small}
% \begin{sc}
% \resizebox{7cm}{!}{
\begin{tabular}{l|cccccc}
\toprule
Data Aug. & Crop          & Flip          & GaussianBlur  & Grayscale     & All            \\ \midrule
SD-VAE    & 0.33          & 0.34          & 0.41          & 0.37          & 0.29           \\
Ours      & \textbf{0.45} & \textbf{0.44} & \textbf{0.46} & \textbf{0.47} &  \textbf{0.41} \\
\bottomrule
\end{tabular}
% }
% \end{sc}
% \end{small}
% \end{center}
\label{tab: Semantic consistency}
% \vskip -0.1in
\end{table}

The first experiment demonstrates that our VAE exhibits better semantic similarity between samples. The second experiment shows that our VAE has stronger feature attention within individual samples. The third experiment qualitatively indicates that our VAE achieves better semantic consistency with different data augmentations.

\subsection{Enhanced Generation Capability}

We compare the baseline models of DiT and SiT under the same training configuration, with Table~\ref{tab:FID} presenting the experimental results without using classifier-free guidance (cfg). The results indicate that under the same model parameters and training steps, diffusion models trained on ReaLS achieve significant performance improvements. Our approach requires no modifications to the diffusion model training process or additional network structures, providing a cost-free enhancement to the diffusion baseline, with an average FID improvement exceeding \textbf{15\%}.


\begin{table}[ht]
\caption{\textbf{FID Comparisons with Vanilla DiTs and SiTs.} Generate on ImageNet $256\times256$ without classifier-free guidance.}
\centering
% \renewcommand{\arraystretch}{0.9}
% \setlength{\tabcolsep}{1.6pt}
\small % \scriptsize % \scriptsize %\footnotesize % \tiny
\resizebox{8cm}{!}{
\begin{tabular}{lcccccccc}
\toprule
Model & \multicolumn{1}{c}{VAE} & Params & \multicolumn{1}{c}{Steps} & \multicolumn{1}{c}{FID$\downarrow$} & \multicolumn{1}{c}{sFID$\downarrow$} & \multicolumn{1}{c}{IS$\uparrow$} & \multicolumn{1}{c}{Pre.$\uparrow$} & \multicolumn{1}{c}{Rec.$\uparrow$} \\ \midrule
DiT-B-2           & SD-VAE                  & 130M   & 400K                      & 43.5                    &      -                   & -                      & -                        & -                        \\
\textbf{DiT-B/2}  & \textbf{Ours}           & 130M   & \textbf{400K}             & \textbf{35.27}          & \textbf{6.30}            & \textbf{37.80}         & \textbf{0.56}            & \textbf{0.62}            \\ \midrule
SiT-B-2           & SD-VAE                  & 130M   & 400K                      & 33.0                    &     -                    &  -                     &  -                       & -                        \\
\textbf{SiT-B/2}  & \textbf{Ours}           & 130M   & \textbf{400K}             & \textbf{27.53}          & \textbf{5.49}            & \textbf{49.70}         & \textbf{0.59}            & \textbf{0.61}            \\
\textbf{SiT-B/2}  & \textbf{Ours}           & 130M   & \textbf{1M}               & \textbf{21.18}          & \textbf{5.42}            & \textbf{64.72}         & \textbf{0.63}            & \textbf{0.62}            \\
\textbf{SiT-B/2}  & \textbf{Ours}           & 130M   & \textbf{4M}               & \textbf{15.83}        & \textbf{5.25}             & \textbf{83.34}              & \textbf{0.65}             & \textbf{0.63}                         \\ \midrule
SiT-L-2           & SD-VAE                  & 458M   & 400K                      & 18.8                    &      -                   & -                      & -                        &  -                       \\
\textbf{SiT-L/2}  & \textbf{Ours}           & 458M   & \textbf{400K}             & \textbf{16.39}          & \textbf{4.77}            & \textbf{76.67}         & \textbf{0.66}                & \textbf{0.61}                \\ \midrule
SiT-XL-2          & SD-VAE                  & 675M   & 400K                      & 17.2                    &     -                    & -                      & -                        & -                        \\
\textbf{SiT-XL/2} & \textbf{Ours}           & 675M   & \textbf{400K}             & \textbf{14.24}          & \textbf{4.71}            & \textbf{83.83}         & \textbf{0.68}            & \textbf{0.62}            \\
\textbf{SiT-XL/2} & \textbf{Ours}           & 675M   & \textbf{2M}               &    \textbf{8.80}                     & \textbf{4.75}                         & \textbf{118.51}                       & \textbf{0.70}                         & \textbf{0.65}                        \\ \bottomrule
\end{tabular}
}
\label{tab:FID}
\vspace{-1em}
\end{table}


Table~\ref{tab: FID with cfg} displays the generation results of our model with cfg. In the comparative experiments with DiT-B/2 (80 epochs, cfg=1.5) and SiT-B/2 (200 epochs, cfg=1.5), the models trained on ReaLS consistently outperformed traditional VAE space, achieving better FID scores. In the SiT-XL/2 experiment, our model reached an impressive FID of \textbf{1.82} after a relatively low number of training epochs (i.e., 400 epochs).

% \arrayrulecolor{black!30}\midrule

\begin{table}[t]
\caption{\textbf{Generation on ImageNet $256\times256$ with classifier-free guidance.} $*[a,b]$ indicates the use of cfg with the guidance interval~\cite{kynkaanniemi2024applying}.}
\centering
\renewcommand{\arraystretch}{0.8} %0.8
\setlength{\tabcolsep}{1.6pt}
\tiny % \scriptsize % \scriptsize %\footnotesize % \tiny
\resizebox{8cm}{!}{
\begin{tabular}{lcccccc}
\toprule
Model & Epochs & FID$\downarrow$ & sFID$\downarrow$ & IS$\uparrow$ & Pre.$\uparrow$ & Rec.$\uparrow$ \\ \midrule

\multicolumn{7}{c}{\emph{GAN-based Generative Model}} \\ \arrayrulecolor{black!30}\midrule


BigGAN-deep~\cite{brock2018large}     & -       & 6.95            & 7.36             & 171.4        & 0.87                & 0.28             \\ 
StyleGAN-XL~\cite{stylegan}                & -       & 2.30            & 4.02             & 265.12       & 0.78                & 0.53             \\ \arrayrulecolor{black!100}\midrule


\multicolumn{7}{c}{\emph{Autoregressive Generative Model}} \\ \arrayrulecolor{black!30}\midrule
Mask-GIT~\cite{chang2022maskgit}    & 555    & 6.18            & -                & 182.1        & -                   & -                \\
MagViT-v2~\cite{yu2023language}   & 1080   & 1.78            & -                & 319.4        & -                   & -                \\
LlamaGen~\cite{sun2024llamagen}     & 300    & 2.18            & 5.97             & 263.3        & 0.81                & 0.58             \\
VAR~\cite{tian2024var}     & 350    & 1.80            & -                & 365.4        & 0.83                & 0.57             \\
MAR~\cite{li2024autoregressive-mar}     & 800    & 1.55            & -                & 303.7        & 0.81                & 0.62             \\ 

\arrayrulecolor{black!100}\midrule
\multicolumn{7}{c}{\emph{Diffusion Model}} \\ \arrayrulecolor{black!30}\midrule


ADM~\cite{beats_gan}    & -       & 10.94           & 6.02             & 100.98       & 0.69                & 0.63             \\
% ADM-U                      & -       & 7.49            & 5.13             & 127.49       & 0.72                & 0.63             \\
% ADM-G                      & -       & 4.59            & 5.25             & 186.70       & 0.82                & 0.52             \\
ADM-G, ADM-U               & 400     & 3.94            & 6.14             & 215.84       & 0.83                & 0.53             \\ \midrule
Simple Diff~\cite{hoogeboom2023simple}    & -    & 3.76            & -                & 171.6        & -                   & -                \\
Simple Diff(U-ViT, L) & 800       & 2.77            & -                & 211.8        & -                   & -                \\  \midrule

CDM~\cite{ho2022cascaded}                        & 2160   & 4.88            & -                & 158.71       & -                   & -                \\ 

U-ViT-H/2~\cite{uvit}                  & 240    & 2.29            & 5.68             & 263.9        & 0.82                & 0.57             \\ 
VDM++~\cite{kingma2024understanding}                      & 560    & 2.12            & -                & 267.7        & -                   & -                \\



\arrayrulecolor{black!100}\midrule
\multicolumn{7}{c}{\emph{Latent Diffusion Model}} \\ \arrayrulecolor{black!30}\midrule
LDM-8~\cite{rombach2022ldm}   & -       & 15.51           & -                & 79.03        & 0.65                & 0.63             \\
LDM-8-G                    & -       & 7.76            & -                & 209.52       & 0.84                & 0.35             \\
LDM-4                      & -    & 10.56           & -                & 103.49       & 0.71                & 0.62             \\
% LDM-4-G (cfg=1.25)         & -       & 3.95            & -                & 178.22       & 0.81                & 0.55             \\
LDM-4-G (cfg=1.50)         & 200       & 3.60            & -                & 247.67       & 0.87                & 0.48             \\ \midrule
RIN~\cite{jabri2022scalable}                        & -       & 3.42            & -                & 182.0        & -                   & -                \\ 

 \midrule

DiT-B/2 (cfg=1.5)~\cite{dit}          & 80     & 22.21           & -                & -            & -                   & -                \\
\rowcolor{blue!15} \textbf{DIT-B/2 + ReaLS (cfg=1.5)}   & \textbf{80}     & \textbf{19.44}           & \textbf{5.45}             & \textbf{70.37}        & \textbf{0.68}                & \textbf{0.55}             \\ 
DiT-XL/2                   & 1400   & 9.62            & 6.85             & 121.50       & 0.67                & 0.67             \\
DiT-XL/2 (cfg=1.25)      & 1400       & 3.22            & 5.28             & 201.77       & 0.76                & 0.62             \\
DiT-XL/2 (cfg=1.50)      & 1400       & 2.27            & 4.60             & 278.24       & 0.83                & 0.57             \\ 
SD-DiT~\cite{zhu2024sd}                     & 480    & 3.23            & -                & -            & -                   & -                \\ 
FasterDiT~\cite{yao2024fasterdit}                  & 400    & 2.03            & 4.63             & 264.0        & 0.81                & 0.60             \\ 
% 下面这三个结果,投稿版本一定要注释掉
FiT-XL/2~\cite{Lu2024FiT} & 400  & 4.21 & 10.01 & 254.87 & 0.84 & 0.51\\
FiTv2-XL~\cite{wang2024fitv2} & 400 & 2.26 & 4.53 & 260.95 & 0.81 & 0.59 \\
DoD-XL~\cite{yue2024dod} & 400 & 1.73 & 5.14 & 304.31 & 0.79 & 0.64 \\
\midrule

MaskDiT~\cite{zheng2023fast}                    & 1600   & 2.28            & 5.67             & 276.6        & 0.80                & 0.61             \\ 
MDT~\cite{gao2023masked}                        & 1300   & 1.79            & 4.57             & 283.0        & 0.81                & 0.61             \\
MDTv2                      & 1080   & 1.58            & 4.52             & 314.7        & 0.79                & 0.65             \\ \midrule

SiT-B/2 (cfg=1.5)~\cite{sit}          & 200      & 9.3           & -                & -            & -                   & -                \\
\rowcolor{blue!15} \textbf{SiT-B/2 + ReaLS (cfg=1.5)}   & \textbf{200}    & \textbf{8.39}            & \textbf{4.64}             & \textbf{131.97}       & \textbf{0.77}                & \textbf{0.53}             \\
% \rowcolor{blue!15} SiT-B/2 + ReaLS (cfg=2.0)*[0,0.75]  & 200    & 4.58            & 4.70             & 184.51       & 0.80                & 0.55             \\ 
\rowcolor{blue!15} SiT-B/2 + ReaLS (cfg=2.0)  & 650    & 4.38            & 4.52             & 239.08       & 0.86                & 0.46             \\ 
\rowcolor{blue!15} SiT-B/2 + ReaLS (cfg=2.0)*[0,0.75]  & 650    & 2.99            & 4.63             & 222.79       & 0.81                & 0.56             \\ 
\rowcolor{blue!15} SiT-B/2 + ReaLS (cfg=2.25)*[0,0.75]  & 650    & 2.74            & 4.58             & 251.02       & 0.83                & 0.54             \\ 
SiT-XL/2(cfg=1.5, ODE)     & 1400   & 2.15            & 4.60             & 258.09       & 0.81                & 0.60             \\
SiT-XL/2(cfg=1.5, SDE)     & 1400       & 2.06            & 4.49             & 277.50       & 0.83                & 0.59             \\ 
% \rowcolor{blue!15} SiT-XL/2 + ReaLS (cfg=2.0)*[0,0.75]  & 310    & 2.01            & 4.36             & 289.63       & 0.83                & 0.58             \\ 
% \rowcolor{blue!15} SiT-XL/2 + ReaLS (cfg=2.25)*[0,0.75]  & 310    & 2.50            & 4.38             & 319.66       & 0.84                & 0.56             \\ 
% \rowcolor{blue!15} SiT-XL/2 + ReaLS (cfg=1.25)  & 360    & 4.30            & 4.36             & 173.25       & 0.78                & 0.61             \\ 
% \rowcolor{blue!15} SiT-XL/2 + ReaLS (cfg=1.5)  & 360    & 2.93            & 4.25             & 226.31       & 0.82                & 0.56             \\ 
% \rowcolor{blue!15} SiT-XL/2 + ReaLS (cfg=1.75)  & 360    & 3.26            & 4.33             & 271.23       & 0.86                & 0.52             \\ 
% \rowcolor{blue!15} SiT-XL/2 + ReaLS (cfg=2.0)  & 360    & 4.31            & 4.53             & 306.87       & 0.89                & 0.47             \\ 
\rowcolor{blue!15} SiT-XL/2 + ReaLS (cfg=1.25)  & 400    & 4.18            & 4.39             & 175.16       & 0.77                & 0.60             \\ 
\rowcolor{blue!15} SiT-XL/2 + ReaLS (cfg=1.4)  & 400    & 3.08            & 4.29             & 208.60       & 0.81                & 0.58             \\ 
\rowcolor{blue!15} SiT-XL/2 + ReaLS (cfg=1.5)  & 400    & 2.83            & 4.26             & 229.59       & 0.82                & 0.56             \\ 
\rowcolor{blue!15} SiT-XL/2 + ReaLS (cfg=1.7)  & 400    & 3.02            & 4.31             & 266.70       & 0.86                & 0.52             \\ 
\rowcolor{blue!15} SiT-XL/2 + ReaLS (cfg=1.8)*[0,0.75]  & 400    & \cellcolor{yellow!20} 1.82            & 4.45             & 268.54       & 0.81                & 0.60             \\ 
\rowcolor{blue!15} SiT-XL/2 + ReaLS (cfg=2.0)*[0,0.75]  & 400    & 1.98            & 4.36             & 294.52       & 0.82                & 0.59             \\ 

\midrule

DiffiT*~\cite{hatamizadeh2025diffit}                    & -       & 1.73            & -                & 276.5        & 0.80                & 0.62             \\ \midrule
REPA~\cite{yu2024representation}           & 200    & 2.06            & 4.50             & 270.3        & 0.82                & 0.59             \\
REPA                       & 800    & 1.80            & 4.50             & 284.0        & 0.81                & 0.61             \\
REPA*                       & 800    & 1.42            & 4.70             & 305.7        & 0.80                & 0.65             \\
\bottomrule
\end{tabular}
}
\vspace{-2em}
\label{tab: FID with cfg}
\end{table}

\begin{figure*}[t]
    \centering
    \includegraphics[width=0.99\linewidth]{figs/visual_1.jpg}
    \vspace{-1em}
    \caption{\textbf{Visualization results} on ImageNet 256×256, from the SiT-XL/2 + ReaLS, with cfg=4.0.}
    \label{fig:visual_1}
    \vspace{-0.5cm}
\end{figure*}


\subsection{Downstream Tasks.}
\begin{figure*}
    \centering
    \includegraphics[width=0.99\linewidth]{figs/visual_2.jpg}
    \vspace{-1em}
    \caption{\textbf{Training-free Downstream Tasks on Latents.} The diffusion model trained in the representation-aligned latent space naturally possesses stronger semantics, enabling more downstream tasks on latents. The latents generated by diffusion can obtain semantic features through the alignment network used during VAE training, and then multiple modalities of output can be achieved through the corresponding task heads. The first row displays the segmentation results, while the second row shows the depth estimation results.}
    \label{fig:visual_2}
\end{figure*}


By inputting the latents generated by the LDM model into the alignment network during VAE training, we obtain high-dimensional features with rich semantics similar to those of DINOv2. Then, through the segmentation head implemented in the \href{https://github.com/itsprakhar/Downstream-Dinov2}{Github repository}, we can achieve training-free generation of object masks.

Similarly, we use the depth estimation head of the MoGe~\cite{wang2024moge} to achieve training-free depth estimation for generated images. Figure~\ref{fig:visual_2} shows the segmentation mask and depth estimation generated when we use the SiT-XL/2 model to generate images. Downstream tasks involving perception in the latent space are still to be explored, and our approach presents a new possibility for unifying generation and perception.


\subsection{Ablation Studies}
\label{sec:kl analysis}
In the ablation study section, we aim to validate the impact of four key settings on the final generation quality. First, we investigate the effect of different KL weights in the latent loss discussed in Section~\ref{sec:loss}. Second, we explore whether aligning with DINOv2's patch features and cls features can each enhance the generation quality. Third, we examine whether the generation results align with different DINO models affect the final generation quality. Finally, we analyze the impact of different depths of the alignment network on the final generation quality.

% kl weight增大,FID先降低,后上升
During model training, we found that the KL loss weight in the VAE significantly affects the final generation quality. Figure~\ref{fig:kl} illustrates the relationship between the KL weight and the FID of SiT-B/2 at 400k optimization steps. In Section~\ref{sec:loss}, we have analyzed how the KL loss constrains the integrity of the latent space, requiring the overall distribution to approximate a standard normal distribution. In contrast, the alignment loss constrains the position of each sample in this latent space, ensuring that samples with similar semantics are closer together. If we rely solely on alignment loss ($\lambda_k=0$ in the Equation~\ref{equ:latent loss}), the latent space becomes overly dispersed (large standard deviation), hindering generation. Conversely, a high KL weight imposes excessive constraints on the standard normal distribution (small standard deviation), limiting the alignment loss's effectiveness in semantic alignment. Therefore, we ultimately chose a KL weight of $\lambda_k=2e-5$ as our experimental setting.

\begin{figure}[ht]
    \centering
    \includegraphics[width=0.8\linewidth]{figs/kl.jpg}
    \vspace{-1em}
    \caption{\textbf{Impact of KL Constraint on Latent Space and FID.} As the KL weight increases from low to high, the FID initially decreases and then begins to rise again. The size of the point represents the standard deviation of the latent space.}
    \label{fig:kl}
    \vspace{-1.0cm}
\end{figure}

% 使用dino的patch feature和cls feature对于生成都有益
Second, both DINOv2 features positively contribute to the final generation quality, as shown in Table~\ref{Ablation 1}. DINO's patch features represent local semantic characteristics of the image, while DINO's cls features reflect the overall characteristics. They guide the enhancement of semantic quality in the latent space at two different levels.


\begin{table}[ht]
\vskip -0.1in
\caption{\textbf{Impact of Aligning Different Features on Generation (400k Steps).}}
\centering
% \renewcommand{\arraystretch}{0.9}
% \setlength{\tabcolsep}{1.6pt}
\small % \scriptsize % \scriptsize %\footnotesize % \tiny
% \begin{center}
% \begin{small}
% \begin{sc}
% \resizebox{6.5cm}{!}{
\begin{tabular}{l|ccccc}
\toprule
                                   & FID$\downarrow$ &  sFID$\downarrow$ & IS$\uparrow$ & Pre.$\uparrow$ & Rec.$\uparrow$ \\ \midrule
SiT-B-2                            & 33.0            & -                   & -            & -                   & -                \\
\  +DINO patch              & 28.91           & 5.65                & 48.46        & 0.59                & \textbf{0.62}             \\
\  \  +DINO cls & \textbf{27.53}           & \textbf{5.49}                & \textbf{49.70}        & \textbf{0.59}                & 0.61             \\ \bottomrule
\end{tabular}
% }
% % \end{sc}
% \end{small}
% \end{center}
\label{Ablation 1}
\vskip -0.1in
\end{table}

% 使用dino large的效果比dino base的效果好
Third, the generation results from DINOv2-large outperform those from DINOv2-base, as shown in Table~\ref{Ablation 2}. This is expected, as DINOv2-large features higher dimensionality and achieves better self-supervised learning metrics, resulting in richer semantic content and improved generation.


\begin{table}[ht]
\centering
\small
\vskip -0.1in
\caption{\textbf{Impact of Aligning Different DINO Models on Generation (400k Steps).}}
% \begin{center}
% \begin{small}
% \begin{sc}
% \resizebox{7cm}{!}{
\begin{tabular}{l|ccccc}
\toprule
  & FID$\downarrow$ & sFID$\downarrow$ & IS$\uparrow$ & Pre.$\uparrow$ & Rec.$\uparrow$ \\ \midrule
\multicolumn{1}{l|}{DINOv2-base} & 29.23           & 5.73             & 48.80        & 0.57                & \textbf{0.62}             \\
\multicolumn{1}{l|}{DINOv2-large} & \textbf{27.53}           & \textbf{5.49}             & \textbf{49.70}        & \textbf{0.59}                & 0.61             \\ 
\bottomrule
\end{tabular}
% }
% \end{sc}
% \end{small}
% \end{center}
% \vskip -0.1in
\label{Ablation 2}
\vspace{-1em}
\end{table}

% 使用两层的mlp做对齐的效果最好。
Finally, the alignment network composed of two linear layers outperforms both single-layer and four-layer configurations, as shown in Table~\ref{Ablation 3}. A shallow network results in poor alignment, while increasing the number of layers enhances the alignment network's nonlinear fitting ability, which may lead to overfitting of semantic information and consequently reduce the semantic content of the VAE's latent space. Therefore, opting for two linear layers as the alignment network is the optimal choice.

\begin{table}[ht]
\centering
% \renewcommand{\arraystretch}{0.9}
% \setlength{\tabcolsep}{5pt}
\small % \scriptsize % \scriptsize %\footnotesize % \tiny
\vskip -0.1in
\caption{\textbf{Impact of Depth of Align Networks on Generation (400k Steps).}}
% \begin{center}
% \begin{small}
% % \begin{sc}
% \resizebox{7cm}{!}{
\begin{tabular}{l|ccccc}
\toprule
        & FID$\downarrow$ & sFID$\downarrow$ & IS$\uparrow$ & Pre.$\uparrow$ & Rec.$\uparrow$ \\ \midrule
1-layer & 33.66           & 7.12             & 42.96        & 0.53                & \textbf{0.64}    \\
2-layer & \textbf{27.53}  & \textbf{5.49}    & \textbf{49.70} & \textbf{0.59}     & 0.61             \\
4-layer & 29.00           & 6.25             & 47.83        & 0.58                & 0.62             \\ \bottomrule
\end{tabular}
% }
% % \end{sc}
% \end{small}
% \end{center}
\label{Ablation 3}
\vskip -0.1in
\end{table}

\subsection{Discussion}
\label{discussion}
\textbf{Better reconstruction does not necessarily lead to better generation.} Table~\ref{rec} shows the reconstruction metrics of our VAE on ImageNet $256\times256$. Although the reconstruction metrics of our VAE show a slight decline compared to SD-VAE, it provides a semantically rich latent space for the diffusion model, enhancing generation performance. This indicates that higher reconstruction quality does not necessarily lead to better generation results.

\begin{table}[ht]
\vskip -0.15in
\caption{\textbf{Reconstruction Metrics of VAEs.}}
\begin{center}
\begin{small}
% \begin{sc}
\resizebox{7.5cm}{!}{
\begin{tabular}{l|ccccc}
\toprule
Model   & rFID$\downarrow$ & PSNR$\uparrow$ & SSIM$\uparrow$ \\ \midrule
SD-VAE~\cite{rombach2022high}  & 0.74             & 25.68          & 0.820          \\
VQGAN~\cite{esser2021taming}   & 1.19             & 23.38          & 0.762          \\
Ours    & 0.85             & 23.45          & 0.768          \\
\bottomrule
\end{tabular}
}
% \end{sc}
\end{small}
\end{center}
\label{rec}
\vskip -0.1in
\end{table}

\textbf{The representation alignment in latent space and feature space can promote each other.} This paper focuses on aligning the VAE with image semantic representations to provide a better latent space with semantic priors for LDM. Additionally, some works have attempted to enhance image generation quality by incorporating semantics at the feature level, such as REPA~\cite{yu2024representation}. This work improves image generation quality by aligning the features of the diffusion model with image semantic representations.

Both this paper and REPA enhance generation quality through semantic augmentation; however, our approach emphasizes enhancing the latent space, while REPA enhances the LDM. This motivate us to explore the combination of both methods, investigating whether enhancing semantics in both the latent space and diffusion model features could further improve image generation quality.

Therefore, we trained the REPA model on ReaLS. The experimental results are as Table~\ref{repa}. It can be seen that the combination of the two methods yields better generation results than either method alone and significantly surpasses the baseline. After training for 1000k steps, the combined approach achieved a \textbf{30\%} improvement in FID compared to the baseline. These experimental findings further validate the importance of semantic alignment for generative tasks.

\begin{table}[ht]
\centering
% \renewcommand{\arraystretch}{0.9}
\setlength{\tabcolsep}{3.6pt}
\small % \scriptsize % \scriptsize %\footnotesize % \tiny
\vskip -0.15in
\caption{\textbf{Combining ReaLS with REPA.} Representation alignment in both latent space and feature space enhances generation quality, with their combination yielding even better results.}
% \begin{center}
% \begin{small}
% % \begin{sc}
% \resizebox{7.5cm}{!}{
\begin{tabular}{lcccccc}
\toprule
       & Steps & FID$\downarrow$ & sFID$\downarrow$ & IS$\uparrow$ & Pre.$\uparrow$ & Rec.$\uparrow$ \\ \midrule
SiT-B-2 & 400k                & 33.0                    &     -                    &  -                     &  -                       & -                        \\
\ +REPA & 400k & 24.4  & -    & - & -     & -             \\
\ +ReaLS & 400k & 27.53          & 5.49            & 49.70         & 0.59            & 0.61           \\
\ +ReaLS, REPA & 400k & \textbf{23.40}          & \textbf{5.49}            & \textbf{57.55}         & \textbf{0.61}            & \textbf{0.62}           \\ \midrule
SiT-B-2 & 1000k                & 27.31                    &     -                    &  -                     &  -                       & -                        \\
\ +ReaLS, REPA & 1000k & \textbf{18.96}          & \textbf{5.54}            & \textbf{70.57}         & \textbf{0.64}            & \textbf{0.63} \\

\bottomrule
\end{tabular}
% }
% % \end{sc}
% \end{small}
% \end{center}
\label{repa}
% \vskip -0.1in
\end{table}

