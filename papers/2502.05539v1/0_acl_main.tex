% This must be in the first 5 lines to tell arXiv to use pdfLaTeX, which is strongly recommended.
\pdfoutput=1
% In particular, the hyperref package requires pdfLaTeX in order to break URLs across lines.

\documentclass[11pt]{article}

% Change "review" to "final" to generate the final (sometimes called camera-ready) version.
% Change to "preprint" to generate a non-anonymous version with page numbers.
\usepackage{acl}
% \usepackage[review]{acl}

% Standard package includes
\usepackage{times}
\usepackage{latexsym}

% For proper rendering and hyphenation of words containing Latin characters (including in bib files)
\usepackage[T1]{fontenc}
% For Vietnamese characters
% \usepackage[T5]{fontenc}
% See https://www.latex-project.org/help/documentation/encguide.pdf for other character sets

% This assumes your files are encoded as UTF8
\usepackage[utf8]{inputenc}

% This is not strictly necessary, and may be commented out,
% but it will improve the layout of the manuscript,
% and will typically save some space.
\usepackage{microtype}

\usepackage{amsfonts}

% This is also not strictly necessary, and may be commented out.
% However, it will improve the aesthetics of text in
% the typewriter font.
\usepackage{inconsolata}

%Including images in your LaTeX document requires adding
%additional package(s)
\usepackage{graphicx}
\usepackage{hyperref}
\usepackage{url}
\usepackage{algorithm}
\usepackage{algorithmic}
\usepackage{amsmath}
% \usepackage{algpseudocode}
% \usepackage[numbers]{natbib}
% \setcitestyle{numbers}
% \usepackage{natbib}
% \usepackage[english]{babel}

\usepackage{colortbl}
% \usepackage[table,xcdraw]{xcolor}
\usepackage{multirow}
\usepackage{multicol}
\usepackage{float}
\usepackage{booktabs}
\usepackage{tcolorbox}

% If the title and author information does not fit in the area allocated, uncomment the following
%
%\setlength\titlebox{<dim>}
%
% and set <dim> to something 5cm or larger.

\title{SSH: Sparse Spectrum Adaptation via Discrete Hartley Transformation}

% Author information can be set in various styles:
% For several authors from the same institution:
% \author{Author 1 \and ... \and Author n \\
%         Address line \\ ... \\ Address line}
% if the names do not fit well on one line use
%         Author 1 \\ {\bf Author 2} \\ ... \\ {\bf Author n} \\
% For authors from different institutions:
% \author{Author 1 \\ Address line \\  ... \\ Address line
%         \And  ... \And
%         Author n \\ Address line \\ ... \\ Address line}
% To start a separate ``row'' of authors use \AND, as in
% \author{Author 1 \\ Address line \\  ... \\ Address line
%         \AND
%         Author 2 \\ Address line \\ ... \\ Address line \And
%         Author 3 \\ Address line \\ ... \\ Address line}

% \author{Yixian Shen \\
%   Affiliation / Address line 1 \\
%   Affiliation / Address line 2 \\
%   Affiliation / Address line 3 \\
%   \texttt{email@domain} \\\And
%   Second Author \\
%   Affiliation / Address line 1 \\
%   Affiliation / Address line 2 \\
%   Affiliation / Address line 3 \\
%   \texttt{email@domain} \\}
%Yixian Shen,, JIA-HONG HUANG, , Andy D. Pimentel, and Anuj Pathania
\author{
 \textbf{Yixian Shen}, 
    \textbf{Qi Bi}, 
    \textbf{Jia-Hong Huang}, 
    \textbf{Hongyi Zhu}, 
    % \\
    \textbf{Andy D. Pimentel}, 
    \textbf{Anuj Pathania} \\
    University of Amsterdam, Amsterdam, the Netherlands \\
\\
 \texttt{\{y.shen, q.bi, j.huang, h.zhu, a.d.pimentel, a.pathania\}@uva.nl}
}

\begin{document}
\maketitle
\begin{abstract}
Low-rank adaptation (LoRA) has been demonstrated effective in reducing the trainable parameter number when fine-tuning a large foundation model (LLM). However, it still encounters computational and memory challenges when scaling to larger models or addressing more complex task adaptation.
In this work, we introduce \textbf{S}parse \textbf{S}pectrum Adaptation via Discrete \textbf{H}artley Transformation (SSH), a novel approach that significantly reduces the number of trainable parameters while enhancing model performance. 
It selects the most informative spectral components across all layers, under the guidance of the initial weights after a discrete Hartley transformation (DHT). 
The lightweight inverse DHT then projects the spectrum back into the spatial domain for updates. 
Extensive experiments across both single-modality tasks—such as language understanding and generation—and multi-modality tasks—such as visual-text understanding—demonstrate that SSH outperforms existing parameter-efficient fine-tuning (PEFT) methods while achieving 
substantial reductions in computational cost and memory requirements. 
% For instance, during instruction tuning on the LLaMA3.1 8B model, SSH achieves higher accuracy with only 0.048M trainable parameters compared to LoRA's 33.5M, while reducing computational intensity up to 55\% compared to FourierFT. 
% The source code will be publicly available.
\end{abstract}

\section{Introduction}

Video generation has garnered significant attention owing to its transformative potential across a wide range of applications, such media content creation~\citep{polyak2024movie}, advertising~\citep{zhang2024virbo,bacher2021advert}, video games~\citep{yang2024playable,valevski2024diffusion, oasis2024}, and world model simulators~\citep{ha2018world, videoworldsimulators2024, agarwal2025cosmos}. Benefiting from advanced generative algorithms~\citep{goodfellow2014generative, ho2020denoising, liu2023flow, lipman2023flow}, scalable model architectures~\citep{vaswani2017attention, peebles2023scalable}, vast amounts of internet-sourced data~\citep{chen2024panda, nan2024openvid, ju2024miradata}, and ongoing expansion of computing capabilities~\citep{nvidia2022h100, nvidia2023dgxgh200, nvidia2024h200nvl}, remarkable advancements have been achieved in the field of video generation~\citep{ho2022video, ho2022imagen, singer2023makeavideo, blattmann2023align, videoworldsimulators2024, kuaishou2024klingai, yang2024cogvideox, jin2024pyramidal, polyak2024movie, kong2024hunyuanvideo, ji2024prompt}.


In this work, we present \textbf{\ours}, a family of rectified flow~\citep{lipman2023flow, liu2023flow} transformer models designed for joint image and video generation, establishing a pathway toward industry-grade performance. This report centers on four key components: data curation, model architecture design, flow formulation, and training infrastructure optimization—each rigorously refined to meet the demands of high-quality, large-scale video generation.


\begin{figure}[ht]
    \centering
    \begin{subfigure}[b]{0.82\linewidth}
        \centering
        \includegraphics[width=\linewidth]{figures/t2i_1024.pdf}
        \caption{Text-to-Image Samples}\label{fig:main-demo-t2i}
    \end{subfigure}
    \vfill
    \begin{subfigure}[b]{0.82\linewidth}
        \centering
        \includegraphics[width=\linewidth]{figures/t2v_samples.pdf}
        \caption{Text-to-Video Samples}\label{fig:main-demo-t2v}
    \end{subfigure}
\caption{\textbf{Generated samples from \ours.} Key components are highlighted in \textcolor{red}{\textbf{RED}}.}\label{fig:main-demo}
\end{figure}


First, we present a comprehensive data processing pipeline designed to construct large-scale, high-quality image and video-text datasets. The pipeline integrates multiple advanced techniques, including video and image filtering based on aesthetic scores, OCR-driven content analysis, and subjective evaluations, to ensure exceptional visual and contextual quality. Furthermore, we employ multimodal large language models~(MLLMs)~\citep{yuan2025tarsier2} to generate dense and contextually aligned captions, which are subsequently refined using an additional large language model~(LLM)~\citep{yang2024qwen2} to enhance their accuracy, fluency, and descriptive richness. As a result, we have curated a robust training dataset comprising approximately 36M video-text pairs and 160M image-text pairs, which are proven sufficient for training industry-level generative models.

Secondly, we take a pioneering step by applying rectified flow formulation~\citep{lipman2023flow} for joint image and video generation, implemented through the \ours model family, which comprises Transformer architectures with 2B and 8B parameters. At its core, the \ours framework employs a 3D joint image-video variational autoencoder (VAE) to compress image and video inputs into a shared latent space, facilitating unified representation. This shared latent space is coupled with a full-attention~\citep{vaswani2017attention} mechanism, enabling seamless joint training of image and video. This architecture delivers high-quality, coherent outputs across both images and videos, establishing a unified framework for visual generation tasks.


Furthermore, to support the training of \ours at scale, we have developed a robust infrastructure tailored for large-scale model training. Our approach incorporates advanced parallelism strategies~\citep{jacobs2023deepspeed, pytorch_fsdp} to manage memory efficiently during long-context training. Additionally, we employ ByteCheckpoint~\citep{wan2024bytecheckpoint} for high-performance checkpointing and integrate fault-tolerant mechanisms from MegaScale~\citep{jiang2024megascale} to ensure stability and scalability across large GPU clusters. These optimizations enable \ours to handle the computational and data challenges of generative modeling with exceptional efficiency and reliability.


We evaluate \ours on both text-to-image and text-to-video benchmarks to highlight its competitive advantages. For text-to-image generation, \ours-T2I demonstrates strong performance across multiple benchmarks, including T2I-CompBench~\citep{huang2023t2i-compbench}, GenEval~\citep{ghosh2024geneval}, and DPG-Bench~\citep{hu2024ella_dbgbench}, excelling in both visual quality and text-image alignment. In text-to-video benchmarks, \ours-T2V achieves state-of-the-art performance on the UCF-101~\citep{ucf101} zero-shot generation task. Additionally, \ours-T2V attains an impressive score of \textbf{84.85} on VBench~\citep{huang2024vbench}, securing the top position on the leaderboard (as of 2025-01-25) and surpassing several leading commercial text-to-video models. Qualitative results, illustrated in \Cref{fig:main-demo}, further demonstrate the superior quality of the generated media samples. These findings underscore \ours's effectiveness in multi-modal generation and its potential as a high-performing solution for both research and commercial applications.
\section{Related Work}

\subsection{Large 3D Reconstruction Models}
Recently, generalized feed-forward models for 3D reconstruction from sparse input views have garnered considerable attention due to their applicability in heavily under-constrained scenarios. The Large Reconstruction Model (LRM)~\cite{hong2023lrm} uses a transformer-based encoder-decoder pipeline to infer a NeRF reconstruction from just a single image. Newer iterations have shifted the focus towards generating 3D Gaussian representations from four input images~\cite{tang2025lgm, xu2024grm, zhang2025gslrm, charatan2024pixelsplat, chen2025mvsplat, liu2025mvsgaussian}, showing remarkable novel view synthesis results. The paradigm of transformer-based sparse 3D reconstruction has also successfully been applied to lifting monocular videos to 4D~\cite{ren2024l4gm}. \\
Yet, none of the existing works in the domain have studied the use-case of inferring \textit{animatable} 3D representations from sparse input images, which is the focus of our work. To this end, we build on top of the Large Gaussian Reconstruction Model (GRM)~\cite{xu2024grm}.

\subsection{3D-aware Portrait Animation}
A different line of work focuses on animating portraits in a 3D-aware manner.
MegaPortraits~\cite{drobyshev2022megaportraits} builds a 3D Volume given a source and driving image, and renders the animated source actor via orthographic projection with subsequent 2D neural rendering.
3D morphable models (3DMMs)~\cite{blanz19993dmm} are extensively used to obtain more interpretable control over the portrait animation. For example, StyleRig~\cite{tewari2020stylerig} demonstrates how a 3DMM can be used to control the data generated from a pre-trained StyleGAN~\cite{karras2019stylegan} network. ROME~\cite{khakhulin2022rome} predicts vertex offsets and texture of a FLAME~\cite{li2017flame} mesh from the input image.
A TriPlane representation is inferred and animated via FLAME~\cite{li2017flame} in multiple methods like Portrait4D~\cite{deng2024portrait4d}, Portrait4D-v2~\cite{deng2024portrait4dv2}, and GPAvatar~\cite{chu2024gpavatar}.
Others, such as VOODOO 3D~\cite{tran2024voodoo3d} and VOODOO XP~\cite{tran2024voodooxp}, learn their own expression encoder to drive the source person in a more detailed manner. \\
All of the aforementioned methods require nothing more than a single image of a person to animate it. This allows them to train on large monocular video datasets to infer a very generic motion prior that even translates to paintings or cartoon characters. However, due to their task formulation, these methods mostly focus on image synthesis from a frontal camera, often trading 3D consistency for better image quality by using 2D screen-space neural renderers. In contrast, our work aims to produce a truthful and complete 3D avatar representation from the input images that can be viewed from any angle.  

\subsection{Photo-realistic 3D Face Models}
The increasing availability of large-scale multi-view face datasets~\cite{kirschstein2023nersemble, ava256, pan2024renderme360, yang2020facescape} has enabled building photo-realistic 3D face models that learn a detailed prior over both geometry and appearance of human faces. HeadNeRF~\cite{hong2022headnerf} conditions a Neural Radiance Field (NeRF)~\cite{mildenhall2021nerf} on identity, expression, albedo, and illumination codes. VRMM~\cite{yang2024vrmm} builds a high-quality and relightable 3D face model using volumetric primitives~\cite{lombardi2021mvp}. One2Avatar~\cite{yu2024one2avatar} extends a 3DMM by anchoring a radiance field to its surface. More recently, GPHM~\cite{xu2025gphm} and HeadGAP~\cite{zheng2024headgap} have adopted 3D Gaussians to build a photo-realistic 3D face model. \\
Photo-realistic 3D face models learn a powerful prior over human facial appearance and geometry, which can be fitted to a single or multiple images of a person, effectively inferring a 3D head avatar. However, the fitting procedure itself is non-trivial and often requires expensive test-time optimization, impeding casual use-cases on consumer-grade devices. While this limitation may be circumvented by learning a generalized encoder that maps images into the 3D face model's latent space, another fundamental limitation remains. Even with more multi-view face datasets being published, the number of available training subjects rarely exceeds the thousands, making it hard to truly learn the full distibution of human facial appearance. Instead, our approach avoids generalizing over the identity axis by conditioning on some images of a person, and only generalizes over the expression axis for which plenty of data is available. 

A similar motivation has inspired recent work on codec avatars where a generalized network infers an animatable 3D representation given a registered mesh of a person~\cite{cao2022authentic, li2024uravatar}.
The resulting avatars exhibit excellent quality at the cost of several minutes of video capture per subject and expensive test-time optimization.
For example, URAvatar~\cite{li2024uravatar} finetunes their network on the given video recording for 3 hours on 8 A100 GPUs, making inference on consumer-grade devices impossible. In contrast, our approach directly regresses the final 3D head avatar from just four input images without the need for expensive test-time fine-tuning.


\section{Study Design}
% robot: aliengo 
% We used the Unitree AlienGo quadruped robot. 
% See Appendix 1 in AlienGo Software Guide PDF
% Weight = 25kg, size (L,W,H) = (0.55, 0.35, 06) m when standing, (0.55, 0.35, 0.31) m when walking
% Handle is 0.4 m or 0.5 m. I'll need to check it to see which type it is.
We gathered input from primary stakeholders of the robot dog guide, divided into three subgroups: BVI individuals who have owned a dog guide, BVI individuals who were not dog guide owners, and sighted individuals with generally low degrees of familiarity with dog guides. While the main focus of this study was on the BVI participants, we elected to include survey responses from sighted participants given the importance of social acceptance of the robot by the general public, which could reflect upon the BVI users themselves and affect their interactions with the general population \cite{kayukawa2022perceive}. 

The need-finding processes consisted of two stages. During Stage 1, we conducted in-depth interviews with BVI participants, querying their experiences in using conventional assistive technologies and dog guides. During Stage 2, a large-scale survey was distributed to both BVI and sighted participants. 

This study was approved by the University’s Institutional Review Board (IRB), and all processes were conducted after obtaining the participants' consent.

\subsection{Stage 1: Interviews}
We recruited nine BVI participants (\textbf{Table}~\ref{tab:bvi-info}) for in-depth interviews, which lasted 45-90 minutes for current or former dog guide owners (DO) and 30-60 minutes for participants without dog guides (NDO). Group DO consisted of five participants, while Group NDO consisted of four participants.
% The interview participants were divided into two groups. Group DO (Dog guide Owner) consisted of five participants who were current or former dog guide owners and Group NDO (Non Dog guide Owner) consisted of three participants who were not dog guide owners. 
All participants were familiar with using white canes as a mobility aid. 

We recruited participants in both groups, DO and NDO, to gather data from those with substantial experience with dog guides, offering potentially more practical insights, and from those without prior experience, providing a perspective that may be less constrained and more open to novel approaches. 

We asked about the participants' overall impressions of a robot dog guide, expectations regarding its potential benefits and challenges compared to a conventional dog guide, their desired methods of giving commands and communicating with the robot dog guide, essential functionalities that the robot dog guide should offer, and their preferences for various aspects of the robot dog guide's form factors. 
For Group DO, we also included questions that asked about the participants' experiences with conventional dog guides. 

% We obtained permission to record the conversations for our records while simultaneously taking notes during the interviews. The interviews lasted 30-60 minutes for NDO participants and 45-90 minutes for DO participants. 

\subsection{Stage 2: Large-Scale Surveys} 
After gathering sufficient initial results from the interviews, we created an online survey for distributing to a larger pool of participants. The survey platform used was Qualtrics. 

\subsubsection{Survey Participants}
The survey had 100 participants divided into two primary groups. Group BVI consisted of 42 blind or visually impaired participants, and Group ST consisted of 58 sighted participants. \textbf{Table}~\ref{tab:survey-demographics} shows the demographic information of the survey participants. 

\subsubsection{Question Differentiation} 
Based on their responses to initial qualifying questions, survey participants were sorted into three subgroups: DO, NDO, and ST. Each participant was assigned one of three different versions of the survey. The surveys for BVI participants mirrored the interview categories (overall impressions, communication methods, functionalities, and form factors), but with a more quantitative approach rather than the open-ended questions used in interviews. The DO version included additional questions pertaining to their prior experience with dog guides. The ST version revolved around the participants' prior interactions with and feelings toward dog guides and dogs in general, their thoughts on a robot dog guide, and broad opinions on the aesthetic component of the robot's design. 

\section{Experiments}
SSH is compared against state-of-the-art parameter-efficient fine-tuning (PEFT) methods. The experiments are conducted across multiple domains, including single-modality tasks such as natural language understanding (NLU) and natural language generation (NLG), as well as instruction tuning, text summarization, and mathematical reasoning. Additionally, SSH is evaluated on multi-modality tasks, such as vision-language image classification. Finally, an ablation study is performed to assess the effectiveness of our approach.



\subsection{Baselines}
We compare SSH with the following baselines:
\begin{itemize}
    \item \textbf{Full Fine-Tuning (FF):} The entire model is fine-tuned, with updates to all parameters.
    \item \textbf{Adapter Tuning~\cite{houlsby2019parameter,lin2020exploring,ruckle2020adapterdrop,pfeiffer2020adapterfusion}:} Methods that introduce adapter layers between the self-attention and MLP modules for parameter-efficient tuning.
    \item \textbf{LoRA~\cite{hu2022lora}:} A method that estimates weight updates via low-rank matrices.
    \item \textbf{AdaLoRA~\cite{zhang2303adaptive}:} An extension of LoRA that dynamically reallocates the parameter budget based on importance scores.
    \item \textbf{DoRA~\cite{liu2024dora}:} Decomposes pretrained weights into magnitude and direction, using LoRA for efficient directional updates.
    \item \textbf{VeRA~\cite{kopiczko2023vera}:} Employs a single pair of low-rank matrices across all layers, to reduce parameters.
    \item \textbf{FourierFT~\cite{gao2024parameter}:} Fine-tunes models by learning a subset of spectral coefficients in the Fourier domain.
    \item 
    \textbf{AFLoRA~\cite{liu2024aflora}:} Freezes low-rank adaptation parameters using a learned freezing score, reducing trainable parameters while maintaining performance.
    \item 
    \textbf{LaMDA~\cite{azizi2024lamda}:} Fine-tunes large models via spectrally decomposed low-dimensional adaptation, reducing trainable parameters and memory usage while maintaining performance.
    
\end{itemize}

\subsection{Natural Language Understanding}

\begin{table*}[!ht]
\centering
\resizebox{0.85\textwidth}{!}{%
\begin{tabular}{cl|r|ccccccccc}
\toprule
& \textbf{Model} & \textbf{\# Trainable} & \textbf{SST-2} & \textbf{MRPC} & \textbf{CoLA} & \textbf{QNLI} & \textbf{RTE} & \textbf{STS-B} & \multirow{2}{*}{\textbf{Avg.}} \\
& \textbf{\& Method} & \textbf{Parameters} & \textbf{(Acc.)} & \textbf{(Acc.)} & \textbf{(MCC)} & \textbf{(Acc.)} & \textbf{(Acc.)} & \textbf{(PCC)} \\
\midrule
\multirow{9}{*}{\rotatebox{90}{\textbf{BASE}}} 
& FF & 125M & 94.8 & 90.2 & 63.6 & 92.8 & 78.7 & 91.2 & 85.22 \\
& BitFit & 0.1M & 93.7 & \textbf{92.7} & 62.0 & 91.8 & \textbf{81.5} & 90.8 & 85.42 \\
& Adpt\textsuperscript{D} & 0.9M & 94.7 & 88.4 & 62.6 & 93.0 & 75.9 & 90.3 & 84.15 \\
& LoRA & 0.3M & \textbf{95.1} & 89.7 & 63.4 & \textbf{93.3} & 78.4 & \textbf{91.5} & 85.23 \\
& AdaLoRA & 0.3M & 94.5 & 88.7 & 62.0 & 93.1 & 81.0 & 90.5 & 84.97 \\
& DoRA & 0.3M & 94.9 & 89.9 & 63.7 & \textbf{93.3} & 78.9 & \textbf{91.5} & 85.37 \\
& AFLoRA & 0.27M & 94.1 & 89.3 & 63.5 & 91.3 & 77.2 & 90.6 & 84.33 \\
& LaMDA & 0.06M &  94.6 & 89.7 & 64.9 & 91.7 & 78.2 & 90.4 & 84.92 \\
& VeRA & 0.043M & 94.6 & 89.5 & \textbf{65.6} & 91.8 & 78.7 & 90.7 & 85.15 \\
& FourierFT & 0.024M & 94.2 & 90.0 & 63.8 & 92.2 & 79.1 & 90.8 & 85.02 \\
\rowcolor{green!17}
& \textbf{SSH} & \textbf{0.018M} & 94.1 & 91.2 & 63.6 & 92.4 & 80.5 & 90.9 & \textbf{85.46} \\
\midrule
\multirow{8}{*}{\rotatebox{90}{\textbf{LARGE}}} 
& FF & 356M & 96.3 & 90.9 & 68.0 & 94.7 & 86.6 & 92.4 & 88.11 \\
& Adpt\textsuperscript{P} & 3M & 96.1 & 90.2 & \textbf{68.3} & 94.7 & 83.8 & 92.1 & 87.55 \\
& Adpt\textsuperscript{P} & 0.8M & \textbf{96.6} & 89.7 & 67.8 & 94.7 & 80.1 & 91.9 & 86.82 \\
& Adpt\textsuperscript{H} & 6M & 96.2 & 88.7 & 66.5 & 94.7 & 83.4 & 91.0 & 86.75 \\
& Adpt\textsuperscript{H} & 0.8M & 96.3 & 87.7 & 66.3 & 94.7 & 72.9 & 91.5 & 84.90 \\
& LoRA & 0.8M & 96.2 & 90.2 & 68.2 & \textbf{94.8} & 85.2 & 92.3 & 87.82 \\
& DoRA & 0.9M & 96.4 & \textbf{91.0} & 67.2 & \textbf{94.8} & 85.4 & 92.1 & 87.82 \\
& AFLoRA & 0.76M & 96.3 & 90.0 & 67.5 & 94.3 & 86.6 & 91.9 & 87.77 \\
& LaMDA & 0.093M &  96.2 & 90.1 & 68.1 & 94.5 & 87.3 & 92.0 & 88.03 \\
& VeRA & 0.061M & 96.1 & 90.9 & 68.0 & 94.4 & 85.9 & 91.7 & 87.83 \\
& FourierFT & 0.048M & 96.0 & 90.9 & 67.1 & 94.4 & \textbf{87.4} & 91.9 & 87.95 \\
\rowcolor{green!17}
& \textbf{SSH} & \textbf{0.036M} & 96.2 & 90.9 & 67.9 & 94.5 & \textbf{87.4} & \textbf{92.2} & \textbf{88.17} \\
\bottomrule
\end{tabular}%
}
\caption{\small Performance of various fine-tuning methods on GLUE benchmark, using base and large models. Metrics include MCC for CoLA, PCC for STS-B, and accuracy for other tasks. Results are medians of 5 runs with different seeds; the best scores in each category are bolded. SSH delivers the best average performance across tasks while using significantly fewer trainable parameters.}
\label{tab:nlup}
\end{table*}





\noindent \textbf{Models and Datasets.}  
We evaluate SSH on the GLUE benchmark~\cite{wang2019glue} using RoBERTa~\cite{liu2019roberta} in both Base and Large configurations. The GLUE benchmark comprises a diverse set of NLU tasks, offering a comprehensive evaluation framework.


\noindent \textbf{Implementation Details.}  
The SSH method uses 750 of the 768\textsuperscript{2} available spectral coefficients for RoBERTa Base and 1024\textsuperscript{2} for RoBERTa Large, ensuring that each layer retains the most important spectral components. This selection remains consistent across all layers. To ensure fair comparison, we follow the same experimental settings as LoRA and FourierFT. Additional hyperparameters and details are provided in Tab.~\ref{tab:nluh} in the appendix~\ref{gluebench}.


\noindent \textbf{Results and Analysis} 
The results in Table~\ref{tab:nlup} indicate that SSH consistently delivers competitive performance across diverse NLU tasks while maintaining a significantly lower number of trainable parameters. Notably, SSH achieves 80.5\% accuracy on RTE, 92.4\% on QNLI, and 90.9 on STS-B, demonstrating its capability to generalize effectively across multiple linguistic tasks.

SSH also maintains robust performance in sentiment classification, achieving 94.1\% accuracy on SST-2, which is on par with other parameter-efficient methods such as LoRA and BitFit. On CoLA, SSH attains a score of 63.6, matching FourierFT and outperforming Adpt\textsuperscript{D} and AdaLoRA. Additionally, SSH exhibits strong generalization on MRPC with 91.2\% accuracy and achieves a 90.9 Pearson correlation on STS-B, further reinforcing its effectiveness across textual similarity and entailment tasks. These findings highlight SSH as a highly efficient and scalable fine-tuning approach, capable of achieving state-of-the-art performance with minimal parameter overhead.




% These results highlight SSH's ability to retain critical information with minimal parameters, making it an effective and resource-efficient method for fine-tuning large-scale models, particularly in computationally constrained environments.


\begin{table}[!t]
\centering
\scalebox{0.63}{
\begin{tabular}{l|lr|crcccccc}
\toprule
 & \textbf{Method} & \textbf{\# Tr. Para.} & \textbf{BLEU} & \textbf{NIST} & \textbf{METE.} & \textbf{ROU-L} & \textbf{CIDEr} \\
\midrule
\multirow{9}{*}{\rotatebox{90}{\textbf{GPT-2 Medium}}} 
& FT\textsuperscript{1} & 354.92M & 68.2 & 8.62 & 46.2 & 71.0 & 2.47 \\
& Adpt\textsuperscript{L\textsuperscript{1}} & 0.37M & 66.3 & 8.41 & 45.0 & 69.8 & 2.40 \\
& Adpt\textsuperscript{L\textsuperscript{1}} & 11.09M & 68.9 & 8.71 & 46.1 & 71.3 & 2.47 \\
& Adpt\textsuperscript{H\textsuperscript{1}} & 11.09M & 67.3 & 8.50 & 46.0 & 70.7 & 2.44 \\
& LoRA & 0.35M & 68.9 & 8.76 & 46.6 & 71.5 & 2.51 \\
& DoRA & 0.36M & 69.2 & 8.79 & 46.9 & 71.7 & 2.52\\
& VeRA & 0.35M & \textbf{70.1} & 8.81 & 46.6 & 71.5 & 2.50 \\
& FourierFT & 0.048M & 69.1 & \textbf{8.82} & 47.0 & 71.8 & 2.51 \\
\rowcolor{green!17}
& \textbf{SSH} & \textbf{0.036M} & \textbf{70.1} & \textbf{8.82} & \textbf{47.2} & \textbf{71.9} & \textbf{2.54} \\
\midrule
\multirow{8}{*}{\rotatebox{90}{\textbf{GPT-2 Large}}} 
& FT\textsuperscript{1} & 774.03M & 68.5 & 8.78 & 46.0 & 69.9 & 2.45 \\
& Adpt\textsuperscript{L\textsuperscript{1}} & 0.88M & 69.1 & 8.68 & 46.1 & 71.0 & 2.49 \\
& Adpt\textsuperscript{L\textsuperscript{1}} & 23.00M & 68.9 & 8.70 & 46.1 & 71.3 & 2.45 \\
& LoRA & 0.77M & 69.4 & 8.81 & 46.5 & \textbf{71.9} & 2.50 \\
& DoRA & 0.79M & 69.8 & 8.83 & 46.9 & \textbf{71.9} & 2.50 \\
& VeRA & 0.17M & \textbf{70.3} & 8.85 & 46.6 & 71.6 & 2.54 \\
& FourierFT & 0.072M & 70.2 & 8.90 & 47.0 & 71.8 & 2.50 \\
\rowcolor{green!17}
& \textbf{SSH} & \textbf{0.054M} & \textbf{70.3} & \textbf{8.93} & \textbf{47.2} & \textbf{71.9} & \textbf{2.55} \\
\bottomrule
\end{tabular}}
\caption{\small Performance comparison of various fine-tuning methods on GPT-2 Medium and Large models, evaluated using BLEU, NIST, METEOR, ROUGE-L, and CIDEr metrics. \textsuperscript{1} denotes results sourced from previous studies. }
\label{tab:e2e}
\end{table}



\subsection{Natural Language Generation}


\noindent \textbf{Models and Datasets.}  
We evaluate SSH on the E2E natural language generation (NLG) task~\cite{novikova2017e2e}, fine-tuning GPT-2 Medium and Large models~\cite{radford2019language}, which consist of 24 and 36 transformer blocks.

\noindent \textbf{Implementation Details.}  
We fine-tune LoRA, DoRA, FourierFT, VeRA, and the proposed SSH on GPT-2 Medium and Large, using a linear learning rate scheduler over 5 epochs. Results are averaged across 3 runs, with detailed hyperparameters provided in Tab.~\ref{tab:nlgh} in the Appendix~\ref{gluebench}.




\noindent \textbf{Results and Analysis.}  
As shown in Tab.~\ref{tab:e2e}, SSH consistently delivers superior or comparable performance across all evaluation metrics, while requiring significantly fewer trainable parameters. For GPT-2 Medium, SSH matches the highest BLEU score (70.1) and outperforms other methods in NIST (8.82), METEOR (47.2), ROUGE-L (71.9), and CIDEr (2.54), all with 10.3\% fewer parameters than LoRA and 25\% fewer than FourierFT. A similar trend is observed for GPT-2 Large, where SSH achieves the highest NIST (8.93) and METEOR (47.2) scores, while maintaining a 7.1\% parameter reduction compared to LoRA. 




\begin{table}[!t]
\centering
\resizebox{0.5\textwidth}{!}{%
\begin{tabular}{l|l|c|crcc}
\toprule
\textbf{Model} & \textbf{Method} & \textbf{\# Tr. Para.} & \textbf{MT-Bench} & \textbf{Vicuna} \\
\midrule
\multirow{5}{*}{\textbf{LLaMA2-7B}} 
& LoRA & 159.9M & 5.19 & 7.37 \\
& DoRA & 163.7M & 5.20 & 7.41 \\
& VeRA & 1.6M & 5.18 & 7.47 \\
& FourierFT & 0.064M & 5.09 & 7.50 \\
\rowcolor{green!17}
& \textbf{SSH} & \textbf{0.048M} & \textbf{5.22} & \textbf{7.51}\\
\midrule
\multirow{5}{*}{\textbf{LLaMA2-13B}} 
& LoRA & 250.3M & 5.77 & 7.89\\
& DoRA & 264.5M & 5.79 & 7.90 \\
& VeRA & 2.4M & \textbf{5.93} & 7.90 \\
& FourierFT & 0.08M & 5.82 & 7.92 \\
\rowcolor{green!17}
& \textbf{SSH} & \textbf{0.06M} & \textbf{5.93} & \textbf{7.95} \\
\midrule
\multirow{5}{*}{\textbf{LLaMA3.1-8B}} 
& LoRA & 183.3M & 5.65 & 7.52 \\
& DoRA & 186.9M & 5.66 & \textbf{7.59} \\
& VeRA & 1.9M & 5.61 & 7.49 \\
& FourierFT & 0.073M & 5.67 & 7.67 \\
\rowcolor{green!17}
& \textbf{SSH} & \textbf{0.055M} & \textbf{5.69} & \textbf{7.71} \\
\bottomrule
\end{tabular}%
}
\caption{\small Performance comparison of fine-tuning methods on LLaMA models using the Alpaca dataset. Evaluation scores on MT-Bench and Vicuna are generated and scored by GPT-4.}
\label{tab:mtbench_vicuna}
\end{table}


\subsection{Instruction Tuning}

\noindent \textbf{Models and Datasets.}  
We fine-tune LLaMA2-7B, LLaMA2-13B, and LLaMA3.1-8B using SSH and baseline methods on the Alpaca dataset~\cite{taori2023stanford}. For evaluation, we generate responses to predefined questions from the MT-Bench~\cite{zheng2024judging} and Vicuna Eval datasets, which are then scored by GPT-4 on a 10-point scale.

\noindent \textbf{Implementation Details.}  
Following previous work~\cite{dettmers2024qlora,dettmers20228bit}, we apply LoRA, DoRA, and VeRA to all linear layers except the top one. For FourierFT, we use the configuration from~\cite{gao2024parameter}, and for SSH, we set \(n = 750\). All models are trained using QLoRA’s quantization technique~\cite{dettmers2024qlora} on a single GPU. Each method is trained for one epoch, and we report the average score across all generated responses. Hyperparameter details are provided in Tab.\ref{tab:hyperparamsIn} in the Appendix~\ref{gluebench}.


\noindent \textbf{Results and Analysis.}  
The results in Tab.~\ref{tab:mtbench_vicuna} clearly demonstrate the significant efficiency of SSH compared to other fine-tuning methods such as LoRA, DoRA, and FourierFT. For LLaMA2-7B, SSH achieves the best MT-Bench (5.22) and Vicuna (7.51) scores while reducing trainable parameters by over 99.7\%, using only 0.048M parameters compared to LoRA's 159.9M. Similarly, in LLaMA2-13B, SSH ties with VeRA for the highest MT-Bench score (5.93) and surpasses all methods in Vicuna (7.95), again achieving this with a drastically lower parameter count (0.06M vs. 250.3M for LoRA). Even in the larger LLaMA3.1-8B model, SSH continues to outperform, leading in MT-Bench (5.69) and maintaining a competitive Vicuna score (7.71) with far fewer parameters (0.055M). 



\begin{table}
\centering
\resizebox{0.47\textwidth}{!}{%
\begin{tabular}{l|l|r|cccccc}
\toprule
\textbf{Model} & \textbf{Method} & \textbf{\# Train. Para.} & \textbf{CIFAR100} & \textbf{DTD} & \textbf{EuroSAT} & \textbf{OxfordPets} \\
\midrule
\multirow{7}{*}{\textbf{ViT-B}} 
& Head & - & 84.3 & 69.8 & 88.7 & 90.3 \\
& Full & 85.8M & \textbf{92.4} & \textbf{77.7} & \textbf{99.1}& \textbf{93.4} \\
& LoRA & 581K & 92.1 & 75.2 & 98.4 & 93.2 \\
& Dora & 594K & 92.3 & 75.3 & 98.7 & 93.2 \\
& VeRA & 57.3K & 91.7 & 74.6 & 98.5 & \textbf{93.4}\\
& FourierFT & 72K & 94.2 & 75.1 & 98.8 & 93.2 \\
\rowcolor{green!17}
& \textbf{SSH} & \textbf{54K} & 91.6 & 76.1 & \textbf{99.1} & \textbf{93.4} \\

\midrule
\multirow{7}{*}{\textbf{ViT-L}} 
& Head & - & 84.7 & 73.3 & 92.6 & 91.1 \\
& Full & 303.3M & 93.6 & 81.8 & \textbf{99.1} & 94.4 \\
& LoRA & 1.57M &  94.9 & 81.8 & 98.63 & \textbf{94.8} \\
& Dora & 1.62M &  \textbf{95.1} & 81.8 &   98.8 & \textbf{94.8}\\
& VeRA & 130.5K & 94.2 & 81.6& 98.6 & 93.7 \\
& FourierFT & 144K & 93.7 & 81.2 & 98.7 & 94.5 \\
\rowcolor{green!17}
& \textbf{SSH} & \textbf{108K} & 94.5 & \textbf{81.9} & 99.0& \textbf{94.8} \\
\bottomrule
\end{tabular}%
}
\caption{\small Performance of various fine-tuning methods on ViT-B and ViT-L models across different datasets. The best results for each dataset are highlighted in bold. The best results are highlighted in bold. SSH offers strong parameter efficiency, excelling on DTD and EuroSAT while delivering competitive performance on CIFAR100 and OxfordPets, making it a balanced solution for various vision tasks.}
\label{tab:vit_results}
\end{table}





\subsection{Text Summarization}
\noindent \textbf{Models and Datasets.}  
We evaluate the effectiveness of SSH against other baseline methods on the BART-Large model~\cite{lewis2019bart} for text summarization tasks. Specifically, we assess its performance on the XSUM~\cite{narayan2018don} and CNN/DailyMail~\cite{hermann2015teaching} datasets.

\begin{table}[!t]
    \centering
    \resizebox{0.52\textwidth}{!}{%
    \begin{tabular}{l|c|c|c}
        \toprule
        \textbf{Method} & \textbf{Para. (M)} & \textbf{XSUM} & \textbf{CNN/DailyMail} \\
        \midrule
        AFLoRA ($r=32$) & 5.27 & 44.71/21.92/37.33 & 44.95/21.87/42.25 \\
        LaMDA ($r=32$) & 0.85 & 43.94/20.69/35.21 & 44.16/21.17/40.48 \\
        \rowcolor{green!17}
        SSH ($n=5000$) & 0.21 & 44.72/22.05/37.42 & 44.89/21.75/42.13 \\
        \bottomrule
    \end{tabular}}
    \caption{Performance comparison of SSH, AFLoRA, and LaMDA on BART-Large for text summarization tasks. Results are reported as ROUGE-1/ROUGE-2/ROUGE-L.}
    \label{tab:nlg_bart}
\end{table}

\noindent \textbf{Implementation Details.}  
We compare SSH against AFLoRA and LaMDA under consistent experimental conditions. For AFLoRA and LaMDA, we set the rank $r=32$, while for SSH, we select $n=5000$ Hartley spectrum points. The models are trained using a learning rate of $2\times10^{-4}$, with a batch size of 32 for XSUM and 64 for CNN/DailyMail. Training is conducted for 25 epochs on XSUM and 15 epochs on CNN/DailyMail.

\noindent \textbf{Results and Analysis.}  
Table~\ref{tab:nlg_bart} presents the ROUGE evaluation scores (ROUGE-1/ROUGE-2/ROUGE-L) for different fine-tuning approaches. SSH achieves competitive performance while utilizing significantly fewer trainable parameters compared to AFLoRA and LaMDA. On the XSUM dataset, SSH attains the highest ROUGE-2 score (22.05), surpassing AFLoRA (21.92) and LaMDA (20.69) by 0.13 and 1.36 points, respectively. Furthermore, SSH achieves the highest ROUGE-L score (37.42), outperforming AFLoRA by 0.09 and LaMDA by 2.21 points.

Similarly, on the CNN/DailyMail dataset, SSH attains a ROUGE-1 score of 44.89, which is marginally lower than AFLoRA (44.95) by 0.06 points, but it outperforms LaMDA (44.16) by 0.73 points. In terms of ROUGE-2, SSH achieves 21.75, trailing AFLoRA (21.87) by 0.12 points but exceeding LaMDA (21.17) by 0.58 points. Additionally, SSH attains a ROUGE-L score of 42.13, which is 0.12 points lower than AFLoRA but significantly higher than LaMDA by 1.65 points. Overall, SSH consistently demonstrates strong performance while requiring significantly fewer trainable parameters (0.21M) compared to AFLoRA (5.27M) and LaMDA (0.85M). 

\subsection{Mathematical Reasoning}

\noindent \textbf{Models and Dataset.} We evaluate the performance of SSH against AFLoRA and LaMDA on the LLaMA3.1-8B model using the GSM8K~\cite{cobbe2021training}, a widely used dataset designed to assess mathematical reasoning abilities.

\noindent \textbf{Implementation Details.} All methods are trained with a learning rate of $3\times10^{-4}$ for six epochs using a batch size of 16. For parameter-efficient fine-tuning, AFLoRA and LaMDA employ a low-rank adaptation setting of $r=32$, while SSH leverages a Hartley spectrum selection with $n=10000$. Table~\ref{tab:llama_gsm8k} presents a comparison of the methods in terms of trainable parameters and accuracy on GSM8K.

\begin{table}[!t]
    \centering
    \resizebox{0.5\textwidth}{!}{%
    \begin{tabular}{l|c|c}
        \toprule
        \textbf{Method} & \textbf{Trainable Parameters (M)} & \textbf{GSM8K Accuracy} \\
        \midrule
        AFLoRA ($r=32$) & 20.23 & 38.63 \\
        LaMDA ($r=32$) & 4.99 & 38.11 \\
        \rowcolor{green!17}
        SSH ($n=10000$) & 1.54 & \textbf{38.67} \\
        \bottomrule
    \end{tabular}}
    \caption{Comparison of SSH with AFLoRA and LaMDA on LLaMA3.1-8B for GSM8K. Accuracy is reported as a percentage.}
    \label{tab:llama_gsm8k}
\end{table}

\noindent \textbf{Results and Analysis.} SSH achieves the highest accuracy (38.67\%), surpassing AFLoRA (38.63\%) and LaMDA (38.11\%) while using significantly fewer trainable parameters. SSH requires only 1.54M parameters, representing a \textbf{92.4\% reduction} compared to AFLoRA and a \textbf{69.1\% reduction} compared to LaMDA. 

Despite having nearly 13 times fewer parameters than AFLoRA, SSH achieves comparable accuracy, demonstrating a superior trade-off between efficiency and performance. While LaMDA exhibits the lowest accuracy, SSH maintains robustness in mathematical reasoning tasks with minimal resource requirements.

% These findings highlight SSH as a highly efficient fine-tuning method, delivering strong reasoning capabilities while significantly reducing computational overhead. This makes SSH a promising approach for scaling large language models in resource-constrained settings.


\subsection{Image Classification}

\noindent \textbf{Models and Datasets.}  
We evaluate our method on the Vision Transformer (ViT)~\cite{dosovitskiy2020image}, using both the Base and Large variants. Image classification is performed on the CIFAR-100~\cite{krause20133d}, DTD~\cite{cimpoi2014describing}, EuroSAT~\cite{helber2019eurosat}, and OxfordPets~\cite{parkhi2012cats} datasets.

\noindent \textbf{Implementation Details.}  
We evaluate SSH, LoRA, DoRA, VeRA, and FourierFT by applying them to the query and value layers of ViT. Training only the classification head is denoted as "Head". We set \( r = 16 \) for LoRA, \( n = 3000 \) for FourierFT, and \( n = 2250 \) for SSH. Learning rates and weight decay are tuned for all methods, with training limited to 10 epochs. Further hyperparameter details are provided in Tab.~\ref{tab:SSH_image} in the Appendix~\ref{gluebench}.






\noindent \textbf{Results and Analysis.}  
Tab.~\ref{tab:vit_results} highlights the performance of various fine-tuning methods on ViT-B and ViT-L across four image classification datasets. For the ViT-B model, SSH delivers competitive results with only 54K trainable parameters, significantly fewer than LoRA and DoRA, which use more than 10 times as many. Notably, SSH matches the full fine-tuning performance on EuroSAT and OxfordPets, achieving 99.1\% and 93.4\% accuracy, respectively. For the ViT-L model, SSH also proves efficient, achieving near-optimal performance with only 108K parameters. It sets the highest score on DTD with 81.9\% accuracy and matches the best performance on OxfordPets at 94.8\%. 


% \begin{table}
% \centering
% \resizebox{0.47\textwidth}{!}{%
% \begin{tabular}{l|l|r|cccccc}
% \toprule
% \textbf{Model} & \textbf{Method} & \textbf{\# Train. Para.} & \textbf{TVQA} & \textbf{How2QA} & \textbf{TVC} & \textbf{YC2C} \\
% \midrule
% \multirow{7}{*}{\textbf{VL-BART}} 
% & Full & 228.9M & \textbf{76.3} & 73.9 & 45.7& \textbf{154} \\
% & LoRA & 11.8M & 75.5 & 72.9 & 44.6 & 140.9 \\
% & Dora & 11.9M & \textbf{76.3} & 74.1 & \textbf{45.8} & 145.4 \\
% & VeRA & 1.3M & 75.9 & 73.8 & 44.7 & 142.6\\
% & FourierFT & 1.5M & 76.2 & 73.1 & 45.5 & 147.3 \\
% \rowcolor{green!17}
% & \textbf{SSH} & \textbf{1.1M} & 76.2 & \textbf{74.2}& \textbf{45.8} & 152 \\

% \bottomrule
% \end{tabular}%
% }
% \caption{\small Multi-task evaluation results on TVQA, How2QA, TVC, and YC2C using the VL-BART backbone. The best results are highlighted in bold. SSH demonstrates strong performance with significantly fewer trainable parameters, achieving top scores on How2QA, TVC, and competitive results on the other tasks.}

% \label{tab:bart_results}
% \end{table}

% \subsection{Video-Text Understanding}

% \noindent \textbf{Models and Datasets.}  
% We compare DoRA, LoRA, and full fine-tuning on VL-BART, a model that integrates a vision encoder (CLIP-ResNet101~\cite{radford2021learning}) with an encoder-decoder language model (BART-Base~\cite{lewis2019bart}). The comparison spans four video-text tasks: TVQA~\cite{lei2018tvqa} and How2QA~\cite{li2020hero} for video question answering, and TVC~\cite{lei2020tvr} and YC2C~\cite{zhou2018towards} for video captioning.

% \noindent \textbf{Implementation Details.}  
% We follow the framework from ~\cite{sung2022vl}, fine-tuning VL-BART in a multi-task setup for both video-text tasks. We set \( r = 128 \) for LoRA, \( n = 6000 \) for FourierFT, and \( n = 4500 \) for SSH. Learning rates and weight decay are tuned for each method, with training capped at 7 epochs. Further details are provided in Tab.~\ref{tab:SSH_video} in the Appendix~\ref{gluebench}.


% \noindent \textbf{Results and Analysis.}  
% Tab.~\ref{tab:bart_results} presents the multi-task evaluation results for video-text tasks using the VL-BART backbone. SSH consistently delivers competitive performance with significantly fewer trainable parameters. For TVQA, SSH achieves a near-best score of 76.2, matching FourierFT and just slightly below the full fine-tuning result of 76.3, despite using 99.5\% fewer parameters.
% For How2QA, SSH records the highest score of 74.2, outperforming all other methods. Similarly, for TVC, SSH ties for the best result with DoRA at 45.8, again with far fewer parameters. For YC2C, SSH comes close to the top score achieved by full fine-tuning, with a score of 152 compared to 154, while maintaining remarkable parameter efficiency.

% \begin{figure}[!t]
%     \centering
%     \includegraphics[width=\linewidth]{data/rank.pdf}
%     \caption{\small Scalability comparative experiments of LoRA, FourierFT, SSH, and rSSH across GLUE tasks. SSH demonstrates robust scalability and strong performance in NLP understanding tasks.}
%     \label{fig:ablation}
% \end{figure}





\subsection{Ablation Study}
% We investigate the relationship between parameter number and model performance across different methods. For LoRA, we evaluate with ranks \( r = \{1, 2, 4, 6, 8, 16\} \). For FourierFT and SSH, we examine \( n = \{50, 100, 200, 1000, 6144, 12288\} \) spectral coefficients. The experiments are conducted on 6 GLUE tasks.





% \noindent \textbf{Parameter Scalability.} Figure~\ref{fig:ablation} demonstrates that SSH consistently outperforms other approaches as the number of spectral coefficients \( n \) increases, highlighting its robust scalability and efficiency. Unlike LoRA, where increasing the number of trainable parameters often yields diminishing returns, SSH maintains strong performance gains with fewer parameters. Additionally, SSH surpasses FourierFT across tasks such as MRPC, CoLA, RTE, SST-2, and QNLI, demonstrating its effective parameter utilization.







% A statistical analysis using the Student t-test confirms that SSH significantly outperforms FourierFT across most tasks: RTE ($p=0.0446$, $t=2.67$), MRPC ($p=0.0167$, $t=3.53$), SST-2 ($p=0.0272$, $t=3.09$), QNLI ($p=0.0066$, $t=4.46$), and CoLA ($p=0.0144$, $t=3.67$). While STS-B shows a smaller advantage ($p=0.0348$, $t=2.87$), SSH consistently demonstrates superior scalability and performance across benchmarks, reinforcing its effectiveness as \( n \) increases.





% \noindent \textbf{Informed Frequency Selection v.s. Random Sampling.}
% We further compare the proposed SSH with the scenario where the frequency points are selected randomly (denoted as rSSH).
% The results illustrate that SSH, which uses a systematic partitioning and hybrid selection strategy, consistently outperforms rSSH. This is statistically confirmed by the student t-test, where significant improvements are observed in CoLA (p=0.0260, t=3.13) and SST-2 (p=0.0050, t=4.77), highlighting the advantage of informed frequency selection. Even in tasks with smaller gaps like MRPC (p=0.0380, t=2.80) and QNLI (p=0.0165, t=3.54), SSH demonstrates clear benefits. Although the STS-B task shows a less pronounced difference (p=0.117, t=1.89), the overall performance trend strongly favors SSH, especially in tasks where precise frequency selection is crucial for effective fine-tuning.
\begin{figure}
    \centering
    \includegraphics[width=\linewidth]{data/ratio.pdf}
    \caption{\small Ablation study of SSH on GLUE tasks illustrating the effect of varying energy ratios ($\delta$) on performance with RoBERTa-base (n=750). Performance is normalized to $\delta = 0.5$, showing optimal balance and diversity in spectral representation at $\delta = 0.7$.
}
    \label{fig:ratio}
\end{figure}



\noindent \textbf{Energy Ratio Ablation Study.}
\label{subsubsec:energyratio}
Figure~\ref{fig:ratio} presents an ablation study of SSH across GLUE tasks with varying energy ratios (\(\delta\)) on RoBERTa-base with \(n=750\), where performance is normalized to \(\delta = 0.5\). The energy ratios considered are \(\delta=0.5\), \(\delta=0.6\), \(\delta=0.7\), \(\delta=0.8\), and \(\delta=0.9\). 

% The results indicate that an energy ratio of \(\delta=0.7\) generally yields stable and improved performance across most tasks. This is notably evident in MRPC and CoLA, where performance peaks with \(\delta=0.7\). However, performance tends to decline when \(\delta\) is set too low (\(\delta=0.5\) or \(\delta=0.6\)), particularly in tasks like QNLI and CoLA, with relative performance drops of up to 1\%. Similarly, higher \(\delta\) values (\(\delta=0.8\) or \(\delta=0.9\)) also lead to decreased performance, especially in tasks such as QNLI and RTE, suggesting that excessively high energy ratios do not correlate with better task performance. 


The ablation study indicates that an energy ratio of \(\delta=0.7\) optimally balances the selection of spectral components, consistently enhancing performance across natural language understanding tasks such as MRPC and CoLA. This balance prevents overfitting and underfitting, ensuring the retention of informative frequencies while excluding those that are redundant. In contrast, lower ratios (\(\delta=0.5\) or \(\delta=0.6\)) result in inadequate frequency representation, adversely affecting performance in tasks that require robust syntactic and semantic analysis, such as QNLI and CoLA. Higher ratios (\(\delta=0.8\) and \(\delta=0.9\)), while expanding the range of considered frequencies, often introduce noise that compromises the model's focus and generalization ability, particularly evident in tasks like QNLI and STS-B.

% This demonstrates that excessive inclusion of spectral components can diminish model efficacy by detracting from its ability to generalize from training to unseen data.


\section{Conclusion}
We introduce a novel approach, \algo, to reduce human feedback requirements in preference-based reinforcement learning by leveraging vision-language models. While VLMs encode rich world knowledge, their direct application as reward models is hindered by alignment issues and noisy predictions. To address this, we develop a synergistic framework where limited human feedback is used to adapt VLMs, improving their reliability in preference labeling. Further, we incorporate a selective sampling strategy to mitigate noise and prioritize informative human annotations.

Our experiments demonstrate that this method significantly improves feedback efficiency, achieving comparable or superior task performance with up to 50\% fewer human annotations. Moreover, we show that an adapted VLM can generalize across similar tasks, further reducing the need for new human feedback by 75\%. These results highlight the potential of integrating VLMs into preference-based RL, offering a scalable solution to reducing human supervision while maintaining high task success rates. 

\section*{Impact Statement}
This work advances embodied AI by significantly reducing the human feedback required for training agents. This reduction is particularly valuable in robotic applications where obtaining human demonstrations and feedback is challenging or impractical, such as assistive robotic arms for individuals with mobility impairments. By minimizing the feedback requirements, our approach enables users to more efficiently customize and teach new skills to robotic agents based on their specific needs and preferences. The broader impact of this work extends to healthcare, assistive technology, and human-robot interaction. One possible risk is that the bias from human feedback can propagate to the VLM and subsequently to the policy. This can be mitigated by personalization of agents in case of household application or standardization of feedback for industrial applications. 


% \clearpage
\bibliography{0_acl_main}
% \bibliographystyle{plainnat}
\clearpage
\label{sec:appendix}
\clearpage
\renewcommand{\thefigure}{A\arabic{figure}}
\renewcommand{\thetable}{A\arabic{table}}
\renewcommand{\theequation}{A\arabic{equation}}
\setcounter{figure}{0}
\setcounter{table}{0}
\setcounter{equation}{0}

Our Appendix is organized as follows. First, we present the pseudocode for the key components of iGCT. We also include the proof for unit variance and boundary conditions in preconditioning iGCT's noiser. Next, we detail the training setups for our CIFAR-10 and ImageNet64 experiments. Additionally, we provide ablation studies on using guided synthesized images as data augmentation in image classification. Finally, we present more uncurated results comparing iGCT and CFG-EDM on inversion, editing and guidance, thoroughly of iGCT.

\vspace{-0.2cm}
\label{appendix:iGCT}
\section{Pseudocode for iGCT}
\vspace{-0.2cm}

iGCT is trained under a continuous-time scheduler similar to the one proposed by ECT \cite{ect}. Our noise sampling function follows a lognormal distribution, \(p(t) = \textit{LogNormal}(P_\textit{mean}, P_\textit{std})\), with \(P_\textit{mean}=-1.1\) and \( P_\textit{std}=2.0\). At training, the sampled noise is clamped at \(t_\text{min} = 0.002\) and \(t_\text{max} = 80.0\). Step function \(\Delta t (t)=\frac{t}{2^{\left\lfloor k/d \right\rfloor}}n(t)\), is used to compute the step size from the sampled noise \(t\), with \(k,d\) being the current training iteration and the number of iterations for halfing \(\Delta t\), and \(n(t) = 1 + 8 \sigma(-t)\) is a sigmoid adjusting function. 

In Guided Consistency Training, the guidance mask function determines whether the sampled noise \( t \) should be supervised for guidance training. With probability \( q(t) \in [0,1] \), the update is directed towards the target sample \( \boldsymbol{x}_0^{\text{tar}} \); otherwise, no guidance is applied. In practice, \( q(t) \) is higher in noisier regions and zero in low-noise regions, 
\begin{equation}
    q(t) = 0.9 \cdot \left( \text{clamp} \left( \frac{t - t_{\text{low}}}{t_{\text{high}} - t_{\text{low}}}, 0, 1 \right) \right)^2,
\end{equation}
where \( t_{\text{low}} = 11.0 \) and \( t_{\text{high}} = 14.3 \). For the range of guidance strength, we set \(w_\text{min} = 1\) and \(w_\text{max} = 15\). Guidance strengths are sampled uniformly at training, with \(w_\text{min} = 1\) means no guidance applied. 


\begin{algorithm}
\caption{Guided Consistency Training}
\label{alg:GCT}
\begin{algorithmic}[1]  % Adds line numbers
\setlength{\baselineskip}{0.9\baselineskip} % Adjust line spacing
\INPUT Dataset $\mathcal{D}$, weighting function $\lambda(t)$, noise sampling function $p(t)$, noise range $[t_\text{min}, t_\text{max}]$, step function $\Delta t(t)$, guidance mask function $q(t)$, guidance range $[w_\text{min}, w_\text{max}]$, denoiser $D_\theta$
\STATE \rule{0.96\textwidth}{0.45pt} 
\STATE Sample $(\boldsymbol{x}_0^{\text{src}}, c^{\text{src}}), (\boldsymbol{x}_0^{\text{tar}}, c^{\text{tar}}) \sim \mathcal{D}$ 
\STATE Sample noise $\boldsymbol{z} \sim \mathcal{N}(\boldsymbol{0},\mathbf{I})$, time step $t \sim p(t)$, and guidance weight $w \sim \mathcal{U}(w_\text{min}, w_\text{max})$
\STATE Clamp $t \leftarrow \text{clamp}(t,t_\text{min}, t_\text{max})$
\STATE Compute noisy sample: $\boldsymbol{x}_t = \boldsymbol{x}_0^{\text{src}} + t\boldsymbol{z}$
\STATE Sample $\rho \sim \mathcal{U}(0,1)$  
\vspace{0.3em}
\IF{$\rho > q(t)$}
    \STATE Compute step as normal CT: $\boldsymbol{x}_r = \boldsymbol{x}_t - \Delta t(t) \boldsymbol{z}$
    \STATE Set target class: $c \leftarrow c^{\text{src}}$
\ELSE
    \STATE Compute guided noise: $\boldsymbol{z}^* = (\boldsymbol{x}_t - \boldsymbol{x}_0^{\text{tar}}) / t$
    \STATE Compute guided step: $\boldsymbol{x}_r = \boldsymbol{x}_t - \Delta t(t) [w \boldsymbol{z}^* + (1-w)\boldsymbol{z}]$
    \STATE Set target class: $c \leftarrow c^{\text{tar}}$
\ENDIF
\vspace{0.3em} % Reduces extra vertical space before the loss line
\STATE Compute loss: 
\[
\mathcal{L}_\text{gct} = \lambda(t) \, d(D_{\theta}(\boldsymbol{x}_t, t, c, w), D_{{\theta}^-}(\boldsymbol{x}_r, r, c, w))
\]
\STATE Return $\mathcal{L}_\text{gct}$ 
\end{algorithmic}
\end{algorithm}



A \textit{noiser} trained under \textit{Inverse Consistency Training} maps an image to its latent noise in a single step. In contrast, DDIM Inversion requires multiple steps with a diffusion model to accurately produce an image's latent representation. Since the boundary signal is reversed, spreading from \( t_\text{max} \) down to \( t_\text{min} \), we design the importance weighting function \( \lambda'(t) \) to emphasize higher noise regions, defined as:
\begin{equation}
    \lambda'(t) = \frac{\Delta t (t)}{t_\text{max}},
\end{equation}
where the step size \( \Delta t (t) \) is proportional to the sampled noise level \(t\), and \( t_\text{max} \) is a constant that normalizes the scale of the inversion loss. The noise sampling function \( p(t) \) and the step function \( \Delta t (t) \) used in computing both \(\mathcal{L}_\text{gct}\) and \(\mathcal{L}_\text{ict}\) are the same.



\begin{algorithm}
\caption{Inverse Consistency Training}
\label{alg:iCT}
\begin{algorithmic}[1]  % Adds line numbers
\setlength{\baselineskip}{0.9\baselineskip} % Adjust line spacing
\INPUT Dataset $\mathcal{D}$, weighting function $\lambda'(t)$, noise sampling function $p(t)$, noise range $[t_\text{min}, t_\text{max}]$, step function $\Delta t(t)$, noiser $N_\varphi$
\STATE \rule{0.96\textwidth}{0.45pt} 
\STATE Sample $\boldsymbol{x}_0, c \sim \mathcal{D}$ 
\STATE Sample noise $\boldsymbol{z} \sim \mathcal{N}(\boldsymbol{0},\mathbf{I})$, time step $t \sim p(t)$
\STATE Clamp $t \leftarrow \text{clamp}(t,t_\text{min}, t_\text{max})$
\STATE Compute noisy sample: $\boldsymbol{x}_t = \boldsymbol{x}_0 + t\boldsymbol{z}$
\STATE Compute cleaner sample: $\boldsymbol{x}_r = \boldsymbol{x}_t - \Delta t(t) \boldsymbol{z}$
\vspace{0.3em} 
\STATE Compute loss: 
\[
\mathcal{L}_\text{ict} = \lambda'(t) \, d(N_{\varphi}(\boldsymbol{x}_r, r, c), D_{{\varphi}^-}(\boldsymbol{x}_t, t, c))
\]
\STATE Return $\mathcal{L}_\text{ict}$ 
\end{algorithmic}
\end{algorithm}

Together, iGCT jointly optimizes the two consistency objectives \(\mathcal{L}_\text{gct}, \mathcal{L}_\text{ict}\), and aligns the noiser and denoiser via a reconstruction loss, \(\mathcal{L}_\text{recon}\). To improve training efficiency, \(\mathcal{L}_\text{recon}\) is computed every \(i_\text{skip}\), reducing the computational cost of back-propagation through both the weights of the \textit{denoiser} \(\theta\) and the \textit{noiser} \(\varphi\). Alg. \ref{alg:iGCT} provides an overview of iGCT. 

\begin{algorithm}
\caption{iGCT}
\label{alg:iGCT}
\begin{algorithmic}[1]  % Adds line numbers
\setlength{\baselineskip}{0.9\baselineskip} % Adjust line spacing
\INPUT Dataset $\mathcal{D}$, learning rate $\eta$, weighting functions $\lambda'(t), \lambda(t), \lambda_{\text{recon}}$, noise sampling function $p(t)$, noise range $[t_\text{min}, t_\text{max}]$, step function $\Delta t(t)$, guidance mask function $q(t)$, guidance range $[w_\text{min}, w_\text{max}]$, reconstruction skip iters $i_\text{skip}$, models $N_\varphi, D_\theta$
\STATE \rule{0.9\textwidth}{0.45pt}  % Horizontal line to separate input from main algorithm
\STATE \textbf{Init:} $\theta, \varphi$, $\text{Iters} = 0$
\REPEAT
\STATE Do guided consistency training 
\[
\mathcal{L}_\text{gct}(\theta;\mathcal{D},\lambda(t),p(t),t_\text{min},t_\text{max},\Delta t(t),q(t),w_\text{min},w_\text{max})
\]
\STATE Do inverse consistency training
\[
\mathcal{L}_\text{ict}(\varphi;\mathcal{D},\lambda'(t),p(t),t_\text{min},t_\text{max},\Delta t(t))
\]
\IF{$(\text{Iters} \ \% \ i_\text{skip}) == 0$}
\STATE Compute reconstruction loss
\[
\mathcal{L}_\text{recon} = d(D_{\theta}(N_{\varphi}(\boldsymbol{x}_0,t_\text{min},c),t_\text{max},c,0), \boldsymbol{x}_0)
\]
\ELSE
\STATE \[
\mathcal{L}_\text{recon} = 0
\]
\ENDIF
\STATE Compute total loss: 
\[
\mathcal{L} = \mathcal{L}_\text{gct} + \mathcal{L}_\text{ict} + \lambda_{\text{recon}}\mathcal{L}_\text{recon}
\]
\STATE $\theta \leftarrow \theta - \eta \nabla_{\theta} \mathcal{L}, \ \varphi \leftarrow \varphi - \eta \nabla_{\varphi} \mathcal{L}$
\STATE $\text{Iters} = \text{Iters} + 1$
\UNTIL{$\Delta t \rightarrow dt$}
\end{algorithmic}
\end{algorithm}



\vspace{-0.3cm}
\section{Preconditioning for Noiser}
\label{appendix:unit-variance}
\vspace{-0.1cm}

We define 
\begin{equation}
    N_{\varphi}(\boldsymbol{x}_t, t, c) = c_\text{skip}(t) \, \boldsymbol{x}_t + c_\text{out}(t) \, F_{\varphi}(c_\text{in}(t) \, \boldsymbol{x}_t, t, c),
\end{equation}
where \( c_\text{in}(t) = \frac{1}{\sqrt{t^2 + \sigma_\text{data}^2}} \), \( c_\text{skip}(t) = 1 \), and \( c_\text{out}(t) = t_\text{max} - t \). This setup naturally serves as a boundary condition. Specifically:

\begin{itemize}
    \item When \( t = 0 \),
    \begin{equation}
        c_\text{out}(0) = t_\text{max} \gg c_\text{skip}(0) = 1,
    \end{equation}
    emphasizing that the model's noise prediction dominates the residual information given a relatively clean sample.

    \item When \( t = t_\text{max} \),
    \begin{equation}
        N_{\varphi}(\boldsymbol{x}_{t_\text{max}}, t_\text{max}, c) = \boldsymbol{x}_{t_\text{max}},
    \end{equation}
    satisfying the condition that \( N_{\varphi} \) outputs \( \boldsymbol{x}_{t_\text{max}} \) at the maximum time step.
\end{itemize}



We show that these preconditions ensure unit variance for the model’s input and target. First, \(\text{Var}_{\boldsymbol{x}_0, z}[\boldsymbol{x}_t] = \sigma_\text{data}^2 + t^2\), so setting \( c_\text{in}(t) = \frac{1}{\sqrt{\sigma_\text{data}^2 + t^2}} \) normalizes the input variance to 1. Second, we require the training target to have unit variance. Given the noise target for \( N_{\varphi} \) is \(\boldsymbol{x}_{t_\text{max}} = \boldsymbol{x}_0 + t_\text{max} z\), by moving of terms, the effective target for \( F_{\varphi} \) can be written as,
\begin{equation}
    \frac{\boldsymbol{x}_{t_\text{max}} - c_\text{skip}(t)\boldsymbol{x}_{t}}{c_\text{out}(t)}
\end{equation}
When \(c_\text{skip}(t) = 1\), \(c_\text{out}(t) = t_\text{max} - t \), we verify that target is unit variance,
\begin{align}
    &\text{Var}_{\boldsymbol{x}_0, \boldsymbol{z}} \left[ \frac{\boldsymbol{x}_{t_\text{max}} - c_\text{skip}(t) \, \boldsymbol{x}_{t}}{c_\text{out}(t)} \right] \\ \notag
    = \ &\text{Var}_{\boldsymbol{x}_0, \boldsymbol{z}} \left[ \frac{\boldsymbol{x}_0 + t_\text{max} \, \boldsymbol{z} - (\boldsymbol{x}_0 + t \, \boldsymbol{z})}{t_\text{max} - t} \right] \notag \\
    = \ &\text{Var}_{\boldsymbol{x}_0, \boldsymbol{z}} \left[ \frac{(t_\text{max} - t) \, \boldsymbol{z}}{t_\text{max} - t} \right] \notag \\
    = \ &\text{Var}_{\boldsymbol{x}_0, \boldsymbol{z}}[\boldsymbol{z}] \notag \\
    = \ &1. \notag
\end{align}

\vspace{-0.3cm}
\section{Baselines \& Training Details}
\label{appendix:bs-config}
\vspace{-0.1cm}

\begin{figure}[t!]  
    \centering
    \begin{subfigure}[b]{0.33\textwidth}
    \includegraphics[width=\textwidth]{fig/appendix/guidance_embed.pdf} 
        \caption{Guidance embedding.}
    \end{subfigure}
    \hfill
    \begin{subfigure}[b]{0.33\textwidth}
    \includegraphics[width=\textwidth]{fig/appendix/adm_arch.pdf} 
        \caption{NCSN++ architecture.}
    \end{subfigure}
    \hfill
    \begin{subfigure}[b]{0.33\textwidth}
    \includegraphics[width=\textwidth]{fig/appendix/ncsnpp_arch.pdf} 
        \caption{ADM architecture.}
    \end{subfigure}
    \hfill
    \caption{Design of guidance embedding, and conditioning under different network architectures.}
    \vspace{-1em}
    \label{fig:guidance_conditioning}
\end{figure}

For our diffusion model baseline, we follow \textit{EDM}'s official repository (\href{https://github.com/NVlabs/edm}{https://github.com/NVlabs/edm}) instructions for training and set \textit{label\_dropout} to 0.1 to optimize a CFG (classifier-free guided) DM. We will use this DM as the teacher model for our consistency model baseline via consistency distillation. 

The consistency model baseline \textit{Guided CD} is trained with a discrete-time schedule. We set the discretization steps \( N = 18 \) and use a Heun ODE solver to predict update directions based on the CFG EDM, as in \cite{song2023consistency}. Following \cite{luo2023latent}, we modify the model's architecture and iGCT's denoiser to accept guidance strength \(w\) by adding an extra linear layer. See the detailed architecture design for guidance conditioning of consistency model in Fig. \ref{fig:guidance_conditioning}. A range of guidance scales \(w \in [1,15]\) is uniformly sampled at training. Following \cite{song2023improved}, we replace LPIPS by Pseudo-Huber loss, with \(c=0.03 \) determining the breadth of the smoothing section between L1 and L2. See Table \ref{tab:training_configs} for a summary of the training configurations for our baseline models.


\begin{table}[t!]
\centering
\renewcommand{\arraystretch}{1.3} % Adjust vertical spacing
\small % Reduce text size
\caption{Summary of training configurations for baseline models.}
\begin{tabular}{lccc}
\toprule
\multirow{2}{*}{} & \multicolumn{2}{c}{\textbf{CIFAR-10}} & \textbf{ImageNet64}  \\
                  & EDM & Guided-CD & EDM \\
\midrule
\multicolumn{4}{l}{\textbf{\small Arch. config.}} \\
\hline
model arch.        & NCSN++ & NCSN++ & ADM     \\
channels mult.     & 2,2,2  & 2,2,2  & 1,2,3,4 \\
UNet size          & 56.4M  & 56.4M  & 295.9M  \\
\midrule
\multicolumn{4}{l}{\textbf{\small Training config.}} \\
\hline
lr             & 1e-3  & 4e-4  & 2e-4 \\
batch          & 512   & 512   & 4096 \\
dropout        & 0.13  & 0     & 0.1 \\
label dropout  & 0.1   & (n.a.) & 0.1 \\
loss           & L2    & Huber & L2    \\
training iterations & 390k  & 800k  & 800K \\
\bottomrule
\end{tabular}
\label{tab:training_configs}
\end{table}


\begin{table}[t!]
\centering
\renewcommand{\arraystretch}{1.3} % Adjust vertical spacing
\small % Reduce text size
\caption{Summary of training configurations for iGCT.}
\begin{tabular}{lcc}
\toprule
\multirow{2}{*}{} & \textbf{CIFAR-10} & \textbf{ImageNet64}  \\
                  & iGCT & iGCT \\
\midrule
\multicolumn{3}{l}{\textbf{\small Arch. config.}} \\
\hline
model arch.        & NCSN++ & ADM \\
channels mult.     & 2,2,2  & 1,2,2,3 \\
UNet size          & 56.4M  & 182.4M \\ 
Total size         & 112.9M & 364.8M \\ 
\midrule
\multicolumn{3}{l}{\textbf{\small Training config.}} \\
\hline
lr              & 1e-4 & 1e-4 \\
batch           & 1024 & 1024 \\
dropout            & 0.2 & 0.3 \\
loss               & Huber   & Huber \\
\(c\)                  & 0.03    &  0.06 \\
\(d\)                  & 40k     &  40k \\
\( P_\textit{mean} \) & -1.1 &  -1.1 \\
\( P_\textit{std} \) &  2.0  &  2.0  \\
\( \lambda_{\text{recon}} \) & 2e-5 & \parbox[t]{3.5cm}{\centering 2e-5, (\(\leq\) 180k)\\ 4e-5, (\(\leq\) 200k)\\ 6e-5, (\(\leq\) 260k) } \\  
\( i_{\text{skip}} \)        & 10 &  10 \\  
training iterations & 360k &  260k \\
\bottomrule
\end{tabular}
\label{tab:igct_training_configs}
\end{table}  

\begin{figure*}[t] 
    \centering
    \includegraphics[width=1.0\textwidth]{fig/appendix/inversion_collapse.pdf} 
    \caption{Inversion collapse observed during training on ImageNet64. The left image shows the input data. The middle image depicts the inversion collapse that occurred at iteration 220k, where leakage of signals in the noise latent can be visualized. The right image shows the inversion results at iteration 220k after appropriately increasing $\lambda_{\text{recon}}$ to 6e-5. The inversion images are generated by scaling the model's outputs by $1/80$, i.e., $ 1/t_\text{max}$, then clipping the values to the range [-3, 3] before denormalizing them to the range [0, 255]. }
    \vspace{-1.5em}
    \label{fig:inversion_collpase}
\end{figure*}

iGCT is trained with a continuous-time scheduler inspired by ECT \cite{ect}. To rigorously assess its independence from diffusion-based models, iGCT is trained from scratch rather than fine-tuned from a pre-trained diffusion model. Consequently, the training curriculum begins with an initial diffusion training stage, followed by consistency training with the step size halved every \(d\) iterations. In practice, we adopt the same noise sampling distribution \(p(t)\), same step function \(\Delta t (t) \), and same distance metric \( d(\cdot, \cdot) \) for both guided consistency training and inverse consistency training. 

For CIFAR-10, iGCT adopts the same UNet architecture as the baseline models. However, the overall model size is doubled, as iGCT comprises two UNets: one for the denoiser and one for the noiser. The Pseudo-Huber loss is employed as the distance metric, with a constant parameter \( c = 0.03 \). Consistency training is organized into nine stages, each comprising 400k iterations with the step size halved from the last stage. We found that training remains stable when the reconstruction weight \( \lambda_{\text{recon}} \) is fixed at \( 2 \times 10^{-5} \) throughout the entire training process.
 
For ImageNet64, iGCT employs a reduced ADM architecture \cite{dhariwal2021diffusionmodelsbeatgans} with smaller channel sizes to address computational constraints. A higher dropout rate and Pseudo-Huber loss with \( c = 0.06 \) is used, following prior works \cite{ect,song2023improved}. During our experiments, we observed that training on ImageNet64 is sensitive to the reconstruction weight. Keeping \(\lambda_{\text{recon}}\) fixed throughout training leads to inversion collapse, with significant signal leaked to the latent noise (see Fig. \ref{fig:inversion_collpase}). We found that increasing \(\lambda_{\text{recon}}\) to \( 4 \times 10^{-5} \) at iteration 1800 and to \( 6 \times 10^{-5} \) at iteration 2000 effectively stabilizes training and prevents collapse. This suggests that the reconstruction loss serves as a regularizer for iGCT. Additionally, we observed diminishing returns when training exceeded 240k iterations, leading us to stop at 260k iterations for our experiments. These findings indicate that alternative training strategies, such as framing iGCT as a multi-task learning problem \cite{kendall2018multi,liu2019loss}, and conducting a more sophisticated analysis of loss weighting, may be necessary to enhance stability and improve convergence. See Table \ref{tab:igct_training_configs} for a summary of the training configurations for iGCT.



\begin{table}[t]
\caption{Comparison of GPU hours across the methods used in our experiments on CIFAR-10.}
\centering
\begin{tabular}{|l|c|}
\hline
\textbf{Methods} & \textbf{A100 (40G) GPU hours} \\ \hline
CFG-EDM \cite{karras2022elucidating} & 312 \\ \hline
Guided-CD \cite{song2023consistency} & 3968 \\ \hline
iGCT (ours) & 2032 \\ \hline
\end{tabular}
\label{table:compute_resources}
\end{table}



\begin{figure*}[t!]  
    \centering
    \begin{subfigure}[b]{0.33\textwidth}
    \includegraphics[width=\textwidth]{fig/cls_exp_w1.png} 
        \caption{Accuracy on various ratios of augmented data, guidance scale w=1.}
    \end{subfigure}
    \begin{subfigure}[b]{0.33\textwidth}
    \includegraphics[width=\textwidth]{fig/cls_exp_w3.png} 
        \caption{Accuracy on various ratios of augmented data, guidance scale w=3.}
    \end{subfigure}
    \begin{subfigure}[b]{0.33\textwidth}
    \includegraphics[width=\textwidth]{fig/cls_exp_w5.png} 
        \caption{Accuracy on various ratios of augmented data, guidance scale w=5.}
    \end{subfigure}
    \begin{subfigure}[b]{0.33\textwidth}
    \includegraphics[width=\textwidth]{fig/cls_exp_w7.png} 
        \caption{Accuracy on various ratios of augmented data, guidance scale w=7.}
    \end{subfigure}
    \begin{subfigure}[b]{0.33\textwidth}
    \includegraphics[width=\textwidth]{fig/cls_exp_w9.png} 
        \caption{Accuracy on various ratios of augmented data, guidance scale w=9.}
    \end{subfigure}
    \caption{Comparison of synthesized methods, CFG-EDM vs iGCT, used for data augmentation in image classification. iGCT consistently improves accuracy. Conversely, augmentation data synthesized from CFG-EDM offers only limited gains.}
    \vspace{-1.5em}
    \label{fig:cls_results}
\end{figure*}


\vspace{-0.1cm}
\section{Application: Data Augmentation Under Different Guidance}
\vspace{-0.2cm}

In this section, we show the effectiveness of data augmentation with diffusion-based models, CFG-EDM and iGCT, across varying guidance scales for image classification on CIFAR-10 \cite{article}. High quality data augmentation has been shown to enhance classification performance \cite{yang2023imagedataaugmentationdeep}. Under high guidance, augmentation data generated from iGCT consistently improves accuracy. Conversely, augmentation data synthesized from CFG-EDM offers only limited gains. We describe the ratios of real to synthesized data, the classifier architecture, and the training setup in the following. 

\noindent{\bf Training Details.} We conduct classification experiments trained on six different mixtures of augmented data synthesized by iGCT and CFG-EDM: \(0\%\), \(20\%\), \(40\%\), \(80\%\), and \(100\%\). The ratio represents \(\textit{synthesized data} / \textit{real data}\). For example, \(0\%\) indicates that the training and validation sets contain only 50k of real samples from CIFAR-10, and \(20\%\) includes 50k real \textit{and} 10k synthesized samples. In terms of guidance scales, we choose \(w=1,3,5,7,9\) to synthesize the augmented data using iGCT and CFG-EDM. 
The augmented dataset is split 80/20 for training and validation. For testing, the model is evaluated on the CIFAR-10 test set with 10k samples and ground truth labels. 

The standard ResNet-18 \cite{he2015deepresiduallearningimage} is used to train on all different augmented datasets. All models are trained for 250 epochs, with batch size 64, using an Adam optimizer \cite{kingma2017adammethodstochasticoptimization}. For each augmentation dataset, we train the model six times under different seeds and report the average classification accuracy.

\noindent{\bf Results.} The classifier's accuracy, trained on augmented data synthesized by CFG-EDM and iGCT, is shown in Fig. \ref{fig:cls_results}. With \(w=1\) (no guidance), both iGCT and CFG-EDM provide comparable performance boosts. As guidance scale increases, iGCT shows more significant improvements than CFG-EDM. At high guidance and augmentation ratios, performance drops, but this effect occurs later for iGCT (e.g., at \(100\%\) augmentation and \(w=9\)), while CFG-EDM stops improving accuracy at \(w=7\). This experiment highlights the importance of high-quality data under high guidance, with iGCT outperforming CFG-EDM in data quality.

\section{Uncurated Results}
In this section, we present additional qualitative results to highlight the performance of our proposed iGCT method compared to the multi-step EDM baseline. These visualizations include both inversion and guidance tasks across the CIFAR-10 and ImageNet64 datasets. The results demonstrate iGCT's ability to maintain competitive quality with significantly fewer steps and minimal artifacts, showcasing the effectiveness of our approach.

\subsection{Inversion Results}
We provide additional visualization of the latent noise on both CIFAR-10 and ImageNet64 datasets. Fig. \ref{fig:CIFAR-10_inversion_reconstruction} and Fig. \ref{fig:im64_inversion_reconstruction} compare our 1-step iGCT with the multi-step EDM on inversion and reconstruction.  

\subsection{Editing Results}
In this section, we dump more uncurated editing results on ImageNet64's subgroups mentioned in Sec. \ref{sec:image-editing}. Fig. \ref{fig:im64_edit_1}--\ref{fig:im64_edit_4} illustrate a comparison between our 1-step iGCT and the multi-step EDM approach.

\subsection{Guidance Results}
In Section \ref{sec:guidance}, we demonstrated that iGCT provides a guidance solution without introducing the high-contrast artifacts commonly observed in CFG-based methods. Here, we present additional uncurated results on CIFAR-10 and ImageNet64. For CIFAR-10, iGCT achieves competitive performance compared to the baseline diffusion model, which requires multiple steps for generation. See Figs. \ref{fig:CIFAR-10_guided_1}--\ref{fig:CIFAR-10_guided_10}. For ImageNet64, although the visual quality of iGCT's generated images falls slightly short of expectations, this can be attributed to the smaller UNet architecture used—only 61\% of the baseline model size—and the need for a more robust training curriculum to prevent collapse, as discussed in Section \ref{appendix:bs-config}. Nonetheless, even at higher guidance levels, iGCT maintains style consistency, whereas CFG-based methods continue to suffer from pronounced high-contrast artifacts. See Figs. \ref{fig:im64_guided_1}--\ref{fig:im64_guided_4}.


\begin{figure*}[t]
    \centering
    \begin{subfigure}{0.48\textwidth}
        \centering
        \includegraphics[width=\linewidth]{fig/appendix/recon_c10_data.png}
        \caption{CIFAR-10: Original data}
    \end{subfigure}
    \begin{subfigure}{0.48\textwidth}
        \centering
        \includegraphics[width=\linewidth]{fig/appendix/recon_im64_data.png}
        \caption{ImageNet64: Original data}
    \end{subfigure}

    \begin{subfigure}{0.48\textwidth}
        \centering
        \includegraphics[width=\linewidth]{fig/appendix/inv_c10_edm.png}
    \end{subfigure}
    \begin{subfigure}{0.48\textwidth}
        \centering
        \includegraphics[width=\linewidth]{fig/appendix/inv_im64_edm.png}
    \end{subfigure}

    \begin{subfigure}{0.48\textwidth}
        \centering
        \includegraphics[width=\linewidth]{fig/appendix/recon_c10_edm.png}
        \caption{CIFAR-10: Inversion + reconstruction, EDM (18 NFE)}
    \end{subfigure}
    \begin{subfigure}{0.48\textwidth}
        \centering
        \includegraphics[width=\linewidth]{fig/appendix/recon_im64_edm.png}
        \caption{ImageNet64: Inversion + reconstruction, EDM (18 NFE)}
    \end{subfigure}

    \begin{subfigure}{0.48\textwidth}
        \centering
        \includegraphics[width=\linewidth]{fig/appendix/inv_c10_igct.png}
    \end{subfigure}
    \begin{subfigure}{0.48\textwidth}
        \centering
        \includegraphics[width=\linewidth]{fig/appendix/inv_im64_igct.png}
    \end{subfigure}

    \begin{subfigure}{0.48\textwidth}
        \centering
        \includegraphics[width=\linewidth]{fig/appendix/recon_c10_igct.png}
        \caption{CIFAR-10: Inversion + reconstruction, iGCT (1 NFE)}
    \end{subfigure}
    \begin{subfigure}{0.48\textwidth}
        \centering
        \includegraphics[width=\linewidth]{fig/appendix/recon_im64_igct.png}
        \caption{ImageNet64: Inversion + reconstruction, iGCT (1 NFE)}
    \end{subfigure}

    \caption{Comparison of inversion and reconstruction for CIFAR-10 (left) and ImageNet64 (right).}
    \label{fig:comparison_CIFAR-10_imagenet64}
\end{figure*}




\begin{figure*}[t]
    \centering

    % Left column: corgi -> golden retriever
    \begin{minipage}{0.48\textwidth}
        \centering
        \begin{subfigure}{0.48\textwidth}
            \includegraphics[width=\linewidth]{fig/appendix_edit_igct/src_corgi.png}
            \caption{Original: "corgi"}
        \end{subfigure}

        \begin{subfigure}{0.48\textwidth}
            \includegraphics[width=\linewidth]{fig/appendix_edit_edm/w=0_src_corgi_tar_golden_retriever.png}
            \caption{EDM (18 NFE), w=1}
        \end{subfigure}
        \begin{subfigure}{0.48\textwidth}
            \includegraphics[width=\linewidth]{fig/appendix_edit_edm/w=6_src_corgi_tar_golden_retriever.png}
            \caption{EDM (18 NFE), w=7}
        \end{subfigure}
        \begin{subfigure}{0.48\textwidth}
            \includegraphics[width=\linewidth]{fig/appendix_edit_igct/w=6_src_corgi_tar_golden_retriever.png}
            \caption{iGCT (1 NFE), w=7}
        \end{subfigure}
        \begin{subfigure}{0.48\textwidth}
            \includegraphics[width=\linewidth]{fig/appendix_edit_igct/w=0_src_corgi_tar_golden_retriever.png}
            \caption{iGCT (1 NFE), w=1}
        \end{subfigure}

        \caption{ImageNet64: "corgi" $\rightarrow$ "golden retriever"}
        \label{fig:im64_edit_1}
    \end{minipage}
    \hfill
    % Right column: zebra -> horse
    \begin{minipage}{0.48\textwidth}
        \centering
        \begin{subfigure}{0.48\textwidth}
            \includegraphics[width=\linewidth]{fig/appendix_edit_igct/src_zebra.png}
            \caption{Original: "zebra"}
        \end{subfigure}

        \begin{subfigure}{0.48\textwidth}
            \includegraphics[width=\linewidth]{fig/appendix_edit_edm/w=0_src_zebra_tar_horse.png}
            \caption{EDM (18 NFE), w=1}
        \end{subfigure}
        \begin{subfigure}{0.48\textwidth}
            \includegraphics[width=\linewidth]{fig/appendix_edit_edm/w=6_src_zebra_tar_horse.png}
            \caption{EDM (18 NFE), w=7}
        \end{subfigure}
        \begin{subfigure}{0.48\textwidth}
            \includegraphics[width=\linewidth]{fig/appendix_edit_igct/w=0_src_zebra_tar_horse.png}
            \caption{iGCT (1 NFE), w=1}
        \end{subfigure}
        \begin{subfigure}{0.48\textwidth}
            \includegraphics[width=\linewidth]{fig/appendix_edit_igct/w=6_src_zebra_tar_horse.png}
            \caption{iGCT (1 NFE), w=7}
        \end{subfigure}

        \caption{ImageNet64: "zebra" $\rightarrow$ "horse"}
        \label{fig:im64_edit_2}
    \end{minipage}

\end{figure*}

\begin{figure*}[t]
    \centering

    % Left column: broccoli -> cauliflower
    \begin{minipage}{0.48\textwidth}
        \centering
        \begin{subfigure}{0.48\textwidth}
            \includegraphics[width=\linewidth]{fig/appendix_edit_igct/src_broccoli.png}
            \caption{Original: "broccoli"}
        \end{subfigure}

        \begin{subfigure}{0.48\textwidth}
            \includegraphics[width=\linewidth]{fig/appendix_edit_edm/w=0_src_broccoli_tar_cauliflower.png}
            \caption{EDM (18 NFE), w=1}
        \end{subfigure}
        \begin{subfigure}{0.48\textwidth}
            \includegraphics[width=\linewidth]{fig/appendix_edit_edm/w=6_src_broccoli_tar_cauliflower.png}
            \caption{EDM (18 NFE), w=7}
        \end{subfigure}
        \begin{subfigure}{0.48\textwidth}
            \includegraphics[width=\linewidth]{fig/appendix_edit_igct/w=0_src_broccoli_tar_cauliflower.png}
            \caption{iGCT (1 NFE), w=1}
        \end{subfigure}
        \begin{subfigure}{0.48\textwidth}
            \includegraphics[width=\linewidth]{fig/appendix_edit_igct/w=6_src_broccoli_tar_cauliflower.png}
            \caption{iGCT (1 NFE), w=7}
        \end{subfigure}

        \caption{ImageNet64: "broccoli" $\rightarrow$ "cauliflower"}
        \label{fig:im64_edit_3}
    \end{minipage}
    \hfill
    % Right column: jaguar -> tiger
    \begin{minipage}{0.48\textwidth}
        \centering
        \begin{subfigure}{0.48\textwidth}
            \includegraphics[width=\linewidth]{fig/appendix_edit_igct/src_jaguar.png}
            \caption{Original: "jaguar"}
        \end{subfigure}

        \begin{subfigure}{0.48\textwidth}
            \includegraphics[width=\linewidth]{fig/appendix_edit_edm/w=0_src_jaguar_tar_tiger.png}
            \caption{EDM (18 NFE), w=1}
        \end{subfigure}
        \begin{subfigure}{0.48\textwidth}
            \includegraphics[width=\linewidth]{fig/appendix_edit_edm/w=6_src_jaguar_tar_tiger.png}
            \caption{EDM (18 NFE), w=7}
        \end{subfigure}
        \begin{subfigure}{0.48\textwidth}
            \includegraphics[width=\linewidth]{fig/appendix_edit_igct/w=0_src_jaguar_tar_tiger.png}
            \caption{iGCT (1 NFE), w=1}
        \end{subfigure}
        \begin{subfigure}{0.48\textwidth}
            \includegraphics[width=\linewidth]{fig/appendix_edit_igct/w=6_src_jaguar_tar_tiger.png}
            \caption{iGCT (1 NFE), w=7}
        \end{subfigure}

        \caption{ImageNet64: "jaguar" $\rightarrow$ "tiger"}
        \label{fig:im64_edit_4}
    \end{minipage}

\end{figure*}






\begin{figure*}[b]
    \centering
    % First image
    \begin{subfigure}{0.25\textwidth}
        \includegraphics[width=\linewidth]{fig/appendix_edm/0_0.0_middle_4x4_grid.png}
        \caption{CFG-EDM (18 NFE), w=1.0}
    \end{subfigure}
    \begin{subfigure}{0.25\textwidth}
        \includegraphics[width=\linewidth]{fig/appendix_edm/0_6.0_middle_4x4_grid.png}
        \caption{CFG-EDM (18 NFE), w=7.0}
    \end{subfigure}
    \begin{subfigure}{0.25\textwidth}
        \includegraphics[width=\linewidth]{fig/appendix_edm/0_12.0_middle_4x4_grid.png}
        \caption{CFG-EDM (18 NFE), w=13.0}
    \end{subfigure}
    \begin{subfigure}{0.25\textwidth}
        \includegraphics[width=\linewidth]{fig/appendix_igct/0_0.0_middle_4x4_grid.png}
        \caption{iGCT (1 NFE), w=1.0}
    \end{subfigure}
    \begin{subfigure}{0.25\textwidth}
        \includegraphics[width=\linewidth]{fig/appendix_igct/0_6.0_middle_4x4_grid.png}
        \caption{iGCT (1 NFE), w=7.0}
    \end{subfigure}
    % Third image
    \begin{subfigure}{0.25\textwidth}
        \includegraphics[width=\linewidth]{fig/appendix_igct/0_12.0_middle_4x4_grid.png}
        \caption{iGCT (1 NFE), w=13.0}
    \end{subfigure}
    \caption{CIFAR-10 "airplane"}
    \label{fig:CIFAR-10_guided_1}
\end{figure*}
\begin{figure*}[t]
    \centering
    % First image
    \begin{subfigure}{0.25\textwidth}
        \includegraphics[width=\linewidth]{fig/appendix_edm/1_0.0_middle_4x4_grid.png}
        \caption{CFG-EDM (18 NFE), w=1.0}
    \end{subfigure}
    \begin{subfigure}{0.25\textwidth}
        \includegraphics[width=\linewidth]{fig/appendix_edm/1_6.0_middle_4x4_grid.png}
        \caption{CFG-EDM (18 NFE), w=7.0}
    \end{subfigure}
    \begin{subfigure}{0.25\textwidth}
        \includegraphics[width=\linewidth]{fig/appendix_edm/1_12.0_middle_4x4_grid.png}
        \caption{CFG-EDM (18 NFE), w=13.0}
    \end{subfigure}
    \begin{subfigure}{0.25\textwidth}
        \includegraphics[width=\linewidth]{fig/appendix_igct/1_0.0_middle_4x4_grid.png}
        \caption{iGCT (1 NFE), w=1.0}
    \end{subfigure}
    % Second image
    \begin{subfigure}{0.25\textwidth}
        \includegraphics[width=\linewidth]{fig/appendix_igct/1_6.0_middle_4x4_grid.png}
        \caption{iGCT (1 NFE), w=7.0}
    \end{subfigure}
    % Third image
    \begin{subfigure}{0.25\textwidth}
        \includegraphics[width=\linewidth]{fig/appendix_igct/1_12.0_middle_4x4_grid.png}
        \caption{iGCT (1 NFE), w=13.0}
    \end{subfigure}
    \caption{CIFAR-10 "car"}
    \label{fig:CIFAR-10_guided_2}
\end{figure*}
\begin{figure*}[t]
    \centering
    % First image
    \begin{subfigure}{0.25\textwidth}
        \includegraphics[width=\linewidth]{fig/appendix_edm/2_0.0_middle_4x4_grid.png}
        \caption{CFG-EDM (18 NFE), w=1.0}
    \end{subfigure}
    \begin{subfigure}{0.25\textwidth}
        \includegraphics[width=\linewidth]{fig/appendix_edm/2_6.0_middle_4x4_grid.png}
        \caption{CFG-EDM (18 NFE), w=7.0}
    \end{subfigure}
    \begin{subfigure}{0.25\textwidth}
        \includegraphics[width=\linewidth]{fig/appendix_edm/2_12.0_middle_4x4_grid.png}
        \caption{CFG-EDM (18 NFE), w=13.0}
    \end{subfigure}
    \begin{subfigure}{0.25\textwidth}
        \includegraphics[width=\linewidth]{fig/appendix_igct/2_0.0_middle_4x4_grid.png}
        \caption{iGCT (1 NFE), w=1.0}
    \end{subfigure}
    % Second image
    \begin{subfigure}{0.25\textwidth}
        \includegraphics[width=\linewidth]{fig/appendix_igct/2_6.0_middle_4x4_grid.png}
        \caption{iGCT (1 NFE), w=7.0}
    \end{subfigure}
    % Third image
    \begin{subfigure}{0.25\textwidth}
        \includegraphics[width=\linewidth]{fig/appendix_igct/2_12.0_middle_4x4_grid.png}
        \caption{iGCT (1 NFE), w=13.0}
    \end{subfigure}
    \caption{CIFAR-10 "bird"}
    \label{fig:CIFAR-10_guided_3}
\end{figure*}
\begin{figure*}[t]
    \centering
    % First image
    \begin{subfigure}{0.25\textwidth}
        \includegraphics[width=\linewidth]{fig/appendix_edm/3_0.0_middle_4x4_grid.png}
        \caption{CFG-EDM (18 NFE), w=1.0}
    \end{subfigure}
    \begin{subfigure}{0.25\textwidth}
        \includegraphics[width=\linewidth]{fig/appendix_edm/3_6.0_middle_4x4_grid.png}
        \caption{CFG-EDM (18 NFE), w=7.0}
    \end{subfigure}
    \begin{subfigure}{0.25\textwidth}
        \includegraphics[width=\linewidth]{fig/appendix_edm/3_12.0_middle_4x4_grid.png}
        \caption{CFG-EDM (18 NFE), w=13.0}
    \end{subfigure}
    \begin{subfigure}{0.25\textwidth}
        \includegraphics[width=\linewidth]{fig/appendix_igct/3_0.0_middle_4x4_grid.png}
        \caption{iGCT (1 NFE), w=1.0}
    \end{subfigure}
    % Second image
    \begin{subfigure}{0.25\textwidth}
        \includegraphics[width=\linewidth]{fig/appendix_igct/3_6.0_middle_4x4_grid.png}
        \caption{iGCT (1 NFE), w=7.0}
    \end{subfigure}
    % Third image
    \begin{subfigure}{0.25\textwidth}
        \includegraphics[width=\linewidth]{fig/appendix_igct/3_12.0_middle_4x4_grid.png}
        \caption{iGCT (1 NFE), w=13.0}
    \end{subfigure}
    \caption{CIFAR-10 "cat"}
    \label{fig:CIFAR-10_guided_4}
\end{figure*}
\begin{figure*}[t]
    \centering
    % First image
    \begin{subfigure}{0.25\textwidth}
        \includegraphics[width=\linewidth]{fig/appendix_edm/4_0.0_middle_4x4_grid.png}
        \caption{CFG-EDM (18 NFE), w=1.0}
    \end{subfigure}
    \begin{subfigure}{0.25\textwidth}
        \includegraphics[width=\linewidth]{fig/appendix_edm/4_6.0_middle_4x4_grid.png}
        \caption{CFG-EDM (18 NFE), w=7.0}
    \end{subfigure}
    \begin{subfigure}{0.25\textwidth}
        \includegraphics[width=\linewidth]{fig/appendix_edm/4_12.0_middle_4x4_grid.png}
        \caption{CFG-EDM (18 NFE), w=13.0}
    \end{subfigure}
    \begin{subfigure}{0.25\textwidth}
        \includegraphics[width=\linewidth]{fig/appendix_igct/4_0.0_middle_4x4_grid.png}
        \caption{iGCT (1 NFE), w=1.0}
    \end{subfigure}
    % Second image
    \begin{subfigure}{0.25\textwidth}
        \includegraphics[width=\linewidth]{fig/appendix_igct/4_6.0_middle_4x4_grid.png}
        \caption{iGCT (1 NFE), w=7.0}
    \end{subfigure}
    % Third image
    \begin{subfigure}{0.25\textwidth}
        \includegraphics[width=\linewidth]{fig/appendix_igct/4_12.0_middle_4x4_grid.png}
        \caption{iGCT (1 NFE), w=13.0}
    \end{subfigure}
    \caption{CIFAR-10 "deer"}
    \label{fig:CIFAR-10_guided_5}
\end{figure*}
\begin{figure*}[t]
    \centering
    % First image
    \begin{subfigure}{0.25\textwidth}
        \includegraphics[width=\linewidth]{fig/appendix_edm/5_0.0_middle_4x4_grid.png}
        \caption{CFG-EDM (18 NFE), w=1.0}
    \end{subfigure}
    \begin{subfigure}{0.25\textwidth}
        \includegraphics[width=\linewidth]{fig/appendix_edm/5_6.0_middle_4x4_grid.png}
        \caption{CFG-EDM (18 NFE), w=7.0}
    \end{subfigure}
    \begin{subfigure}{0.25\textwidth}
        \includegraphics[width=\linewidth]{fig/appendix_edm/5_12.0_middle_4x4_grid.png}
        \caption{CFG-EDM (18 NFE), w=13.0}
    \end{subfigure}
    \begin{subfigure}{0.25\textwidth}
        \includegraphics[width=\linewidth]{fig/appendix_igct/5_0.0_middle_4x4_grid.png}
        \caption{iGCT (1 NFE), w=1.0}
    \end{subfigure}
    % Second image
    \begin{subfigure}{0.25\textwidth}
        \includegraphics[width=\linewidth]{fig/appendix_igct/5_6.0_middle_4x4_grid.png}
        \caption{iGCT (1 NFE), w=7.0}
    \end{subfigure}
    % Third image
    \begin{subfigure}{0.25\textwidth}
        \includegraphics[width=\linewidth]{fig/appendix_igct/5_12.0_middle_4x4_grid.png}
        \caption{iGCT (1 NFE), w=13.0}
    \end{subfigure}
    \caption{CIFAR-10 "dog"}
    \label{fig:CIFAR-10_guided_6}
\end{figure*}
\begin{figure*}[t]
    \centering
    % First image
    \begin{subfigure}{0.25\textwidth}
        \includegraphics[width=\linewidth]{fig/appendix_edm/6_0.0_middle_4x4_grid.png}
        \caption{CFG-EDM (18 NFE), w=1.0}
    \end{subfigure}
    \begin{subfigure}{0.25\textwidth}
        \includegraphics[width=\linewidth]{fig/appendix_edm/6_6.0_middle_4x4_grid.png}
        \caption{CFG-EDM (18 NFE), w=7.0}
    \end{subfigure}
    \begin{subfigure}{0.25\textwidth}
        \includegraphics[width=\linewidth]{fig/appendix_edm/6_12.0_middle_4x4_grid.png}
        \caption{CFG-EDM (18 NFE), w=13.0}
    \end{subfigure}
    \begin{subfigure}{0.25\textwidth}
        \includegraphics[width=\linewidth]{fig/appendix_igct/6_0.0_middle_4x4_grid.png}
        \caption{iGCT (1 NFE), w=1.0}
    \end{subfigure}
    % Second image
    \begin{subfigure}{0.25\textwidth}
        \includegraphics[width=\linewidth]{fig/appendix_igct/6_6.0_middle_4x4_grid.png}
        \caption{iGCT (1 NFE), w=7.0}
    \end{subfigure}
    % Third image
    \begin{subfigure}{0.25\textwidth}
        \includegraphics[width=\linewidth]{fig/appendix_igct/6_12.0_middle_4x4_grid.png}
        \caption{iGCT (1 NFE), w=13.0}
    \end{subfigure}
    \caption{CIFAR-10 "frog"}
    \label{fig:CIFAR-10_guided_7}
\end{figure*}
\begin{figure*}[t]
    \centering
    % First image
    \begin{subfigure}{0.25\textwidth}
        \includegraphics[width=\linewidth]{fig/appendix_edm/7_0.0_middle_4x4_grid.png}
        \caption{CFG-EDM (18 NFE), w=1.0}
    \end{subfigure}
    \begin{subfigure}{0.25\textwidth}
        \includegraphics[width=\linewidth]{fig/appendix_edm/7_6.0_middle_4x4_grid.png}
        \caption{CFG-EDM (18 NFE), w=7.0}
    \end{subfigure}
    \begin{subfigure}{0.25\textwidth}
        \includegraphics[width=\linewidth]{fig/appendix_edm/7_12.0_middle_4x4_grid.png}
        \caption{CFG-EDM (18 NFE), w=13.0}
    \end{subfigure}
    \begin{subfigure}{0.25\textwidth}
        \includegraphics[width=\linewidth]{fig/appendix_igct/7_0.0_middle_4x4_grid.png}
        \caption{iGCT (1 NFE), w=1.0}
    \end{subfigure}
    % Second image
    \begin{subfigure}{0.25\textwidth}
        \includegraphics[width=\linewidth]{fig/appendix_igct/7_6.0_middle_4x4_grid.png}
        \caption{iGCT (1 NFE), w=7.0}
    \end{subfigure}
    % Third image
    \begin{subfigure}{0.25\textwidth}
        \includegraphics[width=\linewidth]{fig/appendix_igct/7_12.0_middle_4x4_grid.png}
        \caption{iGCT (1 NFE), w=13.0}
    \end{subfigure}
    \caption{CIFAR-10 "horse"}
    \label{fig:CIFAR-10_guided_8}
\end{figure*}
\begin{figure*}[t]
    \centering
    % First image
    \begin{subfigure}{0.25\textwidth}
        \includegraphics[width=\linewidth]{fig/appendix_edm/8_0.0_middle_4x4_grid.png}
        \caption{CFG-EDM (18 NFE), w=1.0}
    \end{subfigure}
    \begin{subfigure}{0.25\textwidth}
        \includegraphics[width=\linewidth]{fig/appendix_edm/8_6.0_middle_4x4_grid.png}
        \caption{CFG-EDM (18 NFE), w=7.0}
    \end{subfigure}
    \begin{subfigure}{0.25\textwidth}
        \includegraphics[width=\linewidth]{fig/appendix_edm/8_12.0_middle_4x4_grid.png}
        \caption{CFG-EDM (18 NFE), w=13.0}
    \end{subfigure}
    \begin{subfigure}{0.25\textwidth}
        \includegraphics[width=\linewidth]{fig/appendix_igct/8_0.0_middle_4x4_grid.png}
        \caption{iGCT (1 NFE), w=1.0}
    \end{subfigure}
    % Second image
    \begin{subfigure}{0.25\textwidth}
        \includegraphics[width=\linewidth]{fig/appendix_igct/8_6.0_middle_4x4_grid.png}
        \caption{iGCT (1 NFE), w=7.0}
    \end{subfigure}
    % Third image
    \begin{subfigure}{0.25\textwidth}
        \includegraphics[width=\linewidth]{fig/appendix_igct/8_12.0_middle_4x4_grid.png}
        \caption{iGCT (1 NFE), w=13.0}
    \end{subfigure}
    \caption{CIFAR-10 "ship"}
    \label{fig:CIFAR-10_guided_9}
\end{figure*}
\begin{figure*}[t]
    \centering
    % First image
    \begin{subfigure}{0.25\textwidth}
        \includegraphics[width=\linewidth]{fig/appendix_edm/9_0.0_middle_4x4_grid.png}
        \caption{CFG-EDM (18 NFE), w=1.0}
    \end{subfigure}
    \begin{subfigure}{0.25\textwidth}
        \includegraphics[width=\linewidth]{fig/appendix_edm/9_6.0_middle_4x4_grid.png}
        \caption{CFG-EDM (18 NFE), w=7.0}
    \end{subfigure}
    \begin{subfigure}{0.25\textwidth}
        \includegraphics[width=\linewidth]{fig/appendix_edm/9_12.0_middle_4x4_grid.png}
        \caption{CFG-EDM (18 NFE), w=13.0}
    \end{subfigure}
    \begin{subfigure}{0.25\textwidth}
        \includegraphics[width=\linewidth]{fig/appendix_igct/9_0.0_middle_4x4_grid.png}
        \caption{iGCT (1 NFE), w=1.0}
    \end{subfigure}
    % Second image
    \begin{subfigure}{0.25\textwidth}
        \includegraphics[width=\linewidth]{fig/appendix_igct/9_6.0_middle_4x4_grid.png}
        \caption{iGCT (1 NFE), w=7.0}
    \end{subfigure}
    % Third image
    \begin{subfigure}{0.25\textwidth}
        \includegraphics[width=\linewidth]{fig/appendix_igct/9_12.0_middle_4x4_grid.png}
        \caption{iGCT (1 NFE), w=13.0}
    \end{subfigure}
    \caption{CIFAR-10 "truck"}
    \label{fig:CIFAR-10_guided_10}
\end{figure*}


\begin{figure*}[b]
    \centering
    % First image
    \begin{subfigure}{0.25\textwidth}
        \includegraphics[width=\linewidth]{fig/appendix_im64_edm/edm_class_291_w=0.0.png}
        \caption{CFG-EDM (18 NFE), w=1.0}
    \end{subfigure}
    \begin{subfigure}{0.25\textwidth}
        \includegraphics[width=\linewidth]{fig/appendix_im64_edm/edm_class_291_w=6.0.png}
        \caption{CFG-EDM (18 NFE), w=7.0}
    \end{subfigure}
    \begin{subfigure}{0.25\textwidth}
        \includegraphics[width=\linewidth]{fig/appendix_im64_edm/edm_class_291_w=12.0.png}
        \caption{CFG-EDM (18 NFE), w=13.0}
    \end{subfigure}
    \begin{subfigure}{0.25\textwidth}
        \includegraphics[width=\linewidth]{fig/appendix_im64_igct/class_291_w=0.0.png}
        \caption{iGCT (2 NFE), w=1.0}
    \end{subfigure}
    \begin{subfigure}{0.25\textwidth}
        \includegraphics[width=\linewidth]{fig/appendix_im64_igct/class_291_w=6.0.png}
        \caption{iGCT (2 NFE), w=7.0}
    \end{subfigure}
    % Third image
    \begin{subfigure}{0.25\textwidth}
        \includegraphics[width=\linewidth]{fig/appendix_im64_igct/class_291_w=12.0.png}
        \caption{iGCT (2 NFE), w=13.0}
    \end{subfigure}
    \caption{ImageNet64 "lion"}
    \label{fig:im64_guided_1}
\end{figure*}



\begin{figure*}[b]
    \centering
    % First image
    \begin{subfigure}{0.25\textwidth}
        \includegraphics[width=\linewidth]{fig/appendix_im64_edm/edm_class_292_w=0.0.png}
        \caption{CFG-EDM (18 NFE), w=1.0}
    \end{subfigure}
    \begin{subfigure}{0.25\textwidth}
        \includegraphics[width=\linewidth]{fig/appendix_im64_edm/edm_class_292_w=6.0.png}
        \caption{CFG-EDM (18 NFE), w=7.0}
    \end{subfigure}
    \begin{subfigure}{0.25\textwidth}
        \includegraphics[width=\linewidth]{fig/appendix_im64_edm/edm_class_292_w=12.0.png}
        \caption{CFG-EDM (18 NFE), w=13.0}
    \end{subfigure}
    \begin{subfigure}{0.25\textwidth}
        \includegraphics[width=\linewidth]{fig/appendix_im64_igct/class_292_w=0.0.png}
        \caption{iGCT (2 NFE), w=1.0}
    \end{subfigure}
    \begin{subfigure}{0.25\textwidth}
        \includegraphics[width=\linewidth]{fig/appendix_im64_igct/class_292_w=6.0.png}
        \caption{iGCT (2 NFE), w=7.0}
    \end{subfigure}
    % Third image
    \begin{subfigure}{0.25\textwidth}
        \includegraphics[width=\linewidth]{fig/appendix_im64_igct/class_292_w=12.0.png}
        \caption{iGCT (2 NFE), w=13.0}
    \end{subfigure}
    \caption{ImageNet64 "tiger"}
    \label{fig:im64_guided_2}
\end{figure*}


\begin{figure*}[b]
    \centering
    % First image
    \begin{subfigure}{0.25\textwidth}
        \includegraphics[width=\linewidth]{fig/appendix_im64_edm/edm_class_28_w=0.0.png}
        \caption{CFG-EDM (18 NFE), w=1.0}
    \end{subfigure}
    \begin{subfigure}{0.25\textwidth}
        \includegraphics[width=\linewidth]{fig/appendix_im64_edm/edm_class_28_w=6.0.png}
        \caption{CFG-EDM (18 NFE), w=7.0}
    \end{subfigure}
    \begin{subfigure}{0.25\textwidth}
        \includegraphics[width=\linewidth]{fig/appendix_im64_edm/edm_class_28_w=12.0.png}
        \caption{CFG-EDM (18 NFE), w=13.0}
    \end{subfigure}
    \begin{subfigure}{0.25\textwidth}
        \includegraphics[width=\linewidth]{fig/appendix_im64_igct/class_28_w=0.0.png}
        \caption{iGCT (2 NFE), w=1.0}
    \end{subfigure}
    \begin{subfigure}{0.25\textwidth}
        \includegraphics[width=\linewidth]{fig/appendix_im64_igct/class_28_w=6.0.png}
        \caption{iGCT (2 NFE), w=7.0}
    \end{subfigure}
    % Third image
    \begin{subfigure}{0.25\textwidth}
        \includegraphics[width=\linewidth]{fig/appendix_im64_igct/class_28_w=12.0.png}
        \caption{iGCT (2 NFE), w=13.0}
    \end{subfigure}
    \caption{ImageNet64 "salamander"}
    \label{fig:im64_guided_3}
\end{figure*}


\begin{figure*}[b]
    \centering
    % First image
    \begin{subfigure}{0.25\textwidth}
        \includegraphics[width=\linewidth]{fig/appendix_im64_edm/edm_class_407_w=0.0.png}
        \caption{CFG-EDM (18 NFE), w=1.0}
    \end{subfigure}
    \begin{subfigure}{0.25\textwidth}
        \includegraphics[width=\linewidth]{fig/appendix_im64_edm/edm_class_407_w=6.0.png}
        \caption{CFG-EDM (18 NFE), w=7.0}
    \end{subfigure}
    \begin{subfigure}{0.25\textwidth}
        \includegraphics[width=\linewidth]{fig/appendix_im64_edm/edm_class_407_w=12.0.png}
        \caption{CFG-EDM (18 NFE), w=13.0}
    \end{subfigure}
    \begin{subfigure}{0.25\textwidth}
        \includegraphics[width=\linewidth]{fig/appendix_im64_igct/class_407_w=0.0.png}
        \caption{iGCT (2 NFE), w=1.0}
    \end{subfigure}
    \begin{subfigure}{0.25\textwidth}
        \includegraphics[width=\linewidth]{fig/appendix_im64_igct/class_407_w=6.0.png}
        \caption{iGCT (2 NFE), w=7.0}
    \end{subfigure}
    % Third image
    \begin{subfigure}{0.25\textwidth}
        \includegraphics[width=\linewidth]{fig/appendix_im64_igct/class_407_w=12.0.png}
        \caption{iGCT (2 NFE), w=13.0}
    \end{subfigure}
    \caption{ImageNet64 "ambulance"}
    \label{fig:im64_guided_4}
\end{figure*}

\end{document}
