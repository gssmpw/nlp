% This must be in the first 5 lines to tell arXiv to use pdfLaTeX, which is strongly recommended.
\pdfoutput=1
% In particular, the hyperref package requires pdfLaTeX in order to break URLs across lines.

\documentclass[11pt]{article}

% Change "review" to "final" to generate the final (sometimes called camera-ready) version.
% Change to "preprint" to generate a non-anonymous version with page numbers.
\usepackage{acl}
% \usepackage[review]{acl}

% Standard package includes
\usepackage{times}
\usepackage{latexsym}

% For proper rendering and hyphenation of words containing Latin characters (including in bib files)
\usepackage[T1]{fontenc}
% For Vietnamese characters
% \usepackage[T5]{fontenc}
% See https://www.latex-project.org/help/documentation/encguide.pdf for other character sets

% This assumes your files are encoded as UTF8
\usepackage[utf8]{inputenc}

% This is not strictly necessary, and may be commented out,
% but it will improve the layout of the manuscript,
% and will typically save some space.
\usepackage{microtype}

\usepackage{amsfonts}

% This is also not strictly necessary, and may be commented out.
% However, it will improve the aesthetics of text in
% the typewriter font.
\usepackage{inconsolata}

%Including images in your LaTeX document requires adding
%additional package(s)
\usepackage{graphicx}
\usepackage{hyperref}
\usepackage{url}
\usepackage{algorithm}
\usepackage{algorithmic}
\usepackage{amsmath}
% \usepackage{algpseudocode}
% \usepackage[numbers]{natbib}
% \setcitestyle{numbers}
% \usepackage{natbib}
% \usepackage[english]{babel}

\usepackage{colortbl}
% \usepackage[table,xcdraw]{xcolor}
\usepackage{multirow}
\usepackage{multicol}
\usepackage{float}
\usepackage{booktabs}
\usepackage{tcolorbox}

% If the title and author information does not fit in the area allocated, uncomment the following
%
%\setlength\titlebox{<dim>}
%
% and set <dim> to something 5cm or larger.

\title{SSH: Sparse Spectrum Adaptation via Discrete Hartley Transformation}

% Author information can be set in various styles:
% For several authors from the same institution:
% \author{Author 1 \and ... \and Author n \\
%         Address line \\ ... \\ Address line}
% if the names do not fit well on one line use
%         Author 1 \\ {\bf Author 2} \\ ... \\ {\bf Author n} \\
% For authors from different institutions:
% \author{Author 1 \\ Address line \\  ... \\ Address line
%         \And  ... \And
%         Author n \\ Address line \\ ... \\ Address line}
% To start a separate ``row'' of authors use \AND, as in
% \author{Author 1 \\ Address line \\  ... \\ Address line
%         \AND
%         Author 2 \\ Address line \\ ... \\ Address line \And
%         Author 3 \\ Address line \\ ... \\ Address line}

% \author{Yixian Shen \\
%   Affiliation / Address line 1 \\
%   Affiliation / Address line 2 \\
%   Affiliation / Address line 3 \\
%   \texttt{email@domain} \\\And
%   Second Author \\
%   Affiliation / Address line 1 \\
%   Affiliation / Address line 2 \\
%   Affiliation / Address line 3 \\
%   \texttt{email@domain} \\}
%Yixian Shen,, JIA-HONG HUANG, , Andy D. Pimentel, and Anuj Pathania
\author{
 \textbf{Yixian Shen}, 
    \textbf{Qi Bi}, 
    \textbf{Jia-Hong Huang}, 
    \textbf{Hongyi Zhu}, 
    % \\
    \textbf{Andy D. Pimentel}, 
    \textbf{Anuj Pathania} \\
    University of Amsterdam, Amsterdam, the Netherlands \\
\\
 \texttt{\{y.shen, q.bi, j.huang, h.zhu, a.d.pimentel, a.pathania\}@uva.nl}
}

\begin{document}
\maketitle
\begin{abstract}
Low-rank adaptation (LoRA) has been demonstrated effective in reducing the trainable parameter number when fine-tuning a large foundation model (LLM). However, it still encounters computational and memory challenges when scaling to larger models or addressing more complex task adaptation.
In this work, we introduce \textbf{S}parse \textbf{S}pectrum Adaptation via Discrete \textbf{H}artley Transformation (SSH), a novel approach that significantly reduces the number of trainable parameters while enhancing model performance. 
It selects the most informative spectral components across all layers, under the guidance of the initial weights after a discrete Hartley transformation (DHT). 
The lightweight inverse DHT then projects the spectrum back into the spatial domain for updates. 
Extensive experiments across both single-modality tasks—such as language understanding and generation—and multi-modality tasks—such as visual-text understanding—demonstrate that SSH outperforms existing parameter-efficient fine-tuning (PEFT) methods while achieving 
substantial reductions in computational cost and memory requirements. 
% For instance, during instruction tuning on the LLaMA3.1 8B model, SSH achieves higher accuracy with only 0.048M trainable parameters compared to LoRA's 33.5M, while reducing computational intensity up to 55\% compared to FourierFT. 
% The source code will be publicly available.
\end{abstract}

\section{Introduction}\label{sec:intro}

In computational finance, Monte Carlo simulations are used extensively to estimate the expected value of financial payoffs based on the solution of stochastic differential equations (SDEs) which model the evolution of stock prices, interest rates, exchange rates and other quantities \cite{glasserman04}.  Monte Carlo methods are very general and flexible, but for high accuracy it requires generating a large number of costly SDE path approximations, which has motivated research into a number of variance reduction or, equivalently, cost reduction techniques. One such method is
Multilevel Monte Carlo (MLMC), which was proposed in \cite{GILES2008} and was adapted for various applications that are summarised in \cite{Giles_overview17} and successfully combined with other methods such as quasi-Monte Carlo methods. The main idea of MLMC is to approximate the payoff using different time stepping resolutions when numerically solving the underlying SDE and to generate an optimal number of samples on each level, such that the overall computational cost is minimised subject to the desired bound on the variance. %, such that the total computational cost is minimised. 
The computational savings come from the fact that most samples are computed on the coarser levels and hence are less expensive while only a few samples from the finest levels are required \cite{GILES2008}.


Among the directions in which the computational cost 
of MLMC methods could further be reduced, an important avenue is the use of lower precision calculations, especially for the first Monte Carlo levels where the targeted accuracy is relatively low. 
 An overview of the research on mixed precision for the standard Monte Carlo (MC) framework is provided in \cite{ChowMixedPrecisionStandardMC} but only a few references study the potential of low precision computation in the MLMC framework \cite{Rounding_error_oliver}. To the best of our knowledge, the only MLMC framework with customised precision in the literature is \cite{brugger2014mixed}, but they use a uniform precision for all operations on each Monte Carlo level instead of optimising 
 the precision of each intermediary variable to reduce as much as possible the cost of path generation.
 
An important motivation for an MLMC framework with variable precision would be performing the low precision computations on reconfigurable hardware devices such as Field Programmable Gate Arrays (FPGAs). FPGAs contain customizable logic blocks and connectors that make it easy to adapt the digital circuit architecture for a specific application, leading to a highly parallel and optimised implementation. Therefore they are successfully exploited in applications that require high speed and have high computational workload, such as signal processing \cite{woods2008fpga}, and real time applications like high frequency trading \cite{HFT1,HFT2}. That is why a number of previous works in hardware architecture design implemented the MLMC algorithm to price financial options using FPGAs as accelerators, which resulted in improved speed and power efficiency compared to full CPU architectures \cite{Schryver2013AMM}. The paper \cite{lindsey2016domain} also proposed 
a Domain Specific Language to automate the configuration of FPGAs for this specific application. However, only \cite{brugger2014mixed} proposed a heuristic to reduce the precision in calculations.

In addition, all aforementioned works considered that the random number generation (RNG) is performed in single or double precision. Yet in most cases an important portion of the workload in the overall MLMC simulation comes from the RNG and in \cite{brugger2014mixed} this limited the total computational savings.
To reduce the cost of MLMC simulations in particular those based on the Geometric Brownian Motion (GBM), \cite{approximateICDF_Oliver, NestedOliver} have proposed to use approximate random numbers that are generated by applying an approximation of the inverse CDF to uniform random numbers. In \cite{NestedOliver}, the authors proposed a way to integrate these lower precision random variables into a \textit{nested} MLMC framework and completed a numerical analysis to bound the resulting error at each MC level by a product of the time step and the error in the random number approximation. The same authors show in \cite{approximateICDF_Oliver} that using approximate random variables reduces the cost of path generation by a factor 7.


In this paper we propose a nested MLMC framework that combines the use of approximate random normal variables and lower precision calculations to reduce the computational cost of MLMC even further than \cite{brugger2014mixed,NestedOliver}. We illustrate the efficiency of our framework in Matlab, after making several assumptions on the cost of operations and size of the errors that we carefully justify. We focus on the case of GBM and use the approximate RNG methods presented in \cite{approximateICDF_Oliver} as well as a new slightly modified method that combines CDF inversion and the central limit theorem. To choose the precision of the variables in the low precision path generation, we introduce a novel method to optimise the bit-widths. This optimisation is performed before the main path generation loop is executed and is based on a linear model of the payoff error  
due to rounding when computing in low precision. The error model relies on algorithmic differentiation in a similar manner to \cite{unifying-bwoptim,bitwidth-AD,ADAPT}. The bit-width optimisation procedure can be performed off-line, so this stage can be excluded from the on-line time complexity of our framework. The user specified desired accuracy is then enforced by calculating on-line the number of samples that need to be generated.

In terms of hardware design, we suggest implementing the low precision path generation on FPGAs and the full-precision ones on a CPU or GPU. 
The FPGA offers enough flexibility to define a separate bit-width for every variable in the low precision path generation, and can be reconfigured periodically to update the bit-widths when the market parameters have changed considerably. 


The paper is organized as follows : \Cref{sec:MLMC} introduces MLMC and nested MLMC to make clear the estimator that is implemented in our framework. Then in \Cref{sec:RNG} we detail the methods that could be used to obtain approximate random normally distributed numbers very cheaply for the low precision path generation. In \Cref{sec:error_model} and \Cref{sec:costModel} we propose an error model and a cost model (resp.) that we then use to formulate the optimisation problem that is solved to obtain the optimal bit-widths of fixed point variables in \Cref{sec:optimisation}. Finally we summarise our results and future directions in \Cref{sec:conclusion}.



\section{Related Work}
\label{sec:related_work}

The original investigation \cite{gibson1979ecological} on the relationship between visual perception and human action defines \emph{affordance} as the opportunities for interaction with the surrounding environment. Behavioral studies on regular and cognitively impaired persons have shown evidence that perception results in both visual and motor signals in the human brain. An extended study \cite{anderson2002attentional} shows that visual attention to the spatial characteristics of the perceived objects initiates automatic motor signals for different actions. In computer vision, human affordance learning involves novel pose prediction such that the estimated pose represents a valid human action within the scene context. The task is fundamental to many problems requiring robust semantic reasoning about the environment, such as human motion synthesis \cite{wang2021scene} and scene-aware human pose generation \cite{wang2017binge, roy2016multi, zhang2022inpaint, yao2023scene}.

Earlier methods of affordance learning have explored knowledge mining \cite{zhu2014reasoning} and multimodal feature cues \cite{roy2016multi} to address the problem. In \cite{zhu2014reasoning}, the authors use a Markov Logic Network for constructing a knowledge base by extracting several object attributes from different image and metadata sources, which can perform various downstream visual inference tasks without any additional classifier, including zero-shot affordance prediction. In \cite{roy2016multi}, the authors use depth map, surface normals, and segmentation map as multimodal cues to train a multi-scale convolutional neural network (CNN) for scene-level semantic label assignment associated with specific human actions. In \cite{do2018affordancenet}, the authors design a multi-branch end-to-end CNN with two separate pathways for object detection and affordance label assignment to achieve high real-time inference throughput. Researchers \cite{chuang2018learning} have also explored socially imposed constraints for affordance learning. In \cite{chuang2018learning}, the authors propose a graph neural network (GNN) to propagate contextual scene information from egocentric views for action-object affordance reasoning.

Probabilistic modeling of scene-aware human motion generation also involves semantic reasoning of human interaction with the environment. Initial works on human motion synthesis have taken different architectural approaches, such as sequence-to-sequence models \cite{barsoum2018hp}, generative adversarial networks (GAN) \cite{barsoum2018hp, cai2018deep, yang2018pose}, graph convolutional networks (GCN) \cite{yan2019convolutional}, and variational autoencoders (VAE) \cite{guo2020action2motion}. However, these methods have mostly ignored the role of environmental semantics. Due to potential uncertainty in human motion, in a recent approach \cite{wang2021scene}, the authors address such motion synthesis with a GAN conditioned on scene attributes and motion trajectory to predict probable body pose dynamics.

One key challenge of human affordance generation in 2D scenes is the lack of large-scale datasets with rich pose annotations. In \cite{wang2017binge}, the authors compile the only public dataset of annotated human body poses in complex 2D indoor scenes by extracting frames from sitcom videos. Aiming to generate a contextually valid human affordance at a user-defined location, the authors propose sampling the scale and deformation parameters for an existing human pose template using a VAE conditioned on the localized image patches as scene context. In \cite{zhang2022inpaint}, the authors introduce a two-stage GAN architecture for achieving a similar goal by estimating the affine bounding box parameters to localize a probable human in the scene and then generating a potential body pose at that location. The method uses the input scene, corresponding depth, and segmentation maps as semantic guidance. In \cite{yao2023scene}, the authors propose a transformer-based approach with knowledge distillation for generating human affordances in 2D indoor scenes.




\section{Methodology}
\paragraph{Preliminaries.}
We primarily focus on the homologous model merging, in which $\boldsymbol{\theta}_i$ all come from the same base model $\boldsymbol{\theta}_{\rm{base}}$. Given $K$ tasks $\{T_1,T_2,\cdots,T_K\}$ and $K$ corresponding fine-tuned models with parameters $\{\boldsymbol{\theta}_1,\boldsymbol{\theta}_2,\cdots,\boldsymbol{\theta}_K\}$, model merging aims to combine $K$ fine-tuned models into one single model simultaneously performing on $\{T_1,T_2,\cdots,T_K\}$ without post-training~\cite{method_p1_1,method_p1_2}.
Task vector~\cite{ilharco2023editing,yang2024adamerging} is a key element in merging method which could enhances the base model‘s ability or enable the model to handle other tasks. Specifically, for task $T_i$, the task vector $\boldsymbol\tau_i\in \mathbb{R}^D$ is defined as the vector obtained by subtracting the SFT weights $\boldsymbol{\theta}_i$ from the base model weight
$\boldsymbol{\theta}_{\rm{base}}$, \emph{i.e.}, $\boldsymbol\tau_i=\boldsymbol{\theta}_i-\boldsymbol{\theta}_{\rm{base}}$. The merged model could be denoted as $\boldsymbol{\theta}_m=\boldsymbol{\theta}_{\rm{base}}+\sum_i \lambda_i\boldsymbol{\tau}_i$, which $\lambda_i$ is the scaling factor measuring the importance of task vector. For clarification, we also denote the neuron set in $\boldsymbol{\theta}_i$ as $\mathcal{N}_i$, the neuron set in $\boldsymbol{\tau}_i$ as $\mathcal{T}_i$.



\begin{algorithm}[!ht]
    \caption{LED-Merging}
    \label{alg1}
    \begin{algorithmic}[1]
        \REQUIRE  base model $\boldsymbol{\theta}_{\rm{base}}$, SFT models $\{\boldsymbol{\theta}_{i}\mid i\in [K]\}$, mask ratios \{$r_{i} \mid i\in [K]\}$, scaling factors $\{\lambda_i\mid i\in[K]\}$, location datasets $\{\mathcal{X}_{i}\mid i\in[K]\}$
        \ENSURE merged parameter $\boldsymbol{\theta}_{m}$
        \STATE $\mathcal{M}\leftarrow\phi$
        \STATE $\boldsymbol{\theta}_{m}\leftarrow \boldsymbol{\theta}_{\rm{base}}$
        \FOR{$i\in [K]$}
        \STATE $I(\boldsymbol{\theta}_i)=\mathbb{E}_{x\sim \mathcal{X}_i}|\boldsymbol{\theta}_{i}\odot \nabla_{\boldsymbol{\theta}_i}\mathcal{L}(x)|$
        \STATE $I(\boldsymbol{\theta}_{\rm{base}})=\mathbb{E}_{x\sim \mathcal{X}_i}|\boldsymbol{\theta}_{\rm{base}}\odot \nabla_{\boldsymbol{\theta}_{\rm{base}}}\mathcal{L}(x)|$
        
        \STATE calculate $\mathcal{T}^{r_i}_{i}$ following Equation \ref{vote}
        \STATE  $\mathcal{M}\leftarrow \mathcal{M}\cup\{\mathcal{T}^{r_i}_i\}$
       
        
   
        
        
        \ENDFOR  
        \FOR{$i\in [K]$}
        
        \STATE calculate $\text{Disjoint}(\mathcal{T}_i^{r_i})$ use Equation~\ref{disjoint_safety}
        \STATE $\boldsymbol{m}_i \leftarrow \boldsymbol{0}$
        \FOR{$d\in \mathcal{T}_i^{r_i}$}
        \STATE $\boldsymbol{m}_{i,d}=1$
        \ENDFOR
        \STATE $\boldsymbol{\theta}_{m}\leftarrow \boldsymbol{\theta}_{m}+\lambda_i \boldsymbol{\tau}_i\odot \boldsymbol{m}_{i}$
        \ENDFOR
    \end{algorithmic}
\end{algorithm}
    %\vspace{-5pt}
\begin{figure*}[h!]
    \centering
    \includegraphics[width=\linewidth]{figs/pipeline_v2.pdf}
    \vspace{-40mm}
    \caption{Overview of our two-stage training pipeline {\ours}.}
    \label{fig:pipeline}
\end{figure*}


\paragraph{LED-Merging: Location, Election, and Disjoint Merging}
To address the neuron misidentification and interference issues in existing model merging methods, we propose LED-Merging (Location, Election, and Disjoint Merging). Specifically, previous studies \cite{modelstock, ilharco2023editing, tiesmerging} fail to accurately identify safety-related neurons in task vectors with a single magnitude score, namely \textit{neuron misidentification}. Meanwhile, there exists an interference between safety-related and utility-related task vector neurons during the merging process, namely \textit{neuron interference}. To address neuron misidentification, we first locate important neurons both in the base and fine-tuned models and then elect neurons from the task vector considering these two scores together. Subsequently, to mitigate the interference, we introduce a disjoint step, isolating these important neurons so that they influence different base neurons. The whole process is illustrated in Figure~\ref{fig:method}. 




In the location and election step, we consider the importance score from base and fine-tuned models simultaneously to locate task-specific neurons. In this way, it is more accurate than relying on the magnitude score alone because task-specific neurons with high importance score in the fine-tuned model may not necessarily score high in the base model, and vice versa.

{\textbf{Location}}.  We first calculate importance scores for each neuron in a base/fine-tuned model. Given a location dataset $\mathcal{X}_i=\{(x,y)_k\}$, where $x$ is the question and $y$ is the answer, we calculate the importance scores for the weight $\boldsymbol{\theta}_i\in\mathbb{R}^D$ in any  layer as follows~\cite{snip,spareseGPT,sun2024a}:
\begin{equation}
    I(\boldsymbol{\theta}_i)=\mathbb{E}_{x\sim \mathcal{X}_i}[\boldsymbol{\theta}_i\odot \nabla _{\boldsymbol{\theta}_i}\mathcal{L}(x)],
    \label{location}
\end{equation}
which $\mathcal{L}(x)=-\log p(y\mid x)$ is the conditional negative log-likelihood loss. We choose the SNIP score~\cite{snip} because it balances computational efficiency and performance~\cite{cq}. Please refer to Sec.~\ref{sec:ablation} for the comparison between different location methods. After computing importance scores, we choose top-$r_i$ neurons as the important neuron subset $\mathcal{N}_{i}^{r_i}$ from $I(\boldsymbol{\theta}_i)$.
 
 % After computing locating scores, we select the neurons scoring both high in base and fine-tuned models as important neurons in task vectors. Then in the disjoint step,  with preventing  polysemantic neurons  from receiving gradient updates towards different directions,
 % we use set difference to isolate the safety   and utility-related neurons  and construct corresponding masks for merging process,

{\textbf{Election}}. A natural question is how to select important neurons in the task vector $\boldsymbol{\tau}_i$ based on $I(\boldsymbol{\theta}_{\rm{base}})$ and $I(\boldsymbol{\theta}_{i})$. The important neurons in the base model may be different from neurons in the fine-tuned model. Therefore, we introduce the following election strategy to select neurons with high scores in both base and fine-tuned models:
\begin{equation}
    \mathcal{T}_i^{r_i}=\mathcal{N}_i^{r_i}\cap \mathcal{N}_{\rm{base}}^{r_i}.
    \label{vote}
\end{equation}
\emph{Remark}. We compare different choosing methods, including scoring low or high in base or fine-tuned model in Section~\ref{sec:ablation} and find that Equation \ref{vote} achieves the best performance.





{\textbf{Disjoint}}. As important neurons from different task vectors may conflict with each other at the same position, we use the set difference to disjoint the neurons from others to prevent interference:
\begin{equation}
    \text{Disjoint}(\mathcal{T}^{r_i}_{i})=\mathcal{T}^{r_i}_{i}-\mathop{\cup}\limits_{{J}\subsetneqq [K],|J|\geq 2}\mathop{\cap}\limits_{j\in {J}}\mathcal{T}^{r_j}_{j}.
    \label{disjoint_safety}
\end{equation}

Next, we construct a mask $\boldsymbol{m}_i\in\mathbb{R}^D$ to implement disjoint in the merging process. Specifically, this mask $\boldsymbol{m}_i$ is used to select neurons from $\mathcal{T}_i$. The mask ratio is $r_i$, where $r\in(0,1]$. The mask $\boldsymbol{m}_i$ can be derived from:
\begin{equation}
    \boldsymbol{m}_{i,d}=\begin{aligned} &\left\{ \begin{array}{ll} 1, & \text{if } d\in \text{Disjoint}(\mathcal{T}_{i}^{r_i}), \\ 0, & \text{otherwise}. \end{array} \right. \end{aligned}
    \label{mask_safety}
\end{equation}


% \subsection{Merging Models with Masks}
{\textbf{Merging}}. The final
merged task vector $\boldsymbol{\tau}_m$ is as follows:
\begin{equation}
    \boldsymbol{\tau}_m= \sum_i \lambda_i\boldsymbol{\tau}_{i}\odot\boldsymbol{m}_i.
    \label{merged_task_vector}
\end{equation}
We summarize the workflow in Algorithm \ref{alg1}.



\section{Experiments}
SSH is compared against state-of-the-art parameter-efficient fine-tuning (PEFT) methods. The experiments are conducted across multiple domains, including single-modality tasks such as natural language understanding (NLU) and natural language generation (NLG), as well as instruction tuning, text summarization, and mathematical reasoning. Additionally, SSH is evaluated on multi-modality tasks, such as vision-language image classification. Finally, an ablation study is performed to assess the effectiveness of our approach.



\subsection{Baselines}
We compare SSH with the following baselines:
\begin{itemize}
    \item \textbf{Full Fine-Tuning (FF):} The entire model is fine-tuned, with updates to all parameters.
    \item \textbf{Adapter Tuning~\cite{houlsby2019parameter,lin2020exploring,ruckle2020adapterdrop,pfeiffer2020adapterfusion}:} Methods that introduce adapter layers between the self-attention and MLP modules for parameter-efficient tuning.
    \item \textbf{LoRA~\cite{hu2022lora}:} A method that estimates weight updates via low-rank matrices.
    \item \textbf{AdaLoRA~\cite{zhang2303adaptive}:} An extension of LoRA that dynamically reallocates the parameter budget based on importance scores.
    \item \textbf{DoRA~\cite{liu2024dora}:} Decomposes pretrained weights into magnitude and direction, using LoRA for efficient directional updates.
    \item \textbf{VeRA~\cite{kopiczko2023vera}:} Employs a single pair of low-rank matrices across all layers, to reduce parameters.
    \item \textbf{FourierFT~\cite{gao2024parameter}:} Fine-tunes models by learning a subset of spectral coefficients in the Fourier domain.
    \item 
    \textbf{AFLoRA~\cite{liu2024aflora}:} Freezes low-rank adaptation parameters using a learned freezing score, reducing trainable parameters while maintaining performance.
    \item 
    \textbf{LaMDA~\cite{azizi2024lamda}:} Fine-tunes large models via spectrally decomposed low-dimensional adaptation, reducing trainable parameters and memory usage while maintaining performance.
    
\end{itemize}

\subsection{Natural Language Understanding}

\begin{table*}[!ht]
\centering
\resizebox{0.85\textwidth}{!}{%
\begin{tabular}{cl|r|ccccccccc}
\toprule
& \textbf{Model} & \textbf{\# Trainable} & \textbf{SST-2} & \textbf{MRPC} & \textbf{CoLA} & \textbf{QNLI} & \textbf{RTE} & \textbf{STS-B} & \multirow{2}{*}{\textbf{Avg.}} \\
& \textbf{\& Method} & \textbf{Parameters} & \textbf{(Acc.)} & \textbf{(Acc.)} & \textbf{(MCC)} & \textbf{(Acc.)} & \textbf{(Acc.)} & \textbf{(PCC)} \\
\midrule
\multirow{9}{*}{\rotatebox{90}{\textbf{BASE}}} 
& FF & 125M & 94.8 & 90.2 & 63.6 & 92.8 & 78.7 & 91.2 & 85.22 \\
& BitFit & 0.1M & 93.7 & \textbf{92.7} & 62.0 & 91.8 & \textbf{81.5} & 90.8 & 85.42 \\
& Adpt\textsuperscript{D} & 0.9M & 94.7 & 88.4 & 62.6 & 93.0 & 75.9 & 90.3 & 84.15 \\
& LoRA & 0.3M & \textbf{95.1} & 89.7 & 63.4 & \textbf{93.3} & 78.4 & \textbf{91.5} & 85.23 \\
& AdaLoRA & 0.3M & 94.5 & 88.7 & 62.0 & 93.1 & 81.0 & 90.5 & 84.97 \\
& DoRA & 0.3M & 94.9 & 89.9 & 63.7 & \textbf{93.3} & 78.9 & \textbf{91.5} & 85.37 \\
& AFLoRA & 0.27M & 94.1 & 89.3 & 63.5 & 91.3 & 77.2 & 90.6 & 84.33 \\
& LaMDA & 0.06M &  94.6 & 89.7 & 64.9 & 91.7 & 78.2 & 90.4 & 84.92 \\
& VeRA & 0.043M & 94.6 & 89.5 & \textbf{65.6} & 91.8 & 78.7 & 90.7 & 85.15 \\
& FourierFT & 0.024M & 94.2 & 90.0 & 63.8 & 92.2 & 79.1 & 90.8 & 85.02 \\
\rowcolor{green!17}
& \textbf{SSH} & \textbf{0.018M} & 94.1 & 91.2 & 63.6 & 92.4 & 80.5 & 90.9 & \textbf{85.46} \\
\midrule
\multirow{8}{*}{\rotatebox{90}{\textbf{LARGE}}} 
& FF & 356M & 96.3 & 90.9 & 68.0 & 94.7 & 86.6 & 92.4 & 88.11 \\
& Adpt\textsuperscript{P} & 3M & 96.1 & 90.2 & \textbf{68.3} & 94.7 & 83.8 & 92.1 & 87.55 \\
& Adpt\textsuperscript{P} & 0.8M & \textbf{96.6} & 89.7 & 67.8 & 94.7 & 80.1 & 91.9 & 86.82 \\
& Adpt\textsuperscript{H} & 6M & 96.2 & 88.7 & 66.5 & 94.7 & 83.4 & 91.0 & 86.75 \\
& Adpt\textsuperscript{H} & 0.8M & 96.3 & 87.7 & 66.3 & 94.7 & 72.9 & 91.5 & 84.90 \\
& LoRA & 0.8M & 96.2 & 90.2 & 68.2 & \textbf{94.8} & 85.2 & 92.3 & 87.82 \\
& DoRA & 0.9M & 96.4 & \textbf{91.0} & 67.2 & \textbf{94.8} & 85.4 & 92.1 & 87.82 \\
& AFLoRA & 0.76M & 96.3 & 90.0 & 67.5 & 94.3 & 86.6 & 91.9 & 87.77 \\
& LaMDA & 0.093M &  96.2 & 90.1 & 68.1 & 94.5 & 87.3 & 92.0 & 88.03 \\
& VeRA & 0.061M & 96.1 & 90.9 & 68.0 & 94.4 & 85.9 & 91.7 & 87.83 \\
& FourierFT & 0.048M & 96.0 & 90.9 & 67.1 & 94.4 & \textbf{87.4} & 91.9 & 87.95 \\
\rowcolor{green!17}
& \textbf{SSH} & \textbf{0.036M} & 96.2 & 90.9 & 67.9 & 94.5 & \textbf{87.4} & \textbf{92.2} & \textbf{88.17} \\
\bottomrule
\end{tabular}%
}
\caption{\small Performance of various fine-tuning methods on GLUE benchmark, using base and large models. Metrics include MCC for CoLA, PCC for STS-B, and accuracy for other tasks. Results are medians of 5 runs with different seeds; the best scores in each category are bolded. SSH delivers the best average performance across tasks while using significantly fewer trainable parameters.}
\label{tab:nlup}
\end{table*}





\noindent \textbf{Models and Datasets.}  
We evaluate SSH on the GLUE benchmark~\cite{wang2019glue} using RoBERTa~\cite{liu2019roberta} in both Base and Large configurations. The GLUE benchmark comprises a diverse set of NLU tasks, offering a comprehensive evaluation framework.


\noindent \textbf{Implementation Details.}  
The SSH method uses 750 of the 768\textsuperscript{2} available spectral coefficients for RoBERTa Base and 1024\textsuperscript{2} for RoBERTa Large, ensuring that each layer retains the most important spectral components. This selection remains consistent across all layers. To ensure fair comparison, we follow the same experimental settings as LoRA and FourierFT. Additional hyperparameters and details are provided in Tab.~\ref{tab:nluh} in the appendix~\ref{gluebench}.


\noindent \textbf{Results and Analysis} 
The results in Table~\ref{tab:nlup} indicate that SSH consistently delivers competitive performance across diverse NLU tasks while maintaining a significantly lower number of trainable parameters. Notably, SSH achieves 80.5\% accuracy on RTE, 92.4\% on QNLI, and 90.9 on STS-B, demonstrating its capability to generalize effectively across multiple linguistic tasks.

SSH also maintains robust performance in sentiment classification, achieving 94.1\% accuracy on SST-2, which is on par with other parameter-efficient methods such as LoRA and BitFit. On CoLA, SSH attains a score of 63.6, matching FourierFT and outperforming Adpt\textsuperscript{D} and AdaLoRA. Additionally, SSH exhibits strong generalization on MRPC with 91.2\% accuracy and achieves a 90.9 Pearson correlation on STS-B, further reinforcing its effectiveness across textual similarity and entailment tasks. These findings highlight SSH as a highly efficient and scalable fine-tuning approach, capable of achieving state-of-the-art performance with minimal parameter overhead.




% These results highlight SSH's ability to retain critical information with minimal parameters, making it an effective and resource-efficient method for fine-tuning large-scale models, particularly in computationally constrained environments.


\begin{table}[!t]
\centering
\scalebox{0.63}{
\begin{tabular}{l|lr|crcccccc}
\toprule
 & \textbf{Method} & \textbf{\# Tr. Para.} & \textbf{BLEU} & \textbf{NIST} & \textbf{METE.} & \textbf{ROU-L} & \textbf{CIDEr} \\
\midrule
\multirow{9}{*}{\rotatebox{90}{\textbf{GPT-2 Medium}}} 
& FT\textsuperscript{1} & 354.92M & 68.2 & 8.62 & 46.2 & 71.0 & 2.47 \\
& Adpt\textsuperscript{L\textsuperscript{1}} & 0.37M & 66.3 & 8.41 & 45.0 & 69.8 & 2.40 \\
& Adpt\textsuperscript{L\textsuperscript{1}} & 11.09M & 68.9 & 8.71 & 46.1 & 71.3 & 2.47 \\
& Adpt\textsuperscript{H\textsuperscript{1}} & 11.09M & 67.3 & 8.50 & 46.0 & 70.7 & 2.44 \\
& LoRA & 0.35M & 68.9 & 8.76 & 46.6 & 71.5 & 2.51 \\
& DoRA & 0.36M & 69.2 & 8.79 & 46.9 & 71.7 & 2.52\\
& VeRA & 0.35M & \textbf{70.1} & 8.81 & 46.6 & 71.5 & 2.50 \\
& FourierFT & 0.048M & 69.1 & \textbf{8.82} & 47.0 & 71.8 & 2.51 \\
\rowcolor{green!17}
& \textbf{SSH} & \textbf{0.036M} & \textbf{70.1} & \textbf{8.82} & \textbf{47.2} & \textbf{71.9} & \textbf{2.54} \\
\midrule
\multirow{8}{*}{\rotatebox{90}{\textbf{GPT-2 Large}}} 
& FT\textsuperscript{1} & 774.03M & 68.5 & 8.78 & 46.0 & 69.9 & 2.45 \\
& Adpt\textsuperscript{L\textsuperscript{1}} & 0.88M & 69.1 & 8.68 & 46.1 & 71.0 & 2.49 \\
& Adpt\textsuperscript{L\textsuperscript{1}} & 23.00M & 68.9 & 8.70 & 46.1 & 71.3 & 2.45 \\
& LoRA & 0.77M & 69.4 & 8.81 & 46.5 & \textbf{71.9} & 2.50 \\
& DoRA & 0.79M & 69.8 & 8.83 & 46.9 & \textbf{71.9} & 2.50 \\
& VeRA & 0.17M & \textbf{70.3} & 8.85 & 46.6 & 71.6 & 2.54 \\
& FourierFT & 0.072M & 70.2 & 8.90 & 47.0 & 71.8 & 2.50 \\
\rowcolor{green!17}
& \textbf{SSH} & \textbf{0.054M} & \textbf{70.3} & \textbf{8.93} & \textbf{47.2} & \textbf{71.9} & \textbf{2.55} \\
\bottomrule
\end{tabular}}
\caption{\small Performance comparison of various fine-tuning methods on GPT-2 Medium and Large models, evaluated using BLEU, NIST, METEOR, ROUGE-L, and CIDEr metrics. \textsuperscript{1} denotes results sourced from previous studies. }
\label{tab:e2e}
\end{table}



\subsection{Natural Language Generation}


\noindent \textbf{Models and Datasets.}  
We evaluate SSH on the E2E natural language generation (NLG) task~\cite{novikova2017e2e}, fine-tuning GPT-2 Medium and Large models~\cite{radford2019language}, which consist of 24 and 36 transformer blocks.

\noindent \textbf{Implementation Details.}  
We fine-tune LoRA, DoRA, FourierFT, VeRA, and the proposed SSH on GPT-2 Medium and Large, using a linear learning rate scheduler over 5 epochs. Results are averaged across 3 runs, with detailed hyperparameters provided in Tab.~\ref{tab:nlgh} in the Appendix~\ref{gluebench}.




\noindent \textbf{Results and Analysis.}  
As shown in Tab.~\ref{tab:e2e}, SSH consistently delivers superior or comparable performance across all evaluation metrics, while requiring significantly fewer trainable parameters. For GPT-2 Medium, SSH matches the highest BLEU score (70.1) and outperforms other methods in NIST (8.82), METEOR (47.2), ROUGE-L (71.9), and CIDEr (2.54), all with 10.3\% fewer parameters than LoRA and 25\% fewer than FourierFT. A similar trend is observed for GPT-2 Large, where SSH achieves the highest NIST (8.93) and METEOR (47.2) scores, while maintaining a 7.1\% parameter reduction compared to LoRA. 




\begin{table}[!t]
\centering
\resizebox{0.5\textwidth}{!}{%
\begin{tabular}{l|l|c|crcc}
\toprule
\textbf{Model} & \textbf{Method} & \textbf{\# Tr. Para.} & \textbf{MT-Bench} & \textbf{Vicuna} \\
\midrule
\multirow{5}{*}{\textbf{LLaMA2-7B}} 
& LoRA & 159.9M & 5.19 & 7.37 \\
& DoRA & 163.7M & 5.20 & 7.41 \\
& VeRA & 1.6M & 5.18 & 7.47 \\
& FourierFT & 0.064M & 5.09 & 7.50 \\
\rowcolor{green!17}
& \textbf{SSH} & \textbf{0.048M} & \textbf{5.22} & \textbf{7.51}\\
\midrule
\multirow{5}{*}{\textbf{LLaMA2-13B}} 
& LoRA & 250.3M & 5.77 & 7.89\\
& DoRA & 264.5M & 5.79 & 7.90 \\
& VeRA & 2.4M & \textbf{5.93} & 7.90 \\
& FourierFT & 0.08M & 5.82 & 7.92 \\
\rowcolor{green!17}
& \textbf{SSH} & \textbf{0.06M} & \textbf{5.93} & \textbf{7.95} \\
\midrule
\multirow{5}{*}{\textbf{LLaMA3.1-8B}} 
& LoRA & 183.3M & 5.65 & 7.52 \\
& DoRA & 186.9M & 5.66 & \textbf{7.59} \\
& VeRA & 1.9M & 5.61 & 7.49 \\
& FourierFT & 0.073M & 5.67 & 7.67 \\
\rowcolor{green!17}
& \textbf{SSH} & \textbf{0.055M} & \textbf{5.69} & \textbf{7.71} \\
\bottomrule
\end{tabular}%
}
\caption{\small Performance comparison of fine-tuning methods on LLaMA models using the Alpaca dataset. Evaluation scores on MT-Bench and Vicuna are generated and scored by GPT-4.}
\label{tab:mtbench_vicuna}
\end{table}


\subsection{Instruction Tuning}

\noindent \textbf{Models and Datasets.}  
We fine-tune LLaMA2-7B, LLaMA2-13B, and LLaMA3.1-8B using SSH and baseline methods on the Alpaca dataset~\cite{taori2023stanford}. For evaluation, we generate responses to predefined questions from the MT-Bench~\cite{zheng2024judging} and Vicuna Eval datasets, which are then scored by GPT-4 on a 10-point scale.

\noindent \textbf{Implementation Details.}  
Following previous work~\cite{dettmers2024qlora,dettmers20228bit}, we apply LoRA, DoRA, and VeRA to all linear layers except the top one. For FourierFT, we use the configuration from~\cite{gao2024parameter}, and for SSH, we set \(n = 750\). All models are trained using QLoRA’s quantization technique~\cite{dettmers2024qlora} on a single GPU. Each method is trained for one epoch, and we report the average score across all generated responses. Hyperparameter details are provided in Tab.\ref{tab:hyperparamsIn} in the Appendix~\ref{gluebench}.


\noindent \textbf{Results and Analysis.}  
The results in Tab.~\ref{tab:mtbench_vicuna} clearly demonstrate the significant efficiency of SSH compared to other fine-tuning methods such as LoRA, DoRA, and FourierFT. For LLaMA2-7B, SSH achieves the best MT-Bench (5.22) and Vicuna (7.51) scores while reducing trainable parameters by over 99.7\%, using only 0.048M parameters compared to LoRA's 159.9M. Similarly, in LLaMA2-13B, SSH ties with VeRA for the highest MT-Bench score (5.93) and surpasses all methods in Vicuna (7.95), again achieving this with a drastically lower parameter count (0.06M vs. 250.3M for LoRA). Even in the larger LLaMA3.1-8B model, SSH continues to outperform, leading in MT-Bench (5.69) and maintaining a competitive Vicuna score (7.71) with far fewer parameters (0.055M). 



\begin{table}
\centering
\resizebox{0.47\textwidth}{!}{%
\begin{tabular}{l|l|r|cccccc}
\toprule
\textbf{Model} & \textbf{Method} & \textbf{\# Train. Para.} & \textbf{CIFAR100} & \textbf{DTD} & \textbf{EuroSAT} & \textbf{OxfordPets} \\
\midrule
\multirow{7}{*}{\textbf{ViT-B}} 
& Head & - & 84.3 & 69.8 & 88.7 & 90.3 \\
& Full & 85.8M & \textbf{92.4} & \textbf{77.7} & \textbf{99.1}& \textbf{93.4} \\
& LoRA & 581K & 92.1 & 75.2 & 98.4 & 93.2 \\
& Dora & 594K & 92.3 & 75.3 & 98.7 & 93.2 \\
& VeRA & 57.3K & 91.7 & 74.6 & 98.5 & \textbf{93.4}\\
& FourierFT & 72K & 94.2 & 75.1 & 98.8 & 93.2 \\
\rowcolor{green!17}
& \textbf{SSH} & \textbf{54K} & 91.6 & 76.1 & \textbf{99.1} & \textbf{93.4} \\

\midrule
\multirow{7}{*}{\textbf{ViT-L}} 
& Head & - & 84.7 & 73.3 & 92.6 & 91.1 \\
& Full & 303.3M & 93.6 & 81.8 & \textbf{99.1} & 94.4 \\
& LoRA & 1.57M &  94.9 & 81.8 & 98.63 & \textbf{94.8} \\
& Dora & 1.62M &  \textbf{95.1} & 81.8 &   98.8 & \textbf{94.8}\\
& VeRA & 130.5K & 94.2 & 81.6& 98.6 & 93.7 \\
& FourierFT & 144K & 93.7 & 81.2 & 98.7 & 94.5 \\
\rowcolor{green!17}
& \textbf{SSH} & \textbf{108K} & 94.5 & \textbf{81.9} & 99.0& \textbf{94.8} \\
\bottomrule
\end{tabular}%
}
\caption{\small Performance of various fine-tuning methods on ViT-B and ViT-L models across different datasets. The best results for each dataset are highlighted in bold. The best results are highlighted in bold. SSH offers strong parameter efficiency, excelling on DTD and EuroSAT while delivering competitive performance on CIFAR100 and OxfordPets, making it a balanced solution for various vision tasks.}
\label{tab:vit_results}
\end{table}





\subsection{Text Summarization}
\noindent \textbf{Models and Datasets.}  
We evaluate the effectiveness of SSH against other baseline methods on the BART-Large model~\cite{lewis2019bart} for text summarization tasks. Specifically, we assess its performance on the XSUM~\cite{narayan2018don} and CNN/DailyMail~\cite{hermann2015teaching} datasets.

\begin{table}[!t]
    \centering
    \resizebox{0.52\textwidth}{!}{%
    \begin{tabular}{l|c|c|c}
        \toprule
        \textbf{Method} & \textbf{Para. (M)} & \textbf{XSUM} & \textbf{CNN/DailyMail} \\
        \midrule
        AFLoRA ($r=32$) & 5.27 & 44.71/21.92/37.33 & 44.95/21.87/42.25 \\
        LaMDA ($r=32$) & 0.85 & 43.94/20.69/35.21 & 44.16/21.17/40.48 \\
        \rowcolor{green!17}
        SSH ($n=5000$) & 0.21 & 44.72/22.05/37.42 & 44.89/21.75/42.13 \\
        \bottomrule
    \end{tabular}}
    \caption{Performance comparison of SSH, AFLoRA, and LaMDA on BART-Large for text summarization tasks. Results are reported as ROUGE-1/ROUGE-2/ROUGE-L.}
    \label{tab:nlg_bart}
\end{table}

\noindent \textbf{Implementation Details.}  
We compare SSH against AFLoRA and LaMDA under consistent experimental conditions. For AFLoRA and LaMDA, we set the rank $r=32$, while for SSH, we select $n=5000$ Hartley spectrum points. The models are trained using a learning rate of $2\times10^{-4}$, with a batch size of 32 for XSUM and 64 for CNN/DailyMail. Training is conducted for 25 epochs on XSUM and 15 epochs on CNN/DailyMail.

\noindent \textbf{Results and Analysis.}  
Table~\ref{tab:nlg_bart} presents the ROUGE evaluation scores (ROUGE-1/ROUGE-2/ROUGE-L) for different fine-tuning approaches. SSH achieves competitive performance while utilizing significantly fewer trainable parameters compared to AFLoRA and LaMDA. On the XSUM dataset, SSH attains the highest ROUGE-2 score (22.05), surpassing AFLoRA (21.92) and LaMDA (20.69) by 0.13 and 1.36 points, respectively. Furthermore, SSH achieves the highest ROUGE-L score (37.42), outperforming AFLoRA by 0.09 and LaMDA by 2.21 points.

Similarly, on the CNN/DailyMail dataset, SSH attains a ROUGE-1 score of 44.89, which is marginally lower than AFLoRA (44.95) by 0.06 points, but it outperforms LaMDA (44.16) by 0.73 points. In terms of ROUGE-2, SSH achieves 21.75, trailing AFLoRA (21.87) by 0.12 points but exceeding LaMDA (21.17) by 0.58 points. Additionally, SSH attains a ROUGE-L score of 42.13, which is 0.12 points lower than AFLoRA but significantly higher than LaMDA by 1.65 points. Overall, SSH consistently demonstrates strong performance while requiring significantly fewer trainable parameters (0.21M) compared to AFLoRA (5.27M) and LaMDA (0.85M). 

\subsection{Mathematical Reasoning}

\noindent \textbf{Models and Dataset.} We evaluate the performance of SSH against AFLoRA and LaMDA on the LLaMA3.1-8B model using the GSM8K~\cite{cobbe2021training}, a widely used dataset designed to assess mathematical reasoning abilities.

\noindent \textbf{Implementation Details.} All methods are trained with a learning rate of $3\times10^{-4}$ for six epochs using a batch size of 16. For parameter-efficient fine-tuning, AFLoRA and LaMDA employ a low-rank adaptation setting of $r=32$, while SSH leverages a Hartley spectrum selection with $n=10000$. Table~\ref{tab:llama_gsm8k} presents a comparison of the methods in terms of trainable parameters and accuracy on GSM8K.

\begin{table}[!t]
    \centering
    \resizebox{0.5\textwidth}{!}{%
    \begin{tabular}{l|c|c}
        \toprule
        \textbf{Method} & \textbf{Trainable Parameters (M)} & \textbf{GSM8K Accuracy} \\
        \midrule
        AFLoRA ($r=32$) & 20.23 & 38.63 \\
        LaMDA ($r=32$) & 4.99 & 38.11 \\
        \rowcolor{green!17}
        SSH ($n=10000$) & 1.54 & \textbf{38.67} \\
        \bottomrule
    \end{tabular}}
    \caption{Comparison of SSH with AFLoRA and LaMDA on LLaMA3.1-8B for GSM8K. Accuracy is reported as a percentage.}
    \label{tab:llama_gsm8k}
\end{table}

\noindent \textbf{Results and Analysis.} SSH achieves the highest accuracy (38.67\%), surpassing AFLoRA (38.63\%) and LaMDA (38.11\%) while using significantly fewer trainable parameters. SSH requires only 1.54M parameters, representing a \textbf{92.4\% reduction} compared to AFLoRA and a \textbf{69.1\% reduction} compared to LaMDA. 

Despite having nearly 13 times fewer parameters than AFLoRA, SSH achieves comparable accuracy, demonstrating a superior trade-off between efficiency and performance. While LaMDA exhibits the lowest accuracy, SSH maintains robustness in mathematical reasoning tasks with minimal resource requirements.

% These findings highlight SSH as a highly efficient fine-tuning method, delivering strong reasoning capabilities while significantly reducing computational overhead. This makes SSH a promising approach for scaling large language models in resource-constrained settings.


\subsection{Image Classification}

\noindent \textbf{Models and Datasets.}  
We evaluate our method on the Vision Transformer (ViT)~\cite{dosovitskiy2020image}, using both the Base and Large variants. Image classification is performed on the CIFAR-100~\cite{krause20133d}, DTD~\cite{cimpoi2014describing}, EuroSAT~\cite{helber2019eurosat}, and OxfordPets~\cite{parkhi2012cats} datasets.

\noindent \textbf{Implementation Details.}  
We evaluate SSH, LoRA, DoRA, VeRA, and FourierFT by applying them to the query and value layers of ViT. Training only the classification head is denoted as "Head". We set \( r = 16 \) for LoRA, \( n = 3000 \) for FourierFT, and \( n = 2250 \) for SSH. Learning rates and weight decay are tuned for all methods, with training limited to 10 epochs. Further hyperparameter details are provided in Tab.~\ref{tab:SSH_image} in the Appendix~\ref{gluebench}.






\noindent \textbf{Results and Analysis.}  
Tab.~\ref{tab:vit_results} highlights the performance of various fine-tuning methods on ViT-B and ViT-L across four image classification datasets. For the ViT-B model, SSH delivers competitive results with only 54K trainable parameters, significantly fewer than LoRA and DoRA, which use more than 10 times as many. Notably, SSH matches the full fine-tuning performance on EuroSAT and OxfordPets, achieving 99.1\% and 93.4\% accuracy, respectively. For the ViT-L model, SSH also proves efficient, achieving near-optimal performance with only 108K parameters. It sets the highest score on DTD with 81.9\% accuracy and matches the best performance on OxfordPets at 94.8\%. 


% \begin{table}
% \centering
% \resizebox{0.47\textwidth}{!}{%
% \begin{tabular}{l|l|r|cccccc}
% \toprule
% \textbf{Model} & \textbf{Method} & \textbf{\# Train. Para.} & \textbf{TVQA} & \textbf{How2QA} & \textbf{TVC} & \textbf{YC2C} \\
% \midrule
% \multirow{7}{*}{\textbf{VL-BART}} 
% & Full & 228.9M & \textbf{76.3} & 73.9 & 45.7& \textbf{154} \\
% & LoRA & 11.8M & 75.5 & 72.9 & 44.6 & 140.9 \\
% & Dora & 11.9M & \textbf{76.3} & 74.1 & \textbf{45.8} & 145.4 \\
% & VeRA & 1.3M & 75.9 & 73.8 & 44.7 & 142.6\\
% & FourierFT & 1.5M & 76.2 & 73.1 & 45.5 & 147.3 \\
% \rowcolor{green!17}
% & \textbf{SSH} & \textbf{1.1M} & 76.2 & \textbf{74.2}& \textbf{45.8} & 152 \\

% \bottomrule
% \end{tabular}%
% }
% \caption{\small Multi-task evaluation results on TVQA, How2QA, TVC, and YC2C using the VL-BART backbone. The best results are highlighted in bold. SSH demonstrates strong performance with significantly fewer trainable parameters, achieving top scores on How2QA, TVC, and competitive results on the other tasks.}

% \label{tab:bart_results}
% \end{table}

% \subsection{Video-Text Understanding}

% \noindent \textbf{Models and Datasets.}  
% We compare DoRA, LoRA, and full fine-tuning on VL-BART, a model that integrates a vision encoder (CLIP-ResNet101~\cite{radford2021learning}) with an encoder-decoder language model (BART-Base~\cite{lewis2019bart}). The comparison spans four video-text tasks: TVQA~\cite{lei2018tvqa} and How2QA~\cite{li2020hero} for video question answering, and TVC~\cite{lei2020tvr} and YC2C~\cite{zhou2018towards} for video captioning.

% \noindent \textbf{Implementation Details.}  
% We follow the framework from ~\cite{sung2022vl}, fine-tuning VL-BART in a multi-task setup for both video-text tasks. We set \( r = 128 \) for LoRA, \( n = 6000 \) for FourierFT, and \( n = 4500 \) for SSH. Learning rates and weight decay are tuned for each method, with training capped at 7 epochs. Further details are provided in Tab.~\ref{tab:SSH_video} in the Appendix~\ref{gluebench}.


% \noindent \textbf{Results and Analysis.}  
% Tab.~\ref{tab:bart_results} presents the multi-task evaluation results for video-text tasks using the VL-BART backbone. SSH consistently delivers competitive performance with significantly fewer trainable parameters. For TVQA, SSH achieves a near-best score of 76.2, matching FourierFT and just slightly below the full fine-tuning result of 76.3, despite using 99.5\% fewer parameters.
% For How2QA, SSH records the highest score of 74.2, outperforming all other methods. Similarly, for TVC, SSH ties for the best result with DoRA at 45.8, again with far fewer parameters. For YC2C, SSH comes close to the top score achieved by full fine-tuning, with a score of 152 compared to 154, while maintaining remarkable parameter efficiency.

% \begin{figure}[!t]
%     \centering
%     \includegraphics[width=\linewidth]{data/rank.pdf}
%     \caption{\small Scalability comparative experiments of LoRA, FourierFT, SSH, and rSSH across GLUE tasks. SSH demonstrates robust scalability and strong performance in NLP understanding tasks.}
%     \label{fig:ablation}
% \end{figure}





\subsection{Ablation Study}
% We investigate the relationship between parameter number and model performance across different methods. For LoRA, we evaluate with ranks \( r = \{1, 2, 4, 6, 8, 16\} \). For FourierFT and SSH, we examine \( n = \{50, 100, 200, 1000, 6144, 12288\} \) spectral coefficients. The experiments are conducted on 6 GLUE tasks.





% \noindent \textbf{Parameter Scalability.} Figure~\ref{fig:ablation} demonstrates that SSH consistently outperforms other approaches as the number of spectral coefficients \( n \) increases, highlighting its robust scalability and efficiency. Unlike LoRA, where increasing the number of trainable parameters often yields diminishing returns, SSH maintains strong performance gains with fewer parameters. Additionally, SSH surpasses FourierFT across tasks such as MRPC, CoLA, RTE, SST-2, and QNLI, demonstrating its effective parameter utilization.







% A statistical analysis using the Student t-test confirms that SSH significantly outperforms FourierFT across most tasks: RTE ($p=0.0446$, $t=2.67$), MRPC ($p=0.0167$, $t=3.53$), SST-2 ($p=0.0272$, $t=3.09$), QNLI ($p=0.0066$, $t=4.46$), and CoLA ($p=0.0144$, $t=3.67$). While STS-B shows a smaller advantage ($p=0.0348$, $t=2.87$), SSH consistently demonstrates superior scalability and performance across benchmarks, reinforcing its effectiveness as \( n \) increases.





% \noindent \textbf{Informed Frequency Selection v.s. Random Sampling.}
% We further compare the proposed SSH with the scenario where the frequency points are selected randomly (denoted as rSSH).
% The results illustrate that SSH, which uses a systematic partitioning and hybrid selection strategy, consistently outperforms rSSH. This is statistically confirmed by the student t-test, where significant improvements are observed in CoLA (p=0.0260, t=3.13) and SST-2 (p=0.0050, t=4.77), highlighting the advantage of informed frequency selection. Even in tasks with smaller gaps like MRPC (p=0.0380, t=2.80) and QNLI (p=0.0165, t=3.54), SSH demonstrates clear benefits. Although the STS-B task shows a less pronounced difference (p=0.117, t=1.89), the overall performance trend strongly favors SSH, especially in tasks where precise frequency selection is crucial for effective fine-tuning.
\begin{figure}
    \centering
    \includegraphics[width=\linewidth]{data/ratio.pdf}
    \caption{\small Ablation study of SSH on GLUE tasks illustrating the effect of varying energy ratios ($\delta$) on performance with RoBERTa-base (n=750). Performance is normalized to $\delta = 0.5$, showing optimal balance and diversity in spectral representation at $\delta = 0.7$.
}
    \label{fig:ratio}
\end{figure}



\noindent \textbf{Energy Ratio Ablation Study.}
\label{subsubsec:energyratio}
Figure~\ref{fig:ratio} presents an ablation study of SSH across GLUE tasks with varying energy ratios (\(\delta\)) on RoBERTa-base with \(n=750\), where performance is normalized to \(\delta = 0.5\). The energy ratios considered are \(\delta=0.5\), \(\delta=0.6\), \(\delta=0.7\), \(\delta=0.8\), and \(\delta=0.9\). 

% The results indicate that an energy ratio of \(\delta=0.7\) generally yields stable and improved performance across most tasks. This is notably evident in MRPC and CoLA, where performance peaks with \(\delta=0.7\). However, performance tends to decline when \(\delta\) is set too low (\(\delta=0.5\) or \(\delta=0.6\)), particularly in tasks like QNLI and CoLA, with relative performance drops of up to 1\%. Similarly, higher \(\delta\) values (\(\delta=0.8\) or \(\delta=0.9\)) also lead to decreased performance, especially in tasks such as QNLI and RTE, suggesting that excessively high energy ratios do not correlate with better task performance. 


The ablation study indicates that an energy ratio of \(\delta=0.7\) optimally balances the selection of spectral components, consistently enhancing performance across natural language understanding tasks such as MRPC and CoLA. This balance prevents overfitting and underfitting, ensuring the retention of informative frequencies while excluding those that are redundant. In contrast, lower ratios (\(\delta=0.5\) or \(\delta=0.6\)) result in inadequate frequency representation, adversely affecting performance in tasks that require robust syntactic and semantic analysis, such as QNLI and CoLA. Higher ratios (\(\delta=0.8\) and \(\delta=0.9\)), while expanding the range of considered frequencies, often introduce noise that compromises the model's focus and generalization ability, particularly evident in tasks like QNLI and STS-B.

% This demonstrates that excessive inclusion of spectral components can diminish model efficacy by detracting from its ability to generalize from training to unseen data.


\section{Conclusion}

%In this paper, w
We propose a new PEFT method called DiffoRA, which enables efficient and adaptive LLM fine-tuning based on LoRA. 
Instead of adjusting every interior rank, 
%of the decomposition matrices 
%of all modules, 
we argue that adopting LoRA module-wisely is sufficient. 
To achieve this, we construct a DAM to select the modules that are most suitable and essential to fine-tune. We theoretically analyze how the DAM impacts the convergence rate and generalization capability.
%of the pre-trained model. 
Furthermore, we adopt continuous relaxation and discretization to establish DAM.
%for each task. 
To alleviate the issue of discretization discrepancy, we utilize the weight-sharing strategy for optimization. 
%We fully implement our method and t
The experimental results demonstrate that our DiffoRA works consistently better than the baselines across all benchmarks. 


% \clearpage
\bibliography{0_acl_main}
% \bibliographystyle{plainnat}
\clearpage
\label{sec:appendix}

%%%%%%%%%%%%%%%%%%%%%%%%%%%%%%%%%%%%%%%%%%%%%%%%%%%%%%%%%%%%%%%%%%%%%%%%%%%%%%%
%%%%%%%%%%%%%%%%%%%%%%%%%%%%%%%%%%%%%%%%%%%%%%%%%%%%%%%%%%%%%%%%%%%%%%%%%%%%%%%
% APPENDIX
%%%%%%%%%%%%%%%%%%%%%%%%%%%%%%%%%%%%%%%%%%%%%%%%%%%%%%%%%%%%%%%%%%%%%%%%%%%%%%%
%%%%%%%%%%%%%%%%%%%%%%%%%%%%%%%%%%%%%%%%%%%%%%%%%%%%%%%%%%%%%%%%%%%%%%%%%%%%%%%
\newpage
\appendix
\onecolumn
\section*{Appendix Overview}
\begin{itemize}
    \item Section~\ref{appendix:related}: Related Work.
    \item Section~\ref{appendix:more_dataset}: More Dataset Details.
    \item Section~\ref{appendix:error_analysis}: Error Analysis.
    \item Section~\ref{appendix:more_qualitative}: More Qualitative Examples.
    \item Section~\ref{appendix:eval_setup}: Evaluation Prompts.
\end{itemize}


\section{Related Work}
\label{appendix:related}
\subsection{Large Multimodal Models}
The field of multimodal~\citep{Radford2021LearningTV, li2022blip, openai2023gpt4v, openai2024gpt4o} AI has experienced extraordinary growth, particularly through the development of Large Multimodal Models (LMMs)~\cite{liu2023llava,zhu2023minigpt,lin2023sphinx,Qwen2-VL}. These models build upon the achievements of Large Language Models (LLMs)~\citep{touvron2023llama,qwen2} and advanced vision models~\cite{Radford2021LearningTV}, expanding their capabilities to process multiple kinds of visual input~\cite{li2024llava,guo2023point,li2023videochat}.

Closed-source models, such as OpenAI's GPT-4o~\citep{openai2024gpt4o}, have demonstrated exceptional capabilities in visual understanding and reasoning. However, their closed-source nature creates barriers to widespread adoption and further development by the broader research community. In response, significant progress has been made in developing open-source alternatives. Early approaches like LLaVA~\cite{liu2023llava}, LLaMA-Adapter~\cite{zhang2024llamaadapter}, and MiniGPT-4~\cite{zhu2023minigpt} established a foundation by combining frozen CLIP models for image encoding with LLMs, enabling multimodal instruction tuning. Subsequent developments through projects such as InternVL2~\cite{chen2024far}, Qwen2-VL~\cite{Qwen2-VL}, SPHINX~\cite{gao2024sphinx,lin2023sphinx}, and MiniCPM-V~\cite{yao2024minicpm} have expanded these capabilities by incorporating more diverse visual instruction datasets and broadening application scenarios.

Recently, with the introduction of o1~\cite{o1}, the field of LMMs has also focused on enhancing the reasoning capability. \cite{wang2024enhancing} introduces mixed preference optimization with automatically constructed data. \cite{yao2024mulberry} proposes to leverage collective knowledge from multiple models to identify effective reasoning paths. Besides, several works~\cite{qvq-72b-preview,du2025virgo} have demonstrated the ability to replicate behaviors similar to o1 models, particularly regarding multi-step CoT reasoning with iterative self-reflection and verification processes.

\subsection{Reasoning Evaluation}
Several methods have been developed to evaluate reasoning in natural language processing, including ROSCOE~\cite{golovneva2022roscoe} and ReCEval~\cite{prasad2023receval}, which assess reasoning chains across multiple dimensions such as correctness and informativeness. However, these approaches are limited to text-only scenarios and do not address the unique challenges present in visual reasoning tasks. Furthermore, the emergence of long chain-of-thought (CoT) reasoning has introduced additional considerations, such as output efficiency and reflection quality, which existing evaluation methods do not adequately address.

On the other hand, various multimodal benchmarks have been developed to assess reasoning abilities across specific domains. Current exploration of visual reasoning predominantly focuses on the mathematics~\cite{zhang2024mavis,peng2024chimera} domains. 
MathVista~\cite{Lu2023MathVistaEM} provides a comprehensive collection of mathematical problems that assess mathematical and logical reasoning abilities. 
Building on this, MathVerse~\cite{zhang2024mathverse} introduces a new benchmark by eliminating redundant textual information to evaluate whether LMMs can accurately interpret graphical representations. 
OlympiadBench~\cite{he2024olympiadbench} further raises the complexity bar by incorporating challenging Olympiad-level mathematics and physics problems. Despite these advances in specialized domains, broader applications such as general-scene reasoning remain relatively unexplored.
Recent developments have begun to expand beyond purely scientific reasoning. For instance, M³CoT~\cite{chen-etal-2024-m3cot} and SciVerse~\cite{sciverse} incorporate commonsense tasks alongside scientific reasoning and knowledge-based assessment in the multimodal benchmark. However, most existing benchmarks focus solely on evaluating final answers while overlooking the intermediate steps, thus providing limited insights into the process through which models arrive at their conclusions.


\section{More Dataset Details}
\label{appendix:more_dataset}
\subsection{Data Source Distribution}
We visualize the data source distributions in our benchmark, which consists of 15 sets, including MathVerse~\cite{zhang2024mathverse}, MMMUPro~\cite{yue2024mmmuprorobustmultidisciplinemultimodal}, OlympiadBench~\cite{he2024olympiadbench}, MMT-Bench~\cite{ying2024mmt}, MuirBench~\cite{wang2024muirbench}, ml-rpm-bench~\cite{zhang2024far}, MMSearch~\cite{jiang2024mmsearch}, CharXiv~\cite{wang2024charxiv}, and SciVerse~\cite{sciverse}.

\begin{figure*}[!h]
\centering
\includegraphics[width=0.4\textwidth]{fig/pie_supp.pdf} 
\caption{\textbf{Data Source Distribution of MME-CoT.}}
\label{appendix:more_dataset-source}
\end{figure*}

\newpage

\subsection{Preliminary Categorization Result}
\label{appendix:preliminary_result}
\begin{table}[htbp]
    \centering
    \caption{\textbf{Accuracy of MMT-Bench for different subcategories}. ACT: Action Understanding; AUT: Attribute Similarity; CNT: Cartoon Understanding; CIM: Counting; DOC: Diagram Understanding; EMO: Difference Spotting; HAL: Geographic Understanding; IIT: Image-Text Matching; IRT: Ordering; IQT: Scene Understanding; MEM: Visual Grounding; MIA: Visual Retrieval; OCR: Object Recognition; PLP: Physical Layout Prediction; RRE: Relationship Extraction; TMP: Temporal Reasoning; VCP: Visual Comprehension; VCR: Visual Coherence Reasoning; VGR: Visual Generation; VIL: Visual Identification; VPU: Visual Prediction Understanding; VRE: Visual Reasoning Evaluation.}
    \label{tab:hit_ratio}
    \setlength{\tabcolsep}{4pt} 
    \renewcommand{\arraystretch}{1.2}
    \small 
    \begin{tabularx}{\textwidth}{l *{22}{X}}
        \toprule
        File Name & 
        \rotatebox{90}{ACT} & \rotatebox{90}{AUT} & \rotatebox{90}{CNT} & \rotatebox{90}{CIM} & 
        \rotatebox{90}{DOC} & \rotatebox{90}{EMO} & \rotatebox{90}{HAL} & \rotatebox{90}{IIT} & 
        \rotatebox{90}{IRT} & \rotatebox{90}{IQT} & \rotatebox{90}{MEM} & \rotatebox{90}{MIA} & 
        \rotatebox{90}{OCR} & \rotatebox{90}{PLP} & \rotatebox{90}{RRE} & \rotatebox{90}{TMP} & 
        \rotatebox{90}{VCP} & \rotatebox{90}{VCR} & \rotatebox{90}{VGR} & \rotatebox{90}{VIL} & 
        \rotatebox{90}{VPU} & \rotatebox{90}{VRE} \\
        \midrule
        GPT4o-cot & 0.60 & 0.60 & 0.44 & 0.67 & 0.79 & 0.30 & 0.71 & 0.50 & 0.63 & 0.10 & 0.85 & 0.60 & 0.77 & 0.36 & 0.76 & 0.48 & 0.86 & 0.80 & 0.49 & 0.48 & 0.82 & 0.85 \\
        GPT4-direct & 0.53 & 0.60 & 0.44 & 0.67 & 0.81 & 0.23 & 0.69 & 0.33 & 0.66 & 0.25 & 0.80 & 0.43 & 0.78 & 0.42 & 0.78 & 0.36 & 0.89 & 0.85 & 0.41 & 0.37 & 0.85 & 0.85 \\
        Qwen2-VL-7B-cot & 0.53 & 0.61 & 0.34 & 0.65 & 0.77 & 0.53 & 0.74 & 0.40 & 0.31 & 0.20 & 0.78 & 0.58 & 0.60 & 0.43 & 0.69 & 0.43 & 0.85 & 0.90 & 0.54 & 0.35 & 0.79 & 0.81 \\
        Qwen2-VL-7B-direct & 0.49 & 0.67 & 0.40 & 0.78 & 0.75 & 0.52 & 0.73 & 0.43 & 0.31 & 0.10 & 0.78 & 0.55 & 0.60 & 0.54 & 0.69 & 0.40 & 0.85 & 0.85 & 0.67 & 0.38 & 0.85 & 0.82 \\
        \bottomrule
    \end{tabularx}
\end{table}


\begin{table}[htbp]
    \centering
    \caption{\textbf{Accuracy of MUIRBench for different subcategories}. AU: Action Understanding; AS: Attribute Similarity; CU: Cartoon Understanding; CO: Counting; DU: Diagram Understanding; DS: Difference Spotting; GU: Geographic Understanding; ITM: Image-Text Matching; OR: Ordering; SU: Scene Understanding; VG: Visual Grounding; VR: Visual Retrieval.}

    \label{tab:hit_ratio}
    \setlength{\tabcolsep}{4pt} 
    \renewcommand{\arraystretch}{1.2} 
    \small 
    \begin{tabularx}{\textwidth}{l XXXX XXXX XXXX XXXX}
        \toprule
        File Name & AU & AS & CU & CO & DU & DS & GU & ITM & OR & SU & VG & VR \\
        \midrule
        GPT4o-cot & 0.48 & 0.57 & 0.55 & 0.75 & 0.82 & 0.64 & 0.59 & 0.82 & 0.38 & 0.88 & 0.56 & 0.70 \\
        GPT4o-direct & 0.45 & 0.62 & 0.59 & 0.50 & 0.88 & 0.62 & 0.55 & 0.86 & 0.33 & 0.74 & 0.38 & 0.77 \\
        Qwen2-VL-7B-cot & 0.38 & 0.51 & 0.42 & 0.43 & 0.43 & 0.27 & 0.21 & 0.55 & 0.13 & 0.69 & 0.37 & 0.28 \\
        Qwen2-VL-7B-direct & 0.39 & 0.47 & 0.44 & 0.41 & 0.40 & 0.33 & 0.25 & 0.51 & 0.13 & 0.67 & 0.31 & 0.20 \\
        \bottomrule
    \end{tabularx}
\end{table}



\begin{table}[htbp]
    \centering
    \caption{\textbf{Accuracy of OlympiadBench for the mathematics and physics subcategories}.}
    \label{tab:hit_ratio_oe}
    \small 
    \begin{tabular}{lcc}
        \toprule
        File Name & Mathematics & Physics\\
        \midrule
        GPT4o-cot & 0.25 & 0.04 \\
        GPT4o-direct & 0.07 & 0.03 \\
        Qwen2-VL-7B-cot & 0.05 & 0.01 \\
        Qwen2-VL-7B-direct & 0.07 & 0.01 \\
        \bottomrule
    \end{tabular}
\end{table}

\newpage

\section{Error Analysis}
\label{appendix:error_analysis}
We showcase the examples of the identified error types of reflection in Fig.~\ref{fig:ref_error_example}.
\begin{figure*}[!h]
\centering
\includegraphics[width=\textwidth]{fig/ref_error_example.pdf} 
\caption{\textbf{Examples of Reflection Error Types.}}
\label{fig:ref_error_example}
\end{figure*}


\newpage

\section{More Qualitative Examples}
\label{appendix:more_qualitative}
\begin{figure*}[!h]
\centering
\includegraphics[width=0.6\textwidth]{fig/precision_recall_example_GPT.pdf} 
\caption{\textbf{Examples of Precision and Recall Evaluation.}}
\label{fig:precision_recall_example_GPT}
\end{figure*}
\newpage

\begin{figure*}[!h]
\centering
\includegraphics[width=0.9\textwidth]{fig/precision_recall_example_Qwen.pdf} 
\caption{\textbf{Examples of Precision and Recall Evaluation.}}
\label{fig:precision_recall_example_Qwen}
\end{figure*}
\newpage

\begin{figure*}[!h]
\centering
\includegraphics[width=0.58\textwidth]{fig/precision_recall_example_QVQ.pdf}
\caption{\textbf{Examples of Precision and Recall Evaluation.}}
\label{fig:precision_recall_example_QVQ}
\end{figure*}
\newpage

\begin{figure*}[!h]
\centering
\includegraphics[width=\textwidth]{fig/precision_recall_example_QVQ2.pdf} 
\caption{\textbf{Examples of Precision and Recall Evaluation.}}
\label{fig:precision_recall_example_QVQ2}
\end{figure*}
\newpage

\begin{figure*}[!h]
\centering
\includegraphics[width=0.51\textwidth]{fig/precision_recall_example2_GPT.pdf} 
\caption{\textbf{Examples of Precision and Recall Evaluation.}}
\label{fig:precision_recall_example2_GPT}
\end{figure*}
\newpage

\begin{figure*}[!h]
\centering
\includegraphics[width=0.79\textwidth]{fig/precision_recall_example2_Qwen.pdf} 
\caption{\textbf{Examples of Precision and Recall Evaluation.}}
\label{fig:precision_recall_example2_Qwen}
\end{figure*}
\newpage

\begin{figure*}[!h]
\centering
\includegraphics[width=0.81\textwidth]{fig/precision_recall_example2_QVQ.pdf} 
\caption{\textbf{Examples of Precision and Recall Evaluation.}}
\label{fig:precision_recall_example2_QVQ}
\end{figure*}
\newpage

\begin{figure*}[!h]
\centering
\includegraphics[width=\textwidth]{fig/relevance_example_GPT.pdf} 
\caption{\textbf{Examples of Relevance Rate Evaluation.}}
% \vspace{-1cm}
\label{fig:relevance_example_GPT}
\end{figure*}
\newpage

\begin{figure*}[!h]
\centering
\includegraphics[width=\textwidth]{fig/relevance_example_Qwen.pdf} 
\caption{\textbf{Examples of Relevance Rate Evaluation.}}
% \vspace{-1cm}
\label{fig:relevance_example_Qwen}
\end{figure*}
\newpage

\begin{figure*}[!h]
\centering
\includegraphics[width=\textwidth]{fig/relevance_example_QVQ.pdf} 
\caption{\textbf{Examples of Relevance Rate Evaluation.}}
% \vspace{-1cm}
\label{fig:relevance_example_QVQ}
\end{figure*}
\newpage

\begin{figure*}[!h]
\centering
\includegraphics[width=\textwidth]{fig/ref_example_QVQ.pdf} 
\caption{\textbf{Examples of Reflection Quality Evaluation.}}
% \vspace{-1cm}
\label{fig:ref_example_QVQ}
\end{figure*}
\newpage


\section{Detailed Evaluation Setup}
\label{appendix:eval_setup}
\subsection{CoT Quality Evaluation Prompts}

\begin{tcolorbox}[breakable, colback=gray!5!white, colframe=gray!75!black, 
title=Recall Evaluation Prompt, boxrule=0.5mm, width=\textwidth, arc=3mm, auto outer arc]

You are an expert system to verify solutions to image-based problems. Your task is to match the ground truth middle steps with the provided solution.\\

INPUT FORMAT:\\
1. Problem: The original question/task\\
2. A Solution of a model\\
3. Ground Truth: Essential steps required for a correct answer\\

MATCHING PROCESS:\\

You need to match each ground truth middle step with the solution:\\

Match Criteria:\\
- The middle step should exactly match in the content or is directly entailed by a certain content in the solution\\
- All the details must be matched, including the specific value and content\\
- You should judge all the middle steps for whether there is a match in the solution\\

OUTPUT FORMAT:
\begin{verbatim}
[
  {
    "step_index": \textless integer\textgreater,
    "judgment": "Matched" | "Unmatched"
  }
]
\end{verbatim}

ADDITIONAL RULES:\\
1. Only output the JSON array with no additional information.\\
2. Judge each ground truth middle step in order without omitting any step.\\

Here are the problem, answer, solution, and ground truth middle steps:\\

[Problem]\\

\{question\}\\

[Answer]\\

\{answer\}\\

[Solution]\\

\{solution\}\\

[Ground Truth Information]\\

\{gt\_annotation\}

\end{tcolorbox}

\begin{tcolorbox}[breakable, colback=gray!5!white, colframe=gray!75!black, 
title=Precision Evaluation Prompt, boxrule=0.5mm, width=\textwidth, arc=3mm, auto outer arc]

\# Task Overview\\
Given a solution with multiple reasoning steps for an image-based problem, reformat it into well-structured steps and evaluate their correctness.\\

\# Step 1: Reformatting the Solution\\
Convert the unstructured solution into distinct reasoning steps while:\\
- Preserving all original content and order\\
- Not adding new interpretations\\
- Not omitting any steps\\

\#\# Step Types\\
1. Logical Inference Steps\\
   - Contains exactly one logical deduction\\
   - Must produce a new derived conclusion\\
   - Cannot be just a summary or observation\\
\\
2. Image Observation Steps\\
   - Pure visual observations\\
   - Only includes directly visible elements\\
   - No inferences or assumptions\\
\\
3. Background Information Steps\\
   - External knowledge or question context\\
   - No inference process involved\\

\#\# Step Requirements\\
- Each step must be atomic (one conclusion per step)\\
- No content duplication across steps\\
- Initial analysis counts as background information\\
- Final answer determination counts as logical inference\\

\# Step 2: Evaluating Correctness\\
Evaluate each step against:\\

\#\# Ground Truth Matching\\
For image observations:\\
- Key elements must match ground truth observations\\
\\
For logical inferences:\\
- Conclusion must EXACTLY match or be DIRECTLY entailed by ground truth\\

\#\# Reasonableness Check (if no direct match)\\
Step must:\\
- Premises must not contradict any ground truth or correct answer\\
- Logic is valid\\
- Conclusion must not contradict any ground truth \\
- Conclusion must support or be neutral to correct answer\\

\#\# Judgement Categories\\
- "Match": Aligns with ground truth\\
- "Reasonable": Valid but not in ground truth\\
- "Wrong": Invalid or contradictory\\
- "N/A": For background information steps\\

\# Output Requirements\\
1. The output format must be in valid JSON format without any other content.\\
2. For highly repetitive patterns, output it as a single step.\\
3. Output maximum 40 steps. Always include the final step that contains the answer.\\

Here is the json output format:\\
\#\# Output Format
\begin{verbatim}
[
  {
    "step_type": "image observation|logical inference|background information",
    "premise": "Evidence (only for logical inference)",
    "conclusion": "Step result",
    "judgment": "Match|Reasonable|Wrong|N/A"
  }
]
\end{verbatim}

Here is the problem, and the solution that needs to be reformatted to steps:\\

[Problem]\\

\{question\}\\

[Solution]\\

\{solution\}\\

[Correct Answer]\\

\{answer\}\\

[Ground Truth Information]\\

\{gt\_annotation\}

\end{tcolorbox}

\subsection{CoT Efficiency Prompt}
\begin{tcolorbox}[breakable, colback=gray!5!white, colframe=gray!75!black, 
title=Relevance Rate Evaluation Prompt, boxrule=0.5mm, width=\textwidth, arc=3mm, auto outer arc]
\# Task Overview
Given a solution with multiple reasoning steps for an image-based problem, evaluate the relevance to get a solution (ignore correct or wrong) of each step.\\

\# Step 1: Reformatting the Solution
Convert the unstructured solution into distinct reasoning steps while:\\
- Preserving all original content and order\\
- Not adding new interpretations\\
- Not omitting any steps\\

\#\# Step Types \\
1. Logical Inference Steps\\
  - Contains exactly one logical deduction\\
  - Must produce a new derived conclusion\\
  - Cannot be just a summary or observation

2. Image Description Steps\\
  - Pure visual observations\\
  - Only includes directly visible elements\\
  - No inferences or assumptions

3. Background Information Steps\\
  - External knowledge or question context\\
  - No inference process involved\\

\#\# Step Requirements
- Each step must be atomic (one conclusion per step)\\
- No content duplication across steps\\
- Initial analysis counts as background information\\
- Final answer determination counts as logical inference\\

\# Step 2: Evaluating Relevancy\\
A relevant step is considered as: 75\% content of the step must be related to trying to get a solution (ignore correct or wrong) to the question.\\

IMPORTANT NOTE:\\
Evaluate relevancy independent of correctness. As long as the step is trying to get to a solution, it is considered relevant. Logical fallacy, knowledge mistake, inconsistent with previous steps, or other mistakes do not affect relevance. A logically wrong step can be relevant if the reasoning attempts to address the question.\\

The following behaviour is considered as relevant:\\
i. The step is planning, summarizing, thinking, verifying, calculating, or confirming an intermediate/final conclusion helpful to get a solution.\\
ii. The step is summarizing or reflecting on previously reached conclusion relevant to get a solution.\\
iii. Repeating the information in the question or give the final answer.\\
iv. A relevant image depiction should be in one of following situation:\\
1. help to obtain a conclusion helpful to solve the question later;\\
2. help to identify certain patterns in the image later;\\
3. directly contributes to the answer\\
v. Depicting or analyzing the options of the question is also relevant.\\
vi. Repeating previous relevant steps are also considered relevant.\\

The following behaviour is considered as irrelevant:\\
i. Depicting image information that does not related to what is asking in the question. Example: The question asks how many cars are present in all the images. If the step focuses on other visual elements like the road or building, the step is considered as irrelevant.\\
ii. Self-thought not related to what the question is asking.\\
iii. Other information that is tangential for answering the question.\\

\# Output Format

\begin{verbatim}
[
  {
    "step_type": "image observation|logical inference|background information",
    "conclusion": "A brief summary of step result",
    "relevant": "Yes|No"
  }
]
\end{verbatim}\\

\# Output Rules\\
Direct JSON output without any other output\\
Output at most 40 steps\\

Here is the problem, and the solution that needs to be reformatted to steps:

[Problem]\\

\{question\}\\

[Solution]\\

\{solution\}
\end{tcolorbox}

\begin{tcolorbox}[breakable, colback=gray!5!white, colframe=gray!75!black, 
title=Reflection Quality Evaluation Prompt, boxrule=0.5mm, width=\textwidth, arc=3mm, auto outer arc]

Here\'s a refined prompt that improves clarity and structure:\\

\# Task\\
Evaluate reflection steps in image-based problem solutions, where reflections are self-corrections or reconsideration of previous statements.\\

\# Reflection Step Identification \\
Reflections typically begin with phrases like:\\
- "But xxx"\\
- "Alternatively, xxx" \\
- "Maybe I should"\\
- "Let me double-check"\\
- "Wait xxx"\\
- "Perhaps xxx"\\
It will throw a doubt of its previously reached conclusion or raise a new thought.\\

\# Evaluation Criteria\\
Correct reflections must:\\
1. Reach accurate conclusions aligned with ground truth\\
2. Use new insights to find the mistake of the previous conclusion or verify its correctness. \\

Invalid reflections include:\\
1. Repetition - Restating previous content or method without new insights\\
2. Wrong Conclusion - Reaching incorrect conclusions vs ground truth\\
3. Incompleteness - Proposing but not executing new analysis methods\\
4. Other - Additional error types\\

\# Input Format\\

[Problem]\\

\{question\}\\

[Solution]\\

\{solution\}\\

[Ground Truth]\\

\{gt\_annotation\}\\

\# Output Requirements\\
1. The output format must be in valid JSON format without any other content.\\
2. Output maximum 30 reflection steps.\\

Here is the json output format:\\
\#\# Output Format
\begin{verbatim}
[
  {
    "conclusion": "One-sentence summary of reflection outcome",
    "judgment": "Correct|Wrong",
    "error_type": "N/A|Repetition|Wrong Conclusion|Incompleteness|Other"
  }
]
\end{verbatim}

\# Rules\\
1. Preserve original content and order\\
2. No new interpretations\\
3. Include ALL reflection steps\\
4. Empty list if no reflections found\\
5. Direct JSON output without any other output

\end{tcolorbox}

\subsection{Direct Evaluation Prompt}
\begin{tcolorbox}[breakable, colback=gray!5!white, colframe=gray!75!black, 
title=Answer Extraction Prompt, boxrule=0.5mm, width=\textwidth, arc=3mm, auto outer arc]
You are an AI assistant who will help me to extract an answer of a question. You are provided with a question and a response, and you need to find the final answer of the question. \\

Extract Rule:

[Multiple choice question]

1. The answer could be answering the option letter or the value. You should directly output the choice letter of the answer.

2. You should output a single uppercase character in A, B, C, D, E, F, G, H, I (if they are valid options), and Z.

3. If the meaning of all options are significantly different from the final answer, output Z. \\

[Non Multiple choice question]

1. Output the final value of the answer. It could be hidden inside the last step of calculation or inference. Pay attention to what the question is asking for to extract the value of the answer.

2. The final answer could also be a short phrase or sentence.

3. If the response doesn't give a final answer, output Z.\\

Output Format: 
Directly output the extracted answer of the response. \\

\{In Context Examples\}\\

Question: \{question\}

Answer: \{response\}\\

Your output: 

\end{tcolorbox}

\begin{tcolorbox}[breakable, colback=gray!5!white, colframe=gray!75!black, 
title=Answer Scoring Prompt, boxrule=0.5mm, width=\textwidth, arc=3mm, auto outer arc]

You are an AI assistant who will help me to judge whether two answers are consistent.\\

Input Illustration:
[Standard Answer] is the standard answer to the question. 
[Model Answer] is the answer extracted from a model's output to this question. 

Task Illustration:
Determine whether [Standard Answer] and [Model Answer] are consistent.\\

Consistent Criteria:

[Multiple-Choice questions]

1. If the [Model Answer] is the option letter, then it must completely matches the [Standard Answer].

2. If the [Model Answer] is not an option letter, then the [Model Answer] must completely match the option content of [Standard Answer].

[Nan-Multiple-Choice questions]

1. The [Model Answer] and [Standard Answer] should exactly match.

2. If the meaning is expressed in the same way, it is also considered consistent, for example, 0.5m and 50cm.\\

Output Format: 
1. If they are consistent, output 1; if they are different, output 0.

2. DIRECTLY output 1 or 0 without any other content.

\{In Context Examples\}\\

Question: \{question\}

[Model Answer]: \{extract\_answer\}

[Standard Answer]: \{gt\_answer\}

Your output:

\end{tcolorbox}

\end{document}
