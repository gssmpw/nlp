\section{Introduction}
\label{sec:introduction}

Replication protocols play the foundational role in designing distributed systems, including distributed storage systems~\cite{10.14778/3007263.3007267, 10.14778/2002938.2002939, zippydb, bankDB, NetflixDB, GEDB, HESSDB}, distributed ledgers~\cite{baasAlibaba, baasAWS, baasAzure, baasIBM, baasOracle}, distributed data analytics systems~\cite{NetflixDB}.  For performance and fault tolerance requirements, distributed systems employ a {\em replication protocol} to maintain a consistent view of the datasets while improving the performance for read-heavy workloads and, more importantly, guaranteeing fault tolerance, {\em i.e.,} reliability and availability in the presence of failures~\cite{zippydb, dynamo, lakshman2009, redis, rocksdb, leveldb, memcached2004, bfthyperledger}.



%These systems employ replication protocols to deal with system failures, e.g., network links, machines, that are present in the cloud infrastructure without violating their correctness. 



%Get to the point!

%As such, the software systems
%need to be constructed in a manner that treats failures as the
%norm rather than the exception.
%\addition{Distributed data stores ub}


%These applications require frequent data store accesses to process user requests that typically present read-heavy access patterns~\cite{246158, 10.14778/3523210.3523213} (REMOVE HERE). To improve user experience, such systems strive to provide fault tolerance without sacrificing performance or consistency~\cite{Herlihy:1990, attiya:1991}.

For handling failures, distributed systems predominately employ Crash Fault Tolerant (CFT) replication protocols~\cite{raft,chain-replication, 10.5555/1855807.1855818, Reed2008AST, 10.5555/800253.807732, lynch:1997, 10.1145/279227.279229, Hermes:2020} to provide consistent replication assuming a {\em fail-stop fault model}, {\em i.e.}, replicas are honest and can only fail by crashing~\cite{delporte}. Unfortunately, CFT protocols are {\em inadequate} for modern untrusted cloud environments, where the underlying cloud infrastructure can be compromised by an adversary, {\em e.g.}, co-located tenants or even a misbehaving cloud operator that may eavesdrop or actively influence the replicas' behavior. In such an untrusted environment, the surface of faults and attacks expands beyond the CFT fail-stop model, ranging from software bugs and configuration errors to malicious attacks~\cite{Gunawi_bugs-in-the-cloud, Shinde2016, hahnel2017high}. CFT protocols are fundamentally incapable of providing consistent replication in the presence of non-benign faults in untrusted cloud environments.


 


%\pramod{why do we need the following text?!} CFT protocols can be either leader-based~\cite{raft,chain-replication, 10.5555/1855807.1855818, Reed2008AST, 10.5555/800253.807732} where a designated node, the leader, drives the protocol execution or decentralized~\cite{lynch:1997, 10.1145/279227.279229, Hermes:2020} where there is no leader and participant nodes (or replicas) propose and execute requests coordinating with each other. We provide a taxonomy of stronly-consistent CFT protocols in $\S$~\ref{sec:background:CFT}.

%However, our growing reliance on online services not only does it require high availability and performance but trustworthy replication too. CFT protocols assume a fail-stop fault model; replicas are honest and can only fail by crashing~\cite{delporte}. However, there is evidence that  the surface of faults and attacks in the third-party untrusted cloud infrastructure has now been expanded beyond the CFT model, ranging from software bugs and configuration errors to malicious attacks~\cite{Gunawi_bugs-in-the-cloud, Shinde2016, hahnel2017high}. Unfortunately, CFT protocols are fundamentally incapable of providing consistent replication in the presence of non-benign faults.


To overcome the limitations of  CFT protocols, Byzantine Fault Tolerant (BFT) protocols~\cite{Lamport:1982} offer important foundations for developing distributed data stores with stronger guarantees in the presence of {\em Byzantine failures}, {\em i.e.}, nodes can fail in arbitrary ways. While BFT protocols can tolerate Byzantine failures, {\em including malicious adversaries}, they are unfortunately {\em not adopted in practice} because of their high-performance and replication resource overheads, and implementation complexity~\cite{visigoth-eurosys}. For instance, BFT protocols require $3f+1$ replicas instead of the $2f+1$ replicas required by CFT protocols for tolerating $f$ faults. BFT protocols also require additional communication rounds (at least three phases instead of two phases in the CFT model); thus incurring high latency and reduced read/write throughput. Lastly, BFT protocols are extremely complex and hard to understand, let alone to implement correctly and optimize~\cite{unsafeZyzzyva}: even intuitive algorithmic optimizations can strongly affect other parts of the protocol~\cite{10.1145/2658994}.

The  “CFT vs. BFT" conundrum creates a fundamental design trade-off between the {\em efficiency of CFT protocols} for practical deployments and the {\em robustness of BFT protocols} for Byzantine settings of modern cloud environments. To strike a balance, our work targets the following research question: {\em How can we design robust and efficient replication protocols for distributed systems hosted in untrusted cloud environments?}

%\myparagraph{Limitations of prior work} Previous research~\cite{minBFT, hybster, DBLP:journals/corr/LiuLKA16a, A2M, 10.1145/2168836.2168866} made an effort to optimize BFT protocols by reducing the number of replicas required using trusted hardware. Specifically, they use trusted modules along with the protocol's ``traditional'' (untrusted/Byzantine) counterpart to prevent Byzantine processes from making conflicting statements for the same request (equivocation). Unfortunately, as before, all of these efforts require programmers to understand, transform and optimize such complex protocols; a task as challenging as it is error-prone. Consequently, BFT protocols are still not widely adopted by cloud services~\cite{bft-time-is-now, 10.1145/1711475.1711494} that, instead, sacrifice security and rely on CFT protocols to handle failures~\cite{Hunt:2010, etcd, zippy, redis, rafthyperledger}.


%Unfortunately, traditional BFT protocols present severe limitations that impede their adoption in modern, high-end distributed systems ($\S$~\ref{sec:background:CFT}). First, classical BFT protocols are limited in performance and scalability: they incur high message complexity and mulitple rounds~\cite{} deploying at least $3f+1$ nodes to tolerate up to $f$ Byzantine nodes. Secondly, BFT replication is complex, introducing burdens to developers \dimitra{say something more here to strengthen this}.



%\noindent{\underline{\#1: Performance and scalability.}} BFT protocols are slow due to high message complexity (typically, $O(n^2)$) and multiple protocol rounds while they require at least $3f+1$ nodes to tolerate up to $f$ Byzantine nodes which severely limits scalability. In contrast, widely adopted CFT protocols~\cite{raft, Reed2008AST, primary-backup, engraft} comprise two rounds and $O(n)$ message complexity with $2f+1$ replicas.


%\noindent{\underline{\#2: Complexity and adoption.}} BFT protocols are complex, introducing burdens to developers. Even if system designers wish to use a state-of-the-art BFT protocol, optimizing it for the specific application settings (e.g., network bandwidth,  number of clients and replicas, cryptographic libraries, etc.) can be complicated. Guerraoui et al.~\cite{10.1145/2658994} found that most protocol implementations consist of thousands of lines of code, e.g., PBFT~\cite{Castro:2002} and Zyzzyva~\cite{10.1145/1658357.1658358}. Even trivial changes or intuitive optimizations can be extremely hard and might affect other parts of the protocol (e.g., view-change in Zyzzyva).



%%%All the above in combination with the difficulty of testing distributed protocols~\cite{Chandra2007PaxosML} require great effort for designing and optimizing such protocols. Consequently, BFT protocols are still not widely adopted by cloud services~\cite{bft-time-is-now, 10.1145/1711475.1711494} that, instead, rely on CFT protocols to handle failures~\cite{Hunt:2010,etcd, zippy, 51, redis}.



%The only complete proof of a BFT protocol we knew of is that of PBFT and it involves 35 pages (even without using any formal language).1 This difficulty, together with the impossibility of exhaustively testing distributed protocols [8] would rather plead for never changing a protocol that was tested and proven correct, e.g., PBFT.

%%%%% Previous works~\cite{minBFT, hybster, DBLP:journals/corr/LiuLKA16a, A2M, 10.1145/2168836.2168866} aim at optimizing BFT protocols and mostly focus on reducing the number of replicas required. They use trusted modules along with the protocol's ``traditional'' (untrusted) counterpart to prevent Byzantine processes from making conflicting statements for the same request (equivocation). As before, all of these efforts require programmers to understand, transform and optimize these complex protocols; a task that is quite challenging.




%{\em ``The solution lies within the problem!"}---our work 

Our work leverages modern cloud hardware for transforming existing CFT protocols for Byzantine settings in untrusted cloud environments. Specifically, our transformation underpins the robustness and efficiency axes.
For {\em robustness}, we leverage trusted hardware available as part of confidential cloud computing to harden the security properties of CFT protocols~\cite{intel-sgx, keystone_eurosys, amd-sev, intelTDX}. In particular, we leverage trusted execution environments (TEEs) to provide two key properties for successfully transforming a CFT protocol to operate in Byzantine settings, as identified by Clements et al.~\cite{clement2012}: (a) transferable authentication, {\em i.e.}, the ability to establish trust in nodes in distributed settings by designing a remote attestation protocol, and (b) non-equivocation, i.e., once the trust is established in a node via the remote attestation protocol, the node follows the CFT replication protocol faithfully, and therefore, it cannot send conflicting statements to other nodes. %to reach agreement in the presence of Byzantine faults in the asynchronous communication model.

For {\em efficiency}, we leverage the modern networking hardware, such as RDMA/DPDK for kernel bypass, to design a highly optimized communication protocol for replicating the state across nodes in distributed settings~\cite{rdma, dpdk, erpc}, while overcoming the I/O bottlenecks in trusted computing~\cite{rkt-io}.


%Specifically, we leverage \emph{(1)} trusted hardware to harden the properties of CFT protocols building a transformation of CFT protocols for Byzantine settings and \emph{(2)} modern networking (direct I/O) for efficiency. In addition, we show that our transformation can \emph{transparently} offer confidentiality to applications when they need it~\cite{ciad} ($\S$~\ref{sec:motivation}), an extra property not offered by traditional BFT, without sacrificing performance. More importantly, we give system designers the opportunity to build protocols that tolerate Byzantine faults without having to become BFT experts.


\myparagraph{Our proposal: \projecttitle{}} 
\projecttitle{} leverages TEEs along with direct I/O to resolve the tension between security and performance by building an efficient and practical transformation of unmodified CFT replication protocols for Byzantine settings. \projecttitle{} achieve this by implementing a distributed Trusted Computing Base (TCB) that shields the replication protocol execution and {\em extends} the security properties offered by a single TEE (whose security properties are only effective in a single-node setup) to a distributed setting of TEEs. Our design is comprised of a transferable authentication phase ($\S$~\ref{sec:attestation}) for distributed trust establishment, a high-performant network stack for secure communication over the untrusted network ($\S$~\ref{subsec:networkin}) and a memory-efficient KV store ($\S$~\ref{subsec:KV}). 

% implementing a distributed Trusted Computing Base (TCB) that shields communication .. %..that transforms unmodified CFT protocols for Byzantine settings. We explore and realize the potential of Trusted Execution Environments (TEEs) to build 

%We explore and realize the potential of Trusted Execution Environments (TEEs) to build an efficient and practical transformation, a ``\projecttitle{}'', of a CFT protocol for Byzantine settings.  If used naively, however, TEEs incur great overheads~\cite{10.5555/1924943.1924946}. In this paper, we show how we can mitigate the performance impact of using TEEs by adopting direct network I/O. As a result, our approach offers BFT while providing {\em better performance (and security, i.e., confidentiality) than conventional BFT protocols} and reaping {\em all the benefits of CFT protocols} (fewer nodes, easier implementation, simplicity).

%To sum up, this paper poses the following question: \emph{Can we leverage (and how) the advances in trusted computing and  networking, like Trusted Execution Environments (TEEs) and direct I/O, to harden the properties of CFT protocols to (1) target a weaker fault model (i.e., Byzantine faults) while (2) offering performance and scalability?}

%Our answer is \projecttitle{}, an approach to transform a CFT protocol to handle Byzantine faults. We materialize the \projecttitle{} approach as a generic library, \projecttitle{}-lib, that prevents equivocation and guarantees authentication for the network. Our key insight is the use of modern hardware to achieve both BFT and performance---two often contradictory goals; we use TEEs for trust and direct network I/O for performance.

We materialise \projecttitle{} approach as a generic library, \projecttitle{}-lib ($\S$~\ref{sec:recipe-implementation}), on top of Intel SGX and the \scone{} framework~\cite{arnautov2016scone}. We carefully build \projecttitle{}'s high-performance network stack for TEEs, extending eRPC~\cite{erpc} to enable direct I/O, essentially DMA operations, within the protected TEE domain while shielding the exchanged messages to maintain authenticity and non-equivocation accross the network. Lastly, our lock-free KV store is designed to optimize for the limited trusted area in TEEs~\cite{speicher-fast, shieldstore}.% on top of folly library~\cite{folly}.

%Our , built on top of TEEs and direct network I/O, follows an established RPC-API, e.g., RDMA programming paradigm~\cite{rdma}, allowing system designers to seamlessly transform a CFT protocol for Byzantine settings. We implement \projecttitle{}-lib on top of Intel SGX~\cite{intel-sgx} and \scone{}. In addition, we extend eRPC~\cite{erpc} (on top of DPDK~\cite{dpdk}) and we build a shielded high-performant RPC library, carefully designed for TEEs, that preserves non-equivocation and authentication in the untrusted network.


Our evaluation assesses \projecttitle{}'s generality and efficiency. Specifically, to show the generality of our approach, we apply and evaluate \projecttitle{} on real hardware with four well-known CFT protocols (from now on, an `R-' prefix stands for the transformed protocol); a decentralized (leaderless) linearizable multi-writer multi-reader protocol (ABD)~\cite{lynch:1997} (R-ABD), two leader-based protocols with linearizable reads, Raft~\cite{raft} (R-Raft) and Chain Replication (CR)~\cite{chain-replication} (R-CR), and AllConcur~\cite{Poke2016AllConcurLC} (R-AllConcur), a decentralized consensus protocol with consistent local reads. To evaluate performance, we compare \projecttitle{} protocols with two competitive BFT replication protocols, BFT-smart~\cite{bft-smart}, whose specific implementation has been adopted in industry~\cite{bftsmarthyperledger} and Damysus~\cite{10.1145/3492321.3519568} a state-of-the-art BFT replication protocol.  Our evaluation shows that \projecttitle{} achieves up to $24\times$ and $5.9\times$ better throughput w.r.t. PBFT and Damysus, respectively, while improving scalability---\projecttitle{} requires $2f+1$ replicas, $f$ fewer replicas compared to PBFT ($3f+1$). We further show that \projecttitle{} can offer confidentiality---a security property not provided by traditional BFT protocols---while achieving a speedup of $7\times$---$13\times$ w.r.t. PBFT and up to $4.9\times$ w.r.t. Damysus.

\myparagraph{Artifact} \projecttitle{} is available as an open-source project.%: \url{https://anonymous.4open.science/r/secure_replication-907F/}.


%build on top of Intel SGX. Our \projecttitle{} up to $24\times$ and $5.9\times$ better throughput w.r.t. BFT-smart and Damysus.

\if 0
\myparagraph{Contributions} Overall, we make the following contributions:


\begin{itemize}[leftmargin=*]
    \item {\bf Hardware-assisted transformation of CFT protocols:} We present \projecttitle{}, a generic approach for transforming CFT protocols to tolerate Byzantine failures without any modifications to the core of the protocols, e.g., states, message rounds, and complexity. We realize our approach by implementing \projecttitle{}-lib leveraging the advances in modern hardware; we use trusted hardware to guarantee transferrable authentication and non-equivocation for thwarting Byzantine errors. Further, we combine trusted hardware with direct network I/O (RDMA/DPDK~\cite{rdma, dpdk}) for performance.

    \item {\bf Generic \projecttitle{} APIs:} We propose generic \projecttitle{} APIs to transform the existing codebase of CFT protocols for Byzantine settings. Using \projecttitle{} APIs, we have successfully
transformed a range of leader-/leaderless-based CFT protocols enforcing different (total order/per-key) ordering semantics.  We further show that \projecttitle{} can offer confidentiality---a security property not provided by traditional BFT protocols.

    
    \item {\bf Formal correctness analysis:}  We provide a correctness analysis for the safety and liveness properties of our
transformation of CFT protocols operating in Byzantine settings. % We provide a formal analysis showing the correctness and 
    \item {\bf \projecttitle{} in action:} We present an extensive evaluation of \projecttitle{} by applying it to four CFT protocols: Chain Replication, Raft, ABD, and AllConcur. We evaluate these four protocols against the state-of-the-art BFT protocol implementations and show that \projecttitle{} achieves up to $24\times$ and $5.9\times$ better throughput.
    
  %  BFT-smart~\cite{bft-smart}, an optimized version of PBFT~\cite{Castro:2002} and Damysus~\cite{10.1145/3492321.3519568}, an optimized version of HotStuff~\cite{DBLP:journals/corr/abs-1803-05069}. Our evaluation shows that \projecttitle{} achieves up to $24\times$ and $5.9\times$ better throughput w.r.t. PBFT and Damysus, respectively, while improving scalability---\projecttitle{} requires $2f+1$ replicas, $f$ fewer replicas compared to PBFT ($3f+1$). We further show that \projecttitle{} can offer confidentiality---a security property not provided by traditional BFT protocols---while achieving a speedup of $7\times$---$13\times$ w.r.t. PBFT and up to $4.9\times$ w.r.t. Damysus.
\end{itemize}

\fi 
%  \vspace{10pt}

% \noindent{\bf {Limitations of \projecttitle{}.}} While \projecttitle{} remains generic, it has been build on top of Intel SGX, which has been abandoned. However,  we believe that this unfortunate event is not restrictive for \projecttitle{}'s adoption. Instead, the state-of-the-art TEEs, such as Intel TDX~\cite{intelTDX} and AMD SEV~\cite{amd-sev}, can support \projecttitle{} out-of-the-box. These TEEs can also facilitate the development and improve the performance of new transformed protocols as they overcome some core architectural limitations of Intel SGX ($\S$~\ref{sub:background:TEE}). Particularly, they come with larger trusted memory~\cite{intelTDX2} and trusted timer sources~\cite{intelTSC, amdTSC}.