\section{Introduction}
\label{sec:introduction}

\myparagraph{}

\myparagraph{Motivation} Replication protocols play a foundational role in designing distributed data storage systems~\cite{10.14778/3007263.3007267, 10.14778/2002938.2002939, zippydb, bankDB, NetflixDB, GEDB, HESSDB}, distributed ledgers~\cite{baasAlibaba, baasAWS, baasAzure, baasIBM, baasOracle}, distributed data analytics systems~\cite{NetflixDB}. For performance and fault tolerance requirements, distributed systems employ Crash Fault Tolerant (CFT) {\em replication protocols}~\cite{raft,chain-replication, 10.5555/1855807.1855818, Reed2008AST, 10.5555/800253.807732, lynch:1997, 10.1145/279227.279229, Hermes:2020} to maintain a consistent view of the datasets guaranteeing fault tolerance, i.e., reliability and availability in the presence of failures~\cite{zippydb, dynamo, lakshman2009, redis, rocksdb, leveldb, memcached2004, bfthyperledger}.
%while improving the performance for read-heavy workloads and, more importantly,





Unfortunately, CFT protocols assume a {\em fail-stop model}\manos{It's a bit weird to say that a crash fault tolerant unfortunately only tolerates crash faults}, i.e., replicas are honest and can only fail by crashing~\cite{delporte}. As such, they are {\em inadequate} for modern untrusted cloud environments, where the underlying cloud infrastructure can be compromised by an adversary, e.g., co-located tenants or even a misbehaving cloud operator that may eavesdrop or actively influence the replicas' behavior\manos{Citations here? Some of the follow-up work to Foreshadow was showing that such attacks are possible, for example. Also, is eavesdropping relevant? That implies read-only access and doesn't affect the behavior of the system}. In such an untrusted environment, the surface of faults and attacks expands beyond the CFT fail-stop model, ranging from software bugs and configuration errors to malicious attacks~\cite{Gunawi_bugs-in-the-cloud, Shinde2016, hahnel2017high}. CFT protocols are fundamentally incapable of providing consistent replication in the presence of non-benign ({\em Byzantine}) faults in untrusted cloud environments.


bring 2.1 (BFT vsCFT)
\myparagraph{Limitations of state-of-the-art approaches}While Byzantine Fault Tolerant (BFT) protocols~\cite{Lamport:1982} offer important foundations for developing distributed systems with stronger guarantees in the presence of {\em Byzantine failures} (i.e., nodes can fail in arbitrary ways), they are {\em not adopted \underline{in practice}} because of their high-performance and replication resource overheads, and implementation complexity~\cite{visigoth-eurosys}.

The “CFT vs. BFT" conundrum creates a fundamental design trade-off between the {\em efficiency of CFT protocols} for practical deployments and the {\em robustness of BFT protocols} for Byzantine settings of modern cloud environments. However, traditional BFT protocols design and evaluation has not taken into account {\em modern cloud hardware}. 
 
%To overcome the limitations of  CFT protocols,  While BFT protocols can tolerate Byzantine failures, {\em including malicious adversaries}, they are unfortunately {\em not adopted in practice} because of their high-performance and replication resource overheads, and implementation complexity~\cite{visigoth-eurosys}. For instance, BFT protocols require $3f+1$ replicas instead of the $2f+1$ replicas required by CFT protocols for tolerating $f$ faults. BFT protocols also require additional communication rounds (at least three phases instead of two phases in the CFT model); thus incurring high latency and reduced read/write throughput. Lastly, BFT protocols are extremely complex and hard to understand, let alone to implement correctly and optimize~\cite{unsafeZyzzyva}: even intuitive algorithmic optimizations can strongly affect other parts of the protocol~\cite{10.1145/2658994}.

%While the “CFT vs. BFT" conundrum creates a fundamental design trade-off between the {\em efficiency of CFT protocols} for practical deployments and the {\em robustness of BFT protocols} for Byzantine settings of modern cloud environments. %To strike a balance, our work targets the following research question: {\em How can we design robust and efficient replication protocols for distributed systems hosted in untrusted cloud environments?}


bring 2.3

\myparagraph{Key insights and high-level design} Our work seeks to resolve this trade-off leveraging modern cloud hardware for transforming existing CFT protocols for Byzantine settings in untrusted cloud environments. Our transformation underpins the robustness and efficiency axes.
For {\em robustness}, we leverage trusted hardware available as part of confidential cloud computing to harden the security properties of CFT protocols~\cite{intel-sgx, keystone_eurosys, amd-sev, intelTDX}. In particular, we leverage trusted execution environments (TEEs) to provide two key properties for successfully transforming a CFT protocol to operate in Byzantine settings, as identified by Clement et al.~\cite{clement2012}: (a) transferable authentication, i.e., the ability to establish trust in nodes in distributed settings by designing a remote attestation protocol, and (b) non-equivocation, i.e., once the trust is established in a node via the remote attestation protocol, the node follows the CFT replication protocol faithfully, and therefore, it cannot send conflicting statements to other nodes. %to reach agreement in the presence of Byzantine faults in the asynchronous communication model.

For {\em efficiency}, we leverage the modern networking hardware, such as RDMA/DPDK for kernel bypass, to design a highly optimized communication protocol for replicating the state across nodes in distributed settings~\cite{rdma, dpdk, erpc}, while overcoming the I/O bottlenecks in trusted computing~\cite{rkt-io}.


-> Say that the theoretical transformation has not beeing realised


\myparagraph{Our proposal: \projecttitle{}} 
\projecttitle{} leverages TEEs along with direct I/O to resolve the tension between security and performance by building an efficient and practical transformation of unmodified CFT replication protocols for Byzantine settings. \projecttitle{} achieve this by implementing a distributed trusted computing base (TCB) that shields the replication protocol execution and {\em extends} the security properties offered by a single TEE (whose security properties are only effective in a single-node setup) to a distributed setting of TEEs. Our design is comprised of a transferable authentication phase ($\S$~\ref{sec:attestation}) for distributed trust establishment, a high-performant network stack for secure communication over the untrusted network ($\S$~\ref{subsec:networkin}) and a memory-efficient KV store ($\S$~\ref{subsec:KV}).  We materialise \projecttitle{} approach as a generic library, \projecttitle{}-lib ($\S$~\ref{sec:recipe-implementation}).


%We materialise \projecttitle{} approach as a generic library, \projecttitle{}-lib ($\S$~\ref{sec:recipe-implementation}), on top of Intel SGX and the \scone{} framework~\cite{arnautov2016scone}. We carefully build \projecttitle{}'s high-performance network stack for TEEs, extending eRPC~\cite{erpc} to enable direct I/O, essentially DMA operations, within the protected TEE domain while shielding the exchanged messages to maintain authenticity and non-equivocation accross the network. Lastly, our lock-free KV store is designed to optimize for the limited trusted area in TEEs~\cite{speicher-fast, shieldstore}.% on top of folly library~\cite{folly}.



Our evaluation assesses \projecttitle{}'s generality and efficiency. Specifically, to show the generality of our approach, we apply and evaluate \projecttitle{} on real hardware with four well-known CFT protocols (from now on, an `R-' prefix stands for the transformed protocol); a decentralized (leaderless) linearizable multi-writer multi-reader protocol (ABD)~\cite{lynch:1997} (R-ABD), two leader-based protocols with linearizable reads, Raft~\cite{raft} (R-Raft) and Chain Replication (CR)~\cite{chain-replication} (R-CR), and AllConcur~\cite{Poke2016AllConcurLC} (R-AllConcur), a decentralized consensus protocol with consistent local reads. To evaluate performance, we compare \projecttitle{} protocols with two competitive BFT replication protocols, BFT-smart~\cite{bft-smart} (PBFT~\cite{Castro2002}), whose specific implementation has been adopted in industry~\cite{bftsmarthyperledger} and Damysus~\cite{10.1145/3492321.3519568} a state-of-the-art BFT replication protocol.  Our evaluation shows that \projecttitle{} achieves up to $24\times$ and $5.9\times$ better throughput w.r.t. PBFT and Damysus, respectively, while improving scalability---\projecttitle{} requires $2f+1$ replicas, $f$ fewer replicas compared to PBFT ($3f+1$). We further show that \projecttitle{} can offer confidentiality---a security property not provided by traditional BFT protocols---while achieving a speedup of $7\times$---$13\times$ w.r.t. PBFT and up to $4.9\times$ w.r.t. Damysus.

%\myparagraph{Artifact} \projecttitle{} is available as an open-source project.%: \url{https://anonymous.4open.science/r/secure_replication-907F/}.


%build on top of Intel SGX. Our \projecttitle{} up to $24\times$ and $5.9\times$ better throughput w.r.t. BFT-smart and Damysus.
To summarize, we make the following contributions:

\begin{itemize}[leftmargin=*]
    \item {\bf Hardware-assisted transformation of CFT protocols:} We present \projecttitle{}, a generic approach for transforming CFT protocols to tolerate Byzantine failures without any modifications to the core of the protocols.%, e.g., states, message rounds, and complexity. %We realize our approach by implementing \projecttitle{}-lib leveraging the advances in modern hardware; we use trusted hardware to guarantee transferrable authentication and non-equivocation for thwarting Byzantine errors. Further, we combine trusted hardware with direct network I/O (RDMA/DPDK~\cite{rdma, dpdk}) for performance.

    \item {\bf Generic \projecttitle{} APIs:} We propose generic \projecttitle{} APIs to transform the existing codebase of CFT protocols for Byzantine settings. Using \projecttitle{} APIs, we have successfully
transformed a range of leader-/leaderless-based CFT protocols enforcing different ordering semantics. % We further show that \projecttitle{} can offer confidentiality---a security property not provided by traditional BFT protocols.

    
    \item {\bf Formal correctness analysis:}  We provide a correctness analysis for the safety and liveness properties of our
transformation of CFT protocols operating in Byzantine settings. % We provide a formal analysis showing the correctness and 
    \item {\bf \projecttitle{} in action:} We present an extensive evaluation of \projecttitle{} by applying it to four CFT protocols: Chain Replication, Raft, ABD, and AllConcur. We evaluate these four protocols against the state-of-the-art BFT protocol implementations and show that \projecttitle{} achieves up to $24\times$ and $5.9\times$ better throughput.
    
  %  BFT-smart~\cite{bft-smart}, an optimized version of PBFT~\cite{Castro:2002} and Damysus~\cite{10.1145/3492321.3519568}, an optimized version of HotStuff~\cite{DBLP:journals/corr/abs-1803-05069}. Our evaluation shows that \projecttitle{} achieves up to $24\times$ and $5.9\times$ better throughput w.r.t. PBFT and Damysus, respectively, while improving scalability---\projecttitle{} requires $2f+1$ replicas, $f$ fewer replicas compared to PBFT ($3f+1$). We further show that \projecttitle{} can offer confidentiality---a security property not provided by traditional BFT protocols---while achieving a speedup of $7\times$---$13\times$ w.r.t. PBFT and up to $4.9\times$ w.r.t. Damysus.
\end{itemize}
