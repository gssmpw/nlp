

\subsection{System Model}
%\manos{Rename to model and assumptions/definitions}
\label{sec:system_model}
\myparagraph{Replication protocol}
A replication protocol ensures that a set of $N$  replicas $(R_i,...,R_n)$ maintains a consistent and available state $S$ despite failures or concurrency. A replication protocol operates over a distributed state machine, where replicas execute a deterministic sequence of operations from the set of operations $O$, s.t. $S \times O \rightarrow S$, and apply them in a consistent order.
Since not all replication protocols solve consensus and thus are not state-machine-replication (SMR) protocols, we do not assume the typical consensus properties (termination, agreement, and validity). Instead, we assume that the replication protocols should guarantee the following properties:
%Thus, general CFT protocols solve the consensus problem, where starting with an initial input value, all nodes have to agree on a single output value, and by extension, the CFT protocol has to satisfy the typical consensus properties (termination, agreement, and validity). CFT protocols should additionally guarantee the following properties:

\begin{itemize}
    \item \textbf{Consistency:} All correct replicas agree on the same sequence of operations.
   \item \textbf{Availability:} Replicas process requests as long as $<f$ replicas fail.
    \item \textbf{Fault tolerance:} The system operates correctly with up to $f$ failures (crash or Byzantine).    
\end{itemize}



\myparagraph{Model sketch}
We model the distributed system as a set of $N$ TEEs in $N$ nodes (or replicas), each hosting either a \emph{follower} or a \textit{coordinator} process $P_i$ which executes a CFT protocol. The system is modeled as a state machine, where each replica $R_i$ maintains a local state $S_i$ and transitions between states based on received messages and protocol execution rules.
%and communicates by exchanging messages. %The terms \textit{replica} and \textit{node} 
%and \emph{process}
%can be used interchangeably. 
We assume that \projecttitle{}'s nodes run in a third-party untrusted cloud infrastructure. %We do not make any assumptions about the client processes.% (From now on simply clients.)---however, it makes sense that they live on separate nodes and forward their requests to the coordinator.
A coordinator serves client requests by initiating the implemented CFT replication protocol. Upon completion, it replies back to clients. In leaderless protocols, coordinators are selected randomly (any node can act as a follower and/or a coordinator). In leader-based protocols, only the active leader can act as a coordinator, the rest of the nodes are followers. %A coordinator is always a replica, but not vice versa.

\myparagraph{Communication model} 
Nodes communicate via a fully-con\-nected, bidirectional, point-to-point and unreliable message-passing network, where messages can be arbitrarily delayed, re-ordered or dropped. In line with previous BFT protocols, we adopt the partial synchrony model~\cite{10.1145/42282.42283}, where there is a known bound $\Delta$ and an unknown Global Stabilization Time (GST), such that after GST, all communications arrive within time $\Delta$.

%We consider an asynchronous network where there are unbounded but finite message transmission delays (messages are delivered \emph{eventually}). 

\myparagraph{Fault and threat model}
%We adopt two failure models: crash-failures ~\cite{crash-consistency} and %Byzantine behavior ~\cite{Lamport:1982}.
We say that a node $N_i$ is \textit{faulty} if it exhibits Byzantine behavior~\cite{Lamport:1982}. 
The unprotected (\emph{out-of-the-TEE}) infrastructure (e.g., host memory, OS, NIC, network infrastructure/adversaries) can exhibit arbitrary Byzantine behavior while we assume that the TEEs can only crash-fail. We say that a node is faulty if one of the following holds true:
\textit{(i)} the TEE fails by crashing or \textit{(ii)} the unprotected infrastructure is Byzantine.
%For safety (and liveness), we assume that for $N \ge (2f+1)$ nodes up to $f$ can be \emph{faulty}.%, while the rest are \textit{correct}.
Safety is defined as follows:  If a correct replica $R_j$ delivers a message $m$  from $R_i$, then $R_i$  must have previously sent $m$, and $m$ is consistent with the protocol state. Liveness is defined as follows: If a client submits a request $r$, and a majority of replicas are correct, then $r$ is eventually committed. Last, for safety and liveness, we assume that for $N \ge (2f+1)$ nodes up to $f$ can be \emph{faulty}.

\myparagraph{Cryptographic model} 
We assume collision resistance for the hash functions; no computationally bounded adversary can find two distinct inputs $m \neq m'$ such that $hash(m)=hash(m')$, except with negligible probability. We also make the conventional assumption that signatures and keys are unforgeable, the initial keys are generated securely, and the private keys are stored securely in the TEE. 

\begin{comment}
\addition{\footnote{\addition{This assumption implies that the executing code within the trusted TEE cannot deviate from the protocol specification and thus a process cannot generate and send conflicting statements for the same protocol round to different replicas. Other than that the unprotected area can equivocate in any possible way, i.e., by replaying, re-ordering and compromising in any way the network traffic and the unprotected memory.}}}
\end{comment}


\if 0
\pramod{can be omitted for space issue!}

\myparagraph{Limitations} \projecttitle{} runs the protocol code in a distributed setting of TEEs which increases the trusted computing base (TCB) size. We assume that the TEEs' hardware and the protocols' implementation are correct; the state transitions are in compliance with the protocol specifications while Byzantine nodes/attackers cannot compromise the properties of TEEs.  We do not protect against Denial-of-Service (DoS) attacks. Nevertheless, DoS attacks are indistinguishable from a network partition and can only affect availability but not safety. \projecttitle{} protocols remain live as long as up to f are subset to DoS attacks. Lastly, our work targets RPC-based CFT strongly-consistent protocols as defined in $\S$~\ref{sec:background:CFT} and we do not explore one-sided RDMA-based protocols~\cite{10.1145/3545008.3545041, 10.1145/3127479.3128609, 10.1145/2749246.2749267}. Lastly, since RDMA is not supported in geo-distributed settings, our \projecttitle{} targets single datacenter deployments.

\fi
%In contrast to classical BFT, we do not protect against implementation errors.

%In this work, we show that TEEs can simplify the faults a replica can experience. Our insight is that TEEs can help us eliminate the Byzantine behavior by shielding the executing protocol inside the trusted area. As such, the (trusted) code cannot deviate from the protocol, the state transitions are in compliance with the protocol specifications.  Other than that, a node, i.e., all the infrastructure outside the TEE, can still exhibit Byzantine behavior. 

%\dimitra{here not sure}This means that a node may appear to not send messages to some or all other nodes during a given operation. This is defined as weak and strong non-equivocation, respectively~\cite{madsen2020}. \projecttitle{} is responsible for neither and relies on the underlying CFT protocol to handle them. This is because weak non-equivocation is indistinguishable from network partitioning, while strong is indistinguishable from a crashed node.


%This is in contrast to prior work~\cite{minBFT, hybster, DBLP:journals/corr/LiuLKA16a, A2M, 10.1145/2168836.2168866} on hybrid BFT/TEE protocols which only port a small trusted module (traditionally a trusted counter for non-equivocation) inside the TEE.  Further, \projecttitle{} assumes that the software implementation of the underlying CFT protocol is correct and, in contrast to classical BFT, does not protect against implementation errors.

\begin{comment}
\section{Practical Security Challenges}\label{sec:security_challenges}

\emph{Can we provide a general recipe for hardening the security properties of non-BFT protocols offering both strong security (integrity, confidentiality and freshness) and performance?}

First we present the practical challenges that derive from our thread model.

\subsection{Untrusted network infrastructure}
\myparagraph{Problem} Under our extreme threat model, an adversary can tamper with the network traffic by re-ordering, delaying, re-sending and compromising the messages.

\myparagraph{Solution} We designed and implemented a secure network stack that leverages the advances of networking technologies. Our library extends the trust to the network infrastructure. Precisely, we ensure the integrity and confidentiality of a message and we guarantee that an operation will be executed at-most-once. The latter protects the system from adversaries that re-send old messages and, thus, force the system to re-execute stale operations.

\subsection{Secure bootstrapping and configuration}
\myparagraph{Problem} Nodes come and go. We need to establish a secure configuration and attestation process to offer fast attestation of incoming nodes.

\myparagraph{Solution} ....

\subsection{Run-time security; correct protocols' state transition}
\myparagraph{Problem} Non-BFT protocols assume that all replicas are honest. However, under our threat model this is far from truth. An adversary is able to ``change'' the execution path of the running code -- for example,  by manipulating the registers and the memory. As a result, even a correctly implemented protocol can reply incorrectly to both participant nodes and clients.

\myparagraph{Solution} With the \underline{assumption} that both the TEE and protocol implementations are correct, we can guarantee that the nodes running the protocol inside the TEE either will be honest or will not reply at all. In any case, the execution of the protocol inside the TEE guarantees that no replica will respond intentionally dishonestly. This is our key insight: non-BFT protocols running inside TEEs can tolerate Byzantine settings while guaranteeing strict security.


\subsection{Protocol correctness; equivocation and split-brain problem} 
\myparagraph{Problem} Such a privileged attacker can \emph{easily} compromise the correctness and consistency properties of the protocols since our proposed network stack cannot protect us from forking attacks. Attackers might cause a network partition and further convince the system that both partitions are active. Note that decentralized protocols, e.g. like ABD, which rely on quorum, tolerate forking attacks: at most one partition (w/ majority) would be able to execute both read and write requests. However, all well-known leader-based protocols, such as Raft, CR, CRAQ, etc. cannot directly handle forking attacks.

\underline{A motivating example is the case of Raft}. Let's assume that after a network partition, the old leader with the minority is active in one partition, while a new leader is elected after some time in the partition with the majority. In a trusted system, the old leader is able to serve read requests but not write requests (since writes would require acknowledgments by the majority). The new leader is only allowed to execute writes after the old leader becomes inactive. That is typically achieved through the use of leases -- the elected leader is always aware of the expiration time of the previous leader's lease and delays writes until it is expired. 

Unfortunately, in our untrusted world, we do not have a trusted time source. Attackers might compromise the timers' correctness in many ways (\dimitra{T-Lease}). As a result, attackers could be able to ``keep'' two leaders active in two different network partitions. In that case, both leaders are able to respond to read requests and, additionally, the leader in the partition with the majority can also execute writes. Similar scenarios are also possible to happen in other leader-based protocols like Chain Replication where the attacker can ``fork'' two tail nodes in two different network partitions.


\myparagraph{Solution} To address this major challenge, our recipe should ensure the following: \emph{(i)} nodes that are not part of the configuration/majority should deterministically become inactive and \emph{(ii)} clients cannot get two different responses for the same request. The first property \emph{(i)} safeguards the protocol from forking attacks that violate consistency. The protocol developers can adopt T-Leases that offer with a trusted lease mechanism, exclusively designed to ensure correctness in untrusted systems with our threat model. More precisely, a T-Lease ensures that even with an untrusted timer source the duration of a lease from the granters point of view is always a superset of the duration of this lease from the holder's point of view. Further, T-Lease mechanism protects against TOCTOU attacks. T-Leases have minimal overheads compared to native leases (5\%).

The second property \emph{(ii)} ensures that the attacker cannot tamper with client replies by sending to them an older, yet correct, reply to that request, e.g. an old read-request for the same key. our leader-based protocols should force the client to communicate with the request's leader/coordinator. In particular, a client sends the request to the node that believes or knew that it should be responsible for replying to their request. If this node is the leader, then the client will get a reply. Otherwise the node will send an error message to the client indicating which node is the leader (if they know). In these cases, the client should re-execute the operation.


\myparagraph{Solution} To address this major challenge, our recipe should ensure that in this untrusted environment any nodes that are not part of the configuration/majority should deterministically become inactive. This requirement safeguards the protocol from forking attacks that might violate consistency. To achieve this, protocol developers should adopt T-Leases that offer with a trusted lease mechanism, exclusively designed to ensure correctness in untrusted systems with our threat model. More precisely, a T-Lease ensures that even with an untrusted timer source the duration of a lease from the granters point of view is always a superset of the duration of this lease from the holder's point of view. Further, T-Lease mechanism protects against TOCTOU attacks. T-Leases have minimal overheads compared to native leases (5\%).


\subsection{Performance}
\myparagraph{Problem} TEEs do not directly communicate with the OS since the latter cannot be not trusted. As a result, the application threads should exit the enclave to execute a system call and these enclave transitions are costly.

\myparagraph{Solution} We overcome this limitation by adopting the recent advances in network technologies such as RDMA and DPDK. Both technologies offer kernel bypass decoupling the networking operations from the kernel.
\fi
\end{comment}