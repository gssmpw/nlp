% \if 0
% \begin{abstract}
% Distributed data stores predominantly deploy replication protocols for fault tolerance and improved read-heavy performance. More specifically, Crash Fault Tolerant (CFT) replication protocols maintain a consistent view of their state among the replicas and guarantee availability in the face of failures. As we are moving to untrusted third-party cloud environments, nodes might fail arbitrarily due to configuration errors, bugs, malicious attacks, etc. Unfortunately, CFT protocols cannot guarantee correctness in the face of these non-benign faults.

% While Byzantine Fault Tolerant (BFT) protocols can handle arbitrary failures, BFT replication is very expensive, complex and hard to understand. Our work addresses these limitations by hardening the properties of CFT protocols to tolerate Byzantine faults. Specifically, we realize synergies between recent advances in hardware (trusted computing and networking) and BFT and we ask the following question: \emph{Can we leverage (and how) the state-of-the-art trusted hardware and networking to build a transformation of a CFT protocol for Byzantine settings while offering performance?}

% To answer the question we propose \projecttitle{}, a generic approach to transform CFT protocols to tolerate Byzantine failures. \projecttitle{} leverages the advances in Trusted Execution Environments and direct network I/O to guarantee non-equivocation and  authentication in the presence of Byzantine actors offering performance and generality. Our evaluation against PBFT and Damysus (a derivative of HotStuff) shows that \projecttitle{} can increase throughput up to $24\times$ and up to $5.9\times$, respectively, while decreasing the number of replicas by $f$ (w.r.t. PBFT). Lastly, \projecttitle{} can also provide confidentiality---a property not provided by conventional BFT protocols---with a speedup of $7\times$---$13\times$ w.r.t. PBFT and up to $4.9\times$ w.r.t. Damysus.


% \end{abstract}
% \fi

\begin{abstract}

Distributed systems ubiquitously rely on Crash Fault Tolerance (CFT) replication protocols for performance when (benign) faults occur. Unfortunately, CFT protocols are {\em inadequate} for modern untrusted cloud environments, where the underlying cloud infrastructure can be compromised by an adversary and fail in arbitrary ways (Byzantine failures). %, e.g., co-located tenants or even a misbehaving cloud operator. 
% %that may eavesdrop or actively influence the replicas' behavior
% To overcome these limitations,



%we investigate the impact of
%modern hardware on the performance of strongly-consistent
%replication protocols.

%Byzantine Fault Tolerant (BFT) protocols lay the foundations for maintaining a consistent view of the distributed state among the replicas and guaranteeing availability, {\em even in the presence of malicious adversaries}. While BFT protocols can tolerate arbitrary (Byzantine) failures, they are unfortunately {\em not adopted in practice} because of their high-performance overheads, replication resource overheads, and implementation complexity. 

%Our work addresses these BFT limitations by exploring the impact of modern hardware on the security and performance of strongly consistent replication protocols. Specifically, we target the following question:  

%we realize the synergies between the recent advances in the cloud hardware (trusted computing for confidential computing and RDMA-based/kernel-bypass networking) and replication protocols \manos{Until this point, this paragraph is very vague}: 

Today’s modern hardware, with manycore servers, RDMA-capable networks and trusted execution environments, challenges the conventional wisdom on CFT
protocol design. In this paper, we explore the synergy between modern hardware and the security and performance of strongly consistent replication protocols. In other words, \emph{can we leverage (and how) modern cloud hardware to harden the security properties of a CFT protocol for Byzantine settings while achieving high performance?}

To answer the question, we propose \projecttitle{}, a generic approach to transform {\em existing} CFT protocols to tolerate Byzantine failures in untrusted cloud environments. \projecttitle{} leverages the advances in confidential computing and direct network I/O to guarantee non-equivocation and transferable authentication in the presence of Byzantine actors while offering performance and resource overheads on par with CFT protocols. Importantly, \projecttitle{}'s APIs are generic and can easily be adapted to existing codebases---we have transformed a range of leader-/leaderless-based CFT protocols enforcing different (e.g., total order/per-key) ordering semantics.  Our evaluation based on the transformation of four CFT protocols (Raft, ABD, Chain Replication, and AllConcur) against the state-of-the-art BFT protocols shows that \projecttitle{} can increase throughput up to $5.9\times$---$24\times$, while requiring fewer replicas, i.e., 2$f$+1 replicas instead of 3$f$+1 replicas to tolerate $f$ faults. Lastly, we provide a correctness analysis for the safety and liveness properties of our transformation of CFT protocols operating in Byzantine settings. 


%PBFT and Damysus (a derivative of HotStuff) shows that \projecttitle{} can increase throughput up to $24\times$ and up to $5.9\times$, respectively, while decreasing the number of replicas by $f$ (w.r.t. PBFT). Lastly, \projecttitle{} can also provide confidentiality---a property not provided by conventional BFT protocols---with a speedup of $7\times$---$13\times$ w.r.t. PBFT and up to $4.9\times$ w.r.t. Damysus.


\end{abstract}