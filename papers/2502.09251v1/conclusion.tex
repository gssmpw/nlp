
\section{Summary of the \projecttitle{} Library}
We present the \projecttitle{} library, a generic library for transforming CFT protocols to tolerate Byzantine failures without any modifications to the core of the protocols, e.g., states, message rounds, and complexity. We realize our \projecttitle{} library by leveraging the advances in modern hardware; we use trusted hardware to guarantee transferable authentication and non-equivocation for thwarting Byzantine errors. Further, we combine trusted hardware with direct network I/O~\cite{rdma, dpdk} for performance. We present an extensive evaluation of \projecttitle{} by applying it to four CFT protocols: Chain Replication, Raft, ABD, and AllConcur. We evaluate these four protocols against the state-of-the-art BFT protocol implementations and show that \projecttitle{} achieves up to $24\times$ and $5.9\times$ better throughput.

\if 0

\myparagraph{Contributions} To summarize, we make the following contributions:


\begin{itemize}[leftmargin=*]
    \item {\bf Hardware-assisted transformation of CFT protocols:} We present \projecttitle{}, a generic approach for transforming CFT protocols to tolerate Byzantine failures without any modifications to the core of the protocols.%, e.g., states, message rounds, and complexity. %We realize our approach by implementing \projecttitle{}-lib leveraging the advances in modern hardware; we use trusted hardware to guarantee transferrable authentication and non-equivocation for thwarting Byzantine errors. Further, we combine trusted hardware with direct network I/O (RDMA/DPDK~\cite{rdma, dpdk}) for performance.

    \item {\bf Generic \projecttitle{} APIs:} We propose generic \projecttitle{} APIs to transform the existing codebase of CFT protocols for Byzantine settings. Using \projecttitle{} APIs, we have successfully
transformed a range of leader-/leaderless-based CFT protocols enforcing different ordering semantics. % We further show that \projecttitle{} can offer confidentiality---a security property not provided by traditional BFT protocols.

    
    \item {\bf Formal correctness analysis:}  We provide a correctness analysis for the safety and liveness properties of our
transformation of CFT protocols operating in Byzantine settings. % We provide a formal analysis showing the correctness and 
    \item {\bf \projecttitle{} in action:} We present an extensive evaluation of \projecttitle{} by applying it to four CFT protocols: Chain Replication, Raft, ABD, and AllConcur. We evaluate these four protocols against the state-of-the-art BFT protocol implementations and show that \projecttitle{} achieves up to $24\times$ and $5.9\times$ better throughput.
    
  %  BFT-smart~\cite{bft-smart}, an optimized version of PBFT~\cite{Castro:2002} and Damysus~\cite{10.1145/3492321.3519568}, an optimized version of HotStuff~\cite{DBLP:journals/corr/abs-1803-05069}. Our evaluation shows that \projecttitle{} achieves up to $24\times$ and $5.9\times$ better throughput w.r.t. PBFT and Damysus, respectively, while improving scalability---\projecttitle{} requires $2f+1$ replicas, $f$ fewer replicas compared to PBFT ($3f+1$). We further show that \projecttitle{} can offer confidentiality---a security property not provided by traditional BFT protocols---while achieving a speedup of $7\times$---$13\times$ w.r.t. PBFT and up to $4.9\times$ w.r.t. Damysus.
\end{itemize}
\fi 

%\section*{Software Artifact Availability} \projecttitle{} will be %available as an open-source project.

% \section{Conclusion}
% We present \projecttitle{}, a generic approach for transforming CFT protocols to tolerate Byzantine failures without any modifications to the core of the protocols, e.g., states, message rounds, and complexity. We realize our approach by implementing \projecttitle{}-lib leveraging the advances in modern hardware; we use trusted hardware to guarantee transferable authentication and non-equivocation for thwarting Byzantine errors. Further, we combine trusted hardware with direct network I/O~\cite{rdma, dpdk} for performance. We present an extensive evaluation of \projecttitle{} by applying it to four CFT protocols: Chain Replication, Raft, ABD, and AllConcur. We evaluate these four protocols against the state-of-the-art BFT protocol implementations and show that \projecttitle{} achieves up to $24\times$ and $5.9\times$ better throughput.

% along with the complete experimental setup.%: \url{https://anonymous.4open.science/r/secure_replication-907F/}.

%\myparagraph{Supplementary material}  We provide a correctness analysis for the safety and liveness of our transformation of CFT protocols operating in Byzantine settings. 

\if 0
In conclusion, we make the following contributions:


\begin{itemize}[align=left,leftmargin=*,]
    \item {\bf Hardware-assisted transformation of CFT protocols:} We present \projecttitle{}, a generic approach for transforming CFT protocols to tolerate Byzantine failures without any modifications to the core of the protocols, e.g., states, message rounds, and complexity. We realize our approach by implementing \projecttitle{}-lib leveraging the advances in modern hardware; we use trusted hardware to guarantee transferable authentication and non-equivocation for thwarting Byzantine errors. Further, we combine trusted hardware with direct network I/O~\cite{rdma, dpdk} for performance.

    \item {\bf Generic \projecttitle{} APIs:} We propose generic \projecttitle{} APIs to transform the existing codebase of CFT protocols for Byzantine settings. With \projecttitle{} APIs, we have successfully transformed a range of leader-/leaderless CFT protocols enforcing different (total order/per-key) ordering semantics. 
    
    \item {\bf Confidential replication protocols:} We further show that \projecttitle{} can offer confidentiality---a security property not provided by traditional BFT protocols.
    
    \item {\bf Correctness analysis:}  We provide a correctness analysis for the safety and liveness of our transformation of CFT protocols operating in Byzantine settings. % We provide a formal analysis showing the correctness and 

    \item {\bf \projecttitle{} in action:} We present an extensive evaluation of \projecttitle{} by applying it to four CFT protocols: Chain Replication, Raft, ABD, and AllConcur. We evaluate these four protocols against the state-of-the-art BFT protocol implementations and show that \projecttitle{} achieves up to $24\times$ and $5.9\times$ better throughput.

  %  BFT-smart~\cite{bft-smart}, an optimized version of PBFT~\cite{Castro:2002} and Damysus~\cite{10.1145/3492321.3519568}, an optimized version of HotStuff~\cite{DBLP:journals/corr/abs-1803-05069}. Our evaluation shows that \projecttitle{} achieves up to $24\times$ and $5.9\times$ better throughput w.r.t. PBFT and Damysus, respectively, while improving scalability---\projecttitle{} requires $2f+1$ replicas, $f$ fewer replicas compared to PBFT ($3f+1$). We further show that \projecttitle{} can offer confidentiality---a security property not provided by traditional BFT protocols---while achieving a speedup of $7\times$---$13\times$ w.r.t. PBFT and up to $4.9\times$ w.r.t. Damysus.
\end{itemize}

\fi 
% In this work, we propose \projecttitle{}, a generic approach and library for
% transforming CFT protocols for Byzantine settings without any
% modifications to the protocols' states, message
% rounds and complexity. \projecttitle{} leverages the advances in modern hardware and uses Trusted Execution
% Environments (TEEs) to guarantee correct execution of the protocol withstanding Byzantine failures. To optimize for performance we combine TEEs with direct network I/O. We build with \projecttitle{} four CFT protocols of different protocol families to show the generality of our approach. Our evaluation shows that these four protocols compared to BFT-smart and Damysus achieve up to $24\times$ and $5.9\times$ better performance respectively, with $f$ less replicas w.r.t. BFT-smart. In addition, we show that \projecttitle{} can provide stronger properties than BFT, i.e., confidentiality, without sacrificing performance---indeed, \projecttitle{} with confidentiality achieves up to $13\times$ (w.r.t. BFT-smart) and $4.9\times$ (w.r.t. Damysus) better performance.

\section*{Acknowledgements}
This work was supported in parts by an ERC Starting Grant (ID: 101077577), the Intel Trustworthy Data Center of the Future (TDCoF), and the Chips Joint Undertaking (JU), European Union (EU) HORIZON-JU-IA, under grant agreement No. 101140087 (SMARTY).
% This project has also received funding from the CHIPS Joint Undertaken as part of the European Union's Horizon Europe research and innovation programme, SMARTY Project, grant agreement No. 101140087. 
The authors acknowledge the financial support by the Federal Ministry of Education and Research of Germany in the programme of "Souverän. Digital. Vernetzt.". Joint project 6G-life, project identification number: 16KISK002. 
%This work was also supported by the Intel Trustworthy Data Center of the Future (TDCoF) project.