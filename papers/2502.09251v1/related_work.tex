%\subsection{Related Work}
%\dimitra{Include this paper as well here Engraft: Enclave-guarded Raft on Byzantine Faulty Nodes}
%\pramod{can we frame the research gap based on the related work? Currently it only summarizes the laundry list.}
Recall our research question:

\begin{tcolorbox}[colback=white, colframe=white, boxrule=0.1mm]
\emph{Can we leverage (and how) the advances in trusted computing (TEEs) and networking (direct I/O) to harden the properties of CFT protocols to (1) target a weaker fault model (i.e., Byzantine faults) while (2) offering performance and scalability? }
\end{tcolorbox}

%\vspace{5pt}

Table~\ref{tab:recipe_vs_bft} compares the (most) related work with \projecttitle{} under those two axes: (1) their fault model, including the size of the implementation trusted computing base (TCB) and (2) performance or resource scalability including the number of active replicas, the message complexity as well as the use of novel hardware technologies (TEEs and D-IO). Resilience is the number of faulty nodes a protocol withstands (for safety and liveness). In contrast, direct-IO shows whether the protocols are implemented to work with direct network I/O, such as RDMA~\cite{rdma} or DPDK~\cite{dpdk}. 

We divide the related work that target BFT into two broad categories based on their resource scalability; the first includes classical BFT protocols that require (\emph{at least}) $3f+1$ participating replicas~\cite{DBLP:journals/corr/abs-1803-05069, bft-smart, Suri_Payer_2021, 10.1145/1658357.1658358, 6681599, all_about_eve} and the second category~\cite{10.1145/3552326.3587455, minBFT, hybster, DBLP:journals/corr/LiuLKA16a, A2M, 10.1145/2168836.2168866} refers to \emph{hybrid} BFT protocols that use trusted modules downgrading the replication degree to $2f+1$. In contrast to our \projecttitle{}, these approaches still require a working understanding of BFT; a task as challenging as it is error-prone~\cite{unsafeZyzzyva}.

\begin{table}[t]
%\tiny
\fontsize{7}{10}\selectfont 

\begin{center}
\begin{tabular}{ |c|c|c|c|c|c|c| } 
 \hline
 Protocols & Active/All Replicas & Resilience & Message complexity & TEEs/D-IO & Fault Model & TCB \\ [0.5ex] \hline \hline
 FastBFT~\cite{DBLP:journals/corr/LiuLKA16a}, CheapBFT~\cite{10.1145/2168836.2168866} & $f+1$/$2f+1$ & $0$ & $O(n), O(n^2)$ & Yes/No & Byz. & Small\\
 MinBFT~\cite{minBFT}, Hybster~\cite{hybster} & $2f+1$/$2f+1$ & $f$ &  $O(n^2)$ & Yes/No & Byz. & Small\\
 PBFT~\cite{Castro:2002}, HotStuff~\cite{DBLP:journals/corr/abs-1803-05069} & $3f+1$/$3f+1$ & $f$ & $O(n^2)$ $O(n)$ & No/No & Byz. & N/A\\
 CFT & $2f+1$/$2f+1$ & $f$ & depends on the protocol & No/Yes & Crash-stop. & N/A\\
 {\bf \projecttitle{}} & $\mathbf{2f+1}$/$\mathbf{2f+1}$ & ${\mathbf{f}}$ & depends on the protocol & \bf{Yes}/\bf{Yes} & \bf{Byz.} & \bf{Low}\\ 
 \hline
\end{tabular}
\end{center}
%\vspace{-10pt}
\caption{Replication protocols related work vs \projecttitle{}.}
\label{tab:recipe_vs_bft}
\vspace{0pt}
\end{table}


%\pramod{Systematize the related work a bit.}
%Table~\ref{tab:recipe_vs_bft} compares the (most) related work with \projecttitle{}.
%Resilience is the number of faulty nodes a protocol withstands (for safety and liveness).





%Table~\ref{tab:recipe_vs_bft} compares the (most) related work with \projecttitle{}. 
%We divide these works into two categories; the first includes efficient BFT protocols (w/ $3f+1$)~\cite{DBLP:journals/corr/abs-1803-05069, bft-smart, Suri_Payer_2021, 10.1145/1658357.1658358,6681599, all_about_eve} and the second~\cite{10.1145/3552326.3587455, minBFT, hybster, DBLP:journals/corr/LiuLKA16a, A2M, 10.1145/2168836.2168866} refers to \emph{hybrid} BFT protocols that use trusted modules to prevent equivocation (w/ $2f+1$). In contrast to our \projecttitle{}, all of these efforts require programmers to understand, transform, and optimize such complex protocols; a task as challenging as it is error-prone~\cite{unsafeZyzzyva}.

%these approaches still require a working understanding of BFT.



\myparagraph{Classical BFT protocols} In the first category, PBFT~\cite{Castro:2002} and its variations~\cite{CFF} run a three-phase protocol. Replicas broadcast messages and transit to the next phases after receiving {\emph{quorum certificates}}~\cite{Castro2002} from at least $2f+1$ distinct replicas leading to $O(n^2)$ message complexity. Zyzzyva~\cite{10.1145/1658357.1658358} offloads to the clients the responsibility to correct replicas' state in case of a Byzantine primary. However, prior work~\cite{unsafeZyzzyva} found safety concerns in the protocol.

Streamlined protocols~\cite{DBLP:journals/corr/abs-1803-05069, Chan2018PaLaAS, DBLP:journals/corr/abs-1807-04938, Chan2018PiLiAE} avoid heavy state transfers at the view-change by rotating the leader on each command at the cost of additional rounds. HotStuff~\cite{DBLP:journals/corr/abs-1803-05069} adds two extra phases to commit the latest blocks. Basil~\cite{Suri_Payer_2021} targets \emph{operability} when Byzantine nodes sabotage the execution requiring $5f+1$ replicas.


\myparagraph{Trusted hardware for hybrid BFT protocols} The second category includes {\emph{hybrid}} protocols~\cite{10.1145/3492321.3519568, minBFT, 10.1145/3552326.3587455, 10.1145/3492321.3519568, treaty, avocado, ccf} that leverage trusted hardware to optimize the performance of classical BFT at the cost of generalization and easy adoption. For example, MinBFT~\cite{minBFT} (a PBFT derivative optimized with TEEs), Damysus~\cite{10.1145/3492321.3519568} (a HotStuff derivative optimized with TEEs) and Hybster~\cite{hybster} use TEEs to decrease replication factor whereas others~\cite{levin2009trinc, 10.1145/1323293.1294280, yandamuri} utilize trusted counters and logs. Similarly, CheapBFT~\cite{10.1145/2168836.2168866} and FastBFT~\cite{DBLP:journals/corr/LiuLKA16a} build on trusted modules to use $f$+1 active replicas but transit to fallback BFT protocols in case of Byzantine failures. HotStuff-TPM~\cite{DBLP:journals/corr/abs-1803-05069} uses TPM~\cite{10.1109/SP.2010.32} at the cost of an extra phase. 

Similarly to \projecttitle{}, CCF~\cite{ccf} builds within a distributing setting of TEEs a variation of Raft consensus protocol (which operates under the CFT model). In contrast to \projecttitle{}, CCF builds a ledger, an append-only log, whereas \projecttitle{} is designed as a generic library to strengthen any CFT replication protocol that does not necessarily offer consensus.


\myparagraph{Programmable hardware for hybrid BFT protocols} Other works leverage programmable hardware, e.g. FPGAs~\cite{10.1145/2168836.2168866}, SmartCards~\cite{levin2009trinc} and switches~\cite{10.1145/3603269.3604874} to provide foundational primitives for ensuring BFT. For example, NeoBFT~\cite{10.1145/3603269.3604874} targets the BFT model for permissioned (BFT) blockchain systems~\cite{10.1145/3190508.3190538} by designing an \emph{authenticated ordered multicast} primitive in the programmable switch. To overcome the computation and scalability bottlenecks, they connect an FPGA device that serves as a cryptographic coprocessor to the switch. Compared to NeoBFT, where the switch is a single point of failure, \projecttitle{} offloads security into a distributed setting of TEEs, providing better availability guarantees. At the same time, it allows system designers to transform unmodified CFT protocols.  

Lastly, Trinc~\cite{levin2009trinc} and CheapBFT~\cite{10.1145/2168836.2168866} rely on peripherals to generate attestations for the exchanged message with extremely expensive latencies (50us---105ms)~\cite{levin2009trinc, 10.1145/2168836.2168866}. In addition, similarly to all previous hybrid protocols, they are specifically designed to optimize a particular variation of a BFT system and do not offer a generic methodology for any replication protocol. 

 

%\pramod{Also the following paragraph from the background section can be merged into the related work.}

%In the second category, MinBFT~\cite{minBFT}, Damysus~\cite{10.1145/3492321.3519568} and Hybster~\cite{hybster} use TEEs to decrease replication factor.  

%protocols~\cite{minBFT, hybster, DBLP:journals/corr/LiuLKA16a, A2M, 10.1145/2168836.2168866} made an effort to reduce the number of replicas in BFT protocols using trusted hardware. The derived protocols rely on trusted modules along with their ``traditional'' (untrusted/Byzantine) counterpart to prevent Byzantine nodes from equivocating. Unfortunately, as before, all of these efforts require programmers to understand, transform and optimize such complex protocols; a task as challenging as it is error-prone~\cite{unsafeZyzzyva}. Consequently, BFT protocols are still not widely adopted by cloud services~\cite{bft-time-is-now, 10.1145/1711475.1711494} that, instead, sacrifice security and rely on CFT protocols to handle failures~\cite{Hunt:2010, etcd, zippy, redis, rafthyperledger}.

%Seminal paper~\cite{clement2012} shows that non-equivocation and transferable authentication are the lower bound for a CFT-to-BFT transformation. Their work is not orthogonal to ours---we design \projecttitle{} to satisfy both properties. They focus on the theoretical proof of the transformation (requiring $O(n)$ message size complexity), we show \emph{how} to apply it efficiently ($O(1)$ complexity) in practice with modern hardware.

\if 0
%DIMITRA: This was Eurosys23 version
Seminal paper~\cite{clement2012} shows that the non-equivocation and transferable authentication are the lower bound for a CFT-to-BFT transformation. Their work is not orthogonal to ours---we design \projecttitle{} to satisfy both properties. While in~\cite{clement2012} messages comprise all previous messages ($O(n)$ message size complexity), \projecttitle{} keeps $O(1)$ message size complexity due to TEEs shielded protocol's execution. Lastly, ~\cite{clement2012} focuses on the theoretical proof of the transformation, we show \emph{how} to apply it in practice efficiently using modern hardware.
\fi


%To sum up, \projecttitle{} is differentiated from previous systems in two important aspects. First, we implement a generic library to \emph{seamlessly} transform CFT protocols for Byzantine settings. Prior works optimize existing BFT protocols, we harden the CFT protocols to provide performance and scalability, and minimize developers' effort---system designers can now {\emph {easily}} implement robust replication for the untrusted Byzantine cloud infrastructure. Second, while previous research focuses on algorithmic optimizations sometimes with the help of specialized hardware, we leverage two prominent and widely adopted hardware technologies; \projecttitle{} extends the use of TEEs to distributed settings with direct network I/O and ensures the two key properties, non-equivocation and transferable authentication, improving CFT protocol robustness. Our evaluation with the state-of-the-art BFT protocol, BFT-smart~\cite{bft-smart} showcases the impact of modern hardware in the context of BFT: we offer up to $23\times$ better performance. Interestingly, our approach can also offer confidentiality (which is not provided by BFT protocols) with a $7\times$---$13\times$ speedup compared to BFT and integrity for the exchanged messages while it also allows for local (trusted and linearizable) reads.


%Trustworthy systems The systems suffer from high latencies  build large TCBs~\cite{treaty, avocado}, and rely on specific TEEs~\cite{minBFT, hybster}. In addition, hybrid BFT systems that decrease the replication factor due to TEEs~\cite{minBFT,10.1145/3552326.3587455} allocate extra CPU resources to run the trusted counterparts of the system. 