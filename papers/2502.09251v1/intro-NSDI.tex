\section{Introduction}
\label{sec:introduction}

%\myparagraph{Motivation} \pramod{replication protocols} 
Replication protocols play a foundational role in designing distributed systems, such as distributed data storage systems~\cite{10.14778/3007263.3007267, 10.14778/2002938.2002939, zippydb, bankDB, NetflixDB, GEDB, HESSDB}, distributed coordination services~\cite{Hunt:2010, chubby},  distributed ledgers~\cite{baasAlibaba, baasAWS, baasAzure, baasIBM, baasOracle}, and distributed data analytics systems~\cite{NetflixDB}. For performance and fault tolerance requirements, distributed systems employ Crash Fault Tolerant (CFT) {\em replication protocols}~\cite{raft,chain-replication, 10.5555/1855807.1855818, Reed2008AST, 10.5555/800253.807732, lynch:1997, 10.1145/279227.279229, Hermes:2020} to maintain a consistent view of the datasets, guaranteeing fault tolerance, i.e., reliability and availability in the presence of failures~\cite{zippydb, dynamo, lakshman2009, redis, rocksdb, leveldb, memcached2004, bfthyperledger}.
%while improving the performance for read-heavy workloads and, more importantly,


Unfortunately, CFT protocols assume a {\em fail-stop model}, i.e., replicas are honest and can only fail by crashing~\cite{delporte}. As such, they are {\em inadequate} for modern untrusted cloud environments, where the underlying cloud infrastructure can be compromised by an adversary, e.g., co-located tenants or even a misbehaving cloud operator that may eavesdrop or actively influence the replicas' behavior.
%\manos{Citations here? Some of the follow-up work to Foreshadow was showing that such attacks are possible, for example. Also, is eavesdropping relevant? That implies read-only access and doesn't affect the behavior of the system}. 
In such an untrusted environment, the surface of faults and attacks expands beyond the CFT fail-stop model, ranging from software bugs and configuration errors to malicious attacks~\cite{Gunawi_bugs-in-the-cloud, Shinde2016, hahnel2017high}. CFT protocols are fundamentally incapable of providing consistent replication in the presence of non-benign ({\em Byzantine}) faults in untrusted cloud environments.



\subsection{The CFT Vs. BFT Conundrum} %\pramod{current s2.1} 

\myparagraph{CFT protocols}
CFT protocols assume that the infrastructure is trusted. These protocols tolerate only benign faults; replicas can fail by stopping or by omitting some steps~\cite{delporte}. As such, while having low overheads, they are not suitable for modern applications deployed in third-party untrusted cloud infrastructure~\cite{10.1145/3190508.3190538}. In this paper, we evaluate protocols that enforce either sequential consistency~\cite{1675439} or linearizability~\cite{Herlihy:1990}, also referred to as \emph{strongly-consistent} replication protocols.




We can broadly split strongly-consistent CFT protocols into two categories (see Table~\ref{tab:categories} for the taxonomy):  \emph{(i)} leader-based protocols (e.g., Raft~\cite{raft}, Chain Replication (CR)~\cite{chain-replication}), where a node, designated as a leader, drives the protocol execution and \emph{(ii)} decentralized protocols (e.g., ABD~\cite{lynch:1997},  AllConcur~\cite{Poke2016AllConcurLC}), where there is no leader and all nodes can propose and execute requests. 

We further divide them based on their ordering semantics. First, protocols with total ordering, where the protocols create a total order of all writes across all keys and apply them in that order. Second, protocols with per-key ordering semantics where the protocol enforces the total order of writes on a per-key basis. The evaluation of \projecttitle{} ($\S$~\ref{sec:eval}) relies on this taxonomy to systematically study its protocols' performance, as these two dimensions significantly impact the performance of the CFT protocols~\cite{f04eb9b864204bab958e72055062748c}. %By implementing and evaluating one protocol of each category with \projecttitle{} ($\S$¬\ref{sec:eval}) not only do we study the performance characteristics of protocols of the entire category but we further show the generality of our approach.

%These characteristics lead to a taxonomy of four categories summarized in Table~\ref{tab:categories}.

%We chose these two characteristics to categorise the protocols because they have been proven to impact performance the most~\cite{f04eb9b864204bab958e72055062748c}. Briefly, protocols that offer total order might introduce scalability bottlenecks due to coordination costs whereas leader-based protocols can also limit scalability as a single node, i.e., the leader, is the point of serialization. On the other hand, leader-less or per-key ordered protocols can support more parallelism but they might require distributed coordination, and as a result, more resources (e.g., network, locks) to serialize writes.



\begin{table}[t]
\small
%\fontsize{9}{10}\selectfont 

\fontsize{7}{10}\selectfont 
\begin{center}
\begin{tabular}{ | p{2.8cm} | p{3.6cm} | p{3.6cm} | }
\rowcolor{gray!25}
 \hline
 & \bf{Leader-based} & \bf{Leader-less} \\ \hline
     \multirow{2}{*}{\bf{Total order}}  & {\bf{Raft}}~\cite{raft}, ZAB~\cite{Reed2008AST},            & {\bf{AllConcur}}~\cite{Poke2016AllConcurLC}, \\ 
 & Multi-Paxos~\cite{10.1145/2673577} & Derecho~\cite{derecho}                         \\ \hline
 
    \multirow{2}{*}{\bf{Per-key order}}  & {\bf{CR}}~\cite{chain-replication}, CRAQ~\cite{10.5555/1855807.1855818},& {\bf{ABD}}~\cite{lynch:1997}, CP~\cite{10.1145/279227.279229},  \\ 
      & PB~\cite{primary-backup}, CHT~\cite{cht}  &  Hermes~\cite{Hermes:2020} \\
\hline
\end{tabular}
\end{center}
%\vspace{-10pt}
\caption{CFT protocols taxonomy. Using \projecttitle{}, we transform one protocol (shown in bold) of each category.} \label{tab:categories}
\vspace{-2pt}
\end{table}


\myparagraph{BFT protocols} In contrast to CFT protocols, BFT protocols assume very little about the nodes and the network; faulty nodes may behave arbitrarily while the network is unreliable. To tolerate $f$ arbitrarily faulty processes that may \emph{equivocate} (i.e., make conflicting statements for the same request to different replicas), BFT protocols add $f$ extra replicas to their system model requiring at least $3f+1$ replicas for safety. As such, BFT protocols exhibit worse scalability compared to CFT protocols (which only require at most $2f+1$ replicas).

BFT protocols are also limited in performance. They incur high message complexity ($O(f^2)$)~\cite{10.1145/2168836.2168866, Castro:2002, minBFT}, multiple protocol rounds~\cite{DBLP:journals/corr/LiuLKA16a, 10.1145/2168836.2168866, Castro:2002, DBLP:journals/corr/abs-1803-05069, yandamuri} and complex recovery ($O(f^2)$ in view-change)~\cite{DBLP:journals/corr/LiuLKA16a, Castro:2002, minBFT, 10.1145/2168836.2168866}. As an example of this, PBFT~\cite{Castro:2002}, a well-known BFT protocol, requires at least $3f+1$ nodes, executes three broadcast rounds, and incurs $O(n^2)$ message complexity. %HotStuff~\cite{DBLP:journals/corr/abs-1803-05069}, on the other hand, simplifies message complexity ($O(f)$) at the cost of extra communication steps: it requires 5 phases (\texttt{pre-prepare}, \texttt{prepare} and \texttt{commit} along with two complementary phases, \texttt{new-phase} and \texttt{decide}) at each round. 

%In addition, BFT protocols require at least $3f+1$ nodes to tolerate up to $f$ Byzantine nodes which severely limits scalability. In contrast, widely adopted CFT protocols~\cite{raft, Reed2008AST, primary-backup, engraft} comprise two rounds and $O(n)$ message complexity with $2f+1$ replicas.


Thirdly, BFT protocols are complex, introducing burdens to developers. Guerraoui et al.~\cite{10.1145/2658994} found that most protocol implementations consist of thousands of lines of (non-trivial) code, e.g., PBFT~\cite{Castro:2002} and Zyzzyva~\cite{10.1145/1658357.1658358}. Even if system designers wish to use a state-of-the-art BFT protocol, optimizing it for the specific application settings (e.g., network bandwidth,  number of clients and replicas, cryptographic libraries, etc.) is a rather complicated task.  Even trivial changes or intuitive optimizations can be extremely hard and might affect other parts of the protocol (e.g., view-change in Zyzzyva).


\subsection{Transformation Requirements} The basic requirements for transforming a CFT protocol for Byzantine environments are established in a theoretical result published by Clement et al. in PODC 2012~\cite{clement2012}. This seminal paper shows that non-equivocation and transferable authentication are necessary to go from $3f+1$ to $2f+1$ replicas for a reliable broadcast in Byzantine settings. Our work shows that not only can this lower bound be achieved in practice, but we can do so while providing high performance by leveraging modern hardware in a cloud environment. Next, we discuss how \projecttitle{} satisfies these two fundamental requirements, while $\S$~\ref{sec:motivation} elaborates on how to design \em{practical} and \em{efficient} protocols that meet these requirements.


\noindent{}{\bf{\underline{Property 1:}}} The transferable authentication property refers to the authenticity of a received message, requiring that a replica must be able to verify that the supposed sender indeed had sent the message. The authentication is transferable if the original sender can be verified even for forwarded messages. Formally, a message $m$ received by a correct process $P_j$ from $P_i$ is verifiable by any other correct process $P_k$. That is, given an authentication proof $\sigma_i$: 
\[
\forall P_k : \text{Verify}(m, \sigma_i, P_i) \Rightarrow \text{Accept}(m, P_k)
\]


\noindent{}{\bf{\underline{Property 2:}}} The non-equivocation property guarantees that replicas cannot \emph{accept} conflicting statements for the same request. That implies that \projecttitle{} must detect attacks where adversaries try to compromise the protocol by sending invalid requests or by re-sending valid but stale requests (\emph{replay attacks}). Formally, a Byzantine node $P_i$ cannot produce two different messages (conflicting statements) $m$ and $m'$ for the same operation to different correct replicas $P_j$ and $P_k$: \[
\forall P_j, P_k, \quad (P_i \xrightarrow{} P_j : m) \wedge (P_i \xrightarrow{} P_k : m') \Rightarrow m = m'
\]
 



\subsection{Rethinking CFT Protocols} %\pramod{the role of modern hardware in realizing transformation requirements. Adapt s3.1} 
While Byzantine Fault Tolerant (BFT) protocols~\cite{Lamport:1982} offer important foundations for developing distributed systems with stronger guarantees in the presence of {\em Byzantine failures}, they are {\em not adopted \underline{in practice}} because of their high-performance and replication resource overheads, and implementation complexity~\cite{visigoth-eurosys}.

The “CFT vs. BFT" conundrum creates a fundamental design trade-off between the {\em efficiency of CFT protocols} for practical deployments and the {\em robustness of BFT protocols} for Byzantine settings of modern cloud environments. However, traditional BFT protocols design and evaluation has not taken into account {\em \underline{modern cloud hardware}}. 

\myparagraph{Key insights} Our work seeks to resolve this trade-off by leveraging modern cloud hardware to transform existing CFT protocols for Byzantine settings in untrusted cloud environments. Our transformation underpins the robustness and efficiency axes.
For {\em robustness}, we leverage trusted hardware available to harden the security properties of CFT protocols~\cite{intel-sgx, keystone_eurosys, amd-sev, intelTDX}. 

For {\em efficiency}, we leverage the modern networking hardware, such as RDMA/DPDK for kernel bypass, to design a highly optimized communication protocol for replicating the state across nodes in distributed settings~\cite{rdma, dpdk, erpc}, while overcoming the I/O bottlenecks in trusted computing~\cite{rkt-io}.



%In particular, we leverage trusted execution environments (TEEs) to provide two key properties for successfully transforming a CFT protocol to operate in Byzantine settings, as identified by Clement et al.~\cite{clement2012}: (a) transferable authentication, i.e., the ability to establish trust in nodes in distributed settings by designing a remote attestation protocol, and (b) non-equivocation, i.e., once the trust is established in a node via the remote attestation protocol, the node follows the CFT replication protocol faithfully, and therefore, it cannot send conflicting statements to other nodes. %to reach agreement in the presence of Byzantine faults in the asynchronous communication model.


\myparagraph{Modern hardware in the context of BFT}
Trusted execution environments (TEEs)~\cite{cryptoeprint:2016:086, arm-realm, amd-sev, riscv-multizone, intelTDX} offer a hardware-enforced isolated computing environment that guarantees the integrity and confidentiality of its code and data, remaining resistant against all software attacks even in the presence of a privileged attacker (hypervisor or OS). %Based on this promise, TEEs are now being streamlined by all major CPU manufacturers, i.e., Intel TDX/SGX~\cite{intelTDX, intel-sgx}, RISC-V Keystone~\cite{riscv-multizone}, AMD-SEV~\cite{amd-sev}, Arm Realms~\cite{arm-realm}. etc. and are adopted in cloud~\cite{azure_confidential, alibaba_confidential, googleconfidentialvm, google_amd, google_kubernetes_secure}.

In our work, we leverage trusted execution environments (TEEs) in the context of BFT by realizing the potential of TEEs in hardening the properties of CFT replication protocols in the presence of Byzantine actors (e.g., network adversaries, compromised OS/hypervisor, corrupted host memory, etc.) in the untrusted cloud.
%\projecttitle{} builds on top of \scone{}~\cite{arnautov2016scone} and Intel SGX~\cite{intel-sgx}, but our work remains generic and can be applied to other TEEs too.

%Unfortunately, system designers have spotted three core limitations that complicate the adoption of TEEs for distributed systems. First, syscall execution from a TEE is prohibitively expensive due to transits between the trusted area (enclave) and (untrusted) OS~\cite{arnautov2016scone}. Thus, achieving low-latency communication, which is at the core of any replication protocol, is challenging. Secondly, the enclave is too limited to fit modern applications' needs~\cite{speicher-fast, memcached2004}: in our case we have approximately 94 MiB available to store data. Lastly, and more importantly, TEEs are designed to offer single-node secuirity and isolation whereas distributed trust establishment among all participant nodes is not offered out of the box.

%Treaty relies on Intel SGX, a set of x86 ISA extensions for TEE [46] that offers the abstraction of an isolated memory, the enclave. Enclave pages reside in the Enclave Page Cache (EPC)—a specific memory region (94 MiB in v1, 256 MiB in v2) that is protected by an on-chip Memory Encryption Engine. For larger enclave sizes, SGX implements a, rather expensive, paging mechanism [23].

%Confidential computing frameworks leverage TEEs to secure unmodified applications. They can broadly be categorized as libOS-based systems [25], [47]–[49], and host-based systems [23], [24], [50]. All of these efforts seek to minimise the number of enclave transitions, world switches, due to their high cost (e.g., TLB flushing, security checks [46]). Treaty is built on top of Scone [23] that exposes a modified libc and combines user-level threading and asynchronous syscalls [26] to reduce the cost of syscall execution.

%\myparagraph{Trusted Execution Environments (TEEs)} We build \projecttitle{} leveraging the recent advances in hardware in the context of BFT. Specifically, we realize the potential of Trusted Execution Environments (TEEs)~\cite{intel-sgx, arm-trustzone, keystone_eurosys, amd-sev} in hardening the properties of CFT protocols. TEEs guarantee the authenticity, integrity and confidentiality of their executing code and data, remaining resistant against all software attacks even from a compromised software system stack, e.g., OS, hypervisor. 

%Unfortunately, system designers have spotted two core limitations that complicate the adoption of TEEs for distributed systems. First, syscall execution from a TEE is prohibitively expensive due to transits between the trusted area (enclave) and (untrusted) OS~\cite{arnautov2016scone}. Thus, achieving low-latency communication, which is at the core of any replication protocol, is challenging. Secondly, the enclave is too limited (128 MB) to fit modern applications' needs~\cite{speicher-fast, memcached2004}. 

% 286500, Mitchell2013, memcachedRDMA, pilaf, hbaseRDMA

High-performance distributed systems~\cite{fasst, farm} abandon the traditional kernel-based networking (sockets) to avoid syscalls' overheads~\cite{flexsc}. Instead, they adopt direct network I/O (RDMA~\cite{rdma}, DPDK~\cite{dpdk}) to map the device's address space into userspace,  bypassing the kernel stack.

We also adopt direct network I/O as it is even more well-suited to TEEs where syscall execution is extremely expensive~\cite{treaty, avocado}. We leverage eRPC~\cite{erpc}, a general-purpose and asynchronous remote procedure call (RPC) library for high-speed networking for lossy Ethernet or lossless fabrics. %eRPC provides us with a UDP stack and RDMA, DPDK and RoCE transport layers. We extended eRPC to \emph{(1)} optimize the TEE memory usage and \emph{(2)} shield the network messages against the untrusted network infrastructure. 

To sum up, we leverage TEEs and high-performant network stacks in the context of BFT to provide two key properties for successfully transforming a CFT protocol to operate in Byzantine settings, as identified by Clement et al.~\cite{clement2012}: (a) transferable authentication, i.e., the ability to establish trust in nodes in distributed settings by designing a remote attestation protocol, and (b) non-equivocation, i.e., once the trust is established in a node via the remote attestation protocol, the node follows the CFT replication protocol faithfully, and therefore, it cannot send conflicting statements to other nodes.

To realize the transformation requirements, we use modern hardware to implement the following mechanisms:

\noindent{}{\bf{Mechanism 1:}} We employ cryptographic primitives and an attestation protocol. The cryptographic primitives ensure that nodes can generate and validate authenticated messages while our attestation protocol ($\S$~\ref{sec:attestation}) ensures that only trusted replicas access the cryptographic keys and execute the protocol.

\noindent{}{\bf{Mechanism 2:}} We prevent equivocation by materializing a distributed TCB that shields the protocol's (distributed) execution as well as shielding the network communication based on an authenticated message format ($\S$~\ref{sec:normal_operation}).



\subsection{\projecttitle{} Overview} 



\myparagraph{\projecttitle{} protocol overview} We consider a message-passing system consisting of $N$ nodes running the Recipe protocol. Among the $N$ nodes, $f$ exhibit Byzantine faults, while the remaining $N-f$ nodes are well-behaved. We assume $N \geq 2f + 1$. Well-behaved nodes follow the protocol faithfully, while malicious nodes may arbitrarily deviate. We define any group of $N-f$ or more nodes as a quorum.

\projecttitle{} executes a sequence of epochs, each potentially assigned a unique leader. Each epoch consists of several phases: Transferable authentication, Initialization, Normal operation (including request processing and commitment), View change, and Recovery.

The protocol is segmented into five key components:

\begin{enumerate}
    \item \textbf{Transferable authentication:} Before joining, each node undergoes a remote attestation procedure. The Protocol Designer (PD) interacts with a Configuration and Attestation Service (CAS) to verify the integrity of the node's Trusted Execution Environment (TEE). Only attested nodes receive configuration information and cryptographic keys.
    \item \textbf{Initialization:} Attested nodes establish secure communication channels and initialize their local key-value stores. The implemented CFT protocol (e.g., Raft) relies on its existing mechanisms to elect a leader before executing clients' requests.
    \item \textbf{Normal operation:}
    \begin{itemize}
        \item \textbf{Request processing:} Clients send attested requests $[\hc \sigmac, (metadata, req\_data)]$ to the leader. The leader verifies the request, "shields" it using cryptographic primitives within its TEE, creating $(\alphaa, \kv)$, and broadcasts it to the followers as $[\hr \sigmacq, (metadata', req\_data)]$, where $metadata'$ includes the view identifier $\view$, the communication channel identifier $\cq$ and the message sequence number identifier $\cntcq$ to prevent replay attacks. Followers execute the request within their TEEs and acknowledge the leader. %Sequence numbers ($\cntcq$, $\rcntcq$) are used to ensure message ordering and prevent replay attacks.
        \item \textbf{Commitment:} A multi-phase commit protocol (dependent on the underlying CFT protocol) ensures that updates are consistently applied across all correct replicas.
    \end{itemize}
    \item \textbf{View change:} If the leader fails (detected via a trusted lease mechanism~\cite{t-lease} which is implemented as part of our secure runtime environment), a new leader is elected using the CFT protocol. Committed updates are preserved across view changes.
    \item \textbf{Recovery:} New or recovering nodes undergo the transferable authentication process and join the membership as {\emph fresh} replicas. They then synchronize their state with the existing replicas before fully participating.
\end{enumerate}

In a nutshell, \projecttitle{} guarantees safety by leveraging TEEs and cryptographic primitives to protect data and ensure that only valid operations are executed. The underlying CFT protocol and the trusted lease mechanism for failure detection provide liveness. The transferable authentication phase ensures that only authorized nodes participate in the protocol, preventing Sybil attacks~\cite{sybilAttack}.


% \projecttitle{} protocol is segmented into three key components:

% \begin{itemize}
%     \item \bf{Authentication phase}
%     \item \bf{Initialization phase}
%     \item \bf{Normal execution phase}
%     \item \bf{View-change}
% \end{itemize}
% \projecttitle{} protocol is comprised of three distinct phases: the transferable
% authentication phase, the initialization phase and the normal execution phase.
% where the transformed CFT protocol executes clients’ requests. All participant nodes run the transferable authentication phase to gain confidence on the membership and ensure that only authenticated replicas receive configurations and secrets and participate
% to the protocol. The successful completion of the attestation phase guarantees the transferable authentication property and protects against Sybil attacks~\cite{sybilAttack}. Recipe materializes this phase using a remote attestation protocol. Recovered nodes need also to be attested.


%   The nodes execute the initialization phase,
% initializing their own local KVs B.1 and their network connections (e.g., configures NIC-memory,
% network ports, etc.) and establish connections with other peers B.2 based on the configuration it
% securely received at the attestation process A.7



% \myparagraph{\projecttitle{} primitives and protocol}  \projectitle{} provides a framework for building secure, hardware-accelerated replicated state machines. \projectitle{} uses TEEs (Trusted Execution Environments) and a combination of cryptographic primitives and a custom protocol to ensure safety, integrity, and availability. Key features include:

% \begin{itemize}
%     \item \textbf{Non-equivocation and Authentication:} \projectitle{} uses \texttt{shield\_request()} and \texttt{verify\_request()} primitives, along with sequence numbers and MACs, to prevent replay attacks and ensure message integrity and authenticity. Each message includes metadata like sequence numbers, view, and sender/receiver IDs, along with a MAC over the payload and metadata.
%     \item \textbf{Attestation:} \texttt{remote\_attestation()} allows nodes to verify the trustworthiness of other nodes' TEEs and code before sharing secrets. This process involves nonce generation, DHKE, quote generation, and signature verification.
%     \item \textbf{Client Interaction:} Clients interact through a PUT/GET API, with requests forwarded to a coordinator node.
%     \item \textbf{Normal Operation:} After an initial transferable authentication phase, nodes initialize and then execute the core CFT (Consensus Fault Tolerance) protocol (e.g., Raft). The protocol involves shielded messages, sequence number verification, and a commit phase. Messages exchanged between replicas include a per-request unique tuple (view, communication channel ID, and message counter).
%     \item \textbf{View Change:} Recipe supports leader election and view changes using a trusted lease mechanism to overcome the limitations of untrusted timers.
%     \item \textbf{Transferable Authentication:} Before normal operation, nodes undergo a transferable authentication phase where they prove the genuineness of their TEEs and loaded code to a protocol designer (challenger) via a Configuration and Attestation Service (CAS).
%     \item \textbf{Recovery:} New or recovered nodes are attested before joining the cluster. They send a join request, are attested, receive configuration and secrets, and synchronize with the cluster state before participating fully.
% \end{itemize}

\myparagraph{Formal analysis and verification} We formally verify the safety and security properties of \projecttitle{} using the symbolic model checker Tamarin~\cite{tamarin-prover}, assuming a Dolev-Yao attacker and perfect cryptography.
For this, we model an abstract \projecttitle{} setup as a labeled transition system. We can then verify a set of temporal properties on the transition traces of this system using Tamarin's automated deduction and equational reasoning. 
This allows us to verify the safety, integrity, and freshness properties that are presented in detail in \autoref{sec:formal-verif}.

%\subsection{The \projecttitle{} Library: Implementation and Evaluation} \pramod{We present the system architecture from the current intro. Our generic transformation of the existing protocols, and the evaluation. }



\myparagraph{\projecttitle{} library} We materialise \projecttitle{} approach as a generic library, \projecttitle{}-lib ($\S$~\ref{sec:recipe-implementation}).
The \projecttitle{} library leverages TEEs along with direct I/O to resolve the tension between security and performance by building an efficient and practical transformation of unmodified CFT replication protocols for Byzantine settings. \projecttitle{} achieve this by implementing a distributed trusted computing base (TCB) that shields the replication protocol execution and {\em extends} the security properties offered by a single TEE (whose security properties are only effective in a single-node setup) to a distributed setting of TEEs. Our design is comprised of a transferable authentication phase ($\S$~\ref{sec:attestation}) for distributed trust establishment, a high-performant network stack for secure communication over the untrusted network ($\S$~\ref{subsec:networkin}) and a memory-efficient KV store ($\S$~\ref{subsec:KV}).  


%We materialise \projecttitle{} approach as a generic library, \projecttitle{}-lib ($\S$~\ref{sec:recipe-implementation}), on top of Intel SGX and the \scone{} framework~\cite{arnautov2016scone}. We carefully build \projecttitle{}'s high-performance network stack for TEEs, extending eRPC~\cite{erpc} to enable direct I/O, essentially DMA operations, within the protected TEE domain while shielding the exchanged messages to maintain authenticity and non-equivocation accross the network. Lastly, our lock-free KV store is designed to optimize for the limited trusted area in TEEs~\cite{speicher-fast, shieldstore}.% on top of folly library~\cite{folly}.



\myparagraph{\projecttitle{} evaluation} Our evaluation assesses \projecttitle{}'s generality and efficiency. Specifically, to show the generality of our approach, we apply and evaluate \projecttitle{} on real hardware with four well-known CFT protocols (from now on, an `R-' prefix stands for the transformed protocol); a decentralized (leaderless) linearizable multi-writer multi-reader protocol (ABD)~\cite{lynch:1997} (R-ABD), two leader-based protocols with linearizable reads, Raft~\cite{raft} (R-Raft) and Chain Replication (CR)~\cite{chain-replication} (R-CR), and AllConcur~\cite{Poke2016AllConcurLC} (R-AllConcur), a decentralized consensus protocol with consistent local reads. To evaluate performance, we compare \projecttitle{} protocols with two competitive BFT replication protocols, BFT-smart~\cite{bft-smart} (PBFT~\cite{Castro2002}), whose specific implementation has been adopted in industry~\cite{bftsmarthyperledger} and Damysus~\cite{10.1145/3492321.3519568} a state-of-the-art BFT replication protocol.  Our evaluation shows that \projecttitle{} achieves up to $24\times$ and $5.9\times$ better throughput w.r.t. PBFT and Damysus, respectively, while improving scalability---\projecttitle{} requires $2f+1$ replicas, $f$ fewer replicas compared to PBFT ($3f+1$). We further show that \projecttitle{} can offer confidentiality---a security property not provided by traditional BFT protocols---while achieving a speedup of $7\times$---$13\times$ w.r.t. PBFT and up to $4.9\times$ w.r.t. Damysus.

\subsection{Our Contributions} 
%\pramod{Reuse existing contribution list + formal verification.}
To summarize, we make the following contributions:

\begin{itemize}[leftmargin=*]
    \item {\bf Hardware-assisted transformation of CFT protocols:} We present \projecttitle{}, a generic approach for transforming CFT protocols to tolerate Byzantine failures without any modifications to the core of the protocols.%, e.g., states, message rounds, and complexity. %We realize our approach by implementing \projecttitle{}-lib leveraging the advances in modern hardware; we use trusted hardware to guarantee transferrable authentication and non-equivocation for thwarting Byzantine errors. Further, we combine trusted hardware with direct network I/O (RDMA/DPDK~\cite{rdma, dpdk}) for performance.

        
    \item {\bf Formal analysis and verification:} We formally verify the safety and security properties of \projecttitle{} using the Tamarin symbolic model checker~\cite{tamarin-prover}.  By modeling \projecttitle{}  as a labeled transition system and assuming a Dolev-Yao attacker~\cite{dolev-yao-model} with perfect cryptography, we verify key properties like safety, integrity, and freshness through automated deduction and equational reasoning. Therefore, we provide a correctness analysis for the safety and liveness properties of our
transformation of CFT protocols operating in Byzantine settings. % We provide a formal analysis showing the correctness and 

    \item {\bf Generic \projecttitle{} APIs:} We propose generic \projecttitle{} APIs to transform the existing codebase of CFT protocols for Byzantine settings. Using \projecttitle{} APIs, we have successfully
transformed a range of leader-/leaderless-based CFT protocols enforcing different ordering semantics. % We further show that \projecttitle{} can offer confidentiality---a security property not provided by traditional BFT protocols.


    \item {\bf \projecttitle{} in action:} We present an extensive evaluation of \projecttitle{} by applying it to four CFT protocols: Chain Replication, Raft, ABD, and AllConcur. We evaluate these four protocols against the state-of-the-art BFT protocol implementations and show that \projecttitle{} achieves up to $24\times$ and $5.9\times$ better throughput.
    
  %  BFT-smart~\cite{bft-smart}, an optimized version of PBFT~\cite{Castro:2002} and Damysus~\cite{10.1145/3492321.3519568}, an optimized version of HotStuff~\cite{DBLP:journals/corr/abs-1803-05069}. Our evaluation shows that \projecttitle{} achieves up to $24\times$ and $5.9\times$ better throughput w.r.t. PBFT and Damysus, respectively, while improving scalability---\projecttitle{} requires $2f+1$ replicas, $f$ fewer replicas compared to PBFT ($3f+1$). We further show that \projecttitle{} can offer confidentiality---a security property not provided by traditional BFT protocols---while achieving a speedup of $7\times$---$13\times$ w.r.t. PBFT and up to $4.9\times$ w.r.t. Damysus.
\end{itemize}



\subsection{Software Artifact: Theory Meets Practice}
We have implemented the \projecttitle{} protocol as a generic library, which can be used to transform existing CFT protocols to operate in Byzantine cloud environments.   
\projecttitle{} will be available as an open-source project along with the entire experimental evaluation setup~\cite{receipe_artifact}. Our artifact includes the following:
 \begin{itemize}
 \item The \projecttitle{} library, including the secure runtime based on Intel SGX as our base TEE~\cite{intel-sgx}.

  \item The \projecttitle{} distributed data store architecture with the \projecttitle{}  replicated key-value store.


 \item Based on \projecttitle{}, a generic transformation of four CFT protocols: ABD~\cite{lynch:1997}, Raft~\cite{raft}, Chain Replication (CR)~\cite{chain-replication}, and AllConcur~\cite{Poke2016AllConcurLC}.

\item The formal verification proofs in Tamarin symbolic model checker~\cite{tamarin-prover}.

 \end{itemize}
 
%To overcome the limitations of  CFT protocols,  While BFT protocols can tolerate Byzantine failures, {\em including malicious adversaries}, they are unfortunately {\em not adopted in practice} because of their high-performance and replication resource overheads, and implementation complexity~\cite{visigoth-eurosys}. For instance, BFT protocols require $3f+1$ replicas instead of the $2f+1$ replicas required by CFT protocols for tolerating $f$ faults. BFT protocols also require additional communication rounds (at least three phases instead of two phases in the CFT model); thus incurring high latency and reduced read/write throughput. Lastly, BFT protocols are extremely complex and hard to understand, let alone to implement correctly and optimize~\cite{unsafeZyzzyva}: even intuitive algorithmic optimizations can strongly affect other parts of the protocol~\cite{10.1145/2658994}.

%While the “CFT vs. BFT" conundrum creates a fundamental design trade-off between the {\em efficiency of CFT protocols} for practical deployments and the {\em robustness of BFT protocols} for Byzantine settings of modern cloud environments. %To strike a balance, our work targets the following research question: {\em How can we design robust and efficient replication protocols for distributed systems hosted in untrusted cloud environments?}


%bring 2.3




%\myparagraph{Artifact} \projecttitle{} is available as an open-source project.%: \url{https://anonymous.4open.science/r/secure_replication-907F/}.


%build on top of Intel SGX. Our \projecttitle{} up to $24\times$ and $5.9\times$ better throughput w.r.t. BFT-smart and Damysus.
