\section{Model and Preliminary Definitions}
\label{sec:design}

\subsection{The CFT Vs. BFT Conundrum}
\label{sec:background:CFT}




\myparagraph{CFT protocols}
CFT protocols assume that the infrastructure is trusted. These protocols tolerate only benign faults; replicas can fail by stopping or by omitting some steps~\cite{delporte}. As such, while having low overheads, they are not suitable for modern applications deployed in third-party untrusted cloud infrastructure~\cite{10.1145/3190508.3190538}. In this paper, we evaluate protocols that enforce either sequential consistency~\cite{1675439} or linearizability~\cite{Herlihy:1990}, also referred to as \emph{strongly-consistent} replication protocols.




We can broadly split strongly-consistent CFT protocols into two categories (see Table~\ref{tab:categories} for the taxonomy):  \emph{(i)} leader-based protocols (e.g., Raft~\cite{raft}, Chain Replication (CR)~\cite{chain-replication}), where a node, designated as a leader, drives the protocol execution and \emph{(ii)} decentralized protocols (e.g., ABD~\cite{lynch:1997},  AllConcur~\cite{Poke2016AllConcurLC}), where there is no leader and all nodes can propose and execute requests. 

We further divide them based on their ordering semantics. First, protocols with total ordering, where the protocols create a total order of all writes across all keys and apply them in that order. Second, protocols with per-key ordering semantics where the protocol enforces the total order of writes on a per-key basis. The evaluation of \projecttitle{} ($\S$~\ref{sec:eval}) relies on this taxonomy to systematically study its protocols' performance, as these two dimensions significantly impact the performance of the CFT protocols~\cite{f04eb9b864204bab958e72055062748c}. %By implementing and evaluating one protocol of each category with \projecttitle{} ($\S$¬\ref{sec:eval}) not only do we study the performance characteristics of protocols of the entire category but we further show the generality of our approach.

%These characteristics lead to a taxonomy of four categories summarized in Table~\ref{tab:categories}.

%We chose these two characteristics to categorise the protocols because they have been proven to impact performance the most~\cite{f04eb9b864204bab958e72055062748c}. Briefly, protocols that offer total order might introduce scalability bottlenecks due to coordination costs whereas leader-based protocols can also limit scalability as a single node, i.e., the leader, is the point of serialization. On the other hand, leader-less or per-key ordered protocols can support more parallelism but they might require distributed coordination, and as a result, more resources (e.g., network, locks) to serialize writes.



\begin{table}[t]
\small
%\fontsize{9}{10}\selectfont 

\fontsize{7}{10}\selectfont 
\begin{center}
\begin{tabular}{ | p{2.8cm} | p{3.6cm} | p{3.6cm} | }
\rowcolor{gray!25}
 \hline
 & \bf{Leader-based} & \bf{Leader-less} \\ \hline
     \multirow{2}{*}{\bf{Total order}}  & {\bf{Raft}}~\cite{raft}, ZAB~\cite{Reed2008AST},            & {\bf{AllConcur}}~\cite{Poke2016AllConcurLC}, \\ 
 & Multi-Paxos~\cite{10.1145/2673577} & Derecho~\cite{derecho}                         \\ \hline
 
    \multirow{2}{*}{\bf{Per-key order}}  & {\bf{CR}}~\cite{chain-replication}, CRAQ~\cite{10.5555/1855807.1855818},& {\bf{ABD}}~\cite{lynch:1997}, CP~\cite{10.1145/279227.279229},  \\ 
      & PB~\cite{primary-backup}, CHT~\cite{cht}  &  Hermes~\cite{Hermes:2020} \\
\hline
\end{tabular}
\end{center}
%\vspace{-10pt}
\caption{CFT protocols taxonomy. Using \projecttitle{}, we transform one protocol (shown in bold) of each category.} \label{tab:categories}
\vspace{-2pt}
\end{table}

\myparagraph{BFT protocols} 
In contrast to CFT protocols, BFT protocols assume very little about the nodes and the network; faulty nodes may behave arbitrarily while the network is unreliable. To tolerate $f$ arbitrarily faulty processes that may \emph{equivocate} (i.e., make conflicting statements for the same request to different replicas), BFT protocols add $f$ extra replicas to their system model requiring at least $3f+1$ replicas for safety. As such, BFT protocols exhibit worse scalability compared to CFT protocols (which only require at most $2f+1$ replicas).

BFT protocols are limited in performance, too. They incur high message complexity ($O(f^2)$)~\cite{10.1145/2168836.2168866, Castro:2002, minBFT}, multiple protocol rounds~\cite{DBLP:journals/corr/LiuLKA16a, 10.1145/2168836.2168866, Castro:2002, DBLP:journals/corr/abs-1803-05069, yandamuri} and complex recovery ($O(f^2)$ in view-change)~\cite{DBLP:journals/corr/LiuLKA16a, Castro:2002, minBFT, 10.1145/2168836.2168866}. As an example of this, PBFT~\cite{Castro:2002}, a well-known BFT protocol, requires at least $3f+1$ nodes, executes three broadcast rounds, and incurs $O(n^2)$ message complexity. %HotStuff~\cite{DBLP:journals/corr/abs-1803-05069}, on the other hand, simplifies message complexity ($O(f)$) at the cost of extra communication steps: it requires 5 phases (\texttt{pre-prepare}, \texttt{prepare} and \texttt{commit} along with two complementary phases, \texttt{new-phase} and \texttt{decide}) at each round. 

%In addition, BFT protocols require at least $3f+1$ nodes to tolerate up to $f$ Byzantine nodes which severely limits scalability. In contrast, widely adopted CFT protocols~\cite{raft, Reed2008AST, primary-backup, engraft} comprise two rounds and $O(n)$ message complexity with $2f+1$ replicas.


Thirdly, BFT protocols are complex, introducing burdens to developers. Guerraoui et al.~\cite{10.1145/2658994} found that most protocol implementations consist of thousands of lines of (non-trivial) code, e.g., PBFT~\cite{Castro:2002} and Zyzzyva~\cite{10.1145/1658357.1658358}. Even if system designers wish to use a state-of-the-art BFT protocol, optimizing it for the specific application settings (e.g., network bandwidth,  number of clients and replicas, cryptographic libraries, etc.) is a rather complicated task.  Even trivial changes or intuitive optimizations can be extremely hard and might affect other parts of the protocol (e.g., view-change in Zyzzyva).





\subsection{Transformation Requirements}
\label{subsec:req}

The basic requirements for transforming a CFT protocol for Byzantine environments are established in a theoretical result published by Clement et al. in PODC 2012~\cite{clement2012}. This seminal paper shows that non-equivocation and transferable authentication are necessary to go from $3f+1$ to $2f+1$ replicas for a reliable broadcast in Byzantine settings. Our work shows that not only can this lower bound be achieved in practice, we can do so while providing high performance by leveraging modern hardware in a cloud environment. Next we discuss how \projecttitle{} satisfies these two fundamental requirements, while $\S$~\ref{sec:motivation} elaborates on how to design \em{practical} and \em{efficient} protocols that meet these requirements.


\noindent{}{\bf{\underline{Property 1:}}} The transferable authentication property refers to the authenticity of a received message, requiring that a replica must be able to verify that the supposed sender indeed had sent the message. The authentication is transferable if the original sender can be verified even for forwarded messages. Formally, a message $m$ received by a correct process $P_j$ from $P_i$ is verifiable by any other correct process $P_k$. That is, given an authentication proof $\sigma_i$: 
\[
\forall P_k : \text{Verify}(m, \sigma_i, P_i) \Rightarrow \text{Accept}(m, P_k)
\]

\noindent{}{\bf{Mechanism 1:}} \projecttitle{} employs cryptographic primitives and an attestation protocol. The cryptographic primitives ensure that nodes can generate and validate authenticated messages while \projecttitle{}'s attestation protocol ($\S$~\ref{sec:attestation}) ensures that only trusted replicas access the cryptographic keys and execute the protocol.

\noindent{}{\bf{\underline{Property 2:}}} The non-equivocation property guarantees that replicas cannot \emph{accept} conflicting statements for the same request. That implies that \projecttitle{} must detect attacks where adversaries try to compromise the protocol by sending invalid requests or by re-sending valid but stale requests (\emph{replay attacks}). Formally, a Byzantine node $P_i$ cannot produce two conflicting statements $m$ and $m'$ for the same operation to different correct replicas $P_j$ and $P_k$: \[
\forall P_j, P_k, \quad (P_i \xrightarrow{} P_j : m) \wedge (P_i \xrightarrow{} P_k : m') \Rightarrow m = m'
\]

\noindent{}{\bf{Mechanism 2:}} \projecttitle{} prevents equivocation by materializing a distributed TCB that shields the protocol's (distributed) execution as well as shielding the network communication based on an authenticated message format ($\S$~\ref{sec:normal_operation}). 
