\section{Introduction}
\label{sec:introduction}
Our growing reliance on online services requires highly available, fast and trustworthy systems that remain correct even when failures occur. The vast majority of cloud services provide availability in the presence of faults by deploying Crash Fault Tolerant (CFT)  replication protocols. CFT protocols assume a fail-stop fault
model; the replicas are honest and can only fail by crashing~\cite{delporte}. However, the wide adoption of online services in the third-party untrusted cloud infrastructure has drastically increased the surface of faults and attacks that range from software bugs and configuration errors to malicious attacks~\cite{Gunawi_bugs-in-the-cloud, Shinde2016,hahnel2017high}. Unfortunately, CFT protocols are fundamentally incapable of providing consistent replication in the presence of non-benign faults.
%are insufficient to
%meet all the security properties required by a pplications hosted in the cloud.

\if 0
Indeed, recent research~\cite{Gunawi_bugs-in-the-cloud, Shinde2016,hahnel2017high} shows that malicious attackers are now able to get advantage of software bugs, configuration errors and security vulnerabilities and tamper with the properties and semantics of modern online applications (e.g., databases, KV stores, financial and banking applications, blockchain, etc.). 
All these arbitrary behaviors are modeled under the notion of the Byzantine Fault Model. 
\fi

On the other hand, Byzantine Fault Tolerance (BFT)~\cite{Lamport:1982} offers important foundations for developing protocols with stronger guarantees in the presence of Byzantine actors, i.e., processes that fail in arbitrary ways. %other than silently crashing. %Practical systems~\cite{bft-smart} that implement state machine replication have adopted this model to provide resiliency against arbitrary faults. 
%While BFT protocols (BFTs) target Byzantine settings where the nodes involved can behave arbitrarily and be dishonest, 
Unfortunately, traditional BFT protocols present severe limitations that impede their adoption in high-end distributed systems.
%Unfortunately, traditional BFT protocols present the following severe limitations with respect to high-end distributed systems. 
{\bf First}, BFT protocols are slow due to high message complexity (typically, $O(n^2)$) and multiple protocol rounds. {\bf Second}, they require at least $3f+1$ nodes to tolerate up to $f$ Byzantine nodes which severely limits scalability. In contrast, widely adopted CFT protocols~\cite{raft, Reed2008AST, primary-backup, engraft} comprise two rounds and $O(n)$ message complexity with $2f+1$ replicas.% \pramod{shall we contrast these costs wrt. CFT protocols?} ~\cite{Hunt:2010,etcd}, e.g., Raft

{\bf Finally}, BFT protocols are complex, introducing burdens to developers. Even if system designers wish to use a state-of-the-art BFT protocol, optimizing it for the specific application settings can be complicated. 
Guerraoui et al.~\cite{10.1145/2658994} found that most protocol implementations consist of thousands of lines of code, e.g., PBFT~\cite{Castro:2002} and Zyzzyva~\cite{10.1145/1658357.1658358}. Even trivial changes or intuitively simple modifications for optimizing for the common case, are extremely hard and might affect other parts of the protocol (e.g., view-change in Zyzzyva).

All the above in combination with the difficulty of testing distributed protocols~\cite{Chandra2007PaxosML} require great effort for designing and optimizing such protocols. Consequently, BFT protocols are still not widely adopted by cloud services~\cite{bft-time-is-now, 10.1145/1711475.1711494} that, instead, rely on CFT protocols to handle failures~\cite{Hunt:2010,etcd, zippy, 51, redis}.

%\pramod{should we make a point that because of these limitations BFT protocols are still not widely adopted when compared to CFT? The next para can have a transisition -- To overcome these limitations, previous works.}


%The only complete proof of a BFT protocol we knew of is that of PBFT and it involves 35 pages (even without using any formal language).1 This difficulty, together with the impossibility of exhaustively testing distributed protocols [8] would rather plead for never changing a protocol that was tested and proven correct, e.g., PBFT.

%on transforming current BFT protocols to CFT protocols by 
Previous works~\cite{minBFT, hybster, DBLP:journals/corr/LiuLKA16a, A2M, 10.1145/2168836.2168866} aim at optimizing BFT protocols and mostly focus on reducing the number of replicas required. They use trusted modules along with the protocol's ``traditional'' (untrusted) counterpart to prevent Byzantine processes from making conflicting statements for the same request (equivocation). As before, all of these efforts require programmers to understand, transform and optimize these complex protocols; a task that is quite challenging.%can rapidly turn into a nightmare.



%\myparagraph{{\bf Motivation}}


\if 0
%\begin{itemize}
%    \item[1.] BFT protocols are slow as they occur high message complexity (typically, $O(n^2)$) and multiple protocol rounds/phases (3PC). Further, they traditionally require at least $3f+1$ nodes for tolerating up to $f$ Byzantine nodes which greatly limits scalability.
%    \item[2.] BFT protocols do *not* target strong security properties---specifically confidentiality where only authorized nodes/clients can access the data~\cite{avocado}. In general, it would be safe to say that current BFT does not target a typical, yet realistic, security threat model~\cite{Baumann2014} where the third party cloud provider and infrastructure that hosts the application is untrusted.
%\item[3.] BFT is complex from the programmers perspective which introduces burdens to develop and optimize the protocol.
    %\item \underline{Solution}: Strong consistent CFT protocols \& TEEs
%\end{itemize}
\fi

\if 0
%Prior work~\cite{10.1145/2332432.2332490} formally defines two properties \emph{(i)} non-equivocation and \emph{(i)} transferable authentication and proves that these two properties are the lower bound for transforming a CFT protocol to a BFT protocol.


%We leverage the theory in~\cite{10.1145/2332432.2332490} and we go one step further designing, implementing and evaluating \projecttitle{}, an end-to-end library to systematically transform in practice CFTs to BFTs without further modifications on the process number, protocol rounds and messages complexity.
\fi

%In contrast to previous BFT$\rightarrow$CFT transformations, we go in the opposite direction and build a performance-efficient transformation of CFT protocols for Byzantine settings (CFT$\rightarrow$BFT) that allows designers to build protocols that tolerate Byzantine faults without having to become BFT experts.
In contrast to previous approaches that aim to minimize the number of replicas while offering BFT, we go in the opposite direction: we aim to harden the properties of CFT protocols building a performance-efficient transformation of CFT protocols for Byzantine settings (CFT$\rightarrow$BFT). Our transformation allows designers to build protocols that tolerate Byzantine faults without having to become BFT experts.


%We realize the potential of the recent advances in hardware---trusted computing and direct network I/O---and implement an efficient transformation, a ``\projecttitle{}'', of a CFT protocol for Byzantine settings. Our approach offers BFT while providing better performance than conventional BFT protocols. Note that CFT protocols require fewer nodes for availability, are well-documented and studied (both in research and production) and are easier to implement. 

We realize the potential of Trusted Execution Environments (TEEs) and implement an efficient transformation, a ``\projecttitle{}'', of a CFT protocol for Byzantine settings.  If used naively, however, TEEs incur great overheads~\cite{10.5555/1924943.1924946}. In this paper, we show how we can mitigate the performance impact of using TEEs by adopting direct network I/O. As a result, our approach offers BFT while providing {\em better performance than conventional BFT protocols} and reaping {\em all the benefits of CFT protocols} (fewer nodes, easier implementation, simplicity).
%like requiring fewer nodes for availability, being well-documented and studied, %and being easier to implement.

%Note that CFT protocols require fewer nodes for availability, are well-documented and studied (both in research and production) and are easier to implement. 


To sum up, this paper poses the following question: \emph{Can we leverage (and how) the advances in networking and trusted computing, like Trusted Execution Environments (TEEs) and direct I/O, to harden the properties of CFT protocols to (1) target a weaker fault model (i.e., Byzantine faults) while (2) offering performance and scalability?}  %However, they cannot target Byzantine settings out-of-the-box, as they assume a Fail/Crash fault model; the replicas are \emph{honest} and can only fail by crashing.
%; \emph{with TEEs we can guarantee \textbf{honesty} for the running application. As long as we have honesty, any malicious error can be ``seen'' as crash-stop behavior.} 

%\pramod{We intermix recipe as an approach and library
%6:39
%I think what can we say is the following:
%We present an approach -- recipe.
%Based on our approach, we build recipe-library which allows users to transform protocols %seamlessly.}

%Our answer to this question is positive. We support our claims by building \projecttitle{}, an end-to-end library that systematically transforms a CFT protocol~\cite{f04eb9b864204bab958e72055062748c} to handle Byzantine faults. \projecttitle{} is a generic library that prevents equivocation and guarantees authentication for the network. We leverage the state-of-the-art hardware to target both performance and BFT---two often contradictory notions; we use TEEs for trust and direct network I/O for performance.

Our answer is \projecttitle{}, an approach to transform a CFT protocol to handle Byzantine faults. We realize the \projecttitle{} approach as a generic library, \projecttitle{}-lib, that prevents equivocation and guarantees authentication for the network. Our key insight is the use of modern hardware to achieve both BFT and performance---two often contradictory goals; we use TEEs for trust and direct network I/O for performance.


%\pramod{the following pararaph can actually encapsulate all the implementation details under the "recipe libray". We build a generic library based on.., which allow...}
Our \projecttitle{}-lib, built on top of TEEs and direct network I/O, follows an established RPC-API, e.g., RDMA programming paradigm~\cite{rdma}, allowing system designers to seamlessly transform a CFT protocol for Byzantine settings. Although we make no assumptions about the TEE, we implement \projecttitle{}-lib on top of Intel SGX~\cite{intel-sgx} and \scone{}~\cite{arnautov2016scone}. In addition, we extend eRPC~\cite{erpc} (on top of DPDK~\cite{dpdk}) and we build a shielded high-performant networking library, carefully designed for TEEs, that preserves non-equivocation and authentication in the untrusted network.


To show the generality of \projecttitle{}, we apply and evaluate it on real hardware with four well-known CFT protocols (from now on `R-' prefix stands for the transformed protocol); a decentralized (leaderless) linearizable multi-writer multi-reader protocol (ABD)~\cite{lynch:1997} (R-ABD), two leader-based protocols with linearizable reads, Raft~\cite{raft} (R-Raft) and Chain Replication (CR)~\cite{chain-replication} (R-CR), and AllConcur~\cite{Poke2016AllConcurLC} (R-AllConcur), a decentralized consensus protocol with consistent local reads. %We build and evaluate these protocols with \projecttitle{} on real hardware. 

To sum up, we make the following contributions:


\begin{itemize}
    \item We present \projecttitle{}, a generic approach for transforming CFT protocols to tolerate Byzantine failures without any modifications to the protocols, e.g., states, message rounds and complexity.
    \item We realize our approach by implementing \projecttitle{}-lib leveraging the advances in modern hardware; specifically, we use Trusted Execution Environments (TEEs) to guarantee trust for the executed code and eliminate Byzantine errors. Further, we combine TEEs and direct network I/O (RDMA/DPDK~\cite{rdma, dpdk}) for performance.
    \item We provide proofs of correctness and we show the generality of \projecttitle{} by applying it to four CFT protocols.
    \item Lastly, we evaluate these four protocols against BFT-smart~\cite{bft-smart}, an optimized version of PBFT~\cite{Castro:2002} and Damysus~\cite{10.1145/3492321.3519568}, a transformed \addition{(using TEEs)} version of Hotstuff~\cite{DBLP:journals/corr/abs-1803-05069} with $2f+1$ replicas. Our evaluation shows that \projecttitle{} achieves up to $24\times$ and $5.9\times$ better throughput w.r.t. PBFT and Damysus, respectively, while improving scalability---\projecttitle{} requires $2f+1$ replicas, $f$ fewer replicas compared to PBFT ($3f+1$). We further show that \projecttitle{} can offer confidentiality---a security property not provided by traditional BFT protocols---while achieving a speedup of $7\times$---$13\times$ w.r.t. PBFT and up to $4.9\times$ w.r.t. Damysus.
\end{itemize}


\if 0
\begin{itemize}
    \item We show that a non-equivocation and integrity verification layer, jointly with TEEs, are necessary to transform a strong consistent CFT protocol to handle BFT settings. 
    \item We build \projecttitle{} which systematically transforms strong consistent CFT protocols to handle Byzantine errors without further modifications on the number of processes, protocol rounds and messages complexity. Our \projecttitle{} targets practical systems; we build three TEE-assisted primitives, i.e., non-equivocation, authentication and attestation, to harden the CFT protocols.
    \item We build \projecttitle{} leveraging the modern hardware trends on TEE and user-space networking, emphasizing on performance.
    \item We build and evaluate three widely adopted CFT protocols, R-ABD (\projecttitle{}'s derivative of ABD~\cite{lynch:1997}), R-Raft~\cite{raft}(\projecttitle{}'s derivative of Raft~\cite{raft}), and R-CR (\projecttitle{}'s derivative of CR~\cite{chain-replication}). Our evaluation of these three systems against BFT-smart~\cite{bft-smart} shows that we can achieve up to $100\times$ better throughput.
\end{itemize}

\dimitra{
Contributions:
\begin{itemize}
    \item We use modern software/hardware trends, i.e., Trusted Execution Environments, kernel by-pass networking (DPDK, RDMA, etc.) to offer \projecttitle{}, a powerful API to transform any Crash Fault Tolerant protocol to Byzantine Fault Tolerant settings.
    \item We offer a analysis of the correctness properties of \projecttitle{} (todos: check with Manos here).
    \item We offer an evaluation of four CFT protocols compared to BFT .. .
\end{itemize}
}
\fi
\if 0
\myparagraph{Key insight} While prior work has explored the synergies between TEEs and BFT protocols~\cite{A2M, minBFT} to prevent equivocation, we go one step further. Our key insight is the following: \emph{With TEEs we can guarantee \textbf{honesty} for the application running inside the enclave. As long as we have honesty, any malicious error can be ``seen'' as crash-stop behavior.}

\myparagraph{Solution} \projecttitle{} presents a recipe on how to systematically secure strong consistent CFT protocols~\cite{f04eb9b864204bab958e72055062748c} leveraging TEEs to achieve Byzantine settings tolerance. While BFT allows nodes to go out of spec, we leverage TEEs to build \projecttitle{} that disallows any node to go out of spec.

\myparagraph{Contributions}
\begin{itemize}
    \item We show that a non-equivocation and integrity verification layer, jointly with TEEs, are enough to transform a strong consistent CFT protocol to handle BFT settings. 
    \item We build \projecttitle{} which systematically transforms strong consistent CFT protocols to handle Byzantine errors without further modifications on the number of processes, protocol rounds and messages complexity. Our \projecttitle{} targets practical systems; we build three TEE-assisted primitives.
    \item We build \projecttitle{} leveraging the modern trends on Trusted execution environments and networking, emphasizing on performance.
    \item We build and evaluate three practical, widely adopted CFT protocols, ABD~\cite{}, Raft~\cite{raft}, Chain Replication~\cite{chain-replication} with \projecttitle{}. Our evaluation against PBFT-smart~\cite{bft-smart} shows that we can achieve ....
\end{itemize}
\fi