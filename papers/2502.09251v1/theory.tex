\section{\projecttitle{} Analysis and Formal Verification}
\label{sec:theory}

\subsection{Requirements Analysis}
We show how \projecttitle{} satisfies the non-equivocation and the transferable authentication properties.

\myparagraph{Non-equivocation}
\projecttitle{} prevents equivocation attacks through a trusted monotonically increasing message counter $cnt_{cq}$ that assigns \emph{sequence numbers} to the network messages. The sender assigns a monotonically increasing sequence number to every message of a given round, $cnt_{cq} \leftarrow cnt_{cq} +1$, guaranteeing a total ordering of all network messages between any two communication endpoints. Formally, for any two messages $m1,m2$ over a communication channel $cq$: $cnt_{cq}(m_{1}) < cnt_{cq}(m_2) \Longrightarrow  m_{1} \prec m_2$. 
On the receiving side, it suffices for replicas to verify that the message's counter is \emph{consistent} with their local known sequencer for this communication endpoint, $cnt_{cq} = rcnt_{cq} +1$. %For instance, if $R_i$ sends $m_{c_1}$ and $m_2$ with $cnt_{cq}(m_{c_1})=cnt_{cq}(m_2)$, $R_j$ rejects $m_2$ as invalid.
\projecttitle{}'s sequencer prevents the replays (stale but authenticated messages), which is indistinguishable from equivocation, $cnt_{cq} < rnt_{cq}$. In addition, a Byzantine node may ``appear'' to not send messages to some (weak non-equivocation) or all (strong non-equivocation) other nodes during a given operation~\cite{madsen2020}. \projecttitle{} is responsible for neither---we rely on the CFT as both weak and strong non-equivocation are indistinguishable from crash failures~\cite{madsen2020}.


%In \projecttitle{}, we guarantee non-equivocation via trusted counter assignment and verification. Each replica maintains local sequence tuples $(v, cq, seq)$ where $v$ is the current view number, $cq$ is the communication endpoint(s) between two nodes and $seq$ is the current trusted counter value in that view for the latest request sent in an endpoint. The coordinator node assigns to requests a unique tuple $(v, cq, seq)$ and then increments the $seq$ to guarantee monotonic increments and rollback/forking attacks resilience.  %the request with the correct tuple and increments the $seq$.
%Replicas execute the implemented CFT protocol for verified valid requests using the \texttt{VerifyCounter}$(<req, (v, cq, seq)>, {h_{cq}})$ primitive. Specifically, replicas verify the freshness of a message by examining its counter id. The non-equivocation layer verifies that the message's id (as part of the message metadata) is consistent with its local counter. Replicas in \projecttitle{} are willing to accept ``future'' valid~\footnote{The attestation that takes place at the TEE setup ($\S$~\ref{subsec:attestation}) ensures that only trusted nodes are capable of generating valid messages.} messages as these might come out of order, i.e., messages whose seq number is $> (seq'_{cq}+1)$ ($seq'_{cq}$ is the last seen request number from that communication channel). These messages are processed and committed only when the underlying CFT protocol allows that.


\myparagraph{Transferable authentication} 
\label{transferable-authentication}
\projecttitle{} ensures the following two core properties from its TEE-assisted primitives: property \#1: \projecttitle{} distributes the configuration, keys etc. in a secure manner to trusted nodes, and property \#2: \projecttitle{} preserves the authenticity and integrity of the network messages.% as replicas must be able to check the authenticity and integrity (or lack thereof) of the managed data, messages and requests and deny to process stale or invalid requests.

Transferable authentication is provided implicitly by properties \#1 and \#2. 
 Property \#1 ensures that their signing keys are shared for every communicating pair of processes after their successful attestation. Recall that configuration data, signing keys, and other secrets are securely provisioned only to trusted nodes that have successfully completed remote attestation. This, in turn, follows that only trusted (correct) processes can sign (and generate) valid messages, since every message $m$ is signed by the sender's TEE using a private key $sk$. It also follows that Byzantine adversaries cannot alter or forge messages without the signing key, including ``future'' messages; instead, they are only limited to replaying old messages. Formally, $Pr[Verify(\sigma_c,m,k_{pub})=1] \leq negl(\lambda)$, where $\sigma$ is the signature, $\lambda$ is the security parameter (e.g., key size), and $negl(\lambda)$ is a negligible function, meaning it decreases faster than any polynomial function. Lastly, authenticity is transferable and can be verified in the exact same way that any two directly communicating nodes do.
 
 % Formally, $P[shield\_request() = \sigma_c :$ Verify $(h_{cq}_{\sigma}_q, m] \leq \epsilon$, where $P$ is the probability, $\sigma_c$ is the signature, $m$ is the message, and $\epsilon > 0$ is a negligible number.

%is satisfied by the fact that each request is verified for its authenticity and integrity by decrypting the metadata and comparing the embedded hash. This
%Property \#2 ensures that Byzantine actors can neither alter a message's contents nor forge a new valid message by generating a verifiable hash without the signing key. Lastly, authenticity is transferable and can be verified in the exact same way that any two directly communicating nodes do.
%The combined power of these properties satisfies the transferable authentication and non-equivocation requirements.

%Firstly, for every message received, its sender is already trusted from the attestation process. There are two cases: Either the coordinator receives a message from a client or a replica receives a message from the coordinator. In both cases, the authentication layer ensures that the origin of the message is indeed the supposed sender \manos{how ?} \dimitra{thanks to attestation only trusted replicas (trusted h/w and correct software) have the encryption keys. Therefore only valid processes, processes of the membership are able to generate messages that will pass the authentication layer. Metadata (which are encrypted in the message) might contain some metadata including the src node id}. Secondly, for every message received, the integrity of the message is verified by using the embedded hash. Both mechanisms ensure that any message received cannot be forged by a faulty replica impersonating a correct one or that the message has been intercepted and altered by a third-party adversary. This way, it is guaranteed that the message is a valid state machine message from an authenticated sender. 


%\dimitra{
%\begin{itemize}
%    \item formalize properties as in PODC in the context of TEEs
%    \item Q1: how TEEs/\texttt{} ensures non-equivocation and authentication? (note: PODC establishes the requirement for a reliable broadcast, recipe does not need that)
%    \item Q2: definitions and how do we ensure safety and liveness
%\end{itemize}
%}
%We show that non-equivocation and integrity verification, jointly with TEEs, are the two bounder claims to transform a CFT protocol to handle Byzantine errors.

%\myparagraph{\textsc{Case 1}} \emph{Suppose $N$ enclaves each of which runs a process of a CFT protocol over an untrusted, asynchronous and unreliable network infrastructure.  In prior to native CFT protocols we assume that $N >= 2f +1$ where, $f$ is the maximum number of process that might fail/crash arbitrarily or their untrusted world might be compromised. We further assume that there is an non-equivocation layer. However, there is not an integrity layer, i.e., all data are stored in the untrusted world but no metadata are kept in the enclave.}

%\myparagraph{\textsc{Proof 1}}
%Proof by contradiction. Assume a leader-based CFT protocol, e.g., CR. The head node drives the protocol while the tail node is always responding to the \texttt{GET} requests. As said, all data are stored in the untrusted host memory. The head node executes the protocol for a \texttt{PUT} request $m$. All replicas in the chain execute and then forward the request $m$ to the next node. The tail receives and accepts the $m$ which is now considered committed and visible to readers. A malicious attacker overwrites the value of the request. Subsequent \texttt{GET} operations show incorrect values as the tail cannot verify the integrity property. 

%\vspace{10pt}

%\myparagraph{\textsc{Case 2}} \emph{Suppose $N$ enclaves each of which runs a process of a CFT protocol over an untrusted, asynchronous and unreliable network infrastructure. In prior to native CFT protocols we assume that $N >= 2f +1$ where, $f$ is the maximum number of process that might fail/crash arbitrarily or their untrusted world might be compromised. We further assume that there is an integrity layer, i.e., hashes that verify integrity of the application data are stored inside the trusted enclave. We assume that there is no non-equivocation layer.}

%\myparagraph{\textsc{Proof 2}}
%Proof by contradiction. Assume a simple quorum-based protocol. Process multicasts a message $m1$ for a \texttt{PUT} request to the $N$ replicas. The message, except for the payload, includes the hash of the data payload as it passes through the integrity verification layer. All replicas of the protocol receive and accept the message $m1$ and they execute the \texttt{PUT} request. In the next round a process sends $m2$, a \texttt{PUT} request for the same piece of data to the $N$ replicas. As $m1$ is completed, there is a clear \emph{happen-before} relationship between $m2 \leftarrow m1$. The message $m2$, generated inside the enclave, also includes the calculated hash of the payload. Network adversaries drop $m2$ before $k$ replicas ($k > f+1$) receive and accept it. Instead, they send $m3$ which is a copy of $m1$ to the $k$ replicas. The $k$ replicas receive and accept $m3$. Subsequent \texttt{GET} requests encounter the quorum which will decide on $m1/m3$. This violates the property $m2 \leftarrow m1$.

%\subsection{Safety properties for \projecttitle{}}
\subsection{Correctness Analysis}
%\dimitra{this section not up-to-date}
CFT protocols need to provide the following safety properties regarding the messages delivered by the network~\cite{making_distributed_app_rob, 268272}. We show how these are provided by \projecttitle{}'s non-equivocation and (transferable) authentication layers.

%DIMITRA THAT WAS BEFORE [EuroSys2023]
%\projecttitle{} needs to provide the following safety properties that are expected
%from a CFT protocol regarding the messages delivered by the %network~\cite{making_distributed_app_rob, 268272}. We show how these are provided %by \projecttitle{}'s non-equivocation and (transferable) authentication layers.



\noindent{\bf{Safety.}} If a correct process $p_i$ \emph{receives and accepts} a message $m$ from a process $p_j$, then the sender $p_j$ is correct and has executed the \texttt{send} operation with $m$.

\noindent{\bf{Integrity.}} If a correct process $p_i$ \emph{receives and accepts} a message $m$, then $m$ is a valid and correct message generated according to the protocol specifications.
%\item {\bf Freshness and reliability.} If a process $p_i$ runs the protocol for a valid message $m_{j_x}$ from the sender thread $p_j$ then it will not accept any other message $m_{j_y}$, where $y = x$. Further, a sender continuously re-sends a message until the recipient node acknowledges it.

\noindent{\bf{Freshness.}} If a correct process $p_i$ \emph{receives and accepts} a valid message $m_{j_x}$ sent from a correct process $p_j$, then it will not accept any future message $m_{j_y}$ with the same identifier, $y = x$, $\forall$$x,y \in \mathbb{N^+}$.
%\item{\bf Reliability.} A correct sender continuously re-sends a message until the recipient node acknowledges it. \manos{I think that confuses people and may not be an important property. What if a compromised NIC drops the messages? You cannot guarantee their delivery no matter the number of retries}
%\manos{What about some kind of agreement/consistency ?}
%\end{itemize}

%Next, we explain how \projecttitle{} satisfies these properties. First, \projecttitle{} provides (transferable) authentication through the integrity verification layer. 
%which resides in the TEE ensures that a process \emph{(i)} will be always able to prove the integrity property of the managed data and \emph{(ii)} can never lie to other process, either it will reply with the correct response (by using the metadata stored in the enclave) or it will not respond at all. Therefore, if the correct process accepts a message from $p_i$ then, the sender $p_i$ has indeed sent this message. \projecttitle{} ensures this property by re-sending any non-acknowledged requests and by performing integrity checks upon reception and integrity computations on transmission. 

Next, we explain how \projecttitle{} satisfies these properties. 
Safety and integrity are directly satisfied by our transferable authentication mechanisms. Firstly, every message $m$ is signed by the sender’s TEE using a private key $k_{priv}$, and receivers verify $m$ using the sender’s public key $k_{pub}$. Thus, only trusted and correct processes can generate valid messages (i.e., valid signatures) that can be successfully verified: a message \textit{m} accepted by some correct process $p_i$ must have been generated and sent by a correct process $p_j$. Moreover, correct processes cannot deviate from the protocol's specification to generate messages that do not adhere to it.  Byzantine adversaries can neither forge nor alter messages without $k_{priv}$ ($\S$ \ref{transferable-authentication}).

%In any case, when a correct process $p_i$ receives and accepts a message \textit{m} it must be a valid message generated according to the underlying protocol's specifications. 


Freshness is directly satisfied by our non-equivocation layer that by using monotonically increasing counters, imposes a total order on messages between two communication endpoints. A correct process $p_i$ drops already received messages to sustain replay equivocation attacks. 

%\dimitra{here} However, the underlying protocol specification may permit processing out-of-order messages, e.g., ABD~\cite{lynch:1997} relies on quorums and Lamport timestamps~\cite{Lamport:1982} (that are generated in the TEE in our implementation) for linearizability. 

%Lastly, while~\cite{clement2012} requires messages to include all previous messages for replicas to validate their state (by replaying the messages history), we avoid this linear (w.r.t. the number of rounds) message and computation complexity. TEEs' shielded execution of the protocol ensures valid state transitions, computation, etc. that follow the protocol's specifications. 

%Therefore, \projecttitle{} replicas do not need to replay the history for validating their state.

% DIMITRA -> this was before [Eurosys'23]
%Lastly, while~\cite{clement2012} requires each message to include all previous messages so that replicas can replay the history and validate their state, \projecttitle{} avoids this linear (w.r.t the number of rounds) message and computation complexity due to TEEs. TEEs' shielded execution of the protocol guarantees that state transitions, computation, etc. follow the protocol's specifications. Therefore, \projecttitle{} replicas do not need to replay the history for validating their state.

\subsection{Formal Verification of the \projecttitle{} Protocol}~\label{sec:formal-verif}
We formally verify the previously mentioned safety and additional security properties of \projecttitle{} using Tamarin~\cite{tamarin-prover}. Tamarin operates in the symbolic (Dolev-Yao) model~\cite{dolev-yao-model} and thus requires us to make the following assumptions: (1) we do not consider individual bits, but instead atomic terms, like a counter, cryptographic key, etc., which are composed to derive messages (2) all cryptographic functions are pure, i.e., they have no side effects, and perfect (e.g., no hash collision) (3) a potential attacker can read and delete all messages sent on the network and modify them using the functions built into Tamarin or explicitly provided in the model (e.g., no side channels). Based on these assumptions, we can model the system state as a multiset, and the possible state transitions as multiset rewriting rules, resulting in a labeled transition system. In the case of \projecttitle{}, this involved mapping of the transferable authentication, initialization, and execution phase to \textit{facts} stored in the multiset and \textit{rules} for each operation that modifies the system state. This allows us to consider an unbounded number of processes, messages, and protocol executions. We utilize the results of previous TLS verification work~\cite{tls-proof1,tls-proof2,tamarin-tls-proof}, to abstract the TLS handshake and subsequent connection as a secure channel in our model.

In order to verify the properties of this system, we annotate rules with parameterizable \textit{action facts} and use them to express temporal first-order properties on all possible \textit{traces}, i.e., transition sequences. Tamarin can then use deduction and equational reasoning to derive either a proof of correctness or a counterexample, which violates our property~\cite{tamarin-prover}. To express the temporal relation of action facts we will use \(a\at{}t_i\), to express that action fact \(a\) occurred in the trace at time point \(t_i\). The relation \(t_a \prec t_b\) specifies that \(t_a\) occurred strictly before \(t_b\) in the trace, and \(t_a \equiv t_b\) expresses that both occur at the same time, which implies that they map to the same rule execution. Using this framework, we can express the following safety properties:

We use the action facts \textit{Acc}(ept) and \textit{Send}, which map to the according process operations, as well as the action fact \textit{Tr}(usted), which marks the rule where the processes are finally attested and from which point onward they are trusted. The first property we verify, which corresponds to our safety and integrity properties, is that sent and accepted (by some process) messages always originate from a trusted (correct) process or formally in Tamarin:

\begingroup
% Workaround for more compact equations, update in case it breaks
\setlength{\abovedisplayskip}{0.0em} % above gap
\setlength{\belowdisplayskip}{0.4em} % below gap
\setlength{\jot}{0.1em} % line gap

\begin{equation}
    \begin{split}
    \forall~ p_i, m_{j_x}, t_{t_i}, t_a ~:&~ Tr(p_i) \at{} t_{t_i} \;\wedge\; Acc(p_i, m_{j_x}) \at{} t_a \;\wedge\; t_{t_i} \prec t_a \\
    \implies & \exists~ p_j,  t_{t_j}, t_s ~:~ Tr(p_j) \at{} t_t \;\wedge\; Send(p_j, m_{j_x}) \at{} t_s \;\wedge\; t_{t_j} \prec t_s \prec t_a
    \end{split}
\end{equation}
\endgroup

\noindent{We define property (2) to verify that messages are always accepted in the order they are sent. We express this in Tamarin as:}
% With the second property, we verify that messages are never lost without being noticed; in Tamarin:

% \begingroup
% % Workaround for more compact equations, update in case it breaks
% \setlength{\abovedisplayskip}{0.0em} % above gap
% \setlength{\belowdisplayskip}{0.4em} % below gap
% \setlength{\jot}{0.1em} % line gap

% \begin{equation}
%     \begin{split}
%     \forall~ p_i, p_j, m_{j_x}, t_{t_i}, t_{t_j}, t_{a_x}, t_{s_x} ~:&~ Tr(p_i) \at{} t_{t_i} \;\wedge\; Tr(p_j) \at{} t_{t_j} \;\wedge\; Acc(p_i, m_{j_x}) \at{} t_{a_x} \;\wedge\; Send(p_j, m_{j_x}) \at{} t_{s_x} \\
%     & \;\wedge\; t_{t_i},t_{t_j} \prec t_{s_x} \prec t_{a_x} \\
%     & \implies ( \forall~ m_{j_y}, t_{s_y} ~:~ Send(p_j, m_{j_y}) \at{} t_{s_y} \;\wedge\; t_{s_y} \prec t_{s_x} \\
%     & ~~~\implies \exists t_{a_y} ~:~ Acc(p_i, m_{j_y}) \at{} t_{a_y} \;\wedge\; t_{a_y} \prec t_{a_x})
%     \end{split}
% \end{equation}
% \endgroup

% \noindent{The third property completes our guarantees on the accepted message order by verifying that messages are always accepted in the order they are sent. We express this in Tamarin as:}

\begingroup
% Workaround for more compact equations, update in case it breaks
\setlength{\abovedisplayskip}{0.0em} % above gap
\setlength{\belowdisplayskip}{0.4em} % below gap
\setlength{\jot}{0.1em} % line gap

\begin{equation}
    \begin{split}
    \forall~ p_i, m_{j_x}, m_{j_y}, t_{t_i}, t_{a_x}, t_{a_y} ~:&~ Tr(p_i) \at{} t_{t_i} \;\wedge\; Acc(p_i, m_{j_x}) \at{} t_{a_x} \;\wedge\; Acc(p_i, m_{j_y}) \at{} t_{a_y} \;\wedge\; t_{t_i} \prec t_{a_x} \prec t_{a_y} \implies \\
    & \exists~ p_j, t_{s_x}, t_{s_y} ~:~ Send(p_j, m_{j_x}) \at{} t_{s_x} \;\wedge\; Send(p_j, m_{j_y}) \at{} t_{s_y} \;\wedge\; t_{s_x} \prec t_{s_y} \\
    \end{split}
\end{equation}
\endgroup

\noindent{Finally, we verify that messages are never accepted twice, which corresponds to our freshness property, expressed in Tamarin as:}

\begingroup
% Workaround for more compact equations, update in case it breaks
\setlength{\abovedisplayskip}{0.0em} % above gap
\setlength{\belowdisplayskip}{0.4em} % below gap
\setlength{\jot}{0.1em} % line gap

\begin{equation}
    \begin{split}
    \forall~ p_i, m_{j_x}, m_{j_x}, t_{t_i}, t_{a_x}, t_{a_y} ~:&~ Tr(p_i) \at{} t_{t_i} \;\wedge\; Acc(p_i, m_{j_x}) \at{} t_{a_x} \;\wedge\; Acc(p_i, m_{j_x}) \at{} t_{a_y} \\
    &\;\wedge\; t_{t_i} \prec t_{a_x}, t_{a_y} \implies t_{a_x} \equiv t_{a_y}
    \end{split}
\end{equation}
\endgroup

\noindent{All of these properties are successfully verified by Tamarin for any number of TEEs, processes, and protocol executions, even in the presence of an attacker, as outlined above. A link to the full Tamarin model and proof artifact, which consists of more than seven thousand generated lines, will be provided after the double-blind review.}  
% \julian{Proofs: \url{https://github.com/julianpritzi/recipe\_proofs}}