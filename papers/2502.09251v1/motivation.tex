\subsection{System Design Challenges}
\label{sec:motivation}
\if 0
\subsection{Related work limitations}

\myparagraph{Performance, scalability and trust to individuals} For safety, clients and replicas on BFT protocols build trust collectively; they cannot trust individual replicas, instead, they rely on quorum certificates mandating a specific number of replicas' replies (from $f+1$ to $2f+1$) to agree on an action. As an example of this, in PBFT, replicas need at least $2f+1$ matching replies from different replicas to execute a protocol's phase (to make sure that the majority has ``agreeded'' on the request) while a client needs to gather at least $f+1$ identical replies from different replicas (which implies that at least one correct replica has responded) in order to trust the result of their request. Unfortunately, we find that this is limiting in two ways: First it significantly impacts performance and scalability given that all $3f+1$ replicas need to communicate with each other (in normal operation). Secondly, we argue that it limits any opportunities for further optimizations and specifically linearizable local-to-the-node reads, that have been proven to greatly improve throughput in CFT protocols~\cite{}.
 
In \projecttitle{} we target both limitations. We maintain the CFT protocols properties with $2f+1$ in Byzantine settings. Specifically, we implement a hybrid data structure for a node's data that detects integrity and authenticity violations supporting  local (linearizable) reads when the CFT offers them. Our approach not only does it improve performance, but adoption as well; developers do not have to reason about maintaining protocols' semantics in the Byzantine world.


\myparagraph{Generality yet simplicity} Prior research on BFT protocols either uses traditional techniques (e.g., network optimizations, parallelism, etc.) to optimize for the normal case~\cite{} or re-designs derivatives of traditional BFT agreement protocols with some improved scalability using TEEs~\cite{}.  Unfortunately, in most cases these systems inherit the complexity and some of the performance limitations (e.g., message complexity) of their ancestors which complicates their adoption to real world and introduces burdens when building more specialized-to-an-application versions of these protocols. 

In contrast to this, \projecttitle{} supports a wide-range of commonly used CFT protocols~\cite{} for various real-world use cases that might vary from agreement protocols~\cite{} that are required from (permissioned) blockchain services~\cite{}, to replication protocols for storage systems~\cite{} and atomic RMW protocols for transactional systems~\cite{}. 



\myparagraph{Stricter security: the case for confidentiality} As modernized applications, e.g.~\cite{10.1093/jamia/ocx068}, seek for stricter security guarantees, some cloud providers have recently launched blockchain solutions as part of their infrastructure~\cite{}. Unfortunately, cloud-hosted blockchain systems imply that someone has strong trust in the provider regarding the integrity and confidentiality of the hosted blockchain which is opposed to the de-centralized trust that motivates the use of blockchain systems~\cite{10.1145/3528535.3531516}. In addition, since, confidentiality is not offered by the BFT model~\cite{ciad}, and as a result, the agreement protocols, which are the point of serialization of the ledger in blockchain systems~\cite{}, are compelled to put too much trust in the cloud infrastructure. 

In \projecttitle{}, we exploit TEEs' confidentiality property where we easily offer confidential execution of the protocol showing that we can achieve both BFT and confidentiality without sacrificing performance compare.
\fi




%\subsection{Practical considerations of the transformation}
%\pramod{the following paragraph can be omitted.}
%Our paper builds on Clement et al.'s theoretical work~\cite{clement2012}, which proves that a transformation of any CFT protocol to a BFT protocol is feasible if two properties are guaranteed: non-equivocation and transferable authentication ($\S$~\ref{subsec:req}). Clement et al. argue that trusted hardware (i.e., TEEs) can shield against equivocation~\cite{A2M}. That work, however, merely provides a witness to the feasibility of such a transformation and it does not aim for performance or efficiency. There is still a significant gap between their theoretical findings and how to apply them in practice for real systems. The rest of this section describes what the gap is.


%\manos {this would be the perfect place to explain what that gap is. Especially since the last sentence mentions "this gap", as if we have already explained it.}
%We explain the transformation in $\S$\label{subsec:req}. % are the lower bound to decrease the degree of replication from $3f+1$ to $2f+1$ in Byzantine settings (where $f$ is the
%number of faulty replicas). 

%Our paper builds on Clement et al.~\cite{clement2012}'s work. They prove that a transformation of any CFT protocol to a BFT is feasible. They show that two properties, the non-equivocation
%and the transferable authentication, are the lower bound to decrease the degree of replication from $3f+1$ to $2f+1$ in Byzantine settings (where $f$ is the
%number of faulty replicas). We explain the transformation in $\S$\label{subsec:req}.



Our work shows how to leverage modern hardware to build efficient, robust, and easily adaptable distributed protocols by meeting the aforementioned transformation requirements.
%(see Q1---Q3 below). Motivated by the recently launched cloud-hosted blockchain systems, we also argue that confidential BFT protocols are required to satisfy modern applications' needs for confidentiality (see Q4 below).
Specifically, we address the following research questions.

\noindent{\bf{Q1: How to use TEEs efficiently?}} TEEs are not a panacea: due to their architectural limitations (limited trusted memory and slow syscalls' API)~\cite{shieldstore, treaty}, their naive adoption to build practical systems does not necessarily translate into performance improvements. For example, communication in the state-of-the-art BFT protocols~\cite{10.1145/3492321.3519568, minBFT, 10.1145/3528535.3531516}, which is at the core of any distributed protocol, primarily builds on standardized kernel interfaces (e.g., sockets) suffering from big latencies and not exploiting the full potential of the system~\cite{10.1145/3140659.3080208}. 

In \projecttitle{}, we carefully address these limitations without introducing an additional burden to developers. We build a secure Remote Procedure Call (RPC) framework on top of a direct I/O network stack for TEEs that achieves three goals. First, it boosts performance by avoiding expensive syscalls within TEEs. Secondly, it extends the transferable authentication and non-equivocation primitives across the untrusted network infrastructure, realizing the transformation in practice. Lastly, we follow an established RPC-programming paradigm that has proven to be the most effective one for building distributed protocols~\cite{Hermes:2020, farm, fasst, erpc}.%\manos{Hmm, this last one doesn't sound like a novel contribution.}


\noindent{\bf{Q2: How to use TEEs to transform and build \emph{practical} systems?}} %\manos{What is? We haven't mentioned any strategy; just that they show a translation exists. Maybe replace "is" with "adopts".}
While Clement et al. show that a translation of a CFT protocol to a BFT protocol {\em exists}, this mechanism adopts an impractical strategy when it comes to building real systems. The entire (transformed) system relies on an expensive mechanism to ensure the correct execution of the underlying CFT protocol. In each round, each replica needs to receive the history of all previous messages, verify their authentication, and replay the execution of the protocol's entire history. This way, it is ensured that non-Byzantine replicas rebuild their state correctly and also that the currently executed message is legitimate (i.e., derives from a valid execution scenario of the protocol). 


Secondly, the transformed protocol may amplify the native semantics of the original CFT protocol such as linearizable local reads. As in classical BFT protocols, clients cannot trust individuals, instead, they build collective trust by receiving $f+1$ identical replies from different replicas to ensure that at least one correct replica has responded. 

%To sum up, this translation cannot be applied out-of-the-box; \emph{(i)} it is impractical as, except for the obvious performance implications, the cpu and/or network resources are easily saturated after some protocol's runs and \emph{(ii)} it does not naturally preserve the original's protocol semantics; it still requires programmers to reason about them in the transformed protocol.

We design \projecttitle{} to work out-of-the-box to build real systems. \projecttitle{} leverages the properties offered by TEEs to shield the correct protocol execution while our network stack extends the security properties to the network. As a result, our approach does not impose any dependency on the history execution of the protocol, and the original protocol's message complexity is not affected. We also offer an authenticated, per-node, in-memory KV store to allow replicas to detect integrity and authenticity violations and to support local reads individually. Our approach improves performance, but enables easy adoption as well; developers do not have to worry about maintaining protocols' semantics in Byzantine settings.


\noindent{\bf{Q3: How to realize initialization, recovery, and failure detection?}} While the transformation remains agnostic with respect to the transformed CFT protocol in normal operation, the system designers still need to design recovery mechanisms when failures occur. Specifically, Clement et al. do not address how the system \emph{initializes} its state, \emph{detects} failures, and \emph{recovers} from them.  Different CFT protocols have different mechanisms for recovery and failure detection. Some protocols continue to operate when failures occur~\cite{lynch:1997, primary-backup} while others rely on accurate timeouts to detect non-responsive leaders and nodes~\cite{Hermes:2020, raft, chain-replication}. Unfortunately, TEEs come with neither a trusted initialization mechanism for distributed systems~\cite{ias} nor a trusted timer source~\cite{sgxtrustedtime, monotoniccounterssgx}.

\projecttitle{} builds on a secure substrate that overcomes these limitations. We build on a mechanism for collective attestation and a trusted lease mechanism~\cite{t-lease} which is a foundational primitive for \emph{trusted timeouts}, failure detectors~\cite{222603}, leader election~\cite{815321}, etc.
 


\myparagraph{Q4: Is BFT enough? The case for confidential BFT protocols} Applications that manage sensitive data (e.g., health-care applications~\cite{10.1093/jamia/ocx068}) adopt blockchain solutions for privacy. To this end, cloud-hosted blockchain solutions have been launched~\cite{baasAlibaba, baasAWS, baasAzure, baasIBM, baasOracle}. However, these cloud-hosted blockchain systems that fundamentally build on agreement protocols for serialising the ledger~\cite{rafthyperledger}, jeopardise the blockchain principles of decentralised trust~\cite{10.1145/3528535.3531516}. %Indeed, applications are compelled to put too much trust in the cloud provider regarding the integrity and confidentiality of the hosted protocol.

While BFT protocols offer an important foundation to build trustworthy systems, we argue that more and more modern applications~\cite{9484786, 6038579, Chong2007SIFEC, privacy_nlm, cryptdb, trustedDB, ciad} seek confidentiality beyond the scope of the BFT model. \projecttitle{} satisfies this need. Built on top of TEEs, \projecttitle{} transparently offers confidential execution without sacrificing performance.