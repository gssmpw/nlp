\section{Introduction}
\label{sec:introduction}
Our growing reliance on online services requires highly available, fast and trustworthy systems that remain correct even when failures occur. The vast majority of cloud services provide availability in the presence of faults by deploying Crash Fault Tolerant (CFT)  replication protocols. CFT protocols assume a fail-stop fault
model; the replicas are honest and can only fail by crashing~\cite{delporte}. However, the wide adoption of online services in the third-party untrusted cloud infrastructure has drastically increased the surface of faults and attacks that range from software bugs and configuration errors to malicious attacks~\cite{Gunawi_bugs-in-the-cloud, Shinde2016, hahnel2017high}. Unfortunately, CFT protocols are fundamentally incapable of providing consistent replication in the presence of non-benign faults.


On the other hand, Byzantine Fault Tolerance (BFT)~\cite{Lamport:1982} offers important foundations for developing protocols with stronger guarantees in the presence of Byzantine actors, i.e., processes that fail in arbitrary ways. Unfortunately, traditional BFT protocols present severe limitations that impede their adoption in modern, high-end distributed systems. 

\noindent{\bf {\#1: Performance and scalability.}} BFT protocols are slow due to high message complexity (typically, $O(n^2)$) and multiple protocol rounds while they require at least $3f+1$ nodes to tolerate up to $f$ Byzantine nodes which severely limits scalability. In contrast, widely adopted CFT protocols~\cite{raft, Reed2008AST, primary-backup, engraft} comprise two rounds and $O(n)$ message complexity with $2f+1$ replicas.


\noindent{\bf {\#2: Complexity and adoption.}} BFT protocols are complex, introducing burdens to developers. Even if system designers wish to use a state-of-the-art BFT protocol, optimizing it for the specific application settings can be complicated. Guerraoui et al.~\cite{10.1145/2658994} found that most protocol implementations consist of thousands of lines of code, e.g., PBFT~\cite{Castro:2002} and Zyzzyva~\cite{10.1145/1658357.1658358}. Even trivial changes or intuitively simple modifications for optimizing for the common case, are extremely hard and might affect other parts of the protocol (e.g., view-change in Zyzzyva).


Previous research~\cite{minBFT, hybster, DBLP:journals/corr/LiuLKA16a, A2M, 10.1145/2168836.2168866} made an effort to optimize such protocols by reducing the number of replicas required with trusted hardware. They use trusted modules along with the protocol's ``traditional'' (untrusted) counterpart to prevent Byzantine processes from making conflicting statements for the same request (equivocation). As before, all of these efforts require programmers to understand, transform and optimize such complex protocols; a task that is quite challenging. As an evidence of this, BFT protocols are still not widely adopted by cloud services~\cite{bft-time-is-now, 10.1145/1711475.1711494} that, instead, sacrifice security and rely on CFT protocols to handle failures~\cite{Hunt:2010, etcd, zippy, redis, rafthyperledger}.

%%%All the above in combination with the difficulty of testing distributed protocols~\cite{Chandra2007PaxosML} require great effort for designing and optimizing such protocols. Consequently, BFT protocols are still not widely adopted by cloud services~\cite{bft-time-is-now, 10.1145/1711475.1711494} that, instead, rely on CFT protocols to handle failures~\cite{Hunt:2010,etcd, zippy, 51, redis}.



%The only complete proof of a BFT protocol we knew of is that of PBFT and it involves 35 pages (even without using any formal language).1 This difficulty, together with the impossibility of exhaustively testing distributed protocols [8] would rather plead for never changing a protocol that was tested and proven correct, e.g., PBFT.

%%%%% Previous works~\cite{minBFT, hybster, DBLP:journals/corr/LiuLKA16a, A2M, 10.1145/2168836.2168866} aim at optimizing BFT protocols and mostly focus on reducing the number of replicas required. They use trusted modules along with the protocol's ``traditional'' (untrusted) counterpart to prevent Byzantine processes from making conflicting statements for the same request (equivocation). As before, all of these efforts require programmers to understand, transform and optimize these complex protocols; a task that is quite challenging.



\noindent{\bf {Key insights, contributions and high-level design.}} In contrast to previous approaches that merely aim to minimize the number of replicas while offering BFT, we go in the opposite direction targeting both limitations at once: we aim to harden the properties of CFT protocols building a performance-efficient transformation of CFT protocols for Byzantine settings (CFT$\rightarrow$BFT). In addition, we show that our transformation can \emph{transparently} offer confidentiality to applications when they need it~\cite{ciad} ($\S$~\ref{sec:motivation}), an extra property not offered by traditional BFT, without sacrificing performance. More importantly, we give system designers the opportunity to build protocols that tolerate Byzantine faults without having to become BFT experts.


%We realize the potential of the recent advances in hardware---trusted computing and direct network I/O---and implement an efficient transformation, a ``\projecttitle{}'', of a CFT protocol for Byzantine settings. Our approach offers BFT while providing better performance than conventional BFT protocols. Note that CFT protocols require fewer nodes for availability, are well-documented and studied (both in research and production) and are easier to implement. 

We realize the potential of Trusted Execution Environments (TEEs) and implement an efficient transformation, a ``\projecttitle{}'', of a CFT protocol for Byzantine settings.  If used naively, however, TEEs incur great overheads~\cite{10.5555/1924943.1924946}. In this paper, we show how we can mitigate the performance impact of using TEEs by adopting direct network I/O. As a result, our approach offers BFT while providing {\em better performance (and security, i.e., confidentiality) than conventional BFT protocols} and reaping {\em all the benefits of CFT protocols} (fewer nodes, easier implementation, simplicity).

To sum up, this paper poses the following question: \emph{Can we leverage (and how) the advances in networking and trusted computing, like Trusted Execution Environments (TEEs) and direct I/O, to harden the properties of CFT protocols to (1) target a weaker fault model (i.e., Byzantine faults) while (2) offering performance and scalability?}

Our answer is \projecttitle{}, an approach to transform a CFT protocol to handle Byzantine faults. We realize the \projecttitle{} approach as a generic library, \projecttitle{}-lib, that prevents equivocation and guarantees authentication for the network. Our key insight is the use of modern hardware to achieve both BFT and performance---two often contradictory goals; we use TEEs for trust and direct network I/O for performance.


Our \projecttitle{}-lib, built on top of TEEs and direct network I/O, follows an established RPC-API, e.g., RDMA programming paradigm~\cite{rdma}, allowing system designers to seamlessly transform a CFT protocol for Byzantine settings. We implement \projecttitle{}-lib on top of Intel SGX~\cite{intel-sgx} and \scone{}~\cite{arnautov2016scone}. In addition, we extend eRPC~\cite{erpc} (on top of DPDK~\cite{dpdk}) and we build a shielded high-performant RPC library, carefully designed for TEEs, that preserves non-equivocation and authentication in the untrusted network.


To show the generality of \projecttitle{}, we apply and evaluate it on real hardware with four well-known CFT protocols (from now on `R-' prefix stands for the transformed protocol); a decentralized (leaderless) linearizable multi-writer multi-reader protocol (ABD)~\cite{lynch:1997} (R-ABD), two leader-based protocols with linearizable reads, Raft~\cite{raft} (R-Raft) and Chain Replication (CR)~\cite{chain-replication} (R-CR), and AllConcur~\cite{Poke2016AllConcurLC} (R-AllConcur), a decentralized consensus protocol with consistent local reads. %We build and evaluate these protocols with \projecttitle{} on real hardware. 

To sum up, we make the following contributions:


\begin{itemize}
    \item We present \projecttitle{}, a generic approach for transforming CFT protocols to tolerate Byzantine failures without any modifications to the core of the protocols, e.g., states, message rounds and complexity.
    \item We realize our approach by implementing \projecttitle{}-lib leveraging the advances in modern hardware; we use Trusted Execution Environments (TEEs) to guarantee trust for the executed code and eliminate Byzantine errors. Further, we combine TEEs and direct network I/O (RDMA/DPDK~\cite{rdma, dpdk}) for performance.
    \item We provide proofs of correctness and we show the generality of \projecttitle{} by applying it to four CFT protocols.
    \item Lastly, we evaluate these four protocols against BFT-smart~\cite{bft-smart}, an optimized version of PBFT~\cite{Castro:2002} and Damysus~\cite{10.1145/3492321.3519568}, a transformed \addition{(using TEEs)} version of HotStuff~\cite{DBLP:journals/corr/abs-1803-05069} with $2f+1$ replicas. Our evaluation shows that \projecttitle{} achieves up to $24\times$ and $5.9\times$ better throughput w.r.t. PBFT and Damysus, respectively, while improving scalability---\projecttitle{} requires $2f+1$ replicas, $f$ fewer replicas compared to PBFT ($3f+1$). We further show that \projecttitle{} can offer confidentiality---a security property not provided by traditional BFT protocols---while achieving a speedup of $7\times$---$13\times$ w.r.t. PBFT and up to $4.9\times$ w.r.t. Damysus.
\end{itemize}

 \vspace{10pt}

\noindent{\bf {Limitations of \projecttitle{}.}} While \projecttitle{} remains generic, it has been build on top of Intel SGX, which has been abandoned. However,  we believe that this unfortunate event is not restrictive for \projecttitle{}'s adoption. Instead, the state-of-the-art TEEs, such as Intel TDX~\cite{intelTDX} and AMD SEV~\cite{amd-sev}, can support \projecttitle{} out-of-the-box while also facilitate the development of new transformed protocols even more as they come with trusted timer sources~\cite{intelTSC, amdTSC}.