\section{Related work limitations}
\myparagraph{Performance, scalability and trust to individuals} For safety, clients and replicas on BFT protocols build trust collectively; they cannot trust individual replicas, instead, they rely on quorum certificates mandating a specific number of replicas' replies (from $f+1$ to $2f+1$) to agree on an action. As an example of this, in PBFT, replicas need at least $2f+1$ matching replies from different replicas to execute a protocol's phase (to make sure that the majority has ``agreeded'' on the request) while a client needs to gather at least $f+1$ identical replies from different replicas (which implies that at least one correct replica has responded) in order to trust the result of their request. Unfortunately, we find that this is limiting in two ways: First it significantly impacts performance and scalability given that all $3f+1$ replicas need to communicate with each other (in normal operation). Secondly, we argue that it limits any opportunities for further optimizations and specifically linearizable local-to-the-node reads, that have been proven to greatly improve throughput in CFT protocols____.
 
In \projecttitle{} we target both limitations. We maintain the CFT protocols properties with $2f+1$ in Byzantine settings. Specifically, we implement a hybrid data structure for a node's data that detects integrity and authenticity violations supporting  local (linearizable) reads when the CFT offers them. Our approach not only does it improve performance, but adoption as well; developers do not have to reason about maintaining protocols' semantics in the Byzantine world.


\myparagraph{Generality yet simplicity} Prior research on BFT protocols either uses traditional techniques (e.g., network optimizations, parallelism, etc.) to optimize for the normal case____ or re-designs derivatives of traditional BFT agreement protocols with some improved scalability using TEEs____.  Unfortunately, in most cases these systems inherit the complexity and some of the performance limitations (e.g., message complexity) of their ancestors which complicates their adoption to real world and introduces burdens when building more specialized-to-an-application versions of these protocols. 

In contrast to this, \projecttitle{} supports a wide-range of commonly used CFT protocols____ for various real-world use cases that might vary from agreement protocols____ that are required from (permissioned) blockchain services____, to replication protocols for storage systems____ and atomic RMW protocols for transactional systems____. 



\myparagraph{Stricter security: the case for confidentiality} As modernized applications, e.g.____, seek for stricter security guarantees, some cloud providers have recently launched blockchain solutions as part of their infrastructure____. Unfortunately, cloud-hosted blockchain systems imply that someone has strong trust in the provider regarding the integrity and confidentiality of the hosted blockchain which is opposed to the de-centralized trust that motivates the use of blockchain systems____. In addition, since, confidentiality is not offered by the BFT model____, and as a result, the agreement protocols, which are the point of serialization of the ledger in blockchain systems____, are compelled to put too much trust in the cloud infrastructure. 

In \projecttitle{}, we exploit TEEs' confidentiality property where we easily offer confidential execution of the protocol showing that we can achieve both BFT and confidentiality without sacrificing performance compare.
\fi




%