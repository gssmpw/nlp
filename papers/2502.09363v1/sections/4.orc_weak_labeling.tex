\section{The Label Accuracy and Cost of ORC Weak Labeling}
Let us start with the ORC weak labeling method. This method uses a priori information about the event start and end times and is therefore not available in practice, but should be seen as an upper bound on what can be achieved with weak labeling. The start and end times of the true presence and absence events are used to construct the query segments:
\begin{equation}
\label{eq:orc}
    \sQ_{\text{ORC}} = \{(a_0, b_0), (a_1, b_1), \dots, (a_{B_{\text{ORC}}-1}, b_{B_{\text{ORC}}-1})\} = \{q_0, \dots, q_{B_{\text{ORC}}-1}\},
\end{equation}
where $(a_i, b_i)$ is the $i$th ground truth presence or absence event. The annotator indicates presence or absence for each of these segments, which by construction results in the ground truth annotations, illustrated in Figure~\ref{fig:orc_weak_labeling}. In the example, there are three target events (green), and four absence events, which means that $B_{\text{ORC}}=7$. In general $B_{\text{ORC}}\in\{2M-1, 2M+1\}$, where $M$ denotes the number of presence events. The number of absence events can be fewer than $2M+1$ if the recording starts or ends with a presence event, however, for simplicity and without losing generality, we will consider $B_{\text{ORC}}=2M+1$ as the minimum number of query segments needed for ORC to derive the ground truth. From an annotation cost perspective, this is the most cautious choice, and it is also the most likely outcome. The query accuracy is $1$ for each query segment since by construction the fraction of correctly labeled data in each query segment will be $1$ when given the correct presence or absence labels.

\begin{figure}[H]
    \centering
    \includegraphics[width=0.8\linewidth]{figures/orc_weak_labeling.png}
    \caption{ORC weak labeling of an audio recording with three target events ($M=3$) shown in green and four absence events. The $B_{\text{ORC}}=7$, query segments $q_0, \dots, q_6$ are derived from the ground truth segmentation of the data, and therefore the label accuracy will by definition be $1$.} %And in general we get on average $M$ absence segments where $M$ is the number of events.}
    \label{fig:orc_weak_labeling}
\end{figure}

In summary, the ORC weak labeling method produces annotations with label accuracy $1$, using the minimum number of query segments needed to achieve this. We use this as a reference on what can be achieved for weak labeling data.

% \begin{figure}[H]
%     \centering
%     \includegraphics[width=0.5\textwidth]{figures/orc_sufficient_queries.png}
%     \caption{The four cases with different number of absence segments (numbered) between the events (green) in an audio recording. In this example we have $3$ events, and either (a) $4$, (b) $3$, (c) $3$, or (d) $2$  absence segments. On average $3$ absence segments if assumed equally probable.} %And in general we get on average $M$ absence segments where $M$ is the number of events.}
%     \label{fig:b_orc}
% \end{figure}

%The number of query segments grow linearly with the number of target events $M$ in the given audio recording. 

%In Figure~\ref{fig:b_orc} we show an example with $M=3$ sound events without overlap. Between each sound event there is a segment with the absence of an event giving $M-1$ absence segments in total (see (d) in Figure~\ref{fig:b_orc}) then we potentially have $1$ extra absence segment at each end of the recording (see (b-d) in Figure~\ref{fig:b_orc}). If we assume these cases to be equally probable we end up with an average of $M$ absence segments. The total number of segments needed on average to query the audio recording perfectly and get a label accuracy of $1$ is therefore
%\begin{equation}
%B_{\text{ORC}} = 2M.
%\end{equation}

%This tells us how few queries we can use and still get perfect labels with ORC. 