\section{Appendix}
\label{app:appendix}

We do not include all simplifications of expressions in the proofs, but we do provide the code for a symbolic mathematics solver (SymPy) at GitHub\footnote{link will be added for camera-ready, for now see the supplementary material}, where all results can be verified. The notebook named ``symbolic\_verification\_of\_analysis.ipynb'' can be used to verify the analysis. 


%In fact, the only thing that you need to be convinced of are the entries in Table~\ref{tab:expressions} which are derived in Appendix~\ref{app:details_on_expressions}, all other results directly follow and have been verified in the notebook using the symbolic mathematics solver.


\subsection{Proof of Theorem~\ref{thm:expected_iou}}
\label{app:proof_of_theorem}


% \begin{figure}[H]
%     \centering
%     \includegraphics[width=0.8\textwidth]{figures/metric_at_all_overlap_occurences_1.pdf}
%     \caption{The label accuracy, $F(e_t, q, \gamma)$, for $t\in [t_0, t_5]$ for the two cases: (i) $d_e \geq d_q$, and (ii) $d_e \leq d_q$, where $t_0 = 0$ and $t_5 = d_e + d_q$. In appendix~\ref{app:appendix} we show how $t_1$, $t_2$, $t_3$ and $t_4$, where the discontinuities occur, can be defined as functions of $d_e$, $d_q$ and $\gamma$, and use these together with the function evaluations at the discontinuities to compute $A_1$, $A_2$ and $A_3$ for each case (i) and (ii).}
%     \label{fig:query_iou}
% \end{figure}

% \begin{claim}
% The expected query segment accuracy of an arbitrary query segment $q$ of length $d_q$ which overlaps with an event $e$ of length $d_e$ using annotator presence criterion $\gamma$ is
% \begin{equation}
% \label{eq:expected_iou_proof}
%     P(d_e, d_q, \gamma) = \E_{\rt \sim p}\left[F(e_{\rt}, q, \gamma)\right] = \begin{cases}
%     \frac{d_{e} \left(- 2 d_{e} \gamma^{2} + 2 d_{q} \gamma + d_{q}\right)}{d_{q} \left(d_{e} + d_{q}\right)}, & \text{ if } d_q \geq \gamma d_e, \\
%     \frac{d_q}{d_e + d_q}, & \text{ if } d_q < \gamma d_e.
%     \end{cases}
% \end{equation}
% \end{claim}


\label{app:thm1}

We will derive an expression for the expected query segment accuracy given overlap with a single event in terms of $d_e$, $d_q$, and $\gamma$, under all possible assumptions which will prove Theorem~\ref{thm:expected_iou}. 

\begin{proof}
We need to consider two main assumptions. The first assumption is that the presence criterion for the annotator can be fulfilled, that is, $d_q \geq \gamma d_e$, and the second assumption is that the annotator presence criterion can not be fulfilled, that is, $d_q < \gamma d_e$. This happens if the query segment length is so short that it can never cover a large enough fraction of the event of interest to make presence detection feasible.

\begin{assumption}
    The annotator presence criterion can be fulfilled ($d_q \geq \gamma d_e$).
\end{assumption}
Under this assumption there are two possible cases for the relation between $d_q$ and $d_e$, either the event length is longer or equal to the query segment length, $d_e \geq d_q$ (case i), or the event length is shorter than the query segment length, $d_e < d_q$ (case ii). In Figure~\ref{fig:query_segment_accuracy}, we plot the query segment accuracy, $F(e_t, q, \gamma)$, for $t \in [0, d_e+d_q]$ for case (i) on the left, and case (ii) on the right. 
We describe in more detail in Appendix~\ref{app:details_on_expressions} how the query segment accuracy behaves as a function of different amounts of overlap between the query segment and the event. Briefly, what we see in Figure~\ref{fig:query_segment_accuracy} is that initially there is arbitrarily little overlap ($t_0^{(i)}$ and $t_0^{(ii)}$), an absence label is given to the query segment and the accuracy is therefore $1$. Then the accuracy decrease linearly with the amount of overlap until the presence criterion is fulfilled and a presence label is given ($t_1^{(i)}$ and $t_1^{(ii)}$). After that, the accuracy linearly increase with the amount of overlap between the event and query segment until we reach a ceiling for the accuracy when either the whole query segment is inside the event ($t_2^{(i)}$) or the query segment covers the whole event ($t_2^{(ii)}$). Finally, the overlap between the query segment and the event starts to decrease again ($t_3^{(i)}$ and $t_3^{(ii)})$, and everything is symmetrical.

We continue by dropping the case superscripts show in the figure for $A_1, \dots, A_3$ and $t_0, \dots, t_5$, and only provide the full proof for case (i), but the proof for case (ii) is similar. In both cases the area $A$ in Eq.~\ref{eq:normalized_area} can be divided into five distinct parts:
\begin{equation}
\label{eq:area}
    A = 2A_1 + 2A_2 + A_3,
\end{equation}
where $A_1$ and $A_2$ are counted twice due to symmetry.

\begin{figure}[H]
    \centering
    \includegraphics[width=0.8\textwidth]{figures/metric_at_all_overlap_occurences_1.pdf}
    \caption{Assuming $d_q \geq d_e \gamma$, we plot the query segment accuracy, $F(e_t, q, \gamma)$, for $t\in [0, d_e + d_q]$, where $t_0 = 0$ and $t_5 = d_e + d_q$. Case (i) where $d_e \geq d_q$ is shown in the left panel, and case (ii) where $d_e < d_q$ is shown in the right panel.}
    \label{fig:query_segment_accuracy}
\end{figure}

The variables $t_0, t_1, \dots, t_5$, represent the different states $t$ of overlap where the discontinuities of $F(e_t, q, \gamma)$ occur, and using these we can express the areas as the following integrals:
\begin{equation}
    A_1 = \int_{t_0}^{t_1} F(e_t, q, \gamma)\mathrm{d}t = \int_{t_4}^{t_5} F(e_t, q, \gamma)\mathrm{d}t,
\end{equation}
and
\begin{equation}
    A_2 = \int_{t_1}^{t_2} F(e_t, q, \gamma)\mathrm{d}t = \int_{t_3}^{t_4} F(e_t, q, \gamma)\mathrm{d}t,
\end{equation}
due to symmetry, and
\begin{equation}
    A_3 = \int_{t_2}^{t_3} F(e_q, q, \gamma)\mathrm{d}t.
\end{equation}
% We now need to show that
% \begin{align}
%     P(d_e, d_q, \gamma) &= \E_{\rt \sim p}\left[F(e_{\rt}, q, \gamma)\right] \\
%     \label{eq:case_i}
%     &= \frac{2A^{(i)}_{1} + 2A^{(i)}_{2} + A^{(i)}_{3}}{d_e + d_q} \\
%     \label{eq:case_ii}
%     &= \frac{2A^{(ii)}_{1} + 2A^{(ii)}_{2} + A^{(ii)}_{3}}{d_e + d_q} \\
%     \label{eq:expectation_claim}
%     &= \frac{d_{e} \left(- 2 d_{e} \gamma^{2} + 2 d_{q} \gamma + d_{q}\right)}{d_{q} \left(d_{e} + d_{q}\right)}.
% \end{align}
We use that the query segment accuracy $F(e_t, q, \gamma)$ is linear in each interval, which means that the areas can be expressed as
\begin{equation}
\label{eq:a_1}
    A_1 = \frac{F(e_{t_0}, q, \gamma) + F(e_{t_{1^-}}, q, \gamma)}{2} (t_1 - t_0),
\end{equation}
\begin{equation}
\label{eq:a_2}
    A_2 = \frac{F(e_{t_{1^+}}, q, \gamma) + F(e_{t_2}, q, \gamma)}{2} (t_2 - t_1),
\end{equation}
and
\begin{equation}
\label{eq:a_3}
    A_3 = \frac{F(e_{t_2}, q, \gamma) + F(e_{t_3}, q, \gamma)}{2} (t_3 - t_2),
\end{equation}
where $t^-$ indicate that we approach the discontinuity at $t$ from below and $t^+$ from above. We now only need to express $t_0, \dots, t_3$ and $F(e_{t_0}, q, \gamma), \dots, F(e_{t_3}, q, \gamma)$ in terms of $d_e$, $d_q$ and $\gamma$ to conclude the proof. For brevity, these have been provided in Table~\ref{tab:expressions}. See section~\ref{app:details_on_expressions} for details on how to express these in terms of $d_q$, $d_e$ and $\gamma$.

\begin{table}[]
    \centering
    \begin{tabular}{l l | l l}
         \multicolumn{2}{c|}{Case (i), $d_e \geq d_q$} & \multicolumn{2}{c}{Case (ii), $d_e < d_q$} \\
         \hline
         $t^{(i)}_0 = 0$          & $F(e^{(i)}_{t_0}, q, \gamma) = 1$   & $t^{(ii)}_0 = 0$          & $F(e^{(ii)}_{t_0}, q, \gamma) = 1$ \\
         $t^{(i)}_1 = \gamma d_e$ & $F(e^{(i)}_{t_1^-}, q, \gamma) = \frac{d_q - \gamma d_e}{d_q}$ & $t^{(ii)}_1 = \gamma d_e$ & $F(e^{(ii)}_{t_1^-}, q, \gamma) = \frac{d_q - \gamma d_e}{d_q}$ \\
         $t^{(i)}_2 = d_q$        & $F(e^{(i)}_{t_1^+}, q, \gamma) = \frac{\gamma d_e}{d_q}$ & $t^{(ii)}_2 = d_e$        & $F(e^{(ii)}_{t_1^+}, q, \gamma) = \frac{\gamma d_e}{d_q}$ \\
         $t^{(i)}_3 = d_e$        & $F(e^{(i)}_{t_2}, q, \gamma) = 1$   & $t^{(ii)}_3 = d_q$        & $F(e^{(ii)}_{t_2}, q, \gamma) = \frac{d_e}{d_q}$ \\
                                  & $F(e^{(i)}_{t_3}, q, \gamma) = 1$   &                           & $F(e^{(ii)}_{t_3}, q, \gamma) = \frac{d_e}{d_q}$ \\
    \end{tabular}
    \caption{A summary of the derived expressions for $t_0, \dots, t_3$ and $F(e_{t_0}, q, \gamma), \dots, F(e_{t_3}, q, \gamma)$ for each case. $F(e_{t_1^-}, q, \gamma)$ and $F(e_{t_1^+}, q, \gamma)$ denotes the limits when approaching $t_1$ from below and above respectively.}
    \label{tab:expressions}
\end{table}

We provide the steps for case (i), and leave the derivation for case (ii) to the reader. We substitute the expressions for case (i), provided in Table~\ref{tab:expressions}, into equations Eq.~\ref{eq:a_1}-\ref{eq:a_3}, and the resulting expressions for the areas $A^{(i)}_1$, $A^{(i)}_2$, and $A^{(i)}_3$ into Eq.~\ref{eq:area} which give

% \begin{align}
% A^{(i)}_1 = \frac{1 + (d_q - \gamma d_e)/d_q}{2}(\gamma d_e),
% \end{align}

% \begin{align}
% A^{(i)}_2 = \frac{1 + \gamma d_e/d_q}{2}(d_q - \gamma d_e),
% \end{align}
% and
% \begin{align}
% A^{(i)}_3 = d_e - d_q.
% \end{align}
% Finally, $A^{(i)}_1$, $A^{(i)}_2$, and $A^{(i)}_3$ substituted into Eq.~\ref{eq:area}

\begin{align*}
A^{(i)} &= \frac{2}{2}(1+\frac{d_q-\gamma d_e}{d_q})\gamma d_e
+ \frac{2}{2}(1 + \frac{\gamma d_e}{d_q})(d_q - \gamma d_e)
+ (d_e - d_q) \\
&= (2d_q - \gamma d_e)\frac{\gamma d_e}{d_q} + (d_q + \gamma d_e)(d_q - \gamma d_e)\frac{1}{d_q} + (d_e - d_q) \\
&= \frac{1}{d_q}(2\gamma d_q d_e - \gamma^2 d_e^2 + \cancel{d_q^2} - \gamma^2 d_e^2 + d_e d_q - \cancel{d_q^2}) \\
&= \frac{1}{d_q}(2\gamma d_q d_e - 2\gamma^2 d_e^2 + d_e d_q) \\
&= \frac{d_e}{d_q}(2\gamma d_q - 2\gamma^2 d_e + d_q).
\end{align*}
Finally, by substituting $A$ for $A^{(i)}$ in Eq.~\ref{eq:normalized_area} we arrive at
\begin{equation}
    \frac{A^{(i)}}{d_e + d_q} = \frac{d_e(2\gamma d_q - 2\gamma^2 d_e + d_q)}{d_q(d_e + d_q)}
\end{equation}

which shows that Eq.~\ref{eq:expected_iou} holds for case (i) under the assumption that $d_q \geq \gamma d_e$. Similarly, this also holds for case (ii).

% \begin{equation}
%     \E_{\rt \sim p}\left[F(e_{\rt}, q, \gamma)\right] = \frac{d_{e} \left(- 2 d_{e} \gamma^{2} + 2 d_{q} \gamma + d_{q}\right)}{d_{q} \left(d_{e} + d_{q}\right)}.
% \end{equation}

\begin{assumption}
    The annotator presence criterion can not be fulfilled ($d_q < \gamma d_e$).
\end{assumption}

When the presence criterion can not be fulfilled we never get any presence labels, this means that the fraction of the query segment that overlaps with an event is always incorrectly given an absence label. When the query segment completely overlaps with an event the query segment accuracy will be $0$ (seen between $t_1$ and $t_2$ in Figure~\ref{fig:query_segment_accuracy_2}).

\begin{figure}[H]
    \centering
    \includegraphics[width=0.4\textwidth]{figures/metric_at_all_overlap_occurences_2.pdf}
    \caption{Assuming that $d_q < \gamma d_e$, we plot the query segment accuracy, $F(e_t, q, \gamma)$, for $t\in [0, d_e + d_q]$, where $t_0 = 0$ and $t_3 = d_e + d_q$.}
    \label{fig:query_segment_accuracy_2}
\end{figure}

The area $A_1$ is counted twice due to symmetry. The discontinuity at $t_1$ occurs for the smallest $t\in [0, d_e + d_q]$ for which $F(e_t, q, \gamma)=0$, which happens for the smallest $t$ for which the whole query segment overlaps with the event $|e \cap q| = d_q$ at $t=d_q$. We therefore have that $t_1 - t_0 = t_3 - t_2 = d_q$.

When there is no overlap between the query segment and the event giving a presence label is always correct, thus $F(e_{t_0}, q, \gamma) = 1$. However, giving an absence label to a query segment that completely overlaps with an event gives the query segment accuracy $0$, thus $F(e_{t_1}, q, \gamma) = 0$. The total area under the curve is therefore $2A_1 = d_q$ and by normalizing with $t_3-t_0 = d_e + d_q$, we get $d_q / (d_e + d_q)$, which proves the $d_q < \gamma d_e$ case of Eq.~\ref{eq:expected_iou}, and concludes the proof.

% \begin{equation}
%     \E_{\rt \sim p}\left[F(e_{\rt}, q, \gamma)\right] = \frac{d_q}{d_e + d_q},
% \end{equation}
% which concludes the proof.
\end{proof}

%%%%%%%%%%%%%%%%%%%%%%%%%%%%%%%%%%%%%%%%%%%%%%%%%%%%%%%%%%%%%%%%%%%%%%%%%%%%%%%%%
% \subsubsection{Details on the expressions in Table~\ref{tab:expressions}}
% \label{app:details_on_expressions}

% \textbf{TODO: this section needs a bit more polish.}

% We start by noting that $t^{(i)}_0 = t^{(ii)}_0 = 0$ and $t^{(i)}_1 = t^{(ii)}_1 = \gamma d_e$. By definition, $t^{(i)}_0 = t^{(ii)}_0 = 0$ for both cases since $t\in[0, d_e+d_q]$, when $t=0$ the end of the event is exactly aligned with the start of the query segment. Further, $F(e_{t_0}, q, \gamma)=\frac{d_e - |e_{t_0}\cap q|}{d_q} = 1$, since $|e_{t_0}\cap q|=0$ at $t_0=0$. Also, $t_1 = t^{(i)}_1 = t^{(ii)}_1 = \gamma d_e$ for both cases. The discontinuity we see at $t_1$ in Figure~\ref{fig:query_segment_accuracy} for the respective case is when $F(e_t, q, \gamma)$ goes from $\frac{d_q-|e_t\cap q|}{d_q}$ (criterion for presence not fulfilled) to $\frac{|e_t\cap q|}{d_q}$ (criterion for presence fulfilled), which occurs for the smallest $t\in[0, d_e+d_q]$ for which the annotator presence criterion is fulfilled. This happens when $h(e_t, q) = \gamma$ which by Eq.~\ref{eq:event_fraction} means that $|e_{t} \cap q| = \gamma d_e$. Therefore, $t_1=t^{(i)}_1=t^{(ii)}_1=\gamma d_e$ (see $t=t_1$ in Figure~\ref{fig:case_i} and Figure~\ref{fig:case_ii} for each case). Further,

% \begin{equation}
%     \lim_{t \to t_1^-} F(e_{t}, q, \gamma) = \frac{d_q - \gamma d_e}{d_q}
% \end{equation}

% \begin{equation}
%     \lim_{t \to t_1^+} F(e_{t}, q, \gamma) = \frac{\gamma d_e}{d_q}
% \end{equation}

% Now we look at $t_2$ and $t_3$, which differ for the two cases.

% \begin{figure}[H]
%     \centering
%     \includegraphics[width=0.8\textwidth]{figures/case_i.png}
%     \caption{An illustration of how the sound event $e_t$ and the query segment $q$ overlap at the four distinct states $t = t_0, \dots, t_3$ for case (i) where $d_e \geq d_q$.}
%     \label{fig:case_i}
% \end{figure}

% $t^{(i)}_2 = d_q$.
% The shift we see at $t^{(i)}_2$ in Figure~\ref{fig:query_segment_accuracy} occurs for the smallest $t\in[0, d_e+d_q]$ where the maximum query segment accuracy is achieved. That is, the smallest $t$ where the entire query segment overlaps with the event. This happens when the end of the sound event aligns with the end of the query segment. This happens at $t^{(i)}_2=d_q$ (see $t=t_2$ in Figure~\ref{fig:case_i}).

% $t^{(i)}_3=d_e$.
% This shift we see at $t^{(i)}_3$ in Figure~\ref{fig:query_segment_accuracy} occurs for the maximal $t\in[0, d_e+d_q$ where the maximal query segment accuracy occurs. That is, the largest $t$ where the entire query segment overlaps with the event. This happens when $t^{(i)}_3=d_e$ (see $t=t_3$ in Figure~\ref{fig:case_i}).

% Further, we get 
% \begin{align}
%     F(e_{t^{(i)}_2}, q, \gamma) &= \frac{|e_{t^{(i)}_2} \cap q|}{d_q} \\
%     &= \frac{d_q}{d_q} \\
%     &= \frac{|e_{t^{(i)}_3} \cap q|}{d_q} \\
%     &= F(e_{t^{(i)}_3}, q, \gamma)
% \end{align}
% because the whole query segment overlaps with the event in both cases.

% \begin{figure}[H]
%     \centering
%     \includegraphics[width=0.8\textwidth]{figures/case_ii.png}
%     \caption{An illustration of how the sound event $e_t$ and the query segment $q$ overlap at the four distinct states $t = t_0, \dots, t_3$ for case (ii) where $d_e < d_q$.}
%     \label{fig:case_ii}
% \end{figure}

% $t^{(ii)}_2 = d_q$.
% The shift we see at $t^{(ii)}_2$ in Figure~\ref{fig:query_segment_accuracy} occurs for the smallest $t\in[0, d_e+d_q]$ where the maximum query segment accuracy is achieved. That is, the smallest $t$ where the entire query segment overlaps with the event. That is, the beginning of the sound event aligns with the beginning of the query segment. This happens at $t^{(ii)}_2=d_e$ (see $t=t_2$ in Figure~\ref{fig:case_ii}).

% $t^{(ii)}_3=d_e$.
% This shift we see at $t^{(ii)}_3$ in Figure~\ref{fig:query_segment_accuracy} occurs for the maximal $t\in[0, d_e+d_q$ where the maximal query segment accuracy occurs. That is, the largest $t$ for which the query segment overlaps with the entire event. This happens when $t^{(ii)}_3=d_q$ (see $t=t_3$ in Figure~\ref{fig:case_ii}).

% Further, we get 
% \begin{align}
%     F(e_{t^{(ii)}_2}, q, \gamma) &= \frac{|e_{t^{(ii)}_2} \cap q|}{d_q} \\
%     &= \frac{d_e}{d_q} \\
%     &= \frac{|e_{t^{(i)}_3} \cap q|}{d_q} \\
%     &= F(e_{t^{(ii)}_3}, q, \gamma)
% \end{align}
% because the overlap is with the whole event in both cases.
%%%%%%%%%%%%%%%%%%%%%%%%%%%%%%%%%%%%%%%%%%%%%%%%%%%%%%%%%%%%%%%%%%%%%%%%%%%%%%%%%%
\subsubsection{Details on the expressions in Table~\ref{tab:expressions}}
\label{app:details_on_expressions}

This section provides a detailed explanation of the values presented in Table~\ref{tab:expressions}. For each case (i) and (ii), we will define the specific time points  $t_0, t_1, t_2, t_3$ where the query segment accuracy function $F(e_t, q, \gamma)$ changes, and explain the corresponding value of the function at these points based on the overlap between the event $e_t$ and the query segment $q$. The states $t_4$ and $t_5$ are analogous to $t_1$ and $t_0$, respectively, and therefore not illustrated. The difference is that the amount of overlap between the query segment and event decreases (instead of increases) when approaching these states.

\textbf{Case (i): $d_e \geq d_q$}

\begin{figure}[H]
    \centering
    \includegraphics[width=0.8\textwidth]{figures/case_i.png}
    \caption{An illustration of how the sound event $e_t$ and the query segment $q$ overlap at the four distinct states $t = t_0, \dots, t_3$ for case (i) where $d_e \geq d_q$.}
    \label{fig:case_i}
\end{figure}

\begin{itemize}
    \item $t^{(i)}_0$: At $t^{(i)}_0 = 0$, the end of the event $e_t$ aligns perfectly with the beginning of the query segment $q$. This means there is no overlap between the event and the query segment ($|e_{t^{(i)}_0} \cap q| = 0$). Therefore, assuming the annotator absence criterion applies, the query segment accuracy is $F(e_{t^{(i)}_0}, q, \gamma) = \frac{d_q - |e_{t^{(i)}_0} \cap q|}{d_q} = \frac{d_q - 0}{d_q} = 1$.
    \item $t^{(i)}_1$: The time $t^{(i)}_1 = \gamma d_e$ represents the point where the annotator presence criterion is first met. Before this point ($t < t^{(i)}_1$), the overlap $|e_t \cap q|$ is less than $\gamma d_e$, and the query segment accuracy is given by $F(e_t, q, \gamma) = \frac{d_q - |e_t \cap q|}{d_q}$. As $t$ approaches $t^{(i)}_1$ from the left, $|e_t \cap q|$ approaches $\gamma d_e$, hence $\lim_{t \to t_1^-} F(e_{t}, q, \gamma) = \frac{d_q - \gamma d_e}{d_q}$. At $t = t^{(i)}_1$, the presence criterion is met, and the accuracy function switches to $F(e_t, q, \gamma) = \frac{|e_t \cap q|}{d_q}$. As $t$ approaches $t^{(i)}_1$ from the right, $|e_t \cap q|$ is slightly greater than $\gamma d_e$, and $\lim_{t \to t_1^+} F(e_{t}, q, \gamma) = \frac{\gamma d_e}{d_q}$. This transition is visually represented in Figure~\ref{fig:case_i} at time $t=t_1$.
    \item $t^{(i)}_2$: At $t^{(i)}_2 = d_q$, the entire query segment $q$ is fully contained within the event $e_t$. This means the overlap is maximal: $|e_{t^{(i)}_2} \cap q| = d_q$. Since the presence criterion is met, the query segment accuracy is $F(e_{t^{(i)}_2}, q, \gamma) = \frac{|e_{t^{(i)}_2} \cap q|}{d_q} = \frac{d_q}{d_q} = 1$. This behavior is visually represented in Figure~\ref{fig:case_i} at time $t=t_2$, where the green box representing the event fully covers the red box representing the query segment.
    \item $t^{(i)}_3$: At $t^{(i)}_3 = d_e$, the entire query segment $q$ still fully overlaps with the event $e_t$. Similar to $t_2$, the overlap is $|e_{t^{(i)}_3} \cap q| = d_q$, and therefore $F(e_{t^{(i)}_3}, q, \gamma) = \frac{|e_{t^{(i)}_3} \cap q|}{d_q} = \frac{d_q}{d_q} = 1$. This is depicted in Figure~\ref{fig:case_i} at time $t=t_3$.
\end{itemize}

\textbf{Case (ii): $d_e < d_q$}

\begin{figure}[H]
    \centering
    \includegraphics[width=0.8\textwidth]{figures/case_ii.png}
    \caption{An illustration of how the sound event $e_t$ and the query segment $q$ overlap at the four distinct states $t = t_0, \dots, t_3$ for case (ii) where $d_e < d_q$.}
    \label{fig:case_ii}
\end{figure}

\begin{itemize}
    \item $t^{(ii)}_0$: At $t^{(ii)}_0 = 0$, the end of the event $e_t$ aligns perfectly with the beginning of the query segment $q$. There is no overlap ($|e_{t^{(ii)}_0} \cap q| = 0$). Assuming the annotator absence criterion applies, the query segment accuracy is $F(e_{t^{(ii)}_0}, q, \gamma) = \frac{d_q - |e_{t^{(ii)}_0} \cap q|}{d_q} = \frac{d_q - 0}{d_q} = 1$.
    \item $t^{(ii)}_1$: The time $t^{(ii)}_1 = \gamma d_e$ again marks the point where the annotator presence criterion is first met. Before this ($t < t^{(ii)}_1$), the overlap $|e_t \cap q| < \gamma d_e$, and $F(e_t, q, \gamma) = \frac{d_q - |e_t \cap q|}{d_q}$. Approaching $t^{(ii)}_1$ from the left, $|e_t \cap q| \to \gamma d_e$, thus $\lim_{t \to t_1^-} F(e_{t}, q, \gamma) = \frac{d_q - \gamma d_e}{d_q}$. At $t = t^{(ii)}_1$, the criterion is met, and the function becomes $F(e_t, q, \gamma) = \frac{|e_t \cap q|}{d_q}$. Approaching from the right, $|e_t \cap q|$ is slightly greater than $\gamma d_e$, so $\lim_{t \to t_1^+} F(e_{t}, q, \gamma) = \frac{\gamma d_e}{d_q}$. This transition is shown in Figure~\ref{fig:case_ii} at $t=t_1$.
    \item $t^{(ii)}_2$: At $t^{(ii)}_2 = d_e$, the beginning of the event $e_t$ aligns with the beginning of the query segment $q$. At this point, the overlap is maximal, as the entire event is contained within the query segment: $|e_{t^{(ii)}_2} \cap q| = d_e$. Since the presence criterion is met, the query segment accuracy is $F(e_{t^{(ii)}_2}, q, \gamma) = \frac{|e_{t^{(ii)}_2} \cap q|}{d_q} = \frac{d_e}{d_q}$. This situation is illustrated in Figure~\ref{fig:case_ii} at $t=t_2$.
    \item $t^{(ii)}_3$: At $t^{(ii)}_3 = d_q$, the end of the event $e_t$ aligns with the end of the query segment $q$. Similar to $t^{(ii)}_2$, the entire event is contained within the query segment, so the overlap is $|e_{t^{(ii)}_3} \cap q| = d_e$. Consequently, the query segment accuracy is $F(e_{t^{(ii)}_3}, q, \gamma) = \frac{|e_{t^{(ii)}_3} \cap q|}{d_q} = \frac{d_e}{d_q}$. This corresponds to the state depicted in Figure~\ref{fig:case_ii} at $t=t_3$.
\end{itemize}

Understanding these key time points and the corresponding query segment accuracy values is crucial for calculating the area under the curve, which represents the expected query segment accuracy.
%%%%%%%%%%%%%%%%%%%%%%%%%%%%%%%%%%%%%%%%%%%%%%%%%%%%%%%%%%%%%%%%%%%%%%%%%%%%%%%%%

\subsection{Proof of Theorem~\ref{thm:fix_optimal_query_length}}
\label{app:thm2}

\begin{proof}
We start by finding a unique critical point $d^*_q$ which makes $f'(d^*_q) = 0$ when $d_q \ge \gamma d_e$. We then show that $d_q^*$ is a global maximum by analyzing the boundaries of $f(d_q)$ on its' domain when $d_q \ge \gamma d_e$. We show that $f(d_q^*) \ge f(\gamma d_e)$ and that $f(d_q^*) \ge \lim_{d_q \rightarrow \infty} f(d_q)$. Since $d_q^*$ is a unique critical point we conclude that it must be a global maximum of the function $f(d_q)$ when $d_q \ge \gamma d_e$. Lastly, we show that $f(d_q^*) \ge f(\gamma d_e) \ge f(d_q)$ when $d_q < \gamma d_e$ which proves that $d_q^*$ is a global maximum of the function $f(d_q)$ for $d_q > 0$.

\textbf{1. Finding the unique critical point $d_q^*$.}

To find the critical points, we need to compute the derivative of $f(d_q)$ with respect to $d_q$ and set it to zero. Let $N(d_q) = d_e (-2d_e \gamma^2 + 2d_q \gamma + d_q)$ and $D(d_q) = d_q (d_e + d_q)$. Then $f(d_q) = \frac{N(d_q)}{D(d_q)}$. Using the quotient rule, the derivative is given by:
\begin{equation*}
f'(d_q) = \frac{N'(d_q)D(d_q) - N(d_q)D'(d_q)}{[D(d_q)]^2}
\end{equation*}
First, we find the derivatives of the numerator and the denominator:
\begin{align*}
N'(d_q) &= \frac{d}{dd_q} [d_e (-2d_e \gamma^2 + 2d_q \gamma + d_q)] \\
&= d_e (0 + 2\gamma + 1) \\
&= d_e (2\gamma + 1)
\end{align*}
\begin{align*}
D(d_q) &= d_q (d_e + d_q) = d_e d_q + d_q^2 \\
D'(d_q) &= \frac{d}{dd_q} [d_e d_q + d_q^2] \\
&= d_e + 2d_q
\end{align*}
Now, we plug these into the quotient rule formula:
\begin{align*}
f'(d_q) &= \frac{[d_e (2\gamma + 1)][d_q (d_e + d_q)] - [d_e (-2d_e \gamma^2 + 2d_q \gamma + d_q)][d_e + 2d_q]}{[d_q (d_e + d_q)]^2}
\end{align*}
To find the critical points, we set $f'(d_q) = 0$, which means the numerator must be zero:
\begin{equation*}
[d_e (2\gamma + 1)][d_q (d_e + d_q)] - [d_e (-2d_e \gamma^2 + 2d_q \gamma + d_q)][d_e + 2d_q] = 0
\end{equation*}
Since $d_e > 0$, we can divide by $d_e$:
\begin{equation*}
(2\gamma + 1) d_q (d_e + d_q) - (-2d_e \gamma^2 + 2d_q \gamma + d_q) (d_e + 2d_q) = 0
\end{equation*}
Expanding the terms:
\begin{align*}
(2\gamma + 1) (d_e d_q + d_q^2) &- (-2d_e^2 \gamma^2 - 4d_e d_q \gamma^2 + 2d_e d_q \gamma + 4d_q^2 \gamma + d_e d_q + 2d_q^2) = 0 \\
2\gamma d_e d_q + 2\gamma d_q^2 + d_e d_q + d_q^2 &- (-2d_e^2 \gamma^2 - 4d_e d_q \gamma^2 + 2d_e d_q \gamma + 4d_q^2 \gamma + d_e d_q + 2d_q^2) = 0
\end{align*}
Collecting and rearranging the terms to form a quadratic equation in $d_q$:
\begin{align*}
(2\gamma + 1 - 4\gamma - 2) d_q^2 + (2\gamma + 1 + 4\gamma^2 - 2\gamma - 1) d_e d_q + 2d_e^2 \gamma^2 &= 0 \\
(-2\gamma - 1) d_q^2 + (4\gamma^2) d_e d_q + 2d_e^2 \gamma^2 &= 0 \\
(2\gamma + 1) d_q^2 - 4\gamma^2 d_e d_q - 2d_e^2 \gamma^2 &= 0
\end{align*}
Using the quadratic formula $d_q = \frac{-b \pm \sqrt{b^2 - 4ac}}{2a}$, where $a = 2\gamma + 1$, $b = -4 d_e \gamma^2$, $c = -2 d_e^2 \gamma^2$:
\begin{align*}
d_q &= \frac{4 d_e \gamma^2 \pm \sqrt{(-4 d_e \gamma^2)^2 - 4 (2\gamma + 1) (-2 d_e^2 \gamma^2)}}{2 (2\gamma + 1)} \\
&= \frac{4 d_e \gamma^2 \pm \sqrt{16 d_e^2 \gamma^4 + 8 (2\gamma + 1) d_e^2 \gamma^2}}{4\gamma + 2} \\
&= \frac{4 d_e \gamma^2 \pm \sqrt{16 d_e^2 \gamma^4 + 16 d_e^2 \gamma^3 + 8 d_e^2 \gamma^2}}{4\gamma + 2} \\
&= \frac{4 d_e \gamma^2 \pm \sqrt{8 d_e^2 \gamma^2 (2\gamma^2 + 2\gamma + 1)}}{4\gamma + 2} \\
&= \frac{4 d_e \gamma^2 \pm 2 d_e |\gamma| \sqrt{4\gamma^2 + 4\gamma + 2}}{2(2\gamma + 1)}
\end{align*}
Since $\gamma > 0$, we have $|\gamma| = \gamma$:
\begin{align*}
d_q &= \frac{4 d_e \gamma^2 \pm 2 d_e \gamma \sqrt{4\gamma^2 + 4\gamma + 2}}{2(2\gamma + 1)} \\
&= \frac{2 d_e \gamma^2 \pm d_e \gamma \sqrt{4\gamma^2 + 4\gamma + 2}}{(2\gamma + 1)} \\
&= d_e \gamma \frac{2\gamma \pm \sqrt{4\gamma^2 + 4\gamma + 2}}{2\gamma + 1}
\end{align*}
We note that $\sqrt{4\gamma^2 + 4\gamma + 2} = 2\sqrt{\gamma^2 + \gamma + 0.5} > 2\sqrt{\gamma^2} = 2\gamma$, which means that we need to choose the positive sign for $d_q>0$ to be true.
The value of $d_q$ that makes the derivative zero is therefore uniquely defined by:
\begin{equation*}
d_q = d_e\,\gamma \frac{2\,\gamma + \sqrt{4\,\gamma^2 +4\,\gamma +2}}{2\,\gamma +1} \ge d_e \gamma,
\end{equation*}
where the last inequality holds because $\sqrt{4\gamma^2 + 4\gamma + 2} = 2\sqrt{\gamma^2 + \gamma + 0.5} \ge 1$.

\textbf{2. Analyze the function at the boundaries of its' domain.}

To understand why this critical point corresponds to a maximum, we analyze the function $f(d_q)$ as $d_q$ at the boundaries of its domain.

\textbf{2a. $f(d_q)$ when $d_q = \gamma d_e$ ($d_q \ge \gamma d_e$).}
\begin{align*}
    f(\gamma d_e) &= \frac{d_{e} \left( 2 (\gamma d_e) \gamma - 2 d_{e} \gamma^{2} + (\gamma d_e)\right)}{(\gamma d_e) \left(d_{e} + \gamma d_e\right)} \\
    &= \frac{d_{e} \left( 2 d_e \gamma^2 - 2 d_{e} \gamma^{2} + \gamma d_e\right)}{(\gamma d_e) \left(d_{e} + \gamma d_e\right)} \\
    &= \frac{d_{e} \left(\gamma d_e\right)}{(\gamma d_e) \left(d_{e} + \gamma d_e\right)} \\
    &= \frac{d_{e}^2 \gamma}{(\gamma d_e) d_{e}(1+\gamma)} \\
    &= \frac{1}{1+\gamma}.
\end{align*}

\textbf{2b. $f(d_q)$ as $d_q \rightarrow \infty$ ($d_q \ge \gamma d_e$).}

We want to evaluate the limit of $f(d_q)$ as $d_q$ approaches infinity:
\begin{align*}
\lim_{d_q \rightarrow \infty} f(d_q) &= \lim_{d_q \rightarrow \infty} \frac{d_e (-2d_e \gamma^2 + (2\gamma + 1)d_q)}{d_e d_q + d_q^2}
\end{align*}
Divide the numerator and the denominator by the highest power of $d_q$ in the denominator, which is $d_q^2$:
\begin{align*}
\lim_{d_q \rightarrow \infty} f(d_q) &= \lim_{d_q \rightarrow \infty} \frac{d_e \left(-\frac{2d_e \gamma^2}{d_q^2} + \frac{2\gamma + 1}{d_q}\right)}{\frac{d_e}{d_q} + 1}
\end{align*}
As $d_q \rightarrow \infty$, the terms $\frac{2d_e \gamma^2}{d_q^2}$, $\frac{2\gamma + 1}{d_q}$, and $\frac{d_e}{d_q}$ all approach 0. Thus,
\begin{equation*}
\lim_{d_q \rightarrow \infty} f(d_q) = \frac{d_e (0 + 0)}{0 + 1} = 0
\end{equation*}
This means that as $d_q$ becomes very large, the function $f(d_q)$ approaches 0.

\textbf{2c. Showing that $f(d_q^*) \ge f(\gamma d_e)$.}

We want to show that $f(d^*_q) \ge f(\gamma d_e)$. Or equivalently, that $f(d^*_q) - f(\gamma d_e) \ge 0$. From Theorem~\ref{thm:max_iou} we know that $f(d^*_q) = 2\gamma\left(2\gamma + 1 - \sqrt{4\gamma^2 + 4\gamma + 2}\right) + 1$, and from 2a we know that $f(\gamma d_e) = \frac{1}{1+\gamma}$. After substitution and some algebraic manipulation, we get
$$\gamma\left(\frac{4\gamma^2+6\gamma+3}{1+\gamma} - 2\sqrt{4\gamma^2 + 4\gamma + 2}\right) \ge 0.$$
Since $\gamma > 0$, it suffices to show that 
$$\frac{4\gamma^2+6\gamma+3}{1+\gamma} \geq 2\sqrt{4\gamma^2 + 4\gamma + 2}.$$
Squaring both sides of the above inequality and simplifying, we obtain the equivalent inequality
$$\left(\frac{4\gamma^2+6\gamma+3}{1+\gamma}\right)^2 \geq 4(4\gamma^2 + 4\gamma + 2).$$
After further algebraic manipulations (which we leave to the reader), we arrive at the inequality
$$(2\gamma + 1)^2 \geq 0.$$
Since $(2\gamma + 1)^2 \geq 0$ holds for all $\gamma$, and the previous steps are all equivalences, we conclude that 
$$ f(d_q^*) - f(\gamma d_e) \ge 0$$ for $0<\gamma\le 1$, and therefore,
$$f(d^*_q) \ge f(\gamma d_e).$$

\textbf{2d. Showing that $f(d_q^*) \ge \lim_{d_q\rightarrow\infty} f(d_q)$.}

We combine the results from 2a-2c to get
\begin{align*}
    f(d_q^*) &\ge f(\gamma d_e) \\
    &= \frac{1}{1+\gamma} \\
    &\ge 0  \\
    &= \lim_{d_q \rightarrow \infty} f(d_q).
\end{align*}

\textbf{2e. $f(d_q)$ as $d_q \rightarrow (\gamma d_e)^-$ ($d_q < \gamma d_e$).}

Since we are approaching $\gamma d_e$ from the left, we have that $f(d_q) = d_q/(d_e + d_q)$. This function is continuous for $d_q < \gamma d_e$, so the limit is given by the direct substitution:
\begin{align*}
\lim_{d_q \rightarrow (\gamma d_e)^-} \frac{d_q}{d_e + d_q} &= \frac{\gamma d_e}{d_e + \gamma d_e} \\
&= \frac{\gamma d_e}{d_e(1 + \gamma)} \\
&= \frac{\gamma}{1 + \gamma}
\end{align*}

\textbf{2f. Showing that $f(\gamma d_e) \ge f(d_q)$ when $d_q < \gamma d_e$}.
We start by noting that $f(\gamma d_e) = \frac{1}{1+\gamma} \ge \frac{\gamma}{1+\gamma} = \lim_{d_q\rightarrow(\gamma d_e)^-}$. Now it is sufficient to show that $f(d_q) = d_q/(d_q + d_e)$ is strictly decreasing for decreasing $d_q$, which we do by computing the derivative of $f(d_q)$ with respect to $d_q$ using the quotient rule:
\begin{align*}
    f'(d_q) &= \frac{(d_q+\gamma)(1) - d_q(1)}{(d_q+\gamma)^2} \\
    &= \frac{d_q+\gamma-d_q}{(d_q+\gamma)^2} \\
    &= \frac{\gamma}{(d_q+\gamma)^2}.
\end{align*}

Since $\gamma > 0$ and $(d_q+\gamma)^2 > 0$ for all $d_q > 0$, we have $f'(d_q) > 0$ for all $d_q > 0$. This implies that the function $f(d_q)$ is strictly increasing on the interval $(0, \infty)$. Therefore, if $0 < c \le b$, it must be the case that $f(c) \le f(b)$. Moreover, since $c < b$, $f(c) < f(b)$. Thus, for any $b>0$, $f(b) > f(c)$ for all $0< c \le b$. Now let $0 < d_q = c \leq \gamma d_e = b$.


\textbf{3. Combining everything (2a-2f)}

We have derived a unique critical point $d_q^* \ge \gamma d_e$ by setting the first derivative of $f(d_q)$ to zero. We have then shown that $f(d_q^*)$ is greater than or equal to $f(d_q)$ at the limits of its' domain when $d_q \ge \gamma d_e$. Finally, we show that $f(d_q^*) \ge f(\gamma d_e) \ge f(d_q)$ when $d_q < \gamma d_e$. Therefore, the value of $d_q$ that is the global maximum of $f(d_q)$ when $d_q>0$ is:
\begin{equation*}
\boxed{d_q^* = d_e\,\gamma \frac{2\,\gamma + \sqrt{4\,\gamma^2 +4\,\gamma +2}}{2\,\gamma +1}}
\end{equation*}
\end{proof}



% \subsection{Proof of Theorem~\ref{thm:max_iou}}
% \label{app:thm3}


% From Theorem~\ref{thm:fix_optimal_query_length} we know that $d_q^*$ maximize the expected label accuracy $f(d_q)$ for events of length $d_e$. We can thus define the maximum label accuracy as
% \begin{align}
% \label{eq:max_iou}
%     f^*(\gamma) &= f(d_q^*) \\
%     &= \frac{d_e(2\gamma d_q^*  - 2\gamma^2 d_e+ d_q^*)}{d_q^*(d_e + d_q^*)} \\
%     &= 2\gamma(2\gamma +1 - \sqrt{4\gamma^2+4\gamma + 2}) + 1
% \end{align}

% In the supplementary material, there is a symbolic mathematics solver that verifies that these expressions are equal. This can be derived by substituting $\delta = d_q / d_q$

\subsection{Proof of Theorem~\ref{thm:max_iou}}
\label{app:thm3}

\begin{proof}
From Theorem~\ref{thm:fix_optimal_query_length} we have that
\[
d_q^*
\;=\;
\frac{
  d_e\,\gamma\,\Bigl(2\,\gamma + \sqrt{4\,\gamma^2 +4\,\gamma +2}\Bigr)
}{
  2\,\gamma +1
}
\]
maximizes the function
\[
f(d_q)
\;=\;
\frac{
  d_e \bigl(-2\,d_e\,\gamma^2 \;+\; (2\,\gamma +1)\,d_q\bigr)
}{
  d_q\,\bigl(d_e + d_q\bigr)
}.
\]
We wish to show that the maximum label accuracy given overlap, 
\(\displaystyle f^*(\gamma) = f\bigl(d_q^*\bigr),\)
is 
\[
2\,\gamma\!\Bigl(
  2\,\gamma +1
  \;-\;
  \sqrt{\,4\,\gamma^2 +4\,\gamma +2}
\Bigr)
\;+\;
1.
\]

\medskip

\noindent
\textbf{1. Express $f(d_q)$ in terms of a dimensionless variable.}

Define
\[
\delta 
\;=\; 
\frac{d_q}{d_e}.
\]
Then
\[
d_q 
\;=\; 
\delta\,d_e,
\quad
d_e + d_q
\;=\;
d_e\,(1 + \delta),
\]
and
\[
f(d_q)
\;=\;
f(\delta\,d_e)
\;=\;
\frac{
  d_e \bigl(-2\,d_e\,\gamma^2 + (2\,\gamma +1)\,\delta\,d_e\bigr)
}{
  (\delta\,d_e)\,\bigl(d_e + \delta\,d_e\bigr)
}
=
\frac{
  -2\,\gamma^2 + (2\,\gamma +1)\,\delta
}{
  \delta \,\bigl(1 + \delta\bigr)
}.
\]
We can therefore write
\[
f(\delta)
\;=\;
\frac{
  -2\,\gamma^2 
  \;+\; 
  (2\,\gamma +1)\,\delta
}{
  \delta\,(1 + \delta)
}.
\]

\medskip

\noindent
\textbf{2. Identify the optimal dimensionless query length $\delta^*$.}

From Theorem~\ref{thm:fix_optimal_query_length}, we know that
\[
d_q^*
\;=\;
\frac{
  d_e\,\gamma\,
  \bigl(
    2\,\gamma \;+\; \sqrt{4\,\gamma^2 +4\,\gamma +2}
  \bigr)
}{
  2\,\gamma +1
}.
\]
Dividing both sides by \(d_e\) gives
\[
\delta^*
\;=\;
\frac{d_q^*}{d_e}
\;=\;
\gamma \,\frac{
  2\,\gamma \;+\; \sqrt{\,4\,\gamma^2 +4\,\gamma +2}
}{
  2\,\gamma +1
}.
\]
We need to show that
\[
f\bigl(\delta^*\bigr)
\;=\;
2\,\gamma\!\Bigl(2\,\gamma +1 - \sqrt{4\,\gamma^2 +4\,\gamma +2}\Bigr)
\;+\;
1.
\]

\medskip

\noindent
\textbf{3. Compute $f(\delta^*)$ explicitly.}

Let
\[
N(\delta)
\;=\;
-2\,\gamma^2 
\;+\; 
(2\,\gamma +1)\,\delta,
\quad
D(\delta)
\;=\;
\delta\,(1 + \delta).
\]
Then 
\(\;f(\delta) = \tfrac{N(\delta)}{D(\delta)}.\)

\begin{enumerate}
\item 
\textit{Numerator at \(\delta^*\).}

\[
N\bigl(\delta^*\bigr)
=
-2\,\gamma^2
\;+\;
(2\,\gamma +1)\,\delta^*
=
-2\,\gamma^2
\;+\;
(2\,\gamma +1)
\Bigl[
  \gamma
  \,\frac{
    2\,\gamma + \sqrt{\,4\,\gamma^2 +4\,\gamma +2}
  }{
    2\,\gamma +1
  }
\Bigr].
\]
Inside the brackets, \((2\,\gamma +1)\) cancels:
\[
N\bigl(\delta^*\bigr)
=
-2\,\gamma^2 
\;+\; 
\gamma\,\bigl(2\,\gamma + \sqrt{\,4\,\gamma^2 +4\,\gamma +2}\bigr)
=
-2\,\gamma^2 
\;+\;
2\,\gamma^2
\;+\;
\gamma\;\sqrt{\,4\,\gamma^2 +4\,\gamma +2}
=
\gamma\;\sqrt{\,4\,\gamma^2 +4\,\gamma +2}.
\]

\item 
\textit{Denominator at \(\delta^*\).}

\[
D(\delta)
\;=\;
\delta\,(1 + \delta).
\]
Hence,
\[
D\bigl(\delta^*\bigr)
=
\delta^*
\Bigl(1 + \delta^*\Bigr)
=
\Bigl[
  \gamma \,\frac{2\,\gamma + \sqrt{\,4\,\gamma^2 +4\,\gamma +2}}{2\,\gamma +1}
\Bigr]
\Bigl[
  1
  \;+\;
  \gamma \,\frac{2\,\gamma + \sqrt{\,4\,\gamma^2 +4\,\gamma +2}}{2\,\gamma +1}
\Bigr].
\]
The second bracket becomes a single fraction:
\[
1 
+ 
\gamma \,\frac{2\,\gamma + \sqrt{\,4\,\gamma^2 +4\,\gamma +2}}{2\,\gamma +1}
=
\frac{
  (2\,\gamma +1)
  \;+\;
  \gamma\;\bigl(2\,\gamma + \sqrt{\,4\,\gamma^2 +4\,\gamma +2}\bigr)
}{
  2\,\gamma +1
}.
\]
Combining, we get
\[
D\bigl(\delta^*\bigr)
=
\gamma \,\frac{2\,\gamma + \sqrt{\,4\,\gamma^2 +4\,\gamma +2}}{2\,\gamma +1}
\;\times\;
\frac{
  (2\,\gamma +1)
  + 
  2\,\gamma^2 
  + 
  \gamma\;\sqrt{\,4\,\gamma^2 +4\,\gamma +2}
}{
  2\,\gamma +1
}.
\]
So
\[
D\bigl(\delta^*\bigr)
=
\gamma\,
\frac{
  (2\,\gamma + \sqrt{\,4\,\gamma^2 +4\,\gamma +2})
  \,\bigl(
    2\,\gamma +1 + 2\,\gamma^2 
    + 
    \gamma\,\sqrt{\,4\,\gamma^2 +4\,\gamma +2}
  \bigr)
}{
  (2\,\gamma +1)^2
}.
\]

\item
\textit{Form the ratio.}  
Thus,
\[
f\bigl(\delta^*\bigr)
=
\frac{
  N(\delta^*)
}{
  D(\delta^*)
}
=
\frac{
  \gamma\;\sqrt{\,4\,\gamma^2 +4\,\gamma +2}
}{
  \gamma 
  \,\frac{
    (2\,\gamma + \sqrt{\,4\,\gamma^2 +4\,\gamma +2})
    \,\bigl(
      2\,\gamma +1 + 2\,\gamma^2 
      + 
      \gamma\,\sqrt{\,4\,\gamma^2 +4\,\gamma +2}
    \bigr)
  }{
    (2\,\gamma +1)^2
  }
}.
\]
Cancel the common factor \(\gamma\), invert the denominator and multiply:
\[
f\bigl(\delta^*\bigr)
=
\frac{
  \sqrt{\,4\,\gamma^2 +4\,\gamma +2}\,(2\,\gamma +1)^2
}{
  (2\,\gamma + \sqrt{\,4\,\gamma^2 +4\,\gamma +2})
  \,\bigl(
    2\,\gamma +1 + 2\,\gamma^2 
    + 
    \gamma\,\sqrt{\,4\,\gamma^2 +4\,\gamma +2}
  \bigr)
}.
\]
You can verify by direct expansion (or by a symbolic algebra tool which we provide in the supplementary material) that
\[
\frac{
  \sqrt{\,4\,\gamma^2 +4\,\gamma +2}\,(2\,\gamma +1)^2
}{
  (2\,\gamma + \sqrt{\,4\,\gamma^2 +4\,\gamma +2})
  \,\bigl(
    2\,\gamma +1 + 2\,\gamma^2 
    + 
    \gamma\,\sqrt{\,4\,\gamma^2 +4\,\gamma +2}
  \bigr)
}
=
2\,\gamma\,\Bigl(2\,\gamma +1 - \sqrt{\,4\,\gamma^2 +4\,\gamma +2}\Bigr) 
\;+\;
1.
\]
Thus
\[
f\bigl(\delta^*\bigr)
\;=\;
2\,\gamma\,\Bigl(2\,\gamma +1 - \sqrt{4\,\gamma^2 +4\,\gamma +2}\Bigr) + 1,
\]
which proves that
\[
f^*(\gamma)
=
f\bigl(d_q^*\bigr)
=
2\,\gamma\,\Bigl(2\,\gamma +1 - \sqrt{4\,\gamma^2 +4\,\gamma +2}\Bigr)
\;+\;
1.
\]
\end{enumerate}

Hence, Eq.~\ref{eq:max_iou} holds, completing the proof.
\end{proof}
