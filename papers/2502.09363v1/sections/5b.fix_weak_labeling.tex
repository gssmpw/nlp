\section{Interpretation under $\delta = d_q / d_e$}

\begin{theorem}
\label{thm:optimal_delta}
For a given presence criterion $\gamma \in (0, 1]$, the relative query segment length $\delta > 0$ that maximizes the expected label accuracy $f(\delta) = \frac{2 \delta \gamma + \delta - 2 \gamma^{2}}{\delta \left(\delta + 1\right)}$ is given by
\begin{equation}
\label{eq:optimal_delta}
    \delta^* = \frac{2\gamma^2 + \gamma\sqrt{2}\sqrt{2\gamma^2 + 2\gamma + 1}}{2\gamma + 1}.
\end{equation}
\end{theorem}

\begin{proof}
To find the value of $\delta$ that maximizes $f(\delta)$, we need to find the critical points by taking the derivative of $f(\delta)$ with respect to $\delta$ and setting it to zero. Let's compute $f'(\delta)$:
\begin{align*}
    f(\delta) &= \frac{2 \delta \gamma + \delta - 2 \gamma^{2}}{\delta \left(\delta + 1\right)} = \frac{(2\gamma + 1)\delta - 2\gamma^2}{\delta^2 + \delta}
\end{align*}
Using the quotient rule for differentiation, where $u(\delta) = (2\gamma + 1)\delta - 2\gamma^2$ and $v(\delta) = \delta^2 + \delta$, we have $u'(\delta) = 2\gamma + 1$ and $v'(\delta) = 2\delta + 1$.
\begin{align*}
    f'(\delta) &= \frac{u'(\delta)v(\delta) - u(\delta)v'(\delta)}{[v(\delta)]^2} \\
    &= \frac{(2\gamma + 1)(\delta^2 + \delta) - ((2\gamma + 1)\delta - 2\gamma^2)(2\delta + 1)}{(\delta^2 + \delta)^2} \\
    &= \frac{(2\gamma + 1)\delta^2 + (2\gamma + 1)\delta - [(2\gamma + 1)\delta(2\delta + 1) - 2\gamma^2(2\delta + 1)]}{(\delta^2 + \delta)^2} \\
    &= \frac{(2\gamma + 1)\delta^2 + (2\gamma + 1)\delta - [(4\gamma + 2)\delta^2 + (2\gamma + 1)\delta - 4\gamma^2\delta - 2\gamma^2]}{(\delta^2 + \delta)^2} \\
    &= \frac{(2\gamma + 1)\delta^2 + (2\gamma + 1)\delta - (4\gamma + 2)\delta^2 - (2\gamma + 1)\delta + 4\gamma^2\delta + 2\gamma^2}{(\delta^2 + \delta)^2} \\
    &= \frac{(2\gamma + 1 - 4\gamma - 2)\delta^2 + (2\gamma + 1 - 2\gamma - 1)\delta + 4\gamma^2\delta + 2\gamma^2}{(\delta^2 + \delta)^2} \\
    &= \frac{(-2\gamma - 1)\delta^2 + 4\gamma^2\delta + 2\gamma^2}{(\delta^2 + \delta)^2} \\
    &= \frac{-(2\gamma + 1)\delta^2 + 4\gamma^2\delta + 2\gamma^2}{(\delta^2 + \delta)^2}
\end{align*}
To find the critical points, we set $f'(\delta) = 0$, which means we only need to set the numerator to zero:
\begin{equation}
    -(2\gamma + 1)\delta^2 + 4\gamma^2\delta + 2\gamma^2 = 0
\end{equation}
Multiply by $-1$:
\begin{equation}
    (2\gamma + 1)\delta^2 - 4\gamma^2\delta - 2\gamma^2 = 0
\end{equation}
This is a quadratic equation in $\delta$ of the form $a\delta^2 + b\delta + c = 0$, with $a = 2\gamma + 1$, $b = -4\gamma^2$, and $c = -2\gamma^2$. Using the quadratic formula:
\begin{align*}
    \delta &= \frac{-b \pm \sqrt{b^2 - 4ac}}{2a} \\
    &= \frac{4\gamma^2 \pm \sqrt{(-4\gamma^2)^2 - 4(2\gamma + 1)(-2\gamma^2)}}{2(2\gamma + 1)} \\
    &= \frac{4\gamma^2 \pm \sqrt{16\gamma^4 + 8\gamma^2(2\gamma + 1)}}{2(2\gamma + 1)} \\
    &= \frac{4\gamma^2 \pm \sqrt{16\gamma^4 + 16\gamma^3 + 8\gamma^2}}{2(2\gamma + 1)} \\
    &= \frac{4\gamma^2 \pm \sqrt{8\gamma^2(2\gamma^2 + 2\gamma + 1)}}{2(2\gamma + 1)} \\
    &= \frac{4\gamma^2 \pm 2\gamma\sqrt{2}\sqrt{2\gamma^2 + 2\gamma + 1}}{2(2\gamma + 1)} \\
    &= \frac{2\gamma^2 \pm \gamma\sqrt{2}\sqrt{2\gamma^2 + 2\gamma + 1}}{2\gamma + 1} \\
    &= \frac{\gamma}{2\gamma + 1} \left( 2\gamma \pm \sqrt{2}\sqrt{2\gamma^2 + 2\gamma + 1} \right)
\end{align*}
Since $\delta > 0$, we must consider which root is positive. For $\gamma \in (0, 1]$, $2\gamma + 1 > 0$ and $\gamma > 0$. We need to examine the term $2\gamma \pm \sqrt{2}\sqrt{2\gamma^2 + 2\gamma + 1}$. Let's consider the negative root:
$2\gamma - \sqrt{2}\sqrt{2\gamma^2 + 2\gamma + 1}$. We know that $\sqrt{2} > 1$, so let's compare $\sqrt{2}\sqrt{2\gamma^2 + 2\gamma + 1}$ with $2\gamma$. Squaring both terms, we compare $2(2\gamma^2 + 2\gamma + 1) = 4\gamma^2 + 4\gamma + 2$ with $(2\gamma)^2 = 4\gamma^2$. Since $4\gamma^2 + 4\gamma + 2 > 4\gamma^2$ for $\gamma \in (0, 1]$, we have $\sqrt{2}\sqrt{2\gamma^2 + 2\gamma + 1} > 2\gamma$. Thus, the negative root would lead to a negative $\delta$, which is not allowed. Therefore, we take the positive root:
\begin{equation}
    \delta^* = \frac{2\gamma^2 + \gamma\sqrt{2}\sqrt{2\gamma^2 + 2\gamma + 1}}{2\gamma + 1} = \frac{\gamma}{2\gamma + 1} \left( 2\gamma + \sqrt{2}\sqrt{2\gamma^2 + 2\gamma + 1} \right)
\end{equation}
To confirm that this is a maximum, we can analyze the sign of $f''(\delta)$ at $\delta = \delta^*$ or examine the behavior of $f'(\delta)$ around $\delta^*$. As shown in the thought process above, by analyzing the first derivative's numerator's derivative at $\delta^*$, it can be shown that the second derivative is negative at $\delta^*$, indicating a local maximum. Given the context of expected accuracy, and the function's behavior, this local maximum is indeed the global maximum for $\delta > 0$ in the domain where $d_q \geq \gamma d_e$ (which is $\delta \geq \gamma$).

We should also verify that $\delta^* \geq \gamma$ to ensure it is within the domain where we derived $f(\delta)$.  We need to show that $\frac{2\gamma^2 + \gamma\sqrt{2}\sqrt{2\gamma^2 + 2\gamma + 1}}{2\gamma + 1} \geq \gamma$. This was already verified in the thought process.

Thus, the optimal relative query segment length that maximizes the expected label accuracy $f(\delta)$ is given by Eq.~\ref{eq:optimal_delta}.
\end{proof}

This theorem provides a closed-form expression for the optimal relative query segment length $\delta^*$ that maximizes the expected label accuracy $f(\delta)$ for a given presence criterion $\gamma$. This result is crucial for understanding how to choose the query segment length relative to the event length to achieve the best possible label accuracy when using the FIX weak labeling method under the assumed simplified data distribution.