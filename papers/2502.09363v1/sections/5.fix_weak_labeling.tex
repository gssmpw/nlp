\section{The Label Accuracy and Cost of FIX Weak Labeling}

The outline of this section is as follows. In Section~\ref{sec:fix_weak_labeling_method} we define the FIX labeling method. In Section~\ref{sec:expected_label_accuracy_given_overlap} we derive a closed-form expression for the expected label accuracy of a query segment given that it overlaps with a single event of deterministic event length. We note that it is only in the cases of overlap between a query segment and an event that a presence label can occur under the assumed annotator model, and that the expectation in label accuracy over these cases therefore can be viewed as the expected presence label accuracy. For the remainder of the paper we will simply write expected label accuracy when referring to the expectation over the overlapping cases, unless explicitly stated otherwise.

In the same section we derive the optimal query length with respect to the expected label accuracy, the maximum expected label accuracy and the number of query segments needed (proxy for annotation cost). In Section~\ref{sec:theory_event_distribution} we explain how the expression for expected label accuracy can be used in the case of a single event of stochastic length, and in Section~\ref{sec:theory_multi_events} we explain under which conditions this can be used when multiple events can occur. Finally, we derive a closed form expression for the expected label accuracy of an audio recording with multiple events of stochastic length in Section~\ref{sec:expected_label_accuracy_all_cases}, and provide an alternative interpretation of the theory in Section~\ref{sec:ratio_based}.



\subsection{The FIX Weak Labeling Method}
\label{sec:fix_weak_labeling_method}
The FIX weak labeling method, commonly used in practice, splits the audio recording into fixed and equal length query segments, and then an annotator is asked to provide either a presence or absence label for each of the query segments. 
%This method is commonly used in practice~\citep{Shuyang2017, Shuyang2018, Wang2022, Martin-Morato2023a}, so we want to better understand the limits. 
Let $B_{\text{FIX}}$ denote the number of query segments used, then the query segments for an audio recording of length $T$ are defined as
\begin{equation}
\label{eq:fix}
    \sQ_{\text{FIX}} = \{(a_0, b_0), (a_1, b_1), \dots, (a_{B_{\text{FIX}}-1}, b_{B_{\text{FIX}}-1})\} = \{q_0, \dots, q_{B_{\text{FIX}}-1}\},
\end{equation}
where the start and end timings of each query segment is $q_i = (a_i, b_i) = (id_q, (i+1)d_q)$ and the fixed query segment length is $d_q = T / B_{\text{FIX}} $. We illustrate this in Figure~\ref{fig:fix_weak_labeling}, where the presence criterion for the annotator is $\gamma=0.5$. There are three presence events and four absence events, and using only $B_{\text{FIX}}=7$ query segments results in annotations with an average label accuracy that is lower than $1$. 

\begin{figure}[ht!]
    \centering
    \includegraphics[width=0.8\linewidth]{figures/fix_weak_labeling.png}
    \caption{
    %The FIX weak labeling process. Observe how the fixed-length query segments ($q_0$ to $q_6$) can overlap with presence events (green), leading to potential inaccuracies in the assigned presence labels (red).
    Illustration of the FIX weak labeling method. The audio recording contains presence events (green). The FIX method divides the recording into fixed-length query segments (e.g., $q_0$ to $q_6$). Note how the alignment between segments and presence events affects the accuracy of presence labels (red hatched).
    %FIX weak labeling of an audio recording with three presence events and four absence events. The $B=7$ segments, $q_0, \dots, q_6$ are of fixed and equal length. The label accuracy depends on how well these segments align with the target events and their length.
    } 
    \label{fig:fix_weak_labeling}
\end{figure}

We want to find an expression for the expected label accuracy for a given data distribution and query segment length. In addition, we want to understand the query length that maximize the expected label accuracy.


\subsection{The Expected Label Accuracy of a Query Segment given Event Overlap}
\label{sec:expected_label_accuracy_given_overlap}

To derive a tractable closed-form expression we analyze a simplified data distribution, consisting of audio recordings of length $T$ that always contain a single event of deterministic length $d_e$. This is arguably the simplest data distribution to annotate, and the results can therefore be viewed as an upper bound on the expected label accuracy for any more complex data distribution. %, since it is arguably the simplest data distribution to annotate. 

%In addition, the theoretical results derived for this simplified data distribution can be used to understand the label accuracy of more complex data distributions with stochastic event lengths or multiple events. We expand on this in  Section~\ref{sec:theory_event_distribution} and Section~\ref{sec:theory_multi_events}, and provide simulation studies with more complex data distributions in Section~\ref{sec:event_distribution} and Section~\ref{sec:multi_events} to support these claims. 

\begin{figure}
    \centering
    \includegraphics[width=0.8\linewidth]{figures/occurences_2.png}
    \caption{
    %The bottom panel shows the resulting label accuracy for query segment $q_2$ based on the event occurrences depicted in the upper panel as the event's end time ($t$) varies. Overlap causes label noise (red); the gray area ($A$) represents the label accuracy during overlaps.
    \textit{Top panel:}  A single event ($e_t$) of length $d_e$ can occur at various end times ($t$) within the recording of length $T$. \textit{Bottom panel:} The resulting label accuracy for query segment $q_2$ (arbitrarily chosen for illustration) of length $d_q$ as a function of the event's end time ($t$). Overlap between the event and the query segment leads to segment label noise and a reduced label accuracy, which in this case occur when $t\in[a_2, a_2+d_e+d_q]$ where $a_2$ is the start time of $q_2$. The red hatched area ($A$) represents the cumulative label accuracy during these overlapping scenarios.
    %We illustrate the label accuracy as a function of all possible event occurrences $e_t$ in the bottom panel for a given query segment $q_2$, where $t$ is the offset of event $e_t$, $T$ is the length of the audio recording, $d_e$ is the event length, $d_q$ is the query segment length and $d_e + d_q$ is the total amount of overlapping occurrences. The label accuracy is $1$ in all non-overlapping cases, and less than $1$ in all overlapping cases. That is, segment label noise only occur in cases of overlap (red). We are interested in understanding the label accuracy of the presence labels (gray), and therefore want to compute the gray area $A$. Considering a different query segment $q_i$ will only result in a translation of this area along the x-axis.
    }
    \label{fig:proof_idea}
\end{figure}

The setup is illustrated in the upper panel of Figure~\ref{fig:proof_idea}, where a single event $e_t$ of length $d_e$ can occur at any time $t\in[0, T]$ (indicated by the arrow). The bottom panel of Figure~\ref{fig:proof_idea} shows the label accuracy for a specific query segment ($q_2$) as the end time ($t$) of the event varies. The area (A) highlighted in hatched red indicates the label accuracy in the cases of overlap between the query segment and the event, and the area in hatched green indicate the label accuracy in the cases of no overlap, which is by the definition of the annotator model is always $1$. Crucially, while this figure illustrates the accuracy for query segment $q_2$, the shape of this accuracy function remains the same for other query segments; only its position along the x-axis would change.  %, and then later in Section~\ref{sec:expected_label_accuracy_all_cases} we will explain how this can be used to derive an expression for label accuracy for all cases.

%Computing the expectation over both overlapping and non-overlapping cases would mean that more absence sound results in a higher metric score. In fact, let $A$ denote the gray area in Figure~\ref{fig:proof_idea}, then the expected label accuracy over all cases is $(A + T-(d_e+d_q))/T$, since we always have a score of $1$ for the $T-(d_e+d_q)$ cases of non-overlap. We can see that $(A + T-(d_e+d_q))/T \rightarrow 1 $ if $T\rightarrow \infty$. We are more interested in the labeling errors that occur around the presence events, and therefore choose to only consider the cases with overlap when computing the expected label accuracy. That is, we want to derive an expression for $A / (d_e + d_q)$.

To simplify the mathematical analysis, without loss of generality, we can fix the query segment to start at time $0$, $q=(0, d_q)$, and represent the event with its ending time $t$ as $e_t = (t-d_e, t)$. In this way, $t\in[0, d_e+d_q]$ describes all possible overlap occurrences. That is, when $t=0$ the event ends at the start of the query segment, and when $t=d_e+d_q$ the event starts at the end of the query segment. To formalize this, we can express the expected label accuracy in case of overlap by integrating over all possible event end times ($t$) where overlap occurs: %To get the expected label accuracy given overlap we need to compute
\begin{align}
\label{eq:_integral_1}
    \E_{\rt \sim p}\left[F(e_{\rt}, q, \gamma)\right] &= \int_{0}^{d_e + d_q}F(e_t, q, \gamma)p(t)\mathrm{d}t, \\
    \label{eq:_integral_2}
    &= \frac{1}{d_e + d_q} \int_{0}^{d_e + d_q}F(e_t, q, \gamma)\mathrm{d}t \\
    \label{eq:normalized_area}
    &= \frac{A}{d_e + d_q}.
    %&= \frac{1}{d_e + d_q}\int_{0}^{d_e + d_q}F(e_t, q, \gamma)
\end{align}
where $\rt \sim p$ denotes a random variable $\rt$ distributed according to a distribution $p$, and $p(t)$ denotes the probability of realization $t$. Since we assume that the sound event can occur anywhere in the audio recording with equal probability we get $p(\rt) = 1/(d_e + d_q)$, and by observing that the integral $\int_{0}^{d_e+d_q}F(e_t, q, \gamma)\mathrm{d}t$ describes the hatched red area denoted $A$ in Figure~\ref{fig:proof_idea} we arrive at the final expression in Eq.~\ref{eq:normalized_area}. 

Remember that absence labels can occur when there is no overlap (always correct) and when there is overlap but the presence criterion is not fulfilled, and presence labels can only occur when there is overlap and the presence criterion is fulfilled. Therefore, inaccurate labels only occur in the case of overlap. The expected label accuracy in the case of overlap therefore describes the accuracy of the labels when segment label noise can occur, which happens around the boundaries of the true event.

In Appendix~\ref{app:thm1} we show how to express $A$ in terms of the event length $d_e$, the query segment length $d_q$ and the presence criterion $\gamma$ under the assumption that the annotator presence criterion can be fulfilled ($d_q \geq \gamma d_e$), and that it can not be fulfilled ($d_q < \gamma d_e$). Finally, we arrive at the following four main theorems:

\begin{theorem}
\label{thm:expected_iou}
The expected label accuracy in case of overlap between a query segment $q$ of length $d_q$ and a single event $e$ of deterministic length $d_e$ is
\begin{equation}
\label{eq:expected_iou}
    f(d_q) = \E_{\rt \sim p}\left[F(e_{\rt}, q, \gamma)\right] = \begin{cases}
    \frac{d_{e} \left(2 \gamma d_{q} - 2 \gamma^2 d_{e} + d_{q}\right)}{d_{q} \left(d_{e} + d_{q}\right)}, & \text{ if } d_q \geq \gamma d_e, \\
    \frac{d_q}{d_e + d_q}, & \text{ if } d_q < \gamma d_e,
    \end{cases}
\end{equation}
when the presence criterion for the annotator is $\gamma$.
\end{theorem}
\begin{proof} See Appendix~\ref{app:thm1} for the proof. We show how to express the area $A$ in Eq.~\ref{eq:normalized_area} in terms of $d_e$, $d_q$ and $\gamma$ for the two assumptions: $d_q \geq \gamma d_e$, and $d_q < \gamma d_e$.
\end{proof}

%We use Theorem~\ref{thm:expected_iou} to prove Theorem~\ref{thm:fix_optimal_query_length}, which show the optimal query segment length to maximize the expected label accuracy. 

\begin{theorem}
\label{thm:fix_optimal_query_length}
The query length that maximizes the expected label accuracy in case of overlap for a given event length $d_e$ is 
\begin{equation}
\label{eq:fix_optimal_query_length}
    d_q^* = d_e\gamma\frac{2\gamma + \sqrt{4\gamma^2 + 4\gamma + 2}}{2\gamma + 1}.
\end{equation}
\end{theorem}
\begin{proof}
See Appendix~\ref{app:thm2} for the proof. We compute the derivative of $f(d_q)$ with respect to $d_q$, and show that $d_q^*$ is the maximum.
\end{proof}

%By inserting $d_q^*$ into Eq.~\ref{eq:expected_iou} we get an expression for the maximum expected label accuracy as a function of $\gamma$, which leads to the next theorem.

\begin{theorem}
\label{thm:max_iou}
The maximum expected label accuracy in case of overlap between a query segment of length $d_q$ and an event of length $d_e$ when $d_q \geq \gamma d_e$ is
\begin{equation}
    \label{eq:max_iou}
    f^*(\gamma) = f(d_q^*) = 2\gamma\left(2\gamma + 1 - \sqrt{4\gamma^2+4\gamma + 2} \right) + 1
\end{equation}
\end{theorem}
\begin{proof}
See Appendix~\ref{app:thm3} for the proof. We substitute $d_q$ for $d_q^*$ in Eq.~\ref{eq:expected_iou}.
\end{proof}

%Finally, we use $d_q^*$ to compute how many query segments are needed to maximize the label accuracy for an audio recording of length $T$. 

\begin{theorem}
\label{thm:fix_number_of_queries}
The number of queries $B^*_{\text{FIX}}$ (cost) that are needed by FIX to maximize the expected label accuracy in case of overlap for an audio recording of length $T$ when $d_e=1$ is 
\begin{equation}
\label{eq:b_fix_1}
    B^*_{\text{FIX}} = \frac{T}{d_q^*}.
\end{equation}
\end{theorem}
\begin{proof}
$T/B^*_{\text{FIX}} = d_q^*$, which by Theorem~\ref{thm:fix_optimal_query_length} leads to maximum label accuracy.
\end{proof}


%\textbf{TODO: explain somewhere what each theorem means, and why it is useful information.}
%\paragraph{Summary:}
%\subsection{Summary and more complex data distributions}
%\label{sec:applicability_fix_label_accuracy}
%To summarize, in Theorem~\ref{thm:expected_iou} we derived an expression, $f(d_q)$, for the expected label accuracy of FIX weak labeling. 

%The expression is derived for fixed length segments of length $d_q$ using an annotator model with presence criterion $\gamma$ and assuming a data distribution where recordings are of finite length and only contain a single presence event of deterministic event length $d_e$.

In summary, Theorem~\ref{thm:expected_iou} gives us an expression $f(d_q)$ for the expected label accuracy when query segments of length $d_q$ are used to detect events of length $d_e$ and the presence criterion for the annotator is $\gamma$. We use this to find the query segment length $d_q^*$ that maximize the expected label accuracy, leading to Theorem~\ref{thm:fix_optimal_query_length}. Theorem~\ref{thm:fix_optimal_query_length} show the query segment length $d_q^*$ that maximizes expected label accuracy for a given event length and annotator criterion. Further, by inserting $d_q^*$ into Theorem~\ref{thm:expected_iou}, $f^*(\gamma) = f(d_q^*)$, we get Theorem~\ref{thm:max_iou}, which is the maximum achievable expected label accuracy for a given annotator criterion $\gamma$. We have omitted the case $d_q < \gamma d_e$ when deriving $f^*(\gamma)$, since maximizing the expected label accuracy in the case when the annotator presence criterion can not be fulfilled is not very interesting, since we can not get presence labels. Note that $f^*(\gamma)$ is a function of only $\gamma$, meaning that the maximum expected label accuracy is independent of the target event length when considering a single deterministic event. Finally, Theorem~\ref{thm:fix_number_of_queries} show that an annotator needs to weakly label $B^*_{\text{FIX}}$ query segments for each audio recording to achieve the maximum label accuracy in expectation, which can be seen as a proxy for annotation cost.

There is arguably no simpler audio data distribution to annotate than when recordings only contain a single event of deterministic length (except for when no event occurs at all). We can therefore treat $f^*(\gamma)$ as an upper bound on the maximum expected label accuracy for any audio distribution. We demonstrate this empirically in the results in Section~\ref{sec:results}. However, in practice audio recordings often contain events that vary both in length and number. Let us therefore consider how the derived theory can be useful also in these cases.


%Theorem~\ref{thm:fix_number_of_queries} tells us the number of query segments needed to maximize the expected label accuracy when FIX labeling audio recordings of length $T$ with a single randomly occurring event of length $d_e$. $B^*_{\text{FIX}}$ tells us how many segments the annotator has to weakly label, which is a proxy for annotation cost.

\subsection{Stochastic Event Length}
\label{sec:theory_event_distribution}
Events may vary in length according to some event length distribution. Let $p(d_e)$ denote the probability of the outcome that an event has length $d_e$, and let $d_e\sim p(d_e)$ denote that $d_e$ is a sample from that distribution. The expected label accuracy over a distribution of event lengths for a given $\gamma$ and query segment length $d_q$ can then be computed as
\begin{align}
\E_{d_e\sim p(d_e)}\left[f(d_q)\right]
\label{eq:expected_query_iou_distribution}
&= \int_{0}^{\infty} f(d_q)p(d_e) \mathrm{d}d_e.
\end{align}
While we do not provide a closed form solution for this, we can solve the integral in Eq.~\ref{eq:expected_query_iou_distribution} by numerical integration. Note that $d_q^*$ in Theorem~\ref{thm:fix_optimal_query_length} depends on the single event length $d_e$, and to find it for a distribution we would need to solve Eq.~\ref{eq:expected_query_iou_distribution} for a range of $d_q$ and find the one that leads to the best label accuracy. However, for some event length distributions, setting $d_e$ to the average of the distribution turns out to be a good heuristic. We perform a simulation study in Section~\ref{sec:event_distribution} to support these claims.

%We study the effect of the event length distribution on the maximum expected label accuracy and the optimal query length. 


\subsection{Multiple Events}
\label{sec:theory_multi_events}
There may be multiple ($M$) events present in a given audio recording. In Figure~\ref{fig:multiple_events} we show the label accuracy for all possible occurrences of a query segment $q_t$ in a recording with two events ($M=2$). Note that we have put the subscript $t$ on the query segment ($q_t$) instead of the event as in the prior analysis. This formulation is entirely equivalent, but when talking about multiple events it is more intuitive to consider them as fixed in time for a given recording, and that the query segments occur relative them at random. There are now two regions where overlap occurs, one around $e_1$ and one around $e_2$. On average we get $2A/2(d_e + d_q) = A/(d_e + d_q) = f(d_q)$. That is, the theory we derived for the single event case explains the multiple event case.

\begin{figure}
    \centering
    \includegraphics[width=0.8\linewidth]{figures/occurences_multi_events.png}
    \caption{
    \textit{Top panel:}  Two events ($M=2$) of length $d_e$ that are fixed in time within a recording of length $T$, and a query segment $q_t = (-d_q + t, t)$. \textit{Bottom panel:} The resulting label accuracy of $q_t$ for $t\in [0, T-d_q]$, simulating that the $q_t$ can appear anywhere at random in time in relation to the events. As before, when there is overlap between the query segment and an event the label accuracy is below $1$, otherwise it is always $1$.
    }
    \label{fig:multiple_events}
\end{figure}

However, for this to hold we need to assume that for any event the closest other event is least $d_q$ away in time. In Figure~\ref{fig:multiple_events} this holds since the start of $e_2$ is at least $d_q$ away from the end of $e_1$. If this assumption holds then the expected label accuracy for multiple events is $f(d_q)$. The assumption is plausible if events are sparse in relation to $d_q$. Note that $d_q^* \in (0, d_e\frac{2 + \sqrt{10}}{3}]$ for $\gamma \in (0, 1]$ according to Theorem~\ref{thm:fix_optimal_query_length}. That is, when considering the optimal query length $d_q^*$ this assumption translates to that events should be no closer than approximately $1.72d_e$ for $\gamma = 1$, $0.81d_e$ for $\gamma = 0.5$, and $0$ for $\gamma \rightarrow 0$. We perform a simulation study in Section~\ref{sec:multi_events} to see the effect of breaking this assumption, and we leave it to future work to derive the expected label accuracy in case of overlap for multiple events.

\subsection{The Expected Label Accuracy of an Audio Recording}
\label{sec:expected_label_accuracy_all_cases}
We now know the expected label accuracy of a query segment given event overlap, and how to use this for a stochastic event lengths and multiple events. We can use this to derive an expression for the expected label accuracy of and audio recording of finite length ($T$) that has multiple ($M$) stochastic event lengths ($d_e\sim p(d_e)$).

\begin{theorem}
\label{thm:label_accuracy}
The expected label accuracy for an audio recording of length $T$, with $M$ events of stochastic event length $d_e \sim p(d_e)$ that are spaced at least $d_q$ apart is
\begin{equation}
\label{eq:label_accuracy_recording}
    \E_{d_e\sim p(d_e)}\left[- \frac{2 M d_{e}^{2} \gamma^{2}}{T d_{q}} + \frac{2 M d_{e} \gamma}{T} - \frac{M d_{q}}{T} + 1 \right].
\end{equation}
\begin{proof}
We will do this proof by picture. In Figure~\ref{fig:multiple_events} we have two events ($M=2$), in general for $M$ events the accumulated label accuracy in the cases of overlap is $MA$ (the sum of the hatched red areas), the total amount of overlapping cases is $M(d_e + d_q)$ and the total amount of non-overlapping cases is therefore $T-M(d_e+d_q)$ for an audio recording of length $T$. In the case of no overlap, the label accuracy is always $1$, which means that the accumulated label accuracy in the case of no overlap (sum of the green hatched areas) is $T-M(d_e + d_q)$. Normalizing for the entire duration of the recording we arrive at
\begin{equation}
\frac{AM + T-M(d_e + d_q)}{T} = - \frac{2 M d_{e}^{2} \gamma^{2}}{T d_{q}} + \frac{2 M d_{e} \gamma}{T} - \frac{M d_{q}}{T} + 1,
\end{equation}
and as before we can simply compute an expectation over the event length distribution.
\end{proof}
\end{theorem}

Theorem~\ref{thm:label_accuracy} tells us the expected label accuracy under FIX weak labeling with query segment length $d_q$ for an audio recording of length $T$, with $M$ events of stochastic event length $d_e \sim p(d_e)$. If we want to account for class label noise, where the annotator gives the wrong label with probability $\rho$, this can be included by simply scaling the whole expression in Eq.~\ref{eq:label_accuracy_recording} by $(1-\rho)$. That is, the expected label accuracy for the cases of overlap allows us to express a variety of things about the expected label accuracy of an audio recording. 

However, note that we have $T$ in the denominator of all terms except the term that is $1$, meaning that if we let $T$ approach $\infty$, then the expected label accuracy approaches $1$. That is, considering the accuracy of both absence and presence labels equally can lead to hiding the effect that we want to understand in this paper, which is the effect of $d_q$ on the accuracy of the presence labels. We could derive a balanced accuracy in a similar way as above, but instead we choose to continue our analysis looking only at the expected label accuracy in the case of overlap. %We leave it to future work to extend the analysis in these directions. 


\subsection{Expected Label Accuracy given Overlap when $d_q = \delta d_e$}
\label{sec:ratio_based}
As a result of the proof for Theorem~\ref{thm:max_iou} in Appendix~\ref{app:thm3} we get an alternative dimensionless interpretation of the expected label accuracy when the query segment length is expressed as a factor of the event $d_q = \delta d_e$,
\begin{equation}
    f(\delta d_e) = \frac{(2\gamma + 1)\delta - 2\gamma^2}{\gamma(1+\gamma)},
\end{equation}
and an expression for the ratio that maximizes it
\begin{equation}
    \delta^* = \frac{d_q^*}{d_e} = \gamma \frac{2\gamma + \sqrt{2\gamma^2 + 2\gamma + 1}}{2\gamma + 1}.
\end{equation}
This alternative formulation illustrates that it is the ratio $\delta = d_q/d_e$ that affects the expected label accuracy of a single event, and not the absolute lengths $d_q$ and $d_e$. Further, we can use this interpretation to rewrite Theorem~\ref{thm:label_accuracy} as
\begin{equation}
    \E_{\delta \sim p(\delta)} \left[\frac{M d \delta \left(- \delta + 2 \gamma\right) - 2 M d \gamma^{2} + T \delta}{T \delta}\right],
\end{equation}
where $\delta$ denotes a random variable with probability distribution $p(\delta)$. %It may be less intuitive to sample ratios $\delta = d_q/d_e$ from a distribution, but it does provide an alternative view.