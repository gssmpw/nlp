\section{Simulating the Label Accuracy of FIX Weak Labeling}
\label{sec:simulation}

To validate the theory, we simulated FIX labeling of various audio recording distributions and compared the average simulated label quality with the theoretical results from Section~\ref{sec:expected_label_accuracy_given_overlap}. The code used for these simulations is provided in the supplementary material.

We generated 1000 audio recordings of length $T=100$ seconds for each configuration. The number of events, $M$, and the event length distributions varied across simulations, as detailed below:

\begin{itemize}
    \item \textbf{Single Event with Deterministic Length:} We simulated recordings with $M=1$ event of deterministic length $d_e = 1$ second.

    \item \textbf{Single Event with Stochastic Length from Normal Distributions:} We drew event lengths from two normal distributions with the same mean but different variances ($\mathcal{N}(3, 0.1)$ and $\mathcal{N}(3, 1)$), and from two normal distributions with different means but the same variance ($\mathcal{N}(0.5, 0.1)$ and $\mathcal{N}(5, 0.1)$). For these simulations, $M=1$.
    
    \item \textbf{Single Event with Stochastic Length from Gamma Distributions:} We sample event lengths from two gamma distributions (offset by $0.5$ seconds due to computation cost) with different shape parameters but the same scale parameter ($\text{Gamma}(0.8, 1) + 0.5$ and $\text{Gamma}(0.2, 1) + 0.5$) with $M=1$.
    
    \item \textbf{Single Event with Stochastic Length from Real Length Sample:} We used the event length distributions for dog barks and baby cries from the NIGENS dataset~\citep{Trowitzsch2019} with $M=1$.
    
    \item \textbf{Multiple Events with Deterministic Length:}  We simulated recordings with multiple events ($M=30$ and $M=50$) where each event had a deterministic length of $d_e = 1$ second.
\end{itemize}

For recordings with stochastic event lengths or multiple events, the length of each of the $M$ events was sampled from the specified distribution. Each sampled event was then placed randomly within the recording. The start time $a_e$ of each event was drawn uniformly at random from $[0, T - d_e]$. If multiple events were present, overlapping events were merged into one presence event. For each generated audio recording, we simulated FIX labeling using different annotator presence criteria $\gamma \in [0.01, 0.99]$ and a range of query segment lengths $d_q$. The query segment lengths were linearly spaced between a small fraction of the minimum event length observed in the distribution and a value several times the maximum observed event length. 

We then computed the average label accuracy over the query segments that overlaps with an event in each recording. For each query segment $q$ we check if the annotator presence criterion ($h(e, q) \geq \gamma$) is fulfilled for any event $e \in E$, where $E$ is the set of all events that overlap with $q$. If this is true for any of the events then $q$ is given a presence label ($l_q = 1$) otherwise it is given an absence label ($l_q = 0$). The label accuracy is then computed in a similar way as in Eq.~\ref{eq:query_iou}, but since we can now have multiple events overlapping with the same query segment, we need to consider the union of all overlapping events $\cup_{e\in E}e$ when computing the label accuracy of assigning label $l_q$ to that query segment. The total amount of overlap becomes $|(\cup_{e\in E} e) \cap q|$ instead of $|e \cap q|$. However, when $M=1$ this is equivalent to Eq.~\ref{eq:query_iou} ($|(\cup_{e\in E} e) \cap q| = |e \cap q|$), since $|E| = 1$.

%In this way, we simulate the effect of breaking the assumption that events are spaced at least $d_q$ apart, and can better understand the effect this has when compared to the derived theory. Finally, for each considered $\gamma$, we empirically determined the maximum average label accuracy across all tested query lengths and the corresponding optimal query length. These empirical results were then compared to the theoretical predictions.

In this way, we simulated the effect of breaking the assumption that events are spaced at least $d_q$ apart, and could better understand the effect this had when compared to the derived theory. Finally, for each considered $\gamma$, we empirically determined the maximum average label accuracy across all tested query lengths and the corresponding optimal query length. These empirical results were then compared to the theoretical predictions.

%In this way, we simulate the effect of breaking the assumption that events are spaced at least $d_q$ apart, and can better understand the effect this has when compared to the derived theory. Finally, for each considered $\gamma$, we empirically determine the maximum average label accuracy across all tested query lengths and the corresponding optimal query length. These empirical results are then compared to the theoretical predictions.

%This formulation of label accuracy is different from that used to derive the theory when we allow multiple events. During simulation, we also consider the case where multiple events can overlap with the same query segment, meaning that we need to consider the contribution of all the events that overlap with a query segment to compute accuracy of the label $l_q$. In the single event case, this formulation of label accuracy is equivalent with Eq.~\ref{eq:query_iou}. 


%Our definition of label accuracy emphasizes resolving individual presence events. For a recording of length $T$ containing $M$ non-overlapping presence events of average duration $d_e$, our metric reflects the accuracy of labeling each individual event. For example, if a query correctly identifies one event, its accuracy contribution is related to the duration of that event relative to the recording length. This differs from conventional accuracy metrics which might compute accuracy as $(M d_e) / T$ if the entire recording were labeled as containing presence. Our approach is crucial for evaluating the ability to resolve and correctly label individual events, particularly important for evaluation labels where high accuracy should indicate the precise detection of all events.