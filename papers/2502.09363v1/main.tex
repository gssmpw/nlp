\documentclass[10pt]{article} % For LaTeX2e


% If accepted, instead use the following line for the camera-ready submission:
%\usepackage[accepted]{tmlr}
% To de-anonymize and remove mentions to TMLR (for example for posting to preprint servers), instead use the following:
\usepackage[preprint]{tmlr}
%\usepackage[]{tmlr}

\usepackage{graphicx}
\usepackage{float}
\usepackage{amsmath}
\usepackage{amssymb}
\usepackage{amsthm}
\usepackage{soul}
\usepackage{cancel}



% Define theorem styles and environments
\theoremstyle{definition}
\newtheorem{assumption}{Assumption}
\newtheorem{case}{Case}
\newtheorem{lemma}{Lemma}
\newtheorem{theorem}{Theorem}
\newtheorem{definition}{Definition}
\newtheorem{claim}{Claim}

\theoremstyle{remark}
\newtheorem*{proof*}{Proof}


% Optional math commands from https://github.com/goodfeli/dlbook_notation.
%%%%% NEW MATH DEFINITIONS %%%%%

\usepackage{amsmath,amsfonts,bm}
\usepackage{derivative}
% Mark sections of captions for referring to divisions of figures
\newcommand{\figleft}{{\em (Left)}}
\newcommand{\figcenter}{{\em (Center)}}
\newcommand{\figright}{{\em (Right)}}
\newcommand{\figtop}{{\em (Top)}}
\newcommand{\figbottom}{{\em (Bottom)}}
\newcommand{\captiona}{{\em (a)}}
\newcommand{\captionb}{{\em (b)}}
\newcommand{\captionc}{{\em (c)}}
\newcommand{\captiond}{{\em (d)}}

% Highlight a newly defined term
\newcommand{\newterm}[1]{{\bf #1}}

% Derivative d 
\newcommand{\deriv}{{\mathrm{d}}}

% Figure reference, lower-case.
\def\figref#1{figure~\ref{#1}}
% Figure reference, capital. For start of sentence
\def\Figref#1{Figure~\ref{#1}}
\def\twofigref#1#2{figures \ref{#1} and \ref{#2}}
\def\quadfigref#1#2#3#4{figures \ref{#1}, \ref{#2}, \ref{#3} and \ref{#4}}
% Section reference, lower-case.
\def\secref#1{section~\ref{#1}}
% Section reference, capital.
\def\Secref#1{Section~\ref{#1}}
% Reference to two sections.
\def\twosecrefs#1#2{sections \ref{#1} and \ref{#2}}
% Reference to three sections.
\def\secrefs#1#2#3{sections \ref{#1}, \ref{#2} and \ref{#3}}
% Reference to an equation, lower-case.
\def\eqref#1{equation~\ref{#1}}
% Reference to an equation, upper case
\def\Eqref#1{Equation~\ref{#1}}
% A raw reference to an equation---avoid using if possible
\def\plaineqref#1{\ref{#1}}
% Reference to a chapter, lower-case.
\def\chapref#1{chapter~\ref{#1}}
% Reference to an equation, upper case.
\def\Chapref#1{Chapter~\ref{#1}}
% Reference to a range of chapters
\def\rangechapref#1#2{chapters\ref{#1}--\ref{#2}}
% Reference to an algorithm, lower-case.
\def\algref#1{algorithm~\ref{#1}}
% Reference to an algorithm, upper case.
\def\Algref#1{Algorithm~\ref{#1}}
\def\twoalgref#1#2{algorithms \ref{#1} and \ref{#2}}
\def\Twoalgref#1#2{Algorithms \ref{#1} and \ref{#2}}
% Reference to a part, lower case
\def\partref#1{part~\ref{#1}}
% Reference to a part, upper case
\def\Partref#1{Part~\ref{#1}}
\def\twopartref#1#2{parts \ref{#1} and \ref{#2}}

\def\ceil#1{\lceil #1 \rceil}
\def\floor#1{\lfloor #1 \rfloor}
\def\1{\bm{1}}
\newcommand{\train}{\mathcal{D}}
\newcommand{\valid}{\mathcal{D_{\mathrm{valid}}}}
\newcommand{\test}{\mathcal{D_{\mathrm{test}}}}

\def\eps{{\epsilon}}


% Random variables
\def\reta{{\textnormal{$\eta$}}}
\def\ra{{\textnormal{a}}}
\def\rb{{\textnormal{b}}}
\def\rc{{\textnormal{c}}}
\def\rd{{\textnormal{d}}}
\def\re{{\textnormal{e}}}
\def\rf{{\textnormal{f}}}
\def\rg{{\textnormal{g}}}
\def\rh{{\textnormal{h}}}
\def\ri{{\textnormal{i}}}
\def\rj{{\textnormal{j}}}
\def\rk{{\textnormal{k}}}
\def\rl{{\textnormal{l}}}
% rm is already a command, just don't name any random variables m
\def\rn{{\textnormal{n}}}
\def\ro{{\textnormal{o}}}
\def\rp{{\textnormal{p}}}
\def\rq{{\textnormal{q}}}
\def\rr{{\textnormal{r}}}
\def\rs{{\textnormal{s}}}
\def\rt{{\textnormal{t}}}
\def\ru{{\textnormal{u}}}
\def\rv{{\textnormal{v}}}
\def\rw{{\textnormal{w}}}
\def\rx{{\textnormal{x}}}
\def\ry{{\textnormal{y}}}
\def\rz{{\textnormal{z}}}

% Random vectors
\def\rvepsilon{{\mathbf{\epsilon}}}
\def\rvphi{{\mathbf{\phi}}}
\def\rvtheta{{\mathbf{\theta}}}
\def\rva{{\mathbf{a}}}
\def\rvb{{\mathbf{b}}}
\def\rvc{{\mathbf{c}}}
\def\rvd{{\mathbf{d}}}
\def\rve{{\mathbf{e}}}
\def\rvf{{\mathbf{f}}}
\def\rvg{{\mathbf{g}}}
\def\rvh{{\mathbf{h}}}
\def\rvu{{\mathbf{i}}}
\def\rvj{{\mathbf{j}}}
\def\rvk{{\mathbf{k}}}
\def\rvl{{\mathbf{l}}}
\def\rvm{{\mathbf{m}}}
\def\rvn{{\mathbf{n}}}
\def\rvo{{\mathbf{o}}}
\def\rvp{{\mathbf{p}}}
\def\rvq{{\mathbf{q}}}
\def\rvr{{\mathbf{r}}}
\def\rvs{{\mathbf{s}}}
\def\rvt{{\mathbf{t}}}
\def\rvu{{\mathbf{u}}}
\def\rvv{{\mathbf{v}}}
\def\rvw{{\mathbf{w}}}
\def\rvx{{\mathbf{x}}}
\def\rvy{{\mathbf{y}}}
\def\rvz{{\mathbf{z}}}

% Elements of random vectors
\def\erva{{\textnormal{a}}}
\def\ervb{{\textnormal{b}}}
\def\ervc{{\textnormal{c}}}
\def\ervd{{\textnormal{d}}}
\def\erve{{\textnormal{e}}}
\def\ervf{{\textnormal{f}}}
\def\ervg{{\textnormal{g}}}
\def\ervh{{\textnormal{h}}}
\def\ervi{{\textnormal{i}}}
\def\ervj{{\textnormal{j}}}
\def\ervk{{\textnormal{k}}}
\def\ervl{{\textnormal{l}}}
\def\ervm{{\textnormal{m}}}
\def\ervn{{\textnormal{n}}}
\def\ervo{{\textnormal{o}}}
\def\ervp{{\textnormal{p}}}
\def\ervq{{\textnormal{q}}}
\def\ervr{{\textnormal{r}}}
\def\ervs{{\textnormal{s}}}
\def\ervt{{\textnormal{t}}}
\def\ervu{{\textnormal{u}}}
\def\ervv{{\textnormal{v}}}
\def\ervw{{\textnormal{w}}}
\def\ervx{{\textnormal{x}}}
\def\ervy{{\textnormal{y}}}
\def\ervz{{\textnormal{z}}}

% Random matrices
\def\rmA{{\mathbf{A}}}
\def\rmB{{\mathbf{B}}}
\def\rmC{{\mathbf{C}}}
\def\rmD{{\mathbf{D}}}
\def\rmE{{\mathbf{E}}}
\def\rmF{{\mathbf{F}}}
\def\rmG{{\mathbf{G}}}
\def\rmH{{\mathbf{H}}}
\def\rmI{{\mathbf{I}}}
\def\rmJ{{\mathbf{J}}}
\def\rmK{{\mathbf{K}}}
\def\rmL{{\mathbf{L}}}
\def\rmM{{\mathbf{M}}}
\def\rmN{{\mathbf{N}}}
\def\rmO{{\mathbf{O}}}
\def\rmP{{\mathbf{P}}}
\def\rmQ{{\mathbf{Q}}}
\def\rmR{{\mathbf{R}}}
\def\rmS{{\mathbf{S}}}
\def\rmT{{\mathbf{T}}}
\def\rmU{{\mathbf{U}}}
\def\rmV{{\mathbf{V}}}
\def\rmW{{\mathbf{W}}}
\def\rmX{{\mathbf{X}}}
\def\rmY{{\mathbf{Y}}}
\def\rmZ{{\mathbf{Z}}}

% Elements of random matrices
\def\ermA{{\textnormal{A}}}
\def\ermB{{\textnormal{B}}}
\def\ermC{{\textnormal{C}}}
\def\ermD{{\textnormal{D}}}
\def\ermE{{\textnormal{E}}}
\def\ermF{{\textnormal{F}}}
\def\ermG{{\textnormal{G}}}
\def\ermH{{\textnormal{H}}}
\def\ermI{{\textnormal{I}}}
\def\ermJ{{\textnormal{J}}}
\def\ermK{{\textnormal{K}}}
\def\ermL{{\textnormal{L}}}
\def\ermM{{\textnormal{M}}}
\def\ermN{{\textnormal{N}}}
\def\ermO{{\textnormal{O}}}
\def\ermP{{\textnormal{P}}}
\def\ermQ{{\textnormal{Q}}}
\def\ermR{{\textnormal{R}}}
\def\ermS{{\textnormal{S}}}
\def\ermT{{\textnormal{T}}}
\def\ermU{{\textnormal{U}}}
\def\ermV{{\textnormal{V}}}
\def\ermW{{\textnormal{W}}}
\def\ermX{{\textnormal{X}}}
\def\ermY{{\textnormal{Y}}}
\def\ermZ{{\textnormal{Z}}}

% Vectors
\def\vzero{{\bm{0}}}
\def\vone{{\bm{1}}}
\def\vmu{{\bm{\mu}}}
\def\vtheta{{\bm{\theta}}}
\def\vphi{{\bm{\phi}}}
\def\va{{\bm{a}}}
\def\vb{{\bm{b}}}
\def\vc{{\bm{c}}}
\def\vd{{\bm{d}}}
\def\ve{{\bm{e}}}
\def\vf{{\bm{f}}}
\def\vg{{\bm{g}}}
\def\vh{{\bm{h}}}
\def\vi{{\bm{i}}}
\def\vj{{\bm{j}}}
\def\vk{{\bm{k}}}
\def\vl{{\bm{l}}}
\def\vm{{\bm{m}}}
\def\vn{{\bm{n}}}
\def\vo{{\bm{o}}}
\def\vp{{\bm{p}}}
\def\vq{{\bm{q}}}
\def\vr{{\bm{r}}}
\def\vs{{\bm{s}}}
\def\vt{{\bm{t}}}
\def\vu{{\bm{u}}}
\def\vv{{\bm{v}}}
\def\vw{{\bm{w}}}
\def\vx{{\bm{x}}}
\def\vy{{\bm{y}}}
\def\vz{{\bm{z}}}

% Elements of vectors
\def\evalpha{{\alpha}}
\def\evbeta{{\beta}}
\def\evepsilon{{\epsilon}}
\def\evlambda{{\lambda}}
\def\evomega{{\omega}}
\def\evmu{{\mu}}
\def\evpsi{{\psi}}
\def\evsigma{{\sigma}}
\def\evtheta{{\theta}}
\def\eva{{a}}
\def\evb{{b}}
\def\evc{{c}}
\def\evd{{d}}
\def\eve{{e}}
\def\evf{{f}}
\def\evg{{g}}
\def\evh{{h}}
\def\evi{{i}}
\def\evj{{j}}
\def\evk{{k}}
\def\evl{{l}}
\def\evm{{m}}
\def\evn{{n}}
\def\evo{{o}}
\def\evp{{p}}
\def\evq{{q}}
\def\evr{{r}}
\def\evs{{s}}
\def\evt{{t}}
\def\evu{{u}}
\def\evv{{v}}
\def\evw{{w}}
\def\evx{{x}}
\def\evy{{y}}
\def\evz{{z}}

% Matrix
\def\mA{{\bm{A}}}
\def\mB{{\bm{B}}}
\def\mC{{\bm{C}}}
\def\mD{{\bm{D}}}
\def\mE{{\bm{E}}}
\def\mF{{\bm{F}}}
\def\mG{{\bm{G}}}
\def\mH{{\bm{H}}}
\def\mI{{\bm{I}}}
\def\mJ{{\bm{J}}}
\def\mK{{\bm{K}}}
\def\mL{{\bm{L}}}
\def\mM{{\bm{M}}}
\def\mN{{\bm{N}}}
\def\mO{{\bm{O}}}
\def\mP{{\bm{P}}}
\def\mQ{{\bm{Q}}}
\def\mR{{\bm{R}}}
\def\mS{{\bm{S}}}
\def\mT{{\bm{T}}}
\def\mU{{\bm{U}}}
\def\mV{{\bm{V}}}
\def\mW{{\bm{W}}}
\def\mX{{\bm{X}}}
\def\mY{{\bm{Y}}}
\def\mZ{{\bm{Z}}}
\def\mBeta{{\bm{\beta}}}
\def\mPhi{{\bm{\Phi}}}
\def\mLambda{{\bm{\Lambda}}}
\def\mSigma{{\bm{\Sigma}}}

% Tensor
\DeclareMathAlphabet{\mathsfit}{\encodingdefault}{\sfdefault}{m}{sl}
\SetMathAlphabet{\mathsfit}{bold}{\encodingdefault}{\sfdefault}{bx}{n}
\newcommand{\tens}[1]{\bm{\mathsfit{#1}}}
\def\tA{{\tens{A}}}
\def\tB{{\tens{B}}}
\def\tC{{\tens{C}}}
\def\tD{{\tens{D}}}
\def\tE{{\tens{E}}}
\def\tF{{\tens{F}}}
\def\tG{{\tens{G}}}
\def\tH{{\tens{H}}}
\def\tI{{\tens{I}}}
\def\tJ{{\tens{J}}}
\def\tK{{\tens{K}}}
\def\tL{{\tens{L}}}
\def\tM{{\tens{M}}}
\def\tN{{\tens{N}}}
\def\tO{{\tens{O}}}
\def\tP{{\tens{P}}}
\def\tQ{{\tens{Q}}}
\def\tR{{\tens{R}}}
\def\tS{{\tens{S}}}
\def\tT{{\tens{T}}}
\def\tU{{\tens{U}}}
\def\tV{{\tens{V}}}
\def\tW{{\tens{W}}}
\def\tX{{\tens{X}}}
\def\tY{{\tens{Y}}}
\def\tZ{{\tens{Z}}}


% Graph
\def\gA{{\mathcal{A}}}
\def\gB{{\mathcal{B}}}
\def\gC{{\mathcal{C}}}
\def\gD{{\mathcal{D}}}
\def\gE{{\mathcal{E}}}
\def\gF{{\mathcal{F}}}
\def\gG{{\mathcal{G}}}
\def\gH{{\mathcal{H}}}
\def\gI{{\mathcal{I}}}
\def\gJ{{\mathcal{J}}}
\def\gK{{\mathcal{K}}}
\def\gL{{\mathcal{L}}}
\def\gM{{\mathcal{M}}}
\def\gN{{\mathcal{N}}}
\def\gO{{\mathcal{O}}}
\def\gP{{\mathcal{P}}}
\def\gQ{{\mathcal{Q}}}
\def\gR{{\mathcal{R}}}
\def\gS{{\mathcal{S}}}
\def\gT{{\mathcal{T}}}
\def\gU{{\mathcal{U}}}
\def\gV{{\mathcal{V}}}
\def\gW{{\mathcal{W}}}
\def\gX{{\mathcal{X}}}
\def\gY{{\mathcal{Y}}}
\def\gZ{{\mathcal{Z}}}

% Sets
\def\sA{{\mathbb{A}}}
\def\sB{{\mathbb{B}}}
\def\sC{{\mathbb{C}}}
\def\sD{{\mathbb{D}}}
% Don't use a set called E, because this would be the same as our symbol
% for expectation.
\def\sF{{\mathbb{F}}}
\def\sG{{\mathbb{G}}}
\def\sH{{\mathbb{H}}}
\def\sI{{\mathbb{I}}}
\def\sJ{{\mathbb{J}}}
\def\sK{{\mathbb{K}}}
\def\sL{{\mathbb{L}}}
\def\sM{{\mathbb{M}}}
\def\sN{{\mathbb{N}}}
\def\sO{{\mathbb{O}}}
\def\sP{{\mathbb{P}}}
\def\sQ{{\mathbb{Q}}}
\def\sR{{\mathbb{R}}}
\def\sS{{\mathbb{S}}}
\def\sT{{\mathbb{T}}}
\def\sU{{\mathbb{U}}}
\def\sV{{\mathbb{V}}}
\def\sW{{\mathbb{W}}}
\def\sX{{\mathbb{X}}}
\def\sY{{\mathbb{Y}}}
\def\sZ{{\mathbb{Z}}}

% Entries of a matrix
\def\emLambda{{\Lambda}}
\def\emA{{A}}
\def\emB{{B}}
\def\emC{{C}}
\def\emD{{D}}
\def\emE{{E}}
\def\emF{{F}}
\def\emG{{G}}
\def\emH{{H}}
\def\emI{{I}}
\def\emJ{{J}}
\def\emK{{K}}
\def\emL{{L}}
\def\emM{{M}}
\def\emN{{N}}
\def\emO{{O}}
\def\emP{{P}}
\def\emQ{{Q}}
\def\emR{{R}}
\def\emS{{S}}
\def\emT{{T}}
\def\emU{{U}}
\def\emV{{V}}
\def\emW{{W}}
\def\emX{{X}}
\def\emY{{Y}}
\def\emZ{{Z}}
\def\emSigma{{\Sigma}}

% entries of a tensor
% Same font as tensor, without \bm wrapper
\newcommand{\etens}[1]{\mathsfit{#1}}
\def\etLambda{{\etens{\Lambda}}}
\def\etA{{\etens{A}}}
\def\etB{{\etens{B}}}
\def\etC{{\etens{C}}}
\def\etD{{\etens{D}}}
\def\etE{{\etens{E}}}
\def\etF{{\etens{F}}}
\def\etG{{\etens{G}}}
\def\etH{{\etens{H}}}
\def\etI{{\etens{I}}}
\def\etJ{{\etens{J}}}
\def\etK{{\etens{K}}}
\def\etL{{\etens{L}}}
\def\etM{{\etens{M}}}
\def\etN{{\etens{N}}}
\def\etO{{\etens{O}}}
\def\etP{{\etens{P}}}
\def\etQ{{\etens{Q}}}
\def\etR{{\etens{R}}}
\def\etS{{\etens{S}}}
\def\etT{{\etens{T}}}
\def\etU{{\etens{U}}}
\def\etV{{\etens{V}}}
\def\etW{{\etens{W}}}
\def\etX{{\etens{X}}}
\def\etY{{\etens{Y}}}
\def\etZ{{\etens{Z}}}

% The true underlying data generating distribution
\newcommand{\pdata}{p_{\rm{data}}}
\newcommand{\ptarget}{p_{\rm{target}}}
\newcommand{\pprior}{p_{\rm{prior}}}
\newcommand{\pbase}{p_{\rm{base}}}
\newcommand{\pref}{p_{\rm{ref}}}

% The empirical distribution defined by the training set
\newcommand{\ptrain}{\hat{p}_{\rm{data}}}
\newcommand{\Ptrain}{\hat{P}_{\rm{data}}}
% The model distribution
\newcommand{\pmodel}{p_{\rm{model}}}
\newcommand{\Pmodel}{P_{\rm{model}}}
\newcommand{\ptildemodel}{\tilde{p}_{\rm{model}}}
% Stochastic autoencoder distributions
\newcommand{\pencode}{p_{\rm{encoder}}}
\newcommand{\pdecode}{p_{\rm{decoder}}}
\newcommand{\precons}{p_{\rm{reconstruct}}}

\newcommand{\laplace}{\mathrm{Laplace}} % Laplace distribution

\newcommand{\E}{\mathbb{E}}
\newcommand{\Ls}{\mathcal{L}}
\newcommand{\R}{\mathbb{R}}
\newcommand{\emp}{\tilde{p}}
\newcommand{\lr}{\alpha}
\newcommand{\reg}{\lambda}
\newcommand{\rect}{\mathrm{rectifier}}
\newcommand{\softmax}{\mathrm{softmax}}
\newcommand{\sigmoid}{\sigma}
\newcommand{\softplus}{\zeta}
\newcommand{\KL}{D_{\mathrm{KL}}}
\newcommand{\Var}{\mathrm{Var}}
\newcommand{\standarderror}{\mathrm{SE}}
\newcommand{\Cov}{\mathrm{Cov}}
% Wolfram Mathworld says $L^2$ is for function spaces and $\ell^2$ is for vectors
% But then they seem to use $L^2$ for vectors throughout the site, and so does
% wikipedia.
\newcommand{\normlzero}{L^0}
\newcommand{\normlone}{L^1}
\newcommand{\normltwo}{L^2}
\newcommand{\normlp}{L^p}
\newcommand{\normmax}{L^\infty}

\newcommand{\parents}{Pa} % See usage in notation.tex. Chosen to match Daphne's book.

\DeclareMathOperator*{\argmax}{arg\,max}
\DeclareMathOperator*{\argmin}{arg\,min}

\DeclareMathOperator{\sign}{sign}
\DeclareMathOperator{\Tr}{Tr}
\let\ab\allowbreak

%\section{Proofs}
\label{sec:appendix}




\usepackage{hyperref}
\usepackage{url}

%\newcommand{\githublink}[0]{\href{https://github.com/johnmartinsson/weak-labeling-theory}{github}}


%\title{Modeling the Label Accuracy and Cost of Weak Labeling Fixed Length Data Segments}
%\title{Modeling Label Accuracy and Cost in Weakly Labeling Fixed-Length Segments}
%\title{How Accurate and Cost-Effective is Fixed-Length Weak Labeling for Time-Series Events? A Theoretical Analysis}

\title{The Accuracy Cost of Weakness: A Theoretical Analysis of Fixed-Segment Weak Labeling for Events in Time}

% Authors must not appear in the submitted version. They should be hidden
% as long as the tmlr package is used without the [accepted] or [preprint] options.
% Non-anonymous submissions will be rejected without review.

\author{\name John Martinsson \email john.martinsson@ri.se \\
      \addr Computer Science \\
      RISE Research Institutes of Sweden \\
      \addr Centre for Mathematical Sciences \\
      Lund University \\
      \AND
      \name Olof Mogren \email olof.mogren@ri.se \\
      \addr Computer Science \\
      RISE Research Institutes of Sweden
      \AND
      \name Tuomas Virtanen \email tuomas.virtanen@tuni.fi\\
      \addr Signal Processing Research Centre \\
      Tampere University 
      \AND
      \name Maria Sandsten \email maria.sandsten@matstat.lu.se\\
      \addr Centre for Mathematical Sciences \\
      Lund University \\
      }

% The \author macro works with any number of authors. Use \AND 
% to separate the names and addresses of multiple authors.

\newcommand{\fix}{\marginpar{FIX}}
\newcommand{\new}{\marginpar{NEW}}

\def\month{MM}  % Insert correct month for camera-ready version
\def\year{YYYY} % Insert correct year for camera-ready version
\def\openreview{\url{https://openreview.net/forum?id=XXXX}} % Insert correct link to OpenReview for camera-ready version


\begin{document}


\maketitle

\begin{abstract}
Accurate labels are critical for deriving robust machine learning models. Labels are used to train supervised learning models and to evaluate most machine learning paradigms. In this paper, we model the accuracy and cost of a common weak labeling process where annotators assign presence or absence labels to fixed-length data segments for a given event class. The annotator labels a segment as "present" if it sufficiently covers an event from that class, e.g., a birdsong sound event in audio data.
%
We analyze how the segment length affects the label accuracy and the required number of annotations, and compare this fixed-length labeling approach with an oracle method that uses the true event activations to construct the segments. Furthermore, we quantify the gap between these methods and verify that in most realistic scenarios the oracle method is better than the fixed-length labeling method in both accuracy and cost. Our findings provide a theoretical justification for adaptive weak labeling strategies that mimic the oracle process, and a foundation for optimizing weak labeling processes in sequence labeling tasks.

\end{abstract}



\section{Introduction}
\IEEEPARstart{I}{n} recent years, flourishing of Artificial Intelligence Generated Content (AIGC) has sparked significant advancements in modalities such as text, image, audio, and even video. 
Among these, AI-Generated Image (AGI) has garnered considerable interest from both researchers and the public.
Plenty of remarkable AGI models and online services, such as StableDiffusion\footnote{\url{https://stability.ai/}}, Midjourney\footnote{\url{https://www.midjourney.com/}}, and FLUX\footnote{\url{https://blackforestlabs.ai/}}, offer users an excellent creative experience.
However, users often remain critical of the quality of the AGI due to image distortions or mismatches with user intentions.
Consequently, methods for assessing the quality of AGI are becoming increasingly crucial to help improve the generative capabilities of these models.

Unlike Natural Scene Image (NSI) quality assessment, which focuses primarily on perception aspects such as sharpness, color, and brightness, AI-Generated Image Quality Assessment (AGIQA) encompasses additional aspects like correspondence and authenticity. 
Since AGI is generated on the basis of user text prompts, it may fail to capture key user intentions, resulting in misalignment with the prompt.
Furthermore, authenticity refers to how closely the generated image resembles real-world artworks, as AGI can sometimes exhibit logical inconsistencies.
While traditional IQA models may effectively evaluate perceptual quality, they are often less capable of adequately assessing aspects such as correspondence and authenticity.

\begin{figure}\label{fig:radar}
    \centering
    \includegraphics[width=1.0\linewidth]{figures/radar_plot.pdf}
    \caption{A comparison on quality, correspondence, and authenticity aspects of AIGCIQA2023~\cite{wang2023aigciqa2023} dataset illustrates the superior performance of our method.}
\end{figure}

Several methods have been proposed specifically for the AGIQA task, including metrics designed to evaluate the authenticity and diversity of generated images~\cite{gulrajani2017improved,heusel2017gans}. 
Nevertheless, these methods tend to compare and evaluate grouped images rather than single instances, which limits their utility for single image assessment.
Beginning with AGIQA-1k~\cite{zhang2023perceptual}, a series of AGIQA databases have been introduced, including AGIQA-3k~\cite{li2023agiqa}, AIGCIQA-20k~\cite{li2024aigiqa}, etc.
Concurrently, there has been a surge in research utilizing deep learning methods~\cite{zhou2024adaptive,peng2024aigc,yu2024sf}, which have significantly benefited from pre-trained models such as CLIP~\cite{radford2021learning}. 
These approaches enhance the analysis by leveraging the correlations between images and their descriptive texts.
While these models are effective in capturing general text-image alignments, they may not effectively detect subtle inconsistencies or mismatches between the generated image content and the detailed nuances of the textual description.
Moreover, as these models are pre-trained on large-scale datasets for broad tasks, they might not fully exploit the textual information pertinent to the specific context of AGIQA without task-specific fine-tuning.
To overcome these limitations, methods that leverage Multimodal Large Language Models (MLLMs)~\cite{wang2024large,wang2024understanding} have been proposed.
These methods aim to fully exploit the synergies of image captioning and textual analysis for AGIQA.
Although they benefit from advanced prompt understanding, instruction following, and generation capabilities, they often do not utilize MLLMs as encoders capable of producing a sequence of logits that integrate both image and text context.

In conclusion, the field of AI-Generated Image Quality Assessment (AGIQA) continues to face significant challenges: 
(1) Developing comprehensive methods to assess AGIs from multiple dimensions, including quality, correspondence, and authenticity; 
(2) Enhancing assessment techniques to more accurately reflect human perception and the nuanced intentions embedded within prompts; 
(3) Optimizing the use of Multimodal Large Language Models (MLLMs) to fully exploit their multimodal encoding capabilities.

To address these challenges, we propose a novel method M3-AGIQA (\textbf{M}ultimodal, \textbf{M}ulti-Round, \textbf{M}ulti-Aspect AI-Generated Image Quality Assessment) which leverages MLLMs as both image and text encoders. 
This approach incorporates an additional network to align human perception and intentions, aiming to enhance assessment accuracy. 
Specially, we distill the rich image captioning capability from online MLLMs into a local MLLM through Low-Rank Adaption (LoRA) fine-tuning, and train this model with human-labeled data. The key contributions of this paper are as follows:
\begin{itemize}
    \item We propose a novel AGIQA method that distills multi-aspect image captioning capabilities to enable comprehensive evaluation. Specifically, we use an online MLLM service to generate aspect-specific image descriptions and fine-tune a local MLLM with these descriptions in a structured two-round conversational format.
    \item We investigate the encoding potential of MLLMs to better align with human perceptual judgments and intentions, uncovering previously underestimated capabilities of MLLMs in the AGIQA domain. To leverage sequential information, we append an xLSTM feature extractor and a regression head to the encoding output.
    \item Extensive experiments across multiple datasets demonstrate that our method achieves superior performance, setting a new state-of-the-art (SOTA) benchmark in AGIQA.
\end{itemize}

In this work, we present related works in Sec.~\ref{sec:related}, followed by the details of our M3-AGIQA method in Sec.~\ref{sec:method}. Sec.~\ref{sec:exp} outlines our experimental design and presents the results. Sec.~\ref{sec:limit},~\ref{sec:ethics} and~\ref{sec:conclusion} discuss the limitations, ethical concerns, future directions and conclusions of our study.
\section{Problem Setting}

%How can we effectively label time-localized sound events in real-world scenarios where annotators have limited time and expertise? This question motivates our study, where we model a data distribution and an annotator capable of weakly labeling the presence or absence of events within data segments. Our primary interest lies in evaluating the accuracy of these presence annotations under varying assumptions about the annotator model and the data distribution.

The analysis is framed within a multi-pass binary labeling setting. Here, an annotator assigns binary labels (presence or absence) to data segments based on the occurrence of specific sound events. The annotator model abstracts how an annotator interacts with data by labeling segments, without requiring precise knowledge of event boundaries. While inspired by time-localized and non-stationary sound events, this framework is generalizable to any time series with similar characteristics.

It's important to emphasize that, in this weak labeling setting, the concept of overlapping events is not explicitly modeled. Overlapping events from the same class are treated as a single, longer presence event, because presence/absence labels cannot differentiate between individual event instances. For instance, in an audio recording with two birds calling simultaneously, this weak labeling framework simplifies the overlap into a single 'present' event. While this simplification is necessary when studying weak labeling in this setting, it fundamentally restricts our ability to resolve polyphony (the identification of multiple overlapping sound events). We leave the exploration of annotator models capable of providing richer labels to future work; this is beyond the scope of our study.

%Note that, in the presence/absence labeling setting the concept of overlapping events is abstracted away. Overlapping events from the same class are treated as a single, longer presence event because presence/absence labels cannot differentiate between individual contributors. For instance, in an audio recording with two birds calling simultaneously, the framework simplifies the overlap into a single "present" event. While useful for studying weak labeling, this abstraction inherently limits the ability to resolve polyphony (multiple overlapping sound events). We leave it to future work to explore other annotator models capable of providing more detailed labels, but it is beyond the scope of this work.

\subsection{The Assumed Data Distribution}
\label{sec:label_distribution}

A sound event $e$ is defined by its start time \( a_e \in \mathbb{R} \), end time \( b_e \in \mathbb{R} \), and class \( c_e \in \mathcal{C} \), denoted as \( e = (a_e, b_e, c_e) \). Audio recordings are assumed to have finite length \( T \), and events are uniformly distributed over the recording. 

%A sound event is defined by its start time \( a_e \in \mathbb{R} \), end time \( b_e \in \mathbb{R} \), and class \( c_e \in \mathcal{C} \), denoted as \( e = (a_e, b_e, c_e) \) with event length \( d_e = b_e - a_e \). The event timings \( (a_e, b_e) \) are referred to as the onset and offset, while \( c_e \) is the event class. Audio recordings are assumed to have finite length \( T \), and events \( e \) are assumed uniformly distributed over the recording such that \( a_e \in [0, T - d_e] \). 

The uniform distribution reflects the assumption that events are equally likely to appear anywhere in the recording relative to the recording's start time. Consider a person who wants to record an event but does not know when the event will occur. We assume that they are equally likely to start recording at any time before this event.

\subsection{The Assumed Annotator Model}
\label{sec:annotator_model}

For a given sound event class \( c \in \mathcal{C} \), the annotator decides the presence or absence of an event $e$ of class \( c \) in a data segment \( q = (a_q, b_q) \), where \( d_q = b_q - a_q \) is the fixed-length of the segment. We will refer to $q$ as a query segment because it is queried for a presence or absence label. Let $l_q \in \{0, 1\}$ denote the weak label indicated by the annotator for query segment $q$, where $l_q = 1$ indicates presence of an event of class $c$ in $q$ and $l_q = 0$ indicates absence of that event class in $q$. Detecting the presence of an event requires observing a sufficient fraction of the event within the query segment, formalized as follows:

\begin{definition}
\label{def:event_fraction}
The \textit{event fraction} is the fraction of the total event duration \( d_e = b_e - a_e \) that overlaps with the query segment \( q \),
\begin{equation}
\label{eq:event_fraction}
    h(e, q) = \frac{|e \cap q|}{d_e},
\end{equation}
where \( e \cap q \) is the intersection of \( (a_e, b_e) \) and \( (a_q, b_q) \).
\end{definition}

\begin{definition}
\label{def:annotator_criterion}
The \textit{presence criterion} \( \gamma \in (0, 1] \) is the minimum event fraction required for the annotator to detect the presence of \( e \) in \( q \),
\begin{equation}
\label{eq:annotator_criterion}
    h(e, q) \geq \gamma.
\end{equation}
\end{definition}


The annotator assigns a presence label (\(l_q = 1\)) to \( q \) if there is sufficient overlap with any presence event $e$ of class $c$ (\( h(e, q) \geq \gamma \)); otherwise, it assigns an absence label (\(l_q=0\)). The parameter \( \gamma \) reflects the annotator's sensitivity: lower \( \gamma \) values indicate sensitivity to smaller event fractions, while higher values require larger fractions. 

This framework captures variability in annotator behavior. For example, detecting "human speech" or "bird song" may only require hearing a small fraction of the event (\( \gamma \) closer to 0), while recognizing specific phrases or bird species might demand a near-complete observation (\( \gamma \) closer to 1). The value of \( \gamma \) thus depends on the annotator and the complexity of the event class. This model provides a flexible yet precise way to simulate annotator behavior and quantify their labeling performance. However, it is important to note that this model is deterministic, focusing on temporal alignment between events and the query segment. In practice, human annotation often involves stochastic factors, such as variability in perception and judgment, which are not explicitly modeled here.


\subsection{Label Accuracy}
\label{sec:quality_of_presence_labels}

Label accuracy measures the alignment between annotator-provided labels and ground truth labels:

\begin{definition}
The label accuracy is defined as 
\begin{equation}
\label{eq:query_iou}
    F(e, q, \gamma) = \begin{cases}
        \frac{|e \cap q|}{d_q}, & \text{ if } l_q = 1, \\
        \frac{d_q - |e \cap q|}{d_q}, & \text{ if } l_q = 0.
    \end{cases}
\end{equation}
\end{definition}

For instance, consider a 3-second query segment ($d_q = 3$) that overlaps exactly one second ($|e \cap q|=1$) with a 2-second sound event ($d_e = 2$) of the class bird song  $(c=\text{``bird song''}$). The annotator assigns a presence label ($l_q = 1$) with label accuracy \( \frac{|e \cap q|}{d_q} = \frac{1}{3} \) if half or less of the event needs to be in the query segment ($\gamma \leq 0.5$). Contrary, the annotator assigns an absence label (\(l_q = 0\)) with label accuracy \( \frac{d_q - |e \cap q|}{d_q} = \frac{3 - 1}{3} = \frac{2}{3} \) if more than half of the event ($\gamma > 0.5$) needs to be in the query segment. This formulation isolates the segment label noise (\( 1 - F(e, q, \gamma) \)) introduced by the automated component (fixed-length segments) of the FIX weak labeling method.

%If the presence criterion is that at least half of the bird song event needs to be in the query segment ($\gamma = 0.5$), then this is met (\( \frac{|e \cap q|}{d_e} =\frac{1}{2} = 0.5 \ge \gamma \)), and the annotator assigns a presence label (\(l_q = 1\)) with label accuracy \( \frac{|e \cap q|}{d_q} = \frac{1}{3} \). If the presence criterion, however, is stricter ($\gamma > 0.5$) then it would not be met, and the annotator assigns an absence label (\(l_q = 0\)) with label accuracy \( \frac{d_q - |e \cap q|}{d_q} = \frac{3 - 1}{3} = \frac{2}{3} \). 

\section{The Label Accuracy and Cost of ORC Weak Labeling}
Let us start with the ORC weak labeling method. This method uses a priori information about the event start and end times and is therefore not available in practice, but should be seen as an upper bound on what can be achieved with weak labeling. The start and end times of the true presence and absence events are used to construct the query segments:
\begin{equation}
\label{eq:orc}
    \sQ_{\text{ORC}} = \{(a_0, b_0), (a_1, b_1), \dots, (a_{B_{\text{ORC}}-1}, b_{B_{\text{ORC}}-1})\} = \{q_0, \dots, q_{B_{\text{ORC}}-1}\},
\end{equation}
where $(a_i, b_i)$ is the $i$th ground truth presence or absence event. The annotator indicates presence or absence for each of these segments, which by construction results in the ground truth annotations, illustrated in Figure~\ref{fig:orc_weak_labeling}. In the example, there are three target events (green), and four absence events, which means that $B_{\text{ORC}}=7$. In general $B_{\text{ORC}}\in\{2M-1, 2M+1\}$, where $M$ denotes the number of presence events. The number of absence events can be fewer than $2M+1$ if the recording starts or ends with a presence event, however, for simplicity and without losing generality, we will consider $B_{\text{ORC}}=2M+1$ as the minimum number of query segments needed for ORC to derive the ground truth. From an annotation cost perspective, this is the most cautious choice, and it is also the most likely outcome. The query accuracy is $1$ for each query segment since by construction the fraction of correctly labeled data in each query segment will be $1$ when given the correct presence or absence labels.

\begin{figure}[H]
    \centering
    \includegraphics[width=0.8\linewidth]{figures/orc_weak_labeling.png}
    \caption{ORC weak labeling of an audio recording with three target events ($M=3$) shown in green and four absence events. The $B_{\text{ORC}}=7$, query segments $q_0, \dots, q_6$ are derived from the ground truth segmentation of the data, and therefore the label accuracy will by definition be $1$.} %And in general we get on average $M$ absence segments where $M$ is the number of events.}
    \label{fig:orc_weak_labeling}
\end{figure}

In summary, the ORC weak labeling method produces annotations with label accuracy $1$, using the minimum number of query segments needed to achieve this. We use this as a reference on what can be achieved for weak labeling data.

% \begin{figure}[H]
%     \centering
%     \includegraphics[width=0.5\textwidth]{figures/orc_sufficient_queries.png}
%     \caption{The four cases with different number of absence segments (numbered) between the events (green) in an audio recording. In this example we have $3$ events, and either (a) $4$, (b) $3$, (c) $3$, or (d) $2$  absence segments. On average $3$ absence segments if assumed equally probable.} %And in general we get on average $M$ absence segments where $M$ is the number of events.}
%     \label{fig:b_orc}
% \end{figure}

%The number of query segments grow linearly with the number of target events $M$ in the given audio recording. 

%In Figure~\ref{fig:b_orc} we show an example with $M=3$ sound events without overlap. Between each sound event there is a segment with the absence of an event giving $M-1$ absence segments in total (see (d) in Figure~\ref{fig:b_orc}) then we potentially have $1$ extra absence segment at each end of the recording (see (b-d) in Figure~\ref{fig:b_orc}). If we assume these cases to be equally probable we end up with an average of $M$ absence segments. The total number of segments needed on average to query the audio recording perfectly and get a label accuracy of $1$ is therefore
%\begin{equation}
%B_{\text{ORC}} = 2M.
%\end{equation}

%This tells us how few queries we can use and still get perfect labels with ORC. 
\section{The Label Accuracy and Cost of FIX Weak Labeling}

The outline of this section is as follows. In Section~\ref{sec:fix_weak_labeling_method} we define the FIX labeling method. In Section~\ref{sec:expected_label_accuracy_given_overlap} we derive a closed-form expression for the expected label accuracy of a query segment given that it overlaps with a single event of deterministic event length. We note that it is only in the cases of overlap between a query segment and an event that a presence label can occur under the assumed annotator model, and that the expectation in label accuracy over these cases therefore can be viewed as the expected presence label accuracy. For the remainder of the paper we will simply write expected label accuracy when referring to the expectation over the overlapping cases, unless explicitly stated otherwise.

In the same section we derive the optimal query length with respect to the expected label accuracy, the maximum expected label accuracy and the number of query segments needed (proxy for annotation cost). In Section~\ref{sec:theory_event_distribution} we explain how the expression for expected label accuracy can be used in the case of a single event of stochastic length, and in Section~\ref{sec:theory_multi_events} we explain under which conditions this can be used when multiple events can occur. Finally, we derive a closed form expression for the expected label accuracy of an audio recording with multiple events of stochastic length in Section~\ref{sec:expected_label_accuracy_all_cases}, and provide an alternative interpretation of the theory in Section~\ref{sec:ratio_based}.



\subsection{The FIX Weak Labeling Method}
\label{sec:fix_weak_labeling_method}
The FIX weak labeling method, commonly used in practice, splits the audio recording into fixed and equal length query segments, and then an annotator is asked to provide either a presence or absence label for each of the query segments. 
%This method is commonly used in practice~\citep{Shuyang2017, Shuyang2018, Wang2022, Martin-Morato2023a}, so we want to better understand the limits. 
Let $B_{\text{FIX}}$ denote the number of query segments used, then the query segments for an audio recording of length $T$ are defined as
\begin{equation}
\label{eq:fix}
    \sQ_{\text{FIX}} = \{(a_0, b_0), (a_1, b_1), \dots, (a_{B_{\text{FIX}}-1}, b_{B_{\text{FIX}}-1})\} = \{q_0, \dots, q_{B_{\text{FIX}}-1}\},
\end{equation}
where the start and end timings of each query segment is $q_i = (a_i, b_i) = (id_q, (i+1)d_q)$ and the fixed query segment length is $d_q = T / B_{\text{FIX}} $. We illustrate this in Figure~\ref{fig:fix_weak_labeling}, where the presence criterion for the annotator is $\gamma=0.5$. There are three presence events and four absence events, and using only $B_{\text{FIX}}=7$ query segments results in annotations with an average label accuracy that is lower than $1$. 

\begin{figure}[ht!]
    \centering
    \includegraphics[width=0.8\linewidth]{figures/fix_weak_labeling.png}
    \caption{
    %The FIX weak labeling process. Observe how the fixed-length query segments ($q_0$ to $q_6$) can overlap with presence events (green), leading to potential inaccuracies in the assigned presence labels (red).
    Illustration of the FIX weak labeling method. The audio recording contains presence events (green). The FIX method divides the recording into fixed-length query segments (e.g., $q_0$ to $q_6$). Note how the alignment between segments and presence events affects the accuracy of presence labels (red hatched).
    %FIX weak labeling of an audio recording with three presence events and four absence events. The $B=7$ segments, $q_0, \dots, q_6$ are of fixed and equal length. The label accuracy depends on how well these segments align with the target events and their length.
    } 
    \label{fig:fix_weak_labeling}
\end{figure}

We want to find an expression for the expected label accuracy for a given data distribution and query segment length. In addition, we want to understand the query length that maximize the expected label accuracy.


\subsection{The Expected Label Accuracy of a Query Segment given Event Overlap}
\label{sec:expected_label_accuracy_given_overlap}

To derive a tractable closed-form expression we analyze a simplified data distribution, consisting of audio recordings of length $T$ that always contain a single event of deterministic length $d_e$. This is arguably the simplest data distribution to annotate, and the results can therefore be viewed as an upper bound on the expected label accuracy for any more complex data distribution. %, since it is arguably the simplest data distribution to annotate. 

%In addition, the theoretical results derived for this simplified data distribution can be used to understand the label accuracy of more complex data distributions with stochastic event lengths or multiple events. We expand on this in  Section~\ref{sec:theory_event_distribution} and Section~\ref{sec:theory_multi_events}, and provide simulation studies with more complex data distributions in Section~\ref{sec:event_distribution} and Section~\ref{sec:multi_events} to support these claims. 

\begin{figure}
    \centering
    \includegraphics[width=0.8\linewidth]{figures/occurences_2.png}
    \caption{
    %The bottom panel shows the resulting label accuracy for query segment $q_2$ based on the event occurrences depicted in the upper panel as the event's end time ($t$) varies. Overlap causes label noise (red); the gray area ($A$) represents the label accuracy during overlaps.
    \textit{Top panel:}  A single event ($e_t$) of length $d_e$ can occur at various end times ($t$) within the recording of length $T$. \textit{Bottom panel:} The resulting label accuracy for query segment $q_2$ (arbitrarily chosen for illustration) of length $d_q$ as a function of the event's end time ($t$). Overlap between the event and the query segment leads to segment label noise and a reduced label accuracy, which in this case occur when $t\in[a_2, a_2+d_e+d_q]$ where $a_2$ is the start time of $q_2$. The red hatched area ($A$) represents the cumulative label accuracy during these overlapping scenarios.
    %We illustrate the label accuracy as a function of all possible event occurrences $e_t$ in the bottom panel for a given query segment $q_2$, where $t$ is the offset of event $e_t$, $T$ is the length of the audio recording, $d_e$ is the event length, $d_q$ is the query segment length and $d_e + d_q$ is the total amount of overlapping occurrences. The label accuracy is $1$ in all non-overlapping cases, and less than $1$ in all overlapping cases. That is, segment label noise only occur in cases of overlap (red). We are interested in understanding the label accuracy of the presence labels (gray), and therefore want to compute the gray area $A$. Considering a different query segment $q_i$ will only result in a translation of this area along the x-axis.
    }
    \label{fig:proof_idea}
\end{figure}

The setup is illustrated in the upper panel of Figure~\ref{fig:proof_idea}, where a single event $e_t$ of length $d_e$ can occur at any time $t\in[0, T]$ (indicated by the arrow). The bottom panel of Figure~\ref{fig:proof_idea} shows the label accuracy for a specific query segment ($q_2$) as the end time ($t$) of the event varies. The area (A) highlighted in hatched red indicates the label accuracy in the cases of overlap between the query segment and the event, and the area in hatched green indicate the label accuracy in the cases of no overlap, which is by the definition of the annotator model is always $1$. Crucially, while this figure illustrates the accuracy for query segment $q_2$, the shape of this accuracy function remains the same for other query segments; only its position along the x-axis would change.  %, and then later in Section~\ref{sec:expected_label_accuracy_all_cases} we will explain how this can be used to derive an expression for label accuracy for all cases.

%Computing the expectation over both overlapping and non-overlapping cases would mean that more absence sound results in a higher metric score. In fact, let $A$ denote the gray area in Figure~\ref{fig:proof_idea}, then the expected label accuracy over all cases is $(A + T-(d_e+d_q))/T$, since we always have a score of $1$ for the $T-(d_e+d_q)$ cases of non-overlap. We can see that $(A + T-(d_e+d_q))/T \rightarrow 1 $ if $T\rightarrow \infty$. We are more interested in the labeling errors that occur around the presence events, and therefore choose to only consider the cases with overlap when computing the expected label accuracy. That is, we want to derive an expression for $A / (d_e + d_q)$.

To simplify the mathematical analysis, without loss of generality, we can fix the query segment to start at time $0$, $q=(0, d_q)$, and represent the event with its ending time $t$ as $e_t = (t-d_e, t)$. In this way, $t\in[0, d_e+d_q]$ describes all possible overlap occurrences. That is, when $t=0$ the event ends at the start of the query segment, and when $t=d_e+d_q$ the event starts at the end of the query segment. To formalize this, we can express the expected label accuracy in case of overlap by integrating over all possible event end times ($t$) where overlap occurs: %To get the expected label accuracy given overlap we need to compute
\begin{align}
\label{eq:_integral_1}
    \E_{\rt \sim p}\left[F(e_{\rt}, q, \gamma)\right] &= \int_{0}^{d_e + d_q}F(e_t, q, \gamma)p(t)\mathrm{d}t, \\
    \label{eq:_integral_2}
    &= \frac{1}{d_e + d_q} \int_{0}^{d_e + d_q}F(e_t, q, \gamma)\mathrm{d}t \\
    \label{eq:normalized_area}
    &= \frac{A}{d_e + d_q}.
    %&= \frac{1}{d_e + d_q}\int_{0}^{d_e + d_q}F(e_t, q, \gamma)
\end{align}
where $\rt \sim p$ denotes a random variable $\rt$ distributed according to a distribution $p$, and $p(t)$ denotes the probability of realization $t$. Since we assume that the sound event can occur anywhere in the audio recording with equal probability we get $p(\rt) = 1/(d_e + d_q)$, and by observing that the integral $\int_{0}^{d_e+d_q}F(e_t, q, \gamma)\mathrm{d}t$ describes the hatched red area denoted $A$ in Figure~\ref{fig:proof_idea} we arrive at the final expression in Eq.~\ref{eq:normalized_area}. 

Remember that absence labels can occur when there is no overlap (always correct) and when there is overlap but the presence criterion is not fulfilled, and presence labels can only occur when there is overlap and the presence criterion is fulfilled. Therefore, inaccurate labels only occur in the case of overlap. The expected label accuracy in the case of overlap therefore describes the accuracy of the labels when segment label noise can occur, which happens around the boundaries of the true event.

In Appendix~\ref{app:thm1} we show how to express $A$ in terms of the event length $d_e$, the query segment length $d_q$ and the presence criterion $\gamma$ under the assumption that the annotator presence criterion can be fulfilled ($d_q \geq \gamma d_e$), and that it can not be fulfilled ($d_q < \gamma d_e$). Finally, we arrive at the following four main theorems:

\begin{theorem}
\label{thm:expected_iou}
The expected label accuracy in case of overlap between a query segment $q$ of length $d_q$ and a single event $e$ of deterministic length $d_e$ is
\begin{equation}
\label{eq:expected_iou}
    f(d_q) = \E_{\rt \sim p}\left[F(e_{\rt}, q, \gamma)\right] = \begin{cases}
    \frac{d_{e} \left(2 \gamma d_{q} - 2 \gamma^2 d_{e} + d_{q}\right)}{d_{q} \left(d_{e} + d_{q}\right)}, & \text{ if } d_q \geq \gamma d_e, \\
    \frac{d_q}{d_e + d_q}, & \text{ if } d_q < \gamma d_e,
    \end{cases}
\end{equation}
when the presence criterion for the annotator is $\gamma$.
\end{theorem}
\begin{proof} See Appendix~\ref{app:thm1} for the proof. We show how to express the area $A$ in Eq.~\ref{eq:normalized_area} in terms of $d_e$, $d_q$ and $\gamma$ for the two assumptions: $d_q \geq \gamma d_e$, and $d_q < \gamma d_e$.
\end{proof}

%We use Theorem~\ref{thm:expected_iou} to prove Theorem~\ref{thm:fix_optimal_query_length}, which show the optimal query segment length to maximize the expected label accuracy. 

\begin{theorem}
\label{thm:fix_optimal_query_length}
The query length that maximizes the expected label accuracy in case of overlap for a given event length $d_e$ is 
\begin{equation}
\label{eq:fix_optimal_query_length}
    d_q^* = d_e\gamma\frac{2\gamma + \sqrt{4\gamma^2 + 4\gamma + 2}}{2\gamma + 1}.
\end{equation}
\end{theorem}
\begin{proof}
See Appendix~\ref{app:thm2} for the proof. We compute the derivative of $f(d_q)$ with respect to $d_q$, and show that $d_q^*$ is the maximum.
\end{proof}

%By inserting $d_q^*$ into Eq.~\ref{eq:expected_iou} we get an expression for the maximum expected label accuracy as a function of $\gamma$, which leads to the next theorem.

\begin{theorem}
\label{thm:max_iou}
The maximum expected label accuracy in case of overlap between a query segment of length $d_q$ and an event of length $d_e$ when $d_q \geq \gamma d_e$ is
\begin{equation}
    \label{eq:max_iou}
    f^*(\gamma) = f(d_q^*) = 2\gamma\left(2\gamma + 1 - \sqrt{4\gamma^2+4\gamma + 2} \right) + 1
\end{equation}
\end{theorem}
\begin{proof}
See Appendix~\ref{app:thm3} for the proof. We substitute $d_q$ for $d_q^*$ in Eq.~\ref{eq:expected_iou}.
\end{proof}

%Finally, we use $d_q^*$ to compute how many query segments are needed to maximize the label accuracy for an audio recording of length $T$. 

\begin{theorem}
\label{thm:fix_number_of_queries}
The number of queries $B^*_{\text{FIX}}$ (cost) that are needed by FIX to maximize the expected label accuracy in case of overlap for an audio recording of length $T$ when $d_e=1$ is 
\begin{equation}
\label{eq:b_fix_1}
    B^*_{\text{FIX}} = \frac{T}{d_q^*}.
\end{equation}
\end{theorem}
\begin{proof}
$T/B^*_{\text{FIX}} = d_q^*$, which by Theorem~\ref{thm:fix_optimal_query_length} leads to maximum label accuracy.
\end{proof}


%\textbf{TODO: explain somewhere what each theorem means, and why it is useful information.}
%\paragraph{Summary:}
%\subsection{Summary and more complex data distributions}
%\label{sec:applicability_fix_label_accuracy}
%To summarize, in Theorem~\ref{thm:expected_iou} we derived an expression, $f(d_q)$, for the expected label accuracy of FIX weak labeling. 

%The expression is derived for fixed length segments of length $d_q$ using an annotator model with presence criterion $\gamma$ and assuming a data distribution where recordings are of finite length and only contain a single presence event of deterministic event length $d_e$.

In summary, Theorem~\ref{thm:expected_iou} gives us an expression $f(d_q)$ for the expected label accuracy when query segments of length $d_q$ are used to detect events of length $d_e$ and the presence criterion for the annotator is $\gamma$. We use this to find the query segment length $d_q^*$ that maximize the expected label accuracy, leading to Theorem~\ref{thm:fix_optimal_query_length}. Theorem~\ref{thm:fix_optimal_query_length} show the query segment length $d_q^*$ that maximizes expected label accuracy for a given event length and annotator criterion. Further, by inserting $d_q^*$ into Theorem~\ref{thm:expected_iou}, $f^*(\gamma) = f(d_q^*)$, we get Theorem~\ref{thm:max_iou}, which is the maximum achievable expected label accuracy for a given annotator criterion $\gamma$. We have omitted the case $d_q < \gamma d_e$ when deriving $f^*(\gamma)$, since maximizing the expected label accuracy in the case when the annotator presence criterion can not be fulfilled is not very interesting, since we can not get presence labels. Note that $f^*(\gamma)$ is a function of only $\gamma$, meaning that the maximum expected label accuracy is independent of the target event length when considering a single deterministic event. Finally, Theorem~\ref{thm:fix_number_of_queries} show that an annotator needs to weakly label $B^*_{\text{FIX}}$ query segments for each audio recording to achieve the maximum label accuracy in expectation, which can be seen as a proxy for annotation cost.

There is arguably no simpler audio data distribution to annotate than when recordings only contain a single event of deterministic length (except for when no event occurs at all). We can therefore treat $f^*(\gamma)$ as an upper bound on the maximum expected label accuracy for any audio distribution. We demonstrate this empirically in the results in Section~\ref{sec:results}. However, in practice audio recordings often contain events that vary both in length and number. Let us therefore consider how the derived theory can be useful also in these cases.


%Theorem~\ref{thm:fix_number_of_queries} tells us the number of query segments needed to maximize the expected label accuracy when FIX labeling audio recordings of length $T$ with a single randomly occurring event of length $d_e$. $B^*_{\text{FIX}}$ tells us how many segments the annotator has to weakly label, which is a proxy for annotation cost.

\subsection{Stochastic Event Length}
\label{sec:theory_event_distribution}
Events may vary in length according to some event length distribution. Let $p(d_e)$ denote the probability of the outcome that an event has length $d_e$, and let $d_e\sim p(d_e)$ denote that $d_e$ is a sample from that distribution. The expected label accuracy over a distribution of event lengths for a given $\gamma$ and query segment length $d_q$ can then be computed as
\begin{align}
\E_{d_e\sim p(d_e)}\left[f(d_q)\right]
\label{eq:expected_query_iou_distribution}
&= \int_{0}^{\infty} f(d_q)p(d_e) \mathrm{d}d_e.
\end{align}
While we do not provide a closed form solution for this, we can solve the integral in Eq.~\ref{eq:expected_query_iou_distribution} by numerical integration. Note that $d_q^*$ in Theorem~\ref{thm:fix_optimal_query_length} depends on the single event length $d_e$, and to find it for a distribution we would need to solve Eq.~\ref{eq:expected_query_iou_distribution} for a range of $d_q$ and find the one that leads to the best label accuracy. However, for some event length distributions, setting $d_e$ to the average of the distribution turns out to be a good heuristic. We perform a simulation study in Section~\ref{sec:event_distribution} to support these claims.

%We study the effect of the event length distribution on the maximum expected label accuracy and the optimal query length. 


\subsection{Multiple Events}
\label{sec:theory_multi_events}
There may be multiple ($M$) events present in a given audio recording. In Figure~\ref{fig:multiple_events} we show the label accuracy for all possible occurrences of a query segment $q_t$ in a recording with two events ($M=2$). Note that we have put the subscript $t$ on the query segment ($q_t$) instead of the event as in the prior analysis. This formulation is entirely equivalent, but when talking about multiple events it is more intuitive to consider them as fixed in time for a given recording, and that the query segments occur relative them at random. There are now two regions where overlap occurs, one around $e_1$ and one around $e_2$. On average we get $2A/2(d_e + d_q) = A/(d_e + d_q) = f(d_q)$. That is, the theory we derived for the single event case explains the multiple event case.

\begin{figure}
    \centering
    \includegraphics[width=0.8\linewidth]{figures/occurences_multi_events.png}
    \caption{
    \textit{Top panel:}  Two events ($M=2$) of length $d_e$ that are fixed in time within a recording of length $T$, and a query segment $q_t = (-d_q + t, t)$. \textit{Bottom panel:} The resulting label accuracy of $q_t$ for $t\in [0, T-d_q]$, simulating that the $q_t$ can appear anywhere at random in time in relation to the events. As before, when there is overlap between the query segment and an event the label accuracy is below $1$, otherwise it is always $1$.
    }
    \label{fig:multiple_events}
\end{figure}

However, for this to hold we need to assume that for any event the closest other event is least $d_q$ away in time. In Figure~\ref{fig:multiple_events} this holds since the start of $e_2$ is at least $d_q$ away from the end of $e_1$. If this assumption holds then the expected label accuracy for multiple events is $f(d_q)$. The assumption is plausible if events are sparse in relation to $d_q$. Note that $d_q^* \in (0, d_e\frac{2 + \sqrt{10}}{3}]$ for $\gamma \in (0, 1]$ according to Theorem~\ref{thm:fix_optimal_query_length}. That is, when considering the optimal query length $d_q^*$ this assumption translates to that events should be no closer than approximately $1.72d_e$ for $\gamma = 1$, $0.81d_e$ for $\gamma = 0.5$, and $0$ for $\gamma \rightarrow 0$. We perform a simulation study in Section~\ref{sec:multi_events} to see the effect of breaking this assumption, and we leave it to future work to derive the expected label accuracy in case of overlap for multiple events.

\subsection{The Expected Label Accuracy of an Audio Recording}
\label{sec:expected_label_accuracy_all_cases}
We now know the expected label accuracy of a query segment given event overlap, and how to use this for a stochastic event lengths and multiple events. We can use this to derive an expression for the expected label accuracy of and audio recording of finite length ($T$) that has multiple ($M$) stochastic event lengths ($d_e\sim p(d_e)$).

\begin{theorem}
\label{thm:label_accuracy}
The expected label accuracy for an audio recording of length $T$, with $M$ events of stochastic event length $d_e \sim p(d_e)$ that are spaced at least $d_q$ apart is
\begin{equation}
\label{eq:label_accuracy_recording}
    \E_{d_e\sim p(d_e)}\left[- \frac{2 M d_{e}^{2} \gamma^{2}}{T d_{q}} + \frac{2 M d_{e} \gamma}{T} - \frac{M d_{q}}{T} + 1 \right].
\end{equation}
\begin{proof}
We will do this proof by picture. In Figure~\ref{fig:multiple_events} we have two events ($M=2$), in general for $M$ events the accumulated label accuracy in the cases of overlap is $MA$ (the sum of the hatched red areas), the total amount of overlapping cases is $M(d_e + d_q)$ and the total amount of non-overlapping cases is therefore $T-M(d_e+d_q)$ for an audio recording of length $T$. In the case of no overlap, the label accuracy is always $1$, which means that the accumulated label accuracy in the case of no overlap (sum of the green hatched areas) is $T-M(d_e + d_q)$. Normalizing for the entire duration of the recording we arrive at
\begin{equation}
\frac{AM + T-M(d_e + d_q)}{T} = - \frac{2 M d_{e}^{2} \gamma^{2}}{T d_{q}} + \frac{2 M d_{e} \gamma}{T} - \frac{M d_{q}}{T} + 1,
\end{equation}
and as before we can simply compute an expectation over the event length distribution.
\end{proof}
\end{theorem}

Theorem~\ref{thm:label_accuracy} tells us the expected label accuracy under FIX weak labeling with query segment length $d_q$ for an audio recording of length $T$, with $M$ events of stochastic event length $d_e \sim p(d_e)$. If we want to account for class label noise, where the annotator gives the wrong label with probability $\rho$, this can be included by simply scaling the whole expression in Eq.~\ref{eq:label_accuracy_recording} by $(1-\rho)$. That is, the expected label accuracy for the cases of overlap allows us to express a variety of things about the expected label accuracy of an audio recording. 

However, note that we have $T$ in the denominator of all terms except the term that is $1$, meaning that if we let $T$ approach $\infty$, then the expected label accuracy approaches $1$. That is, considering the accuracy of both absence and presence labels equally can lead to hiding the effect that we want to understand in this paper, which is the effect of $d_q$ on the accuracy of the presence labels. We could derive a balanced accuracy in a similar way as above, but instead we choose to continue our analysis looking only at the expected label accuracy in the case of overlap. %We leave it to future work to extend the analysis in these directions. 


\subsection{Expected Label Accuracy given Overlap when $d_q = \delta d_e$}
\label{sec:ratio_based}
As a result of the proof for Theorem~\ref{thm:max_iou} in Appendix~\ref{app:thm3} we get an alternative dimensionless interpretation of the expected label accuracy when the query segment length is expressed as a factor of the event $d_q = \delta d_e$,
\begin{equation}
    f(\delta d_e) = \frac{(2\gamma + 1)\delta - 2\gamma^2}{\gamma(1+\gamma)},
\end{equation}
and an expression for the ratio that maximizes it
\begin{equation}
    \delta^* = \frac{d_q^*}{d_e} = \gamma \frac{2\gamma + \sqrt{2\gamma^2 + 2\gamma + 1}}{2\gamma + 1}.
\end{equation}
This alternative formulation illustrates that it is the ratio $\delta = d_q/d_e$ that affects the expected label accuracy of a single event, and not the absolute lengths $d_q$ and $d_e$. Further, we can use this interpretation to rewrite Theorem~\ref{thm:label_accuracy} as
\begin{equation}
    \E_{\delta \sim p(\delta)} \left[\frac{M d \delta \left(- \delta + 2 \gamma\right) - 2 M d \gamma^{2} + T \delta}{T \delta}\right],
\end{equation}
where $\delta$ denotes a random variable with probability distribution $p(\delta)$. %It may be less intuitive to sample ratios $\delta = d_q/d_e$ from a distribution, but it does provide an alternative view.
\section{Simulating the Label Accuracy of FIX Weak Labeling}
\label{sec:simulation}

To validate the theory, we simulated FIX labeling of various audio recording distributions and compared the average simulated label quality with the theoretical results from Section~\ref{sec:expected_label_accuracy_given_overlap}. The code used for these simulations is provided in the supplementary material.

We generated 1000 audio recordings of length $T=100$ seconds for each configuration. The number of events, $M$, and the event length distributions varied across simulations, as detailed below:

\begin{itemize}
    \item \textbf{Single Event with Deterministic Length:} We simulated recordings with $M=1$ event of deterministic length $d_e = 1$ second.

    \item \textbf{Single Event with Stochastic Length from Normal Distributions:} We drew event lengths from two normal distributions with the same mean but different variances ($\mathcal{N}(3, 0.1)$ and $\mathcal{N}(3, 1)$), and from two normal distributions with different means but the same variance ($\mathcal{N}(0.5, 0.1)$ and $\mathcal{N}(5, 0.1)$). For these simulations, $M=1$.
    
    \item \textbf{Single Event with Stochastic Length from Gamma Distributions:} We sample event lengths from two gamma distributions (offset by $0.5$ seconds due to computation cost) with different shape parameters but the same scale parameter ($\text{Gamma}(0.8, 1) + 0.5$ and $\text{Gamma}(0.2, 1) + 0.5$) with $M=1$.
    
    \item \textbf{Single Event with Stochastic Length from Real Length Sample:} We used the event length distributions for dog barks and baby cries from the NIGENS dataset~\citep{Trowitzsch2019} with $M=1$.
    
    \item \textbf{Multiple Events with Deterministic Length:}  We simulated recordings with multiple events ($M=30$ and $M=50$) where each event had a deterministic length of $d_e = 1$ second.
\end{itemize}

For recordings with stochastic event lengths or multiple events, the length of each of the $M$ events was sampled from the specified distribution. Each sampled event was then placed randomly within the recording. The start time $a_e$ of each event was drawn uniformly at random from $[0, T - d_e]$. If multiple events were present, overlapping events were merged into one presence event. For each generated audio recording, we simulated FIX labeling using different annotator presence criteria $\gamma \in [0.01, 0.99]$ and a range of query segment lengths $d_q$. The query segment lengths were linearly spaced between a small fraction of the minimum event length observed in the distribution and a value several times the maximum observed event length. 

We then computed the average label accuracy over the query segments that overlaps with an event in each recording. For each query segment $q$ we check if the annotator presence criterion ($h(e, q) \geq \gamma$) is fulfilled for any event $e \in E$, where $E$ is the set of all events that overlap with $q$. If this is true for any of the events then $q$ is given a presence label ($l_q = 1$) otherwise it is given an absence label ($l_q = 0$). The label accuracy is then computed in a similar way as in Eq.~\ref{eq:query_iou}, but since we can now have multiple events overlapping with the same query segment, we need to consider the union of all overlapping events $\cup_{e\in E}e$ when computing the label accuracy of assigning label $l_q$ to that query segment. The total amount of overlap becomes $|(\cup_{e\in E} e) \cap q|$ instead of $|e \cap q|$. However, when $M=1$ this is equivalent to Eq.~\ref{eq:query_iou} ($|(\cup_{e\in E} e) \cap q| = |e \cap q|$), since $|E| = 1$.

%In this way, we simulate the effect of breaking the assumption that events are spaced at least $d_q$ apart, and can better understand the effect this has when compared to the derived theory. Finally, for each considered $\gamma$, we empirically determined the maximum average label accuracy across all tested query lengths and the corresponding optimal query length. These empirical results were then compared to the theoretical predictions.

In this way, we simulated the effect of breaking the assumption that events are spaced at least $d_q$ apart, and could better understand the effect this had when compared to the derived theory. Finally, for each considered $\gamma$, we empirically determined the maximum average label accuracy across all tested query lengths and the corresponding optimal query length. These empirical results were then compared to the theoretical predictions.

%In this way, we simulate the effect of breaking the assumption that events are spaced at least $d_q$ apart, and can better understand the effect this has when compared to the derived theory. Finally, for each considered $\gamma$, we empirically determine the maximum average label accuracy across all tested query lengths and the corresponding optimal query length. These empirical results are then compared to the theoretical predictions.

%This formulation of label accuracy is different from that used to derive the theory when we allow multiple events. During simulation, we also consider the case where multiple events can overlap with the same query segment, meaning that we need to consider the contribution of all the events that overlap with a query segment to compute accuracy of the label $l_q$. In the single event case, this formulation of label accuracy is equivalent with Eq.~\ref{eq:query_iou}. 


%Our definition of label accuracy emphasizes resolving individual presence events. For a recording of length $T$ containing $M$ non-overlapping presence events of average duration $d_e$, our metric reflects the accuracy of labeling each individual event. For example, if a query correctly identifies one event, its accuracy contribution is related to the duration of that event relative to the recording length. This differs from conventional accuracy metrics which might compute accuracy as $(M d_e) / T$ if the entire recording were labeled as containing presence. Our approach is crucial for evaluating the ability to resolve and correctly label individual events, particularly important for evaluation labels where high accuracy should indicate the precise detection of all events.
\section{Results}
\label{sec:results}
In this section we present the results of the simulated annotation process, and show how these connect to the derived theory. We start by looking at the expected label accuracy and the query segment length that maximize the expected label accuracy for FIX and ORC weak labeling, and then we relate this to the annotation cost.

\subsection{Expected Label Accuracy given Overlap}

%Understanding how expected label accuracy given overlap and the query length that maximize this is affected by the annotator presence criterion is useful for optimizing the FIX weak labeling strategy. 
We evaluate how different annotator presence criteria ($\gamma$) influence the achievable label accuracy given overlap under FIX weak labeling. We first examine the case of a single event with a deterministic length, then extend our simulation study to stochastic event lengths, and finally to multiple events occurring within the same recording.


\subsubsection{Single Event with Deterministic Length}
The simulated results are derived using the simulation setup described in section~\ref{sec:simulation}, with $M=1$ (a single event) and $d_e=1$ (deterministic length). In Figure~\ref{fig:simple_simulation}, we show the maximum expected label accuracy given overlap (left) and the corresponding query length that maximize the label accuracy (right) for different $\gamma$. $f^*(\gamma)$ is the maximum expected label accuracy achievable with annotator presence criterion $\gamma$ for the considered event length. We can see that the simulated average label accuracy closely follows the expected label accuracy, and that the corresponding segment length leading to this maximum is the same in theory and simulation.

\begin{figure}[H]
    \centering
    \includegraphics[width=0.66\textwidth]{figures/uniform_FIX_accuracy.png}
    % \includegraphics[width=0.49\textwidth]{figures/iou_max_vs_gamma.png}
    \caption{In the left panel we show the maximum expected label accuracy, $f^*(\gamma)$, for different $\gamma$, and the average maximum label accuracy from the simulations. In the right panel we show the query length that leads to this maximum label accuracy in theory, for $d_e = 1$, and in simulation. The theory follows the simulations well.}
    \label{fig:simple_simulation}
\end{figure}

In Figure~\ref{fig:simple_simulation} we see that if the annotator needs to hear more than $50$\% of the sound event to detect presence ($\gamma=0.5$) then the highest achievable label accuracy is $f^*(0.5) \approx 0.76$. This means that on average there is around $34$\% segment label noise around the presence labels. We also see that the query length that gives the maximum label accuracy is $d_q^* \approx 0.81$. The gap to the ORC weak labeling method which always gives a label accuracy of $1$, is large especially for large $\gamma$. In general, we can see how the maximum label accuracy deteriorates with a growing $\gamma$, and which query segment length to choose to maximize label accuracy in expectation. 

\subsubsection{Single Event with Stochastic Length}
\label{sec:event_distribution}
We now consider stochastic event lengths. We do this to better understand the effect of the event length distribution on the maximum expected label accuracy and the optimal query length. We solve the integral in Eq.~\ref{eq:expected_query_iou_distribution} by numerical integration over different event length distributions, and compare with the theory derived for a single deterministic event length and simulations. In each figure we present the derived theoretical rules $f^*(\gamma)$ and $d_q^*$ for the simplified event length distribution, the results from integration of Eq.~\ref{eq:expected_query_iou_distribution} with different event length distributions $p(d_e)$ (numerical), and the simulated results using the procedure described in section~\ref{sec:simulation} (simulated) where event lengths are sampled from different distributions. Note that, since $d_q^*$ is derived for a deterministic event length $d_e$, and require a choice of this value, we set $d_e$ to the average event length ($\mu$) for each distribution in these experiments as a heuristic. We then present the maximum expected label accuracy for different $\gamma$ (left in figures) and the query segment length that maximizes the expected label accuracy (middle in figures), and the histogram for the considered event length distributions (right in figures).

\begin{figure}
    \centering
    \includegraphics[width=\textwidth]{figures/normal_mean_FIX_accuracy.png}
    \caption{We validate the theory for stochastic event lengths drawn from two normal distributions with different means, but the same variance. We show the expected label accuracy (left panel), the optimal query length (middle panel), and the considered event length distributions (right panel).}
    \label{fig:normal_mean}
\end{figure}


\begin{figure}
    \centering
    \includegraphics[width=\textwidth]{figures/normal_variance_FIX_accuracy.png}
    \caption{We validate the theory for stochastic event lengths drawn from two normal distributions with different variance, but the same mean. We show the expected label accuracy (left panel), the optimal query length (middle panel), and the considered event length distributions (right panel).}
    \label{fig:normal_variance}
\end{figure}


In Figure~\ref{fig:normal_mean} and Figure~\ref{fig:normal_variance} we see that the mean and variance of the normal distribution have a small (if any) effect on the maximum expected label accuracy, but the mean does affect which query segment length that maximizes the expected label accuracy. We also see that $d_q^*$ follows the simulated and numerical optimal query length well for all considered normal distributions, when $d_e$ is set to the average event length ($\mu$) for the considered event length distribution. The average event length can be used as a heuristic value if we only know the average and not the true distribution to integrate over.

\begin{figure}
    \centering
    \includegraphics[width=\textwidth]{figures/gamma_FIX_accuracy.png}
    \caption{We validate the theory for stochastic event lengths drawn from two gamma distributions with different shape parameters, but the same scale parameter. We show the expected label accuracy (left panel), the optimal query length (middle panel), and the considered event length distributions (right panel).}
    \label{fig:gamma}
\end{figure}

In Figure~\ref{fig:gamma} we can see that a gamma distribution does affect the maximum expected label accuracy, and that simply setting $d_e$ to the average event length of the distribution leads to underestimating the optimal query length. Since it is not possible to optimize for both short and long events at the same time using FIX weak labeling, this type of distribution is quite challenging.

\begin{figure}
    \centering
    \includegraphics[width=\textwidth]{figures/dog_and_baby_FIX_accuracy.png}
    \caption{Barking dog and crying baby event length distributions from the NIGENS dataset~\citep{Trowitzsch2019}. These annotations have been made with a strong guarantee for high quality onsets and offsets.}
    \label{fig:dog_and_baby}
\end{figure}

In Figure~\ref{fig:dog_and_baby} we validate the theory against a real sample of event lengths from either baby cries or dog barks. Numerical integration between the derived expression and the histogram predicts the simulations well.

\subsubsection{Multiple Events with Stochastic Length}
\label{sec:multi_events}

In these simulations we allow multiple events to occur in the same recording ($M>1$). In Figure~\ref{fig:uniform_30} we show the results of sampling $30$ events of length $d_e=1$ for each audio recording. This does have a an effect on the expected maximum label accuracy and the corresponding query length, but not (that) large. In Figure~\ref{fig:uniform_50} we show the results of sampling $50$ events of length $d_e=1$ for each audio recording. This is an extreme case, where the event density of the recording is very high. %We would not expect this to happen often in practice. %But, now the effect starts to show. However, the theory still matches the simulations.

\begin{figure}
    \centering
    \includegraphics[width=\textwidth]{figures/uniform_30_FIX_accuracy.png}
    \caption{We validate the theory for multiple events of length $d_e=1$. We show the expected label accuracy (left panel), the optimal query length (middle panel), and the considered event length distributions (right panel). Note that presence events longer than $1$ can occur if two or more events overlap. We sample $30$ events with event length $d_e=1$ occur at random for each audio recording in this simulation.}
    \label{fig:uniform_30}
\end{figure}


\begin{figure}
    \centering
    \includegraphics[width=\textwidth]{figures/uniform_50_FIX_accuracy.png}
    \caption{We validate the theory for multiple events of length $d_e=1$. We show the expected label accuracy (left panel), the optimal query length (middle panel), and the considered event length distributions (right panel). Note that presence events longer than $1$ can occur if two or more events overlap. We sample $50$ events with event length $d_e=1$ occur at random for each audio recording in this simulation.}
    \label{fig:uniform_50}
\end{figure}

% In all experiments we can see that $f^*(\gamma)$ acts as an upper bound on the expected label accuracy for any of the other considered event length distributions. Arguably, any recording distribution that has more than a single event, or events of varying length should be at least as hard to annotate with FIX as the one with only a single event of deterministic event length.

\subsection{Annotation Cost for Maximum Expected Label Accuracy given Overlap}

Achieving maximum expected label accuracy comes at a cost, and understanding this cost trade-off is essential for practical annotation efforts. The cost model we employ accounts for both the time spent listening to audio and the effort required to label presence or absence events. %The key difference between FIX and ORC weak labeling lies in the number of queries required and the resulting label accuracy: FIX uses a theoretically derived optimal query length $d_q^*$ to construct $T/d_q^*$ query segments for an audio recording of length $T$ and achieve non-perfect label accuracy, while ORC requires at least $2M + 1$ queries to ensure perfect label accuracy.

\subsubsection{Formalizing the Cost Model}

The derived theory for the optimal query length allows us to analyze the cost of achieving maximum expected label accuracy under different annotator models for FIX weak labeling. We assume that the whole audio recording of length $T$ is listened to. The key difference in cost between the FIX and ORC weak labeling method is the number of segments ($B$) that need to be given a presence or absence label. We formalize a cost model as:
\begin{equation}
\label{eq:cost}
    C(T, B) = (1-r)T + rB,
\end{equation}
where $1-r$ represents the cost of listening to one second of audio (cost per second), and $r$ represents the cost of answering a query (cost per query). The term $(1-r)T$ therefore represents the cost of listening to $T$ seconds of audio, and the term $rB$ the cost of assigning $B$ presence or absence labels. Using this cost model, we calculate the cost of annotating an audio recording of length $T$ with $M$ sound events of length $d_e=1$ using either FIX or ORC weak labeling. For FIX, the number of queries that maximize expected label accuracy is given by $B^*_{\text{FIX}} = T/d_q^*$ (see Theorem~\ref{thm:fix_number_of_queries}). For ORC, achieving an expected label accuracy of $1$ requires at least $B^*_{\text{ORC}} = 2M+1$ queries.

In practice, we do not know the number of events $M$. To explore potential overestimation of $M$ when, for example, using a weak labeling process that tries to mimic ORC weak labeling, we model $B_{\text{ORC}}$ as a multiple of the necessary number of queries: $B_{\text{ORC}} = sB^*_{\text{ORC}}$, where $s \in \{1, 2, 4, 8\}$ represents the degree of overestimation. This approach captures scenarios where the number of events are either precisely estimated ($s=1$) or significantly overestimated ($s=8$) during the annotation process. In practice, $B_{\text{ORC}}$ could be set based on a bound on $M$. For example, by estimating a maximum expected number of sound events in a recording, $M_{\max}$, based on knowledge of typical event density, or characteristics of the audio recording. We assume that overestimation by more than a factor of $8$ is unlikely. The relative cost between FIX and ORC weak labeling can then be computed as:
\begin{equation}
\label{eq:cost_ratio}
    \frac{C_{\text{FIX}}}{C_{\text{ORC}}} = \frac{C(T, B^*_{\text{FIX}})}{C(T, B_{\text{ORC}})},
\end{equation}
where a ratio larger than $1$ indicates that FIX is more costly than ORC, and a ratio smaller than $1$ indicates that FIX is less costly than ORC.



\begin{figure}
    \centering
    \includegraphics[width=0.49\textwidth]{figures/cost_orc_fix_gamma.png}
    \includegraphics[width=0.49\textwidth]{figures/cost_orc_fix_r.png}
    \caption{The relative cost of FIX and ORC for varying annotator criteria $\gamma$ (left), and cost ratios $r$ (right). The default parameters are: $T=100$, $r=0.5$, $M=1$ and $\gamma=0.5$. We simulate overestimating the number of needed queries $B_{\text{ORC}} = s(2M+1)$ by a factor of $s$ for $s \in \{1, 2, 4, 8\}$ to see how this affects the relative cost. The cost of FIX is greater than the cost of ORC above the dashed red line where the cost ratio is $1$.}
    \label{fig:cost_1}
\end{figure}

\subsubsection{Effect of annotator criteria ($\gamma$) and cost ratio ($r$).} Figure~\ref{fig:cost_1} (left) shows the relative cost for varying annotator criteria $\gamma \in [0.1, 1]$. As $\gamma \rightarrow 0.1$, the cost of FIX increases sharply, reflecting the need for an infinitely large number of queries to achieve an expected label accuracy of $1$. In practice, achieving perfect accuracy with FIX is infeasible due to the associated cost. For higher $\gamma$, the cost of FIX becomes more comparable to ORC. However, combining this with Theorem~\ref{thm:max_iou} reveals that FIX can either match ORC in cost but with lower expected accuracy or achieve similar accuracy at a much higher cost.

The right panel of Figure~\ref{fig:cost_1} examines the impact of the cost ratio $r$. Across all tested values, ORC remains less costly than FIX in the default setting ($T=100$, $r=0.5$, $\gamma=0.5$, $M=1$). This confirms that the relative cost advantage of ORC is robust to changes in $r$.

\begin{figure}
    \centering
    \includegraphics[width=0.49\textwidth]{figures/cost_orc_fix_M.png}
    \includegraphics[width=0.49\textwidth]{figures/cost_orc_fix_T.png}
    \caption{The relative cost of FIX and ORC for varying number of sound events $M$ (left) and recording lengths $T$ (right). The default parameters are: $T=100$, $r=0.5$, $M=1$ and $\gamma=0.5$. We simulate overestimating the number of needed queries $B_{\text{ORC}} = s(2M+1)$ by a factor of $s$ for $s \in \{1, 2, 4, 8\}$ to see how this affects the relative cost. The cost of FIX is greater than the cost of ORC above the dashed red line where the cost ratio is $1$.}
    \label{fig:cost_2}
\end{figure}

\subsubsection{Effect of number of events ($M$) and recording length ($T$).} Figure~\ref{fig:cost_2} explores the impact of $M$ and $T$ on the relative cost. In the left panel, we see that for $s=1$, ORC is less costly than FIX when the number of events is below $60$. However, as $s$ increases to $8$, FIX becomes less costly when at most $10$ events are present. These results indicate that the relative cost depends heavily on the density of sound events in the recording and the estimated annotation budget for ORC. 

In the right panel, varying $T$ shows a similar trend. For shorter recordings (high event density), ORC loses its cost advantage. However, it’s important to note that the maximum achievable expected label accuracy with FIX under default settings ($\gamma=0.5$) is $f^*(0.5) \approx 0.76$, whereas ORC achieves $1.0$. In such cases, the additional cost of ORC may be justified by the significantly higher label quality.

While these results indicate that the relative cost depends on the sound event density, we should remember that we are considering weak labeling of presence events. This implies that all $M$ events in this analysis are treated as non-overlapping, as the annotation task does not consider temporal overlaps for this analysis. The scenario of $M > 60$ non-overlapping events of length $1$ in a recording of length $T=100$ is therefore unlikely in practice. Similarly, estimating $10$ events as $80$ (modeled by $s=8$) for an audio recording of length $T=100$ represents a substantial overestimation and seems improbable given the capabilities of modern sound event detection tools. %Modern sound event detection tools, which leverage advances in machine learning, typically provide reliable upper bounds on the number of events, making extreme overestimations less likely.
\section{Related Work}
\label{sec:related}


\noindentbold{2D visual foundation models}
In recent years, we have witnessed the emergence of large pretrained models—so-called foundation models that are trained on large-scale datasets and serve as a \textit{foundation} for many downstream tasks.
These models demonstrate remarkable versatility across multiple modalities, including language~\cite{team2023gemini,touvron2023llama,touvron2023llama2,dubey2024llama3,vicuna2023,radford2019language,brown2020language,chung2024scaling,achiam2023gpt,bai2023qwen,yang2024qwen2,jiang2023mistral,jiang2024mixtral}, vision~\cite{sam,ravi2024sam,dino_v1,oquab2023dinov2,zou2024segment,rombach2022high,ho2020denoising,nichol2021improved,songdenoising,songscore}, audio~\cite{deshmukh2023pengi,zhang2023speechgpt,rubenstein2023audiopalm,borsos2023audiolm}. 
Furthermore, they enable multi-modal reasoning capabilities that bridge across different modalities~\cite{girdharImageBindOneEmbedding2023,Qwen-VL,llava,radfordLearningTransferableVisual2021,jia2021scaling,team2024gemini}.
Among these models, those that operate on visual modalities are known as visual foundation models (VFM).
VFMs excel in various computer vision tasks such as image segmentation~\cite{sam,ravi2024sam,zou2024segment,zou2023generalized,cheng2021per,cheng2022masked,jain2023oneformer,li2024semantic}, object detection~\cite{liu2023grounding,carion2020end}, representation learning~\cite{dino_v1,oquab2023dinov2}, and open-vocabulary understanding~\cite{radfordLearningTransferableVisual2021,li2022language,ghiasi2022scaling,ram,ram_pp,yu2023convolutions,kang2024defense,naeem2024silc,cho2024cat}.
When integrated with large language models, they enable sophisticated visual reasoning and natural language interactions~\cite{llava,Qwen-VL,girdharImageBindOneEmbedding2023,team2024gemini,guo2024regiongpt,yuan2024osprey,you2023ferret}.
We use such vision language models to construct open vocabulary segmentation and captions for point clouds based on multiview images.







\noindentbold{Open-vocabulary 3D segmentation}
Building on the success of 2D VFMs, recent work have extended open-vocabulary capabilities to 3D scene understanding.
OpenScene~\cite{Peng2023OpenScene} first introduced zero-shot 3D semantic segmentation by distilling knowledge from language-aligned image encoders~\cite{li2022language,ghiasi2022scaling}.
Subsequent methods~\cite{ding2022pla,yang2024regionplc,jiang2024open} leverage multiview images to generate textual captions, which then serve as training supervision.
However, these methods face challenges in generating high-quality 3D mask-text pairs at scale.
For open-vocabulary 3D instance segmentation, existing methods~\cite{takmaz2023openmask3d,nguyen2024open3dis,huang2024openins3d} typically rely on closed-vocabulary proposal networks such as Mask3D~\cite{schult2023mask3d}, which inherently constrains their ability to detect novel object categories. 
Moreover, these methods leverage 2D VFMs like CLIP~\cite{radfordLearningTransferableVisual2021} for region classification by projecting 3D regions onto multiple 2D views.
This approach requires both 2D images and 3D point clouds during inference. Additionally, it necessitates multiple inferences of large 2D models on projected masks, resulting in high computational costs. 
We address these limitations by developing the first single-stage open-vocabulary 3D instance segmentation model that operates directly in 3D without ground truth labels, using our \dataname dataset and Segment3D~\cite{huang2024segment3d} proposals.

\noindentbold{3D vision-language datasets}
Several datasets align 3D scenes with textual annotations to facilitate language-driven 3D understanding. 
ScanRefer~\cite{chen2020scanrefer}, ReferIt3D~\cite{achlioptas2020referit_3d} and EmbodiedScan~\cite{wangEmbodiedScanHolisticMultiModal2023} provide fine-grained object-level localization through detailed referential phrases, while ScanQA~\cite{azuma2022scanqa} targets spatially grounded question-answering. 
In contrast, SceneVerse~\cite{jiaSceneVerseScaling3D2024} and MMScan~\cite{lyu2024mmscan} employ large-language models or vision-language models to partially automate annotation.
Despite leveraging advanced models, these datasets depend significantly on costly human annotations derived from closed-vocabulary sources, limiting their support for open-vocabulary and scalability for large-scale 3D segmentation tasks.

\section{Discussion}
\label{sec:discussion}

FIX labeling has been employed in many works, with varying degrees of complexity. Theorem~\ref{thm:fix_optimal_query_length} provides a useful rule of thumb for selecting the best segmentation length for a given event length, and Eq.~\ref{eq:expected_query_iou_distribution} provides a way to use this theorem to analyze stochastic event length distributions. Our results suggest that, in most cases, knowing the average event length provides a good estimate, but understanding the (approximate) distribution of event lengths improves the analysis.

\textbf{Implications for practical annotation.} The analysis highlights the trade-offs in label accuracy and annotation cost between FIX and ORC weak labeling. While FIX can be less costly under specific conditions (e.g., high event density), these conditions are unlikely to occur in real-world annotation tasks. Furthermore, even in cases where FIX is less costly, its significantly lower label accuracy ($f^*(0.5) \approx 0.76$ vs. $1.0$ for ORC) can negate its cost advantage. ORC, on the other hand, guarantees higher accuracy at a potentially higher cost, which is sensitive to overestimation of $B_{\text{ORC}}$. 

Given the rarity of extreme event densities and the importance of high-quality labels, ORC is likely the better theoretical choice for most annotation tasks. However, ORC weak labeling is not available in practice since it uses the true change points of the events. An interesting research direction is to model the ORC weak labeling process by estimating these change points, construct query segments from these, and then let the annotator weakly label these query segments. \citet{Martinsson2024} propose to actively model the ORC weak labeling process during the annotation for sound event detection, and \citet{Kim2023} propose a related framework for image segmentation. However, how to model ORC weak labeling reliably in practice remains an open research question, as it may introduce unwanted bias due to annotation errors, overfitting to sparse events, or limitations in budget estimation models. In this regard, FIX weak labeling is very robust and provides a straightforward baseline that prioritizes simplicity and consistency.

Future research should focus on mitigating the potential biases when modeling ORC weak labeling while retaining its theoretical advantages. By addressing these challenges, annotation methods can better align with the practical needs of sound event detection and related applications.

\textbf{Understanding the consequences of evaluating with noisy labels.}
Despite the extensive focus on noisy training labels, evaluation labels are often implicitly assumed to be perfect. However, as emphasized in the introduction, inaccurate evaluation labels present a significant challenge. Crucially, when noise is present in both training and evaluation data, we risk selecting models that merely replicate the evaluation noise, potentially overlooking those with superior generalization abilities. This echoes the central concern highlighted by \citet{Görnitz2014}, which will be described below. To illustrate this point concretely, consider the implications of evaluating with noisy labels as described by our theorem.

We can use Theorem~\ref{thm:max_iou} to understand the properties of the best performing model when the evaluation data contains FIX weak labels. For example, we now know from Theorem~\ref{thm:max_iou} that for $\gamma=0.5$ the annotations will at most have an expected label accuracy of $f^*(0.5) \approx 0.76$. The ``best'' performing model will therefore be a model that mimics the noise in these labels. We may end up rejecting a model that has learned the clean (accurate) labels simply because our evaluation labels are inaccurate. The theory therefore provides a better understanding for the ``best'' performing model under FIX weak labels.

\textbf{$f^*(\gamma)$ as an upper bound.}
The expression for expected label accuracy derived in this paper applies to the simplest scenario, where only a single event with deterministic length is present. In all of our results, we observe that $f^*(\gamma)$ is greater than or equal to the expected and average label accuracy that FIX weak labeling achieve for more complex distributions. This suggests that $f^*(\gamma)$ can be considered an upper bound. However, a formal proof showing that adding more events or introducing event length variability leads to a harder distribution to annotate is beyond the scope of this paper.

\textbf{Extending the theory to more dimensions.}
The theory we have derived is for data annotation in one dimension, in particular we exemplify with events in time for audio data. However, the framework can be extended to more dimensions. Annotation in more than three dimensions is hard, since human intuition starts to break down and designing annotation interfaces will be hard, so for most practical applications extending this theory to events that occur in three dimensions should be enough. With minor adjustments, this framework can be applied to FIX weak labeling of rectangles in images or cubes in point clouds, expanding its relevance to broader areas. This extension could open up new avenues for research in multi-dimensional label noise and help optimize annotation strategies in these domains.

\textbf{Multiple different presence classes.}
If the same presence criterion $\gamma$ is applicable for all classes then Theorem~\ref{thm:expected_iou} is applicable to the joint event length distribution of these different classes. The integration performed in Eq.~\ref{eq:expected_query_iou_distribution} can be adapted for the joint distribution, assuming it can be estimated. However, real-world presence criteria for different event classes may vary, requiring more complex models to account for these discrepancies. Future empirical studies on annotator behavior could help refine this model and improve its practical applicability.

%\textbf{Understanding segment-based evaluation criteria.}
%The theory presented here is also relevant for evaluating sound event detection (SED) methods. Common evaluation metrics, such as the segment-based F$_1$ score~\citep{Mesaros2016}, divide audio into fixed-length segments and label them based on overlap with ground truth. Assuming ground truth labels for evaluation we effectively have an annotator that can detect the presence of arbitrarily small event fractions ($\gamma \rightarrow 0$). The expected label accuracy becomes $f(d_q) = d_e / (d_e + d_q)$, where $d_q$ is the segment length. This formula suggests that a small $d_q$ (approaching zero) leads to minimal segment label noise, but choosing a very small segment length comes at a computational cost even during evaluation. The theory can help inform these trade-offs, ensuring that the accepted level of absence sound in presence labels is satisfactory for the given application.

\textbf{Understanding segment-based evaluation criteria.}
The theory presented here is also relevant for evaluating sound event detection (SED) methods. Common evaluation metrics, such as the segment-based F$_1$ score~\citep{Mesaros2016}, divide audio into fixed-length segments and label them based on overlap with ground truth. A key reason for using segment-based evaluation is that ground truth annotations are often temporally noisy or imprecise, and evaluating over longer segments helps to mitigate the impact of this temporal uncertainty. Assuming ground truth labels for evaluation, we effectively have an annotator that can detect the presence of arbitrarily small event fractions ($\gamma \rightarrow 0$). The expected label accuracy becomes $f(d_q) = d_e / (d_e + d_q)$, where $d_q$ is the segment length. This formula suggests that a small $d_q$ (approaching zero) leads to minimal segment label noise, but choosing a very small segment length negates the noise reduction effect and also comes at a computational cost. The theory presented here can help inform such trade-offs. %, ensuring that the accepted level of absence sound in presence labels is satisfactory for the given application.


\section{Conclusions}\label{sec:conclusions}

Our examination of the landscape of software testing within \iot platforms derived substantive insights, with 10 key findings from both a mining-based study and a survey with developers. 

We took a closer look at two specific platforms – \homeassistant and \openhab. 
We find notable evidence signaling the difficulty that developers face with testing IoT platforms, with the majority of \addons and integration apps of both platforms falling short of the 50\% test coverage threshold.
On average, only 5\% \addons contains any test methods for \openhab and well-known brand like Amazon Alexa exhibits a maximum test ratio of 59\%.

A majority of our survey participants prefer automated testing, to try to catch problems early in the development process to save time and make things more reliable. They stress the need for consistent, repeatable tests and detailed logs. But interestingly, some developers still prefer manual testing as it facilitates human intuition and adaptability for real-world situations.

Finally, our research identifies challenges in fixing problems caused by software updates, cross-platform testing, and dealing with issues in low-power devices -- among others. In summary, our study sheds light on the testing practices, tools, perceptions, and challenges for IoT platforms, illustrating promising pathways for future research to improve testing for this rapidly growing domain. 



% TODO: remove for submission (anonymize)
%\textbf{!!!!!!!!!!!!!!!! TODO: remove section 10 before submission !!!!!!!!!!!!!!!!!}

% \subsubsection*{Broader Impact Statement}
% In this optional section, TMLR encourages authors to discuss possible repercussions of their work,
% notably any potential negative impact that a user of this research should be aware of. 
% Authors should consult the TMLR Ethics Guidelines available on the TMLR website
% for guidance on how to approach this subject.

\subsubsection*{Author Contributions}
If you'd like to, you may include a section for author contributions as is done
in many journals. This is optional and at the discretion of the authors. Only add
this information once your submission is accepted and deanonymized. 

\subsubsection*{Acknowledgments}

The authors would like to express their gratitude to the following individuals and organizations for their contributions and support:

\begin{itemize}
    \item \textbf{Annamaria Mesaros}: Provided insightful questions about an early draft of this work that provoked a revision of the metric used to evaluate label quality.
    \item \textbf{Magnus Oskarsson}: Contributed a proof sketch for Theorem~\ref{thm:fix_optimal_query_length} and posed insightful questions about the paper.
    \item \textbf{Edvin Listo Zec}: Assisted in deriving a proof of Theorem~\ref{thm:fix_optimal_query_length} by effectively prompting a large language model.
    \item This work was supported by The Swedish Foundation for Strategic Research (SSF; FID20-0028) and Sweden’s Innovation Agency (2023-01486).
\end{itemize}














\bibliography{main}
\bibliographystyle{tmlr}

\appendix
\section{Appendix}
\label{app:appendix}

We do not include all simplifications of expressions in the proofs, but we do provide the code for a symbolic mathematics solver (SymPy) at GitHub\footnote{link will be added for camera-ready, for now see the supplementary material}, where all results can be verified. The notebook named ``symbolic\_verification\_of\_analysis.ipynb'' can be used to verify the analysis. 


%In fact, the only thing that you need to be convinced of are the entries in Table~\ref{tab:expressions} which are derived in Appendix~\ref{app:details_on_expressions}, all other results directly follow and have been verified in the notebook using the symbolic mathematics solver.


\subsection{Proof of Theorem~\ref{thm:expected_iou}}
\label{app:proof_of_theorem}


% \begin{figure}[H]
%     \centering
%     \includegraphics[width=0.8\textwidth]{figures/metric_at_all_overlap_occurences_1.pdf}
%     \caption{The label accuracy, $F(e_t, q, \gamma)$, for $t\in [t_0, t_5]$ for the two cases: (i) $d_e \geq d_q$, and (ii) $d_e \leq d_q$, where $t_0 = 0$ and $t_5 = d_e + d_q$. In appendix~\ref{app:appendix} we show how $t_1$, $t_2$, $t_3$ and $t_4$, where the discontinuities occur, can be defined as functions of $d_e$, $d_q$ and $\gamma$, and use these together with the function evaluations at the discontinuities to compute $A_1$, $A_2$ and $A_3$ for each case (i) and (ii).}
%     \label{fig:query_iou}
% \end{figure}

% \begin{claim}
% The expected query segment accuracy of an arbitrary query segment $q$ of length $d_q$ which overlaps with an event $e$ of length $d_e$ using annotator presence criterion $\gamma$ is
% \begin{equation}
% \label{eq:expected_iou_proof}
%     P(d_e, d_q, \gamma) = \E_{\rt \sim p}\left[F(e_{\rt}, q, \gamma)\right] = \begin{cases}
%     \frac{d_{e} \left(- 2 d_{e} \gamma^{2} + 2 d_{q} \gamma + d_{q}\right)}{d_{q} \left(d_{e} + d_{q}\right)}, & \text{ if } d_q \geq \gamma d_e, \\
%     \frac{d_q}{d_e + d_q}, & \text{ if } d_q < \gamma d_e.
%     \end{cases}
% \end{equation}
% \end{claim}


\label{app:thm1}

We will derive an expression for the expected query segment accuracy given overlap with a single event in terms of $d_e$, $d_q$, and $\gamma$, under all possible assumptions which will prove Theorem~\ref{thm:expected_iou}. 

\begin{proof}
We need to consider two main assumptions. The first assumption is that the presence criterion for the annotator can be fulfilled, that is, $d_q \geq \gamma d_e$, and the second assumption is that the annotator presence criterion can not be fulfilled, that is, $d_q < \gamma d_e$. This happens if the query segment length is so short that it can never cover a large enough fraction of the event of interest to make presence detection feasible.

\begin{assumption}
    The annotator presence criterion can be fulfilled ($d_q \geq \gamma d_e$).
\end{assumption}
Under this assumption there are two possible cases for the relation between $d_q$ and $d_e$, either the event length is longer or equal to the query segment length, $d_e \geq d_q$ (case i), or the event length is shorter than the query segment length, $d_e < d_q$ (case ii). In Figure~\ref{fig:query_segment_accuracy}, we plot the query segment accuracy, $F(e_t, q, \gamma)$, for $t \in [0, d_e+d_q]$ for case (i) on the left, and case (ii) on the right. 
We describe in more detail in Appendix~\ref{app:details_on_expressions} how the query segment accuracy behaves as a function of different amounts of overlap between the query segment and the event. Briefly, what we see in Figure~\ref{fig:query_segment_accuracy} is that initially there is arbitrarily little overlap ($t_0^{(i)}$ and $t_0^{(ii)}$), an absence label is given to the query segment and the accuracy is therefore $1$. Then the accuracy decrease linearly with the amount of overlap until the presence criterion is fulfilled and a presence label is given ($t_1^{(i)}$ and $t_1^{(ii)}$). After that, the accuracy linearly increase with the amount of overlap between the event and query segment until we reach a ceiling for the accuracy when either the whole query segment is inside the event ($t_2^{(i)}$) or the query segment covers the whole event ($t_2^{(ii)}$). Finally, the overlap between the query segment and the event starts to decrease again ($t_3^{(i)}$ and $t_3^{(ii)})$, and everything is symmetrical.

We continue by dropping the case superscripts show in the figure for $A_1, \dots, A_3$ and $t_0, \dots, t_5$, and only provide the full proof for case (i), but the proof for case (ii) is similar. In both cases the area $A$ in Eq.~\ref{eq:normalized_area} can be divided into five distinct parts:
\begin{equation}
\label{eq:area}
    A = 2A_1 + 2A_2 + A_3,
\end{equation}
where $A_1$ and $A_2$ are counted twice due to symmetry.

\begin{figure}[H]
    \centering
    \includegraphics[width=0.8\textwidth]{figures/metric_at_all_overlap_occurences_1.pdf}
    \caption{Assuming $d_q \geq d_e \gamma$, we plot the query segment accuracy, $F(e_t, q, \gamma)$, for $t\in [0, d_e + d_q]$, where $t_0 = 0$ and $t_5 = d_e + d_q$. Case (i) where $d_e \geq d_q$ is shown in the left panel, and case (ii) where $d_e < d_q$ is shown in the right panel.}
    \label{fig:query_segment_accuracy}
\end{figure}

The variables $t_0, t_1, \dots, t_5$, represent the different states $t$ of overlap where the discontinuities of $F(e_t, q, \gamma)$ occur, and using these we can express the areas as the following integrals:
\begin{equation}
    A_1 = \int_{t_0}^{t_1} F(e_t, q, \gamma)\mathrm{d}t = \int_{t_4}^{t_5} F(e_t, q, \gamma)\mathrm{d}t,
\end{equation}
and
\begin{equation}
    A_2 = \int_{t_1}^{t_2} F(e_t, q, \gamma)\mathrm{d}t = \int_{t_3}^{t_4} F(e_t, q, \gamma)\mathrm{d}t,
\end{equation}
due to symmetry, and
\begin{equation}
    A_3 = \int_{t_2}^{t_3} F(e_q, q, \gamma)\mathrm{d}t.
\end{equation}
% We now need to show that
% \begin{align}
%     P(d_e, d_q, \gamma) &= \E_{\rt \sim p}\left[F(e_{\rt}, q, \gamma)\right] \\
%     \label{eq:case_i}
%     &= \frac{2A^{(i)}_{1} + 2A^{(i)}_{2} + A^{(i)}_{3}}{d_e + d_q} \\
%     \label{eq:case_ii}
%     &= \frac{2A^{(ii)}_{1} + 2A^{(ii)}_{2} + A^{(ii)}_{3}}{d_e + d_q} \\
%     \label{eq:expectation_claim}
%     &= \frac{d_{e} \left(- 2 d_{e} \gamma^{2} + 2 d_{q} \gamma + d_{q}\right)}{d_{q} \left(d_{e} + d_{q}\right)}.
% \end{align}
We use that the query segment accuracy $F(e_t, q, \gamma)$ is linear in each interval, which means that the areas can be expressed as
\begin{equation}
\label{eq:a_1}
    A_1 = \frac{F(e_{t_0}, q, \gamma) + F(e_{t_{1^-}}, q, \gamma)}{2} (t_1 - t_0),
\end{equation}
\begin{equation}
\label{eq:a_2}
    A_2 = \frac{F(e_{t_{1^+}}, q, \gamma) + F(e_{t_2}, q, \gamma)}{2} (t_2 - t_1),
\end{equation}
and
\begin{equation}
\label{eq:a_3}
    A_3 = \frac{F(e_{t_2}, q, \gamma) + F(e_{t_3}, q, \gamma)}{2} (t_3 - t_2),
\end{equation}
where $t^-$ indicate that we approach the discontinuity at $t$ from below and $t^+$ from above. We now only need to express $t_0, \dots, t_3$ and $F(e_{t_0}, q, \gamma), \dots, F(e_{t_3}, q, \gamma)$ in terms of $d_e$, $d_q$ and $\gamma$ to conclude the proof. For brevity, these have been provided in Table~\ref{tab:expressions}. See section~\ref{app:details_on_expressions} for details on how to express these in terms of $d_q$, $d_e$ and $\gamma$.

\begin{table}[]
    \centering
    \begin{tabular}{l l | l l}
         \multicolumn{2}{c|}{Case (i), $d_e \geq d_q$} & \multicolumn{2}{c}{Case (ii), $d_e < d_q$} \\
         \hline
         $t^{(i)}_0 = 0$          & $F(e^{(i)}_{t_0}, q, \gamma) = 1$   & $t^{(ii)}_0 = 0$          & $F(e^{(ii)}_{t_0}, q, \gamma) = 1$ \\
         $t^{(i)}_1 = \gamma d_e$ & $F(e^{(i)}_{t_1^-}, q, \gamma) = \frac{d_q - \gamma d_e}{d_q}$ & $t^{(ii)}_1 = \gamma d_e$ & $F(e^{(ii)}_{t_1^-}, q, \gamma) = \frac{d_q - \gamma d_e}{d_q}$ \\
         $t^{(i)}_2 = d_q$        & $F(e^{(i)}_{t_1^+}, q, \gamma) = \frac{\gamma d_e}{d_q}$ & $t^{(ii)}_2 = d_e$        & $F(e^{(ii)}_{t_1^+}, q, \gamma) = \frac{\gamma d_e}{d_q}$ \\
         $t^{(i)}_3 = d_e$        & $F(e^{(i)}_{t_2}, q, \gamma) = 1$   & $t^{(ii)}_3 = d_q$        & $F(e^{(ii)}_{t_2}, q, \gamma) = \frac{d_e}{d_q}$ \\
                                  & $F(e^{(i)}_{t_3}, q, \gamma) = 1$   &                           & $F(e^{(ii)}_{t_3}, q, \gamma) = \frac{d_e}{d_q}$ \\
    \end{tabular}
    \caption{A summary of the derived expressions for $t_0, \dots, t_3$ and $F(e_{t_0}, q, \gamma), \dots, F(e_{t_3}, q, \gamma)$ for each case. $F(e_{t_1^-}, q, \gamma)$ and $F(e_{t_1^+}, q, \gamma)$ denotes the limits when approaching $t_1$ from below and above respectively.}
    \label{tab:expressions}
\end{table}

We provide the steps for case (i), and leave the derivation for case (ii) to the reader. We substitute the expressions for case (i), provided in Table~\ref{tab:expressions}, into equations Eq.~\ref{eq:a_1}-\ref{eq:a_3}, and the resulting expressions for the areas $A^{(i)}_1$, $A^{(i)}_2$, and $A^{(i)}_3$ into Eq.~\ref{eq:area} which give

% \begin{align}
% A^{(i)}_1 = \frac{1 + (d_q - \gamma d_e)/d_q}{2}(\gamma d_e),
% \end{align}

% \begin{align}
% A^{(i)}_2 = \frac{1 + \gamma d_e/d_q}{2}(d_q - \gamma d_e),
% \end{align}
% and
% \begin{align}
% A^{(i)}_3 = d_e - d_q.
% \end{align}
% Finally, $A^{(i)}_1$, $A^{(i)}_2$, and $A^{(i)}_3$ substituted into Eq.~\ref{eq:area}

\begin{align*}
A^{(i)} &= \frac{2}{2}(1+\frac{d_q-\gamma d_e}{d_q})\gamma d_e
+ \frac{2}{2}(1 + \frac{\gamma d_e}{d_q})(d_q - \gamma d_e)
+ (d_e - d_q) \\
&= (2d_q - \gamma d_e)\frac{\gamma d_e}{d_q} + (d_q + \gamma d_e)(d_q - \gamma d_e)\frac{1}{d_q} + (d_e - d_q) \\
&= \frac{1}{d_q}(2\gamma d_q d_e - \gamma^2 d_e^2 + \cancel{d_q^2} - \gamma^2 d_e^2 + d_e d_q - \cancel{d_q^2}) \\
&= \frac{1}{d_q}(2\gamma d_q d_e - 2\gamma^2 d_e^2 + d_e d_q) \\
&= \frac{d_e}{d_q}(2\gamma d_q - 2\gamma^2 d_e + d_q).
\end{align*}
Finally, by substituting $A$ for $A^{(i)}$ in Eq.~\ref{eq:normalized_area} we arrive at
\begin{equation}
    \frac{A^{(i)}}{d_e + d_q} = \frac{d_e(2\gamma d_q - 2\gamma^2 d_e + d_q)}{d_q(d_e + d_q)}
\end{equation}

which shows that Eq.~\ref{eq:expected_iou} holds for case (i) under the assumption that $d_q \geq \gamma d_e$. Similarly, this also holds for case (ii).

% \begin{equation}
%     \E_{\rt \sim p}\left[F(e_{\rt}, q, \gamma)\right] = \frac{d_{e} \left(- 2 d_{e} \gamma^{2} + 2 d_{q} \gamma + d_{q}\right)}{d_{q} \left(d_{e} + d_{q}\right)}.
% \end{equation}

\begin{assumption}
    The annotator presence criterion can not be fulfilled ($d_q < \gamma d_e$).
\end{assumption}

When the presence criterion can not be fulfilled we never get any presence labels, this means that the fraction of the query segment that overlaps with an event is always incorrectly given an absence label. When the query segment completely overlaps with an event the query segment accuracy will be $0$ (seen between $t_1$ and $t_2$ in Figure~\ref{fig:query_segment_accuracy_2}).

\begin{figure}[H]
    \centering
    \includegraphics[width=0.4\textwidth]{figures/metric_at_all_overlap_occurences_2.pdf}
    \caption{Assuming that $d_q < \gamma d_e$, we plot the query segment accuracy, $F(e_t, q, \gamma)$, for $t\in [0, d_e + d_q]$, where $t_0 = 0$ and $t_3 = d_e + d_q$.}
    \label{fig:query_segment_accuracy_2}
\end{figure}

The area $A_1$ is counted twice due to symmetry. The discontinuity at $t_1$ occurs for the smallest $t\in [0, d_e + d_q]$ for which $F(e_t, q, \gamma)=0$, which happens for the smallest $t$ for which the whole query segment overlaps with the event $|e \cap q| = d_q$ at $t=d_q$. We therefore have that $t_1 - t_0 = t_3 - t_2 = d_q$.

When there is no overlap between the query segment and the event giving a presence label is always correct, thus $F(e_{t_0}, q, \gamma) = 1$. However, giving an absence label to a query segment that completely overlaps with an event gives the query segment accuracy $0$, thus $F(e_{t_1}, q, \gamma) = 0$. The total area under the curve is therefore $2A_1 = d_q$ and by normalizing with $t_3-t_0 = d_e + d_q$, we get $d_q / (d_e + d_q)$, which proves the $d_q < \gamma d_e$ case of Eq.~\ref{eq:expected_iou}, and concludes the proof.

% \begin{equation}
%     \E_{\rt \sim p}\left[F(e_{\rt}, q, \gamma)\right] = \frac{d_q}{d_e + d_q},
% \end{equation}
% which concludes the proof.
\end{proof}

%%%%%%%%%%%%%%%%%%%%%%%%%%%%%%%%%%%%%%%%%%%%%%%%%%%%%%%%%%%%%%%%%%%%%%%%%%%%%%%%%
% \subsubsection{Details on the expressions in Table~\ref{tab:expressions}}
% \label{app:details_on_expressions}

% \textbf{TODO: this section needs a bit more polish.}

% We start by noting that $t^{(i)}_0 = t^{(ii)}_0 = 0$ and $t^{(i)}_1 = t^{(ii)}_1 = \gamma d_e$. By definition, $t^{(i)}_0 = t^{(ii)}_0 = 0$ for both cases since $t\in[0, d_e+d_q]$, when $t=0$ the end of the event is exactly aligned with the start of the query segment. Further, $F(e_{t_0}, q, \gamma)=\frac{d_e - |e_{t_0}\cap q|}{d_q} = 1$, since $|e_{t_0}\cap q|=0$ at $t_0=0$. Also, $t_1 = t^{(i)}_1 = t^{(ii)}_1 = \gamma d_e$ for both cases. The discontinuity we see at $t_1$ in Figure~\ref{fig:query_segment_accuracy} for the respective case is when $F(e_t, q, \gamma)$ goes from $\frac{d_q-|e_t\cap q|}{d_q}$ (criterion for presence not fulfilled) to $\frac{|e_t\cap q|}{d_q}$ (criterion for presence fulfilled), which occurs for the smallest $t\in[0, d_e+d_q]$ for which the annotator presence criterion is fulfilled. This happens when $h(e_t, q) = \gamma$ which by Eq.~\ref{eq:event_fraction} means that $|e_{t} \cap q| = \gamma d_e$. Therefore, $t_1=t^{(i)}_1=t^{(ii)}_1=\gamma d_e$ (see $t=t_1$ in Figure~\ref{fig:case_i} and Figure~\ref{fig:case_ii} for each case). Further,

% \begin{equation}
%     \lim_{t \to t_1^-} F(e_{t}, q, \gamma) = \frac{d_q - \gamma d_e}{d_q}
% \end{equation}

% \begin{equation}
%     \lim_{t \to t_1^+} F(e_{t}, q, \gamma) = \frac{\gamma d_e}{d_q}
% \end{equation}

% Now we look at $t_2$ and $t_3$, which differ for the two cases.

% \begin{figure}[H]
%     \centering
%     \includegraphics[width=0.8\textwidth]{figures/case_i.png}
%     \caption{An illustration of how the sound event $e_t$ and the query segment $q$ overlap at the four distinct states $t = t_0, \dots, t_3$ for case (i) where $d_e \geq d_q$.}
%     \label{fig:case_i}
% \end{figure}

% $t^{(i)}_2 = d_q$.
% The shift we see at $t^{(i)}_2$ in Figure~\ref{fig:query_segment_accuracy} occurs for the smallest $t\in[0, d_e+d_q]$ where the maximum query segment accuracy is achieved. That is, the smallest $t$ where the entire query segment overlaps with the event. This happens when the end of the sound event aligns with the end of the query segment. This happens at $t^{(i)}_2=d_q$ (see $t=t_2$ in Figure~\ref{fig:case_i}).

% $t^{(i)}_3=d_e$.
% This shift we see at $t^{(i)}_3$ in Figure~\ref{fig:query_segment_accuracy} occurs for the maximal $t\in[0, d_e+d_q$ where the maximal query segment accuracy occurs. That is, the largest $t$ where the entire query segment overlaps with the event. This happens when $t^{(i)}_3=d_e$ (see $t=t_3$ in Figure~\ref{fig:case_i}).

% Further, we get 
% \begin{align}
%     F(e_{t^{(i)}_2}, q, \gamma) &= \frac{|e_{t^{(i)}_2} \cap q|}{d_q} \\
%     &= \frac{d_q}{d_q} \\
%     &= \frac{|e_{t^{(i)}_3} \cap q|}{d_q} \\
%     &= F(e_{t^{(i)}_3}, q, \gamma)
% \end{align}
% because the whole query segment overlaps with the event in both cases.

% \begin{figure}[H]
%     \centering
%     \includegraphics[width=0.8\textwidth]{figures/case_ii.png}
%     \caption{An illustration of how the sound event $e_t$ and the query segment $q$ overlap at the four distinct states $t = t_0, \dots, t_3$ for case (ii) where $d_e < d_q$.}
%     \label{fig:case_ii}
% \end{figure}

% $t^{(ii)}_2 = d_q$.
% The shift we see at $t^{(ii)}_2$ in Figure~\ref{fig:query_segment_accuracy} occurs for the smallest $t\in[0, d_e+d_q]$ where the maximum query segment accuracy is achieved. That is, the smallest $t$ where the entire query segment overlaps with the event. That is, the beginning of the sound event aligns with the beginning of the query segment. This happens at $t^{(ii)}_2=d_e$ (see $t=t_2$ in Figure~\ref{fig:case_ii}).

% $t^{(ii)}_3=d_e$.
% This shift we see at $t^{(ii)}_3$ in Figure~\ref{fig:query_segment_accuracy} occurs for the maximal $t\in[0, d_e+d_q$ where the maximal query segment accuracy occurs. That is, the largest $t$ for which the query segment overlaps with the entire event. This happens when $t^{(ii)}_3=d_q$ (see $t=t_3$ in Figure~\ref{fig:case_ii}).

% Further, we get 
% \begin{align}
%     F(e_{t^{(ii)}_2}, q, \gamma) &= \frac{|e_{t^{(ii)}_2} \cap q|}{d_q} \\
%     &= \frac{d_e}{d_q} \\
%     &= \frac{|e_{t^{(i)}_3} \cap q|}{d_q} \\
%     &= F(e_{t^{(ii)}_3}, q, \gamma)
% \end{align}
% because the overlap is with the whole event in both cases.
%%%%%%%%%%%%%%%%%%%%%%%%%%%%%%%%%%%%%%%%%%%%%%%%%%%%%%%%%%%%%%%%%%%%%%%%%%%%%%%%%%
\subsubsection{Details on the expressions in Table~\ref{tab:expressions}}
\label{app:details_on_expressions}

This section provides a detailed explanation of the values presented in Table~\ref{tab:expressions}. For each case (i) and (ii), we will define the specific time points  $t_0, t_1, t_2, t_3$ where the query segment accuracy function $F(e_t, q, \gamma)$ changes, and explain the corresponding value of the function at these points based on the overlap between the event $e_t$ and the query segment $q$. The states $t_4$ and $t_5$ are analogous to $t_1$ and $t_0$, respectively, and therefore not illustrated. The difference is that the amount of overlap between the query segment and event decreases (instead of increases) when approaching these states.

\textbf{Case (i): $d_e \geq d_q$}

\begin{figure}[H]
    \centering
    \includegraphics[width=0.8\textwidth]{figures/case_i.png}
    \caption{An illustration of how the sound event $e_t$ and the query segment $q$ overlap at the four distinct states $t = t_0, \dots, t_3$ for case (i) where $d_e \geq d_q$.}
    \label{fig:case_i}
\end{figure}

\begin{itemize}
    \item $t^{(i)}_0$: At $t^{(i)}_0 = 0$, the end of the event $e_t$ aligns perfectly with the beginning of the query segment $q$. This means there is no overlap between the event and the query segment ($|e_{t^{(i)}_0} \cap q| = 0$). Therefore, assuming the annotator absence criterion applies, the query segment accuracy is $F(e_{t^{(i)}_0}, q, \gamma) = \frac{d_q - |e_{t^{(i)}_0} \cap q|}{d_q} = \frac{d_q - 0}{d_q} = 1$.
    \item $t^{(i)}_1$: The time $t^{(i)}_1 = \gamma d_e$ represents the point where the annotator presence criterion is first met. Before this point ($t < t^{(i)}_1$), the overlap $|e_t \cap q|$ is less than $\gamma d_e$, and the query segment accuracy is given by $F(e_t, q, \gamma) = \frac{d_q - |e_t \cap q|}{d_q}$. As $t$ approaches $t^{(i)}_1$ from the left, $|e_t \cap q|$ approaches $\gamma d_e$, hence $\lim_{t \to t_1^-} F(e_{t}, q, \gamma) = \frac{d_q - \gamma d_e}{d_q}$. At $t = t^{(i)}_1$, the presence criterion is met, and the accuracy function switches to $F(e_t, q, \gamma) = \frac{|e_t \cap q|}{d_q}$. As $t$ approaches $t^{(i)}_1$ from the right, $|e_t \cap q|$ is slightly greater than $\gamma d_e$, and $\lim_{t \to t_1^+} F(e_{t}, q, \gamma) = \frac{\gamma d_e}{d_q}$. This transition is visually represented in Figure~\ref{fig:case_i} at time $t=t_1$.
    \item $t^{(i)}_2$: At $t^{(i)}_2 = d_q$, the entire query segment $q$ is fully contained within the event $e_t$. This means the overlap is maximal: $|e_{t^{(i)}_2} \cap q| = d_q$. Since the presence criterion is met, the query segment accuracy is $F(e_{t^{(i)}_2}, q, \gamma) = \frac{|e_{t^{(i)}_2} \cap q|}{d_q} = \frac{d_q}{d_q} = 1$. This behavior is visually represented in Figure~\ref{fig:case_i} at time $t=t_2$, where the green box representing the event fully covers the red box representing the query segment.
    \item $t^{(i)}_3$: At $t^{(i)}_3 = d_e$, the entire query segment $q$ still fully overlaps with the event $e_t$. Similar to $t_2$, the overlap is $|e_{t^{(i)}_3} \cap q| = d_q$, and therefore $F(e_{t^{(i)}_3}, q, \gamma) = \frac{|e_{t^{(i)}_3} \cap q|}{d_q} = \frac{d_q}{d_q} = 1$. This is depicted in Figure~\ref{fig:case_i} at time $t=t_3$.
\end{itemize}

\textbf{Case (ii): $d_e < d_q$}

\begin{figure}[H]
    \centering
    \includegraphics[width=0.8\textwidth]{figures/case_ii.png}
    \caption{An illustration of how the sound event $e_t$ and the query segment $q$ overlap at the four distinct states $t = t_0, \dots, t_3$ for case (ii) where $d_e < d_q$.}
    \label{fig:case_ii}
\end{figure}

\begin{itemize}
    \item $t^{(ii)}_0$: At $t^{(ii)}_0 = 0$, the end of the event $e_t$ aligns perfectly with the beginning of the query segment $q$. There is no overlap ($|e_{t^{(ii)}_0} \cap q| = 0$). Assuming the annotator absence criterion applies, the query segment accuracy is $F(e_{t^{(ii)}_0}, q, \gamma) = \frac{d_q - |e_{t^{(ii)}_0} \cap q|}{d_q} = \frac{d_q - 0}{d_q} = 1$.
    \item $t^{(ii)}_1$: The time $t^{(ii)}_1 = \gamma d_e$ again marks the point where the annotator presence criterion is first met. Before this ($t < t^{(ii)}_1$), the overlap $|e_t \cap q| < \gamma d_e$, and $F(e_t, q, \gamma) = \frac{d_q - |e_t \cap q|}{d_q}$. Approaching $t^{(ii)}_1$ from the left, $|e_t \cap q| \to \gamma d_e$, thus $\lim_{t \to t_1^-} F(e_{t}, q, \gamma) = \frac{d_q - \gamma d_e}{d_q}$. At $t = t^{(ii)}_1$, the criterion is met, and the function becomes $F(e_t, q, \gamma) = \frac{|e_t \cap q|}{d_q}$. Approaching from the right, $|e_t \cap q|$ is slightly greater than $\gamma d_e$, so $\lim_{t \to t_1^+} F(e_{t}, q, \gamma) = \frac{\gamma d_e}{d_q}$. This transition is shown in Figure~\ref{fig:case_ii} at $t=t_1$.
    \item $t^{(ii)}_2$: At $t^{(ii)}_2 = d_e$, the beginning of the event $e_t$ aligns with the beginning of the query segment $q$. At this point, the overlap is maximal, as the entire event is contained within the query segment: $|e_{t^{(ii)}_2} \cap q| = d_e$. Since the presence criterion is met, the query segment accuracy is $F(e_{t^{(ii)}_2}, q, \gamma) = \frac{|e_{t^{(ii)}_2} \cap q|}{d_q} = \frac{d_e}{d_q}$. This situation is illustrated in Figure~\ref{fig:case_ii} at $t=t_2$.
    \item $t^{(ii)}_3$: At $t^{(ii)}_3 = d_q$, the end of the event $e_t$ aligns with the end of the query segment $q$. Similar to $t^{(ii)}_2$, the entire event is contained within the query segment, so the overlap is $|e_{t^{(ii)}_3} \cap q| = d_e$. Consequently, the query segment accuracy is $F(e_{t^{(ii)}_3}, q, \gamma) = \frac{|e_{t^{(ii)}_3} \cap q|}{d_q} = \frac{d_e}{d_q}$. This corresponds to the state depicted in Figure~\ref{fig:case_ii} at $t=t_3$.
\end{itemize}

Understanding these key time points and the corresponding query segment accuracy values is crucial for calculating the area under the curve, which represents the expected query segment accuracy.
%%%%%%%%%%%%%%%%%%%%%%%%%%%%%%%%%%%%%%%%%%%%%%%%%%%%%%%%%%%%%%%%%%%%%%%%%%%%%%%%%

\subsection{Proof of Theorem~\ref{thm:fix_optimal_query_length}}
\label{app:thm2}

\begin{proof}
We start by finding a unique critical point $d^*_q$ which makes $f'(d^*_q) = 0$ when $d_q \ge \gamma d_e$. We then show that $d_q^*$ is a global maximum by analyzing the boundaries of $f(d_q)$ on its' domain when $d_q \ge \gamma d_e$. We show that $f(d_q^*) \ge f(\gamma d_e)$ and that $f(d_q^*) \ge \lim_{d_q \rightarrow \infty} f(d_q)$. Since $d_q^*$ is a unique critical point we conclude that it must be a global maximum of the function $f(d_q)$ when $d_q \ge \gamma d_e$. Lastly, we show that $f(d_q^*) \ge f(\gamma d_e) \ge f(d_q)$ when $d_q < \gamma d_e$ which proves that $d_q^*$ is a global maximum of the function $f(d_q)$ for $d_q > 0$.

\textbf{1. Finding the unique critical point $d_q^*$.}

To find the critical points, we need to compute the derivative of $f(d_q)$ with respect to $d_q$ and set it to zero. Let $N(d_q) = d_e (-2d_e \gamma^2 + 2d_q \gamma + d_q)$ and $D(d_q) = d_q (d_e + d_q)$. Then $f(d_q) = \frac{N(d_q)}{D(d_q)}$. Using the quotient rule, the derivative is given by:
\begin{equation*}
f'(d_q) = \frac{N'(d_q)D(d_q) - N(d_q)D'(d_q)}{[D(d_q)]^2}
\end{equation*}
First, we find the derivatives of the numerator and the denominator:
\begin{align*}
N'(d_q) &= \frac{d}{dd_q} [d_e (-2d_e \gamma^2 + 2d_q \gamma + d_q)] \\
&= d_e (0 + 2\gamma + 1) \\
&= d_e (2\gamma + 1)
\end{align*}
\begin{align*}
D(d_q) &= d_q (d_e + d_q) = d_e d_q + d_q^2 \\
D'(d_q) &= \frac{d}{dd_q} [d_e d_q + d_q^2] \\
&= d_e + 2d_q
\end{align*}
Now, we plug these into the quotient rule formula:
\begin{align*}
f'(d_q) &= \frac{[d_e (2\gamma + 1)][d_q (d_e + d_q)] - [d_e (-2d_e \gamma^2 + 2d_q \gamma + d_q)][d_e + 2d_q]}{[d_q (d_e + d_q)]^2}
\end{align*}
To find the critical points, we set $f'(d_q) = 0$, which means the numerator must be zero:
\begin{equation*}
[d_e (2\gamma + 1)][d_q (d_e + d_q)] - [d_e (-2d_e \gamma^2 + 2d_q \gamma + d_q)][d_e + 2d_q] = 0
\end{equation*}
Since $d_e > 0$, we can divide by $d_e$:
\begin{equation*}
(2\gamma + 1) d_q (d_e + d_q) - (-2d_e \gamma^2 + 2d_q \gamma + d_q) (d_e + 2d_q) = 0
\end{equation*}
Expanding the terms:
\begin{align*}
(2\gamma + 1) (d_e d_q + d_q^2) &- (-2d_e^2 \gamma^2 - 4d_e d_q \gamma^2 + 2d_e d_q \gamma + 4d_q^2 \gamma + d_e d_q + 2d_q^2) = 0 \\
2\gamma d_e d_q + 2\gamma d_q^2 + d_e d_q + d_q^2 &- (-2d_e^2 \gamma^2 - 4d_e d_q \gamma^2 + 2d_e d_q \gamma + 4d_q^2 \gamma + d_e d_q + 2d_q^2) = 0
\end{align*}
Collecting and rearranging the terms to form a quadratic equation in $d_q$:
\begin{align*}
(2\gamma + 1 - 4\gamma - 2) d_q^2 + (2\gamma + 1 + 4\gamma^2 - 2\gamma - 1) d_e d_q + 2d_e^2 \gamma^2 &= 0 \\
(-2\gamma - 1) d_q^2 + (4\gamma^2) d_e d_q + 2d_e^2 \gamma^2 &= 0 \\
(2\gamma + 1) d_q^2 - 4\gamma^2 d_e d_q - 2d_e^2 \gamma^2 &= 0
\end{align*}
Using the quadratic formula $d_q = \frac{-b \pm \sqrt{b^2 - 4ac}}{2a}$, where $a = 2\gamma + 1$, $b = -4 d_e \gamma^2$, $c = -2 d_e^2 \gamma^2$:
\begin{align*}
d_q &= \frac{4 d_e \gamma^2 \pm \sqrt{(-4 d_e \gamma^2)^2 - 4 (2\gamma + 1) (-2 d_e^2 \gamma^2)}}{2 (2\gamma + 1)} \\
&= \frac{4 d_e \gamma^2 \pm \sqrt{16 d_e^2 \gamma^4 + 8 (2\gamma + 1) d_e^2 \gamma^2}}{4\gamma + 2} \\
&= \frac{4 d_e \gamma^2 \pm \sqrt{16 d_e^2 \gamma^4 + 16 d_e^2 \gamma^3 + 8 d_e^2 \gamma^2}}{4\gamma + 2} \\
&= \frac{4 d_e \gamma^2 \pm \sqrt{8 d_e^2 \gamma^2 (2\gamma^2 + 2\gamma + 1)}}{4\gamma + 2} \\
&= \frac{4 d_e \gamma^2 \pm 2 d_e |\gamma| \sqrt{4\gamma^2 + 4\gamma + 2}}{2(2\gamma + 1)}
\end{align*}
Since $\gamma > 0$, we have $|\gamma| = \gamma$:
\begin{align*}
d_q &= \frac{4 d_e \gamma^2 \pm 2 d_e \gamma \sqrt{4\gamma^2 + 4\gamma + 2}}{2(2\gamma + 1)} \\
&= \frac{2 d_e \gamma^2 \pm d_e \gamma \sqrt{4\gamma^2 + 4\gamma + 2}}{(2\gamma + 1)} \\
&= d_e \gamma \frac{2\gamma \pm \sqrt{4\gamma^2 + 4\gamma + 2}}{2\gamma + 1}
\end{align*}
We note that $\sqrt{4\gamma^2 + 4\gamma + 2} = 2\sqrt{\gamma^2 + \gamma + 0.5} > 2\sqrt{\gamma^2} = 2\gamma$, which means that we need to choose the positive sign for $d_q>0$ to be true.
The value of $d_q$ that makes the derivative zero is therefore uniquely defined by:
\begin{equation*}
d_q = d_e\,\gamma \frac{2\,\gamma + \sqrt{4\,\gamma^2 +4\,\gamma +2}}{2\,\gamma +1} \ge d_e \gamma,
\end{equation*}
where the last inequality holds because $\sqrt{4\gamma^2 + 4\gamma + 2} = 2\sqrt{\gamma^2 + \gamma + 0.5} \ge 1$.

\textbf{2. Analyze the function at the boundaries of its' domain.}

To understand why this critical point corresponds to a maximum, we analyze the function $f(d_q)$ as $d_q$ at the boundaries of its domain.

\textbf{2a. $f(d_q)$ when $d_q = \gamma d_e$ ($d_q \ge \gamma d_e$).}
\begin{align*}
    f(\gamma d_e) &= \frac{d_{e} \left( 2 (\gamma d_e) \gamma - 2 d_{e} \gamma^{2} + (\gamma d_e)\right)}{(\gamma d_e) \left(d_{e} + \gamma d_e\right)} \\
    &= \frac{d_{e} \left( 2 d_e \gamma^2 - 2 d_{e} \gamma^{2} + \gamma d_e\right)}{(\gamma d_e) \left(d_{e} + \gamma d_e\right)} \\
    &= \frac{d_{e} \left(\gamma d_e\right)}{(\gamma d_e) \left(d_{e} + \gamma d_e\right)} \\
    &= \frac{d_{e}^2 \gamma}{(\gamma d_e) d_{e}(1+\gamma)} \\
    &= \frac{1}{1+\gamma}.
\end{align*}

\textbf{2b. $f(d_q)$ as $d_q \rightarrow \infty$ ($d_q \ge \gamma d_e$).}

We want to evaluate the limit of $f(d_q)$ as $d_q$ approaches infinity:
\begin{align*}
\lim_{d_q \rightarrow \infty} f(d_q) &= \lim_{d_q \rightarrow \infty} \frac{d_e (-2d_e \gamma^2 + (2\gamma + 1)d_q)}{d_e d_q + d_q^2}
\end{align*}
Divide the numerator and the denominator by the highest power of $d_q$ in the denominator, which is $d_q^2$:
\begin{align*}
\lim_{d_q \rightarrow \infty} f(d_q) &= \lim_{d_q \rightarrow \infty} \frac{d_e \left(-\frac{2d_e \gamma^2}{d_q^2} + \frac{2\gamma + 1}{d_q}\right)}{\frac{d_e}{d_q} + 1}
\end{align*}
As $d_q \rightarrow \infty$, the terms $\frac{2d_e \gamma^2}{d_q^2}$, $\frac{2\gamma + 1}{d_q}$, and $\frac{d_e}{d_q}$ all approach 0. Thus,
\begin{equation*}
\lim_{d_q \rightarrow \infty} f(d_q) = \frac{d_e (0 + 0)}{0 + 1} = 0
\end{equation*}
This means that as $d_q$ becomes very large, the function $f(d_q)$ approaches 0.

\textbf{2c. Showing that $f(d_q^*) \ge f(\gamma d_e)$.}

We want to show that $f(d^*_q) \ge f(\gamma d_e)$. Or equivalently, that $f(d^*_q) - f(\gamma d_e) \ge 0$. From Theorem~\ref{thm:max_iou} we know that $f(d^*_q) = 2\gamma\left(2\gamma + 1 - \sqrt{4\gamma^2 + 4\gamma + 2}\right) + 1$, and from 2a we know that $f(\gamma d_e) = \frac{1}{1+\gamma}$. After substitution and some algebraic manipulation, we get
$$\gamma\left(\frac{4\gamma^2+6\gamma+3}{1+\gamma} - 2\sqrt{4\gamma^2 + 4\gamma + 2}\right) \ge 0.$$
Since $\gamma > 0$, it suffices to show that 
$$\frac{4\gamma^2+6\gamma+3}{1+\gamma} \geq 2\sqrt{4\gamma^2 + 4\gamma + 2}.$$
Squaring both sides of the above inequality and simplifying, we obtain the equivalent inequality
$$\left(\frac{4\gamma^2+6\gamma+3}{1+\gamma}\right)^2 \geq 4(4\gamma^2 + 4\gamma + 2).$$
After further algebraic manipulations (which we leave to the reader), we arrive at the inequality
$$(2\gamma + 1)^2 \geq 0.$$
Since $(2\gamma + 1)^2 \geq 0$ holds for all $\gamma$, and the previous steps are all equivalences, we conclude that 
$$ f(d_q^*) - f(\gamma d_e) \ge 0$$ for $0<\gamma\le 1$, and therefore,
$$f(d^*_q) \ge f(\gamma d_e).$$

\textbf{2d. Showing that $f(d_q^*) \ge \lim_{d_q\rightarrow\infty} f(d_q)$.}

We combine the results from 2a-2c to get
\begin{align*}
    f(d_q^*) &\ge f(\gamma d_e) \\
    &= \frac{1}{1+\gamma} \\
    &\ge 0  \\
    &= \lim_{d_q \rightarrow \infty} f(d_q).
\end{align*}

\textbf{2e. $f(d_q)$ as $d_q \rightarrow (\gamma d_e)^-$ ($d_q < \gamma d_e$).}

Since we are approaching $\gamma d_e$ from the left, we have that $f(d_q) = d_q/(d_e + d_q)$. This function is continuous for $d_q < \gamma d_e$, so the limit is given by the direct substitution:
\begin{align*}
\lim_{d_q \rightarrow (\gamma d_e)^-} \frac{d_q}{d_e + d_q} &= \frac{\gamma d_e}{d_e + \gamma d_e} \\
&= \frac{\gamma d_e}{d_e(1 + \gamma)} \\
&= \frac{\gamma}{1 + \gamma}
\end{align*}

\textbf{2f. Showing that $f(\gamma d_e) \ge f(d_q)$ when $d_q < \gamma d_e$}.
We start by noting that $f(\gamma d_e) = \frac{1}{1+\gamma} \ge \frac{\gamma}{1+\gamma} = \lim_{d_q\rightarrow(\gamma d_e)^-}$. Now it is sufficient to show that $f(d_q) = d_q/(d_q + d_e)$ is strictly decreasing for decreasing $d_q$, which we do by computing the derivative of $f(d_q)$ with respect to $d_q$ using the quotient rule:
\begin{align*}
    f'(d_q) &= \frac{(d_q+\gamma)(1) - d_q(1)}{(d_q+\gamma)^2} \\
    &= \frac{d_q+\gamma-d_q}{(d_q+\gamma)^2} \\
    &= \frac{\gamma}{(d_q+\gamma)^2}.
\end{align*}

Since $\gamma > 0$ and $(d_q+\gamma)^2 > 0$ for all $d_q > 0$, we have $f'(d_q) > 0$ for all $d_q > 0$. This implies that the function $f(d_q)$ is strictly increasing on the interval $(0, \infty)$. Therefore, if $0 < c \le b$, it must be the case that $f(c) \le f(b)$. Moreover, since $c < b$, $f(c) < f(b)$. Thus, for any $b>0$, $f(b) > f(c)$ for all $0< c \le b$. Now let $0 < d_q = c \leq \gamma d_e = b$.


\textbf{3. Combining everything (2a-2f)}

We have derived a unique critical point $d_q^* \ge \gamma d_e$ by setting the first derivative of $f(d_q)$ to zero. We have then shown that $f(d_q^*)$ is greater than or equal to $f(d_q)$ at the limits of its' domain when $d_q \ge \gamma d_e$. Finally, we show that $f(d_q^*) \ge f(\gamma d_e) \ge f(d_q)$ when $d_q < \gamma d_e$. Therefore, the value of $d_q$ that is the global maximum of $f(d_q)$ when $d_q>0$ is:
\begin{equation*}
\boxed{d_q^* = d_e\,\gamma \frac{2\,\gamma + \sqrt{4\,\gamma^2 +4\,\gamma +2}}{2\,\gamma +1}}
\end{equation*}
\end{proof}



% \subsection{Proof of Theorem~\ref{thm:max_iou}}
% \label{app:thm3}


% From Theorem~\ref{thm:fix_optimal_query_length} we know that $d_q^*$ maximize the expected label accuracy $f(d_q)$ for events of length $d_e$. We can thus define the maximum label accuracy as
% \begin{align}
% \label{eq:max_iou}
%     f^*(\gamma) &= f(d_q^*) \\
%     &= \frac{d_e(2\gamma d_q^*  - 2\gamma^2 d_e+ d_q^*)}{d_q^*(d_e + d_q^*)} \\
%     &= 2\gamma(2\gamma +1 - \sqrt{4\gamma^2+4\gamma + 2}) + 1
% \end{align}

% In the supplementary material, there is a symbolic mathematics solver that verifies that these expressions are equal. This can be derived by substituting $\delta = d_q / d_q$

\subsection{Proof of Theorem~\ref{thm:max_iou}}
\label{app:thm3}

\begin{proof}
From Theorem~\ref{thm:fix_optimal_query_length} we have that
\[
d_q^*
\;=\;
\frac{
  d_e\,\gamma\,\Bigl(2\,\gamma + \sqrt{4\,\gamma^2 +4\,\gamma +2}\Bigr)
}{
  2\,\gamma +1
}
\]
maximizes the function
\[
f(d_q)
\;=\;
\frac{
  d_e \bigl(-2\,d_e\,\gamma^2 \;+\; (2\,\gamma +1)\,d_q\bigr)
}{
  d_q\,\bigl(d_e + d_q\bigr)
}.
\]
We wish to show that the maximum label accuracy given overlap, 
\(\displaystyle f^*(\gamma) = f\bigl(d_q^*\bigr),\)
is 
\[
2\,\gamma\!\Bigl(
  2\,\gamma +1
  \;-\;
  \sqrt{\,4\,\gamma^2 +4\,\gamma +2}
\Bigr)
\;+\;
1.
\]

\medskip

\noindent
\textbf{1. Express $f(d_q)$ in terms of a dimensionless variable.}

Define
\[
\delta 
\;=\; 
\frac{d_q}{d_e}.
\]
Then
\[
d_q 
\;=\; 
\delta\,d_e,
\quad
d_e + d_q
\;=\;
d_e\,(1 + \delta),
\]
and
\[
f(d_q)
\;=\;
f(\delta\,d_e)
\;=\;
\frac{
  d_e \bigl(-2\,d_e\,\gamma^2 + (2\,\gamma +1)\,\delta\,d_e\bigr)
}{
  (\delta\,d_e)\,\bigl(d_e + \delta\,d_e\bigr)
}
=
\frac{
  -2\,\gamma^2 + (2\,\gamma +1)\,\delta
}{
  \delta \,\bigl(1 + \delta\bigr)
}.
\]
We can therefore write
\[
f(\delta)
\;=\;
\frac{
  -2\,\gamma^2 
  \;+\; 
  (2\,\gamma +1)\,\delta
}{
  \delta\,(1 + \delta)
}.
\]

\medskip

\noindent
\textbf{2. Identify the optimal dimensionless query length $\delta^*$.}

From Theorem~\ref{thm:fix_optimal_query_length}, we know that
\[
d_q^*
\;=\;
\frac{
  d_e\,\gamma\,
  \bigl(
    2\,\gamma \;+\; \sqrt{4\,\gamma^2 +4\,\gamma +2}
  \bigr)
}{
  2\,\gamma +1
}.
\]
Dividing both sides by \(d_e\) gives
\[
\delta^*
\;=\;
\frac{d_q^*}{d_e}
\;=\;
\gamma \,\frac{
  2\,\gamma \;+\; \sqrt{\,4\,\gamma^2 +4\,\gamma +2}
}{
  2\,\gamma +1
}.
\]
We need to show that
\[
f\bigl(\delta^*\bigr)
\;=\;
2\,\gamma\!\Bigl(2\,\gamma +1 - \sqrt{4\,\gamma^2 +4\,\gamma +2}\Bigr)
\;+\;
1.
\]

\medskip

\noindent
\textbf{3. Compute $f(\delta^*)$ explicitly.}

Let
\[
N(\delta)
\;=\;
-2\,\gamma^2 
\;+\; 
(2\,\gamma +1)\,\delta,
\quad
D(\delta)
\;=\;
\delta\,(1 + \delta).
\]
Then 
\(\;f(\delta) = \tfrac{N(\delta)}{D(\delta)}.\)

\begin{enumerate}
\item 
\textit{Numerator at \(\delta^*\).}

\[
N\bigl(\delta^*\bigr)
=
-2\,\gamma^2
\;+\;
(2\,\gamma +1)\,\delta^*
=
-2\,\gamma^2
\;+\;
(2\,\gamma +1)
\Bigl[
  \gamma
  \,\frac{
    2\,\gamma + \sqrt{\,4\,\gamma^2 +4\,\gamma +2}
  }{
    2\,\gamma +1
  }
\Bigr].
\]
Inside the brackets, \((2\,\gamma +1)\) cancels:
\[
N\bigl(\delta^*\bigr)
=
-2\,\gamma^2 
\;+\; 
\gamma\,\bigl(2\,\gamma + \sqrt{\,4\,\gamma^2 +4\,\gamma +2}\bigr)
=
-2\,\gamma^2 
\;+\;
2\,\gamma^2
\;+\;
\gamma\;\sqrt{\,4\,\gamma^2 +4\,\gamma +2}
=
\gamma\;\sqrt{\,4\,\gamma^2 +4\,\gamma +2}.
\]

\item 
\textit{Denominator at \(\delta^*\).}

\[
D(\delta)
\;=\;
\delta\,(1 + \delta).
\]
Hence,
\[
D\bigl(\delta^*\bigr)
=
\delta^*
\Bigl(1 + \delta^*\Bigr)
=
\Bigl[
  \gamma \,\frac{2\,\gamma + \sqrt{\,4\,\gamma^2 +4\,\gamma +2}}{2\,\gamma +1}
\Bigr]
\Bigl[
  1
  \;+\;
  \gamma \,\frac{2\,\gamma + \sqrt{\,4\,\gamma^2 +4\,\gamma +2}}{2\,\gamma +1}
\Bigr].
\]
The second bracket becomes a single fraction:
\[
1 
+ 
\gamma \,\frac{2\,\gamma + \sqrt{\,4\,\gamma^2 +4\,\gamma +2}}{2\,\gamma +1}
=
\frac{
  (2\,\gamma +1)
  \;+\;
  \gamma\;\bigl(2\,\gamma + \sqrt{\,4\,\gamma^2 +4\,\gamma +2}\bigr)
}{
  2\,\gamma +1
}.
\]
Combining, we get
\[
D\bigl(\delta^*\bigr)
=
\gamma \,\frac{2\,\gamma + \sqrt{\,4\,\gamma^2 +4\,\gamma +2}}{2\,\gamma +1}
\;\times\;
\frac{
  (2\,\gamma +1)
  + 
  2\,\gamma^2 
  + 
  \gamma\;\sqrt{\,4\,\gamma^2 +4\,\gamma +2}
}{
  2\,\gamma +1
}.
\]
So
\[
D\bigl(\delta^*\bigr)
=
\gamma\,
\frac{
  (2\,\gamma + \sqrt{\,4\,\gamma^2 +4\,\gamma +2})
  \,\bigl(
    2\,\gamma +1 + 2\,\gamma^2 
    + 
    \gamma\,\sqrt{\,4\,\gamma^2 +4\,\gamma +2}
  \bigr)
}{
  (2\,\gamma +1)^2
}.
\]

\item
\textit{Form the ratio.}  
Thus,
\[
f\bigl(\delta^*\bigr)
=
\frac{
  N(\delta^*)
}{
  D(\delta^*)
}
=
\frac{
  \gamma\;\sqrt{\,4\,\gamma^2 +4\,\gamma +2}
}{
  \gamma 
  \,\frac{
    (2\,\gamma + \sqrt{\,4\,\gamma^2 +4\,\gamma +2})
    \,\bigl(
      2\,\gamma +1 + 2\,\gamma^2 
      + 
      \gamma\,\sqrt{\,4\,\gamma^2 +4\,\gamma +2}
    \bigr)
  }{
    (2\,\gamma +1)^2
  }
}.
\]
Cancel the common factor \(\gamma\), invert the denominator and multiply:
\[
f\bigl(\delta^*\bigr)
=
\frac{
  \sqrt{\,4\,\gamma^2 +4\,\gamma +2}\,(2\,\gamma +1)^2
}{
  (2\,\gamma + \sqrt{\,4\,\gamma^2 +4\,\gamma +2})
  \,\bigl(
    2\,\gamma +1 + 2\,\gamma^2 
    + 
    \gamma\,\sqrt{\,4\,\gamma^2 +4\,\gamma +2}
  \bigr)
}.
\]
You can verify by direct expansion (or by a symbolic algebra tool which we provide in the supplementary material) that
\[
\frac{
  \sqrt{\,4\,\gamma^2 +4\,\gamma +2}\,(2\,\gamma +1)^2
}{
  (2\,\gamma + \sqrt{\,4\,\gamma^2 +4\,\gamma +2})
  \,\bigl(
    2\,\gamma +1 + 2\,\gamma^2 
    + 
    \gamma\,\sqrt{\,4\,\gamma^2 +4\,\gamma +2}
  \bigr)
}
=
2\,\gamma\,\Bigl(2\,\gamma +1 - \sqrt{\,4\,\gamma^2 +4\,\gamma +2}\Bigr) 
\;+\;
1.
\]
Thus
\[
f\bigl(\delta^*\bigr)
\;=\;
2\,\gamma\,\Bigl(2\,\gamma +1 - \sqrt{4\,\gamma^2 +4\,\gamma +2}\Bigr) + 1,
\]
which proves that
\[
f^*(\gamma)
=
f\bigl(d_q^*\bigr)
=
2\,\gamma\,\Bigl(2\,\gamma +1 - \sqrt{4\,\gamma^2 +4\,\gamma +2}\Bigr)
\;+\;
1.
\]
\end{enumerate}

Hence, Eq.~\ref{eq:max_iou} holds, completing the proof.
\end{proof}



\end{document}