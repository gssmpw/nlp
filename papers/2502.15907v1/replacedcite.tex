\section{Related Work}
\label{sec:related}

This section first reviews recent studies in semantic segmentation in flood monitoring. Next, core transfer learning and model reprogramming works are discussed. Finally, as we employ graph networks in our approach, the prominent works in graph network works are presented.

The highly accurate and efficient flood monitoring systems have increasingly been utilizing deep learning (DL) methods ____, such as convolutional neural network (CNN)-based architectures were very popular for semantic segmentation before being overtaken by U-Net-based architectures ____. The introduction of skip connections between the encoder and decoder further substantially improved the performance of U-Net for semantic segmentation ____. Rafi et al. ____ proposed an explainable deep CNN that utilizes multi-spectral optical and Synthetic Aperture Radar (SAR) images for flood inundation mapping.  Mahadi et al. proposed a U-Net-like hybrid model namely DeepLabv3+ that also applied atrous convolution for water region segmentation from surveillance footage ____ that outperformed U-Net. Hern{\'a}ndez at al. ____ utilized a U-Net-like model with unmanned aerial vehicles (UAVs) equipped with on-board edge computing to process flood-related data locally, consequently enabling faster response times and reducing dependency on distant computational resources. Recently transformer-based architectures have been used for segmentation as well ____. Another similar study by Roy et al. ____ introduced FloodTransformer, a transformer-based model specifically designed for segmenting flood scenes from aerial images, which demonstrated outstanding performance across several benchmarks.  

Transfer learning has been used in multiple applications to improve performance when limited data is available. Wu et al. ____ utilized transfer learning to adapt pre-trained deep learning models to the task of near-real-time flood detection using Synthetic Aperture Radar (SAR) images. Another notable work by Ghosh et al. ____ applied transfer learning to fine-tune CNN-based architecture which is optimized for analyzing image data, resulting in substantial enhancements in automatic flood detection capabilities. Recently, a new concept of model reprogramming emerged. Reprogramming enables re-purposing a pre-trained model without fine-tuning by introducing input transformation and output mapping____. Model reprogramming has been successfully applied in speech, computer vision, NLP, and time series domain ____.

Graph Neural Networks (GNNs) have shown substantial promise in various domains. For instance, in the field of medical imaging, ViG-UNet integrates vision graph neural networks to enhance medical image segmentation by showcasing potential pathways for similar adaptations in environmental scenes ____. Another study explored unsupervised image segmentation using GNNs, which maximizes mutual information to achieve segmentation without labelled data, a method that could dramatically reduce the need for annotated flood images ____. Furthermore, a novel approach employing GNNs for dynamic scene segmentation in videos presented methodologies that could be adapted for analyzing temporal changes in flood events by providing a foundation for real-time flood monitoring ____. 

GNNs and their variants such as Graph Attention Networks (GAT) ____ are very successful in various fields but their capabilities in semantic segmentation, particularly for environmental monitoring such as flood detection, are still unexplored. Building upon existing advancements in GNNs, there exists a compelling research gap in the integration of GNNs with established architectures like U-Net for semantic segmentation. While U-Net and its variants have set benchmarks in medical and natural scene segmentation, the unique capabilities of GNNs to capture complex spatial relationships remain underutilized in these models ____. Consequently, our study integrates GNN with U-Net to take advantage of both and demonstrates the abilities of such architecture for flood segmentation.