\section{Related Work}
Several studies have explored the delay performance of wireless networks with and without retransmissions.
From a queuing theoretic perspective, the trade-off between error probability and delay of multi-access systems over AWGN channel is analysed in____, and analytical models are developed to compute end-to-end delay in wireless networks modelled as a G/G/1 queue in____.
Much work has been done to characterise and derive bounds on the performance of wireless networks using network calculus or large-deviation theory____.
Some of these include analysing delay and error performance using effective bandwidth____, 
deriving delay bounds using effective capacity and service curve approaches____,
and deriving delay bounds and solutions for delay distributions using stochastic network calculus____ and (min,×) algebra____.

% From a queing theretic perspective, Telatar et.al.____ analysed the  trade-off between error probabiliy and delay of multi-access systems over AWGN channel by combining traditional information theory and queueing theory, and Bisnik et.al.____ developed analytical models to compute end-to-end delay in multihop wireless networks modelled as a G/G/1 queue.

% Hassan et al.____ applied effective bandwidth____ to analyze delay and error performance in a single-link wireless system, modeling the system as a finite state Markov chain (FSMC). 
% Wu et al.____ introduced the concept of effective capacity to derive an asymptotic approximation of delay bound violation probabilities for Rayleigh fading channels, where the authors focused on a regime of very low SNR for simplicity.
% Fidler____ extended this line of work by using a probabilistic service curve approach on a two-state FSMC model to derive delay bounds for wireless links.
% Other studies have proposed methods to evaluate delay and throughput distributions under various network conditions. Verticale et al.____ proposed a statistical envelope for wireless channels subject to Rayleigh fading and evaluated an approximate expression for the probability tail of the queueing delay. 
% Ciucu et al.____ provided closed-form solutions for throughput and delay distribution bounds using stochastic network calculus____, which gave better predictions of delay behavior in wireless systems. Al-Zubaidi et al.____ extended the field by introducing a novel (min,×) algebra, deriving delay bounds for multi-hop fading channels, and providing a new approach to analyzing delays in more complex scenarios.

The performance of ARQ and HARQ retransmission schemes has been widely studied in low-latency environments. 
An effective-capacity____ analysis of general HARQ systems is given in Larsson et.al.____. However, this analysis relies on an asymptotic information-theoretic approach requiring large packets____.
Akin et al.____ introduced a state transition model for HARQ systems and derived the effective capacity, modelling packet error rates using outage probability based on Shannon capacity____. However, outage and ergodic capacity are more suited only for long packets and are not appropriate otherwise____. 
Further, Schiessl et al.____ analysed the delay of finite blocklength wireless fading channels and showed that the Shannon capacity model significantly overestimates the delay performance in low-latency applications. 


To address this, The authors later studied the sensitivity of delay under the finite blocklength regime in ____ and derived an approximation for the decoding error probability under certain assumptions. 
Specifically on ARQ, Devassy et.al.____ used finite blocklength capacity over fading channels____ to study the performance of short packet communication.
In their work, they extended the concept of the slotted Gaussian collision channel with feedback____ and studied the throughput and delay as a function of the coded packet size and HARQ as a special case. The authors showed the existence of significantly different DVP for the same average delay, thus cementing the fact that studies on average delay are not sufficient for providing useful QoS guarantees. Similar studies by Sahin et al.____ focused on HARQ incremental redundancy (HARQ-IR)____ and analyzed its performance over Gilbert-Elliott channels with Rayleigh fading. They modelled HARQ as a Markov chain where the fading coefficients were discretized into states, with decoding errors modelled as outages on these discrete thresholds.


All the works are either restricted to a single-process retransmission scheme or model the multi-process ARQ/HARQ using a single-server queue. 
These limitations worsen the modelling inaccuracies for systems with larger RTTs, and fail to address practical implementation aspects of slot-based 5G systems, where inescapable decoding complexity and non-negligible feedback delays over multiple transmissions significantly contribute to the DVP.
While some works, such as____, include waiting delays in their analysis, they argue that cumulative transmission delays dominate the total delay. However, in slot-based 5G systems, even a single slot for decoding and feedback can constitute at least 50\% of the RTT, making this assumption less valid.
These studies are information-theoretic, lacking considerations for resource allocation and modulation and coding schemes (MCS), or are not sufficiently aligned with 3GPP specifications.
This limits their practical applicability, as real-world systems must account for the effects of resource allocation, coding schemes, and feedback delays on system performance.