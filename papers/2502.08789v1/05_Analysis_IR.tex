\section{HARQ: Incremental Redundancy}\label{sec:harq}
In this section, we consider the HARQ scenario with incremental redundancy. 
As discussed earlier, we assume that the coded packet length for all transmissions remains constant, thereby attaining the maximum increment in redundancy with reach retransmission. 
The PER is represented by the vector $\Vec{p}=\begin{bmatrix}p_1 & p_2 & \dots & p_M\end{bmatrix}$. We assume a maximum of $M$ transmissions and a maximum of $Q_{\mathrm{max}}$ parallel HARQ processes. Typically, $M=4$, and $Q_{\mathrm{max}}$ is 8 or 16 in real HARQ implementations.
\begin{table}[b]
\centering
\renewcommand{\arraystretch}{1.25}
\renewcommand{\arraystretch}{1.25}
\begin{tabular}{|l|l|l|l|l|}
\hline
\multicolumn{1}{|c|}{\multirow{2}{*}{State}} & \multicolumn{1}{c|}{\multirow{2}{*}{Next state}} & \multicolumn{1}{c|}{Range} & \multicolumn{1}{c|}{Range} & \multicolumn{1}{c|}{Proba-} \\
\multicolumn{1}{|c|}{}                       & \multicolumn{1}{c|}{}                            & \multicolumn{1}{c|}{$(q)$}   & \multicolumn{1}{c|}{$(m)$}   & \multicolumn{1}{c|}{bility} \\ \hline
$(0,1)$                                      & $(0,1)$                                          &   -                        &   -                        & $1-fp_1$                    \\
$(0,1)$                                      & $(1,2)$                                          &   -                        &   -                        & $fp_1$                      \\
$(q,m)$                                      & $(q,m+1)$                                        & $[1,\qmax)$                & $[1,M)$                    & $f'p_m$                     \\
$(q,m)$                                      & $(q+1,m+1)$                                      & $[1,\qmax)$                & $[1,M)$                    & $fp_m$                      \\
$(q,m)$                                      & $(q,1)$                                          & $[1,\qmax]$                & $[1,M)$                    & $fp_m'$                     \\
$(q,m)$                                      & $(q-1,1)$                                        & $[1,\qmax]$                & $[1,M)$                    & $f'p_m'$                    \\
$(q,M)$                                      & $(q,1)$                                          & $[1,\qmax]$                &    -                       & $f$                         \\
$(q,M)$                                      & $(q-1,1)$                                        & $[1,\qmax]$                &    -                       & $f'$                        \\
$(\qmax ,m)$                                 & $(\qmax ,m+1)$                                   &        -                   & $[1,M)$                    & $p_m$                       \\ \hline
\end{tabular}
\caption{Non-zero probabilities of the transition probability matrix $P$ for a given $q$ and $m$. State number $s=qM+m$ for the state $(q,m)$. $f'=1-f,\;p_m'=1-p_m.$}
\label{table:IR}
\end{table}

To compute the DVP, we proceed similarly to the previous section by combining the wait delay and service delay, which are computed separately. 
We propose an algorithmic approach to compute the wait delay, as this is more suited for HARQ with a relatively small $M$, $\qmax$, and a non-iid PER across the retransmissions.
% Adapting the ARQ queuing model to include maximum retransmissions and non-identical PERs is not straightforward, so we propose an algorithmic approach to address this.
As before in Section~\ref{sec:arq}, we bound the wait delay bound using the immediate feedback scenario where the retransmissions happen in the immediate next slot.
We now construct the Markov chain transition probability matrix of this scenario.

\subsection{Queueing Model}
We define $(\qmax+1)M$ states, denoted by the tuple $(q,m)$, where $0 \leq q \leq Q_{\mathrm{max}}$ and $1 \leq m \leq M$, measured at the slot boundary. The states represent the current queue length $q$ (observed by a newly arriving packet) and the transmission number $m$ of the packet that will be transmitted in the next slot. 
For example, the state $(3,2)$ indicates that the queue length of $3$ and the packet to be transmitted has already failed once. 
% In this section, we directly for the adapted model discussed Section~\ref{sec:arq}, where we look at the Markov chain whose steady state distribution dominates the actual queue distribution.


The non-zero transition probabilities for all states are given in Table~\ref{table:IR}. For ease in constructing and using the transition probability matrix $P_{(q+1)M\times (q+1)M}$, we number the states as $s=1,2,\dots,(qM+m),\dots,(q+1)M$. The states $(0,m), m \geq 2$ are never reached and are included for uniformity and simplicity. These states are defined with a self-loop probability of 1 and have a steady-state probability of 0.

% \begin{table*}[t]
% \centering
% \begin{tabular}{|l|l|l|l|l|}
% \hline
% State                  & Next State               & Range ($q$)                & Range ($m$)  & Probability \\ \hline
% $(0,1)$                & $(0,1)$                  &                            &              & $1-fp_1$    \\ \hline
% $(0,1)$                & $(1,2)$                  &                            &              & $fp_1$      \\ \hline
% $(q,m)$                & $(q,m+1)$                & $1\leq  q< \qmax $   & $1\leq m<M$  & $f'p_m$     \\ \hline
% $(q,m)$                & $(q+1,m+1)$              & $1\leq  q<\qmax $    & $1\leq m<M$  & $fp_m$      \\ \hline
% $(q,m)$                & $(q,1)$                  & $1\leq q\leq \qmax $ & $1\leq m< M$ & $fp_m'$     \\ \hline
% $(q,m)$                & $(q-1,1)$                & $1\leq q\leq \qmax $ & $1\leq m< M$ & $f'p_m'$    \\ \hline
% $(q,M)$                & $(q,1)$                  & $1\leq q\leq \qmax $ &              & $f$         \\ \hline
% $(q,M)$                & $(q-1,1)$                & $1\leq q\leq \qmax $ &              & $f'$        \\ \hline
% $(\qmax ,m)$ & $(\qmax ,m+1)$ &                            & $1\leq m< M$ & $p_m$       \\ \hline
% \end{tabular}
% \caption{Non-zero probabilities of the transition probability matrix $P$. State number $s=qM+m$ for the state $(q,m)$. $f'=1-f,\;p_m'=1-p_m.$}
% \label{table:IR}
% \end{table*}


The probabilities can be explained as follows: $f$ represents arrival, and $p_m$ represents the PER at the $m^{\mathrm{th}}$ attempt. For state $(0,1)$, an arrival and transmission failure lead to a transition to state $(1,2)$, while other possibilities result in a loop. In states with $m=M$, the PER becomes irrelevant because the packet is either successfully transmitted or discarded. Similarly, a packet is dropped when an arrival and transmission failure occurs at state $(Q_{\mathrm{max}},m)$ due to a queue overflow. For other states, the transitions follow the typical pattern: failures increase $m$, successes reset $m$ to 0, and arrivals/departures adjust the queue size based on the transmission outcome.

Once $P$ is constructed, the steady-state probabilities, denoted by $\Tilde{\pi}$, can be computed by finding the eigenvector of $P^T$ corresponding to the unit eigenvalue. This can be done using standard algorithms or by iterating $P$ until $P^i \approx P^{i+1}$, with the rows converging to the steady-state probabilities.

The steady-state probabilities $\Tilde{\pi}$ are for the modified Markov chain with $(Q_{\mathrm{max}}+1)M$ states. To obtain the steady-state probabilities ${\pi}$ for each queue length $q=1,2,\dots, Q_{\mathrm{max}}$, we sum the probabilities of all states with the same queue length but different $m$ values:
\begin{equation}
    \pi_q = \sum_{s=qM}^{(q+1)M}\Tilde{\pi}_s.
\end{equation}
We assume that $\qmax$ is chosen such that packet drops due to queue overflow are negligible, typical in a high-reliability setting. Otherwise, one could repeat with a larger $\qmax$. That being said, we do consider the drops emerging from packets reaching the retransmission limit of HARQ ($m=M$), which cannot be neglected.

\subsection{Wait Delay}
To compute the wait delay, we start by finding $f_W(k|q)$, the conditional wait probability given queue length $q$. We propose ALGORITHM~\ref{Alg:getWaitProbability} to compute this for a given $k,q,\Vec{p}$ and $M$ using combinatorics.
% recursively by summing $q$ i.i.d. random variables, each representing transmission attempts with probabilities $p_1, p_2, \dots, p_M$. 
% The wait delay is 0 if and only if the queue length is 0, so $f_W(0)~=~\pi_0$. 
The unconditional wait delay pmf is obtained by marginalizing the queue length probabilities:
\begin{equation}
    f_{\dwait}(k)=\mathbb{P}(\dwait=k) \leq \sum_{q=0}^{\infty}\pi_qf_W(k|q).\label{eq:secIR_waitDelay}
\end{equation}
\begin{algorithm}[t]
\caption{Recursive function \textit{getWaitProbability} to compute the conditional probability of wait delay of $k$ slots given a queuelength of $q$ packets. The global constant $M_0 = M$ in the first call of the recursion.}
\begin{algorithmic}[th]
% \State $M_0 \gets M$\Comment{$M_0$ is a global constant.}
\Function{getWaitProbability}{$k, q, \Vec{p}, M, M_0$}
    \If{$k == q$}
        \State\Return $(1 - \Vec{p}_1)^q$
        \Comment{$k=q\Rightarrow$ all success.}
    \ElsIf{$k < q$ \textbf{or} $k > M \cdot q$}
        \State\Return $0$
        \Comment{Out of range, $prob = 0$.}
    \EndIf
    \State $prob \gets 0$
    \State $N \gets \min(\text{floor}((k - q) / (M - 1)), q)$\
        \\\Comment{Max \#packets with max attempts $=M$.}
    \For{$n = 0$ \textbf{to} $N$}
        \State $numSeqs \gets \dbinom{q}{n}$
        \State $seqProbFail \gets \prod_{i=1}^{M-1}\Vec{p}_{i}$
        
        \If{$M == M_0$}
            \State $seqProbSucc \gets 1$
            \\\Comment{Handle discard case when $M = M_0$.}
        \Else
             \State $seqProbSucc \gets (1 - \Vec{p}_M)$
        \EndIf
        
        \State $seqProb\gets(seqProbFail \cdot seqProbSucc)^n$
        \State $subSeqProb \gets$ \Call{getWaitProbability}{}
        \State \qquad\qquad\qquad\qquad($k-Mn, q - n, \Vec{p}, M - 1, M_0$)
        \\\Comment{Recursion.}
        \State $prob \gets prob + numSeqs\!\cdot\!seqProb\!\cdot\!subSeqProb$
    \EndFor
    \State \Return $prob$
\EndFunction
\end{algorithmic}
\label{Alg:getWaitProbability}
\end{algorithm}






\subsection{Delay Violation Probability}
We now compute the distributions of the service delay, similar to Section~\ref{NoIR_dvp}. 
The service delay is determined by the PER vector and $k_d$, the maximum number of transmissions allowed before exceeding the delay target. 
Unlike \eqref{eq:secModel_Kd}, where we assumed infinite retransmissions, here we limit $k_d$ by $M$:
\begin{align}
k_{d}&=\min\left(M,\left\lfloor\frac{\nicefrac{d}{T}+\delta}{\delta+\ddecod+1}\right\rfloor\right),\label{eq:secIR_kd}\\
\mathbb{P}(\dserv>d)&=\prod_{i=1}^{k_{d}}p_i.\label{eq:secIR_servDelay}
\end{align}
% To compute the queue length or wait time probabilities in practice, we terminate the algorithm once the probability reaches zero (or a sufficiently small value), as these probabilities decrease monotonically. 
Let $k_{d-kT}$ denote the $k_d$ for the delay target $d-kT$. 
\begin{align*}
    k_{d-kT}&=\min\left(M,\left\lfloor\frac{\nicefrac{(d-kT)}{T}+\delta}{\delta+1}\right\rfloor\right),\nonumber\\
    &=\min\left(M,\left\lfloor\frac{\nicefrac{d}{T}-k+\delta}{\delta+\ddecod+1}\right\rfloor\right).
\end{align*}

Finally, the total DVP is computed as before in Section~\ref{NoIR_dvp}, using the wait delay and service delay violation probabilities:
\begin{align}
    \mathbb{P}(\dtot>d) &=\sum_k \mathbb{P}(\dwait=k)\mathbb{P}(\dserv>d-k)\nonumber\\
    % \sum_kf_{D_w}(k)\prod_{i=1}^{ \min\left(M,1 + \left\lfloor\frac{\nicefrac{d}{T}-1}{\delta+1}\right\rfloor\right)}p_i
    &\leq\sum_kf_{\dwait}(k)\prod_{i=1}^{k_{d-kT}}p_i.\label{eq:secIR_totDelay}
\end{align}
% \todo[inline]{Outro?}