\section{Conclusion}\label{sec:conclusion}
In this work, we aimed to characterise the QoS in a 5G system focusing on ARQ and HARQ-IR retransmission schemes by accurately evaluating the delay violation probability (DVP) for a given target delay. 
Unlike existing methods, we proposed a novel delay model that incorporated decoding and feedback delay into it. This also demanded the inclusion of a multi-server queueing model with multiple parallel ARQ/HARQ processes where the packets do not wait for feedback from previous transmissions, thereby saving valuable transmission opportunities. 
Using this delay model and a novel packet error rate (PER) model based on finite blocklength packet transmission theory, we computed closed-form expressions and algorithms to compute DVP for ARQ and HARQ schemes. 
Our assumptions closely followed 3GPP standards and can be adapted to various scenarios, thus enhancing the usability of this work.

Our numerical evaluations demonstrated that the proposed evaluation schemes significantly outperform state-of-the-art immediate feedback (IF) models in terms of accuracy, with the performance gap widening as decoding and feedback delays increase. 
We observed that While HARQ achieves better DVP outcomes than persistent ARQ under normal circumstances, persistent ARQ is better in some specific cases due to the practical constraints of allowing an arbitrary number of retransmission attempts for HARQ. 
We illustrated how parameter tuning affects DVP and emphasised the importance of balancing MCS and resource allocation to regulate QoS in 5G networks. 
We observed that sufficient resource allocation per byte of packet size can help achieve low DVP levels, even for larger packet sizes. 
Additionally, we saw that the throughput of the system initially increases with the arrival rate but eventually decreases due to the increase in delay violation. This revealed the existence of an optimum arrival rate that maximises the throughput.

Beyond DVP analysis, our numerical results can inform resource allocation algorithms, enabling them to guarantee QoS under specific system configurations. 
These findings underscore the importance of optimizing resource allocation and MCS selection to meet the stringent delay and reliability requirements of latency and reliability sensitive 5G applications, marking a step toward real-world implementation of 5G networks. 
