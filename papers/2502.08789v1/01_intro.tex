
\section{Introduction}\label{sec:intro}
The advent of 5G networks has marked a significant transformation in wireless communication, for instance, by supporting ultra-reliable and low-latency communication (URLLC), enhanced mobile broadband (eMBB), and massive machine-type communications (mMTC) services~\cite{3gpp2017study,popovski20185G,3gpp.22.104}. 
URLLC demands arguably the strictest quality of service (QoS) requirements in 5G in terms of delay and reliability and is poised to be the main enabler for real-time applications such as autonomous driving, virtual reality, and Industry 4.0\cite{sisinni2018industrial}. 
These applications are typically characterised by short packets transmitted with moderately low throughput\cite{durisi2016URLLC} and require delays in milliseconds with very low packet error rates (PER) of at most $10^{-3}$~\cite{popovski20226g} to $10^{-5}$~\cite{3gpp.38.913}, between various devices like machines, sensors, actuators and controllers.

To meet such strict QoS requirements in 5G, increasing the coding capabilities and reducing the PER beyond a limit is neither feasible with the timing constraints nor cheap in terms of resource costs. 
It is not effective either, as the decoding complexity has a significant negative effect on fulfilling the QoS requirements~\cite{celebi2021latency}. It has been argued that for given channel conditions, it is suboptimal to aim solely to minimise the PER~\cite{Peng2011ReliabilityPHY}, but it is better to aim for a moderate error rate with a good retransmission mechanism.

Automatic repeat request (ARQ)~\cite{ArqCommEngDeskRef} and hybrid automatic repeat request (HARQ)~\cite{Frenger2001HarqHSDPA,DAHLMAN2014299} retransmission schemes have already become ubiquitous in wireless communication. They enhance reliability and reduce latency by effective feedback, selective retransmissions, and incremental redundancy (IR) in the case of HARQ. HARQ, for instance, significantly outperforms no-feedback schemes for low-latency targets under the assumption of limited frequency diversity and no time diversity~\cite{Ostman2018Harq}, typical of the short packet URLLC.
Further, to reduce the latency, these retransmission schemes are implemented as a multi-process transmit queue, where the packets do not wait for feedback from the previous packets. These parallel transmissions of unacknowledged packets are called \textit{ARQ/HARQ processes}.
The need for these ARQ/HARQ processes arises from the decoding complexity and feedback scheduling. This is because, with a single ARQ/HARQ process, the packets have to wait for feedback, wasting all the valuable transmission opportunities during the round-trip time (RTT) of the packet.


In a 5G system, a stricter target latency typically comes at the cost of reduced reliability.
Achieving a sweet spot in the reliability-latency trade-off is thus essential, which is generally measured using the delay violation probability (DVP) of a target delay. 
Current state-of-the-art methods for computing DVP rely on single-server approximations of multi-process ARQ/HARQ schemes, which provide accurate estimates only when the inherent decoding and feedback delays are neglected. 
In this work, we address these critical gaps and explore the relation between DVP and 5G retransmission schemes by modelling ARQ and HARQ as a multi-server queue in the presence of decoding complexity and feedback delay. 



\subsection{Related Work}
Several studies have explored the delay performance of wireless networks with and without retransmissions.
From a queuing theoretic perspective, the trade-off between error probability and delay of multi-access systems over AWGN channel is analysed in~\cite{Telatar1995}, and analytical models are developed to compute end-to-end delay in wireless networks modelled as a G/G/1 queue in~\cite{bisnik2006queuing}.
Much work has been done to characterise and derive bounds on the performance of wireless networks using network calculus or large-deviation theory~\cite{devassy2019reliable,Zubaidi2016,yeh2012fundamental,petreska2019bound}.
Some of these include analysing delay and error performance using effective bandwidth~\cite{chang1995EB,hassan2004markov}, 
deriving delay bounds using effective capacity and service curve approaches~\cite{wu2003effectiveCap,fidler2006wlc15},
and deriving delay bounds and solutions for delay distributions using stochastic network calculus~\cite{jiang2008SNC,Ciucu2010} and (min,×) algebra~\cite{Zubaidi2016}.

% From a queing theretic perspective, Telatar et.al.~\cite{Telatar1995} analysed the  trade-off between error probabiliy and delay of multi-access systems over AWGN channel by combining traditional information theory and queueing theory, and Bisnik et.al.~\cite{bisnik2006queuing} developed analytical models to compute end-to-end delay in multihop wireless networks modelled as a G/G/1 queue.

% Hassan et al.~\cite{hassan2004markov} applied effective bandwidth~\cite{chang1995EB} to analyze delay and error performance in a single-link wireless system, modeling the system as a finite state Markov chain (FSMC). 
% Wu et al.~\cite{wu2003effectiveCap} introduced the concept of effective capacity to derive an asymptotic approximation of delay bound violation probabilities for Rayleigh fading channels, where the authors focused on a regime of very low SNR for simplicity.
% Fidler~\cite{fidler2006wlc15} extended this line of work by using a probabilistic service curve approach on a two-state FSMC model to derive delay bounds for wireless links.
% Other studies have proposed methods to evaluate delay and throughput distributions under various network conditions. Verticale et al.~\cite{verticale2009closed} proposed a statistical envelope for wireless channels subject to Rayleigh fading and evaluated an approximate expression for the probability tail of the queueing delay. 
% Ciucu et al.~\cite{Ciucu2010} provided closed-form solutions for throughput and delay distribution bounds using stochastic network calculus~\cite{jiang2008SNC}, which gave better predictions of delay behavior in wireless systems. Al-Zubaidi et al.~\cite{Zubaidi2016} extended the field by introducing a novel (min,×) algebra, deriving delay bounds for multi-hop fading channels, and providing a new approach to analyzing delays in more complex scenarios.

The performance of ARQ and HARQ retransmission schemes has been widely studied in low-latency environments. 
An effective-capacity\cite{wu2003effectiveCap} analysis of general HARQ systems is given in Larsson et.al.~\cite{larsson2016HARQ}. However, this analysis relies on an asymptotic information-theoretic approach requiring large packets\cite{devassy2019reliable}.
Akin et al.~\cite{akin2015backlog} introduced a state transition model for HARQ systems and derived the effective capacity, modelling packet error rates using outage probability based on Shannon capacity~\cite{Shannon}. However, outage and ergodic capacity are more suited only for long packets and are not appropriate otherwise~\cite{durisi2016URLLC}. 
Further, Schiessl et al.~\cite{schiessl2015delay} analysed the delay of finite blocklength wireless fading channels and showed that the Shannon capacity model significantly overestimates the delay performance in low-latency applications. 


To address this, The authors later studied the sensitivity of delay under the finite blocklength regime in \cite{schiessl2018delay} and derived an approximation for the decoding error probability under certain assumptions. 
Specifically on ARQ, Devassy et.al.~\cite{devassy2014finite,devassy2018delay,devassy2019reliable} used finite blocklength capacity over fading channels~\cite{polyanskiy2010FBL,polyanskiy2011feedback,Wei2013FBLblockfading} to study the performance of short packet communication.
In their work, they extended the concept of the slotted Gaussian collision channel with feedback~\cite{caire2000modulation,caire2001throughput} and studied the throughput and delay as a function of the coded packet size and HARQ as a special case. The authors showed the existence of significantly different DVP for the same average delay, thus cementing the fact that studies on average delay are not sufficient for providing useful QoS guarantees. Similar studies by Sahin et al.~\cite{Sahin2014Harq,Sahin2015Harq,Sahin2019Harq} focused on HARQ incremental redundancy (HARQ-IR)~\cite{DAHLMAN2014299} and analyzed its performance over Gilbert-Elliott channels with Rayleigh fading. They modelled HARQ as a Markov chain where the fading coefficients were discretized into states, with decoding errors modelled as outages on these discrete thresholds.


All the works are either restricted to a single-process retransmission scheme or model the multi-process ARQ/HARQ using a single-server queue. 
These limitations worsen the modelling inaccuracies for systems with larger RTTs, and fail to address practical implementation aspects of slot-based 5G systems, where inescapable decoding complexity and non-negligible feedback delays over multiple transmissions significantly contribute to the DVP.
While some works, such as~\cite{Sahin2019Harq}, include waiting delays in their analysis, they argue that cumulative transmission delays dominate the total delay. However, in slot-based 5G systems, even a single slot for decoding and feedback can constitute at least 50\% of the RTT, making this assumption less valid.
These studies are information-theoretic, lacking considerations for resource allocation and modulation and coding schemes (MCS), or are not sufficiently aligned with 3GPP specifications.
This limits their practical applicability, as real-world systems must account for the effects of resource allocation, coding schemes, and feedback delays on system performance.

\subsection{Contributions}
Our contributions are summarised as follows:
\begin{enumerate} 
\item We propose a framework consisting of a delay model and an error model to accurately compute the DVP for ARQ and HARQ retransmission schemes in 5G. This framework has the potential to aid resource allocation and link adaptation algorithms targeting specific DVPs at given delay thresholds.
\item The delay model employs multi-server transmit queues, enabling support for multiple ARQ/HARQ processes while accounting for decoding and feedback delays. The model is grounded in 3GPP standards and incorporates realistic configurations, providing a key advancement over existing works.
\item The error model, while simple, uses realistic finite blocklength theory to evaluate the PER of ARQ and HARQ-IR with sufficient accuracy. The error model is isolated from the delay model, enabling the results to be used with measured PER values, further increasing the model's flexibility for practical use.
\item Our numerical evaluations demonstrate accuracy improvements over single-server models that rely on immediate feedback (IF) assumptions.
We analyse the effect of parameter tuning on DVP across various delay regimes and target delays and highlight the impact of resource allocation and packet sizes, especially under tight delay constraints. 
Additionally, we reveal the existence of an optimal arrival rate that maximises the system throughput. 
\end{enumerate}


The remainder of this work is organised as follows: In Section~\ref{sec:model}, we introduce the system model and the error model. 
In Sections \ref{sec:arq} and \ref{sec:harq}, we propose closed-form expressions and algorithms to compute the DVP for ARQ and HARQ retransmission schemes. Within each of these sections, we (1) discuss the queuing model and compute the steady-state queue probabilities, (2) compute the wait delay distribution, and (3) compute the service delay distribution and use it to calculate the DVP. 
% Section~\ref{sec:detArr} explains the adaptations necessary for deterministic arrivals.\todo[]{Remove/ Appendix?} 
Finally, in Section~\ref{sec:simulations}, we show the numerical evaluation in detail and conclude in Section~\ref{sec:conclusion}.
