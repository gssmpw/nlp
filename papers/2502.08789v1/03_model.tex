\section{System Model and Problem Statement}\label{sec:model}
\begin{figure*}[th]
\centering
\begin{minipage}[t]{.5\textwidth}
  \centering
  \includegraphics[width=\textwidth]{Figures/model.eps}
  \caption{Illustration of the closed-loop communication system studied showing the retransmission process. Different delay components are shown where the packets experience them.}
  \label{fig:BasicModel}
\end{minipage}%
\hspace{0.03\textwidth}%
\begin{minipage}[t]{0.46\textwidth}
  \centering
  \includegraphics[width=\textwidth]{Figures/timingDiagram1.eps}
  \caption{Timing diagram showing the order and positions of different slot-based arrival and departure events with respect to the corresponding slot boundary.}
  \label{fig:timingdiagram}
\end{minipage}
\end{figure*}


Consider a 5G communication system as depicted in Fig.~\ref{fig:BasicModel}. 
A User Equipment (UE) generates or receives uplink (UL) packets and queues them for transmission\footnote{The analysis and results apply to uplink and downlink scenarios; we focus on the uplink for clarity and consistency.}. 
These packets are sent to a dedicated gNB via a 5G-NR wireless link, utilizing a fixed number of resources scheduled to the UE in each time slot.
The packet is encoded over these pre-allocated frequency domain resource blocks (RB) using the configured modulation and coding scheme (MCS).

If the queue is non-empty, the UE uses the entire slot to transmit the head-of-the-queue packet. 
The gNB attempts to decode the packet using the implemented coding scheme. Successful packets are used for their intended purpose, and an acknowledgement is fed back in the downlink (DL)
The UE receives this feedback, and retransmission is triggered if necessary. 
For this, a static retransmission scheme is configured, and the packets are discarded after a maximum number of transmission attempts.

The process involves various delays, as shown in Fig.~\ref{fig:BasicModel}. 
Encoding delay is ignored, as it occurs only once per packet and can be performed while the packet waits in the queue. Similarly, propagation delays are neglected due to the short distances typical in high-reliability applications. 
However, if needed, one can include the encoding delay by subtracting it from the delay target, as it is endured only once per packet, and the propagation delay by adding it to the decoding delay, as they are always encountered as a pair.



% \begin{figure*}[th]
% \centering
% \begin{minipage}{.48\textwidth}
%   \centering
%   \includegraphics[width=\textwidth]{Figures/model.eps}
%   \caption{Illustration showing the communication process. The different delay components are shown.}
%   \label{fig:BasicModel}
% \end{minipage}%
% \hspace{0.01\textwidth}%
% \begin{minipage}{0.33\textwidth}
%   \centering
%   \includegraphics[width=\textwidth]{Figures/timingDiagram.png}
%   \caption{Timing diagram showing the positions and order of different arrival and departure events.}
%   \label{fig:timingdiagram}
% \end{minipage}%
% \hspace{0.01\textwidth}%
% \begin{minipage}{0.16\textwidth}
%   \centering
%   \includegraphics[width=\textwidth]{Figures/harq.png}
%   \caption{Illustration of the HARQ-IR process with 4 redundancy versions.}
%   \label{fig:harqprocess}
% \end{minipage}
% \end{figure*}


\subsection{System Model}
Arrivals (or generation) of packets of size $n$ bits are modelled to occur randomly with an arrival probability of $f$.
While earlier arriving packets are given initial transmission opportunities, they do not wait for the feedback of previously transmitted packets, forming multiple simultaneous transmit-retransmit processes.
These packets form a queue awaiting transmission opportunities, which is modelled as a multi-server queue, with each server representing a packet undergoing a transmission process.
The maximum queue size is $\qmax$, and the slot length is $T$. 


New or retransmitted packets are added to the queue at the UE immediately after the slot boundary. 
Each packet transmission uses all time domain resources within the slot, with departures or feedback generation occurring at the gNB just before the subsequent slot boundary, as illustrated in Fig.~\ref{fig:timingdiagram}. Although the slot timings at the UE and gNB may not align perfectly in terms of absolute clock time due to propagation delay, their offset remains constant. Synchronization of slot indices is maintained through the application of a timing advance (TA) computed during the initial handshake, ensuring that a transmission in slot $k$ of the UE is received within slot $k$ at the gNB.
In this work, retransmissions always have priority, and retransmission schedules are added to the head of the queue. 

We assume that packet failures in the UL manifest as decoding failures at the gNB, and negative acknowledgements (NACKs) are always successfully transmitted in the DL for these failed packets. The maximum number of retransmissions allowed is denoted by $M$ corresponding. Depending on whether $M$ is finite or infinite, we refer to this as a \textit{truncated} or \textit{persistent} retransmission scheme, respectively. 
The packet experiences a decoding delay of $\ddecod$ regardless of success, and the feedback incurs a delay of $\delta$ before being received by the UE, both measured in slots.
Transmitted packets are stored separately outside of the queue for potential retransmissions, preventing queue overflow even when $\qmax < \infty$. Therefore, a failed packet sent in slot $k$ is up for retransmission in slot $k + \ddecod + \delta + 1$.
% \todo[inline]{cite decoding and feedback constraints and standards at gNB}

The packet error rate (PER) is modelled in two ways. First, we consider the ARQ scheme, where failures are independent and identically distributed (i.i.d.) across different packets and transmission attempts. Second, we consider the HARQ scheme, where failures are identical only between packets but not between transmission attempts. ARQ discards information from failed transmissions and retries decoding independently, whereas HARQ retains this information to improve decoding of subsequent attempts. HARQ can be implemented in various ways, for example, Chase Combining (CC) and Incremental Redundancy (IR)\cite{DAHLMAN2014299}. In this work, we focus on HARQ-IR, the method that is predominantly used today. While the PER for ARQ is denoted by $p$, the PER for different transmission attempts in HARQ is represented by the vector $\bm{p} = [p_1, p_2, \dots, p_M]$, where $p_{m+1} \leq p_m,\,\forall m= 1, 2, \dots, M$. Thus, ARQ can be considered a special case of HARQ, where $p_m = p,\,\forall m$.

The slot-based packet transmission model above suffices for computing DVP given a known PER. However, to calculate PER and fully characterize DVP, we model transmissions based on a simplified 3GPP specification.
OFDM Resources are allocated in quanta of resource blocks (RBs), the number of which is denoted by $\nrb$. 
Each RB contains 12 sub-carriers separated in frequency with a sub-carrier spacing (SCS) of $15 \times 2^\nu$ kHz, indexed by $\nu$, referred to as numerology~\cite{3gpp.38.211}.
One OFDM sub-carrier defines a so-called resource element (RE), the number of which is denoted by $\nre$, resulting in $\nre = 12\nrb$.
A slot of duration $T = 2^{-\nu}$ ms contains 15 time-domain symbols, yielding $12\times15 = 180$ symbols per RB per slot and a blocklength of $180\times\nrb$ for each transmission.
In practice, only 12 or 14 symbols are present in a slot instead of 15 due to the cyclic prefix that we ignore in this work for simplicity. We also fix $\nu = 0$, setting SCS to 15 kHz. Nonetheless, these assumptions can be easily removed using the parameterised slot duration $T$ and symbols per slot. In addition, we assume a transport block size (TBS) not exceeding 8448 bits, avoiding code block segmentation~\cite{3gpp.38.212}.

Packets are transmitted over a Rayleigh block fading channel, with the assumption that the fading is constant within a time slot and changes independently between slots. Let $\mathcal{N}$ be the complex AWGN noise, and $h_k$ the Rayleigh-distributed fading coefficient at slot $k$ with $\mathbb{E}[\vert h_k\vert^2] = \hvar$. The channel input/output relation is:
$$y_k = h_k x_k + \mathcal{N}.$$
The transmission is carried out with a fixed MCS from the 3GPP 38.214-Table 5.1.3.1-1~\cite{3gpp.38.214}, which directly gives the spectral efficiency $\eta$ defined as the rate per symbol. Let $V$ denote the channel dispersion coefficient, $\snrInst$ the instantaneous SNR at the receiver, and $Q(x)$ the Q-function. The PER of an ARQ scheme with a given instantaneous SNR is given by\cite{polyanskiy2010FBL,Wei2013FBLblockfading}:
\begin{equation}
    p(\snrInst)=Q\left(\left(\log_2\left(1+\snrInst\right)-\eta\right)\sqrt{\frac{\nre}{V}}\right).\label{eq:secModel_perARQ_instSNR}
\end{equation}
% \todo[inline]{Explain about this more?}
% \todo[inline]{there is slightly improved version of this in \cite{devassy2014finite} where they use $(\frac{\log n}{2n})$ for the additional term $+O(\frac{\log n}{n})$ in the FBL approximation. Consider using this?}

HARQ-IR initially encodes with a low-rate code and generates a finite number $M$ of redundancy versions (RVs) through puncturing~\cite{HarqIRCommEngDeskRef}. 
Various methods exist for generating these RVs~\cite{HarqIRCommEngDeskRef,szczecinski2013rate,heidarzadeh2018systematic}, which differ in aspects such as RV overlap, whether the packet length changes with each RV and the number of higher layer packets combined in each physical layer packet. 
For simplicity, in this work, we assume that RVs are of equal length and non-overlapping, as illustrated in Fig.~\ref{fig:harqprocess2}, with 4 RVs corresponding to $M=4$.
Each RV has a coding rate of $Mr_c$ where $r_c$ is the coding rate of the unpunctured version. These are transmitted consecutively, thus completing the unpunctured code by the final attempt. 
Since HARQ uses previous transmissions for decoding, we get an effective spectral efficiency of $\nicefrac{\eta}{m}$ and a blocklength of $m\nre$ after $m^{\text{th}}$ transmission.
Using this, the PER for the $m^{\text{th}}$ transmission could be written as:
\begin{equation}
    p_m(\snrInst)=Q\left(\left(\log_2\left(1+\snrInst\right)-\frac{\eta}{m}\right)\sqrt{\frac{m\nre}{V}}\right).\label{eq:secModel_perHARQ_instSNR}
\end{equation}
This approach models a HARQ sufficiently well in terms of (only) the parameters of our interest while being much simpler than some existing approaches.
\begin{figure}[t]
\centering
\includegraphics[width=0.95\linewidth]{Figures/harq2.eps}
\caption{Illustration of the HARQ-IR process. The $r_c$-rate channel coded bits are punctured to obtain 4 equal-length and non-overlapping RVs with a coding rate of $4r_c$ each. Effective spectral efficiency and block length after each decoding attempt are shown.}
\label{fig:harqprocess2}
\end{figure}
% \todo[inline]{This modeling is novel. Move it outside system model?}
% \todo[inline]{Name this model?}

%%%%%%%%%% EXPLANATION, IF NEEDED, FOR THE INTEGRAL THAT FOLLOWS. BASIC VARIABLE SUBSTITUTION
% Denote the average SNR with $\snr$. The instantaneous SNR random variable $S$ is thus distributed exponentially with mean $\snr$. Thus
% \begin{align}
%  p_m=\mathbb{E}_{S}\left[p_m(S)\right]&=\int_{0}^{\infty}\frac{1}{\snr }\,Q\left(\left(\log_2\left(1+s\right)-\frac{\eta}{m}\right)\sqrt{\frac{m\nre}{V}}\right)\,e^{\nicefrac{-s}{\snr }}\,\mathrm{d}s
% \end{align}
% Let the received SNR be defined as $S=P_{T}\alpha\vert h\vert^2$, 
% where $P_{T}$ is the transmit power, $\alpha$ is the path-loss component and $h$ is the fading coefficient. Define $\kappa=P_T\alpha,\,x=\vert h\vert^2$. Thus $s=\kappa x$ and
% \begin{align}
%  p_m&=\int_{0}^{\infty}\frac{\kappa}{\snr }\,Q\left(\left(\log_2\left(1+\kappa x\right)-\frac{\eta}{m}\right)\sqrt{\frac{m\nre}{V}}\right)\,e^{\nicefrac{-\kappa x}{\snr }}\,\mathrm{d}x.
%  \intertext{We also have,}
%  \mathbb{E}[S]&=\kappa\mathbb{E}\left[\vert h\vert^2\right]\coloneqq\gamma\\
%  \Rightarrow \kappa &=\frac{\gamma}{\mathbb{E}\left[\vert h\vert^2\right]} =\frac{\gamma}{\hvar}.\\
%  \intertext{Thus,}
%   p_m&=\int_{0}^{\infty}\frac{1}{\hvar }\,Q\left(\left(\log_2\left(1+\frac{\gamma}{\hvar} x\right)-\frac{\eta}{m}\right)\sqrt{\frac{m\nre}{V}}\right)\,e^{\nicefrac{-x}{\hvar }}\,\mathrm{d}x.
% \end{align}

Recall that with a Rayleigh fading channel,  $\snrInst$ is exponentially distributed. Rewrite $S = \frac{\snr}{\hvar }\vert h\vert^2$, where $\snr \coloneqq \mathbb{E}[S]$, the average SNR at the receiver. To derive the PER for the average SNR $\snr$, one can compute the expectation using numerical integration or various Monte Carlo methods. In Section~\ref{sec:simulations}, we apply a simple Monte Carlo approach. We have,
\begin{multline}
    p=\int_{0}^{\infty}\frac{1}{\hvar }\,Q\Bigg(\bigg(\log_2\bigg(1+\frac{\snr}{\hvar }s\bigg)\\
    -\eta\bigg)\,\sqrt{\frac{\nre}{V}}\Bigg)\,e^{{-s}/{\hvar}}\,\mathrm{d}s,\qquad\text{(for ARQ)}\label{eq:secModel_perARQ}
\end{multline}
\begin{multline}
    \quad p_m=\int_{0}^{\infty}\frac{1}{\hvar }\,
    Q\Bigg(\bigg(\log_2\bigg(1+\frac{\snr}{\hvar }s\bigg)\\
    -\frac{\eta}{m}\bigg)\sqrt{\frac{m\nre}{V}}\Bigg)\,e^{{-s}/{\hvar }}\,\mathrm{d}s.\qquad\text{(for HARQ)}\label{eq:secModel_perHARQ}
\end{multline}
% \begin{align}
%     p&=\int_{0}^{\infty}\frac{1}{\hvar }\,Q\left(\left(\log_2\left(1+\frac{\snr}{\hvar }s\right)-\eta\right)\sqrt{\frac{\nre}{V}}\right)\,e^{\nicefrac{-s}{\hvar  }}\,\mathrm{d}s,&&\text{(for ARQ)}\label{eq:secModel_perARQ}\\
%     p_m&=\int_{0}^{\infty}\frac{1}{\hvar }\,
%     Q\left(\left(\log_2\left(1+\frac{\snr}{\hvar }s\right)-\frac{\eta}{m}\right)\sqrt{\frac{m\nre}{V}}\right)\,e^{\nicefrac{-s}{\hvar }}\,\mathrm{d}s.&&\text{(for HARQ)}\label{eq:secModel_perHARQ}
% \end{align}

\begin{table*}[t]
\centering
\renewcommand{\arraystretch}{1.25}
\caption{Table of abbreviations.}
\label{Tab:abbr}
\begin{tabular}{|ll|ll|ll|}
\hline
DVP & Delay Violation Probability    & IR  & Incremental Redundancy   & PER  & Packet Error Rate \\
MCS & Modulation and Coding Scheme & ARQ & Automatic Repeat Request & HARQ & Hybrid ARQ        \\
SCS & sub-carrier Spacing   & RE  & Resource Element         & RB   & Resource Block    \\ \hline
\end{tabular}
\end{table*}
\begin{table*}[t]
\centering
\renewcommand{\arraystretch}{1.25}
\caption{Table of notations.}
\label{Tab:notations}
\begin{tabular}{|ll|ll|ll|}
\hline
$n$       & Packet length (bits)     & $T$      & Slot duration (s)             & $f$                    & Frequency of random arrivals, with $f<1$   \\
$\qmax$   & Maximum queue size       & $M$      & Maximum transmission attempts & $c$                    & Cycle of deterministic arrivals (in slots) \\
$\nrb$    & Number of RBs            & $\nre$   & Number of REs                 & $k$                    & Slot index                                 \\
$\dwait$  & Waiting delay (in slots) & $\dserv$ & Serving delay (in slots)      & $\dtot$                & Total delay (in slots)                     \\
$d$       & Delay target (s)         & $\delta$ & Feedback delay (in slots)     & $\ddecod$              & Decoding delay (in slots)                  \\
$m$       & Transmission index       & $p$      & PER of the ARQ scheme         & $p_m$                  & PER of $m^{\text{th}}$ attempt of HARQ-IR  \\
$\dvp(d)$ & DVP for target $d$       & $\snr$   & Average SNR                   & $\hvar$                & Mean of $\vert h\vert^2$                   \\
$V$       & Channel dispersion       & $\eta$   & Spectral efficiency           & \multirow{2}{*}{$k_d$} & Maximum transmission attempts possible     \\
          &                          &          &                               &                        & without violating the target delay         \\ \hline
\end{tabular}
\end{table*}

\subsection{Problem Statement}
We consider three delay components: wait delay ($\dwait$), service delay ($\dserv$), and total delay ($\dtot$), all random variables measured in slots. Wait delay is the time between a packet’s arrival and its first transmission opportunity. Service delay is the time from the first transmission to the final transmission, and total delay is their sum:
\begin{equation*}
    \dtot = \dwait+\dserv.
\end{equation*}
As the feedback of the successful (or discarded) transmission is irrelevant, for $m$ transmission attempts:
\begin{equation}
    \dserv = m+m\ddecod+(m-1)\delta.\label{eq:SecModel_Dserv}
\end{equation}
The delay violation probability (DVP) associated with a delay target $d$ is defined as:
\begin{equation}
\dvp(d) = \mathbb{P}\left(D>d\right).\label{eq:SecModel_dvpDefinition}
\end{equation}
Our goal is to characterize the DVP as a function of various system parameters for ARQ and HARQ-IR. 
% While we primarily focus on random packet arrivals, we also briefly address deterministic arrivals.\todo{appendix}

\subsubsection{Warm-up: Bounded Arrival Retransmission Model}
For illustrative purposes, we consider a simplified ARQ scheme with no waiting time ($\dwait = 0$), resulting in $\dtot = \dserv$. 
This scheme, referred to as the Bounded Arrival Retransmission (BAR) scheme, assumes arrivals are either deterministic with a cycle $c\geq M \cdot \text{RTT}$ or are triggered only after the successful transmission of the previous packet, ensuring that a queue never forms. 
Here, the round-trip time (in slots) is given by $\text{RTT}=1+\ddecod+\delta$. 
Let $k_d$ denote the maximum number of transmission attempts allowable without violating the delay target $d$. Thus, the DVP corresponds to $k_d$ failed transmissions, that is, a probability of $p^{k_d}$. 
We have,
\begin{align}
    k_d &= \left\lfloor \frac{\nicefrac{d}{T} + \delta}{\delta + \ddecod + 1} \right\rfloor, \label{eq:secModel_Kd} \\
    \dvp(d) &= p^{k_d}. \label{eq:secModel_BARdvp}
\end{align}
It is worthwhile to observe that $k_d = \left\lfloor \nicefrac{d}{T} \right\rfloor$ when $\delta = \ddecod = 0$, where each attempt takes exactly 1 slot.\\

Note that with minimal effort, one can modify the general ARQ/HARQ results for a deterministic arrival process with a cycle time of every $c = f^{-1}$ time slots. We omit this part due to space constraints.
The results can be extended with minimal adjustments to a deterministic arrival process with a cycle time of $c = f^{-1}$ time slots. This extension is omitted due to space constraints.

We summarize important abbreviations in TABLE~\ref{Tab:abbr} and notations in TABLE~\ref{Tab:notations}.

