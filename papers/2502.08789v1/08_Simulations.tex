

\section{Numerical Evaluation}\label{sec:simulations}
We begin this section on numerical evaluation by detailing the parameter configuration, including default settings, MCS selection processes, and PER computation methods. 
We then compare the proposed ARQ and HARQ DVP evaluation schemes with the state-of-the-art IF approximation, showing the importance of not ignoring the decoding and feedback delay. 
Following this, we study key DVP trends across varying system parameters by examining the impact of RTT, resource allocation, and arrival rate on the evaluated DVP. Throughout this section, we consider a persistent ARQ with unlimited retransmissions and queue size, i.e., $M = \qmax = \infty$ and a typical HARQ configuration of $M = 4$ and $\qmax = 16$. 

\subsubsection*{Parameter Configuration}
We proposed DVP evaluation for ARQ and HARQ across various system parameters, leading to numerous permutations of parameter settings and illustrations. However, for clarity and conciseness, we limit our evaluation to key configurations.
Since incorporating decoding and feedback delays and supporting parallel ARQ/HARQ processes is the key novelty of this work, we choose RTT and the delay threshold $d$, which directly influences the DVP. 
We consider allocated $\nrb$ and packet length to highlight resource allocation implications and the arrival frequency $f$ to study system throughput. 
Default parameter values and ranges are listed in TABLE~\ref{table:params}. We set $\hvar=1$, $\gamma = 10$ dB, and $V=1$ as $\vert V-1\vert<0.0414,\,\forall\,\text{SNR}>4.1$ dB in an AWGN channel~\cite{polyanskiy2010FBL}. In each figure, a subset of parameters is varied, and the default values are used for the rest.

While some parameters like $\nrb$ are configurable, others are application-specific or come from the device capabilities. For example, \cite{3gpp.22.261} outlines KPI requirements for various applications. Similarly, RTT depends on factors such as decoding capabilities, feedback scheduling delays, and priority levels assigned by the gNB.
To ensure broad applicability, we use parameters within a typical range and present results independent of specific applications or scenarios.
\begin{table}[t]
\centering
\renewcommand{\arraystretch}{1.25}
\begin{tabular}{|l|l|l|}
\hline
Parameter             & Default value  & Range                                                                                                                            \\ \hline
$n$                   & 100$\times$8 b & $\{30, 50, 100, 200\}\times8$ b                                                                                                  \\
$\eta_{\mathrm{min}}$ & 0.2344         & -                                                                                                                                \\
$\eta_{\mathrm{max}}$ & 5.5547         & -                                                                                                                                \\
$\nrb$                & 10             & $\left\{\left\lceil\frac{n}{180\eta_{\mathrm{max}}}\right\rceil,\left\lceil\frac{n}{180\eta_{\mathrm{min}}}\right\rceil\right\}$ \\
$\snr$                & 10 dB          & -                                                                                                              \\
$\ddecod$             & 1 slot         & -                                                                                                                                \\
$\delta$              & 2 slots        & -                                                                                                                                \\
RTT                   & 4 slots        & $[1,7]$ slots                                                                                                                    \\
$d$                   & 8.5 ms         & $[2, 20]$ ms                                                                                                                     \\
$f$                   & $\frac{1}{3}$  & -                                                                                                                                \\
$\hvar$               & 1              & -                                                                                                                                \\
$V$                   & 1              & -                                                                                                                                \\ \hline
\end{tabular}
\caption{Default values and range of important parameters.}
\label{table:params}
\end{table}
\begin{figure}[b]
\centering
\includegraphics[width=0.98\linewidth]{Figures/comparison.eps}
\caption{Performance comparison of the proposed DVP evaluation schemes with the immediate feedback (IF) schemes for default configuration. The simulated DVP is also shown.}
\label{fig.Comparison}
\end{figure}
\begin{figure*}[t]
\centering
\begin{subfigure}[t]{0.49\linewidth}
\centering
\includegraphics[width=\textwidth]{Figures/blerVsNrb_vsRtt.eps}
\caption{DVP vs. $\nrb$ for different RTT. $d=8.5$ ms.}
\label{fig:Sim_dvp-nrb-rtt}
\end{subfigure}
\begin{subfigure}[t]{0.49\linewidth}
\centering
\includegraphics[width=\textwidth]{Figures/blerVsNrb_vsDelayTarget.eps}
\caption{DVP vs. $\nrb$ for different target delay $d$. RTT $=4$ slots.}
\label{fig:Sim_dvp-nrb-d}
\end{subfigure}
\caption{DVP vs. allocated $\nrb$ per slot for different delay parameters, namely RTT and $d$.}
\label{fig:Sim_dvp-nrb-nAndRtt}
\end{figure*}

Now, we move on to the MCS selection mechanism. 
The MCS selection follows 3GPP standards~\cite{3gpp.38.214}, where we choose an MCS index from 0 to 28, and obtain the modulation order, coding rate and spectral efficiency $\eta$ corresponding to it.
This range of MCS corresponds to a lower and upper limit of possible $\nrb$ for a given uncoded packet length $n$.
To minimize DVP for a given $\nrb$, the smallest MCS with $180\nrb\eta\geq n$ is chosen, that is, an MCS capable of supporting $n$ bits on $\nrb$ resources. 

We compute PER for ARQ and HARQ using \eqref{eq:secModel_perARQ} and \eqref{eq:secModel_perHARQ}, respectively. 
For DVP calculation, we use \eqref{dvpbound} for ARQ and ALGORITHM~\ref{Alg:getWaitProbability} with \eqref{eq:secIR_waitDelay} and \eqref{eq:secIR_totDelay} for HARQ. Note that we observe the DVP changing in steps at various points in all figures. This results from the finite and discrete nature of MCS selection and RB allocation in the 5G standard. 


\subsubsection*{Performance Comparison}
To evaluate performance, we compare the proposed methods with state-of-the-art single-server models, which are accurate only under the assumption of zero decoding and feedback delays, thereby eliminating the need for multiple processes. In this section, these models are referred to as the immediate feedback models (IF), where feedback is assumed to be available immediately after the transmission slot, effectively setting RTT to one slot.
The IF serves as the benchmark because the key novelty of this work lies in addressing the unrealistic assumptions of zero RTT and single-process ARQ/HARQ implementations.
We use two IF benchmarks, ARQ-IF and HARQ-IF, derived by setting $\ddecod = \delta = 0$ in the ARQ and HARQ schemes, respectively.


In Fig.~\ref{fig.Comparison}, we present the DVP for ARQ, HARQ, and IF under the default parameter settings, with the HARQ results validated using an event-based numerical simulation in MATLAB. The simulation confirms that, under the model assumptions, HARQ accurately computes the DVP across different values of $\nrb$, reflecting a realistic 5G scenario.
Next, we observe that ARQ consistently produces a worse DVP than HARQ, as expected, due to the absence of incremental redundancy, a key feature of modern 5G systems. This trend holds throughout the section, except for specific corner cases, such as those observed for IF when $\nrb>4$. In this case, this results from the 1-slot RTT that allows up to 8 attempts within the default target delay of 8.5 ms. While ARQ can fully utilize these opportunities, HARQ is constrained to $M=4$ retransmissions.
Finally, IF shows a DVP that is 4 to 6 orders of magnitude smaller than that of HARQ, which is the price one has to pay for the added delay. This shows that IF comes with a large sacrifice in terms of DVP accuracy if used to approximate the DVP of a realistic 5G HARQ.
This comparison also highlights that ARQ is a much better closed-form approximation for HARQ than the IF.

In the remainder of this section, we evaluate the DVP trends across various parameters. As the accuracy improvement with respect to IF remains consistent across configurations, we focus only on the proposed ARQ and HARQ schemes to maintain clarity.
\begin{figure}[t]
\centering
\includegraphics[width=0.98\linewidth]{Figures/bler_vs_dByRttRatio.eps}
\caption{DVP vs. $\nicefrac{d}{1+\ddecod+\delta}$, the delay target to RTT ratio.}
% \caption{DVP vs. delay target to RTT ratio.}
\label{fig:Sim_dvp-dByRtt}
\end{figure}

\begin{figure*}[t]
\centering
\begin{subfigure}[t]{0.49\textwidth}
\centering
\includegraphics[width=\textwidth]{Figures/blerVsNrb_vsPktLen.eps}
\caption{DVP vs. allocated $\nrb$ per slot.}
\label{fig:Sim_dvp-nrb-n}
\end{subfigure}
\begin{subfigure}[t]{0.49\textwidth}
\centering
\includegraphics[width=\textwidth]{Figures/blerVsNrbUsed_vsPktLen.eps}
\caption{DVP vs. average consumed $\nrb$ per packet.}
\label{fig:Sim_dvp-nrbUsed-n}
\end{subfigure}
\caption{DVP vs. resource blocks $\nrb$ for different uncoded packet lengths $n$.}
\label{fig:Sim_dvp-nrb-nAndUsedn}
\end{figure*}

\subsubsection*{The Effect of Delay Parameters:}
Now, we examine the impact of various delay components in the DVP of ARQ and HARQ.
In Fig.~\ref{fig:Sim_dvp-nrb-rtt} and Fig.~\ref{fig:Sim_dvp-nrb-d}, we show the DVP as a function of allocated $\nrb$ per slot across different RTTs and target delays.
Note that the RTT=1 corresponds to the IF approximation. 
We focus on two key observations. 
First, we observe that RTT substantially influences DVP, and a larger RTT increases the significance of this work over IF assumptions, especially with larger resource allocations.
This is because a large RTT could quickly eat into the delay margin with each retransmission. We also observe that the performance gap between ARQ and HARQ increases when the delay margin is not tight, corresponding to a larger target delay or a smaller RTT.
Second, the improved DVP through a larger $\nrb$ (and a lower MCS resulting from it) becomes more pronounced with a longer delay target. 
For example, the 3-order magnitude DVP improvement between $\nrb=1$ (MCS-28) and $\nrb=19$ (MCS-0) at a default $d=8.5$ ms grows to 7 orders at $d=16.5$ ms, i.e., doubling the delay target provides an additional DVP improvement of up to 4 orders.

Having observed that an increase in target delay or a decrease in RTT similarly affects DVP, it becomes useful to assess DVP as a function of their ratio.
To this end, we show DVP against the target delay-to-RTT ratio $\frac{d\times10^3}{1+\ddecod+\delta}$ in Fig.~\ref{fig:Sim_dvp-dByRtt}, for a fixed $\nrb$ of 10 and a packet length of 100 Bytes, corresponding to MCS-3. 
The RTT is converted to milliseconds for consistency, with 1 ms slots in the chosen numerology of 0 (see Section.\ref{sec:model}).
To obtain this ratio, we fix RTT at various values, as depicted, and vary the target delay from 2.5 to 10 ms.
The plot shows a consistent improvement of approximately one order of magnitude in DVP per unit increase in the ratio across RTT values. 
Another interesting observation comes from the comparison of ARQ and HARQ. While HARQ understandably provides better DVP performance in general, it saturates at around $10^{-7}$ for this default configuration. 
This limitation arises because HARQ, unlike ARQ, is restricted in its retransmission attempts and thus cannot fully leverage larger delay margins, as seen by comparing $k_d$ from \eqref{eq:secARQ_kd} and \eqref{eq:secIR_kd}.
\begin{figure}[t]
\centering
\includegraphics[width=0.98\linewidth]{Figures/blerVsNrb_vsNrbPerPktLen.eps}
\caption{DVP vs. allocated resources per slot per Byte of packet length, $\nicefrac{8\nrb}{n}$ for different $n$.}
\label{fig:Sim_dvp-nrb-nrbByn}
\end{figure}

\subsubsection*{The Effect of Resource Allocation:}
Now, we study the effect of packet length and resource allocation in DVP. In Fig.~\ref{fig:Sim_dvp-nrb-n}, we show the DVP variation with allocated $\nrb$ per slot for different uncoded packet lengths. 
As seen already, increasing $\nrb$ generally improves DVP, and the improvement rate is significant, providing up to a four-order reduction in DVP across the $\nrb$ range, corresponding to an equivalent MCS range from 28 down to 0. This range of $\nrb$ depends on the packet length. Thus, as observed in the figure, this DVP reduction with respect to $\nrb$ is much steeper for smaller packet sizes.


Note that when the PER is high, resulting from a frugal resource allocation, the number of retransmissions required for success also gets higher.
This increases the average number of resources consumed per packet as illustrated in Fig.~\ref{fig:Sim_dvp-nrbUsed-n}, where DVP is plotted against the average resource consumption per packet for different packet lengths.
We used the same data from Fig.~\ref{fig:Sim_dvp-nrb-n} for comparison, and one can observe that the curve gets steeper for smaller $\nrb$. 
This effect becomes extreme for persistent ARQ with unlimited retransmission attempts, where the expected number of attempts is given by ${1}/{1-p}$.

In Fig.~\ref{fig:Sim_dvp-nrb-nrbByn}, we analyze the combined effect of $\nrb$ and $n$ by studying resource allocation normalized by packet length. The DVP is plotted against $\nrb$ per byte for different fixed packet lengths. 
Notably, the plots for different packet lengths align closely, indicating that the DVP depends primarily on the resources allocated per byte rather than on the individual values of $\nrb$ or $n$. This insight shows that with properly allocating resources, lower DVP can be achieved even for larger packets.
\begin{figure*}[t]
\centering
\begin{subfigure}[t]{0.49\textwidth}
\centering
\includegraphics[width=\textwidth]{Figures/DVP_vsArrRate_vsRTTAndNrb.eps}
\caption{DVP vs. of arrival rate.}
\label{fig:DvpvsArrivalrate}
\end{subfigure}
\begin{subfigure}[t]{0.49\textwidth}
\centering
\includegraphics[width=\textwidth]{Figures/throughput_fordifferentRTTAndNrb.eps}
\caption{Throughput vs. arrival rate showing the optimums.}
\label{fig:ThroughputVsArrivalrate}
\end{subfigure}
\caption{DVP and throughput vs. arrival rate for HARQ with different RTT and $\nrb$ with a fixed $n$ and varying $f$.}
\label{fig:DvpThroughputVsArrivalRate}
\end{figure*}

% \begin{figure}[t]
% \centering
% \includegraphics[width=0.98\linewidth]{Figures/DVP_vsArrRate_vsRTTAndNrb.eps}
% \caption{DVP vs. of arrival rates for different RTT and $\nrb$.}
% % \caption{DVP vs. delay target to RTT ratio.}
% \label{fig:DvpvsArrivalrate}
% \end{figure}
% \begin{figure}[t]
% \centering
% \includegraphics[width=0.98\linewidth]{Figures/throughput_fordifferentRTTAndNrb.eps}
% \caption{Throughput vs. arrival rate for different RTT and $\nrb$, showing throughput-maximising arrival rate.}
% % \caption{DVP vs. delay target to RTT ratio.}
% \label{fig:ThroughputVsArrivalrate}
% \end{figure}


\subsubsection*{Effect of Arrival Rate:}
A 5G system with strict latency requirements must discard all packets that violate the target delay, leaving only the remaining packets to contribute to the throughput. 
The throughput thus depends primarily on the arrival rate and the DVP. 
To study this relationship, we show the DVP and throughput of HARQ as a function of the arrival rate in Fig.~\ref{fig:DvpThroughputVsArrivalRate}.
Here, we vary the arrival rate $fn/T$ by adjusting $f$, while keeping the packet length $n$ fixed at its default value.
The dotted lines correspond to variations in $\nrb$ for a fixed RTT, while the solid lines represent variations in RTT for a fixed $\nrb$.

In Fig~\ref{fig:DvpvsArrivalrate}, observe the region of arrival rate at which the DVP rises sharply toward 1 from its asymptotic lower bound.
This rise in DVP lowers the throughput, leading to the emergence of an optimal arrival rate that maximizes the throughput, as illustrated in Fig.~\ref{fig:ThroughputVsArrivalrate}.
Notably, while RTT is one of the key parameters deciding the DVP, the arrival rate at which the DVP goes to a very high value (ca. $750$ kbps in this example), and thus the optimum arrival rate, appears largely independent of the RTT. This, however, is not the case for different $\nrb$, where the optimum arrival rate increases with more resource allocation. 
These observations are useful in the resource allocation tailored for RTT and packet length.






