\section{Deterministic Arrivals}\label{sec:detArr}
\todo[inline]{Remove this, or move to an appendix?}
In this section, we look at a URLLC-enabled system where the arrivals are deterministic with a cycle time $c$. This could, for instance, be a UE with sensor capabilities that samples the environment and send the data to the gNB. 
The DVP is straightforward when $c\geq M\cdot\text{RTT}$ and resembles the BAR model discussed as a warm-up in Section~\ref{sec:model}, where the queue is never formed.
Otherwise, a numerical approach similar to HARQ is required.
\subsection{Queing Model}
We observe the queue state only at the boundary of every $c$ slots. Between observations, exactly one packet arrives, and up to $c$ packets may succeed. A transition from state $q$ to state $(q-k+1)$ occurs when exactly $k$ slots succeed, where $0 \leq k \leq \min(q+1, c)$, with the following probabilities:
\begin{align}
\mathbb{P}(q \to q-k+1) &= \binom{c}{k} (1-p)^k p^{c-k},\;q\geq c,\,0\leq k \leq c,\label{eq:secDet_transProb1}\\   
\mathbb{P}(q \to q-k+1) &= \binom{c}{k} (1-p)^k p^{c-k},\;q<c,\,0\leq k\leq q,\label{eq:secDet_transProb2}\\  
\mathbb{P}(q \to 0) &= \sum_{k=q+1}^{c}\binom{c}{k} (1-p)^k p^{c-k},q<c.\label{eq:secDet_transProb3}
\end{align}
When $q < c$, a transition to state 0 is realized by summing the probabilities for multiple outcomes, as shown in \eqref{eq:secDet_transProb2} and \eqref{eq:secDet_transProb3}.
% For easier understanding, this single expression can be split into multiple cases as follows:
% \begin{itemize}
% \item For state $q = 0 $:
% \begin{itemize}
%     \item $\mathbb{P}(0 \to 1) = p^c$
%     \item $ \mathbb{P}(0 \to 0) = 1-p^c$
% \end{itemize}
% \item For states $1\leq q\leq c-1$:
% \begin{itemize}
%     \item $\mathbb{P}(q \to q+1) = p^c$
%     \item$\mathbb{P}(q \to q-(k-1)) = \binom{c}{k} (1-p)^k p^{c-k},\,\forall k\leq q$
%     \item $\mathbb{P}(q \to 0) =\sum_{k=q+1}^{c}\binom{c}{k} (1-p)^k p^{c-k}$
% \end{itemize}
% \item For states $q\geq c$:
% \begin{itemize}
%     \item     $\mathbb{P}(q \to q-(k-1)) = \binom{c}{k} (1-p)^k p^{c-k}$
% \end{itemize}
% \end{itemize}

Define $(x)^+ \coloneqq \mathrm{max}(0, x)$. The steady-state probabilities $\pi_q$ for $q \geq 0$ can be computed by solving the following equations:
% \begin{align*}
%         \pi_0 &=\sum_{k=1}^{\min(1+c, c)} \pi_{k-1} \binom{c}{k} (1-p)^k p^{c-k}\\
%         \pi_1 &= \pi_0 p^c + \pi_1 \binom{c}{1} (1-p) p^{c-1} + \sum_{k=3}^{\min(2+c, c)} \pi_{k-1} \binom{c}{k} (1-p)^k p^{c-k}\\
%         \pi_q &= \pi_{q-1} p^c + \pi_q \binom{c}{1} (1-p) p^{c-1} + \sum_{j=q+1}^{\min(+1+c, c)} \pi_{j-1} \binom{c}{j} (1-p)^j p^{c-j}
% \end{align*}
\begin{align*}
\pi_q &=\sum_{k=0}^{c}\binom{c}{k}\,(1-p)^k p^{c-k}\;\pi_{(q+k-1)^{+}}\\
&=p^c\,\pi_{(q-1)^+}\;+\;\sum_{k=1}^{c}\binom{c}{k}\,(1-p)^k p^{c-k}\,\pi_{q+k-1}
\intertext{That is,}
\pi_q &=\begin{cases}
    &p^c\,\pi_{0}\;+\;\sum_{k=1}^{c}\binom{c}{k}\,(1-p)^k p^{c-k}\,\pi_{q+k-1},\\
    &\hspace{16em}\quad if\;q=0,\\ 
    &p^c\,\pi_{q-1}\;+\;\sum_{k=1}^{c}\binom{c}{k}\,(1-p)^k p^{c-k}\,\pi_{q+k-1},\\
    &\hspace{16em}\quad if\;q>0,
\end{cases}
\end{align*}
where, $\pi_q=0,\,\forall q>\qmax.$
We Solve this by choosing an appropriate maximum queue size $Q$ and finding $\{\pi_q\}$ using either the equations or the transition probability matrix. If $\pi_{Q_{\max}}$ is not sufficiently small, we repeat the process with a larger $Q_{\max}$. 

With the queue length statistics available, finding the service delay and wait delay conditioned on the queue length follows the same steps as in previous sections. We marginalize these over the steady state queue length probabilities to get the unconditioned wait and service delays and then compute the DVP.

Recall that the comparison between deterministic arrivals and random arrivals is interesting. Even if the average arrival rate is the same, the queues can be different. How different, of course, depends on the parameters. This is because the variance of the inter-arrival time in terms of a queue with deterministic arrivals is zero and, hence, does not result in larger queues due to burst arrivals. Thus the average queue length -- and as a result, wait times -- of a queue with deterministic arrivals are typically smaller than that of a queue with random arrivals with the same arrival and service rates. 
