\section{Related work}
\subsection{On the environmental impacts of data and AI technologies}

In recent decades, AI has been promoted as a key tool for tackling climate change____, with claims such as `AI is essential for solving the climate crisis'____. Yet, critical analyses of the environmental costs have raised doubts about the sustainability of AI____.  Recent developments in Deep Learning (DL) are considered computationally intensive, meaning they require vast amounts of energy and water to train algorithmic systems.\footnote{The electronic circuits used to train GenAI, such as GPUs, NPUs, and TPUs, require more electricity than previous circuits to analyse data and train algorithms. Increased electricity consumption also translates into higher water usage to cool these chips.} If this energy comes from burning fossil fuels, training algorithmic systems results in carbon emissions, which is widely recognised as an environmental impact____. Within DL, GenAI models have raised significant concerns about the benefits of AI technologies in addressing climate change, due to their computational costs and the resulting environmental impacts. Indeed, Google, Meta, and Microsoft have acknowledged in their sustainability reports that GenAI technologies are driving up their environmental impacts. As a result, the growing literature on the environmental effects of GenAI has shown that, rather than being a solution to climate change, this technology is contributing to it____.

Within the FAccT community, Emily Bender et al. (2021) were pioneers in raising awareness about the environmental harms of LLMs in their influential work, on \textit{Stochastic Parrots}____. As a result, FAccT scholarship has increasingly focused not only on the social and technical dimensions of AI but also on its environmental impact____. As a result, taxonomies of algorithmic harms are now incorporating ecological harms as part of the risks associated with AI implementation, such as carbon emissions and water usage____.  In this context, Luccioni, Viguier, and Ligozat (2023) investigated the carbon footprint of large models such as BLOOM, unveiling that it emits 50 tonnes CO$_2$eq emissions____. As this body of work demonstrates, examining the infrastructure of GenAI technology reveals harms that data centres pose not only to the environment but also to local communities. For example, media scholar Mél Hogan (2015) highlighted that data centres in the US extract large amounts of water____, while Libertson et al. (2021) and Bresnihan and Brodie (2021) criticised the substantial energy demands of the data centre industry, which disrupts energy access and has led to electricity grid collapses in local communities in Sweden and Ireland, respectively____. Furthermore, other critical scholars have argued that the environmental harms of AI should be considered across the entire production process, rather than focusing solely on data centres. Examining other industrial activities crucial to AI development, such as mineral extraction, chip manufacturing, and e-waste dumping, uncovers additional environmental harms____. These perspectives suggest that attention to the environmental harms of AI should also encompass issues such as soil degradation and the waste generated by the AI industry, extending beyond carbon emissions and water withdrawal just during the algorithmic training phase.


\subsection{An ecofeminist approach towards data and AI}

Feminism and AI are two interconnected concepts that have been widely analysed, but what about ecofeminism and AI? There have been numerous studies highlighting the need to incorporate feminist perspectives into AI. As an example, the \textit{Data Feminism} framework proposed by scholars D'Ignazio and Klein (2020)____ draws on seven principles to expose how feminism can help rebalance power in AI/ML research, including embracing pluralism, which is strongly aligned with ecofeminist values. However, three years after their book was published, they revisited these principles, emphasising the importance of analysing the environmental issues posed by AI development and deployment, which should be integrated into data feminist approaches____. Despite this, AI feminist perspectives have not yet explored the connection between gender oppression and environmental injustice in the technological context. 

As discussed earlier, an ecofeminist perspective on AI must encompass its values, such as plurality, acknowledge the connection between gender and environmental struggles, and critique the hegemony of Western science. This latter ecofeminist value has been crucial in critical feminist and STS scholarship, such as the work of Judy Wajcman, Ruha Benjamin, and Donna Haraway, who have shed light on how technological artifacts embody white and masculine hegemonic values and ideologies____. For instance, in the early 1990s, Haraway discussed how `Operations Research began with the WWII and efforts to coordinate radar devices… Statistical models were increasingly applied to problems of simulation and prediction for making key decisions'~\cite[p. 58]{haraway1991simians}. This body of scholarship led critical scholars to recognise that AI technologies, which are based in Operational Research and Statistics, among other fields, also reproduce white and masculine epistemologies____ and show connections to the US military-industrial complex____. In fact, OpenAI has recently announced contracts to deploy AI in the battlefield____. During the Radical Book Fair held in Barcelona, Yayo Herrero and Vandana Shiva (2024) discussed the past and future of the ecofeminist movement. Yayo Herrero, an Spanish scholar and ecofeminist, criticised the use of AI in military operations, emphasising that through the AI industry, we are witnessing a `new wrinkle in a technology of death and war, which is able to detect whose life is dispensable'____. Herrero referred to \textit{Lavender}, a Machine Learning (ML)-based algorithmic system implemented by the Israeli army to identify and kill individuals allegedly linked to Hamas____. With a 10\% error rate, this system was deployed to target missiles at individuals previously detected by the algorithm, which could result in the deaths of innocent people. This algorithm directly challenges ecofeminist epistemologies, which prioritise the value of all life, human or non-human, on Earth. Moreover, the increasing environmental harms caused by GenAI-intensive computing present an opportunity to integrate an environmental perspective into feminism and AI. This paper responds to that call by drawing on \textit{Data Feminism} and proposing seven principles that add to them and guide us toward \textit{Data Ecofeminism}.