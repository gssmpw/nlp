\begin{table}[t]
\footnotesize
\centering
\setlength{\tabcolsep}{2.5pt}
\begin{tabular}{lcccc}
\toprule
\Th{Autoencoder} & $\tau$ & \Th{gFID$\downarrow$} & \Th{rFID$\downarrow$} & \Th{ID} \\ 
\midrule 
\sdvae & - & $43.5$ & $0.90$ & $62.2$ \\ \cmidrule(lr){1-5}
+ \texttt{EQ-VAE}  & $R(\theta)$ & $41.2$ & $0.73$ & $57.9$ \\
+ \texttt{EQ-VAE} & $S(s,s)$ & $35.8$ & $0.78$& $41.0$ \\
\rowcolor{TableColor} 
+ \texttt{EQ-VAE} & $R(\theta) \cdot S(s,s)$ & $34.1$ & $0.82$ &$39.4$ \\
+ \texttt{EQ-VAE} & $R(\theta) \cdot S(s_x,s_y)$ & $33.2$ &$0.92$& $38.9$ \\  
\bottomrule
\end{tabular}%
\vspace{-3pt}
\caption{\textbf{Spatial Transformation Ablation in \texttt{EQ-VAE}.}
We measure \Th{gFID}, \Th{rFID}, and intrinsic dimension (ID) for latents regularized via rotations, isotropic scaling, anisotropic scaling, and combinations. Combining transformations lowers ID and enhances generative performance, though anisotropic scaling can slightly degrade reconstruction.}
\label{tab:ablation-trans}
%\vspace{-3pt}
\end{table}
