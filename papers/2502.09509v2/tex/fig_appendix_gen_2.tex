{
\small
\centering
\newcommand{\resultsfignew}[1]{\includegraphics[width=0.09\textwidth,valign=t]{#1}}

\newcommand{\resultsfignewmedium}[1]{\includegraphics[width=0.184\textwidth,valign=t]{#1}}

\newcommand{\resultsfignewbig}[1]{\includegraphics[width=0.372\textwidth,valign=t]{#1}}
\setlength{\tabcolsep}{1pt}

\setlength{\tabcolsep}{1pt}

\begin{tabular}{@{}ccccccc@{}}  % 7 columns total

%=== 1) First caption row across all 7 columns
\mc{7}{Class label = "panda" (388)}\\

%=== 2) Multirow spanning 4 rows (covering columns 1--4)
%    Then columns 5--7 have small images.
\multicolumn{4}{c}{%
    \multirow{4}{*}{%
       \resultsfignewbig{fig/classes/388/000079.png}%
    }%
} & 

\multicolumn{2}{c}{%
    \multirow{2}{*}{%
       \resultsfignewmedium{fig/classes/388/000080.png}%
    }%
}
&
\resultsfignew{fig/classes/388/000017.png}
\\
\mc{7}{\vspace{-2.0ex}}\\

%=== 3) Since the above \multirow spans 4 rows, for rows 2--4,
%    columns 1--4 are 'occupied' by the big image. We do “& & & &” ...
& & & & 
& & 
\resultsfignew{fig/classes/388/000081.png} \\
\mc{7}{\vspace{-2.0ex}}\\

& & & & 
\multicolumn{2}{c}{%
    \multirow{2}{*}{%
       \resultsfignewmedium{fig/classes/388/000082.png}%
    }%
} & 
\resultsfignew{fig/classes/388/000083.png} \\
\mc{7}{\vspace{-2.0ex}}\\

& & & & 
& & 
\resultsfignew{fig/classes/388/000084.png} \\

%=== 4) Some extra vertical space or a rule
\mc{7}{\vspace{-1.ex}}\\

%=== 1) First caption row across all 7 columns
\mc{7}{Class label = “golden retriever” (207)}\\

%=== 2) Multirow spanning 4 rows (covering columns 1--4)
%    Then columns 5--7 have small images.
\multicolumn{4}{c}{%
    \multirow{4}{*}{%
       \resultsfignewbig{fig/classes/207/000099.png}%
    }%
} & 

\multicolumn{2}{c}{%
    \multirow{2}{*}{%
       \resultsfignewmedium{fig/classes/207/000100.png}%
    }%
}
&
\resultsfignew{fig/classes/207/000101.png}
\\
\mc{7}{\vspace{-2.0ex}}\\

%=== 3) Since the above \multirow spans 4 rows, for rows 2--4,
%    columns 1--4 are 'occupied' by the big image. We do “& & & &” ...
& & & & 
& & 
\resultsfignew{fig/classes/207/000102.png} \\
\mc{7}{\vspace{-2.0ex}}\\

& & & & 
\multicolumn{2}{c}{%
    \multirow{2}{*}{%
       \resultsfignewmedium{fig/classes/207/000102.png}%
    }%
} & 
\resultsfignew{fig/classes/207/000104.png} \\
\mc{7}{\vspace{-2.0ex}}\\

& & & & 
& & 
\resultsfignew{fig/classes/207/000103.png} \\

%=== 4) Some extra vertical space or a rule
\mc{7}{\vspace{-1.ex}}\\

%=== 1) First caption row across all 7 columns
\mc{7}{Class label = “macaw” (88)}\\

%=== 2) Multirow spanning 4 rows (covering columns 1--4)
%    Then columns 5--7 have small images.
\multicolumn{4}{c}{%
    \multirow{4}{*}{%
       \resultsfignewbig{fig/classes/88/000033.png}%
    }%
} & 

\multicolumn{2}{c}{%
    \multirow{2}{*}{%
       \resultsfignewmedium{fig/classes/88/000034.png}%
    }%
}
&
\resultsfignew{fig/classes/88/000035.png}
\\
\mc{7}{\vspace{-2.0ex}}\\

%=== 3) Since the above \multirow spans 4 rows, for rows 2--4,
%    columns 1--4 are 'occupied' by the big image. We do “& & & &” ...
& & & & 
& & 
\resultsfignew{fig/classes/88/000006.png} \\
\mc{7}{\vspace{-2.0ex}}\\

& & & & 
\multicolumn{2}{c}{%
    \multirow{2}{*}{%
       \resultsfignewmedium{fig/classes/88/000001.png}%
    }%
} & 
\resultsfignew{fig/classes/88/000003.png} \\
\mc{7}{\vspace{-2.0ex}}\\

& & & & 
& & 
\resultsfignew{fig/classes/88/000005.png} \\
    
\end{tabular}
}