\begin{figure}[t]
    \centering
    \setlength{\tabcolsep}{1.5pt}
    \begin{tabular}{cccc}
      \multirow{2}{*}{\small Input Image \textbf{x}}
      & \multicolumn{2}{c}{{\small \texttt{SD-VAE}}}
      & {\texttt{Ours}} \\
      \cmidrule(lr){2-3} \cmidrule(lr){4-4}
       
      & {\small$\mathcal{D} ( \mathcal{E}(\tau \circ \mathbf{x}))$}
      & {\small$\mathcal{D}(\tau \circ \mathcal{E}(\mathbf{x}))$}
      & {\small$\mathcal{D}(\tau \circ \mathcal{E}(\mathbf{x}))$} \\
      
      \vspace{-0.4cm} \\
      
     \includegraphics[width=0.225\linewidth]{fig/equi_latex_new/000051.png} 
      & \includegraphics[width=0.225\linewidth]{fig/equi_latex_new/51_before.JPEG} 
      & \includegraphics[width=0.225\linewidth]{fig/equi_latex_new/51_after.JPEG} 
      & \includegraphics[width=0.225\linewidth]{fig/equi_latex_new/51_ours.JPEG} \\
      
      \includegraphics[width=0.225\linewidth]{fig/equi_plots/images/00000046.JPEG} 
      & \includegraphics[width=0.225\linewidth]{fig/equi_plots/sd-vae/befre_s05/00000046.JPEG} 
      & \includegraphics[width=0.225\linewidth]{fig/equi_plots/sd-vae/s05/00000046.JPEG} 
      & \includegraphics[width=0.225\linewidth]{fig/equi_plots/ours/s05/00000046.JPEG} \\

    \end{tabular}
  \caption{
  \textbf{Latent Space Equivariance.} 
  Reconstructed images using \texttt{SD-VAE}~\cite{rombach2022high} and our \our when applying scaling transformation $\tau$, with factor $s=0.5$, to the input images $\mathcal{D}(\mathcal{E}(\tau \circ \mathbf{x}))$ versus directly to the latent representations $\mathcal{D}(\tau \circ \mathcal{E}(\mathbf{x}))$. Our approach preserves reconstruction quality under latent transformations, whereas \texttt{SD-VAE} exhibits significant degradation. See ~\autoref{fig:qualitative-equivariance-appendix} for additional examples.
  }
  \label{fig:qualitative-equivariance-main}
\end{figure}
