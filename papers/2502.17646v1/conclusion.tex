% \section{Conclusion and Future Work}  

% This paper presented the architecture of the \digit platform, a Digital Twin framework for ITS that integrates predictive modeling, simulations, and automated workflows to address real-time traffic management challenges. Built on a Domain Concept Model, the architecture models key components such as vehicles, sensors, and communication networks, enabling seamless integration between physical and digital systems. Evaluation results demonstrate the framework’s ability to deliver accurate forecasts and replicate traffic patterns efficiently, validating its adaptability and scalability for dynamic traffic environments.  

% Future work will focus on expanding the architecture to support multi-modal transportation systems, including public transit and pedestrian flows. Further enhancements will include developing models for communication networks and integrating them to perform analyses on key metrics such as expected range, throughput, and latency. Improvements in real-time analytics leveraging IoT and 5G technologies will also be explored to enhance data acquisition and communication reliability. Additionally, adaptive modeling techniques and dynamic calibration methods will be investigated to improve performance during peak traffic and disruptions, ensuring robustness and scalability across diverse and complex transportation scenarios.  
\section{Future Work}  

\noindent Future efforts will focus on extending the architecture for \digit platform to support multi-modal transportation systems, including public transit and pedestrian flows. Enhancements in modeling communication networks will enable evaluations of key metrics such as latency, throughput, and range, improving performance in connected and autonomous vehicle scenarios. Additional focus will be on integrating adaptive modeling techniques and dynamic calibration methods to enhance scalability during peak traffic and disruptions. Testing across diverse traffic scenarios, including urban roads, highways, and mixed-mode networks, will validate generalizability. Finally, advancements in real-time analytics leveraging IoT and 5G technologies will be explored to strengthen data acquisition, communication reliability, and responsiveness, ensuring the platform can handle dynamic traffic environments effectively.
