\section{Related Work}


Digital Twin (DT) technology has shown significant potential across diverse domains by enabling real-time monitoring, predictive analytics, and scenario-based simulations. Its adaptability has made it particularly effective in applications ranging from infrastructure management to smart cities, offering a foundation for addressing the challenges of Intelligent Transportation Systems (ITS).

In infrastructure management, Likhit et al. \cite{kanigolla2024architecting} demonstrated DTs' effectiveness in optimizing water distribution networks through real-time data and simulations. Their work highlights leveraging predictable flow patterns to improve resource management. However, transportation systems add complexity due to unpredictable traffic dynamics and human behavior, as noted by Glaessgen and Stargel \cite{glaessgen2012digital}, requiring robust frameworks to address uncertainties.

Expanding to urban-scale applications, Mohammadi et al. \cite{MohammadiSmartCity} applied DTs for smart city optimization, enabling better decision-making and resource allocation. Similarly, Barat et al. \cite{barat2022agent} modeled pandemic dynamics, emphasizing adaptability in crisis scenarios. Despite these advances, Barn et al. \cite{balbir} highlighted gaps in integrating socio-technical interactions and behavioral modeling, limiting realism in existing frameworks.

To address scalability and interoperability, Redelinghuys et al. \cite{redelinghuys2020six} proposed layered DT architectures, promoting modularity. Complementing this, Kulkarni et al. \cite{kulkarni2023dtworkflow} introduced automated workflows for predictive modeling, ensuring responsiveness under evolving conditions. Yet, these methods often rely on static configurations, posing challenges for dynamic systems requiring continuous retraining.

Despite advancements, Digital Twin (DT) systems still struggle with real-time synchronization between virtual and physical environments, often relying on offline simulations \cite{baolixia}. Recent efforts, such as Ge et al. \cite{kuvsic2023digital}, introduced cyber-physical frameworks with IoT sensors for synchronized traffic simulations. However, scalability to multi-modal urban systems remains a challenge \cite{lee2015cyber}.

To tackle these limitations, Domain-Driven Design (DDD) offers a modular approach for structuring complex DTs. By leveraging bounded contexts, DDD improves scalability and adaptability. Macias et al. \cite{macias2023architecting} applied DDD principles for scalable architectures, Evans \cite{evans2004domain} emphasized adaptability and automated updates, highlighting DDD's potential to meet real-time demands and evolving requirements.

Building on this foundation, we propose an architecture for the \digit platform—a DT framework for ITS that integrates predictive modeling, simulations, and automated workflows. The architecture applies DDD principles to identify domains, which are modeled using a Decision-Component Model (DCM). The DCM captures key entities such as vehicles, sensors, communication networks, and user behaviors, ensuring seamless integration between physical and digital systems. ML models handle traffic forecasting, supported by MLOps pipelines for scalability through automated retraining. Simulations enable scenario testing, providing insights for decision-making and interventions. By combining these elements, our framework addresses modern transportation complexities with a dynamic and scalable approach.

