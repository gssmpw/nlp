\section{Introduction}  

The increasing complexity of modern transportation systems, characterized by growing vehicular density, diverse communication protocols, dynamic and distinct traffic patterns, and complex road networks poses significant challenges in traffic management, safety, and operational efficiency. To this end, Intelligent Transport System (ITS) has emerged as a potential solution as they serve as an ecosystem that integrates technology, communication, and data analytics to improve the efficiency, safety, and sustainability of transportation systems.
ITS solutions are used to manage and optimize traffic flow, enhance user experience, and support advanced applications like autonomous driving and vehicular communication~\cite{perallos2015intelligent, andersen2000intelligent}. Most importantly, the ITS infrastructure helps reduce the accidents that occur on the roads. Hence, it is important to deploy more ITS infrastructure as fatal accidents are on the rise in several countries~\cite{NHTSA_Traffic_Stats, Road_Accidents_India}.


However, deploying an ITS infrastructure, which achieves the earlier mentioned objectives while ensuring safety, is non trivial due to its scale, dynamic environment, and real-time demands. The development of a formally sound ITS solution thus becomes challenging as it is difficult to verify all the what-if scenarios in the physical deployment. In response, Digital Twins (DTs) have emerged as a transformative paradigm, enabling the creation of virtual replicas of physical systems to support real-time monitoring, predictive analytics, and adaptive control mechanisms \cite{grieves2017digital}\cite{liu2021review}\cite{barricelli2019survey}\cite{feng2023resilience}. In the context of ITS, DTs leverage real-time data from IoT-enabled sensors, vehicular networks, and environmental monitoring systems to improve situational awareness and enhance decision-making processes \cite{jafari2023review}. Existing implementations, such as Mobility DT \cite{mobilityWang}, have demonstrated their utility in modeling traffic flows, evaluating rerouting strategies, and predicting congestion patterns through simulation-based approaches. These advancements highlight the potential of DTs to transform urban mobility and transportation planning.  

Although modeling traffic flows works well in several countries, the challenge is significantly higher in many other countries that witness highly unstructured traffic. This scenario also demands robust traffic management solutions to mitigate the social, economic, and environmental impacts of congestion. Traffic flow prediction enables proactive decision-making and efficient traffic control. However, the irregular and dynamic nature of traffic patterns, presents unique challenges in identifying consistent trends. To address these complexities, this project leverages temporal and spatio-temporal Deep Learning (DL) models, such as LSTM (Long Short-Term Memory) \cite{lstmbase} and BiLSTM (Bidirectional LSTM) \cite{bilstm}. LSTM, a variant of Recurrent Neural Networks (RNNs), excels in capturing long-term dependencies in sequential data, making it highly effective for modeling traffic patterns that evolve over time. Its ability to retain and utilize information over long sequences enables accurate predictions of future traffic flows based on historical trends. By integrating real-time data from traffic nodes at intersections in Hyderabad, India, the system ensures precise predictions and real-time adaptability. Automated retraining mechanisms further enhance robustness, contributing to smarter and more efficient traffic management systems.


This paper introduces architecture for \digit (DT for ITS) platform, addressing the limitations of existing frameworks by providing a modular, scalable, and adaptive solution for traffic management. To systematically model the components and behaviors of ITS, the architecture adopts a \textit{Domain Concept Model (DCM)}, which captures key entities such as vehicles, sensors, communication networks, and user behaviors. The DCM serves as the foundation for designing the architecture and defining the interactions required to simulate traffic scenarios and predict outcomes effectively.  

The proposed architecture integrates real-time data streams with Machine Learning (ML) based predictive modeling to optimize traffic flow and enhance decision-making. It employs ML models to forecast traffic patterns based on historical and real-time data. The predictions are validated through simulations, enabling scenario testing and evaluation of intervention strategies. By bridging modeling and simulation, this architecture demonstrates the potential of integrating predictive analytics into real-time traffic management workflows. The architecture for \digit platform incorporates adaptive \textit{Machine Learning Operations (MLOps)} \cite{symeonidis} practices, automating the deployment and lifecycle management of predictive models. This framework supports dynamic adjustments to evolving traffic conditions while maintaining computational efficiency and scalability. The results presented in this work validate the system’s accuracy and responsiveness, demonstrating its suitability for real-world ITS applications.  

