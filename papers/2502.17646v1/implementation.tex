% \section{Implementation}

% The architecture of \digit platform was implemented by integrating multiple technologies to enable real-time traffic monitoring, simulation, prediction, and visualization. Real-time traffic data, including timestamps, flow rates, and vehicle speeds, was collected using sensors deployed near the IIIT Hyderabad campus. These sensors continuously captured traffic data and transmitted it to an API, which handled data preprocessing and storage. The data pipeline ensured the integration of incoming information into the system for further analysis. For traffic simulation, we employed the Simulation of Urban Mobility (SUMO) tool. SUMO enabled the modeling of transportation networks based on real-world sensor data and facilitated testing of intervention strategies. The predictive models were implemented in Python using the TensorFlow library. Two models, LSTM and BiLSTM, were used to temporal patterns in traffic data, enabling short-term traffic forecasting. These models were integrated with an MLOps pipeline that automated training, validation, and deployment. Performance metrics, including Mean Absolute Error (MAE), Mean Absolute Percentage Error (MAPE) and Root Mean Squared Error (RMSE), were monitored to ensure accuracy, and retraining was triggered when performance degradation was detected \cite{bhatt2024towards}. For visualization, a dashboard was developed using React for the frontend and Node.js for the backend. It provided real-time monitoring and analysis, displaying key metrics such as vehicle count, average speed, and traffic intensity levels through interactive graphs. The dashboard was connected to the backend API, which ensured continuous updates based on incoming sensor data and predictions. Cloud infrastructure was used to provide scalable storage and computational resources, supporting data processing and analytics tasks required for the operation of \digit.
\section{Implementation}

The implementation of the \digit platform follows the architecture outlined in Section~\ref{sec:architecture}, operationalizing its components to enable real-time traffic monitoring, prediction, and visualization for ITS. By leveraging a modular and scalable design, the system ensures integration between data acquisition, predictive analytics, simulation, and user-facing dashboards. 

Real-time traffic data was obtained using IoT-enabled sensors deployed near the IIIT Hyderabad campus, capturing key attributes such as timestamps, flow rates, average speeds, and congestion levels. These sensors, corresponding to the \textit{Sensors on the road} in the architecture (Figure~\ref{fig:digit_architecture}), transmitted preprocessed data to a public API hosted on a Virtual Private Server (VPS). This API acted as the primary data pipeline, ensuring scalability and seamless integration of real-time and historical data. This data was further communicated to the \textit{DT Manager} via the \textit{Communication Layer} as shown in (Figure~\ref{fig:digit_architecture}).

Traffic \textit{Simulator}, a critical component of the \textit{Digital Twin Manager} (Section~\ref{sec:architecture}), was implemented using the SUMO. SUMO utilized the processed sensor data to model transportation networks and simulate real-world scenarios, such as peak traffic hours, roadblocks, and intervention strategies. These simulations allowed for scenario-based testing of adaptive traffic management measures, including dynamic signal timing and rerouting strategies, in alignment with the behavioral models defined in the DCM (Section~\ref{sec:architecture}).

Predictive analytics was powered by ML models, which were handled by the \textit{Model Manager}, as described in Section~\ref{sec:architecture}. Specifically, Long Short-Term Memory (LSTM) and Bidirectional LSTM (BiLSTM) networks, implemented in Python using TensorFlow, were used. These models were designed to analyze temporal patterns in traffic data and provide short-term forecasts, such as predicting congestion levels at specific intersections. The models were integrated with an \textit{MLOps pipeline} to automate processes such as training, validation, and deployment. This pipeline continuously monitored model performance using metrics like Root Mean Squared Error (RMSE). Retraining was dynamically triggered when performance degradation exceeded predefined thresholds, ensuring sustained accuracy under evolving traffic conditions, as described in Section~\ref{sec:architecture}.


The \textit{visualization layer} was implemented using a React-based frontend and a Node.js backend, connected to the API for real-time updates. The dashboard provided an intuitive interface for monitoring key metrics such as vehicle counts, average speeds, and traffic intensity. It also visualized simulation results and predictions through interactive graphs and heatmaps, reflecting the actionable insights derived from the DT's predictive and simulation engines. This implementation extended the DCM’s state and behavioral models into user-facing applications, ensuring operational transparency and stakeholder engagement. The system leveraged the cloud infrastructure for scalable storage and computational resources, supporting both real-time analytics and batch processing. This infrastructure enabled seamless integration across all layers of the architecture, ensuring robust and efficient operation. Overall, the implementation demonstrated the practical realization of the proposed architecture, maintaining synchronization between physical and digital systems while addressing the complex requirements of ITS.
