\section{ITS Deployment Case Study}
\label{sec:usecase}
The creation of a Digital Twin (DT) for Intelligent Transportation Systems (ITS) entails the deployment of IoT-enabled traffic monitoring infrastructures in urban settings. This approach aims to establish a robust and scalable framework for analyzing and enhancing traffic flow efficiency. The primary objectives include real-time data acquisition, traffic scenario simulation, and predictive modeling to tackle issues such as congestion, delays, and inefficiencies in traffic control.

For illustrative purposes, an ITS prototype has been implemented through a network of edge devices positioned strategically at high-traffic intersections near the IIIT Hyderabad campus. These devices are outfitted with cameras and processing units that gather real-time traffic metrics, such as vehicle counts, density levels, and congestion indicators, at regular time intervals. Advanced temporal models, including Long Short-Term Memory (LSTM) and Bidirectional LSTM (BiLSTM), are utilized for on-device data processing. These models excel in recognizing temporal patterns in traffic data and forecasting future trends, leveraging LSTM's capability to capture long-term dependencies and BiLSTM's strength in analyzing data sequences from both directions.

To enhance data quality and storage efficiency, the processed information is consolidated into 5-minute intervals before being transmitted to a public API hosted on a Virtual Private Server (VPS). This centralized API facilitates seamless data access for various system components. Moreover, the localized processing on edge devices reduces latency, enabling prompt responses to dynamic traffic situations.

This configuration offers continuous, real-time insights into traffic dynamics, particularly valuable during peak hours or unforeseen events like accidents or road closures. The combination of LSTM and BiLSTM models significantly improves the system's adaptability to fluctuating traffic conditions. The architecture also supports periodic retraining of the models, triggered either by scheduled routines or performance-based criteria, ensuring sustained accuracy and relevance of predictions.

The accumulated data serves as the foundation for simulations conducted with tools like the Simulation of Urban MObility (SUMO) \cite{SUMO2018}, which replicate actual traffic scenarios to evaluate vehicle behavior at monitored intersections. These simulations can assess the impact of strategies such as modifying traffic light schedules or redirecting traffic flow to alleviate congestion. Additionally, machine learning models analyze both real-time and historical datasets to forecast future traffic conditions, identifying trends like anticipated congestion spikes during evening rush hours.

This case study illustrates the potential of DT in enhancing traffic management in urban environments. By integrating real-time data monitoring, scenario simulation, and predictive analytics, the system equips traffic authorities with the capability to observe current conditions and virtually test potential interventions before applying them in the real world. Throughout this paper, we refer to this case study to contextualize and elaborate on the architecture proposed for developing and deploying a DT in ITS applications.

