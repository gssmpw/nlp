\section{Evaluation}

The evaluation of the architecture of \digit platform is guided by three key aspects: 

\begin{itemize}
    \item \textit{Accuracy}: The predictive performance of the analytics component, measured against observed traffic flow data.  
    \item \textit{Fidelity}: The ability of the simulation environment to replicate real-world traffic patterns and responses to interventions.
    \item \textit{Efficiency}: The computational performance of the system, including the time required to execute predictions and run simulations.  
\end{itemize}

\subsection{Data Collection and Setup}

The data used for evaluating \digit was collected from IoT-enabled sensors deployed at intersections near the IIIT Hyderabad campus, as described in Section~\ref{sec:usecase}. These sensors capture traffic parameters, including vehicle counts, speeds, and congestion levels, at fixed intervals of \textit{5 minutes}. This frequency ensures that the system captures real-time variations in traffic conditions, providing a foundation for both predictive analytics and simulations. The collected data is transmitted via the \textit{Communication Layer} to the \textit{Data Lake}, where it is preprocessed to handle missing values, normalize readings, and structure data for modeling. The dataset spans several days, encompassing both peak and non-peak traffic conditions, making it suitable for evaluating system performance across varying levels of congestion. Using this detailed data set, the DT ensures accurate modeling of traffic dynamics, allowing more realistic simulations and precise predictions. To develop and validate predictive models, the data set was divided into a \textit{ training set (75\%)}, \textit{validation set (15\%)}, and a \textit{testing set (10\%)}. The data was structured into sequences of \textit{15 input timesteps} to predict the \textit{16th timestep}, reflecting the temporal dependencies required for short-term forecasting. For the current implementation, the model takes traffic flow data as input and outputs predicted flow values for future time intervals. 

\subsection{Predictive Model Validation: Accuracy}  

The predictive analytics component of the architecture of \digit platform employs deep learning models, specifically Long Short-Term Memory (LSTM) and Bidirectional Long Short-Term Memory (BiLSTM), implemented within the \textit{MLOps Pipeline}, as shown in Figure~\ref{fig:digit_architecture}. These models are designed to capture temporal dependencies in traffic data, enabling accurate short-term predictions of traffic flow and congestion patterns. Both models were trained using preprocessed data collected from IoT-enabled sensors, as described in Section~\ref{sec:usecase}. The data was structured into sequences of \textit{15 input timesteps} to predict \textit{16th timesteps}, reflecting the temporal dependencies required for short-term forecasting. Traffic flow rates, measured in vehicles per 5 minutes, were used as the target variable to ensure precise predictions suitable for real-time decision-making and intervention strategies. For example, the models can predict increased congestion at an intersection, triggering appropriate interventions such as rerouting traffic or adjusting signal timings. The performance of the LSTM and BiLSTM models was evaluated using standard metrics, including Mean Absolute Error (MAE) (vehicles), Root Mean Squared Error (RMSE) (vehicles), and Mean Absolute Percentage Error (MAPE) (percentage). The results are presented in Table~\ref{tab:model_metrics}.  

\begin{table}[h!]
\centering
\caption{Performance Metrics for Predictive Models}
\label{tab:model_metrics}
\begin{tabular}{|c|c|c|c|}
\hline
\textbf{Model} & \textbf{RMSE (vehicles)} & \textbf{MAE (vehicles)} & \textbf{MAPE (\%)} \\ \hline
LSTM           & 25.522                   & 18.394                  & 13.3               \\ \hline
BiLSTM         & 24.451                   & 17.255                  & 19.1               \\ \hline
\end{tabular}
\end{table}

\noindent The results demonstrate that both models effectively capture traffic dynamics. The BiLSTM model achieved slightly lower RMSE and MAE values, indicating higher predictive accuracy. However, the LSTM model exhibited a lower MAPE, suggesting it may be more robust for percentage-based error evaluation. These findings highlight the complementary strengths of the two models, depending on specific performance criteria. Predictions generated by the models were visualized through the \textit{Visualization Layer} using the \textit{Dashboard for predictions}, as shown in Figure~\ref{fig:traffic_prediction_dashboard}, providing insights into both immediate and future traffic patterns. These insights allow stakeholders to assess the effectiveness of interventions in real-time, ensuring timely and informed traffic management decisions.
\begin{figure}[ht!]
    \centering
    \includegraphics[width=\linewidth]{images/dashboard/PredictionDashLight.png}
    \caption{Traffic Prediction Dashboard displaying accuracy, prediction errors, and performance metrics.}
    \label{fig:traffic_prediction_dashboard}
\end{figure}


\subsection{Simulation Fidelity}  

The simulation fidelity of the architecture of \digit platform was evaluated by analyzing its ability to replicate observed traffic patterns and predict traffic flow dynamics accurately. This evaluation combined machine learning predictions with simulation outputs to validate the system’s ability to model real-world traffic behavior effectively. Traffic predictions were generated using Long Short-Term Memory (LSTM) and Bidirectional LSTM (BiLSTM) models, implemented within the \textit{MLOps Pipeline}. These predictions were compared against actual traffic flow data collected at 5-minute intervals from IoT-enabled sensors, as shown in Figure~\ref{fig:pred_vs_actual}. The results demonstrated a strong correlation between predicted and observed values, capturing key traffic features such as congestion buildup, clearance, and peak-hour variations. To ensure consistency, the modeling assumptions between SUMO simulations and the DT were carefully aligned, including vehicle flow rates, signal timing protocols, and road network configurations. For example, vehicle flow rates from IoT sensors were matched with the input parameters in SUMO to reflect observed traffic volumes. This alignment ensures that the virtual simulations reflect real-world dynamics as closely as possible. Some spikes were observed during periods of high congestion, as shown in Figure~\ref{fig:pred_vs_actual}, due to insufficient data for such scenarios; however, the model is expected to improve with additional data collection. Simulations were executed using the SUMO platform, integrated within the \textit{Simulation and Digital Twin Manager}, to model virtual scenarios reflecting real-world dynamics. These simulation outputs were visualized in the \textit{Visualization Layer} via an interactive dashboard, providing insights into traffic behavior and intervention strategies, as shown in Figure~\ref{fig:simulation_dashboard}.

\begin{figure}[ht!]
    \centering
    \includegraphics[width=\linewidth]{images/LSTMgraph.png}
    \caption{Predicted vs Actual Traffic Flow.}
    \label{fig:pred_vs_actual}
\end{figure}

\begin{figure}[ht!]
    \centering
    \includegraphics[width=\linewidth]{images/dashboard/TrafficSimulationLight.png}
    \caption{Simulation Dashboard displaying traffic flow timelines, speed distributions, and junction traffic visualizations.}
    \label{fig:simulation_dashboard}
\end{figure}

\subsection{Computational Efficiency}  

The computational performance of \digit was evaluated to validate its suitability for real-time applications. The assessment measured execution times for both predictive modeling and simulation tasks, ensuring responsiveness under time-sensitive conditions. The LSTM and BiLSTM models, implemented within the \textit{MLOps Pipeline}, required an average of \textit{7 milliseconds} to process 15 input timesteps and generate predictions for the next 16 timesteps. Simulations covering the same interval were executed within \textit{15 seconds} using SUMO. These results demonstrate that the system operates within time constraints appropriate for real-time traffic management, enabling timely responses to evolving traffic conditions. The integration of the \textit{MLOps Pipeline} further enhances computational efficiency by automating the retraining and deployment of predictive models. This process ensures that the system adapts dynamically to changes in traffic behavior without manual intervention, maintaining both accuracy and scalability as data volumes increase.

