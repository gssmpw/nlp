\section{Snowball Adversarial Attack on Street Signs}

This section introduces the Snowball Adversarial Attack, an adversarial strategy that incrementally amplifies perturbations to mislead traffic sign classification systems. Inspired by the accumulation of snow or snowball patch on a street sign, the attack progressively builds upon small, natural-looking modifications -- such as occlusions from snow -- until misclassification occurs. Tested on real-world traffic sign images from Street View, the Snowball Adversarial Attack demonstrates its effectiveness against robust recognition models, highlighting the growing threat of evolving adversarial manipulations.

\subsection{Street Signs}

Street signs serve as essential components of traffic regulation and road safety, providing critical information for both human drivers and autonomous systems. However, machine learning-based traffic sign recognition models remain susceptible to adversarial attacks, where minor perturbations—such as occlusions from snow, dirt, or natural wear—can significantly impact classification accuracy. The Snowball Adversarial Attack exploits this vulnerability by gradually accumulating adversarial perturbations, simulating realistic environmental changes that degrade model performance over time. To evaluate the effectiveness of this attack, we utilize real Street View images~\cite{googleExploreStreet}, enabling the simulation of real-world adversarial scenarios without the need for physical modifications to street signs. This methodology ensures a practical yet ethical assessment of the attack’s impact, demonstrating the potential risks posed to autonomous driving systems in uncontrolled environments. Figure~\ref{fig:Test_Images} illustrates  the sample street sign images.

\subsection{Street Sign Masks}

To evaluate the effectiveness of the Snowball Attack, we generated custom binary masks that precisely control the placement of accumulated perturbations on traffic signs. Our approach ensures that each perturbation aligns naturally with boundaries of the street signs and looks like natural snow, to maintain visual plausibility.

The mask generation process begins by processing input images to identify mask regions. We first convert the image to grayscale and apply Gaussian blur~\cite{hummel1987deblurring} to reduce noise. Next, Canny edge detection~\cite{rong2014improved} highlights key contours, which are refined using morphological operations to create a continuous outline. The largest contour is selected to generate a binary mask, which defines the occlusion area on the traffic sign. This method allows for precise and adaptive placement of perturbations, ensuring that occlusions appear natural under varying angles and lighting conditions. Figure~\ref{fig:Output Masks} highlights the various  masks used in the experiments.

\subsection{Generated Snowballs}

To create a diverse and representative dataset of snowball images, we utilized three commonly available image generation tools: DALL-E3~\cite{dalle3}, Adobe Firefly~\cite{adobefirefly} and Midjourney~\cite{midjourney}. Each of these platforms employs distinct generative models and rendering techniques, resulting in variations in texture, lighting, shading, and overall realism. By generating three unique snowball images from each tool, we aimed to capture a range of artistic interpretations and stylistic nuances inherent to each model. Figure~\ref{fig:snowball_adversarial_image} shows the generated snowball images.

DALL-E3, developed by OpenAI, is known for its structured and text-conditioned image synthesis, often producing images with well-defined edges and fine-grained details. Adobe Firefly, with its emphasis on creative control and photorealism, introduces additional refinements, particularly in the handling of lighting and material textures. Midjourney, on the other hand, leverages its unique diffusion-based approach to create highly stylized and visually rich compositions. The resulting images from these tools provide a basis for evaluating the consistency, diversity, and fidelity of AI-generated snowball representations.

By leveraging multiple generation sources, we ensure that our dataset includes a variety of visual perspectives, helping us assess the strengths and limitations of each AI model in replicating realistic snow formations. These images serve as the foundation for further qualitative and quantitative analysis in subsequent sections.

\subsection{Search Algorithm for Snowball Placement}

The proposed search algorithm systematically identifies the optimal placement and orientation of the adversarial snowball overlay to maximize the likelihood of misclassification. Given a traffic sign image and its corresponding mask, the algorithm first determines the valid regions for perturbation by analyzing the non-zero pixels within the mask. The patch size is computed as a function of the available perturbable area, ensuring proportional scaling across different sign types. The algorithm then iterates over feasible placement positions, testing various patch rotations when applicable. For each candidate perturbation, an adversarial image is generated and evaluated using a pre-trained classifier. The confidence score of the incorrect prediction is recorded, and the perturbation configuration yielding the highest misclassification confidence is selected. This iterative search process ensures an efficient exploration of the perturbation space while optimizing adversarial effectiveness. Figures~\ref{fig:stop_snowball_adversarial_image} to~\ref{fig:right_snowball_adversarial_image} show the generated adversarial snowball images on each of the traffic signs.

\subsection{Optimized Search Algorithm}

The optimized search algorithm enhances the efficiency of adversarial perturbation generation by refining the search space iteratively. Initially, the algorithm determines valid perturbation regions using a binary mask and selects potential overlay positions. The snowball perturbation is applied across multiple patch sizes and orientations, with each configuration evaluated for its ability to induce misclassification. To improve search efficiency, the algorithm prioritizes high-confidence misclassifications and iteratively refines the mask by shrinking it around the most effective perturbation locations. This targeted reduction of the search space enables faster convergence towards an optimal adversarial configuration. Additionally, the use of structured patch scaling and selective angle testing reduces redundant computations while maintaining adversarial effectiveness. By iterating through progressively refined masks, the algorithm minimizes unnecessary evaluations and reduces the time to find best location for the snowball patch. 
