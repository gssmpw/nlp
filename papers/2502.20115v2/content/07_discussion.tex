
\section{Discussion}
\label{sec:discussion}

\subsection{Related work} 

While the Introduction~\ref{sec:intro} already discussed related work, we discuss next the connection to~\citet{shimizu2012joint} in more detail.
The fundamental difference is that we assume the disturbances have some shared information across views and can be Gaussian, whereas \citet{shimizu2012joint} estimates disturbances that are view-dependent and necessarily non-Gaussian.
Unlike \citet{shimizu2012joint}, we prove identifiability results requiring only Gaussianity.
Moreover, the estimation methods are different: \citet{shimizu2012joint} use an extension of DirectLiNGAM~\citep{shimizu2011directlingam} to multiple views, while we combine methods from Shared ICA \citet{richard2021sharedica} and the original ICA-based LiNGAM~\citep{shimizu2006linear}. 

A related very general framework, allowing for various kinds of interventions, was proposed by \citet{mooij2020joint}. On the other hand, a related ``multi-context" approach, where the disturbances are not necessarily related at all while the DAGs are, was proposed by \citet{sturma2023unpaired}.


\subsection{Future work}
Typically, optimizing over permutations has a worst-case complexity that is cubic in the number of variables, $O(p^3)$, but that plagues much of LiNGAM literature. We leave the design of more computationally efficient algorithms for future work. Likewise, methods such as DirectLiNGAM \citep{shimizu2011directlingam} or PairwiseLiNGAM \citep{Hyva13JMLR} might be useful and adaptable to the multi-view context.

One variant of the model that might be useful in a very low-data regime is the one in which the $\mB^i$ are the same over $i$. This should be almost trivial to implement on the algorithmic level, while the identifiability is already implied by our results under the same assumptions.
