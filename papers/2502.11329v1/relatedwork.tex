\section{Literature review}
In their study, ~\cite{puliga2021microsatellite} demonstrated the positive impacts of immunotherapy in treating patients with GI cancer who have MSI status. Their research emphasizes the importance of accurately classifying patients into MSI and MSS categories in order to identify those who could benefit from specific treatment approaches, particularly immunotherapy. In a thorough investigation by ~\cite{zhao2022identification}, various tasks related to gastric cancer, including classification and segmentation, were extensively studied. In their research, ~\cite{li2019early} employed fluorescence hyperspectral imaging to capture fluorescence spectral images. They integrated deep learning into spectral-spatial classification methods to effectively identify and extract combined ``spectral + spatial" features. These features were utilized to construct an early diagnosis model for GI cancer. The model achieved an overall accuracy of 96.5\% when classifying non-precancerous lesions, precancerous lesions, and gastric cancer groups.

~\cite{sai2022modified} introduced a modified ResNet model for classifying MSI vs. MSS in GI cancer. Their proposed model achieved a test accuracy of 89.81\%. In their study, ~\cite{khan2022transfer} utilized a transfer learning-based approach to construct a model for classifying MSI vs. MSS for cancers. The model was trained on histological images obtained from formalin-fixed paraffin-embedded (FFPE) samples. The results demonstrated an accuracy of 90.91\%. In their work, ~\cite {chilukoti2022privacy} described the threats related to medical applications and developed a DP-friendly model for Covid19 detection to nullify the threats posed to the non-private model.

Several important observations can be made based on the review of existing literature. First, classifying MSI vs. MSS in GI cancer is crucial in ensuring patients receive appropriate treatment. It is important to acknowledge that deep learning models are susceptible to various attacks. Interestingly, prior research in the context of GI cancer classification has often overlooked privacy-related concerns during the training of classifiers. Drawing inspiration from the medical applications of DP, we develop a model incorporating DP and leveraging a privacy-friendly pre-trained model known as NF-Net.