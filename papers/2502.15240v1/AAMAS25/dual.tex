\section{Dual-Based Algorithm}\label{sec:dual}
We now present a heuristic dual-based algorithm inspired by the dual formulation of our optimization problem P1 (Eq. \ref{eq:POne}). This algorithm pulls each arm in  round-robin fashion for the first $O(\sqrt{T})$ rounds  and solves the problem using a Lagrangian dual formulation with the appropriate  estimates of $A$ using pulls from the  first phase. 
% \piyushi{Sir, its not completely the UCB estimate of A. Can we write, UCB estimate of a function of A obtained from the Lagrangian?} 


We start  with formulating the dual of our Problem P1. %First note that the solution of P1 is the same as the following primal problem.
\begin{align}
    \textup{\textbf{Primal:}}& -\min_{\pi\in\Delta_m} \sum_{i=1}^n -\langle A_i, \pi\rangle \nonumber
    \\
    &\textup{s.t. }-(\langle A_i, \pi\rangle - C_i\Amax_i)\leq 0 \ \forall i\in[n].
\end{align}
We derive the Lagrangian dual with Lagrange parameters $\lambda_i|_{i=1}^n$ corresponding to the fairness constraints.
\begin{align}
    \textup{\textbf{Dual:}}\quad &-\max_{\lambda \in \mathbb{R}^n \geq 0}\ \min_{\pi\in \Delta_m} \sum_{i=1}^n-\langle A_i, \pi\rangle - \sum_{i=1}^n \lambda_i(\langle A_i, \pi\rangle - C_i\Amax_i) \nonumber\\
    =& -\max_{\lambda \in \mathbb{R}^n \geq 0}\ \min_{\pi\in \Delta_m} -\left\langle\sum_{i=1}^n(1 +\lambda_i)A_i, \pi\right\rangle + \sum_{i=1}^n \lambda_i C_i\Amax_i \nonumber\\
    =& -\max_{\lambda \in \mathbb{R}^n \geq 0} \ -\max_{j\in[m]}\left(\sum_{i=1}^n(1+\lambda_i)A_i\right)_j + C_i\langle\lambda, \Amax \rangle\nonumber\\
    =&-\max_{\lambda \in \mathbb{R}^n \geq 0} \ - \|\left(\textup{Diag}(1+\lambda)A\right)^\top \mathbf{1}_n\|_\infty + C_i\langle\lambda, \Amax \rangle \label{dualcp}
\end{align}
where $\textup{Diag}(\cdot)$ denotes the diagonal matrix formed by the entries $(1+\lambda_i)$.
Motivated by the simplification we obtain in the last step, our dual algorithm is designed to pick the arms based on the UCB estimate of $\|\left(\textup{Diag}(1+\lambda)A\right)^\top \mathbf{1}_n\|_\infty$ with $\lambda$ as the solution of Eq.~(\ref{dualcp}).

\begin{algorithm}
\caption{\textsc{Dual-Inspired Algorithm.}}
\label{alg-dual}
\begin{algorithmic}[1]
\STATE \textbf{Require:} $T, n, m , C, N_{j}^{0}=0 \ \forall j \in [m]$.
\STATE $t'=m\lceil \sqrt{T} \rceil, \ t=1$,  $\widehat{A}=\mathbf{0}_{m\times n}$.
\FOR{$t\leq t'$}
\STATE Pull arm $j'=t \textup{ mod } m + 1$.
\STATE $\forall i \in [n]$, observe reward $X_{i, j'}^{t}\sim \mathcal{D}(\mu_{i,j'})$.
\STATE $\forall i \in[n], \forall j \in[m]$, $\widehat{A}_{i,j} = 
    \begin{cases}
     \widehat{A}_{i,j} & \text{ if }  j\neq j'  \\
      \frac{(N_{j}^{t-1}) \widehat{A}_{i,j} + X_{i,j}^t}{N_{j}^{t}}  & \text{ if } j =j'. \\ 
      \end{cases}$
\STATE $N_{j'}^ t = N_{j'}^{t-1}+1.$
\ENDFOR
\STATE Compute $\mathcal{E}$ with entries $\epsilon_{i,j}^t=\sigma\sqrt{\frac{2\log{(8mnT)}}{N_{j}^{t}}}.$
\STATE Compute $\hat{\lambda}$ by solving the convex program in Eq. (\ref{dualcp}) with $\hat{A}$.
\FOR{$t\leq T$}
% \STATE Compute $\hat{\lambda}$ by solving the convex program in Eq. (\ref{dualcp}) with $\hat{A}$.
\STATE $j'\in \argmax \left( \left(\textup{Diag}(1+\hat{\lambda})\hat{A}\right)^\top \mathbf{1}_n + \mathcal{E}\right)$.
\STATE $\forall i \in [n]$, observe reward $X_{i, j'}^t\sim \mathcal{D}(\mu_{i,j'})$.
\STATE $N_{j'}^{t} = N_{j'}^{t-1}+1.$
\STATE  $\forall i \in[n], \forall j \in[m],$
$\widehat{A}_{i,j} = 
    \begin{cases}
     \widehat{A}_{i,j} & \text{ if }  j\neq j'  \\
      \frac{(N_{j}^{t-1}) \widehat{A}_{i,j} + X_{i,j}^t}{N_{j}^{t}}  & \text{ if } j =j'. \\ 
      \end{cases}$
\STATE \STATE Update entries of $\mathcal{E}$.
\ENDFOR

 \end{algorithmic}
\end{algorithm}