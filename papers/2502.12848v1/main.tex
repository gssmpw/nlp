\documentclass[conference]{IEEEtran}
\IEEEoverridecommandlockouts

\usepackage[hyphens]{url}
\usepackage{hyperref}
\usepackage[hyphenbreaks]{breakurl}

\usepackage{booktabs}
\usepackage{multirow}
\usepackage{amsmath,amssymb,amsfonts}
\usepackage{graphicx}
\usepackage{amsthm}
\usepackage{url}
\usepackage{listings}
\usepackage{mathtools}
\usepackage[matrix,arrow,curve]{xy}
\usepackage{enumitem}
\usepackage{extarrows}
\usepackage{comment}
\usepackage{tikzsymbols}
\usepackage[utf8]{inputenc}
\usepackage{csquotes}
\usepackage{pbalance}
\usepackage{nowidow}

\newcommand*{\CAMERAREADY}{}%

\ifdefined\COLORDIFF
  \newcommand{\diff}[1]{{\color{cbred} #1}}
\else
  \newcommand{\diff}[1]{{#1}}
\fi



\usepackage{tikz}
\usetikzlibrary{positioning,shapes,arrows,calc,fit,arrows.meta}

\tikzset{
  invisible/.style={opacity=0},
  visible on/.style={alt={#1{}{invisible}}},
  alt/.code args={<#1>#2#3}{%
    \alt<#1>{\pgfkeysalso{#2}}{\pgfkeysalso{#3}} %
  },
}


\usepackage[T1]{fontenc}
\usepackage{amsmath,amssymb}
\usepackage{url}




\newtheorem{example}{Example}

\newtheorem{conjecture}{Conjecture}

\newtheorem{theorem}{Theorem}

\newtheorem{lemma}{Lemma}

\theoremstyle{definition}
\newtheorem{definition}{Definition}

\theoremstyle{proposition}
\newtheorem{proposition}{Proposition}
\usepackage[numbers,compress]{natbib}

\usepackage[capitalise]{cleveref}



\usepackage{strandstyle}

\newcommand{\myparagraph}[1]{\smallskip\subsubsection*{#1}}
\usepackage{zi4}            %
\usepackage{xcolor}
\usepackage{listings}
\usepackage{lstautogobble}  %
\usepackage{xspace}
\definecolor{dkgreen}{rgb}{0,0.6,0}
\definecolor{ltblue}{rgb}{0,0.4,0.4}
\definecolor{dkviolet}{rgb}{0.3,0,0.5}

\lstdefinelanguage{Coq}{
    mathescape=false,
    texcl=false,
    morekeywords=[1]{Section, Module, End, Require, Import, Export,
        Variable, Variables, Parameter, Parameters, Axiom, Hypothesis,
        Hypotheses, Notation, Local, Tactic, Reserved, Scope, Open, Close,
        Bind, Delimit, Definition, Let, Ltac, Fixpoint, CoFixpoint, Add,
        Morphism, Relation, Implicit, Arguments, Unset, Contextual,
        Strict, Prenex, Implicits, Inductive, CoInductive, Record,
        Structure, Canonical, Coercion, Context, Class, Global, Instance,
        Program, Infix, Theorem, Lemma, Corollary, Coro, llary, Proposition, Fact,
        Remark, Example, Proof, Goal, Save, Qed, Defined, Hint, Resolve,
        Rewrite, View, Search, Show, Print, Printing, All, Eval, Check,
        Projections, inside, outside, Def},
    morekeywords=[2]{forall, exists, exists2, fun, fix, cofix, struct,
        match, with, end, as, in, return, let, if, is, then, else, for, of,
        nosimpl, when},
    morekeywords=[3]{Type,Prop, Set, true, false, option},
    morekeywords=[4]{pose, set, move, case, elim, apply, clear, hnf,
        intro, intros, generalize, rename, pattern, after, destruct,
        induction, using, refine, inversion, injection, rewrite, congr,
        unlock, compute, ring, field, fourier, replace, fold, unfold,
        change, cutrewrite, simpl, have, suff, wlog, suffices, without,
        loss, nat_norm, assert, cut, trivial, revert, bool_congr, nat_congr,
        symmetry, transitivity, auto, split, left, right, autorewrite},
    morekeywords=[5]{by, done, exact, reflexivity, tauto, romega, omega,
        assumption, solve, contradiction, discriminate},
    morekeywords=[6]{do, last, first, try, idtac, repeat},
    morecomment=[s]{(*}{*)},
    showstringspaces=false,
    morestring=[b]",
    morestring=[d]’,
    tabsize=2,
    extendedchars=true,
    inputencoding=utf8,
    sensitive=true,
    breaklines=true,
    basicstyle=\small,
    captionpos=b,
    columns=[l]flexible,
    identifierstyle={\ttfamily\color{black}},
    keywordstyle=[1]{\ttfamily\color{dkviolet}},
    keywordstyle=[2]{\ttfamily\color{dkgreen}},
    keywordstyle=[3]{\ttfamily\color{ltblue}},
    keywordstyle=[5]{\ttfamily\color{dkred}},
    stringstyle=\ttfamily,
    commentstyle={\ttfamily\color{dkgreen}},
    keepspaces,
    xleftmargin=2mm,
    literate=
    {≺}{{$\prec$}}1
    {Σ}{{$\Sigma$}}1
    {ℓ}{{$\ell$}}1
    {Π}{{$\Pi$}}1
    {π}{{$\pi$}}1
    {⊖}{{$\ominus$}}1
    {⊕}{{$\oplus$}}1
    {𝔸}{{$\mathbb{A}$}}1
    {⟨}{{$\langle$}}1
    {⟩}{{$\rangle$}}1
    {⋅}{{$\cdot$}}1
    {ϕ}{{$\phi$}}1
    {ℜ}{{$\mathcal{R}$}}1
    {⊢}{{$\vdash$}}1
    {∈}{{$\in$}}1
    {⊏}{{$\sqsubset$}}1
    {τ}{{$\tau$}}1
    {'}{{$^\prime$}}1
    {forall}{{$\forall$}}1
    {exists}{{$\exists$}}1
    {<-}{{$\leftarrow$}}1
    {=>+}{{$\Rightarrow^+$}}1
    {==}{{\code{==}}}1
    {->}{{$\rightarrow$}}1
    {<->}{{$\leftrightarrow$}}1
    {\#}{{\texttt{\#}}}1
    {\/\\}{{$\wedge$}}1
    {\\\/}{{$\vee$}}1
    {<>}{{$\neq$}}1
    {~}{{$\lnot$}}1
}[keywords,comments,strings]
\lstset{language=Coq}


\newcommand{\arxivonly}[1]{{}}

  \makeatletter
  \def\ps@IEEEtitlepagestyle{
    \def\@oddfoot{\mycopyrightnotice}
    \def\@evenfoot{}
  }

  \def\mycopyrightnotice{
    {\footnotesize
    \begin{minipage}{\textwidth}
    To appear at IEEE CSF'25, June 16-20, 2025, Santa Cruz, CA, USA.
    \copyright~2025 IEEE.
    Personal use of this material is permitted.
    Permission from IEEE must be obtained for all other uses, in any current or future media, including reprinting/republishing this material for advertising or promotional purposes, creating new collective works, for resale or redistribution to servers or lists, or reuse of any copyrighted component of this work in other works.
    The definitive Version of Record is going to appear in the proceedings of the
    38th IEEE Computer Security Foundations Symposium (IEEE CSF'25), June 16-20, 2025, Santa Cruz, CA, USA.
    \end{minipage}
    }
  }


\begin{document}

\newcommand{\easystrands}{\texttt{StrandsRocq}}

\title{
  Strands Rocq:\\  %
  \huge Why is a Security Protocol Correct, Mechanically?
}

\ifdefined\CAMERAREADY
  \author{
      \IEEEauthorblockN{Matteo Busi}
      \IEEEauthorblockA{
      \textit{DAIS, Ca' Foscari University}\\
      Venice, Italy \\
      matteo.busi@unive.it}
  \and
      \IEEEauthorblockN{Riccardo Focardi}
      \IEEEauthorblockA{
      \textit{DAIS, Ca' Foscari University}\\
      Venice, Italy \\
      focardi@unive.it}
  \and
      \IEEEauthorblockN{Flaminia L. Luccio}
      \IEEEauthorblockA{
      \textit{DAIS, Ca' Foscari University}\\
      Venice, Italy \\
      luccio@unive.it}
  }
\else
  \author{Anonymous author(s)}
\fi

\maketitle

\newcommand{\enc}[2]{{\ensuremath {\langle #1 \rangle _{#2}}}}

\begin{abstract}
Strand spaces are a formal framework for symbolic protocol verification that allows for pen-and-paper proofs of security \cite{FHG98}. While extremely insightful, pen-and-paper proofs are error-prone, and it is hard to gain confidence on their correctness. To overcome this problem, we developed \easystrands, a full mechanization of the strand spaces in Coq (soon to be renamed Rocq). The mechanization was designed to be faithful to the original pen-and-paper development, and it was engineered to be modular and extensible. \easystrands{} incorporates new original proof techniques, a novel notion of maximal penetrator that enables protocol compositionality, and a set of Coq tactics tailored to the domain, facilitating proof automation and reuse, and simplifying the work of protocol analysts. To demonstrate the versatility of our approach, we modelled and analyzed a family of authentication protocols, drawing inspiration from ISO/IEC 9798-2 two-pass authentication, the classical Needham-Schroeder-Lowe protocol, as well as a recently-proposed static analysis for a key management API. The analyses in \easystrands{} confirmed the high degree of proof reuse, and enabled us to distill the minimal requirements for protocol security. Through mechanization, we identified and addressed several issues in the original proofs and we were able to significantly improve the precision of the static analysis for the key management API. Moreover, we were able to leverage the novel notion of maximal penetrator to provide a compositional proof of security for two simple authentication protocols.
\end{abstract}


\begin{IEEEkeywords}
Formal Methods, Strand Spaces, security protocols, Coq.
\end{IEEEkeywords}


\section{Introduction}
\section{Introduction}
\label{sec:introduction}
The business processes of organizations are experiencing ever-increasing complexity due to the large amount of data, high number of users, and high-tech devices involved \cite{martin2021pmopportunitieschallenges, beerepoot2023biggestbpmproblems}. This complexity may cause business processes to deviate from normal control flow due to unforeseen and disruptive anomalies \cite{adams2023proceddsriftdetection}. These control-flow anomalies manifest as unknown, skipped, and wrongly-ordered activities in the traces of event logs monitored from the execution of business processes \cite{ko2023adsystematicreview}. For the sake of clarity, let us consider an illustrative example of such anomalies. Figure \ref{FP_ANOMALIES} shows a so-called event log footprint, which captures the control flow relations of four activities of a hypothetical event log. In particular, this footprint captures the control-flow relations between activities \texttt{a}, \texttt{b}, \texttt{c} and \texttt{d}. These are the causal ($\rightarrow$) relation, concurrent ($\parallel$) relation, and other ($\#$) relations such as exclusivity or non-local dependency \cite{aalst2022pmhandbook}. In addition, on the right are six traces, of which five exhibit skipped, wrongly-ordered and unknown control-flow anomalies. For example, $\langle$\texttt{a b d}$\rangle$ has a skipped activity, which is \texttt{c}. Because of this skipped activity, the control-flow relation \texttt{b}$\,\#\,$\texttt{d} is violated, since \texttt{d} directly follows \texttt{b} in the anomalous trace.
\begin{figure}[!t]
\centering
\includegraphics[width=0.9\columnwidth]{images/FP_ANOMALIES.png}
\caption{An example event log footprint with six traces, of which five exhibit control-flow anomalies.}
\label{FP_ANOMALIES}
\end{figure}

\subsection{Control-flow anomaly detection}
Control-flow anomaly detection techniques aim to characterize the normal control flow from event logs and verify whether these deviations occur in new event logs \cite{ko2023adsystematicreview}. To develop control-flow anomaly detection techniques, \revision{process mining} has seen widespread adoption owing to process discovery and \revision{conformance checking}. On the one hand, process discovery is a set of algorithms that encode control-flow relations as a set of model elements and constraints according to a given modeling formalism \cite{aalst2022pmhandbook}; hereafter, we refer to the Petri net, a widespread modeling formalism. On the other hand, \revision{conformance checking} is an explainable set of algorithms that allows linking any deviations with the reference Petri net and providing the fitness measure, namely a measure of how much the Petri net fits the new event log \cite{aalst2022pmhandbook}. Many control-flow anomaly detection techniques based on \revision{conformance checking} (hereafter, \revision{conformance checking}-based techniques) use the fitness measure to determine whether an event log is anomalous \cite{bezerra2009pmad, bezerra2013adlogspais, myers2018icsadpm, pecchia2020applicationfailuresanalysispm}. 

The scientific literature also includes many \revision{conformance checking}-independent techniques for control-flow anomaly detection that combine specific types of trace encodings with machine/deep learning \cite{ko2023adsystematicreview, tavares2023pmtraceencoding}. Whereas these techniques are very effective, their explainability is challenging due to both the type of trace encoding employed and the machine/deep learning model used \cite{rawal2022trustworthyaiadvances,li2023explainablead}. Hence, in the following, we focus on the shortcomings of \revision{conformance checking}-based techniques to investigate whether it is possible to support the development of competitive control-flow anomaly detection techniques while maintaining the explainable nature of \revision{conformance checking}.
\begin{figure}[!t]
\centering
\includegraphics[width=\columnwidth]{images/HIGH_LEVEL_VIEW.png}
\caption{A high-level view of the proposed framework for combining \revision{process mining}-based feature extraction with dimensionality reduction for control-flow anomaly detection.}
\label{HIGH_LEVEL_VIEW}
\end{figure}

\subsection{Shortcomings of \revision{conformance checking}-based techniques}
Unfortunately, the detection effectiveness of \revision{conformance checking}-based techniques is affected by noisy data and low-quality Petri nets, which may be due to human errors in the modeling process or representational bias of process discovery algorithms \cite{bezerra2013adlogspais, pecchia2020applicationfailuresanalysispm, aalst2016pm}. Specifically, on the one hand, noisy data may introduce infrequent and deceptive control-flow relations that may result in inconsistent fitness measures, whereas, on the other hand, checking event logs against a low-quality Petri net could lead to an unreliable distribution of fitness measures. Nonetheless, such Petri nets can still be used as references to obtain insightful information for \revision{process mining}-based feature extraction, supporting the development of competitive and explainable \revision{conformance checking}-based techniques for control-flow anomaly detection despite the problems above. For example, a few works outline that token-based \revision{conformance checking} can be used for \revision{process mining}-based feature extraction to build tabular data and develop effective \revision{conformance checking}-based techniques for control-flow anomaly detection \cite{singh2022lapmsh, debenedictis2023dtadiiot}. However, to the best of our knowledge, the scientific literature lacks a structured proposal for \revision{process mining}-based feature extraction using the state-of-the-art \revision{conformance checking} variant, namely alignment-based \revision{conformance checking}.

\subsection{Contributions}
We propose a novel \revision{process mining}-based feature extraction approach with alignment-based \revision{conformance checking}. This variant aligns the deviating control flow with a reference Petri net; the resulting alignment can be inspected to extract additional statistics such as the number of times a given activity caused mismatches \cite{aalst2022pmhandbook}. We integrate this approach into a flexible and explainable framework for developing techniques for control-flow anomaly detection. The framework combines \revision{process mining}-based feature extraction and dimensionality reduction to handle high-dimensional feature sets, achieve detection effectiveness, and support explainability. Notably, in addition to our proposed \revision{process mining}-based feature extraction approach, the framework allows employing other approaches, enabling a fair comparison of multiple \revision{conformance checking}-based and \revision{conformance checking}-independent techniques for control-flow anomaly detection. Figure \ref{HIGH_LEVEL_VIEW} shows a high-level view of the framework. Business processes are monitored, and event logs obtained from the database of information systems. Subsequently, \revision{process mining}-based feature extraction is applied to these event logs and tabular data input to dimensionality reduction to identify control-flow anomalies. We apply several \revision{conformance checking}-based and \revision{conformance checking}-independent framework techniques to publicly available datasets, simulated data of a case study from railways, and real-world data of a case study from healthcare. We show that the framework techniques implementing our approach outperform the baseline \revision{conformance checking}-based techniques while maintaining the explainable nature of \revision{conformance checking}.

In summary, the contributions of this paper are as follows.
\begin{itemize}
    \item{
        A novel \revision{process mining}-based feature extraction approach to support the development of competitive and explainable \revision{conformance checking}-based techniques for control-flow anomaly detection.
    }
    \item{
        A flexible and explainable framework for developing techniques for control-flow anomaly detection using \revision{process mining}-based feature extraction and dimensionality reduction.
    }
    \item{
        Application to synthetic and real-world datasets of several \revision{conformance checking}-based and \revision{conformance checking}-independent framework techniques, evaluating their detection effectiveness and explainability.
    }
\end{itemize}

The rest of the paper is organized as follows.
\begin{itemize}
    \item Section \ref{sec:related_work} reviews the existing techniques for control-flow anomaly detection, categorizing them into \revision{conformance checking}-based and \revision{conformance checking}-independent techniques.
    \item Section \ref{sec:abccfe} provides the preliminaries of \revision{process mining} to establish the notation used throughout the paper, and delves into the details of the proposed \revision{process mining}-based feature extraction approach with alignment-based \revision{conformance checking}.
    \item Section \ref{sec:framework} describes the framework for developing \revision{conformance checking}-based and \revision{conformance checking}-independent techniques for control-flow anomaly detection that combine \revision{process mining}-based feature extraction and dimensionality reduction.
    \item Section \ref{sec:evaluation} presents the experiments conducted with multiple framework and baseline techniques using data from publicly available datasets and case studies.
    \item Section \ref{sec:conclusions} draws the conclusions and presents future work.
\end{itemize}

\section{Background on Strand Spaces}\label{sec:background-new}
  \section{Background}\label{sec:backgrnd}

\subsection{Cold Start Latency and Mitigation Techniques}

Traditional FaaS platforms mitigate cold starts through snapshotting, lightweight virtualization, and warm-state management. Snapshot-based methods like \textbf{REAP} and \textbf{Catalyzer} reduce initialization time by preloading or restoring container states but require significant memory and I/O resources, limiting scalability~\cite{dong_catalyzer_2020, ustiugov_benchmarking_2021}. Lightweight virtualization solutions, such as \textbf{Firecracker} microVMs, achieve fast startup times with strong isolation but depend on robust infrastructure, making them less adaptable to fluctuating workloads~\cite{agache_firecracker_2020}. Warm-state management techniques like \textbf{Faa\$T}~\cite{romero_faa_2021} and \textbf{Kraken}~\cite{vivek_kraken_2021} keep frequently invoked containers ready, balancing readiness and cost efficiency under predictable workloads but incurring overhead when demand is erratic~\cite{romero_faa_2021, vivek_kraken_2021}. While these methods perform well in resource-rich cloud environments, their resource intensity challenges applicability in edge settings.

\subsubsection{Edge FaaS Perspective}

In edge environments, cold start mitigation emphasizes lightweight designs, resource sharing, and hybrid task distribution. Lightweight execution environments like unikernels~\cite{edward_sock_2018} and \textbf{Firecracker}~\cite{agache_firecracker_2020}, as used by \textbf{TinyFaaS}~\cite{pfandzelter_tinyfaas_2020}, minimize resource usage and initialization delays but require careful orchestration to avoid resource contention. Function co-location, demonstrated by \textbf{Photons}~\cite{v_dukic_photons_2020}, reduces redundant initializations by sharing runtime resources among related functions, though this complicates isolation in multi-tenant setups~\cite{v_dukic_photons_2020}. Hybrid offloading frameworks like \textbf{GeoFaaS}~\cite{malekabbasi_geofaas_2024} balance edge-cloud workloads by offloading latency-tolerant tasks to the cloud and reserving edge resources for real-time operations, requiring reliable connectivity and efficient task management. These edge-specific strategies address cold starts effectively but introduce challenges in scalability and orchestration.

\subsection{Predictive Scaling and Caching Techniques}

Efficient resource allocation is vital for maintaining low latency and high availability in serverless platforms. Predictive scaling and caching techniques dynamically provision resources and reduce cold start latency by leveraging workload prediction and state retention.
Traditional FaaS platforms use predictive scaling and caching to optimize resources, employing techniques (OFC, FaasCache) to reduce cold starts. However, these methods rely on centralized orchestration and workload predictability, limiting their effectiveness in dynamic, resource-constrained edge environments.



\subsubsection{Edge FaaS Perspective}

Edge FaaS platforms adapt predictive scaling and caching techniques to constrain resources and heterogeneous environments. \textbf{EDGE-Cache}~\cite{kim_delay-aware_2022} uses traffic profiling to selectively retain high-priority functions, reducing memory overhead while maintaining readiness for frequent requests. Hybrid frameworks like \textbf{GeoFaaS}~\cite{malekabbasi_geofaas_2024} implement distributed caching to balance resources between edge and cloud nodes, enabling low-latency processing for critical tasks while offloading less critical workloads. Machine learning methods, such as clustering-based workload predictors~\cite{gao_machine_2020} and GRU-based models~\cite{guo_applying_2018}, enhance resource provisioning in edge systems by efficiently forecasting workload spikes. These innovations effectively address cold start challenges in edge environments, though their dependency on accurate predictions and robust orchestration poses scalability challenges.

\subsection{Decentralized Orchestration, Function Placement, and Scheduling}

Efficient orchestration in serverless platforms involves workload distribution, resource optimization, and performance assurance. While traditional FaaS platforms rely on centralized control, edge environments require decentralized and adaptive strategies to address unique challenges such as resource constraints and heterogeneous hardware.



\subsubsection{Edge FaaS Perspective}

Edge FaaS platforms adopt decentralized and adaptive orchestration frameworks to meet the demands of resource-constrained environments. Systems like \textbf{Wukong} distribute scheduling across edge nodes, enhancing data locality and scalability while reducing network latency. Lightweight frameworks such as \textbf{OpenWhisk Lite}~\cite{kravchenko_kpavelopenwhisk-light_2024} optimize resource allocation by decentralizing scheduling policies, minimizing cold starts and latency in edge setups~\cite{benjamin_wukong_2020}. Hybrid solutions like \textbf{OpenFaaS}~\cite{noauthor_openfaasfaas_2024} and \textbf{EdgeMatrix}~\cite{shen_edgematrix_2023} combine edge-cloud orchestration to balance resource utilization, retaining latency-sensitive functions at the edge while offloading non-critical workloads to the cloud. While these approaches improve flexibility, they face challenges in maintaining coordination and ensuring consistent performance across distributed nodes.



\section{Mechanizing Strand Spaces: \easystrands}\label{sec:running}
  
In this section,
we present \easystrands, a complete mechanization of strand spaces in Coq.
We briefly introduce the structure and engineering of the library (\cref{sec:library}). Then, we demonstrate the process of specifying and proving the correctness of the protocol presented in \cref{ex:simpleprotocol} through simple steps, illustrating the specification phase (\cref{sec:specification}), the underlying proof technique, its mechanization, and our novel proof automation techniques (\cref{sec:proofprotocol}) that allow for compact and reusable proofs (\cref{sec:reusing}).
Finally in \cref{sec:newproofs} we present a new proof technique that simplifies the one presented by Fabrega et al. \cite{FHG98}.
During this journey, we start with the basic authentication protocol from \cref{ex:simpleprotocol}, inspired by the ISO/IEC 9798-2 two-pass authentication protocol~\cite{ISO97982}, and successfully analyze five different variants, uncovering the minimal security assumptions for each of them.

\ifdefined\CAMERAREADY
  \myparagraph{Note for the readers} The complete mechanization and proofs are available online~\cite{strandsrocqcode}.
\else
  \myparagraph{Note for the reviewers} The complete mechanization and proofs are included as supplementary material with the submission.
\fi

\subsection{The \easystrands{} library}
\label{sec:library}
We organized the library into modules, separating the theory of strands based on abstract domains, as in the original paper (folder \lstinline|Common|), from an implementation that we believe is more convenient for verifying protocols.
Implementing the abstract domains is an important sanity check to remove all axioms and assumptions, ensuring that such assumptions are realistic (folder \lstinline|Instances|).
For example, concrete terms are part of \lstinline|Instances|, which makes the library very flexible if one wishes to model new cryptographic primitives: the entire \lstinline|Common| section remains unchanged, and it is only necessary to instantiate a specific \lstinline|Module Type|.
Unlike the abstract strand definition from~\cref{sec:background-new}, strands are instantiated here as \lstinline|Σ := nat * list sT|.
The natural number serves as a strand identifier and \lstinline|list sT| is a list of signed terms denoting the trace associated with the strand.
This choice is particularly convenient for protocol specification as it allows for specifying strands and their traces in a single place.
In the original paper, traces are bound to strands through a separate function \lstinline|tr|.
In our implementation we just have that \lstinline|tr s| is defined as \lstinline|snd s|, i.e., the second field of the strand instantiation.

\subsection{Modelling Protocols}
\label{sec:specification}

We define the roles in the protocol by inductively listing all the possible strands they can undertake.
This might seem overly intricate since, in most cases, honest principals follow a single execution trace that is quantified over parameters and payload values.
Nevertheless, in general, a principal could engage in more than one trace.
For instance, a penetrator may carry out various potential traces (\cref{sec:background-new}).
Additionally, when modeling key management APIs (\cref{subsec:KMP}), a single principal/device can implement various functionalities, each represented by a distinct trace.

Starting now, we directly present the notation employed in \easystrands, which deviates slightly from the mathematical notation used so far.
We use \lstinline{Na} to represent the nonce $N_a$, \lstinline{SK A B} to denote the key $\mathit{\SK{AB}}$, \lstinline{⟨ M ⟩_(K)} to indicate $\enc{M}{K}$, and \lstinline{⊕}, \lstinline{⊖} to respectively denote $+$ and $-$.
Since the type of \lstinline{A}, \lstinline{B} and \lstinline{Na} is \T, representing atomic terms, we respectively write \lstinline{$A}, \lstinline{$B} and \lstinline{$Na} to represent their values as general terms of type \terms.
For the protocol of \cref{ex:simpleprotocol},
the initiator strands are defined as follows:
\begin{lstlisting}
Inductive SA_initiator_strand (A B Na : T) : Σ -> Prop :=
  | SAS_Init : forall i,
      SA_initiator_strand A B Na
        (i, [ ⊕ $A ⋅ $B ⋅ $Na; ⊖ ⟨ $Na ⋅ $A ⟩_(SK A B) ]).
\end{lstlisting}
Dually, the responder strands have swapped inputs and outputs:
\begin{lstlisting}
Inductive SA_responder_strand (A B Na : T) : Σ -> Prop :=
  | SAS_Resp : forall i,
      SA_responder_strand A B Na
        (i, [ ⊖ $A ⋅ $B ⋅ $Na; ⊕ ⟨ $Na ⋅ $A ⟩_(SK A B) ]).
\end{lstlisting}
To analyze this protocol we will restrict ourselves to strands of three types: \lstinline{penetrator_strand} (defined along \cref{sec:background-new}), \lstinline{SA_initiator_strand}, or \lstinline{SA_responder_strand}:
\begin{lstlisting}
Inductive SA_StrandSpace (K__P : K -> Prop) : Σ -> Prop :=
  | SASS_Pen  : forall s,
    penetrator_strand K__P s -> SA_StrandSpace K__P s
  | SASS_Init : forall (A B Na : T) s,
    SA_initiator_strand A B Na s -> SA_StrandSpace K__P s
  | SASS_Resp : forall (A B Na : T) s,
    SA_responder_strand A B Na s -> SA_StrandSpace K__P s
\end{lstlisting}
where \lstinline|K__P| encodes the knowledge of the penetrator at the beginning of the execution.
For our purposes the following minimal definition suffices:
\begin{lstlisting}
Definition K__P_AB (A B : T) (k : K) := k <> SK A B.
\end{lstlisting}
Intuitively, we assume that the only key the penetrator should not know is the actual key used by the two honest parties.





\subsection{Proof Automation}
\label{sec:proofprotocol}

We have developed a Coq library and some tactics to efficiently implement case analysis over strands, searching for a minimal element over a given strand.
We illustrate their usage below.
From now on we assume to have two honest parties \lstinline|A| and \lstinline|B|, a nonce \lstinline|Na| and a bundle \lstinline|C| whose nodes belong to the protocol strands \lstinline|SA_StrandSpace (K__P_AB A B)| in which the attacker does not know the key \lstinline|SK A B|. Since we want to prove authentication for the initiator, we assume that \lstinline|C| contains an initiator strand \lstinline|s| with the appropriate parameters, i.e., \lstinline|SA_initiator_strand A B Na s|.
All of these variables and hypotheses are specified locally using Coq \lstinline|Variable| and \lstinline|Hypothesis| commands and make propositions and lemmas more succinct and readable.

We consider \emph{non-injective agreement} requiring that, under the above assumptions,
there exists a responder strand \lstinline{s'}, and the initiator and responder traces agree on parameters
\lstinline{A}, \lstinline{B} and \lstinline{Na}.
Formally,
\begin{lstlisting}
Proposition noninjective_agreement :
  exists s' : Σ,
    SA_responder_strand A B Na s' /\ is_strand_of s' C.
\end{lstlisting}





\noindent
As illustrated in \cref{ex:proofs},
the proof in the strand spaces model revolves around showing that only the responder, with parameters \lstinline{A}, \lstinline{B} and \lstinline{Na}, can generate the expected ciphertext \lstinline{c = (⟨ $Na ⋅ $A ⟩_(SK A B))}.
The proof
is based on lemma \texttt{\small exists\_minimal\_bundle} (see \lstinline{Common/Bundles.v}) stating that each nonempty subset of nodes has a minimal w.r.t.\ the $\preceq_C$ relation (\cref{sec:background-new}).

The proof inspects all possible kinds of strands for \lstinline{s}: penetrator, initiator and responder.
Doing this in Coq is tedious and requires repetitive proofs even for cases that are deemed as trivial in pen-and-paper proofs.
For this reason, \easystrands{} provides a characterization of the minimal element of set of nodes in terms of a logical proposition covering all the possible cases.
For example, for the first strand of the penetrator, which is the output of an atomic term \lstinline{t} written \lstinline{[⊕ $t]} we obtain the
proposition \lstinline{False \/ c = $t /\ True /\ index m = 0}
which is false since \lstinline{c} is a ciphertext and it cannot be that \lstinline{c = $t}.
Other cases are more complex, e.g., for pair generation \lstinline{[⊖ g; ⊖ h; ⊕ g ⋅ h]} we get
\begin{lstlisting}
((False \/ subterm c g /\ False /\ index m = 0) \/
~ subterm c g /\ subterm c h /\ False /\ index m = 1) \/
~ subterm c g /\ ~ subterm c h /\
(c = g ⋅ h \/ subterm c g \/ subterm c h) /\
True /\ index m = 2
\end{lstlisting}
that is less trivial to analyze manually.
Therefore, we have implemented a tactic called \lstinline{simplify_prop}, which recursively simplifies propositions, leveraging the decidability of underlying statements.
It also attempts to automatically prove straightforward
(in)equalities, such as \lstinline{c <> $t} in the first case of the penetrator.


When applied to the penetrator case, the \lstinline{simplify_prop} tactic eliminates seven out of eighth cases, leaving only the interesting one, i.e., the encryption case with trace
\begin{lstlisting}
[⊖ #(SK A B); ⊖ $Na ⋅ $A; ⊕ (⟨ $Na ⋅ $A ⟩_(SK A B))]
\end{lstlisting}
Intuitively, this refers to the case where the penetrator generates the ciphertext \lstinline{c}, which is used by \lstinline{A} to confirm the identity of \lstinline{B}.
We eliminate this case by exploiting the fact that the penetrator can never learn a secure symmetric key.
This can be proved using a general property regarding the penetrator,
which asserts that the key read in the first node of a penetrator's encryption strand, in this case \lstinline{SK A B}, cannot be equal to a key that is not initially known by the penetrator and does not originate on a honest participant strand.
The fact that \lstinline{SK A B} is not initially known by the attacker is a direct consequence of the definition of \lstinline{K__P_AB A B} as \lstinline|k <> SK A B|. Additionally, the fact that \lstinline{SK A B} is not generated by the honest participants is demonstrated through a simple lemma, which can be proved using the same proof automation technique in just 8 lines of Coq.
So, we conclude that it must be \lstinline{SK A B <> SK A B}, leading to a contradiction.

The initiator case is automatically resolved by the \lstinline{simplify_prop} tactic, while the responder case leaves us with two subcases, depending on whether \lstinline{A} and \lstinline{B} are equal or not.
Both cases are resolved easily, as they yield a valid binding for the protocol parameters.
Interestingly, thanks to our proof automation techniques, the whole proof of \lstinline|noninjective_agreement| amounts to about 60 lines, as is greatly reusable as we will se next.

We also prove that each responder session corresponds to a different initiator session, i.e., that authentication is \emph{injective} and cannot be reused in a replay attack.
\begin{lstlisting}
Proposition injectivity :
  uniquely_originates $Na ->
    forall U U' s',
      SA_initiator_traces U U' Na (tr s') -> s' = s.
\end{lstlisting}
Notice that this property only holds if \lstinline{Na} is freshly generated which, in the strand spaces model, is captured by the \lstinline{uniquely_originates} definition stating that \lstinline{Na} originates, i.e., appears for the first time, in a unique node in a given bundle.
Injective agreement follows as a corollary from \lstinline{noninjective_agreement} and \lstinline{injectivity} (see \lstinline{injective_agreement} in \lstinline{Examples/simple_auth/SimpleAuth.v}).
\subsection{Proof Reuse}\label{sec:reusing}
An important feature of protocol analysis tools is the ability to \emph{play} with protocol specifications by quickly exploring various protocol variants.
This process is useful and insightful, as it allows us to observe how modifying the protocol affects its security.
We have incorporated this feature into \easystrands{} through proof automation via Coq tactics that perform case analysis, and eliminate the easy cases, as illustrated in the previous section.
Even though this does not guarantee that proofs can be reused when a specification is modified, in practice, we have observed that it is often the case.
Below, we provide examples supporting this fact.
Moreover, we point out that the proofs for the protocol in \cref{sec:proofprotocol} were mostly reused for the proofs of the NSL protocol, which is entirely different and relies on an asymmetric key cryptosystem (\cref{subsec:NSL}).

\myparagraph{Replacing $A$ with $B$ in the ciphertext}
The role of $A$ in the second protocol message is crucial for the security of the protocol, as it clarifies the direction of the message.
This is attributed to our consideration of the symmetric key $\SK{AB}$ as \emph{bidirectional}, meaning it remains the same whether the protocol is run by $A$ with $B$ or by $B$ with $A$.
Without an identifier in the ciphertext, the protocol would be vulnerable to what is commonly known as a \emph{reflection attack}, which we will discuss in the next section.
Here, we demonstrate that using either $A$ or $B$ in the ciphertext achieves the same result, as both identifiers disambiguate the protocol's direction.
To this aim, we consider a variant where $B$ replaces $A$ in the second message:
\vspace*{-0.2cm}
\begin{align*}
  A \rightarrow B & : A \cdot B \cdot N_a \\
  B \rightarrow A & : \enc{N_a \cdot B}{ \SK{AB}}
\end{align*}
Interestingly, when we make this modification, the security proof of the original protocol remains valid for this variant: we just need to change the ciphertext \lstinline{c} from \lstinline{(⟨ $Na ⋅ $A ⟩_(SK A B))} to \lstinline{(⟨ $Na ⋅ $B ⟩_(SK A B))} and the name of one hypothesis in a single rewrite statement.
This can be attributed to our characterization of the minimal element of the set of nodes using a logical proposition that covers all possible cases, along with the utilization of the \lstinline{simplify_prop} tactic in our proof automation.
This tactic automatically resolves most cases, even if they differ for some terms.
The example can be found in \lstinline{Examples/simple_auth/SimpleAuthWithB.v}.

\myparagraph{A flawed version of the protocol}
If we remove both $A$ and $B$ identifiers from the ciphertext the protocol is subject to a well-known reflection attack.
\begin{align*}
  A \rightarrow B & : A \cdot B \cdot N_a \\
  B \rightarrow A & : \enc{N_a }{ \SK{AB}}
\end{align*}
In this case we can copy-paste the proof of the original protocol to check where and why it fails.
The problem arises in the responder case, which has the goal
\begin{lstlisting}
  SA_responder_strand A B Na [⊖ ($B ⋅ $A) ⋅ $Na; ⊕ c]
\end{lstlisting}
but in the hypotheses, we have
\begin{lstlisting}
  SA_responder_strand B A Na [⊖ ($B ⋅ $A) ⋅ $Na; ⊕ c]
\end{lstlisting}
with \lstinline{A} and \lstinline{B} swapped, indicating a (known) reflection attack where \lstinline{c} is generated by \lstinline{A} itself acting as the responder.
The proof can only be closed when \lstinline{A = B}.
In this particular case, \lstinline{A} is persuaded to communicate with itself, which holds true even if the attacker reflects messages between two distinct sessions.
This example can be found in \lstinline{Examples/simple_auth/SimpleAuthFlawed.v}.

\myparagraph{Relaxing the Term Typing}
A common strategy for aiding automated verification involves constraining term types. In our current example, for example, we assume that \Na belongs to the set \T of atomic terms. A notable advantage of strand spaces lies in the insightful nature of their proofs, allowing the addition of assumptions only when necessary. Consequently, it becomes feasible to establish minimal assumptions for protocol security. This, coupled with our proof automation enabling the reuse of proofs, facilitates experimentation with type relaxation over terms to identify missing assumptions when needed. We conducted such an experiment by relaxing the typing, considering \Na as a general term belonging to \terms, not necessarily atomic,
The first lemma that cannot be proven is the one stating that \lstinline{(SKA A B)} never originates on a honest participant strand. In other words, we cannot prove that honest participants do not leak the symmetric key.
In fact, it might be the case that \Nap, for a given initiator, contains \lstinline{#(SKA A B)} as a subterm.
Therefore, the first restriction we need is:
\begin{lstlisting}
forall U U', ~ #(SK U U') ⊏ Na'
\end{lstlisting}
Intuitively, we impose the requirement that a nonce does not covertly transport the secure key  \lstinline{(SKA A B)} as a subterm. Should this occur, the initial message of the initiator would originate such a key, potentially exposing it to the penetrator.

The second point where the proof for the original protocol fails is in the initiator case of the \lstinline{noninjective_agreement} proposition. At this stage of the proof, we aim to eliminate the possibility that an initiator with parameters \Ap, \Bp, \Nap originates the ciphertext \lstinline{⟨ Na ⋅ A ⟩_(SKA A B)}. Once again, this scenario could arise if this ciphertext is a subterm of \Ap, \Bp, or \Nap.
To address this, we require the following:
\begin{lstlisting}
forall N U U', ~ (⟨ N ⋅ $U ⟩_(SK U U')) ⊏ Na'
\end{lstlisting}
We conclude that the protocol remains secure even when nonces are general terms, as long as they do not covertly transport the secure key and the corresponding ciphertext, the two fundamental ingredients for the security of the protocol.
These conditions are included in the specification of the strands for honest participants.
For example for the initiator (and similarly for the responder):
\begin{lstlisting}
Inductive SA_initiator_strand (A B : T) (Na : 𝔸) :
  Σ -> Prop :=
  | SAS_Init : forall i,
    (forall U U', ~ #(SK U U') ⊏ Na) ->
    (forall N U U', ~ (⟨ N ⋅ $U ⟩_(SK U U')) ⊏ Na) ->
    SA_initiator_strand A B Na
      (i, [ ⊕ $A ⋅ $B ⋅ Na; ⊖ ⟨ Na ⋅ $A ⟩_(SK A B) ]).
\end{lstlisting}

The example can be found in \lstinline{Examples/simple_auth/SimpleAuthUntyped.v}.

\subsection{A New Proof Technique}\label{sec:newproofs}
All proofs are based on the minimality lemma that we described in \cref{sec:background-new} and \cref{ex:proofs}.
However, it is up to the analyst to specify the specific set whose minimal elements exhibit witnesses for certain strands, such as in agreement properties, or whose emptiness proves a particular property, as in secrecy proofs that we will examine next (\cref{subsec:KMP}).

\easystrands{} has allowed us to experiment with various approaches to improve the proof techniques of \cite{FHG98}.
To illustrate this, we consider the dual protocol of \cref{ex:simpleprotocol} in which Alice sends an encrypted nonce to Bob, who decrypts it and sends it back in the clear.
Here, authentication is testified by the unique ability of the responder to decrypt an encrypted random challenge, so there is no ciphertext proving the presence of the responder.
Instead, the fact that the nonce has been decrypted needs to be considered as proof of the presence of Bob. Even here, we need one of the identifiers in the ciphertext to prevent reflection attacks.
The protocol is:
\begin{align*}
  A \rightarrow B & : \enc{N_a \cdot A}{\SK{AB}} \\
  B \rightarrow A & : N_a
\end{align*}
We let \lstinline|c := ⟨ Na ⋅ A ⟩_(SK A B)| and we consider the set of nodes whose term \lstinline|t| satisfies the proposition \lstinline|P := $Na ⊏ t /\ ~ c ⊏ t|.
Intuitively, these nodes contain the nonce \Na but do not contain the ciphertext \lstinline|c|.
Thus, they are $\preceq_C$-preceded by the node where the decryption happens.
Therefore, the minimal element of such a set should identify the responder node that performs the decryption and effectively binds all the responder parameters to the expected values \A, \B, \Na.

This proof technique for encrypted challenges, while effective, has a limitation: to eliminate the penetrator strand that destructs pairs, we need to prove that neither the initiator nor the responder originate pairs \lstinline|g ⋅ h| such that \linebreak\lstinline{c ⊏ h} or \lstinline|c ⊏ g|.
This enables the elimination of the pair destruction case of the penetrator, mainly because the penetrator is the only one that might have generated the problematic pairs containing \lstinline|c| in one element and \Na in the other.
These cases are problematic in general because we could have instances such as \linebreak\lstinline|P g /\ c ⊏ h|, implying \lstinline|~P (g ⋅ h)|. This observation is also mentioned in the proof of the NSL protocol in \cite{FHG98}.

Even though in \easystrands{} we have devised a general lemma to handle these cases uniformly and simply, this property on pairs really depends on the protocol syntax and is unrelated to its security.
Protocols that violate this property cannot be proved secure using this technique.
To overcome this limitation, we have explored a new proof technique, which we call the \emph{protected predicate} technique, and we now illustrate with a variant of the above protocol:
\begin{align*}
  A \rightarrow B & : B \cdot \enc{N_a \cdot A}{\SK{AB}} \\
  B \rightarrow A & : N_a
\end{align*}
This protocol adds \B in the clear in the first message, breaking the requirement that a honest participant strand never originates pairs \lstinline|g ⋅ h| such that \lstinline{c ⊏ h} or \lstinline|c ⊏ g|.
Thus, to prove the security of this protocol we define the following predicate:

\begin{lstlisting}
Fixpoint protected a :=
  match a with
  | $t => t <> Na
  | #_ => True
  | ⟨g ⋅ h⟩_(k) =>
      (k = SK A B /\ g = $Na /\ h = $A) \/
      (protected g /\ protected h)
  | ⟨g⟩_(k) => protected g
  | g⋅h => protected g /\ protected h
  end.
\end{lstlisting}
Intuitively, the condition \lstinline|protected A B Na a| holds if and only if \Na appears in \lstinline|a| in the form \lstinline|⟨ Na ⋅ A ⟩_(SK A B)|, or if it does not appear in \lstinline|a| at all.
Now we consider the set of nodes whose terms do not satisfy this condition and use its minimal element to prove agreement.
In fact,
we can prove that the first node where \Na appears unprotected is the responder node that performs the decryption.

This notion is less demanding than the previous predicate \lstinline|$Na ⊏ t /\ ~c ⊏ t|.
For example, term \lstinline|t = Na ⋅ ⟨ Na ⋅ A ⟩_(SK A B)| does not satisfy \lstinline|$Na ⊏ t /\ ~c ⊏ t| as \lstinline|c ⊏ t|, but satisfies \lstinline|~protected A B Na t| since \Na appears in \lstinline|t| in a form different from \lstinline|c|.
It is easy to see that \lstinline|~protected A B Na g| or \lstinline{~protected A B Na h} imply \lstinline|~protected A B Na (g⋅h)|, which solves the pair destruction case of the penetrator without any extra lemma.
We have used this technique to prove the security of the above protocol, and we have also applied it to the NSL protocol (\cref{subsec:NSL}).

Interestingly, regardless of the proof technique used, it is necessary for this protocol to assume that \lstinline|Na| uniquely originates, even
for
noninjective agreement.
Without this assumption, the attacker could simply guess \lstinline|Na| and impersonate Bob.
In the initial protocol of \cref{sec:proofprotocol}, nonce freshness is only required for injective agreement.
This illustrates the elegance of strand spaces, enabling the distillation of the minimal requirements for security proofs.

The example with the original proof technique can be found in \lstinline{Examples/simple_auth/SimpleAuthDual.v} and the variant using the \lstinline|protected| predicate can be found in \lstinline{Examples/simple_auth/SimpleAuthDualBProtected.v}.

\subsection{Maximal Penetrators and Compositionality}\label{sec:maximal}
We have seen that proofs in \easystrands{} rely on case analysis of the various strands belonging to the penetrator and the honest participants.
The goal is to show that a certain subset of nodes is either empty, as it does not contain a minimal element (e.g., for secrecy), or that it admits a minimal element on a specific honest strand (e.g., for authentication).
For penetrator strands, we typically need to demonstrate that none of them admits a minimal element, thus proving that the penetrator cannot interfere with the desired security property.

While performing our many mechanized proofs, we realized that a more general and efficient way to specify the penetrator would be to take a dual approach.
Instead of listing all possible penetrator strands in the classic Dolev-Yao style, we could define the penetrator in terms of what they cannot do with respect to sensitive cryptographic operations.
In other words, penetrator strands would include all those that do not violate specific cryptographic constraints.
This idea resembles the intriguing approach proposed in \cite{banaSymbolic} to achieve computational soundness results, and, in fact, is commonly used in computational models of cryptography.
Here, we explore this concept in a purely symbolic setting, which, to the best of our knowledge, is novel and unexplored in the literature.

This approach, which we call the \emph{maximal penetrator}, offers several advantages.
First, it allows for proving security without the need to specify a Dolev-Yao attacker, which depends on the specific structure of terms and requires updates whenever new terms, such as cryptographic primitives, are introduced.
Second, it enables the penetrator to be maximized by only specifying what is strictly forbidden in order to achieve the security of a given protocol.
As a result, if the security of two protocols has been proven with respect to their maximal penetrators, they can be composed when they mutually respect each other's maximal penetrator conditions.
Intuitively, given two protocols,
if the behavior of each protocol is fully subsumed by the maximal penetrator of the other, we can safely combine them and derive a security proof for the combined protocol from the individual proofs.
In other words, this approach provides protocol compositionality for free.

We have implemented this technique on the protocol of \cref{ex:simpleprotocol} and its variant presented in \cref{sec:reusing}, where $A$ is replaced by $B$.
We then proved the security of their composition by fully reusing the individual security proofs for each protocol, as explained below.

We begin by defining the concept of maximal penetrator strands.
The key challenge is defining a property that ensures the penetrator does not compromise the cryptographic primitives required for the protocol's security.
Ideally, this property should be minimal in order to maximize the penetrator’s capabilities.
In the simple authentication protocol of \cref{ex:simpleprotocol}, security relies on the ability to encrypt using \lstinline|SK A B|.
Thus, the following definition asserts that encryption by the penetrator should only be allowed if the key is known, i.e., it is readable in cleartext from the network.
\begin{lstlisting}
Definition NoForgeCipher A B n :=
  forall p, originates (⟨ p ⟩_(SK A B)) n ->
    exists n', n' =>+ n /\ term n' = ⊖ #(SK A B).
\end{lstlisting}
The above definition states that for given \A and \B, if a ciphertext \lstinline|⟨ p ⟩_(SK A B)| originates in an output node, there must exist a preceding input node in the same strand where the key \lstinline|SK A B| is read in the clear.

The only other property needed for security is that the penetrator never originates \lstinline|SK A B|.
Therefore, we specify maximal penetrator strands as those whose nodes do not originate \lstinline|SK A B| and satisfy \lstinline|NoForgeCipher A B|:
\begin{lstlisting}
Inductive SA_maximal_penetrator_strand (A B : T) : Σ -> Prop :=
  | SAS_Pen : forall s,
      ( forall n, s = strand n ->
        ~originates #(SK A B) n /\ NoForgeCipher A B n ) ->
        SA_maximal_penetrator_strand A B s.
\end{lstlisting}
Under this maximal penetrator, we were able to prove the same authentication properties that we established using the standard Dolev-Yao penetrator (see \lstinline{Examples/simple_auth/SimpleAuthMaximalEnc.v}).

We then demonstrated several interesting results.
First%
, the Dolev-Yao penetrator is subsumed by the maximal penetrator, confirming that we are not overlooking any significant attacks.
\begin{lstlisting}
Lemma DY_is_SA_maximal_penetrator:
  forall A B s, penetrator_strand (K__P_AB A B) s -> SA_maximal_penetrator_strand A B s.
\end{lstlisting}
This also implies that the results we proved under the Dolev-Yao penetrator can now be derived from those established under the maximal penetrator by simply applying the lemma above.
\ifdefined\COLORDIFF
    \color{cbred}
\else
\fi
Crucially, our maximal penetrator is strictly stronger than the Dolev-Yao penetrator but still allows the protocol to be proved secure.
Consider for example the strand \lstinline{[⊖ ⟨ M ⟩_(SK A B); ⊕ M ]}, where an encrypted message is decrypted without knowledge of the secret key \lstinline{SK A B}.
This strand cannot be constructed by a Dolev-Yao penetrator because such intruders cannot break cryptography.
However, since the strand satisfies \lstinline{NoForgeCipher} and does not originate \lstinline{SK A B}, the maximal penetrator can produce it.
This intuition is formalized in file \lstinline{SimpleAuthMaximalEnc.v} by the lemma \lstinline{SA_maximal_penetrator_not_eq_DY}, whose proof is based on the above example.
\ifdefined\COLORDIFF
    \color{black}
\else
\fi

For compositionality, it is useful to demonstrate that certain honest participants can be mimicked by the maximal penetrator.
In particular, we find that the initiator is always subsumed by the penetrator, as it neither originates ciphertexts nor sensitive keys.
In contrast, for the responder, this holds only when the initiator key \lstinline|SK A' B'| is different from the one the attacker cannot forge, namely \lstinline|SK A B|.
This situation arises when neither \lstinline|A = A' /\ B = B'| nor \lstinline|A = B' /\ B = A'|.
This latter property is also the basis for our compositionality result.
\begin{lstlisting}
Lemma ini_penetrator :
  forall s A B Na A' B',
    SA_initiator_strand A' B' Na s ->
    SA_maximal_penetrator_strand A B s.

Lemma res_penetrator :
  forall s A B Na A' B',
    ~((A = A' /\ B = B') \/ (A = B' /\ B = A')) ->
    SA_responder_strand A' B' Na s ->
    SA_maximal_penetrator_strand A B s.
\end{lstlisting}
We have finally composed the protocol of \cref{ex:simpleprotocol} and its variant presented in \cref{sec:reusing}, where $A$ is replaced by $B$ in the protocol, under the same maximal penetrator.
This is done by simply placing in the same strand space the maximal penetrator strands and the initiator and responder strands of the two protocols, whose identities are required to respectively satisfy two predicates \lstinline|p1| and \lstinline|p2|.
Fixed a maximal penetrator for \lstinline|A| and \lstinline|B|, if we pick
\begin{lstlisting}
Definition p1 (A' B' : T) := True.
Definition p2 (A' B' : T) := ~((A = A' /\ B = B') \/ (A = B' /\ B = A')).
\end{lstlisting}
we allow all participants for protocol 1, as well as all participants with \lstinline|SK A' B'| that is disjoint from \lstinline|SK A B| for protocol 2.
Due to this restriction, protocol 2 can be emulated by the maximal penetrator, thanks to the \lstinline|res_penetrator| lemma.
Consequently, we can prove that the composed strand space is essentially the same as the strand space of protocol 1, allowing us to directly reuse all the results that have already been established for protocol 1.
\begin{lstlisting}
Lemma comp_is_protocol1:
  C_is_SS C (SA_StrandSpace p1 p2 A B) ->
  C_is_SS C (SimpleAuthMaximalEnc.SA_StrandSpace A B).
\end{lstlisting}
The same result also holds for protocol 2 by swapping predicates \lstinline|p1| and \lstinline|p2|.
All details are available in \lstinline{Examples/simple_auth/SimpleAuthMaximalEnc*.v}.







\section{Case Studies}\label{sec:casestudies}
  In addition to the family of simple authentication protocols inspired by the ISO/IEC 9798-2 two-pass authentication protocol (\cref{sec:running}), we have applied \easystrands{} to two nontrivial case studies:
the classic Needham-Schoeder-Lowe protocol and its original flawed version (\cref{subsec:NSL}) \cite{lowe1995attack}, and a recently proposed solution for secure key management policies (\cref{subsec:KMP}) \cite{focardi2021secure}.
Due to space constraints we only briefly describe the highlights and refer the interested reader to the files respectively in \lstinline{Examples/nsl}, \lstinline{Examples/ns_original} and \lstinline{Examples/kmp}.

\subsection{Case Study 1: Needham-Schroeder-Lowe Protocol}\label{subsec:NSL}
The NSL protocol is a standard protocol that has been widely analyzed \cite{lowe1995attack}.
The protocol assumes that $A$ and $B$ know their respective public keys, $\mathit{PK} A$ and $\mathit{PK} B$:
\begin{align*}
    &A \rightarrow B: \langle N_a \cdot A \rangle_{\mathit{PK} B}\\
    &B \rightarrow A: \langle N_a \cdot N_b \cdot B \rangle_{\mathit{PK} A}\\
    &A \rightarrow B: \langle N_b \rangle_{\mathit{PK} B}
\end{align*}
This protocol can be used to mutually authenticate the \emph{initiator} $A$ and the \emph{responder} $B$, while allowing them to share two secret values (the nonces, $N_a$ and $N_b$) that can be used together to generate a shared session key.
Intuitively, the authentication guarantee arises from the fact that only $A$ and $B$ can decrypt the nonces using their private keys and send them back to each other. Meanwhile, cryptography ensures the secrecy of the fresh nonces. This protocol, along with its original flawed version (in which $B$ was absent in the second message), has been used in \cite{FHG98} to illustrate the strand spaces model.

\myparagraph{Results} We have successfully mechanized all the proofs in \cite{FHG98} spotting and fixing two problems, described below. We also applied our new \emph{protected predicate} proof technique from \cref{sec:newproofs} to simplify some proofs.

\myparagraph{First issue}
When proving Lemma 4.4 in \cite{FHG98}, Fabrega et al. consider the set
$T = \{ m \in C \mid m \prec_C n_2 \land g \cdot h \sqsubset \mathit{term} (m) \}$ (for some $n_2$, $g$, and $h$),
and they implicitly assume that $T$ is \emph{sign closed}, i.e., is such that for any pair of nodes $m, m'$ with the same unsigned term, it holds $m \in S$ iff $m' \in S$. However, this is not true when $m \prec_C n_2$ but $m' \not\prec_C n_2$. This means that the authors could not have applied Lemma 2.7 from~\cite{FHG98} that states that minimal elements of sign closed sets are always positive. Fortunately, the conclusion of this lemma still holds when weakening the requirements of being sign closed, and in \easystrands{} we have devised a general lemma to handle these cases uniformly and simply. In particular, we have a lemma similar to 2.7 from~\cite{FHG98} which only requires that for each negative node in a set there exists a preceding positive node that also belongs to the set (see \lstinline|Lemma minimal_is_positive_weak| in \lstinline|Common/Bundles.v|).

\myparagraph{Second issue}
Initiator's nonce secrecy is only sketched in \cite{FHG98}.
Reformulated in Coq, the first part of the initiator's nonce secrecy
\mbox{proposition
from~\cite{FHG98} would
be:}
\begin{lstlisting}
forall m, is_node_of m C -> $Na ⊏ uns_term m ->
    (⟨ $Na ⋅ $A ⟩_(PK B)) ⊏ uns_term m \/
      (⟨ $Na ⋅ $Nb ⋅ $B ⟩_(PK A)) ⊏ uns_term m.
\end{lstlisting}
Intuitively, whenever \Na appears in a node, one of the two above ciphertexts should also appear in the node.
Unfortunately, this proposition fails in (at least) two cases:
$(i)$ consider a node \lstinline{m} that lies on an \emph{initiator} strand with parameters \lstinline{A}, \lstinline{B}, \lstinline{Na}, and \lstinline{Na}.
Then, the third message \lstinline{⟨ $Na ⟩_(PK B)} contradicts the proposition;
$(ii)$ consider \lstinline{m} lying on a \emph{responder} strand with parameters \lstinline{A}, \lstinline{B}, \lstinline{Na}, and \lstinline{Nb' <> Nb}.
Here, the second message \lstinline{$Na ⊏ ⟨ $Na ⋅ $Nb' ⋅ $B ⟩_(PK A)} contradicts the proposition.
To solve this issue and prove the initiator's nonce secrecy, we weakened the theorem just enough by accounting for the missing case \lstinline{(⟨ $Na ⟩_(PK B)) ⊏ uns_term m}, and by letting \lstinline|Nb| free in \lstinline|⟨ $Na ⋅ $Nb ⋅ $B ⟩_(PK A)| 
(full proof in \lstinline{Examples/nsl/NSL_secrecy_initiator_simple.v}).

\subsection{Case Study 2: Key Management Policies}\label{subsec:KMP}
Key management encompasses the practices involved in generating, distributing, storing, and revoking cryptographic keys. To ensure security, keys are commonly stored in tamper-resistant hardware like Hardware Security Modules (HSMs) and accessed through suitable APIs, such as \texttt{PKCS\#11}.
Unfortunately, incorrect key management or overly liberal APIs, which do not allow to provide a policy that precisely determines the intended use of a certain class of keys, may hinder the security of the stored keys ~\cite{anderson00correctness,clulow03pkcs11}.
Among others~\cite{CentenaroFL13,KunPOST15}, Focardi and Luccio \cite{focardi2021secure} proposed security solutions based on {typed key management policies}.
The idea is to dynamically keep track of key types by encrypting a key and its type under a device master key. The policy dictates which key can wrap/unwrap which other key based on the respective types.

The proof in \cite{focardi2021secure} is developed in strand spaces and, due to the overapproximation result, we claim that such general soundness result would be hard if not impossible to achieve using state-of-the-art fully automated tools.
Preliminary tests with Tamarin allowed us to prove the security of specific policies, disregarding the overapproximation part.
Scalability became an issue as the policy size increased,  since the tool had to traverse all policy states for the analysis.

\myparagraph{Results} We fully mechanized the soundness theorem of \cite{focardi2021secure} and uncovered an ambiguous usage of the proof technique in the pen-and-paper development and a redundant case in the original notion of policy closure that we simplified. We improved the precision of the analysis by providing a more accurate closure operation, which allowed us to prove the security of the \emph{secure templates} example \cite{BCFS-ccs10}, previously rejected by the analysis in \cite{focardi2021secure}.

\myparagraph{First issue} The security theorem presented in \cite{focardi2021secure}
is a soundness result.
It establishes that the policy closure overapproximates the key types at runtime and at all bundle nodes.
Focardi and Luccio achieve this by considering the dual set of nodes violating the properties and demonstrating its emptiness through an inductive examination of all possible strands. During our analysis, we found that in the pen-and-paper development, the definition of this set did not encompass all possible cases for subterms. To address this, we employed our novel \emph{protected predicate} proof technique, which centers around the \lstinline|protected| predicate as outlined in \cref{sec:newproofs}. This approach inherently covers all subterms by construction and simplify the treatment of pair terms, especially in penetrator strands.

\myparagraph{Second issue}
While developing the proof mechanization we realized that one of the condition in the policy closure (item 5 in Definition 6 of \cite{focardi2021secure}) dealing with decryption operations was never
used in the proof and could be safely removed (see below for more detail).

\myparagraph{Improving the analysis precision}
While mechanizing  the proof by~\citet{focardi2021secure} we realized that the closure operation could be made more precise, simpler and more intuitive.
In the following we briefly present our improved closure operation and show that it is more precise than the original one by validating a particular policy, proposed in \cite{BCFS-ccs10}, that was rejected by the original analysis.

We need to provide more details about the model presented in \cite{focardi2021secure}.
When a key is created, a type is assigned to it and encrypted along with the key under a secret master key $\mkey$ to enforce the policy at execution time.
For example, key $k_1$ of type $\ckey_1$ is  modeled as $\enc{k_1,\ckey_1}{\mkey}$.
Keys can be used to encrypt and decrypt other keys to securely export them out of the device and possibly import them into another one.
These two operations are usually referred to as \emph{wrap} and \emph{unwrap}.
When a key is unwrapped any type admitted by the policy is assigned to the unwrapped key, making it possible to have multiple types for the same key.
This is modeled by creating another ciphertext with the new assigned type, e.g., $\enc{k_1,\ckey_2}{\mkey}$.

A key management policy is specified as a set of directives $\policyold{\ckey_1}{\lenc}{\ckey_2}$ and $\policyold{\ckey_1}{\ldec}{\ckey_2}$ respectively indicating that keys of type $\ckey_1$ can encrypt keys of type $\ckey_2$, and keys of type $\ckey_1$ can decrypt wrapped keys and assign them type $\ckey_2$.
We also let $D$ denote the type for generic data so $\policyold{\ckey_1}{\lenc}{D}$ and $\policyold{\ckey_1}{\ldec}{D}$ indicate that keys of type $\ckey_1$ can perform standard encryption and decryption operations on messages.
Let $\termDKey$ denote the keys originated in the device,
then the key management API strands have the following form:
\begin{description}[leftmargin=6em,style=nextline]
\item [~~\rm Create:] \strand{+\enc{k,\ckey}{\mkey}} with $k \in \termDKey$ uniquely originating
\item [~~\rm Encrypt:] \strand{-m,~ {-\enc{k,\ckey}{\mkey}},~ {+\enc{m}{k}}} if $\policyold{\ckey}{\lenc}{D}$
\item [~~\rm Decrypt:] \strand{-\enc{m}{k},~ {-\enc{k,\ckey}{\mkey}},~ {+m}} if $\policyold{\ckey}{\ldec}{D}$
\item [~~\rm Wrap:] \strand{-\enc{k_1,\ckey_1}{\mkey},~ {-\enc{k_2,\ckey_2}{\mkey}},~ {+\enc{k_1}{k_2}}} if $\policyold{\ckey_2}{\lenc}{\ckey_1}$
\item [~~\rm Unwrap:] \strand{-\enc{k_1}{k_2},~ {-\enc{k_2,\ckey_2}{\mkey}},~ {+\enc{k_1,\ckey_1}{\mkey}}} if $\policyold{\ckey_2}{\ldec}{\ckey_1}$
\end{description}
Intuitively, Create generates a new device key of type $K$, Encrypt and Decrypt perform standard encrypt and decrypt operations on messages if the policy enables them, Wrap and Unwrap model key management operations in which a key encrypts/decrypts other keys along the policy directives.

A {closure operation} applied to the key management policy yields an overapproximation of the types that a particular key may assume during runtime, and a security theorem establishes the soundness of this overapproximation, ensuring that keys never assuming the insecure \emph{Data} type $D$ are guaranteed to remain undisclosed. The set of types that are \emph{reachable} from an initial type $\ckey$ is noted $\R_\ckey$. To compute this set, a new policy denoted by $\Rightarrow$ is defined, extending $\rightarrow$ to overapproximate all possible key types that can be reached when executing the key management APIs.

The original closure of \cite{focardi2021secure} defines $\Rightarrow$ as the smallest relation such that:
\begin{enumerate}
    \item $\policy{\ckey}{l}{\cdkey}$ implies $\policyC{\ckey}{l}{\cdkey}$;
      \label{item1}
    \item $\ckey \in \R_\ckey$;
      \label{item2}
    \item $\policyC{D}{l}{D}$;
      \label{item2bis}
    \item $\policyC{\ckey}{\lenc}{\cdkey}$ and $\policyC{\ckey}{\ldec}{\czkey}$ implies $\czkey \in \R_\cdkey$;
      \label{item3}
      \item $\policyold{\ckey}{\ldec}{\cdkey}$ and $\ckey \in \R_\czkey$ implies $\policyC{\czkey}{\ldec}{\cdkey}$
     \label{item3bis}
  \item $\policyC{\ckey}{\lenc}{\cdkey}$ and ($\ckey \in \R_\czkey$ or $\czkey \in \R_\ckey$) implies $\policyC{\czkey}{\lenc}{\cdkey}$
       \label{item5}
  \item $\policyC{\cdkey}{\lenc}{\ckey}$ and ($\ckey \in \R_\czkey$ or $\czkey \in \R_\ckey$) implies $\policyC{\cdkey}{\lenc}{\czkey}$
       \label{item6}
  \end{enumerate}
Intuitively,  whatever is allowed by $\rightarrow$ is also allowed by $\Rightarrow$ (item \ref{item1}); a type $\ckey$ is always reachable by itself (item \ref{item2}); $D$ can perform any operation over  $D$, in order to account for penetrator's behaviour (item \ref{item2bis}); if a type $\ckey$ can acquire the capability of wrapping $\cdkey$ and then decrypt it as $\czkey$, then $\czkey$ should belong to the types $\R_\cdkey$ that are reachable from $\cdkey$ (item \ref{item3}). Item \ref{item3bis} propagates decryption capability from $K$ to $J$ if $K$ is reachable from $J$. Similarily, items \ref{item5} and \ref{item6} propagate encryption capabilities bidirectionally.

Developing our mechanized proof we first realized that item \ref{item3bis} was unnecessary, as discussed above, and we removed it.
Moreover,
while this closure can be proved to soundly approximate the propagation of key types and so it is enough for security, the last two items look overly conservative and not very intuitive.
We then devised a more accurate closure which replaces original rules from \ref{item3bis} to \ref{item6} with the following:
\begin{enumerate}
    \setcounter{enumi}{4}
\item $\policyC{\ckey}{\lenc}{\cdkey}$ and $\ckey \in \R_\czkey$ and $\cdkey \in \R_\cwkey$ implies $\policyC{\czkey}{\lenc}{\cwkey}$
     \label{item5b}
\item \vspace*{-.2cm}$\policyC{\ckey}{\ldec}{\cdkey}$ and $\ckey \in \R_\czkey$ implies $\policyC{\czkey}{\ldec}{\cdkey}$
     \label{item6b}
\end{enumerate}
Intuitively, when $K$ and $J$ can be reached by $Z$ and $W$, the encryption capabilities between $K$ and $J$ are inherited by $Z$ and $W$ (item \ref{item5b}).
Similarly, for decryption, the capability to decrypt to a type $J$ is inherited from $K$ by $Z$ if $K$ is reachable from $Z$ (item \ref{item6b}).
These two rules model more accurately the fact that encryption and decryption capabilities are acquired when a certain type $K$ is reached by another type $Z$.
\begin{figure}[t]
    \centering
    \begin{tikzpicture}[node distance=10mm and 20mm, main/.style = {semithick, inner sep=0, draw, circle,minimum width =0.7cm}]
    \node[main] (1) {$\ckey_1$};
    \node[main] (2) [below= of 1]{$\ckey_2$};
    \node[main] (3) [below= of 2]{$\ckey_3$};
    \node[main] (4) [right= of 2]{$D$};
    \path (1) edge [semithick, loop above, ->]  node[midway, above] {\lenc} (1);
    \path (1) edge [semithick, ->]  node[midway, right] {\lenc/\ldec} (2);
    \path (3) edge [semithick, bend right, ->]  node[midway, right] {\lenc/\ldec} (4);
    \draw[->] (1) [semithick, bend right, ->] to[in=225]++ (-2,-2) to[out=315]  node[at start, left] {\lenc} (3);
    \draw[semithick, ->] (2) -- node[midway, above] {\lenc} (4);
    \path (2) edge [semithick, loop left, ->]  node[midway, left] {\ldec} (2);
    \end{tikzpicture}
  \caption{Secure templates of \cite{BCFS-ccs10} as specified in \cite{focardi2021secure}.}
  \label{fig:templates}
\end{figure}

By largely reusing the mechanization of the original proof, we were able to demonstrate that this closure is also sound.
We applied it to all the examples in \cite{focardi2021secure}, reproducing all the results and additionally proving the security of the \emph{secure templates} policy shown in~\cref{fig:templates}.
Intuitively, this policy has a unique type for unwrapped keys ($\ckey_2$) that prevent conflicting roles.
Keys can be generated either as wrap/unwrap keys ($\ckey_1$) or as encrypt/decrypt keys ($\ckey_3$).
When unwrap happens, the imported key assumes type $\ckey_2$ which is only allowed to unwrap and encrypt. The rationale is that unwrap and encrypt operations do not conflict with the initial key roles.

The refined closure provides the following reachable types:
\begin{eqnarray*}
    \R_{\ckey_1} & = & \{ \ckey_1, \ckey_2 \} \\
    \R_{\ckey_2} & = & \{ \ckey_2 \} \\
    \R_{\ckey_3} & = & \{ \ckey_2, \ckey_3 \} \\
    \R_{\ckey_D} & = & \{ \ckey_2, D \}
\end{eqnarray*}
This confirms the intuition that from each type, it is possible to reach only the unwrapped key $\ckey_2$ and nothing else. In turn, this proves the confidentiality of key types $\ckey_1$, $\ckey_2$, and $\ckey_3$, since $D$, the insecure \emph{Data} type, does not appear in their reachable sets. Consequently, values of keys with those initial types will never appear as plaintext. In \cite{focardi2021secure}, instead, it is shown that the original closure computes reachable sets that all contain $D$, making it impossible to draw any conclusions about key confidentiality for this particular policy.
To the best of our knowledge, this is the first proof of security of the \emph{secure templates} policy from \cite{BCFS-ccs10}.





\section{Summary of Analyzed Protocols}
\label{sec:summary}
\subsection{Lloyd-Max Algorithm}
\label{subsec:Lloyd-Max}
For a given quantization bitwidth $B$ and an operand $\bm{X}$, the Lloyd-Max algorithm finds $2^B$ quantization levels $\{\hat{x}_i\}_{i=1}^{2^B}$ such that quantizing $\bm{X}$ by rounding each scalar in $\bm{X}$ to the nearest quantization level minimizes the quantization MSE. 

The algorithm starts with an initial guess of quantization levels and then iteratively computes quantization thresholds $\{\tau_i\}_{i=1}^{2^B-1}$ and updates quantization levels $\{\hat{x}_i\}_{i=1}^{2^B}$. Specifically, at iteration $n$, thresholds are set to the midpoints of the previous iteration's levels:
\begin{align*}
    \tau_i^{(n)}=\frac{\hat{x}_i^{(n-1)}+\hat{x}_{i+1}^{(n-1)}}2 \text{ for } i=1\ldots 2^B-1
\end{align*}
Subsequently, the quantization levels are re-computed as conditional means of the data regions defined by the new thresholds:
\begin{align*}
    \hat{x}_i^{(n)}=\mathbb{E}\left[ \bm{X} \big| \bm{X}\in [\tau_{i-1}^{(n)},\tau_i^{(n)}] \right] \text{ for } i=1\ldots 2^B
\end{align*}
where to satisfy boundary conditions we have $\tau_0=-\infty$ and $\tau_{2^B}=\infty$. The algorithm iterates the above steps until convergence.

Figure \ref{fig:lm_quant} compares the quantization levels of a $7$-bit floating point (E3M3) quantizer (left) to a $7$-bit Lloyd-Max quantizer (right) when quantizing a layer of weights from the GPT3-126M model at a per-tensor granularity. As shown, the Lloyd-Max quantizer achieves substantially lower quantization MSE. Further, Table \ref{tab:FP7_vs_LM7} shows the superior perplexity achieved by Lloyd-Max quantizers for bitwidths of $7$, $6$ and $5$. The difference between the quantizers is clear at 5 bits, where per-tensor FP quantization incurs a drastic and unacceptable increase in perplexity, while Lloyd-Max quantization incurs a much smaller increase. Nevertheless, we note that even the optimal Lloyd-Max quantizer incurs a notable ($\sim 1.5$) increase in perplexity due to the coarse granularity of quantization. 

\begin{figure}[h]
  \centering
  \includegraphics[width=0.7\linewidth]{sections/figures/LM7_FP7.pdf}
  \caption{\small Quantization levels and the corresponding quantization MSE of Floating Point (left) vs Lloyd-Max (right) Quantizers for a layer of weights in the GPT3-126M model.}
  \label{fig:lm_quant}
\end{figure}

\begin{table}[h]\scriptsize
\begin{center}
\caption{\label{tab:FP7_vs_LM7} \small Comparing perplexity (lower is better) achieved by floating point quantizers and Lloyd-Max quantizers on a GPT3-126M model for the Wikitext-103 dataset.}
\begin{tabular}{c|cc|c}
\hline
 \multirow{2}{*}{\textbf{Bitwidth}} & \multicolumn{2}{|c|}{\textbf{Floating-Point Quantizer}} & \textbf{Lloyd-Max Quantizer} \\
 & Best Format & Wikitext-103 Perplexity & Wikitext-103 Perplexity \\
\hline
7 & E3M3 & 18.32 & 18.27 \\
6 & E3M2 & 19.07 & 18.51 \\
5 & E4M0 & 43.89 & 19.71 \\
\hline
\end{tabular}
\end{center}
\end{table}

\subsection{Proof of Local Optimality of LO-BCQ}
\label{subsec:lobcq_opt_proof}
For a given block $\bm{b}_j$, the quantization MSE during LO-BCQ can be empirically evaluated as $\frac{1}{L_b}\lVert \bm{b}_j- \bm{\hat{b}}_j\rVert^2_2$ where $\bm{\hat{b}}_j$ is computed from equation (\ref{eq:clustered_quantization_definition}) as $C_{f(\bm{b}_j)}(\bm{b}_j)$. Further, for a given block cluster $\mathcal{B}_i$, we compute the quantization MSE as $\frac{1}{|\mathcal{B}_{i}|}\sum_{\bm{b} \in \mathcal{B}_{i}} \frac{1}{L_b}\lVert \bm{b}- C_i^{(n)}(\bm{b})\rVert^2_2$. Therefore, at the end of iteration $n$, we evaluate the overall quantization MSE $J^{(n)}$ for a given operand $\bm{X}$ composed of $N_c$ block clusters as:
\begin{align*}
    \label{eq:mse_iter_n}
    J^{(n)} = \frac{1}{N_c} \sum_{i=1}^{N_c} \frac{1}{|\mathcal{B}_{i}^{(n)}|}\sum_{\bm{v} \in \mathcal{B}_{i}^{(n)}} \frac{1}{L_b}\lVert \bm{b}- B_i^{(n)}(\bm{b})\rVert^2_2
\end{align*}

At the end of iteration $n$, the codebooks are updated from $\mathcal{C}^{(n-1)}$ to $\mathcal{C}^{(n)}$. However, the mapping of a given vector $\bm{b}_j$ to quantizers $\mathcal{C}^{(n)}$ remains as  $f^{(n)}(\bm{b}_j)$. At the next iteration, during the vector clustering step, $f^{(n+1)}(\bm{b}_j)$ finds new mapping of $\bm{b}_j$ to updated codebooks $\mathcal{C}^{(n)}$ such that the quantization MSE over the candidate codebooks is minimized. Therefore, we obtain the following result for $\bm{b}_j$:
\begin{align*}
\frac{1}{L_b}\lVert \bm{b}_j - C_{f^{(n+1)}(\bm{b}_j)}^{(n)}(\bm{b}_j)\rVert^2_2 \le \frac{1}{L_b}\lVert \bm{b}_j - C_{f^{(n)}(\bm{b}_j)}^{(n)}(\bm{b}_j)\rVert^2_2
\end{align*}

That is, quantizing $\bm{b}_j$ at the end of the block clustering step of iteration $n+1$ results in lower quantization MSE compared to quantizing at the end of iteration $n$. Since this is true for all $\bm{b} \in \bm{X}$, we assert the following:
\begin{equation}
\begin{split}
\label{eq:mse_ineq_1}
    \tilde{J}^{(n+1)} &= \frac{1}{N_c} \sum_{i=1}^{N_c} \frac{1}{|\mathcal{B}_{i}^{(n+1)}|}\sum_{\bm{b} \in \mathcal{B}_{i}^{(n+1)}} \frac{1}{L_b}\lVert \bm{b} - C_i^{(n)}(b)\rVert^2_2 \le J^{(n)}
\end{split}
\end{equation}
where $\tilde{J}^{(n+1)}$ is the the quantization MSE after the vector clustering step at iteration $n+1$.

Next, during the codebook update step (\ref{eq:quantizers_update}) at iteration $n+1$, the per-cluster codebooks $\mathcal{C}^{(n)}$ are updated to $\mathcal{C}^{(n+1)}$ by invoking the Lloyd-Max algorithm \citep{Lloyd}. We know that for any given value distribution, the Lloyd-Max algorithm minimizes the quantization MSE. Therefore, for a given vector cluster $\mathcal{B}_i$ we obtain the following result:

\begin{equation}
    \frac{1}{|\mathcal{B}_{i}^{(n+1)}|}\sum_{\bm{b} \in \mathcal{B}_{i}^{(n+1)}} \frac{1}{L_b}\lVert \bm{b}- C_i^{(n+1)}(\bm{b})\rVert^2_2 \le \frac{1}{|\mathcal{B}_{i}^{(n+1)}|}\sum_{\bm{b} \in \mathcal{B}_{i}^{(n+1)}} \frac{1}{L_b}\lVert \bm{b}- C_i^{(n)}(\bm{b})\rVert^2_2
\end{equation}

The above equation states that quantizing the given block cluster $\mathcal{B}_i$ after updating the associated codebook from $C_i^{(n)}$ to $C_i^{(n+1)}$ results in lower quantization MSE. Since this is true for all the block clusters, we derive the following result: 
\begin{equation}
\begin{split}
\label{eq:mse_ineq_2}
     J^{(n+1)} &= \frac{1}{N_c} \sum_{i=1}^{N_c} \frac{1}{|\mathcal{B}_{i}^{(n+1)}|}\sum_{\bm{b} \in \mathcal{B}_{i}^{(n+1)}} \frac{1}{L_b}\lVert \bm{b}- C_i^{(n+1)}(\bm{b})\rVert^2_2  \le \tilde{J}^{(n+1)}   
\end{split}
\end{equation}

Following (\ref{eq:mse_ineq_1}) and (\ref{eq:mse_ineq_2}), we find that the quantization MSE is non-increasing for each iteration, that is, $J^{(1)} \ge J^{(2)} \ge J^{(3)} \ge \ldots \ge J^{(M)}$ where $M$ is the maximum number of iterations. 
%Therefore, we can say that if the algorithm converges, then it must be that it has converged to a local minimum. 
\hfill $\blacksquare$


\begin{figure}
    \begin{center}
    \includegraphics[width=0.5\textwidth]{sections//figures/mse_vs_iter.pdf}
    \end{center}
    \caption{\small NMSE vs iterations during LO-BCQ compared to other block quantization proposals}
    \label{fig:nmse_vs_iter}
\end{figure}

Figure \ref{fig:nmse_vs_iter} shows the empirical convergence of LO-BCQ across several block lengths and number of codebooks. Also, the MSE achieved by LO-BCQ is compared to baselines such as MXFP and VSQ. As shown, LO-BCQ converges to a lower MSE than the baselines. Further, we achieve better convergence for larger number of codebooks ($N_c$) and for a smaller block length ($L_b$), both of which increase the bitwidth of BCQ (see Eq \ref{eq:bitwidth_bcq}).


\subsection{Additional Accuracy Results}
%Table \ref{tab:lobcq_config} lists the various LOBCQ configurations and their corresponding bitwidths.
\begin{table}
\setlength{\tabcolsep}{4.75pt}
\begin{center}
\caption{\label{tab:lobcq_config} Various LO-BCQ configurations and their bitwidths.}
\begin{tabular}{|c||c|c|c|c||c|c||c|} 
\hline
 & \multicolumn{4}{|c||}{$L_b=8$} & \multicolumn{2}{|c||}{$L_b=4$} & $L_b=2$ \\
 \hline
 \backslashbox{$L_A$\kern-1em}{\kern-1em$N_c$} & 2 & 4 & 8 & 16 & 2 & 4 & 2 \\
 \hline
 64 & 4.25 & 4.375 & 4.5 & 4.625 & 4.375 & 4.625 & 4.625\\
 \hline
 32 & 4.375 & 4.5 & 4.625& 4.75 & 4.5 & 4.75 & 4.75 \\
 \hline
 16 & 4.625 & 4.75& 4.875 & 5 & 4.75 & 5 & 5 \\
 \hline
\end{tabular}
\end{center}
\end{table}

%\subsection{Perplexity achieved by various LO-BCQ configurations on Wikitext-103 dataset}

\begin{table} \centering
\begin{tabular}{|c||c|c|c|c||c|c||c|} 
\hline
 $L_b \rightarrow$& \multicolumn{4}{c||}{8} & \multicolumn{2}{c||}{4} & 2\\
 \hline
 \backslashbox{$L_A$\kern-1em}{\kern-1em$N_c$} & 2 & 4 & 8 & 16 & 2 & 4 & 2  \\
 %$N_c \rightarrow$ & 2 & 4 & 8 & 16 & 2 & 4 & 2 \\
 \hline
 \hline
 \multicolumn{8}{c}{GPT3-1.3B (FP32 PPL = 9.98)} \\ 
 \hline
 \hline
 64 & 10.40 & 10.23 & 10.17 & 10.15 &  10.28 & 10.18 & 10.19 \\
 \hline
 32 & 10.25 & 10.20 & 10.15 & 10.12 &  10.23 & 10.17 & 10.17 \\
 \hline
 16 & 10.22 & 10.16 & 10.10 & 10.09 &  10.21 & 10.14 & 10.16 \\
 \hline
  \hline
 \multicolumn{8}{c}{GPT3-8B (FP32 PPL = 7.38)} \\ 
 \hline
 \hline
 64 & 7.61 & 7.52 & 7.48 &  7.47 &  7.55 &  7.49 & 7.50 \\
 \hline
 32 & 7.52 & 7.50 & 7.46 &  7.45 &  7.52 &  7.48 & 7.48  \\
 \hline
 16 & 7.51 & 7.48 & 7.44 &  7.44 &  7.51 &  7.49 & 7.47  \\
 \hline
\end{tabular}
\caption{\label{tab:ppl_gpt3_abalation} Wikitext-103 perplexity across GPT3-1.3B and 8B models.}
\end{table}

\begin{table} \centering
\begin{tabular}{|c||c|c|c|c||} 
\hline
 $L_b \rightarrow$& \multicolumn{4}{c||}{8}\\
 \hline
 \backslashbox{$L_A$\kern-1em}{\kern-1em$N_c$} & 2 & 4 & 8 & 16 \\
 %$N_c \rightarrow$ & 2 & 4 & 8 & 16 & 2 & 4 & 2 \\
 \hline
 \hline
 \multicolumn{5}{|c|}{Llama2-7B (FP32 PPL = 5.06)} \\ 
 \hline
 \hline
 64 & 5.31 & 5.26 & 5.19 & 5.18  \\
 \hline
 32 & 5.23 & 5.25 & 5.18 & 5.15  \\
 \hline
 16 & 5.23 & 5.19 & 5.16 & 5.14  \\
 \hline
 \multicolumn{5}{|c|}{Nemotron4-15B (FP32 PPL = 5.87)} \\ 
 \hline
 \hline
 64  & 6.3 & 6.20 & 6.13 & 6.08  \\
 \hline
 32  & 6.24 & 6.12 & 6.07 & 6.03  \\
 \hline
 16  & 6.12 & 6.14 & 6.04 & 6.02  \\
 \hline
 \multicolumn{5}{|c|}{Nemotron4-340B (FP32 PPL = 3.48)} \\ 
 \hline
 \hline
 64 & 3.67 & 3.62 & 3.60 & 3.59 \\
 \hline
 32 & 3.63 & 3.61 & 3.59 & 3.56 \\
 \hline
 16 & 3.61 & 3.58 & 3.57 & 3.55 \\
 \hline
\end{tabular}
\caption{\label{tab:ppl_llama7B_nemo15B} Wikitext-103 perplexity compared to FP32 baseline in Llama2-7B and Nemotron4-15B, 340B models}
\end{table}

%\subsection{Perplexity achieved by various LO-BCQ configurations on MMLU dataset}


\begin{table} \centering
\begin{tabular}{|c||c|c|c|c||c|c|c|c|} 
\hline
 $L_b \rightarrow$& \multicolumn{4}{c||}{8} & \multicolumn{4}{c||}{8}\\
 \hline
 \backslashbox{$L_A$\kern-1em}{\kern-1em$N_c$} & 2 & 4 & 8 & 16 & 2 & 4 & 8 & 16  \\
 %$N_c \rightarrow$ & 2 & 4 & 8 & 16 & 2 & 4 & 2 \\
 \hline
 \hline
 \multicolumn{5}{|c|}{Llama2-7B (FP32 Accuracy = 45.8\%)} & \multicolumn{4}{|c|}{Llama2-70B (FP32 Accuracy = 69.12\%)} \\ 
 \hline
 \hline
 64 & 43.9 & 43.4 & 43.9 & 44.9 & 68.07 & 68.27 & 68.17 & 68.75 \\
 \hline
 32 & 44.5 & 43.8 & 44.9 & 44.5 & 68.37 & 68.51 & 68.35 & 68.27  \\
 \hline
 16 & 43.9 & 42.7 & 44.9 & 45 & 68.12 & 68.77 & 68.31 & 68.59  \\
 \hline
 \hline
 \multicolumn{5}{|c|}{GPT3-22B (FP32 Accuracy = 38.75\%)} & \multicolumn{4}{|c|}{Nemotron4-15B (FP32 Accuracy = 64.3\%)} \\ 
 \hline
 \hline
 64 & 36.71 & 38.85 & 38.13 & 38.92 & 63.17 & 62.36 & 63.72 & 64.09 \\
 \hline
 32 & 37.95 & 38.69 & 39.45 & 38.34 & 64.05 & 62.30 & 63.8 & 64.33  \\
 \hline
 16 & 38.88 & 38.80 & 38.31 & 38.92 & 63.22 & 63.51 & 63.93 & 64.43  \\
 \hline
\end{tabular}
\caption{\label{tab:mmlu_abalation} Accuracy on MMLU dataset across GPT3-22B, Llama2-7B, 70B and Nemotron4-15B models.}
\end{table}


%\subsection{Perplexity achieved by various LO-BCQ configurations on LM evaluation harness}

\begin{table} \centering
\begin{tabular}{|c||c|c|c|c||c|c|c|c|} 
\hline
 $L_b \rightarrow$& \multicolumn{4}{c||}{8} & \multicolumn{4}{c||}{8}\\
 \hline
 \backslashbox{$L_A$\kern-1em}{\kern-1em$N_c$} & 2 & 4 & 8 & 16 & 2 & 4 & 8 & 16  \\
 %$N_c \rightarrow$ & 2 & 4 & 8 & 16 & 2 & 4 & 2 \\
 \hline
 \hline
 \multicolumn{5}{|c|}{Race (FP32 Accuracy = 37.51\%)} & \multicolumn{4}{|c|}{Boolq (FP32 Accuracy = 64.62\%)} \\ 
 \hline
 \hline
 64 & 36.94 & 37.13 & 36.27 & 37.13 & 63.73 & 62.26 & 63.49 & 63.36 \\
 \hline
 32 & 37.03 & 36.36 & 36.08 & 37.03 & 62.54 & 63.51 & 63.49 & 63.55  \\
 \hline
 16 & 37.03 & 37.03 & 36.46 & 37.03 & 61.1 & 63.79 & 63.58 & 63.33  \\
 \hline
 \hline
 \multicolumn{5}{|c|}{Winogrande (FP32 Accuracy = 58.01\%)} & \multicolumn{4}{|c|}{Piqa (FP32 Accuracy = 74.21\%)} \\ 
 \hline
 \hline
 64 & 58.17 & 57.22 & 57.85 & 58.33 & 73.01 & 73.07 & 73.07 & 72.80 \\
 \hline
 32 & 59.12 & 58.09 & 57.85 & 58.41 & 73.01 & 73.94 & 72.74 & 73.18  \\
 \hline
 16 & 57.93 & 58.88 & 57.93 & 58.56 & 73.94 & 72.80 & 73.01 & 73.94  \\
 \hline
\end{tabular}
\caption{\label{tab:mmlu_abalation} Accuracy on LM evaluation harness tasks on GPT3-1.3B model.}
\end{table}

\begin{table} \centering
\begin{tabular}{|c||c|c|c|c||c|c|c|c|} 
\hline
 $L_b \rightarrow$& \multicolumn{4}{c||}{8} & \multicolumn{4}{c||}{8}\\
 \hline
 \backslashbox{$L_A$\kern-1em}{\kern-1em$N_c$} & 2 & 4 & 8 & 16 & 2 & 4 & 8 & 16  \\
 %$N_c \rightarrow$ & 2 & 4 & 8 & 16 & 2 & 4 & 2 \\
 \hline
 \hline
 \multicolumn{5}{|c|}{Race (FP32 Accuracy = 41.34\%)} & \multicolumn{4}{|c|}{Boolq (FP32 Accuracy = 68.32\%)} \\ 
 \hline
 \hline
 64 & 40.48 & 40.10 & 39.43 & 39.90 & 69.20 & 68.41 & 69.45 & 68.56 \\
 \hline
 32 & 39.52 & 39.52 & 40.77 & 39.62 & 68.32 & 67.43 & 68.17 & 69.30  \\
 \hline
 16 & 39.81 & 39.71 & 39.90 & 40.38 & 68.10 & 66.33 & 69.51 & 69.42  \\
 \hline
 \hline
 \multicolumn{5}{|c|}{Winogrande (FP32 Accuracy = 67.88\%)} & \multicolumn{4}{|c|}{Piqa (FP32 Accuracy = 78.78\%)} \\ 
 \hline
 \hline
 64 & 66.85 & 66.61 & 67.72 & 67.88 & 77.31 & 77.42 & 77.75 & 77.64 \\
 \hline
 32 & 67.25 & 67.72 & 67.72 & 67.00 & 77.31 & 77.04 & 77.80 & 77.37  \\
 \hline
 16 & 68.11 & 68.90 & 67.88 & 67.48 & 77.37 & 78.13 & 78.13 & 77.69  \\
 \hline
\end{tabular}
\caption{\label{tab:mmlu_abalation} Accuracy on LM evaluation harness tasks on GPT3-8B model.}
\end{table}

\begin{table} \centering
\begin{tabular}{|c||c|c|c|c||c|c|c|c|} 
\hline
 $L_b \rightarrow$& \multicolumn{4}{c||}{8} & \multicolumn{4}{c||}{8}\\
 \hline
 \backslashbox{$L_A$\kern-1em}{\kern-1em$N_c$} & 2 & 4 & 8 & 16 & 2 & 4 & 8 & 16  \\
 %$N_c \rightarrow$ & 2 & 4 & 8 & 16 & 2 & 4 & 2 \\
 \hline
 \hline
 \multicolumn{5}{|c|}{Race (FP32 Accuracy = 40.67\%)} & \multicolumn{4}{|c|}{Boolq (FP32 Accuracy = 76.54\%)} \\ 
 \hline
 \hline
 64 & 40.48 & 40.10 & 39.43 & 39.90 & 75.41 & 75.11 & 77.09 & 75.66 \\
 \hline
 32 & 39.52 & 39.52 & 40.77 & 39.62 & 76.02 & 76.02 & 75.96 & 75.35  \\
 \hline
 16 & 39.81 & 39.71 & 39.90 & 40.38 & 75.05 & 73.82 & 75.72 & 76.09  \\
 \hline
 \hline
 \multicolumn{5}{|c|}{Winogrande (FP32 Accuracy = 70.64\%)} & \multicolumn{4}{|c|}{Piqa (FP32 Accuracy = 79.16\%)} \\ 
 \hline
 \hline
 64 & 69.14 & 70.17 & 70.17 & 70.56 & 78.24 & 79.00 & 78.62 & 78.73 \\
 \hline
 32 & 70.96 & 69.69 & 71.27 & 69.30 & 78.56 & 79.49 & 79.16 & 78.89  \\
 \hline
 16 & 71.03 & 69.53 & 69.69 & 70.40 & 78.13 & 79.16 & 79.00 & 79.00  \\
 \hline
\end{tabular}
\caption{\label{tab:mmlu_abalation} Accuracy on LM evaluation harness tasks on GPT3-22B model.}
\end{table}

\begin{table} \centering
\begin{tabular}{|c||c|c|c|c||c|c|c|c|} 
\hline
 $L_b \rightarrow$& \multicolumn{4}{c||}{8} & \multicolumn{4}{c||}{8}\\
 \hline
 \backslashbox{$L_A$\kern-1em}{\kern-1em$N_c$} & 2 & 4 & 8 & 16 & 2 & 4 & 8 & 16  \\
 %$N_c \rightarrow$ & 2 & 4 & 8 & 16 & 2 & 4 & 2 \\
 \hline
 \hline
 \multicolumn{5}{|c|}{Race (FP32 Accuracy = 44.4\%)} & \multicolumn{4}{|c|}{Boolq (FP32 Accuracy = 79.29\%)} \\ 
 \hline
 \hline
 64 & 42.49 & 42.51 & 42.58 & 43.45 & 77.58 & 77.37 & 77.43 & 78.1 \\
 \hline
 32 & 43.35 & 42.49 & 43.64 & 43.73 & 77.86 & 75.32 & 77.28 & 77.86  \\
 \hline
 16 & 44.21 & 44.21 & 43.64 & 42.97 & 78.65 & 77 & 76.94 & 77.98  \\
 \hline
 \hline
 \multicolumn{5}{|c|}{Winogrande (FP32 Accuracy = 69.38\%)} & \multicolumn{4}{|c|}{Piqa (FP32 Accuracy = 78.07\%)} \\ 
 \hline
 \hline
 64 & 68.9 & 68.43 & 69.77 & 68.19 & 77.09 & 76.82 & 77.09 & 77.86 \\
 \hline
 32 & 69.38 & 68.51 & 68.82 & 68.90 & 78.07 & 76.71 & 78.07 & 77.86  \\
 \hline
 16 & 69.53 & 67.09 & 69.38 & 68.90 & 77.37 & 77.8 & 77.91 & 77.69  \\
 \hline
\end{tabular}
\caption{\label{tab:mmlu_abalation} Accuracy on LM evaluation harness tasks on Llama2-7B model.}
\end{table}

\begin{table} \centering
\begin{tabular}{|c||c|c|c|c||c|c|c|c|} 
\hline
 $L_b \rightarrow$& \multicolumn{4}{c||}{8} & \multicolumn{4}{c||}{8}\\
 \hline
 \backslashbox{$L_A$\kern-1em}{\kern-1em$N_c$} & 2 & 4 & 8 & 16 & 2 & 4 & 8 & 16  \\
 %$N_c \rightarrow$ & 2 & 4 & 8 & 16 & 2 & 4 & 2 \\
 \hline
 \hline
 \multicolumn{5}{|c|}{Race (FP32 Accuracy = 48.8\%)} & \multicolumn{4}{|c|}{Boolq (FP32 Accuracy = 85.23\%)} \\ 
 \hline
 \hline
 64 & 49.00 & 49.00 & 49.28 & 48.71 & 82.82 & 84.28 & 84.03 & 84.25 \\
 \hline
 32 & 49.57 & 48.52 & 48.33 & 49.28 & 83.85 & 84.46 & 84.31 & 84.93  \\
 \hline
 16 & 49.85 & 49.09 & 49.28 & 48.99 & 85.11 & 84.46 & 84.61 & 83.94  \\
 \hline
 \hline
 \multicolumn{5}{|c|}{Winogrande (FP32 Accuracy = 79.95\%)} & \multicolumn{4}{|c|}{Piqa (FP32 Accuracy = 81.56\%)} \\ 
 \hline
 \hline
 64 & 78.77 & 78.45 & 78.37 & 79.16 & 81.45 & 80.69 & 81.45 & 81.5 \\
 \hline
 32 & 78.45 & 79.01 & 78.69 & 80.66 & 81.56 & 80.58 & 81.18 & 81.34  \\
 \hline
 16 & 79.95 & 79.56 & 79.79 & 79.72 & 81.28 & 81.66 & 81.28 & 80.96  \\
 \hline
\end{tabular}
\caption{\label{tab:mmlu_abalation} Accuracy on LM evaluation harness tasks on Llama2-70B model.}
\end{table}

%\section{MSE Studies}
%\textcolor{red}{TODO}


\subsection{Number Formats and Quantization Method}
\label{subsec:numFormats_quantMethod}
\subsubsection{Integer Format}
An $n$-bit signed integer (INT) is typically represented with a 2s-complement format \citep{yao2022zeroquant,xiao2023smoothquant,dai2021vsq}, where the most significant bit denotes the sign.

\subsubsection{Floating Point Format}
An $n$-bit signed floating point (FP) number $x$ comprises of a 1-bit sign ($x_{\mathrm{sign}}$), $B_m$-bit mantissa ($x_{\mathrm{mant}}$) and $B_e$-bit exponent ($x_{\mathrm{exp}}$) such that $B_m+B_e=n-1$. The associated constant exponent bias ($E_{\mathrm{bias}}$) is computed as $(2^{{B_e}-1}-1)$. We denote this format as $E_{B_e}M_{B_m}$.  

\subsubsection{Quantization Scheme}
\label{subsec:quant_method}
A quantization scheme dictates how a given unquantized tensor is converted to its quantized representation. We consider FP formats for the purpose of illustration. Given an unquantized tensor $\bm{X}$ and an FP format $E_{B_e}M_{B_m}$, we first, we compute the quantization scale factor $s_X$ that maps the maximum absolute value of $\bm{X}$ to the maximum quantization level of the $E_{B_e}M_{B_m}$ format as follows:
\begin{align}
\label{eq:sf}
    s_X = \frac{\mathrm{max}(|\bm{X}|)}{\mathrm{max}(E_{B_e}M_{B_m})}
\end{align}
In the above equation, $|\cdot|$ denotes the absolute value function.

Next, we scale $\bm{X}$ by $s_X$ and quantize it to $\hat{\bm{X}}$ by rounding it to the nearest quantization level of $E_{B_e}M_{B_m}$ as:

\begin{align}
\label{eq:tensor_quant}
    \hat{\bm{X}} = \text{round-to-nearest}\left(\frac{\bm{X}}{s_X}, E_{B_e}M_{B_m}\right)
\end{align}

We perform dynamic max-scaled quantization \citep{wu2020integer}, where the scale factor $s$ for activations is dynamically computed during runtime.

\subsection{Vector Scaled Quantization}
\begin{wrapfigure}{r}{0.35\linewidth}
  \centering
  \includegraphics[width=\linewidth]{sections/figures/vsquant.jpg}
  \caption{\small Vectorwise decomposition for per-vector scaled quantization (VSQ \citep{dai2021vsq}).}
  \label{fig:vsquant}
\end{wrapfigure}
During VSQ \citep{dai2021vsq}, the operand tensors are decomposed into 1D vectors in a hardware friendly manner as shown in Figure \ref{fig:vsquant}. Since the decomposed tensors are used as operands in matrix multiplications during inference, it is beneficial to perform this decomposition along the reduction dimension of the multiplication. The vectorwise quantization is performed similar to tensorwise quantization described in Equations \ref{eq:sf} and \ref{eq:tensor_quant}, where a scale factor $s_v$ is required for each vector $\bm{v}$ that maps the maximum absolute value of that vector to the maximum quantization level. While smaller vector lengths can lead to larger accuracy gains, the associated memory and computational overheads due to the per-vector scale factors increases. To alleviate these overheads, VSQ \citep{dai2021vsq} proposed a second level quantization of the per-vector scale factors to unsigned integers, while MX \citep{rouhani2023shared} quantizes them to integer powers of 2 (denoted as $2^{INT}$).

\subsubsection{MX Format}
The MX format proposed in \citep{rouhani2023microscaling} introduces the concept of sub-block shifting. For every two scalar elements of $b$-bits each, there is a shared exponent bit. The value of this exponent bit is determined through an empirical analysis that targets minimizing quantization MSE. We note that the FP format $E_{1}M_{b}$ is strictly better than MX from an accuracy perspective since it allocates a dedicated exponent bit to each scalar as opposed to sharing it across two scalars. Therefore, we conservatively bound the accuracy of a $b+2$-bit signed MX format with that of a $E_{1}M_{b}$ format in our comparisons. For instance, we use E1M2 format as a proxy for MX4.

\begin{figure}
    \centering
    \includegraphics[width=1\linewidth]{sections//figures/BlockFormats.pdf}
    \caption{\small Comparing LO-BCQ to MX format.}
    \label{fig:block_formats}
\end{figure}

Figure \ref{fig:block_formats} compares our $4$-bit LO-BCQ block format to MX \citep{rouhani2023microscaling}. As shown, both LO-BCQ and MX decompose a given operand tensor into block arrays and each block array into blocks. Similar to MX, we find that per-block quantization ($L_b < L_A$) leads to better accuracy due to increased flexibility. While MX achieves this through per-block $1$-bit micro-scales, we associate a dedicated codebook to each block through a per-block codebook selector. Further, MX quantizes the per-block array scale-factor to E8M0 format without per-tensor scaling. In contrast during LO-BCQ, we find that per-tensor scaling combined with quantization of per-block array scale-factor to E4M3 format results in superior inference accuracy across models. 


\section{Discussion and Conclusions}
  
\section{CONCLUSION}
\label{sec:concl}
The rapid rise of AI-generated media challenges information authenticity and societal trust, necessitating robust detection mechanisms. This survey examines the evolution of AI-generated media detection, focusing on the shift from Non-MLLM-based domain-specific detectors to MLLM-based general-purpose approaches. We compare these methods across authenticity, explainability, and localization tasks from both single-modal and multi-modal perspectives. Additionally, we review datasets, methodologies, and evaluation metrics, identifying key limitations and research challenges.
Beyond technical concerns, MLLM-based detection raises ethical and security issues. As GenAI sees broader deployment, regulatory frameworks vary significantly across jurisdictions, complicating governance. By summarizing these regulations, we provide insights for researchers navigating legal and ethical challenges.
While many challenges remain, We hope this survey sparks further discussion, informs future research, and contributes to a more secure and trustworthy AI ecosystem.

\section*{Acknowledgments}
We would like to thank the anonymous reviewers for their comments and suggestions, which greatly helped us in improving this paper.
This work is partially supported by projects ``SEcurity and RIghts In the CyberSpace - SERICS'' (PE00000014 - CUP H73C2200089001), ``Interconnected Nord-Est Innovation Ecoscheme - iNEST'' (ECS00000043 - CUP H43C22000540006), and PRIN/PNRR ``Automatic Modelling and \(\forall\)erification of Dedicated sEcUrity deviceS - AM\(\forall\)DEUS'' (P2022EPPHM - CUP H53D23008130001), all under the National Recovery and Resilience Plan (NRRP) funded by the European Union - NextGenerationEU.



\bibliographystyle{IEEEtranN}

\small{
\bibliography{biblio}
}






\end{document}
