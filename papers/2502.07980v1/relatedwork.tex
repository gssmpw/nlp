\section{Related Work}
Task-specific evaluation plays a crucial role in advancing research in LLM applications by providing precise insights into model capabilities and limitations within defined contexts.
The scalability of general-purpose models has demonstrated enhanced task performance in various domains, including language \citep{brown2020languagemodelsfewshotlearners}, mathematics \citep{MathPaper2023, Mao2024CHAMP}, and code generation \citep{humanEvalpaper}\footnote{\href{https://paperswithcode.com/sota/code-generation-on-humaneval}{HumanEval}}. 

In the realm of digital circuit design, noteworthy progress has been made in harnessing LLMs for tasks such as generating Verilog Code, as explored by \citet{VerilogEval2023}. Moreover, Cadence's JedAI \footnote{\href{https://community.cadence.com/cadence_blogs_8/b/corporate/posts/cadence-creates-industry-s-first-llm-technology-for-chip-design}{JedAI}} platform exemplifies the first application of LLM technology in chip design, illustrating the feasibility of integrating LLMs into digital design workflows.

In the realm of analog design, LLMs have already been integrated into frameworks that automate aspects of the design process \citep{chang2024lamagic, lai2024analogcoder}. While these works focus on leveraging LLMs directly for circuit design, an essential precursor is to evaluate the knowledge and reasoning capabilities of LLMs on fundamental analog circuit knowledge. Without a deep understanding of their foundational capabilities, the effectiveness and versatility of LLMs in real-world circuit design may be limited. To address this gap, we introduce the \textbf{CIRCUIT} dataset, which serves as a critical first step in the analog design pipeline.

When reviewing existing datasets for other domains, we notice that evaluation proves difficult on complex tasks. Coding tasks utilize unit testing with automatic evaluation, while other fields necessitate human evaluation. LLMs have also been used as evaluating agents. \citep{Mao2024CHAMP, TruthfulQA} While LLMs can evaluate large volumes of data, do not suffer from fatigue, and are cheaper to utilize, our initial experiments showed that they struggle with understanding and interpreting complex reasoning about analog circuits.
Inspired by unit testing, we introduce a simple dataset design and evaluation metric combination that shows promise for the assessment of LLMs across various fields and tasks. This framework is inherently scalable, suitable for cost-effective automatic evaluation, adaptable to more complex analog design tasks, and transferable to other reasoning domains.


%--------------------------------------------------------------------------------------