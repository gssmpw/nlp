\documentclass[draft,12pt,cleveref]{colt2025} % Anonymized submission
%\documentclass[final,12pt]{colt2025} % Include author names

% The following packages will be automatically loaded:
% amsmath, amssymb, natbib, graphicx, url, algorithm2e


% \usepackage[capitalise]{cleveref}
\title[Confidence Estimation via Sequential Likelihood Mixing]{Confidence Estimation via Sequential Likelihood Mixing}
% \title[Sequential Confidence Estimation]{Universal Confidence Estimation via Sequential Likelihood Mixing}
% \title[Sequential Confidence Estimation]{Universal Confidence Estimation via Sequential Likelihood Mixing\\Applications, Connections and New Perspectives}
\usepackage{times}

% Custom package with our own macros %Michele
\usepackage{mystyle}
\usepackage{algorithm2e}
\usepackage{subcaption}
\usepackage{caption}


% \makeatletter
% \renewenvironment{quotation}
%                {\list{}{\listparindent=15pt%whatever you need
%                         \itemindent    \listparindent
%                         \leftmargin=10pt%  whatever you need
%                         \rightmargin=10pt%whatever you need
%                         \topsep=5pt%%%%%  whatever you need
%                         \parsep        \z@ \@plus\p@}%
%                 \item\relax}
%                {\endlist}
% \makeatother

% Use \Name{Author Name} to specify the name.
% If the surname contains spaces, enclose the surname
% in braces, e.g. \Name{John {Smith Jones}} similarly
% if the name has a "von" part, e.g \Name{Jane {de Winter}}.
% If the first letter in the forenames is a diacritic
% enclose the diacritic in braces, e.g. \Name{{\'E}louise Smith}

% Two authors with the same address
% \coltauthor{\Name{Author Name1} \Email{abc@sample.com}\and
%  \Name{Author Name2} \Email{xyz@sample.com}\\
%  \addr Address}

% Three or more authors with the same address:
% \coltauthor{\Name{Author Name1} \Email{an1@sample.com}\\
%  \Name{Author Name2} \Email{an2@sample.com}\\
%  \Name{Author Name3} \Email{an3@sample.com}\\
%  \addr Address}

% Authors with different addresses:
\coltauthor{%
 \Name{Johannes Kirschner} \Email{johannes.kirschner@sdsc.ethz.ch}\\
 \addr Swiss Data Science Center%, Andreasstrasse 5, 8092 Zürich, Switzerland
 \AND
 \Name{Andreas Krause} \Email{krausea@ethz.ch}\\
 \addr ETH Zurich, Department of Computer Science%
 \AND
 \Name{Michele Meziu} \Email{mezium@student.ethz.ch}\\
 \addr ETH Zurich, Department of Computer Science%
 \AND
 \Name{Mojmir Mutný} \Email{mmutny@inf.ethz.ch}\\
 \addr ETH Zurich, Department of Computer Science%
}


% todonotes
\usepackage[textsize=tiny]{todonotes}
\usepackage{layouts}
\setlength{\marginparwidth}{2cm}
\newcommand{\todom}[2][]{\todo[size=\scriptsize,color=blue!20!white,#1]{M: #2}}
\newcommand{\todoj}[2][]{\todo[size=\scriptsize,color=green!20!white,#1]{J: #2}}
\newcommand{\todomo}[2][]
{\todo[size=\scriptsize,color=purple!20!white,#1]{Mo: #2}}

\newtheorem{assumption}{Assumption}

\begin{document}

\maketitle

\begin{abstract}%
We present a universal framework for constructing confidence sets based on sequential likelihood mixing. Building upon classical results from sequential analysis, we provide a unifying perspective on several recent lines of work, and establish fundamental connections between sequential mixing, Bayesian inference and regret inequalities from online estimation. The framework applies to any realizable family of likelihood functions and allows for non-i.i.d.~data and anytime validity. Moreover, the framework seamlessly integrates standard approximate inference techniques, such as variational inference and sampling-based methods, and extends to misspecified model classes, while preserving provable coverage guarantees. We illustrate the power of the framework by deriving tighter confidence sequences for classical settings, including sequential linear regression and sparse estimation, with simplified proofs.
\end{abstract}

\begin{keywords}%
  confidence sequences, likelihood ratios, Bayesian inference, online estimation
\end{keywords}

\documentclass[../main.tex]{subfiles}
\graphicspath{{../images/}}
\makeatletter
\def\input@path{{../images/}}
\makeatother
\begin{document}
\section{Introduction}
\begin{figure}
\centering
\begin{tikzpicture}
\node[inner sep=0pt] (ws) at (0, 0) {
\includegraphics[height=.4\textwidth, trim={10cm 0 10cm 0},clip]{world_space.png}};
\node[inner sep=0pt] (cs) at (6,0) {\includegraphics[height=.4\textwidth, trim={10cm 1cm 10cm 4cm},clip]{conf_space.png}};
\end{tikzpicture}
\vspace{-5pt}
\label{fig:pbrm_intro}
\caption{\textbf{Left}: Shows world space obstacles as grey spheres. Robots start and goal configuration is colored red and green, respectively. Configurations along the computed path are colored transparent blue. \textbf{Right:} Mapped world space scenario to configuration space. Obstacle region is the grey mesh. Red spheres are collision-free regions computed by the neural SCDF. The optimized shortest path in the convex corridor is the blue curve.}
\vspace{-25pt}
\end{figure}
Motion planning is the problem of finding a collision-free trajectory that connects a given start and goal configuration. The planning takes place in the configuration space of the robot. For single body robots, like mobile robots or drones, the configuration space and the world space are usually the same. This simplifies the planning, since explicit obstacle representations are available which enables geometrical tools like separating hyperplanes, smallest distance to obstacles etc., to be used when designing motion planning algorithms. For multi-body robots like manipulators, the situation is completely different. The world space obstacles are usually mapped to non-convex regions, and to make the problem even harder, the mapping is usually not known. Forming explicit representations of the obstacle region in the configuration space is usually too expensive or intractable. Despite all of this, sampling based planners are used with great success, which mainly is due to their use of implicit representations of the obstacle region. The basic idea is to construct a graph in the configuration space that covers and connects the collision-free region. From this graph, a path can be extracted that connects a given start and goal configuration. The approach is computationally expensive, since the graph is constructed with the smallest geometrical building block available, points, which represents a collision-check. Furthermore, the extracted paths from the graph are non-smooth and jagged due to the stochastic nature of the approach. This adds an additional post-processing step to the process, where the paths are shortcutted and smoothened, before the path can be used for tracking. Clearly a lot of time is invested to form this graph and produce smooth paths. Thus, if the obstacles start to move, then all of this work is done in no use, since all points that make up this graph need to be re-verified, which is simply too time consuming to be done in real time.
\\\\
In this work, we want to address the existing drawbacks of the sampling based planners. Our main contribution is an improved motion planner where each vertex in the graph covers a collision-free region in the form of a sphere instead of a point and where the edges are formed with neighboring intersecting spheres. This representation has the advantage of instead of returning piecewise linear paths, returning a sequence of overlapping spheres, i.e. a convex corridor, that connects a given start and goal configuration, illustrated in Figure \ref{fig:pbrm_intro}. This convex corridor allows us to use convex optimization to produce smooth trajectories, instead of computationally expensive post-processing methods. The representation further allows us to estimate the coverage of the collision-free space, which gives us awareness and feedback in the offline roadmap construction phase. Finally, our representation is simple to adapt to moving obstacles, simply requery for the new radii and recheck for intersections. 
\\\\
The spherical collision-free regions are formed using a signed distance function (SDF), which is a function that returns the smallest distance from an arbitrary point to the boundary of an obstacle. As the name implies, the distance is signed, thus if the point is inside the obstacle it is negative otherwise positive. If the distance is positive, a sphere with radius equal to the distance is guaranteed to cover a collision-free region. Using an SDF in motion planning is not new, but what is novel about our approach is that we express the distance in the configuration space instead of the world space and by doing so allows us to form these convex collision-free regions. We refer to the resulting SDF as a signed configuration distance function (SCDF). Computing an SCDF analytically is non-trivial, our approach is therefore to parameterize the SCDF with a deep neural network and learn the mapping by supervised learning. Our resulting neural SCDF can compute distances for different parameter values of obstacle shapes and we also show how multiple distances can be combined, thus making our approach flexible.
\section{Related work}
Motion planning algorithms can roughly be divided into three families, grid-based, sampling based and optimization based methods. Grid-based methods (GBM) discretize the planning space from which a graph is then compiled. A standard search method is A$^\star$ \citep{a_star}, which is classified as an \textit{informed} search method, since it employs a heuristic function to speed up the search. A$^\star$ guarantees to return an optimal path at the level of discretization used. GBMs usually discretize the planning space by a regular lattice and this limits the GBMs to problems with low dimensionality due to the curse of dimensionality. Thus, GBMs are usually limited to single-body robots where the degrees of freedom (DOF) are low. To overcome the inherent scaling problem with the GBMs, stochastic methods are usually used for multi-body robots. These methods are termed as sampling-based methods (SBM) and core members within this family are the rapidly-exploring random trees (RRT) \citep{rrt} and the probabilistic roadmap (PRM) \citep{prm}. RRT grows a tree from the start configuration and explores the collision-free region in a rapid way until it is able to connect to the goal region. RRT is usually improved by bi-directional planning \citep{rrt_connect}, i.e. an additional tree is grown from the goal configuration and the trees are tested for connection after any tree has been expanded. RRT is a single-query method, thus it searches for a path from scratch each time it is queried. Contrary to this, PRM is a multi-query method, which solves for multiple queries without starting from scratch. PRM does this by creating a roadmap (graph) that covers the collision-free space as an offline step. The graph is then used to solve for multiple queries. PRMs are used in cases where the environment does not change since the extra offline step is too computationally costly and needs to be re-done if the environment is changed. In our work, we address this inherent issue by using a different roadmap representation. Our vertices in the graph cover a collision-free region in the form of spheres and we form the edges by checking for intersecting spheres. If something in the environment changes, we recompute the spheres radii and recheck the intersections, without relying on collision detection. We use a trained neural network to compute the sphere radius, therefore querying for the radius can be done fast, hence our representation enables the PRM for dynamic environments.
\\\\
In the recent decades, optimization based methods (OBM) \citep{chomp, schulman, itomp, stomp} have been introduced as an alternative to SBM for multi-body robots. Like the SBM, the OBMs scale well to higher dimensional problems and produce smoother motion. It is common to use a SDF in the optimization since it is a smooth function, thus enabling gradient-based methods. However, the standard way of expressing the SDF is in world space. The distance therefore needs to be mapped to the configuration space by the forward kinematics. This mapping makes the optimization problem a non-linear program (NLP), which is computationally expensive to solve. Recently, a different approach has been proposed. In \cite{mp_gcs} motion planning is formulated as a convex optimization problem by using the graph of convex sets framework \citep{gcs}. The underlying idea is to decompose the collision-free space into intersecting convex sets from which a convex optimization problem is formulated. In cases where an explicit representation of the obstacles in the configuration space exists, like for single-body robots, creating collision-free convex regions can be done fast \citep{iris}. For multi-body robots, this is non-trivial. Existing work does this successfully \citep{iris_nlp, iris_c} by an optimization based approach, but the methods are still too time consuming to be used in the presence of moving obstacles. Our approach is instead to use deep learning to learn an SDF expressed in the configuration space. With this, we can query for shortest distances to the collision boundary, which allows us to expand spherical regions which are collision-free. Our approach is fast and therefore enables our suggested roadmap planner to be used in dynamic environments.
\\\\
Recent research has focused on learning collision detection \citep{fk_kernel_distance, diffco, graphdistnet} by predicting the signed distance between the robot links and the surrounding obstacles in the world space. The learned SDF is used in trajectory optimization but since the distance is expressed in the world space, the problem becomes an NLP and therefore takes a long time to solve. We take a novel approach and suggest to instead express the signed distance in the configuration space. This allows us to improve the PRM at the same time as it enables convex optimization for trajectory optimization, which runs faster and is more reliable than NLP solvers. In \cite{cspf} a learned signed distance function in the configuration space is proposed similar to our approach. However, their approach is restricted to point cloud representations, while we propose to represent the obstacles as parameterized geometric shapes, e.g. spheres. Furthermore, we also show how to use our learned SCDF to improve an existing roadmap planner.
\section{Problem formulation}
A robot is located in the world space, $\W \subset \R^3 $. The unique location of the robot is given by its configuration $\q \in \C$, where $\C$ is the configuration space. The set of points covered by the robots bodies at a certain configuration is expressed as $\B(\q) \subset \W$. The robot is surrounded by $\NrObst$ obstacles $\O = \bigcup_{i=1}^{\NrObst} \O_i$, where  $\O_i \subset \W$. The representation of the obstacle in the configuration space is the set $\C\O_i = \{\q \in \C \: |\: \B(\q) \cap \O_i \neq \emptyset \}$. The obstacle space is formed as $\Co = \bigcup_{i=1}^{\NrObst} \C \O_i$. The complement is referred to as the free space, $\Cf = \C \setminus \Co$. The path planning problem is a tuple, ($\Cf$, $\qStart$, $\qGoal$), where we want to connect a query pair, consisting of a start, $\qStart$, and goal configuration, $\qGoal$, with a geometric path, $\q(s): [0, 1] \mapsto \Cf$, such that $\q(0)=\qStart$ and $\q(1)=\qGoal$, or report correctly when such a path does not exist.
\end{document}

% We study (stochastic) gradient descent on the empirical risk
\begin{equation*}
\cL(w) = \frac{1}{n}\sum_{i=1}^n l(p_i(w))\, ,
\end{equation*}
where the loss function $l$ and the functions  $(p_i)_{i=1}^n$  are specified in the following assumptions. Note that the empirical risk for binary classification from Equation~\eqref{def:emp_risk_intro} is a special case of the above objective.

\begin{assumption}\label{hyp:loss_exp_log}\phantom{=}
  \begin{enumerate}[label=\roman*)]
    \item The loss is either the exponential loss, $l(q) = e^{-q}$, or the logistic loss, $l(q) = \log(1{+}e^{-q})$.
    \item There exists an integer $L \in \mathbb{N}^*$  such that, for all $1 \leq i \leq n$, the function $p_i$ is $L$-homogeneous\footnote{We recall that a mapping $f : \mathbb{R}^d \rightarrow \mathbb{R}$ is positively $L$-homogeneous if $f(\lambda w) = \lambda^L f(w)$ for all $w \in \mathbb{R}^d$ and $\lambda >0$.}, locally Lipschitz continuous and semialgebraic.
  \end{enumerate}
\end{assumption}
If the $p_i$'s were differentiable with respect to $w$, the chain rule would guarantee that
\begin{align*}
\nabla \mathcal{L}(w) = \frac{1}{n}\sum_{i=1}^n  l'(p_i(w)) \nabla p_i(w)\enspace.
\end{align*}
However, we only assume that the $p_i$'s are semialgebraic. While we could consider Clarke subgradients, the Clarke subgradient of operations on functions (e.g., addition, composition, and minimum) is only contained within the composition of the respective Clarke subgradients. This, as noted in Section~\ref{sec:cons_field}, implies that the output of backpropagation is usually not an element of a Clarke subgradient but a selection of some conservative set-valued field.
Consequently, for $1\leq i \leq n$, we consider $D_i : \bbR^d \rightrightarrows\bbR^d$, a conservative set-valued field of $p_i$, and a function $\sa_i : \bbR^d \rightarrow \bbR^d$ such that for all $w \in \bbR^d$, $\sa_i(w) \in D_i(w)$. Given a step-size $\gamma >0$, gradient descent (GD)\footnote{More precisely, this refers to conservative gradient descent. We use the term GD for simplicity, as conservative gradients behave similarly to standard gradients.} is then expressed as
\begin{equation*}\label{eq:gd_new}\tag{GD}
  w_{k+1} = w_k - \frac{\gamma}{n} \sum_{i=1}^n l'(p_i(w_k))\sa_i(w_k)\,.
\end{equation*}
For its stochastic counterpart, stochastic gradient descent (SGD), we fix a batch-size $1\leq n_b \leq n$. At each iteration $k \in \bbN$, we randomly and uniformly draw a batch $B_k \subset \{1, \ldots, n \}$ of size $n_b$. The update rule is then given by 
\begin{equation*}\label{eq:sgd_new}\tag{SGD}
  w_{k+1} = w_k -  \frac{\gamma}{n_b}\sum_{i\in B_k} l'(p_i(w_k)) \sa_i(w_k)\, .
\end{equation*}
The considered conservative set-valued fields will satisfy an Euler lemma-type assumption.
%\nic{Smoother transition}
\begin{assumption}\phantom{=}\label{hyp:conserv}
  For every $i \leq n$, $\sa_i$ is measurable and $D_i$ is semialgebraic. Moreover, for every $w \in \bbR^d$ and $\lambda \geq 0$, $\sa_i(w)  \in D_i(w)$,
  \begin{equation*}
    D_i(\lambda w) = \lambda^{L-1} D_i(w)\, , \textrm{ and } \quad   L p_i(w) = \scalarp{\sa_i(w)}{w}\, .
  \end{equation*}
\end{assumption}
%\nic{Smoother transition}
Having in mind the binary classification setting, in which $p_i(w) = y_i \Phi(x_i, w)$, we define the margin
\begin{equation}\label{def:marg}
  \sm: \bbR^d \rightarrow \bbR, \quad \sm(w) = \min_{1\leq i \leq n} p_i(w)\, .
\end{equation}
It quantifies the quality of a prediction rule $\Phi(\cdot, w)$. In particular,  the training data is perfectly separated when $\sm(w) >0$. A binary prediction for $x$ is given by the sign of $\Phi(x, w)$, and under the homogeneity assumption, it depends only on the normalized direction $w / \norm{w}$. Consequently, we will focus on the sequence of directions $u_k := w_k / \norm{w_k}$. Our final assumption ensures that the normalized directions $(u_k)$ have stabilized in a region where the training data is correctly classified.

\begin{assumption}\label{hyp:marg_lowb}
  Almost surely, $\liminf \sm(u_k) >0$.
\end{assumption}
Before presenting our main result, we comment on our assumptions.

\paragraph{On Assumption~\ref{hyp:loss_exp_log}.} As discussed in the introduction, the primary example we consider is when $p_i(w) = y_i \Phi(x_i;w)$ is the signed prediction of a feedforward neural network without biases and with piecewise linear activation functions on a labeled dataset $((x_i,y_i))_{i \leq n}$. In this case,
\begin{equation}\label{eq:NN}
 p_i(w) = y_i \Phi(w;x_i) = y_i V_L(W_L) \sigma(V_{L-1}(W_{L-1}) \sigma(V_{L-1}(W_{L-2}) \ldots \sigma(V_{1}(W_1 x_i))))\, ,
\end{equation}
where $w = [W_1, \ldots, W_L]$, $W_i$ represents the weights of the $i$-th layer, $V_i$ is a linear function in the space of matrices (with $V_i$ being the identity for fully-connected layers) and $\sigma$ is a coordinate-wise activation function such as $z \mapsto \max(0,z)$ ($\ReLU$), $z \mapsto \max(az, z)$ for a small parameter $a>0$ (LeakyReLu) or $z \mapsto z$. Note that the mapping $w \mapsto p_i(w)$ is semialgebraic and $L$-homogeneous for any of these activation functions. Regarding the loss functions, the logistic and exponential losses are among the most commonly studied and widely used. In Appendix~\ref{app:gen_sett}, we extend our results to a broader class of losses, including $l(q) = e^{-q^a}$ and $l(q) = \ln (1 + e^{-q^a})$ for any $a \geq 1$.

\paragraph{On Assumption~\ref{hyp:conserv}.} Assumption~\ref{hyp:conserv} holds automatically  if $D_i$ is the Clarke subgradient of $p_i$. Indeed, at any vector $w \in \bbR^d$, where $p_i$ is differentiable it holds that $p_i(\lambda w) = \lambda^{L} p_i(w)$. Differentiating relatively to $w$ and $\lambda$ (noting that $p_i$ remains differentiable at $\lambda w$ due to homogeneity), we obtain $\lambda \nabla p_i(\lambda w) = \lambda^{L} \nabla p_i(w)$ and $\scalarp{\nabla p_i(\lambda w)}{w} = L \lambda^{L-1} p_i(w)$. The expression for any element of the Clarke subgradient then follows from~\eqref{eq:def_clarke}. 

However, for an arbitrary conservative set-valued field, Assumption~\ref{hyp:conserv} does not necessarily hold. For instance, $D(x) = \mathds{1}(x \in \mathbb{N})$ is a conservative set-valued field for $p \equiv 0$, which does not satisfy Assumption~\ref{hyp:conserv}. Nevertheless, in practice, conservative set-valued fields naturally arise from a formal application of the chain rule. For a non-smooth but homogeneous activation function $\sigma$, one selects an element $e \in \partial \sigma (0)$, and computes $\sa_i(w)$ via backpropagation. Whenever a gradient candidate of $\sigma$ at zero is required (i.e., in~\eqref{eq:NN}, for some $j$, $V_j(W_j)$ contains a zero entry), it is replaced by $e$. 
Since $V_j(W_j)$ and $V_j(\lambda W_j)$ have the same zero elements, it follows that for every such $w$, $
\sa_i(\lambda w) = \lambda^L \sa_i(w)$. The conservative set-valued field $D_i$ is then obtained by associating to each $w$ the set of all possible outcomes of the chain rule, with $e$ ranging over all elements of $\partial \sigma(0)$. Thus, for such fields, Assumption~\ref{hyp:conserv} holds.


\paragraph{On Assumption~\ref{hyp:marg_lowb}.} Training typically continues even after the training error reaches zero.
Assumption~\ref{hyp:marg_lowb} characterizes this late-training phase, where our result applies. 
As noted earlier, since $\sm$ is $L$-homogeneous, the classification rule is determined by the direction of the  iterates $u_k=w_k/\norm{w_k}$. Assumption~\ref{hyp:marg_lowb} then states that, beyond some iteration, the normalized margin remains positive. 
This assumption is natural in the context of studying the implicit bias of SGD: we \emph{assume} that we reached the phase in which the dataset is correctly classified and \emph{then} characterize the limit points. A similar perspective was taken in  \cite{nacson2019lexicographic}, where the implicit bias of GF was analyzed under the assumption that the sequence of directions and the loss converge. However, unlike their approach, ours does not require assuming such convergence a priori.

Earlier works such as \cite{ji2020directional,Lyu_Li_maxmargin}, which analyze subgradient flow or smooth GD, establish convergence by assuming the existence of a single iterate $w_{k_0}$ satisfying $\sm(w_{k_0}) > \varepsilon$ and then proving that $\lim \sm(u_{k}) > 0$. Their approach relies on constructing a smooth approximation of the margin, which increases during training, ensuring that $\sm(u_k) > 0$ for all iterates with $k \geq k_0$. This is feasible in their setting, as they study either subgradient flow or GD with smooth $p_i$’s, allowing them to leverage the descent lemma.

In contrast, our analysis considers a nonsmooth and stochastic setting, in which, even if an iterate $w_{k_0}$ satisfying $\sm(w_{k_0}) > \varepsilon$ exists, there is no a priori assurance that subsequent iterates remain in the region where Assumption~\ref{hyp:marg_lowb} holds. From this perspective, Assumption~\ref{hyp:marg_lowb} can be viewed as a stability assumption, ensuring that iterates continue to classify the dataset correctly. Establishing stability for stochastic and nonsmooth algorithms is notoriously hard, and only partial results in restrictive settings exist \cite{borkar2000ode,ramaswamy2017generalization,josz2024global}.

%Finally, note that Assumption~\ref{hyp:marg_lowb} only needs to hold almost surely. Specifically, with probability 1, there exist $k_0$ and $\varepsilon$ such that for all $k \geq k_0$, $\sm(u_k) \geq \varepsilon > 0$. In the case of~\eqref{eq:sgd_new}, $k_0$ and $\delta$ are random variables and may take different values across different realizations. 

%\paragraph{On constant stepsizes.}
%We allow the step size to be a constant of arbitrary magnitude, subject to the stability Assumption~\ref{hyp:marg_lowb}. This may seem surprising in a nonsmooth and stochastic setting, where a vanishing step size is typically required to ensure convergence (see, e.g., \cite{majewski2018analysis, dav-dru-kak-lee-19, bolte2023subgradient, le2024nonsmooth}).
% TODO more details in methodology and data processing
% merge methodology and 
\section{Framework for Analyzing Emotion}
In this section, we present our framework for analyzing emotion. We first establish a basic understanding of emotion polarity by determining the sentiment valence of each root tweet and comment. We then use multi-label emotion detection to predict the emotion categories associated with each post. Based on this data, we explore the interactive nature of emotions, by identifying common patterns in emotion transition pairs between temporally-adjacent posts. Finally we investigate the emotional trajectory within threads to understand how emotional intensity and type shift over time, by aggregating the predicted labels for posts at each time stamp in a given thread. As part of this, we contrast rumour with non-rumour threads, to gain a holistic understanding of emotional expression in rumours and non-rumours on Twitter.

% elaborate a bit on why we choose EmoLLM, compared with other automatic emotion detection methods
\paragraph{Affective Computing: Automatic Emotion Detection}
Manually annotating emotions is both costly and time-consuming, so we use an LLM-based emotion detection model, EmoLLM~\citep{liu2024emollms}, which is specifically designed for sentiment analysis and emotion detection. The model was instruction-tuned on SemEval 2018 Task1 using a comprehensive emotion labeling scheme grounded in established theoretical frameworks. We prompt the model to perform Valence Ordinal Classification (V-oc), Emotion Classification (E-c), and Emotion Intensity regression (E-i). Detailed prompts are shown in \Cref{tab:emollm_ins}.

\paragraph{Categorical Emotion Labeling Scheme} \label{para:emotion_label}
Numerous emotion label sets  have been proposed~\citep{Ekman1992AnAF, Plutchik1980AGP, Russell1980ACM}. According to \citet{Ekman1992AnAF, Plutchik1980AGP}, certain emotions, such as joy, fear, and sadness, are considered more fundamental than others, both physiologically and cognitively. The Valence-Arousal-Dominance (VAD) model \citep{Russell1980ACM} categorizes emotions within a three-dimensional space of valence (positivity-negativity), arousal (active-passive), and dominance (dominant-submissive). Inspired by \citet{mohammad-etal-2018-semeval}, we incorporate elements from both basic emotion theories and the VAD model, and further ground EmoLLM emotion predictions to develop the following emotion label schemes: (1) \textit{neutral or no emotion}; (2) \textit{negative emotions}: anger (also includes annoyance and rage),  disgust (also includes disinterest, dislike, and loathing), fear (also includes apprehension, anxiety, and terror), pessimism (also includes cynicism, and no confidence), sadness (also includes pensiveness and grief); 3) \textit{positive emotions}: joy (also includes serenity and ecstasy), love (also includes affection), optimism (also includes hopefulness and confidence), anticipation (also includes interest and vigilance), surprise (also includes distraction and amazement) and trust (also includes acceptance, liking, and admiration). 


\paragraph{Emotion Polarity: Sentiment Valence} 
To understand the basic emotion polarity expressed in rumour and non-rumour content, we begin with sentiment valence analysis. Sentiment valence aims to capture the overall emotional tone conveyed by a post, in terms of how positive or negative it is~\citep{liu2024emosurvey}. We frame the sentiment valence task as ordinal regression~\citep{mohammad-etal-2018-semeval}. As shown in \Cref{tab:emollm_ins}, for a given tweet post, we classify it into one of seven ordinal levels of sentiment intensity, spanning varying degrees of positive and negative valence, that best represents the tweeter's mental state. The tweet posts within a thread can be divided into two categories: root tweets, which are posted by the publisher, and follow posts, which include all subsequent replies under the root post. We begin by conducting sentiment valence analysis on each post within the thread conversation. 
% TJB: confused by how comments can include all subsequent replies; we seem to be overloading the terminology, for comments to be both individual posts and series of posts
% RX: yes, I am unifiying all terms.
For each category, we compute the mean sentiment valence to enable further investigation into the specific emotions associated with different sentiment valences over a thread.
% TJB: clarify for comments whether the classification is done over the combined meta-document (i.e. the root + all comments to that point) or individually over the separate documents and then combined ... or over individual documents, in which case the statement about "all replies" needs clarification
% RX: we separate root and comments for each tweet conversation, the former is the root tweet posted by the publisher while the rest are comments. "all replies" mean all comments under root tweet, we aggregate them by computing the mean sentiment, and then average over each part.

\paragraph{Emotion Distribution} 
Following sentiment valence analysis, we then examine specific emotions and their distribution in rumour and non-rumour tweet posts.
Motivated by the fact that a certain tweet might exhibit more than one emotion, we frame the task as multi-label emotion detection problem. As shown as V-oc in \Cref{tab:emollm_ins}, given a tweet, we classify it into one of seven ordinal classes, corresponding to various levels of positive and negative sentiment intensity. To reduce noise from automatic emotion detectors, we take the top-three predicted emotions for each tweet. We then aggregate and plot the emotion distribution to provide an overview of dominant emotional trends across the rumour and non-rumour posts. Given that the follow posts make up the majority of the data compared to the root posts, we will focus on using follow posts in our next analysis.
% TJB: what is the basis of saying that the signal is richer? simply that there are more reply posts than root posts? clarify
% RX: yes, and we are more interested in interaction in comments.

\paragraph{Emotion Transitions} 
Emotions are contagious and highly interactive~\citep{Ferrara_2015}. When publishers write tweets that convey their emotions, readers are likely to respond with emotional reactions of their own~\citep{Ferrara_2015,emotion_dynamics}. In this part, we model this interactive nature of emotions in the form of emotion transition pairs, which are built from two chronologically-adjacent tweets. In each pair, the first element represents the emotion inferred from the initial content published at a given time, and the second element represents the emotion inferred from the reply content published immediately after. For example, if the first tweet exhibits \textit{joy} \textit{trust} and \textit{anticipation}, and the second tweet shows \textit{anger}, \textit{disgust} and \textit{surprise}, we form the pairs (\textit{joy}, \textit{anger}), (\textit{joy}, \textit{disgust}), (\textit{joy}, \textit{surprise}), (\textit{trust}, \textit{surprise}), (\textit{trust}, \textit{surprise}), (\textit{trust}, \textit{disgust}), (\textit{anticipation}, \textit{anger}), (\textit{anticipation}, \textit{surprise}) and (\textit{anticipation}, \textit{disgust}). We create transitions for all combinations of emotion pairs and explore the likelihood of emotion transition pairs occurring in rumour and non-rumour content. Exploring emotion transitions allows us to understand the emotional flow in social media conversations and uncover typical patterns of rumour and non-rumour content, and any differences between the two.

\paragraph{Emotion Trajectories} 
We explore the cumulative trajectory of emotion over time to observe how emotions evolve during the conversational thread. We collect all detected emotion labels for each tweet from both rumour and non-rumour content, then track cumulative emotion counts at each chronological step. Finally, we visualize these trends and apply regression models to analyze the growth of emotions over time. This temporal analysis reveals how emotions accumulate or intensify across time, offering insight into the trajectory of emotions in rumour and non-rumour content.

\begin{table*}[!h]
    \centering
    \small
    \begin{tabular}{cccccccccccc}
        \toprule
        \textbf{Setting} & \textbf{Ru} & \textbf{Non} & \textbf{p} & \textbf{\#Ru/Non} & \textbf{T} & \textbf{F} & \textbf{U} & \textbf{$p$ (U vs T)} & \textbf{$p$ (U vs F)} & \textbf{\#T/\#F/\#U} \\
        \midrule
        \textbf{PHEME root} & \textbf{$-$0.25} & $-$0.17 & 0.00 & 2602/2602 & $-$0.21 & $-$0.11 & \textbf{$-$0.39} & 7.75e-11 & 4.41e-11 & 629/629/629 \\
        \textbf{PHEME follow} & \textbf{$-$0.33} & $-$0.26 & 6.47e-09 & & $-$0.35 & $-$0.20 & \textbf{$-$0.39} & 0.03 & 8.38e-15 & \\
        \textbf{Twitter15 root} & \textbf{$-$0.26} & $-$0.01 & 3.51e-05 & 372/372 & $-$0.21 & $-$0.20 & \textbf{$-$0.34} & 0.01 & 0.01 & 359/359/359 \\
        \textbf{Twitter15 follow} & \textbf{$-$0.27} & $-$0.06 & 1.65e-09 & & $-$0.24 & $-$0.25 & \textbf{$-$0.30} & 0.16 & 0.21 & \\
        \textbf{Twitter16 root} & \textbf{$-$0.18} & \z0.07 & 0.00 & 205/205 & \z0.11 & $-$0.22 & \textbf{$-$0.30} & 1.35e-06 & 0.18 & 63/63/63 \\
        \textbf{Twitter16 follow} & \textbf{$-$0.31} & $-$0.12 & 9.19e-06 & & $-$0.30 & \textbf{$-$0.36} & $-$0.27 & 0.67 & 0.90 & \\
        % \textbf{CoAID root} & \textbf{$-$0.34} & $-$0.16 & 0.01 & 167/167 & - & - & - & - & - & - \\
        % \textbf{CoAID follow} & \textbf{$-$0.24} & $-$0.13 & 0.01 & & - & - & - & - & - & \\
        \bottomrule
    \end{tabular}
    \caption{Valence Ordinal Regression results for all datasets. root = root posts, follow = follow posts, Ru = rumour, Non = Non-rumour, T = True rumour, F = False rumour, U = Unverified rumour; $p$ values indicates significance of a one-tailed t-test.}
\label{tab:voc_results}
\end{table*}

\begin{algorithm}[ht!] 
\caption{PC Algorithm}
\label{pc}
\begin{algorithmic}[1] 
\State \textbf{Input:} Data $\mathbf{X}$, significance level $\alpha$
\State \textbf{Output:} Completed Partially Directed Acyclic Graph (CPDAG)

\State Initialize a complete undirected graph $G$ with all variables as nodes.

\State \textbf{Step 1: Skeleton Identification}
\For{each pair of variables $(X, Y)$ in $G$}
    \State Find the subset $S \subseteq \text{Adj}(X, G) \setminus \{Y\}$ such that 
    $X \indep Y \mid S$ with significance $\alpha$.
    \If{such a subset $S$ exists}
        \State Remove the edge $X - Y$ from $G$.
    \EndIf
\EndFor

\State \textbf{Step 2: Edge Orientation}
\For{each triple of variables $(X, Y, Z)$ in $G$ where $X - Z - Y$ and $X, Y$ are not adjacent}
    \If{$Z \notin S$ for all separating sets $S$ for $X$ and $Y$}
        \State Orient as $X \to Z \leftarrow Y$ (identify a collider).
    \EndIf
\EndFor

\While{possible}
    \For{each edge $(X - Y)$ in $G$}
        \If{there exists a directed path $X \to \dots \to Z$ such that $Z - Y$}
            \State Orient as $X \to Y$ (acyclicity rule).
        \ElsIf{orienting $X - Y$ as $X \to Y$ creates a new v-structure}
            \State Orient as $X \to Y$ (v-structure rule).
        \EndIf
    \EndFor
\EndWhile

\State \textbf{return} the CPDAG representing the equivalence class of causal graphs.

% how we frame the task, compute the emotion intensity, how to aggregate on conversation level

\end{algorithmic}
\end{algorithm}


\paragraph{Causal Relationship of Emotions in Rumour \& Non-Rumour Threads}
To gain a deeper insight into the relationship between rumours and the emotions underlying them, we extend our analysis beyond statistical correlation by conducting a causal analysis. Specifically, we apply the Peter-Clark (PC) algorithm \cite{Spirtes2000}, a classical constraint-based causal discovery algorithm on the three merged datasets. 

Uncovering causal relations between variables of interest is never an easy problem. Under the fundamental assumption of \textit{causal Markov condition} that a variable is conditionally independent of all its non-effects given its direct cause, \textit{faithfulness} ensures that the casual graph exactly encodes the independence and conditional independence relations among variables. These two assumptions allow us to infer causal relationships from observed statistical independencies, forming the cornerstone of constraint-based causal discovery methods. 

The PC algorithm identifies causal relationships among the variables of interest, represented as a directed acyclic graph (DAG), by numerating the independence and conditional independence relationships. The algorithm consists of two main steps: 
\begin{enumerate}
    \item \textbf{Skeleton Identification}: Starting with a complete undirected graph where all variables are connected, edges are iteratively removed based on conditional independence and independence relationships among variables, inferred by a conditional independence test. This step returns an undirected graph, which we call a skeleton. 
    \item \textbf{Edge Orientation}: After constructing the skeleton, edges are oriented by a set of predefined rules (Meek's Rule \cite{meek1997graphical}) to avoid cycles and orient collider structures.
\end{enumerate}

The complete PC algorithm is provided in algorithm \ref{pc}. It returns a  completed partially directed acyclic graph (CPDAG), which represents an equivalence class of causal graphs that are consistent with the observed data’s independence and conditional independence relations. In our implementation, we adopt the  Fisher-z test \cite{fisher_probable_1921} to infer the conditional independence relations.

%%% Local Variables:
%%% mode: latex
%%% TeX-master: "../main_anonymous"
%%% End:


\subsection{From Bayesian to Frequentist Inference}\label{sec:bayes}

A natural choice for the mixing distribution is the Bayesian posterior, which establishes a fundamental connection between frequentist confidence estimation and Bayesian inference. To explore this relationship, we first formally define the Bayesian inference model.
\begin{assumption}[Bayesian Inference]\label{a:bayes}
    In the Bayesian inference model, the learner defines a prior distribution $\mu_0 \in \sP(\Theta)$ over model parameters (independent of the data), and predicts using the posterior distribution $\mu_t(\theta) \propto \prod_{s=1}^{t-1} p_s(y_s|\theta) \mu_0(\theta)$.
\end{assumption}
The main result of this section establishes that if the mixing distributions are computed according to Bayes' rule, then prior likelihood mixing (\cref{result:prior_mixing}) and sequential likelihood mixing (\cref{result:posterior_mixing}) are equivalent. A further application of Bayes rule shows that any (realizable) Bayesian model can be turned into a $(1-\delta)$-confidence sequence by comparing the log posterior probability $\log \mu_t(\theta)$ to the log prior probability $\log \mu_0(\theta)$. This is known as \emph{prior-posterior ratio confidence set} \citep{waudby2020confidence}: 
\begin{align*}
    C_t =  \left\{ \theta \in \Theta: - \log \mu_t(\theta) \leq  \log \frac{1}{\delta} - \log \mu_0(\theta) \right\} \,.
\end{align*}
The equivalence result is foreshadowed in the works by \citet{waudby2020confidence} and \citet{neiswanger2021uncertainty}, who establish the posterior-ratio confidence set and the connection to the marginal likelihood. The explicit equivalence to the sequential mixing framework, however seems to be absent in prior works, and is formally given in \cref{result:mixing-equivalence} below. 
%As a consequence, the concentration bounds for sequential linear regression by \citet{neiswanger2021uncertainty,flynn2024improved,flynn2024tighter} and earlier work by \cite{abbasi2011improved} are essentially equivalent, as we illustrate below. \todoj{maybe move below Lemma 6, so that 'below' makes sense}

\begin{theorem}[Mixing Equivalence]\label{result:mixing-equivalence} If the mixing distributions are chosen according to Bayes' rule, prior likelihood mixing (\cref{result:prior_mixing}) and sequential mixing (\cref{result:posterior_mixing}) are equivalent.
\end{theorem}
\begin{proof}
   The result follows by applying Bayes' rule recursively to show the following equality, $\sum_{s=1}^t \log \int p_s(y_s|\nu) d\mu_{s-1}(\nu) = \log \int \prod_{s=1}^t p_s(y_s|\nu) d\mu_{0}(\nu)$.
    % \begin{align*}
    %     \sum_{s=1}^t \log \int p_s(y_s|\nu) d\mu_{s-1}(\nu) = \log \int \prod_{s=1}^t p_s(y_s|\nu) d\mu_{0}(\nu) \,. 
    % \end{align*}
\end{proof}

The surprising consequence is, that within the Bayesian inference model, sequential mixing provides no statistical advantage compared to averaging the likelihood over the prior. Less surprisingly though, Bayes' rule can be understood as an incremental update rule to compute the marginal likelihood. In this sense, the equivalence can be re-stated as recovering prior mixing (\cref{result:prior_mixing}) as a special case of sequential mixing (\cref{result:posterior_mixing}). However, note that for mixing distributions outside the Bayesian model, the equivalence does not hold in general, leaving the possibility to find non-Bayesian mixing distributions that achieve faster convergence. We come back to this idea in \cref{sec:oco}.

Next, we state a second implication of Bayes' rule, the prior-posterior ratio confidence set. 
\begin{lemma}[Prior-Posterior Ratio Confidence Set \citep{waudby2020confidence}] \label{lem:posterior_ratio_confidence_set}\\
    For any realizable Bayesian model, the following defines a $(1-\delta)$-confidence sequence:
    % The confidence sequence $C_t = \{\theta \in \Theta: L_t(\theta) \leq \log \frac{1}{\delta} - \log \int \prod_{s=1}^t p_s( y_s|\nu) d\mu_0(\nu)  \}$ can be equivalently written as follows:
\begin{align*}
    C_t &=  \left\{ \theta \in \Theta: - \log \mu_t(\theta) \leq  \log \frac{1}{\delta} - \log \mu_0(\theta) \right\} \,.
\end{align*}
Moreover, the confidence set is equivalent to the construction in \cref{result:prior_mixing,result:posterior_mixing}.
\end{lemma}
\begin{proof}
    Note that $\log \mu_t(\theta) = \log \mu_0(\theta)  + L_t(\theta) - \log \int \prod_{s=1}^t p_s(y_s|\nu) d\mu_{0}(\nu)$ holds for all $\theta \in \Theta$ by Bayes' rule. Substituting the equality into \cref{result:prior_mixing} gives the result.
\end{proof}
The remarkable conclusion is that any realizable Bayesian model can be turned into a frequentist confidence set by thresholding the log posterior probability relative to the log prior probability. As a caveat, it is tempting to think of $C_t$ as a Bayesian credible region, however, the posterior credible probability $\mu_{t-1}(C_t)$ is typically not $1-\delta$. Further, the confidence set, by construction, never rejects parameters in the null set of the prior distribution, unlike in classical Bayesian inference. In any case, a sensible choice is $\Theta = \supp(\mu_0)$, as long as the realizability condition (\cref{a:realizability}) is satisfied, that is, $\theta^* \in \Theta$ defines the true likelihood of the data. For an application of the prior-posterior confidence set to sequential sampling without replacement, we refer to \citet{waudby2020confidence}.

As a consequence of the prior-posterior ratio confidence set and the mixing equivalence, the confidence sets for sequential linear regression by \citet{neiswanger2021uncertainty,flynn2024improved,flynn2024tighter} and earlier work by \cite{abbasi2011improved} are essentially equivalent, as we demonstrate below. Moreover, a lower bound by \citet{lattimore2020bandit} shows that the construction is tight without further assumptions on the data generation distribution. 

\paragraph{Sequential Linear Regression} 
To illustrate the utility of the Bayesian perspective, we consider sequential linear regression with a Gaussian prior and likelihood. To preempt any concerns, we remark that the Gaussian assumption can be relaxed to sub-Gaussian distributions, as we explain in \cref{sec:subgaussian}. Formally, let $\theta^* \in \Theta = \bR^d$, with multivariate Gaussian prior $\cN(\theta_0, V_0^{-1})$ centered at $\theta_0 \in \bR^d$ and prior precision matrix $V_0 \in \bR^{d \times d}$, where commonly $V_0 = \lambda \eye_d \in \bR^{d\times d}$ for a regularizer $\lambda > 0$. The observation likelihood is Gaussian,  $y_t \sim \cN(x_t^\top\theta^*, \sigma^2)$ for a feature vector $x_t \in \bR^d$ and known observation variance $\sigma^2 > 0$. The Gaussian posterior is $\mu_t = \cN(\hat \theta_t^\RLS, V_t^{-1})$, where $\hat \theta_t^\RLS$ is the regularized least squares (RLS) estimate,
\begin{align*}
\hat \theta_t^\RLS = \argmin_{\theta \in \bR^d} \frac{1}{2 \sigma^2} \sum_{s=1}^t \big(\ip{x_s, \theta} - y_s\big)^2 + \frac{1}{2} \|\theta - \theta_0\|_{V_0}^2\,.
\end{align*}
Here, $V_t = \frac{1}{\sigma^2}\sum_{s=1}^t x_s x_s^\top + V_0$ is the posterior precision matrix, and we use the notation $\|\nu\|_A^2 = \nu^\top A \nu$ for $\nu \in \bR^d$ and $A \in \bR^{d\times d}$. The prior and posterior densities are explicitly given as follows:
\begin{align*}
    \mu_0(\theta) &= (2 \pi)^{-2/k} (\det V_0)^{1/2} \exp\big(- \tfrac{1}{2}\|\theta - \theta_0\|_{V_0}^2 \big) \\
    \mu_t(\theta) &= (2 \pi)^{-2/k} (\det V_t)^{1/2} \exp\big(- \tfrac{1}{2}\|\theta - \hat \theta_t^\RLS\|_{V_t}^2 \big)
\end{align*}
Applying \cref{lem:posterior_ratio_confidence_set} with the Gaussian posterior, we get the following $(1-\delta)$-confidence sequence:
\begin{align*}
    C_t^\RLS = \left\{ \theta \in \bR^d : \frac{1}{2}\|\theta - \hat \theta_t^\RLS\|_{V_t}^2 \leq \log \frac{1}{\delta} + \frac{1}{2}\log \det V_t - \frac{1}{2}\log \det V_0 + \frac{1}{2}\|\theta  - \theta_0\|_{V_0}^2 \right\}\,.
\end{align*}
An important feature of the bound is that it scales with the \emph{effective dimension} or \emph{total information gain} $\gamma_t = \frac{1}{2}\log \det V_t - \frac{1}{2}\log \det V_0$ of the data \citep[c.f.~][]{huang2021short}, which can be much smaller than the parameter dimension $d$ \citep{srinivas2009gaussian}. 
Note also that the confidence set does \emph{not} require a known bound on the norm $\|\theta^*\|_2 \leq S$, which is required in all prior work that we are aware of. If such a bound is available, a direct approach is to define the Gaussian prior and posterior directly over the restricted set $\cB_S = \{\theta \in \bR^d : \|\theta\|^2 \leq S\}$. However, in this case, the normalization constant is not easily computed in closed form. Instead, we intersect $C_t^\RLS$ with the norm ball $\cB_S$. Relaxing the confidence set further, and choosing $V_0 = \lambda \eye_d$ and $\theta_0 = 0$, we eventually arrive at
\begin{align*}
    % C_t &\subset \{ \theta \in \Theta : \frac{1}{2 \sigma^2} \|\theta - \hat \theta_t\|_{V_t}^2 \leq \log \frac{1}{\delta} + \log \det V_t - \log \det V_0 + S^2 \}\nonumber\\
    C_t^\RLS \cap \cB_S \subset \left\{ \theta \in \bR^d : \frac{1}{2} \|\theta - \hat \theta_t^\RLS\|_{V_t}^2 \leq \log \frac{1}{\delta} + \frac{1}{2}\log \det V_t - \frac{d}{2}\log \lambda+ \frac{\lambda}{2}S^2 \right\} \,.
\end{align*}
The last line essentially recovers the influential result by \citet{abbasi2011improved}, albeit avoiding a lower-order cross-term, improving the bound by up to a factor of two. 
The proof of \citet{abbasi2011improved} uses the method of mixtures, but mixing the noise martingale $S_t = \sum_{s=1}^t \epsilon_s x_t$ over a centered prior, instead of directly mixing the likelihood ratio. 
More recent work by \cite{flynn2024improved} achieves the tighter result using a similar sequential mixing idea, however, the likelihood framework and connection to Bayesian inference is not mentioned. A direct extension is to heteroscedastic noise, $y_t \sim \cN(x_t^\top\theta^*, \sigma_t^2)$ with known variance $\sigma_t^2$ \citep[c.f.,][]{kirschner2018information}. Another, more involved extension is to unknown mean and variance \cite[c.f.,][]{chowdhury2023bregman}. \looseness=-1

\paragraph{Gaussian Process Regression}
We remark that the confidence set for sequential linear regression can be equivalently stated for non-parametric Gaussian processes regression in infinite-dimensional kernel Hilbert spaces (RKHS) using the `kernel trick'. Our derivation improves (up to a factor of two) the results by \cite{abbasi2012thesis,chowdhury2017kernelized,whitehouse2023sublinear} and recovers more recent results by \cite{neiswanger2021uncertainty,flynn2024tighter}, who do not state the equivalence.

% In particular, we can restate the above confidence set $C_t$ on a separable RKHS space, and project the confidence set onto a specific evaluation $x$, via the reproducing kernel operation $f(x) = f^\top k(\cdot,x)$, to arrive at
% \[  C_t(x) = \{ f(x) | |f(x) - \hat{f}_t(x)| \leq  \} \]
% \todoj{make Gaussian processes explicit}


% Lastly, we remark that discussion extends to the more general class of sub-Gaussian likelihoods, which we discuss in \cref{sec:subgaussian}. 

\subsection{Variational Confidence Sets}\label{sec:elbo}
While the confidence set construction and its relation to Bayesian inference is universal and holds for any prior and family of likelihood functions, the price to pay is that computing the marginal likelihood, or equivalently, the posterior distribution, is intractable in general. Fortunately, approximate inference methods have been well studied in the field of Bayesian inference. 
Our starting point is a variational inequality for the marginal likelihood $\int \prod_{s=1}^t p_s(y_s|\theta)d\mu_0(\theta) = \int \exp(- L_t(\theta)) d\mu_0(\theta)$ and the Kullback-Leibner (KL) divergence, often attributed to \citet{donsker1983asymptotic}. 
\begin{lemma}[Variational Inequality]\label{lemma:variational-kl}
For any two distributions $\mu,\rho \in \sP(\Theta)$ and any measurable function $g : \Theta \rightarrow \bR$,
    \begin{align*}
    \log \int \exp(g(\theta)) d\mu \geq \int g(\theta) d\rho(\theta) - \KL(\rho\|\mu) \,.
    \end{align*}
    Moreover, the inequality becomes an equality for $d\rho(\theta) \propto \exp(g(\theta)) d\mu(\theta)$.
\end{lemma}
The inequality plays a central role in variational inference, and is typically stated as the \emph{evidence lower bound} (ELBO), by specializing \cref{lemma:variational-kl} using $g(\theta) = - L_t (\theta)$ and $\mu = \mu_0$,
\begin{align*}
 \log \int \exp(- L_t(\theta)) d\mu_0 \geq \ELBO_t(\rho) := -\int L_t(\theta) d\rho(\theta) - \KL(\rho\|\mu_0) \,.
\end{align*}
For the given choices, the inequality becomes tight when $\rho = \mu_t$ is the Bayesian posterior.
Variational inference aims at numerically maximizing the evidence lower bound over a parametric family of posterior distributions \citep{jordan1999introduction}, see also \citep{blei2017variational}. In the context of confidence estimation, the key insight is that the variational inequality allows to relax the marginal likelihood that defines the confidence sequence in \cref{result:prior_mixing}. This result has been recently stated by \citet{lee2024improved} in the specialized context of logistic and multinomial bandits, and similar bounds are well-known in the PAC-Bayes literature \citep[e.g.,][]{zhang2006varepsilon,chen2022unified,alquier2024user}. Here, we emphasize the connection to variational inference and the evidence lower bound as a way to define a confidence coefficient with valid anytime coverage.
\begin{theorem}[Evidence Lower Bound Confidence Set]\label{result:elbo_confidence_set}
    For any $\cF_t$-adapted sequence of distributions $\mu_t \in \sP(\Theta)$ and a data-independent prior $\mu_0 \in \sP(\Theta)$, define\looseness=-1
    % the $(1-\delta)$-confidence sequence $C_t = \left\{\theta \in \Theta : L_t(\theta) \leq  \log \frac{1}{\delta} - \log \int \prod_{s=1}^t p_s(y_s|\nu) d\mu_0(\nu) \right\}$, it holds that
    \begin{align*}
		 C_t = \left\{\theta \in \Theta : L_t(\theta) \leq  \log \frac{1}{\delta} - \ELBO_t(\mu_t)  \right\} \,.
	\end{align*}
    Then $C_t$ defines a $(1-\delta)$-confidence sequence. Moreover, if $\rho_t$ is chosen as the Bayesian posterior, the result is equivalent to the marginal likelihood confidence set in \cref{result:prior_mixing}.
    % Moreover, if $\rho$ is chosen as the Bayesian posterior, the inclusion becomes an equality.
\end{theorem}
The practical implication of this result is that it provides a tool to trade off computational tractability and statistical efficiency. In particular, standard variational inference methods can be converted into a confidence set with provable coverage, simply by thresholding the negative log-likelihood by the attained evidence lower bound. Another possibility is to make ad-hoc choices for the posterior to obtain closed-form expressions for confidence sets, e.g.~for logistic regression \citep{lee2024unified}. 



\subsection{Oracle Complexity Bounds via Online Estimation}\label{sec:oco}

The size of the sequential mixing confidence set in \cref{result:posterior_mixing} depends on the ability of the learner to produce a sequence of mixing distributions that minimize the cumulative log loss,
\begin{align*}
   \sum_{s-1}^t l_s(\mu_{s-1}) =  - \sum_{s=1}^t \log \int p_s(y_s|\nu) d\mu_{s-1}(\nu) \,.
\end{align*}
The field of \emph{online density estimation} studies algorithms for minimizing the cumulative log loss, and offers a rich literature on complexity bounds for the regret \citep[e.g.,][]{vovk1990aggregating,vovk1997competitive,zhang2006varepsilon,rakhlin2014online}. Here, regret is defined as the difference between the cumulative log loss, and the best prediction in hindsight, which, in the simplest case, coincides with the maximum likelihood estimate. Specifically, we assume that the mixing distributions $\mu_1, \mu_2, \dots$ are chosen by an \emph{online estimation} algorithm, in a way that ensures a bound $B_t \geq 0$ on the log-regret,
\begin{align*}
    \Lambda_t = \sum_{s=1}^t l_s(\mu_{s-1}) - \min_{\theta} L_t(\theta) \leq B_t \,.
\end{align*}
For a complete introduction, we refer the reader to the standard literature \citep[e.g.,][]{cesa2006prediction,orabona2019modern,shalev2012online}. 
% More generally, the field of online convex optimization (OCO) studies algorithms for minimizing the regret, with minimal assumptions on  the sequence of loss functions. The OCO setting is typically formulated as a game between an \emph{adversary}, choosing the loss $l_t$, and an \emph{online learning} algorithm choosing a prediction (in our case, $\mu_{t-1})$. At the end of each round, the learner occurs the loss $l_t(\mu_{t-1})$, and the loss function is revealed to the learner, allowing to update the prediction. Our main result is the following.

The next result demonstrates how the sequential likelihood mixing framework relates to maximum likelihood estimation, using regret inequalities from online estimation. 
\begin{theorem}[Regret-To-Confidence]\label{result:regret}
    Assume there exists an online estimation algorithm such that the log-regret is bounded almost-surely, $\Lambda_t \leq B_t$, for a predictable sequence $B_t \geq 0$. Define 
    \begin{align*}
    C_t = \left\{ \theta \in \Theta: L_t(\theta) \leq  \log \frac{1}{\delta} + L_t(\hat \theta_t^\MLE) + B_t \right\} \,.
    \end{align*}
    Then $C_t$ defines a $(1-\delta)$-confidence sequence.
\end{theorem}
\begin{proof}
The result follows directly from \cref{result:posterior_mixing}, by introducing $L_t(\hat \theta_t^\MLE)$ and using the definition of the regret.
\end{proof}
The importance of the result is that the \emph{existence} of an online estimation algorithm with a \emph{known} regret bound, allows to define a valid $(1-\delta)$-confidence sequence, relative to the log-likelihood of the MLE and the complexity bound for online estimation. In particular, the construction does \emph{not} require access to the predictions of the online learning algorithm.  Moreover, the confidence coefficient can be computed using standard supervised learning algorithms, eliminating the need for computing the marginal likelihood. Lastly, the complexity term offers a more interpretable bound on the confidence coefficient. We will come back to several concrete examples below.


The use of regret inequalities to derive concentration inequalities goes back to at least \cite{dekel2010robust,abbasi2012online}. The ``Online-to-Confidence-Set'' conversion by \citet{abbasi2012online} is, however, different in a subtle way, and requires access to the predictions of the online learning algorithm. We recover (and improve upon) their result using a generalization of \cref{result:regret} in \cref{sec:sparse}. Later works extend this idea, for example, to (multinomial) logistic bandits \citep{lee2024improved}. A similarly flavoured result is by \cite{abeles2024generalization} in the context of PAC-Bayes generalization bounds. Worth mentioning is also the work by \cite{rakhlin2017equivalence}, who prove an equivalence between regret bounds and tail inequalities.

\paragraph{Logistic Regression} We illustrate \cref{result:regret} in the sequential logistic regression setting. Let $\Theta = \{\theta \in \bR^d : \|\theta\|_2 \leq S\}$ for a norm bound $S > 0$. The likelihood is a Bernoulli distribution, $p_t(y|\theta) = \Ber(\phi(\ip{\theta, x_t}))$ with the logistic link function $\phi(z) = (1 + e^{-z})^{-1}$  and the covariates $x_t \in \bR^d$. Following along the lines of \citet{lee2024improved}, we invoke a regret bound for logistic regression by \cite{foster2018logistic}, who prove an online learning algorithm that achieves $\Lambda_t \leq 10d \log \left( e + \frac{St}{2d} \right)$. \cref{result:regret} immediately implies the following $(1-\delta)$-confidence sequence:
\begin{align*}
    C_t = \left\{ \theta \in \Theta: L_t(\theta) \leq  \log \frac{1}{\delta} + L_t(\hat \theta_t^\MLE) + 10d \log \left( e + \frac{St}{2d}  \right)\right\} \,.
\end{align*}
We remark that the confidence coefficient above improves upon the result of \citet[Theorem 1]{lee2024improved}, as a consequence of directly applying Ville's inequality to the likelihood ratio.


\paragraph{Compressed Sensing} We come back to the sequential linear regression setting described below \cref{lem:posterior_ratio_confidence_set}, with the additional assumption that the true parameter $\theta^*$ is $k$-sparse, i.e.~$\|\theta^*\|_0 \leq k$. One way to account for the sparsity assumption is to set $\Theta_k = \{\theta \in \bR^d : \|\theta\|_0 < k, \|\theta\|_2 \leq S\}$. For sparse linear regression, \citep{gerchinovitz2011sparsity} proposes online learning algorithm  that achieves $\Lambda_t \leq C_0 k \log(t)$ for a constant $C_0 > 0$. \cref{result:regret} implies the following a confidence sequence,
\begin{align*}
    C_t = \left\{ \theta \in \Theta: L_t(\theta) \leq  \log \frac{1}{\delta} + L_t(\hat \theta_t^\MLE) + C_0 k \log(t)\right\} \,.
\end{align*}
Note that here MLE is defined over the sparse set $\Theta_k$, which poses computational challenges. On the other hand, the confidence set by \citet{abbasi2012online} requires access to the predictions of the online learning algorithm. The construction therefore suffers a similar fate, as finding a computationally efficient algorithm for sparse linear prediction is still an open problem.

% Specialized regret bounds, such as for sparse linear bandits or logistic regression are useful as the reveal concrete dependencies on problem parameters. Perhaps surprisingly, it is possible to obtain regret bounds without any assumptions on the model class.



\paragraph{Finite Model Identification} Assume that the parameter set $\Theta$ is finite, i.e.~$|\Theta|< \infty$. The key insight is to note that the log-loss is $\eta$-exp-concave for $\eta \leq 1$, i.e.~$\exp(-\eta l_t(\mu))$ is concave in $\mu \in \sP(\Theta)$ for all $\eta \leq 1$ (in fact, it is linear for $\eta=1$). This highlights the importance of using mixing distributions, as, in general, $-\log p_t(y_t|\theta)$ is \emph{not} exp-concave as a function of $\theta$.

The standard approach for online learning with exp-concave functions is the \emph{exponential weights algorithm} \citep[EWA, ][]{littlestone1994weighted,freund1997decision}, see also the book by \citet{cesa2006prediction}. For $\eta=1$, EWA is equivalent to Bayesian inference and the prediction $\mu_t$ is equal to the Bayesian posterior (see \cref{alg:cew}). With a uniform prior, the regret of EWA satisfies $\Lambda_t \leq \log(|\Theta|)$, uniformly over all data sequences. The proof is provided for completeness in \cref{sec:cew}. Using \cref{result:regret}, we obtain the following $(1-\delta)$-confidence sequence\looseness=-1
\begin{align*}
    C_t = \{\theta \in \Theta : L_t(\theta) \leq \log \frac{1}{\delta} + L_t(\hat \theta_t^\MLE) + \log |\Theta| \} \,.
\end{align*}
The result should be compared to the standard union bound argument (\cref{sec:mle}). While the bound is the same, note that we obtained \cref{result:regret} as a \emph{relaxation} of \cref{result:posterior_mixing}. Hence, the sequential mixing confidence set (with the Bayesian posterior as mixing distributions) is never worse, and possibly tighter for benign data and structured model classes \citep[e.g.,][]{auer2002adaptive,cesa2007improved,de2014follow}. The confidence set can also be written for the maximum a-posteriori estimate (MAP), in which case the role of the prior distribution becomes more apparent:
\begin{align*}
    C_t = \left\{\theta \in \Theta : L_t(\theta) \leq \log \frac{1}{\delta} + \min_{\nu \in \Theta} \big( L_t(\nu) - \log \mu_0(\nu) \big)\right\} \,.
\end{align*}
Lastly, we remark that the exponential weights algorithm can be generalized to the continuous setting. The confidence sequence derived from the regret bound of continuous exponential weights is equivalent to the ELBO confidence set in \cref{result:elbo_confidence_set}. This is another consequence of the mixing equivalence. We refer to \cref{sec:cew} for further details.

\section*{Conclusion}
This paper aims to enhance our understanding of the computational complexity of computing various Shapley value variants. We found that for various ML models --- including decision trees, regression tree ensembles, weighted automata, and linear regression --- both local and global interventional and baseline SHAP can be computed in polynomial time under HMM modeled distributions. This extends popular algorithms, such as TreeSHAP, beyond their empirical distributional scope. We also establish strict complexity gaps between the various SHAP variants (baseline, interventional, and conditional) and prove the intractability of computing SHAP for tree ensembles and neural networks in simplified scenarios. Overall, we present SHAP as a versatile framework whose complexity depends on four key factors: \begin{inparaenum}[(i)] \item model type, \item SHAP variant, \item distribution modeling approach, \item and local vs. global explanations\end{inparaenum}. We believe this perspective provides deeper insight into the computational complexity of SHAP, paving the way for future work.




%We believe that our framework provides a more intricate understanding of SHAP computation complexity across different models, distributions, and variants, paving the way for further research.

Our work opens promising directions for future research. First, expanding our computational analysis to other SHAP-related metrics, such as asymmetric SHAP~\citep{frye20} and SAGE~\citep{covert2020understanding}, would be valuable. Additionally, we aim to explore more expressive distribution classes and relaxed assumptions beyond those in Section \ref{sec:tractable} while maintaining tractable SHAP computation. Finally, when exact computation is intractable (Section \ref{sec:intractable}), investigating the approximability of SHAP metrics through approximation and parameterized complexity theory~\citep{downey2012parameterized} is an important direction.

%Our work opens several promising avenues for future research on the computational properties of explainable AI methods, with a particular focus on SHAP. First, it would be interesting to broaden the computational analysis conducted in this work to include other popular SHAP-related metrics in the literature, such as asymmetric SHAP \cite{frye20} and SAGE \cite{covert2020understanding}. Also, in the future, we aim to explore more expressive distribution classes and relaxed distributional assumptions—extending beyond those examined in Section \ref{sec:tractable} —that still yield tractable SHAP computation. Finally, when exact computation proves intractable (Section \ref{sec:intractable}), it is worthwhile to theoretically investigate the question of the approximability of computing the SHAP metrics across various configurations, through the lens of approximation and parametrized complexity theory \cite{arora2009computational}.

%This paper aims to deepen our understanding of the computational complexity involved in obtaining different Shapley value variants. We found that for a variety of ML models, including decision trees, tree ensembles for regression, weighted automata, and linear regression models — computing both local and global interventional and baseline SHAP can be done in polynomial time when distributions are modeled by HMMs. This extends the distributional scope of popular algorithms like TreeSHAP, which is limited to empirical distributions. Additionally, we demonstrate a strict complexity gap between SHAP variants, showing that interventional and baseline SHAP can be strictly easier to compute than conditional SHAP. Despite these positive results, we uncovered intractability for various SHAP variants in neural networks and tree ensembles. Finally, we provided generalized complexity relations across SHAP variants. We believe that our framework offers a deeper understanding of the complexity involved in computing SHAP across various variants, models, distributions, as well as in both local and global computations, laying the groundwork for future research.


% Acknowledgments---Will not appear in anonymized version
\acks{We thank Guillaume Obozinski for helpful discussions on the topic.}

% \newpage
\bibliography{bibliography}

\newpage
\appendix
\crefalias{section}{appendix} % uncomment if you are using cleveref

The advancement of artificial intelligence in the legal domain has led to the development of various tools that assist in legal research, document retrieval, and automated legal reasoning. Several studies have explored the use of Natural Language Processing (NLP)\cite{khurana2023natural}, machine learning models, and vector-based search mechanisms to enhance the efficiency of legal chatbots. The primary focus of this literature review is on retrieval-augmented generation (RAG) models, FAISS-based document retrieval, deep learning for legal applications, and the use of large language models (LLMs) in legal AI.  

Recent research on Retrieval-Augmented Generation (RAG)\cite{gao2023retrieval} for legal AI has demonstrated its potential in enhancing legal text retrieval and summarization. S. S. Manathunga, Y. and A. Illangasekara\cite{manathunga2023retrieval} proposed a RAG-based model that improves legal text summarization by dynamically fetching relevant documents before generating responses. Similarly, Lee and Ryu \cite{ryu-etal-2023-retrieval} explored the application of RAG in case law retrieval, demonstrating its superiority over traditional keyword-based search engines. The introduction of RAG has significantly improved response accuracy by grounding AI-generated text in authoritative legal documents, reducing hallucinations in AI-driven legal assistance.  

% \begin{figure}[h]
%     \centering
%     \includegraphics[width=8cm]{FAISS.png}
%     \caption{Faiss: Efficient Similarity Search and Clustering of Dense Vectors}
%     \label{Overall Result of comparing FAISS and Chroma with different number of top documents}
% \end{figure}

The efficiency of FAISS (Facebook AI Similarity Search) in legal document retrieval has also been widely studied. Zhao et al. \cite{devlin-etal-2019-bert} implemented FAISS to enhance large-scale legal question answering systems, achieving significant improvements in retrieval speed and relevance. N. Goyal and D. Chen \cite{inbook} demonstrated that FAISS-based vector search mechanisms outperform conventional database searches in legal information retrieval, reducing query response time while maintaining high accuracy. The integration of FAISS with transformer-based models, as seen in the work of Hsieh and Wu, further enhances semantic retrieval, ensuring that chatbot responses align with actual legal texts.  

Transformer-based models such as BERT and GPT-based architecture have also contributed to the evolution of AI-driven legal research. Devlin et al. introduced BERT (Bidirectional Encoder Representations from Transformers), which significantly improved the understanding of legal language. RoBERTa, an optimized version of BERT, was later developed by Liu et al. \cite{liu2019roberta} to enhance contextual understanding and document similarity matching in legal queries. These models have been integrated into legal chatbots for contract analysis and legal decision-making, as demonstrated in the studies of Li et al. and Jin and Liu, where fine-tuned transformers improved legal text comprehension and summarization.  
The role of deep learning in legal AI has also been investigated extensively. Radford et al. introduced GPT-3, which paved the way for legal AI assistants capable of generating human-like responses. However, researchers such as Firth and Lee emphasized the limitations of LLMs in legal reasoning, arguing that these models require external verification mechanisms to prevent misinformation. The use of contrastive learning and fine-tuning for legal text retrieval has been explored by Arabi and Akbari \cite{article}, who demonstrated that embedding-based retrieval significantly improves chatbot response accuracy.  

Another significant area of research involves evaluating AI-generated legal responses using automated metrics. Zhang and Wu introduced BLEU\cite{10.3115/1073083.1073135} and ROUGE\cite{lin-2004-rouge} scores as a means to evaluate AI-generated legal text summaries, ensuring their quality and relevance. Similarly, Zhao et al. \cite{yuan2024rag} examined the effectiveness of RAG-based models in handling complex legal queries, highlighting the importance of legal consistency scores (LCS) in evaluating AI-driven responses.  

The practical applications of legal AI chatbots have been studied extensively in the context of access to justice and AI ethics. Wang and Cheng et al. \cite{xue2024bias} highlighted the potential of AI-driven legal assistants in bridging the justice gap, particularly in countries where legal resources are not easily accessible. Chan conducted a systematic review of retrieval-based legal chatbots, noting that while these systems improve accessibility, they also raise ethical concerns regarding legal misinformation and bias. Research by Min \cite{Min2023ARTIFICIALIA} explored methods for bias detection and mitigation in legal AI, ensuring fairness in AI-generated legal advice.  

Comparative studies between rule-based legal bots, keyword-driven legal search engines, and AI-powered legal chatbots further illustrate the superiority of retrieval-augmented approaches. In a study conducted by Zeng \cite{zeng2024scalable}, FAISS-based retrieval mechanisms significantly outperformed traditional Boolean keyword searches, reducing irrelevant document retrieval by 40\%. Singh \cite{10760929} further demonstrated that AI-powered legal research tools using NLP provide faster and more contextually accurate responses compared to standard legal databases.  

Despite these advancements, challenges remain in AI-driven legal research. Existing chatbots still struggle with multi-jurisdictional legal queries, as noted by Weichbroth \cite{Weichbroth2025AIAT}, who emphasized the need for jurisdiction-aware legal AI models. Additionally, legal AI models often lack the ability to process long-context legal arguments effectively, a limitation discussed by Gupta, who proposed memory-based retrieval techniques to improve long-form legal text processing.  

Research continues to refine AI-driven legal assistance, particularly in retrieval-augmented generation, FAISS-based search, transformer models, and deep learning techniques for legal research. However, further improvements are needed in bias mitigation, jurisdiction-specific adaptations, and long-context legal understanding. Future developments in multilingual legal AI, enhanced retrieval mechanisms, and AI-powered contract analysis will be crucial in making legal AI tools more accessible, reliable, and widely applicable in legal practice.
\section{Laplace's method}\label{sec:laplace}
Recall from section \cref{sec:mle} that $\max_{\nu} R_t(\nu;\theta^*)$ is not a martingale, which prevents a direct application of Ville's inequality. Laplace's method uses the observation that, under suitable regularity assumptions on a sequence functions $f_n : \bR^d \rightarrow \bR$ with unique maximizer $x_n^*$ and positive definite Hessian $H_n(x) = \frac{\partial^2}{\partial x^2} f_n(x) \in \bR^{d \times d}$, the following asymptotic expansion provides an approximation of the maximizer $f_n(x_n^*)$, 
\begin{align*}
    \int_\Theta h(x) e^{- f_n(x)} dx \sim  \frac{(2\pi)^{d/2} h(x^*_n)}{\sqrt{\det H_n(x_n^*)}} e^{- f_n(x^*_n)}
    % \max_{\theta} f_n(\theta) = \log \lim_{n \rightarrow \infty} \int \exp( n f_n(\theta)) d\mu_0(\theta)
\end{align*}
To apply this idea to the marginal likelihood that appears in the confidence coefficient in \cref{result:prior_mixing}, assume that $\Theta \subset \bR^d$ and $\mu_0$ admits a density $h(\theta)$ w.r.t. to the Lebesgue measure. Let $\hat \theta_n^\MLE$ be the maximum likelihood estimate, and $I_t(\theta) = \frac{\partial^2}{\partial \theta^2} L_t(\theta)$ the empirical Fischer information matrix. Laplace's methods gives the following approximation
\begin{align*}
\beta_t(\delta) &= \log \frac{1}{\delta} - \log \int \exp(-L_t(\nu))  h(\nu) d\theta\\
&\approx\log \frac{1}{\delta} + L_t(\hat \theta_t^{\MLE})  + \frac{1}{2}\log \det I_t(\hat \theta_n^\MLE) - \frac{d}{2} \log(2\pi) - \log \mu_0(\hat \theta_n^\MLE)  
\end{align*}
Note that the confidence coefficient is smaller (resulting in a smaller confidence set) the more mass $h(\hat \theta^*_n)$ places on the maximizer.
An similar (and perhaps more natural) argument can be made for the maximum a-posteriori estimate.
Unfortunately, making these approximation rigorous is challenging without placing further assumptions on the data generating distribution and function class \citep[e.g.,][]{shun1995laplace}. We will not pursue this any further here, however point out that regret inequalities (\cref{sec:oco}) provide an alternative way to control the error w.r.t.~the MLE, including in finite time.
 
\section{Continuous Exponential Weights}\label{sec:cew}

Continuous exponential weights (\cref{alg:cew}) is a direct generalization of the classical exponential weights algorithm (also known as Hedge) by \cite{freund1997decision,littlestone1994weighted}. For a standard reference, see the book by \cite{cesa2006prediction}. Recall the definition of the log-loss, defined for distributions $\mu \in \sP(\Theta)$,
\begin{align*}
 l_t(\mu) = - \log \left(\int_{\Theta} p_t(y_t|\theta)d\mu(\theta)\right)
\end{align*}
Note that $l_t(\mu)$ is $\eta$-exp-concave for $\eta \leq 1$, since $\exp(-\eta l_t(\mu))$ is concave in $\mu : \sP(\Theta) \rightarrow \bR$. The (continuous) exponential weights algorithm satisfies the following regret bound, that holds for any sequence of $\eta$-exp-concave loss functions.
\begin{theorem}[Regret of Continuous Exponential Weights]\label{thm:cew}
    For any distribution $\rho \in \sP(\Theta)$, and any sequence of $\eta$-exp-concave loss functions $l_1, \dots, l_t$, the regret of the exponential weights algorithms with prior $\mu_0 \in \sP(\Theta)$ and learning rate $\eta$ satisfies
    \begin{align*}
        \sum_{s=1}^t l_s(\mu_{s-1}) - \int L_t(\theta) d\rho \leq \frac{1}{\eta}\KL(\rho, \mu_0)
    \end{align*}
    % So in particular, for any $\theta \in \Theta$ and assuming that $\delta_\theta \ll \mu_0$,
    % \begin{align}
    %     \sum_{t=1}^n \tilde l_t(\mu_t) - \sum_{t=1}^n l_t(\theta) \leq \log\left(\frac{1}{\mu_0(\theta)}\right)
    % \end{align}
    Moreover, for finite $\Theta$ and any $\nu \in \Theta$,
    \begin{align*}
        \sum_{s=1}^t  l_s(\mu_{s-1}) -  L_t(\nu) \leq \frac{1}{\eta}\log \frac{1}{\rho(\nu)}
    \end{align*}
\end{theorem}


% \begin{remark}
%     Note that $\tilde l_t(\mu_t)$ is the `posterior' log likelihood of $p(y|x,\theta)\mu_t(\theta)$.
% \end{remark}
\begin{proof} Denote by $\delta_{\theta} \in \sP(\Theta)$ the Dirac measure on $\theta \in \Theta$. Define the unnormalized measure 
    \begin{align*}
    %    \tilde \mu_t(d\theta) = \exp(-\eta L_{t}(\theta)) \mu_0(d\theta) = \prod_{s=1}^{t} \exp(-\eta l_s(\delta_\theta)) \mu_0(d\theta)
       \tilde \mu_t(d\theta) = \prod_{s=1}^{t} \exp(-\eta l_s(\delta_\theta)) \mu_0(d\theta)
    \end{align*}
    Note that $\mu_t(d\theta) = \frac{\tilde \mu_t(d\theta)}{\tilde \mu_t(\Theta)}$.
To prove the regret bound note that by $\eta$-exp-concavity of $l_t$,
\begin{align*}
    \frac{\tilde \mu_t(\Theta)}{\tilde \mu_{t-1}(\Theta)} &= \int \exp(-\eta l_t(\delta_\theta)) d\mu_{t-1}(\theta) \leq \exp (- \eta  l_t(\mu_{t-1}) )
\end{align*}
Further, for any $\rho \in \sP(\Theta)$, the variational inequality (\cref{lemma:variational-kl}) implies
\begin{align*}
\tilde \mu_t(\Theta) \geq \exp \Big( - \eta \ip{L_n, \rho} - \KL(\rho, \mu_0)\Big)
\end{align*}
Combining the last two inequalities and telescoping, we get
\begin{align*}
    \sum_{s=1}^t  l_s(\mu_{s-1}) -  \int L_s(\theta) d\rho(\theta) \leq \frac{1}{\eta}\KL(\rho, \mu_0)
\end{align*}
This completes the first part of the proof. The second part follows by setting $\rho=\delta_{\nu}$.
\end{proof}


\LinesNumbered
\RestyleAlgo{ruled}
\begin{algorithm2e}[t]
	\DontPrintSemicolon
	\SetAlgoVlined
	\SetAlgoNoLine
	\SetAlgoNoEnd
	\caption{Continuous Exponential Weights} \label{alg:cew}
    \SetKwInput{KwInput}{Input}
    
    \KwInput{Prior $\mu_0 \in \sP(\Theta)$, learning rate $\eta > 0$}

    \For{$t \gets 1, 2. \dots$} {
        \textbf{Predict:} $\mu_{t-1}$\;
        \textbf{Observe:} $x_t, y_t$\;
        \textbf{Receive (log) loss:}$$l_t(\mu_{t-1}) = - \log \left(\int p_t(y_t|\theta)d\mu_{t-1}(\theta)\right)$$\;
        \textbf{Update} $\mu_t(d\theta) \propto \exp\big(-\eta \sum_{s=1}^t l_s(d\theta)\big) \mu_0(d\theta)$ \textbf{:}
        $$\mu_t(d\theta)  = \frac{\exp\big(\eta \log p_t(y_t|\theta)\big) \mu_{t-1}(d\theta)}{\int \exp\big(\eta \log p_t(y_t|\nu)\big) d\mu_{t-1}(\nu)}$$
    }
\end{algorithm2e}

\paragraph{Confidence Sequences using Continuous Exponential Weights} 
Substituting the regret bound from \cref{thm:cew} into \cref{result:posterior_mixing} yields the following $(1-\delta)$-confidence sequence, which holds for any $\cF_t$-adapted sequence $\mu_1,\mu_2, \dots \in \sP(\Theta)$:
\begin{align*}
    C_t  = \left\{ \theta \in \Theta: L_t(\theta) \leq  \log \frac{1}{\delta} + \int L_t(\theta) d\mu_t(\theta) + \KL(\mu_t \| \mu_0)\right\} \,.
\end{align*}
Note that we have recovered the ELBO-confidence set from \cref{result:elbo_confidence_set}. In other words, the regret bound of continuous exponential weights is the sequential analog of the variational inequality \cref{lemma:variational-kl}, and because of the mixing equivalence (\cref{result:mixing-equivalence}), the resulting confidence sets are the same. Unfortunately, for continuous $\Theta$, the bound becomes vacuous when $\rho$ is set to a Dirac measure. This prevents us from using $\mu_t = \delta_{\hat \theta_t^\MLE}$ to derive a confidence set that directly compares to the maximum likelihood estimate. A more careful analysis using additional smoothness assumptions is a possible way forward \citep[c.f.,][]{lee2024unified}.




\section{Misspecified Model Classes}\label{sec:misspecified}

So far, we have assumed that the model class is realizable, that is, there exists a parameter $\theta^* \in \Theta$ such that $p_t(y|\theta^*) = \frac{d\bP_t}{d\xi}$ represents the density of the true data generating distribution. The realizability assumption is used to show that the sequential likelihood ratio is a (super)martingale. The can be significantly relaxed, and there are several ways in which we can construct supermartingales for misspecified model classes, as we elaborate now. 





% \subsubsection{Sub-Likelihood Robustness}
% In particular, we will assume that the following condition holds for the true data-generating process, we will refer to it as sub-likelihood condition. 

% \begin{definition}[Sub-likelihood]\label{def:2} Let $L$ be the log-likelihood of the true distribution of $\epsilon_t$ for all $t$, be $p$. We call $p$ to be sub-Likelihood $L$ if, 
%   \begin{equation}
%       \EE_{p}[e^{\nabla  L(\epsilon) \eta + B_{ L}(\epsilon-\eta,\epsilon)}] \leq 1,
%   \end{equation}  
%   where $B_{\log L}$ denotes Bregman divergence of $\log L$. 
% \end{definition}
% This sub-likelihood condition can be linked to a broader treatment of sub-processes by \cite{Howard2020}, where they examine a similar problem to define sub-families and construct confidence sets for these sub-families. Conceptually, our approaches are opposite. While they define a family of probability distributions that includes a given likelihood, we start with a fixed likelihood and identify robustness conditions for other distributions.

% In order to have a valid confidence interval with likelihood ratios, we need that 
% \begin{align}\label{eq:miss_lk}
%     E_t(\theta) = \prod_{s=1}^t \frac{\exp(L(f_{\hat \theta_{s-1}}(x_s) - y_s))}{\exp(L(f_\theta(x_s) - )y_s)}
% \end{align}
% is a super-martingale, as we show now. 
% \begin{theorem} Assume that the true data-generating satisfies Definition \ref{def:2}, then $E_t(\theta^*)$ in Eq. \eqref{eq:miss_lk} is a super-martingale adapted to the usual filtration $\cF_{t-1}$. 
% \end{theorem}
% \begin{proof}
% Let $\eta = f_{\hat \theta_t}(x_{t-1})$, and note that Bregman divergence, 
% $L(\epsilon-\eta) - L(\epsilon) = -\nabla L (\epsilon)^\top \eta - B_L(\epsilon-\eta,\epsilon)$. Then, 

% \begin{eqnarray}
% \EE_{\epsilon_t}[E_t(\theta^*)| \mathcal{F}_{t-1}]  &  = & E_{t-1}(\theta^*)\EE_{\epsilon}[\exp(L(\epsilon-\eta) - L(\epsilon))] \\ & = & E_{t-1}(\theta^*) \EE_{\epsilon}[\exp( \nabla L (\epsilon)^\top \eta - B_L(\epsilon-\eta,\epsilon))] \leq E_{t-1}(\theta^*)
% \end{eqnarray}

% \end{proof} 

% This result allows us to construct confidence intervals for $\theta^*$ that parametrizes the mean of the $y_t$ irrespective of the misspecification, albeit at loss of power since the $E_t(\theta^*)$ is only a super-martingale instead of a martingale as in the well-specified case. 

% \paragraph{Example 1: Sub-Gaussian Likelihoods}\label{sec:subgaussian}
% Assume that the true data generating distribution $p(y_t|x_t)$ is $\sigma$-sub-Gaussian, that is $\epsilon_t = y_t - \EE[y_t]$ satisfies
% \begin{align*}
%     \EE[\exp\left(\epsilon \eta - \frac{\sigma^2 \eta^2}{2}\right)] \leq 1 \quad \text{for all} \quad \eta \in \bR
% \end{align*}
% This assumption exactly coincides with Definition \ref{def:2} as $B_{L} = \frac{\eta^2}{2\sigma^2}$, and $\nabla L = \frac{\epsilon}{\sigma^2}$, which is equivalent to the above.

% \paragraph{Example 2: Sub-Poisson Likelihood}\label{sec:sub} Suppose that the mean is parametrized as $f_\theta(x) = \exp(\theta^\top x) $, and we are observing integer values as $y_t \in \{0,1,\dots \infty\}$. Now the true generating distribution $p_i$ for data point $y_i$ is sub-Poisson if, that is when using Poisson likelihood, we maintain coverage even if $p$ satisfied the following,
% \[ \EE [\exp\left( (\eta + \epsilon)\log (\lambda_i) + \log\left(\frac{(\epsilon + \eta)!}{\epsilon!}\right)\right) ] \leq 1 \quad \text{for all}  \quad \eta \in \bR. \]

\subsection{Sub-Gaussian Distributions}\label{sec:subgaussian}
In this section, we let $\cY = \bR$ and assume that the true data generating distribution $\bP_t$ is $\sigma$-sub-Gaussian for all $t \geq 1$, that is $\epsilon_t = y_t - \EE[y_t|\cF_{t}]$ satisfies,
\begin{align*}
\EE[e^{\epsilon_t \eta}|\cF_t] \leq e^{\frac{\sigma^2 \eta^2}{2}} \quad \text{for all} \quad \eta \in \bR \,.
\end{align*}
Further assume that we have a parametrized family of mean functions $f_\theta : \cX \rightarrow \bR$, and there exists a $\theta^* \in \Theta$ such that $\EE[y_t|\cF_t] = f_{\theta^*}(x_t)$.
For any $\cF$-adapted sequence of mixing distributions $\mu_0, \mu_1, \dots \in \sP(\Theta)$, define
    \begin{align*}
        E_t(\theta) = \prod_{s=1}^t \frac{ \int \exp(- \frac{1}{2 \sigma^2} (f_{\nu}(x_s) - y_s)^2) d\mu_{s-1}(\nu)}{\exp(- \frac{1}{2 \sigma^2} (f_\theta(x_s) - y_s)^2)} \,.
    \end{align*}
As we will see shortly, $E_t(\theta^*)$ is a $\cF_t$ adapted supermartingale, and $\EE[E_1(\theta^*)] \leq 1$.
Two remarks before we prove the claim. First, if the true data distribution is Gaussian with mean $f_\theta(x)$ and variance $\sigma^2$, then $E_t(\theta)$ is just the sequential marginal likelihood ratio, and the $\sigma$-sub-Gaussian condition holds with equality. Second, the result implies that we can proceed in constructing our confidence set as if the the Gaussian likelihood model was correct, and the coverage results remain true.\looseness=-1
\begin{theorem}
    For any $\cF$-adapted sequence of distributions $\mu_0, \mu_1, \mu_2, \dots$ in $\sP(\Theta)$, define
    \begin{align*}
        C_t = \left\{ \theta \in \Theta:   \sum_{s=1}^t \tfrac{1}{2 \sigma^2}(f_\theta(x_s) - y_s)^2 \leq \log \frac{1}{\delta} + \sum_{s=1}^t \log \int  \exp( -\tfrac{1}{2 \sigma^2} (f_{\nu}(x_s) - y_s)^2) d\mu_{s-1}(\nu)\right\}
    \end{align*}
    Then $C_t$ defines a $(1-\delta)$-confidence sequence.
\end{theorem}
\begin{proof}
We start by showing that $E_t(\theta^*)$ is a super-martingale. Fubini's theorem implies that
\begin{align*}
    \EE[E_t(\theta^*)|\cF_{t-1}] &= E_{t-1}(\theta^*) \int \EE[\exp\left( - \tfrac{1}{2\sigma^2} \big(f_\nu(x_t) - y_t\big)^2 + \tfrac{1}{2\sigma^2} \big(f_{\theta^*}(x_t) - y_t\big)^2 \right) |\cF_t ]  d \mu_{t-1}(\nu)%\\
    % &= E_{t-1}(\theta^*) \EE[\exp\left( - \tfrac{1}{2\sigma^2} f_{\hat \theta_t}(x_t)^2 - \tfrac{1}{\sigma^2} y_t f_{\hat \theta_t}(x_t) + \tfrac{1}{\sigma^2} y_t f_{\theta^*}(x_t) - \tfrac{1}{2\sigma^2} f_{\theta^*}(x_t)^2  \right) ]\\
    % &= E_{t-1}(\theta^*) \exp\left( - \tfrac{\sigma^2}{2} \big( f_{\hat \theta_s}(x_s)^2 - f_{\theta^*}(x_t)^2 \big) \right) \EE[\exp\left( - \tfrac{1}{2\sigma^2} \big( - 2 y_s f_{\hat \theta_s}(x_s) + 2 y_s f_\theta(x_s)  \big) \right) ]\\
\end{align*}
From here, we compute the conditional expectation inside the integral. We expand the squares, simplify and substitute $y_t = f_{\theta^*}(x_t) + \epsilon_t$. After a bit of work we arrive at 
\begin{align*}
    &\EE[\exp\left( - \tfrac{1}{2\sigma^2} \big(f_\nu(x_t) - y_t\big)^2 + \tfrac{1}{2\sigma^2} \big(f_{\theta^*}(x_t) - y_t\big)^2 \right) |\cF_t ] \\
    &= \exp\left(- \tfrac{1}{2\sigma^2} \big(f_{\hat \theta_{t-1}}(x_t) - f_{\theta^*}(x_t)\big)^2 \right) \EE[\exp\left( \epsilon_t \cdot \tfrac{1}{\sigma^2} \big(f_{\hat \theta_{t-1}}(x_t) - f_{\theta^*}(x_t)\big) \right) |\cF_t] \,.
\end{align*}
Next, we use that $\epsilon_t$ is $\sigma^2$-sub-Gaussian, which, by definition, implies that 
\begin{align*}
    \EE[\exp\left( \epsilon_t \cdot \tfrac{1}{\sigma^2} \big(f_{\hat \theta_{t-1}}(x_t) - f_{\theta^*}(x_t)\big) \right) ] \leq \exp\left(\tfrac{1}{2\sigma^2} \big(f_{\hat \theta_{t-1}}(x_t) - f_{\theta^*}(x_t)\big)^2 \right) \,.
\end{align*}
We conclude that $\EE[E_t(\theta^*)|\cF_{t-1}] \leq  E_{t-1}(\theta^*)$ and $\EE[E_1(\theta^*)] \leq 1$. The claim follows using Ville's inequality.
\end{proof}

% Lastly, we remark that one can reverse the setup, and start with a loss function $l_t : \Theta \times \cY \rightarrow \bR$. Let $\cP$ be the set of distributions for which the process $E_t(\theta) = \prod_{s=1}^t \exp(l_s(\theta, y_s) - l_s(\hat \theta_{s-1}, y_s) )$ is a supermartingale\todoj{how is $\theta^*$ and $\bP$ related?}. Then $\{ \theta \in \Theta : \log E_t(\theta) \leq \log \frac{1}{\delta} \}$ defines a confidence set, as long as $\bP \in \cP$. Setting  $l_t(\theta, y) = \frac{1}{2\sigma^2}(f_\theta(x_t) - y)^2$ recovers the sub-Gaussian case, and choosing the log-loss $l_t(\theta, y) = - \log p_t(y|\theta)$ recovers the standard likelihood ratio. 

% $$\theta^*_t = \argmin_{\theta \in \Theta} \sum_{s=1}^t \bE_s[l_s(\theta, y_s)]$$

% Define the \emph{generalized sequential likelihood ratio},
% $$E_t(\nu, \theta) = \prod_{s=1}^t \exp(l_s(\theta, y_s) - l_s(\nu, y_s) )$$
% Note that for the log loss, $E_t(\nu, \theta) = R_t(\nu, \theta)$ recovers the standard sequential likelihood ratio. Define
% % $$D_t^E(\nu \| \theta) = \sum_{s=1}^t \bE_s[\log E_s(\nu, \theta)]$$
% % and assume that $D_t^E$ is a divergence under the true distribution
% % Assume that there exists a par
% $$\theta^*_t = \argmin_{\theta \in \Theta} \sum_{s=1}^t \bE_s[\log E_s(\theta^*, \theta)]$$
% Note that the expectation recovers the KL (minimized by $\theta^*$) for the log loss, and the square loss over means for the sub-Gaussian case (offset by a variance term). 
% $$\theta^* = \argmin_{\theta \in \Theta} \sum_{s=1}^t \bE_s[\log E_s(\bP, \theta)]$$

\subsection{Convex Model Classes}\label{sec:convex}
We return to the original definition of the model class as a parameterized family of conditional densities $\cM = \{ p_\theta(y|x) : \theta \in \Theta\}$. Moreover, assume that there exists a conditional density $p^*(y|x)$ such that for all $t \geq 1$, $\frac{d\bP_t}{d\xi} = p^*(\cdot|x_t) d\xi$. Crucially, we do \emph{not} require that $p^* \in \cM$. 

Throughout this section, we make the assumption that $\cM$ is convex. Note that convexity is required in the space of distributions, i.e., all finite mixtures of densities in $\cM$ are contained in $\cM$. In general, convexity of $\Theta$ does not imply convexity of $\cM$. Nevertheless, there are many examples of convex model classes, including, for example, all finite mixtures of any family of distributions. 

Our main tool is the \emph{reverse information projection} theorem by \citet{li1999estimation}, see also \citet{lardy2024reverse}. Applied to our setup, the theorem states that for any sequence $q_n \in \cM$ such that $$\lim_{n \rightarrow \infty} \KL(p^*\|q_n) = \inf_{q \in \cM} \KL(p^*\|q) < \infty\,,$$ there exists a unique (sub-)probability measure $q^* d\xi$ such that $\KL(p^*\|q^*) = \inf_{q \in \cM} \KL(p^*\|q)$. Moreover, the reverse information projection theorem shows that for any $q_\theta \in \cM$,
\begin{align}
    \EE[\frac{p_\theta(y_t|x_t)}{q^*(y_t|x_t)}\big | \cF_t] \leq 1 \,. \label{eq:rips-e-value}
\end{align}
A technical condition is required to ensure that the limiting element $q^*$ is contained $\cM$. If we require that $\cY$ is a complete separable metric space, and $\cM$ is sequentially compact (with respect to the weak topology), then Prokhorov's theorem implies that $q^* \in \cM$. A similarly flavoured result (stated without mixing distributions) is by \citet[Proposition 7]{wasserman2020universal}

\begin{theorem}[Convex Model Classes] Assume that $\cM$ is convex and there exists $q^* \in \cM$ such that $\KL(p^*\|q^*) = \inf_{q \in \cM} \KL(p^*\|q)$. Then the sequential likelihood mixing confidence set, defined for any $\cF_t$-adapted sequence of distributions $\mu_0, \mu_1, \mu_2, \dots$ in $\sP(\Theta)$ (see \cref{result:posterior_mixing}), defines a $(1-\delta)$-confidence sequence for $q^* \in \cM$, i.e., $\bP[q^* \in C_t, \forall t \geq 1] \geq 1-\delta$.
\end{theorem}
\begin{proof}
Define the sequential marginal likelihood ratio w.r.t. to a conditional density $q(\cdot|x)$,
\begin{align*}
J_t(q) = \prod_{s=1}^t\frac{\int p_\nu(y_s|x_s) d\mu_{s-1}(\nu)}{q(y_s|x_s)}
\end{align*}
The reverse information projection theorem, specifically \cref{eq:rips-e-value}, implies that $J_t(q^*)$ is a non-negative supermartingale with $\EE[J_1] \leq 1$. The theorem follows using Ville's inequality.
\end{proof}

\section{Tempered Likelihood Ratios}\label{sec:tempered}

The Bayesian update rule can be generalized by introducing a \emph{temperature} parameter $\beta > 0$,
\begin{align*}
    \mu_t(\theta) \propto \prod_{s=1}^t p_s(y_s|\theta)^\beta \mu_0(\theta) \,.
\end{align*}
The generalized update rule has been studied under various names, e.g. as \emph{fractional posteriors} \citep{bhattacharya2019bayesian}, \emph{powered likelihoods} \citet{holmes2017assigning} and \emph{tempered posteriors} \citep{alquier2020concentration}, often in the context of adding robustness to misspecification in the Bayesian model \citep{grunwald2012safe}. The result presented below are similar in spirit to the work by \citet{zhang2006varepsilon}, who studies complexity bounds for density estimation in the classical i.i.d.~setting.

\paragraph{Divergences} A few more definitions will be useful, we follow the exposition of \citet{van2014renyi}. Let $p,q \in \sP(\cY)$ be two distributions over the observation space $\cY$. We assume that $p,q$ admit densities w.r.t. a common base measure. We define the Rényi divergence for  $p,q$ and parameter $0 \leq \zeta \leq 1$,
\begin{align*}
    D_\zeta(p\|q) = \frac{1}{\zeta - 1} \log \int p(x)^{\zeta} q(x)^{1- \zeta} dx \,.
\end{align*}
For $0< \zeta < 1$, it holds that
\begin{align*}
   (1-\zeta)  D_\zeta(p\|q) = \zeta D_{1-\zeta}(q\|p)\,.
\end{align*}
Moreover, the Hellinger distance is given by 
\begin{align*}
    H^2(p\|q) = \int \big(\sqrt{p(x)} - \sqrt{q(x)}\big)^2 dx\,.
\end{align*}
Hellinger and Rényi divergences satisfy the following relation:
\begin{align*}
    \frac{1}{2} H^2(p\|q) \leq D_{1/2}(p\|q)\,.
\end{align*}
For notational convenience, we define
\begin{align*}
    D_{\zeta,t}(\theta\|\nu) &:= D_\zeta(p_t(\cdot|\theta)\|p_t(\cdot|\nu)) \,,\\
    H^2_t(\theta\|\nu) &:= H^2(p_t(\cdot|\theta)\|p_t(\cdot|\nu)) \,.\\
\end{align*}

\subsection{Tempered Confidence Sequences}
Let $\hat \theta_0, \hat \theta_1, \dots$ be an $\cF_t$-adapted sequence of estimators.
Define the \emph{tempered} log-ratio, 
\begin{align}
    A_t^\beta(\theta) = - \beta \log \frac{p_t(y_t|\hat \theta_{t-1})}{p_t(y_t|\theta)} \,.
\end{align}
In particular, by applying the exponential function, we get the \emph{tempered likelihood ratio},
\begin{align*}
\exp(-A_t^\beta(\theta)) = \left(\frac{p_t(y_t|\hat \theta_{t-1})}{p_t(y_t|\theta)}\right)^\beta
\end{align*}
As a side remark, Jensen's inequality implies that 
% \begin{align*}
$\bE[\exp(-A_t^\beta(\theta^*))] \leq  \bE\big[\frac{p_t(y_t|\hat \theta_{t-1})}{p_t(y_t|\theta^*)}\big]^\beta = 1$.
% \end{align*}
Hence $\prod_{s=1}^t A_t^\beta(\theta^*)$ is a super-martingale, and all results presented in the main paper continue to hold for the tempered ratio. However, we can strengthen the construction by enforcing the martingale property. Define
\begin{align*}
% M_t(\theta) = \frac{\exp(- \sum_{s=1}^t A_t^\beta(\theta))}{\textstyle \prod_{s=1}^t \bE_\theta[\exp(- A_t^\beta(\theta))|\cF_t]}
M_t(\theta) = \frac{\exp(- \sum_{s=1}^t A_t^\beta(\theta))}{\textstyle \prod_{s=1}^t \int \exp(- A_t^\beta(\theta)) p_t(y|\theta) dy} \,.
\end{align*}
By definition, $M_t(\theta^*)$ is a non-negative martingale. 
Hence, Ville's inequality implies that
\begin{align*}
   \bP\left[- \sum_{s=1}^t \log \int \exp(- A_s^\beta(\theta^*)) p_s(y|\theta^*) dy - \sum_{s=1}^t A_s(\theta^*) \geq \log \frac{1}{\delta}\right] \leq \delta \,.
\end{align*}
Finally, we note that 
\begin{align*}
   - \log \int \exp(- A_t^\beta(\theta)) p_t(y|\theta) dy = (1 - \beta) D_{\beta,t}(\hat \theta_{t-1} \| \theta)= \beta D_{1-\beta,t}(\theta\|\hat \theta_{t-1}) \,.
\end{align*}

\begin{theorem}[Tempered Confidence Set]\label{result:tempered} Let $\hat \theta_0, \hat \theta_1, \dots$ be an $\cF_t$-adapted sequence of estimators. Define the log regret $\Lambda_t(\theta) = - \sum_{s=1}^t \log p_s(y_s|\hat \theta_{s-1}) - L_t(\theta)$ for $\theta \in \Theta$, and
    \begin{align*}
        C_t^\beta = \left\{ \theta \in \Theta : \sum_{s=1}^t D_{1-\beta,s}(\theta\|\hat \theta_{s-1}) - \Lambda_t(\theta)\leq \frac{1}{\beta} \log \frac{1}{\delta}\right\} \,.
    \end{align*}
    Then $C_t^\beta$ defines a $(1-\delta)$-confidence sequence. Moreover, define
    \begin{align*}
        C_t^H = \left\{ \theta \in \Theta :\sum_{s=1}^t  H_s^2(\theta\|\hat \theta_{s-1}) - \Lambda_t(\theta)\leq 2 \log \frac{1}{\delta}\right\} \,.
    \end{align*}
    Then $C_t^H$ defines a $(1-\delta)$-confidence sequence.
\end{theorem}
As a consequence of the result, assume that the estimation sequence is constructed to achieve bounded regret, $\Lambda_t(\theta) \leq B_t$ for a predictable sequence $B_t \geq 0$, typically related to the complexity of the model class (c.f.,~\cref{sec:oco} and \cref{result:regret}), see also \citet{zhang2006varepsilon}. Then \cref{result:tempered} provides confidence sequences that only depend on the Hellinger or Renyi divergences. The advantage is, that unlike the likelihood ratios, the divergence is predictable quantity under the filtration $\cF_t$, hence can be used in sequential decision making setting to control the state $x_t$. This is one of the reasons why similar bounds have recently gained interest in the sequential decision-making literature \citep[e.g.,][]{chen2022unified,foster2021statistical,foster2023tight,wagenmaker2023instance}. In the next section, we provide another application to sparse estimation.

% \todoj{is there any advantage of keeping both the expectation and realized terms?}
% Note that $\sum_{s=1}^t A_t(\theta^*)$ is the log-loss regret and is controlled with probability 1. 
% Moreover, for $\beta=1/2$
% \begin{align*}
%     -\log \EE_t[ \exp(-A_t(\theta^*))] \geq 1 - \EE_t[ \exp(-A_t(\theta^*))] = \frac{1}{2} H^2(p_{\theta^*}, p_{\theta_t})
% \end{align*}
% Therefore
% \begin{align*}
%     \sum_{s=1}^t \frac{1}{2} H^2(p_{\theta^*}, p_{\theta_t}) \leq \log \frac{1}{\alpha} + \sum_{s=1}^t A_t(\theta^*) 
% \end{align*}
% The important difference is that the left-hand side does not depend on the last observation $y_t$, hence the confidence set is ``predictable'', unlike the likelihood ratio.

% Another way to proceed is

% note on introducing mixing distributions
% For $\beta \leq 1$,\todoj{mixing the tempered posterior vs tempering the mixed ratio}
% \begin{align}
%     \EE_{\theta \sim \mu_t}[ \left(\frac{p_{\theta}(y_t)}{p_{\theta^*}(y_t)} \right)^\beta] \leq  \left(\frac{\ip{\mu_t, p_t}}{p_{\theta^*}(y_t)} \right)^\beta 
% \end{align}


\subsection{Online Linear Prediction}\label{sec:sparse}
Recall the sequential linear regression setting from \cref{sec:bayes}. Specifically, assume that $\Theta \subset \bR^d$, and Gaussian likelihood $p_t(y|\theta) \sim \cN(\ip{\theta, x_t}, 1)$ for $\theta \in \Theta$ and covariates $x_t \in \bR^d$. Everything in this section generalizes to $\sigma$-sub-Gaussian noise distribution using the same arguments as in \cref{sec:subgaussian}. In a slight generalization to earlier results, assume that an online learning algorithm produces predictions $\hat y_t \in \cY$ (opposed to $\hat \theta_t \in \Theta$), in way such that the regret satisfies the following bound for any $\theta \in \Theta$,
\begin{align*}
    \Lambda_t(\theta) = \sum_{s=1}^t \frac{1}{2}(\hat y_{s-1} - y_s)^2 - \frac{1}{2}(\ip{\theta, x_s} - y_s)^2 \leq B_t(\theta) \,.
\end{align*}
For a concrete instantiation in the linear setting, we refer to the famous Vovk-Azoury-Warmuth forecaster \citep{vovk1997competitive,azoury2001relative}. We remark that the bound has been further generalized to non-parametric settings by \citet{rakhlin2014online}. We are now in place to generalize the online-to-confidence set conversion by \citet{abbasi2012online}.

\begin{lemma}[Online-To-Confidence Convertion] Assume the sequential linear regression setup defined above with a known bound $\Lambda_t(\theta^*) \leq B_t$. For any $\cF_t$-adapted sequence $\hat y_0, \hat y_1, \hat y_2, \dots,$ and  $0 < \beta < 1$, let
    \begin{align*}
        C_t = \left\{ \theta \in \Theta : \sum_{s=1}^t \frac{1}{2} \big(\hat y_{s-1} - \ip{\theta, x_s}\big)^2 \leq \frac{1}{\beta - \beta^2 } \log \frac{1}{\delta} + \frac{\beta}{\beta - \beta^2} B_t \right\}
    \end{align*}
    Then $C_t$ defines a $(1-\delta)$-confidence sequence.
\end{lemma}
\begin{proof}
The plan is to compute  $- \log \int \exp(- A_t^\beta(\theta)) p_t(y|\theta) dy$ for the Gaussian distribution, and then invoke \cref{result:tempered}. It is useful to note that for Gaussian distributed variable $\epsilon \sim \cN(0, \sigma^2)$ the moment generating function is $\bE[\exp(t \epsilon)] = \exp(\frac{1}{2}\sigma^2 t^2)$.
A short calculation reveals that 
\begin{align*}
    & \int \exp\big(- A_t^\beta(\theta)\big) p_t(y|\theta) dy\\
    &= \int \exp\Big(-  \frac{\beta}{2} \big(\hat y_{s-1} - y\big)^2 + \frac{\beta}{2} \big(\ip{\theta,x_t} - y\big)^2 \Big)  p_t(y|\theta) dy\\
    &= \int \exp\Big(-  \frac{\beta}{2} \big(\hat y_{s-1} - {\theta,x_t}\big)^2 + \beta \epsilon_t \big(\ip{\theta,x_t} - \ip{\theta,x_t}\big) \Big)  p_t(y|\theta) dy\\
    &= \exp\Big(-  \frac{\beta - \beta^2}{2} \big(\hat y_{s-1} - \ip{\theta,x_t}\big)^2 \Big) \,.
\end{align*}
Hence,
\begin{align*}
    & - \log \int \exp\big(- A_t^\beta(\theta)\big) p_t(y|\theta) dy =  \frac{\beta - \beta^2}{2} \big(\hat Y_{s-1} - \ip{\theta,x_t}\big)^2 \,.
\end{align*}
Lastly, we make use of the assumption that $\Lambda_t(\theta^*) \leq B_t$. The result follows by intersecting the confidence set $C_t^\beta$ from \cref{result:tempered} with  $\{\theta \in \Theta : \Lambda_t(\theta) \leq B_t\}$.
\end{proof}

\paragraph{Sparse Linear Regression} We make the additional assumption that the true parameter $\theta^* \in \Theta$ is $k$-sparse, i.e. $\|\theta^*\|_0 \leq k$. The key insight is that \citet{gerchinovitz2011sparsity} provides an online algorithm, producing the sequence $\hat y_0, \hat y_1, \dots$, for which $B_t(\theta^*) \leq  \cO( k \log(t))$. We compare to the result by \citet{abbasi2012online} in the same setting. For $\beta = \frac{1}{2}$, our result reads
\begin{align*}
   C_t = \left\{ \theta \in \Theta : \sum_{s=1}^t \frac{1}{2} (\hat y_{s-1} - \ip{\theta, x_s})^2 \leq 4\log \frac{1}{\delta} + 2 B_t(\theta^*) \right\} \,.
\end{align*}
In comparison, the result by \citet[Theorem 1]{abbasi2012online} reads, in our notation,
\begin{align*}
C_t = \left\{ \theta \in \mathbb{R}^d : \sum_{s=1}^{t}  \frac{1}{2}(\hat y_{s-1} - \langle \theta, x_t \rangle)^2 \leq 16 \log \frac{1}{\delta} + \frac{1}{2} + 2B_t(\theta^*) + 16\log (\sqrt{8} + \sqrt{1 + 2 B_t(\theta^*)}) \right\}
\end{align*}
The additional terms stem from using recursive inequality on the square-error, which we avoid using the more direct argument via the tempered likelihood ratio. This demonstrates the benefit of the sequential likelihood framework for deriving confidence sets via the online-to-confidence set conversion.
% Assume a sparse linear model $y_t = \ip{x_t, \theta^*} + \epsilon_t$ with $\sigma$-sub-Gaussian noise, i.e. satisfying $\bE[\exp(t \cdot \epsilon_t)] \leq \exp(\frac{\sigma^2 t^2}{2})$. The true parameter satisfies $\|\theta\|_0 \leq m$.
% Let $\hat Y_t \in \bR$ be a sequence of predictions from an online learning algorithm with regret bound
% \begin{align*}
%     \rho_n(\theta^*) = \sum_{t=1}^n (\hat Y_t - y_t)^2 - (\ip{\theta^*, x_t} - y_t)^2 = \sum_{t=1}^n r_t(\theta^*) \leq B_n
% \end{align*}
% where we defined $r_t(\theta^*) = (\hat Y_t - y_t)^2 - (\ip{\theta^*, x_t} - y_t)^2$. Note that
% \begin{align}
%     q_t := (\hat Y_t - \ip{\theta^*, x_t})^2 = r_t(\theta^*) + 2 \epsilon_t (\hat Y_t - \ip{\theta^*, x_t}) \label{eq:q_t-to-r_t}
% \end{align}
% Further define 
% \begin{align*}
% Q_n = \sum_{t=1}^n q_t = \sum_{t=1}^n (\hat Y_t - \ip{\theta^*, x_t})^2 
% \end{align*}
% Our goal is to find a high-probability upper bound for $Q_n$ and turn the bound into a confidence set for $\theta^*$.
% We define the following process for parameters $a,b \geq 0$:
% \begin{align}
%     M_t^{a,b} = \exp\left(\tfrac{b}{2} Q_t - \tfrac{b + a}{2} \rho_t(\theta^*)\right)
% \end{align}
% The choice that we make shortly is $a = b = \frac{1}{4 \sigma^2}$. Note that
% \begin{align*}
%     \EE[M_t^{a,b}|\cF_{t-1}] &= \EE[\exp\left(\tfrac{b}{2} Q_t - \tfrac{b + a}{2} \rho_t(\theta^*)\right) | \cF_t]\\
%     &= M_{t-1}^{a,b} \,\EE[\exp\left(\tfrac{b}{2} q_t - \tfrac{b + a}{2} r_t(\theta^*)\right) | \cF_t]\\
%     &= M_{t-1}^{a,b} \,\EE[\exp\left(- \tfrac{a}{2} q_t  +\tfrac{b + a}{2} q_t - \tfrac{b + a}{2} r_t(\theta^*)\right) | \cF_t]\\
%     &\stackrel{(1)}{=} M_{t-1}^{a,b} \,\EE[\exp\left(- \tfrac{a}{2} q_t  + (a+b) (\hat Y_t - \ip{\theta^*, x_t}) \epsilon_t \right) | \cF_t]\\
%     &\stackrel{(2)}{\leq} M_{t-1}^{a,b} \,\EE[\exp\left(- \tfrac{a}{2} q_t  + \sigma^2 \tfrac{(a+b)^2}{2} (\hat Y_t - \ip{\theta^*, x_t})^2 \right) | \cF_t]\\
%     &= M_{t-1}^{a,b}\, \EE[\exp\left(- \tfrac{a}{2} q_t  + \sigma^2 \tfrac{(a+b)^2}{2} q_t \right) | \cF_t]
% \end{align*}
% Equation $(1)$ follows from \ref{eq:q_t-to-r_t}; the $\frac{a+b}{2}$-factor was introduced to precisely cancel $r_t(\theta^*)$. The inequality $(2)$ follows by assumption that the noise is $\sigma$-sub-Gaussian. We now choose $a$ and $b$ to cancel the remaining terms out, i.e.

% \begin{align*}
% a = \sigma^2 (a+b)^2
% \end{align*}
% We can obtain the values of $a$ and $b$ by solving a quadratic, which gives as a general formula $a = \frac{1}{\sigma^2} - b \pm \frac{1}{2\sigma}\sqrt{\frac{1}{\sigma^2} - 4b}$ . An easy choice is $a=b=\frac{1}{4 \sigma^2}$. This is probably close to optimal (after writing out Ville's inequality it becomes intuitively clear  that we want large $b$ and small $a$), and $b=\frac{1}{4 \sigma^2}$ is also the largest feasible value for $b$. Applying Ville's inequality leads to the following bound:
% \begin{align}
% \bP[ M_t^{a,b} \geq \tfrac{1}{\alpha}] \leq \alpha
% \end{align}
% In other words, with probability at least $1-\delta$,
% \begin{align}
% \frac{1}{2} \sum_{t=1}^n (\hat Y_t - \ip{\theta^*, x_t})^2 \leq  4\sigma^2 \log \frac{1}{\alpha} + \rho_n(\theta^*) \leq  4\sigma^2  \log \frac{1}{\alpha} + B_n
% \end{align}


% For $a=b=\frac{1}{4 \sigma^2}$ we have
% \begin{align}
%     C_t^{\frac{1}{4},\frac{1}{4}} &= \left\{\theta : \frac{1}{2} \sum_{t=1}^n (\hat Y_t - \ip{\theta, x_t})^2 \leq  4 \sigma^2 \log \frac{1}{\alpha} + \rho_n(\theta^*)\right\} \\
%     &\subset \left\{\theta : \frac{1}{2} \sum_{t=1}^n (\hat Y_t - \ip{\theta, x_t})^2 \leq  4 \sigma^2 \log \frac{1}{\alpha} + B_n \right\} 
% \end{align}
% On the other hand, $a = 1$ and $b=0$ also leads to a valid confidence set, since $\exp(-\frac{1}{2}\rho_t(\theta^*))$ is the usual likelihood ratio for (sub-)Gaussian likelihood. Writing explicitly,
% \begin{align}
%     C_t = C_t^{0,1} &= \left\{\theta : \frac{1}{2}\rho(\theta) \leq  \log \frac{1}{\alpha} \right\} \\
%     &= \left\{\theta : \frac{1}{2} \sum_{t=1}^n (\hat Y_t - \ip{\theta, x_t})^2 \leq \log \frac{1}{\alpha} + \frac{1}{2} \sum_{t=1}^n (\hat Y_t - y_t)^2\right\} 
% \end{align}
% \section{Experiments: Planning outperforms Heuristics}
\label{sec:experiment}

We begin our empirical demonstrations by showcasing the effectiveness of our planning framework on both synthetic and real datasets. We focus on the simplest planning algorithm, 1-step lookaheads (Algorithm~\ref{alg:complete}), and show that even basic planning can hold great promise. 
We illustrate our framework using two uncertainty quantification modules---GPs and 
\ensembles/ \ensembleplus. 

Throughout this section, we focus on evaluating the mean squared error of 
a regression model $\model$,  and develop adaptive policies that minimize uncertainty on $g(f)$ defined in~\eqref{eqn:l2-g-f}.
When GPs provide a valid model of uncertainty, 
our experiments show that our planning framework significantly outperforms other baselines. 
We further demonstrate that our conceptual framework extends to deep learning-based uncertainty quantification methods such as  \ensembleplus while highlighting computational challenges that need to be resolved in order to scale our ideas. 
For simplicity, we assume a naive predictor, i.e., $\psi(\cdot) \equiv 0$. However, we emphasize that this problem is just as complex as if we were using a sophisticated model $\psi(.)$. The performance gap between the algorithms 
primarily depends
on the level  of uncertainty in our prior beliefs.

To evaluate the performance of our algorithm, we benchmark it against several baselines. 
%Active learning baselines use an acquisition function $\ac$ to select points that have the highest   function value: $X\opt_t \in \argmax_{X \in \xpoolj{t}} \ac({X})$ at every step $t$. These methods may also need an UQ module, which we simply use the same UQ module as in our algorithm, and it  outputs $V(X)$ that measures the the uncertainty of each point $X \in \xpoolj{t}$.
Our first set of baselines are from active learning~\citep{AggarwalKoGuHaPh14}:
\\ % \noindent\textbf{Active Learning Heuristics:} 
\textbf{(1)} 
\textsf{Uncertainty Sampling (Static):}  In this approach, we query the samples for which the model is least certain about. Specifically, we estimate the variance of the latent output $f(X)$ for each $X \in \xpool$ using the UQ module and select the top-$K$ points with the highest uncertainty. \\
\textbf{(2)} \textsf{Uncertainty Sampling (Sequential):} This is a greedy heuristic that sequentially selects the points with the highest uncertainty within a batch, while updating the posterior beliefs using pseudo labels from the current posterior state. Unlike \textsf{Uncertainty Sampling (Static)}, this method takes into account the information gained from each point within batch, and hence tries to diversify the selected points within a batch. 

 
We also compare our approach to the  \textbf{(3)} \textsf{Random Sampling}, which selects each batch uniformly at random from the pool. Additionally, we compare solving the planning problem using  \textsf{REINFORCE}-based policy gradients with   $\mathsf{Smoothed\text{-}Autodiff}$ policy gradients.\footnote{Our code repository is available at
  \url{https://github.com/namkoong-lab/adaptive-labeling}.}
%Detailed experimental setups are provided in Section \ref{sec:details-experiments}.

%We repeat all experiments with 10 random seeds.




\begin{figure}[t]
\centering
\begin{minipage}[b]{0.49\textwidth}
\centering
\includegraphics[width=\textwidth, height=5cm]{figures/original_scale/Var_of_l_2_loss.pdf}
\caption{(Synthetic data) Variance of mean squared loss evaluated through the posterior belief $\mu_t$ at each horizon $t$. This is the objective that policy gradient methods like \textsf{REINFORCE} and $\ouralgo$ optimizes. 1-step lookaheads are surprisingly effective even in long horizons.}
\label{fig:var-l2-sim}
\end{minipage}
\hfill
\begin{minipage}[b]{0.49\textwidth}
\centering \includegraphics[width=\textwidth, height=5cm]{figures/original_scale/Error_of_estimated_model_l_2_loss.pdf}
\caption{(Synthetic data) Error between MSE calculated based on collected data $\mc{D}^{0:T}$ vs. population oracle MSE over $\mc{D}_{\rm eval} \sim P_X$. Reducing uncertainty over posteriors directly leads to better OOD evaluations. 1-step lookaheads significantly outperform active learning heuristics in small horizons.}
\label{fig:mean-l2-sim}
\end{minipage}
%\caption{Simulated data for GPs}
%\label{fig:both_plots}
\end{figure}

\subsection{Planning with Gaussian processes}
\label{sec:experiment-plan-GP}
We now briefly describe the data generation process for the GP experiments,  deferring a more detailed discussion of the dataset generation to Section~\ref{sec:details-experiments}. 
We use both the synthetic data and the real data to test our methodology.
For the \emph{simulated data},  we construct a setting where the general population is distributed across \emph{51 non-overlapping clusters} while the initial labeled data $\dtrain$ just comes from one cluster. In contrast, both $\dpool \defeq (\xpool,\ypool),\deval \defeq (\xeval,\yeval)$ are generated   from all the clusters. 
We begin with a low-dimensional scenario, generating a one-dimensional regression setting using a GP. %Gaussian Process (GP).
Although the data-generating process is not known to the algorithms,  we assume that the GP hyperparameters are known to all the algorithms
to ensure fair comparisons. This can be viewed as a setting where our prior is well-specified, allowing us to isolate the effects
of different policy optimization approaches
 without any concerns about the misspecified priors. We select $10$ batches, each of size $K=5$ across $T = 10$ time horizons.

To examine the robustness of our method against the distributional assumptions made  in the simulated case, we then move to a real dataset where the correct prior is not known. We simulate selection bias from the eICU dataset~\citep{PollardJoRaCeMaBa18}, which contains real-world patient data with in-hospital mortality outcomes. 
We conduct a $k$-means clustering to generate 51 clusters and then select data from those clusters. We view this to be a credible replication of practice, as severe distribution shifts are common due to selection bias in clinical labels.  To convert the binary mortality labels into a regression setting, we train a  random forest classifier and fit a GP on predicted scores, which serves as the UQ module for all the algorithms. As before, the task is to select 10 batches, each consisting of 5 samples, across 10 time horizons.

 In Figures~\ref{fig:var-l2-sim} and~\ref{fig:mean-l2-sim}, we present results for the simulated data. 
Figure~\ref{fig:var-l2-sim} shows the variance of $\ell_2$ loss, and Figure~\ref{fig:mean-l2-sim} presents the error in the estimated $\ell_2$ loss using $\mu_t$ (relative to true $\ell_2$ loss, that is unknown to the algorithm). 
As we can see from these plots, our method one-step lookahead  gives substantial improvements  over active learning baselines and random sampling. In addition,
compared to the one-step lookahead planning approach using \textsf{REINFORCE}-based policy gradients, 
we observe that $\mathsf{Smoothed\text{-}Autodiff}$-based policy gradients provide significantly more robust performance over all horizons.

In Figures~\ref{fig:var-l2-real}~and~\ref{fig:mean-l2-real}, we observe similar findings on the eICU data. We see that planning policies (\textsf{REINFORCE} and $\mathsf{Smoothed\text{-}Autodiff}$) consistently outperform other heuristics by a large margin.  Active learning baselines perform poorly in these small-horizon batched problems and can sometimes be even worse than the random search baselines.  Overall, our results show the importance of careful planning in adaptive labeling for reliable model evaluation. 

We offer some intuition as to why one-step lookahead planning may outperform other heuristic algorithms. 
 First,  \textsf{Uncertainty sampling (Static)} while myopically selects the
 top-$K$ inputs with the highest uncertainty, it fails to consider 
the overlap in information content among the ``best” instances; see \citep{AggarwalKoGuHaPh14} for more details. 
In other words,  it might acquire points from the same region with high uncertainty while failing to induce diversity among the batch.
Although \textsf{Uncertainty Sampling (Sequential)} somewhat addresses the issue of information overlap, a significant drawback of 
this algorithm
is the disconnect between the objective we aim to optimize and the algorithm. For example, it might sample from a region with high uncertainty but very low density. 

\begin{figure}[t]
\centering
\begin{minipage}[b]{0.48\textwidth}
\centering
\includegraphics[width=\textwidth, height=5cm]{figures/original_scale/Var_of_l_2_loss_real.pdf}
\caption{(Real-world eICU data) Variance of mean squared loss evaluated through the posterior belief $\mu_t$ at each horizon $t$. Even 1-step lookaheads are extremely effective planners, and auto-differentiation-based pathwise policy gradients provide a reliable optimization algorithm based on low-variance gradient estimates.}
\label{fig:var-l2-real}
\end{minipage}
\hfill
\begin{minipage}[b]{0.48\textwidth}
\centering \includegraphics[width=\textwidth, height=5cm]{figures/original_scale/Error_of_estimated_model_l_2_loss_real.pdf}
\caption{(Real-world eICU data) Error between MSE calculated based on collected data $\mc{D}^{0:T}$ vs. population oracle MSE over $\mc{D}_{\rm eval} \sim P_X$. Reducing uncertainty over posteriors directly leads to better OOD evaluations. Our method significantly outperforms active learning-based heuristics, and random sampling.}
\label{fig:mean-l2-real}
\end{minipage}
%\caption{Real data for GPs}
\end{figure}
 
%\vspace{-1.5cm}
% \begin{wrapfigure}{r}{.32\columnwidth}
%   \vspace{-.5cm} 
%   \centering
% \includegraphics[scale=.29]{figures/Var of l2l_2 loss.pdf}
%   \vspace{-0.2cm}
%   \caption{Results of GP}
% \label{fig:var-l2-gp}
%   \vspace{-0.1cm}
% \end{wrapfigure}


% Attempts have been made  in the past to address these  drawbacks heuristically  (see \citep{AggarwalKoGuHaPh14}). We give a unified computational framework while approaching the problem in a more principled manner and solving it more optimally.




\subsection{Planning with  neural network-based uncertainty quantification methods ($\ensembleplus$)}


We now provide a proof-of-concept that shows the generalizability of our conceptual framework  to the deep learning-based UQ modules, specifically focusing on $\ensembleplus$ due to their previously observed superior performance~\citep{OsbandWenAsDwIbLuRo23}. Recall that implementing our framework with deep learning-based UQ modules  requires us to retrain the model across multiple possible random actions $\bm{a}(\theta)$ sampled from the current policy $\pi_\theta$.
This requires significant computational resources, in sharp contrast to the GPs where the posteriors are in closed form and can be readily updated and differentiated. 

Due to the computational constraints, we test $\ensembleplus$ on a toy setting to demonstrate the generalizability of our framework. We consider a setting where the general population consists of four clusters, while the initial labeled data only comes from one cluster. Again we generate data using GPs.  The task is to select a batch of 2 points in one horizon. We detail the $\ensembleplus$ architecture in Section \ref{sec:details-experiments}, and we assume prior uncertainty to be large (depends on the scaling of the prior generating functions). 
The results are summarized in the Table~\ref{tab:UQ_ensemble}.

% \begin{table}[H]
% \vspace{-10pt}
% \caption{Performance under \ensembleplus as UQ module}
%     \centering
%     \begin{tabular}{|m{3cm}|m{2.5cm}|m{2cm}|} 
%     \hline
%       Algorithm   & Variance of $\loss_2$ loss estimate & Error of $\loss_2$ loss estimate  \\ \hline Random Sampling 
%          & $1710.9 \pm 1352.1$ & $8.67\pm6.62$ 
%       \\ \hline \ouralgo & $1.30 \pm 0.68$ & $0.91\pm0.25$ \\ \hline
%     \end{tabular}
%     \label{tab:UQ_ensemble}
%     %\vspace{-10pt}
% \end{table}




\begin{table}[h]
\vspace{-10pt}
\caption{Performance under \ensembleplus as the UQ module}
\centering
\begin{tabular}{|l|l|l|}
\hline
Algorithm   & Variance of $\loss_2$ loss estimate & Error of $\loss_2$ loss estimate  \\
\hline
\textsf{Random sampling} & 7129.8 $\pm$ 1027.0 & 136.2 $\pm$ 8.28 \\ \hline
\textsf{Uncertainty sampling (Static)} & 10852 $\pm$ 0.0 & 162.156 $\pm$ 0.0 \\ \hline
\textsf{Uncertainty sampling (Sequential)} & 8585.5 $\pm$ 898.9 & 144 $\pm$ 6.93 \\ \hline
\textsf{REINFORCE} & 1697.1 $\pm$ 0.0 & 45.27 $\pm$ 0.0 \\ \hline
\ouralgo & 1697.1 $\pm$ 0.0 & 45.27 $\pm$ 0.0 \\ \hline
\end{tabular}
%\caption{Comparison of different algorithms based on variance   and   error in $\ell_2$ loss estimation with Ensemble $+$ as the UQ module. Our results demonstrate that {\ouralgo} and REINFORCE outperformthe other active learning based heuristics, confirming the benefits of our MDP formulation for the adaptive labeling problem, as also demonstrated in Section 4.\\
%\footnotesize{Experimental details: We use Gaussian Processes as our data generating process, GP parameters are the same as in Section D.3.  The task is to select a batch of 2 points along one horizon.The marginal distribution $p_X$ has 4 \textit{non-overlapping} clusters. Initial data comes from one cluster, while pool and evaluation points comes from all the clusters. We have $20$ initial labeled data points, $10$ pool points, and $252$ evaluation points.  Training procedures are similar to the one in Section D.3.} }
\label{tab:UQ_ensemble}
\end{table}



% We faced  issues in scaling up these experiments which will be our focus in the future. 





% \begin{itemize}
%     \item Posteriors should be consistent. Two dimensions: even with less training,  
%     \item the inference should be  fast enough
% \end{itemize}


% Potential research directions for uncertainty quantification

% In this section we consider a simple setting We consider a simpler setting and 


% For synthetic dataset generation, we use ...... For real datasets, we use ...... We compare our methodolgy to several baselines ()    This Section is structured as follows:
% \begin{itemize}
%     \item \textbf{GPs, square loss objective} (Section \ref{}): 
%     %the broad aim of the experiments  in this section is to isolate the performance of our methodology without any concerns for the inefficiencies induced due to a mis-specified prior or imperfect posterior inference. To accomplish this we generate synthetic datasets using GPs (detailed later). We use the well specified prior (GPs - with same hyperparameter setting) as our UQ module.   
%      As GPs provide differentaible posterior inference - any errors induced due to imperfect posterior updates are also isolated. We note that under this setting
%      \item In Section\ref{} we demonstrate why our methodology performs better than other baselines - by devising various synthetic experiments ()
%     \item  \textbf{UQ Benchmarking }(Section \ref{}): Before diving into the experiments using $\ensembleplus$ and ENNs,  we showcase our benchmarking experiments in Section \ref{}. We use real datasets We observe that ENNs perform better
%      \item \textbf{Ensemble $+$}, objective: recall, accuracy
%     \item \textbf{ENN}, objective: recall, accuracy
% \end{itemize}




% In Section {}, we test 
% \subsection{Experimental details}

% \begin{itemize}
%     \item UQ methodologies - GPs, ENNs
%     \item Objectives - Recall,  ATE
%     \item Datasets - ATE-synthetic datasets, Recall-synthetic, real datasets
%     \item Baselines - 
%     \begin{itemize}
%         \item Random sampling
%         \item Active learning - Uncertainty based sampling - In regression setting almost all of the 
%         \item Myopic greedy - Greedy Batch based sampling
%         \item Policy Gradient
%     \end{itemize}
    
% \end{itemize}

% \subsection{Experiments}
%     \begin{itemize}
%     \item GPs with square loss
%     \item Benchmarking ENN
%         \item ENNs with ATE
%         \item ENNs with Recall
%     \end{itemize}

% \subsection{Benefits over other algorithms - intuition and experiments}

%Active learning - Myopic greedy / Don't rely on the objective rather some entropy version.


%%% Local Variables:
%%% mode: latex
%%% TeX-master: "main"
%%% End:


\end{document}
