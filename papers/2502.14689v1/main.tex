\documentclass[draft,12pt,cleveref]{colt2025} % Anonymized submission
%\documentclass[final,12pt]{colt2025} % Include author names

% The following packages will be automatically loaded:
% amsmath, amssymb, natbib, graphicx, url, algorithm2e


% \usepackage[capitalise]{cleveref}
\title[Confidence Estimation via Sequential Likelihood Mixing]{Confidence Estimation via Sequential Likelihood Mixing}
% \title[Sequential Confidence Estimation]{Universal Confidence Estimation via Sequential Likelihood Mixing}
% \title[Sequential Confidence Estimation]{Universal Confidence Estimation via Sequential Likelihood Mixing\\Applications, Connections and New Perspectives}
\usepackage{times}

% Custom package with our own macros %Michele
\usepackage{mystyle}
\usepackage{algorithm2e}
\usepackage{subcaption}
\usepackage{caption}


% \makeatletter
% \renewenvironment{quotation}
%                {\list{}{\listparindent=15pt%whatever you need
%                         \itemindent    \listparindent
%                         \leftmargin=10pt%  whatever you need
%                         \rightmargin=10pt%whatever you need
%                         \topsep=5pt%%%%%  whatever you need
%                         \parsep        \z@ \@plus\p@}%
%                 \item\relax}
%                {\endlist}
% \makeatother

% Use \Name{Author Name} to specify the name.
% If the surname contains spaces, enclose the surname
% in braces, e.g. \Name{John {Smith Jones}} similarly
% if the name has a "von" part, e.g \Name{Jane {de Winter}}.
% If the first letter in the forenames is a diacritic
% enclose the diacritic in braces, e.g. \Name{{\'E}louise Smith}

% Two authors with the same address
% \coltauthor{\Name{Author Name1} \Email{abc@sample.com}\and
%  \Name{Author Name2} \Email{xyz@sample.com}\\
%  \addr Address}

% Three or more authors with the same address:
% \coltauthor{\Name{Author Name1} \Email{an1@sample.com}\\
%  \Name{Author Name2} \Email{an2@sample.com}\\
%  \Name{Author Name3} \Email{an3@sample.com}\\
%  \addr Address}

% Authors with different addresses:
\coltauthor{%
 \Name{Johannes Kirschner} \Email{johannes.kirschner@sdsc.ethz.ch}\\
 \addr Swiss Data Science Center%, Andreasstrasse 5, 8092 Zürich, Switzerland
 \AND
 \Name{Andreas Krause} \Email{krausea@ethz.ch}\\
 \addr ETH Zurich, Department of Computer Science%
 \AND
 \Name{Michele Meziu} \Email{mezium@student.ethz.ch}\\
 \addr ETH Zurich, Department of Computer Science%
 \AND
 \Name{Mojmir Mutný} \Email{mmutny@inf.ethz.ch}\\
 \addr ETH Zurich, Department of Computer Science%
}


% todonotes
\usepackage[textsize=tiny]{todonotes}
\usepackage{layouts}
\setlength{\marginparwidth}{2cm}
\newcommand{\todom}[2][]{\todo[size=\scriptsize,color=blue!20!white,#1]{M: #2}}
\newcommand{\todoj}[2][]{\todo[size=\scriptsize,color=green!20!white,#1]{J: #2}}
\newcommand{\todomo}[2][]
{\todo[size=\scriptsize,color=purple!20!white,#1]{Mo: #2}}

\newtheorem{assumption}{Assumption}

\begin{document}

\maketitle

\begin{abstract}%
We present a universal framework for constructing confidence sets based on sequential likelihood mixing. Building upon classical results from sequential analysis, we provide a unifying perspective on several recent lines of work, and establish fundamental connections between sequential mixing, Bayesian inference and regret inequalities from online estimation. The framework applies to any realizable family of likelihood functions and allows for non-i.i.d.~data and anytime validity. Moreover, the framework seamlessly integrates standard approximate inference techniques, such as variational inference and sampling-based methods, and extends to misspecified model classes, while preserving provable coverage guarantees. We illustrate the power of the framework by deriving tighter confidence sequences for classical settings, including sequential linear regression and sparse estimation, with simplified proofs.
\end{abstract}

\begin{keywords}%
  confidence sequences, likelihood ratios, Bayesian inference, online estimation
\end{keywords}

\section{Introduction}
\label{sec:introduction}
The business processes of organizations are experiencing ever-increasing complexity due to the large amount of data, high number of users, and high-tech devices involved \cite{martin2021pmopportunitieschallenges, beerepoot2023biggestbpmproblems}. This complexity may cause business processes to deviate from normal control flow due to unforeseen and disruptive anomalies \cite{adams2023proceddsriftdetection}. These control-flow anomalies manifest as unknown, skipped, and wrongly-ordered activities in the traces of event logs monitored from the execution of business processes \cite{ko2023adsystematicreview}. For the sake of clarity, let us consider an illustrative example of such anomalies. Figure \ref{FP_ANOMALIES} shows a so-called event log footprint, which captures the control flow relations of four activities of a hypothetical event log. In particular, this footprint captures the control-flow relations between activities \texttt{a}, \texttt{b}, \texttt{c} and \texttt{d}. These are the causal ($\rightarrow$) relation, concurrent ($\parallel$) relation, and other ($\#$) relations such as exclusivity or non-local dependency \cite{aalst2022pmhandbook}. In addition, on the right are six traces, of which five exhibit skipped, wrongly-ordered and unknown control-flow anomalies. For example, $\langle$\texttt{a b d}$\rangle$ has a skipped activity, which is \texttt{c}. Because of this skipped activity, the control-flow relation \texttt{b}$\,\#\,$\texttt{d} is violated, since \texttt{d} directly follows \texttt{b} in the anomalous trace.
\begin{figure}[!t]
\centering
\includegraphics[width=0.9\columnwidth]{images/FP_ANOMALIES.png}
\caption{An example event log footprint with six traces, of which five exhibit control-flow anomalies.}
\label{FP_ANOMALIES}
\end{figure}

\subsection{Control-flow anomaly detection}
Control-flow anomaly detection techniques aim to characterize the normal control flow from event logs and verify whether these deviations occur in new event logs \cite{ko2023adsystematicreview}. To develop control-flow anomaly detection techniques, \revision{process mining} has seen widespread adoption owing to process discovery and \revision{conformance checking}. On the one hand, process discovery is a set of algorithms that encode control-flow relations as a set of model elements and constraints according to a given modeling formalism \cite{aalst2022pmhandbook}; hereafter, we refer to the Petri net, a widespread modeling formalism. On the other hand, \revision{conformance checking} is an explainable set of algorithms that allows linking any deviations with the reference Petri net and providing the fitness measure, namely a measure of how much the Petri net fits the new event log \cite{aalst2022pmhandbook}. Many control-flow anomaly detection techniques based on \revision{conformance checking} (hereafter, \revision{conformance checking}-based techniques) use the fitness measure to determine whether an event log is anomalous \cite{bezerra2009pmad, bezerra2013adlogspais, myers2018icsadpm, pecchia2020applicationfailuresanalysispm}. 

The scientific literature also includes many \revision{conformance checking}-independent techniques for control-flow anomaly detection that combine specific types of trace encodings with machine/deep learning \cite{ko2023adsystematicreview, tavares2023pmtraceencoding}. Whereas these techniques are very effective, their explainability is challenging due to both the type of trace encoding employed and the machine/deep learning model used \cite{rawal2022trustworthyaiadvances,li2023explainablead}. Hence, in the following, we focus on the shortcomings of \revision{conformance checking}-based techniques to investigate whether it is possible to support the development of competitive control-flow anomaly detection techniques while maintaining the explainable nature of \revision{conformance checking}.
\begin{figure}[!t]
\centering
\includegraphics[width=\columnwidth]{images/HIGH_LEVEL_VIEW.png}
\caption{A high-level view of the proposed framework for combining \revision{process mining}-based feature extraction with dimensionality reduction for control-flow anomaly detection.}
\label{HIGH_LEVEL_VIEW}
\end{figure}

\subsection{Shortcomings of \revision{conformance checking}-based techniques}
Unfortunately, the detection effectiveness of \revision{conformance checking}-based techniques is affected by noisy data and low-quality Petri nets, which may be due to human errors in the modeling process or representational bias of process discovery algorithms \cite{bezerra2013adlogspais, pecchia2020applicationfailuresanalysispm, aalst2016pm}. Specifically, on the one hand, noisy data may introduce infrequent and deceptive control-flow relations that may result in inconsistent fitness measures, whereas, on the other hand, checking event logs against a low-quality Petri net could lead to an unreliable distribution of fitness measures. Nonetheless, such Petri nets can still be used as references to obtain insightful information for \revision{process mining}-based feature extraction, supporting the development of competitive and explainable \revision{conformance checking}-based techniques for control-flow anomaly detection despite the problems above. For example, a few works outline that token-based \revision{conformance checking} can be used for \revision{process mining}-based feature extraction to build tabular data and develop effective \revision{conformance checking}-based techniques for control-flow anomaly detection \cite{singh2022lapmsh, debenedictis2023dtadiiot}. However, to the best of our knowledge, the scientific literature lacks a structured proposal for \revision{process mining}-based feature extraction using the state-of-the-art \revision{conformance checking} variant, namely alignment-based \revision{conformance checking}.

\subsection{Contributions}
We propose a novel \revision{process mining}-based feature extraction approach with alignment-based \revision{conformance checking}. This variant aligns the deviating control flow with a reference Petri net; the resulting alignment can be inspected to extract additional statistics such as the number of times a given activity caused mismatches \cite{aalst2022pmhandbook}. We integrate this approach into a flexible and explainable framework for developing techniques for control-flow anomaly detection. The framework combines \revision{process mining}-based feature extraction and dimensionality reduction to handle high-dimensional feature sets, achieve detection effectiveness, and support explainability. Notably, in addition to our proposed \revision{process mining}-based feature extraction approach, the framework allows employing other approaches, enabling a fair comparison of multiple \revision{conformance checking}-based and \revision{conformance checking}-independent techniques for control-flow anomaly detection. Figure \ref{HIGH_LEVEL_VIEW} shows a high-level view of the framework. Business processes are monitored, and event logs obtained from the database of information systems. Subsequently, \revision{process mining}-based feature extraction is applied to these event logs and tabular data input to dimensionality reduction to identify control-flow anomalies. We apply several \revision{conformance checking}-based and \revision{conformance checking}-independent framework techniques to publicly available datasets, simulated data of a case study from railways, and real-world data of a case study from healthcare. We show that the framework techniques implementing our approach outperform the baseline \revision{conformance checking}-based techniques while maintaining the explainable nature of \revision{conformance checking}.

In summary, the contributions of this paper are as follows.
\begin{itemize}
    \item{
        A novel \revision{process mining}-based feature extraction approach to support the development of competitive and explainable \revision{conformance checking}-based techniques for control-flow anomaly detection.
    }
    \item{
        A flexible and explainable framework for developing techniques for control-flow anomaly detection using \revision{process mining}-based feature extraction and dimensionality reduction.
    }
    \item{
        Application to synthetic and real-world datasets of several \revision{conformance checking}-based and \revision{conformance checking}-independent framework techniques, evaluating their detection effectiveness and explainability.
    }
\end{itemize}

The rest of the paper is organized as follows.
\begin{itemize}
    \item Section \ref{sec:related_work} reviews the existing techniques for control-flow anomaly detection, categorizing them into \revision{conformance checking}-based and \revision{conformance checking}-independent techniques.
    \item Section \ref{sec:abccfe} provides the preliminaries of \revision{process mining} to establish the notation used throughout the paper, and delves into the details of the proposed \revision{process mining}-based feature extraction approach with alignment-based \revision{conformance checking}.
    \item Section \ref{sec:framework} describes the framework for developing \revision{conformance checking}-based and \revision{conformance checking}-independent techniques for control-flow anomaly detection that combine \revision{process mining}-based feature extraction and dimensionality reduction.
    \item Section \ref{sec:evaluation} presents the experiments conducted with multiple framework and baseline techniques using data from publicly available datasets and case studies.
    \item Section \ref{sec:conclusions} draws the conclusions and presents future work.
\end{itemize}
% We study (stochastic) gradient descent on the empirical risk
\begin{equation*}
\cL(w) = \frac{1}{n}\sum_{i=1}^n l(p_i(w))\, ,
\end{equation*}
where the loss function $l$ and the functions  $(p_i)_{i=1}^n$  are specified in the following assumptions. Note that the empirical risk for binary classification from Equation~\eqref{def:emp_risk_intro} is a special case of the above objective.

\begin{assumption}\label{hyp:loss_exp_log}\phantom{=}
  \begin{enumerate}[label=\roman*)]
    \item The loss is either the exponential loss, $l(q) = e^{-q}$, or the logistic loss, $l(q) = \log(1{+}e^{-q})$.
    \item There exists an integer $L \in \mathbb{N}^*$  such that, for all $1 \leq i \leq n$, the function $p_i$ is $L$-homogeneous\footnote{We recall that a mapping $f : \mathbb{R}^d \rightarrow \mathbb{R}$ is positively $L$-homogeneous if $f(\lambda w) = \lambda^L f(w)$ for all $w \in \mathbb{R}^d$ and $\lambda >0$.}, locally Lipschitz continuous and semialgebraic.
  \end{enumerate}
\end{assumption}
If the $p_i$'s were differentiable with respect to $w$, the chain rule would guarantee that
\begin{align*}
\nabla \mathcal{L}(w) = \frac{1}{n}\sum_{i=1}^n  l'(p_i(w)) \nabla p_i(w)\enspace.
\end{align*}
However, we only assume that the $p_i$'s are semialgebraic. While we could consider Clarke subgradients, the Clarke subgradient of operations on functions (e.g., addition, composition, and minimum) is only contained within the composition of the respective Clarke subgradients. This, as noted in Section~\ref{sec:cons_field}, implies that the output of backpropagation is usually not an element of a Clarke subgradient but a selection of some conservative set-valued field.
Consequently, for $1\leq i \leq n$, we consider $D_i : \bbR^d \rightrightarrows\bbR^d$, a conservative set-valued field of $p_i$, and a function $\sa_i : \bbR^d \rightarrow \bbR^d$ such that for all $w \in \bbR^d$, $\sa_i(w) \in D_i(w)$. Given a step-size $\gamma >0$, gradient descent (GD)\footnote{More precisely, this refers to conservative gradient descent. We use the term GD for simplicity, as conservative gradients behave similarly to standard gradients.} is then expressed as
\begin{equation*}\label{eq:gd_new}\tag{GD}
  w_{k+1} = w_k - \frac{\gamma}{n} \sum_{i=1}^n l'(p_i(w_k))\sa_i(w_k)\,.
\end{equation*}
For its stochastic counterpart, stochastic gradient descent (SGD), we fix a batch-size $1\leq n_b \leq n$. At each iteration $k \in \bbN$, we randomly and uniformly draw a batch $B_k \subset \{1, \ldots, n \}$ of size $n_b$. The update rule is then given by 
\begin{equation*}\label{eq:sgd_new}\tag{SGD}
  w_{k+1} = w_k -  \frac{\gamma}{n_b}\sum_{i\in B_k} l'(p_i(w_k)) \sa_i(w_k)\, .
\end{equation*}
The considered conservative set-valued fields will satisfy an Euler lemma-type assumption.
%\nic{Smoother transition}
\begin{assumption}\phantom{=}\label{hyp:conserv}
  For every $i \leq n$, $\sa_i$ is measurable and $D_i$ is semialgebraic. Moreover, for every $w \in \bbR^d$ and $\lambda \geq 0$, $\sa_i(w)  \in D_i(w)$,
  \begin{equation*}
    D_i(\lambda w) = \lambda^{L-1} D_i(w)\, , \textrm{ and } \quad   L p_i(w) = \scalarp{\sa_i(w)}{w}\, .
  \end{equation*}
\end{assumption}
%\nic{Smoother transition}
Having in mind the binary classification setting, in which $p_i(w) = y_i \Phi(x_i, w)$, we define the margin
\begin{equation}\label{def:marg}
  \sm: \bbR^d \rightarrow \bbR, \quad \sm(w) = \min_{1\leq i \leq n} p_i(w)\, .
\end{equation}
It quantifies the quality of a prediction rule $\Phi(\cdot, w)$. In particular,  the training data is perfectly separated when $\sm(w) >0$. A binary prediction for $x$ is given by the sign of $\Phi(x, w)$, and under the homogeneity assumption, it depends only on the normalized direction $w / \norm{w}$. Consequently, we will focus on the sequence of directions $u_k := w_k / \norm{w_k}$. Our final assumption ensures that the normalized directions $(u_k)$ have stabilized in a region where the training data is correctly classified.

\begin{assumption}\label{hyp:marg_lowb}
  Almost surely, $\liminf \sm(u_k) >0$.
\end{assumption}
Before presenting our main result, we comment on our assumptions.

\paragraph{On Assumption~\ref{hyp:loss_exp_log}.} As discussed in the introduction, the primary example we consider is when $p_i(w) = y_i \Phi(x_i;w)$ is the signed prediction of a feedforward neural network without biases and with piecewise linear activation functions on a labeled dataset $((x_i,y_i))_{i \leq n}$. In this case,
\begin{equation}\label{eq:NN}
 p_i(w) = y_i \Phi(w;x_i) = y_i V_L(W_L) \sigma(V_{L-1}(W_{L-1}) \sigma(V_{L-1}(W_{L-2}) \ldots \sigma(V_{1}(W_1 x_i))))\, ,
\end{equation}
where $w = [W_1, \ldots, W_L]$, $W_i$ represents the weights of the $i$-th layer, $V_i$ is a linear function in the space of matrices (with $V_i$ being the identity for fully-connected layers) and $\sigma$ is a coordinate-wise activation function such as $z \mapsto \max(0,z)$ ($\ReLU$), $z \mapsto \max(az, z)$ for a small parameter $a>0$ (LeakyReLu) or $z \mapsto z$. Note that the mapping $w \mapsto p_i(w)$ is semialgebraic and $L$-homogeneous for any of these activation functions. Regarding the loss functions, the logistic and exponential losses are among the most commonly studied and widely used. In Appendix~\ref{app:gen_sett}, we extend our results to a broader class of losses, including $l(q) = e^{-q^a}$ and $l(q) = \ln (1 + e^{-q^a})$ for any $a \geq 1$.

\paragraph{On Assumption~\ref{hyp:conserv}.} Assumption~\ref{hyp:conserv} holds automatically  if $D_i$ is the Clarke subgradient of $p_i$. Indeed, at any vector $w \in \bbR^d$, where $p_i$ is differentiable it holds that $p_i(\lambda w) = \lambda^{L} p_i(w)$. Differentiating relatively to $w$ and $\lambda$ (noting that $p_i$ remains differentiable at $\lambda w$ due to homogeneity), we obtain $\lambda \nabla p_i(\lambda w) = \lambda^{L} \nabla p_i(w)$ and $\scalarp{\nabla p_i(\lambda w)}{w} = L \lambda^{L-1} p_i(w)$. The expression for any element of the Clarke subgradient then follows from~\eqref{eq:def_clarke}. 

However, for an arbitrary conservative set-valued field, Assumption~\ref{hyp:conserv} does not necessarily hold. For instance, $D(x) = \mathds{1}(x \in \mathbb{N})$ is a conservative set-valued field for $p \equiv 0$, which does not satisfy Assumption~\ref{hyp:conserv}. Nevertheless, in practice, conservative set-valued fields naturally arise from a formal application of the chain rule. For a non-smooth but homogeneous activation function $\sigma$, one selects an element $e \in \partial \sigma (0)$, and computes $\sa_i(w)$ via backpropagation. Whenever a gradient candidate of $\sigma$ at zero is required (i.e., in~\eqref{eq:NN}, for some $j$, $V_j(W_j)$ contains a zero entry), it is replaced by $e$. 
Since $V_j(W_j)$ and $V_j(\lambda W_j)$ have the same zero elements, it follows that for every such $w$, $
\sa_i(\lambda w) = \lambda^L \sa_i(w)$. The conservative set-valued field $D_i$ is then obtained by associating to each $w$ the set of all possible outcomes of the chain rule, with $e$ ranging over all elements of $\partial \sigma(0)$. Thus, for such fields, Assumption~\ref{hyp:conserv} holds.


\paragraph{On Assumption~\ref{hyp:marg_lowb}.} Training typically continues even after the training error reaches zero.
Assumption~\ref{hyp:marg_lowb} characterizes this late-training phase, where our result applies. 
As noted earlier, since $\sm$ is $L$-homogeneous, the classification rule is determined by the direction of the  iterates $u_k=w_k/\norm{w_k}$. Assumption~\ref{hyp:marg_lowb} then states that, beyond some iteration, the normalized margin remains positive. 
This assumption is natural in the context of studying the implicit bias of SGD: we \emph{assume} that we reached the phase in which the dataset is correctly classified and \emph{then} characterize the limit points. A similar perspective was taken in  \cite{nacson2019lexicographic}, where the implicit bias of GF was analyzed under the assumption that the sequence of directions and the loss converge. However, unlike their approach, ours does not require assuming such convergence a priori.

Earlier works such as \cite{ji2020directional,Lyu_Li_maxmargin}, which analyze subgradient flow or smooth GD, establish convergence by assuming the existence of a single iterate $w_{k_0}$ satisfying $\sm(w_{k_0}) > \varepsilon$ and then proving that $\lim \sm(u_{k}) > 0$. Their approach relies on constructing a smooth approximation of the margin, which increases during training, ensuring that $\sm(u_k) > 0$ for all iterates with $k \geq k_0$. This is feasible in their setting, as they study either subgradient flow or GD with smooth $p_i$’s, allowing them to leverage the descent lemma.

In contrast, our analysis considers a nonsmooth and stochastic setting, in which, even if an iterate $w_{k_0}$ satisfying $\sm(w_{k_0}) > \varepsilon$ exists, there is no a priori assurance that subsequent iterates remain in the region where Assumption~\ref{hyp:marg_lowb} holds. From this perspective, Assumption~\ref{hyp:marg_lowb} can be viewed as a stability assumption, ensuring that iterates continue to classify the dataset correctly. Establishing stability for stochastic and nonsmooth algorithms is notoriously hard, and only partial results in restrictive settings exist \cite{borkar2000ode,ramaswamy2017generalization,josz2024global}.

%Finally, note that Assumption~\ref{hyp:marg_lowb} only needs to hold almost surely. Specifically, with probability 1, there exist $k_0$ and $\varepsilon$ such that for all $k \geq k_0$, $\sm(u_k) \geq \varepsilon > 0$. In the case of~\eqref{eq:sgd_new}, $k_0$ and $\delta$ are random variables and may take different values across different realizations. 

%\paragraph{On constant stepsizes.}
%We allow the step size to be a constant of arbitrary magnitude, subject to the stability Assumption~\ref{hyp:marg_lowb}. This may seem surprising in a nonsmooth and stochastic setting, where a vanishing step size is typically required to ensure convergence (see, e.g., \cite{majewski2018analysis, dav-dru-kak-lee-19, bolte2023subgradient, le2024nonsmooth}).
\begin{figure*}[ht]
    \centering
    \includegraphics[width=\textwidth, trim=79 280 93 123, clip]{figures/framework_img.pdf}
    \caption{The pipeline of the \ENDow{} framework 
    %where each component is specified in a given configuration. 
    which yields a downstream task score and a WER score of the transcript set input to the task. The pipeline is executed for several severeties of noising and types of cleaning techniques. %Acoustic noising is applied at $k$ intensities, providing $k+1$ audio versions (including the non-noised version), eventually producing $k+2$ transcript versions (including the source transcript). Applying transcript cleaning reveals the effect of \textit{types} of noise. 
    Resulting scores are plotted on a graph for the analyses, as in, e.g., \autoref{fig_cleaning_graphs}.}
    %The pipeline is executed on $k+1$ intensities of acoustic noising (including the non-noised version), producing $k+2$ scores for the downstream task (including execution on the source transcripts). This process eventually describes the effect of the \textit{intensity} of transcript noise on the downstream task. The process is repeated for $m$ cleaning techniques ($m+1$ when including no cleaning), to analyze the benefit of a cleaning approach and the effect of the \textit{types} of transcript noise.}
    \label{fig_framework}
\end{figure*}

\subsection{From Bayesian to Frequentist Inference}\label{sec:bayes}

A natural choice for the mixing distribution is the Bayesian posterior, which establishes a fundamental connection between frequentist confidence estimation and Bayesian inference. To explore this relationship, we first formally define the Bayesian inference model.
\begin{assumption}[Bayesian Inference]\label{a:bayes}
    In the Bayesian inference model, the learner defines a prior distribution $\mu_0 \in \sP(\Theta)$ over model parameters (independent of the data), and predicts using the posterior distribution $\mu_t(\theta) \propto \prod_{s=1}^{t-1} p_s(y_s|\theta) \mu_0(\theta)$.
\end{assumption}
The main result of this section establishes that if the mixing distributions are computed according to Bayes' rule, then prior likelihood mixing (\cref{result:prior_mixing}) and sequential likelihood mixing (\cref{result:posterior_mixing}) are equivalent. A further application of Bayes rule shows that any (realizable) Bayesian model can be turned into a $(1-\delta)$-confidence sequence by comparing the log posterior probability $\log \mu_t(\theta)$ to the log prior probability $\log \mu_0(\theta)$. This is known as \emph{prior-posterior ratio confidence set} \citep{waudby2020confidence}: 
\begin{align*}
    C_t =  \left\{ \theta \in \Theta: - \log \mu_t(\theta) \leq  \log \frac{1}{\delta} - \log \mu_0(\theta) \right\} \,.
\end{align*}
The equivalence result is foreshadowed in the works by \citet{waudby2020confidence} and \citet{neiswanger2021uncertainty}, who establish the posterior-ratio confidence set and the connection to the marginal likelihood. The explicit equivalence to the sequential mixing framework, however seems to be absent in prior works, and is formally given in \cref{result:mixing-equivalence} below. 
%As a consequence, the concentration bounds for sequential linear regression by \citet{neiswanger2021uncertainty,flynn2024improved,flynn2024tighter} and earlier work by \cite{abbasi2011improved} are essentially equivalent, as we illustrate below. \todoj{maybe move below Lemma 6, so that 'below' makes sense}

\begin{theorem}[Mixing Equivalence]\label{result:mixing-equivalence} If the mixing distributions are chosen according to Bayes' rule, prior likelihood mixing (\cref{result:prior_mixing}) and sequential mixing (\cref{result:posterior_mixing}) are equivalent.
\end{theorem}
\begin{proof}
   The result follows by applying Bayes' rule recursively to show the following equality, $\sum_{s=1}^t \log \int p_s(y_s|\nu) d\mu_{s-1}(\nu) = \log \int \prod_{s=1}^t p_s(y_s|\nu) d\mu_{0}(\nu)$.
    % \begin{align*}
    %     \sum_{s=1}^t \log \int p_s(y_s|\nu) d\mu_{s-1}(\nu) = \log \int \prod_{s=1}^t p_s(y_s|\nu) d\mu_{0}(\nu) \,. 
    % \end{align*}
\end{proof}

The surprising consequence is, that within the Bayesian inference model, sequential mixing provides no statistical advantage compared to averaging the likelihood over the prior. Less surprisingly though, Bayes' rule can be understood as an incremental update rule to compute the marginal likelihood. In this sense, the equivalence can be re-stated as recovering prior mixing (\cref{result:prior_mixing}) as a special case of sequential mixing (\cref{result:posterior_mixing}). However, note that for mixing distributions outside the Bayesian model, the equivalence does not hold in general, leaving the possibility to find non-Bayesian mixing distributions that achieve faster convergence. We come back to this idea in \cref{sec:oco}.

Next, we state a second implication of Bayes' rule, the prior-posterior ratio confidence set. 
\begin{lemma}[Prior-Posterior Ratio Confidence Set \citep{waudby2020confidence}] \label{lem:posterior_ratio_confidence_set}\\
    For any realizable Bayesian model, the following defines a $(1-\delta)$-confidence sequence:
    % The confidence sequence $C_t = \{\theta \in \Theta: L_t(\theta) \leq \log \frac{1}{\delta} - \log \int \prod_{s=1}^t p_s( y_s|\nu) d\mu_0(\nu)  \}$ can be equivalently written as follows:
\begin{align*}
    C_t &=  \left\{ \theta \in \Theta: - \log \mu_t(\theta) \leq  \log \frac{1}{\delta} - \log \mu_0(\theta) \right\} \,.
\end{align*}
Moreover, the confidence set is equivalent to the construction in \cref{result:prior_mixing,result:posterior_mixing}.
\end{lemma}
\begin{proof}
    Note that $\log \mu_t(\theta) = \log \mu_0(\theta)  + L_t(\theta) - \log \int \prod_{s=1}^t p_s(y_s|\nu) d\mu_{0}(\nu)$ holds for all $\theta \in \Theta$ by Bayes' rule. Substituting the equality into \cref{result:prior_mixing} gives the result.
\end{proof}
The remarkable conclusion is that any realizable Bayesian model can be turned into a frequentist confidence set by thresholding the log posterior probability relative to the log prior probability. As a caveat, it is tempting to think of $C_t$ as a Bayesian credible region, however, the posterior credible probability $\mu_{t-1}(C_t)$ is typically not $1-\delta$. Further, the confidence set, by construction, never rejects parameters in the null set of the prior distribution, unlike in classical Bayesian inference. In any case, a sensible choice is $\Theta = \supp(\mu_0)$, as long as the realizability condition (\cref{a:realizability}) is satisfied, that is, $\theta^* \in \Theta$ defines the true likelihood of the data. For an application of the prior-posterior confidence set to sequential sampling without replacement, we refer to \citet{waudby2020confidence}.

As a consequence of the prior-posterior ratio confidence set and the mixing equivalence, the confidence sets for sequential linear regression by \citet{neiswanger2021uncertainty,flynn2024improved,flynn2024tighter} and earlier work by \cite{abbasi2011improved} are essentially equivalent, as we demonstrate below. Moreover, a lower bound by \citet{lattimore2020bandit} shows that the construction is tight without further assumptions on the data generation distribution. 

\paragraph{Sequential Linear Regression} 
To illustrate the utility of the Bayesian perspective, we consider sequential linear regression with a Gaussian prior and likelihood. To preempt any concerns, we remark that the Gaussian assumption can be relaxed to sub-Gaussian distributions, as we explain in \cref{sec:subgaussian}. Formally, let $\theta^* \in \Theta = \bR^d$, with multivariate Gaussian prior $\cN(\theta_0, V_0^{-1})$ centered at $\theta_0 \in \bR^d$ and prior precision matrix $V_0 \in \bR^{d \times d}$, where commonly $V_0 = \lambda \eye_d \in \bR^{d\times d}$ for a regularizer $\lambda > 0$. The observation likelihood is Gaussian,  $y_t \sim \cN(x_t^\top\theta^*, \sigma^2)$ for a feature vector $x_t \in \bR^d$ and known observation variance $\sigma^2 > 0$. The Gaussian posterior is $\mu_t = \cN(\hat \theta_t^\RLS, V_t^{-1})$, where $\hat \theta_t^\RLS$ is the regularized least squares (RLS) estimate,
\begin{align*}
\hat \theta_t^\RLS = \argmin_{\theta \in \bR^d} \frac{1}{2 \sigma^2} \sum_{s=1}^t \big(\ip{x_s, \theta} - y_s\big)^2 + \frac{1}{2} \|\theta - \theta_0\|_{V_0}^2\,.
\end{align*}
Here, $V_t = \frac{1}{\sigma^2}\sum_{s=1}^t x_s x_s^\top + V_0$ is the posterior precision matrix, and we use the notation $\|\nu\|_A^2 = \nu^\top A \nu$ for $\nu \in \bR^d$ and $A \in \bR^{d\times d}$. The prior and posterior densities are explicitly given as follows:
\begin{align*}
    \mu_0(\theta) &= (2 \pi)^{-2/k} (\det V_0)^{1/2} \exp\big(- \tfrac{1}{2}\|\theta - \theta_0\|_{V_0}^2 \big) \\
    \mu_t(\theta) &= (2 \pi)^{-2/k} (\det V_t)^{1/2} \exp\big(- \tfrac{1}{2}\|\theta - \hat \theta_t^\RLS\|_{V_t}^2 \big)
\end{align*}
Applying \cref{lem:posterior_ratio_confidence_set} with the Gaussian posterior, we get the following $(1-\delta)$-confidence sequence:
\begin{align*}
    C_t^\RLS = \left\{ \theta \in \bR^d : \frac{1}{2}\|\theta - \hat \theta_t^\RLS\|_{V_t}^2 \leq \log \frac{1}{\delta} + \frac{1}{2}\log \det V_t - \frac{1}{2}\log \det V_0 + \frac{1}{2}\|\theta  - \theta_0\|_{V_0}^2 \right\}\,.
\end{align*}
An important feature of the bound is that it scales with the \emph{effective dimension} or \emph{total information gain} $\gamma_t = \frac{1}{2}\log \det V_t - \frac{1}{2}\log \det V_0$ of the data \citep[c.f.~][]{huang2021short}, which can be much smaller than the parameter dimension $d$ \citep{srinivas2009gaussian}. 
Note also that the confidence set does \emph{not} require a known bound on the norm $\|\theta^*\|_2 \leq S$, which is required in all prior work that we are aware of. If such a bound is available, a direct approach is to define the Gaussian prior and posterior directly over the restricted set $\cB_S = \{\theta \in \bR^d : \|\theta\|^2 \leq S\}$. However, in this case, the normalization constant is not easily computed in closed form. Instead, we intersect $C_t^\RLS$ with the norm ball $\cB_S$. Relaxing the confidence set further, and choosing $V_0 = \lambda \eye_d$ and $\theta_0 = 0$, we eventually arrive at
\begin{align*}
    % C_t &\subset \{ \theta \in \Theta : \frac{1}{2 \sigma^2} \|\theta - \hat \theta_t\|_{V_t}^2 \leq \log \frac{1}{\delta} + \log \det V_t - \log \det V_0 + S^2 \}\nonumber\\
    C_t^\RLS \cap \cB_S \subset \left\{ \theta \in \bR^d : \frac{1}{2} \|\theta - \hat \theta_t^\RLS\|_{V_t}^2 \leq \log \frac{1}{\delta} + \frac{1}{2}\log \det V_t - \frac{d}{2}\log \lambda+ \frac{\lambda}{2}S^2 \right\} \,.
\end{align*}
The last line essentially recovers the influential result by \citet{abbasi2011improved}, albeit avoiding a lower-order cross-term, improving the bound by up to a factor of two. 
The proof of \citet{abbasi2011improved} uses the method of mixtures, but mixing the noise martingale $S_t = \sum_{s=1}^t \epsilon_s x_t$ over a centered prior, instead of directly mixing the likelihood ratio. 
More recent work by \cite{flynn2024improved} achieves the tighter result using a similar sequential mixing idea, however, the likelihood framework and connection to Bayesian inference is not mentioned. A direct extension is to heteroscedastic noise, $y_t \sim \cN(x_t^\top\theta^*, \sigma_t^2)$ with known variance $\sigma_t^2$ \citep[c.f.,][]{kirschner2018information}. Another, more involved extension is to unknown mean and variance \cite[c.f.,][]{chowdhury2023bregman}. \looseness=-1

\paragraph{Gaussian Process Regression}
We remark that the confidence set for sequential linear regression can be equivalently stated for non-parametric Gaussian processes regression in infinite-dimensional kernel Hilbert spaces (RKHS) using the `kernel trick'. Our derivation improves (up to a factor of two) the results by \cite{abbasi2012thesis,chowdhury2017kernelized,whitehouse2023sublinear} and recovers more recent results by \cite{neiswanger2021uncertainty,flynn2024tighter}, who do not state the equivalence.

% In particular, we can restate the above confidence set $C_t$ on a separable RKHS space, and project the confidence set onto a specific evaluation $x$, via the reproducing kernel operation $f(x) = f^\top k(\cdot,x)$, to arrive at
% \[  C_t(x) = \{ f(x) | |f(x) - \hat{f}_t(x)| \leq  \} \]
% \todoj{make Gaussian processes explicit}


% Lastly, we remark that discussion extends to the more general class of sub-Gaussian likelihoods, which we discuss in \cref{sec:subgaussian}. 

\subsection{Variational Confidence Sets}\label{sec:elbo}
While the confidence set construction and its relation to Bayesian inference is universal and holds for any prior and family of likelihood functions, the price to pay is that computing the marginal likelihood, or equivalently, the posterior distribution, is intractable in general. Fortunately, approximate inference methods have been well studied in the field of Bayesian inference. 
Our starting point is a variational inequality for the marginal likelihood $\int \prod_{s=1}^t p_s(y_s|\theta)d\mu_0(\theta) = \int \exp(- L_t(\theta)) d\mu_0(\theta)$ and the Kullback-Leibner (KL) divergence, often attributed to \citet{donsker1983asymptotic}. 
\begin{lemma}[Variational Inequality]\label{lemma:variational-kl}
For any two distributions $\mu,\rho \in \sP(\Theta)$ and any measurable function $g : \Theta \rightarrow \bR$,
    \begin{align*}
    \log \int \exp(g(\theta)) d\mu \geq \int g(\theta) d\rho(\theta) - \KL(\rho\|\mu) \,.
    \end{align*}
    Moreover, the inequality becomes an equality for $d\rho(\theta) \propto \exp(g(\theta)) d\mu(\theta)$.
\end{lemma}
The inequality plays a central role in variational inference, and is typically stated as the \emph{evidence lower bound} (ELBO), by specializing \cref{lemma:variational-kl} using $g(\theta) = - L_t (\theta)$ and $\mu = \mu_0$,
\begin{align*}
 \log \int \exp(- L_t(\theta)) d\mu_0 \geq \ELBO_t(\rho) := -\int L_t(\theta) d\rho(\theta) - \KL(\rho\|\mu_0) \,.
\end{align*}
For the given choices, the inequality becomes tight when $\rho = \mu_t$ is the Bayesian posterior.
Variational inference aims at numerically maximizing the evidence lower bound over a parametric family of posterior distributions \citep{jordan1999introduction}, see also \citep{blei2017variational}. In the context of confidence estimation, the key insight is that the variational inequality allows to relax the marginal likelihood that defines the confidence sequence in \cref{result:prior_mixing}. This result has been recently stated by \citet{lee2024improved} in the specialized context of logistic and multinomial bandits, and similar bounds are well-known in the PAC-Bayes literature \citep[e.g.,][]{zhang2006varepsilon,chen2022unified,alquier2024user}. Here, we emphasize the connection to variational inference and the evidence lower bound as a way to define a confidence coefficient with valid anytime coverage.
\begin{theorem}[Evidence Lower Bound Confidence Set]\label{result:elbo_confidence_set}
    For any $\cF_t$-adapted sequence of distributions $\mu_t \in \sP(\Theta)$ and a data-independent prior $\mu_0 \in \sP(\Theta)$, define\looseness=-1
    % the $(1-\delta)$-confidence sequence $C_t = \left\{\theta \in \Theta : L_t(\theta) \leq  \log \frac{1}{\delta} - \log \int \prod_{s=1}^t p_s(y_s|\nu) d\mu_0(\nu) \right\}$, it holds that
    \begin{align*}
		 C_t = \left\{\theta \in \Theta : L_t(\theta) \leq  \log \frac{1}{\delta} - \ELBO_t(\mu_t)  \right\} \,.
	\end{align*}
    Then $C_t$ defines a $(1-\delta)$-confidence sequence. Moreover, if $\rho_t$ is chosen as the Bayesian posterior, the result is equivalent to the marginal likelihood confidence set in \cref{result:prior_mixing}.
    % Moreover, if $\rho$ is chosen as the Bayesian posterior, the inclusion becomes an equality.
\end{theorem}
The practical implication of this result is that it provides a tool to trade off computational tractability and statistical efficiency. In particular, standard variational inference methods can be converted into a confidence set with provable coverage, simply by thresholding the negative log-likelihood by the attained evidence lower bound. Another possibility is to make ad-hoc choices for the posterior to obtain closed-form expressions for confidence sets, e.g.~for logistic regression \citep{lee2024unified}. 



\subsection{Oracle Complexity Bounds via Online Estimation}\label{sec:oco}

The size of the sequential mixing confidence set in \cref{result:posterior_mixing} depends on the ability of the learner to produce a sequence of mixing distributions that minimize the cumulative log loss,
\begin{align*}
   \sum_{s-1}^t l_s(\mu_{s-1}) =  - \sum_{s=1}^t \log \int p_s(y_s|\nu) d\mu_{s-1}(\nu) \,.
\end{align*}
The field of \emph{online density estimation} studies algorithms for minimizing the cumulative log loss, and offers a rich literature on complexity bounds for the regret \citep[e.g.,][]{vovk1990aggregating,vovk1997competitive,zhang2006varepsilon,rakhlin2014online}. Here, regret is defined as the difference between the cumulative log loss, and the best prediction in hindsight, which, in the simplest case, coincides with the maximum likelihood estimate. Specifically, we assume that the mixing distributions $\mu_1, \mu_2, \dots$ are chosen by an \emph{online estimation} algorithm, in a way that ensures a bound $B_t \geq 0$ on the log-regret,
\begin{align*}
    \Lambda_t = \sum_{s=1}^t l_s(\mu_{s-1}) - \min_{\theta} L_t(\theta) \leq B_t \,.
\end{align*}
For a complete introduction, we refer the reader to the standard literature \citep[e.g.,][]{cesa2006prediction,orabona2019modern,shalev2012online}. 
% More generally, the field of online convex optimization (OCO) studies algorithms for minimizing the regret, with minimal assumptions on  the sequence of loss functions. The OCO setting is typically formulated as a game between an \emph{adversary}, choosing the loss $l_t$, and an \emph{online learning} algorithm choosing a prediction (in our case, $\mu_{t-1})$. At the end of each round, the learner occurs the loss $l_t(\mu_{t-1})$, and the loss function is revealed to the learner, allowing to update the prediction. Our main result is the following.

The next result demonstrates how the sequential likelihood mixing framework relates to maximum likelihood estimation, using regret inequalities from online estimation. 
\begin{theorem}[Regret-To-Confidence]\label{result:regret}
    Assume there exists an online estimation algorithm such that the log-regret is bounded almost-surely, $\Lambda_t \leq B_t$, for a predictable sequence $B_t \geq 0$. Define 
    \begin{align*}
    C_t = \left\{ \theta \in \Theta: L_t(\theta) \leq  \log \frac{1}{\delta} + L_t(\hat \theta_t^\MLE) + B_t \right\} \,.
    \end{align*}
    Then $C_t$ defines a $(1-\delta)$-confidence sequence.
\end{theorem}
\begin{proof}
The result follows directly from \cref{result:posterior_mixing}, by introducing $L_t(\hat \theta_t^\MLE)$ and using the definition of the regret.
\end{proof}
The importance of the result is that the \emph{existence} of an online estimation algorithm with a \emph{known} regret bound, allows to define a valid $(1-\delta)$-confidence sequence, relative to the log-likelihood of the MLE and the complexity bound for online estimation. In particular, the construction does \emph{not} require access to the predictions of the online learning algorithm.  Moreover, the confidence coefficient can be computed using standard supervised learning algorithms, eliminating the need for computing the marginal likelihood. Lastly, the complexity term offers a more interpretable bound on the confidence coefficient. We will come back to several concrete examples below.


The use of regret inequalities to derive concentration inequalities goes back to at least \cite{dekel2010robust,abbasi2012online}. The ``Online-to-Confidence-Set'' conversion by \citet{abbasi2012online} is, however, different in a subtle way, and requires access to the predictions of the online learning algorithm. We recover (and improve upon) their result using a generalization of \cref{result:regret} in \cref{sec:sparse}. Later works extend this idea, for example, to (multinomial) logistic bandits \citep{lee2024improved}. A similarly flavoured result is by \cite{abeles2024generalization} in the context of PAC-Bayes generalization bounds. Worth mentioning is also the work by \cite{rakhlin2017equivalence}, who prove an equivalence between regret bounds and tail inequalities.

\paragraph{Logistic Regression} We illustrate \cref{result:regret} in the sequential logistic regression setting. Let $\Theta = \{\theta \in \bR^d : \|\theta\|_2 \leq S\}$ for a norm bound $S > 0$. The likelihood is a Bernoulli distribution, $p_t(y|\theta) = \Ber(\phi(\ip{\theta, x_t}))$ with the logistic link function $\phi(z) = (1 + e^{-z})^{-1}$  and the covariates $x_t \in \bR^d$. Following along the lines of \citet{lee2024improved}, we invoke a regret bound for logistic regression by \cite{foster2018logistic}, who prove an online learning algorithm that achieves $\Lambda_t \leq 10d \log \left( e + \frac{St}{2d} \right)$. \cref{result:regret} immediately implies the following $(1-\delta)$-confidence sequence:
\begin{align*}
    C_t = \left\{ \theta \in \Theta: L_t(\theta) \leq  \log \frac{1}{\delta} + L_t(\hat \theta_t^\MLE) + 10d \log \left( e + \frac{St}{2d}  \right)\right\} \,.
\end{align*}
We remark that the confidence coefficient above improves upon the result of \citet[Theorem 1]{lee2024improved}, as a consequence of directly applying Ville's inequality to the likelihood ratio.


\paragraph{Compressed Sensing} We come back to the sequential linear regression setting described below \cref{lem:posterior_ratio_confidence_set}, with the additional assumption that the true parameter $\theta^*$ is $k$-sparse, i.e.~$\|\theta^*\|_0 \leq k$. One way to account for the sparsity assumption is to set $\Theta_k = \{\theta \in \bR^d : \|\theta\|_0 < k, \|\theta\|_2 \leq S\}$. For sparse linear regression, \citep{gerchinovitz2011sparsity} proposes online learning algorithm  that achieves $\Lambda_t \leq C_0 k \log(t)$ for a constant $C_0 > 0$. \cref{result:regret} implies the following a confidence sequence,
\begin{align*}
    C_t = \left\{ \theta \in \Theta: L_t(\theta) \leq  \log \frac{1}{\delta} + L_t(\hat \theta_t^\MLE) + C_0 k \log(t)\right\} \,.
\end{align*}
Note that here MLE is defined over the sparse set $\Theta_k$, which poses computational challenges. On the other hand, the confidence set by \citet{abbasi2012online} requires access to the predictions of the online learning algorithm. The construction therefore suffers a similar fate, as finding a computationally efficient algorithm for sparse linear prediction is still an open problem.

% Specialized regret bounds, such as for sparse linear bandits or logistic regression are useful as the reveal concrete dependencies on problem parameters. Perhaps surprisingly, it is possible to obtain regret bounds without any assumptions on the model class.



\paragraph{Finite Model Identification} Assume that the parameter set $\Theta$ is finite, i.e.~$|\Theta|< \infty$. The key insight is to note that the log-loss is $\eta$-exp-concave for $\eta \leq 1$, i.e.~$\exp(-\eta l_t(\mu))$ is concave in $\mu \in \sP(\Theta)$ for all $\eta \leq 1$ (in fact, it is linear for $\eta=1$). This highlights the importance of using mixing distributions, as, in general, $-\log p_t(y_t|\theta)$ is \emph{not} exp-concave as a function of $\theta$.

The standard approach for online learning with exp-concave functions is the \emph{exponential weights algorithm} \citep[EWA, ][]{littlestone1994weighted,freund1997decision}, see also the book by \citet{cesa2006prediction}. For $\eta=1$, EWA is equivalent to Bayesian inference and the prediction $\mu_t$ is equal to the Bayesian posterior (see \cref{alg:cew}). With a uniform prior, the regret of EWA satisfies $\Lambda_t \leq \log(|\Theta|)$, uniformly over all data sequences. The proof is provided for completeness in \cref{sec:cew}. Using \cref{result:regret}, we obtain the following $(1-\delta)$-confidence sequence\looseness=-1
\begin{align*}
    C_t = \{\theta \in \Theta : L_t(\theta) \leq \log \frac{1}{\delta} + L_t(\hat \theta_t^\MLE) + \log |\Theta| \} \,.
\end{align*}
The result should be compared to the standard union bound argument (\cref{sec:mle}). While the bound is the same, note that we obtained \cref{result:regret} as a \emph{relaxation} of \cref{result:posterior_mixing}. Hence, the sequential mixing confidence set (with the Bayesian posterior as mixing distributions) is never worse, and possibly tighter for benign data and structured model classes \citep[e.g.,][]{auer2002adaptive,cesa2007improved,de2014follow}. The confidence set can also be written for the maximum a-posteriori estimate (MAP), in which case the role of the prior distribution becomes more apparent:
\begin{align*}
    C_t = \left\{\theta \in \Theta : L_t(\theta) \leq \log \frac{1}{\delta} + \min_{\nu \in \Theta} \big( L_t(\nu) - \log \mu_0(\nu) \big)\right\} \,.
\end{align*}
Lastly, we remark that the exponential weights algorithm can be generalized to the continuous setting. The confidence sequence derived from the regret bound of continuous exponential weights is equivalent to the ELBO confidence set in \cref{result:elbo_confidence_set}. This is another consequence of the mixing equivalence. We refer to \cref{sec:cew} for further details.

\section{Conclusion}
In this work, we propose a simple yet effective approach, called SMILE, for graph few-shot learning with fewer tasks. Specifically, we introduce a novel dual-level mixup strategy, including within-task and across-task mixup, for enriching the diversity of nodes within each task and the diversity of tasks. Also, we incorporate the degree-based prior information to learn expressive node embeddings. Theoretically, we prove that SMILE effectively enhances the model's generalization performance. Empirically, we conduct extensive experiments on multiple benchmarks and the results suggest that SMILE significantly outperforms other baselines, including both in-domain and cross-domain few-shot settings.


% Acknowledgments---Will not appear in anonymized version
\acks{We thank Guillaume Obozinski for helpful discussions on the topic.}

% \newpage
\bibliography{bibliography}

\newpage
\appendix
\crefalias{section}{appendix} % uncomment if you are using cleveref

The advancement of artificial intelligence in the legal domain has led to the development of various tools that assist in legal research, document retrieval, and automated legal reasoning. Several studies have explored the use of Natural Language Processing (NLP)\cite{khurana2023natural}, machine learning models, and vector-based search mechanisms to enhance the efficiency of legal chatbots. The primary focus of this literature review is on retrieval-augmented generation (RAG) models, FAISS-based document retrieval, deep learning for legal applications, and the use of large language models (LLMs) in legal AI.  

Recent research on Retrieval-Augmented Generation (RAG)\cite{gao2023retrieval} for legal AI has demonstrated its potential in enhancing legal text retrieval and summarization. S. S. Manathunga, Y. and A. Illangasekara\cite{manathunga2023retrieval} proposed a RAG-based model that improves legal text summarization by dynamically fetching relevant documents before generating responses. Similarly, Lee and Ryu \cite{ryu-etal-2023-retrieval} explored the application of RAG in case law retrieval, demonstrating its superiority over traditional keyword-based search engines. The introduction of RAG has significantly improved response accuracy by grounding AI-generated text in authoritative legal documents, reducing hallucinations in AI-driven legal assistance.  

% \begin{figure}[h]
%     \centering
%     \includegraphics[width=8cm]{FAISS.png}
%     \caption{Faiss: Efficient Similarity Search and Clustering of Dense Vectors}
%     \label{Overall Result of comparing FAISS and Chroma with different number of top documents}
% \end{figure}

The efficiency of FAISS (Facebook AI Similarity Search) in legal document retrieval has also been widely studied. Zhao et al. \cite{devlin-etal-2019-bert} implemented FAISS to enhance large-scale legal question answering systems, achieving significant improvements in retrieval speed and relevance. N. Goyal and D. Chen \cite{inbook} demonstrated that FAISS-based vector search mechanisms outperform conventional database searches in legal information retrieval, reducing query response time while maintaining high accuracy. The integration of FAISS with transformer-based models, as seen in the work of Hsieh and Wu, further enhances semantic retrieval, ensuring that chatbot responses align with actual legal texts.  

Transformer-based models such as BERT and GPT-based architecture have also contributed to the evolution of AI-driven legal research. Devlin et al. introduced BERT (Bidirectional Encoder Representations from Transformers), which significantly improved the understanding of legal language. RoBERTa, an optimized version of BERT, was later developed by Liu et al. \cite{liu2019roberta} to enhance contextual understanding and document similarity matching in legal queries. These models have been integrated into legal chatbots for contract analysis and legal decision-making, as demonstrated in the studies of Li et al. and Jin and Liu, where fine-tuned transformers improved legal text comprehension and summarization.  
The role of deep learning in legal AI has also been investigated extensively. Radford et al. introduced GPT-3, which paved the way for legal AI assistants capable of generating human-like responses. However, researchers such as Firth and Lee emphasized the limitations of LLMs in legal reasoning, arguing that these models require external verification mechanisms to prevent misinformation. The use of contrastive learning and fine-tuning for legal text retrieval has been explored by Arabi and Akbari \cite{article}, who demonstrated that embedding-based retrieval significantly improves chatbot response accuracy.  

Another significant area of research involves evaluating AI-generated legal responses using automated metrics. Zhang and Wu introduced BLEU\cite{10.3115/1073083.1073135} and ROUGE\cite{lin-2004-rouge} scores as a means to evaluate AI-generated legal text summaries, ensuring their quality and relevance. Similarly, Zhao et al. \cite{yuan2024rag} examined the effectiveness of RAG-based models in handling complex legal queries, highlighting the importance of legal consistency scores (LCS) in evaluating AI-driven responses.  

The practical applications of legal AI chatbots have been studied extensively in the context of access to justice and AI ethics. Wang and Cheng et al. \cite{xue2024bias} highlighted the potential of AI-driven legal assistants in bridging the justice gap, particularly in countries where legal resources are not easily accessible. Chan conducted a systematic review of retrieval-based legal chatbots, noting that while these systems improve accessibility, they also raise ethical concerns regarding legal misinformation and bias. Research by Min \cite{Min2023ARTIFICIALIA} explored methods for bias detection and mitigation in legal AI, ensuring fairness in AI-generated legal advice.  

Comparative studies between rule-based legal bots, keyword-driven legal search engines, and AI-powered legal chatbots further illustrate the superiority of retrieval-augmented approaches. In a study conducted by Zeng \cite{zeng2024scalable}, FAISS-based retrieval mechanisms significantly outperformed traditional Boolean keyword searches, reducing irrelevant document retrieval by 40\%. Singh \cite{10760929} further demonstrated that AI-powered legal research tools using NLP provide faster and more contextually accurate responses compared to standard legal databases.  

Despite these advancements, challenges remain in AI-driven legal research. Existing chatbots still struggle with multi-jurisdictional legal queries, as noted by Weichbroth \cite{Weichbroth2025AIAT}, who emphasized the need for jurisdiction-aware legal AI models. Additionally, legal AI models often lack the ability to process long-context legal arguments effectively, a limitation discussed by Gupta, who proposed memory-based retrieval techniques to improve long-form legal text processing.  

Research continues to refine AI-driven legal assistance, particularly in retrieval-augmented generation, FAISS-based search, transformer models, and deep learning techniques for legal research. However, further improvements are needed in bias mitigation, jurisdiction-specific adaptations, and long-context legal understanding. Future developments in multilingual legal AI, enhanced retrieval mechanisms, and AI-powered contract analysis will be crucial in making legal AI tools more accessible, reliable, and widely applicable in legal practice.
\section{Laplace's method}\label{sec:laplace}
Recall from section \cref{sec:mle} that $\max_{\nu} R_t(\nu;\theta^*)$ is not a martingale, which prevents a direct application of Ville's inequality. Laplace's method uses the observation that, under suitable regularity assumptions on a sequence functions $f_n : \bR^d \rightarrow \bR$ with unique maximizer $x_n^*$ and positive definite Hessian $H_n(x) = \frac{\partial^2}{\partial x^2} f_n(x) \in \bR^{d \times d}$, the following asymptotic expansion provides an approximation of the maximizer $f_n(x_n^*)$, 
\begin{align*}
    \int_\Theta h(x) e^{- f_n(x)} dx \sim  \frac{(2\pi)^{d/2} h(x^*_n)}{\sqrt{\det H_n(x_n^*)}} e^{- f_n(x^*_n)}
    % \max_{\theta} f_n(\theta) = \log \lim_{n \rightarrow \infty} \int \exp( n f_n(\theta)) d\mu_0(\theta)
\end{align*}
To apply this idea to the marginal likelihood that appears in the confidence coefficient in \cref{result:prior_mixing}, assume that $\Theta \subset \bR^d$ and $\mu_0$ admits a density $h(\theta)$ w.r.t. to the Lebesgue measure. Let $\hat \theta_n^\MLE$ be the maximum likelihood estimate, and $I_t(\theta) = \frac{\partial^2}{\partial \theta^2} L_t(\theta)$ the empirical Fischer information matrix. Laplace's methods gives the following approximation
\begin{align*}
\beta_t(\delta) &= \log \frac{1}{\delta} - \log \int \exp(-L_t(\nu))  h(\nu) d\theta\\
&\approx\log \frac{1}{\delta} + L_t(\hat \theta_t^{\MLE})  + \frac{1}{2}\log \det I_t(\hat \theta_n^\MLE) - \frac{d}{2} \log(2\pi) - \log \mu_0(\hat \theta_n^\MLE)  
\end{align*}
Note that the confidence coefficient is smaller (resulting in a smaller confidence set) the more mass $h(\hat \theta^*_n)$ places on the maximizer.
An similar (and perhaps more natural) argument can be made for the maximum a-posteriori estimate.
Unfortunately, making these approximation rigorous is challenging without placing further assumptions on the data generating distribution and function class \citep[e.g.,][]{shun1995laplace}. We will not pursue this any further here, however point out that regret inequalities (\cref{sec:oco}) provide an alternative way to control the error w.r.t.~the MLE, including in finite time.
 
\section{Continuous Exponential Weights}\label{sec:cew}

Continuous exponential weights (\cref{alg:cew}) is a direct generalization of the classical exponential weights algorithm (also known as Hedge) by \cite{freund1997decision,littlestone1994weighted}. For a standard reference, see the book by \cite{cesa2006prediction}. Recall the definition of the log-loss, defined for distributions $\mu \in \sP(\Theta)$,
\begin{align*}
 l_t(\mu) = - \log \left(\int_{\Theta} p_t(y_t|\theta)d\mu(\theta)\right)
\end{align*}
Note that $l_t(\mu)$ is $\eta$-exp-concave for $\eta \leq 1$, since $\exp(-\eta l_t(\mu))$ is concave in $\mu : \sP(\Theta) \rightarrow \bR$. The (continuous) exponential weights algorithm satisfies the following regret bound, that holds for any sequence of $\eta$-exp-concave loss functions.
\begin{theorem}[Regret of Continuous Exponential Weights]\label{thm:cew}
    For any distribution $\rho \in \sP(\Theta)$, and any sequence of $\eta$-exp-concave loss functions $l_1, \dots, l_t$, the regret of the exponential weights algorithms with prior $\mu_0 \in \sP(\Theta)$ and learning rate $\eta$ satisfies
    \begin{align*}
        \sum_{s=1}^t l_s(\mu_{s-1}) - \int L_t(\theta) d\rho \leq \frac{1}{\eta}\KL(\rho, \mu_0)
    \end{align*}
    % So in particular, for any $\theta \in \Theta$ and assuming that $\delta_\theta \ll \mu_0$,
    % \begin{align}
    %     \sum_{t=1}^n \tilde l_t(\mu_t) - \sum_{t=1}^n l_t(\theta) \leq \log\left(\frac{1}{\mu_0(\theta)}\right)
    % \end{align}
    Moreover, for finite $\Theta$ and any $\nu \in \Theta$,
    \begin{align*}
        \sum_{s=1}^t  l_s(\mu_{s-1}) -  L_t(\nu) \leq \frac{1}{\eta}\log \frac{1}{\rho(\nu)}
    \end{align*}
\end{theorem}


% \begin{remark}
%     Note that $\tilde l_t(\mu_t)$ is the `posterior' log likelihood of $p(y|x,\theta)\mu_t(\theta)$.
% \end{remark}
\begin{proof} Denote by $\delta_{\theta} \in \sP(\Theta)$ the Dirac measure on $\theta \in \Theta$. Define the unnormalized measure 
    \begin{align*}
    %    \tilde \mu_t(d\theta) = \exp(-\eta L_{t}(\theta)) \mu_0(d\theta) = \prod_{s=1}^{t} \exp(-\eta l_s(\delta_\theta)) \mu_0(d\theta)
       \tilde \mu_t(d\theta) = \prod_{s=1}^{t} \exp(-\eta l_s(\delta_\theta)) \mu_0(d\theta)
    \end{align*}
    Note that $\mu_t(d\theta) = \frac{\tilde \mu_t(d\theta)}{\tilde \mu_t(\Theta)}$.
To prove the regret bound note that by $\eta$-exp-concavity of $l_t$,
\begin{align*}
    \frac{\tilde \mu_t(\Theta)}{\tilde \mu_{t-1}(\Theta)} &= \int \exp(-\eta l_t(\delta_\theta)) d\mu_{t-1}(\theta) \leq \exp (- \eta  l_t(\mu_{t-1}) )
\end{align*}
Further, for any $\rho \in \sP(\Theta)$, the variational inequality (\cref{lemma:variational-kl}) implies
\begin{align*}
\tilde \mu_t(\Theta) \geq \exp \Big( - \eta \ip{L_n, \rho} - \KL(\rho, \mu_0)\Big)
\end{align*}
Combining the last two inequalities and telescoping, we get
\begin{align*}
    \sum_{s=1}^t  l_s(\mu_{s-1}) -  \int L_s(\theta) d\rho(\theta) \leq \frac{1}{\eta}\KL(\rho, \mu_0)
\end{align*}
This completes the first part of the proof. The second part follows by setting $\rho=\delta_{\nu}$.
\end{proof}


\LinesNumbered
\RestyleAlgo{ruled}
\begin{algorithm2e}[t]
	\DontPrintSemicolon
	\SetAlgoVlined
	\SetAlgoNoLine
	\SetAlgoNoEnd
	\caption{Continuous Exponential Weights} \label{alg:cew}
    \SetKwInput{KwInput}{Input}
    
    \KwInput{Prior $\mu_0 \in \sP(\Theta)$, learning rate $\eta > 0$}

    \For{$t \gets 1, 2. \dots$} {
        \textbf{Predict:} $\mu_{t-1}$\;
        \textbf{Observe:} $x_t, y_t$\;
        \textbf{Receive (log) loss:}$$l_t(\mu_{t-1}) = - \log \left(\int p_t(y_t|\theta)d\mu_{t-1}(\theta)\right)$$\;
        \textbf{Update} $\mu_t(d\theta) \propto \exp\big(-\eta \sum_{s=1}^t l_s(d\theta)\big) \mu_0(d\theta)$ \textbf{:}
        $$\mu_t(d\theta)  = \frac{\exp\big(\eta \log p_t(y_t|\theta)\big) \mu_{t-1}(d\theta)}{\int \exp\big(\eta \log p_t(y_t|\nu)\big) d\mu_{t-1}(\nu)}$$
    }
\end{algorithm2e}

\paragraph{Confidence Sequences using Continuous Exponential Weights} 
Substituting the regret bound from \cref{thm:cew} into \cref{result:posterior_mixing} yields the following $(1-\delta)$-confidence sequence, which holds for any $\cF_t$-adapted sequence $\mu_1,\mu_2, \dots \in \sP(\Theta)$:
\begin{align*}
    C_t  = \left\{ \theta \in \Theta: L_t(\theta) \leq  \log \frac{1}{\delta} + \int L_t(\theta) d\mu_t(\theta) + \KL(\mu_t \| \mu_0)\right\} \,.
\end{align*}
Note that we have recovered the ELBO-confidence set from \cref{result:elbo_confidence_set}. In other words, the regret bound of continuous exponential weights is the sequential analog of the variational inequality \cref{lemma:variational-kl}, and because of the mixing equivalence (\cref{result:mixing-equivalence}), the resulting confidence sets are the same. Unfortunately, for continuous $\Theta$, the bound becomes vacuous when $\rho$ is set to a Dirac measure. This prevents us from using $\mu_t = \delta_{\hat \theta_t^\MLE}$ to derive a confidence set that directly compares to the maximum likelihood estimate. A more careful analysis using additional smoothness assumptions is a possible way forward \citep[c.f.,][]{lee2024unified}.




\section{Misspecified Model Classes}\label{sec:misspecified}

So far, we have assumed that the model class is realizable, that is, there exists a parameter $\theta^* \in \Theta$ such that $p_t(y|\theta^*) = \frac{d\bP_t}{d\xi}$ represents the density of the true data generating distribution. The realizability assumption is used to show that the sequential likelihood ratio is a (super)martingale. The can be significantly relaxed, and there are several ways in which we can construct supermartingales for misspecified model classes, as we elaborate now. 





% \subsubsection{Sub-Likelihood Robustness}
% In particular, we will assume that the following condition holds for the true data-generating process, we will refer to it as sub-likelihood condition. 

% \begin{definition}[Sub-likelihood]\label{def:2} Let $L$ be the log-likelihood of the true distribution of $\epsilon_t$ for all $t$, be $p$. We call $p$ to be sub-Likelihood $L$ if, 
%   \begin{equation}
%       \EE_{p}[e^{\nabla  L(\epsilon) \eta + B_{ L}(\epsilon-\eta,\epsilon)}] \leq 1,
%   \end{equation}  
%   where $B_{\log L}$ denotes Bregman divergence of $\log L$. 
% \end{definition}
% This sub-likelihood condition can be linked to a broader treatment of sub-processes by \cite{Howard2020}, where they examine a similar problem to define sub-families and construct confidence sets for these sub-families. Conceptually, our approaches are opposite. While they define a family of probability distributions that includes a given likelihood, we start with a fixed likelihood and identify robustness conditions for other distributions.

% In order to have a valid confidence interval with likelihood ratios, we need that 
% \begin{align}\label{eq:miss_lk}
%     E_t(\theta) = \prod_{s=1}^t \frac{\exp(L(f_{\hat \theta_{s-1}}(x_s) - y_s))}{\exp(L(f_\theta(x_s) - )y_s)}
% \end{align}
% is a super-martingale, as we show now. 
% \begin{theorem} Assume that the true data-generating satisfies Definition \ref{def:2}, then $E_t(\theta^*)$ in Eq. \eqref{eq:miss_lk} is a super-martingale adapted to the usual filtration $\cF_{t-1}$. 
% \end{theorem}
% \begin{proof}
% Let $\eta = f_{\hat \theta_t}(x_{t-1})$, and note that Bregman divergence, 
% $L(\epsilon-\eta) - L(\epsilon) = -\nabla L (\epsilon)^\top \eta - B_L(\epsilon-\eta,\epsilon)$. Then, 

% \begin{eqnarray}
% \EE_{\epsilon_t}[E_t(\theta^*)| \mathcal{F}_{t-1}]  &  = & E_{t-1}(\theta^*)\EE_{\epsilon}[\exp(L(\epsilon-\eta) - L(\epsilon))] \\ & = & E_{t-1}(\theta^*) \EE_{\epsilon}[\exp( \nabla L (\epsilon)^\top \eta - B_L(\epsilon-\eta,\epsilon))] \leq E_{t-1}(\theta^*)
% \end{eqnarray}

% \end{proof} 

% This result allows us to construct confidence intervals for $\theta^*$ that parametrizes the mean of the $y_t$ irrespective of the misspecification, albeit at loss of power since the $E_t(\theta^*)$ is only a super-martingale instead of a martingale as in the well-specified case. 

% \paragraph{Example 1: Sub-Gaussian Likelihoods}\label{sec:subgaussian}
% Assume that the true data generating distribution $p(y_t|x_t)$ is $\sigma$-sub-Gaussian, that is $\epsilon_t = y_t - \EE[y_t]$ satisfies
% \begin{align*}
%     \EE[\exp\left(\epsilon \eta - \frac{\sigma^2 \eta^2}{2}\right)] \leq 1 \quad \text{for all} \quad \eta \in \bR
% \end{align*}
% This assumption exactly coincides with Definition \ref{def:2} as $B_{L} = \frac{\eta^2}{2\sigma^2}$, and $\nabla L = \frac{\epsilon}{\sigma^2}$, which is equivalent to the above.

% \paragraph{Example 2: Sub-Poisson Likelihood}\label{sec:sub} Suppose that the mean is parametrized as $f_\theta(x) = \exp(\theta^\top x) $, and we are observing integer values as $y_t \in \{0,1,\dots \infty\}$. Now the true generating distribution $p_i$ for data point $y_i$ is sub-Poisson if, that is when using Poisson likelihood, we maintain coverage even if $p$ satisfied the following,
% \[ \EE [\exp\left( (\eta + \epsilon)\log (\lambda_i) + \log\left(\frac{(\epsilon + \eta)!}{\epsilon!}\right)\right) ] \leq 1 \quad \text{for all}  \quad \eta \in \bR. \]

\subsection{Sub-Gaussian Distributions}\label{sec:subgaussian}
In this section, we let $\cY = \bR$ and assume that the true data generating distribution $\bP_t$ is $\sigma$-sub-Gaussian for all $t \geq 1$, that is $\epsilon_t = y_t - \EE[y_t|\cF_{t}]$ satisfies,
\begin{align*}
\EE[e^{\epsilon_t \eta}|\cF_t] \leq e^{\frac{\sigma^2 \eta^2}{2}} \quad \text{for all} \quad \eta \in \bR \,.
\end{align*}
Further assume that we have a parametrized family of mean functions $f_\theta : \cX \rightarrow \bR$, and there exists a $\theta^* \in \Theta$ such that $\EE[y_t|\cF_t] = f_{\theta^*}(x_t)$.
For any $\cF$-adapted sequence of mixing distributions $\mu_0, \mu_1, \dots \in \sP(\Theta)$, define
    \begin{align*}
        E_t(\theta) = \prod_{s=1}^t \frac{ \int \exp(- \frac{1}{2 \sigma^2} (f_{\nu}(x_s) - y_s)^2) d\mu_{s-1}(\nu)}{\exp(- \frac{1}{2 \sigma^2} (f_\theta(x_s) - y_s)^2)} \,.
    \end{align*}
As we will see shortly, $E_t(\theta^*)$ is a $\cF_t$ adapted supermartingale, and $\EE[E_1(\theta^*)] \leq 1$.
Two remarks before we prove the claim. First, if the true data distribution is Gaussian with mean $f_\theta(x)$ and variance $\sigma^2$, then $E_t(\theta)$ is just the sequential marginal likelihood ratio, and the $\sigma$-sub-Gaussian condition holds with equality. Second, the result implies that we can proceed in constructing our confidence set as if the the Gaussian likelihood model was correct, and the coverage results remain true.\looseness=-1
\begin{theorem}
    For any $\cF$-adapted sequence of distributions $\mu_0, \mu_1, \mu_2, \dots$ in $\sP(\Theta)$, define
    \begin{align*}
        C_t = \left\{ \theta \in \Theta:   \sum_{s=1}^t \tfrac{1}{2 \sigma^2}(f_\theta(x_s) - y_s)^2 \leq \log \frac{1}{\delta} + \sum_{s=1}^t \log \int  \exp( -\tfrac{1}{2 \sigma^2} (f_{\nu}(x_s) - y_s)^2) d\mu_{s-1}(\nu)\right\}
    \end{align*}
    Then $C_t$ defines a $(1-\delta)$-confidence sequence.
\end{theorem}
\begin{proof}
We start by showing that $E_t(\theta^*)$ is a super-martingale. Fubini's theorem implies that
\begin{align*}
    \EE[E_t(\theta^*)|\cF_{t-1}] &= E_{t-1}(\theta^*) \int \EE[\exp\left( - \tfrac{1}{2\sigma^2} \big(f_\nu(x_t) - y_t\big)^2 + \tfrac{1}{2\sigma^2} \big(f_{\theta^*}(x_t) - y_t\big)^2 \right) |\cF_t ]  d \mu_{t-1}(\nu)%\\
    % &= E_{t-1}(\theta^*) \EE[\exp\left( - \tfrac{1}{2\sigma^2} f_{\hat \theta_t}(x_t)^2 - \tfrac{1}{\sigma^2} y_t f_{\hat \theta_t}(x_t) + \tfrac{1}{\sigma^2} y_t f_{\theta^*}(x_t) - \tfrac{1}{2\sigma^2} f_{\theta^*}(x_t)^2  \right) ]\\
    % &= E_{t-1}(\theta^*) \exp\left( - \tfrac{\sigma^2}{2} \big( f_{\hat \theta_s}(x_s)^2 - f_{\theta^*}(x_t)^2 \big) \right) \EE[\exp\left( - \tfrac{1}{2\sigma^2} \big( - 2 y_s f_{\hat \theta_s}(x_s) + 2 y_s f_\theta(x_s)  \big) \right) ]\\
\end{align*}
From here, we compute the conditional expectation inside the integral. We expand the squares, simplify and substitute $y_t = f_{\theta^*}(x_t) + \epsilon_t$. After a bit of work we arrive at 
\begin{align*}
    &\EE[\exp\left( - \tfrac{1}{2\sigma^2} \big(f_\nu(x_t) - y_t\big)^2 + \tfrac{1}{2\sigma^2} \big(f_{\theta^*}(x_t) - y_t\big)^2 \right) |\cF_t ] \\
    &= \exp\left(- \tfrac{1}{2\sigma^2} \big(f_{\hat \theta_{t-1}}(x_t) - f_{\theta^*}(x_t)\big)^2 \right) \EE[\exp\left( \epsilon_t \cdot \tfrac{1}{\sigma^2} \big(f_{\hat \theta_{t-1}}(x_t) - f_{\theta^*}(x_t)\big) \right) |\cF_t] \,.
\end{align*}
Next, we use that $\epsilon_t$ is $\sigma^2$-sub-Gaussian, which, by definition, implies that 
\begin{align*}
    \EE[\exp\left( \epsilon_t \cdot \tfrac{1}{\sigma^2} \big(f_{\hat \theta_{t-1}}(x_t) - f_{\theta^*}(x_t)\big) \right) ] \leq \exp\left(\tfrac{1}{2\sigma^2} \big(f_{\hat \theta_{t-1}}(x_t) - f_{\theta^*}(x_t)\big)^2 \right) \,.
\end{align*}
We conclude that $\EE[E_t(\theta^*)|\cF_{t-1}] \leq  E_{t-1}(\theta^*)$ and $\EE[E_1(\theta^*)] \leq 1$. The claim follows using Ville's inequality.
\end{proof}

% Lastly, we remark that one can reverse the setup, and start with a loss function $l_t : \Theta \times \cY \rightarrow \bR$. Let $\cP$ be the set of distributions for which the process $E_t(\theta) = \prod_{s=1}^t \exp(l_s(\theta, y_s) - l_s(\hat \theta_{s-1}, y_s) )$ is a supermartingale\todoj{how is $\theta^*$ and $\bP$ related?}. Then $\{ \theta \in \Theta : \log E_t(\theta) \leq \log \frac{1}{\delta} \}$ defines a confidence set, as long as $\bP \in \cP$. Setting  $l_t(\theta, y) = \frac{1}{2\sigma^2}(f_\theta(x_t) - y)^2$ recovers the sub-Gaussian case, and choosing the log-loss $l_t(\theta, y) = - \log p_t(y|\theta)$ recovers the standard likelihood ratio. 

% $$\theta^*_t = \argmin_{\theta \in \Theta} \sum_{s=1}^t \bE_s[l_s(\theta, y_s)]$$

% Define the \emph{generalized sequential likelihood ratio},
% $$E_t(\nu, \theta) = \prod_{s=1}^t \exp(l_s(\theta, y_s) - l_s(\nu, y_s) )$$
% Note that for the log loss, $E_t(\nu, \theta) = R_t(\nu, \theta)$ recovers the standard sequential likelihood ratio. Define
% % $$D_t^E(\nu \| \theta) = \sum_{s=1}^t \bE_s[\log E_s(\nu, \theta)]$$
% % and assume that $D_t^E$ is a divergence under the true distribution
% % Assume that there exists a par
% $$\theta^*_t = \argmin_{\theta \in \Theta} \sum_{s=1}^t \bE_s[\log E_s(\theta^*, \theta)]$$
% Note that the expectation recovers the KL (minimized by $\theta^*$) for the log loss, and the square loss over means for the sub-Gaussian case (offset by a variance term). 
% $$\theta^* = \argmin_{\theta \in \Theta} \sum_{s=1}^t \bE_s[\log E_s(\bP, \theta)]$$

\subsection{Convex Model Classes}\label{sec:convex}
We return to the original definition of the model class as a parameterized family of conditional densities $\cM = \{ p_\theta(y|x) : \theta \in \Theta\}$. Moreover, assume that there exists a conditional density $p^*(y|x)$ such that for all $t \geq 1$, $\frac{d\bP_t}{d\xi} = p^*(\cdot|x_t) d\xi$. Crucially, we do \emph{not} require that $p^* \in \cM$. 

Throughout this section, we make the assumption that $\cM$ is convex. Note that convexity is required in the space of distributions, i.e., all finite mixtures of densities in $\cM$ are contained in $\cM$. In general, convexity of $\Theta$ does not imply convexity of $\cM$. Nevertheless, there are many examples of convex model classes, including, for example, all finite mixtures of any family of distributions. 

Our main tool is the \emph{reverse information projection} theorem by \citet{li1999estimation}, see also \citet{lardy2024reverse}. Applied to our setup, the theorem states that for any sequence $q_n \in \cM$ such that $$\lim_{n \rightarrow \infty} \KL(p^*\|q_n) = \inf_{q \in \cM} \KL(p^*\|q) < \infty\,,$$ there exists a unique (sub-)probability measure $q^* d\xi$ such that $\KL(p^*\|q^*) = \inf_{q \in \cM} \KL(p^*\|q)$. Moreover, the reverse information projection theorem shows that for any $q_\theta \in \cM$,
\begin{align}
    \EE[\frac{p_\theta(y_t|x_t)}{q^*(y_t|x_t)}\big | \cF_t] \leq 1 \,. \label{eq:rips-e-value}
\end{align}
A technical condition is required to ensure that the limiting element $q^*$ is contained $\cM$. If we require that $\cY$ is a complete separable metric space, and $\cM$ is sequentially compact (with respect to the weak topology), then Prokhorov's theorem implies that $q^* \in \cM$. A similarly flavoured result (stated without mixing distributions) is by \citet[Proposition 7]{wasserman2020universal}

\begin{theorem}[Convex Model Classes] Assume that $\cM$ is convex and there exists $q^* \in \cM$ such that $\KL(p^*\|q^*) = \inf_{q \in \cM} \KL(p^*\|q)$. Then the sequential likelihood mixing confidence set, defined for any $\cF_t$-adapted sequence of distributions $\mu_0, \mu_1, \mu_2, \dots$ in $\sP(\Theta)$ (see \cref{result:posterior_mixing}), defines a $(1-\delta)$-confidence sequence for $q^* \in \cM$, i.e., $\bP[q^* \in C_t, \forall t \geq 1] \geq 1-\delta$.
\end{theorem}
\begin{proof}
Define the sequential marginal likelihood ratio w.r.t. to a conditional density $q(\cdot|x)$,
\begin{align*}
J_t(q) = \prod_{s=1}^t\frac{\int p_\nu(y_s|x_s) d\mu_{s-1}(\nu)}{q(y_s|x_s)}
\end{align*}
The reverse information projection theorem, specifically \cref{eq:rips-e-value}, implies that $J_t(q^*)$ is a non-negative supermartingale with $\EE[J_1] \leq 1$. The theorem follows using Ville's inequality.
\end{proof}

\section{Tempered Likelihood Ratios}\label{sec:tempered}

The Bayesian update rule can be generalized by introducing a \emph{temperature} parameter $\beta > 0$,
\begin{align*}
    \mu_t(\theta) \propto \prod_{s=1}^t p_s(y_s|\theta)^\beta \mu_0(\theta) \,.
\end{align*}
The generalized update rule has been studied under various names, e.g. as \emph{fractional posteriors} \citep{bhattacharya2019bayesian}, \emph{powered likelihoods} \citet{holmes2017assigning} and \emph{tempered posteriors} \citep{alquier2020concentration}, often in the context of adding robustness to misspecification in the Bayesian model \citep{grunwald2012safe}. The result presented below are similar in spirit to the work by \citet{zhang2006varepsilon}, who studies complexity bounds for density estimation in the classical i.i.d.~setting.

\paragraph{Divergences} A few more definitions will be useful, we follow the exposition of \citet{van2014renyi}. Let $p,q \in \sP(\cY)$ be two distributions over the observation space $\cY$. We assume that $p,q$ admit densities w.r.t. a common base measure. We define the Rényi divergence for  $p,q$ and parameter $0 \leq \zeta \leq 1$,
\begin{align*}
    D_\zeta(p\|q) = \frac{1}{\zeta - 1} \log \int p(x)^{\zeta} q(x)^{1- \zeta} dx \,.
\end{align*}
For $0< \zeta < 1$, it holds that
\begin{align*}
   (1-\zeta)  D_\zeta(p\|q) = \zeta D_{1-\zeta}(q\|p)\,.
\end{align*}
Moreover, the Hellinger distance is given by 
\begin{align*}
    H^2(p\|q) = \int \big(\sqrt{p(x)} - \sqrt{q(x)}\big)^2 dx\,.
\end{align*}
Hellinger and Rényi divergences satisfy the following relation:
\begin{align*}
    \frac{1}{2} H^2(p\|q) \leq D_{1/2}(p\|q)\,.
\end{align*}
For notational convenience, we define
\begin{align*}
    D_{\zeta,t}(\theta\|\nu) &:= D_\zeta(p_t(\cdot|\theta)\|p_t(\cdot|\nu)) \,,\\
    H^2_t(\theta\|\nu) &:= H^2(p_t(\cdot|\theta)\|p_t(\cdot|\nu)) \,.\\
\end{align*}

\subsection{Tempered Confidence Sequences}
Let $\hat \theta_0, \hat \theta_1, \dots$ be an $\cF_t$-adapted sequence of estimators.
Define the \emph{tempered} log-ratio, 
\begin{align}
    A_t^\beta(\theta) = - \beta \log \frac{p_t(y_t|\hat \theta_{t-1})}{p_t(y_t|\theta)} \,.
\end{align}
In particular, by applying the exponential function, we get the \emph{tempered likelihood ratio},
\begin{align*}
\exp(-A_t^\beta(\theta)) = \left(\frac{p_t(y_t|\hat \theta_{t-1})}{p_t(y_t|\theta)}\right)^\beta
\end{align*}
As a side remark, Jensen's inequality implies that 
% \begin{align*}
$\bE[\exp(-A_t^\beta(\theta^*))] \leq  \bE\big[\frac{p_t(y_t|\hat \theta_{t-1})}{p_t(y_t|\theta^*)}\big]^\beta = 1$.
% \end{align*}
Hence $\prod_{s=1}^t A_t^\beta(\theta^*)$ is a super-martingale, and all results presented in the main paper continue to hold for the tempered ratio. However, we can strengthen the construction by enforcing the martingale property. Define
\begin{align*}
% M_t(\theta) = \frac{\exp(- \sum_{s=1}^t A_t^\beta(\theta))}{\textstyle \prod_{s=1}^t \bE_\theta[\exp(- A_t^\beta(\theta))|\cF_t]}
M_t(\theta) = \frac{\exp(- \sum_{s=1}^t A_t^\beta(\theta))}{\textstyle \prod_{s=1}^t \int \exp(- A_t^\beta(\theta)) p_t(y|\theta) dy} \,.
\end{align*}
By definition, $M_t(\theta^*)$ is a non-negative martingale. 
Hence, Ville's inequality implies that
\begin{align*}
   \bP\left[- \sum_{s=1}^t \log \int \exp(- A_s^\beta(\theta^*)) p_s(y|\theta^*) dy - \sum_{s=1}^t A_s(\theta^*) \geq \log \frac{1}{\delta}\right] \leq \delta \,.
\end{align*}
Finally, we note that 
\begin{align*}
   - \log \int \exp(- A_t^\beta(\theta)) p_t(y|\theta) dy = (1 - \beta) D_{\beta,t}(\hat \theta_{t-1} \| \theta)= \beta D_{1-\beta,t}(\theta\|\hat \theta_{t-1}) \,.
\end{align*}

\begin{theorem}[Tempered Confidence Set]\label{result:tempered} Let $\hat \theta_0, \hat \theta_1, \dots$ be an $\cF_t$-adapted sequence of estimators. Define the log regret $\Lambda_t(\theta) = - \sum_{s=1}^t \log p_s(y_s|\hat \theta_{s-1}) - L_t(\theta)$ for $\theta \in \Theta$, and
    \begin{align*}
        C_t^\beta = \left\{ \theta \in \Theta : \sum_{s=1}^t D_{1-\beta,s}(\theta\|\hat \theta_{s-1}) - \Lambda_t(\theta)\leq \frac{1}{\beta} \log \frac{1}{\delta}\right\} \,.
    \end{align*}
    Then $C_t^\beta$ defines a $(1-\delta)$-confidence sequence. Moreover, define
    \begin{align*}
        C_t^H = \left\{ \theta \in \Theta :\sum_{s=1}^t  H_s^2(\theta\|\hat \theta_{s-1}) - \Lambda_t(\theta)\leq 2 \log \frac{1}{\delta}\right\} \,.
    \end{align*}
    Then $C_t^H$ defines a $(1-\delta)$-confidence sequence.
\end{theorem}
As a consequence of the result, assume that the estimation sequence is constructed to achieve bounded regret, $\Lambda_t(\theta) \leq B_t$ for a predictable sequence $B_t \geq 0$, typically related to the complexity of the model class (c.f.,~\cref{sec:oco} and \cref{result:regret}), see also \citet{zhang2006varepsilon}. Then \cref{result:tempered} provides confidence sequences that only depend on the Hellinger or Renyi divergences. The advantage is, that unlike the likelihood ratios, the divergence is predictable quantity under the filtration $\cF_t$, hence can be used in sequential decision making setting to control the state $x_t$. This is one of the reasons why similar bounds have recently gained interest in the sequential decision-making literature \citep[e.g.,][]{chen2022unified,foster2021statistical,foster2023tight,wagenmaker2023instance}. In the next section, we provide another application to sparse estimation.

% \todoj{is there any advantage of keeping both the expectation and realized terms?}
% Note that $\sum_{s=1}^t A_t(\theta^*)$ is the log-loss regret and is controlled with probability 1. 
% Moreover, for $\beta=1/2$
% \begin{align*}
%     -\log \EE_t[ \exp(-A_t(\theta^*))] \geq 1 - \EE_t[ \exp(-A_t(\theta^*))] = \frac{1}{2} H^2(p_{\theta^*}, p_{\theta_t})
% \end{align*}
% Therefore
% \begin{align*}
%     \sum_{s=1}^t \frac{1}{2} H^2(p_{\theta^*}, p_{\theta_t}) \leq \log \frac{1}{\alpha} + \sum_{s=1}^t A_t(\theta^*) 
% \end{align*}
% The important difference is that the left-hand side does not depend on the last observation $y_t$, hence the confidence set is ``predictable'', unlike the likelihood ratio.

% Another way to proceed is

% note on introducing mixing distributions
% For $\beta \leq 1$,\todoj{mixing the tempered posterior vs tempering the mixed ratio}
% \begin{align}
%     \EE_{\theta \sim \mu_t}[ \left(\frac{p_{\theta}(y_t)}{p_{\theta^*}(y_t)} \right)^\beta] \leq  \left(\frac{\ip{\mu_t, p_t}}{p_{\theta^*}(y_t)} \right)^\beta 
% \end{align}


\subsection{Online Linear Prediction}\label{sec:sparse}
Recall the sequential linear regression setting from \cref{sec:bayes}. Specifically, assume that $\Theta \subset \bR^d$, and Gaussian likelihood $p_t(y|\theta) \sim \cN(\ip{\theta, x_t}, 1)$ for $\theta \in \Theta$ and covariates $x_t \in \bR^d$. Everything in this section generalizes to $\sigma$-sub-Gaussian noise distribution using the same arguments as in \cref{sec:subgaussian}. In a slight generalization to earlier results, assume that an online learning algorithm produces predictions $\hat y_t \in \cY$ (opposed to $\hat \theta_t \in \Theta$), in way such that the regret satisfies the following bound for any $\theta \in \Theta$,
\begin{align*}
    \Lambda_t(\theta) = \sum_{s=1}^t \frac{1}{2}(\hat y_{s-1} - y_s)^2 - \frac{1}{2}(\ip{\theta, x_s} - y_s)^2 \leq B_t(\theta) \,.
\end{align*}
For a concrete instantiation in the linear setting, we refer to the famous Vovk-Azoury-Warmuth forecaster \citep{vovk1997competitive,azoury2001relative}. We remark that the bound has been further generalized to non-parametric settings by \citet{rakhlin2014online}. We are now in place to generalize the online-to-confidence set conversion by \citet{abbasi2012online}.

\begin{lemma}[Online-To-Confidence Convertion] Assume the sequential linear regression setup defined above with a known bound $\Lambda_t(\theta^*) \leq B_t$. For any $\cF_t$-adapted sequence $\hat y_0, \hat y_1, \hat y_2, \dots,$ and  $0 < \beta < 1$, let
    \begin{align*}
        C_t = \left\{ \theta \in \Theta : \sum_{s=1}^t \frac{1}{2} \big(\hat y_{s-1} - \ip{\theta, x_s}\big)^2 \leq \frac{1}{\beta - \beta^2 } \log \frac{1}{\delta} + \frac{\beta}{\beta - \beta^2} B_t \right\}
    \end{align*}
    Then $C_t$ defines a $(1-\delta)$-confidence sequence.
\end{lemma}
\begin{proof}
The plan is to compute  $- \log \int \exp(- A_t^\beta(\theta)) p_t(y|\theta) dy$ for the Gaussian distribution, and then invoke \cref{result:tempered}. It is useful to note that for Gaussian distributed variable $\epsilon \sim \cN(0, \sigma^2)$ the moment generating function is $\bE[\exp(t \epsilon)] = \exp(\frac{1}{2}\sigma^2 t^2)$.
A short calculation reveals that 
\begin{align*}
    & \int \exp\big(- A_t^\beta(\theta)\big) p_t(y|\theta) dy\\
    &= \int \exp\Big(-  \frac{\beta}{2} \big(\hat y_{s-1} - y\big)^2 + \frac{\beta}{2} \big(\ip{\theta,x_t} - y\big)^2 \Big)  p_t(y|\theta) dy\\
    &= \int \exp\Big(-  \frac{\beta}{2} \big(\hat y_{s-1} - {\theta,x_t}\big)^2 + \beta \epsilon_t \big(\ip{\theta,x_t} - \ip{\theta,x_t}\big) \Big)  p_t(y|\theta) dy\\
    &= \exp\Big(-  \frac{\beta - \beta^2}{2} \big(\hat y_{s-1} - \ip{\theta,x_t}\big)^2 \Big) \,.
\end{align*}
Hence,
\begin{align*}
    & - \log \int \exp\big(- A_t^\beta(\theta)\big) p_t(y|\theta) dy =  \frac{\beta - \beta^2}{2} \big(\hat Y_{s-1} - \ip{\theta,x_t}\big)^2 \,.
\end{align*}
Lastly, we make use of the assumption that $\Lambda_t(\theta^*) \leq B_t$. The result follows by intersecting the confidence set $C_t^\beta$ from \cref{result:tempered} with  $\{\theta \in \Theta : \Lambda_t(\theta) \leq B_t\}$.
\end{proof}

\paragraph{Sparse Linear Regression} We make the additional assumption that the true parameter $\theta^* \in \Theta$ is $k$-sparse, i.e. $\|\theta^*\|_0 \leq k$. The key insight is that \citet{gerchinovitz2011sparsity} provides an online algorithm, producing the sequence $\hat y_0, \hat y_1, \dots$, for which $B_t(\theta^*) \leq  \cO( k \log(t))$. We compare to the result by \citet{abbasi2012online} in the same setting. For $\beta = \frac{1}{2}$, our result reads
\begin{align*}
   C_t = \left\{ \theta \in \Theta : \sum_{s=1}^t \frac{1}{2} (\hat y_{s-1} - \ip{\theta, x_s})^2 \leq 4\log \frac{1}{\delta} + 2 B_t(\theta^*) \right\} \,.
\end{align*}
In comparison, the result by \citet[Theorem 1]{abbasi2012online} reads, in our notation,
\begin{align*}
C_t = \left\{ \theta \in \mathbb{R}^d : \sum_{s=1}^{t}  \frac{1}{2}(\hat y_{s-1} - \langle \theta, x_t \rangle)^2 \leq 16 \log \frac{1}{\delta} + \frac{1}{2} + 2B_t(\theta^*) + 16\log (\sqrt{8} + \sqrt{1 + 2 B_t(\theta^*)}) \right\}
\end{align*}
The additional terms stem from using recursive inequality on the square-error, which we avoid using the more direct argument via the tempered likelihood ratio. This demonstrates the benefit of the sequential likelihood framework for deriving confidence sets via the online-to-confidence set conversion.
% Assume a sparse linear model $y_t = \ip{x_t, \theta^*} + \epsilon_t$ with $\sigma$-sub-Gaussian noise, i.e. satisfying $\bE[\exp(t \cdot \epsilon_t)] \leq \exp(\frac{\sigma^2 t^2}{2})$. The true parameter satisfies $\|\theta\|_0 \leq m$.
% Let $\hat Y_t \in \bR$ be a sequence of predictions from an online learning algorithm with regret bound
% \begin{align*}
%     \rho_n(\theta^*) = \sum_{t=1}^n (\hat Y_t - y_t)^2 - (\ip{\theta^*, x_t} - y_t)^2 = \sum_{t=1}^n r_t(\theta^*) \leq B_n
% \end{align*}
% where we defined $r_t(\theta^*) = (\hat Y_t - y_t)^2 - (\ip{\theta^*, x_t} - y_t)^2$. Note that
% \begin{align}
%     q_t := (\hat Y_t - \ip{\theta^*, x_t})^2 = r_t(\theta^*) + 2 \epsilon_t (\hat Y_t - \ip{\theta^*, x_t}) \label{eq:q_t-to-r_t}
% \end{align}
% Further define 
% \begin{align*}
% Q_n = \sum_{t=1}^n q_t = \sum_{t=1}^n (\hat Y_t - \ip{\theta^*, x_t})^2 
% \end{align*}
% Our goal is to find a high-probability upper bound for $Q_n$ and turn the bound into a confidence set for $\theta^*$.
% We define the following process for parameters $a,b \geq 0$:
% \begin{align}
%     M_t^{a,b} = \exp\left(\tfrac{b}{2} Q_t - \tfrac{b + a}{2} \rho_t(\theta^*)\right)
% \end{align}
% The choice that we make shortly is $a = b = \frac{1}{4 \sigma^2}$. Note that
% \begin{align*}
%     \EE[M_t^{a,b}|\cF_{t-1}] &= \EE[\exp\left(\tfrac{b}{2} Q_t - \tfrac{b + a}{2} \rho_t(\theta^*)\right) | \cF_t]\\
%     &= M_{t-1}^{a,b} \,\EE[\exp\left(\tfrac{b}{2} q_t - \tfrac{b + a}{2} r_t(\theta^*)\right) | \cF_t]\\
%     &= M_{t-1}^{a,b} \,\EE[\exp\left(- \tfrac{a}{2} q_t  +\tfrac{b + a}{2} q_t - \tfrac{b + a}{2} r_t(\theta^*)\right) | \cF_t]\\
%     &\stackrel{(1)}{=} M_{t-1}^{a,b} \,\EE[\exp\left(- \tfrac{a}{2} q_t  + (a+b) (\hat Y_t - \ip{\theta^*, x_t}) \epsilon_t \right) | \cF_t]\\
%     &\stackrel{(2)}{\leq} M_{t-1}^{a,b} \,\EE[\exp\left(- \tfrac{a}{2} q_t  + \sigma^2 \tfrac{(a+b)^2}{2} (\hat Y_t - \ip{\theta^*, x_t})^2 \right) | \cF_t]\\
%     &= M_{t-1}^{a,b}\, \EE[\exp\left(- \tfrac{a}{2} q_t  + \sigma^2 \tfrac{(a+b)^2}{2} q_t \right) | \cF_t]
% \end{align*}
% Equation $(1)$ follows from \ref{eq:q_t-to-r_t}; the $\frac{a+b}{2}$-factor was introduced to precisely cancel $r_t(\theta^*)$. The inequality $(2)$ follows by assumption that the noise is $\sigma$-sub-Gaussian. We now choose $a$ and $b$ to cancel the remaining terms out, i.e.

% \begin{align*}
% a = \sigma^2 (a+b)^2
% \end{align*}
% We can obtain the values of $a$ and $b$ by solving a quadratic, which gives as a general formula $a = \frac{1}{\sigma^2} - b \pm \frac{1}{2\sigma}\sqrt{\frac{1}{\sigma^2} - 4b}$ . An easy choice is $a=b=\frac{1}{4 \sigma^2}$. This is probably close to optimal (after writing out Ville's inequality it becomes intuitively clear  that we want large $b$ and small $a$), and $b=\frac{1}{4 \sigma^2}$ is also the largest feasible value for $b$. Applying Ville's inequality leads to the following bound:
% \begin{align}
% \bP[ M_t^{a,b} \geq \tfrac{1}{\alpha}] \leq \alpha
% \end{align}
% In other words, with probability at least $1-\delta$,
% \begin{align}
% \frac{1}{2} \sum_{t=1}^n (\hat Y_t - \ip{\theta^*, x_t})^2 \leq  4\sigma^2 \log \frac{1}{\alpha} + \rho_n(\theta^*) \leq  4\sigma^2  \log \frac{1}{\alpha} + B_n
% \end{align}


% For $a=b=\frac{1}{4 \sigma^2}$ we have
% \begin{align}
%     C_t^{\frac{1}{4},\frac{1}{4}} &= \left\{\theta : \frac{1}{2} \sum_{t=1}^n (\hat Y_t - \ip{\theta, x_t})^2 \leq  4 \sigma^2 \log \frac{1}{\alpha} + \rho_n(\theta^*)\right\} \\
%     &\subset \left\{\theta : \frac{1}{2} \sum_{t=1}^n (\hat Y_t - \ip{\theta, x_t})^2 \leq  4 \sigma^2 \log \frac{1}{\alpha} + B_n \right\} 
% \end{align}
% On the other hand, $a = 1$ and $b=0$ also leads to a valid confidence set, since $\exp(-\frac{1}{2}\rho_t(\theta^*))$ is the usual likelihood ratio for (sub-)Gaussian likelihood. Writing explicitly,
% \begin{align}
%     C_t = C_t^{0,1} &= \left\{\theta : \frac{1}{2}\rho(\theta) \leq  \log \frac{1}{\alpha} \right\} \\
%     &= \left\{\theta : \frac{1}{2} \sum_{t=1}^n (\hat Y_t - \ip{\theta, x_t})^2 \leq \log \frac{1}{\alpha} + \frac{1}{2} \sum_{t=1}^n (\hat Y_t - y_t)^2\right\} 
% \end{align}
% \section{Experiments}
\label{sec:experiments}
The experiments are designed to address two key research questions.
First, \textbf{RQ1} evaluates whether the average $L_2$-norm of the counterfactual perturbation vectors ($\overline{||\perturb||}$) decreases as the model overfits the data, thereby providing further empirical validation for our hypothesis.
Second, \textbf{RQ2} evaluates the ability of the proposed counterfactual regularized loss, as defined in (\ref{eq:regularized_loss2}), to mitigate overfitting when compared to existing regularization techniques.

% The experiments are designed to address three key research questions. First, \textbf{RQ1} investigates whether the mean perturbation vector norm decreases as the model overfits the data, aiming to further validate our intuition. Second, \textbf{RQ2} explores whether the mean perturbation vector norm can be effectively leveraged as a regularization term during training, offering insights into its potential role in mitigating overfitting. Finally, \textbf{RQ3} examines whether our counterfactual regularizer enables the model to achieve superior performance compared to existing regularization methods, thus highlighting its practical advantage.

\subsection{Experimental Setup}
\textbf{\textit{Datasets, Models, and Tasks.}}
The experiments are conducted on three datasets: \textit{Water Potability}~\cite{kadiwal2020waterpotability}, \textit{Phomene}~\cite{phomene}, and \textit{CIFAR-10}~\cite{krizhevsky2009learning}. For \textit{Water Potability} and \textit{Phomene}, we randomly select $80\%$ of the samples for the training set, and the remaining $20\%$ for the test set, \textit{CIFAR-10} comes already split. Furthermore, we consider the following models: Logistic Regression, Multi-Layer Perceptron (MLP) with 100 and 30 neurons on each hidden layer, and PreactResNet-18~\cite{he2016cvecvv} as a Convolutional Neural Network (CNN) architecture.
We focus on binary classification tasks and leave the extension to multiclass scenarios for future work. However, for datasets that are inherently multiclass, we transform the problem into a binary classification task by selecting two classes, aligning with our assumption.

\smallskip
\noindent\textbf{\textit{Evaluation Measures.}} To characterize the degree of overfitting, we use the test loss, as it serves as a reliable indicator of the model's generalization capability to unseen data. Additionally, we evaluate the predictive performance of each model using the test accuracy.

\smallskip
\noindent\textbf{\textit{Baselines.}} We compare CF-Reg with the following regularization techniques: L1 (``Lasso''), L2 (``Ridge''), and Dropout.

\smallskip
\noindent\textbf{\textit{Configurations.}}
For each model, we adopt specific configurations as follows.
\begin{itemize}
\item \textit{Logistic Regression:} To induce overfitting in the model, we artificially increase the dimensionality of the data beyond the number of training samples by applying a polynomial feature expansion. This approach ensures that the model has enough capacity to overfit the training data, allowing us to analyze the impact of our counterfactual regularizer. The degree of the polynomial is chosen as the smallest degree that makes the number of features greater than the number of data.
\item \textit{Neural Networks (MLP and CNN):} To take advantage of the closed-form solution for computing the optimal perturbation vector as defined in (\ref{eq:opt-delta}), we use a local linear approximation of the neural network models. Hence, given an instance $\inst_i$, we consider the (optimal) counterfactual not with respect to $\model$ but with respect to:
\begin{equation}
\label{eq:taylor}
    \model^{lin}(\inst) = \model(\inst_i) + \nabla_{\inst}\model(\inst_i)(\inst - \inst_i),
\end{equation}
where $\model^{lin}$ represents the first-order Taylor approximation of $\model$ at $\inst_i$.
Note that this step is unnecessary for Logistic Regression, as it is inherently a linear model.
\end{itemize}

\smallskip
\noindent \textbf{\textit{Implementation Details.}} We run all experiments on a machine equipped with an AMD Ryzen 9 7900 12-Core Processor and an NVIDIA GeForce RTX 4090 GPU. Our implementation is based on the PyTorch Lightning framework. We use stochastic gradient descent as the optimizer with a learning rate of $\eta = 0.001$ and no weight decay. We use a batch size of $128$. The training and test steps are conducted for $6000$ epochs on the \textit{Water Potability} and \textit{Phoneme} datasets, while for the \textit{CIFAR-10} dataset, they are performed for $200$ epochs.
Finally, the contribution $w_i^{\varepsilon}$ of each training point $\inst_i$ is uniformly set as $w_i^{\varepsilon} = 1~\forall i\in \{1,\ldots,m\}$.

The source code implementation for our experiments is available at the following GitHub repository: \url{https://anonymous.4open.science/r/COCE-80B4/README.md} 

\subsection{RQ1: Counterfactual Perturbation vs. Overfitting}
To address \textbf{RQ1}, we analyze the relationship between the test loss and the average $L_2$-norm of the counterfactual perturbation vectors ($\overline{||\perturb||}$) over training epochs.

In particular, Figure~\ref{fig:delta_loss_epochs} depicts the evolution of $\overline{||\perturb||}$ alongside the test loss for an MLP trained \textit{without} regularization on the \textit{Water Potability} dataset. 
\begin{figure}[ht]
    \centering
    \includegraphics[width=0.85\linewidth]{img/delta_loss_epochs.png}
    \caption{The average counterfactual perturbation vector $\overline{||\perturb||}$ (left $y$-axis) and the cross-entropy test loss (right $y$-axis) over training epochs ($x$-axis) for an MLP trained on the \textit{Water Potability} dataset \textit{without} regularization.}
    \label{fig:delta_loss_epochs}
\end{figure}

The plot shows a clear trend as the model starts to overfit the data (evidenced by an increase in test loss). 
Notably, $\overline{||\perturb||}$ begins to decrease, which aligns with the hypothesis that the average distance to the optimal counterfactual example gets smaller as the model's decision boundary becomes increasingly adherent to the training data.

It is worth noting that this trend is heavily influenced by the choice of the counterfactual generator model. In particular, the relationship between $\overline{||\perturb||}$ and the degree of overfitting may become even more pronounced when leveraging more accurate counterfactual generators. However, these models often come at the cost of higher computational complexity, and their exploration is left to future work.

Nonetheless, we expect that $\overline{||\perturb||}$ will eventually stabilize at a plateau, as the average $L_2$-norm of the optimal counterfactual perturbations cannot vanish to zero.

% Additionally, the choice of employing the score-based counterfactual explanation framework to generate counterfactuals was driven to promote computational efficiency.

% Future enhancements to the framework may involve adopting models capable of generating more precise counterfactuals. While such approaches may yield to performance improvements, they are likely to come at the cost of increased computational complexity.


\subsection{RQ2: Counterfactual Regularization Performance}
To answer \textbf{RQ2}, we evaluate the effectiveness of the proposed counterfactual regularization (CF-Reg) by comparing its performance against existing baselines: unregularized training loss (No-Reg), L1 regularization (L1-Reg), L2 regularization (L2-Reg), and Dropout.
Specifically, for each model and dataset combination, Table~\ref{tab:regularization_comparison} presents the mean value and standard deviation of test accuracy achieved by each method across 5 random initialization. 

The table illustrates that our regularization technique consistently delivers better results than existing methods across all evaluated scenarios, except for one case -- i.e., Logistic Regression on the \textit{Phomene} dataset. 
However, this setting exhibits an unusual pattern, as the highest model accuracy is achieved without any regularization. Even in this case, CF-Reg still surpasses other regularization baselines.

From the results above, we derive the following key insights. First, CF-Reg proves to be effective across various model types, ranging from simple linear models (Logistic Regression) to deep architectures like MLPs and CNNs, and across diverse datasets, including both tabular and image data. 
Second, CF-Reg's strong performance on the \textit{Water} dataset with Logistic Regression suggests that its benefits may be more pronounced when applied to simpler models. However, the unexpected outcome on the \textit{Phoneme} dataset calls for further investigation into this phenomenon.


\begin{table*}[h!]
    \centering
    \caption{Mean value and standard deviation of test accuracy across 5 random initializations for different model, dataset, and regularization method. The best results are highlighted in \textbf{bold}.}
    \label{tab:regularization_comparison}
    \begin{tabular}{|c|c|c|c|c|c|c|}
        \hline
        \textbf{Model} & \textbf{Dataset} & \textbf{No-Reg} & \textbf{L1-Reg} & \textbf{L2-Reg} & \textbf{Dropout} & \textbf{CF-Reg (ours)} \\ \hline
        Logistic Regression   & \textit{Water}   & $0.6595 \pm 0.0038$   & $0.6729 \pm 0.0056$   & $0.6756 \pm 0.0046$  & N/A    & $\mathbf{0.6918 \pm 0.0036}$                     \\ \hline
        MLP   & \textit{Water}   & $0.6756 \pm 0.0042$   & $0.6790 \pm 0.0058$   & $0.6790 \pm 0.0023$  & $0.6750 \pm 0.0036$    & $\mathbf{0.6802 \pm 0.0046}$                    \\ \hline
%        MLP   & \textit{Adult}   & $0.8404 \pm 0.0010$   & $\mathbf{0.8495 \pm 0.0007}$   & $0.8489 \pm 0.0014$  & $\mathbf{0.8495 \pm 0.0016}$     & $0.8449 \pm 0.0019$                    \\ \hline
        Logistic Regression   & \textit{Phomene}   & $\mathbf{0.8148 \pm 0.0020}$   & $0.8041 \pm 0.0028$   & $0.7835 \pm 0.0176$  & N/A    & $0.8098 \pm 0.0055$                     \\ \hline
        MLP   & \textit{Phomene}   & $0.8677 \pm 0.0033$   & $0.8374 \pm 0.0080$   & $0.8673 \pm 0.0045$  & $0.8672 \pm 0.0042$     & $\mathbf{0.8718 \pm 0.0040}$                    \\ \hline
        CNN   & \textit{CIFAR-10} & $0.6670 \pm 0.0233$   & $0.6229 \pm 0.0850$   & $0.7348 \pm 0.0365$   & N/A    & $\mathbf{0.7427 \pm 0.0571}$                     \\ \hline
    \end{tabular}
\end{table*}

\begin{table*}[htb!]
    \centering
    \caption{Hyperparameter configurations utilized for the generation of Table \ref{tab:regularization_comparison}. For our regularization the hyperparameters are reported as $\mathbf{\alpha/\beta}$.}
    \label{tab:performance_parameters}
    \begin{tabular}{|c|c|c|c|c|c|c|}
        \hline
        \textbf{Model} & \textbf{Dataset} & \textbf{No-Reg} & \textbf{L1-Reg} & \textbf{L2-Reg} & \textbf{Dropout} & \textbf{CF-Reg (ours)} \\ \hline
        Logistic Regression   & \textit{Water}   & N/A   & $0.0093$   & $0.6927$  & N/A    & $0.3791/1.0355$                     \\ \hline
        MLP   & \textit{Water}   & N/A   & $0.0007$   & $0.0022$  & $0.0002$    & $0.2567/1.9775$                    \\ \hline
        Logistic Regression   &
        \textit{Phomene}   & N/A   & $0.0097$   & $0.7979$  & N/A    & $0.0571/1.8516$                     \\ \hline
        MLP   & \textit{Phomene}   & N/A   & $0.0007$   & $4.24\cdot10^{-5}$  & $0.0015$    & $0.0516/2.2700$                    \\ \hline
       % MLP   & \textit{Adult}   & N/A   & $0.0018$   & $0.0018$  & $0.0601$     & $0.0764/2.2068$                    \\ \hline
        CNN   & \textit{CIFAR-10} & N/A   & $0.0050$   & $0.0864$ & N/A    & $0.3018/
        2.1502$                     \\ \hline
    \end{tabular}
\end{table*}

\begin{table*}[htb!]
    \centering
    \caption{Mean value and standard deviation of training time across 5 different runs. The reported time (in seconds) corresponds to the generation of each entry in Table \ref{tab:regularization_comparison}. Times are }
    \label{tab:times}
    \begin{tabular}{|c|c|c|c|c|c|c|}
        \hline
        \textbf{Model} & \textbf{Dataset} & \textbf{No-Reg} & \textbf{L1-Reg} & \textbf{L2-Reg} & \textbf{Dropout} & \textbf{CF-Reg (ours)} \\ \hline
        Logistic Regression   & \textit{Water}   & $222.98 \pm 1.07$   & $239.94 \pm 2.59$   & $241.60 \pm 1.88$  & N/A    & $251.50 \pm 1.93$                     \\ \hline
        MLP   & \textit{Water}   & $225.71 \pm 3.85$   & $250.13 \pm 4.44$   & $255.78 \pm 2.38$  & $237.83 \pm 3.45$    & $266.48 \pm 3.46$                    \\ \hline
        Logistic Regression   & \textit{Phomene}   & $266.39 \pm 0.82$ & $367.52 \pm 6.85$   & $361.69 \pm 4.04$  & N/A   & $310.48 \pm 0.76$                    \\ \hline
        MLP   &
        \textit{Phomene} & $335.62 \pm 1.77$   & $390.86 \pm 2.11$   & $393.96 \pm 1.95$ & $363.51 \pm 5.07$    & $403.14 \pm 1.92$                     \\ \hline
       % MLP   & \textit{Adult}   & N/A   & $0.0018$   & $0.0018$  & $0.0601$     & $0.0764/2.2068$                    \\ \hline
        CNN   & \textit{CIFAR-10} & $370.09 \pm 0.18$   & $395.71 \pm 0.55$   & $401.38 \pm 0.16$ & N/A    & $1287.8 \pm 0.26$                     \\ \hline
    \end{tabular}
\end{table*}

\subsection{Feasibility of our Method}
A crucial requirement for any regularization technique is that it should impose minimal impact on the overall training process.
In this respect, CF-Reg introduces an overhead that depends on the time required to find the optimal counterfactual example for each training instance. 
As such, the more sophisticated the counterfactual generator model probed during training the higher would be the time required. However, a more advanced counterfactual generator might provide a more effective regularization. We discuss this trade-off in more details in Section~\ref{sec:discussion}.

Table~\ref{tab:times} presents the average training time ($\pm$ standard deviation) for each model and dataset combination listed in Table~\ref{tab:regularization_comparison}.
We can observe that the higher accuracy achieved by CF-Reg using the score-based counterfactual generator comes with only minimal overhead. However, when applied to deep neural networks with many hidden layers, such as \textit{PreactResNet-18}, the forward derivative computation required for the linearization of the network introduces a more noticeable computational cost, explaining the longer training times in the table.

\subsection{Hyperparameter Sensitivity Analysis}
The proposed counterfactual regularization technique relies on two key hyperparameters: $\alpha$ and $\beta$. The former is intrinsic to the loss formulation defined in (\ref{eq:cf-train}), while the latter is closely tied to the choice of the score-based counterfactual explanation method used.

Figure~\ref{fig:test_alpha_beta} illustrates how the test accuracy of an MLP trained on the \textit{Water Potability} dataset changes for different combinations of $\alpha$ and $\beta$.

\begin{figure}[ht]
    \centering
    \includegraphics[width=0.85\linewidth]{img/test_acc_alpha_beta.png}
    \caption{The test accuracy of an MLP trained on the \textit{Water Potability} dataset, evaluated while varying the weight of our counterfactual regularizer ($\alpha$) for different values of $\beta$.}
    \label{fig:test_alpha_beta}
\end{figure}

We observe that, for a fixed $\beta$, increasing the weight of our counterfactual regularizer ($\alpha$) can slightly improve test accuracy until a sudden drop is noticed for $\alpha > 0.1$.
This behavior was expected, as the impact of our penalty, like any regularization term, can be disruptive if not properly controlled.

Moreover, this finding further demonstrates that our regularization method, CF-Reg, is inherently data-driven. Therefore, it requires specific fine-tuning based on the combination of the model and dataset at hand.

\end{document}
