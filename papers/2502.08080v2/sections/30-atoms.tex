
\begin{figure*}[ht!]
    \centering
    \includegraphics[width=0.99\linewidth]{plots/squares_p4_and_z1z2_d5000_10seeds_overlaps.pdf}
    \caption{Overlap $\normf{\bM}^2 / \Tr(\bQ)$ as a function of the sample complexity $\alpha$. The dots represent numerical simulation results, computed for $n = 5000$ (for the asymmetric method) or $d = 5000$ (for the symmetric method) and averaging over $10$ instances. (\textbf{Left}) Link function $g(z_1,z_2) = z_1z_2$. Solid lines are obtained from state evolution predictions eq. (\ref{eq:overlap_prod_zk},\ref{eq:examples_symmetric_general}). Dashed line at $\alpha_c \approx 0.59375$. (\textbf{Right}) Link function $g(\bz) = p^{-1}\norm{\bz}^2$, $p = 4$. Solid lines are obtained from state evolution predictions eq. (\ref{eq:overlap_asymmetric_squares},\ref{eq:examples_symmetric_general}). Dashed line at $\alpha_c = 2$.}
    \label{fig:z1z2_overlap}
\end{figure*}
In this section we illustrate the framework introduced in Section \ref{sec:main_results} to predict the asymptotic performance of the spectral estimators (\ref{eq:def:spectral_asymmetric},\ref{eq:def:spectral_symmetric}) for specific examples of link functions, providing a comparison between our asymptotic analytical results and finite size numerical simulations for the overlap between the spectral estimators and the weights $\mat{W}_\star$, defined as $m \coloneqq \nicefrac{\normf{\bM}}{\sqrt{\Tr(\bQ)}}$, where $\bM$ and $\bQ$ are the overlap matrices defined in eq. (\ref{eq:def:overlaps_amp}) correspondent to the fixed points in Lemmas \ref{result:1}, \ref{result:2}, \ref{result:3}, \ref{result:4}. In Figure \ref{fig:z1z2_overlap} we compare these theoretical predictions to numerical simulations at finite dimensions, respectively for the link functions $g(z_1,z_2) = z_1z_2$ and $g(\bz) = p^{-1}\norm{\bz}^2$. Additional numerical experiments are presented in Appendix \ref{app:example_details}.
\subsection{Asymmetric spectral method}\label{sec:examples_asymmetric}
%\subsection{$\dgout(y)$ jointly diagonalizable $\forall y$}
We provide closed-form expressions for the overlap parameter $m \coloneqq \nicefrac{\normf{\bM}}{\sqrt{\Tr(\bQ)}}$ of the spectral estimator $\matwhat_{\tens{L}}$ (\ref{eq:def:spectral_asymmetric}), for a selection of examples of link functions. The details of the derivation are given in Appendix \ref{app:example_details}.
\begin{itemize}[leftmargin=2em,wide=1pt]
    \item $g(z\in\R)$ (single-index model):
    \begin{align}
        \alpha_c &= \left(\E_{\rdm{y}\sim\Zout}\left[\left(\Var[z\big|\rdm{y}] - 1\right)^2\right]\right)^{-1},\\
        m^2 &= \Bigg(\frac{\alpha - \alpha_c}{\alpha + \alpha_c^2\E_{\rdm{y}\sim\Zout}\left[\left(\Var[z\big|\rdm{y}] - 1\right)^3\right]}\Bigg)_+
    \end{align}
    \item $g(\bz) = p^{-1}||\bz||^2$:
    \begin{equation}\label{eq:overlap_asymmetric_squares}
        \alpha_c = \frac{p}{2},\quad m^2 = \left(\frac{(2\alpha-p)}{2(\alpha+2)}\right)_+
    \end{equation}
    \item $g(\bz) = \operatorname{sign}(z_1z_2)$: 
    \begin{equation}
        \alpha_c = \frac{\pi^2}{4},\quad m^2 = \left(1-\frac{\pi^2}{4\alpha}\right)_+
    \end{equation}
    \item $g(\bz) = \prod_{k=1}^pz_k$:
   \begin{align}\label{eq:overlap_prod_zk}
        \alpha_c &= \left(\E_{\rdm{y}\sim\Zout}\left[\lambda(\rdm{y})^2\right]\right)^{-1},\\
        m^2 &= \left(\frac{\alpha - \alpha_c}{\alpha + \alpha_c^2\E_{\rdm{y}\sim\Zout}\left[\lambda(\rdm{y})^3\right]}\right)_+,
    \end{align}
    where
    \begin{equation}\label{eq:lambda_prod_zk}
        \lambda(y) = \begin{cases}
            \begin{array}{ll}
              |y|\frac{K_1(|y|)}{K_0(|y|)} +\rdm{y}- 1,   & p = 2 \\
              \frac{2G^{p, 0}_{0, p} \left( y^22^{-p}  \, \bigg| \, \begin{array}{c}
0 \\
\vect{e}_p
\end{array} \right)}{ G^{p, 0}_{0, p} \left( y^22^{-p} \, \bigg| \, \begin{array}{c}
0 \\
\bzero_p
\end{array} \right)} - 1,   & p\geq 3
            \end{array}
        \end{cases}
    \end{equation}
    and the previous expression are written in terms of the modified Bessel function of the second kind and Meijer $G$-function, with the notations $\bzero_p\in\R^p = (0,\ldots,0)^T$ and $\vect{e}_p\in\R^p=(0,\ldots,0,1)^T$.
    \item $g(z_1,z_2) = z_1z_2^{-1}$: $
        \alpha_c = 1,\quad m^2 = (1 -\alpha^{-1})_+$
\end{itemize}

\subsection{Symmetric spectral method}
We provide expressions for the overlap parameter $m \coloneqq \nicefrac{\normf{\bM}}{\sqrt{\Tr(\bQ)}}$ of the spectral estimator $\matwhat_{\tens{T}}$ (\ref{eq:def:spectral_symmetric}), for a selection of examples of link functions.
In all the following cases, the state evolution equations simplify, allowing to write the results as functionals of $\lambda:\R\to\R$, which is a specific to each problem:
\begin{itemize}
    \item $g(z\in\R)$ (single-index model): $\lambda(y) = \Var[z\big|y] - 1$;
    
    \item $g(\bz) = p^{-1}||\bz||^2$: $\lambda(y) = y - 1$;
    \item $g(\bz) = \operatorname{sign}(z_1z_2)$: $\lambda(y) = 2\pi^{-1}y$;
    \item $g(\bz) = \prod_{k=1}^pz_k$: $\lambda(y)$ defined in eq. (\ref{eq:lambda_prod_zk}).
\end{itemize}
For all these example, the value $\alpha_c$ can be found in Section \ref{sec:examples_asymmetric}.
For $\alpha>\alpha_c$, consider $a$ and $\gamma$ solutions of
\begin{align}\label{eq:examples_symmetric_a}
&\E_{\rdm{y}\sim\Zout}\left[\frac{\lambda(\rdm{y})^2}{a(1 + \lambda(\rdm{y})) - \lambda(\rdm{y}) }\right] = \frac{1}{\alpha}\\
\label{eq:examples_symmetric_gamma}
&\gamma = 1 + \alpha\E_{\rdm{y}\sim\Zout}\left[\frac{\lambda(\rdm{y})}{a(1 + \lambda(\rdm{y})) - \lambda(\rdm{y}) }\right].
\end{align}
Then, for any $\alpha$, the overlap $m \coloneqq \nicefrac{\normf{\bM}}{\sqrt{\Tr(\bQ)}} $ is given by
\begin{equation}\label{eq:examples_symmetric_general}
    m^2 = \Bigg(\frac{1 - \alpha\E_{\rdm{y}\sim\Zout}[\lambda^2(\rdm{y})\left(a(1 + \lambda(\rdm{y})) - \lambda(\rdm{y})\right)^{-2}]}{1 + \alpha\E_{\rdm{y}\sim\Zout}[\lambda^3(\rdm{y})\left(a(1+ \lambda(\rdm{y})) - \lambda(\rdm{y})\right)^{-2}] } \Bigg)_+,
\end{equation}
which is strictly positive $\forall \alpha > \alpha_c$. Additional details on the derivation of this result can be found in Appendix \ref{app:details_examples_symmetric}.


Reasoning about situations often involves weighing multiple pieces of information to draw inferences. 
%
Consider the last row in Table~\ref{table:atom-examples}. 
%
A human determining that $H$ contradicts $P$ will attribute the contradiction to the fact that $P$ mentions a father and daughter, but $H$ mentions two men.
%
Implicitly, they will have also weighed the fact that $H$'s mention of ``cutting grass'' is entailed by $P$'s mention of a lawnmower, and so it does not contribute to the contradiction.
%
We treat these two determinations as distinct \textit{atomic sub-problems}.


Hypotheses in \snli, and in turn, \dsnli, can be complex sentences, and while solving inference problems, models must weigh all pieces of information in both $P$ and $H$.
% 
We expect humans to make inferences about constituent pieces of information in a manner that is consistent with their overall judgment, an equally desirable property in models.
%
Not only does it signal holistic understanding of the situation described in the problem, but it can help pinpoint exactly what types of inferences models struggle with.
%
Identifying atomic sub-problems also allows us to understand the granular inferences that are evaluated in benchmark datasets.
%
In turn, this helps to understand the \textit{diversity} in the dataset: despite there being thousands of examples, certain inferences may come up repeatedly.

%
To identify the constituent sub-problems, we break \textit{hypotheses} in \snli~and \dsnli~into atomic propositions~\cite{wanner2024closer} to use in subsequent analyses. 
%
Each atomic decomposition represents a single piece of information.
%
Formally, given an \snli~example with $P$ and $H$, we generate atomic decompositions of $H$ represented by $a_1...a_n$. 
%
Each atomic sub-problem then involves predicting the relation between a ($P$, $a_i$) tuple (\S\ref{sec:snli-atoms:rules}).
%
Given a \dnli~example with $P$, $H$, and $U$, atomic sub-problems involve a ($P$, $a_i$, $U$) tuple where the task is to determine whether $U$ strengthens, weakens, or has no effect on $a_i$ (\S\ref{sec:dnli-atoms}).


\subsection{Generating Atomic Propositions}
\label{subsec:generate-atoms}
To generate atomic propositions, we draw on Neo-Davidsonian event-based semantic representations of sentences~\cite{castaneda1967, parsons1990events}.
%
Sentences can be represented in first-order logical form as conjunctions of predicates representing entities, where actions are explicitly represented with event variables and predicate arguments are mapped to semantic roles~\cite{dowty1991thematic}. 
%
For example, the sentence \textit{``The juggler performs at a party''} could be represented as: 
\[
\small
\begin{aligned}
    &\exists x_1 \exists e \ (\text{Juggler}(x_1) \land \text{Perform}(e) \land \text{Agent}(e, x_1) \land \\
    &\quad \exists x_2 \ (\text{Party}(x_2) \land \text{At}(e, x_2)))
\end{aligned}
\]

\noindent Each conjunct can then be mapped to a natural language expression, called an \textit{atom}.
%
This ensures that both arguments of actions \textit{and} the actions themselves are included as separate atoms.

We draw on this intuition to carefully hand-construct exemplars, a methodology shown to improve the atomicity and groundedness of decompositions~\cite{wanner2024closer}.
%
We prompt \texttt{llama-3-8b-instruct} with these exemplars (Appendix~\ref{appendix:atom-generation}) to generate atoms for each example in the \dsnli~test set (henceforth, \dsnlitest),  as well as for 1000 randomly sampled examples in the SNLI test set (\snlitest). See Table~\ref{tab:dataset-sizes} for dataset statistics.

\subsection{Validating Atomic Decompositions}
\label{subsec:validate-atoms}
Valid atomic decompositions of hypotheses must be logically entailed from the hypothesis they were decomposed from.
%
For our experiments on \snlitest~(\S\ref{sec:snli-atoms}), we do not validate atom entailment ourselves, letting each model determine whether $H$ entails each $a_i$ itself (\S\ref{sec:snli-atoms:rules}), and only measuring consistency on the atomic sub-problems that the model itself admits as ``valid''.
%

However, we \textit{do} validate all generated atoms in \dsnlitest, since non-monotonic reasoning does not give rise to clear constraints between an original problem and its constituent atomic sub-problems.
%
Our two-step validation process involves pruning decompositions with a strong, finetuned NLI model followed by human validation.

\paragraph{Pruning.} For each example in \dsnlitest, we use a \abr{DeBERTa}-large model finetuned on popular NLI datasets\footnote{MNLI~\cite{williams-etal-2018-broad}, Fever-NLI~\cite{thorne-etal-2018-fever}, Adversarial NLI~\cite{nie-etal-2020-adversarial}, LingNLI~\cite{parrish-etal-2021-putting-linguist}, and WANLI~\cite{liu-etal-2022-wanli}} and remove all generated atoms that are not entailed by the hypothesis.
%
By design, $P$-$H$ pairs in \dsnli~have a neutral relation, and updates strengthen or weaken propositions in the \textit{hypothesis}.
%
Hence, we run a secondary pruning stage to retain only those atoms that are \textit{not} entailed by the premise (see Table~\ref{tab:dataset-sizes}).
%
See Appendix~\ref{appendix:atom-generation} for a discussion of coverage.
%
\paragraph{Human Validation.} An author annotated all atoms that survived pruning as either \textit{invalid} or \textit{valid} (see Table~\ref{table:atom-examples} for examples of invalid atoms).
%
Valid \dsnli~atoms had to (1) be grammatical, (2) entail from $H$, (3) not entail from $P$.
%
Atoms introducing new information were considered invalid, including those that were \textit{pragmatic} inferences of $H$~\cite{jeretic-etal-2020-natural, srikanth-etal-2024-pregnant}.
%

95.7\% of pruned atoms were determined as valid by the author annotator.
%
An external annotator also annotated a sample of 100 atoms for validity for an agreement of $\kappa=0.82$ measured by Cohen's Kappa~\cite{cohen1960coefficient}.
%
The remaining analysis in this work is done on the set of \textit{valid} \dsnli~atoms.