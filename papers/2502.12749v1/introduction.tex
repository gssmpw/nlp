% !TEX root = ./main.tex
\section{Introduction}

A typical reconfiguration problem on a graph consists of two feasible 
solutions $S$, $T$, and a modification rule. 
The objective is to determine whether $S$ can be transformed 
{into} $T$ by repeated applications of the modification rule 
such that at any intermediate step, the set remains a 
feasible solution. 
It {is helpful to imagine} that a token is located {at} every vertex of $S$. 
{Therefore,} the modification operation corresponds to how 
the tokens can be moved. 
Three well-studied operations are 
\emph{token addition and removal}, \emph{token jumping}, and 
\emph{token sliding}. 
In the first operation, one can add or remove the token 
{during} the intermediate stages. 
In the second operation, a token can jump from one vertex to another vertex (usually to a vertex that doesn't have a token). 
In \emph{token sliding}, one can move tokens along edges of the graph, i.e., vertices can only be replaced with {adjacent} vertices.

The reconfiguration version of various classical problems like 
\textsc{Satisfiability}~\cite{DBLP:journals/siamcomp/GopalanKMP09}, 
\textsc{Dominating Set}, \textsc{Independent Set}, 
\textsc{Colouring's}~\cite{DBLP:journals/endm/BonamyB13,DBLP:journals/jgt/FeghaliJP16}, 
\textsc{Kempe Chains}~\cite{DBLP:journals/jct/BonamyBFJ19,DBLP:journals/ejc/Feghali0P17}, 
\textsc{Shortest Paths}~\cite{DBLP:journals/tcs/Bonsma13}, etc., 
{has received significant attention in recent years.} 
{See surveys~\cite{DBLP:journals/algorithms/Nishimura18,DBLP:books/cu/p/Heuvel13} for a comprehensive introduction to these topics.} 
Among these problems, \textsc{Independent Set Reconfiguration} is 
{one of} {the most well-studied} problems with respect to 
all the operations mentioned above, 
{namely} token addition and removal~\cite{DBLP:journals/tcs/KaminskiMM12,DBLP:journals/algorithmica/MouawadN0SS17}, 
token jumping~\cite{DBLP:conf/swat/BonsmaKW14,DBLP:conf/fct/BousquetMP17,DBLP:journals/tcs/ItoDHPSUU11,DBLP:conf/tamc/ItoKOSUY14,DBLP:conf/isaac/ItoKO14}, and 
token sliding~\cite{DBLP:conf/wg/BonamyB17,DBLP:journals/tcs/DemaineDFHIOOUY15,DBLP:conf/isaac/Fox-EpsteinHOU15,DBLP:journals/tcs/HearnD05,DBLP:conf/isaac/HoangU16,DBLP:journals/talg/LokshtanovM19}. 
Recall that for an integer $k$, a $k$-independent set of $G$ is a subset 
$S \subseteq V(G)$ of size $k$ of pairwise non-adjacent vertices. 
In this article, we consider the complexity of the reconfiguration of 
\textsc{Independent Set} problem under the token sliding rule. 
As we focus only on {\textsc{Independent Set},} we refer to it simply as \textsc{Token Sliding} and consider the following two problems.

\defproblem{\textsc{Token Sliding-Connectivity (TS-Connectivity)}}{Graph $G$ and integer $k$}{
Is the $k$-$TS$ reconfiguration graph $TS_k(G)$ of $G$ connected?

The vertices of $TS_k(G)$ are $k$-independent sets of $G$ and 
$(I, J)$ is an edge of $TS_k(G)$ if and only if 
$J \setminus I = \{u\}$, $I \setminus J = \{v\}$ and 
$(u, v)$ is an edge of $G$ for some $u, v\in V(G)$.}

\defproblem{\textsc{Token Sliding-Rechability (TS-Rechability)}}{Graph $G$ and
two $k$-independent set of $I, J$ of $G$.}{Are the vertices corresponding to $I$ and $J$ in the same connected components of $TS_k(G)$?
%Alternately, does there exist a sequence of independent sets that transform $I$ into $J$ so that every pair of consecutive independent sets 
%of the sequence can be obtained via sliding a token?
}

\noindent 
Alternately, in the second variation, the objective is to determine whether
we can transform one independent set to the other by 
token sliding, while the set remains independent.

%These reconfiguration problems turned out to be
%much harder than the corresponding graph problem.
Hearn and Demaine~\cite
{DBLP:journals/tcs/HearnD05} proved  
that \textsc{TS-Reachability}
is \PSPACE-complete even for planar graphs. 
On the positive side, Kaminski et al.~\cite{DBLP:journals/tcs/KaminskiMM12} and Bonsma et al.~\cite{DBLP:journals/mst/BelmonteKLMOS21} presented polynomial time algorithms 
when the input graph is restricted to cographs and claw-free graphs,
respectively.
Yamada and Uehara~\cite{DBLP:conf/walcom/YamadaU16} showed that any 
two $k$-independent sets
in proper interval graphs can be transformed to each other 
in polynomial steps.
Demaine et al.~\cite{DBLP:conf/isaac/DemaineDFHIOOUY14} described a linear algorithm when
the input graph is a tree.
In the same paper, the authors asked if determining whether 
\textsc{TS-Reachability} is polynomial time for 
interval graphs and then more generally for chordal graphs, 
a strict superclass of interval graphs. 
Belmonte et al.~\cite{DBLP:journals/mst/BelmonteKLMOS21} answered the second
part in the negative by proving that the problem is 
\PSPACE-complete on split graphs, a strict subset
of chordal graphs.
Bonamy and Bousquet~\cite{DBLP:conf/wg/BonamyB17} answered 
the first part 
positively by presenting a polynomial-time 
algorithm for \textsc{TS-Reachability} when input is an
interval graph.
They also proved the following result about  
\textsc{TS-Connectivity}.

\begin{proposition*}[Theorem~$2$ and~$3$ in \cite{DBLP:conf/wg/BonamyB17}]
\label{prop:token-sliding-known-results}
\emph{\textsc{Token Sliding-Connectivity}}
%Deciding whether $TS_k(G)$ if connected
%\begin{itemize}
%\item 
is in \emph{\P} for interval graphs, but
%\item 
is \emph{\co-\NP-hard}
%and co-W[2]-hard parameterized by the size $k$ 
for split graphs.
%\end{itemize}
\end{proposition*}

%Before proceeding, we introduce some notations.
It is known that for each chordal graph there is a tree, 
called \emph{clique-tree}, whose
vertices are the cliques of the graph and for every vertex 
in the graph, the set of cliques that contain it form a subtree of clique-tree.
See, for example, the note by Blair and Peyton~\cite{10.1007/978-1-4613-8369-7_1}.
The maximal cliques in an interval graph can be ordered such 
that for every vertex, the maximal cliques containing that vertex occur consecutively~\cite{DBLP:journals/algorithmica/GalbyMSST24}.
Hence, for interval graphs, such the maximum clique-tree degree is two.
Whereas, for split graphs the maximum degree of a clique-tree 
can be unbounded, as demonstrated by a star graph.
This prompted Bonamy and Bousquet~\cite{DBLP:conf/wg/BonamyB17}
to explicitly ask the following question.

\smallskip
\noindent \textbf{Question:} \emph{Let $d$ be a fixed constant. 
For an integer $k$ and chordal graph $G$ of the maximum
clique-tree degree at most $d$, 
can the connectivity of $TS_k(G)$ be decided in polynomial time?}
\smallskip

The questions like this have been systematically addressed in the realm 
of \emph{parameterized complexity}.
This paradigm allows for a more refined analysis of the problem’s complexity.
In this setting, we associate each instance $I$ with a parameter $\ell$
which can be arbitrarily smaller than the input size.
The objective is to check whether the problem is `tractable' when 
the parameter is `small'.
A parameter may originate from the formulation of the problem itself (called \emph{natural parameters}) or it can be a property of the input graph (called \emph{structural parameters}).
Parameterized problems that admit an algorithm with running time
$f(\ell) \cdot |I|^{\OO(1)}$ for some computable function $f$, are called 
\emph{fixed parameter tractable} (\FPT).
On the other hand, parameterized problems that are hard for the complexity class \W[1] do not admit such fixed-parameter algorithms,
under standard complexity assumptions.
A problem is \para-\NP-hard if it remains \NP-hard even when the parameter 
$\ell$ is fixed. 
In other words, fixing the parameter does not simplify the problem to a
polynomial-time solution.
%, implying that the problem's complexity is inherently high regardless of the parameter. 

For the above problems, the natural parameter is the number of tokens $k$
whereas the maximum clique-tree degree 
is one of the structural parameters. 
It is known that \textsc{TS-Connectivity}~\cite{DBLP:conf/wg/BonamyB17}
and \textsc{TS-Reachability}~\cite{DBLP:conf/tamc/ItoKOSUY14} are \co-\W[2]-hard and
\W[1]-hard, respectively, when parameterized by the number of tokens $k$.
However, the latter problem admits an \FPT~algorithm for planar graphs~\cite{DBLP:conf/isaac/ItoKO14}.
In this article, we study the structural parameterization
of these problems when the input is a chordal graph.
The first parameter we consider is the maximum clique-tree
degree of the chordal graph and answer the question by 
Bonamy and Bousquet~\cite{DBLP:conf/wg/BonamyB17}
in the negative.

\begin{restatable}{theorem}{tsconnnphard}
\label{thm:np-hardness-clique-tree-degree-TS-Conn}
On chordal graphs of maximum clique-tree degree $4$,
\emph{\textsc{Token Sliding-Connectivity}} problem is 
\emph{\co-\NP-hard}.
\end{restatable} 

Bonamy and Bousquet~\cite{DBLP:conf/wg/BonamyB17} 
adapted their algorithm for \textsc{TS-Connectivity} on interval graphs
to solve \textsc{TS-Reachability} on interval graphs in polynomial time.
Also, Belmonte et al.~\cite{DBLP:journals/mst/BelmonteKLMOS21} 
proved that \textsc{TS-Reachability} is \PSPACE-complete on split graphs.
Hence, it is natural to ask the analogous question regarding
\textsc{TS-Reachability}: 
\emph{Does \emph{\textsc{TS-Reachability}} admit a 
polynomial-time algorithm when the input is a chordal graph 
of the maximum clique-tree degree at most $d$?}
We answer even this question in the negative.
%We prove that small but critical modifications in the reduction used
%in the above theorem implies the following result.

\begin{restatable}{theorem}{tsreachnphard}
\label{thm:np-hardness-clique-tree-degree-TS-Reach}
On chordal graphs of maximum clique-tree degree $3$,
\emph{\textsc{Token Sliding-Reachability}} problem is 
\emph{\NP-\hard}.
\end{restatable}

Next, we move to our second structural parameter \emph{leafage}.
The {leafage} of chordal graph $G$, denoted by $\ell$, is defined as 
the minimum number of leaves in the tree of a tree representation of $G$.
Habib and Stacho~\cite{DBLP:conf/esa/HabibS09} showed that we can 
compute the leafage of a connected chordal graph in polynomial time. 
In recent years, researchers have studied the structural parameterization of various graph problems on chordal graphs parameterized by the leafage.
Fomin et al.~\cite{DBLP:journals/algorithmica/FominGR20} and Arvind et al.~\cite{DBLP:journals/corr/ArvindNPZ22} proved, respectively, that the \textsc{Dominating Set} and \textsc{Graph Isomorphism} problems on chordal graphs are \FPT\ parameterized by the leafage.
Galby et al.~\cite{DBLP:journals/algorithmica/GalbyMSST24} presented 
an improved \FPT\ algorithm for {\sc Dominating Set} parameterized by the 
leafage, and adapted it to obtain 
similar \FPT\ algorithms for {\sc Connected Dominating Set} and 
{\sc Steiner Tree} on chordal graphs.
They also proved that
{\sc MultiCut with Undeletable Terminals} on chordal graphs
is {\W[1]-hard} when parameterized by the leafage $\ell$.
Barnetson et al.~\cite{DBLP:journals/networks/BarnetsonBEHPR21} and Papadopoulos and Tzimas \cite{DBLP:conf/iwoca/PapadopoulosT22} presented \XP-algorithms running in time $n^{\calO(\ell)}$ for \textsc{Fire Break} and \textsc{Subset Feedback Vertex Set} 
on chordal graphs, respectively.
Papadopoulos and Tzimas \cite{DBLP:conf/iwoca/PapadopoulosT22} also 
proved that the latter problem is \W[1]-\hard\ when 
parameterized by the leafage.
Hochst{\"{a}}ttler et al~\cite{DBLP:journals/dam/HochstattlerHMP21} showed that we can compute the neighborhood polynomial of a chordal graph in $n^{\calO(\ell)}$-time.

For a chordal graph, leafage is a larger parameter than
the maximum clique-tree degree.
However, our next results imply that this parameter is not large enough
to obtain \FPT\ algorithms for either of the problems. 

\begin{restatable}{theorem}{thm:W-hardness-leafage-TS-Conn}
\label{thm:W-hardness-leafage-TS-Conn}
On chordal graphs, the 
\emph{\textsc{Token Sliding-Connectivity}} problem is 
\emph{co-\W[1]-\hard} when parameterized by the leafage $\ell$
of the input graph.
\end{restatable}

\begin{restatable}{theorem}{thm:W-hardness-leafage-TS-Reach}
\label{thm:W-hardness-leafage-TS-Reach}
On chordal graphs, the 
\emph{\textsc{Token Sliding-Reachability}} problem is 
\emph{\W[1]-\hard} when parameterized by the leafage $\ell$
of the input graph.
\end{restatable}

%\subparagraph*{Organization}
\noindent \textbf{Organization}
We use the standard notations that are specified in 
\cref{prelims}.
In Section~\ref{sec:max-clique-tree-degree} and 
Section~\ref{sec:leafage-hardness}, we present the hardness
results with respect to parameter maximum clique tree degree
and leafage, respectively.
We conclude the paper in Section~\ref{sec:conclusion}.




