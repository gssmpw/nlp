% !TEX root = ./main.tex
\section{Conclusion}
\label{sec:conclusion}
In this article, we studied the \textsc{Token Sliding Connectivity} 
and \textsc{Token Sliding Reachability} problems when the input 
is restricted to a chordal graph. 
We prove that both these problems are \para-\NP-hard 
when parameterized by the maximum clique-degree of the chordal graph. 
This answers the open question posed by 
Bonamy and Bousquet~\cite{DBLP:conf/wg/BonamyB17}. 
We then consider the complexity of the problems when 
parameterized by the larger parameter {`leafage'} and 
prove that \textsc{Token Sliding Connectivity} and 
\textsc{Token Sliding Reachability} are \co-\W[1]-hard and \W[1]-hard, 
respectively. 
We conjecture that both these problems admit an \XP-algorithm, i.e., an 
algorithm with running time $n^{f(\ell)}$, when parameterized by the leafage. 
It would also be interesting to {investigate} whether well-partitioned chordal 
graphs, introduced in \cite{DBLP:journals/dm/AhnJKL22}, can be used to 
narrow down the complexity gaps for these problems. 
Formally, do these problems admit \FPT\ algorithms when 
parameterized by the leafage when the input graph is restricted 
to a well-partitioned chordal graph?


