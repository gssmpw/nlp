%!TEX root = ./main.tex
\section{Preliminaries}
\label{prelims}

For a positive integer $q$, we denote the set $\{1, 2, \dots, q\}$ by $[q]$,
and for any $0 \leq p \leq q$, we denote the set $\{p,\dots,q\}$ by $[p,q]$.


\subparagraph*{Graph Theory.}
For a graph $G$, we denote by $V(G)$ and $E(G)$
the set of vertices and edges of $G$, respectively.
Unless specified otherwise, we use $n$ to denote the number of vertices in $G$.
{We denote the edge with endpoints $u, v$ by $uv$}.
For any $v \in V(G)$,
$N_G(v) = \{u \mid uv \in E(G)\}$ denotes the (open) neighbourhood of $v$,
and $N_G[v]=N_G (v) \cup \{v\}$ denotes the closed neighbourhood of $v$.
When the graph $G$ is clear from the context, we omit the subscript $G$.
For any $S \subseteq V(G)$, $G - S$ denotes the graph obtained from $G$ by deleting vertices in $S$.
We denote the subgraph of $G$ induced by $S$, i.e., the graph $G - (V(G) \setminus S)$, by $G[S]$.
We say graph $G$ contains graph $H$ as in \emph{induced subgraph}
if $H$ can be obtained from $G$ by a series of vertex-deletions. A {\em tree} $T$ is a connected acyclic graph. The sets $V_{\geq 3}(T)$ and $V_{=1}(T)$
denote the set of vertices of degree at least $3$,
and of degree equal to $1$, respectively.
The set $V_{\geq 3}(T)$ is also called the set of {\em branching vertices} of $T$
and the set $V_{=1}(T)$ is called the set of {\em leaves} of $T$.
Note that $|V_{\geq 3}(T)| \le |V_{=1}(T)|-1$.
Any node of $T$ which is not a leaf is called \emph{internal}.
For any further notation from basic graph theory, we refer the reader to~\cite{DBLP:books/daglib/Diestel12}.



\subparagraph*{Chordal graphs and Tree representations.}
A graph is called a chordal graph if it contains no induced cycle of length at least four.
It is well-known that chordal graphs can be represented as intersection graphs of subtrees in a tree, that is, 
for every chordal graph $G$, there exists a tree $T$ and a collection $\calM$ of subtrees of $T$ 
in one-to-one correspondence with $V(G)$ 
such that two vertices in $G$ are adjacent if and only if their corresponding subtrees intersect.
%The pair $(T, \calM)$ is called a \emph{tree representation} of $G$.
%For every $v \in V(G)$, we denote by $\mld(v)$ the subtree corresponding to $v$ and
%refer to $\mld(v)$ as the {\em model} of $v$ in $T$.
%Throughout this article, we use \emph{nodes} to refer to the vertices of the tree $T$
%to avoid confusion with the vertices of the graph $G$.
%Furthermore, we use the Greek alphabet to denote nodes of $T$ and the Latin alphabet to denote vertices of $G$.
%For notational convenience, for any node $\alpha \in V(T)$ and edge $e \in E(T)$, we may abuse notation and write $\alpha \in \mld(v)$ in place of $\alpha \in V(\mld(v))$ as well as $e \in \mld(v)$ in place of $e \in E(\mld(v))$.

%Given a subtree $T'$ of $T$, we denote by $G_{|T'}$ the subgraph of $G$ 
%induced by those vertices $x \in V(G)$ such that $V(\mld(x)) \subseteq V(T')$.
%If $T$ is rooted, then for each vertex $v \in V(G)$, we call the node in $\mld(v)$ that is closest to the root of $T$, 
%the {\em topmost} node of $\mld(v)$ and denote it by $\topnode_{\mld}(v)$. 
 
The \emph{leafage} of chordal graph $G$, denoted by $\ell$, is defined as 
the minimum number of leaves in the tree of a tree representation of $G$.
A tree representation $(T,\calM)$ for $G$ such that the number of leaves in $T$ is $\ell$, can be computed in time $O(|V(G)|^3)$ \cite{DBLP:conf/esa/HabibS09}. 
%Furthermore, the number of nodes in $T$ is at most $O(|V(G)|)$. 

Let $G$ be a connected graph, and consider $\mathcal{K}_G$ to be the set of its maximal cliques. A tree on $\mathcal{K}_G$ is said to satisfy the \emph{clique-intersection} property if, for every pair of distinct cliques $K, K' \in \mathcal{K}_G$, the set $K\cap K'$ is contained in every clique on the path connecting $K$ and $K'$ in the tree. The tree $T^c$ for a chordal graph $G$ whose vertices are the cliques in $\mathcal{K}_G$ and which satisfies the clique-intersection property is called the \emph{clique-tree} of $G$. It is shown in \cite{10.1007/978-1-4613-8369-7_1} that the clique-tree $T_c$ for a chordal graph $G$ is isomorphic to its intersection tree $T$. Since in any tree the number of leaves is always greater than or equal to the maximum degree of a vertex, the leafage of a chordal graph $G$ is at least the 
maximum degree of its clique-tree $T^c$.
 
% $T$ such that $G$ is a $T$-graph.
%An intersection graph of a set family is a graph obtained by constructing a vertex for each set of the family and putting an edge between two vertices if and only if the corresponding sets intersect.
%In this article, we focus on the 
%A tree $T$ is said to {\em represent} a chordal graph $G$ if $G$ is the intersection graph of sub-trees of $T$.
%A pair $(T,\mld)$, where $T$ is a tree and $\mld$ is a map from the vertex set of $G$ to sub-trees of $T$, is called a {\em tree representation} of $G$. 
%For any $v \in V(G)$, we refer to $\mld(v)$ as its {\em model} in $T$.
%
%
%
%
%
%
%We use greek characters to denote nodes in $T$
%and latin characters to denote the vertices of the graph.
%
%For a given tree $T$, let $\branchN(T)$ denote the set of branching nodes.
%Let $\branchE(T)$ denote the pairs of branching nodes in $T$,
%i.e.\ pairs between nodes in $\branchN(T)$,
%that are connected via a path that does not contain another branching node.

%\ppp{Do we need the following notation somewhere?}
%For a pair $e = (\alpha,\beta)$ in $\branchE(T)$
%we say that graph $G^{\star}$ is obtained by \emph{contracting $e$ in $T$}
%if $G^{\star}$ is the $(T/e)$-graph
%with the model obtained as follows:
%\begin{itemize}
%  \item
%  contract the path from $\alpha$ to $\beta$ in $T$
%  % and, respectively, the $t_1$ to $t_2$ path $P$ in $T'$
%  and denote the node obtained from $\alpha$ and $\beta$ by $\gamma_{\alpha\beta}$,
%  \item
%  delete all $\alpha\beta$-vertices of $G$, and
%  \item
%  for each remaining vertex $v \in V(G)$,
%  delete from $\mld(v)$ the subdivision nodes of $P$
%  and replace $\alpha$ and $\beta$ by $\gamma_{\alpha\beta}$
%  if at least one of these nodes is in $\mld(v)$.
%\end{itemize}
%Note that $V(G^{\star}) \subseteq V(G)$ and two vertices in $V(G^{\star})$
%that are not adjacent in $G$ could be adjacent in $G^{\star}$.

%\subparagraph*{$H_{\ell}$-induced-subgraph-free chordal graphs.}

\subparagraph*{Parameterized Complexity.}
The input of a parameterized problem comprises an instance $I$,
which is an input of the classical instance of the problem,
and an integer $k$, which is called the parameter.
A parameterized problem $\Pi$ is said to be \emph{fixed-parameter tractable} (\FPT\ for short) %
if, given an instance $(I,k)$ of $\Pi$,
we can decide whether $(I,k)$ is a \yes-instance of $\Pi$
in time $f(k)\cdot |I|^{\OO(1)}$
for some computable function $f$ depending only on $k$.
A parameterized problem $\Pi$ is said to be \para\NPH\ %
if it is \NPH even for a constant value of $k$. 
For the purposes of this paper, we define a problem to be 
\WoneH if it is at least as hard as 
\textsc{MultiColored Independent Set (MultiCol Ind-Set)}
or \textsc{MultiColored Clique (MultiCol Clique)}.
In the first problem, an input is a graph $G$ with a 
partition $\langle V_1,V_2,\ldots,V_k\rangle$ of $V(G)$ such that each $V_i$ 
is an independent set and contains exactly $n$ vertices, and integer $k$.
The question is to determine whether $G$ contain 
an independent set of size $k$ that contains exactly one vertex from 
each partition.
The second problem is defined in a similar way.
If a problem is \WoneH, then it is unlikely to have an \FPT algorithm.
For more details on parameterized algorithms, and in particular parameterized branching algorithms,
we refer the reader to the book by Cygan et al.~\cite{DBLP:books/sp/CyganFKLMPPS15}.


%\defproblem{\textsc{MultiColored Independent Set (MultiCol Ind-Set)}}{A graph $G$ with a 
%partition $\langle V_1,V_2,\ldots,V_k\rangle$ of $V(G)$ such that each $V_i$ 
%is an independent set and contains exactly $n$ vertices, and integer $k$.}{
%Does $G$ contain an independent set of size $k$ that contains exactly one vertex from each partition?}
%The \textsc{Multicolored Clique} problem is defined as follows:
%\defproblem{\textsc{MultiColored Clique (MultiCol Clique)}}{A graph $G$ with a partition $\langle V_1,V_2,\ldots,V_k\rangle$ of $V(G)$ such that each $V_i$ is an independent set and contains exactly $n$ vertices, integer $k$}{Does $G$ have a clique of size $k$ which contains exactly one vertex from each partition?}
