% !TEX root = ./main.tex
\usepackage{amssymb}
\usepackage{amsmath}
\usepackage{amsthm}
\usepackage{hyperref}
% \hypersetup{
%     colorlinks=true,
%     linkcolor=blue,
%     filecolor=magenta,
%     urlcolor=cyan,
% }
\usepackage{xspace}
\usepackage{mathtools}
\nolinenumbers
\usepackage[algoruled,boxed,lined]{algorithm2e}
\usepackage{complexity} % For symbols in complexity class.
\usepackage{todonotes} % For todo notes
\presetkeys{todonotes}{inline}{}
%\usepackage[shortlabels]{enumitem}
%\setlist{nosep}
\usepackage{comment}
\usepackage[capitalise,noabbrev]{cleveref}
\usepackage{tikz}
\usetikzlibrary{shapes.geometric}
\usetikzlibrary{positioning}

%\usepackage[numbers]{natbib}
%\bibliographystyle{plainnat}
%\usepackage{thm-restate}
%\usepackage{authblk}
%\usepackage{vmargin}
%\setmarginsrb{1in}{1in}{1in}{1in}{0pt}{0pt}{0pt}{6mm}

\newcommand{\parent}{\operatorname{\mathtt{parent}}}
\newcommand{\gray}{\textsf{gray}}
\newcommand{\white}{\textsf{white}}
\newcommand{\forb}{\textsf{forb}}
% Deprecated use \botnode_\mld instead:
\newcommand{\minmod}{\operatorname{\mathsf{min-mod}}}
% Deprecated use \topnode_\mld instead:
\newcommand{\maxmod}{\operatorname{\mathsf{max-mod}}}
\newcommand{\spn}{\textsf{span}}
\newcommand{\intvl}{\beta}
\newcommand{\deff}{\coloneqq} % Symbol for definition
\DeclareMathOperator{\cba}{\textsf{cba}}

\newcommand{\cWoneH}{\textup{\textsf{coW[1]}}-hard\xspace}

% Vertices whose model contains nodes from the tree rooted at #1
\newcommand{\interV}[1]{V_{#1}^{\cap}} % All vertices
\newcommand{\interR}[1]{R_{#1}^{\cap}} % Red vertices
\newcommand{\interB}[1]{B_{#1}^{\cap}} % Blue vertices
% Vertices whose model only contains nodes from the tree rooted at #1
\newcommand{\belowV}[1]{V_{#1}^{\subseteq}} % All vertices
\newcommand{\belowR}[1]{R_{#1}^{\subseteq}} % Red vertices
\newcommand{\belowB}[1]{B_{#1}^{\subseteq}} % Blue vertices
% Vertices whose model only contains nodes from the tree rooted at #1 but excluding #1
\newcommand{\underV}[1]{V_{#1}^{\subseteq\dag}} % All vertices
\newcommand{\underR}[1]{R_{#1}^{\subseteq\dag}} % Red vertices
\newcommand{\underB}[1]{B_{#1}^{\subseteq\dag}} % Blue vertices
% Vertices whose model contains the node #1
\newcommand{\containV}[1]{V_{#1}^{\in}} % All vertices
\newcommand{\containR}[1]{R_{#1}^{\in}} % Red vertices
\newcommand{\containB}[1]{B_{#1}^{\in}} % Blue vertices

\usepackage{color}
% Annotations
%\newcommand{\prelim}[1]{\textcolor{cyan}{#1}}
%\newcommand{\tocheck}[1]{\textcolor{cyan}{#1}}
% \newcommand{\todo}[1]{\textcolor{red}{\textbf{#1}}}
%\newcommand{\blue}[1]{{\color{blue}{#1}}}
%\newcommand{\red}[1]{{\color{red}[#1]}}
%\newcommand{\sn}[1]{\textcolor{violet}{#1 -{SN}}}
%\newcommand{\rs}[1]{\textcolor{magenta}{#1 -{RS}}}
%\newcommand{\pt}[1]{\textcolor{purple}{#1 -PT}}
%\newcommand{\ra}[1]{\textcolor{magenta}{#1 -RA}}




\DeclarePairedDelimiter\abs{\lvert}{\rvert}

\newcommand{\at}{\texttt{at}}
\newcommand{\lf}{\texttt{lf}}

\newcommand{\N}{\mathbb{N}}

\newcommand{\calA}{\mathcal{A}}
\newcommand{\B}{\ensuremath{\mathcal B}}
\newcommand{\bc}{\texttt{bc}}
\newcommand{\calB}{\mathcal{B}}
\newcommand{\calC}{\mathcal{C}}
\newcommand{\ClC}{\textsc{ClC}}
\newcommand{\calF}{\mathcal{F}}
\newcommand{\F}{\ensuremath{\mathbb F}}
\newcommand{\calG}{\mathcal{G}}
\newcommand{\calH}{{\mathcal H}}
\newcommand{\calI}{\mathcal I}
\newcommand{\calK}{\mathcal{K}}
\renewcommand{\L}{\ensuremath{\mathcal{L}}}
\newcommand{\calM}{\ensuremath{{\mathcal M}}}
\newcommand{\calO}{\ensuremath{{\mathcal O}}}
\newcommand{\OO}{\mathcal{O}}
\newcommand{\Oh}{\mathcal{O}} % Symbol for big-O notation.
\newcommand{\OPT}{\textsc{OPT}}
\newcommand{\calP}{\mathcal{P}}
\newcommand{\Q}{\ensuremath{\mathcal{Q}}}
\newcommand{\sStar}{\mathcal{S}^{\star}}
\newcommand{\calR}{\mathcal{R}}
\newcommand{\rank}{\texttt{rank}}
\newcommand{\calS}{\mathcal{S}}
\newcommand{\T}{\ensuremath{\mathbb T}}
\newcommand{\tw}{\texttt{tw}}
\newcommand{\calT}{\mathcal{T}}
\newcommand{\bbT}{\mathbb{T}_{\ell}}
\newcommand{\calU}{\mathcal{U}}
\newcommand{\calV}{\mathcal{V}}
\newcommand{\vc}{\texttt{vc}}
\newcommand{\calW}{\mathcal{W}}
\newcommand{\WC}{W^c}
\newcommand{\calX}{\mathcal{X}}
\newcommand{\calZ}{\mathcal{Z}}

\newcommand{\true}{\texttt{True}}
\newcommand{\false}{\texttt{False}}

\newcommand{\para}{\textsf{para}}
\newcommand{\POLY}{\textsf{poly}}
\newcommand{\Complete}{\textsf{Complete}}
\newcommand{\NPH}{\NP-hard\xspace}
\newcommand{\WoneH}{\textup{\textsf{W[1]}}-hard\xspace}
\newcommand{\ETH}{\textsf{ETH}\xspace}
\newcommand{\Woc}{\textsf{W[1]}-Complete}
\newcommand{\CONP}{\textsf{coNP}}
\newcommand{\CONPpoly}{\textsf{coNP/poly}}
\newcommand{\CONPH}{\textsf{coNP}-hard}
\newcommand{\CONPC}{\textsf{coNP}-complete}
\newcommand{\NPComplete}{\NP-complete}
\newcommand{\hard}{{hard}}
\newcommand{\NPpoly}{\textsf{NP/poly}}
\newcommand{\coNPpoly}{\textsf{coNP/poly}}
\newcommand{\paraNP}{\textsf{paraNP}}

\newcommand{\yes}{\textsc{Yes}}
\newcommand{\no}{\textsc{No}}
\newcommand{\YES}{\textsc{Yes}}
\newcommand{\NO}{\textsc{No}}

\newcommand{\bd}{\textsf{bd}}
\newcommand{\upbd}{\textsf{up-bd}}
\newcommand{\lobd}{\textsf{lo-bd}}
\newcommand{\combd}{\textsf{com-bd}}
\newcommand{\pvt}{\textsf{pvt}}

%\newtheorem{theorem}{Theorem}[section]
%\newtheorem{lemma}[theorem]{Lemma}
%\newtheorem{corollary}[theorem]{Corollary}
%\newtheorem{claim}[theorem]{Claim}
%\newtheorem{observation}[theorem]{Observation}
%\newtheorem{remark}[theorem]{Remark}
\newtheorem*{proposition*}{Proposition}
\newtheorem{reduction rule}[theorem]{Reduction Rule}
\Crefname{reduction rule}{Reduction~Rule}{Reduction~Rules}
\newtheorem{greedy select}[theorem]{Greedy~Select}\Crefname{greedy select}{Greedy~Select}{Greedy~Select}

\newtheorem{marking-scheme}[theorem]{Marking Scheme}
%\newtheorem{definition}[theorem]{Definition}
%\newenvironment{claimproof}[1][\proofname]{ \begin{proof}[#1] \renewcommand{\qedsymbol}{$\diamond$} }{ \end{proof} }



% Problem Definitions
\newcommand{\GroupST}{\textsc{Group Steiner Tree}\xspace}
\newcommand{\BinPack}{\textsc{Bin Packing}\xspace}
\newcommand{\DifBinPack}{\textsc{Differing Bin Packing}\xspace}
\newcommand{\DS}{\textsc{DomSet}\xspace}% Dominating Set problem
\newcommand{\connDS}{\textsc{connDomSet}\xspace}% connected Dominating Set
\newcommand{\RBDS}{\textsc{Red-Blue-DomSet}\xspace}
\newcommand{\RestRBDS}{\textsc{Rest-Red-Blue-DomSet}\xspace}
\newcommand{\conRBDS}{\textsc{Connected Red-Blue-DomSet}\xspace}
\newcommand{\SetCov}{\textsc{Set~Cover}\xspace}
\newcommand{\HitSet}{\textsc{Hitting~Set}\xspace}
\newcommand{\SteinTree}{\textsc{Steiner~Tree}\xspace}
\newcommand{\MC}{\textsc{Multicut}\xspace}
\newcommand{\MCundel}{\textsc{MultiCut with UnDel Term}\xspace}


% Definitions specific for Group Steiner Tree
\newcommand{\vers}[3]{\ensuremath{\langle #1, #2, #3 \rangle}}


% ################ NEW ENVIROMENT STARTS ##########################
\newcommand{\defparproblem}[4]{
  \vspace{1mm}
\noindent\fbox{
  \begin{minipage}{0.96\textwidth}
  \begin{tabular*}{\textwidth}{@{\extracolsep{\fill}}lr} #1  & {\bf{Parameter:}} #3
\\ \end{tabular*}
  {\bf{Input:}} #2  \\
  {\bf{Question:}} #4
  \end{minipage}
  }
  \vspace{1mm}
}

\newcommand{\defparproblemoutput}[4]{
  \vspace{1mm}
\noindent\fbox{
  \begin{minipage}{0.96\textwidth}
  \begin{tabular*}{\textwidth}{@{\extracolsep{\fill}}lr} #1  & {\bf{Parameter:}} #3
\\ \end{tabular*}
  {\bf{Input:}} #2  \\
  {\bf{Output:}} #4
  \end{minipage}
  }
  \vspace{1mm}
}


\newcommand{\defproblem}[3]{
  \vspace{1mm}
\noindent\fbox{
  \begin{minipage}{0.96\textwidth}
  \begin{tabular*}{\textwidth}{@{\extracolsep{\fill}}lr} #1 \\ \end{tabular*}
  {\bf{Input:}} #2  \\
  {\bf{Question:}} #3
  \end{minipage}
  }
  \vspace{1mm}
}

\newcommand{\defpromiseproblem}[4]{
  \vspace{1mm}
\noindent\fbox{
  \begin{minipage}{0.96\textwidth}
  \begin{tabular*}{\textwidth}{@{\extracolsep{\fill}}lr} #1 \\ \end{tabular*}
  {\bf{Input:}} #2  \\
  {\bf{Promise:}} #3\\
  {\bf{Question:}} #4
  \end{minipage}
  }
  \vspace{1mm}
}

\newcommand{\defproblemout}[3]{
  \vspace{1mm}
\noindent\fbox{
  \begin{minipage}{0.96\textwidth}
  \begin{tabular*}{\textwidth}{@{\extracolsep{\fill}}lr} #1 \\ \end{tabular*}
  {\bf{Input:}} #2  \\
  {\bf{Output:}} #3
  \end{minipage}
  }
  \vspace{1mm}
}

\newcommand{\parnamedefn}[4]{
\begin{tabbing}
\name{#1} \\
\emph{Input:} \hspace{1cm} \= \parbox[t]{10cm}{#2} \\
\emph{Parameter:}            \> \parbox[t]{10cm}{#3} \\
\emph{Question:}             \> \parbox[t]{10cm}{#4} \\
% \emph{Input:} \hspace{1.2cm} \= \parbox[t]{9.5cm}{#2} \\
% \emph{Parameter:}            \> \parbox[t]{9.5cm}{#3} \\
% \emph{Question:}             \> \parbox[t]{9.5cm}{#4} \\
\end{tabbing}
\vspace{-0.5cm}
}

\newcommand{\decnamedefn}[3]{
  \begin{tabbing} #1\\
    \emph{Input:} \hspace{1.2cm} \= \parbox[t]{12cm}{#2} \\
    \emph{Question:}             \> \parbox[t]{12cm}{#3} \\
  \end{tabbing}
}
% ################ NEW ENVIROMENT ENDS ##########################


\newcommand{\defproblemques}[3]{
  \vspace{1mm}
\noindent\fbox{
  \begin{minipage}{0.96\textwidth}
  \begin{tabular*}{\textwidth}{@{\extracolsep{\fill}}lr} #1 \\ \end{tabular*}
  {\bf{Input:}} #2  \\
  {\bf{Question:}} #3
  \end{minipage}
  }
  \vspace{1mm}
}



\newcommand{\wt}{\texttt{wt}}

%%%%%%%%%%%%%%%%%%%%MWC%%%%%%%%%%%%%%%%%%%%%%%%%%%%%

\newcommand{\mwcfull}{\textup{\textsc{Multiway Cut with Undeletable Terminals}}}
\newcommand{\mwc}{\textsc{MWC}}
\newcommand{\mwcset}{multiway-cut}
\newcommand{\stcut}{\textsc{$(s,t)$-Cut}}
\newcommand{\first}{\texttt{first}}
\newcommand{\leaves}{\texttt{leaves}}
\newcommand{\w}{\texttt{w}}
\newcommand{\connection}{connection}
\newcommand{\ver}{\textnormal{\texttt{ver}}}
\newcommand{\mld}{\mathcal{M}}
\newcommand{\rr}{\textsf{r}}
\newcommand{\node}{\gamma}
\newcommand{\topnode}{\mathtt{top}}
\newcommand{\botnode}{\mathtt{bot}}
\newcommand{\sink}{sink}
\newcommand{\Pterm}{P}
\newcommand{\destroys}{destroys}
\newcommand{\destroy}{destroy}
\newcommand{\truncated}{truncated}


%%%% Dominating Set
\newcommand{\branchN}{V^-} % Set of branching nodes in the input
\newcommand{\branchE}{E^-} % Set of edges between branching nodes in the input
\newcommand{\spnV}{\operatorname{\mathsf{span}}^V} % Function on \branchV
\newcommand{\spnE}{\operatorname{\mathsf{span}}^E} % Function on \branchE

%%%% MultiCut
\newcommand{\Pairs}{\Pterm}