% !TEX root = ./main.tex

\label{sec:max-clique-tree-degree}

\subsection{Hardness for \textsc{Token Sliding-Connectivity}}

In this subsection, we prove Theorem~\ref{thm:np-hardness-clique-tree-degree-TS-Conn}.
% which we restate for reader's convenience.
%\tsconnnphard*
%\begin{proof}
We present a polynomial time reduction from the
\textsc{Dominating Set on Non-Blocking Graphs}
which is \NPH~\cite{DBLP:conf/wg/BonamyB17}.
%to the \textsc{TS-Connectivity} problem 
%on chordal graphs with a maximum clique-tree degree of $4$. 
For a graph $G$ and a subset $S$ of $V(G)$, a vertex $x\in V(G)$ is said to 
be a \emph{private neighbour} of $s\in S$ if $x\in N(s)$ and 
$x\notin N[t]$ for any $t\in S\setminus \{s\}$. 
A set of vertices $S\subset V(G)$ is a blocking set if no vertex in $S$ 
has a private neighbour with respect to $S$. 
%A graph $G$ is $k$-blocking if it has a blocking set of size at most $k$.
In the \textsc{Dominating Set on Non-Blocking Graphs} problem,
an input is a graph $G$ and integer $k$ such that $G$ has no blocking set of size at most $2k-1$.
The objective is to determine whether $G$ contains a dominating set
of size at most $k$.

\noindent\textbf{Reduction:} 
Our reduction closely follows the one presented 
in~\cite{DBLP:conf/wg/BonamyB17}.
Let $(G,k)$ be an input instance of 
\textsc{Dominating Set on Non-Blocking Graphs}. 
The reduction constructs an instance 
$(G^\prime,k^\prime)$ of the \textsc{TS- Connectivity} problem.
Suppose $V(G) = \{v_1,v_2,\ldots,v_n\}$. 
\begin{itemize}
\item Add the vertices in the sets $C=\{c_1,c_2,\ldots,c_{n+k+1}\}$ 
and $W=\{w_1,w_2,\ldots,w_{n+k+2}\}$ in $V(G^\prime)$.
Add the edges $c_ic_j$ in $E(G^\prime)$ for $i\neq j\in [n+k+1]$
to ensure that $G^\prime[C]$ is a clique.
Set $W$ will remain an independent set in $G^{\prime}$.
\item For each $i,j\in [n]$, if $v_i$ and $v_j$ are adjacent in $G$ then add 
the edges $c_iw_j$ and $c_jw_i$ to $E(G^\prime)$.
\item Add the edge $c_iw_i$ in $E(G^\prime)$ for $i \in [n+k+1]$. 
\item Make vertex $w_{n+k+2}$ adjacent to all the vertices in $C$
that correspond to vertices in $V(G)$, i.e., 
add edges $c_iw_{n+k+2}$ in $E(G^\prime)$ for each $i\in [n]$.
\item Add sets of vertices $X=\{x_1,x_2,\ldots,x_{n+k+2}\}$ 
and $Y=\{y_1,y_2,\ldots,y_{n+k+2}\}$ in $V(G^\prime)$.
Also, add the edges $x_iy_i$ for $i\in[n+k+2]$ and edges 
$x_ic_j$ for each $i\in [n+k+2]$ and $j\in [n+k+1]$ in $G^\prime$.
That is, each vertex in $X$ is adjacent to all the vertices in $C$,
and $G^\prime[X \cup Y]$ consists of a matching as in 
\cref{TSC_deg_coNPH}. 
\end{itemize}
\begin{figure}[t]
    \centering
    \includegraphics[scale=0.35]{./images/TSC_deg_red_coNPH}

    \includegraphics[scale=0.35]{./images/clique_tree_1.png}
    \caption{The reduced instance for the \textsc{Token Sliding Connectivity} problem parameterized by the maximum clique-tree degree and its corresponding clique-tree.}
    \label{TSC_deg_coNPH}
\end{figure}

The reduction sets $k^\prime=k+1$. 
This completes the description of the reduction.  
Next, we prove its correctness and graph $G^{\prime}$ satisfies
the desired properties.

\begin{lemma}
$G^\prime$ is a chordal graph with maximum clique-tree degree at most $4$.
\end{lemma}
\begin{proof}
Note that $G^\prime[C\cup X\cup W]$ is a split graph as 
$C$ induces a clique and $X\cup W$ induces an independent set 
in $G^\prime$. 
The graph $G^\prime$ is formed by adding a pendant vertex 
to each vertex in $X$ in the graph $G^\prime[C\cup X\cup W]$;
hence the graph $G^\prime$ is a chordal graph. 

%Now we show that the clique tree decomposition of $G^\prime$ 
%has degree at most $4$. 
We now construct a clique-tree $\calT_c$ for $G^\prime$, 
and show that it has a maximum degree of $4$. 
Add a vertex $U_i$ in $\calT_c$ corresponding to the clique 
induced by the vertices $\{x_i,c_1,c_2,\ldots,c_{n+k+1}\}$ 
in $G^\prime$ for $i \in [n+k+2]$. 
Similarly, add a vertex $W_i$ in $\calT_c$ corresponding 
to the clique induced by the vertices in $N[w_i]$ in $G^\prime$ 
for $i\in [n+k+2]$. 
Add a vertex $Y_i$ corresponding to the clique (i.e., edge) induced 
by the vertices $\{x_i,y_i\}$ for $i\in[n+k+2]$. 
Add an edge from $U_i$ to $U_{i+1}$ for each $i\in [n+k+1]$. 
Add an edge from $U_i$ to $W_i$ for each $i\in [n+k+2]$.
Add an edge from $U_i$ to $Y_i$ for each $i\in [n+k+2]$. 
Thus, the maximum degree of $\calT_c$ is $4$. 
Each vertex in $V(\calT_c)$ corresponds to a maximal clique in $G^\prime$.
This tree is also shown in \cref{TSC_deg_coNPH}.

Next, we prove that this tree satisfies the clique-intersection property.
Each vertex in $Y\cup W$ appears in only one vertex (bag) of $\calT_c$.
Each vertex $x_i\in X$, for some $i \in [n+k+2]$, 
appears in the vertex (bag) $Y_i$ and all the vertices $U_j$ 
where $j\in [n+k+2]$. 
Since the vertices $U_j$ where $j\in [n+k+2]$ induce a path in 
$\calT_c$ and the vertex $Y_i$ is adjacent to the vertex $U_i$;
the set of vertices containing $x_i$ induces a connected sub-tree of 
$\calT_c$. 
Now any vertex $c_i\in C$ (for some $i\in [n+k+1]$) appears 
in all the bags $U_j$ where $j\in [n+k+2]$, and possibly some 
of the bags $W_\ell$ where $\ell\in [n+k+2]$.
This induces a path with some pendant vertices, 
which is a connected subtree of $\calT_c$. 
Thus, $G^\prime$ has a clique-tree with maximum degree $4$.
\end{proof}
\begin{lemma}
\label{lemma:TS-conn-max-deg}
$(G,k)$ is a \yes\ instance of the
\emph{\textsc{Dominating Set on Non-Blocking Graphs}} 
problem if and only if $(G^\prime,k^\prime)$ is a \no\ instance of the
\emph{\textsc{TS-Connectivity}} problem. 
\end{lemma}
\begin{proof}
Suppose $(G,k)$ is a \yes\ instance of the
\textsc{Dominating Set on Non-Blocking Graphs} problem. 
Without loss of generality, let $D=\{v_1,v_2,\ldots,v_k\}$
be a dominating set in $G$. 
We construct an independent set $I$ of size $k+1$ in $G^\prime$ 
such that no token on $I$ can be moved. 
Define $I :=\{w_1,w_2,\ldots,w_k,w_{n+k+2}\}$. 
Clearly, the token on $w_{n+k+2}$ cannot be moved because 
$N(w_{n+k+2})=\{c_i \mid i\in [n]\}$ and 
if we move this token to $c_j$ where $j\in [n]$, 
it will be adjacent to the token on $w_\ell$ for some 
$\ell \in [k]$ as $v_j\in N[v_\ell]$ for some $\ell\in [k]$ as $D$
is a dominating set in $G$. 
Now, if we move the token on $w_i$ for some $i\in [k]$ to any vertex 
in $N(w_i)\subset \{c_1,c_2,\ldots,c_n\}$, this will be adjacent to the 
token on $w_{n+k+2}$ because $N(w_{n+k+2})= \{c_1,c_2,\ldots,c_n\}$.
Thus $I$ will form an isolated vertex in the configuration graph 
$TS_{k+1}(G^\prime)$. 
As there are another independent sets of size $k+1$ in $G^\prime$,
such as $\{w_{n+1}, w_{n+2}, \dots, w_{n+k+1}\}$, 
$(G^{\prime}, k)$ is a \no\ instance of the \textsc{TS-Connectivity} problem. 

Suppose we have a \no\ instance of the
\textsc{Dominating Set on Non-Blocking Graphs} problem, 
then we show that $(G^\prime,k+1)$ is a \yes\ instance of the
\textsc{TS-Connectivity} problem. 
Define independent set $J:=\{w_{n+1},w_{n+2},\ldots,w_{n+k+1}\}$.
We show that any independent set $I$ of size $k+1$ in 
$G^\prime$ can be reconfigured to $J$ using a sequence 
of valid token sliding moves. 
It is sufficient to show that the independent set $I$ can be reconfigured 
into an independent set $I^\prime$ such that $|I^\prime\cap J|>|I\cap J|$. 
We will demonstrate this through a case analysis depending 
on how $I$ intersects the partitions $C, X, Y$, and $W$.

\emph{Case $(I)$: $I\cap C\neq \emptyset$:}
Let $\{c_i\}=|I\cap C|$ as $I$ is an independent set and 
hence it cannot contain two or more vertices from $C$. 
Now the other tokens in $I$ can only be in $Y\cup W$. 
If $i>n$, move the token on $c_i$ to $w_i$. 
This will give the desired independent set $I^\prime$. 
If $i\leq n$, bring the token on $I$ to $c_j$ where $j\in [n+1,n+k+1]$
and $w_j$ has no token. 
This is a valid move since $c_j$ has no neighbours in $Y\cup W$ 
other than $w_j$. 
Now move this token from $c_j$ to $w_j$ to get the desired 
independent set $I^\prime$.

\emph{Case $(II)$: $I\cap C=\emptyset$ but $I\cap X\neq \emptyset$:}
Let $x_i\in I\cap X$ for some $i\in [n+k+2]$. 
If there are more than one tokens in $I$ on the vertices in $X$, 
slide the tokens on vertices $x_j$ where $j\neq i, j\in [n+k+2]$ to $y_j$. 
There can be at most $k$ tokens in $W$ 
as there is at least one token in $X$.
Now bring the token on $x_i$ to $c_\ell$ where $\ell\in [n+1,n+k+1]$ and there is no token on $w_\ell$. This is a valid move since $c_\ell$ has no neighbours in $Y\cup W$ other than $w_\ell$. Now bring this token from $c_\ell$ to $w_\ell$ to get the desired independent set $I^\prime$.

\emph{Case $(III)$: $I\cap C=I\cap X=\emptyset$ but $I\cap Y\neq \emptyset$:} 
Suppose $y_i\in I \cap Y$. Bring the token on $y_i$ to $x_i$. This move is valid because $X\cup C$ has no token. Now proceed as in Case~$(II)$.

\emph{Case $(IV)$: $I\cap C=I\cap X=I\cap Y=\emptyset$ and $w_{n+k+2}\in I\cap W$:}
Consider the set $D_1=\{v_i| \,w_i~\text{has a token and}~ i\in [n]\}$.
The cardinality of this set is at most $k$.
The set $D_1\subset V(G)$ cannot be a dominating set as $G$ does not have a dominating set of size at most $k$. 
Therefore there exists $v_j$ for some $j\in [n]$ such that ${N[v_j]\cap D_1}=\emptyset$. 
The vertex $c_j$ is not be adjacent to any vertex in $W\cap I$ 
except $w_{n+k+2}$. 
Move the token on $w_{n+k+2}$ to $c_j$ and proceed as in Case~$(I)$.

\emph{Case $(V)$: $I\cap C=I\cap X=I\cap Y=\emptyset$ and $w_{n+k+2}\notin I\cap W$:} 
Consider the set $D_2=\{v_i| \,w_i~\text{has a token and}~i\in [n]\}$. 
It will have size at most $k+1$. 
The set $D_2 \subset V(G)$ cannot be a blocking set as 
$G$ does not have a blocking set of size at most $2k-1$. 
Therefore, there exists $v_j\in D_2$ for some $j\in [n]$ 
such that $v_j$ has a private neighbour $v_\ell$ for some $\ell\in[n]$. 
Move the token on $w_j$ to $c_\ell$. 
This will not be adjacent to any other token in $W$. 
Now proceed as in Case~$(I)$.

Hence, if $(G,k)$ is a \no\ instance of the \textsc{Dominating Set} problem, then $(G^\prime,k)$ is a \yes\ instance of the \textsc{TS-Connectivity} problem.
This completes the proof of the claim.
\end{proof}

These arguments, along with the fact that reduction 
takes polynomial-time, imply that \textsc{TS-Connectivity} is {\co-\NP-hard} on chordal graphs with a maximum clique-tree degree of $4$.
%\end{proof}



\subsection{Hardness for \textsc{Token Sliding Reachability}}
In this subsection, we present a reduction used to prove \Cref{thm:np-hardness-clique-tree-degree-TS-Reach}.
% which we restate for reader's convenience. 
%\tsreachnphard*
%\begin{proof}
We present a polynomial time reduction from
\textsc{TS-Reachability} on split graphs
which is known to be \NPH~\cite{DBLP:journals/mst/BelmonteKLMOS21}.


 \begin{figure}
    \centering
    \includegraphics[width=5cm]{./images/TSR_deg}
    \caption{The input Instance of \textsc{TS-Reachability}, $G$ is a split and $C$ is a clique graph.}
    \label{TSR_deg}
\end{figure}


\textbf{Reduction}
Let $(G,I,J)$ be the input instance of \textsc{TS-Reachability} 
problem where $G$ is a split graph with a clique $C=\{c_1,c_2,\ldots,c_p\}$ 
and an independent set $U=\{u_1,u_2,\ldots,u_q\}$ and
$I,J$ are two $k$-independent sets in $G$.
The reduction constructs an instance 
$(G^\prime,I^\prime,J^\prime)$ of the \textsc{TS-Reachability} problem
as follows.
%on chordal graphs of maximum clique-tree degree $3$. 
\begin{itemize}
\item For each vertex $c_i\in V(G)$ construct a vertex $d_i$, 
and for each vertex $u_j\in V(G)$ construct a vertex 
$w_j$ where $i\in[p]$ and $j\in[q]$.
Add edges between all pairs of vertices $d_i$ and $d_j$ where 
$i\neq j\in [p]$ to construct the clique $C^{\prime}$.
Denote the set of vertices $\{w_1,w_2,\ldots,w_q\}$ by $W$. 
This will be an independent set in $G^\prime$.
\item For each edge $c_iu_j$ in $E(G)$ (where $i\in [p]$ and $j\in [q]$), add an edge $d_iw_j$.
\item Construct $q$ vertices $s_1,s_2,\ldots,s_q$. We will denote the set $\{s_1,s_2,\ldots,s_q\}$ by $S$ and refer to the vertices in $S$ as \emph{special} vertices. Now add an edge from each $s_i$ to each $d_j$ such that $i\in [q]$ and $j \in [p]$. That is, each vertex $s_i$ in $S$ should be adjacent to all the vertices $d_j$ in $C^\prime$, as shown in \cref{TSR_deg_NPH}.
\end{itemize}
\begin{figure}
    \centering
    \includegraphics[scale=0.40]{./images/TSR_deg_red_NPH}
    \includegraphics[scale=0.3]{./images/clique_tree_2.png}
    \caption{Reduction for \textsc{Token Sliding Reachability} problem parameterized by the maximum clique-tree degree and the corresponding clique-tree.}
    \label{TSR_deg_NPH}
\end{figure}
This completes the description of the reduced instance. 
\begin{lemma}
$G^\prime$ is a chordal graph with a maximum clique-tree degree of
at most $3$.
\end{lemma}
\begin{proof}
Clearly, $G^\prime$ is a split graph with clique $C^\prime$ 
and independent set $S\cup W$. 
Hence $G^\prime$ is chordal. 
We construct a clique-tree $\calT_c$ for $G^\prime$ as follows: 
Add a vertex $U_i$ in $\calT_c$ corresponding to the clique 
induced by the vertices $\{w_i,d_1,d_2,\ldots,d_p\}$ in $G^\prime$ 
for $i\in [q]$. 
Similarly, add a vertex $W_i$ in $\calT_c$ corresponding 
to the clique induced by the vertices in $N[w_i]$ in $G^\prime$ for $i\in [q]$. 
Add an edge from $U_i$ to $U_{i+1}$ for each $i\in [q-1]$. 
Add an edge from $U_i$ to $W_i$ for each $i\in [q]$. 
Thus, the maximum degree of $\calT_c$ is $3$. 
Each vertex in $V(\calT_c)$ corresponds to a maximal clique in 
$G^\prime$. 
Each vertex in $S\cup W$ appears in just one vertex (bag) of $\calT_c$,
and each vertex in $C^\prime$ appears in all the vertices $U_i$ for 
$i\in [q]$ and possibly some of the vertices $W_j$ for $j\in [q]$, 
which induces a path in $\calT_c$ with possibly some pendant edges.
Thus, for each vertex $v\in V(G^\prime)$, the set of vertices (bags) of 
$\calT_c$ which contain $v$, 
induces a connected sub-tree of $\calT_c$. 
Hence, $G^\prime$ has a clique-tree of maximum degree $3$. 
\end{proof}
\begin{lemma}
$(G,I,J)$ is a \yes\ instance of the \textsc{Token Sliding Reachability} problem
if and only if $(G^\prime, I^\prime, J^\prime)$ is a \yes\ instance of the \textsc{Token Sliding Reachability} problem.
\end{lemma}
\begin{proof}
The subgraph induced by $C^\prime \cup W$ in $G^\prime$ 
is isomorphic to $G$ where the isomorphism $\phi$ is given by 
$\phi(c_i)=d_i$ for $i\in [p]$ and $\phi(u_j)=w_j$ for $j \in [q]$.
Define $\phi(T)=\bigcup_{v\in T}{\phi(v)}$ for any subset $T$ of $G$. 
%Clearly, $\phi(T)$ will be a subset of $C^\prime \cup W$ and 
%$\phi(T)$ will be an independent set in $G^\prime$ if and only if 
%$T$ is an independent set in $G$. 
Define $I^\prime=\phi(I)$ and $J^\prime=\phi(J)$. 
Since $I$ and $J$ are independent sets in $G$, $I^\prime$ and $J^\prime$ are independent sets in $G^\prime$.

In the forward direction,
it is sufficient to show that if any independent set $I_1$ in $G$ 
can be reconfigured to an independent set $I_2$ in $G$ in 
one valid move of token sliding, then the independent set 
$\phi(I_1)$ in $G^\prime$ can be reconfigured to an 
independent set $\phi(I_2)$ in $G$ in one valid move of token sliding.
If $I_2=I_1\setminus \{u\}\cup \{v\}$, then $\phi(I_2)=\phi(I_1)\setminus \{\phi(u)\}\cup \{\phi(v)\}$. 
Thus $\phi(I_2)$ can be reconfigured from $\phi(I_1)$ by moving 
the token on $\phi(u)$ to $\phi(v)$. 
Since $\phi$ is an isomorphism, this is a valid move.

In the reverse direction, if $I^\prime=J^\prime$, 
then $I=J$ and we trivially have a \yes\ instance.
Therefore, assume $I^\prime \neq J^\prime$ and thus at least 
one token needs to be moved to reconfigure $I^\prime$ to $J^\prime$.
If at any intermediate step, a special vertex contains a token, 
then there cannot be a token in $C^\prime$ and thus no other 
token can move. 
Therefore we will need to move this token to $C^\prime$. 
Thus any sequence of valid token sliding moves in $G^\prime$, 
where the initial and final independent sets do not contain 
a special vertex, can be modified to a sequence of valid token 
sliding moves in $G^\prime$ where no intermediate configuration 
has a token on a special vertex, with the same initial and final independent 
sets.
Hence we can get a sequence of valid token sliding moves from $I^\prime$ 
to $J^\prime$ which places tokens only on vertices of $C^\prime \cup W$. 
Thus, we can get a sequence of independent sets obtained by valid token
sliding moves in $G$ by replacing each intermediate independent set 
$K$ in this sequence by $\phi^{-1}(K)$.
\end{proof}



The above two claims, along with the fact that reduction takes polynomial-time,
imply that \textsc{Token Sliding Reachability} is \NP-\hard\ on chordal graphs with maximum clique-tree degree $3$.


%\end{proof}