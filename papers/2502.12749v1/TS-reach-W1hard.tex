% !TEX root = ./main.tex

\subsection{Hardness of Token Sliding Reachability}
In this subsection, we prove \Cref{thm:W-hardness-leafage-TS-Reach}.
We present a parameter-preserving reduction
that takes an instance
$(G,\langle V_1,V_2,\ldots,V_k\rangle, k)$ of the
\textsc{MultiColored Clique} problem as an input and returns an instance
$(G^\prime, I,J)$ of the \textsc{TS-Reachability} problem.
%where $G^\prime$ is a chordal graph with leafage $\ell=3\cdot {{k}\choose{2}}+2k+2$. The instance $(G^\prime,I,J)$ is a \yes\ instance if and only if $(G,\langle V_1,V_2,\ldots,V_k\rangle, k)$ is a \yes\ instance.
Without loss of generality, we assume that each $V_i$
has at least one edge incident to it.
%First we describe the construction of the instance $(G^\prime,nk)$ of  \textsc{Token Sliding-Reachability} from a given instance $(G,\langle V_1,V_2,\ldots,V_k\rangle, k)$ of \textsc{Multicolored Clique}.
As before, we find it convenient to describe a chordal graph
$G^\prime$ as a tree model $\mathcal{T}$.
Recall that each vertex of $G^\prime$ is associated
with a specific subtree of
$\mathcal{T}$ and two vertices of $G^\prime$ share
an edge if and only if their corresponding subtrees have a non-empty intersection.
%Then, we specify independent sets $I$ and $J$ in $G^\prime$.
We describe the reduction 
%in steps and prove necessary claims in between
after an informal overview.

\begin{figure}[t]
\centering
\includegraphics[scale=0.5]{images/TSR_leaf_Info_Red.png}
\caption{Outline of the resulting graph $G^{\prime}$. See the informal overview of the reduction.}
\label{TSR_leafW[1]H_Red}
\vspace{-5mm}
\end{figure}

\subparagraph*{Informal Overview of the Reduction:}
\Cref{TSR_leafW[1]H_Red} shows the outline of the resulting graph $G^{\prime}$.
The green box, denoted by $V$, corresponds to the encoding of vertices in $G$.
The purple and blue boxes contain additional vertices.
Most of the vertices encoding edges in $G$,
are across purple, blue, and green boxes,
and are used to move tokens between these boxes.
The initial independent set $I$ is
$I_{\text{index}}\cup I_{\text{park}}$ and
the final independent set $J$ is
$J_{\text{index}}\cup J_{\text{park}}$.
The reduction constructs graph $G^\prime$ that satisfies
the following properties.
\begin{enumerate}
\item If there is a token in
$I_{\text{park}} \cup J_{\text{park}} \cup K_{\text{index}}$,
then one can not move a token out of $I_{\text{index}}$.
\item {All} the tokens in $I_{\text{index}}$ can be
moved to $K_{\text{index}}$ via $V$ only if
tokens in $V$ (which are moved from $I_{\text{park}}$)
corresponds to a multicolored clique in $G$.
\item Tokens in $K_{\text{index}}$ impose restrictions on movements of tokens in $V$.
\end{enumerate}
These properties imply that to move \emph{all} the tokens
from $I$ to $J$, one needs to move the tokens in the following phases:
In the first phase, one needs to move
all the tokens from $I_{\text{park}}$
to $V$ in such a way that their position corresponds to
a multicolored clique in $G$.
In the second phase, one needs to move all the tokens from
$I_{\text{index}}$ to $K_{\text{index}}$.
This move \emph{locks} the tokens at their places in $V$
and the `clique'-configuration is maintained throughout the movements
of the tokens.
As all the tokens are now out of $I_{\text{index}}$,
one can move all the tokens from
$K_{\text{index}}$ to $J_{\text{index}}$ in the third phase.
Finally, as there are no tokens in $K_{\text{index}}$ now,
one can move all the tokens in $V$ to $J_{\text{index}}$
in the fourth phase.

\begin{figure}[t]
\centering
\includegraphics[scale=0.4]{images/TS_Reach_reduction_reach_tree_structure.png}
\caption{The tree model used in the reduction to prove \Cref{thm:W-hardness-leafage-TS-Reach}. The central vertex is $t_0$.
The vertices in the green rectangles are used to encode vertices
in $V(G)$.
The purple rectangles are parking structures for the initial and final
independent set.
The vertices in the blue rectangle are index vertices, making
all of them can only be moved if placements of tokens in
the green rectangle corresponds to a clique in $V(G)$.}
\label{fig:TS-Reach-reduction-reach-tree-structure}
\vspace{-5mm}
\end{figure}

\subparagraph*{Structure of the model tree $\mathcal{T}$:}
To begin with, the model tree $\mathcal{T}$ consists of
a central vertex $t_0$, which is the root and has
$k$ children $t_1,t_2,\ldots,t_k$.
See \Cref{fig:TS-Reach-reduction-reach-tree-structure} for an illustration.
For each $i\in[k]$, $t_i$ has two children $t_i^a$ and $t_i^b$.
The reduction subdivides each edge $t_it_i^a$ and $t_it_i^b$ of the tree $\mathcal{T}$ by adding $2n-1$ new vertices on this edge.
Denote the path between $t_i^a$ and $t_i^b$ by $T_i$ for each $i \in [k]$.
We use $T_i$ to encode vertices in $V_i$.
The collection of vertices on the subtrees of the tree induced by the union of
the vertex sets $\{t_1,t_2,\ldots,t_k\}$, $\{t^a_1,t^a_2,\ldots,t^a_k\}$,
and $\{t^b_1,t^b_2,\ldots,t^b_k\}$ is denoted by $V$.
These vertices encode the vertices $V(G)$ of the input graph $G$.
Add two more children of $t_0$ and label them as $t_P^I$ and $t_P^J$.
We will use $t_P^I$ to park the tokens at the beginning and
$t_P^J$ to park the tokens at the end.

For every $k' \in [\binom{k}{2}]$, add a child of $t_0$ labelled $t^I_{k'}$
and a child of $t_0$ labelled $t^J_{k'}$.
For every $i < j \in [k]$, a child of $t_0$ labelled $t^K_{ij}$.
{We denote $I^{\calT }_{\text{index}} := \{t^I_{k'} \mid\ k' \in [\binom{k}{2}]\} $,
$J^{\calT}_{\text{index}} := \{t^J_{k'} \mid\ k' \in [\binom{k}{2}]\}$ but
$K^{\calT }_{\text{index}} := \{t^K_{ij} \mid\ i < j \in [k]\}$.}
{We highlight the two different ways of indexing these sets.}
A vertex corresponding to $t^K_{ij}$ is associated with
a collection of edges with one endpoint in $V_i$ and another endpoint
in $V_j$.
Whereas, the vertices in $I^{\calT }_{\text{index}}$ and 
$ J^{\calT}_{\text{index}}$
are a collection of $\binom{k}{2}$-many vertices each.
Subdivide each edge of the form $t^I_{k'}t_0$ and $t^J_{ij}t_0$
by adding intermediate vertices $\ell^I_{k'}$ and $\ell^J_{k'}$, respectively.
Similarly, subdivide each edge of the form $t^K_{ij}t_0$ by adding an
intermediate vertex $\ell^K_{ij}$.

\subparagraph*{Encoding the vertices of $G$:}
We add the following two types of vertices in $G^{\prime}$
to encode $n$ vertices in $V_i$ for each $i \in [k]$.
Recall that we have subdivided edge $t_it_i^a$ and edge $t_it_i^b$
of the tree $\mathcal{T}$ and added $2n-1$ new vertices
on each of these edges.
\begin{itemize}
\item Add $n$ new vertices $u_i^1,u_i^2,\ldots,u_i^n$
in $G^\prime$ corresponding to the $n$ disjoint intervals
on this edge, starting from the interval from $t_i^a$
to the first new vertex as shown in \Cref{fig:Encoding_Ver_TSR}.
Also, for every $p \in [n-1]$, add a connector vertex that connects
$u_i^p$ and $u_i^{p+1}$.
\item  Add $n$ new vertices $w_i^1,w_i^2,\ldots,w_i^n$
in $G^\prime$ corresponding to the $n$ disjoint intervals
on this edge, starting from the first new vertex to the interval
before $t_i^b$ as shown in \Cref{fig:Encoding_Ver_TSR}.
As before, for every $p \in [n-1]$, add a connector vertex
that connects $w_i^p$ and $w_i^{p+1}$.
\end{itemize}
Note that if there are $p$ tokens on the left-hand side of $T_i$
(i.e., between $t_i^a$ and $t_i$) it is safe to assume that they are on
$u_i^1,u_i^2,\ldots,u_i^p$. If not, then we can move them to
$u_i^1,u_i^2,\ldots,u_i^p$ by valid token sliding operations
without disturbing the tokens on the rest of the graph.
Similarly, if there are $p$ tokens on the right hand side of $T_i$
(i.e., between $t_i^b$ and $t_i$) it is safe to assume that
they are on $w_i^{n-p+1},w_i^{n-p+2},\ldots,w_i^n$.
If tokens are placed on $u_i^1,u_i^2,\ldots,u_i^p$ and
$w_i^{p+1},w_i^{p+2},\ldots,w_i^n$, it corresponds to
selecting the $p^{th}$ vertex in $V_i$ as a part of the clique.

\begin{figure}[t]
\centering
\includegraphics[scale=0.5]{images/Encoding_Ver_TSR.png}
\caption{Adding vertices corresponding to vertices in $V_i$ to $G'$ when
$n = 5$. Note that on the left side, the indexing starts from the
vertex farthest from $t_i$, whereas on the right side, the indexing
starts from the vertex closest to $t_i$. \label{fig:Encoding_Ver_TSR}
\vspace{-5mm}}
\end{figure}


\subparagraph*{Encoding the edges of $G$:}
Whenever there is an edge from the $p^{th}$ vertex in $V_i$ to the
$q^{th}$ vertex in $V_j$ where $p, q\in[n]$ and $i\neq j\in [k]$,
add a red vertex in $G^\prime$ as described below.
%See \Cref{red-2.4} for an illustration.
\begin{itemize}
\item Add a line segment starting from $u_i^{p+1}$
(including the starting point of the interval corresponding to $u_i^{p+1}$)
to the endpoint of the interval $w_i^p$
(including the endpoint of the interval corresponding to $w_i^p$).
\item Similarly, add a line segment starting from $u_j^{q+1}$
(including the starting point of the interval corresponding to $u_j^{q+1}$)
to the starting point of the interval $w_j^q$
(including the endpoint of the interval corresponding to $w_j^q$).
\item Add a horizontal line segment from $t_0$ to $t_{ij}^K$
in $K_{\text{index}}$.
\item Connect these three horizontal line segments by
a subpath from $t_i$ to $t_j$ (containing $t_0$).
\end{itemize}
See \Cref{Encoding_edge_TSR}.
We denote the vertices of $G^\prime$ which are added corresponding
to the edges of $G$ by $\calH$-type vertices as we add two structures
similar to a horizontal letter $H$ (with an extra line segment joining the
centre of this structure to $t_0$) corresponding to each edge of $G$.
We denote the $\calH$-type vertex which corresponds to the edge in $G$
corresponding to the edge from the $p^{th}$ vertex in $V_i$ and
the $q^{th}$ vertex in $V_j$ by $r_{ij}^{pq}$.
We use these vertices to move the tokens from $I_{\text{index}}$
to $K_{\text{index}}$ and from $K_{\text{index}}$ to $J_{\text{index}}$.
We make the following modifications to make the first type of movement possible.
Recall that we have subdivided each edge of the
form $t^K_{ij}t_0$ by adding an intermediate vertex $\ell^K_{ij}$.
\begin{itemize}
\item
Extend the sub-tree corresponding to each red vertex
$r^{pq}_{ij}$ till this intermediate vertex
$\ell^K_{ij}$ (or $\ell^K_{ji}$ if $i>j$).
\end{itemize}

\begin{figure}
\centering
\includegraphics[scale=0.45]{images/Encoding_edge_TSR.png}
\caption{Encoding the edge between $3^{rd}$ vertex in $V_j$ and $2^{nd}$ vertex of $V_i$.}
\label{Encoding_edge_TSR}
\vspace{-5mm}
\end{figure}

\begin{figure}[t]
\centering
\includegraphics[scale=0.45]{images/park_struc_TSR.png}
\caption{Parking Structure; $nk$ many tokens can be parked in $J_{\text{park}}$ and $I_{\text{park}}$ each.}
\label{park_struc_TSR}
\vspace{-5mm}
\end{figure}

\subparagraph*{Auxiliary vertices:}
We start with the parking structure.
We add two parking structures to the graph
viz one by subdividing $t_0t^I_p$ and another one by
subdividing $t_0t^J_p$.
\begin{itemize}
\item We add a path in $G^\prime$ with $nk$ blue vertices
(disjoint intervals) between $t_0$ and $t_P^J$ as shown \Cref{park_struc_TSR}.
We connect these vertices using $(nk-1)$ green vertices as shown.
This allows us to park $nk$ many tokens in the
parking structure $J_\text{park}$.
Similarly, add a path with $nk$ blue vertices (disjoint intervals)
connected by $nk-1$ green vertices between $t_0$ and $t_P^I$.
\item We add a purple vertex $b^\star$ whose model is
$\{t_0,t_1,t_2,\ldots,t_k\}$.
This vertex acts as a bridge that takes tokens from the
$T_i$ part (for some $i\in [k]$) if we are able to get a token to $t_i$.
\end{itemize}
Define $I_{\text{park}}$ as the collection of the vertices
corresponding to the subpath of $\mathcal{T}$ from
$t_P^I$ to $t_0$ (including $t_P^I$ but excluding $t_0$).
Similarly, $J_{\text{park}}$ denotes the vertices corresponding
the subpath of $\mathcal{T}$ from $t_P^J$ to $t_0$
(including $t_P^J$ but excluding $t_0$).
Next, we add vertices in $I_{\text{index}}, J_{\text{index}}$ and
$K_{\text{index}}$.
\begin{itemize}
\item For every $k' \in [\binom{k}{2}]$,
add two vertices in $G^{\prime}$ whose models are
$\{t^I_{k'}\}$ and $\{\ell^I_{k'}\}$, respectively.
These are demonstrated by the dark blue and
orange intervals
in \Cref{fig:TS-Reach-auxillary-vertices}.
\item For every pair $i < j \in [k]$, add a vertex in $G^{\prime}$,
denoted $g_{ij}$, whose model is $\{t^K_{ij},\ell^K_{ij}\}$.
These are shown by the green vertices in \Cref{fig:TS-Reach-auxillary-vertices}.
\item For every $k' \in [\binom{k}{2}]$,
add a vertex, denoted by $p_{k'}$, in $G^{\prime}$
whose model is $\{t^J_{k'},\ell^J_{k'}\}$.
These are depicted by the purple vertices in
\Cref{fig:TS-Reach-auxillary-vertices}.
\end{itemize}
We now define $I_{\text{index}}:=\{\ell^I_{k^\prime}\mid k^\prime \in [\binom{k}{2}]\}$,
$J_{\text{index}}:= \{p_{k^\prime}\mid k^\prime \in [\binom{k}{2}] \}$, and
$K_{\text{index}}:=\{g_{ij}\mid i<j \in [k]\}$.
\begin{figure}
\centering
\includegraphics[scale=0.45]{images/auxilary_ver_TSR.png}
\caption{Adding the Auxillary vertices}
\label{fig:TS-Reach-auxillary-vertices}
\vspace{-5mm}
\end{figure}
Next, we describe the construction of the
structure, which restricts the movement of tokens.
\begin{itemize}
\item Add the cyan vertex as in \Cref{fig:TS-Reach-auxillary-vertices}
which corresponds to a  star-like structure
centered at $t_0$ and covering all the vertices on the path from
$t_0$ to $t_P^I$ and on the path from $t_0$ to $t_P^J$.
This vertex also covers the paths from $t_0$ to $\ell^J_{k^\prime}$
for $k^\prime \in [\binom{k}{2}]$.
We refer to this vertex as the first \emph{choke-type vertex}
and denote it by $\calC_1$.
\end{itemize}
Finally, we describe the additional vertices in $G$ which
will facilitate the movement of the tokens from
$K_{\text{index}}$ to $J_{\text{index}}$
in case of a \yes\ instance of \textsc{Multicolored Clique}.
%Once this shift is done, then all the tokens in $V$
%can move to $J_{\text{park}}$,
%completing the reconfiguration of tokens from $I$ to $J$.
\begin{itemize}
\item Add a connector vertex $c^{ij}$ corresponding
to the sub-tree formed by $\ell^K_{ij},t_0$ and $\ell^J_{k^\prime}$
for each $i<j \in [k]$ and $k^\prime \in [\binom{k}{2}]$. This is shown in
 \Cref{fig:TS-Reach-auxillary-vertices}.
\end{itemize}

This completes the description of the construction of graph $G^\prime$.
The reduction sets the initial independent set $I$ is
$I_{\text{index}}\cup I_{\text{park}}$ and
the final independent set $J$ is
$J_{\text{index}}\cup J_{\text{park}}$.
Finally, it returns $(G^\prime, I, J)$ is an instance of \textsc{TS-Reachability}.
In the next two lemmas, we prove that the reduction is safe.

\begin{restatable}{lemma}{tsReachForward}
\label{lemma:forward-reduction-leafage-TS-Reach}

If $(G,\langle V_1,V_2,\ldots,V_k\rangle, k)$ is a \yes\ instance of \textsc{MultiCol Clique},
then $(G^\prime, I, J)$ is a \yes\ instance of \textsc{TS-Reachability}.
\end{restatable}
\begin{proof}
Suppose $G$ has a multicolored clique consisting of the $p_i^{th}$
vertex from each $V_i$.
%As mentioned in the informal overview,
%in the first phase, one needs to move
We move all the tokens from $I_{\text{park}}$
to $V$ in such a way that their position corresponds to
a multicolored clique in $G$.
Formally, we move the vertices in $I_\text{index}$ to $V$ such that each
$T_i$ has tokens on $\{u_i^1,u_i^2,\ldots,u_i^{p_i}\}$ and
$\{w_i^{p_i+1},w_i^{p_i+2},\ldots,w_i^n\}$.
We say an $\calH$-type vertex is \emph{usable} (to move tokens)
if it is not adjacent to any token.
Consider an edge incident on $q^{th}$ vertex in $V_i$
and the $\calH$-type vertex added to $G^{\prime}$ to encode it.
By the construction, the interval starting from $u_i^{q+1}$ to $w_i^q$
(both inclusive) is a part of the model of $\calH$-type vertex.
This implies edges whose both endpoints are in the multicolored clique,
have no token adjacent to them even when some are moved to $V$.
Hence, the edges in the multiclique mentioned above
are always usable to move (remaining) tokens from $I$ to $V$.
And hence, we can move all $k \cdot n$ tokens from $I$ to $V$.

Using the same argument as above,
one can move all the tokens from $I_{\text{index}}$ to $K_{\text{index}}$.
Formally, consider an index $k' \in \binom{k}{2}$ and $i, j \in [k]$.
One can move a token on $\ell_{k'}^I$ in $I_{index}$
to $g_{ij}^K$ in $K_{index}$ by moving it to $b^*$
and then using the red $\calH$-type vertex corresponding to
the edge in multicolored clique across the vertex in $V_i$ and $V_j$.

In the third phase, one can move all ${k}\choose{2}$ tokens in
$K_\text{index}$ to $J_\text{index}$ by first moving a token
on $g^{ij}$ to $c^{ij}$ and then to $p^{k^\prime}$.
Note that this is possible as before starting this phase,
all the tokens are out of $I_{\text{index}}$.

Finally, as there are no tokens in $K_{\text{index}}$,
one can move all the tokens in $V$ to $J_{\text{index}}$
in the fourth phase.
If there are ${k}\choose{2}$ tokens in $J_\text{index}$ and
$nk$ tokens in $V$, all the tokens in $V$ can be brought to $J_\text{park}$.
Each token in $V$ (in some $T_i$) can be brought to the vertex
$b^*$ and then to the last unoccupied blue vertex in $J_\text{park}$.
%Thus all the tokens in $V$ can be brought to $J_\text{park}$.
Hence, we can move all the tokens in $K_{\text{index}}$ to $J_{\text{index}}$
and then all the tokens in $V$ to $J_{\text{park}}$.
Thus, the new independent set occupied by the tokens is $J$
which completes the proof.
\end{proof}

\begin{lemma}
\label{lemma:backward-reduction-leafage-TS-Reach}
If $(G^\prime, I, J)$ is a \yes\ instance of \textsc{TS-Reachability}
then $(G,\langle V_1,V_2,\ldots,V_k\rangle, k)$ is a \yes\ instance of \textsc{MultiCol Clique}.
\end{lemma}
\begin{proof}
We first prove some claims regarding the possible movements
of the tokens.


\begin{restatable}{claim}{tsReachclChoke}
\label{cl:choke1}
If there is a token in 
$I_{\text{park}} \cup J_{\text{park}} \cup K_{\text{index}}$,
then one can not move any token out of $I_{\text{index}}$.
\end{restatable}
\begin{claimproof}
If a token from $I_\text{index}$ must be moved outside
$I_\text{index}$, then it must be first brought to an orange vertex.
The orange vertex is adjacent to $\calC_1$.
Therefore, no vertex adjacent to $\calC_1$ can already
contain a token.
Since all the vertices in
$I_{\text{park}}\cup J_{\text{park}} \cup J_{\text{index}}$
are adjacent to $\calC_1$, if
$I_{\text{park}}\cup J_{\text{park}} \cup J_{\text{index}}$
has at least one token, then all the tokens in
$I_\text{index}$ must be on the dark blue vertices
no token can be moved to an orange vertex.
Hence no token can be moved outside $I_\text{index}$.
\end{claimproof}

\noindent Hence, to move the vertices from $I_{\text{index}}$,
we must move the vertices out of $I_{\text{park}}$.
Moreover, we can not move these tokens from $I_{\text{park}}$ 
to $J_{\text{park}}$ or to $J_{\text{index}}$ 
before all the tokens out of $I_{\text{index}}$.
We are also not be able to move any of these tokens to
$K_\text{index}$ as once a token lands on
$g_{ij}\in K_\text{index}$, the red vertex $r_{ij }$
cannot be used for any other token as it is adjacent to $g_{ij}$.
This implies that no token can be brought inside or outside
$T_i\cup T_j$.
Thus some of the tokens in $I_{\text{index}}$ will get stuck
if all of them are not moved to $K_\text{index}$ in the first phase.
Hence, one needs to move all the tokens in $I_{\text{park}}$
to $V$.
The next two claims argue about the movements
of such tokens.

\begin{restatable}{claim}{tsReachclAtmostNtokens}
\label{cl:atmostn+1tokens} 
For any $i \in [k]$,
{for an $\calH$-type vertex adjacent to some $T_i$ to remain
usable, one can move at most $n$ tokens to $T_i$.}
\end{restatable}
\begin{claimproof}
Suppose one has moved $n + 1$ tokens to  $T_i$.
Without loss of generality, we can assume that the tokens
are on $u_i^1,u_i^2,\ldots,u_i^p$ and
$w_i^n,w_i^{n-1},\ldots,w_i^p$
i.e., $p$ vertices on the left hand side of $T_i$
and $n+1-p$ vertices on the right hand side of $T_i$
where $p \in [n]$.
%Recall that there is at least one edge adjacent to a vertex in $V_i$ in $G$.
%Let the index of this vertex be $q \in [n]$.
Consider an edge incident on $q^{th}$ vertex in $V_i$
and the $\calH$-type vertex added to $G^{\prime}$ to encode it.
By the construction, the interval starting from $u_i^{q+1}$ to $w_i^q$
(both inclusive) is a part of the model of $\calH$-type vertex.
Define $S_p = \{u_i^1,u_i^2,\ldots,u_i^p\}
\cup \{w_i^n,w_i^{n-1},\ldots,w_i^p\}$.
%For any $p\in [n]$, the intersection set $S$ of
%$\{u_i^1,u_i^2,\ldots,u_i^p,w_i^n,w_i^{n-1},\ldots,w_i^p\}$
%and $\{u_i^{q+1},u_i^{q+2},\ldots,u_i^n,w_i^1,w_i^2,\ldots,w_i^q\}$
%will be of size at least $1$.
If $p\geq q+1$, then $S_p$ contains $u_i^{q+1}$.
Otherwise, $p\leq q$ and $S_p$ contains $w_q$.
Thus, if there are $n+1$ tokens in $T_i$, then no $\calH$-type vertex
adjacent to $T_i$ is usable.
\end{claimproof}



\noindent By the construction, at least one $\calH$-type vertex is needed
to move tokens from $t_i$ to other vertices in $T_i$.
Hence, the above claim implies that
one can move at most $n + 1$ tokens to $T_i$.

\begin{restatable}{claim}{tsReachclUsable}
\label{cl:usable}
If $T_i$ contains $n$ tokens, then $\calH$-type vertex corresponding to an edge adjacent to the $q^{th}$ vertex in $V_i$ is be usable if and only 
if it has no tokens outside $T_i$ and the tokens in $T_i$ are on 
$\{u_i^1,u_i^2,\ldots,u_i^q,w_i^n,w_i^{n-1},\ldots,w_i^{q+1}\}$.
\end{restatable}
\begin{claimproof}
Without loss of generality, we can assume that the tokens are present on
$u_i^1,u_i^2,\ldots,u_i^p$ and $w_i^n,w_i^{n-1},\ldots,w_i^{n-p+1}$
i.e., $p$ vertices on the left hand side of $T_i$ and $n-p$ vertices on the
right hand side of $T_i$ where $p\in [n]$.
If there is an edge adjacent to the $q^{th}$ vertex in $V_i$ in $G$,
then the $\calH$-type vertex corresponding to this edge
intersects $T_i$ at $\{u_i^{q+1},u_i^{q+2} \ldots,u_i^n\} \cup \
\{w_i^1,w_i^2,\ldots,w_i^q\}$.
If $p=q$, then there are $n$ tokens on
$\{u_i^1,u_i^2,\ldots,u_i^q\} \cup
\{w_i^n,w_i^{n-1},\ldots,w_i^{q+1}\}$,
then this has an empty intersection with
$\{u_i^{q+1},u_i^{q+2},\ldots,u_i^n\} \cup \{w_i^1,w_i^2,\ldots,w_i^q\}$
and thus the corresponding $\calH$-type vertex is usable
if there is no token outside $T_i$ adjacent to it.
If $p\neq q$, the intersection set $S$ of the tokens in $T_i$, i.e.,
$\{u_i^1,u_i^2,\ldots,u_i^p\} \cup
\{w_i^n,w_i^{n-1},\ldots,w_i^{n-p+1}\}$
and the vertices of $T_i$ which are adjacent to the $\calH$-type
vertex corresponding to the edge incident on the $q^{th}$ vertex
of $V_i$ given by
$\{u_i^{q+1},u_i^{q+2} \ldots,u_i^n\} \cup \{w_i^1,w_i^2,\ldots,w_i^q\}$
will be of size at least one.
If $p<q$, then $S$ will contain $w_q$ and if $p>q$,
then $S$ will contain $u_i^{q+1}$.
Therefore the $\calH$-type vertex cannot be usable.
\end{claimproof}

\noindent Now all the tokens in $I_{\text{index}}$ can be moved to
$K_{\text{index}}$ if and only if all the tokens in
$I_{\text{park}}$ are moved to $V$ such that
${k}\choose{2}$ of the red ($\calH$-type) vertices are usable
(i.e., do not have any token adjacent to them).
The red ($\calH$-type) vertices corresponding to the edges between the 
$p^{th}$ vertex in $V_i$ and the $q^{th}$ vertex in $V_j$ are usable 
for $i<j$. 
Thus we have ${k}\choose{2}$ red ($\calH$-type) vertices which are usable.


If there are ${k}\choose{2}$ red ($\calH$-type) vertices which are usable,
then each distinct pair of $i\neq j\in [k]$ must contribute one
usable red vertex.
This is because any arrangement of $n+1$ tokens on $T_i$ cannot
contribute a usable $\calH$-type vertex by \Cref{cl:atmostn+1tokens},
and an arrangement of $n$ tokens on $T_i$ can contribute at most one
usable $\calH$-type vertex.
A pair of distinct indices $i,j\in [k]$ can contribute a usable
$\calH$ type vertex $r^{pq}_{ij}$ if and only if there are tokens
on $\{u_i^1,u_i^2,\ldots,u_i^{p}\}$ and
$\{w_i^{p+1},w_i^{p+2},\ldots,w_i^n\}$ in $T_i$;
and $\{u_j^1,u_j^2,\ldots,u_j^{q}\}$ and
$\{w_j^{q+1},w_j^{q+2},\ldots,w_i^n\}$ in $T_j$.
The existence of $r^{pq}_{ij}$ implies the existence of
an edge between the $p^{th}$ vertex of $V_i$ and
the $q^{th}$ vertex of $V_j$ in $G$.
Thus the existence of ${k}\choose{2}$ red ($\calH$-type)
vertices that are usable imply the existence of a 
multicolored clique in $G$.
\end{proof}

\Cref{lemma:forward-reduction-leafage-TS-Reach} and
\Cref{lemma:backward-reduction-leafage-TS-Reach}
implies that the reduction is safe.
This, along with the fact that the reduction works in polynomial time
and the leafage $\ell$ of the chordal graph $G^\prime$
is $3\cdot {{k}\choose{2}}+2k+2$
implies the 
proof of \Cref{thm:W-hardness-leafage-TS-Reach}.
