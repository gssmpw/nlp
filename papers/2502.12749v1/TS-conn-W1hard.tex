% !TEX root = ./main.tex
\subsection{Hardness for \textsc{Token Sliding Connectivity}}

In this subsection, we prove \Cref{thm:W-hardness-leafage-TS-Conn}.
We present a parameter-preserving reduction that takes as input an 
instance $(G,\langle V_1,V_2,\ldots,V_k\rangle, k)$ of 
\textsc{MultiCol Ind-Set} and returns an instance 
$(G^\prime,nk)$ of \textsc{TS-Connectivity} 
where $G^\prime$ is a chordal graph with leafage $\ell = 2 \cdot k$. 
We find it convenient to describe a tree model $\mathcal{T}$ of the 
chordal graph $G^\prime$.
Recall that in this model,
each vertex of $G^\prime$ corresponds to a specified subtree of 
$\mathcal{T}$, and two vertices of $G^\prime$ are adjacent 
if and only if their corresponding subtrees have a non-empty intersection.

\subparagraph*{Structure of the model tree $\mathcal{T}$, Parking Structure, and Conditional Free Pass }

The model tree $\mathcal{T}$ consists of a central vertex $t_0$, which is the root and has $k+1$ children $t_1,t_2,\ldots,t_k$, and $t_p$.
For each $i\in[k]$, $t_i$ has two children $t_i^a$ and $t_i^b$.
The reduction subdivides each edge $t_i t_i^a$ and $t_i t_i^b$ of the tree $\mathcal{T}$ by adding $2n-1$ new vertices on this edge.
Denote the path between {$t_i^a$} and $t_i^b$ by $T_i$ for each $i \in [k]$.
We use $T_i$ to encode vertices in $V_i$.
It also subdivides $t_0 t_p$ and uses it to park the tokens as
described below.
\begin{itemize}
\item {We subdivide $t_0t_p$ of the tree $\mathcal{T}$ by adding} $nk$ blue vertices and $nk-1$
green vertices as shown in \Cref{structure}.
This allows us to park $nk$ tokens on blue vertices in this `parking structure'.
\item We add a purple vertex $b^\star$ in $G$ whose model contains
${t_0,t_1,t_2,\ldots,t_k}$ and the first vertex in the path from
$t_0$ to $t_p$, i.e., it is a star centered at $t_0$ with $k + 1$ leaves.
\end{itemize}
Vertex $b^{\star}$ will act as a bridge and allow us to take
tokens from any $t_i$ (for some $i\in [k]$) to the parking spot
provided it is possible to move a token to $t_i$.

Next, we move to the conditional free pass.
The idea is to ensure that if there is no token in $T_i$
(which is the condition), then all the tokens in $T_j$
can be moved to $t_j$, then to $b^{\star}$ and eventually
to the parking structure.
For each $i\neq j$ where $i,j\in [k]$, we add a orange vertex that has the following model.
\begin{itemize}
\item In $T_i$, add a line segment from $t_i^a$ to $t_i^b$.
In $T_j$, add a line segment from the first vertex from $t_j$ towards
$t_j^a$ to the first vertex from $t_j$ towards $t_j^b$.
Finally, connect these two line segments by an inverted $\calV$-like
structure with the top of the inverted $\calV$ corresponding to the vertex
$t_0$ as shown in \Cref{structure}.
\end{itemize}

\begin{figure}[t]
    \centering
    \includegraphics[scale=0.4]{./images/Structure.png}
    \includegraphics[scale=0.5]{./images/Conditional.png}
    \caption{(Top) Structure of the model tree. Node $t_0$ is the central vertex and $t_1, t_2, \dots, t_k$ and $t_p$ are its children.
     The structure between $t_0$ and $t_p$ is to park the tokens.
     (Bottom) The orange vertex denotes the conditional free pass
     between $T_i$ and $T_j$.
     \label{structure}}
     \vspace{-5mm}
\end{figure}
%Before moving forward, we remark that it might be possible that
%$b^{\star}$ is adjacent with some other token and hence the 
%tokens in $T_j$ can not be moved to it.
%The above claim only implies the condition on the 
%usability of the red vertex intersecting $T_j$.

\subparagraph*{Encoding the vertices of $G$:}
We add the following three types of vertices in $G^{\prime}$ to encode
$n$ vertices in $V_i$ for each $i \in [k]$.
Recall that we have subdivided the edge $t_it_i^a$ and the edge $t_it_i^b$
of the tree $\mathcal{T}$ and added $2n-1$ new vertices on each of
these edges.
\begin{itemize}
\item
Add $n$ new vertices $u_i^1,u_i^2,\ldots,u_i^n$ in $G^\prime$
corresponding to the $n$ disjoint intervals on this edge starting
from the vertex $t_i^a$ 
to the first new vertex, from the second new vertex to the third new vertex and so on as shown in
\Cref{Encoding_Vertices_TSC}. Here, the vertex $u_1$ corresponds to the interval between the vertex $t_i^a$ and the first new vertex and so on.
Also, add $n-1$ connector vertices that
connect $u_i^p$ and $u_i^{p+1}$ where $p\in [n]$.
\item
Similarly, add $n$ new vertices $w_i^1,w_i^2,\ldots,w_i^n$ in $G^\prime$
corresponding to the $n$ disjoint intervals on this edge starting
from the first new vertex to the second new vertex and so on such that $w_i^n$
corresponds to the interval from the last new vertex to the vertex $t_i^b$ as shown in \Cref{Encoding_Vertices_TSC}.
As before, add $n-1$  connector vertices for each
$i\in[k]$ which connect $w_i^p$ and $w_i^{p+1}$.
\item As shown in \Cref{Encoding_Vertices_TSC}, add $(n-1)$ pink vertices
for each $t_i$ where $i\in [k]$ as follows:
Each pink vertex $y_i^p$ intersects ${u_i^p,u_i^{p+1},\ldots,u_i^n}$
and ${w_i^1,w_i^2,\ldots,w_i^{p+1}}$ for $p\in [n-1]$.
\end{itemize}

Note that if there are $p$ tokens on the left-hand side of $T_i$
(i.e., between $t_i^a$ and $t_i$) it is safe to assume that they are
on $u_i^1,u_i^2,\ldots,u_i^p$.
If not, then we can move them to $u_i^1,u_i^2,\ldots,u_i^p$ by valid
token sliding operations without disturbing the tokens on
the rest of the graph.
Similarly, if there are $p$ tokens on the right-hand side of $T_i$
(i.e., between $t_i^b$ and $t_i$) it is safe to assume that they are on
$w_i^{n-p+1},w_i^{n-p+2},\ldots,w_i^n$.
The pink vertices are added to ensure the following two
claims hold.
\begin{restatable}{claim}{tsConnClaimNTokens}
\label{cl:ntokens}
For any $i \in [k]$, if there are $n$ tokens on $T_i$, then no token
can be moved to a pink vertex.
\end{restatable}
\begin{claimproof}
We first consider the case when all the tokens are on
${w_i^1,w_i^2,\ldots,w_i^n}$.
If we move the token on some $w_i^q$ to some pink vertex $y_i^r$
where $r\neq q$, it will be adjacent to the token on the vertex $w_i^r$.
If we move the token on some $w_i^q$ to $y_i^q$ then it
will be adjacent to the token on some $w_i^r$ where $r\neq q$.
Hence, in this case, no token can be moved to a pink vertex.
Using similar arguments, if all tokens are
on ${u_i^1,u_i^2,\ldots,u_i^n}$, then no token can be moved to a pink vertex.

Consider the case when tokens are on ${u_i^1,u_i^2,\ldots,u_i^p}$
and on ${w_i^{n-p+1},w_i^{n-p+2},\ldots,w_i^n}$
for some $0<p<n$.
Suppose we move the token on $u_i^1$ to the pink vertex $y_i^1$,
this token will be adjacent to the token on $u_i^p$ if $p\geq2$
otherwise it will be adjacent to the token on $w_i^2$.
If we move the token on $u_i^q$ where $2\leq q \leq p$
to a pink vertex $y_i^r$ where $r\leq q$,
then it will be adjacent to $u_i^1$.
If we move the token on $w_i^q$ to a pink vertex $y_i^r$
such that $r\geq q\geq p+2$, then it will be adjacent to
the token on $w_i^{p+1}$.
If $p\neq n-1$ and we move the token on $w_i^{p+1}$
to a pink vertex $y_i^r$ such that $r\neq p+1$,
then it will be adjacent to the token on $w_i^n$.
If we move the token on $w_i^{p+1}$ to the pink vertex
$y_i^{p+1}$, then it will be adjacent to the token on the vertex $u_i^p$.
Thus, no token from $T_i$ can be moved to a pink vertex via a valid token sliding move.
\end{claimproof}
\begin{restatable}{claim}{tsConnNrTokenEachSide}
\label{cl:lessthann}
For any $i \in [k]$, if there are fewer than $n$ tokens on $T_i$, then
the pink vertices intersecting $T_i$ can be used to move tokens to $b^{\star}$ (and then to the parking structure).
\end{restatable}
\begin{claimproof}
As before, we remark that the claim only implies the condition on the
usability of the pink vertices intersecting $T_i$.
It might be possible that $b^{\star}$ is adjacent to some other
token and hence the tokens in $T_i$ cannot be moved to it.
However, for the sake of clarity, we assume that this is not the case
in the rest of the proof.

Without loss of generality, we assume that there are tokens on
$u_i^1,u_i^2,\ldots,u_i^p$ and $v_i^{n-q+1},v_i^{n-q+2},\ldots,v_i^n$
where $p+q<n$.
Move the token on $u_i^p$ to $y_i^p$ which is not adjacent
to any other token.
Now move this token to the purple vertex $b^{\star}$
and then to the last vertex of the parking space.
Similarly move each $u_j$ (where $j<p$ and $j$ is the index
of the largest remaining token between $t_i^a$ and $t_i$) to $b^{\star}$.
Then move it to the last empty blue vertex.
Thus all the tokens on $u_i^1,u_i^2,\ldots,u_i^p$ can be moved
to $b^{\star}$.
Similarly all the tokens on $v_i^{n-q+1},v_i^{n-q+2},\ldots,v_i^n$
can be moved to the parking space via $b^{\star}$ by starting
from the token on the vertex with the smallest index.
\end{claimproof}

\begin{figure}[t]
    \centering
    \includegraphics[scale=0.5]{./images/yellow1.png}
    \caption{Add $(n-1)$ pink vertices for each $t_i$ where $i\in[k]$ as follows: Each pink vertex $y_i^p$ intersects $\{u_i^p,u_i^{p+1},\ldots,u_i^n\}$ and $\{w_i^1,w_i^2,\ldots,w_i^{p+1}\}$. In the figure $n$ is taken to be $5$.
    No token from $T_i$ can be moved to a pink vertex.}
    \label{Encoding_Vertices_TSC}
    \vspace{-5mm}
\end{figure}

\subparagraph*{Encoding the edges of $G$:}
Whenever there is an edge from the $p^{th}$ vertex in $V_i$ to the $q^{th}$ vertex in $V_j$ where $p,q\in[n]$ and $i,j\in [k]$, add a vertex in $G^\prime$ as described below. Refer \Cref{Encoding_edges_TSC} for an illustration.
\begin{itemize}
\item Add a horizontal line segment from the endpoint of the interval corresponding to $u_i^p$ to the endpoint of the interval corresponding to $w_i^p$ (both inclusive). Add another horizontal line segment from the endpoint of the interval corresponding to $u_j^{q+1}$ to the starting point of the interval corresponding to $w_j^q$ (both inclusive). Join these two horizontal line segments by a vertical line segment.
\item Similarly, add a horizontal line segment from the endpoint of the interval corresponding to $u_j^q$ to the starting point of the interval corresponding to $w_j^q$ (both inclusive). Add another horizontal line segment from the endpoint of the interval corresponding to $u_i^{p+1}$ to the starting point of the interval corresponding to $w_i^p$. Join these two horizontal line segments by a vertical line segment.
\end{itemize}
We denote the vertices of $G^\prime$ which are added corresponding to the edges of $G$ by $\calH$-type vertices as we add two structures similar to letter $H$.

If there are $n$ tokens in $T_i$ for some $i$, none of the tokens in $T_i$
can be moved to the parking structure using the pink vertices as shown in 
 \Cref{cl:ntokens}.
Thus in this case we will need to use the $\calH$-type vertices to move at least one of these tokens to the parking space.

\begin{restatable}{claim}{tsConnclEdge}
\label{cl:edge}
Consider an edge $e$ of $G$ from the $p^{th}$ vertex in $V_i$ 
to the $q^{th}$ vertex in $V_j$ where $p,q\in [n]$ and $i,j\in [k]$.
Suppose $n$ tokens in $T_i$ are placed on $\{u_i^1,u_i^2,\ldots,u_i^p\}$ 
and $\{w_i^{p+1},w_i^{p+2},\ldots,w_i^n\}$. 
A token in $T_i$ can be moved to $b^{\star}$ and then to the parking 
space using the $\calH$-type structure corresponding to $e$ 
if and only if $T_j$ contains $n$ tokens placed on 
$\{u_j^1,u_j^2,\ldots,u_j^q\}$ and 
$\{w_j^{q+1},w_j^{q+2},\ldots,w_j^n\}$.
\end{restatable}
\begin{claimproof}
Suppose the $n$ tokens in $T_j$ are on $\{u_j^1,u_j^2,\ldots,u_j^q\}$ 
and $\{w_j^{q+1},w_j^{q+2},\ldots,w_j^n\}$. 
None of these are adjacent to the $\calH$-type structure 
corresponding to $e$. 
The only token on $T_i$ which is adjacent to the 
$\calH$-type structure corresponding to $e$ is the 
token on $u_i^p$, hence it can move to the 
$\calH$-type structure without violating the independence.

Now we show the converse. Without loss of generality, 
suppose the $n$ tokens in $T_j$ are on 
$\{u_j^1,u_j^2,\ldots,u_j^r\}$ and $\{w_j^{r+1},w_j^{r+2},\ldots,w_j^n\}$ 
where $r\neq q$. 
If $r>q$, then the token on $u_j^r$ will 
be adjacent to the $\calH$- type structure. 
Similarly if $r<q$, the token on $w_j^q$ 
will be adjacent to the $\calH$-type structure. 
Thus, no token in $T_i$ can be moved to the $\calH$-type structure.
\end{claimproof}

\begin{figure}
    \centering
    \includegraphics[scale=.55]{./images/edge1.png}
    \caption{For $n = 5$, if there exists an edge between the $2^{nd}$
    vertex of $V_i$ and $3^{rd}$ vertex of $V_j$, 
    this is the $\calH$-type structure corresponding to this edge.
    This is the $\calH$-type structure when $n=5$ and 
    there is an edge between $2^{nd}$ vertex of $V_i$ and 
    the $3^{rd}$ vertex of $V_j$. 
    Notice that one of the tokens on $u_i^{2}$ (or $u_j^{3}$) 
    can move to $t_0$ using the cyan (or the red) vertex.}
\label{Encoding_edges_TSC} 
\vspace{-5mm}
\end{figure}

This completes the description of the reduction and the necessarily claims.
Next, we show that $(G,\langle V_1,V_2,\ldots,V_k\rangle, k)$
is a \yes\ instance of \textsc{MultiCol Ind-Set} if and only if
the instance $(G^\prime,nk)$ is a \no\ instance of 
\textsc{TS-Connectivity}. 
Let the independent set in $G^\prime$ which consists of the 
$nk$ vertices in the parking structure be denoted by $I^\star$.
We use the fact that $(G^\prime,nk)$ is a \yes\ instance of 
\textsc{TS-Connectivity} if and only if any 
independent set $I$ of size $nk$ in $G^\prime$ can be 
modified to $I^\star$ via a sequence of valid token sliding operations.

\begin{restatable}{lemma}{tsConnforward}
\label{lemma:forward-reduction-leafage}
If $(G,\langle V_1,V_2,\ldots,V_k\rangle, k)$ is a \yes\ instance of \textsc{MultiCol Ind-Set},
then $(G^\prime,nk)$ is a \no\ instance of \textsc{TS-Connectivity}.
\end{restatable}
\begin{proof}
%Suppose $(G,\langle V_1,V_2,\ldots,V_k\rangle, k)$ is a \yes\ 
%instance of \textsc{MultiCol Ind-Set}. 
Supose $X=\{v_1^{p_1},v_2^{p_2},\ldots,v_k^{p_k}\}$ be a solution of 
\textsc{MultiCol Ind-Set} where the $p_i^{th}$ vertex from 
partition $V_i$ is in the solution for every $i\in [k]$. 
We construct an independent set $I$ in $G^\prime$ of size 
$nk$ as follows:
For each $i \in [k]$, place $n$ tokens in each $T_i$ on the vertices 
$u_i^1,u_i^2,\ldots,u_i^{p_i}$ and $w_i^{p_i+1},w_i^{p_i+2},\ldots,w_i^n$. 
We show that $I^\star$ is not reachable from $I$ via a set 
of valid token sliding moves.
Since there are $n$ tokens in $T_i$, pink vertices cannot 
be used to move a token in $T_i$ to the parking structure by 
\Cref{cl:ntokens}.
Moreover, by the construction, the orange vertex intersecting both 
$T_i$ and $T_j$ can be used to move tokens in $T_j$ to $b^{\star}$
(and then to the parking structure) if and only if 
$T_i$ does not contain any token.
And hence, the orange vertex also cannot be used 
to move a token in $T_i$ to the parking structure.
Hence it will be possible to move a token to the parking structure 
only using the $\calH$-type structure. 
Using \Cref{cl:edge}, when there are $n$ tokens each in both 
$T_i$ and $T_j$ such that the tokens in $T_i$ are at 
$\{u_i^1,u_i^2,\ldots,u_i^p\}$ and 
$\{w_i^{p+1},w_i^{p+2},\ldots,w_i^n\}$ and 
the tokens in $T_j$ are at $\{u_j^1,u_j^2,\ldots,u_j^p\}$ and 
$\{w_j^{p+1},w_j^{p+2},\ldots,w_j^n\}$, 
the tokens can be moved to the parking structure using 
the $\calH$-type structure corresponding to the edge between $p^{th}$ 
vertex in $V_i$ and $q^{th}$ vertex in $V_j$. 
This is not possible because $X$ is an independent set.
Hence, no token in $I$ can be moved to the parking space and thus 
$I^\star$ cannot be reached from $I$.
\end{proof}


\begin{lemma}
\label{lemma:backward-reduction-leafage}
If $(G,\langle V_1,V_2,\ldots,V_k\rangle, k)$ is a \no\ instance of \textsc{MultiCol Ind-Set},
then $(G^\prime,nk)$ is a \yes\ instance of \textsc{TS-Connectivity}.
\end{lemma}
\begin{proof}
Suppose $(G,\langle V_1,V_2,\ldots,V_k\rangle, k)$ is a \no\ 
instance of \textsc{MultiCol Ind-Set}. 
We show that any independent set $I$ in $G^\prime$
of size $nk$ can be transformed to $I^\star$ via a 
sequence of valid sequence of token sliding operations. 

Note that a token on $b^{\star}$ can be moved to a parking structure.
Also, any token on any of the $\calH$-type structured vertex, or pink
or orange vertices can either be moved to $b^{\star}$ if possible or
can be moved to the blue vertices on $T_i$ 
(and this will only free up space for other tokens to move). 
Hence without loss of generality we can assume that all the tokens in $I$ are 
on $T_i$ for some $i\in[k]$ or in the parking structure. 
If there are some tokens in the parking structure, 
we will move them as close to $t_p$ as possible.
We now consider the following two mutually disjoint 
and exhaustive cases.

\emph{Case $I$:} There exists $i\in [k]$ such that $T_i$ contains 
at most $(n-1)$ tokens. 
This can happen either because some tokens are already in the parking 
structure or there exists $j\in [k]$ such that $T_j$ 
contains at least $n+1$ tokens. 
In this case, move all the tokens in $T_i$ to the parking structure 
using the pink vertices (as seen in \Cref{cl:lessthann}). 
After this, move all the vertices in each $T_j$ such that 
$j\neq i$ to the parking structure using the orange vertices.
Note that by the construction, the orange vertex intersecting both 
$T_i$ and $T_j$ can be used to move tokens in $T_j$ to $b^{\star}$
(and then to the parking structure) if and only if 
$T_i$ does not contain any token.
Thus we can reach from $I$ to $I^\star$.

\emph{Case $II$:} For every $i\in [k]$, 
there are exactly $n$ tokens in each $T_i$. 
Without loss of generality, we can assume that these 
tokens are on $\{u_i^1,u_i^2,\ldots,u_i^{p_i}\}$ and 
$\{w_i^{p_i+1},w_i^{p_i+2},\ldots,w_i^n\}$ for some $p_i\in[n]$.
Consider set formed by taking the $p_i^{th}$ vertex $v_i^{p_i}$
in the $i^{th}$ partition in $G$ for $i\in [k]$. 
This forms a $k$-sized multicolored subset of $V(G)$ 
and thus it cannot be an independent set as we 
have a \no\ instance of \textsc{MultiCol Ind-Set}.
Thus there must be an edge between two of the vertices in this set.
We will use the $\calH$-type structure corresponding to 
this edge in $G^\prime$ as seen in \Cref{cl:edge} 
to move a token in some $T_i$ 
(where the edge is incident on a vertex from the $i^{th}$ partition in $G$) 
to the parking structure. 
Now there are $n-1$ tokens in $T_i$ and we proceed similar to 
Case $I$ and thus we can reconfigure $I$ to $I^\star$ 
via a sequence of valid token sliding moves.
\end{proof}
The proof of \Cref{thm:W-hardness-leafage-TS-Conn} follows
from \Cref{lemma:forward-reduction-leafage},
\Cref{lemma:backward-reduction-leafage} and the facts
that the reduction takes polynomial time and the leafage of
the resulting chordal graph is $2k + 1$.