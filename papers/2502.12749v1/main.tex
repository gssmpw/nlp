\documentclass[a4paper,UKenglish,cleveref, autoref, thm-restate]{lipics-v2021}
%This is a template for producing LIPIcs articles. 
%See lipics-v2021-authors-guidelines.pdf for further information.
%for A4 paper format use option "a4paper", for US-letter use option "letterpaper"
%for british hyphenation rules use option "UKenglish", for american hyphenation rules use option "USenglish"
%for section-numbered lemmas etc., use "numberwithinsect"
%for enabling cleveref support, use "cleveref"
%for enabling autoref support, use "autoref"
%for anonymousing the authors (e.g. for double-blind review), add "anonymous"
%for enabling thm-restate support, use "thm-restate"
%for enabling a two-column layout for the author/affilation part (only applicable for > 6 authors), use "authorcolumns"
%for producing a PDF according the PDF/A standard, add "pdfa"

\pdfoutput=1 %uncomment to ensure pdflatex processing (mandatatory e.g. to submit to arXiv)
\hideLIPIcs  %uncomment to remove references to LIPIcs series (logo, DOI, ...), e.g. when preparing a pre-final version to be uploaded to arXiv or another public repository

\graphicspath{{./images/}}%helpful if your graphic files are in another directory

%\bibliographystyle{apalike}
% the mandatory bibstyle haha no

\title{Revisiting Token Sliding on Chordal Graphs} %TODO Please add

%\titlerunning{Dummy short title} %TODO optional, please use if title is longer than one line

\author{Rajat Adak}{Indian Institute of Science Bangalore, India  \and \url{https://sites.google.com/view/rajatadak}}{rajatadak@iisc.ac.in}{}{}%TODO mandatory, please use full name; only 1 author per \author macro; first two parameters are mandatory, other parameters can be empty. Please provide at least the name of the affiliation and the country. The full address is optional. Use additional curly braces to indicate the correct name splitting when the last name consists of multiple name parts.

\author{Saraswati Girish Nanoti}{Indian Institute of Technology Gandhinagar, India}{nanoti_saraswati@iitgn.ac.in}{}{}

\author{Prafullkumar Tale}{Indian Institute of Science Education and Research Bhopal, India \and \url{https://pptale.github.io/}}{prafullkumar@iiserb.ac.in}{}{}

\authorrunning{Adak, Nanoti and Tale} %TODO mandatory. First: Use abbreviated first/middle names. Second (only in severe cases): Use first author plus 'et al.'

\Copyright{Rajat Adak, Saraswati Girish Nanoti and Prafullkumar Tale} %TODO mandatory, please use full first names. LIPIcs license is "CC-BY";  http://creativecommons.org/licenses/by/3.0/


\begin{CCSXML}
<ccs2012>
<concept>
<concept_id>10003752.10003809</concept_id>
<concept_desc>Theory of computation~Design and analysis of algorithms</concept_desc>
<concept_significance>500</concept_significance>
</concept>
</ccs2012>
\end{CCSXML}

\ccsdesc[500]{Theory of computation~Design and analysis of algorithms}

%TODO mandatory: Please choose ACM 2012 classifications from https://dl.acm.org/ccs/ccs_flat.cfm 

\keywords{Independent Set, Token Sliding, Chordal Graphs, Leafage, W[1]-hardness} %TODO mandatory; please add comma-separated list of keywords

\category{} %optional, e.g. invited paper

\relatedversion{} %optional, e.g. full version hosted on arXiv, HAL, or other respository/website
%\relatedversiondetails[linktext={opt. text shown instead of the URL}, cite=DBLP:books/mk/GrayR93]{Classification (e.g. Full Version, Extended Version, Previous Version}{URL to related version} %linktext and cite are optional

%\supplement{}%optional, e.g. related research data, source code, ... hosted on a repository like zenodo, figshare, GitHub, ...
%\supplementdetails[linktext={opt. text shown instead of the URL}, cite=DBLP:books/mk/GrayR93, subcategory={Description, Subcategory}, swhid={Software Heritage Identifier}]{General Classification (e.g. Software, Dataset, Model, ...)}{URL to related version} %linktext, cite, and subcategory are optional

\funding{}
%optional, to capture a funding statement, which applies to all authors. Please enter author specific funding statements as fifth argument of the \author macro.

\acknowledgements{}
%optional

%\nolinenumbers 
%uncomment to disable line numbering

\usepackage[svgnames]{xcolor}

\usepackage{etoolbox}
\makeatletter
\patchcmd{\BR@backref}{\newblock}{\newblock($\uparrow$~}{}{}
\patchcmd{\BR@backref}{\par}{)\par}{}{}
\makeatother

\PassOptionsToPackage{numbers,sort&compress}{natbib}
\usepackage{natbib}
\bibliographystyle{plainnat}

\hypersetup{
    colorlinks=true,
    linkcolor=IndianRed,
    filecolor=magenta,      
    urlcolor=DodgerBlue,
    citecolor = SeaGreen
}


%Editor-only macros:: begin (do not touch as author)%%%%%%%%%%%%%%%%%%%%%%%%%%%%%%%%%%
\EventEditors{John Q. Open and Joan R. Access}
\EventNoEds{2}
\EventLongTitle{42nd Conference on Very Important Topics (CVIT 2016)}
\EventShortTitle{CVIT 2016}
\EventAcronym{CVIT}
\EventYear{2016}
\EventDate{December 24--27, 2016}
\EventLocation{Little Whinging, United Kingdom}
\EventLogo{}
\SeriesVolume{42}
\ArticleNo{23}
%%%%%%%%%%%%%%%%%%%%%%%%%%%%%%%%%%%%%%%%%%%%%%%%%%%%%%

\newcommand{\CG}{\mathcal{G}\xspace}
\newcommand{\CV}{\mathcal{V}\xspace}
\newcommand{\CE}{\mathcal{E}\xspace}
\newcommand{\CA}{\mathcal{A}\xspace}
\newcommand{\CF}{\mathcal{F}\xspace}
\newcommand{\CR}{\mathcal{R}\xspace}
\newcommand{\CB}{\mathcal{B}\xspace}
\newcommand{\CX}{\mathcal{X}\xspace}
\newcommand{\CK}{\mathcal{K}\xspace}
\newcommand{\CM}{\mathcal{M}\xspace}
\newcommand{\CC}{\mathcal{C}\xspace}
\newcommand{\CL}{\mathcal{L}\xspace}
\newcommand{\CI}{\mathcal{I}\xspace}
\newcommand{\CQ}{\mathcal{Q}\xspace}
\newcommand{\CO}{\mathcal{O}\xspace}
\newcommand{\CP}{\mathcal{P}\xspace}
\newcommand{\CS}{\mathcal{S}\xspace}
\newcommand{\CT}{\mathcal{T}\xspace}
\newcommand{\CJ}{\mathcal{J}\xspace}
\usepackage[para]{footmisc}
\usepackage{subfig}
% \usepackage{subcaption}
% \usepackage{array}
% \usepackage{colortbl}



\begin{document}

\maketitle

\begin{abstract}
In this article, we revisit the complexity of the reconfiguration 
of independent sets under the token sliding rule on chordal graphs.
In the \textsc{Token Sliding-Connectivity} problem, 
the input is a graph $G$ and an integer $k$, and 
the objective is to determine whether the reconfiguration graph 
$TS_k(G)$ of $G$ is connected. 
The vertices of $TS_k(G)$ are $k$-independent sets of $G$, 
and two vertices are adjacent if and only if one can transform one 
of the two corresponding independent sets into the other by sliding 
a vertex (also called a \emph{token}) along an edge. 
Bonamy and Bousquet [WG'17] proved that the 
\textsc{Token Sliding-Connectivity} problem is 
polynomial-time solvable on interval graphs but \NP-hard 
on split graphs.
In light of these two results, the authors asked: can we decide the connectivity of $TS_k(G)$ in polynomial time for chordal graphs with 
\emph{maximum clique-tree degree} $d$? 
We answer this question in the negative and prove that the problem is \para-\NP-hard when parameterized by $d$. 
More precisely, the problem is \NP-hard even when $d = 4$. 
We then study the parameterized complexity of the problem for a larger 
parameter called \emph{leafage} and prove that the problem is 
\co-\W[1]-hard.
We prove similar results for a closely related problem called 
\textsc{Token Sliding-Reachability}. 
In this problem, the input is a graph $G$ with two of its 
$k$-independent sets $I$ and $J$, and 
the objective is to determine whether there is a sequence of 
valid token sliding moves that transform $I$ into $J$.
\end{abstract}

\documentclass[../main.tex]{subfiles}
\graphicspath{{../images/}}
\makeatletter
\def\input@path{{../images/}}
\makeatother
\begin{document}
\section{Introduction}
\begin{figure}
\centering
\begin{tikzpicture}
\node[inner sep=0pt] (ws) at (0, 0) {
\includegraphics[height=.4\textwidth, trim={10cm 0 10cm 0},clip]{world_space.png}};
\node[inner sep=0pt] (cs) at (6,0) {\includegraphics[height=.4\textwidth, trim={10cm 1cm 10cm 4cm},clip]{conf_space.png}};
\end{tikzpicture}
\vspace{-5pt}
\label{fig:pbrm_intro}
\caption{\textbf{Left}: Shows world space obstacles as grey spheres. Robots start and goal configuration is colored red and green, respectively. Configurations along the computed path are colored transparent blue. \textbf{Right:} Mapped world space scenario to configuration space. Obstacle region is the grey mesh. Red spheres are collision-free regions computed by the neural SCDF. The optimized shortest path in the convex corridor is the blue curve.}
\vspace{-25pt}
\end{figure}
Motion planning is the problem of finding a collision-free trajectory that connects a given start and goal configuration. The planning takes place in the configuration space of the robot. For single body robots, like mobile robots or drones, the configuration space and the world space are usually the same. This simplifies the planning, since explicit obstacle representations are available which enables geometrical tools like separating hyperplanes, smallest distance to obstacles etc., to be used when designing motion planning algorithms. For multi-body robots like manipulators, the situation is completely different. The world space obstacles are usually mapped to non-convex regions, and to make the problem even harder, the mapping is usually not known. Forming explicit representations of the obstacle region in the configuration space is usually too expensive or intractable. Despite all of this, sampling based planners are used with great success, which mainly is due to their use of implicit representations of the obstacle region. The basic idea is to construct a graph in the configuration space that covers and connects the collision-free region. From this graph, a path can be extracted that connects a given start and goal configuration. The approach is computationally expensive, since the graph is constructed with the smallest geometrical building block available, points, which represents a collision-check. Furthermore, the extracted paths from the graph are non-smooth and jagged due to the stochastic nature of the approach. This adds an additional post-processing step to the process, where the paths are shortcutted and smoothened, before the path can be used for tracking. Clearly a lot of time is invested to form this graph and produce smooth paths. Thus, if the obstacles start to move, then all of this work is done in no use, since all points that make up this graph need to be re-verified, which is simply too time consuming to be done in real time.
\\\\
In this work, we want to address the existing drawbacks of the sampling based planners. Our main contribution is an improved motion planner where each vertex in the graph covers a collision-free region in the form of a sphere instead of a point and where the edges are formed with neighboring intersecting spheres. This representation has the advantage of instead of returning piecewise linear paths, returning a sequence of overlapping spheres, i.e. a convex corridor, that connects a given start and goal configuration, illustrated in Figure \ref{fig:pbrm_intro}. This convex corridor allows us to use convex optimization to produce smooth trajectories, instead of computationally expensive post-processing methods. The representation further allows us to estimate the coverage of the collision-free space, which gives us awareness and feedback in the offline roadmap construction phase. Finally, our representation is simple to adapt to moving obstacles, simply requery for the new radii and recheck for intersections. 
\\\\
The spherical collision-free regions are formed using a signed distance function (SDF), which is a function that returns the smallest distance from an arbitrary point to the boundary of an obstacle. As the name implies, the distance is signed, thus if the point is inside the obstacle it is negative otherwise positive. If the distance is positive, a sphere with radius equal to the distance is guaranteed to cover a collision-free region. Using an SDF in motion planning is not new, but what is novel about our approach is that we express the distance in the configuration space instead of the world space and by doing so allows us to form these convex collision-free regions. We refer to the resulting SDF as a signed configuration distance function (SCDF). Computing an SCDF analytically is non-trivial, our approach is therefore to parameterize the SCDF with a deep neural network and learn the mapping by supervised learning. Our resulting neural SCDF can compute distances for different parameter values of obstacle shapes and we also show how multiple distances can be combined, thus making our approach flexible.
\section{Related work}
Motion planning algorithms can roughly be divided into three families, grid-based, sampling based and optimization based methods. Grid-based methods (GBM) discretize the planning space from which a graph is then compiled. A standard search method is A$^\star$ \citep{a_star}, which is classified as an \textit{informed} search method, since it employs a heuristic function to speed up the search. A$^\star$ guarantees to return an optimal path at the level of discretization used. GBMs usually discretize the planning space by a regular lattice and this limits the GBMs to problems with low dimensionality due to the curse of dimensionality. Thus, GBMs are usually limited to single-body robots where the degrees of freedom (DOF) are low. To overcome the inherent scaling problem with the GBMs, stochastic methods are usually used for multi-body robots. These methods are termed as sampling-based methods (SBM) and core members within this family are the rapidly-exploring random trees (RRT) \citep{rrt} and the probabilistic roadmap (PRM) \citep{prm}. RRT grows a tree from the start configuration and explores the collision-free region in a rapid way until it is able to connect to the goal region. RRT is usually improved by bi-directional planning \citep{rrt_connect}, i.e. an additional tree is grown from the goal configuration and the trees are tested for connection after any tree has been expanded. RRT is a single-query method, thus it searches for a path from scratch each time it is queried. Contrary to this, PRM is a multi-query method, which solves for multiple queries without starting from scratch. PRM does this by creating a roadmap (graph) that covers the collision-free space as an offline step. The graph is then used to solve for multiple queries. PRMs are used in cases where the environment does not change since the extra offline step is too computationally costly and needs to be re-done if the environment is changed. In our work, we address this inherent issue by using a different roadmap representation. Our vertices in the graph cover a collision-free region in the form of spheres and we form the edges by checking for intersecting spheres. If something in the environment changes, we recompute the spheres radii and recheck the intersections, without relying on collision detection. We use a trained neural network to compute the sphere radius, therefore querying for the radius can be done fast, hence our representation enables the PRM for dynamic environments.
\\\\
In the recent decades, optimization based methods (OBM) \citep{chomp, schulman, itomp, stomp} have been introduced as an alternative to SBM for multi-body robots. Like the SBM, the OBMs scale well to higher dimensional problems and produce smoother motion. It is common to use a SDF in the optimization since it is a smooth function, thus enabling gradient-based methods. However, the standard way of expressing the SDF is in world space. The distance therefore needs to be mapped to the configuration space by the forward kinematics. This mapping makes the optimization problem a non-linear program (NLP), which is computationally expensive to solve. Recently, a different approach has been proposed. In \cite{mp_gcs} motion planning is formulated as a convex optimization problem by using the graph of convex sets framework \citep{gcs}. The underlying idea is to decompose the collision-free space into intersecting convex sets from which a convex optimization problem is formulated. In cases where an explicit representation of the obstacles in the configuration space exists, like for single-body robots, creating collision-free convex regions can be done fast \citep{iris}. For multi-body robots, this is non-trivial. Existing work does this successfully \citep{iris_nlp, iris_c} by an optimization based approach, but the methods are still too time consuming to be used in the presence of moving obstacles. Our approach is instead to use deep learning to learn an SDF expressed in the configuration space. With this, we can query for shortest distances to the collision boundary, which allows us to expand spherical regions which are collision-free. Our approach is fast and therefore enables our suggested roadmap planner to be used in dynamic environments.
\\\\
Recent research has focused on learning collision detection \citep{fk_kernel_distance, diffco, graphdistnet} by predicting the signed distance between the robot links and the surrounding obstacles in the world space. The learned SDF is used in trajectory optimization but since the distance is expressed in the world space, the problem becomes an NLP and therefore takes a long time to solve. We take a novel approach and suggest to instead express the signed distance in the configuration space. This allows us to improve the PRM at the same time as it enables convex optimization for trajectory optimization, which runs faster and is more reliable than NLP solvers. In \cite{cspf} a learned signed distance function in the configuration space is proposed similar to our approach. However, their approach is restricted to point cloud representations, while we propose to represent the obstacles as parameterized geometric shapes, e.g. spheres. Furthermore, we also show how to use our learned SCDF to improve an existing roadmap planner.
\section{Problem formulation}
A robot is located in the world space, $\W \subset \R^3 $. The unique location of the robot is given by its configuration $\q \in \C$, where $\C$ is the configuration space. The set of points covered by the robots bodies at a certain configuration is expressed as $\B(\q) \subset \W$. The robot is surrounded by $\NrObst$ obstacles $\O = \bigcup_{i=1}^{\NrObst} \O_i$, where  $\O_i \subset \W$. The representation of the obstacle in the configuration space is the set $\C\O_i = \{\q \in \C \: |\: \B(\q) \cap \O_i \neq \emptyset \}$. The obstacle space is formed as $\Co = \bigcup_{i=1}^{\NrObst} \C \O_i$. The complement is referred to as the free space, $\Cf = \C \setminus \Co$. The path planning problem is a tuple, ($\Cf$, $\qStart$, $\qGoal$), where we want to connect a query pair, consisting of a start, $\qStart$, and goal configuration, $\qGoal$, with a geometric path, $\q(s): [0, 1] \mapsto \Cf$, such that $\q(0)=\qStart$ and $\q(1)=\qGoal$, or report correctly when such a path does not exist.
\end{document}


\section{Preliminaries}\label{sec:preliminaries}



%We denote by $(\Ac(x_\Ac),\Bc(x_\Bc))(z)$ a random execution of $\pi$ with private inputs $(x_\Ac,y_\Ac)$, and common input $z$.

%\Jnote{Move to DP}
% At the end of such an execution, the protocol outputs a public transcript denoted by the random variable $\trans_\pi(x_\Ac,x_\Ac,z)$ we denotes the common as $\out(\trans_\pi(x_\Ac,x_\Ac,z)$, and each party $\Pc \in \set{\Ac,\Bc}$ obtains his view denoted $\view^\Pc_\pi(x_\Ac,x_\Bc,z)$, which may also contain a ``local output'' \Jnote{Local} $\out^\Pc(x_\Ac,x_\Bc,z)$ (if the protocol specifies such an output). \Jnote{Common output, and parties output}


\subsection{Distributions and Random Variables}\label{sec:prelim:dist}
The support of a distribution $P$ over a finite set $\cS$ is defined by $\Supp(P) \eqdef \set{x\in \cS: P(x)>0}$. For a distribution or a random variable $D$, let $d\from D$ denote that $d$ was sampled according to $D$. Similarly,  for a set $\cS$, let $x \from \cS$ denote that $x$ is drawn uniformly from $\cS$, and denote by $\cU_{\cS}$ the uniform distribution over $\cS$. For a finite set $\cX$ and a distribution $C_X$ over $\cX$, we use the capital letter $X$ to denote the random variable that takes values in $\cX$ and is sampled according to $C_X$. The {\sf statistical distance} (\aka {\sf~variation distance}) of two distributions $P$ and $Q$ over a discrete domain $\cX$ is defined by $\sdist{P}{Q} \eqdef \max_{\cS\subseteq \cX} \size{P(\cS)-Q(\cS)} = \frac{1}{2} \sum_{x \in \cS}\size{P(x)-Q(x)}$. 
For a vector $x = (x_1,\ldots,x_n)$ and index $i\in [n]$, we let $x_{-i} = (x_1,\ldots,x_{i-1},x_{i+1},\ldots,x_n)$ and $x^{(i)} = (x_1,\ldots,x_{i-1}, -x_i, x_{i+1},\ldots,x_n)$, for a set $\cS \subseteq [n]$ we let $x_{\cS} = (x_i)_{i \in \cS}$ and $x_{-\cS} = (x_i)_{i \in [n]\setminus \cS}$, and for a vector $r \in \zo^n$ we let $x_r = (x_i)_{\set{i \colon r_i = 1}}$ and $x_{-r} = (x_i)_{\set{i \colon r_i = 0}}$.

%For $n \in \N$ we let $U_n$ be the uniform distribution over $\oo^n$, and let $S_n$ be the distribution induces by the sum of $n$ i.i.d.\ random variables, each is distributed according to $U_1$. Let $\cN(0,1)$ be the standard normal distribution.
%For a distribution $\cD$ and a function $f$, we define by $f(\cD)$ the distribution that is induced by the output of $f(x)$ for $x \from \cD$. 





% \begin{theorem}[\cite{McGregorMPRTV10}]\label{thm:sv-extracotr}
% 	\Enote{Remove if not needed}
% 	There is a constant $c$ to make the following holds. Let $X$ be an $\alpha$-SV source on $\{0,1\}^n$, let $Y$ be a source on $\{0,1\}^n$ with min-entropy at least $\beta n$ (independent from $X$), and let $Z=\ip{X,Y}\mbox{mod m}$ for some $m\in\mathbb{N}$. Then for every $\delta\in[0,1]$, the random variable $(Y,Z)$ is $\delta$-close to $(Y,U)$ where $U$ is uniform on $\mathbb{Z}_m$ and independent of $Y$, provided that
% 	$$
% 	n\geq c\cdot\frac{m^2}{\alpha\beta}\cdot\log(\frac{m}{\beta})\cdot\log(\frac{m}{\delta}).
% 	$$
% \end{theorem}



\Enote{I removed the definition of DP since it already appears in the intro}
\remove{
\subsection{Differential Privacy}\label{sec:prelim:DP}
We use the following standard definition of (information theoretic) differential privacy, due to \citet{DMNS06}. For notational convenience, we focus on databases over $\oo$.
\begin{definition}[Differentially private mechanisms]\label{def:mech}
	A randomized function $f\colon\oo^n\mapsto \zs$ is an {\sf $n$-size, $(\eps,\delta)$-differentially private mechanism} (denoted $(\eps,\delta)$-\DP) if for every neighboring $w,w'\in \oo^n$ and every function $g\colon \zs\mapsto \zo$, it holds that 
	$$
	\pr{g(f(w))=1}\leq e^{\eps}\cdot \pr{g(f(w'))=1} +\delta.
	$$ 	
	If $\delta=0$, we omit it from the notation.
\end{definition}
}


\subsubsection{Computational Differential Privacy}
There are several ways for defining computational differential privacy (see \cref{sec:related-works}). We use the most relaxed version due to \cite{BNO08}. For notational convenience, we focus on databases over $\oo$.
\begin{definition}[Computational differentially private mechanisms]\label{def:ComMech}
	A randomized function ensemble $f=\set{f_\pk\colon\oo^{n(\pk)}\mapsto \zs}$ is an {\sf $n$-size, $(\eps,\delta)$-computationally differentially private} (denoted $(\eps,\delta)$-$\CDP$) if for every poly-size circuit family $\set{\Ac_\pk}_{\pk\in \N}$, the following holds for every large enough $\pk$ and every neighboring $w,w'\in\oo^{n(\pk)}$:
	$$
	\pr{\Ac_\pk(f_\pk(w))=1}\leq e^{\eps(\pk)}\cdot \pr{\Ac_\pk(f_\pk(w'))=1} +\delta(\pk).
	$$ 
	If $\delta(\pk) = \negl(\pk)$, we omit it from the notation. 
\end{definition}



\subsubsection{Two-Party Differential Privacy}\label{sec:DP}
In this section we formally define distributed differential privacy mechanism (\ie protocols). %For the ease of notation, we consider protocol with no common input.

\begin{definition}\label{def:DP}%\Nnote{fix security parameter}
	A two-party protocol $\Pi=(\Ac,\Bc)$ is {\sf $(\eps,\delta)$-differentially private}, denoted $(\eps,\delta)$-$\DP$, if the following holds for every algorithm $\Dc$: let $\V^\Pc(x,y)(\pk)$ be the view of party $\Pc$ in a random execution of $\Pi(x,y)(1^\pk)$. Then for every $\pk,n \in \N$, $x\in \oo^n$ and neighboring $y,y'\in\oo^n$:
	\begin{align*}
	\pr{\Dc(V^\Ac(x,y)(\pk))=1}\le e^{\eps(\pk)}\cdot \pr{\Dc(V^\Ac (x,y')(\pk))=1}+\delta(\pk),
	\end{align*} 
	and for every $y\in \oo^n$ and neighboring $x,x'\in\oo^{n}$:
	\begin{align*}
	\pr{\Dc(V^\Bc(x,y)(\pk))=1}\le e^{\eps(\pk)}\cdot \pr{\Dc(V^\Bc (x',y)(\pk))=1}+\delta(\pk).
	\end{align*} 	
	Protocol $\Pi$ is {\sf $(\eps,\delta)$-computational differentially private}, denoted $(\eps,\delta)$-$\CDP$, if the above inequalities only hold for a non-uniform \ppt $\Dc$ and large enough $\pk$. We omit $\delta = \negl(\pk)$ from the notation. \footnote{Note that define we give for two-party differentially private protocols is a semi-honest definition, in which we ask for the security to hold when the parties interact in an honest execution of the protocol. Since we are proving a lower bound, starting from this weaker guarantee (as opposed to security against malicious players), yields a stronger result.}
\end{definition}
%We omit $\delta$ from the notation if $\delta$ is a negligible function of $n$.

%\Enote{simulation-based}
\begin{remark}[The definition for computational differential privacy we use]\label{rem:comDPChannel} 
	An alternative, stronger definition of computational differential privacy, known as simulation-based computational differential privacy, requires that the distribution of each party’s view be computationally indistinguishable from a distribution that ensures privacy in an information-theoretic sense. \cref{def:DP} is a weaker notion in comparison. Consequently, establishing a lower bound for a protocol that satisfies this weaker guarantee (as we do in this work) yields a stronger result.%Actually, our lower bound only requires the privacy to hold against \emph{uniform} external observer.
	%\Nnote{Maybe add: When only interesting in \Dp against external observer, the two definitions can be achieve using key-agreement and (single-party) \Dp mechanism. }
\end{remark}




\subsection{Useful Claims}
\remove{
In this section, we state generic lemmas and propositions that we will use later in our proofs.

The following lemma which we prove in \cref{sec:missing-proofs:distance-I}, measures the distance between two uniform stings conditioned one a random index $i$ either being fixed to $0$ or to $1$.

\def\distanceILemma{
    Let $R \la \zo^n$. For any (randomized) function $f:\{0,1\}^n\rightarrow \{0,1\}$ and $\alpha > 0$, it holds that
    \begin{align}\label{eq:f-alpha}
        \ppr{i \la [n]}{\size{\:\ex{f(R) \mid R_i = 0}-\ex{f(R) \mid R_i = 1}\:}\geq \alpha} \leq \frac{2}{n \alpha^2},
    \end{align}
    where the expectations are taken over $R$ and the randomness of $f$.
}

\begin{lemma}\label{lem:distance-I}
    \distanceILemma
\end{lemma}
}

The following two propositions state that given the output of a differentially private function, it is not possible to predict well even a random index (even if all other indexes are leaked). The first proposition handles the information-theoretic case and the second handles the computation case. Both propositions are proven in \cref{sec:missing-proofs:hard-to-guess}. 

\def\propHardToGuessInf{
    Let $f\colon \oo^n \rightarrow \cY$ be an $(\eps,\delta)$-\DP function, let $g \colon [n] \times \oo^{n-1} \times \cY \rightarrow \set{-1,1,\bot}$ be a (randomized) function, and let $X = (X_1,\ldots,X_n) \la \oo^n$. Then the following holds for every $i \in [n]$ where $X_i^* = g(i,X_{-i},f(X_1,\ldots,X_n))$:
    \begin{align*}
        \pr{X_i^* = X_i} \leq e^{\eps}\cdot \pr{X_i^* = -X_i} + \delta.
    \end{align*}
}

\begin{proposition}\label{prop:hard-to-guess-inf}
    \propHardToGuessInf
\end{proposition}


\def\propHardToGuessComp{
    Let $f = \set{f_{\pk} \colon \oo^{n(\pk)} \rightarrow \zo^{m(\pk)}}_{\pk \in \bbN}$ be an $(\eps,\delta)$-\CDP function ensemble, and let $\set{g_{\pk}}_{\pk \in \bbN}$ be a poly-size circuit family. Then, for large enough $\pk$ and $X = (X_1,\ldots,X_{n(\pk)}) \la \oo^{n(\pk)}$, the following holds for every $i \in [n(\pk)]$ where $X_i^* = g_{\pk}(i,X_{-i},f_{\pk}(X_1,\ldots,X_n))$:
    \begin{align*}
        \pr{X_i^* = X_i} \leq e^{\eps(\pk)}\cdot \pr{X_i^* = -X_i} + \delta(\pk).
    \end{align*}
}

\begin{proposition}\label{prop:hard-to-guess-comp}
    \propHardToGuessComp
\end{proposition}





\remove{
\Enote{Chao's old statement:}
\begin{lemma}\label{lem:distance-I-old}
        Let $R \la \zo^n$. 
	For any function $f:\{0,1\}^n\rightarrow \{0,1\}$ and $\alpha<0.01$, it holds that
	$$
	\Pr_{i\la[n]}\left[\: \size{\:\mathbb{E}[f(R) \mid R_i = 0]-\mathbb{E}[f(R) \mid R_i = 1]\:}\geq \alpha\right]\leq \frac{2+2\log(\frac{1}{\alpha})}{n\alpha^2}.
	$$
\end{lemma}
\begin{proof}
	Define $S_1=\{r \in \zo^n \colon f(r)=1\}$. Then for any $i\in[n]$, we have
	$$
	\begin{array}{rl}
		\size{\mathbb{E}[f(R) \mid R_i = 0]-\mathbb{E}[f(R) \mid R_i = 1]}
		&=\size{\Pr[R\in S_1|R_i=0]-\Pr[R\in S_1|R_i=1]}\\
		&=\size{\frac{\Pr[R_i=0|R\in S_1]\cdot\Pr[R\in S_1]}{\Pr[R_i=0]}-\frac{\Pr[R_i=1|R\in S_1]\cdot\Pr[R\in S_1]}{\Pr[R_i=1]}}\\
		&=\frac{2\size{S_1}}{2^n}\size{\Pr[R_i=0|R\in S_1]-\Pr[R_i=1|R\in S_1]}
	\end{array}
	$$
	When $|S_1|\leq \alpha\cdot 2^{n-1}$, we have $\size{\mathbb{E}[f(R) \mid R_i = 0]-\mathbb{E}[f(R) \mid R_i = 1]}\leq\frac{2\size{S_1}}{2^n}\leq \alpha$ for any $i\in[n]$. Hence, in the following, we assume $|S_1|> \alpha\cdot 2^{n-1}$.

	%Define $I_{bad}=\{i|\size{\Pr[R_i=0|R\in S_1]-\Pr[R_i=1|R\in S_1]}>2\alpha\}$ and $k=\size{I_{bad}}$, then for any $i\notin I_{bad}$, we have 
    %$$
    %\begin{array}{rl}
    %    2\alpha&\geq \size{\Pr[R_i=0|R\in S_1]-\Pr[R_i=1|R\in S_1]}\\
    %    &=\size{\frac{\Pr[R\in S_1|R_i=0]\cdot\Pr[R_i=0]}{\Pr[R\in S_1]}-\frac{\Pr[R\in S_1|R_i=1]\cdot\Pr[R_i=1]}{\Pr[R\in S_1]}}\\
    %    &=\size{\Pr[R\in S_1|R_i=0]-\Pr[R\in S_1|R_i=1]}\cdot\frac{1}{2\Pr[R\in S_1]}\\
    %    &\geq \size{\mathbb{E}[f(R) \mid R_i = 0]-\mathbb{E}[f(R) \mid R_i = 1]}\cdot \frac{1}{2},
    %\end{array}
    %$$ 
    %where the last inequality is because $\Pr[R\in S_1]\leq 1$. So that $\size{\mathbb{E}}[f(R) \mid R_i = 0]-\mathbb{E}[f(R) \mid R_i = 1]\leq %4\alpha$.
    Define $I_{bad}=\{i \colon \size{\Pr[R_i=0|R\in S_1]-\Pr[R_i=1|R\in S_1]} \geq 2\alpha\}$ and $k=\size{I_{bad}}$, and denote $I_{bad}=\{i_1,\dots,i_k\}$. Define $(X_{i_1}, \ldots X_{i_k}) = (R_{i_1},\dots,R_{i_k})\mid_{R \in S_1}$. 
    Consider the min-entropy
	$$
	\begin{array}{rl}
		H_{min}(X_{i_1},\dots,X_{i_k})&\leq H(X_{i_1},\dots,X_{i_k})\\
		&\leq \sum_{j=1}^k H(X_{i_j})\\
		&\leq k\cdot \left(-(\frac{1}{2}+2\alpha)\cdot\log(\frac{1}{2}+2\alpha)-(\frac{1}{2}-2\alpha)\cdot\log(\frac{1}{2}-2\alpha)\right)\\
            &=k\cdot \left(-(\frac{1}{2}+2\alpha)\cdot(\log(1+4\alpha)-1)-(\frac{1}{2}-2\alpha)\cdot(\log(1-4\alpha)-1)\right)\\
            &=k\cdot \left(1-(\frac{1}{2}+2\alpha)\cdot\log(1+4\alpha)-(\frac{1}{2}-2\alpha)\cdot\log(1-4\alpha)\right),
		
	\end{array}
	$$
	where $H_{min}(Y)$ is the minimum entropy of $Y$ and $H(Y)$ is the Shannon entropy of $Y$.\Enote{add to preliminaries.}
        The third inequality holds since by the definition of $I_{bad}$, for every $j \in [k]$ it holds that $\size{\pr{X_{i_j} = 1}-\pr{X_{i_j} = 0}} > 2\alpha$, and therefore $H(X_{i_j}) \leq H(1/2 + 2\alpha)$\Enote{define}.
	
	Therefore, there exists $b_1,\dots,b_k\in\{0,1\}$, such that 
	
	\begin{align}\label{eq:min-entropy-result}
		\Pr\left[(R_{i_1},\ldots,R_{i_k}) = (b_1,\ldots,b_k) \mid R\in S_1\right]
		&= \pr{(X_{i_1},\ldots,X_{i_k}) = (b_1,\ldots,b_k)}\\
		&= 2^{-H_{min}(X_{i_1},\dots,X_{i_k})}\nonumber\\
		&\geq 2^{k\cdot \left(-1+(\frac{1}{2}+2\alpha)\cdot\log(1+4\alpha)+(\frac{1}{2}-2\alpha)\cdot\log(1-4\alpha)\right)}.\nonumber
	\end{align}
	
	Let $S_{bad}=\{r \in \zo^n  \colon \set{(r_{i_1},\ldots,r_{i_k}) = (b_1,\ldots,b_k)} \land \set{r\in S_1}\}$.
	It holds that
	\begin{align*}
		|S_{bad}|
		&= \size{S_1} \cdot \Pr\left[(R_{i_1},\ldots,R_{i_k}) = (b_1,\ldots,b_k) \mid R\in S_1\right]\\
		&\geq \alpha\cdot 2^{n-1}\cdot2^{k\cdot \left(-1+(\frac{1}{2}+2\alpha)\cdot\log(1+4\alpha)+(\frac{1}{2}-2\alpha)\cdot\log(1-4\alpha)\right)},
	\end{align*} 
	where the inequality holds by \cref{eq:min-entropy-result} and since $\size{S_1} \geq \alpha\cdot 2^{n-1}$.
	Notice that any string in $S_{bad}$ depends on at most $n-k$ bits. It implies that $|S_{bad}|\leq 2^{n-k}$. Therefore, we have
	$$
	\begin{array}{rl}
		&2^{n-k}\geq \alpha\cdot 2^{n-1}\cdot2^{k\cdot \left(-1+(\frac{1}{2}+2\alpha)\cdot\log(1+4\alpha)+(\frac{1}{2}-2\alpha)\cdot\log(1-4\alpha)\right)} \\
		\Rightarrow& n-k \geq \log \alpha+n-1+k\cdot \left(-1+(\frac{1}{2}+2\alpha)\cdot\log(1+4\alpha)+(\frac{1}{2}-2\alpha)\cdot\log(1-4\alpha)\right)\\
		\Rightarrow& 1-\log \alpha \geq k\cdot((\frac{1}{2}+2\alpha)\cdot\log(1+4\alpha)+(\frac{1}{2}-2\alpha)\cdot\log(1-4\alpha))\\
		\Rightarrow& 1-\log \alpha \geq k\cdot(4\alpha\cdot\log(1+4\alpha)+(\frac{1}{2}-2\alpha)\cdot\log(1-16\alpha^2))\\
        \Rightarrow& 1-\log\alpha \geq k\cdot(15.9\alpha^2-8\alpha^2+32\alpha^3)=k\cdot(7.9\alpha^2+32\alpha^3)>0.5k\alpha^2\\
		\Rightarrow& k\leq \frac{2-2\log \alpha}{\alpha^2} = \frac{2+2\log (1/\alpha)}{\alpha^2},
	\end{array}
	$$
	Where the third transition holds since 
	\begin{align*}
		\lefteqn{(\frac{1}{2}+2\alpha)\cdot\log(1+4\alpha)+(\frac{1}{2}-2\alpha)\cdot\log(1-4\alpha)}\\
		&= 4\alpha\cdot\log(1+4\alpha) + (\frac{1}{2}-2\alpha)\paren{\log(1+4\alpha)+\log(1-4\alpha)}\\
		&= 4\alpha\cdot\log(1+4\alpha)+(\frac{1}{2}-2\alpha)\cdot\log(1-16\alpha^2),
	\end{align*}
	and the forth transition holds since $4\alpha\cdot\log(1+4\alpha)+(\frac{1}{2}-2\alpha)\cdot\log(1-16\alpha^2) > 15.9\alpha^2-8\alpha^2+32\alpha^3$ for $\alpha < 0.01$.
	Thus, we conclude that 
	$$
	\Pr_{i\la[n]}\left[\size{\mathbb{E}[f(R) \mid R_i=0]-\mathbb{E}[f(R) \mid R_i = 1]}\geq \alpha\right]\leq \frac{k}{n}\leq \frac{2+2\log (1/\alpha)}{n\alpha^2}.
	$$
\end{proof}
}


\subsection{Channels and Two-Party Protocols}\label{sec:protocol}

\paragraph{Channels.}A channel is simply a distribution of a pair of tuples defined as follows. 
\begin{definition}[Channels]\label{def:channel} A {\sf channel} $C_{(X,U)(Y,V)}$ of size $\isize$ over alphabet $\Sigma$ is a probability distribution over $(\Sigma^\isize \times\zo^\ast) \times(\Sigma^\isize \times\zo^\ast)$. The ensemble $C_{(X,U)(Y,V)}= \set{C_{(X_\pk,U_\pk)(Y_\pk,V_\pk)}}_{\pk\in \N}$ is an $\isize$-size channel ensemble, if for every $\pk\in \N$, $C_{(X_\pk,U_\pk)(Y_\pk,V_\pk)}$ is an $\isize(\pk)$-size channel. %We denote a channel of size one by a \emph{single-bit} channel. 
We refer to $X$ and $Y$ as the {\sf local outputs}, and to $U$ and $V$ as the {\sf views}.	
\end{definition}

We view a  channel as the experiment in which there are two parties $\Ac$ and $\Bc$.  Party $\Ac$ receives ``output'' $X$ and ``view'' $U$, and party $\Bc$ receives ``output'' $Y$ and ``view'' $V$. Unless stated otherwise, the channels we consider are over the alphabet $\Sigma = \oo$. We naturally identify channels with the distribution that characterizes their output.








\subsubsection{Two-Party Protocols}

A two-party protocol $\Pi=(\Ac,\Bc)$ is \ppt if the running time of both parties is polynomial in their input length. We let $\Pi(x,y)(z)$ or $(\Ac(x),\Bc(y))(z)$ denote a random execution of $\Pi$ on a common input $z$, and private inputs $x,y$.%We assume \wlg that a protocol has a common output (part of its transcript).\Jnote{This is not really the case we consider in this paper..}

\begin{definition}[Oracle-aided protocols]\label{def:ChannelAidedProtocol}
	In a two-party protocol $\Pi$ with oracle access to a {\sf protocol} $\Psi$, denoted $\Pi^\Psi$, the parties make use of the \textit{next-message function} of $\Psi$.\footnote{The function that on a partial view of one of the parties, returns its next message.} In a two-party protocol $\Pi$ with oracle access to a {\sf channel} $C_{Z W}$, denoted $\Pi^C$, the parties can jointly invoke $C$ for several times. In each call, an independent pair $(z,w)$ is sampled according to $C_{Z W}$, one party gets $z$, the other gets $w$.
\end{definition}


\begin{definition}[The channel of a protocol]\label{def:ChannlOfProtocol}
	For a no-input two-party protocol $\Pi= (\Ac,\Bc)$, we associate the channel $C_\Pi$, defined by $\C_\Pi= C_{(X, U),(Y, V)}$, where $X$ and $Y$ are the local outputs of $\Ac$ and $\Bc$ (respectively) and
	$U$ and $V$ are the local views of $\Ac$ and $\Bc$ (respectively).
    
	For a two-party protocol $\Pi$ that gets a security parameter $1^\pk$ as its (only, common) input, we associate the channel ensemble $ \set{C_{\Pi(1^\pk)}}_{\pk\in \N}$. 
\end{definition}

\begin{definition}[$(\alpha,\gamma)$-Accurate channel]\label{def:accurate-func}
	A channel $C = C_{(X, U),(Y, V)}$ is {\sf $(\alpha,\gamma)$-accurate for the function $f$}, if $\ppr{C}{\size{\out(V)-f(X,Y)}\leq \alpha}\ge \gamma$, where $\out(V)$ is the designated output.
    A channel ensemble $C_{(X, U),(Y, V)}= \set{C_{(X_\pk, U_\pk),(Y_\pk, V_\pk)}}_{\pk\in \N}$ is  $(\alpha,\gamma)$-accurate for  $f$ if $C_{(X_\pk, U_\pk),(Y_\pk, V_\pk)}$ is $(\alpha(\pk),\gamma(\pk))$-accurate for $f$, for every $\pk \in \N$.
\end{definition}

\subsubsection{Differentially Private Channels}\label{sec:DPChannel}
Differentially private channels are naturally defined as follows:
\begin{definition}[Differentially private channels]\label{def:DPChannel}
	An $n$-size channel $C = C_{(X, U),(Y, V)}$ with $X, Y$ over $\oo^n$ 
	is {\sf$(\eps,\delta)$-differentially private} (denoted $(\eps,\delta)$-$\DP$) if for every $x \in \Supp(X)$ there exists an $n$-size $(\eps,\delta)$-$\DP$ mechanisms $\Mc_x$ such that $(X,Y,U) \equiv (X,Y,\Mc_X(Y))$, and for every $y \in \Supp(Y)$ there exists an $n$-size $(\eps,\delta)$-$\DP$ mechanisms $\Mc_y'$ such that $(X,Y,V) \equiv (X,Y,\Mc_Y'(X))$. In addition, we say that the channel is \emph{uniform} if $X$ and $Y$ are independent random variables uniformly distributed in $\oo^n$. 
\end{definition}

\begin{definition}[Computational differentially private channels]\label{def:CDPChannel}
	An $n$-size channel ensemble $C = \set{C_{(X_\pk, U_\pk),(Y_\pk, V_\pk)}}_{\pk\in\N}$ with $X_\pk, Y_\pk$ over $\oo^n$ 
	is {\sf$(\eps,\delta)$-computationally differentially private} (denoted $(\eps,\delta)$-$\CDP$) if for every ensemble $\set{x_\pk \in \Supp(X_\pk)}_{\pk\in\N}$ there exists an $n$-size $(\eps,\delta)$-\CDP mechanisms ensemble $\set{\Mc_{x_\pk}}_{\pk\in\N}$ such that $(X_\pk,Y_\pk,U_\pk) \equiv (X_\pk,Y_\pk,\Mc_{X_\pk}(Y_\pk))$, for every $\pk\in\N$, and for every ensemble $\set{y_\pk \in \Supp(Y_\pk)}_{\pk\in\N}$ there exists an $n$-size $(\eps,\delta)$-$\CDP$ mechanisms ensemble $\set{\Mc'_{y_\pk}}_{\pk\in\N}$ such that $(X_\pk,Y_\pk,V_\pk) \equiv (X_\pk,Y_\pk,\Mc_{Y_\pk}'(X_\pk))$ for every $\pk\in \N$. In addition, we say that the channel is \emph{uniform} if $X_\pk$ and $Y_\pk$ are independent random variables uniformly distributed in $\{\pm 1\}^n$ for all $\pk\in\N$.
\end{definition}




% \begin{lemma}~\label{lem:dp-sv-source}
% 	Let $P$ be an $\varepsilon$-DP randomized protocol. Let $X$ and $Y$ be independent random variables uniformly distributed in $\{\pm 1\}^n$ and let random variable $\Pi(X,Y)$ denote the transcript of running $P(X,y)$. Then for every $\pi\in Supp(\Pi)$, the random variables corresponding to the inputs conditioned on transcript $\pi$, $X_\pi$ and $Y_\pi$, are independent $e^{-\varepsilon}$-strong SV source.
% \end{lemma}





\subsubsection{Weak Erasure Channel (\WEC)}

\begin{definition}[\WEC]\label{def:WEC}
	A channel $((O_A,V_A), (O_B,V_B))$ with $O_A \in \set{0,1}$ and $O_B \in \set{0,1,\bot}$ is a {\sf weak erasure channel}, denoted $(\alpha,p,q)$-$\WEC$, if:
	\begin{itemize}
		%\item $O_A\in \set{-1,1}$ and $O_B\in \set{-1,1,\bot}$.
		\item Random erasure: $\pr{O_B = \perp} = 1/2$.
		
		\item Agreement: $\pr{O_A\ne O_B\mid O_B\ne \bot}\le \alpha$.
		
		\item Secrecy:
		
		\begin{enumerate}
			\item For every algorithm $\Dc$ it holds that\label{WEC:item:A}
			\begin{align*}
				%\size{\pr{\Ac(O_A,V_A) = 1 \mid O_B \neq \perp} - \pr{\Ac(O_A,V_A) = 1 \mid O_B = \perp}} \le p
				\size{\pr{\Dc(V_A) = 1 \mid O_B \neq \perp} - \pr{\Dc(V_A) = 1 \mid O_B = \perp}} \le p
			\end{align*}
			(Alice doesn't know if $O_B = \perp$.)
			
			\item For every algorithm $\Dc$ it holds that\label{WEC:item:B}
			\begin{align*}
				\pr{\Dc(V_B) = O_A \mid O_B=\bot} \leq \frac{1+q}{2}.
			\end{align*}
			(i.e., if $O_B=\bot$, Bob don't know what is the value of $O_A$).
			
			%\item $SD((O_A U|O_B=\bot),(O_A U|O_B\ne \bot))\le p$ (The sender don't know if $O_B=\bot$).
			
			%\item $SD(V O_A|O_B=\bot,V(-O_A)|O_B=\bot)\le q$ (If $O_B=\bot$, Bob don't know what the value of $O_A$).
		\end{enumerate}
	\end{itemize}
   We say that a channel ensemble $C=\set{C_\pk}_{\pk\in N}$ is a {\sf computational weak erasure channel}, denoted $(\alpha,p,q)$-\CompWEC, if for every \ppt algorithm $\Dc$ and every sufficiently large $\pk\in\N$, $C_\pk$ satisfies the properties stated in the items above, where the secrecy property holds with respect to a \ppt algorithm $\Dc$. A protocol $\Lambda$ is said to be $(\alpha,p,q)$-$\CompWEC$, if the ensemble induces by the protocol (that is, $C=\set{C_{\Lambda(\pk)}}_{\pk\in\N}$) is $(\alpha,p,q)$-$\CompWEC$.  
\end{definition}



\subsubsection{Approximate Weak Erasure Channel (\AWEC)}\label{sec:AWEC}

\begin{definition}[\AWEC]\label{def:AWEC}
	A channel $C = ((O_A,V_A), (O_B,V_B))$ over $([-n,n] \times \zo^*) \times (([-n,n] \cup \bot)  \times \zo^*)$ is an {\sf approximate weak erasure channel}, denoted $(\ell,\alpha,p,q)$-\AWEC if:
	\begin{itemize}
		
		\item Random erasure: $\pr{O_B = \perp} = 1/2$.
		
		\item Accuracy: $\pr{\size{O_A - O_B} > \ell \mid O_B \ne \bot}\le \alpha$.
		
		\item Secrecy:
		
		\begin{enumerate}
			\item For every algorithm $\Dc$ it holds that\label{AWEC:item:A}
			\begin{align*}
				%\size{\pr{\Ac(O_A,V_A) = 1 \mid O_B \neq \perp} - \pr{\Ac(O_A,V_A) = 1 \mid O_B = \perp}} \le p
				\size{\pr{\Dc(V_A) = 1 \mid O_B \neq \perp} - \pr{\Dc(V_A) = 1 \mid O_B = \perp}} \le p
			\end{align*}
			(Alice doesn't know if $O_B=\bot$).
			
			\item For every algorithm $\Dc$ it holds that\label{AWEC:item:B}
			\begin{align*}
				\pr{\size{\Dc(V_B) - O_A} \leq 1000 \ell \mid O_B=\bot} \leq q.
			\end{align*}
			(i.e., if $O_B=\bot$, Bob can't estimate the value of $O_A$ with error $\leq 1000 \ell$).
		\end{enumerate}
	\end{itemize}
     We say that a channel ensemble $C=\set{C_\pk}_{\pk\in N}$ is a {\sf computational approximate weak erasure channel}, denoted $(\ell,\alpha,p,q)$-\CompAWEC, if for every \ppt algorithm $\Dc$ and every sufficiently large $\pk\in\N$, $C_\pk$ satisfies the properties stated in the items above. A protocol $\Gamma$ is said to be $(\ell,\alpha,p,q)$-$\CompAWEC$, if the ensemble induced by the protocol (that is, $C=\set{C_{\Gamma(\pk)}}_{\pk\in\N}$) is $(\ell,\alpha,p,q)$-$\CompAWEC$.  
\end{definition}

We will make use of the following lemma, which shows that for some choices of the parameters, \AWEC implies \WEC. The lemma is proven in \cref{sec:AWEC-to-WEC}.

\begin{lemma}\label{lemma:AWEC-to-WEC}
	For every $\ell> 0$, there exists a \ppt protocol $\Lambda = (\Pc_1,\Pc_2)$ such that given an oracle access to an $(\ell,\alpha,p,q)$-\AWEC $C$, the channel $\tilde{C}$ induced by $\Lambda^C$ is $(\alpha'=\alpha+0.001,\: p' = p ,\:  q' = 1/2 + 2(q+0.01))$-\WEC.
	Furthermore, the proof is constructive in a black-box manner:
	\begin{enumerate}
		\item There exists an oracle-aided \ppt algorithm $\Ec_1$ such that for every channel $C = ((\OA,\VA), (\OB,\VB))$ and algorithm $\Dc$ violating the \WEC secrecy property~\ref{WEC:item:A} of $\tilde{C}$, algorithm $\Ec_1^{\Dc}$ violates the \AWEC secrecy property~\ref{AWEC:item:A} of $C$.
		
		\item There exists an oracle-aided \ppt algorithm $\Ec_2$ such that for every channel $C = ((\OA,\VA), (\OB,\VB))$ and algorithm $\Dc$ violating the \WEC secrecy property~\ref{WEC:item:B} of $\tilde{C}$, algorithm $\Ec_2^{\Dc}$ violates the \AWEC secrecy property~\ref{AWEC:item:B} of $C$.
	\end{enumerate}
\end{lemma}

Since \cref{lemma:AWEC-to-WEC} is constructive, the following is an immediate corollary.
\begin{corollary}\label{cor:CompAWEC to CompWEC}
There exists an oracle aided \ppt protocol $\Lambda$, such that given a protocol $\Gamma$ that induces $(\ell,\alpha,p,q)$-\CompAWEC, it holds that $\Lambda^\Gamma$ is $(\alpha'=\alpha+0.001,\: p' = p ,\:  q' = 1/2 + 2(q+0.01))$-\CompWEC.  
\end{corollary}
\begin{proof}[Proof of \ref{cor:CompAWEC to CompWEC}]
Let $\Lambda$ be the \ppt algorithm guaranteed  by Lemma \ref{lemma:AWEC-to-WEC}. Given an $(\ell,\alpha,p,q)$-\CompAWEC protocol $\Gamma$, we define $\Lambda(\pk)={\Lambda^{\Gamma(\pk)}(\pk)}$. Assume towards a contradiction that $\Lambda$ is not a $(\alpha',p',q')$-\CompWEC. It follows that there exists a \ppt $\Dc$ that for infinity many $\pk\in\N$ contradicts one of the \WEC secrecy properties of channel ensemble $\set{C_{\Lambda(\pk)}}_{\pk\in\N}$. Fix $\pk\in\N$ for which this holds. By Lemma \ref{lemma:AWEC-to-WEC}, there exists a \ppt $\Ec^\Dc$ that for every such $\pk$  contradicts one of the secrecy properties of the channel $C_{\Gamma(\pk)}$. This implies that for infinity many $\pk\in\N$, $\Ec^\Dc$  contradict the secrecy of the channel ensemble $\set{C_{\Gamma(\pk)}}_{\pk\in\N}$, which is a contradiction since this would means that $\Gamma$ is not a $(\ell,\alpha,p,q)$-\CompAWEC.       
\end{proof}



\subsection{Oblivious Transfer (\OT)}

\paragraph{Secure Computation.}
We use the standard notion of securely computing a functionality, \cf  \cite{Goldreich04}.
\begin{definition}[Secure computation]\label{def:SFE}
	A two-party protocol {\sf securely computes a functionality $f$}, if it does so according to the real/ideal paradigm.   We add the term perfectly/statistically/computationally/non-uniform computationally, if the simulator's output is  perfect/statistical/computationally indistinguishable/  non-uniformly indistinguishable from  the real distribution.  The protocol have the above notions of security {\sf against semi-honest  adversaries}, if its security only  guaranteed to holds against an adversary that follows the prescribed protocol.   Finally, for the case of perfectly secure computation, we naturally apply the above notion also to the non-asymptotic case: the protocol with no security parameter perfectly  compute a functionality $f$.
	
	A two-party protocol {\sf securely computes a functionality ensemble $f$ with oracle to a channel $C$}, if it does so according to the above definition when the parties have access to a trusted party computing $C$. All the above adjectives naturally extend to this setting.
\end{definition}

\paragraph{Oblivious Transfer.}
The (one-out-of-two) oblivious transfer functionality is defined as follows.
\begin{definition}[oblivious transfer functionality $f_{\OT}$]\label{def:OTfunc}
	The oblivious transfer functionality over $\zo \times (\zs)^2$ is defined by  $f_{\OT} (i,(\sigma_0,\sigma_1)) = (\perp,\sigma_i)$.
\end{definition}
A protocol is $\ast$ secure OT,   for \\$\ast\in \set{\text{semi-honest statistically/computationally/computationally non-uniform}}$, if it  compute the $f_{\OT}$  functionality with $\ast$ security.





% \begin{definition}[Computational oblivious transfer, semi-honest model]
% A protocol $\Pi=(\Ac,\Bc)$ is a semi-honest 1-out-of-2 computational oblivious transfer (comp-OT) protocol if the following holds. Given a common input $1^{\pk}$, the parties $\Ac$ and $\Bc$ run the protocol $\Pi(1^\pk)$ (in an honest manner) and    
% $\Ac$ outputs $X=(m_1,m_2)\in \zo\times\zo$ and has a view $U$ and $\Bc$ outputs $Y=(i,\hat{m})\in\zo\times\zo$ and has a view $V$, and the following properties are satisfied:
% \begin{enumerate}
%     \item \textbf{Correctness:} 
%     $\pr{\hat{m}\neq m_i}<\negl(\pk).$ 
    
%     \item \textbf{A's Privacy:} For every \ppt $\Dc$ and every sufficiently large $\pk$:
%     $\pr{\Dc(V)=m_{i-1}}<(1+\negl(\pk))/2$
    
%     \item \textbf{B's Privacy:} For every \ppt $\Dc$ and every sufficiently large $\pk$:
%     $\pr{\Dc(U)=i}<(1+\negl(\pk))/2$  
% \end{enumerate}
% \end{definition}

We make use of the following useful results by Wullschleger on oblivious transfer amplification from weak channels.
\begin{theorem}[\cite{Wullschleger09}, from \WEC to statistically secure \OT]\label{thm:WEC TO OT IT}
    There exists an oracle aided protocol $\Pi$ such that the following holds: Given a $(\alpha,p,q)$-\WEC $C$, if $44(\alpha+p)\le 1-q$ then $\Pi^{C}(1^\pk)$ is a semi-honest statistically secure \OT.
\end{theorem}

The following computational version of \cref{thm:WEC TO OT IT} is implicit in \cite{Wullschleger09} and is based on the computational proof explicitly stated in \cite{Wul07} (see Section 6 in \cite{Wullschleger09} for discussion).   

\begin{theorem}[\cite{Wullschleger09,   Wul07}, from \CompWEC to computinally secure \OT]\label{thm:WEC TO OT Comp}
    There exists an oracle aided protocol $\Pi$ such that the following holds: Given a $(\alpha,p,q)$-\CompWEC protocol $\Lambda$, if $44(\alpha+p)\le 1-q$ then $\Pi^{\Lambda}$ is a semi-honest computational secure \OT.
\end{theorem}



% \begin{definition}[Computational 1-out-of-2 Oblivious Transfer, semi-honest model]
% A protocol $\Pi=(\Ac,\Bc)$ is a semi-honest 1-out-of-2 $(\eps,\alpha,\beta)$-oblivious transfer (OT) protocol if the following holds. 

% The parties $\Ac$ and $\Bc$ run the protocol (in an honest manner) and    
% $\Ac$ outputs $X=(m_1,m_2)\in \zo\times\zo$ and has a view $U$ and $\Bc$ outputs $Y=(i,\hat{m})\in\zo\times\zo$ and has a view $V$, and following properties are satisfied:
% \begin{enumerate}
%     \item \textbf{Correctness:} 
%     $\pr{\hat{m}\neq m_i}<\eps.$ 
    
%     \item \textbf{A's Privacy:} For every adversary $\Dc$:
%     $\pr{\Dc(V)=m_{i-1}}<(1+\alpha)/2$
    
%     \item \textbf{B's Privacy:} For every adversary $\Dc$: $\pr{\Dc(U)=i}<(1+\beta)/2$  
% \end{enumerate}
% \end{definition}
\section{Parameterized by Maximum Clique-Tree Degree}
% !TEX root = ./main.tex

\label{sec:max-clique-tree-degree}

\subsection{Hardness for \textsc{Token Sliding-Connectivity}}

In this subsection, we prove Theorem~\ref{thm:np-hardness-clique-tree-degree-TS-Conn}.
% which we restate for reader's convenience.
%\tsconnnphard*
%\begin{proof}
We present a polynomial time reduction from the
\textsc{Dominating Set on Non-Blocking Graphs}
which is \NPH~\cite{DBLP:conf/wg/BonamyB17}.
%to the \textsc{TS-Connectivity} problem 
%on chordal graphs with a maximum clique-tree degree of $4$. 
For a graph $G$ and a subset $S$ of $V(G)$, a vertex $x\in V(G)$ is said to 
be a \emph{private neighbour} of $s\in S$ if $x\in N(s)$ and 
$x\notin N[t]$ for any $t\in S\setminus \{s\}$. 
A set of vertices $S\subset V(G)$ is a blocking set if no vertex in $S$ 
has a private neighbour with respect to $S$. 
%A graph $G$ is $k$-blocking if it has a blocking set of size at most $k$.
In the \textsc{Dominating Set on Non-Blocking Graphs} problem,
an input is a graph $G$ and integer $k$ such that $G$ has no blocking set of size at most $2k-1$.
The objective is to determine whether $G$ contains a dominating set
of size at most $k$.

\noindent\textbf{Reduction:} 
Our reduction closely follows the one presented 
in~\cite{DBLP:conf/wg/BonamyB17}.
Let $(G,k)$ be an input instance of 
\textsc{Dominating Set on Non-Blocking Graphs}. 
The reduction constructs an instance 
$(G^\prime,k^\prime)$ of the \textsc{TS- Connectivity} problem.
Suppose $V(G) = \{v_1,v_2,\ldots,v_n\}$. 
\begin{itemize}
\item Add the vertices in the sets $C=\{c_1,c_2,\ldots,c_{n+k+1}\}$ 
and $W=\{w_1,w_2,\ldots,w_{n+k+2}\}$ in $V(G^\prime)$.
Add the edges $c_ic_j$ in $E(G^\prime)$ for $i\neq j\in [n+k+1]$
to ensure that $G^\prime[C]$ is a clique.
Set $W$ will remain an independent set in $G^{\prime}$.
\item For each $i,j\in [n]$, if $v_i$ and $v_j$ are adjacent in $G$ then add 
the edges $c_iw_j$ and $c_jw_i$ to $E(G^\prime)$.
\item Add the edge $c_iw_i$ in $E(G^\prime)$ for $i \in [n+k+1]$. 
\item Make vertex $w_{n+k+2}$ adjacent to all the vertices in $C$
that correspond to vertices in $V(G)$, i.e., 
add edges $c_iw_{n+k+2}$ in $E(G^\prime)$ for each $i\in [n]$.
\item Add sets of vertices $X=\{x_1,x_2,\ldots,x_{n+k+2}\}$ 
and $Y=\{y_1,y_2,\ldots,y_{n+k+2}\}$ in $V(G^\prime)$.
Also, add the edges $x_iy_i$ for $i\in[n+k+2]$ and edges 
$x_ic_j$ for each $i\in [n+k+2]$ and $j\in [n+k+1]$ in $G^\prime$.
That is, each vertex in $X$ is adjacent to all the vertices in $C$,
and $G^\prime[X \cup Y]$ consists of a matching as in 
\cref{TSC_deg_coNPH}. 
\end{itemize}
\begin{figure}[t]
    \centering
    \includegraphics[scale=0.35]{./images/TSC_deg_red_coNPH}

    \includegraphics[scale=0.35]{./images/clique_tree_1.png}
    \caption{The reduced instance for the \textsc{Token Sliding Connectivity} problem parameterized by the maximum clique-tree degree and its corresponding clique-tree.}
    \label{TSC_deg_coNPH}
\end{figure}

The reduction sets $k^\prime=k+1$. 
This completes the description of the reduction.  
Next, we prove its correctness and graph $G^{\prime}$ satisfies
the desired properties.

\begin{lemma}
$G^\prime$ is a chordal graph with maximum clique-tree degree at most $4$.
\end{lemma}
\begin{proof}
Note that $G^\prime[C\cup X\cup W]$ is a split graph as 
$C$ induces a clique and $X\cup W$ induces an independent set 
in $G^\prime$. 
The graph $G^\prime$ is formed by adding a pendant vertex 
to each vertex in $X$ in the graph $G^\prime[C\cup X\cup W]$;
hence the graph $G^\prime$ is a chordal graph. 

%Now we show that the clique tree decomposition of $G^\prime$ 
%has degree at most $4$. 
We now construct a clique-tree $\calT_c$ for $G^\prime$, 
and show that it has a maximum degree of $4$. 
Add a vertex $U_i$ in $\calT_c$ corresponding to the clique 
induced by the vertices $\{x_i,c_1,c_2,\ldots,c_{n+k+1}\}$ 
in $G^\prime$ for $i \in [n+k+2]$. 
Similarly, add a vertex $W_i$ in $\calT_c$ corresponding 
to the clique induced by the vertices in $N[w_i]$ in $G^\prime$ 
for $i\in [n+k+2]$. 
Add a vertex $Y_i$ corresponding to the clique (i.e., edge) induced 
by the vertices $\{x_i,y_i\}$ for $i\in[n+k+2]$. 
Add an edge from $U_i$ to $U_{i+1}$ for each $i\in [n+k+1]$. 
Add an edge from $U_i$ to $W_i$ for each $i\in [n+k+2]$.
Add an edge from $U_i$ to $Y_i$ for each $i\in [n+k+2]$. 
Thus, the maximum degree of $\calT_c$ is $4$. 
Each vertex in $V(\calT_c)$ corresponds to a maximal clique in $G^\prime$.
This tree is also shown in \cref{TSC_deg_coNPH}.

Next, we prove that this tree satisfies the clique-intersection property.
Each vertex in $Y\cup W$ appears in only one vertex (bag) of $\calT_c$.
Each vertex $x_i\in X$, for some $i \in [n+k+2]$, 
appears in the vertex (bag) $Y_i$ and all the vertices $U_j$ 
where $j\in [n+k+2]$. 
Since the vertices $U_j$ where $j\in [n+k+2]$ induce a path in 
$\calT_c$ and the vertex $Y_i$ is adjacent to the vertex $U_i$;
the set of vertices containing $x_i$ induces a connected sub-tree of 
$\calT_c$. 
Now any vertex $c_i\in C$ (for some $i\in [n+k+1]$) appears 
in all the bags $U_j$ where $j\in [n+k+2]$, and possibly some 
of the bags $W_\ell$ where $\ell\in [n+k+2]$.
This induces a path with some pendant vertices, 
which is a connected subtree of $\calT_c$. 
Thus, $G^\prime$ has a clique-tree with maximum degree $4$.
\end{proof}
\begin{lemma}
\label{lemma:TS-conn-max-deg}
$(G,k)$ is a \yes\ instance of the
\emph{\textsc{Dominating Set on Non-Blocking Graphs}} 
problem if and only if $(G^\prime,k^\prime)$ is a \no\ instance of the
\emph{\textsc{TS-Connectivity}} problem. 
\end{lemma}
\begin{proof}
Suppose $(G,k)$ is a \yes\ instance of the
\textsc{Dominating Set on Non-Blocking Graphs} problem. 
Without loss of generality, let $D=\{v_1,v_2,\ldots,v_k\}$
be a dominating set in $G$. 
We construct an independent set $I$ of size $k+1$ in $G^\prime$ 
such that no token on $I$ can be moved. 
Define $I :=\{w_1,w_2,\ldots,w_k,w_{n+k+2}\}$. 
Clearly, the token on $w_{n+k+2}$ cannot be moved because 
$N(w_{n+k+2})=\{c_i \mid i\in [n]\}$ and 
if we move this token to $c_j$ where $j\in [n]$, 
it will be adjacent to the token on $w_\ell$ for some 
$\ell \in [k]$ as $v_j\in N[v_\ell]$ for some $\ell\in [k]$ as $D$
is a dominating set in $G$. 
Now, if we move the token on $w_i$ for some $i\in [k]$ to any vertex 
in $N(w_i)\subset \{c_1,c_2,\ldots,c_n\}$, this will be adjacent to the 
token on $w_{n+k+2}$ because $N(w_{n+k+2})= \{c_1,c_2,\ldots,c_n\}$.
Thus $I$ will form an isolated vertex in the configuration graph 
$TS_{k+1}(G^\prime)$. 
As there are another independent sets of size $k+1$ in $G^\prime$,
such as $\{w_{n+1}, w_{n+2}, \dots, w_{n+k+1}\}$, 
$(G^{\prime}, k)$ is a \no\ instance of the \textsc{TS-Connectivity} problem. 

Suppose we have a \no\ instance of the
\textsc{Dominating Set on Non-Blocking Graphs} problem, 
then we show that $(G^\prime,k+1)$ is a \yes\ instance of the
\textsc{TS-Connectivity} problem. 
Define independent set $J:=\{w_{n+1},w_{n+2},\ldots,w_{n+k+1}\}$.
We show that any independent set $I$ of size $k+1$ in 
$G^\prime$ can be reconfigured to $J$ using a sequence 
of valid token sliding moves. 
It is sufficient to show that the independent set $I$ can be reconfigured 
into an independent set $I^\prime$ such that $|I^\prime\cap J|>|I\cap J|$. 
We will demonstrate this through a case analysis depending 
on how $I$ intersects the partitions $C, X, Y$, and $W$.

\emph{Case $(I)$: $I\cap C\neq \emptyset$:}
Let $\{c_i\}=|I\cap C|$ as $I$ is an independent set and 
hence it cannot contain two or more vertices from $C$. 
Now the other tokens in $I$ can only be in $Y\cup W$. 
If $i>n$, move the token on $c_i$ to $w_i$. 
This will give the desired independent set $I^\prime$. 
If $i\leq n$, bring the token on $I$ to $c_j$ where $j\in [n+1,n+k+1]$
and $w_j$ has no token. 
This is a valid move since $c_j$ has no neighbours in $Y\cup W$ 
other than $w_j$. 
Now move this token from $c_j$ to $w_j$ to get the desired 
independent set $I^\prime$.

\emph{Case $(II)$: $I\cap C=\emptyset$ but $I\cap X\neq \emptyset$:}
Let $x_i\in I\cap X$ for some $i\in [n+k+2]$. 
If there are more than one tokens in $I$ on the vertices in $X$, 
slide the tokens on vertices $x_j$ where $j\neq i, j\in [n+k+2]$ to $y_j$. 
There can be at most $k$ tokens in $W$ 
as there is at least one token in $X$.
Now bring the token on $x_i$ to $c_\ell$ where $\ell\in [n+1,n+k+1]$ and there is no token on $w_\ell$. This is a valid move since $c_\ell$ has no neighbours in $Y\cup W$ other than $w_\ell$. Now bring this token from $c_\ell$ to $w_\ell$ to get the desired independent set $I^\prime$.

\emph{Case $(III)$: $I\cap C=I\cap X=\emptyset$ but $I\cap Y\neq \emptyset$:} 
Suppose $y_i\in I \cap Y$. Bring the token on $y_i$ to $x_i$. This move is valid because $X\cup C$ has no token. Now proceed as in Case~$(II)$.

\emph{Case $(IV)$: $I\cap C=I\cap X=I\cap Y=\emptyset$ and $w_{n+k+2}\in I\cap W$:}
Consider the set $D_1=\{v_i| \,w_i~\text{has a token and}~ i\in [n]\}$.
The cardinality of this set is at most $k$.
The set $D_1\subset V(G)$ cannot be a dominating set as $G$ does not have a dominating set of size at most $k$. 
Therefore there exists $v_j$ for some $j\in [n]$ such that ${N[v_j]\cap D_1}=\emptyset$. 
The vertex $c_j$ is not be adjacent to any vertex in $W\cap I$ 
except $w_{n+k+2}$. 
Move the token on $w_{n+k+2}$ to $c_j$ and proceed as in Case~$(I)$.

\emph{Case $(V)$: $I\cap C=I\cap X=I\cap Y=\emptyset$ and $w_{n+k+2}\notin I\cap W$:} 
Consider the set $D_2=\{v_i| \,w_i~\text{has a token and}~i\in [n]\}$. 
It will have size at most $k+1$. 
The set $D_2 \subset V(G)$ cannot be a blocking set as 
$G$ does not have a blocking set of size at most $2k-1$. 
Therefore, there exists $v_j\in D_2$ for some $j\in [n]$ 
such that $v_j$ has a private neighbour $v_\ell$ for some $\ell\in[n]$. 
Move the token on $w_j$ to $c_\ell$. 
This will not be adjacent to any other token in $W$. 
Now proceed as in Case~$(I)$.

Hence, if $(G,k)$ is a \no\ instance of the \textsc{Dominating Set} problem, then $(G^\prime,k)$ is a \yes\ instance of the \textsc{TS-Connectivity} problem.
This completes the proof of the claim.
\end{proof}

These arguments, along with the fact that reduction 
takes polynomial-time, imply that \textsc{TS-Connectivity} is {\co-\NP-hard} on chordal graphs with a maximum clique-tree degree of $4$.
%\end{proof}



\subsection{Hardness for \textsc{Token Sliding Reachability}}
In this subsection, we present a reduction used to prove \Cref{thm:np-hardness-clique-tree-degree-TS-Reach}.
% which we restate for reader's convenience. 
%\tsreachnphard*
%\begin{proof}
We present a polynomial time reduction from
\textsc{TS-Reachability} on split graphs
which is known to be \NPH~\cite{DBLP:journals/mst/BelmonteKLMOS21}.


 \begin{figure}
    \centering
    \includegraphics[width=5cm]{./images/TSR_deg}
    \caption{The input Instance of \textsc{TS-Reachability}, $G$ is a split and $C$ is a clique graph.}
    \label{TSR_deg}
\end{figure}


\textbf{Reduction}
Let $(G,I,J)$ be the input instance of \textsc{TS-Reachability} 
problem where $G$ is a split graph with a clique $C=\{c_1,c_2,\ldots,c_p\}$ 
and an independent set $U=\{u_1,u_2,\ldots,u_q\}$ and
$I,J$ are two $k$-independent sets in $G$.
The reduction constructs an instance 
$(G^\prime,I^\prime,J^\prime)$ of the \textsc{TS-Reachability} problem
as follows.
%on chordal graphs of maximum clique-tree degree $3$. 
\begin{itemize}
\item For each vertex $c_i\in V(G)$ construct a vertex $d_i$, 
and for each vertex $u_j\in V(G)$ construct a vertex 
$w_j$ where $i\in[p]$ and $j\in[q]$.
Add edges between all pairs of vertices $d_i$ and $d_j$ where 
$i\neq j\in [p]$ to construct the clique $C^{\prime}$.
Denote the set of vertices $\{w_1,w_2,\ldots,w_q\}$ by $W$. 
This will be an independent set in $G^\prime$.
\item For each edge $c_iu_j$ in $E(G)$ (where $i\in [p]$ and $j\in [q]$), add an edge $d_iw_j$.
\item Construct $q$ vertices $s_1,s_2,\ldots,s_q$. We will denote the set $\{s_1,s_2,\ldots,s_q\}$ by $S$ and refer to the vertices in $S$ as \emph{special} vertices. Now add an edge from each $s_i$ to each $d_j$ such that $i\in [q]$ and $j \in [p]$. That is, each vertex $s_i$ in $S$ should be adjacent to all the vertices $d_j$ in $C^\prime$, as shown in \cref{TSR_deg_NPH}.
\end{itemize}
\begin{figure}
    \centering
    \includegraphics[scale=0.40]{./images/TSR_deg_red_NPH}
    \includegraphics[scale=0.3]{./images/clique_tree_2.png}
    \caption{Reduction for \textsc{Token Sliding Reachability} problem parameterized by the maximum clique-tree degree and the corresponding clique-tree.}
    \label{TSR_deg_NPH}
\end{figure}
This completes the description of the reduced instance. 
\begin{lemma}
$G^\prime$ is a chordal graph with a maximum clique-tree degree of
at most $3$.
\end{lemma}
\begin{proof}
Clearly, $G^\prime$ is a split graph with clique $C^\prime$ 
and independent set $S\cup W$. 
Hence $G^\prime$ is chordal. 
We construct a clique-tree $\calT_c$ for $G^\prime$ as follows: 
Add a vertex $U_i$ in $\calT_c$ corresponding to the clique 
induced by the vertices $\{w_i,d_1,d_2,\ldots,d_p\}$ in $G^\prime$ 
for $i\in [q]$. 
Similarly, add a vertex $W_i$ in $\calT_c$ corresponding 
to the clique induced by the vertices in $N[w_i]$ in $G^\prime$ for $i\in [q]$. 
Add an edge from $U_i$ to $U_{i+1}$ for each $i\in [q-1]$. 
Add an edge from $U_i$ to $W_i$ for each $i\in [q]$. 
Thus, the maximum degree of $\calT_c$ is $3$. 
Each vertex in $V(\calT_c)$ corresponds to a maximal clique in 
$G^\prime$. 
Each vertex in $S\cup W$ appears in just one vertex (bag) of $\calT_c$,
and each vertex in $C^\prime$ appears in all the vertices $U_i$ for 
$i\in [q]$ and possibly some of the vertices $W_j$ for $j\in [q]$, 
which induces a path in $\calT_c$ with possibly some pendant edges.
Thus, for each vertex $v\in V(G^\prime)$, the set of vertices (bags) of 
$\calT_c$ which contain $v$, 
induces a connected sub-tree of $\calT_c$. 
Hence, $G^\prime$ has a clique-tree of maximum degree $3$. 
\end{proof}
\begin{lemma}
$(G,I,J)$ is a \yes\ instance of the \textsc{Token Sliding Reachability} problem
if and only if $(G^\prime, I^\prime, J^\prime)$ is a \yes\ instance of the \textsc{Token Sliding Reachability} problem.
\end{lemma}
\begin{proof}
The subgraph induced by $C^\prime \cup W$ in $G^\prime$ 
is isomorphic to $G$ where the isomorphism $\phi$ is given by 
$\phi(c_i)=d_i$ for $i\in [p]$ and $\phi(u_j)=w_j$ for $j \in [q]$.
Define $\phi(T)=\bigcup_{v\in T}{\phi(v)}$ for any subset $T$ of $G$. 
%Clearly, $\phi(T)$ will be a subset of $C^\prime \cup W$ and 
%$\phi(T)$ will be an independent set in $G^\prime$ if and only if 
%$T$ is an independent set in $G$. 
Define $I^\prime=\phi(I)$ and $J^\prime=\phi(J)$. 
Since $I$ and $J$ are independent sets in $G$, $I^\prime$ and $J^\prime$ are independent sets in $G^\prime$.

In the forward direction,
it is sufficient to show that if any independent set $I_1$ in $G$ 
can be reconfigured to an independent set $I_2$ in $G$ in 
one valid move of token sliding, then the independent set 
$\phi(I_1)$ in $G^\prime$ can be reconfigured to an 
independent set $\phi(I_2)$ in $G$ in one valid move of token sliding.
If $I_2=I_1\setminus \{u\}\cup \{v\}$, then $\phi(I_2)=\phi(I_1)\setminus \{\phi(u)\}\cup \{\phi(v)\}$. 
Thus $\phi(I_2)$ can be reconfigured from $\phi(I_1)$ by moving 
the token on $\phi(u)$ to $\phi(v)$. 
Since $\phi$ is an isomorphism, this is a valid move.

In the reverse direction, if $I^\prime=J^\prime$, 
then $I=J$ and we trivially have a \yes\ instance.
Therefore, assume $I^\prime \neq J^\prime$ and thus at least 
one token needs to be moved to reconfigure $I^\prime$ to $J^\prime$.
If at any intermediate step, a special vertex contains a token, 
then there cannot be a token in $C^\prime$ and thus no other 
token can move. 
Therefore we will need to move this token to $C^\prime$. 
Thus any sequence of valid token sliding moves in $G^\prime$, 
where the initial and final independent sets do not contain 
a special vertex, can be modified to a sequence of valid token 
sliding moves in $G^\prime$ where no intermediate configuration 
has a token on a special vertex, with the same initial and final independent 
sets.
Hence we can get a sequence of valid token sliding moves from $I^\prime$ 
to $J^\prime$ which places tokens only on vertices of $C^\prime \cup W$. 
Thus, we can get a sequence of independent sets obtained by valid token
sliding moves in $G$ by replacing each intermediate independent set 
$K$ in this sequence by $\phi^{-1}(K)$.
\end{proof}



The above two claims, along with the fact that reduction takes polynomial-time,
imply that \textsc{Token Sliding Reachability} is \NP-\hard\ on chordal graphs with maximum clique-tree degree $3$.


%\end{proof}

\section{Parameterized by leafage}
\label{sec:leafage-hardness}
% !TEX root = ./main.tex
\subsection{Hardness for \textsc{Token Sliding Connectivity}}

In this subsection, we prove \Cref{thm:W-hardness-leafage-TS-Conn}.
We present a parameter-preserving reduction that takes as input an 
instance $(G,\langle V_1,V_2,\ldots,V_k\rangle, k)$ of 
\textsc{MultiCol Ind-Set} and returns an instance 
$(G^\prime,nk)$ of \textsc{TS-Connectivity} 
where $G^\prime$ is a chordal graph with leafage $\ell = 2 \cdot k$. 
We find it convenient to describe a tree model $\mathcal{T}$ of the 
chordal graph $G^\prime$.
Recall that in this model,
each vertex of $G^\prime$ corresponds to a specified subtree of 
$\mathcal{T}$, and two vertices of $G^\prime$ are adjacent 
if and only if their corresponding subtrees have a non-empty intersection.

\subparagraph*{Structure of the model tree $\mathcal{T}$, Parking Structure, and Conditional Free Pass }

The model tree $\mathcal{T}$ consists of a central vertex $t_0$, which is the root and has $k+1$ children $t_1,t_2,\ldots,t_k$, and $t_p$.
For each $i\in[k]$, $t_i$ has two children $t_i^a$ and $t_i^b$.
The reduction subdivides each edge $t_i t_i^a$ and $t_i t_i^b$ of the tree $\mathcal{T}$ by adding $2n-1$ new vertices on this edge.
Denote the path between {$t_i^a$} and $t_i^b$ by $T_i$ for each $i \in [k]$.
We use $T_i$ to encode vertices in $V_i$.
It also subdivides $t_0 t_p$ and uses it to park the tokens as
described below.
\begin{itemize}
\item {We subdivide $t_0t_p$ of the tree $\mathcal{T}$ by adding} $nk$ blue vertices and $nk-1$
green vertices as shown in \Cref{structure}.
This allows us to park $nk$ tokens on blue vertices in this `parking structure'.
\item We add a purple vertex $b^\star$ in $G$ whose model contains
${t_0,t_1,t_2,\ldots,t_k}$ and the first vertex in the path from
$t_0$ to $t_p$, i.e., it is a star centered at $t_0$ with $k + 1$ leaves.
\end{itemize}
Vertex $b^{\star}$ will act as a bridge and allow us to take
tokens from any $t_i$ (for some $i\in [k]$) to the parking spot
provided it is possible to move a token to $t_i$.

Next, we move to the conditional free pass.
The idea is to ensure that if there is no token in $T_i$
(which is the condition), then all the tokens in $T_j$
can be moved to $t_j$, then to $b^{\star}$ and eventually
to the parking structure.
For each $i\neq j$ where $i,j\in [k]$, we add a orange vertex that has the following model.
\begin{itemize}
\item In $T_i$, add a line segment from $t_i^a$ to $t_i^b$.
In $T_j$, add a line segment from the first vertex from $t_j$ towards
$t_j^a$ to the first vertex from $t_j$ towards $t_j^b$.
Finally, connect these two line segments by an inverted $\calV$-like
structure with the top of the inverted $\calV$ corresponding to the vertex
$t_0$ as shown in \Cref{structure}.
\end{itemize}

\begin{figure}[t]
    \centering
    \includegraphics[scale=0.4]{./images/Structure.png}
    \includegraphics[scale=0.5]{./images/Conditional.png}
    \caption{(Top) Structure of the model tree. Node $t_0$ is the central vertex and $t_1, t_2, \dots, t_k$ and $t_p$ are its children.
     The structure between $t_0$ and $t_p$ is to park the tokens.
     (Bottom) The orange vertex denotes the conditional free pass
     between $T_i$ and $T_j$.
     \label{structure}}
     \vspace{-5mm}
\end{figure}
%Before moving forward, we remark that it might be possible that
%$b^{\star}$ is adjacent with some other token and hence the 
%tokens in $T_j$ can not be moved to it.
%The above claim only implies the condition on the 
%usability of the red vertex intersecting $T_j$.

\subparagraph*{Encoding the vertices of $G$:}
We add the following three types of vertices in $G^{\prime}$ to encode
$n$ vertices in $V_i$ for each $i \in [k]$.
Recall that we have subdivided the edge $t_it_i^a$ and the edge $t_it_i^b$
of the tree $\mathcal{T}$ and added $2n-1$ new vertices on each of
these edges.
\begin{itemize}
\item
Add $n$ new vertices $u_i^1,u_i^2,\ldots,u_i^n$ in $G^\prime$
corresponding to the $n$ disjoint intervals on this edge starting
from the vertex $t_i^a$ 
to the first new vertex, from the second new vertex to the third new vertex and so on as shown in
\Cref{Encoding_Vertices_TSC}. Here, the vertex $u_1$ corresponds to the interval between the vertex $t_i^a$ and the first new vertex and so on.
Also, add $n-1$ connector vertices that
connect $u_i^p$ and $u_i^{p+1}$ where $p\in [n]$.
\item
Similarly, add $n$ new vertices $w_i^1,w_i^2,\ldots,w_i^n$ in $G^\prime$
corresponding to the $n$ disjoint intervals on this edge starting
from the first new vertex to the second new vertex and so on such that $w_i^n$
corresponds to the interval from the last new vertex to the vertex $t_i^b$ as shown in \Cref{Encoding_Vertices_TSC}.
As before, add $n-1$  connector vertices for each
$i\in[k]$ which connect $w_i^p$ and $w_i^{p+1}$.
\item As shown in \Cref{Encoding_Vertices_TSC}, add $(n-1)$ pink vertices
for each $t_i$ where $i\in [k]$ as follows:
Each pink vertex $y_i^p$ intersects ${u_i^p,u_i^{p+1},\ldots,u_i^n}$
and ${w_i^1,w_i^2,\ldots,w_i^{p+1}}$ for $p\in [n-1]$.
\end{itemize}

Note that if there are $p$ tokens on the left-hand side of $T_i$
(i.e., between $t_i^a$ and $t_i$) it is safe to assume that they are
on $u_i^1,u_i^2,\ldots,u_i^p$.
If not, then we can move them to $u_i^1,u_i^2,\ldots,u_i^p$ by valid
token sliding operations without disturbing the tokens on
the rest of the graph.
Similarly, if there are $p$ tokens on the right-hand side of $T_i$
(i.e., between $t_i^b$ and $t_i$) it is safe to assume that they are on
$w_i^{n-p+1},w_i^{n-p+2},\ldots,w_i^n$.
The pink vertices are added to ensure the following two
claims hold.
\begin{restatable}{claim}{tsConnClaimNTokens}
\label{cl:ntokens}
For any $i \in [k]$, if there are $n$ tokens on $T_i$, then no token
can be moved to a pink vertex.
\end{restatable}
\begin{claimproof}
We first consider the case when all the tokens are on
${w_i^1,w_i^2,\ldots,w_i^n}$.
If we move the token on some $w_i^q$ to some pink vertex $y_i^r$
where $r\neq q$, it will be adjacent to the token on the vertex $w_i^r$.
If we move the token on some $w_i^q$ to $y_i^q$ then it
will be adjacent to the token on some $w_i^r$ where $r\neq q$.
Hence, in this case, no token can be moved to a pink vertex.
Using similar arguments, if all tokens are
on ${u_i^1,u_i^2,\ldots,u_i^n}$, then no token can be moved to a pink vertex.

Consider the case when tokens are on ${u_i^1,u_i^2,\ldots,u_i^p}$
and on ${w_i^{n-p+1},w_i^{n-p+2},\ldots,w_i^n}$
for some $0<p<n$.
Suppose we move the token on $u_i^1$ to the pink vertex $y_i^1$,
this token will be adjacent to the token on $u_i^p$ if $p\geq2$
otherwise it will be adjacent to the token on $w_i^2$.
If we move the token on $u_i^q$ where $2\leq q \leq p$
to a pink vertex $y_i^r$ where $r\leq q$,
then it will be adjacent to $u_i^1$.
If we move the token on $w_i^q$ to a pink vertex $y_i^r$
such that $r\geq q\geq p+2$, then it will be adjacent to
the token on $w_i^{p+1}$.
If $p\neq n-1$ and we move the token on $w_i^{p+1}$
to a pink vertex $y_i^r$ such that $r\neq p+1$,
then it will be adjacent to the token on $w_i^n$.
If we move the token on $w_i^{p+1}$ to the pink vertex
$y_i^{p+1}$, then it will be adjacent to the token on the vertex $u_i^p$.
Thus, no token from $T_i$ can be moved to a pink vertex via a valid token sliding move.
\end{claimproof}
\begin{restatable}{claim}{tsConnNrTokenEachSide}
\label{cl:lessthann}
For any $i \in [k]$, if there are fewer than $n$ tokens on $T_i$, then
the pink vertices intersecting $T_i$ can be used to move tokens to $b^{\star}$ (and then to the parking structure).
\end{restatable}
\begin{claimproof}
As before, we remark that the claim only implies the condition on the
usability of the pink vertices intersecting $T_i$.
It might be possible that $b^{\star}$ is adjacent to some other
token and hence the tokens in $T_i$ cannot be moved to it.
However, for the sake of clarity, we assume that this is not the case
in the rest of the proof.

Without loss of generality, we assume that there are tokens on
$u_i^1,u_i^2,\ldots,u_i^p$ and $v_i^{n-q+1},v_i^{n-q+2},\ldots,v_i^n$
where $p+q<n$.
Move the token on $u_i^p$ to $y_i^p$ which is not adjacent
to any other token.
Now move this token to the purple vertex $b^{\star}$
and then to the last vertex of the parking space.
Similarly move each $u_j$ (where $j<p$ and $j$ is the index
of the largest remaining token between $t_i^a$ and $t_i$) to $b^{\star}$.
Then move it to the last empty blue vertex.
Thus all the tokens on $u_i^1,u_i^2,\ldots,u_i^p$ can be moved
to $b^{\star}$.
Similarly all the tokens on $v_i^{n-q+1},v_i^{n-q+2},\ldots,v_i^n$
can be moved to the parking space via $b^{\star}$ by starting
from the token on the vertex with the smallest index.
\end{claimproof}

\begin{figure}[t]
    \centering
    \includegraphics[scale=0.5]{./images/yellow1.png}
    \caption{Add $(n-1)$ pink vertices for each $t_i$ where $i\in[k]$ as follows: Each pink vertex $y_i^p$ intersects $\{u_i^p,u_i^{p+1},\ldots,u_i^n\}$ and $\{w_i^1,w_i^2,\ldots,w_i^{p+1}\}$. In the figure $n$ is taken to be $5$.
    No token from $T_i$ can be moved to a pink vertex.}
    \label{Encoding_Vertices_TSC}
    \vspace{-5mm}
\end{figure}

\subparagraph*{Encoding the edges of $G$:}
Whenever there is an edge from the $p^{th}$ vertex in $V_i$ to the $q^{th}$ vertex in $V_j$ where $p,q\in[n]$ and $i,j\in [k]$, add a vertex in $G^\prime$ as described below. Refer \Cref{Encoding_edges_TSC} for an illustration.
\begin{itemize}
\item Add a horizontal line segment from the endpoint of the interval corresponding to $u_i^p$ to the endpoint of the interval corresponding to $w_i^p$ (both inclusive). Add another horizontal line segment from the endpoint of the interval corresponding to $u_j^{q+1}$ to the starting point of the interval corresponding to $w_j^q$ (both inclusive). Join these two horizontal line segments by a vertical line segment.
\item Similarly, add a horizontal line segment from the endpoint of the interval corresponding to $u_j^q$ to the starting point of the interval corresponding to $w_j^q$ (both inclusive). Add another horizontal line segment from the endpoint of the interval corresponding to $u_i^{p+1}$ to the starting point of the interval corresponding to $w_i^p$. Join these two horizontal line segments by a vertical line segment.
\end{itemize}
We denote the vertices of $G^\prime$ which are added corresponding to the edges of $G$ by $\calH$-type vertices as we add two structures similar to letter $H$.

If there are $n$ tokens in $T_i$ for some $i$, none of the tokens in $T_i$
can be moved to the parking structure using the pink vertices as shown in 
 \Cref{cl:ntokens}.
Thus in this case we will need to use the $\calH$-type vertices to move at least one of these tokens to the parking space.

\begin{restatable}{claim}{tsConnclEdge}
\label{cl:edge}
Consider an edge $e$ of $G$ from the $p^{th}$ vertex in $V_i$ 
to the $q^{th}$ vertex in $V_j$ where $p,q\in [n]$ and $i,j\in [k]$.
Suppose $n$ tokens in $T_i$ are placed on $\{u_i^1,u_i^2,\ldots,u_i^p\}$ 
and $\{w_i^{p+1},w_i^{p+2},\ldots,w_i^n\}$. 
A token in $T_i$ can be moved to $b^{\star}$ and then to the parking 
space using the $\calH$-type structure corresponding to $e$ 
if and only if $T_j$ contains $n$ tokens placed on 
$\{u_j^1,u_j^2,\ldots,u_j^q\}$ and 
$\{w_j^{q+1},w_j^{q+2},\ldots,w_j^n\}$.
\end{restatable}
\begin{claimproof}
Suppose the $n$ tokens in $T_j$ are on $\{u_j^1,u_j^2,\ldots,u_j^q\}$ 
and $\{w_j^{q+1},w_j^{q+2},\ldots,w_j^n\}$. 
None of these are adjacent to the $\calH$-type structure 
corresponding to $e$. 
The only token on $T_i$ which is adjacent to the 
$\calH$-type structure corresponding to $e$ is the 
token on $u_i^p$, hence it can move to the 
$\calH$-type structure without violating the independence.

Now we show the converse. Without loss of generality, 
suppose the $n$ tokens in $T_j$ are on 
$\{u_j^1,u_j^2,\ldots,u_j^r\}$ and $\{w_j^{r+1},w_j^{r+2},\ldots,w_j^n\}$ 
where $r\neq q$. 
If $r>q$, then the token on $u_j^r$ will 
be adjacent to the $\calH$- type structure. 
Similarly if $r<q$, the token on $w_j^q$ 
will be adjacent to the $\calH$-type structure. 
Thus, no token in $T_i$ can be moved to the $\calH$-type structure.
\end{claimproof}

\begin{figure}
    \centering
    \includegraphics[scale=.55]{./images/edge1.png}
    \caption{For $n = 5$, if there exists an edge between the $2^{nd}$
    vertex of $V_i$ and $3^{rd}$ vertex of $V_j$, 
    this is the $\calH$-type structure corresponding to this edge.
    This is the $\calH$-type structure when $n=5$ and 
    there is an edge between $2^{nd}$ vertex of $V_i$ and 
    the $3^{rd}$ vertex of $V_j$. 
    Notice that one of the tokens on $u_i^{2}$ (or $u_j^{3}$) 
    can move to $t_0$ using the cyan (or the red) vertex.}
\label{Encoding_edges_TSC} 
\vspace{-5mm}
\end{figure}

This completes the description of the reduction and the necessarily claims.
Next, we show that $(G,\langle V_1,V_2,\ldots,V_k\rangle, k)$
is a \yes\ instance of \textsc{MultiCol Ind-Set} if and only if
the instance $(G^\prime,nk)$ is a \no\ instance of 
\textsc{TS-Connectivity}. 
Let the independent set in $G^\prime$ which consists of the 
$nk$ vertices in the parking structure be denoted by $I^\star$.
We use the fact that $(G^\prime,nk)$ is a \yes\ instance of 
\textsc{TS-Connectivity} if and only if any 
independent set $I$ of size $nk$ in $G^\prime$ can be 
modified to $I^\star$ via a sequence of valid token sliding operations.

\begin{restatable}{lemma}{tsConnforward}
\label{lemma:forward-reduction-leafage}
If $(G,\langle V_1,V_2,\ldots,V_k\rangle, k)$ is a \yes\ instance of \textsc{MultiCol Ind-Set},
then $(G^\prime,nk)$ is a \no\ instance of \textsc{TS-Connectivity}.
\end{restatable}
\begin{proof}
%Suppose $(G,\langle V_1,V_2,\ldots,V_k\rangle, k)$ is a \yes\ 
%instance of \textsc{MultiCol Ind-Set}. 
Supose $X=\{v_1^{p_1},v_2^{p_2},\ldots,v_k^{p_k}\}$ be a solution of 
\textsc{MultiCol Ind-Set} where the $p_i^{th}$ vertex from 
partition $V_i$ is in the solution for every $i\in [k]$. 
We construct an independent set $I$ in $G^\prime$ of size 
$nk$ as follows:
For each $i \in [k]$, place $n$ tokens in each $T_i$ on the vertices 
$u_i^1,u_i^2,\ldots,u_i^{p_i}$ and $w_i^{p_i+1},w_i^{p_i+2},\ldots,w_i^n$. 
We show that $I^\star$ is not reachable from $I$ via a set 
of valid token sliding moves.
Since there are $n$ tokens in $T_i$, pink vertices cannot 
be used to move a token in $T_i$ to the parking structure by 
\Cref{cl:ntokens}.
Moreover, by the construction, the orange vertex intersecting both 
$T_i$ and $T_j$ can be used to move tokens in $T_j$ to $b^{\star}$
(and then to the parking structure) if and only if 
$T_i$ does not contain any token.
And hence, the orange vertex also cannot be used 
to move a token in $T_i$ to the parking structure.
Hence it will be possible to move a token to the parking structure 
only using the $\calH$-type structure. 
Using \Cref{cl:edge}, when there are $n$ tokens each in both 
$T_i$ and $T_j$ such that the tokens in $T_i$ are at 
$\{u_i^1,u_i^2,\ldots,u_i^p\}$ and 
$\{w_i^{p+1},w_i^{p+2},\ldots,w_i^n\}$ and 
the tokens in $T_j$ are at $\{u_j^1,u_j^2,\ldots,u_j^p\}$ and 
$\{w_j^{p+1},w_j^{p+2},\ldots,w_j^n\}$, 
the tokens can be moved to the parking structure using 
the $\calH$-type structure corresponding to the edge between $p^{th}$ 
vertex in $V_i$ and $q^{th}$ vertex in $V_j$. 
This is not possible because $X$ is an independent set.
Hence, no token in $I$ can be moved to the parking space and thus 
$I^\star$ cannot be reached from $I$.
\end{proof}


\begin{lemma}
\label{lemma:backward-reduction-leafage}
If $(G,\langle V_1,V_2,\ldots,V_k\rangle, k)$ is a \no\ instance of \textsc{MultiCol Ind-Set},
then $(G^\prime,nk)$ is a \yes\ instance of \textsc{TS-Connectivity}.
\end{lemma}
\begin{proof}
Suppose $(G,\langle V_1,V_2,\ldots,V_k\rangle, k)$ is a \no\ 
instance of \textsc{MultiCol Ind-Set}. 
We show that any independent set $I$ in $G^\prime$
of size $nk$ can be transformed to $I^\star$ via a 
sequence of valid sequence of token sliding operations. 

Note that a token on $b^{\star}$ can be moved to a parking structure.
Also, any token on any of the $\calH$-type structured vertex, or pink
or orange vertices can either be moved to $b^{\star}$ if possible or
can be moved to the blue vertices on $T_i$ 
(and this will only free up space for other tokens to move). 
Hence without loss of generality we can assume that all the tokens in $I$ are 
on $T_i$ for some $i\in[k]$ or in the parking structure. 
If there are some tokens in the parking structure, 
we will move them as close to $t_p$ as possible.
We now consider the following two mutually disjoint 
and exhaustive cases.

\emph{Case $I$:} There exists $i\in [k]$ such that $T_i$ contains 
at most $(n-1)$ tokens. 
This can happen either because some tokens are already in the parking 
structure or there exists $j\in [k]$ such that $T_j$ 
contains at least $n+1$ tokens. 
In this case, move all the tokens in $T_i$ to the parking structure 
using the pink vertices (as seen in \Cref{cl:lessthann}). 
After this, move all the vertices in each $T_j$ such that 
$j\neq i$ to the parking structure using the orange vertices.
Note that by the construction, the orange vertex intersecting both 
$T_i$ and $T_j$ can be used to move tokens in $T_j$ to $b^{\star}$
(and then to the parking structure) if and only if 
$T_i$ does not contain any token.
Thus we can reach from $I$ to $I^\star$.

\emph{Case $II$:} For every $i\in [k]$, 
there are exactly $n$ tokens in each $T_i$. 
Without loss of generality, we can assume that these 
tokens are on $\{u_i^1,u_i^2,\ldots,u_i^{p_i}\}$ and 
$\{w_i^{p_i+1},w_i^{p_i+2},\ldots,w_i^n\}$ for some $p_i\in[n]$.
Consider set formed by taking the $p_i^{th}$ vertex $v_i^{p_i}$
in the $i^{th}$ partition in $G$ for $i\in [k]$. 
This forms a $k$-sized multicolored subset of $V(G)$ 
and thus it cannot be an independent set as we 
have a \no\ instance of \textsc{MultiCol Ind-Set}.
Thus there must be an edge between two of the vertices in this set.
We will use the $\calH$-type structure corresponding to 
this edge in $G^\prime$ as seen in \Cref{cl:edge} 
to move a token in some $T_i$ 
(where the edge is incident on a vertex from the $i^{th}$ partition in $G$) 
to the parking structure. 
Now there are $n-1$ tokens in $T_i$ and we proceed similar to 
Case $I$ and thus we can reconfigure $I$ to $I^\star$ 
via a sequence of valid token sliding moves.
\end{proof}
The proof of \Cref{thm:W-hardness-leafage-TS-Conn} follows
from \Cref{lemma:forward-reduction-leafage},
\Cref{lemma:backward-reduction-leafage} and the facts
that the reduction takes polynomial time and the leafage of
the resulting chordal graph is $2k + 1$.
% !TEX root = ./main.tex

\subsection{Hardness of Token Sliding Reachability}
In this subsection, we prove \Cref{thm:W-hardness-leafage-TS-Reach}.
We present a parameter-preserving reduction
that takes an instance
$(G,\langle V_1,V_2,\ldots,V_k\rangle, k)$ of the
\textsc{MultiColored Clique} problem as an input and returns an instance
$(G^\prime, I,J)$ of the \textsc{TS-Reachability} problem.
%where $G^\prime$ is a chordal graph with leafage $\ell=3\cdot {{k}\choose{2}}+2k+2$. The instance $(G^\prime,I,J)$ is a \yes\ instance if and only if $(G,\langle V_1,V_2,\ldots,V_k\rangle, k)$ is a \yes\ instance.
Without loss of generality, we assume that each $V_i$
has at least one edge incident to it.
%First we describe the construction of the instance $(G^\prime,nk)$ of  \textsc{Token Sliding-Reachability} from a given instance $(G,\langle V_1,V_2,\ldots,V_k\rangle, k)$ of \textsc{Multicolored Clique}.
As before, we find it convenient to describe a chordal graph
$G^\prime$ as a tree model $\mathcal{T}$.
Recall that each vertex of $G^\prime$ is associated
with a specific subtree of
$\mathcal{T}$ and two vertices of $G^\prime$ share
an edge if and only if their corresponding subtrees have a non-empty intersection.
%Then, we specify independent sets $I$ and $J$ in $G^\prime$.
We describe the reduction 
%in steps and prove necessary claims in between
after an informal overview.

\begin{figure}[t]
\centering
\includegraphics[scale=0.5]{images/TSR_leaf_Info_Red.png}
\caption{Outline of the resulting graph $G^{\prime}$. See the informal overview of the reduction.}
\label{TSR_leafW[1]H_Red}
\vspace{-5mm}
\end{figure}

\subparagraph*{Informal Overview of the Reduction:}
\Cref{TSR_leafW[1]H_Red} shows the outline of the resulting graph $G^{\prime}$.
The green box, denoted by $V$, corresponds to the encoding of vertices in $G$.
The purple and blue boxes contain additional vertices.
Most of the vertices encoding edges in $G$,
are across purple, blue, and green boxes,
and are used to move tokens between these boxes.
The initial independent set $I$ is
$I_{\text{index}}\cup I_{\text{park}}$ and
the final independent set $J$ is
$J_{\text{index}}\cup J_{\text{park}}$.
The reduction constructs graph $G^\prime$ that satisfies
the following properties.
\begin{enumerate}
\item If there is a token in
$I_{\text{park}} \cup J_{\text{park}} \cup K_{\text{index}}$,
then one can not move a token out of $I_{\text{index}}$.
\item {All} the tokens in $I_{\text{index}}$ can be
moved to $K_{\text{index}}$ via $V$ only if
tokens in $V$ (which are moved from $I_{\text{park}}$)
corresponds to a multicolored clique in $G$.
\item Tokens in $K_{\text{index}}$ impose restrictions on movements of tokens in $V$.
\end{enumerate}
These properties imply that to move \emph{all} the tokens
from $I$ to $J$, one needs to move the tokens in the following phases:
In the first phase, one needs to move
all the tokens from $I_{\text{park}}$
to $V$ in such a way that their position corresponds to
a multicolored clique in $G$.
In the second phase, one needs to move all the tokens from
$I_{\text{index}}$ to $K_{\text{index}}$.
This move \emph{locks} the tokens at their places in $V$
and the `clique'-configuration is maintained throughout the movements
of the tokens.
As all the tokens are now out of $I_{\text{index}}$,
one can move all the tokens from
$K_{\text{index}}$ to $J_{\text{index}}$ in the third phase.
Finally, as there are no tokens in $K_{\text{index}}$ now,
one can move all the tokens in $V$ to $J_{\text{index}}$
in the fourth phase.

\begin{figure}[t]
\centering
\includegraphics[scale=0.4]{images/TS_Reach_reduction_reach_tree_structure.png}
\caption{The tree model used in the reduction to prove \Cref{thm:W-hardness-leafage-TS-Reach}. The central vertex is $t_0$.
The vertices in the green rectangles are used to encode vertices
in $V(G)$.
The purple rectangles are parking structures for the initial and final
independent set.
The vertices in the blue rectangle are index vertices, making
all of them can only be moved if placements of tokens in
the green rectangle corresponds to a clique in $V(G)$.}
\label{fig:TS-Reach-reduction-reach-tree-structure}
\vspace{-5mm}
\end{figure}

\subparagraph*{Structure of the model tree $\mathcal{T}$:}
To begin with, the model tree $\mathcal{T}$ consists of
a central vertex $t_0$, which is the root and has
$k$ children $t_1,t_2,\ldots,t_k$.
See \Cref{fig:TS-Reach-reduction-reach-tree-structure} for an illustration.
For each $i\in[k]$, $t_i$ has two children $t_i^a$ and $t_i^b$.
The reduction subdivides each edge $t_it_i^a$ and $t_it_i^b$ of the tree $\mathcal{T}$ by adding $2n-1$ new vertices on this edge.
Denote the path between $t_i^a$ and $t_i^b$ by $T_i$ for each $i \in [k]$.
We use $T_i$ to encode vertices in $V_i$.
The collection of vertices on the subtrees of the tree induced by the union of
the vertex sets $\{t_1,t_2,\ldots,t_k\}$, $\{t^a_1,t^a_2,\ldots,t^a_k\}$,
and $\{t^b_1,t^b_2,\ldots,t^b_k\}$ is denoted by $V$.
These vertices encode the vertices $V(G)$ of the input graph $G$.
Add two more children of $t_0$ and label them as $t_P^I$ and $t_P^J$.
We will use $t_P^I$ to park the tokens at the beginning and
$t_P^J$ to park the tokens at the end.

For every $k' \in [\binom{k}{2}]$, add a child of $t_0$ labelled $t^I_{k'}$
and a child of $t_0$ labelled $t^J_{k'}$.
For every $i < j \in [k]$, a child of $t_0$ labelled $t^K_{ij}$.
{We denote $I^{\calT }_{\text{index}} := \{t^I_{k'} \mid\ k' \in [\binom{k}{2}]\} $,
$J^{\calT}_{\text{index}} := \{t^J_{k'} \mid\ k' \in [\binom{k}{2}]\}$ but
$K^{\calT }_{\text{index}} := \{t^K_{ij} \mid\ i < j \in [k]\}$.}
{We highlight the two different ways of indexing these sets.}
A vertex corresponding to $t^K_{ij}$ is associated with
a collection of edges with one endpoint in $V_i$ and another endpoint
in $V_j$.
Whereas, the vertices in $I^{\calT }_{\text{index}}$ and 
$ J^{\calT}_{\text{index}}$
are a collection of $\binom{k}{2}$-many vertices each.
Subdivide each edge of the form $t^I_{k'}t_0$ and $t^J_{ij}t_0$
by adding intermediate vertices $\ell^I_{k'}$ and $\ell^J_{k'}$, respectively.
Similarly, subdivide each edge of the form $t^K_{ij}t_0$ by adding an
intermediate vertex $\ell^K_{ij}$.

\subparagraph*{Encoding the vertices of $G$:}
We add the following two types of vertices in $G^{\prime}$
to encode $n$ vertices in $V_i$ for each $i \in [k]$.
Recall that we have subdivided edge $t_it_i^a$ and edge $t_it_i^b$
of the tree $\mathcal{T}$ and added $2n-1$ new vertices
on each of these edges.
\begin{itemize}
\item Add $n$ new vertices $u_i^1,u_i^2,\ldots,u_i^n$
in $G^\prime$ corresponding to the $n$ disjoint intervals
on this edge, starting from the interval from $t_i^a$
to the first new vertex as shown in \Cref{fig:Encoding_Ver_TSR}.
Also, for every $p \in [n-1]$, add a connector vertex that connects
$u_i^p$ and $u_i^{p+1}$.
\item  Add $n$ new vertices $w_i^1,w_i^2,\ldots,w_i^n$
in $G^\prime$ corresponding to the $n$ disjoint intervals
on this edge, starting from the first new vertex to the interval
before $t_i^b$ as shown in \Cref{fig:Encoding_Ver_TSR}.
As before, for every $p \in [n-1]$, add a connector vertex
that connects $w_i^p$ and $w_i^{p+1}$.
\end{itemize}
Note that if there are $p$ tokens on the left-hand side of $T_i$
(i.e., between $t_i^a$ and $t_i$) it is safe to assume that they are on
$u_i^1,u_i^2,\ldots,u_i^p$. If not, then we can move them to
$u_i^1,u_i^2,\ldots,u_i^p$ by valid token sliding operations
without disturbing the tokens on the rest of the graph.
Similarly, if there are $p$ tokens on the right hand side of $T_i$
(i.e., between $t_i^b$ and $t_i$) it is safe to assume that
they are on $w_i^{n-p+1},w_i^{n-p+2},\ldots,w_i^n$.
If tokens are placed on $u_i^1,u_i^2,\ldots,u_i^p$ and
$w_i^{p+1},w_i^{p+2},\ldots,w_i^n$, it corresponds to
selecting the $p^{th}$ vertex in $V_i$ as a part of the clique.

\begin{figure}[t]
\centering
\includegraphics[scale=0.5]{images/Encoding_Ver_TSR.png}
\caption{Adding vertices corresponding to vertices in $V_i$ to $G'$ when
$n = 5$. Note that on the left side, the indexing starts from the
vertex farthest from $t_i$, whereas on the right side, the indexing
starts from the vertex closest to $t_i$. \label{fig:Encoding_Ver_TSR}
\vspace{-5mm}}
\end{figure}


\subparagraph*{Encoding the edges of $G$:}
Whenever there is an edge from the $p^{th}$ vertex in $V_i$ to the
$q^{th}$ vertex in $V_j$ where $p, q\in[n]$ and $i\neq j\in [k]$,
add a red vertex in $G^\prime$ as described below.
%See \Cref{red-2.4} for an illustration.
\begin{itemize}
\item Add a line segment starting from $u_i^{p+1}$
(including the starting point of the interval corresponding to $u_i^{p+1}$)
to the endpoint of the interval $w_i^p$
(including the endpoint of the interval corresponding to $w_i^p$).
\item Similarly, add a line segment starting from $u_j^{q+1}$
(including the starting point of the interval corresponding to $u_j^{q+1}$)
to the starting point of the interval $w_j^q$
(including the endpoint of the interval corresponding to $w_j^q$).
\item Add a horizontal line segment from $t_0$ to $t_{ij}^K$
in $K_{\text{index}}$.
\item Connect these three horizontal line segments by
a subpath from $t_i$ to $t_j$ (containing $t_0$).
\end{itemize}
See \Cref{Encoding_edge_TSR}.
We denote the vertices of $G^\prime$ which are added corresponding
to the edges of $G$ by $\calH$-type vertices as we add two structures
similar to a horizontal letter $H$ (with an extra line segment joining the
centre of this structure to $t_0$) corresponding to each edge of $G$.
We denote the $\calH$-type vertex which corresponds to the edge in $G$
corresponding to the edge from the $p^{th}$ vertex in $V_i$ and
the $q^{th}$ vertex in $V_j$ by $r_{ij}^{pq}$.
We use these vertices to move the tokens from $I_{\text{index}}$
to $K_{\text{index}}$ and from $K_{\text{index}}$ to $J_{\text{index}}$.
We make the following modifications to make the first type of movement possible.
Recall that we have subdivided each edge of the
form $t^K_{ij}t_0$ by adding an intermediate vertex $\ell^K_{ij}$.
\begin{itemize}
\item
Extend the sub-tree corresponding to each red vertex
$r^{pq}_{ij}$ till this intermediate vertex
$\ell^K_{ij}$ (or $\ell^K_{ji}$ if $i>j$).
\end{itemize}

\begin{figure}
\centering
\includegraphics[scale=0.45]{images/Encoding_edge_TSR.png}
\caption{Encoding the edge between $3^{rd}$ vertex in $V_j$ and $2^{nd}$ vertex of $V_i$.}
\label{Encoding_edge_TSR}
\vspace{-5mm}
\end{figure}

\begin{figure}[t]
\centering
\includegraphics[scale=0.45]{images/park_struc_TSR.png}
\caption{Parking Structure; $nk$ many tokens can be parked in $J_{\text{park}}$ and $I_{\text{park}}$ each.}
\label{park_struc_TSR}
\vspace{-5mm}
\end{figure}

\subparagraph*{Auxiliary vertices:}
We start with the parking structure.
We add two parking structures to the graph
viz one by subdividing $t_0t^I_p$ and another one by
subdividing $t_0t^J_p$.
\begin{itemize}
\item We add a path in $G^\prime$ with $nk$ blue vertices
(disjoint intervals) between $t_0$ and $t_P^J$ as shown \Cref{park_struc_TSR}.
We connect these vertices using $(nk-1)$ green vertices as shown.
This allows us to park $nk$ many tokens in the
parking structure $J_\text{park}$.
Similarly, add a path with $nk$ blue vertices (disjoint intervals)
connected by $nk-1$ green vertices between $t_0$ and $t_P^I$.
\item We add a purple vertex $b^\star$ whose model is
$\{t_0,t_1,t_2,\ldots,t_k\}$.
This vertex acts as a bridge that takes tokens from the
$T_i$ part (for some $i\in [k]$) if we are able to get a token to $t_i$.
\end{itemize}
Define $I_{\text{park}}$ as the collection of the vertices
corresponding to the subpath of $\mathcal{T}$ from
$t_P^I$ to $t_0$ (including $t_P^I$ but excluding $t_0$).
Similarly, $J_{\text{park}}$ denotes the vertices corresponding
the subpath of $\mathcal{T}$ from $t_P^J$ to $t_0$
(including $t_P^J$ but excluding $t_0$).
Next, we add vertices in $I_{\text{index}}, J_{\text{index}}$ and
$K_{\text{index}}$.
\begin{itemize}
\item For every $k' \in [\binom{k}{2}]$,
add two vertices in $G^{\prime}$ whose models are
$\{t^I_{k'}\}$ and $\{\ell^I_{k'}\}$, respectively.
These are demonstrated by the dark blue and
orange intervals
in \Cref{fig:TS-Reach-auxillary-vertices}.
\item For every pair $i < j \in [k]$, add a vertex in $G^{\prime}$,
denoted $g_{ij}$, whose model is $\{t^K_{ij},\ell^K_{ij}\}$.
These are shown by the green vertices in \Cref{fig:TS-Reach-auxillary-vertices}.
\item For every $k' \in [\binom{k}{2}]$,
add a vertex, denoted by $p_{k'}$, in $G^{\prime}$
whose model is $\{t^J_{k'},\ell^J_{k'}\}$.
These are depicted by the purple vertices in
\Cref{fig:TS-Reach-auxillary-vertices}.
\end{itemize}
We now define $I_{\text{index}}:=\{\ell^I_{k^\prime}\mid k^\prime \in [\binom{k}{2}]\}$,
$J_{\text{index}}:= \{p_{k^\prime}\mid k^\prime \in [\binom{k}{2}] \}$, and
$K_{\text{index}}:=\{g_{ij}\mid i<j \in [k]\}$.
\begin{figure}
\centering
\includegraphics[scale=0.45]{images/auxilary_ver_TSR.png}
\caption{Adding the Auxillary vertices}
\label{fig:TS-Reach-auxillary-vertices}
\vspace{-5mm}
\end{figure}
Next, we describe the construction of the
structure, which restricts the movement of tokens.
\begin{itemize}
\item Add the cyan vertex as in \Cref{fig:TS-Reach-auxillary-vertices}
which corresponds to a  star-like structure
centered at $t_0$ and covering all the vertices on the path from
$t_0$ to $t_P^I$ and on the path from $t_0$ to $t_P^J$.
This vertex also covers the paths from $t_0$ to $\ell^J_{k^\prime}$
for $k^\prime \in [\binom{k}{2}]$.
We refer to this vertex as the first \emph{choke-type vertex}
and denote it by $\calC_1$.
\end{itemize}
Finally, we describe the additional vertices in $G$ which
will facilitate the movement of the tokens from
$K_{\text{index}}$ to $J_{\text{index}}$
in case of a \yes\ instance of \textsc{Multicolored Clique}.
%Once this shift is done, then all the tokens in $V$
%can move to $J_{\text{park}}$,
%completing the reconfiguration of tokens from $I$ to $J$.
\begin{itemize}
\item Add a connector vertex $c^{ij}$ corresponding
to the sub-tree formed by $\ell^K_{ij},t_0$ and $\ell^J_{k^\prime}$
for each $i<j \in [k]$ and $k^\prime \in [\binom{k}{2}]$. This is shown in
 \Cref{fig:TS-Reach-auxillary-vertices}.
\end{itemize}

This completes the description of the construction of graph $G^\prime$.
The reduction sets the initial independent set $I$ is
$I_{\text{index}}\cup I_{\text{park}}$ and
the final independent set $J$ is
$J_{\text{index}}\cup J_{\text{park}}$.
Finally, it returns $(G^\prime, I, J)$ is an instance of \textsc{TS-Reachability}.
In the next two lemmas, we prove that the reduction is safe.

\begin{restatable}{lemma}{tsReachForward}
\label{lemma:forward-reduction-leafage-TS-Reach}

If $(G,\langle V_1,V_2,\ldots,V_k\rangle, k)$ is a \yes\ instance of \textsc{MultiCol Clique},
then $(G^\prime, I, J)$ is a \yes\ instance of \textsc{TS-Reachability}.
\end{restatable}
\begin{proof}
Suppose $G$ has a multicolored clique consisting of the $p_i^{th}$
vertex from each $V_i$.
%As mentioned in the informal overview,
%in the first phase, one needs to move
We move all the tokens from $I_{\text{park}}$
to $V$ in such a way that their position corresponds to
a multicolored clique in $G$.
Formally, we move the vertices in $I_\text{index}$ to $V$ such that each
$T_i$ has tokens on $\{u_i^1,u_i^2,\ldots,u_i^{p_i}\}$ and
$\{w_i^{p_i+1},w_i^{p_i+2},\ldots,w_i^n\}$.
We say an $\calH$-type vertex is \emph{usable} (to move tokens)
if it is not adjacent to any token.
Consider an edge incident on $q^{th}$ vertex in $V_i$
and the $\calH$-type vertex added to $G^{\prime}$ to encode it.
By the construction, the interval starting from $u_i^{q+1}$ to $w_i^q$
(both inclusive) is a part of the model of $\calH$-type vertex.
This implies edges whose both endpoints are in the multicolored clique,
have no token adjacent to them even when some are moved to $V$.
Hence, the edges in the multiclique mentioned above
are always usable to move (remaining) tokens from $I$ to $V$.
And hence, we can move all $k \cdot n$ tokens from $I$ to $V$.

Using the same argument as above,
one can move all the tokens from $I_{\text{index}}$ to $K_{\text{index}}$.
Formally, consider an index $k' \in \binom{k}{2}$ and $i, j \in [k]$.
One can move a token on $\ell_{k'}^I$ in $I_{index}$
to $g_{ij}^K$ in $K_{index}$ by moving it to $b^*$
and then using the red $\calH$-type vertex corresponding to
the edge in multicolored clique across the vertex in $V_i$ and $V_j$.

In the third phase, one can move all ${k}\choose{2}$ tokens in
$K_\text{index}$ to $J_\text{index}$ by first moving a token
on $g^{ij}$ to $c^{ij}$ and then to $p^{k^\prime}$.
Note that this is possible as before starting this phase,
all the tokens are out of $I_{\text{index}}$.

Finally, as there are no tokens in $K_{\text{index}}$,
one can move all the tokens in $V$ to $J_{\text{index}}$
in the fourth phase.
If there are ${k}\choose{2}$ tokens in $J_\text{index}$ and
$nk$ tokens in $V$, all the tokens in $V$ can be brought to $J_\text{park}$.
Each token in $V$ (in some $T_i$) can be brought to the vertex
$b^*$ and then to the last unoccupied blue vertex in $J_\text{park}$.
%Thus all the tokens in $V$ can be brought to $J_\text{park}$.
Hence, we can move all the tokens in $K_{\text{index}}$ to $J_{\text{index}}$
and then all the tokens in $V$ to $J_{\text{park}}$.
Thus, the new independent set occupied by the tokens is $J$
which completes the proof.
\end{proof}

\begin{lemma}
\label{lemma:backward-reduction-leafage-TS-Reach}
If $(G^\prime, I, J)$ is a \yes\ instance of \textsc{TS-Reachability}
then $(G,\langle V_1,V_2,\ldots,V_k\rangle, k)$ is a \yes\ instance of \textsc{MultiCol Clique}.
\end{lemma}
\begin{proof}
We first prove some claims regarding the possible movements
of the tokens.


\begin{restatable}{claim}{tsReachclChoke}
\label{cl:choke1}
If there is a token in 
$I_{\text{park}} \cup J_{\text{park}} \cup K_{\text{index}}$,
then one can not move any token out of $I_{\text{index}}$.
\end{restatable}
\begin{claimproof}
If a token from $I_\text{index}$ must be moved outside
$I_\text{index}$, then it must be first brought to an orange vertex.
The orange vertex is adjacent to $\calC_1$.
Therefore, no vertex adjacent to $\calC_1$ can already
contain a token.
Since all the vertices in
$I_{\text{park}}\cup J_{\text{park}} \cup J_{\text{index}}$
are adjacent to $\calC_1$, if
$I_{\text{park}}\cup J_{\text{park}} \cup J_{\text{index}}$
has at least one token, then all the tokens in
$I_\text{index}$ must be on the dark blue vertices
no token can be moved to an orange vertex.
Hence no token can be moved outside $I_\text{index}$.
\end{claimproof}

\noindent Hence, to move the vertices from $I_{\text{index}}$,
we must move the vertices out of $I_{\text{park}}$.
Moreover, we can not move these tokens from $I_{\text{park}}$ 
to $J_{\text{park}}$ or to $J_{\text{index}}$ 
before all the tokens out of $I_{\text{index}}$.
We are also not be able to move any of these tokens to
$K_\text{index}$ as once a token lands on
$g_{ij}\in K_\text{index}$, the red vertex $r_{ij }$
cannot be used for any other token as it is adjacent to $g_{ij}$.
This implies that no token can be brought inside or outside
$T_i\cup T_j$.
Thus some of the tokens in $I_{\text{index}}$ will get stuck
if all of them are not moved to $K_\text{index}$ in the first phase.
Hence, one needs to move all the tokens in $I_{\text{park}}$
to $V$.
The next two claims argue about the movements
of such tokens.

\begin{restatable}{claim}{tsReachclAtmostNtokens}
\label{cl:atmostn+1tokens} 
For any $i \in [k]$,
{for an $\calH$-type vertex adjacent to some $T_i$ to remain
usable, one can move at most $n$ tokens to $T_i$.}
\end{restatable}
\begin{claimproof}
Suppose one has moved $n + 1$ tokens to  $T_i$.
Without loss of generality, we can assume that the tokens
are on $u_i^1,u_i^2,\ldots,u_i^p$ and
$w_i^n,w_i^{n-1},\ldots,w_i^p$
i.e., $p$ vertices on the left hand side of $T_i$
and $n+1-p$ vertices on the right hand side of $T_i$
where $p \in [n]$.
%Recall that there is at least one edge adjacent to a vertex in $V_i$ in $G$.
%Let the index of this vertex be $q \in [n]$.
Consider an edge incident on $q^{th}$ vertex in $V_i$
and the $\calH$-type vertex added to $G^{\prime}$ to encode it.
By the construction, the interval starting from $u_i^{q+1}$ to $w_i^q$
(both inclusive) is a part of the model of $\calH$-type vertex.
Define $S_p = \{u_i^1,u_i^2,\ldots,u_i^p\}
\cup \{w_i^n,w_i^{n-1},\ldots,w_i^p\}$.
%For any $p\in [n]$, the intersection set $S$ of
%$\{u_i^1,u_i^2,\ldots,u_i^p,w_i^n,w_i^{n-1},\ldots,w_i^p\}$
%and $\{u_i^{q+1},u_i^{q+2},\ldots,u_i^n,w_i^1,w_i^2,\ldots,w_i^q\}$
%will be of size at least $1$.
If $p\geq q+1$, then $S_p$ contains $u_i^{q+1}$.
Otherwise, $p\leq q$ and $S_p$ contains $w_q$.
Thus, if there are $n+1$ tokens in $T_i$, then no $\calH$-type vertex
adjacent to $T_i$ is usable.
\end{claimproof}



\noindent By the construction, at least one $\calH$-type vertex is needed
to move tokens from $t_i$ to other vertices in $T_i$.
Hence, the above claim implies that
one can move at most $n + 1$ tokens to $T_i$.

\begin{restatable}{claim}{tsReachclUsable}
\label{cl:usable}
If $T_i$ contains $n$ tokens, then $\calH$-type vertex corresponding to an edge adjacent to the $q^{th}$ vertex in $V_i$ is be usable if and only 
if it has no tokens outside $T_i$ and the tokens in $T_i$ are on 
$\{u_i^1,u_i^2,\ldots,u_i^q,w_i^n,w_i^{n-1},\ldots,w_i^{q+1}\}$.
\end{restatable}
\begin{claimproof}
Without loss of generality, we can assume that the tokens are present on
$u_i^1,u_i^2,\ldots,u_i^p$ and $w_i^n,w_i^{n-1},\ldots,w_i^{n-p+1}$
i.e., $p$ vertices on the left hand side of $T_i$ and $n-p$ vertices on the
right hand side of $T_i$ where $p\in [n]$.
If there is an edge adjacent to the $q^{th}$ vertex in $V_i$ in $G$,
then the $\calH$-type vertex corresponding to this edge
intersects $T_i$ at $\{u_i^{q+1},u_i^{q+2} \ldots,u_i^n\} \cup \
\{w_i^1,w_i^2,\ldots,w_i^q\}$.
If $p=q$, then there are $n$ tokens on
$\{u_i^1,u_i^2,\ldots,u_i^q\} \cup
\{w_i^n,w_i^{n-1},\ldots,w_i^{q+1}\}$,
then this has an empty intersection with
$\{u_i^{q+1},u_i^{q+2},\ldots,u_i^n\} \cup \{w_i^1,w_i^2,\ldots,w_i^q\}$
and thus the corresponding $\calH$-type vertex is usable
if there is no token outside $T_i$ adjacent to it.
If $p\neq q$, the intersection set $S$ of the tokens in $T_i$, i.e.,
$\{u_i^1,u_i^2,\ldots,u_i^p\} \cup
\{w_i^n,w_i^{n-1},\ldots,w_i^{n-p+1}\}$
and the vertices of $T_i$ which are adjacent to the $\calH$-type
vertex corresponding to the edge incident on the $q^{th}$ vertex
of $V_i$ given by
$\{u_i^{q+1},u_i^{q+2} \ldots,u_i^n\} \cup \{w_i^1,w_i^2,\ldots,w_i^q\}$
will be of size at least one.
If $p<q$, then $S$ will contain $w_q$ and if $p>q$,
then $S$ will contain $u_i^{q+1}$.
Therefore the $\calH$-type vertex cannot be usable.
\end{claimproof}

\noindent Now all the tokens in $I_{\text{index}}$ can be moved to
$K_{\text{index}}$ if and only if all the tokens in
$I_{\text{park}}$ are moved to $V$ such that
${k}\choose{2}$ of the red ($\calH$-type) vertices are usable
(i.e., do not have any token adjacent to them).
The red ($\calH$-type) vertices corresponding to the edges between the 
$p^{th}$ vertex in $V_i$ and the $q^{th}$ vertex in $V_j$ are usable 
for $i<j$. 
Thus we have ${k}\choose{2}$ red ($\calH$-type) vertices which are usable.


If there are ${k}\choose{2}$ red ($\calH$-type) vertices which are usable,
then each distinct pair of $i\neq j\in [k]$ must contribute one
usable red vertex.
This is because any arrangement of $n+1$ tokens on $T_i$ cannot
contribute a usable $\calH$-type vertex by \Cref{cl:atmostn+1tokens},
and an arrangement of $n$ tokens on $T_i$ can contribute at most one
usable $\calH$-type vertex.
A pair of distinct indices $i,j\in [k]$ can contribute a usable
$\calH$ type vertex $r^{pq}_{ij}$ if and only if there are tokens
on $\{u_i^1,u_i^2,\ldots,u_i^{p}\}$ and
$\{w_i^{p+1},w_i^{p+2},\ldots,w_i^n\}$ in $T_i$;
and $\{u_j^1,u_j^2,\ldots,u_j^{q}\}$ and
$\{w_j^{q+1},w_j^{q+2},\ldots,w_i^n\}$ in $T_j$.
The existence of $r^{pq}_{ij}$ implies the existence of
an edge between the $p^{th}$ vertex of $V_i$ and
the $q^{th}$ vertex of $V_j$ in $G$.
Thus the existence of ${k}\choose{2}$ red ($\calH$-type)
vertices that are usable imply the existence of a 
multicolored clique in $G$.
\end{proof}

\Cref{lemma:forward-reduction-leafage-TS-Reach} and
\Cref{lemma:backward-reduction-leafage-TS-Reach}
implies that the reduction is safe.
This, along with the fact that the reduction works in polynomial time
and the leafage $\ell$ of the chordal graph $G^\prime$
is $3\cdot {{k}\choose{2}}+2k+2$
implies the 
proof of \Cref{thm:W-hardness-leafage-TS-Reach}.

\section*{Conclusion}
This paper aims to enhance our understanding of the computational complexity of computing various Shapley value variants. We found that for various ML models --- including decision trees, regression tree ensembles, weighted automata, and linear regression --- both local and global interventional and baseline SHAP can be computed in polynomial time under HMM modeled distributions. This extends popular algorithms, such as TreeSHAP, beyond their empirical distributional scope. We also establish strict complexity gaps between the various SHAP variants (baseline, interventional, and conditional) and prove the intractability of computing SHAP for tree ensembles and neural networks in simplified scenarios. Overall, we present SHAP as a versatile framework whose complexity depends on four key factors: \begin{inparaenum}[(i)] \item model type, \item SHAP variant, \item distribution modeling approach, \item and local vs. global explanations\end{inparaenum}. We believe this perspective provides deeper insight into the computational complexity of SHAP, paving the way for future work.




%We believe that our framework provides a more intricate understanding of SHAP computation complexity across different models, distributions, and variants, paving the way for further research.

Our work opens promising directions for future research. First, expanding our computational analysis to other SHAP-related metrics, such as asymmetric SHAP~\citep{frye20} and SAGE~\citep{covert2020understanding}, would be valuable. Additionally, we aim to explore more expressive distribution classes and relaxed assumptions beyond those in Section \ref{sec:tractable} while maintaining tractable SHAP computation. Finally, when exact computation is intractable (Section \ref{sec:intractable}), investigating the approximability of SHAP metrics through approximation and parameterized complexity theory~\citep{downey2012parameterized} is an important direction.

%Our work opens several promising avenues for future research on the computational properties of explainable AI methods, with a particular focus on SHAP. First, it would be interesting to broaden the computational analysis conducted in this work to include other popular SHAP-related metrics in the literature, such as asymmetric SHAP \cite{frye20} and SAGE \cite{covert2020understanding}. Also, in the future, we aim to explore more expressive distribution classes and relaxed distributional assumptions—extending beyond those examined in Section \ref{sec:tractable} —that still yield tractable SHAP computation. Finally, when exact computation proves intractable (Section \ref{sec:intractable}), it is worthwhile to theoretically investigate the question of the approximability of computing the SHAP metrics across various configurations, through the lens of approximation and parametrized complexity theory \cite{arora2009computational}.

%This paper aims to deepen our understanding of the computational complexity involved in obtaining different Shapley value variants. We found that for a variety of ML models, including decision trees, tree ensembles for regression, weighted automata, and linear regression models — computing both local and global interventional and baseline SHAP can be done in polynomial time when distributions are modeled by HMMs. This extends the distributional scope of popular algorithms like TreeSHAP, which is limited to empirical distributions. Additionally, we demonstrate a strict complexity gap between SHAP variants, showing that interventional and baseline SHAP can be strictly easier to compute than conditional SHAP. Despite these positive results, we uncovered intractability for various SHAP variants in neural networks and tree ensembles. Finally, we provided generalized complexity relations across SHAP variants. We believe that our framework offers a deeper understanding of the complexity involved in computing SHAP across various variants, models, distributions, as well as in both local and global computations, laying the groundwork for future research.

%\bibliographystyle{abbrvnat}

\bibliography{references}



\end{document}
