\textbf{Completion Bias.} Users selected the first completion 86\% of the time, revealing a completion order bias. We investigated whether the bias was due to users instinctively pressing Tab for the first completion, as it requires a simpler keystroke than Shift-Tab. Analysis of decision times revealed that users spent a median of 6 seconds selecting the first completion, indicating this action was not automatic. However, users still took longer (9 seconds) to select the second completion, suggesting they deliberated more between the two options.

\begin{figure}[h]
\centering
\includegraphics[width=0.45\textwidth]{figures/similarity_decision_time.pdf}
\caption{Completion similarity vs. decision time, grouped by selection of the first or second code completion.}
\label{fig:similarity_decision}
\end{figure}

We hypothesized that the extended deliberation resulted from users comparing completion differences. To validate this, we evaluated code similarity between completion pairs using the Levenshtein ratio. The dataset was refined by removing outliers (identical or very dissimilar completions) and excluding comments to minimize the impact of documentation differences. As shown in Figure \ref{fig:similarity_decision}, decision time increased with completion similarity for the second completion, indicating greater deliberation for highly similar completions. This trend was absent for the first completion, suggesting this extra deliberation did not occur for these cases.

