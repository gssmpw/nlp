\section{System Deployment}\label{sec:deployment}

\textbf{Deployment Details.} The \systemName extension is advertised in online open-source communities and made available on the VSCode extension store, where it is free to download. 
Similar to the set-up employed by~\citet{chiang2024chatbot} and \citet{lu2024wildvision}, participants are not compensated for using the extension, as in a traditional user study, but instead receive free access to state-of-the-art models.
In addition to logging all preference judgments made by users of \systemName, we also log the latency of each model response, the type of file the user is writing, the prefix and suffix length (characters and tokens), each completion length, which model was in the top versus bottom position, and a unique userID---all of which allows users to utilize the extension without revealing the content of what the user is working on.
Given the sensitive nature of programming, we established clear privacy controls to give users the ability to restrict our access to their data.
Depending on privacy settings, we also collect the user's code context and model responses.
Appendix~\ref{appendix:data_release} provides a copy of the specific user instructions and privacy guidelines.
Our data collection process was reviewed and approved by CMU's Institutional Review Board.


\textbf{Data collection process.} We select 10 state-of-the-art models to balance a set of open and commercial models, as well as generalist and code-specific models.
In latter analysis, we refer to LLMs by shortened names to conserve space: please check Table~\ref{tab:model-comparison} for full model names.
Across \userCount~users, we have served over~\completions suggestions and collected \sampleCount~votes over the course of 3 months.
Overall, we find that all models received between 2-5K votes, providing sufficient coverage. 
In general, the median time to vote---the time taken after the completion is displayed to the user---was 7 seconds, suggesting that users did not accept all suggestions immediately and considered both completions.
A more in-depth overview of data analysis is in Appendix~\ref{appdx:data_analysis}.

\textbf{Data Release.} Despite giving users full control over their privacy, we take a conservative approach to data release given the potential sensitivity of coding data.
To demonstrate the type of code users write using \systemName, we also release a hand-curated set of examples that contain the prefix, suffix, and both completions in the GitHub repository.
This portion of the dataset captures a variety of downstream tasks and languages---Appendix~\ref{appdx:data_analysis} also shows multiple examples.
Two authors carefully checked this set of examples to ensure the code also contained no sensitive information or personally identifiable information. 
We intend to continue a slow release in the future.













