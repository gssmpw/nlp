\subsection{Baseline model architecture}\label{sec:Overview:FeedRanking}
The baseline Feed ranking model \cite{LiRank_paper} employs a point-wise ranking approach, predicting multiple action probabilities including like, comment, share, vote, long dwell, and click for each <member, candidate post> pair. These predictions are linearly combined to generate the final post score. A neural network with a multi-task learning (MTL) architecture generates these probabilities using two towers: the Click Tower, which predicts the probabilities of Click and Long Dwell \cite{dwell_linkedin}, and the Contributions Tower. The Contributions Tower predicts multiple labels, including Likes, Comments, Shares, and Votes. Additionally, it includes an aggregated "Contribution" label that represents overall member engagement. This aggregated label assigns equal weight to actions such as Likes, Comments, and Shares, and is set to 1.0 if any of these actions occur. Throughout the paper, we report the AUC for Contributions, Click, and Long Dwell as our evaluation metrics. Both towers use the same set of dense and sparse features normalized based on their distribution\cite{AirBnB_Search}, and apply multiple fully-connected layers. Sparse ID embedding features \cite{LiRank_paper} are transformed into dense embeddings \cite{DLRM19} through lookup in embedding tables of Actor (the author of the post) and Hashtag Embedding Table as in Figure ~\ref{fig:FeedContributionTower}. For reproducibility we provided a diagram showing how different architectures are connected together into a single model in \cite{LiRank_paper}.

\begin{figure}
  \centering
  \includegraphics[height=5cm, width=7cm]{figures/FeedContributionTower.png}
  \caption{Contribution tower of the main Feed ranking model}
  \label{fig:FeedContributionTower}
  %\vspace{-1.5em}
\end{figure}