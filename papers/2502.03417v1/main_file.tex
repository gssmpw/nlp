%%
%% This is file `sample-authordraft.tex',
%% generated with the docstrip utility.
%%
%% The original source files were:
%%
%% samples.dtx  (with options: `authordraft')
%% 
%% IMPORTANT NOTICE:
%% 
%% For the copyright see the source file.
%% 
%% Any modified versions of this file must be renamed
%% with new filenames distinct from sample-authordraft.tex.
%% 
%% For distribution of the original source see the terms
%% for copying and modification in the file samples.dtx.
%% 
%% This generated file may be distributed as long as the
%% original source files, as listed above, are part of the
%% same distribution. (The sources need not necessarily be
%% in the same archive or directory.)
%%
%% Commands for TeXCount
%TC:macro \cite [option:text,text]
%TC:macro \citep [option:text,text]
%TC:macro \citet [option:text,text]
%TC:envir table 0 1
%TC:envir table* 0 1
%TC:envir tabular [ignore] word
%TC:envir displaymath 0 word
%TC:envir math 0 word
%TC:envir comment 0 0
%%
%%
%% The first command in your LaTeX source must be the \documentclass command.
%\documentclass[sigconf, review]{acmart}
\documentclass[sigconf]{acmart}
\settopmatter{authorsperrow=3}
%% NOTE that a single column version may required for 
%% submission and peer review. This can be done by changing
%% the \doucmentclass[...]{acmart} in this template to 
%% \documentclass[manuscript,screen]{acmart}
%% 
%% To ensure 100% compatibility, please check the white list of
%% approved LaTeX packages to be used with the Master Article Template at
%% https://www.acm.org/publications/taps/whitelist-of-latex-packages 
%% before creating your document. The white list page provides 
%% information on how to submit additional LaTeX packages for 
%% review and adoption.
%% Fonts used in the template cannot be substituted; margin 
%% adjustments are not allowed.

%%
%% \BibTeX command to typeset BibTeX logo in the docs
\AtBeginDocument{%
  \providecommand\BibTeX{{%
    \normalfont B\kern-0.5em{\scshape i\kern-0.25em b}\kern-0.8em\TeX}}}

%% Rights management information.  This information is sent to you
%% when you complete the rights form.  These commands have SAMPLE
%% values in them; it is your responsibility as an author to replace
%% the commands and values with those provided to you when you
%% complete the rights form.

\copyrightyear{2025}
\acmYear{2025}
\setcopyright{acmlicensed}\acmConference[KDD '25]{Proceedings of the 30th ACM SIGKDD Conference on Knowledge Discovery and Data Mining}{August 25--29, 2025}{Toronto, Canada}
\acmBooktitle{Proceedings of the 31th ACM SIGKDD Conference on Knowledge Discovery and Data Mining (KDD '25), August 25--29, 2024, Toronto, Canada}
\acmDOI{10.1145/3637528.3671561}
\acmISBN{979-8-4007-0490-1/24/08}

%% These commands are for a PROCEEDINGS abstract or paper.
%\acmConference[KDD'24]{}{August 25--29,
%  2024}{Barcelona, Spain}
%
%  Uncomment \acmBooktitle if th title of the proceedings is different
%  from ``Proceedings of ...''!
%
%\acmBooktitle{Woodstock '18: ACM Symposium on Neural Gaze Detection,
%  June 03--05, 2018, Woodstock, NY} 
%\acmISBN{978-1-4503-XXXX-X/18/06}


%%
%% Submission ID.
%% Use this when submitting an article to a sponsored event. You'll
%% receive a unique submission ID from the organizers
%% of the event, and this ID should be used as the parameter to this command.
%%\acmSubmissionID{123-A56-BU3}

%%
%% For managing citations, it is recommended to use bibliography
%% files in BibTeX format.
%%
%% You can then either use BibTeX with the ACM-Reference-Format style,
%% or BibLaTeX with the acmnumeric or acmauthoryear sytles, that include
%% support for advanced citation of software artefact from the
%% biblatex-software package, also separately available on CTAN.
%%
%% Look at the sample-*-biblatex.tex files for templates showcasing
%% the biblatex styles.
%%

%%
%% For managing citations, it is recommended to use bibliography
%% files in BibTeX format.
%%
%% You can then either use BibTeX with the ACM-Reference-Format style,
%% or BibLaTeX with the acmnumeric or acmauthoryear sytles, that include
%% support for advanced citation of software artefact from the
%% biblatex-software package, also separately available on CTAN.
%%
%% Look at the sample-*-biblatex.tex files for templates showcasing
%% the biblatex styles.
%%

%%
%% The majority of ACM publications use numbered citations and
%% references.  The command \citestyle{authoryear} switches to the
%% "author year" style.
%%
%% If you are preparing content for an event
%% sponsored by ACM SIGGRAPH, you must use the "author year" style of
%% citations and references.
%% Uncommenting
%% the next command will enable that style.
%%\citestyle{acmauthoryear}

%%
%% end of the preamble, start of the body of the document source.
\usepackage{makecell}
\usepackage{multirow}
\usepackage{algorithm}
\usepackage{algpseudocode}
\usepackage{tabularx}
\usepackage{graphicx}
\usepackage{enumitem}
\usepackage{refcount}
\usepackage{multicol}% http://ctan.org/pkg/multicols
\newcommand{\systemname}{\textit{LiGR}}
\newcounter{savealgorithm}
\newenvironment{subalgorithms}
 {%
  \stepcounter{algorithm}%
  \edef\currentthealgorithm{\thealgorithm}%
  \setcounter{savealgorithm}{\value{algorithm}}%
  \setcounter{algorithm}{0}%
  \renewcommand{\thealgorithm}{\currentthealgorithm\alph{algorithm}}%
 }
 {%
  \setcounter{algorithm}{\value{savealgorithm}}%
 }

\makeatother

%\renewcommand{\baselinestretch}{0.965}
\algnewcommand\algorithmicforeach{\textbf{for each}}
\algdef{S}[FOR]{ForEach}[1]{\algorithmicforeach\ #1\ \algorithmicdo}

\algdef{SE}[REPEATN]{RepeatN}{End}[1]{\algorithmicrepeat\ #1 \textbf{times}}{\algorithmicend}

\settopmatter{printacmref=true}
\begin{document}

%%
%% The "title" command has an optional parameter,
%% allowing the author to define a "short title" to be used in page headers.
\title{From Features to Transformers: \\ Redefining Ranking for Scalable Impact}

%%
%% The "author" command and its associated commands are used to define
%% the authors and their affiliations.
%% Of note is the shared affiliation of the first two authors, and the
%% "authornote" and "authornotemark" commands
%% used to denote shared contribution to the research.

\author{Fedor Borisyuk}
\authornote{Authors contributed equally to this research.}
\author{Lars Hertel}
\authornotemark[1]
\author{Ganesh Parameswaran}
\authornotemark[1]
\author{Gaurav Srivastava}
\authornotemark[1]
\author{Sudarshan Ramanujam}
\authornotemark[1]
\affiliation{%\small
 \institution{LinkedIn}
 \city{Mountain View}
 \state{CA}
 \country{USA}}
\email{fedorvb@gmail.com}

\author{Borja Ocejo}
\authornotemark[1]
\author{Peng Du}
\author{Andrei Akterskii}
\author{Neil Daftary}
\author{Shao Tang}
\affiliation{%\small
 \institution{LinkedIn}
 \city{Mountain View}
 \state{CA}
 \country{USA}}
 \email{bocejo@linkedin.com}
%\email{pedu@linkedin.com}

% bocejo@linkedin.com

\author{Daqi Sun}
\author{Charles Xiao}
\author{Deepesh Nathani}
\author{Mohit Kothari}
\author{Yun Dai}
\author{Aman Gupta}
\affiliation{%\small
 \institution{LinkedIn}
 \city{Mountain View}
 \state{CA}
 \country{USA}}
\email{daqsun@linkedin.com}

%\author{Cyrus Asgari}
%\author{Aditya Ayier}
%\author{Deepesh Nathani}
%\author{Mohit Kothari}
%\author{Yun Dai}
%\author{Aman Gupta}
%\affiliation{%\small
% \institution{LinkedIn}
% \city{Mountain View}
% \state{CA}
% \country{USA}}
%\email{dnathani@linkedin.com}

%%
%% By default, the full list of authors will be used in the page
%% headers. Often, this list is too long, and will overlap
%% other information printed in the page headers. This command allows
%% the author to define a more concise list
%% of authors' names for this purpose.
\renewcommand{\shortauthors}{Fedor Borisyuk et al.}
%% No italics, no superscripts
%% Use footnote or author note to identify equal contribution and/or contact author info

%%
%% The abstract is a short summary of the work to be presented in the
%% article.
\begin{abstract}
\begin{abstract}
Retrieval-Augmented Generation (RAG) is often used with Large Language Models (LLMs) to infuse domain knowledge or user-specific information. In RAG, given a user query, a retriever extracts chunks of relevant text from a knowledge base. These chunks are sent to an LLM as part of the input prompt. Typically, any given chunk is repeatedly retrieved across user questions. However, currently, for every question, attention-layers in LLMs fully compute the key values (KVs) repeatedly for the input chunks, as state-of-the-art methods cannot reuse KV-caches when chunks appear at arbitrary locations with arbitrary contexts. Naive reuse leads to output quality degradation.  This leads to potentially redundant computations on expensive GPUs and increases latency. In this work, we propose \sys, a system for managing and reusing precomputed KVs corresponding to the text chunks (we call \textit{chunk-caches}) in RAG-based systems. We present how to identify \hl{\textit{chunk-caches} that are reusable}, how to efficiently perform a small fraction of recomputation to \textit{fix} the cache to maintain output quality, and how to efficiently store and evict \textit{chunk-caches} in the hardware for maximizing reuse while masking any overheads. With real production workloads as well as synthetic datasets, we show that \sys reduces redundant computation by \textbf{51\%} over SOTA prefix-caching and \textbf{75\%} over full recomputation.
\hl{Additionally, with continuous batching on a real production workload, we get a \textbf{1.6$\times$} speedup in throughput and a \textbf{2$\times$} reduction in end-to-end response latency over prefix-caching while maintaining quality, for both the \llama-3-8B and \llama-3-70B models. 
}
\end{abstract}






\end{abstract}

%%
%% The code below is generated by the tool at http://dl.acm.org/ccs.cfm.
%% Please copy and paste the code instead of the example below.
%%
\begin{CCSXML}
<ccs2012>
<concept>
<concept_id>10010147.10010257.10010293.10010294</concept_id>
<concept_desc>Computing methodologies~Neural networks</concept_desc>
<concept_significance>500</concept_significance>
</concept>
<concept>
<concept_id>10002951.10003317.10003347.10003350</concept_id>
<concept_desc>Information systems~Recommender systems</concept_desc>
<concept_significance>500</concept_significance>
</concept>
<concept>
<concept_id>10002951.10003317.10003338.10003343</concept_id>
<concept_desc>Information systems~Learning to rank</concept_desc>
<concept_significance>500</concept_significance>
</concept>
</ccs2012>
\end{CCSXML}

\ccsdesc[500]{Computing methodologies~Neural networks}
\ccsdesc[500]{Information systems~Recommender systems}
\ccsdesc[500]{Information systems~Learning to rank}


%%
%% Keywords. The author(s) should pick words that accurately describe
%% the work being presented. Separate the keywords with commas.
\keywords{Large Scale Ranking, Deep Neural Networks}

%%
%% This command processes the author and affiliation and title
%% information and builds the first part of the formatted document.
\maketitle
%\def\thefootnote{*}\footnotetext{Work done while at LinkedIn.}
\section{Introduction}\label{sec:intro}
\section{Introduction}
\label{sec:intro}

\begin{figure*}[tb]
    \centering
    \includegraphics[width=0.848\linewidth]{figs/circuitnn.pdf} 
    \caption{Illustration of differentiable CircuitNN. CircuitNN is designed based on differentiable NAND gates. After DAS is guided by PI and PO pairs of the truth table, CircuitNN can get the precise circuit architecture logic equivalent to the truth table.}
    \label{fig:circuitnn}
\end{figure*}

% 1. Describe the importance of logic synthesis
% 2. Existing Problems
% (a) Neural Architecture Search: Unstable, Predefined Setting, etc.
% (b) Circuit Generation: Probabilistic Model, Logic Equivalence

With the rapid advancement of technology, the scale of integrated circuits (ICs) has expanded exponentially. 
This expansion has introduced significant challenges in chip manufacturing, particularly concerning power and area metrics.
A primary objective in IC design is achieving the same circuit function with fewer transistors, thereby reducing power usage and area occupancy.

Logic synthesis~\cite{hachtel2005logicsynth}, a critical step in electronic design automation (EDA), transforms behavioral-level circuit designs into optimized gate-level circuits, ultimately yielding the final IC layout. 
The primary goal of logic synthesis is to identify the physical implementation with the fewest gates for a given circuit function. 
This task constitutes a challenging NP-hard combinatorial optimization problem. 
Current logic synthesis tools~\cite{brayton2010abc, wolf2013yosys} rely on human-designed heuristics, often leading to sub-optimal outcomes.

Differentiable architecture search (DAS) techniques~\cite{liu2018darts, chu2020darts} offer novel perspectives on addressing challenges in this problem.
Circuit functions can be represented through truth tables, which map binary inputs to their corresponding outputs. 
Truth tables provide a precise representation of input-output relationships, ensuring the design of functionally equivalent circuits.
Inspired by this, researchers~\cite{deepmind2024ai4sys, wang2024tnet} have begun exploring the application of DAS to synthesize circuits directly from truth tables.
Specifically, \citet{deepmind2024ai4sys} proposed CircuitNN, a framework that learns differentiable connection structures with logic gates, enabling the automatic generation of logic circuits from truth tables.
This approach significantly reduces the complexity of traditional circuit generation. 
Building on this, \citet{wang2024tnet} introduced T-Net, a triangle-shaped variant of CircuitNN, incorporating regularization techniques to enhance the efficiency of DAS.

Despite these advancements, several challenges remain. 
The computational complexity of DAS grows quadratically with the number of gates, posing scalability issues.
Although triangle-shaped architecture~\cite{wang2024tnet} partially mitigates this problem, redundancy persists. 
%Additionally, DAS is susceptible to converging to local optima, limiting the ability to search architectures that satisfy the given truth tables~\cite{liu2018darts}. 
%Furthermore, hyperparameters (network depth and layer width) require extensive searches, introducing complexity and prolonging the synthesis process. 
Additionally, DAS is susceptible to converging to local optima~\cite{liu2018darts} and hyperparameters (network depth and layer width) require extensive searches. 
The challenges arise from the vast search space in DAS. 
% Even with predefined settings for CircuitNN, finding a configuration that meets the truth table requires extensive trial and error during the DAS process. 
Intuitively, limiting the search space through predefined parameters (network depth, gates per layer, and connection probabilities) can significantly reduce the complexity.

Recent advances~\cite{openai2023gpt4, abramson2024alphafold3, esser2024sd3, li2024mar} in conditional generative models have demonstrated remarkable performance across language, vision, and graph generation tasks. 
Motivated by these developments, we propose a novel approach to circuit generation that generates preliminary circuit structures to guide DAS in generating refined circuits matching specified truth tables. 
Firstly, we introduce CircuitVQ, a tokenizer with a discrete codebook for circuit tokenization. 
Built upon our Circuit AutoEncoder framework~\cite{hou2022graphmae,li2023maskgae,wu2025mgvga}, CircuitVQ is trained through a circuit reconstruction task. 
Specifically, the CircuitVQ encoder encodes input circuits into discrete tokens using a learnable codebook, while the decoder reconstructs the circuit adjacency matrix based on these tokens.
Subsequently, the CircuitVQ encoder serves as a circuit tokenizer for CircuitAR pretraining, which employs a masked autoregressive modeling paradigm~\cite{chang2022maskgit, li2023mage}. 
In this process, the discrete codes function as supervision signals. 
After training, CircuitAR can generate discrete tokens progressively, which can be decoded into initial circuit structures by the decoder of the CircuitVQ. 
These prior insights can guide DAS in producing refined circuits that match the target truth tables precisely.

Our key contributions can be summarized as follows:
\begin{itemize}
\item We introduce CircuitVQ, a circuit tokenizer that facilitates graph autoregressive modeling for circuit generation, based on our Circuit AutoEncoder framework;
\item Develop CircuitAR, a model trained using masked autoregressive modeling, which generates initial circuit structures conditioned on given truth tables;
\item Propose a refinement framework that integrates differentiable architecture search to produce functionally equivalent circuits guided by target truth tables;
\item Comprehensive experiments demonstrating the scalability and capability emergence of our CircuitAR and the superior performance of the proposed circuit generation approach.
\end{itemize}

% Motivation
% (a) Diffusion (Vision, Graph), Autoregressive (Language, Vision)
% (b) Circuit Generation for Predefined Setting
% (c) Neural Architecture Search for Strict Logic Equivalence

% Contribution
% (a) Circuit Tokenizer (new transformer arch, training strategy)
% (b) CircuitAR (train and gen strategies, post-ar strategy)
% (c) Extensive Evaluation including BitD (Bit Distance) for Scalability


\section{Related Work}
% \subsection{Vision Language Model}
% 시각장애인에서 상황을 설명할 DB가 없으니 만들었다. 그리고 이를 VLM에 튜닝했다.
\subsection{Technical approaches for assisting the visually-impaired}


\subsection{Datasets for visual instruction tuning}


\section{Model Architecture}\label{sec:overview}
\begin{figure*}[t]
\begin{center}
\includegraphics[width=.85\linewidth]{fig_overview_v3.pdf}
\end{center}
\caption{
FastAtlas Overview: In each frame, we compute charts spanning fully or partially visible triangles (a), determine texture space bounding boxes for the visible portions of the view-space projections of each chart, and tightly pack these boxes into atlases (b, here $2K \times 2K$). We simultaneously bijectively parameterize and shade the charts into their atlas boxes, obtaining high quality texture space shading (c), and use this shading to render the shaded frames (d).}
\label{fig:overview}
\label{fig:alg_overview}
\end{figure*}

\section{Overview}
\label{sec:overview}
Our work has two core contributions: a real-time, GPU-based algorithm for tight packing of general parameterized charts into compact atlases; and a real-time TSS method that
utilizes this packing.  

\paragraph*{FastAtlas Packing.}
FastAtlas runs entirely on the GPU as a series of compute shaders. It takes the bounding boxes of parameterized charts as input, and packs them into an atlas (Fig~\ref{fig:overview}b, Sec.~\ref{sec:pack}). As such, the only input it requires are the dimensions of the bounding boxes.
Its outputs are deterministic; identical input charts are packed into identical atlases. This is critical for TSS and similar applications, as it ensures that consecutive frames taken from the same camera view have the same shading. Even minute shading differences across such frames can cause sampling jitter, leading to undesirable flicker \cite{baker2012rock}. 
While prior methods such as \cite{mueller2018shading,hladky2019tessellated,hladky2021snakebinning,Neff2022MSA} cap the dimensions of the charts that can be packed as-is for a given atlas size, and scale down all charts that exceed these dimensions, we scale all charts by the same factor, and do so only when strictly necessary to achieve packing success (Figs~\ref{fig:atlas},~\ref{fig:sas_issues}). 

\paragraph*{TSS using FastAtlas.}
Our end-to-end TSS atlas generation method combines the packing method above with a novel approach for computing seamless per-frame charts. 
We define our charts as the connected components of the visible surfaces in each frame (Fig.~\ref{fig:overview}a), and efficiently compute them using a parallel union-find algorithm (Sec.~\ref{sec:visible}). Since the boundaries of these charts coincide with the contours of the rendered surface, they are {\em invisible} to the viewer. This approach 
eliminates the artifacts caused by shading discontinuities along visible seams (Fig.~\ref{fig:seams}). 

\begin{parWithWrapFigure}
\begin{wrapfigure}{l}{.27\columnwidth}%
\includegraphics[width=\linewidth]{fig_inset_view_plane.pdf}%
\end{wrapfigure}
We bijectively parametrize the {\em visible portions} of our charts by projecting them to view space (inset). This maps a constant number of texels to each pixel in the final rendered output, evenly distributing residual undersampling error across all image pixels. While conceptually straightforward, efficiently parameterizing charts containing partially visible triangles using viewspace projection is non-trivial, as the visible portions may no longer be triangular (e.g. green triangle in the inset); applying naive projection to triangles with vertices behind the camera may produce ill-posed results. Clipping triangles before projection is both computationally expensive and significantly complicates downstream operations. We avoid explicit clipping by observing that all that is required for atlas packing is the dimensions of, potentially conservative, bounding boxes of these projected visible portions. We compute such bounding boxes without explicit chart clipping by adapting a conservative screen coverage estimator \shortcite{Blinn:CalculatingScreenCoverage} (Sec.~\ref{sec:box}). We then pack the computed boxes using FastAtlas. 
\end{parWithWrapFigure}

Finally, we shade the visible portion of each chart into its corresponding atlas bounding box (Fig~\ref{fig:overview}c). 
The resulting texture is then used during rasterization as a standard texture map (Fig. ~\ref{fig:overview}d). 
Our framework is compatible with all existing approaches for texture space shading, including forward shading, raytraced illumination, or deferred shading in texture space \cite{baker:2016}. In the examples shown, we use the standard forward shading based rendering pipeline included in the G3D Innovation Engine \cite{G3D17}, a commercial grade renderer.


% \section{Content Modeling}\label{sec:model}
% \input{modeling_techniques.tex}

%\vspace{-0.8em}
\section{System Architecture}\label{sec:system_arch}
{\systemname} helps to significantly simplify our system. As we are able to reduce dependency on the counter features, and reduce number to less than ten features from hundreds. 
Serving of transformer model requires to store near-line member and item activity as shown on the Figure \ref{fig:system_arch}.
Within the system we utilize variety of item and member embeddings generated by GNN \cite{LiGNN_paper}, LLM \cite{PEv3_paper}, as well as ID embeddings stored and served within the model. Given the user request, the system retrieves the top candidates using the model-based retrieval system \cite{Linr_paper}, then the top K candidates are evaluated using the second layer {\systemname} scoring model. {\systemname} takes as input Items features and Member Context.


%\textcolor{red}{\textbf{Fedor}} intro to system arch TO FILL IN 
%\subsection{System design}
%\textcolor{red}{\textbf{Sudarshan / Borja}} TO FILL IN 


\subsubsection{{\systemname} training}
Training of the {\systemname} model is GPU memory intensive, requiring us to leverage several optimizations. Firstly, long user item histories interleaved with corresponding actions result in sequence lengths of up to 2048 for our largest model. We used Flash Attention and mixed precision to reduce memory consumption.

Secondly, {\systemname} leverages several ID embedding features as shown in Table \ref{tab:single_feature_evaluation}. It is important to scale up ID embedding dimensions alongside other hyper-parameters. In order to accommodate embedding tables larger than the memory of one GPU we use column-wise splitting of embedding tables and distribution across multiple devices. The latter allowed us to scale up embedding tables to arbitrarily large sizes.


\begin{figure}
    \centering
    \includegraphics[width=\linewidth]{figures/ARTAL_arch.png}
    \caption{{\systemname} system architecture.}
    \label{fig:system_arch}
\end{figure}
%\textcolor{red}{\textbf{Fedor}} TO FILL IN 

\subsubsection{Optimization of inference}\label{optimization_inference}
Naive inference can be expensive in the {\systemname} model due to the complexity of applying many transformer layers on a long user action history for many candidate items. \cite{HSTU_paper_zhai24a, efficient_transact_paper} provide ways to amortize the computation of the history across all candidate items. In our case, the historical attention implementation allows items to attend to other items only if they are from earlier sessions. For this reason we can easily infer all candidates at the same time, automatically amortizing the inference of the history across all candidates. 

%The pointwise model used multithreading for independent item scoring at inference, but introducing a setwise attention increased latency due to: %(1) additional inference calls, (2) 
%loss of multithreading since items needed to be scored together, and  latency scaling with the number of setwise-scored items. LinkedIn's ranking inference remains CPU-based, with GPU used for retrieval \cite{Linr_paper}. 

Feed metrics are sensitive to latency, requiring careful optimizations to stay within bounds: (1) Split Scoring: Setwise scoring was applied only to the top 10 posts after point-wise scoring, where we decompose point-wise and setwise model weights separately; (2) Score Combination: Setwise scores were added to pointwise scores for items 1–10, preserving the order for items 11–end; (3) Simplification: Rule-based diversity rerankers were disabled; (4) Unified Inference: Pointwise and setwise parts of the model were stitched with a rewriter, enabling single inference to score the entire model.
To scale online scoring, all items were passed using the batch size dimension, repackaged as a listwise dataset, and scored via the trained model. These optimizations reduced incremental latency cost to 10ms (p90), ensuring minimal impact on member experience.

%The pointwise model leveraged multithreading extensively to score items at inference time since scoring each item was independent. After training a new setwise model that operates on top of the existing pointwise model, the model latency goes up due to the following reasons: (1) Extra inference call, (2) Multi-threading was no longer possible since all items needed to be scored together, (3) Latency increase was proportional to the number of items that were scored set-wise.

%One point to note is that the LinkedIn inference for ranking currently is still done mostly on CPUs and GPU inference is used for retrieval \cite{Linr_paper}. Feed metrics are sensitive to latency and we need to be careful to not overshoot the latency bounds. In order to achieve it, the following optimizations were done:

%LinkedIn's ranking inference is mostly CPU-based, with GPU inference used for retrieval [5]. Feed metrics are sensitive to latency, requiring careful optimizations to stay within bounds: (1) Split Scoring: Setwise scoring is applied only to the top 10 organic updates after pointwise scoring; (2) Score Combination: Setwise scores are added to pointwise scores for items 1–10, ensuring their scores remain higher than items 11–end, preserving the pointwise order beyond the top 10; (3) Simplification: Existing rule-based diversity rerankers were disabled;
%(4) Unified Inference: Pointwise and setwise models were stitched with a rewriter, enabling a single-threaded inference call to score the entire model at once, reducing latency.

%\begin{enumerate}
 %   \item Instead of scoring all the items setwise, we chose to split point-wise and setwise models, and apply setwise scoring to the top 10 organic updates following after the pointwise scoring.
 %   \item The setwise model scores was added to the pointwise model scores from item 1-10 in order to ensure that the top 10 scores were always greater than item 11 - end and hence ordering of item 11 - end was preserved based on pointwise model score.
 %   \item Disabled existing rule based diversity rerankers.
 %   \item Instead of having 2 separate calls to the model cloud \cite{Linr_paper}, we chose to stitch the point-wise model with the set-wise model using a rewriter. During inference time, a single thread is now leveraged to score the entire model in one shot. 
%\end{enumerate}

%In industry, TF rankling is usually leveraged to train and score setwise models.

%In order to simplify our online scoring at scale, we chose to simplify the problem. The scoring engine simply passes all the items using the batch size dimension. We repackage this to a listwise dataset and then pass it through the trained model for inference. With the mentioned optimizations, we were able to bring down the latency increase to ~10ms (p90m measurement) which did not severely impact the member experience.
%However, in order to simplify our online scoring at scale, we chose to simplify the problem. The scoring engine simply passes all the items using the batch size dimension that is available for TF models. Internally in the TF model, we repackage this to a listwise dataset and then pass it through the trained model for inference. 

%With the mentioned optimizations, we were able to bring down the latency increase to ~10ms (p90m measurement) which did not severely impact the member experience.

%\subsubsection{Optimization of {\systemname} inference}\label{optimization_inference_GR}
%Naive inference can be expensive in the GR model due to the complexity of applying many transformer layers on a long user action history for many candidate items. \cite{HSTU_paper_zhai24a, efficient_transact_paper} provide ways to amortize the computation of the history across all candidate items. In our case, the historical attention implementation allows items to attend to other items only if they are from earlier sessions. For this reason we can easily infer all candidates at the same time, automatically amortizing the inference of the history across all candidates. 


In this section, we empirically compare the proposed algorithm on both sequence windows and time windows with existing methods.
\paragraph{Datasets} For the sequence-based model, we used two synthetic datasets and two cross-language datasets. The statistics of the datasets are provided in Table \ref{table:statistics}:

\begin{table}[t]
    \centering
    \caption{The statistics of the datasets. The datasets satisfy $1 \leq \|\vx\|\|\vy\| \leq R $.}
    \label{table:statistics}
    \begin{tabular}{|c|c|c|c|c|c|}
    \hline
        Dataset & $n$ & $m_x$ & $m_y$ & $N$ & $R$ \\ \hline
        SYNTHETIC(1) & 100,000 & 1,000 & 2,000 & 50,000 & 65 \\ \hline
        SYNTHETIC(2) & 100,000 & 1,000 & 2,000 & 50,000 & 724 \\ \hline
        APR & 23,235 & 28,017 & 42,833 & 10,000 & 773 \\ \hline
        PAN11 & 88,977 & 5,121 & 9,959 & 10,000 & 5,548 \\ \hline
        EURO & 475,834 & 7,247 & 8,768 & 100,000 & 107,840 \\ \hline
    \end{tabular}
\end{table}

\begin{itemize}
    \item Synthetic: The elements of the two synthetic datasets are initially uniformly sampled from the range (0,1), then multiplied by a coefficient to adjust the maximum column squared norm $R$. The X matrix has 1,000 rows, and the Y matrix has 2,000 rows, each with 100,000 columns. The window size is set to 50,000.
    \item APR: The Amazon Product Reviews (APR) dataset is a publicly available collection containing product reviews and related information from the Amazon website. This dataset consists of millions of sentences in both English and French. We structured it into a review matrix where the X matrix has 28,017 rows, and the Y matrix has 42,833 rows, with both matrices sharing 23,235 columns. The window size is 10,000.
    \item PAN11: PANPC-11 (PAN11) is a dataset designed for text analysis, particularly for tasks such as plagiarism detection, author identification, and near-duplicate detection. The dataset includes texts in English and French. The X and Y matrices contain 5,121 and 9,959 rows, respectively, with both matrices having 88,977 columns. The window size is 10,000.
\end{itemize}
We evaluate the time-based model on another real-world dataset:
\begin{itemize}
    \item EURO: The Europarl (EURO) dataset is a widely used multilingual parallel corpus, comprising the proceedings of the European Parliament. We selected a subset of its English and French text portions. The X and Y matrices contain 7,247 and 8,768 rows, respectively, and both matrices share 475,834 columns. Timestamps are generated using the $Poisson$ $Arrival$ $Process$ with a rate parameter of $\lambda=2$. The window size is set to 100,000, with approximately 30,000 columns of data on average in each window.
\end{itemize}

\paragraph{Setup} For the sequence-based model, we compare the proposed hDS-COD and  aDS-COD with EH-COD~\cite{yao2024approximate} and DI-COD~\cite{yao2024approximate}. We do not consider the Sampling algorithm as a baseline, as its performance is inferior to that of EH-COD and DI-CID, as demonstrated in \cite{yao2024approximate}. %The hDS-COD is adjusted by the parameter $\ell$ and the maximum number of levels $L = \log{R}$, where $R$ is the prior estimate of the maximum squared column norm of the dataset. DI-COD similarly requires a prior estimate of $R$ to limit the maximum number of levels $L = \log{(R/\varepsilon})$. In contrast, aDS-COD and EH-COD do not require an estimate of $R$; their error-space balance is controlled by the parameter $\ell = \frac{1}{\varepsilon}$. 
For the time-based model, we compare the proposed hDS-COD and  aDS-COD with EH-COD and the Sampling algorithm since DI-COD cannot be applied to time-based sliding window model. To achieve the same error bound, the maximum number of levels for hDS-COD is set to $L = \log{(\varepsilon NR)}$, and the initial threshold for aDS-COD is set to $1$.

Our experiments aim to illustrate the trade-offs between space and approximation errors. The x-axis represents two metrics for space: final sketch size and total space cost. The final sketch size refers to the number of columns in the result sketches $\mA$ and $\mB$ generated by the algorithm, representing a compression ratio. The total space cost refers to the maximum space required during the algorithm's execution, measured by the number of columns.We evaluate the approximation performance of all algorithms based on correlation errors $\operatorname{corr-err}(\mathbf{X}_W \mathbf{Y}_W^\top, \mathbf{A} \mathbf{B}^\top)$, which is reflected on the y-axis. Every 1,000 iterations, all algorithms query the window and record the average and maximum errors across all sampled windows.

The experiments for all algorithms were conducted using MATLAB (R2023a), with all algorithms running on a Windows server equipped with 32GB of memory and a single processor of Intel i9-13900K.

\paragraph{Performance} Figure \ref{fig:error vs l} and Figure \ref{fig:error vs space} illustrate the space efficiency comparison of the algorithms on sequence-based datasets. Panels (a-d) show the average errors across all sampled windows, while panels (e-h) display the maximum errors.

Figure \ref{fig:error vs l} evaluates the compression effect of the final sketch. The hDS-COD, aDS-COD, and EH-COD show similar compression performances. But the DS series is more stable, particularly on the synthetic datasets, where they significantly outperform EH-COD and DI-COD. The performance of hDS-COD and aDS-COD is nearly the same, indicating that the adaptive threshold trick in aDS-COD does not have a noticeable negative impact on it, maintaining the same error as hDS-COD.

Figure \ref{fig:error vs space} measures the total space cost of the algorithms. hDS-COD and aDS-COD show a significant advantage over existing methods, as they can achieve the  $\varepsilon$-approximation error with much less space. For the same space cost, the correlation errors of hDS-COD and aDS-COD are much smaller than those of EH-COD and DI-COD. Also, aDS-COD has better space efficiency than hDS-COD because aDS only uses a single-level structure while hDS requires $\log R+1$ levels. We find that hDS-COD requires more space on  SYNTHETIC(2) dataset compared to SYNTHETIC(1) dataset. This phenomenon occurs because SYNTHETIC(2) dataset has a larger $R$, which confirms the dependence on $R$ as stated in Theorem~\ref{thm:hds}. 

Figure \ref{fig:time-based} compares the performance of algorithms on time-based windows. Panels (a) and (b) present the error against the final sketch size, which show that our aDS-COD and hDS-COD algorithms enjoy similar performance as EH-COD and significantly outperform the sampling algorithm. On the other hand, as shown in panels (c) and (d), our methods outperform baselines in terms of total space cost.


%\vspace{-0.8em}


\section{Lessons learnt}
Over the time of development of {\systemname} we learnt many lessons. Here we present couple of interesting examples.
\subsection{Diversity of topics in Recommendations}
%Remove diversity re-rankers - tell the story how we removed diversity rerankers \textcolor{red}{\textbf{Sudarshan \& Borja}}

Over the years, the LinkedIn feed ranking model has been a pure pointwise model and list level interactions are not taken into account by the model. This could however yield a subpar experience of the entire feed session as a whole for our members. Some examples include seeing too many updates from the same actors, back to back out of network content, back to back instances of posts that are surfaced because one of their connections liked an activity of their connection. 

Diversifying this experience through a rule based approach has helped provide a much better session experience as a whole for our members which could be seen from the metric impact. 
We did a simple ablation study of removing all the diversity rules and checked the member impact. As expected, we do see a drop in the DAU in LinkedIn (-0.18\%).

While the rule based diversity for organizing the session does help, we believe it is suboptimal and a model powered solution could yield a better experience for our members. The rules tend to assume that the same template would work for everyone. In our approach we believe that replacing this legacy solution with a model that could learn the required list level diversity attribute is a superior solution.  In fact a model based approach could help us quantify and keep diversity as an objective in the longer term. In this work, we show how setwise models could replace diversity rules with a superior member experience in LinkedIn.


\subsection{Our approach to develop {\systemname}}
%\textcolor{red}{\textbf{Lars - tell the story how we built GR}} to FILL IN

{\systemname} and the traditional DLRM approach are fundamentally different in data format, features, and model training. {\systemname} therefore required a full rebuilding of our training pipeline. The goals for this rebuilding were to use few features, to outperform the baseline, and to demonstrate scaling laws.

We started this work by verifying that our most important count features can be emulated by ID embedding features alone. Then we built a small {\systemname} model and added the top features from the DLRM model or corresponding ID feature counterparts until adding more features gave only small AUC improvements. While the resulting model had ten times less features than the baseline, we did find that with too few features the model can suffer from item cold start and is not able to beat the baseline. In order to reduce member cold start in less frequent members we also increased the user history time window, but retained item ID features only for a limited window.

\begin{figure}
    \centering
    \includegraphics[width=1\linewidth]{figures/gr-progress-chart_new.png}
    \caption{GR AUC improvement over successive iterations.}
    \label{fig:gr-progress}
    \vspace{-1.0em}
\end{figure}

Having added enough features, we scaled up the model along the dimensions of sequence length, embedding dimension, and number of layers. During this phase the main challenges were to maintain training stability and manage GPU memory. To ensure stable training we used different learning rates for dense vs. sparse model parameters, dense gating of transformer layers, and  transformers with gating as shown in Figure \ref{fig:custom-transformer}.

Overall we found that aside from ensuring training stability, most architecture changes we experimented with had little impact on the prediction AUC. The majority of performance appeared to be driven by features and model scale.



\section{Conclusion}\label{sec:conclusion}
\section*{Conclusion}
This paper aims to enhance our understanding of the computational complexity of computing various Shapley value variants. We found that for various ML models --- including decision trees, regression tree ensembles, weighted automata, and linear regression --- both local and global interventional and baseline SHAP can be computed in polynomial time under HMM modeled distributions. This extends popular algorithms, such as TreeSHAP, beyond their empirical distributional scope. We also establish strict complexity gaps between the various SHAP variants (baseline, interventional, and conditional) and prove the intractability of computing SHAP for tree ensembles and neural networks in simplified scenarios. Overall, we present SHAP as a versatile framework whose complexity depends on four key factors: \begin{inparaenum}[(i)] \item model type, \item SHAP variant, \item distribution modeling approach, \item and local vs. global explanations\end{inparaenum}. We believe this perspective provides deeper insight into the computational complexity of SHAP, paving the way for future work.




%We believe that our framework provides a more intricate understanding of SHAP computation complexity across different models, distributions, and variants, paving the way for further research.

Our work opens promising directions for future research. First, expanding our computational analysis to other SHAP-related metrics, such as asymmetric SHAP~\citep{frye20} and SAGE~\citep{covert2020understanding}, would be valuable. Additionally, we aim to explore more expressive distribution classes and relaxed assumptions beyond those in Section \ref{sec:tractable} while maintaining tractable SHAP computation. Finally, when exact computation is intractable (Section \ref{sec:intractable}), investigating the approximability of SHAP metrics through approximation and parameterized complexity theory~\citep{downey2012parameterized} is an important direction.

%Our work opens several promising avenues for future research on the computational properties of explainable AI methods, with a particular focus on SHAP. First, it would be interesting to broaden the computational analysis conducted in this work to include other popular SHAP-related metrics in the literature, such as asymmetric SHAP \cite{frye20} and SAGE \cite{covert2020understanding}. Also, in the future, we aim to explore more expressive distribution classes and relaxed distributional assumptions—extending beyond those examined in Section \ref{sec:tractable} —that still yield tractable SHAP computation. Finally, when exact computation proves intractable (Section \ref{sec:intractable}), it is worthwhile to theoretically investigate the question of the approximability of computing the SHAP metrics across various configurations, through the lens of approximation and parametrized complexity theory \cite{arora2009computational}.

%This paper aims to deepen our understanding of the computational complexity involved in obtaining different Shapley value variants. We found that for a variety of ML models, including decision trees, tree ensembles for regression, weighted automata, and linear regression models — computing both local and global interventional and baseline SHAP can be done in polynomial time when distributions are modeled by HMMs. This extends the distributional scope of popular algorithms like TreeSHAP, which is limited to empirical distributions. Additionally, we demonstrate a strict complexity gap between SHAP variants, showing that interventional and baseline SHAP can be strictly easier to compute than conditional SHAP. Despite these positive results, we uncovered intractability for various SHAP variants in neural networks and tree ensembles. Finally, we provided generalized complexity relations across SHAP variants. We believe that our framework offers a deeper understanding of the complexity involved in computing SHAP across various variants, models, distributions, as well as in both local and global computations, laying the groundwork for future research.

%%
%% The next two lines define the bibliography style to be used, and
%% the bibliography file.

\bibliographystyle{ACM-Reference-Format}
\balance
\bibliography{bibliography}

%%
%% If your work has an appendix, this is the place to put it.
%\vspace{-2.7em}



\appendix
%\section{INFORMATION FOR REPRODUCIBILITY}\label{sec:reproducability}

\section{APPENDIX}

\subsection{Reproducibility notes}
Training stability emerged as a critical factor in optimizing our model architecture, and we implemented several techniques to ensure it. These included using a transformer architecture with gating, where layer normalization is applied before multi-head attention (MHA) or MLP layers, and gating applied on top of MHA/MLP outputs with a sigmoid(XW) layer. Additionally, we employed separate learning rates for embeddings (0.01) and dense layers (0.001) to further enhance stability. Architectural changes, while having limited direct impact on performance, were crucial for maintaining training stability, enabling effective scaling of model and input size — both of which proved more impactful for performance improvements.

Our exploration of features revealed that models with even a single feature could achieve performance near baseline production model levels. However, surpassing baseline performance required the inclusion of multiple features. Notably, we tracked performance on cold-start items—those with few or no interactions in the training data—and found that adding more features significantly boosted performance on these items, demonstrating the value of feature richness for addressing sparse data challenges.

Investigations into position embeddings showed no significant AUC difference between relative attention bias and learned position embeddings for history sequence, suggesting that this choice has minimal impact. Similarly, replacing MLPs with DCNv2 or incorporating DCNv2 at the input level did not result in performance gains, despite its theoretical potential to model interactions across embedding dimensions. Even removing MLPs from transformer blocks caused only minor performance drops, presenting an opportunity to trade off MLPs for additional MHA layers within the same memory budget.

Our experiments with alternative attention activation functions, such as SiLU and Sigmoid, also yielded subpar results compared to Softmax. While these activations were hypothesized to better model extreme affinity cases, Softmax's normalization proved more effective at emphasizing the relative importance of sequence elements. This aligns with our findings on HSTU, which did not improve performance and showed lower AUC. Additionally, HSTU's reliance on SiLU attention made it incompatible with FlashAttention, a critical component in larger-scale experiments.

Regarding vocabulary size, we found no AUC difference between item ID embedding vocabularies of 33 million and 66 million, despite the training data encompassing approximately 150 million unique object IDs. This suggests that a smaller vocabulary suffices without compromising performance.

%Lastly, the team's focus on metrics and rapid iteration proved instrumental in achieving and surpassing MME AUC within a single quarter. By prioritizing fast experimentation, weekly performance improvements, and deferring non-critical tasks, two engineers were able to deliver unexpected results in a remarkably short timeframe. This approach underscores the value of agility and metric-driven development in advancing model performance.

\subsection{Analysis of alternative solutions}\label{wukong_layer}
We also explored alternative methods for scaling the model. Wukong \cite{zhang2024wukongscalinglawlargescale} has demonstrated effective dense parameter scaling by allowing the model to capture higher-order feature interactions efficiently. However, directly applying Wukong to the Feed models did not result in any observable improvement in AUCs, even when we scale the number of layers to 8. Our hypothesis suggests that the numerical features may have experienced information degradation during deep feature interactions with embedding features, potentially due to feature heterogeneity. The diverse nature of features, including numerical and categorical data, likely contributed to challenges in effectively capturing complex interactions. 

We attempted to change the design for Wukong, where we employ a dual-pathway approach to feature processing to a baseline architecture (see \S\ref{sec:Overview:FeedRanking}). Embedding features are passed through a deep Wukong layer, facilitating deep vector-wise interactions. Concurrently, numerical and categorical features are fed into a two-layer DCNv2 to enable element-wise interactions. This bifurcated structure allows for specialized handling of different feature types, potentially mitigating the challenges posed by feature heterogeneity. With this updated architecture, we observed a 0.36\% increase in Contributions AUC on top of baseline model (see \S\ref{sec:Overview:FeedRanking}), when scaling the number of layers to 8, which brings additional 2 million parameters.  With a two-tower structure, we extend the scaling to accommodate multiple objectives. For example, our model can simultaneously optimize for various user interactions, such as clicks, comments, shares, and more. Additionally, we explore different mechanisms for concatenating embeddings to achieve optimal performance. It's important to note that the Wukong layer is not restricted to stacked Factorization Machines; it can also incorporate stacked CrossNet for enhanced flexibility. The original Wukong Layer includes a Linear Compress Block, which linearly recombines embeddings — an important element for performance. When incorporated into the stacked CrossNet variant of Wukong, this block is adapted into a wide layer. This provides an interesting perspective, as each layer in Wukong effectively embodies the wide and deep architecture. 

As we developed {\systemname}, our proposed architecture (Figure \ref{fig:custom-transformer}) achieved a 1.2\% increase in Contributions AUC and demonstrated better scalability compared to the 0.36\% improvement observed with the Wukong layer. We believe that Wukong could primarily be used to extend DLRM-style models \cite{DLRM19} or to be used in combination with DLRM and transformer-based models, such as \cite{zeng2024interformer}.

\end{document}
\endinput
%%
%% End of file `sample-authordraft.tex'.
