%%
%% This is file `sample-authordraft.tex',
%% generated with the docstrip utility.
%%
%% The original source files were:
%%
%% samples.dtx  (with options: `authordraft')
%% 
%% IMPORTANT NOTICE:
%% 
%% For the copyright see the source file.
%% 
%% Any modified versions of this file must be renamed
%% with new filenames distinct from sample-authordraft.tex.
%% 
%% For distribution of the original source see the terms
%% for copying and modification in the file samples.dtx.
%% 
%% This generated file may be distributed as long as the
%% original source files, as listed above, are part of the
%% same distribution. (The sources need not necessarily be
%% in the same archive or directory.)
%%
%% Commands for TeXCount
%TC:macro \cite [option:text,text]
%TC:macro \citep [option:text,text]
%TC:macro \citet [option:text,text]
%TC:envir table 0 1
%TC:envir table* 0 1
%TC:envir tabular [ignore] word
%TC:envir displaymath 0 word
%TC:envir math 0 word
%TC:envir comment 0 0
%%
%%
%% The first command in your LaTeX source must be the \documentclass command.
%\documentclass[sigconf, review]{acmart}
\documentclass[sigconf]{acmart}
\settopmatter{authorsperrow=3}
%% NOTE that a single column version may required for 
%% submission and peer review. This can be done by changing
%% the \doucmentclass[...]{acmart} in this template to 
%% \documentclass[manuscript,screen]{acmart}
%% 
%% To ensure 100% compatibility, please check the white list of
%% approved LaTeX packages to be used with the Master Article Template at
%% https://www.acm.org/publications/taps/whitelist-of-latex-packages 
%% before creating your document. The white list page provides 
%% information on how to submit additional LaTeX packages for 
%% review and adoption.
%% Fonts used in the template cannot be substituted; margin 
%% adjustments are not allowed.

%%
%% \BibTeX command to typeset BibTeX logo in the docs
\AtBeginDocument{%
  \providecommand\BibTeX{{%
    \normalfont B\kern-0.5em{\scshape i\kern-0.25em b}\kern-0.8em\TeX}}}

%% Rights management information.  This information is sent to you
%% when you complete the rights form.  These commands have SAMPLE
%% values in them; it is your responsibility as an author to replace
%% the commands and values with those provided to you when you
%% complete the rights form.

\copyrightyear{2025}
\acmYear{2025}
\setcopyright{acmlicensed}\acmConference[KDD '25]{Proceedings of the 30th ACM SIGKDD Conference on Knowledge Discovery and Data Mining}{August 25--29, 2025}{Toronto, Canada}
\acmBooktitle{Proceedings of the 31th ACM SIGKDD Conference on Knowledge Discovery and Data Mining (KDD '25), August 25--29, 2024, Toronto, Canada}
\acmDOI{10.1145/3637528.3671561}
\acmISBN{979-8-4007-0490-1/24/08}

%% These commands are for a PROCEEDINGS abstract or paper.
%\acmConference[KDD'24]{}{August 25--29,
%  2024}{Barcelona, Spain}
%
%  Uncomment \acmBooktitle if th title of the proceedings is different
%  from ``Proceedings of ...''!
%
%\acmBooktitle{Woodstock '18: ACM Symposium on Neural Gaze Detection,
%  June 03--05, 2018, Woodstock, NY} 
%\acmISBN{978-1-4503-XXXX-X/18/06}


%%
%% Submission ID.
%% Use this when submitting an article to a sponsored event. You'll
%% receive a unique submission ID from the organizers
%% of the event, and this ID should be used as the parameter to this command.
%%\acmSubmissionID{123-A56-BU3}

%%
%% For managing citations, it is recommended to use bibliography
%% files in BibTeX format.
%%
%% You can then either use BibTeX with the ACM-Reference-Format style,
%% or BibLaTeX with the acmnumeric or acmauthoryear sytles, that include
%% support for advanced citation of software artefact from the
%% biblatex-software package, also separately available on CTAN.
%%
%% Look at the sample-*-biblatex.tex files for templates showcasing
%% the biblatex styles.
%%

%%
%% For managing citations, it is recommended to use bibliography
%% files in BibTeX format.
%%
%% You can then either use BibTeX with the ACM-Reference-Format style,
%% or BibLaTeX with the acmnumeric or acmauthoryear sytles, that include
%% support for advanced citation of software artefact from the
%% biblatex-software package, also separately available on CTAN.
%%
%% Look at the sample-*-biblatex.tex files for templates showcasing
%% the biblatex styles.
%%

%%
%% The majority of ACM publications use numbered citations and
%% references.  The command \citestyle{authoryear} switches to the
%% "author year" style.
%%
%% If you are preparing content for an event
%% sponsored by ACM SIGGRAPH, you must use the "author year" style of
%% citations and references.
%% Uncommenting
%% the next command will enable that style.
%%\citestyle{acmauthoryear}

%%
%% end of the preamble, start of the body of the document source.
\usepackage{makecell}
\usepackage{multirow}
\usepackage{algorithm}
\usepackage{algpseudocode}
\usepackage{tabularx}
\usepackage{graphicx}
\usepackage{enumitem}
\usepackage{refcount}
\usepackage{multicol}% http://ctan.org/pkg/multicols
\newcommand{\systemname}{\textit{LiGR}}
\newcounter{savealgorithm}
\newenvironment{subalgorithms}
 {%
  \stepcounter{algorithm}%
  \edef\currentthealgorithm{\thealgorithm}%
  \setcounter{savealgorithm}{\value{algorithm}}%
  \setcounter{algorithm}{0}%
  \renewcommand{\thealgorithm}{\currentthealgorithm\alph{algorithm}}%
 }
 {%
  \setcounter{algorithm}{\value{savealgorithm}}%
 }

\makeatother

%\renewcommand{\baselinestretch}{0.965}
\algnewcommand\algorithmicforeach{\textbf{for each}}
\algdef{S}[FOR]{ForEach}[1]{\algorithmicforeach\ #1\ \algorithmicdo}

\algdef{SE}[REPEATN]{RepeatN}{End}[1]{\algorithmicrepeat\ #1 \textbf{times}}{\algorithmicend}

\settopmatter{printacmref=true}
\begin{document}

%%
%% The "title" command has an optional parameter,
%% allowing the author to define a "short title" to be used in page headers.
\title{From Features to Transformers: \\ Redefining Ranking for Scalable Impact}

%%
%% The "author" command and its associated commands are used to define
%% the authors and their affiliations.
%% Of note is the shared affiliation of the first two authors, and the
%% "authornote" and "authornotemark" commands
%% used to denote shared contribution to the research.

\author{Fedor Borisyuk}
\authornote{Authors contributed equally to this research.}
\author{Lars Hertel}
\authornotemark[1]
\author{Ganesh Parameswaran}
\authornotemark[1]
\author{Gaurav Srivastava}
\authornotemark[1]
\author{Sudarshan Ramanujam}
\authornotemark[1]
\affiliation{%\small
 \institution{LinkedIn}
 \city{Mountain View}
 \state{CA}
 \country{USA}}
\email{fedorvb@gmail.com}

\author{Borja Ocejo}
\authornotemark[1]
\author{Peng Du}
\author{Andrei Akterskii}
\author{Neil Daftary}
\author{Shao Tang}
\affiliation{%\small
 \institution{LinkedIn}
 \city{Mountain View}
 \state{CA}
 \country{USA}}
 \email{bocejo@linkedin.com}
%\email{pedu@linkedin.com}

% bocejo@linkedin.com

\author{Daqi Sun}
\author{Charles Xiao}
\author{Deepesh Nathani}
\author{Mohit Kothari}
\author{Yun Dai}
\author{Aman Gupta}
\affiliation{%\small
 \institution{LinkedIn}
 \city{Mountain View}
 \state{CA}
 \country{USA}}
\email{daqsun@linkedin.com}

%\author{Cyrus Asgari}
%\author{Aditya Ayier}
%\author{Deepesh Nathani}
%\author{Mohit Kothari}
%\author{Yun Dai}
%\author{Aman Gupta}
%\affiliation{%\small
% \institution{LinkedIn}
% \city{Mountain View}
% \state{CA}
% \country{USA}}
%\email{dnathani@linkedin.com}

%%
%% By default, the full list of authors will be used in the page
%% headers. Often, this list is too long, and will overlap
%% other information printed in the page headers. This command allows
%% the author to define a more concise list
%% of authors' names for this purpose.
\renewcommand{\shortauthors}{Fedor Borisyuk et al.}
%% No italics, no superscripts
%% Use footnote or author note to identify equal contribution and/or contact author info

%%
%% The abstract is a short summary of the work to be presented in the
%% article.
\begin{abstract}
\begin{abstract}  
Test time scaling is currently one of the most active research areas that shows promise after training time scaling has reached its limits.
Deep-thinking (DT) models are a class of recurrent models that can perform easy-to-hard generalization by assigning more compute to harder test samples.
However, due to their inability to determine the complexity of a test sample, DT models have to use a large amount of computation for both easy and hard test samples.
Excessive test time computation is wasteful and can cause the ``overthinking'' problem where more test time computation leads to worse results.
In this paper, we introduce a test time training method for determining the optimal amount of computation needed for each sample during test time.
We also propose Conv-LiGRU, a novel recurrent architecture for efficient and robust visual reasoning. 
Extensive experiments demonstrate that Conv-LiGRU is more stable than DT, effectively mitigates the ``overthinking'' phenomenon, and achieves superior accuracy.
\end{abstract}  

\end{abstract}

%%
%% The code below is generated by the tool at http://dl.acm.org/ccs.cfm.
%% Please copy and paste the code instead of the example below.
%%
\begin{CCSXML}
<ccs2012>
<concept>
<concept_id>10010147.10010257.10010293.10010294</concept_id>
<concept_desc>Computing methodologies~Neural networks</concept_desc>
<concept_significance>500</concept_significance>
</concept>
<concept>
<concept_id>10002951.10003317.10003347.10003350</concept_id>
<concept_desc>Information systems~Recommender systems</concept_desc>
<concept_significance>500</concept_significance>
</concept>
<concept>
<concept_id>10002951.10003317.10003338.10003343</concept_id>
<concept_desc>Information systems~Learning to rank</concept_desc>
<concept_significance>500</concept_significance>
</concept>
</ccs2012>
\end{CCSXML}

\ccsdesc[500]{Computing methodologies~Neural networks}
\ccsdesc[500]{Information systems~Recommender systems}
\ccsdesc[500]{Information systems~Learning to rank}


%%
%% Keywords. The author(s) should pick words that accurately describe
%% the work being presented. Separate the keywords with commas.
\keywords{Large Scale Ranking, Deep Neural Networks}

%%
%% This command processes the author and affiliation and title
%% information and builds the first part of the formatted document.
\maketitle
%\def\thefootnote{*}\footnotetext{Work done while at LinkedIn.}
\section{Introduction}\label{sec:intro}
\section{Introduction}


\begin{figure}[t]
\centering
\includegraphics[width=0.6\columnwidth]{figures/evaluation_desiderata_V5.pdf}
\vspace{-0.5cm}
\caption{\systemName is a platform for conducting realistic evaluations of code LLMs, collecting human preferences of coding models with real users, real tasks, and in realistic environments, aimed at addressing the limitations of existing evaluations.
}
\label{fig:motivation}
\end{figure}

\begin{figure*}[t]
\centering
\includegraphics[width=\textwidth]{figures/system_design_v2.png}
\caption{We introduce \systemName, a VSCode extension to collect human preferences of code directly in a developer's IDE. \systemName enables developers to use code completions from various models. The system comprises a) the interface in the user's IDE which presents paired completions to users (left), b) a sampling strategy that picks model pairs to reduce latency (right, top), and c) a prompting scheme that allows diverse LLMs to perform code completions with high fidelity.
Users can select between the top completion (green box) using \texttt{tab} or the bottom completion (blue box) using \texttt{shift+tab}.}
\label{fig:overview}
\end{figure*}

As model capabilities improve, large language models (LLMs) are increasingly integrated into user environments and workflows.
For example, software developers code with AI in integrated developer environments (IDEs)~\citep{peng2023impact}, doctors rely on notes generated through ambient listening~\citep{oberst2024science}, and lawyers consider case evidence identified by electronic discovery systems~\citep{yang2024beyond}.
Increasing deployment of models in productivity tools demands evaluation that more closely reflects real-world circumstances~\citep{hutchinson2022evaluation, saxon2024benchmarks, kapoor2024ai}.
While newer benchmarks and live platforms incorporate human feedback to capture real-world usage, they almost exclusively focus on evaluating LLMs in chat conversations~\citep{zheng2023judging,dubois2023alpacafarm,chiang2024chatbot, kirk2024the}.
Model evaluation must move beyond chat-based interactions and into specialized user environments.



 

In this work, we focus on evaluating LLM-based coding assistants. 
Despite the popularity of these tools---millions of developers use Github Copilot~\citep{Copilot}---existing
evaluations of the coding capabilities of new models exhibit multiple limitations (Figure~\ref{fig:motivation}, bottom).
Traditional ML benchmarks evaluate LLM capabilities by measuring how well a model can complete static, interview-style coding tasks~\citep{chen2021evaluating,austin2021program,jain2024livecodebench, white2024livebench} and lack \emph{real users}. 
User studies recruit real users to evaluate the effectiveness of LLMs as coding assistants, but are often limited to simple programming tasks as opposed to \emph{real tasks}~\citep{vaithilingam2022expectation,ross2023programmer, mozannar2024realhumaneval}.
Recent efforts to collect human feedback such as Chatbot Arena~\citep{chiang2024chatbot} are still removed from a \emph{realistic environment}, resulting in users and data that deviate from typical software development processes.
We introduce \systemName to address these limitations (Figure~\ref{fig:motivation}, top), and we describe our three main contributions below.


\textbf{We deploy \systemName in-the-wild to collect human preferences on code.} 
\systemName is a Visual Studio Code extension, collecting preferences directly in a developer's IDE within their actual workflow (Figure~\ref{fig:overview}).
\systemName provides developers with code completions, akin to the type of support provided by Github Copilot~\citep{Copilot}. 
Over the past 3 months, \systemName has served over~\completions suggestions from 10 state-of-the-art LLMs, 
gathering \sampleCount~votes from \userCount~users.
To collect user preferences,
\systemName presents a novel interface that shows users paired code completions from two different LLMs, which are determined based on a sampling strategy that aims to 
mitigate latency while preserving coverage across model comparisons.
Additionally, we devise a prompting scheme that allows a diverse set of models to perform code completions with high fidelity.
See Section~\ref{sec:system} and Section~\ref{sec:deployment} for details about system design and deployment respectively.



\textbf{We construct a leaderboard of user preferences and find notable differences from existing static benchmarks and human preference leaderboards.}
In general, we observe that smaller models seem to overperform in static benchmarks compared to our leaderboard, while performance among larger models is mixed (Section~\ref{sec:leaderboard_calculation}).
We attribute these differences to the fact that \systemName is exposed to users and tasks that differ drastically from code evaluations in the past. 
Our data spans 103 programming languages and 24 natural languages as well as a variety of real-world applications and code structures, while static benchmarks tend to focus on a specific programming and natural language and task (e.g. coding competition problems).
Additionally, while all of \systemName interactions contain code contexts and the majority involve infilling tasks, a much smaller fraction of Chatbot Arena's coding tasks contain code context, with infilling tasks appearing even more rarely. 
We analyze our data in depth in Section~\ref{subsec:comparison}.



\textbf{We derive new insights into user preferences of code by analyzing \systemName's diverse and distinct data distribution.}
We compare user preferences across different stratifications of input data (e.g., common versus rare languages) and observe which affect observed preferences most (Section~\ref{sec:analysis}).
For example, while user preferences stay relatively consistent across various programming languages, they differ drastically between different task categories (e.g. frontend/backend versus algorithm design).
We also observe variations in user preference due to different features related to code structure 
(e.g., context length and completion patterns).
We open-source \systemName and release a curated subset of code contexts.
Altogether, our results highlight the necessity of model evaluation in realistic and domain-specific settings.






\section{RELATED WORK}
\label{sec:relatedwork}
In this section, we describe the previous works related to our proposal, which are divided into two parts. In Section~\ref{sec:relatedwork_exoplanet}, we present a review of approaches based on machine learning techniques for the detection of planetary transit signals. Section~\ref{sec:relatedwork_attention} provides an account of the approaches based on attention mechanisms applied in Astronomy.\par

\subsection{Exoplanet detection}
\label{sec:relatedwork_exoplanet}
Machine learning methods have achieved great performance for the automatic selection of exoplanet transit signals. One of the earliest applications of machine learning is a model named Autovetter \citep{MCcauliff}, which is a random forest (RF) model based on characteristics derived from Kepler pipeline statistics to classify exoplanet and false positive signals. Then, other studies emerged that also used supervised learning. \cite{mislis2016sidra} also used a RF, but unlike the work by \citet{MCcauliff}, they used simulated light curves and a box least square \citep[BLS;][]{kovacs2002box}-based periodogram to search for transiting exoplanets. \citet{thompson2015machine} proposed a k-nearest neighbors model for Kepler data to determine if a given signal has similarity to known transits. Unsupervised learning techniques were also applied, such as self-organizing maps (SOM), proposed \citet{armstrong2016transit}; which implements an architecture to segment similar light curves. In the same way, \citet{armstrong2018automatic} developed a combination of supervised and unsupervised learning, including RF and SOM models. In general, these approaches require a previous phase of feature engineering for each light curve. \par

%DL is a modern data-driven technology that automatically extracts characteristics, and that has been successful in classification problems from a variety of application domains. The architecture relies on several layers of NNs of simple interconnected units and uses layers to build increasingly complex and useful features by means of linear and non-linear transformation. This family of models is capable of generating increasingly high-level representations \citep{lecun2015deep}.

The application of DL for exoplanetary signal detection has evolved rapidly in recent years and has become very popular in planetary science.  \citet{pearson2018} and \citet{zucker2018shallow} developed CNN-based algorithms that learn from synthetic data to search for exoplanets. Perhaps one of the most successful applications of the DL models in transit detection was that of \citet{Shallue_2018}; who, in collaboration with Google, proposed a CNN named AstroNet that recognizes exoplanet signals in real data from Kepler. AstroNet uses the training set of labelled TCEs from the Autovetter planet candidate catalog of Q1–Q17 data release 24 (DR24) of the Kepler mission \citep{catanzarite2015autovetter}. AstroNet analyses the data in two views: a ``global view'', and ``local view'' \citep{Shallue_2018}. \par


% The global view shows the characteristics of the light curve over an orbital period, and a local view shows the moment at occurring the transit in detail

%different = space-based

Based on AstroNet, researchers have modified the original AstroNet model to rank candidates from different surveys, specifically for Kepler and TESS missions. \citet{ansdell2018scientific} developed a CNN trained on Kepler data, and included for the first time the information on the centroids, showing that the model improves performance considerably. Then, \citet{osborn2020rapid} and \citet{yu2019identifying} also included the centroids information, but in addition, \citet{osborn2020rapid} included information of the stellar and transit parameters. Finally, \citet{rao2021nigraha} proposed a pipeline that includes a new ``half-phase'' view of the transit signal. This half-phase view represents a transit view with a different time and phase. The purpose of this view is to recover any possible secondary eclipse (the object hiding behind the disk of the primary star).


%last pipeline applies a procedure after the prediction of the model to obtain new candidates, this process is carried out through a series of steps that include the evaluation with Discovery and Validation of Exoplanets (DAVE) \citet{kostov2019discovery} that was adapted for the TESS telescope.\par
%



\subsection{Attention mechanisms in astronomy}
\label{sec:relatedwork_attention}
Despite the remarkable success of attention mechanisms in sequential data, few papers have exploited their advantages in astronomy. In particular, there are no models based on attention mechanisms for detecting planets. Below we present a summary of the main applications of this modeling approach to astronomy, based on two points of view; performance and interpretability of the model.\par
%Attention mechanisms have not yet been explored in all sub-areas of astronomy. However, recent works show a successful application of the mechanism.
%performance

The application of attention mechanisms has shown improvements in the performance of some regression and classification tasks compared to previous approaches. One of the first implementations of the attention mechanism was to find gravitational lenses proposed by \citet{thuruthipilly2021finding}. They designed 21 self-attention-based encoder models, where each model was trained separately with 18,000 simulated images, demonstrating that the model based on the Transformer has a better performance and uses fewer trainable parameters compared to CNN. A novel application was proposed by \citet{lin2021galaxy} for the morphological classification of galaxies, who used an architecture derived from the Transformer, named Vision Transformer (VIT) \citep{dosovitskiy2020image}. \citet{lin2021galaxy} demonstrated competitive results compared to CNNs. Another application with successful results was proposed by \citet{zerveas2021transformer}; which first proposed a transformer-based framework for learning unsupervised representations of multivariate time series. Their methodology takes advantage of unlabeled data to train an encoder and extract dense vector representations of time series. Subsequently, they evaluate the model for regression and classification tasks, demonstrating better performance than other state-of-the-art supervised methods, even with data sets with limited samples.

%interpretation
Regarding the interpretability of the model, a recent contribution that analyses the attention maps was presented by \citet{bowles20212}, which explored the use of group-equivariant self-attention for radio astronomy classification. Compared to other approaches, this model analysed the attention maps of the predictions and showed that the mechanism extracts the brightest spots and jets of the radio source more clearly. This indicates that attention maps for prediction interpretation could help experts see patterns that the human eye often misses. \par

In the field of variable stars, \citet{allam2021paying} employed the mechanism for classifying multivariate time series in variable stars. And additionally, \citet{allam2021paying} showed that the activation weights are accommodated according to the variation in brightness of the star, achieving a more interpretable model. And finally, related to the TESS telescope, \citet{morvan2022don} proposed a model that removes the noise from the light curves through the distribution of attention weights. \citet{morvan2022don} showed that the use of the attention mechanism is excellent for removing noise and outliers in time series datasets compared with other approaches. In addition, the use of attention maps allowed them to show the representations learned from the model. \par

Recent attention mechanism approaches in astronomy demonstrate comparable results with earlier approaches, such as CNNs. At the same time, they offer interpretability of their results, which allows a post-prediction analysis. \par



\section{Model Architecture}\label{sec:overview}
\section{Overview}

\revision{In this section, we first explain the foundational concept of Hausdorff distance-based penetration depth algorithms, which are essential for understanding our method (Sec.~\ref{sec:preliminary}).
We then provide a brief overview of our proposed RT-based penetration depth algorithm (Sec.~\ref{subsec:algo_overview}).}



\section{Preliminaries }
\label{sec:Preliminaries}

% Before we introduce our method, we first overview the important basics of 3D dynamic human modeling with Gaussian splatting. Then, we discuss the diffusion-based 3d generation techniques, and how they can be applied to human modeling.
% \ZY{I stopp here. TBC.}
% \subsection{Dynamic human modeling with Gaussian splatting}
\subsection{3D Gaussian Splatting}
3D Gaussian splatting~\cite{kerbl3Dgaussians} is an explicit scene representation that allows high-quality real-time rendering. The given scene is represented by a set of static 3D Gaussians, which are parameterized as follows: Gaussian center $x\in {\mathbb{R}^3}$, color $c\in {\mathbb{R}^3}$, opacity $\alpha\in {\mathbb{R}}$, spatial rotation in the form of quaternion $q\in {\mathbb{R}^4}$, and scaling factor $s\in {\mathbb{R}^3}$. Given these properties, the rendering process is represented as:
\begin{equation}
  I = Splatting(x, c, s, \alpha, q, r),
  \label{eq:splattingGA}
\end{equation}
where $I$ is the rendered image, $r$ is a set of query rays crossing the scene, and $Splatting(\cdot)$ is a differentiable rendering process. We refer readers to Kerbl et al.'s paper~\cite{kerbl3Dgaussians} for the details of Gaussian splatting. 



% \ZY{I would suggest move this part to the method part.}
% GaissianAvatar is a dynamic human generation model based on Gaussian splitting. Given a sequence of RGB images, this method utilizes fitted SMPLs and sampled points on its surface to obtain a pose-dependent feature map by a pose encoder. The pose-dependent features and a geometry feature are fed in a Gaussian decoder, which is employed to establish a functional mapping from the underlying geometry of the human form to diverse attributes of 3D Gaussians on the canonical surfaces. The parameter prediction process is articulated as follows:
% \begin{equation}
%   (\Delta x,c,s)=G_{\theta}(S+P),
%   \label{eq:gaussiandecoder}
% \end{equation}
%  where $G_{\theta}$ represents the Gaussian decoder, and $(S+P)$ is the multiplication of geometry feature S and pose feature P. Instead of optimizing all attributes of Gaussian, this decoder predicts 3D positional offset $\Delta{x} \in {\mathbb{R}^3}$, color $c\in\mathbb{R}^3$, and 3D scaling factor $ s\in\mathbb{R}^3$. To enhance geometry reconstruction accuracy, the opacity $\alpha$ and 3D rotation $q$ are set to fixed values of $1$ and $(1,0,0,0)$ respectively.
 
%  To render the canonical avatar in observation space, we seamlessly combine the Linear Blend Skinning function with the Gaussian Splatting~\cite{kerbl3Dgaussians} rendering process: 
% \begin{equation}
%   I_{\theta}=Splatting(x_o,Q,d),
%   \label{eq:splatting}
% \end{equation}
% \begin{equation}
%   x_o = T_{lbs}(x_c,p,w),
%   \label{eq:LBS}
% \end{equation}
% where $I_{\theta}$ represents the final rendered image, and the canonical Gaussian position $x_c$ is the sum of the initial position $x$ and the predicted offset $\Delta x$. The LBS function $T_{lbs}$ applies the SMPL skeleton pose $p$ and blending weights $w$ to deform $x_c$ into observation space as $x_o$. $Q$ denotes the remaining attributes of the Gaussians. With the rendering process, they can now reposition these canonical 3D Gaussians into the observation space.



\subsection{Score Distillation Sampling}
Score Distillation Sampling (SDS)~\cite{poole2022dreamfusion} builds a bridge between diffusion models and 3D representations. In SDS, the noised input is denoised in one time-step, and the difference between added noise and predicted noise is considered SDS loss, expressed as:

% \begin{equation}
%   \mathcal{L}_{SDS}(I_{\Phi}) \triangleq E_{t,\epsilon}[w(t)(\epsilon_{\phi}(z_t,y,t)-\epsilon)\frac{\partial I_{\Phi}}{\partial\Phi}],
%   \label{eq:SDSObserv}
% \end{equation}
\begin{equation}
    \mathcal{L}_{\text{SDS}}(I_{\Phi}) \triangleq \mathbb{E}_{t,\epsilon} \left[ w(t) \left( \epsilon_{\phi}(z_t, y, t) - \epsilon \right) \frac{\partial I_{\Phi}}{\partial \Phi} \right],
  \label{eq:SDSObservGA}
\end{equation}
where the input $I_{\Phi}$ represents a rendered image from a 3D representation, such as 3D Gaussians, with optimizable parameters $\Phi$. $\epsilon_{\phi}$ corresponds to the predicted noise of diffusion networks, which is produced by incorporating the noise image $z_t$ as input and conditioning it with a text or image $y$ at timestep $t$. The noise image $z_t$ is derived by introducing noise $\epsilon$ into $I_{\Phi}$ at timestep $t$. The loss is weighted by the diffusion scheduler $w(t)$. 
% \vspace{-3mm}

\subsection{Overview of the RTPD Algorithm}\label{subsec:algo_overview}
Fig.~\ref{fig:Overview} presents an overview of our RTPD algorithm.
It is grounded in the Hausdorff distance-based penetration depth calculation method (Sec.~\ref{sec:preliminary}).
%, similar to that of Tang et al.~\shortcite{SIG09HIST}.
The process consists of two primary phases: penetration surface extraction and Hausdorff distance calculation.
We leverage the RTX platform's capabilities to accelerate both of these steps.

\begin{figure*}[t]
    \centering
    \includegraphics[width=0.8\textwidth]{Image/overview.pdf}
    \caption{The overview of RT-based penetration depth calculation algorithm overview}
    \label{fig:Overview}
\end{figure*}

The penetration surface extraction phase focuses on identifying the overlapped region between two objects.
\revision{The penetration surface is defined as a set of polygons from one object, where at least one of its vertices lies within the other object. 
Note that in our work, we focus on triangles rather than general polygons, as they are processed most efficiently on the RTX platform.}
To facilitate this extraction, we introduce a ray-tracing-based \revision{Point-in-Polyhedron} test (RT-PIP), significantly accelerated through the use of RT cores (Sec.~\ref{sec:RT-PIP}).
This test capitalizes on the ray-surface intersection capabilities of the RTX platform.
%
Initially, a Geometry Acceleration Structure (GAS) is generated for each object, as required by the RTX platform.
The RT-PIP module takes the GAS of one object (e.g., $GAS_{A}$) and the point set of the other object (e.g., $P_{B}$).
It outputs a set of points (e.g., $P_{\partial B}$) representing the penetration region, indicating their location inside the opposing object.
Subsequently, a penetration surface (e.g., $\partial B$) is constructed using this point set (e.g., $P_{\partial B}$) (Sec.~\ref{subsec:surfaceGen}).
%
The generated penetration surfaces (e.g., $\partial A$ and $\partial B$) are then forwarded to the next step. 

The Hausdorff distance calculation phase utilizes the ray-surface intersection test of the RTX platform (Sec.~\ref{sec:RT-Hausdorff}) to compute the Hausdorff distance between two objects.
We introduce a novel Ray-Tracing-based Hausdorff DISTance algorithm, RT-HDIST.
It begins by generating GAS for the two penetration surfaces, $P_{\partial A}$ and $P_{\partial B}$, derived from the preceding step.
RT-HDIST processes the GAS of a penetration surface (e.g., $GAS_{\partial A}$) alongside the point set of the other penetration surface (e.g., $P_{\partial B}$) to compute the penetration depth between them.
The algorithm operates bidirectionally, considering both directions ($\partial A \to \partial B$ and $\partial B \to \partial A$).
The final penetration depth between the two objects, A and B, is determined by selecting the larger value from these two directional computations.

%In the Hausdorff distance calculation step, we compute the Hausdorff distance between given two objects using a ray-surface-intersection test. (Sec.~\ref{sec:RT-Hausdorff}) Initially, we construct the GAS for both $\partial A$ and $\partial B$ to utilize the RT-core effectively. The RT-based Hausdorff distance algorithms then determine the Hausdorff distance by processing the GAS of one object (e.g. $GAS_{\partial A}$) and set of the vertices of the other (e.g. $P_{\partial B}$). Following the Hausdorff distance definition (Eq.~\ref{equation:hausdorff_definition}), we compute the Hausdorff distance to both directions ($\partial A \to \partial B$) and ($\partial B \to \partial A$). As a result, the bigger one is the final Hausdorff distance, and also it is the penetration depth between input object $A$ and $B$.


%the proposed RT-based penetration depth calculation pipeline.
%Our proposed methods adopt Tang's Hausdorff-based penetration depth methods~\cite{SIG09HIST}. The pipeline is divided into the penetration surface extraction step and the Hausdorff distance calculation between the penetration surface steps. However, since Tang's approach is not suitable for the RT platform in detail, we modified and applied it with appropriate methods.

%The penetration surface extraction step is extracting overlapped surfaces on other objects. To utilize the RT core, we use the ray-intersection-based PIP(Point-In-Polygon) algorithms instead of collision detection between two objects which Tang et al.~\cite{SIG09HIST} used. (Sec.~\ref{sec:RT-PIP})
%RT core-based PIP test uses a ray-surface intersection test. For purpose this, we generate the GAS(Geometry Acceleration Structure) for each object. RT core-based PIP test takes the GAS of one object (e.g. $GAS_{A}$) and a set of vertex of another one (e.g. $P_{B}$). Then this computes the penetrated vertex set of another one (e.g. $P_{\partial B}$). To calculate the Hausdorff distance, these vertex sets change to objects constructed by penetrated surface (e.g. $\partial B$). Finally, the two generated overlapped surface objects $\partial A$ and $\partial B$ are used in the Hausdorff distance calculation step.

% \section{Content Modeling}\label{sec:model}
% \input{modeling_techniques.tex}

%\vspace{-0.8em}
\section{System Architecture}\label{sec:system_arch}
{\systemname} helps to significantly simplify our system. As we are able to reduce dependency on the counter features, and reduce number to less than ten features from hundreds. 
Serving of transformer model requires to store near-line member and item activity as shown on the Figure \ref{fig:system_arch}.
Within the system we utilize variety of item and member embeddings generated by GNN \cite{LiGNN_paper}, LLM \cite{PEv3_paper}, as well as ID embeddings stored and served within the model. Given the user request, the system retrieves the top candidates using the model-based retrieval system \cite{Linr_paper}, then the top K candidates are evaluated using the second layer {\systemname} scoring model. {\systemname} takes as input Items features and Member Context.


%\textcolor{red}{\textbf{Fedor}} intro to system arch TO FILL IN 
%\subsection{System design}
%\textcolor{red}{\textbf{Sudarshan / Borja}} TO FILL IN 


\subsubsection{{\systemname} training}
Training of the {\systemname} model is GPU memory intensive, requiring us to leverage several optimizations. Firstly, long user item histories interleaved with corresponding actions result in sequence lengths of up to 2048 for our largest model. We used Flash Attention and mixed precision to reduce memory consumption.

Secondly, {\systemname} leverages several ID embedding features as shown in Table \ref{tab:single_feature_evaluation}. It is important to scale up ID embedding dimensions alongside other hyper-parameters. In order to accommodate embedding tables larger than the memory of one GPU we use column-wise splitting of embedding tables and distribution across multiple devices. The latter allowed us to scale up embedding tables to arbitrarily large sizes.


\begin{figure}
    \centering
    \includegraphics[width=\linewidth]{figures/ARTAL_arch.png}
    \caption{{\systemname} system architecture.}
    \label{fig:system_arch}
\end{figure}
%\textcolor{red}{\textbf{Fedor}} TO FILL IN 

\subsubsection{Optimization of inference}\label{optimization_inference}
Naive inference can be expensive in the {\systemname} model due to the complexity of applying many transformer layers on a long user action history for many candidate items. \cite{HSTU_paper_zhai24a, efficient_transact_paper} provide ways to amortize the computation of the history across all candidate items. In our case, the historical attention implementation allows items to attend to other items only if they are from earlier sessions. For this reason we can easily infer all candidates at the same time, automatically amortizing the inference of the history across all candidates. 

%The pointwise model used multithreading for independent item scoring at inference, but introducing a setwise attention increased latency due to: %(1) additional inference calls, (2) 
%loss of multithreading since items needed to be scored together, and  latency scaling with the number of setwise-scored items. LinkedIn's ranking inference remains CPU-based, with GPU used for retrieval \cite{Linr_paper}. 

Feed metrics are sensitive to latency, requiring careful optimizations to stay within bounds: (1) Split Scoring: Setwise scoring was applied only to the top 10 posts after point-wise scoring, where we decompose point-wise and setwise model weights separately; (2) Score Combination: Setwise scores were added to pointwise scores for items 1–10, preserving the order for items 11–end; (3) Simplification: Rule-based diversity rerankers were disabled; (4) Unified Inference: Pointwise and setwise parts of the model were stitched with a rewriter, enabling single inference to score the entire model.
To scale online scoring, all items were passed using the batch size dimension, repackaged as a listwise dataset, and scored via the trained model. These optimizations reduced incremental latency cost to 10ms (p90), ensuring minimal impact on member experience.

%The pointwise model leveraged multithreading extensively to score items at inference time since scoring each item was independent. After training a new setwise model that operates on top of the existing pointwise model, the model latency goes up due to the following reasons: (1) Extra inference call, (2) Multi-threading was no longer possible since all items needed to be scored together, (3) Latency increase was proportional to the number of items that were scored set-wise.

%One point to note is that the LinkedIn inference for ranking currently is still done mostly on CPUs and GPU inference is used for retrieval \cite{Linr_paper}. Feed metrics are sensitive to latency and we need to be careful to not overshoot the latency bounds. In order to achieve it, the following optimizations were done:

%LinkedIn's ranking inference is mostly CPU-based, with GPU inference used for retrieval [5]. Feed metrics are sensitive to latency, requiring careful optimizations to stay within bounds: (1) Split Scoring: Setwise scoring is applied only to the top 10 organic updates after pointwise scoring; (2) Score Combination: Setwise scores are added to pointwise scores for items 1–10, ensuring their scores remain higher than items 11–end, preserving the pointwise order beyond the top 10; (3) Simplification: Existing rule-based diversity rerankers were disabled;
%(4) Unified Inference: Pointwise and setwise models were stitched with a rewriter, enabling a single-threaded inference call to score the entire model at once, reducing latency.

%\begin{enumerate}
 %   \item Instead of scoring all the items setwise, we chose to split point-wise and setwise models, and apply setwise scoring to the top 10 organic updates following after the pointwise scoring.
 %   \item The setwise model scores was added to the pointwise model scores from item 1-10 in order to ensure that the top 10 scores were always greater than item 11 - end and hence ordering of item 11 - end was preserved based on pointwise model score.
 %   \item Disabled existing rule based diversity rerankers.
 %   \item Instead of having 2 separate calls to the model cloud \cite{Linr_paper}, we chose to stitch the point-wise model with the set-wise model using a rewriter. During inference time, a single thread is now leveraged to score the entire model in one shot. 
%\end{enumerate}

%In industry, TF rankling is usually leveraged to train and score setwise models.

%In order to simplify our online scoring at scale, we chose to simplify the problem. The scoring engine simply passes all the items using the batch size dimension. We repackage this to a listwise dataset and then pass it through the trained model for inference. With the mentioned optimizations, we were able to bring down the latency increase to ~10ms (p90m measurement) which did not severely impact the member experience.
%However, in order to simplify our online scoring at scale, we chose to simplify the problem. The scoring engine simply passes all the items using the batch size dimension that is available for TF models. Internally in the TF model, we repackage this to a listwise dataset and then pass it through the trained model for inference. 

%With the mentioned optimizations, we were able to bring down the latency increase to ~10ms (p90m measurement) which did not severely impact the member experience.

%\subsubsection{Optimization of {\systemname} inference}\label{optimization_inference_GR}
%Naive inference can be expensive in the GR model due to the complexity of applying many transformer layers on a long user action history for many candidate items. \cite{HSTU_paper_zhai24a, efficient_transact_paper} provide ways to amortize the computation of the history across all candidate items. In our case, the historical attention implementation allows items to attend to other items only if they are from earlier sessions. For this reason we can easily infer all candidates at the same time, automatically amortizing the inference of the history across all candidates. 


\section{Experimental Analysis}
\label{sec:exp}
We now describe in detail our experimental analysis. The experimental section is organized as follows:
%\begin{enumerate}[noitemsep,topsep=0pt,parsep=0pt,partopsep=0pt,leftmargin=0.5cm]
%\item 

\noindent In {\bf 
Section~\ref{exp:setup}}, we introduce the datasets and methods to evaluate the previously defined accuracy measures.

%\item
\noindent In {\bf 
Section~\ref{exp:qual}}, we illustrate the limitations of existing measures with some selected qualitative examples.

%\item 
\noindent In {\bf 
Section~\ref{exp:quant}}, we continue by measuring quantitatively the benefits of our proposed measures in terms of {\it robustness} to lag, noise, and normal/abnormal ratio.

%\item 
\noindent In {\bf 
Section~\ref{exp:separability}}, we evaluate the {\it separability} degree of accurate and inaccurate methods, using the existing and our proposed approaches.

%\item
\noindent In {\bf 
Section~\ref{sec:entropy}}, we conduct a {\it consistency} evaluation, in which we analyze the variation of ranks that an AD method can have with an accuracy measures used.

%\item 
\noindent In {\bf 
Section~\ref{sec:exectime}}, we conduct an {\it execution time} evaluation, in which we analyze the impact of different parameters related to the accuracy measures and the time series characteristics. 
We focus especially on the comparison of the different VUS implementations.
%\end{enumerate}

\begin{table}[tb]
\caption{Summary characteristics (averaged per dataset) of the public datasets of TSB-UAD (S.: Size, Ano.: Anomalies, Ab.: Abnormal, Den.: Density)}
\label{table:charac}
%\vspace{-0.2cm}
\footnotesize
\begin{center}
\scalebox{0.82}{
\begin{tabular}{ |r|r|r|r|r|r|} 
 \hline
\textbf{\begin{tabular}[c]{@{}c@{}}Dataset \end{tabular}} & 
\textbf{\begin{tabular}[c]{@{}c@{}}S. \end{tabular}} & 
\textbf{\begin{tabular}[c]{c@{}} Len.\end{tabular}} & 
\textbf{\begin{tabular}[c]{c@{}} \# \\ Ano. \end{tabular}} &
\textbf{\begin{tabular}[c]{c@{}c@{}} \# \\ Ab. \\ Points\end{tabular}} &
\textbf{\begin{tabular}[c]{c@{}c@{}} Ab. \\ Den. \\ (\%)\end{tabular}} \\ \hline
Dodgers \cite{10.1145/1150402.1150428} & 1 & 50400   & 133.0     & 5612.0  &11.14 \\ \hline
SED \cite{doi:10.1177/1475921710395811}& 1 & 100000   & 75.0     & 3750.0  & 3.7\\ \hline
ECG \cite{goldberger_physiobank_2000}   & 52 & 230351  & 195.6     & 15634.0  &6.8 \\ \hline
IOPS \cite{IOPS}   & 58 & 102119  & 46.5     & 2312.3   &2.1 \\ \hline
KDD21 \cite{kdd} & 250 &77415   & 1      & 196.5   &0.56 \\ \hline
MGAB \cite{markus_thill_2020_3762385}   & 10 & 100000  & 10.0     & 200.0   &0.20 \\ \hline
NAB \cite{ahmad_unsupervised_2017}   & 58 & 6301   & 2.0      & 575.5   &8.8 \\ \hline
NASA-M. \cite{10.1145/3449726.3459411}   & 27 & 2730   & 1.33      & 286.3   &11.97 \\ \hline
NASA-S. \cite{10.1145/3449726.3459411}   & 54 & 8066   & 1.26      & 1032.4   &12.39 \\ \hline
SensorS. \cite{YAO20101059}   & 23 & 27038   & 11.2     & 6110.4   &22.5 \\ \hline
YAHOO \cite{yahoo}  & 367 & 1561   & 5.9      & 10.7   &0.70 \\ \hline 
\end{tabular}}
\end{center}
\end{table}











\subsection{Experimental Setup and Settings}
\label{exp:setup}
%\vspace{-0.1cm}

\begin{figure*}[tb]
  \centering
  \includegraphics[width=1\linewidth]{figures/quality.pdf}
  %\vspace{-0.7cm}
  \caption{Comparison of evaluation measures (proposed measures illustrated in subplots (b,c,d,e); all others summarized in subplots (f)) on two examples ((A)AE and OCSM applied on MBA(805) and (B) LOF and OCSVM applied on MBA(806)), illustrating the limitations of existing measures for scores with noise or containing a lag. }
  \label{fig:quality}
  %\vspace{-0.1cm}
\end{figure*}

We implemented the experimental scripts in Python 3.8 with the following main dependencies: sklearn 0.23.0, tensorflow 2.3.0, pandas 1.2.5, and networkx 2.6.3. In addition, we used implementations from our TSB-UAD benchmark suite.\footnote{\scriptsize \url{https://www.timeseries.org/TSB-UAD}} For reproducibility purposes, we make our datasets and code available.\footnote{\scriptsize \url{https://www.timeseries.org/VUS}}
\newline \textbf{Datasets: } For our evaluation purposes, we use the public datasets identified in our TSB-UAD benchmark. The latter corresponds to $10$ datasets proposed in the past decades in the literature containing $900$ time series with labeled anomalies. Specifically, each point in every time series is labeled as normal or abnormal. Table~\ref{table:charac} summarizes relevant characteristics of the datasets, including their size, length, and statistics about the anomalies. In more detail:

\begin{itemize}
    \item {\bf SED}~\cite{doi:10.1177/1475921710395811}, from the NASA Rotary Dynamics Laboratory, records disk revolutions measured over several runs (3K rpm speed).
	\item {\bf ECG}~\cite{goldberger_physiobank_2000} is a standard electrocardiogram dataset and the anomalies represent ventricular premature contractions. MBA(14046) is split to $47$ series.
	\item {\bf IOPS}~\cite{IOPS} is a dataset with performance indicators that reflect the scale, quality of web services, and health status of a machine.
	\item {\bf KDD21}~\cite{kdd} is a composite dataset released in a SIGKDD 2021 competition with 250 time series.
	\item {\bf MGAB}~\cite{markus_thill_2020_3762385} is composed of Mackey-Glass time series with non-trivial anomalies. Mackey-Glass data series exhibit chaotic behavior that is difficult for the human eye to distinguish.
	\item {\bf NAB}~\cite{ahmad_unsupervised_2017} is composed of labeled real-world and artificial time series including AWS server metrics, online advertisement clicking rates, real time traffic data, and a collection of Twitter mentions of large publicly-traded companies.
	\item {\bf NASA-SMAP} and {\bf NASA-MSL}~\cite{10.1145/3449726.3459411} are two real spacecraft telemetry data with anomalies from Soil Moisture Active Passive (SMAP) satellite and Curiosity Rover on Mars (MSL).
	\item {\bf SensorScope}~\cite{YAO20101059} is a collection of environmental data, such as temperature, humidity, and solar radiation, collected from a sensor measurement system.
	\item {\bf Yahoo}~\cite{yahoo} is a dataset consisting of real and synthetic time series based on the real production traffic to some of the Yahoo production systems.
\end{itemize}


\textbf{Anomaly Detection Methods: }  For the experimental evaluation, we consider the following baselines. 

\begin{itemize}
\item {\bf Isolation Forest (IForest)}~\cite{liu_isolation_2008} constructs binary trees based on random space splitting. The nodes (subsequences in our specific case) with shorter path lengths to the root (averaged over every random tree) are more likely to be anomalies. 
\item {\bf The Local Outlier Factor (LOF)}~\cite{breunig_lof_2000} computes the ratio of the neighbor density to the local density. 
\item {\bf Matrix Profile (MP)}~\cite{yeh_time_2018} detects as anomaly the subsequence with the most significant 1-NN distance. 
\item {\bf NormA}~\cite{boniol_unsupervised_2021} identifies the normal patterns based on clustering and calculates each point's distance to normal patterns weighted using statistical criteria. 
\item {\bf Principal Component Analysis (PCA)}~\cite{aggarwal_outlier_2017} projects data to a lower-dimensional hyperplane. Outliers are points with a large distance from this plane. 
\item {\bf Autoencoder (AE)} \cite{10.1145/2689746.2689747} projects data to a lower-dimensional space and reconstructs it. Outliers are expected to have larger reconstruction errors. 
\item {\bf LSTM-AD}~\cite{malhotra_long_2015} use an LSTM network that predicts future values from the current subsequence. The prediction error is used to identify anomalies.
\item {\bf Polynomial Approximation (POLY)} \cite{li_unifying_2007} fits a polynomial model that tries to predict the values of the data series from the previous subsequences. Outliers are detected with the prediction error. 
\item {\bf CNN} \cite{8581424} built, using a convolutional deep neural network, a correlation between current and previous subsequences, and outliers are detected by the deviation between the prediction and the actual value. 
\item {\bf One-class Support Vector Machines (OCSVM)} \cite{scholkopf_support_1999} is a support vector method that fits a training dataset and finds the normal data's boundary.
\end{itemize}

\subsection{Qualitative Analysis}
\label{exp:qual}



We first use two examples to demonstrate qualitatively the limitations of existing accuracy evaluation measures in the presence of lag and noise, and to motivate the need for a new approach. 
These two examples are depicted in Figure~\ref{fig:quality}. 
The first example, in Figure~\ref{fig:quality}(A), corresponds to OCSVM and AE on the MBA(805) dataset (named MBA\_ECG805\_data.out in the ECG dataset). 

We observe in Figure~\ref{fig:quality}(A)(a.1) and (a.2) that both scores identify most of the anomalies (highlighted in red). However, the OCSVM score points to more false positives (at the end of the time series) and only captures small sections of the anomalies. On the contrary, the AE score points to fewer false positives and captures all abnormal subsequences. Thus we can conclude that, visually, AE should obtain a better accuracy score than OCSVM. Nevertheless, we also observe that the AE score is lagged with the labels and contains more noise. The latter has a significant impact on the accuracy of evaluation measures. First, Figure~\ref{fig:quality}(A)(c) is showing that AUC-PR is better for OCSM (0.73) than for AE (0.57). This is contradictory with what is visually observed from Figure~\ref{fig:quality}(A)(a.1) and (a.2). However, when using our proposed measure R-AUC-PR, OCSVM obtains a lower score (0.83) than AE (0.89). This confirms that, in this example, a buffer region before the labels helps to capture the true value of an anomaly score. Overall, Figure~\ref{fig:quality}(A)(f) is showing in green and red the evolution of accuracy score for the 13 accuracy measures for AE and OCSVM, respectively. The latter shows that, in addition to Precision@k and Precision, our proposed approach captures the quality order between the two methods well.

We now present a second example, on a different time series, illustrated in Figure~\ref{fig:quality}(B). 
In this case, we demonstrate the anomaly score of OCSVM and LOF (depicted in Figure~\ref{fig:quality}(B)(a.1) and (a.2)) applied on the MBA(806) dataset (named MBA\_ECG806\_data.out in the ECG dataset). 
We observe that both methods produce the same level of noise. However, LOF points to fewer false positives and captures more sections of the abnormal subsequences than OCSVM. 
Nevertheless, the LOF score is slightly lagged with the labels such that the maximum values in the LOF score are slightly outside of the labeled sections. 
Thus, as illustrated in Figure~\ref{fig:quality}(B)(f), even though we can visually consider that LOF is performing better than OCSM, all usual measures (Precision, Recall, F, precision@k, and AUC-PR) are judging OCSM better than AE. On the contrary, measures that consider lag (Rprecision, Rrecall, RF) rank the methods correctly. 
However, due to threshold issues, these measures are very close for the two methods. Overall, only AUC-ROC and our proposed measures give a higher score for LOF than for OCSVM.

\subsection{Quantitative Analysis}
\label{exp:case}

\begin{figure}[t]
  \centering
  \includegraphics[width=1\linewidth]{figures/eval_case_study.pdf}
  %\vspace*{-0.7cm}
  \caption{\commentRed{
  Comparison of evaluation measures for synthetic data examples across various scenarios. S8 represents the oracle case, where predictions perfectly align with labeled anomalies. Problematic cases are highlighted in the red region.}}
  %\vspace*{-0.5cm}
  \label{fig:eval_case_study}
\end{figure}
\commentRed{
We present the evaluation results for different synthetic data scenarios, as shown in Figure~\ref{fig:eval_case_study}. These scenarios range from S1, where predictions occur before the ground truth anomaly, to S12, where predictions fall within the ground truth region. The red-shaded regions highlight problematic cases caused by a lack of adaptability to lags. For instance, in scenarios S1 and S2, a slight shift in the prediction leads to measures (e.g., AUC-PR, F score) that fail to account for lags, resulting in a zero score for S1 and a significant discrepancy between the results of S1 and S2. Thus, we observe that our proposed VUS effectively addresses these issues and provides robust evaluations results.}

%\subsection{Quantitative Analysis}
%\subsection{Sensitivity and Separability Analysis}
\subsection{Robustness Analysis}
\label{exp:quant}


\begin{figure}[tb]
  \centering
  \includegraphics[width=1\linewidth]{figures/lag_sensitivity_analysis.pdf}
  %\vspace*{-0.7cm}
  \caption{For each method, we compute the accuracy measures 10 times with random lag $\ell \in [-0.25*\ell,0.25*\ell]$ injected in the anomaly score. We center the accuracy average to 0.}
  %\vspace*{-0.5cm}
  \label{fig:lagsensitivity}
\end{figure}

We have illustrated with specific examples several of the limitations of current measures. 
We now evaluate quantitatively the robustness of the proposed measures when compared to the currently used measures. 
We first evaluate the robustness to noise, lag, and normal versus abnormal points ratio. We then measure their ability to separate accurate and inaccurate methods.
%\newline \textbf{Sensitivity Analysis: } 
We first analyze the robustness of different approaches quantitatively to different factors: (i) lag, (ii) noise, and (iii) normal/abnormal ratio. As already mentioned, these factors are realistic. For instance, lag can be either introduced by the anomaly detection methods (such as methods that produce a score per subsequences are only high at the beginning of abnormal subsequences) or by human labeling approximation. Furthermore, even though lag and noises are injected, an optimal evaluation metric should not vary significantly. Therefore, we aim to measure the variance of the evaluation measures when we vary the lag, noise, and normal/abnormal ratio. We proceed as follows:

\begin{enumerate}[noitemsep,topsep=0pt,parsep=0pt,partopsep=0pt,leftmargin=0.5cm]
\item For each anomaly detection method, we first compute the anomaly score on a given time series.
\item We then inject either lag $l$, noise $n$ or change the normal/abnormal ratio $r$. For 10 different values of $l \in [-0.25*\ell,0.25*\ell]$, $n \in [-0.05*(max(S_T)-min(S_T)),0.05*(max(S_T)-min(S_T))]$ and $r \in [0.01,0.2]$, we compute the 13 different measures.
\item For each evaluation measure, we compute the standard deviation of the ten different values. Figure~\ref{fig:lagsensitivity}(b) depicts the different lag values for six AD methods applied on a data series in the ECG dataset.
\item We compute the average standard deviation for the 13 different AD quality measures. For example, figure~\ref{fig:lagsensitivity}(a) depicts the average standard deviation for ten different lag values over the AD methods applied on the MBA(805) time series.
\item We compute the average standard deviation for the every time series in each dataset (as illustrated in Figure~\ref{fig:sensitivity_per_data}(b to j) for nine datasets of the benchmark.
\item We compute the average standard deviation for the every dataset (as illustrated in Figure~\ref{fig:sensitivity_per_data}(a.1) for lag, Figure~\ref{fig:sensitivity_per_data}(a.2) for noise and Figure~\ref{fig:sensitivity_per_data}(a.3) for normal/abnormal ratio).
\item We finally compute the Wilcoxon test~\cite{10.2307/3001968} and display the critical diagram over the average standard deviation for every time series (as illustrated in Figure~\ref{fig:sensitivity}(a.1) for lag, Figure~\ref{fig:sensitivity}(a.2) for noise and Figure~\ref{fig:sensitivity}(a.3) for normal/abnormal ratio).
\end{enumerate}

%height=8.5cm,

\begin{figure}[tb]
  \centering
  \includegraphics[width=\linewidth]{figures/sensitivity_per_data_long.pdf}
%  %\vspace*{-0.3cm}
  \caption{Robustness Analysis for nine datasets: we report, over the entire benchmark, the average standard deviation of the accuracy values of the measures, under varying (a.1) lag, (a.2) noise, and (a.3) normal/abnormal ratio. }
  \label{fig:sensitivity_per_data}
\end{figure}

\begin{figure*}[tb]
  \centering
  \includegraphics[width=\linewidth]{figures/sensitivity_analysis.pdf}
  %\vspace*{-0.7cm}
  \caption{Critical difference diagram computed using the signed-rank Wilkoxon test (with $\alpha=0.1$) for the robustness to (a.1) lag, (a.2) noise and (a.3) normal/abnormal ratio.}
  \label{fig:sensitivity}
\end{figure*}

The methods with the smallest standard deviation can be considered more robust to lag, noise, or normal/abnormal ratio from the above framework. 
First, as stated in the introduction, we observe that non-threshold-based measures (such as AUC-ROC and AUC-PR) are indeed robust to noise (see Figure~\ref{fig:sensitivity_per_data}(a.2)), but not to lag. Figure~\ref{fig:sensitivity}(a.1) demonstrates that our proposed measures VUS-ROC, VUS-PR, R-AUC-ROC, and R-AUC-PR are significantly more robust to lag. Similarly, Figure~\ref{fig:sensitivity}(a.2) confirms that our proposed measures are significantly more robust to noise. However, we observe that, among our proposed measures, only VUS-ROC and R-AUC-ROC are robust to the normal/abnormal ratio and not VUS-PR and R-AUC-PR. This is explained by the fact that Precision-based measures vary significantly when this ratio changes. This is confirmed by Figure~\ref{fig:sensitivity_per_data}(a.3), in which we observe that Precision and Rprecision have a high standard deviation. Overall, we observe that VUS-ROC is significantly more robust to lag, noise, and normal/abnormal ratio than other measures.




\subsection{Separability Analysis}
\label{exp:separability}

%\newline \textbf{Separability Analysis: } 
We now evaluate the separability capacities of the different evaluation metrics. 
\commentRed{The main objective is to measure the ability of accuracy measures to separate accurate methods from inaccurate ones. More precisely, an appropriate measure should return accuracy scores that are significantly higher for accurate anomaly scores than for inaccurate ones.}
We thus manually select accurate and inaccurate anomaly detection methods and verify if the accuracy evaluation scores are indeed higher for the accurate than for the inaccurate methods. Figure~\ref{fig:separability} depicts the latter separability analysis applied to the MBA(805) and the SED series. 
The accurate and inaccurate anomaly scores are plotted in green and red, respectively. 
We then consider 12 different pairs of accurate/inaccurate methods among the eight previously mentioned anomaly scores. 
We slightly modify each score 50 different times in which we inject lag and noises and compute the accuracy measures. 
Figure~\ref{fig:separability}(a.4) and Figure~\ref{fig:separability}(b.4) are divided into four different subplots corresponding to 4 pairs (selected among the twelve different pairs due to lack of space). 
Each subplot corresponds to two box plots per accuracy measure. 
The green and red box plots correspond to the 50 accuracy measures on the accurate and inaccurate methods. 
If the red and green box plots are well separated, we can conclude that the corresponding accuracy measures are separating the accurate and inaccurate methods well. 
We observe that some accuracy measures (such as VUS-ROC) are more separable than others (such as RF). We thus measure the separability of the two box-plots by computing the Z-test. 

\begin{figure*}[tb]
  \centering
  \includegraphics[width=1\linewidth]{figures/pairwise_comp_example_long.pdf}
  %\vspace*{-0.5cm}
  \caption{Separability analysis applied on 4 pairs of accurate (green) and inaccurate (red) methods on (a) the MBA(805) data series, and (b) the SED data series.}
  %\vspace*{-0.3cm}
  \label{fig:separability}
\end{figure*}

We now aggregate all the results and compute the average Z-test for all pairs of accurate/inaccurate datasets (examples are shown in Figures~\ref{fig:separability}(a.2) and (b.2) for accurate anomaly scores, and in Figures~\ref{fig:separability}(a.3) and (b.3) for inaccurate anomaly scores, for the MBA(805) and SED series, respectively). 
Next, we perform the same operation over three different data series: MBA (805), MBA(820), and SED. 
Then, we depict the average Z-test for these three datasets in Figure~\ref{fig:separability_agg}(a). 
Finally, we show the average Z-test for all datasets in Figure~\ref{fig:separability_agg}(b). 


We observe that our proposed VUS-based and Range-based measures are significantly more separable than other current accuracy measures (up to two times for AUC-ROC, the best measures of all current ones). Furthermore, when analyzed in detail in Figure~\ref{fig:separability} and Figure~\ref{fig:separability_agg}, we confirm that VUS-based and Range-based are more separable over all three datasets. 

\begin{figure}[tb]
  \centering
  \includegraphics[width=\linewidth]{figures/agregated_sep_analysis.pdf}
  %\vspace*{-0.5cm}
  \caption{Overall separability analysis (averaged z-test between the accuracy values distributions of accurate and inaccurate methods) applied on 36 pairs on 3 datasets.}
  \label{fig:separability_agg}
\end{figure}


\noindent \textbf{Global Analysis: } Overall, we observe that VUS-ROC is the most robust (cf. Figure~\ref{fig:sensitivity}) and separable (cf. Figure~\ref{fig:separability_agg}) measure. 
On the contrary, Precision and Rprecision are non-robust and non-separable. 
Among all previous accuracy measures, only AUC-ROC is robust and separable. 
Popular measures, such as, F, RF, AUC-ROC, and AUC-PR are robust but non-separable.

In order to visualize the global statistical analysis, we merge the robustness and the separability analysis into a single plot. Figure~\ref{fig:global} depicts one scatter point per accuracy measure. 
The x-axis represents the averaged standard deviation of lag and noise (averaged values from Figure~\ref{fig:sensitivity_per_data}(a.1) and (a.2)). The y-axis corresponds to the averaged Z-test (averaged value from Figure~\ref{fig:separability_agg}). 
Finally, the size of the points corresponds to the sensitivity to the normal/abnormal ratio (values from Figure~\ref{fig:sensitivity_per_data}(a.3)). 
Figure~\ref{fig:global} demonstrates that our proposed measures (located at the top left section of the plot) are both the most robust and the most separable. 
Among all previous accuracy measures, only AUC-ROC is on the top left section of the plot. 
Popular measures, such as, F, RF, AUC-ROC, AUC-PR are on the bottom left section of the plot. 
The latter underlines the fact that these measures are robust but non-separable.
Overall, Figure~\ref{fig:global} confirms the effectiveness and superiority of our proposed measures, especially of VUS-ROC and VUS-PR.


\begin{figure}[tb]
  \centering
  \includegraphics[width=\linewidth]{figures/final_result.pdf}
  \caption{Evaluation of all measures based on: (y-axis) their separability (avg. z-test), (x-axis) avg. standard deviation of the accuracy values when varying lag and noise, (circle size) avg. standard deviation of the accuracy values when varying the normal/abnormal ratio.}
  \label{fig:global}
\end{figure}




\subsection{Consistency Analysis}
\label{sec:entropy}

In this section, we analyze the accuracy of the anomaly detection methods provided by the 13 accuracy measures. The objective is to observe the changes in the global ranking of anomaly detection methods. For that purpose, we formulate the following assumptions. First, we assume that the data series in each benchmark dataset are similar (i.e., from the same domain and sharing some common characteristics). As a matter of fact, we can assume that an anomaly detection method should perform similarly on these data series of a given dataset. This is confirmed when observing that the best anomaly detection methods are not the same based on which dataset was analyzed. Thus the ranking of the anomaly detection methods should be different for different datasets, but similar for every data series in each dataset. 
Therefore, for a given method $A$ and a given dataset $D$ containing data series of the same type and domain, we assume that a good accuracy measure results in a consistent rank for the method $A$ across the dataset $D$. 
The consistency of a method's ranks over a dataset can be measured by computing the entropy of these ranks. 
For instance, a measure that returns a random score (and thus, a random rank for a method $A$) will result in a high entropy. 
On the contrary, a measure that always returns (approximately) the same ranks for a given method $A$ will result in a low entropy. 
Thus, for a given method $A$ and a given dataset $D$ containing data series of the same type and domain, we assume that a good accuracy measure results in a low entropy for the different ranks for method $A$ on dataset $D$.

\begin{figure*}[tb]
  \centering
  \includegraphics[width=\linewidth]{figures/entropy_long.pdf}
  %\vspace*{-0.5cm}
  \caption{Accuracy evaluation of the anomaly detection methods. (a) Overall average entropy per category of measures. Analysis of the (b) averaged rank and (c) averaged rank entropy for each method and each accuracy measure over the entire benchmark. Example of (b.1) average rank and (c.1) entropy on the YAHOO dataset, KDD21 dataset (b.2, c.2). }
  \label{fig:entropy}
\end{figure*}

We now compute the accuracy measures for the nine different methods (we compute the anomaly scores ten different times, and we use the average accuracy). 
Figures~\ref{fig:entropy}(b.1) and (b.2) report the average ranking of the anomaly detection methods obtained on the YAHOO and KDD21 datasets, respectively. 
The x-axis corresponds to the different accuracy measures. We first observe that the rankings are more separated using Range-AUC and VUS measures for these two datasets. Figure~\ref{fig:entropy}(b) depicts the average ranking over the entire benchmark. The latter confirms the previous observation that VUS measures provide more separated rankings than threshold-based and AUC-based measures. We also observe an interesting ranking evolution for the YAHOO dataset illustrated in Figure~\ref{fig:entropy}(b.1). We notice that both LOF and MatrixProfile (brown and pink curve) have a low rank (between 4 and 5) using threshold and AUC-based measures. However, we observe that their ranks increase significantly for range-based and VUS-based measures (between 2.5 and 3). As we noticed by looking at specific examples (see Figure~\ref{exp:qual}), LOF and MatrixProfile can suffer from a lag issue even though the anomalies are well-identified. Therefore, the range-based and VUS-based measures better evaluate these two methods' detection capability.


Overall, the ranking curves show that the ranks appear more chaotic for threshold-based than AUC-, Range-AUC-, and VUS-based measures. 
In order to quantify this observation, we compute the Shannon Entropy of the ranks of each anomaly detection method. 
In practice, we extract the ranks of methods across one dataset and compute Shannon's Entropy of the different ranks. 
Figures~\ref{fig:entropy}(c.1) and (c.2) depict the entropy of each of the nine methods for the YAHOO and KDD21 datasets, respectively. 
Figure~\ref{fig:entropy}(c) illustrates the averaged entropy for all datasets in the benchmark for each measure and method, while Figure~\ref{fig:entropy}(a) shows the averaged entropy for each category of measures.
We observe that both for the general case (Figure~\ref{fig:entropy}(a) and Figure~\ref{fig:entropy}(c)) and some specific cases (Figures~\ref{fig:entropy}(c.1) and (c.2)), the entropy is reducing when using AUC-, Range-AUC-, and VUS-based measures. 
We report the lowest entropy for VUS-based measures. 
Moreover, we notice a significant drop between threshold-based and AUC-based. 
This confirms that the ranks provided by AUC- and VUS-based measures are consistent for data series belonging to one specific dataset. 


Therefore, based on the assumption formulated at the beginning of the section, we can thus conclude that AUC, range-AUC, and VUS-based measures are providing more consistent rankings. Finally, as illustrated in Figure~\ref{fig:entropy}, we also observe that VUS-based measures result in the most ordered and similar rankings for data series from the same type and domain.










\subsection{Execution Time Analysis}
\label{sec:exectime}

In this section, we evaluate the execution time required to compute different evaluation measures. 
In Section~\ref{sec:synthetic_eval_time}, we first measure the influence of different time series characteristics and VUS parameters on the execution time. In Section~\ref{sec:TSB_eval_time}, we  measure the execution time of VUS (VUS-ROC and VUS-PR simultaneously), R-AUC (R-AUC-ROC and R-AUC-PR simultaneously), and AUC-based measures (AUC-ROC and AUC-PR simultaneously) on the TSB-UAD benchmark. \commentRed{As demonstrated in the previous section, threshold-based measures are not robust, have a low separability power, and are inconsistent. 
Such measures are not suitable for evaluating anomaly detection methods. Thus, in this section, we do not consider threshold-based measures.}


\subsubsection{Evaluation on Synthetic Time Series}\hfill\\
\label{sec:synthetic_eval_time}

We first analyze the impact that time series characteristics and parameters have on the computation time of VUS-based measures. 
to that effect, we generate synthetic time series and labels, where we vary the following parameters: (i) the number of anomalies {\bf$\alpha$} in the time series, (ii) the average \textbf{$\mu(\ell_a)$} and standard deviation $\sigma(\ell_a)$ of the anomalies lengths in the time series (all the anomalies can have different lengths), (iii) the length of the time series \textbf{$|T|$}, (iv) the maximum buffer length \textbf{$L$}, and (v) the number of thresholds \textbf{$N$}.


We also measure the influence on the execution time of the R-AUC- and AUC- related parameter, that is, the number of thresholds ($N$).
The default values and the range of variation of these parameters are listed in Table~\ref{tab:parameter_range_time}. 
For VUS-based measures, we evaluate the execution time of the initial VUS implementation, as well as the two optimized versions, VUS$_{opt}$ and VUS$_{opt}^{mem}$.

\begin{table}[tb]
    \centering
    \caption{Value ranges for the parameters: number of anomalies ($\alpha$), average and standard deviation anomaly length ($\mu(\ell_a)$,$\sigma(\ell_a)$), time series length ($|T|$), maximum buffer length ($L$), and number of thresholds ($N$).}
    \begin{tabular}{|c|c|c|c|c|c|c|} 
 \hline
 Param. & $\alpha$ & $\mu(\ell_a)$ & $\sigma(\ell_{a})$ & $|T|$ & $L$ & $N$ \\ [0.5ex] 
 \hline\hline
 \textbf{Default} & 10 & 10 & 0 & $10^5$ & 5 & 250\\ 
 \hline
 Min. & 0 & 0 & 0 & $10^3$ & 0 & 2 \\
 \hline
 Max. & $2*10^3$ & $10^3$ & $10$ & $10^5$ & $10^3$ & $10^3$ \\ [1ex] 
 \hline
\end{tabular}
    \label{tab:parameter_range_time}
\end{table}


Figure~\ref{fig:sythetic_exp_time} depicts the execution time (averaged over ten runs) for each parameter listed in Table~\ref{tab:parameter_range_time}. 
Overall, we observe that the execution time of AUC-based and R-AUC-based measures is significantly smaller than VUS-based measures.
In the following paragraph, we analyze the influence of each parameter and compare the experimental execution time evaluation to the theoretical complexity reported in Table~\ref{tab:complexity_summary}.

\vspace{0.2cm}
\noindent {\bf [Influence of $\alpha$]}:
In Figure~\ref{fig:sythetic_exp_time}(a), we observe that the VUS, VUS$_{opt}$, and VUS$_{opt}^{mem}$ execution times are linearly increasing with $\alpha$. 
The increase in execution time for VUS, VUS$_{opt}$, and VUS$_{opt}^{mem}$ is more pronounced when we vary $\alpha$, in contrast to $l_a$ (which nevertheless, has a similar effect on the overall complexity). 
We also observe that the VUS$_{opt}^{mem}$ execution time grows slower than $VUS_{opt}$ when $\alpha$ increases. 
This is explained by the use of 2-dimensional arrays for the storage of predictions, which use contiguous memory locations that allow for faster access, decreasing the dependency on $\alpha$.

\vspace{0.2cm}
\noindent {\bf [Influence of $\mu(\ell_a)$]}:
As shown in Figure~\ref{fig:sythetic_exp_time}(b), the execution time variation of VUS, VUS$_{opt}$, and VUS$_{opt}^{mem}$ caused by $\ell_a$ is rather insignificant. 
We also observe that the VUS$_{opt}$ and VUS$_{opt}^{mem}$ execution times are significantly lower when compared to VUS. 
This is explained by the smaller dependency of the complexity of these algorithms on the time series length $|T|$. 
Overall, the execution time for both VUS$_{opt}$ and VUS$_{opt}^{mem}$ is significantly lower than VUS, and follows a similar trend. 

\vspace{0.2cm}
\noindent {\bf [Influence of $\sigma(\ell_a)$]}: 
As depicted in Figure~\ref{fig:sythetic_exp_time}(d) and inferred from the theoretical complexities in Table~\ref{tab:complexity_summary}, none of the measures are affected by the standard deviation of the anomaly lengths.

\vspace{0.2cm}
\noindent {\bf [Influence of $|T|$]}:
For short time series (small values of $|T|$), we note that O($T_1$) becomes comparable to O($T_2$). 
Thus, the theoretical complexities approximate to $O(NL(T_1+T_2))$, $O(N*(T_1+T_2))+O(NLT_2)$ and $O(N(T_1+T_2))$ for VUS, VUS$_{opt}$, and VUS$_{opt}^{mem}$, respectively. 
Indeed, we observe in Figure~\ref{fig:sythetic_exp_time}(c) that the execution times of VUS, VUS$_{opt}$, and VUS$_{opt}^{mem}$ are similar for small values of $|T|$. However, for larger values of $|T|$, $O(T_1)$ is much higher compared to $O(T_2)$, thus resulting in an effective complexity of $O(NLT_1)$ for VUS, and $O(NT_1)$ for VUS$_{opt}$, and VUS$_{opt}^{mem}$. 
This translates to a significant improvement in execution time complexity for VUS$_{opt}$ and VUS$_{opt}^{mem}$ compared to VUS, which is confirmed by the results in Figure~\ref{fig:sythetic_exp_time}(c).

\vspace{0.2cm}
\noindent {\bf [Influence of $N$]}: 
Given the theoretical complexity depicted in Table~\ref{tab:complexity_summary}, it is evident that the number of thresholds affects all measures in a linear fashion.
Figure~\ref{fig:sythetic_exp_time}(e) demonstrates this point: the results of varying $N$ show a linear dependency for VUS, VUS$_{opt}$, and VUS$_{opt}^{mem}$ (i.e., a logarithmic trend with a log scale on the y axis). \commentRed{Moreover, we observe that the AUC and range-AUC execution time is almost constant regardless of the number of thresholds used. The latter is explained by the very efficient implementation of AUC measures. Therefore, the linear dependency on the number of thresholds is not visible in Figure~\ref{fig:sythetic_exp_time}(e).}

\vspace{0.2cm}
\noindent {\bf [Influence of $L$]}: Figure~\ref{fig:sythetic_exp_time}(f) depicts the influence of the maximum buffer length $L$ on the execution time of all measures. 
We observe that, as $L$ grows, the execution time of VUS$_{opt}$ and VUS$_{opt}^{mem}$ increases slower than VUS. 
We also observe that VUS$_{opt}^{mem}$ is more scalable with $L$ when compared to VUS$_{opt}$. 
This is consistent with the theoretical complexity (cf. Table~\ref{tab:complexity_summary}), which indicates that the dependence on $L$ decreases from $O(NL(T_1+T_2+\ell_a \alpha))$ for VUS to $O(NL(T_2+\ell_a \alpha)$ and $O(NL(\ell_a \alpha))$ for $VUS_{opt}$, and $VUS_{opt}^{mem}$.





\begin{figure*}[tb]
  \centering
  \includegraphics[width=\linewidth]{figures/synthetic_res.pdf}
  %\vspace*{-0.5cm}
  \caption{Execution time of VUS, R-AUC, AUC-based measures when we vary the parameters listed in Table~\ref{tab:parameter_range_time}. The solid lines correspond to the average execution time over 10 runs. The colored envelopes are to the standard deviation.}
  \label{fig:sythetic_exp_time}
\end{figure*}


\vspace{0.2cm}
In order to obtain a more accurate picture of the influence of each of the above parameters, we fit the execution time (as affected by the parameter values) using linear regression; we can then use the regression slope coefficient of each parameter to evaluate the influence of that parameter. 
In practice, we fit each parameter individually, and report the regression slope coefficient, as well as the coefficient of determination $R^2$.
Table~\ref{tab:parameter_linear_coeff} reports the coefficients mentioned above for each parameter associated with VUS, VUS$_{opt}$, and VUS$_{opt}^{mem}$.



\begin{table}[tb]
    \centering
    \caption{Linear regression slope coefficients ($C.$) for VUS execution times, for each parameter independently. }
    \begin{tabular}{|c|c|c|c|c|c|c|} 
 \hline
 Measure & Param. & $\alpha$ & $l_a$ & $|T|$ & $L$ & $N$\\ [0.5ex] 
 \hline\hline
 \multirow{2}{*}{$VUS$} & $C.$ & 21.9 & 0.02 & 2.13 & 212 & 6.24\\\cline{2-7}
 & {$R^2$} & 0.99 & 0.15 & 0.99 & 0.99 & 0.99 \\   
 \hline
  \multirow{2}{*}{$VUS_{opt}$} & $C.$ & 24.2  & 0.06 & 0.19 & 27.8 & 1.23\\\cline{2-7}
  & $R^2$& 0.99 & 0.86 & 0.99 & 0.99 & 0.99\\ 
 \hline
 \multirow{2}{*}{$VUS_{opt}^{mem}$} & $C.$ & 21.5 & 0.05 & 0.21 & 15.7 & 1.16\\\cline{2-7}
  & $R^2$ & 0.99 & 0.89 & 0.99 & 0.99 & 0.99\\[1ex] 
 \hline
\end{tabular}
    \label{tab:parameter_linear_coeff}
\end{table}

Table~\ref{tab:parameter_linear_coeff} shows that the linear regression between $\alpha$ and the execution time has a $R^2=0.99$. Thus, the dependence of execution time on $\alpha$ is linear. We also observe that VUS$_{opt}$ execution time is more dependent on $\alpha$ than VUS and VUS$_{opt}^{mem}$ execution time.
Moreover, the dependence of the execution time on the time series length ($|T|$) is higher for VUS than for VUS$_{opt}$ and VUS$_{opt}^{mem}$. 
More importantly, VUS$_{opt}$ and VUS$_{opt}^{mem}$ are significantly less dependent than VUS on the number of thresholds and the maximal buffer length. 







\subsubsection{Evaluation on TSB-UAD Time Series}\hfill\\
\label{sec:TSB_eval_time}

In this section, we verify the conclusions outlined in the previous section with real-world time series from the TSB-UAD benchmark. 
In this setting, the parameters $\alpha$, $\ell_a$, and $|T|$ are calculated from the series in the benchmark and cannot be changed. Moreover, $L$ and $N$ are parameters for the computation of VUS, regardless of the time series (synthetic or real). Thus, we do not consider these two parameters in this section.

\begin{figure*}[tb]
  \centering
  \includegraphics[width=\linewidth]{figures/TSB2.pdf}
  \caption{Execution time of VUS, R-AUC, AUC-based measures on the TSB-UAD benchmark, versus $\alpha$, $\ell_a$, and $|T|$.}
  \label{fig:TSB}
\end{figure*}

Figure~\ref{fig:TSB} depicts the execution time of AUC, R-AUC, and VUS-based measures versus $\alpha$, $\mu(\ell_a)$, and $|T|$.
We first confirm with Figure~\ref{fig:TSB}(a) the linear relationship between $\alpha$ and the execution time for VUS, VUS$_{opt}$ and VUS$_{opt}^{mem}$.
On further inspection, it is possible to see two separate lines for almost all the measures. 
These lines can be attributed to the time series length $|T|$. 
The convergence of VUS and $VUS_{opt}$ when $\alpha$ grows shows the stronger dependence that $VUS_{opt}$ execution time has on $\alpha$, as already observed with the synthetic data (cf. Section~\ref{sec:synthetic_eval_time}). 

In Figure~\ref{fig:TSB}(b), we observe that the variation of the execution time with $\ell_a$ is limited when compared to the two other parameters. We conclude that the variation of $\ell_a$ is not a key factor in determining the execution time of the measures.
Furthermore, as depicted in Figure~\ref{fig:TSB}(c), $VUS_{opt}$ and $VUS_{opt}^{mem}$ are more scalable than VUS when $|T|$ increases. 
We also confirm the linear dependence of execution time on the time series length for all the accuracy measures, which is consistent with the experiments on the synthetic data. 
The two abrupt jumps visible in Figure~\ref{fig:TSB}(c) are explained by significant increases of $\alpha$ in time series of the same length. 

\begin{table}[tb]
\centering
\caption{Linear regression slope coefficients ($C.$) for VUS execution time, for all time series parameters all-together.}
\begin{tabular}{|c|ccc|c|} 
 \hline
Measure & $\alpha$ & $|T|$ & $l_a$ & $R^2$ \\ [0.5ex] 
 \hline\hline
 \multirow{1}{*}{${VUS}$} & 7.87 & 13.5 & -0.08 & 0.99  \\ 
 %\cline{2-5} & $R^2$ & \multicolumn{3}{c|}{ 0.99}\\
 \hline
 \multirow{1}{*}{$VUS_{opt}$} & 10.2 & 1.70 & 0.09 & 0.96 \\
 %\cline{2-5} & $R^2$ & \multicolumn{3}{c|}{0.96}\\
\hline
 \multirow{1}{*}{$VUS_{opt}^{mem}$} & 9.27 & 1.60 & 0.11 & 0.96 \\
 %\cline{2-5} & $R^2$ & \multicolumn{3}{c|}{0.96} \\
 \hline
\end{tabular}
\label{tab:parameter_linear_coeff_TSB}
\end{table}



We now perform a linear regression between the execution time of VUS, VUS$_{opt}$ and VUS$_{opt}^{mem}$, and $\alpha$, $\ell_a$ and $|T|$.
We report in Table~\ref{tab:parameter_linear_coeff_TSB} the slope coefficient for each parameter, as well as the $R^2$.  
The latter shows that the VUS$_{opt}$ and VUS$_{opt}^{mem}$ execution times are impacted by $\alpha$ at a larger degree than $\alpha$ affects VUS. 
On the other hand, the VUS$_{opt}$ and VUS$_{opt}^{mem}$ execution times are impacted to a significantly smaller degree by the time series length when compared to VUS. 
We also confirm that the anomaly length does not impact the execution time of VUS, VUS$_{opt}$, or VUS$_{opt}^{mem}$.
Finally, our experiments show that our optimized implementations VUS$_{opt}$ and VUS$_{opt}^{mem}$ significantly speedup the execution of the VUS measures (i.e., they can be computed within the same order of magnitude as R-AUC), rendering them practical in the real world.











\subsection{Summary of Results}


Figure~\ref{fig:overalltable} depicts the ranking of the accuracy measures for the different tests performed in this paper. The robustness test is divided into three sub-categories (i.e., lag, noise, and Normal vs. abnormal ratio). We also show the overall average ranking of all accuracy measures (most right column of Figure~\ref{fig:overalltable}).
Overall, we see that VUS-ROC is always the best, and VUS-PR and Range-AUC-based measures are, on average, second, third, and fourth. We thus conclude that VUS-ROC is the overall winner of our experimental analysis.

\commentRed{In addition, our experimental evaluation shows that the optimized version of VUS accelerates the computation by a factor of two. Nevertheless, VUS execution time is still significantly slower than AUC-based approaches. However, it is important to mention that the efficiency of accuracy measures is an orthogonal problem with anomaly detection. In real-time applications, we do not have ground truth labels, and we do not use any of those measures to evaluate accuracy. Measuring accuracy is an offline step to help the community assess methods and improve wrong practices. Thus, execution time should not be the main criterion for selecting an evaluation measure.}


%\vspace{-0.8em}


\section{Lessons learnt}
Over the time of development of {\systemname} we learnt many lessons. Here we present couple of interesting examples.
\subsection{Diversity of topics in Recommendations}
%Remove diversity re-rankers - tell the story how we removed diversity rerankers \textcolor{red}{\textbf{Sudarshan \& Borja}}

Over the years, the LinkedIn feed ranking model has been a pure pointwise model and list level interactions are not taken into account by the model. This could however yield a subpar experience of the entire feed session as a whole for our members. Some examples include seeing too many updates from the same actors, back to back out of network content, back to back instances of posts that are surfaced because one of their connections liked an activity of their connection. 

Diversifying this experience through a rule based approach has helped provide a much better session experience as a whole for our members which could be seen from the metric impact. 
We did a simple ablation study of removing all the diversity rules and checked the member impact. As expected, we do see a drop in the DAU in LinkedIn (-0.18\%).

While the rule based diversity for organizing the session does help, we believe it is suboptimal and a model powered solution could yield a better experience for our members. The rules tend to assume that the same template would work for everyone. In our approach we believe that replacing this legacy solution with a model that could learn the required list level diversity attribute is a superior solution.  In fact a model based approach could help us quantify and keep diversity as an objective in the longer term. In this work, we show how setwise models could replace diversity rules with a superior member experience in LinkedIn.


\subsection{Our approach to develop {\systemname}}
%\textcolor{red}{\textbf{Lars - tell the story how we built GR}} to FILL IN

{\systemname} and the traditional DLRM approach are fundamentally different in data format, features, and model training. {\systemname} therefore required a full rebuilding of our training pipeline. The goals for this rebuilding were to use few features, to outperform the baseline, and to demonstrate scaling laws.

We started this work by verifying that our most important count features can be emulated by ID embedding features alone. Then we built a small {\systemname} model and added the top features from the DLRM model or corresponding ID feature counterparts until adding more features gave only small AUC improvements. While the resulting model had ten times less features than the baseline, we did find that with too few features the model can suffer from item cold start and is not able to beat the baseline. In order to reduce member cold start in less frequent members we also increased the user history time window, but retained item ID features only for a limited window.

\begin{figure}
    \centering
    \includegraphics[width=1\linewidth]{figures/gr-progress-chart_new.png}
    \caption{GR AUC improvement over successive iterations.}
    \label{fig:gr-progress}
    \vspace{-1.0em}
\end{figure}

Having added enough features, we scaled up the model along the dimensions of sequence length, embedding dimension, and number of layers. During this phase the main challenges were to maintain training stability and manage GPU memory. To ensure stable training we used different learning rates for dense vs. sparse model parameters, dense gating of transformer layers, and  transformers with gating as shown in Figure \ref{fig:custom-transformer}.

Overall we found that aside from ensuring training stability, most architecture changes we experimented with had little impact on the prediction AUC. The majority of performance appeared to be driven by features and model scale.



\section{Conclusion}\label{sec:conclusion}
\section{Conclusion}
In this work, we propose a simple yet effective approach, called SMILE, for graph few-shot learning with fewer tasks. Specifically, we introduce a novel dual-level mixup strategy, including within-task and across-task mixup, for enriching the diversity of nodes within each task and the diversity of tasks. Also, we incorporate the degree-based prior information to learn expressive node embeddings. Theoretically, we prove that SMILE effectively enhances the model's generalization performance. Empirically, we conduct extensive experiments on multiple benchmarks and the results suggest that SMILE significantly outperforms other baselines, including both in-domain and cross-domain few-shot settings.

%%
%% The next two lines define the bibliography style to be used, and
%% the bibliography file.

\bibliographystyle{ACM-Reference-Format}
\balance
\bibliography{bibliography}

%%
%% If your work has an appendix, this is the place to put it.
%\vspace{-2.7em}



\appendix
%\section{INFORMATION FOR REPRODUCIBILITY}\label{sec:reproducability}

\section{APPENDIX}

\subsection{Reproducibility notes}
Training stability emerged as a critical factor in optimizing our model architecture, and we implemented several techniques to ensure it. These included using a transformer architecture with gating, where layer normalization is applied before multi-head attention (MHA) or MLP layers, and gating applied on top of MHA/MLP outputs with a sigmoid(XW) layer. Additionally, we employed separate learning rates for embeddings (0.01) and dense layers (0.001) to further enhance stability. Architectural changes, while having limited direct impact on performance, were crucial for maintaining training stability, enabling effective scaling of model and input size — both of which proved more impactful for performance improvements.

Our exploration of features revealed that models with even a single feature could achieve performance near baseline production model levels. However, surpassing baseline performance required the inclusion of multiple features. Notably, we tracked performance on cold-start items—those with few or no interactions in the training data—and found that adding more features significantly boosted performance on these items, demonstrating the value of feature richness for addressing sparse data challenges.

Investigations into position embeddings showed no significant AUC difference between relative attention bias and learned position embeddings for history sequence, suggesting that this choice has minimal impact. Similarly, replacing MLPs with DCNv2 or incorporating DCNv2 at the input level did not result in performance gains, despite its theoretical potential to model interactions across embedding dimensions. Even removing MLPs from transformer blocks caused only minor performance drops, presenting an opportunity to trade off MLPs for additional MHA layers within the same memory budget.

Our experiments with alternative attention activation functions, such as SiLU and Sigmoid, also yielded subpar results compared to Softmax. While these activations were hypothesized to better model extreme affinity cases, Softmax's normalization proved more effective at emphasizing the relative importance of sequence elements. This aligns with our findings on HSTU, which did not improve performance and showed lower AUC. Additionally, HSTU's reliance on SiLU attention made it incompatible with FlashAttention, a critical component in larger-scale experiments.

Regarding vocabulary size, we found no AUC difference between item ID embedding vocabularies of 33 million and 66 million, despite the training data encompassing approximately 150 million unique object IDs. This suggests that a smaller vocabulary suffices without compromising performance.

%Lastly, the team's focus on metrics and rapid iteration proved instrumental in achieving and surpassing MME AUC within a single quarter. By prioritizing fast experimentation, weekly performance improvements, and deferring non-critical tasks, two engineers were able to deliver unexpected results in a remarkably short timeframe. This approach underscores the value of agility and metric-driven development in advancing model performance.

\subsection{Analysis of alternative solutions}\label{wukong_layer}
We also explored alternative methods for scaling the model. Wukong \cite{zhang2024wukongscalinglawlargescale} has demonstrated effective dense parameter scaling by allowing the model to capture higher-order feature interactions efficiently. However, directly applying Wukong to the Feed models did not result in any observable improvement in AUCs, even when we scale the number of layers to 8. Our hypothesis suggests that the numerical features may have experienced information degradation during deep feature interactions with embedding features, potentially due to feature heterogeneity. The diverse nature of features, including numerical and categorical data, likely contributed to challenges in effectively capturing complex interactions. 

We attempted to change the design for Wukong, where we employ a dual-pathway approach to feature processing to a baseline architecture (see \S\ref{sec:Overview:FeedRanking}). Embedding features are passed through a deep Wukong layer, facilitating deep vector-wise interactions. Concurrently, numerical and categorical features are fed into a two-layer DCNv2 to enable element-wise interactions. This bifurcated structure allows for specialized handling of different feature types, potentially mitigating the challenges posed by feature heterogeneity. With this updated architecture, we observed a 0.36\% increase in Contributions AUC on top of baseline model (see \S\ref{sec:Overview:FeedRanking}), when scaling the number of layers to 8, which brings additional 2 million parameters.  With a two-tower structure, we extend the scaling to accommodate multiple objectives. For example, our model can simultaneously optimize for various user interactions, such as clicks, comments, shares, and more. Additionally, we explore different mechanisms for concatenating embeddings to achieve optimal performance. It's important to note that the Wukong layer is not restricted to stacked Factorization Machines; it can also incorporate stacked CrossNet for enhanced flexibility. The original Wukong Layer includes a Linear Compress Block, which linearly recombines embeddings — an important element for performance. When incorporated into the stacked CrossNet variant of Wukong, this block is adapted into a wide layer. This provides an interesting perspective, as each layer in Wukong effectively embodies the wide and deep architecture. 

As we developed {\systemname}, our proposed architecture (Figure \ref{fig:custom-transformer}) achieved a 1.2\% increase in Contributions AUC and demonstrated better scalability compared to the 0.36\% improvement observed with the Wukong layer. We believe that Wukong could primarily be used to extend DLRM-style models \cite{DLRM19} or to be used in combination with DLRM and transformer-based models, such as \cite{zeng2024interformer}.

\end{document}
\endinput
%%
%% End of file `sample-authordraft.tex'.
