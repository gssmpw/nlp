\appendix
\beginsupplement
\clearpage
\onecolumn
\section{Further Methodological Details} \label{sec:appendix-methodol}
\FloatBarrier
\begin{figure*}[ht]
\captionsetup{justification=raggedright, singlelinecheck=false, skip=2pt}
\centering
\begin{subfigure}[t]{0.3\linewidth}
   
    \subcaption{}
    \includegraphics[width=\linewidth]{sections/images/biden.jpg}
\end{subfigure}
\hspace{1cm}
\begin{subfigure}[t]{0.31\linewidth}

   \subcaption{}
   \includegraphics[width=\linewidth]{sections/images/rock.jpg}
\end{subfigure}
\caption{\mybold{AI-Generated Images from New York Times Quiz} \normalfont{\textbf{A}. NYT's explanation for evidence pointing to this image as AI-generated is: ``Though the resemblance to President Biden is striking, he would not be wearing military fatigues as a civilian.''~\cite{nytimes2024deepfake} \textbf{B}. NYT's explanation for evidence pointing to this image as AI-generated is ``One giveaway in this image is the badge, which includes garbled text.''~\cite{nytimes2024deepfake}}}
\label{fig:nytimes}
\Description{AI-generated image of Joe Biden in a conference room and an AI-generated image of the Rock in military uniform in a mall.}
\end{figure*}
\begin{figure*}[ht]
\centering
\captionsetup{justification=raggedright, singlelinecheck=false, skip=2pt}
\begin{subfigure}[t]{0.65\textwidth}
    \caption{}
    \vtop{\vskip0pt\hbox{\includegraphics[width=\linewidth]{sections/images/refiningpipeline.jpg}}}
\end{subfigure}
\hspace{0.01\textwidth} 
\begin{minipage}[t]{0.23\textwidth}
    \begin{subfigure}[t]{\textwidth}
        \caption{}
        \vtop{\vskip0pt\hbox{\includegraphics[width=\linewidth]{sections/images/facerefine.jpg}}}
    \end{subfigure}
    \vspace{0.5cm}
    \begin{subfigure}[t]{\textwidth}
        \caption{}\hfill\vtop{\vskip0pt\hbox{\includegraphics[width=0.8\linewidth]{sections/images/handrefine.jpg}}}
    \end{subfigure}
\end{minipage}
 \caption{\textbf{Image generation process in Stable Diffusion} A. \normalfont{Four stage image generation pipeline where the image is first generated in SD1.5. The output image is then encoded as latent and upscaled to be re-generated in SDXL with ControlNets applied for pose consistency. This is passed to the face refiner \cite{comfyuiimpactpack} which detects dominant and background faces in the image via YOLOv8 \cite{yolov8} and re-generates them using an SDXL pipeline. Finally, the resulting image is passed to the hand refiner \cite{comfyuiimpactpack} which detects hands in the image via YOLOv8  and predicts the hand pose used to guide the re-generation of the hands. \textbf{B}. Faces in the image before and after the face refining process \textbf{C}. Hand refining process. The left image shows the initial generation of the hand. The center image shows a predicted skeleton for the hand that is used for a ControlNet that guides the re-generation of the hand shown in the image on the right.}}
\label{fig:refiningpipe}
\Description{4 stage image generation pipeline}
\end{figure*}
\FloatBarrier
\twocolumn
According to a New York Times (NYT) quiz, qualities that typically signify AI generation include missing fingers, misaligned eyes, repeated elements, and garbled or nonsensical details~\cite{nytimes2024deepfake}.  Examples are shown in \ref{fig:nytimes}. The NYT quiz also discusses qualities that may cause a real image to look AI-generated, such as repeated cropping and compression that often happens over social media.

A screenshot of the pipeline, along with images before and after refinement, is shown in Figure~\ref{fig:refiningpipe}.
\FloatBarrier

\FloatBarrier
Figure~\ref{fig:pose-comprehensive} displays more examples of the four pose complexities and their average accuracies.
\onecolumn
\begin{figure}[H]
\centering
\resizebox{0.85\textwidth}{!}{
% This ensures the figure fits the page
\begin{minipage}{\textwidth}
% Portraits
\begin{subfigure}{0.22\linewidth}
    \includegraphics[width=\linewidth]{sections/images/ff_portrait3_014.jpeg}
    \caption{Acc: 25\%}
\end{subfigure}
\hfill
\begin{subfigure}{0.22\linewidth}
    \includegraphics[width=\linewidth]{sections/images/sd_portrait3_003.jpg}
    \caption{Acc: 37\%}
\end{subfigure}
\hfill
\begin{subfigure}{0.22\linewidth}
    \includegraphics[width=\linewidth]{sections/images/ff_portrait1_002.jpeg}
    \caption{Acc: 66\%}
\end{subfigure}
\hfill
\begin{subfigure}{0.22\linewidth}
    \includegraphics[width=\linewidth]{sections/images/sd_portrait3_075.jpg}
    \caption{Acc: 80\%}
\end{subfigure}

\vspace{0.3cm}

% Full Body
\begin{subfigure}{0.22\linewidth}
    \includegraphics[width=\linewidth]{sections/images/mj_fullbody3_028.jpg}
    \caption{Acc: 37\%}
\end{subfigure}
\hfill
\begin{subfigure}{0.22\linewidth}
    \includegraphics[width=\linewidth]{sections/images/mj_fullbody3_012.jpg}
    \caption{Acc: 57\%}
\end{subfigure}
\hfill
\begin{subfigure}{0.22\linewidth}
    \includegraphics[width=\linewidth]{sections/images/mj_fullbody3_022.jpg}
    \caption{Acc: 66\%}
\end{subfigure}
\hfill
\begin{subfigure}{0.22\linewidth}
    \includegraphics[width=\linewidth]{sections/images/mj_fullbody3_029.jpg}
    \caption{Acc: 83\%}
\end{subfigure}

\vspace{0.3cm}

% Posed Groups
\begin{subfigure}{0.22\linewidth}
    \includegraphics[width=\linewidth]{sections/images/mj_pg3_017.jpg}
    \caption{Acc: 37\%}
\end{subfigure}
\hfill
\begin{subfigure}{0.22\linewidth}
    \includegraphics[width=\linewidth]{sections/images/mj_pg2_012.jpg}
    \caption{Acc: 57\%}
\end{subfigure}
\hfill
\begin{subfigure}{0.22\linewidth}
    \includegraphics[width=\linewidth]{sections/images/mj_pg3_003.jpg}
    \caption{Acc: 66\%}
\end{subfigure}
\hfill
\begin{subfigure}{0.22\linewidth}
    \includegraphics[width=\linewidth]{sections/images/sd_pg3_013.jpg}
    \caption{Acc: 83\%}
\end{subfigure}

\vspace{0.3cm}

% Candid Groups
\begin{subfigure}{0.22\linewidth}
    \includegraphics[width=\linewidth]{sections/images/mj_ng3_016.jpg}
    \caption{Acc: 31\%}
\end{subfigure}
\hfill
\begin{subfigure}{0.22\linewidth}
    \includegraphics[width=\linewidth]{sections/images/mj_ng2_007.jpg}
    \caption{Acc: 66\%}
\end{subfigure}
\hfill
\begin{subfigure}{0.22\linewidth}
    \includegraphics[width=\linewidth]{sections/images/mj_ng4_003.jpg}
    \caption{Acc: 75\%}
\end{subfigure}
\hfill
\begin{subfigure}{0.22\linewidth}
    \includegraphics[width=\linewidth]{sections/images/mj_ng3_005.jpg}
    \caption{Acc: 87\%}
\end{subfigure}

\end{minipage}
}
\vspace{-2mm}
\caption{\textbf{More examples of the four pose complexities and their average accuracies.} \normalfont{The first row shows Portraits, the second row Full Body images, the third row Posed Groups, and the last row Candid Groups.}}
\label{fig:pose-comprehensive}
\Description{Examples of AI-generated images in different pose complexities: Portraits, Full Body, Posed Groups, and Candid Groups, with participant accuracy percentages.}
\end{figure}
\FloatBarrier
\twocolumn
\clearpage
\subsection{Robustness Check: Dataset Comparison}

To ensure the validity of our conclusions, we conducted a robustness check comparing the results from our full dataset against a subset excluding data collected before May 10th, 2024. This comparison addresses potential biases introduced by the initial experimental design, which did not implement stratified randomization as mentioned in Section \ref{exp-design}.

Table~\ref{tab:accuracy-comparison-dataset} presents the accuracy metrics for both the full dataset and the dataset excluding pre-May 10th data. The table includes overall accuracy, as well as specific accuracy for AI-generated and real images, along with their respective 95\% confidence intervals.

\begin{table}[H]
\centering
\caption{Comparison of accuracy: Full Dataset vs. Dataset excluding data before May 10th}
\label{tab:accuracy-comparison-dataset}
\resizebox{\linewidth}{!}{
\begin{tabular}{lcccccc}
\hline
Dataset & \multicolumn{2}{c}{Overall} & \multicolumn{2}{c}{AI-generated} & \multicolumn{2}{c}{Real} \\
 & Accuracy & 95\% CI & Accuracy & 95\% CI & Accuracy & 95\% CI \\
\hline
Full Dataset & 0.75 & [0.74, 0.76] & 0.76 & [0.74, 0.77] & 0.73 & [0.71, 0.75] \\
Dataset excluding data before May 10th & 0.75 & [0.74, 0.76] & 0.76 & [0.75, 0.77] & 0.7201 & [0.70, 0.74] \\
\hline
\end{tabular}}
\Description{A robustness check by comparing accuracy in full dataset vs. dataset excluding data before May 10th}
\end{table}

Figure~\ref{fig:accuracy-comparison-dstaset} visualizes the distribution of image accuracies for both datasets. This comparison allows for direct observation of any potential shifts in accuracy distributions between the full dataset and the subset, excluding early data.
 This robustness check supports the validity of using the full dataset in our main analysis.

\begin{figure}[H]
\centering
\includegraphics[width=\linewidth]{sections/images/dataset_accuracy_distribution_comparison.jpg}
\vspace{-10mm}
\caption{Comparison of accuracies between the full dataset and the dataset excluding data before 10th data.}
\label{fig:accuracy-comparison-dstaset}
\Description{Figure compares the distribution of accuracy of images from the full dataset vs. from the dataset excluding data before May 10th}
\end{figure}

\clearpage
\onecolumn
\section{Curated and Uncurated AI-generated Images}

\begin{figure*}[!htb]
\centering
\resizebox{1.0\textwidth}{!}{ % Scale figure to 95% of text width
\begin{minipage}{\textwidth} % Ensures correct alignment
\captionsetup{justification=raggedright, singlelinecheck=false, skip=2pt}

% Row A

\begin{subfigure}[t]{0.23\linewidth}  
\subcaption{}
\vtop{\vskip0pt\hbox{\includegraphics[width=\linewidth]{sections/images/ff_portrait3_001.jpeg}}}
\end{subfigure}
\hfill
\begin{subfigure}[t]{0.23\linewidth}  
\subcaption{}
\vtop{\vskip0pt\hbox{\includegraphics[width=\linewidth]{sections/images/american_faculty1.jpg}}}
\end{subfigure}
\hfill
\begin{subfigure}[t]{0.23\linewidth}  
\subcaption{}
\vtop{\vskip0pt\hbox{\includegraphics[width=\linewidth]{sections/images/american_faculty2.jpg}}}
\end{subfigure}
\hfill
\begin{subfigure}[t]{0.23\linewidth}  
\subcaption{}
\vtop{\vskip0pt\hbox{\includegraphics[width=\linewidth]{sections/images/american_faculty3.jpg}}}
\end{subfigure}

\vspace{8pt} % Reduced spacing between rows

% Row B
\begin{subfigure}[t]{0.23\linewidth}  
\subcaption{}
\vtop{\vskip0pt\hbox{\includegraphics[width=\linewidth]{sections/images/ff_pg4_001.jpeg}}}
\end{subfigure}
\hfill
\begin{subfigure}[t]{0.23\linewidth}  
\subcaption{}
\vtop{\vskip0pt\hbox{\includegraphics[width=\linewidth]{sections/images/astronaut1.jpg}}}
\end{subfigure}
\hfill
\begin{subfigure}[t]{0.23\linewidth}  
\subcaption{}
\vtop{\vskip0pt\hbox{\includegraphics[width=\linewidth]{sections/images/astronaut2.jpg}}}
\end{subfigure}
\hfill
\begin{subfigure}[t]{0.23\linewidth}  
\subcaption{}
\vtop{\vskip0pt\hbox{\includegraphics[width=\linewidth]{sections/images/astronaut3.jpg}}}
\end{subfigure}

\end{minipage}
}
\vspace{-2mm}
\caption{\mybold{Example images generated by consistently photorealistic and consistently detectable prompts.} \normalfont{
\textbf{A.} Curated image generated with a consistently photorealistic prompt: ``American woman faculty portrait, not a close-up, blond." 
\textbf{B-D} Reprompted images generated with the same consistently photorealistic prompts. 
\textbf{E.} Curated image generated with a consistently detectable prompt: ``Persian woman astronaut in astronaut clothes, family photo with husband and two toddlers, high resolution, realistic." 
\textbf{F-H} Reprompted images of the same consistently detectable prompts.}}
\label{fig:goodandbadprompt}
\Description{Two example images where A shows a portrait image of an American woman faculty with few visible artifacts and B shows a Persian woman and her child and husband in a space suit with noticeable artifacts in all of their faces.}
\end{figure*}

\clearpage
\onecolumn
\section{Future Work on Videos}
\begin{figure*}[ht]
\centering
\resizebox{1.0\textwidth}{!}{ 
\begin{minipage}{\textwidth} 
\captionsetup{justification=raggedright, singlelinecheck=false, skip=2pt}

% Row A
\begin{subfigure}[t]{0.33\linewidth}  
\subcaption{}
\vtop{\vskip0pt\hbox{\includegraphics[width=\linewidth]{sections/images/out_7.jpg}}}
\end{subfigure}
\hfill
\begin{subfigure}[t]{0.33\linewidth}  
\subcaption{}
\vtop{\vskip0pt\hbox{\includegraphics[width=\linewidth]{sections/images/out_9.jpg}}}
\end{subfigure}
\hfill
\begin{subfigure}[t]{0.33\linewidth}  
\subcaption{}
\vtop{\vskip0pt\hbox{\includegraphics[width=\linewidth]{sections/images/out_11.jpg}}}
\end{subfigure}

\vspace{10pt} % Spacing between rows

\begin{subfigure}[t]{0.33\linewidth}  
\subcaption{}
\vtop{\vskip0pt\hbox{\includegraphics[width=\linewidth]{sections/images/out_13.jpg}}}
\end{subfigure}
\hfill
\begin{subfigure}[t]{0.33\linewidth}  
\subcaption{}
\vtop{\vskip0pt\hbox{\includegraphics[width=\linewidth]{sections/images/out_15.jpg}}}
\end{subfigure}
\hfill
\begin{subfigure}[t]{0.33\linewidth}  
\subcaption{}
\vtop{\vskip0pt\hbox{\includegraphics[width=\linewidth]{sections/images/out_17.jpg}}}
\end{subfigure}

\vspace{10pt} % Spacing between rows

\begin{subfigure}[t]{0.33\linewidth}  
\subcaption{}
\vtop{\vskip0pt\hbox{\includegraphics[width=\linewidth]{sections/images/out_19.jpg}}}
\end{subfigure}
\hfill
\begin{subfigure}[t]{0.33\linewidth}  
\subcaption{}
\vtop{\vskip0pt\hbox{\includegraphics[width=\linewidth]{sections/images/out_21.jpg}}}
\end{subfigure}
\hfill
\begin{subfigure}[t]{0.33\linewidth}  
\subcaption{}
\vtop{\vskip0pt\hbox{\includegraphics[width=\linewidth]{sections/images/out_23.jpg}}}
\end{subfigure}

\vspace{10pt} % Spacing between rows

% Row C

\begin{subfigure}[t]{0.33\linewidth}  
\subcaption{}
\vtop{\vskip0pt\hbox{\includegraphics[width=\linewidth]{sections/images/out_25.jpg}}}
\end{subfigure}
\hfill
\begin{subfigure}[t]{0.33\linewidth}  
\subcaption{}
\vtop{\vskip0pt\hbox{\includegraphics[width=\linewidth]{sections/images/out_27.jpg}}}
\end{subfigure}
\hfill
\begin{subfigure}[t]{0.33\linewidth}  
\subcaption{}
\vtop{\vskip0pt\hbox{\includegraphics[width=\linewidth]{sections/images/out_29.jpg}}}
\end{subfigure}

\end{minipage}
}
\vspace{-2mm}
\caption{\mybold{Example frames from an AI-generated video with a temporal anatomical implausibility.} \normalfont{
9 frames from a video generated by OpenAI's Sora diffusion-transformer model where the subject's right leg morphs into the left leg somewhere between E and J. Each frame is separated by 1/10 of a second. This particular artifact fits into the anatomical implausibility category of the taxonomy, but it's different from any anatomical plausibility seen in diffusion model-generated images. In particular, this implausibility has a temporal element: the transition from A to L involves the subject's right leg becoming her left in a split second, which does not fit with what we know about human anatomy.}}
\label{fig:sora}
\Description{9 frames from a video generated by OpenAI's Sora diffusion-transformer model where the subject's right morphs into the left leg somewhere between E and J. Each frame is separated by 1/10 of a second. This particular artifact fits into the anatomical implausibility category of the taxonomy, but it's different than any anatomical plausibility in diffusion model-generated images. In particular, this implausibility has a temporal element: the transition from A to L involves the subject's right leg becoming her left in a split second, which does not fit with what we know about human anatomy.}
\end{figure*}
