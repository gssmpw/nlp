%%
%% This is file `sample-manuscript.tex',
%% generated with the docstrip utility.
%%
%% The original source files were:
%%
%% samples.dtx  (with options: `manuscript')
%% 
%% IMPORTANT NOTICE:
%% 
%% For the copyright see the source file.
%% 
%% Any modified versions of this file must be renamed
%% with new filenames distinct from sample-manuscript.tex.
%% 
%% For distribution of the original source see the terms
%% for copying and modification in the file samples.dtx.
%% 
%% This generated file may be distributed as long as the
%% original source files, as listed above, are part of the
%% same distribution. (The sources need not necessarily be
%% in the same archive or directory.)
%%
%%
%% Commands for TeXCount
%TC:macro \cite [option:text,text]
%TC:macro \citep [option:text,text]
%TC:macro \citet [option:text,text]
%TC:envir table 0 1
%TC:envir table* 0 1
%TC:envir tabular [ignore] word
%TC:envir displaymath 0 word
%TC:envir math 0 word
%TC:envir comment 0 0
%%
%%
%% The first command in your LaTeX source must be the \documentclass
%% command.
%%
%% For submission and review of your manuscript please change the
%% command to \documentclass[manuscript, screen, review]{acmart}.
%%
%% When submitting camera ready or to TAPS, please change the command
%% to \documentclass[sigconf]{acmart} or whichever template is required
%% for your publication.
%%
%%
% \documentclass[manuscript,review,anonymous]{acmart}
% \documentclass[review,anonymous,sigconf]{acmart}
\documentclass[sigconf]{acmart}

%%
%% \BibTeX command to typeset BibTeX logo in the docs
\AtBeginDocument{%
  \providecommand\BibTeX{{%
    Bib\TeX}}}

\newcommand{\beginsupplement}{%
        \setcounter{table}{0}
        \renewcommand{\thetable}{S\arabic{table}}%
        \setcounter{figure}{0}
        \renewcommand{\thefigure}{S\arabic{figure}}%
     }
\usepackage{xcolor}


\definecolor{brown}{rgb}{0.59, 0.29, 0.0}
\definecolor{darkblue}{rgb}{0.0, 0.0, 0.55}
\definecolor{darkgreen}{rgb}{0.0, 0.5, 0.0}

%% Rights management information.  This information is sent to you
%% when you complete the rights form.  These commands have SAMPLE
%% values in them; it is your responsibility as an author to replace
%% the commands and values with those provided to you when you
%% complete the rights form.
% \setcopyright{acmcopyright}
% \copyrightyear{2025}
% \acmYear{2025}
% \acmDOI{https://doi.org/10.1145/3706598.3713962}

%% These commands are for a PROCEEDINGS abstract or paper.
% \acmConference[CHI '25]{Make sure to enter the correct
%   conference title from your rights confirmation email}{April 26-May 1,
%   2025}{Yokohama, Japan}

%%
%%  Uncomment \acmBooktitle if the title of the proceedings is different
%%  from ``Proceedings of ...''!
%%
%%\acmBooktitle{Woodstock '18: ACM Symposium on Neural Gaze Detection,
%%  June 03--05, 2018, Woodstock, NY}
%\acmPrice{15.00}
%\acmISBN{978-1-4503-XXXX-X/18/06}


%%
%% Submission ID.
%% Use this when submitting an article to a sponsored event. You'll
%% receive a unique submission ID from the organizers
%% of the event, and this ID should be used as the parameter to this command.
%\acmSubmissionID{3375}

%%
%% For managing citations, it is recommended to use bibliography
%% files in BibTeX format.
%%
%% You can then either use BibTeX with the ACM-Reference-Format style,
%% or BibLaTeX with the acmnumeric or acmauthoryear sytles, that include
%% support for advanced citation of software artefact from the
%% biblatex-software package, also separately available on CTAN.
%%
%% Look at the sample-*-biblatex.tex files for templates showcasing
%% the biblatex styles.
%%
\usepackage{placeins}
\usepackage{multirow}
\usepackage{verbatim}
\usepackage{afterpage}
\usepackage{float}
\usepackage{graphicx}
\usepackage[table]{colortbl}% http://ctan.org/pkg/xcolor
\usepackage{tikz}
\usepackage{nth}
\usetikzlibrary{calc}
\usepackage{zref-savepos}
\usepackage{xcolor}
\usepackage{booktabs,caption}
\usepackage{setspace}
\usepackage{placeins}
\usepackage{xcolor}
\usepackage{listings}% http://ctan.org/pkg/listings
\lstset{
  basicstyle=\ttfamily,
  mathescape
}
% \usepackage{silence}
% \WarningsOff[images] % This suppresses warnings related to images

\usepackage{algorithm}
\usepackage{algpseudocode}
\usepackage[algo2e, ruled,vlined]{algorithm2e}
\usepackage{tikz}
\usepackage{cancel}
\usepackage{amsmath}
\let\Bbbk\relax
\usepackage{amsfonts}
\usepackage{amssymb}
\usepackage{enumitem}
\usepackage{subcaption}
\usepackage{todonotes}
\newcounter{cpModel}
\makeatletter
\newenvironment{cpModel}[1][htb]{%
  \let\c@algorithm\c@cpModel
  \renewcommand{\ALG@name}{Model}% Update algorithm name
  \begin{algorithm}[#1]%
  }{\end{algorithm}
}
\makeatother 

\definecolor{tableauOrange}{HTML}{F28E2B}
\definecolor{tableauRed}{HTML}{E15759}
\definecolor{tableauBlue}{HTML}{4D79A7}
\definecolor{tableauGreen}{HTML}{77B7B3}
\definecolor{highlightPurple}{HTML}{b17aa1}

\newcommand{\tbOrangeMono}[1]{\textcolor{tableauOrange}{\textbf{\texttt{#1}}}}
\newcommand{\tbRedMono}[1]{\textcolor{tableauRed}{\textbf{\texttt{#1}}}}
\newcommand{\tbBlueMono}[1]{\textcolor{tableauBlue}{\textbf{\texttt{#1}}}}
\newcommand{\tbGreenMono}[1]{\textcolor{tableauGreen}{\textbf{\texttt{#1}}}}
\newcommand{\footnotenum}{\arabic{footnote}}
\setlength{\intextsep}{3pt plus 1pt minus 1pt} 
\raggedbottom
% \newcounter{spmodel}
% \makeatletter
% \newenvironment{spmodel}[1][htb]{%
%   \let\c@algorithm\c@spmodel
%    \renewcommand{\thealgorithm}{} % Reset counter format locally
%   \renewcommand{\ALG@name}{Shortest Path Model}% Update algorithm name
%   \begin{algorithm}[#1]%
%   }{\end{algorithm}
% }
% \makeatother 

%%
%% The majority of ACM publications use numbered citations and
%% references.  The command \citestyle{authoryear} switches to the
%% "author year" style.
%%
%% If you are preparing content for an event
%% sponsored by ACM SIGGRAPH, you must use the "author year" style of
%% citations and references.
%% Uncommenting
%% the next command will enable that style.
%%\citestyle{acmauthoryear}


% custom commands
% \usepackage{xargs} % Used for new commands with optional arguments
\usepackage{soul}  % Used for custom comments
\usepackage{color} % Used for custom colors in comments
% \usepackage{comment}
\usepackage{xspace}
\usepackage{listings}
\usepackage{enumitem}
% \usepackage{ulem}
%% Note: Some commands for spacing Latin letters/abbreviations
\newcommand{\eg}{{\it e.g.,\ }}
\newcommand{\etal}{{\it et al.\ }}
\newcommand{\etc}{{\it etc.}}
\newcommand{\ie}{{\it i.e.,\ }}
\newcommand{\cf}{{c.f.}\xspace}
\newcommand{\aka}{{a.k.a.}\xspace}

%%%%%%%%%%%%%%%%%%%%%%%%%%%%%%%%%%%%%%%%%%%%%%%%
%% Commands for adding comments to the paper. %%
%%%%%%%%%%%%%%%%%%%%%%%%%%%%%%%%%%%%%%%%%%%%%%%%

\usepackage{booktabs}
\definecolor{oxfordblue}{rgb}{0.0, 0.13, 0.28}
\definecolor{harvardcrimson}{rgb}{0.79, 0.0, 0.09}
\definecolor{dartmouthgreen}{rgb}{0.05, 0.5, 0.06}
\definecolor{princetonorange}{rgb}{1.0, 0.56, 0.0}
\definecolor{yaleblue}{rgb}{0.06, 0.3, 0.57}
\definecolor{usccardinal}{rgb}{0.6, 0.0, 0.0}
\definecolor{uclablue}{rgb}{0.33, 0.41, 0.58}
\definecolor{msugreen}{rgb}{0.09, 0.27, 0.23}
\definecolor{cornellred}{rgb}{0.7, 0.11, 0.11}
\definecolor{pomegranate}{RGB}{192, 57, 43}
\definecolor{anti-pomegranate}{RGB}{43,178,192}
\definecolor{alizarin}{RGB}{231, 76, 60}
\definecolor{anti-belize}{RGB}{185, 41, 56}
\definecolor{belize}{RGB}{41, 128, 185}
\definecolor{peter}{RGB}{52, 152, 219}
\definecolor{green}{RGB}{22, 160, 133}
\definecolor{anti-green}{RGB}{160,22,118}
\definecolor{turquoise}{RGB}{26, 188, 156}
\definecolor{pumpkin}{RGB}{211, 84, 0}
\definecolor{anti-pumpkin}{RGB}{0,22,211}
\definecolor{carrot}{RGB}{230, 126, 34}
\definecolor{wisteria}{RGB}{142, 68, 173}
\definecolor{anti-wisteria}{RGB}{99,173,68}
\definecolor{amethyst}{RGB}{155, 89, 182}
\definecolor{nephritis}{RGB}{39, 174, 96}
\definecolor{anti-nephritis}{RGB}{174,39,117}

% \newcommand{\pzh}[1]{{#1}}
% \newcommand{\peng}[1]{{\color{red} #1}}
% \newcommand{\xingbo}[1]{{\textcolor{black}{#1}}}
% \newcommand{\wxb}[1]{{\textcolor{orange}{#1}}}
% \newcommand{\chen}[1]{{\color{green} #1}}

\newcommand{\penguin}[1]{{#1}}
\newcommand{\pzh}[1]{{#1}}
\newcommand{\peng}[1]{{#1}}
\newcommand{\zhenhui}[1]{{#1}}
\newcommand{\haoxiang}[1]{{#1}}
\newcommand{\yh}[1]{{#1}}
\newcommand{\fhx}[1]{{{#1}}}

% \newcommand{\penguin}[1]{{\color{red} #1}}
% \newcommand{\fhx}[1]{{\textcolor{orange}{#1}}}
% \newcommand{\zhenhui}[1]{{\color{carrot} #1}}

% \newcommand{\penguin}[1]{{\color{blue} #1}}
% \newcommand{\fhx}[1]{{\color{blue}{#1}}}
% \newcommand{\zhenhui}[1]{{\color{blue} #1}}

% \newcommand{\fanhx}[1]{{\color{blue}{#1}}}
\newcommand{\fanhx}[1]{{{#1}}}

%% Note: Comment this in to see all comments and unfinished text.
\newcommand{\todo}[1]{\textcolor{red}{[TODO] \emph{#1}}}
\newcommand{\cut}[1]{\textcolor{red}{\st{#1}}}
\newcommand{\sout}[1]{\cut{#1}}
\newcommand{\gray}[1]{\textcolor{gray}{#1}}


\newcommand{\systemname}{{\textit{SystemName}}}
\newcommand{\name}{{\textit{LitLinker}}}
    
% Capitalizing the first letter for section autorefs
\renewcommand{\sectionautorefname}{Section}
\renewcommand{\subsectionautorefname}{Section}
\renewcommand{\subsubsectionautorefname}{Section}



%%
%% end of the preamble, start of the body of the document source.

\newcommand{\revision}[1]{\textcolor{blue}{#1}}

\renewcommand{\figureautorefname}{Figure}
\renewcommand{\tableautorefname}{Table}
\renewcommand{\partautorefname}{Part}
\renewcommand{\appendixautorefname}{Appendix}
\renewcommand{\chapterautorefname}{Chapter}
\renewcommand{\sectionautorefname}{Section}
\renewcommand{\subsectionautorefname}{Section}
\renewcommand{\subsubsectionautorefname}{Section}

\copyrightyear{2025}
\acmYear{2025}
\setcopyright{cc}
\setcctype{by}
\acmConference[CHI '25]{CHI Conference on Human Factors in Computing Systems}{April 26-May 1, 2025}{Yokohama, Japan}
\acmBooktitle{CHI Conference on Human Factors in Computing Systems (CHI '25), April 26-May 1, 2025, Yokohama, Japan}\acmDOI{10.1145/3706598.3713962}
\acmISBN{979-8-4007-1394-1/25/04}

\begin{document}

%%
%% The "title" command has an optional parameter,
%% allowing the author to define a "short title" to be used in page headers.
% \title{An Ontological Framework for Human-Centered AI Image Analysis}
\title[Characterizing Photorealism in AI-Generated Images]{Characterizing Photorealism and Artifacts in Diffusion Model-Generated Images}

% \title[Categorizing Implausibilities in Diffusion Model-Generated Images]{Designing a Framework for Categorizing Implausibilities in Diffusion Model Generated Images}

%%
%% The "author" command and its associated commands are used to define
%% the authors and their affiliations.
%% Of note is the shared affiliation of the first two authors, and the
%% "authornote" and "authornotemark" commands
%% used to denote shared contribution to the research.
% \author{Negar Kamali}
% \email{negar.kamali@anonymous.xxx}
% \orcid{}
% \affiliation{%
%   \institution{Anonymous Institution}
%   \city{Anonymous City}
%   \state{Anonymous State}
%   \country{Anonymous Country}
% }
\author{Negar Kamali}
\email{negar.kamali@u.northwestern.edu}
\orcid{0000-0002-1086-6735}
\affiliation{%
  \institution{Northwestern University}
  \city{Evanston}
  \state{Illinois}
  \country{USA}
}

\author{Karyn Nakamura}
\email{karynnakamura68@gmail.com}
\orcid{0009-0005-4419-0701}
\affiliation{%
  \institution{Northwestern University}
  \city{Evanston}
  \state{Illinois}
  \country{USA}
}

\author{Aakriti Kumar}
\email{aakriti.kumar@kellogg.northwestern.edu}
\orcid{0000-0002-9502-013X}
\affiliation{%
  \institution{Northwestern University}
  \city{Evanston}
  \state{Illinois}
  \country{USA}
}

\author{Angelos Chatzimparmpas}
\email{a.chatzimparmpas@uu.nl}
\orcid{0000-0002-9079-2376 }
\affiliation{%
  \institution{Utrecht University}
  \city{Utrecht}
  \state{}
  \country{Netherlands}
}

\author{Jessica Hullman}
\email{jhullman@northwestern.edu}
\orcid{0000-0001-6826-3550}
\affiliation{%
  \institution{Northwestern University}
  \city{Evanston}
  \state{Illinois}
  \country{USA}
}

\author{Matthew Groh}
\email{matthew.groh@kellogg.northwestern.edu}
\orcid{0000-0002-9029-0157}
\affiliation{%
  \institution{Northwestern University}
  \city{Evanston}
  \state{Illinois}
  \country{USA}
}
%%
%% By default, the full list of authors will be used in the page
%% headers. Often, this list is too long, and will overlap
%% other information printed in the page headers. This command allows
%% the author to define a more concise list
%% of authors' names for this purpose.
\renewcommand{\shortauthors}{Kamali et al.}

% \renewcommand{\shortauthors}{Anonymous et al.}
\renewcommand{\thesubfigure}{\textbf{\Alph{subfigure}}}
\captionsetup[sub]{labelformat=simple}  % Remove parenthese
\newcommand{\mybold}[1]{\textbf{#1}}

% \received{20 February 2007}
% \received[revised]{12 March 2009}
% \received[accepted]{5 June 2009}

%%
%% This command processes the author and affiliation and title
%% information and builds the first part of the formatted document.
\begin{abstract}
Humor is a social binding agent. It is an act of creativity that can provoke emotional reactions on a broad range of topics. Humor has long been thought to be “too human” for AI to generate. However, humans are complex, and humor requires our complex set of skills: cognitive reasoning, social understanding, a broad base of knowledge, creative thinking, and audience understanding. We explore whether giving AI such skills enables it to write humor. We target one audience: Gen Z humor fans. We ask people to rate meme caption humor from three sources: highly upvoted human captions, 2) basic LLMs, and 3) LLMs captions with humor skills. We find that users like LLMs captions with humor skills more than basic LLMs and almost on par with top-rated humor written by people. We discuss how giving AI human-like skills can help it generate communication that resonates with people. 

\end{abstract}

\begin{CCSXML}
<ccs2012>
   <concept>
       <concept_id>10003120.10003121.10011748</concept_id>
       <concept_desc>Human-centered computing~Empirical studies in HCI</concept_desc>
       <concept_significance>500</concept_significance>
       </concept>
   <concept>
       <concept_id>10003120.10003121</concept_id>
       <concept_desc>Human-centered computing~Human computer interaction (HCI)</concept_desc>
       <concept_significance>500</concept_significance>
       </concept>
 </ccs2012>
\end{CCSXML}

\ccsdesc[500]{Human-centered computing~Empirical studies in HCI}
\ccsdesc[500]{Human-centered computing~Human computer interaction (HCI)}

\keywords{photorealism, diffusion models, generative AI, synthetic media, deepfakes, misinformation}

\maketitle

%!TEX root=main.tex

\section{Introduction}
% Decision-makers, analysts, data scientists, and policymakers frequently rely on data to draw conclusions and extract insights. This data-driven approach helps them identify actionable recommendations aimed at influencing an outcome of interest, such as increasing product satisfaction or income levels or decreasing the likelihood of experiencing serious health conditions \cite{galhotra2022hyper,lakkaraju2016interpretable,agrawal1994fast}. 
\revc{Prescriptions, or actionable recommendations, are commonly generated across various fields to influence key outcomes such as improving product satisfaction, enhancing economic policies, or increasing business efficiency. 
%Decision- or policy-makers, analysts, data scientists, and 
Policymakers in government, decision-makers in businesses, and data scientists in various fields, often rely on data-driven approaches to identify 
%actionable recommendations 
potential actions to influence an outcome of interest, such as increasing income levels or loan approval rates}.
% , or decreasing the likelihood of experiencing serious health conditions. 
%
While association or prediction-based methods are extensively used in practice to draw useful insights from data, they typically identify correlations among variables and may fail to reveal the underlying causal factors, i.e., which actions may result in an improved outcome, needed for informed decision-making. 
%For recommendations to be truly impactful, there must be a clear  explanation that justifies why a particular decision is appropriate for a specific subpopulation~\cite{sun2021treatment,plecko2022causal}. 

\emph{Causal analysis} or {\em causal inference}, therefore, is considered one of the most important requirements to generate prescriptions that are {\em actionable} and aligned with human reasoning~\cite{imbens2024causal}. Causal inference, and in particular {\em observational studies} for causal inference on collected data (when controlled trials are impossible due to cost or ethical reasons), have been extensively studied in the statistics and artificial intelligence (AI) literature for several decades \cite{rubin2005causal, pearl2009causal}. Motivated by this foundational work on causal inference, the notion of causality has also influenced the field of database research. The causal models from AI have been extended to relational databases \cite{salimi2020causal},  and causality has been incorporated into various data management tasks such as finding responsibilities of inputs toward query answers ~\cite{meliou2010causality, meliou2009so, meliou2014causality}, explanations for query answers \cite{roy2014formal, DBLP:journals/pacmmod/YoungmannCGR24}, data discovery~\cite{galhotra2023metam,youngmann2023causal}, data cleaning~\cite{pirhadi2024otclean,salimi2019interventional}, hypothetical reasoning \cite{galhotra2022causal}, and large system diagnostics~\cite{markakis2024sawmill,causalsim,sage, gudmundsdottir2017demonstration}. 


\revc{If-then rules are generally considered interpretable by humans~\cite{lakkaraju2016interpretable,guidotti2018local,van2021evaluating,pradhan2022interpretable,chen2018optimization}.
We give a concrete example of the difference between association and causation in generating prescriptions or recommended actions in the form of if-then rules below}:
\begin{example}	%
\label{example:ex1} {\bf Importance of causal prescriptions:}
Consider the Stack Overflow (SO) annual developer survey
\cite{stackoverflowreport}, where respondents from around the world answer
questions about their jobs and demographics. A sample of the dataset \reva{with a subset of the
attributes (there are 20 attributes)} is presented in \cref{tab:data}.
%
Alice, a researcher in the United Nations (UN) finance department, is interested in discovering ways to increase the salaries of high-tech employees worldwide. She is looking for a set of actionable recommendations 
%(that we call a prescription rules) 
to raise the overall average salary.
%
Using association-based approaches~\cite{chen2018optimization,lakkaraju2016interpretable}, she may discover that individuals residing in the US who identify as straight or heterosexual tend to earn higher salaries (see \cref{exp:quality} for full details). However, this observation merely indicates a correlation: people living in the US, for example, generally earn more than those outside the country. Their comparatively higher salaries are primarily attributable to the country's economy and are unrelated to their sexual orientation. Thus, this observation cannot be used as a prescription rule to increase salary. 
Our causal analysis, on the other hand, reveals that individuals aged 25-34 with dependents would benefit from working as front-end developers.
This results in a \$44,009 annual salary increase on average. \reva{Another observation is that students should pursue an
undergraduate major in CS. %Computer Science (CS). 
This can boost their salary by \$22,174 per year} (see details in \cref{sec:casestudy}).
\end{example}

%It has been incorporated into various tasks including . 
%Causal interventions are often more relatable and easier to understand, as they offer insight into the underlying reasons behind the recommendations and allow unraveling complex cause-effect relationships that govern our world~\cite{pearl2009causality}. Furthermore, causal interventions often have long-lasting effects~\cite{imbens2024causal}.

%, making it essential that the prescribed actions are not only actionable but also 

%causally consistent. 

%Decision makings, in particular, high-stak

\cut{
In this work, {we study the problem of generating causal insights (referred to as \emph{prescription rules}), which serve as actionable recommendations} to improve an outcome of interest.
Recent works have introduced causality to the field of database research~\cite{meliou2010causality,  meliou2014causality,salimi2020causal,10.14778/3554821.3554902}. It has been incorporated into various tasks including data discovery~\cite{galhotra2023metam,youngmann2023causal}, data cleaning~\cite{pirhadi2024otclean,salimi2019interventional}, and large system diagnostics~\cite{markakis2024sawmill,causalsim,sage, gudmundsdottir2017demonstration}. 
We propose using causal inference to generate prescription rules that are both actionable and justifiable.
}

While generating prescriptions based on causal inference may help in robust decision-making, causal prescriptions that solely consider the betterment of an outcome (like salary) are not enough in practice. 
It is well-known that decision-making in many high-stake applications (like hiring policy, or policy for approving loans by banks) may lead to disparate societal or economic impact on different sub-populations. 
As a shocking example from a recent work called 
%For example, 
CauSumX~\cite{DBLP:journals/pacmmod/YoungmannCGR24} that generates a set of causal explanations for an aggregated view, the explanations generated %by CauSumX %recommendations which 
suggest that male individuals do a Bachelor's degree to increase their salary while %suggesting that 
being an unmarried woman 
%the recommendation for women includes getting married 
has the most adverse effect on salary
(borrowed directly 
from Fig.~19 in~\cite{youngmann2024summarizedcausalexplanationsaggregate}). 
%We demonstrate the advantage of using causal reasoning to generate actionable recommendations and the limitations of not considering fairness requirements in the following example. 
We explored this further in the context of generating prescriptions and observed that prescriptions that are not fairness-aware can generate unfair outcomes to some subpopulations which we refer to as the {\em protected group}. Examples include women, Black, Latino, or Native Americans, individuals with a disability, countries with a weaker economy, or other protected groups specific to an application. %Here is a concrete example:


% Understanding the causal factors behind these recommendations is crucial to ensuring that decisions lead to fair and equitable outcomes, particularly in sensitive applications where biased decisions can perpetuate or even exacerbate societal inequalities.
% While prior work has extensively explored techniques for association rule mining~\cite{kumbhare2014overview}, recent efforts have focused on deriving causal explanations for individual data points or entire datasets~\cite{salimi2018bias,youngmann2022explaining,ma2023xinsight}. Although some of these methods produce causally consistent insights, the absence of fairness considerations in the process can lead to unfair outcomes, further reinforcing existing biases. For example, CauSumX~\cite{DBLP:journals/pacmmod/YoungmannCGR24} generates causal recommendation which suggest male individuals to do a Bachelor's degree to increase salary while the recommendation for women include getting married (borrowed directly from Figure~19 in the paper~\cite{youngmann2024summarizedcausalexplanationsaggregate}). 





%\emph{Causal inference} has been thoroughly studied in AI and Statistics~\cite{pearl2009causal,rubin2005causal}. Causal analysis is a vital tool in determining the effect of a \emph{treatment} on an \emph{outcome}, and has been used in decision-making in medicine \cite{robins2000marginal}, economics \cite{banerjee2011poor}, biology \cite{shipley2016cause}, and in high-stakes areas such as identifying the root causes of failures in critical infrastructure systems to prevent catastrophic outcomes. Recent works have introduced causality to the field of database research~\cite{meliou2010causality,  meliou2014causality,salimi2020causal,10.14778/3554821.3554902}. It has been incorporated into various tasks including data discovery~\cite{galhotra2023metam,youngmann2023causal}, query result explanation~\cite{salimi2018bias,youngmann2022explaining,DBLP:journals/pacmmod/YoungmannCGR24}, and large system diagnostics~\cite{markakis2024sawmill,causalsim,sage, gudmundsdottir2017demonstration}. We propose leveraging causal inference to generate interpretable and justifiable insights (referred to as \emph{prescription rules}), which serve as actionable recommendations to improve an outcome of interest. Causal reasoning is considered one of the most important requirements,  to generate insights that are actionable and aligned with human reasoning.




\begin{table*}[]
\footnotesize
    \centering
    	\caption{\textnormal{A subset of the Stack Overflow dataset.}}
         \label{tab:data}
    	% \vspace{-4mm}
  			\begin{tabular}[b]{|l|l|l|c|l|l|c|l|c|}
  			
				%\multicolumn{9}{c}{\textbf{Users}}\\ 
				\hline

				\textbf{ID}
    
    % \textbf{Country}& \textbf{Continent} 
    
    &\textbf{Gender} &\textbf{Ethnicity}&
				\textbf{Age} &\textbf{Role} &
				\textbf{Education} &\textbf{Country}&\textbf{Undergrad Major}&\textbf{Salary}
				\\ \hline

				1 &Male&White&26&Data Scientist & PhD& US&Computer Science&180k\\
    		2 &Non-binary&White&32&QA developer & Bachelor's degree& US&Mechanical Eng.&83k\\

 3 &Male&South Asian&29&C-suite executive  & Bachelor's degree & India&Computer Science&24k\\

  % 4 &Female&South Asian&25&Back-end developer  & Master's degree & India&Mathematics&7.5k\\

  4 &Female&East Asian&21&Back-end developer & Bachelor's degree & China&Computer Science&19k\\
  

        % $\ldots$ &  $\ldots$&  $\ldots$&  $\ldots$&  $\ldots$&  $\ldots$&  $\ldots$&  $\ldots$&  $\ldots$&  $\ldots$&  $\ldots$\\
    \hline
			\end{tabular}
            \vspace{-5mm}
\end{table*}




\begin{example}	%
\label{example:ex2}
{\bf Importance of fair prescriptions:}
Continuing Example~\ref{example:ex1}, while those causal prescription rules are highly beneficial for the overall population, they are considerably less effective for individuals residing in countries with a low GDP (indicating a weaker economy). For this group, the average expected increase in salary is only approximately \$13,000 per year (in contrast to \$44,009 for the entire group). % \sr{add which rule 44k or 25k} 
Consequently, implementing these rules would exacerbate the disparity between those living in countries with strong economies and those in countries with weaker economies.
\end{example}




% Our objective is to generate a small set of prescription rules aimed at increasing (or decreasing) an outcome of interest. This is framed as an optimization problem where the goal is to select the fewest prescription rules that maximize utility (i.e., the expected increase or decrease in the outcome). However, 

The example above shows that focusing solely on maximizing utility (\revc{i.e., increasing income}) can result in a scenario where only some of the population receive significant improvement, while others experience no benefit (\revc{only a small benefit for individuals from countries with weaker economies in our example}). Additionally, even if a large portion of the population receives recommendations, a protected subpopulation might not share the benefits and, worse, their situation could deteriorate, exacerbating inequalities.

Examples~\ref{example:ex1} and \ref{example:ex2} show that it is crucial to provide recommendations that are (1) {\em causal} for the outcome (beyond associations),  and (2) also {\em fair or equitable} in terms of the outcome for both the protected and non-protected groups. While recent work in database research
has focused on deriving {\em causal explanations} for individual data points, aggregated view, or entire datasets~\cite{salimi2018bias,youngmann2022explaining,ma2023xinsight, DBLP:journals/pacmmod/YoungmannCGR24}, and in particular \cite{DBLP:journals/pacmmod/YoungmannCGR24} has considered generating a set of causal explanations for an aggregated view that resemble a ruleset, 
%Although some of these methods produce causally consistent insights, 
the absence of fairness considerations in generating these causal explanations can lead to unfair outcomes for the protected group.
%further reinforcing existing biases.


%\red{We, therefore, enable users to incorporate various \emph{coverage and fairness constraints} along with the overall objective of improving an outcome of interest. }

\medskip
\noindent
\textbf{Our contributions.~} 
Motivated by the dual goals of generating causal and fair prescriptions for the betterment of an outcome, we introduce a {\em fairness-aware framework leveraging causal reasoning for generating a set of actionable prescription rules (ruleset)} called \sysName\ (\underline{Fair} \underline{CA}usal \underline{P}rescription).
%
Following research on fairness in data management~\cite{stoyanovich2020responsible,galhotra2022causal}, we assume the existence of a \emph{protected subpopulation}, defined by an attribute such as gender or race for people, or GDP of a country. Motivated by the causal explanation rules for an aggregated view \cite{DBLP:journals/pacmmod/YoungmannCGR24}, each prescription rule in our ruleset applies to a sub-population defined by a {\em grouping attribute}, and prescribes a {\em treatment or intervention} to improve the {\em outcome} for this sub-population. Fairness constraints ensure that the expected utility of the protected population is {\em comparable} to the utility of the unprotected individuals. We borrow the notions of \emph{group and individual fairness} from the fairness literature but tailor them for prescription rules. In addition to the fairness constraints, our coverage constraints ensure that a substantial fraction of the population and protected subpopulation receives at least one recommendation. 
%We demonstrate how such constraints ensure that the generated rules apply to a large portion of the population and ensure fairness through the following example.

\begin{example}
\label{ex:intro_example_3}
Continuing Examples~\ref{example:ex1} and \ref{example:ex2}, Alice uses our proposed system, called \sysName, to impose fairness and coverage constraints to discover useful and equitable recommendations for increasing salaries worldwide. In particular,
Alice chooses to implement a coverage constraint to ensure that the selected rules apply to a significant portion of people worldwide, including a sufficiently large number of individuals from countries with low GDP (the protected group). She also imposes a fairness constraint to ensure that the expected gains for both protected and non-protected groups are comparable.
\reva{She discovers, for example, that for individuals with 6-8 years of coding experience (a subpopulation comprising 21\% of the entire dataset and 25\% of the protected group), pursuing a bachelor’s degree in computer science will increase the expected salary by $\$14.9k$ for protected and by $\$17.8k$ for non-protected}. (See \cref{sec:casestudy} for more details.) This prescription rule applies to a large portion of the population and ensures fairness by providing a similar expected gain for both protected and non-protected groups, and the allowed difference of outcomes between these two populations may be adjusted by choosing appropriate thresholds in the fairness definitions. 
\end{example}


\noindent
Our main contributions are as follows. \\
%\begin{itemize}[leftmargin=*,topsep=0pt]
{\bf (1)} We {\bf develop a framework that generates a set of prescription rules to enhance an outcome of interest (Section~\ref{sec:problem})}. A prescription rule consists of a \emph{grouping pattern} and an \emph{intervention pattern}, representing the target subpopulation and the actionable recommendation for that group, respectively. The strength of the {\em conditional causal effect} (Section~\ref{sec:background-causal}) of this intervention on the subgroup is used to measure the expected utility of a rule. Our objective is to identify the smallest set of rules that maximizes overall expected utility. We refer to this problem as the {\em \probName} problem.
We adopt several notions of fairness (individual vs. group, statistical parity vs. bounded group loss) from the literature to define the {\bf fairness constraints} for our problem. In addition, {\bf coverage constraints} (for individual rules or for a group) ensure that the solution for the \probName\ problem is applied to a sufficient number of individuals and to minimize inequalities. We show NP-hardness for different variants of the problems and properties (matroid) useful in our algorithms. 
%We establish several definitions for group and individual fairness constraints tailored for prescription rules.
\smallskip
    \par
    \noindent
{\bf (2)} We {\bf develop a general three-step algorithm named \sysName to solve the optimization problem of selecting a fair prescription ruleset (Section~\ref{sec:algo})}. The first step involves mining frequent grouping patterns using the Apriori algorithm~\cite{agrawal1994fast}. In the second step, we employ a lattice-based algorithm to find high utility and fair intervention patterns for grouping patterns identified in the previous step. Finally, the third step applies a greedy approach to determine a solution. \sysName\ can be easily adapted to accommodate all variants of the \probName\ problem.

\smallskip
\par
\noindent
{\bf (3) We provide a detailed  case study  (Section~\ref{sec:casestudy}) and experimental analysis (Section~\ref{sec:experiments}) to evaluate our framework and algorithms.}
The case study shows the qualitative difference of different variants of our problem for different choices of the fairness and coverage constraints. The experiments include two datasets, three baselines, and 18 variations of our problem with different constraints. Our evaluations suggest that fairness may come at the cost of expected
utility for everyone. However, without fairness constraints, we often observe a significant disparity between the protected and non-protected. We also observe that
achieving individual fairness is harder than group fairness,
as most high-utility or high-coverage rules are unfair. Lastly, we show that \sysName\ can generate  prescription rules over large datasets in a reasonable time. 

%\end{itemize}


%\paragraph*{Paper outline} 
We discuss related work in \cref{sec:related}, review background on causal inference in \Cref{sec:background-causal}, %and our problem formulation can be found in \cref{sec:problem}. Our algorithmic framework is presented in \cref{sec:algo}. A case study demonstrating the impact of different constraint configurations on the solution is given in \cref{exp:problem_variants}, and our experimental evaluation is detailed in \cref{sec:experiments}. Finally, we 
and discuss the limitations of our framework and future work in \cref{sec:conc}.

% \noindent
% \boxed{\parbox{\columnwidth}{$\bullet$ 
% For people with a professional degree, move to the United Kingdom
%  (coverage = 435 (20), coverage-protected = 20 (13), utility = 186855, utility-protected = 0.)\\
% $\bullet$ For graphic developers, move to the	United States
%  (coverage = 116 (29), coverage-protected = 8 (2), utility = 169431, utility-protected = 0).\\
% $\bullet$ For people who have no formal education, move to the United States
%  (coverage = 123 (34), coverage-protected = 7 (2), utility = 206742, utility-protected = 0).\\
% % \textcolor{red}{size = 38, length = 76, overlap = 64029181, utility = 1659307}\\
% \textcolor{blue}{overall coverage =674, expected utility = 187485
% coverage-protected = 35, expected utility-protected = 0}
% \sr{should mention protected group, and possibly not mention coverage in the intro or just intuitively like high coverage}
% }}


% Alice notes that although these rules result in a \$187,485 increase in the overall salary for those to whom they apply, they only affect a small fraction of the population, specifically 674 individuals. Additionally, although the expected salary increase is substantial, there is no expected increase in salary for non-males, a subpopulation of particular interest to Alice. In other words, applying these rules would result in no gain for non-males.
% \end{example}

% \begin{example}[Episode 2 - coverage and fairness constraints]
% Alice introduces coverage and fairness constraints to ensure that enough people will benefit from the rules and that they will be \emph{fair} with respect to non-males. Specifically, she demands that the benefit for a randomly chosen individual to whom one of the rules applies is nearly the same as the benefit for a randomly chosen individual who identifies as non-male and to whom one of the rules applies.

% After adding these constraints, \sysName\ recommends the following set of prescription rules:



% \noindent
% \boxed{\parbox{\columnwidth}{$\bullet$ 
% For people who have no formal education, move to the United States
%  (coverage = 123 (34), coverage-protected = 7 (2), utility = 206742, utility-protected = 0)\\
% $\bullet$ 
% For females, change role to	DevOps specialist (coverage = 2256 (47), coverage-protected = 2256 (47), utility = 90023, utility-protected = 90023).\\
% $\bullet$ For people with a Master's degree, move to the	United States
%  (coverage = 9097 (2222), coverage-protected = 642 (236), utility = 85390, utility-protected = 84201).\\
% % \textcolor{red}{size = 38, length = 76, overlap = 64029181, utility = 1659307}\\
% \textcolor{blue}{overall coverage =11476	
% , expected utility = 87601,
% coverage-protected = 2905, expected utility-protected = 88519}
% }} 







% \begin{figure}[t]
%         \centering
%         \begin{minipage}[b]{1.0\linewidth}
%             \small
%             \begin{tcolorbox}[colback=white]
%             \vspace{-2mm}
% $\bullet$ For backend developers, the treatment with the highest effect on salary is “Country = US” effect size = 78646
% \begin{itemize}
%     \item For non-male the effect is only: 59429
%     \item For male the effect is 80454
% \end{itemize}

% $\bullet$ For frontend developers, the treatment with the highest effect is :Formal Education = Bachelor's degree” effect size: 17340
% \begin{itemize}
%     \item For white the effect is 33464
%     \item For non-white the effect is 15320
% \end{itemize}


% $\bullet$ For people in Europe, the treatment with the highest effect on salary is “DevType = C-suite executive” effect size = 53254
% \begin{itemize}
%     \item For white the effect is 55112
%     \item For non-white 35249
% \end{itemize}



%             \vspace{-2mm}
%             \end{tcolorbox}
%         \end{minipage}%%
%          % \vspace{-4mm}
%         \caption{Set of prescription rules.}
%         \label{fig:so-explanation}
%     \end{figure}

\section{Background}\label{sec:relwork}

\subsection{Limitations of machine learning approaches to detect AI-generated images}

Machine learning models for detecting AI-generated images are brittle and lack robustness to simple data transformations. Corvi et al.~\cite{corvi2023intriguingpropertiessyntheticimages} compare four different machine learning approaches to deepfake detection and demonstrate that recropping and compression – simple modifications common on social media – lead to drops in accuracy such that the classifiers are nearly just as good as random guessing. Dong et al.~\cite{9879575}reveal the ease with which spectral artifacts used in the identification of GAN-generated images can be mitigated via blurring and resizing, demonstrating a noticeable decrease in accuracy under basic modifications. Cozzolino et al.~\cite{cozzolino2024raisingbaraigeneratedimage} demonstrate that post--processing images by random--cropping, resizing, and compression lead to a drop in AI-generated image detection from 90\% accuracy to 85\% accuracy. The fundamental problem is that machine learning models for deepfake detection lack robustness to context shift, out--of--distribution data and adversarial perturbations ~\cite{wang2023deepfakedetectioncomprehensivestudy, ha2024, groh2022identifying, hulzebosch2020detectingcnngeneratedfacialimages}. 

How an image is generated influences the ability of deepfake detection classifiers to accurately identify it as AI-generated. Classifiers trained to detect GAN-generated images tend to fail to detect diffusion model-generated images. For example, the approach to detecting GAN-generated images based on frequency spectra~\cite{marra2018gansleaveartificialfingerprints, yu2019attributingfakeimagesgans, 9035107, durall2020watchupconvolutioncnnbased, bi2023detectinggeneratedimagesreal, pmlr-v119-frank20a} and inconsistencies in head poses and facial landmark positions~\cite{yang2018exposingdeepfakesusing, yang2019exposinggansynthesizedfacesusing, Mundra_2023_CVPR}, do not generalize to detecting images generated by diffusion models~\cite{ojha2024universalfakeimagedetectors}. GAN-trained detection models miss these patterns because they have learned patterns for identifying GAN-generated images~\cite{wang2020cnn, ricker2024detectiondiffusionmodeldeepfakes}. Likewise, it is possible to learn the statistical regularities in diffusion model-generated images but these regularities are not invariant to image post-processing.~\cite{xi2023aigeneratedimagedetectionusing, 10334046, wang2023dirediffusiongeneratedimagedetection, ma2023exposingfakeeffectivediffusiongenerated, yang2023diffusion}. 

Moreover, machine learning models' lack of robustness for detection is exacerbated by the changing architectures of generative AI models~\cite{lin2024, Mirsky2021}. Vision transformers~\cite{radford2021learning, ojha2024universalfakeimagedetectors} and multi--architecture training~\cite{epstein2023onlinedetectionaigeneratedimages, porcile2024findingaigeneratedfaceswild, jia2024can} show promise for enhancing the detection of AI-generated images, but adversarial attacks and large architectural changes in generative models continue to affect robustness of detection.  

Figure~\ref{fig:AI-faces} highlights the increasing complexity of AI-generated images over the past decade. The changing architectures and increasing photorealism pose a challenge for both humans and machines to distinguish real from AI-generated images. However, humans and machines are fundamentally different. For example, humans can critically reason about an image's elements and its context~\cite{wang2023context}. On the other hand, machine learning classifiers for detecting AI-generated images often oversimplify image authenticity as a question of real versus fake and ignore the critical reasoning about component parts and sub--questions that an ordinary person or digital forensics expert may consider when evaluating an image's authenticity~\cite{jacobsen2024deepfakes}.


\begin{figure*}[h]
\centering
\captionsetup{justification=raggedright, singlelinecheck=false, skip=2pt}
\begin{subfigure}[t]{0.13\textwidth}
    \subcaption{}\vtop{\vskip0pt\hbox{\includegraphics[width=\linewidth]{sections/images/ganfacesgoodfellow.jpg}}}
    \caption*{\footnotesize GAN '14}
\end{subfigure}
\hfill
\begin{subfigure}[t]{0.13\textwidth}
    \subcaption{}\vtop{\vskip0pt\hbox{\includegraphics[width=\linewidth]{sections/images/dcgan.jpg}}}
    \caption*{\footnotesize DCGAN '15}
\end{subfigure}
\hfill
\begin{subfigure}[t]{0.13\textwidth}
    \subcaption{}\vtop{\vskip0pt\hbox{\includegraphics[width=\linewidth]{sections/images/PGGAN.jpg}}}
    \caption*{\footnotesize PGGAN '18}
\end{subfigure}
\hfill
\begin{subfigure}[t]{0.13\textwidth}
    \subcaption{}\vtop{\vskip0pt\hbox{\includegraphics[width=\linewidth]{sections/images/stylegan.jpg}}}
    \caption*{\footnotesize StyleGAN '19}
\end{subfigure}
\hfill
\begin{subfigure}[t]{0.13\textwidth}
    \subcaption{}\vtop{\vskip0pt\hbox{\includegraphics[width=\linewidth]{sections/images/stylegan2.jpg}}}
    \caption*{\footnotesize StyleGAN2 '20}
\end{subfigure}
\hfill
\begin{subfigure}[t]{0.13\textwidth}
    \subcaption{}\vtop{\vskip0pt\hbox{\includegraphics[width=\linewidth]{sections/images/sdxl.jpg}}}
    \caption*{\footnotesize SDXL '23}
\end{subfigure}
\hfill
\begin{subfigure}[t]{0.13\textwidth}
    \subcaption{}\vtop{\vskip0pt\hbox{\includegraphics[width=\linewidth]{sections/images/ff_portrait3_021.jpeg}}}
    \caption*{\footnotesize Firefly '24}
\end{subfigure}
\caption{\mybold{Exemplar images of photorealism across a range of generative models.} \normalfont{Examples of AI-generated images from 2014 to 2024~\cite{goodfellow2014generativeadversarialnetworks,radford2015unsupervised, faceimageforgery,karras2018progressivegrowinggansimproved,karras2019style,karras2020analyzingimprovingimagequality, podell2024sdxl,adobe_firefly}.}}
\label{fig:AI-faces}
\Description{Portrait images from various image generation models that improve in quality and complexity over the years.}
\end{figure*}


\subsection{Human perception and evaluation of AI-generated media}

In response to the increasing realism of AI-generated media, researchers have been examining the degree to which humans can distinguish between authentic and AI-generated media. For example, researchers found that GAN-generated images of faces are indistinguishable from real face portraits~\cite{nightingale2022ai, Lago_2022}. However, for video deepfakes, humans are much better than random guessing~\cite{deepfakedetectionbyhumancrowds}, which may in part be due to humans' specialized ability to process the temporal elements of faces~\cite{deepfakedetectionbyhumancrowds, sinha2006face}. Researchers found that text--to--speech voices were rated as lower in quality and clarity than human voices in 2020~\cite{cambre2020choice} but have reached the point where research participants cannot tell the difference between short 20-second recordings of AI-generated voices and authentically recorded voices~\cite{barrington2024people}.

Recent research has identified specific cues and heuristics that people use to evaluate AI-generated media. For example, cues such as recording settings in the detection of text-to-speech audio~\cite{han2024uncovering} and speaking patterns in political deepfake videos~\cite{groh2024human}. However, two studies found that participants rarely attributed their judgments to specific visual features~\cite{hameleers2024they, wohler2021towards}, and in one of these deepfake studies, researchers found that participants are noticing the artifacts but rarely linking these to manipulation~\cite{wohler2021towards}. With respect to AI-generated text, research has highlighted that people tend to use flawed heuristics when attempting to distinguish AI-generated text from human--written text, like associating grammatical errors with AI-generation~\cite{Jakesch2022HumanHF}. 

Social context also plays a significant role in both what diffusion models generate~\cite{luccioni2024stable} and how people form beliefs about AI-generated images and their content. For example, researchers have found detection ability is influenced by shared identity between the viewer and subject of the content~\cite{mink2024s}. Furthermore, researchers have found that white AI-generated faces were disproportionately judged as human more frequently than their real counterparts~\cite{miller2023ai}. GAN-generated faces in portrait images were often perceived as more trustworthy than real faces~\cite{nightingale2022ai}, and as a result, people were less likely to question their authenticity~\cite{Lago_2022}. In instances where AI-generated images are linked to misinformation, researchers find that labeling AI-generated images and the associated content as ``potentially misleading" instead of simply ``AI-generated" had a stronger influence on curtailing participants' self--reported intentions to share misinformation~\cite{epstein2023label, wittenberg2024labeling}.

Researchers have approached a number of methods for measuring photorealism perceived by humans. For example, prior research has examined photorealism with carefully worded questions such as ``Is the image photorealistic?''~\cite{liang2024rich}, ``Does the image look like a real photo or an AI-generated image?''~\cite{lee2024holistic, otani2023toward} and ``Whether this image could be taken with a camera?''\cite{yan2024sanitycheckaigeneratedimage}. These questions are useful for assessing participants' subjective opinions but do not capture the human ability to distinguish real images from fake images and can potentially suffer from demand characteristic bias. Another approach has been to characterize photorealism by examining the features that can influence realism, such as aesthetics and semantically meaningful content of an image~\cite{peng2024crafting}. A third approach involves simply defining images as photorealistic if they are rendered with computer graphics software~\cite{lyu2005realistic}. In this paper, we approach photorealism from the psychophysics perspective, examining participants' objective performance at distinguishing real images from fake images~\cite{zhou2019hype}. 

\subsection{Categorizing artifacts and implausibilities in diffusion model-generated images} \label{sec:artifimpl}

Previous research on earlier versions of diffusion models categorized the kinds of qualitative failures of diffusion model-generated images as distorted body parts, impossible geometry, physics violations, illogical relationships in a scene, and noise~\cite{borji2023qualitative}. In addition to obvious issues with hands, feet, eyes, and teeth, research at the intersection of digital foresnics and AI-generated images shows details such as corneal reflections~\cite{hu2021exposing} and irregular pupil shapes~\cite{guo2022eyes} can also be artifacts.  Likewise, violations of physics like implausible shadows, lighting, and perspective errors~\cite{farid2022perspectiveinconsistencypainttext, farid2022lighting, sarkar2024shadows} often occur in diffusion model generated images that otherwise appear photorealistic. 
\section{Synthesizing Attribution Data}

\begin{figure*}[ht]
    \centering
    \includegraphics[width=\textwidth]{img/pipeline.drawio.pdf}
    \caption{\textbf{Top:} The \synatt baseline method for synthetic attribution data generation. Given context and question-answer pairs, we prompt an LLM to identify supporting sentences, which are then used to train a smaller attribution model. However, this discriminative approach may yield noisy training data as LLMs are less suited for classification tasks (see \S\ref{sec:experiments-zero-shot}). \textbf{Bottom:} The \synqa data generation pipeline leverages LLMs' generative strengths through four steps: (1) collection of Wikipedia articles as source data; (2) extraction of context attributions by creating chains of sentences that form hops between articles; (3) generation of QA pairs by prompting an LLM with only these context attribution sentences; (4) compilation of the final training samples, each containing the generated QA pair, its context attributions, and the original articles enriched with related distractors.}
    % \caption{\textbf{Top:} The \synatt baseline. Intuitively, we can prompt an LLM for context-attribution by providing the context and question-answer pairs. Then, we train a smaller model on the obtained synthetic data. However, LLMs are less suitable for discriminative (i.e., classification) tasks, and may yield noisy training data (see \S\ref{sec:experiments-zero-shot}). \textbf{Bottom:} The \synqa data generation pipeline consists of four main steps: (1) collection of Wikipedia articles as the source data; (2) extracting the context attributions by creating chains of sentences that form hops between articles; (3) generation of QA pairs by prompting an LLM with only the context attribution sentences; (4) we obtain the resulting \synqa training sample containing three components: the generated QA pair, the context attributions, and the original articles supplemented with related distractor articles.}
    \label{fig:method}
\end{figure*}

Context attribution identifies which parts of a reference text support a given question-answer pair~\cite{rashkin2023measuring}. Formally, given a question $q$, its answer $a$, and a context text $c$ consisting of sentences ${s_1, ..., s_n}$, the task is to identify the subset of sentences $S \subseteq c$ that fully support the answer $a$ to question $q$. To train efficient attribution models without requiring expensive human annotations, we explore synthetic data generation approaches using LLMs.
% Context attribution poses the following question~\cite{rashkin2023measuring}: given a generated text $t_g$ and a context text $t_c$, is $t_g$ attributable to $t_c$? To train models to perform well on this task, we explore how to best generate synthetic attribution data using LLMs. We implement two methods: a discriminative and generative method. 
We implement two methods for synthetic data generation. Our baseline method (\synatt) is discriminative: given existing question-answer pairs and their context, an LLM identifies supporting sentences, which are then used to train a smaller attribution model. Our proposed method (\synqa) takes a generative approach: given selected context sentences, an LLM generates question-answer pairs that are fully supported by these sentences. This approach better leverages LLMs' natural strengths in text generation while ensuring clear attribution paths in the synthetic training data.

%The first method is relatively straightforward and termed \synatt. A simple way to generate synthetic data for context attribution is to ask an LLM to pick out the sentences that support a given question-answer pair. 

% \subsection{Discriminative and Generative Synthetic Data Generation}

% The first method (\synatt) is relatively straightforward: ask the LLM to pick relevant sentences from a provided context that support a given question-answer pair. However, this \textit{discriminative} approach of performing sentence classification overlooks the fact that LLMs excel at \textit{generating} text. Therefore, we design a second data generation method (\synqa) that is generative and thus capitalizes on the strength of LLMs. It involves the following pipeline steps (see also Fig.~\ref{fig:method}): context collection, question-answering generation and distractor mining, which increases the difficulty of the task, thus reflecting more realistic scenarios.

%\textbf{Attribution Synthesis.} The most straightforward approach to generating synthetic data for context attribution is discriminative: prompting an LLM to identify relevant sentences from context documents given a question-answer pair. While intuitive, this approach underutilizes LLMs' capabilities, as they excel at generative rather than discriminative tasks. LLMs are fundamentally designed to generate coherent text following instructions rather than perform binary classification of sentences. In our experiments (\S\ref{sec:experiments}) we dub this method as \synatt.

\subsection{\synqa: Generative Synthetic Data Generation Method}

\synqa consists of three parts: context selection, QA generation, and distractors mining (for an illustration of the method, see Figure~\ref{fig:method}). In what follows, we describe each part in detail.

\textbf{Context Collection.} We use Wikipedia as our data source, as each article consists of sentences about a coherent and connected topic, with two collection strategies. In the first, we select individual Wikipedia articles for dialogue-centric generation and use their sentences as context. In the second, for multi-hop reasoning, we identify sentences containing Wikipedia links and follow these links to create ``hops'' between articles, limiting to a maximum of two paths to maintain semantic coherence, while enabling more complex reasoning patterns (for more details, see Appendix~\ref{app:synthetic_data}).
% \textbf{Context Collection.}  The first step is to select a dataset where each data point is a set of sentences about a coherent and connected topic. These sentences will serve as the context in which we want to find relevant attributions later. We use Wikipedia as the data source
%To better leverage LLMs' generative capabilities, we propose \synqa, a novel and simple approach for synthesizing context attribution data (see Fig.~\ref{fig:method}). 
%We first collect Wikipedia articles that are not present in our testing datasets\footnote{We detect potential data leakage by representing each Wikipedia article as a MinHash signature. Then, for each training Wikipedia article, we retrieve candidates from the testing datasets via Locality Sensitivity Hashing and compute their Jaccard similarity \cite{dasgupta2011fast}. Pairs exceeding a tunable threshold (empirically set to 0.8) are flagged as potential leaks.}.
%For each article, 
% we implement two distinct collection strategies that differ in difficulty. First, we select individual Wikipedia articles and randomly select multiple sentences within each article. Second, we start from a randomly selected sentence containing at least one Wikipedia link
%\footnote{These are human annotated in the Wikipedia articles, or alternatively, can be obtained from entity linking methods \cite{de-cao-etal-2022-multilingual}.} 
% and follow the links to other articles, creating ``hops'' between related content. We limit the chain to a maximum of two hops (connecting up to three articles) to maintain semantic coherence while enabling the more difficult multi-hop reasoning scenarios (for more details, see Appendix~\ref{app:synthetic_data}). 
%In the second strategy, we select individual Wikipedia articles and randomly select multiple sentences within each article that can serve as evidence for generated questions.

\textbf{Question-Answer Generation.} Given the set of contexts, an LLM can now generate question-answer pairs. For single articles, we prompt the model to generate multiple question-answer pairs, each grounded in specific sentences. This creates a set of dialogue-centric samples where questions build upon the previous context. For linked articles, we prompt the model to generate questions that necessitate connecting information across the articles, encouraging multi-hop reasoning.
%\footnote{Note that multi-hop reasoning is not guranteed here; rather, the LLM has the ability to decide whether the question-answer pair involves multiple hops of reasoning. See App. for details.}. 
This yields multi-hop samples requiring integration of information across documents, as well as samples that mimic a dialogue about a specific topic given the context. We provide the full prompts used for generation in Appendix \ref{app:prompts}.

\textbf{Distractors Mining.} To make the attribution task more realistic, we augment each sample with distractor articles. With E5 \cite{wang2022text}, we embed each Wikipedia article in our collection. For each article in the training sample, we randomly select up to three distractors with the highest semantic similarity to the source articles. These distractors share thematic elements with the source articles, but lack information to answer the questions.%do not contain the information necessary to answer the generated questions.

\subsection{Advantages of \synqa}
The \synqa approach has three key advantages:
%over discriminative data generation:
% (1) it leverages LLMs' natural strength in generative tasks; (2) produces diverse multi-hop reasoning scenarios; and (3) creates coherent question-answer pairs with clear attribution paths.
(1) it leverages LLMs' strength in generation rather than classification; (2) creates diverse training samples requiring both dialogue understanding and multi-hop reasoning; and (3) ensures generated questions have clear attribution paths since they are derived from specific context sentences.
By generating both entity-centric and dialogue-centric samples, \synqa produces training data that reflects the variety of real-world QA scenarios, helping models develop robust attribution capabilities, which our experiments demonstrate to generalize across different contexts and domains.
% We formalize the problem of Context Attribution QA as follows: Given a pre-defined context $T_c=\lbrace s_1, s_2, \ldots , s_n \rbrace$---where $s_i$ is a sentence---and an answer text $t_a$ generated by an LLM, the context attribution model should provide a vector $a=(a_1, \ldots , a_n)$, where each element $a_i$ has the following possible values:
% \[
% a_i =
% \begin{cases}
%     1, & \text{if } s_i \text{ supports the generated answer } t_a\\
%     0,  & \text{otherwise} 
% \end{cases}
% \]
% In our setup, we should have at least one entry $a_i = 1$.
% \begin{itemize}
%     \item The simplest way to generate synthetic data for context-attribution is in a discriminative manner: we prompt an LLM to provide the sentence level context attributions given the context documents, question and answer. We deem this generation as discriminative as the model effectively classifies the sentences that are most relevant to the question-answer pair.
%     \item The issue with this approach is that LLM are not best suitable for discriminative tasks, but rather generative. That is, an LLM is better at generating text by following instructions, than classifing sentences/etc.
%     \item To leverage what LLMs are good for, we create a simple context attribution data generation approach where we perform the following: (1) We find wikipedia articles (which are not contained in the testing datasets)\footnote{Describe the approach for dealing with data leakage}; (2) We select a random sentence in a wikipedia article, and find the links to other wikipedia articles (the hops). We select that sentence, and hop to the other Wikipedia article (given by the link). (3) We perform the hop step for maximum of 2 times (i.e., we connect at most 3 articles, and 1 at least). We end up with 3 Wikipedia articles which constitute the hops.
%     \item We provide Llama70B with either 1 wikipedia article or the hops and ask the model to generate a multi-hop question-answer pair which ideally connects all connected articles, or as many as it can; alternatively, if we provide the model with only 1 wikipedia article, we ask the model to select as many sentences as possible in the article, and for each, generate a question-answer pair (we provide the full prompts we use in Appendix).
%     \item The output of the model is a set of question-answer pairs (or a single one), that is grounded in the evidence provided by the sentence(s). We dub the entire approach as \synqa.
%     \item In summary, we develop two settings to generate synthetic data for context attribution in question answering: one is entity-centric and yield data which might be multi-hop; and the other is dialog-centric where subsequent questions build on top of previous ones.
%     \item Finally, to all context + question + answer + context-attribution samples we add distractors: we obtain embeddings using E5 of each wikipedia page, and for each sample we select up to 3 distractors which we add to the data sample. These distractors are similar are document with similar context as the one from which the context-attributions are.
% \end{itemize}


\section{Accuracy in Distinguishing AI-generated Images from Real Photographs}\label{sec:results}

In the main phase of the experiment, we collected 539,749 responses on 599 images from 37,568 participants from February 5, 2024 to June 22, 2024. Sections~\ref{sec:acc-general} through \ref{sec:acc-model} focus on data from the main phase of the experiment. The second phase of the experiment started on June 22 and ended on August 30, with 83,577 responses on 482 images from 3,787 participants. Sections~\ref{sec:imagestimuli} and~\ref{sec:human-curation} describe the influence of human curation of the stimuli on how accurately participants identify the stimuli as AI-generated or real. 

The design of our experiment involves several important design choices. First, we selected the three models
of Midjourney, Firefly, and Stable Diffusion as the diffusion models. Second, we crafted prompts to produce realistic
outputs across various pose categories and content types. Third, we curated 450 images from over 3000 images generated
to use as image stimuli in the experiment. These images were selected to maximize realism while also representing
different visual artifacts and implausibilities. Inevitably, these design choices on models, prompts, and stimuli introduce some selection bias  into the experiment.

Additionally, we implemented two exclusion criteria that should be considered when interpreting our results. First, for all the analyses in Section ~\ref{sec:results}, we excluded observations where participants checked the box on the website ``I have seen this before''. These observations, which account for 2\% of the total observations, were excluded because of the strong possibility that participants who had previously seen the images were already aware of whether they were fake or real.
%the experiment aimed to evaluate whether participants could identify if an unfamiliar image was AI-generated.. 
For these observations marked as having been seen before, 38\% of these observations were on AI-generated stimuli and 62\% were on real images. The image most frequently reported as 'seen before' is a real portrait of Martin Luther King Jr, which was one of the few real images of a well-known celebrity included in the experiment. 

Second, in line with our goals of studying detection ability on images for which there was some ambiguity, we excluded all images where participants' accuracy suggested very little ambiguity. We operationalized this as accuracy above 90\%. 

These exclusion criteria remove all observations on 68 fake images and 4 real images, which represent 14\% of observations from the entire experiment. 

In the human-coded analysis of artifacts discussed in Section~\ref{sec:acc-presence-artifacts}, we apply an additional exclusion criterion to make the coding tractable. Specifically, we exclude all images accurately identified in more than 80\% of observations. This exclusion criterion focuses the analysis on the most challenging images by excluding the most egregious distortions that lead to low photorealism (i.e., high participant accuracy).
%and offers a lower bound on the full extent of the differences between image categories because .

\subsection{Overall Accuracy} \label{sec:acc-general}

In the main study, participants correctly identified AI-generated images and authentic photographs in 76\% and 74\% of observations, respectively. Accuracy varied substantially across images. Prior to implementing our accuracy-based exclusion described above, we found that for AI-generated images, accuracy ranged from 32\% to 99\%. Similarly, accuracy on real photographs ranged from 28\% to 92\%. Figure~\ref{fig:accuracy_real_fake} shows the distribution of accuracy in both AI-generated and real images with example images selected from the top, bottom, and middle deciles of each distribution. At the image level, the mean accuracy for identifying AI-generated and real images was 76\% (95\% CI:[74,77]) and 74\% (95\% CI:[72,76]), respectively. 

Despite our efforts to minimize obvious artifacts, some images - particularly non-portraits - were challenging to generate without noticeable artifacts. As a result, participants achieved nearly 100\% accuracy on a few AI-generated images with obvious features. We present examples of these images in Figure~\ref{fig:three-fake-images}.  %that further motivate the exclusion criteria that we apply to the rest of the results section. 
In contrast to AI-generated images, real photographs rarely contain definitive artifacts and visual cues often seen in AI-generated images, which limits participants from achieving near-perfect accuracy on real photographs.
\begin{figure}[H]
    \centering
    \includegraphics[width=\linewidth]{sections/images/general_accuracy.pdf}
    \caption{Distribution of accuracy scores for real and AI-generated images with example images representing different accuracy levels.}
    \label{fig:accuracy_real_fake}
    \Description{Histograms showing the distribution of accuracy scores for real and AI-generated images, accompanied by example images representing various accuracy levels.}
\end{figure}
\begin{figure}[H]
\centering
\captionsetup{justification=raggedright, singlelinecheck=false, skip=2pt, font=small}
\begin{subfigure}[t]{0.3\linewidth}
    \subcaption{}\vtop{\vskip0pt\hbox{\includegraphics[width=\linewidth]{sections/images/sd_portrait3_040.jpg}}}
\end{subfigure}
\hfill
\begin{subfigure}[t]{0.3\linewidth}
    \subcaption{}\vtop{\vskip0pt\hbox{\includegraphics[width=\linewidth]{sections/images/ff_pg3_010.jpeg}}}
\end{subfigure}
\hfill
\begin{subfigure}[t]{0.3\linewidth}
    \subcaption{}\vtop{\vskip0pt\hbox{\includegraphics[width=\linewidth]{sections/images/ff_fullbody3_007.jpeg}}}
\end{subfigure}
\caption{\mybold{Examples of obviously AI-generated images and their corresponding accuracy.} \normalfont{\textbf{A.} AI-generated portrait with 92\% accuracy. \textbf{B.} AI-generated posed group image with 95\% accuracy. \textbf{C.} AI-generated full-body image with 99\% accuracy.}}
\label{fig:three-fake-images}
\Description{Three examples of obviously AI-generated images with corresponding accuracy scores: A. Portrait with 92\% accuracy, B. Posed group image with 95\% accuracy, C. Full-body image with 99\% accuracy. }
\end{figure}

\subsection{Participant Level Accuracy}\label{sec:indiv-acc}

\begin{figure*}[h]
    \centering
    \captionsetup{justification=raggedright, singlelinecheck=false, skip=2pt, font=small}

    % Top Row - A and B
    \begin{subfigure}[t]{0.4\textwidth}  
        \centering
        \subcaption[]{}  
        \vspace{-3pt}  
        \includegraphics[width=\linewidth]{sections/images/scatter_plot_fake_images.jpg} 
        % \subcaption{}
    \end{subfigure}
    \hspace{0.05\textwidth} 
    \begin{subfigure}[t]{0.38\textwidth}  
        \centering
        \subcaption[]{}  
        \vspace{-3pt}
        \includegraphics[width=\linewidth]{sections/images/scatter_plot_real_images.jpg}
        % \subcaption{}
    \end{subfigure}

    % Bottom Row - C and D
    \begin{subfigure}[t]{0.36\textwidth}  
        \centering
        \subcaption[]{}  
        \vspace{-3pt}
        \includegraphics[width=\linewidth]{sections/images/accuracy_distribution.jpg}
        % \subcaption{}
    \end{subfigure}
    \hspace{0.05\textwidth}  
    \begin{subfigure}[t]{0.36\textwidth}  
        \centering
        \subcaption[]{}  
        \vspace{-3pt}
        \includegraphics[width=\linewidth]{sections/images/learning_curve_with_fake_real_bias.png}
        % \subcaption{}
    \end{subfigure}

    \caption{\textbf{Participant-level accuracy and learning trends.}  
    \normalfont{ \textbf{A.} Scatterplot of participant-level accuracy for AI-generated images. \textbf{B.} Scatterplot of participant-level accuracy for real images. \textbf{C.} Histogram showing the distribution of accuracy across the first ten images seen by participants who viewed at least 10 images. \textbf{D.} Learning curve illustrating accuracy trends and classification biases when detecting AI-generated and real images.}}
    
    \label{fig:combined-participant-accuracy}

    \Description{A composite figure showing participant accuracy trends:  
    A. Scatterplot displaying accuracy levels for detecting AI-generated images, with points representing individual participants.  
    B. Scatterplot for real images, structured similarly to A.  
    C. Histogram showing the distribution of accuracy for participants' first 10 images.  
    D. Learning curve tracking accuracy trends over time, highlighting biases in AI-generated and real image classification.}
\end{figure*}

Given the organic nature of participants' engagement with this experiment, we did not impose restrictions on the number of images a participant saw. Most participants in this study provided responses to at least seven images, but some participants only provided a single response, and one participant provided 502 responses. 

The vast majority of participants (75\%) saw 16 or fewer images. Figure~\ref{fig:combined-participant-accuracy}A and B present the distribution of participant--level accuracy by number of viewed images. 

In order to compare participant performance and avoid issues that arise with differential attrition, we focus on the first ten images seen by participants who saw at least 10 images, which includes 152,050 observations from 15,205 participants. First, we note that 34\% of these participants achieved 90\% accuracy or higher on the first ten images seen. If the AI-generated images were perfectly photorealistic such that the human ability to distinguish is no higher than random guessing, then we would have expected only 1\% of participants to achieve this threshold of accuracy (assuming random guessing at 50\% accuracy, with participants evaluating 10 images each, achieving at least 9 out of 10 correct responses would occur with a probability of approximately 1.07\%, based on the binomial probability distribution). Figure~\ref{fig:combined-participant-accuracy}C shows the distribution of accuracy across the first ten images seen by participants who saw at least 10 images.

In Figure~\ref{fig:combined-participant-accuracy}D, we present accuracy rates by the number of images seen. We find that on average, participants begin the experiment by disproportionally identifying images as fake in 63\% of observations. Notably, this bias is reduced after only a few images.


\subsection{Accuracy by Scene Complexity} \label{sec:acc-scene-complexity}

We find that on average, participants' accuracy increases as scene complexity increases. For example, we find that 16\% of portraits appear in the bottom decile of accuracy scores (representing the highest level of photorealism), whereas only 3\% of AI-generated posed group images appear in the bottom decile. Figure~\ref{fig:pose-complexity} presents the distribution of accuracy for each category, separately for real and AI-generated images. For AI-generated images, the mean accuracy was 72.7\% (95\% CI: [72.4, 72.9]) for portraits, 77.2\% (95\% CI: [76.8, 78.6]) for full body, 76.2\% (95\% CI: [75.8, 76.7]) for posed groups, and 73.4\% (95\% CI: [73, 73.8]) for candid groups. For real images, the accuracy was 71.1\% (95\% CI: [70, 71.4]) for portraits, 75.5\% (95\% CI: [75.1, 75.8]) for full body, 76.7\% (95\% CI: [76.3, 77]) for posed groups, and 74.8\% (95\% CI: [74.4, 75.1]) for candid groups. 

As exemplified in Figure~\ref{fig:pose-complexity}C, we note that portraits, relative to the other levels of scene complexity, typically have less detail, simpler and more standardized poses, more blurred backgrounds, and fewer available cues than full-body or group images. 
\begin{figure}[h]
    \captionsetup{justification=raggedright, singlelinecheck=false, skip=2pt, font=small}
    \centering

    % First subfigure - Reduce space
    \begin{subfigure}[t]{\linewidth}
        \subcaption{}
        \vspace{-12pt}  % Reduce vertical space
        \includegraphics[width=\linewidth]{sections/images/scene_complexity_filtered90_Real.png}
    \end{subfigure}
    
    % Second subfigure - Reduce space
    \begin{subfigure}[t]{\linewidth}
        \subcaption{}
        \vspace{-12pt}  % Reduce vertical space
        \includegraphics[width=\linewidth]{sections/images/scene_complexity_filtered90_AI-generated.png}
    \end{subfigure}
    
    % Third subfigure - Reduce space
    \begin{subfigure}[t]{\linewidth}
    \centering
        \subcaption{}
        \vspace{-12pt}  % Reduce vertical space
        \includegraphics[width=0.9\linewidth]{sections/images/scene_complexity_B.pdf}
    \end{subfigure}
    
    \caption{\textbf{Scene complexity}: Accuracy of real and AI-generated images by scene complexity levels. 
    \normalfont{Beeswarm plots of image-level accuracy for each dimension of scene complexity with bootstrapped 95\% confidence intervals. We exclude images identified with above 90\% accuracy in this analysis. \textbf{A.} Real images \textbf{B.} AI-generated images \textbf{C.} AI-generated images across scene complexities.}}
    
    \label{fig:pose-complexity}
    
    \Description{Beeswarm plots showing the accuracy of real and AI-generated images by pose complexity. Each dot represents an individual image, with error bars indicating the bootstrapped 95\% confidence interval around the mean.}
\end{figure}


\subsection{Accuracy by Presence of Artifacts}\label{sec:acc-presence-artifacts}
\begin{figure*}[h]
\centering
\captionsetup{justification=raggedright, singlelinecheck=false, skip=2pt, font=small}

% Full PDF Image
\includegraphics[width=\linewidth]{sections/images/combined_artifact_trends.pdf}

\caption{\textbf{Accuracy by artifact types and display times} \normalfont{\textbf{A. Mean accuracy for different artifact types.} Distribution of accuracy scores by artifact type
for images with at least one artifact. \textbf{B. Mean accuracy over display time.} Change in mean accuracy across different display time assignments (1 second, 5 seconds, 10 seconds, 20 seconds, and unlimited) with 95\% confidence intervals and bee swarm plots of image accuracy for AI-generated and real images. \textbf{C. Mean accuracy over time for different artifact types.} Change in mean accuracy across different time assignments (1 second, 5 seconds, 10 seconds, 20 seconds, and unlimited) with 95\% confidence intervals and bee swarm plots of image accuracy for
images with anatomical (pink), functional (green), and stylistic
(blue) artifacts. The x–axis shows the display time intervals, and the
y–axis shows accuracy.}}

\label{fig:combined-artifact-trends}

\Description{A composite figure showing accuracy-related analyses:  
(A) A beeswarm plot displaying the distribution of accuracy scores for images containing at least one artifact, categorized by artifact type.  
(B) A line plot illustrating mean accuracy across different display time conditions (1 second, 5 seconds, 10 seconds, 20 seconds, and unlimited), with 95\% confidence intervals. Overlaid bee swarm plots represent individual accuracy scores for AI-generated and real images.  
(C) A line plot showing mean accuracy over time for different artifact types (Anatomical, Functional, and Stylistic). Each artifact type is color-coded (pink for Anatomical, green for Functional, and blue for Stylistic). Bee swarm plots depict individual accuracy scores for images within each artifact category. The x-axis represents display time intervals, and the y-axis represents accuracy.}

\end{figure*}


In order to analyze accuracy by artifact type, we annotated images with diffusion model artifact categories from the taxonomy based on a three-step process. First, four co-authors independently annotated all 218 images with accuracy below 80\%, identifying artifacts and providing detailed explanations for their annotations. Second, each of these annotations was reviewed and edited by two additional co-authors. Third, a fifth co-author reviewed all annotations for consistency. Figures~\ref{fig:combined-varying-artifacts-visibility}A--C and \ref{fig:combined-varying-artifacts-visibility}D--F provide examples of how we annotated images, displaying the identified artifact categories, the reasoning behind their identification, and the associated detection accuracy for each image. During this process, we observed that the three main artifact types---anatomical implausibilities, stylistic artifacts, and functional artifacts---each appeared in nearly a third of the images we annotated. In contrast, violations of physics and sociocultural implausibilities were less common, appearing in only 20 and 12 images, respectively. In light of this distribution of artifacts, Figure~\ref{fig:combined-artifact-trends}A presents the distribution of accuracy scores across images containing at least the three listed artifact types.

Based on our annotations of artifacts in images, we find participants are less accurate on images with functional implausibilities than images with anatomical implausibilities or stylistic artifacts. The mean accuracy on images with at least one functional implausibility, one anatomical implausibility, and one stylistic artifact is 64.1\% (95\% CI: [63.8, 64.5]), 65\% (95\% CI: [64.6, 65.4]), and 64.9\% (95\% CI: [64.5, 65.3]), respectively. While the accuracy on images with functional implausibilities is lower than on images with other implausibilities and artifacts, the mean accuracy scores are similar. However, this similarity in means masks the differences in the distribution of accuracy scores, as shown in Figure~\ref{fig:combined-artifact-trends}A. We find that images with participant accuracy scores in the 40--60\% range (which represent images approaching indistinguishability between real and AI-generated) make up 32.8\% of images annotated with functional implausibilities compared to 21.4\% and 22.4\% of images annotated with anatomical implausibilities and stylistic artifacts, respectively. 


We find that images that we annotated as containing multiple artifacts can still appear photorealistic enough to make detection difficult for most people. Artifacts vary in levels of visibility, as shown in Figure~\ref{fig:combined-varying-artifacts-visibility}A--C. While Figure~\ref{fig:combined-varying-artifacts-visibility}A and C contain stylistic artifacts, they are far more apparent in Figure~\ref{fig:combined-varying-artifacts-visibility}B, which is reflected in its higher detection accuracy. Despite Figure~\ref{fig:combined-varying-artifacts-visibility}A and C containing multiple artifact categories, they had low detection accuracy, suggesting that the presence of multiple artifacts does not necessarily make images easier to identify and that artifact visibility is also a contributing factor. 


The visibility of artifacts is highly variable, and Figure~\ref{fig:combined-varying-artifacts-visibility}D--F present examples highlighting this variability. The anatomical implausibility in the fingers in image Figure~\ref{fig:combined-varying-artifacts-visibility}D is very noticeable, whereas the functional implausibilities in the tennis racket and shirt design of Figure~\ref{fig:combined-varying-artifacts-visibility}F are more subtle. The corresponding accuracy scores for these images--- 62\% for Figure~\ref{fig:combined-varying-artifacts-visibility}E and 54\% for Figure~\ref{fig:combined-varying-artifacts-visibility}F —reinforce the observation that anatomical artifacts tend to be more easily detected, while functional implausibilities often require closer attention and familiarity with depicted objects. The stylistic artifacts in the cinematization of Figure~\ref{fig:combined-varying-artifacts-visibility}E and plastic-like skin texture fall in between, further showing the spectrum of detectability across different artifact categories and visibility. 


\begin{figure*}[h!]
\centering
\captionsetup{justification=raggedright, singlelinecheck=false, skip=2pt, font=small}

% First Row of Images
\begin{subfigure}[t]{0.27\linewidth}
\centering
    \subcaption{}
    \includegraphics[width=\linewidth]{sections/images/8059c316907c586bdf33ad3cb9ca3f95.jpeg}
\end{subfigure}
\hspace{1cm}
\begin{subfigure}[t]{0.27\linewidth}
\centering
    \subcaption{}
    \includegraphics[width=\linewidth]{sections/images/616ba73f50088eb13244a807076248f7.jpeg}
\end{subfigure}
\hspace{1cm}
\begin{subfigure}[t]{0.27\linewidth}
\centering
    \subcaption{}
    \includegraphics[width=\linewidth]{sections/images/2c4c0b171577884f5c0991cacb5c5ebc.jpeg}
\end{subfigure}

\vskip 5mm % Adds vertical spacing between rows

% Second Row of Images
\begin{subfigure}[t]{0.27\linewidth}
\centering
    \subcaption{}
    \includegraphics[width=\linewidth]{sections/images/ff_pg3_009.jpeg}
\end{subfigure}
\hspace{1cm}
\begin{subfigure}[t]{0.27\linewidth}
\centering
    \subcaption{}
    \includegraphics[width=\linewidth]{sections/images/mj_portrait3_010.jpeg}
\end{subfigure}
\hspace{1cm}
\begin{subfigure}[t]{0.27\linewidth}
\centering
    \subcaption{}
    \includegraphics[width=\linewidth]{sections/images/ff_portrait3_004.jpeg}
\end{subfigure}

\caption{\textbf{Examples of images with varying artifact visibility.}  
\normalfont{\textbf{Top row (A--C):} Example images showcasing stylistic and functional artifacts with varying visibility.  
\textbf{A.} A subtle stylistic artifact in the soft and wispy textures of the woman's hair and a minor functional implausibility in the atypical design of her shirt collar (Accuracy: 47\%). \textbf{B.} An obvious stylistic artifact due to the overall cinematization of the image (Accuracy: 73\%). \textbf{C.} A combination of multiple artifacts, including anatomical implausibilities in the woman's hand, functional implausibilities in the table shape and wall panels, and a stylistic artifact in the soft texture of the woman's face (Accuracy: 38\%). \textbf{Bottom row (D--F):} Images with anatomical, stylistic, and functional artifacts of varying visibility. \textbf{D.} Anatomical implausibilities in the fingers of the three students (Accuracy: 84\%). \textbf{E.} A stylistic artifact in the cinematized look and plastic-like texture of the woman's skin (Accuracy: 62\%). \textbf{F.} No obvious anatomical or stylistic artifacts, but closer inspection reveals functional implausibilities: the tennis racket is asymmetrical, its strings are not taut, and the shirt has irregularly shaped designs with glitch-like inconsistencies (Accuracy: 54\%).}}

\label{fig:combined-varying-artifacts-visibility}

\Description{A composite figure showing six images with varying visibility of AI-generated artifacts.  
(A--C) The first row highlights stylistic and functional artifacts, including wispy hair, cinematized lighting, and a distorted table.  
(D--F) The second row focuses on anatomical, stylistic, and functional artifacts, including distorted fingers, plastic-like textures, and inconsistencies in objects like a tennis racket.}
\end{figure*}

\subsection{Accuracy by Randomized Display Time}\label{sec:acc-time}
\begin{figure*}[h]
\centering
\captionsetup{justification=raggedright, singlelinecheck=false, skip=2pt, font=small}
\begin{subfigure}[t]{0.27\linewidth}
\centering
    \subcaption{}
    \includegraphics[width=\linewidth]{sections/images/bbbfb2a12cd66783ce7e4015ec0084b9.jpg}
\end{subfigure}
\hspace{1cm}
\begin{subfigure}[t]{0.27\linewidth}
\centering
  \subcaption{}
    \includegraphics[width=\linewidth]{sections/images/5204545de13342cbefdc0e9022d821d2.jpg}
\end{subfigure}
\hspace{1cm}
\begin{subfigure}[t]{0.27\linewidth}
\centering
  \subcaption{}
    \includegraphics[width=\linewidth]{sections/images/ccc04b661d52c055a44fc01718c6a2bc.jpg}
\end{subfigure}
\caption{\textbf{Exemplar AI-generated images for which a closer look improves accuracy.} \normalfont{\textbf{A.} Accuracy: 38\% at 1 second display time to 65\% at 20 second display time. \textbf{B.} Accuracy: 44\% at 1 second display time to 82\% at 20 second display time. \textbf{C.} Accuracy: 27\% at 1 second display time to 70\% at 20 second display time.}}
\label{fig:displaytime}
\Description{Three images showing AI-generated images for which a closer look improves accuracy. (A) Image of woman generated by Stable Diffusion: Accuracy is 38\% at 1 second display time and it improves to 65\% at 20 second display time with  (B) Image of people generated by Stable Diffusion: Accuracy is 44\% at 1 second display time and it improves to 82\% at 20 second display time with (C) Image of people generated by Stable Diffusion: Accuracy is 27\% at 1 second display time and it improves to 70\% at 20 second display time with.}
\end{figure*}

By randomizing the display time of images in this experiment, our results support evaluating how viewing duration influences participants' accuracy. We find that longer viewing times improve performance. With just 1 second of display time, participants are  72\% accurate (95\% CI=[71.6, 72.5], 95\% CI=[71.3, 72.2]) on AI-generated and real images, respectively. With 5 seconds of display time, accuracy increases to 77\% (95\% CI=[77.0, 77.8], 95\% CI=[76.6, 77.4]) for both AI-generated and real images, respectively. While accuracy on real images appears to plateau by 5 seconds of display time, accuracy on AI-generated images increases up to 80\% (95\% CI=[79.6, 80.4]) at 10 seconds and 82\% (95\% CI=[81.2, 81.9]) at 20 seconds. Figure~\ref{fig:combined-artifact-trends}B presents the distribution of accuracy scores across display time conditions. Across the observations where display time was randomized, we find that the proportion of AI-generated images that are identified below random chance decreases from 43\% when participants only have 1 second to view the image to 30\%, 25\%, 17\%, and 17\% when participants have 5, 10, 20 seconds, and unlimited time to view the image.

In some images, AI artifacts can be noticed with a quick glance, but for others, careful attention to detail is necessary to spot the artifact. Figure~\ref{fig:displaytime} presents three images that require careful attention, as evidenced by the fact that most participants mark as real when they are limited to seeing the image for a second but
fake once they take into account the details of the scene.

Accuracy across all artifact types improved with increased display time. As shown in Figure~\ref{fig:combined-artifact-trends}C, participants showed higher accuracy when images were displayed for longer time (anatomical artifacts: 63\% at 5 seconds vs. 59\% at 1 second; stylistic artifacts: 63\% at 5 seconds vs. 60\% at 1 second; functional artifacts: 60\% at 5 seconds vs. 55\% at 1 second). For all artifacts, there is a significant improvement in detection accuracy when increasing display time from 5 seconds to unlimited. 

In Figure~\ref{fig:combined-artifact-trends}C, we observe that participants improved the most in identifying functional artifacts, with an 18\% improvement from 1 second to unlimited viewing time. In comparison, anatomical and stylistic artifacts showed smaller improvements of 11\% each over the same time interval. Unlike anatomical and stylistic implausibilities that can be identified at first glance, functional artifacts often require a closer look and familiarity with the elements in the image as they often appear in parts of the image that are not the main subject. 


\subsection{Qualitative Analysis of Participant Comments}\label{sec:qualitative-analysis}

We collected 34,675 comments from participants who filled out the optional text input box asking participants: ``If you think this is AI-generated, please explain why.'' In order to identify themes from these 34,675 comments, we prompted GPT-3.5 Turbo to identify 10 main themes across these comments. GPT-3.5 Turbo responded with the following ten themes, which we manually reviewed and refined to mitigate the ambiguities and generalization typical of large language models \cite{stephan2024rlvflearningverbalfeedback}: (1) Image quality focusing on the overall appearance, smoothness, and sometimes unrealistic perfection of image elements; (2) Facial and anatomical inconsistencies where participants pointed to irregularities in eyes, mouths, noses, skin texture, expressions, and general human anatomy; (3) Anatomical and functional anomalies such as deformities, misplaced body parts, and irregularities in objects or environments; (4) Lighting and environmental inconsistencies including unnatural lighting, inconsistent shadows, and reflections; (5) Digital manipulation indicators suggesting suspicions of AI-generation or digital alteration; (6) Biometric discrepancies particularly unnatural or imperfect body parts like hands and fingers; (7) Uncanny valley perceptions where images almost looked human but had subtle unnatural features that caused discomfort; (8) Contextual incongruities such as unrealistic scenarios and mismatched social elements; (9) Physical anomalies highlighting illogical physical interactions within the images; and (10) holistic authenticity assessment making overall judgments based on a combination of multiple cues and inconsistencies. Based on these ten main themes, we prompted GPT-3.5 to label each comment with one of the ten themes. Figure~\ref{fig:comments-all} illustrates examples of participant comments for four images and how they were categorized into themes. Figure~\ref{fig:themes} displays the distribution of themes across the comments and the related concept from our taxonomy in parentheses.

Based on GPT-3.5 Turbo, we find that 61\% of participants' comments mentioned relying on anatomical implausibilities. The next most common concept referred to is stylistic artifacts, which is mentioned in 30\% of comments. Participants mentioned functional implausibilities in 21\% of comments, violations of physics in 15\% of comments, and sociocultural implausibilities in only 4\% of comments. 

Based on the authors' annotations of artifacts, we find functional implausibilities to be the most prevalent, appearing in 58.7\% of images, followed by anatomical implausibilities in 51.4\% and stylistic artifacts in 39.0\% of images. We identify violations of physics and sociocultural implausibilities in only 9.17\% and 5.50\% of images, respectively. 
\begin{figure}[H]
\centering
\captionsetup{justification=raggedright, singlelinecheck=false, skip=2pt}
\begin{subfigure}[t]{0.9\linewidth}
    % \subcaption{}
    \includegraphics[width=\linewidth]{sections/images/theme_distribution_single_column.jpg}
\end{subfigure}
\caption{\mybold{Distribution of themes identified in participant comments.}}
\label{fig:themes}
\Description{A horizontal bar chart showing the distribution of themes identified in participant comments explaining their reasoning for AI image detection. The themes include Image Quality, Facial and Anatomical Inconsistencies, Anatomical and Functional Anomalies, Lighting and Environmental Inconsistencies, Digital Manipulation Indicators, Biometric Discrepancies, Uncanny Valley Perceptions, Contextual Incongruities, Physical Anomalies, and Holistic Authenticity Assessment.}
\end{figure}
While functional artifacts were the most prevalent in human researcher annotated images, they were less frequently mentioned in participant comments annotated by GPT--3.5. Conversely, anatomical artifacts were emphasized more in participant comments than in their prevalence in annotated images. 

\begin{figure}[htb]
\centering
\captionsetup{justification=raggedright, singlelinecheck=false, skip=2pt, font=small}
\begin{subfigure}[t]{0.22\textwidth}
    \subcaption{}\vtop{\vskip0pt\hbox{\includegraphics[width=\linewidth]{sections/images/mj_ng3_007.jpg}}}
\end{subfigure}
\hfill
\begin{subfigure}[t]{0.22\textwidth}
    \subcaption{}\vtop{\vskip0pt\hbox{\includegraphics[width=\linewidth]{sections/images/mj_portrait3_001.jpg}}}
\end{subfigure}
\hfill
\begin{subfigure}[t]{0.22\textwidth}
    \subcaption{}\vtop{\vskip0pt\hbox{\includegraphics[width=\linewidth]{sections/images/mj_fullbody3_003.jpg}}}
\end{subfigure}
% \hfill
% \begin{subfigure}[t]{0.19\textwidth}\subcaption{}\vtop{\vskip0pt\hbox{\includegraphics[width=\linewidth]{sections/images/fullbody3_023.jpeg}}}
% \end{subfigure}
\hfill
\begin{subfigure}[t]{0.22\textwidth}
    \subcaption{}\vtop{\vskip0pt\hbox{\includegraphics[width=\linewidth]{sections/images/mj_pg3_002.jpg}}}
\end{subfigure}
\caption{\mybold{Examples of participant comments mapped to themes.} \normalfont{\textbf{A.} ``Cosmetic style out of character with vintage setting": Contextual Incongruities. \textbf{B.} ``Skin too smooth, depth of field shallow.": Image Quality, Lighting Inconsistencies. \textbf{C.} ``If this is not AI then it is a staged photograph like a movie set because of the lighting and he is an actor.": Lighting inconsistencies, Contextual Incongruities.
\textbf{D.} ``Group looks pasted onto background.": Digital Manipulation Indicators.}}
\label{fig:comments-all}
\Description{Four images with participant comments mapped to themes.(A) AI-generated candid image with a comment on cosmetic style being out of character with a vintage setting.(B) AI image with smooth skin, with a comment on skin being too smooth and shallow depth of field.(C) AI-generated full body shot of a man, with a comment suggesting it resembles a staged photograph due to lighting.
(D) AI image of a group, with a comment on the group looking pasted onto the background.}
\end{figure}

\subsection{Accuracy by Models} \label{sec:acc-model}

In the process of generating the images for this experiment's stimuli set, we noticed that Midjourney, Firefly, and Stable Diffusion have different capabilities and limitations. For example, we noticed that Midjourney often produced images with persistent stylistic artifacts that were challenging to eliminate. Firefly, on the other hand, frequently exhibited a tendency toward synthetic emotional expressions, with subjects often appearing unnaturally and overly cheerful, necessitating multiple iterations to produce more realistic results. Stable Diffusion struggled significantly with generating group images, often introducing artifacts such as anatomical inconsistencies. In light of the limitations to generate non-portrait images with Stable Diffusion, 75\% of the Stable Diffusion-generated stimuli in this experiment were portraits. On the other hand, 30\% of Midjourney and Firefly-generated images in this experiment depict portraits. In order to compare the three models fairly, we focus our comparison on portrait images. Figure~\ref{fig:models} presents accuracy shown on portraits by each of the three models and reveals that participants' mean accuracy on Midjourney, Stable Diffusion, and Firefly were  76\% (95\% CI: [75.2, 75.8]), 74\% (95\% CI: [73.9, 74.8]), and 73\% (95\% CI: [72.7, 73.3]), respectively. 


\begin{figure}[H]
    \centering
    \includegraphics[width=0.8\linewidth]{sections/images/model_accuracy_combined.jpg} 
    \caption{\textbf{Accuracy across generative AI models} \normalfont{Each point represents an image. The black dots and error bars show the mean accuracy and 95\% bootstrapped confidence intervals for each model}}
    \label{fig:models}
    \Description{Bee swarm chart showing accuracy across different generative AI models. The chart compares the accuracy rates for identifying AI-generated content among various models along with bootstrapped 95\% confidence intervals}
\end{figure}


\subsection{Accuracy on Human Curated Images vs. Uncurated Images} \label{sec:human-curation}
\begin{figure*}[h]
\centering
\captionsetup{justification=raggedright, singlelinecheck=false, skip=2pt, font=small}
\begin{subfigure}[t]{0.24\linewidth}
    \subcaption{}\vtop{\vskip0pt\hbox{\includegraphics[width=\linewidth]{sections/images/sd_portrait3_003.jpg}}}
\end{subfigure}
\hfill
\begin{subfigure}[t]{0.24\linewidth}
    \subcaption{}\vtop{\vskip0pt\hbox{\includegraphics[width=\linewidth]{sections/images/0bce6c35c24ec5ce8ae8ad5bb4f67d59_r4.jpg}}}
\end{subfigure}
\hfill
\begin{subfigure}[t]{0.24\linewidth}
    \subcaption{}\vtop{\vskip0pt\hbox{\includegraphics[width=\linewidth]{sections/images/0bce6c35c24ec5ce8ae8ad5bb4f67d59_r11.jpg}}}
\end{subfigure}
\hfill
\begin{subfigure}[t]{0.24\linewidth}
    \subcaption{}\vtop{\vskip0pt\hbox{\includegraphics[width=\linewidth]{sections/images/0bce6c35c24ec5ce8ae8ad5bb4f67d59_r2.jpg}}}
\end{subfigure}
\caption{\mybold{Re-generated images from the same prompt.} \normalfont{ \textbf{A.} Stage 1 image generated by Stable Diffusion and curated by our team (37\% accuracy) \textbf{B.} Most photorealistic of 12 prompt-matched image generations by Stable Diffusion (42\% accuracy) \textbf{C.} Median photorealistic of 12 prompt-matched image generations by Stable Diffusion (59\% accuracy) \textbf{D.} Least photorealistic of 12 prompt-matched image generations by Stable Diffusion(83\% accuracy)}}
\label{fig:regeneration}
\Description{Four images showing re-generated outputs from the same prompt.\textbf{A.} Stage 1 image generated by Stable Diffusion and curated by our team (37\% accuracy) \textbf{B.} Most photorealistic of 12 prompt-matched image generations by Stable Diffusion (42\% accuracy) \textbf{C.} Median photorealistic of 12 prompt-matched image generations by Stable Diffusion (59\% accuracy) \textbf{D.} Least photorealistic of 12 prompt-matched image generations by Stable Diffusion(83\% accuracy)}
\end{figure*}

\begin{figure*}[h]
\centering
\captionsetup{justification=raggedright, singlelinecheck=false, skip=2pt}
\begin{subfigure}[t]{0.48\textwidth}  
\subcaption{}
\vtop{\vskip0pt\hbox{\includegraphics[width=\linewidth]{sections/images/curation_value_add_min.jpg}}}
\end{subfigure}
\hfill
\begin{subfigure}[t]{0.48\textwidth}
\subcaption{}
\vtop{\vskip0pt\hbox{\includegraphics[width=\linewidth]{sections/images/curation_value_add_mean.jpg}}}
\end{subfigure}
\caption{\mybold{Comparing accuracy scores on curated images and uncurated prompt-matched images.} \normalfont{\textbf{A.} Scatterplot showing human detection accuracy of the original curated image compared to human detection accuracy of its most photorealistic regeneration out of 11 to 24 prompt-matched images labeled as re-generations. \textbf{B.} Scatterplot showing human detection accuracy of the original curated image compared to human detection accuracy of its mean photorealistic regeneration out of 11 to 24 re-generations.}}
\label{fig:curation-value}
\Description{Two scatterplots showing curation value analysis.(A) Minimum curation value added.(B) Mean curation value added. Each chart illustrates the impact of curation on the overall value added to the dataset.}
\end{figure*}

Generating photorealistic AI-generated images involves three key ingredients: the diffusion model, the prompt, and human curation. In this section, we examine how human curation of diffusion model-generated images influences the aggregate accuracy scores of human participants. In order to show this influence, we compare diffusion model images from the main experiment, which were curated by our research team, with multiple diffusion model images generated from the same prompt as the curated images. This comparison reveals the increase in photorealism (as measured by the decrease in participants' accuracy) on the curated images relative to the prompt-matched images.

In this second phase of the experiment, we randomly sampled 39 AI-generated images from the main stimuli set, where the sample was stratified on 10 percentage point wide bins on human detection accuracy. For each of these 39 images, we generated at least 11 prompt-matched images using Midjourney, Firefly, and the same pipeline in Stable Diffusion. Figure~\ref{fig:regeneration} displays a Stable Diffusion-generated image from our original stimuli set and three of the twelve generations using the same prompt. We generated 482 total additional images, with at least 11 per prompt. These 482 images were included alongside the 149 real images on the experiment website.

In Figure~\ref{fig:curation-value}, we present scatterplots comparing human detection accuracy on the initial curated images and the best prompt-matched images in panel A, and mean prompt-matched images in panel B. We find the human-curated images have lower human detection accuracy than the best regenerated image in 18 of 39 instances and the mean re-generated image in 35 of 39 instances. In total, the human-curated images were perceived to be more photorealistic than 408 of the 482 (84\%) uncurated prompt-matched images. Specifically, we find the marginal value added by human curation for images that were initially detected in the range of 30\% to 50\% is 31 percentage points, 50 to 60\% is 23 percentage points, 60-70\% is 11 percentage points, 70-80\% is 8 percentage points, and 80+\% is 4 percentage points. Across the stimuli selected from Midjourney, Firefly, and Stable Diffusion, the marginal value of human curation is 7.8, 19.0, and 16.9 percentage points, respectively. 


The two panels in Figure~\ref{fig:curation-value} illustrate the positive correlation between accuracy on the human-curated image and accuracy on the regeneration. This reveals how the prompt influences photorealism. The Pearson Correlation Coefficient between accuracy on curated images and their best, mean, and worst re-generations are .58, .53, and .32, respectively. This positive correlation suggests the choice of a prompt plays a significant role in the photorealism of an image. Figure~\ref{fig:goodandbadprompt} displays two original curated images where A is generated by a prompt in which re-generations achieved low human detection accuracy (a `good' prompt), and B is generated by a prompt in which re-generations achieved a high human detection accuracy (a `bad' prompt). Prompts that consistently generate easily detectable images often have elements that are difficult to generate and result in artifacts. The prompt ``Persian woman astronaut in astronaut clothes, family photo with husband and two toddlers, high resolution, realistic" for Figure~\ref{fig:goodandbadprompt}B  generates a posed group image that tends to be easy to detect. On the other hand, the prompt ``American woman faculty portrait, not a close-up, blond" for Figure~\ref{fig:goodandbadprompt}A generates a portrait image that tends to be perceived as more photorealistic.
We analyze the experimental results in this section. We highlight two interesting findings: Test-time scaling in Multi-Agent system (Section \ref{sec:result_test_time_scaling}), and modality-tailored critiques enhance the self-correction ability (Section \ref{sec:result_multimodal_self_critique}). Additionally, we discuss the advantage of \model{} (Section \ref{sec:why_metal}), and the benefit of agentic design (Section \ref{sec:result_agentic_vs_modular})


\subsection{Test-Time Scaling} 
\label{sec:result_test_time_scaling}

We investigate the relationship between the test-time computational budget and model performance. As illustrated in Figure~\ref{fig:acc_by_iter}, our analysis reveals an interesting trend:  increasing the logarithm of the computational budget leads to continuous performance improvements. This near-linear relationship indicates the test-time scaling phenomenon, demonstrating that allowing more iterations during inference could potentially enhance performance.

One potential reason for this phenomenon is the strong self-improvement capability of \model{}. Our framework is designed so that specialized agents iteratively collaborate, allowing each agent to refine its output based on feedback from others. With each iteration, errors are corrected and insights from different modalities are integrated, leading to incremental performance gains. This continual refinement process leverages the strengths of individual agents, resulting in the self-improvement capability that drives the observed performance enhancements as computational resources increase. 

Due to limited resources, we have not extended the experiment range further. However, the observed scaling implies that the framework can benefit from more iterations of collaborative self-improvement. We leave a more comprehensive exploration of this potential to the future work.

% Figure~\ref{fig:acc_by_iter} reveals a near-linear relationship between performance and the logarithm of the computational budget. \kw{one question might be why don't we further increase the budget and stop at 4096 tokens.} As more computations are performed, performance improves continuously, suggesting that prolonged multi-agent collaboration could eventually achieve near-perfect results. This finding confirms that the test-time scaling law applies to the chart generation task with the multi-agent framework. \kw{I would not use "confirm" as it might be too strong. We can say we observe this trend in our experiments. }


\subsection{Modality-Tailored Critiques} \label{sec:result_multimodal_self_critique}

\begin{figure} [tbp]
    \centering
    \includegraphics[width=1\linewidth]{figs/average_improvement_by_difficulty.pdf}
    \caption{Performance gain after 5 compute recurrences of \model{} over different  difficulty.}
    \vspace{-0.2in}
    \label{fig:improv_by_diff}
\end{figure}

From the ablation study result shown in Table \ref{tbl:ablation_study}, we observed that separating visual and code critiques enhances the model's self-correction capabilities. In contrast, \model{}$_S$ struggles to effectively self-improve in the chart-to-code generation task.

We identify two potential reasons for this observation. First, combining both visual and code inputs results in an extended context that can overwhelm the model, leading to information loss. This dilution makes it difficult to capture key details from each modality, resulting in less accurate critiques and a reduction in overall self-correction effectiveness. Second, the self-critique process for chart generation involves distinct requirements: visual data demands spatial understanding, color analysis, and fine detail recognition, while code data requires strict adherence to syntax and logical consistency. A unified critique approach is ill-suited to address these differing needs. Without modality-specific feedback, the model struggles to detect and correct errors unique to each data type.

These findings suggest that self-correction in the multimodal context can be enhanced by leveraging tailored critique strategies for each modality.

\subsection{Why \model{}} 
\label{sec:why_metal}

We believe \model{} provides three advantages. 
First, by assigning specialized tasks to individual agents, the system effectively reduces error propagation. During inference, each agent evaluates whether to take action based on the available information and insights from other agents. This process enables each agent to serve as a safeguard, detecting and correcting mistakes before they escalate. 

Second, the modular design of \model{} enables easy modification and adaptation. For instance, one can integrate different base models tailored for specific tasks—such as employing a critique-trained model for critique agents and a generation-trained model for generation agents—to maximize overall performance. 

Third, \model{} is robust with the strong base model. Figure~\ref{fig:improv_by_diff} compares the performance of \model{} to that of Direct Prompting over five iterations across varying chart difficulty levels. \model{} with the \gpt base model achieved consistent improvements regardless of difficulty. When using \llama as the base model, the performance gains tend to diminish with increasing reference chart complexity, but the improvements remain substantial. This drop might be due to the limited critique capabilities of the \llama base model. Nonetheless, the flexibility of \model{} to replace the base model for different agents allows us to tailor the system optimally—using, for example, a critique-optimized model for critique agents and a generation-focused model for generation agents—to maximize overall performance.

\begin{figure*}[ht]
    \centering
    \includegraphics[width=1\linewidth]{figs/case_study.pdf}
    \caption{Case study of \model{}'s progressive refinement from initial generation to perfect. Starting from Round 0's initial generation (60\% color score , 84\% text score), the system iteratively improves the output. In Round 1, the system identifies and corrects Y-axis scale issues and missing annotations, achieving 100\% text score. Round 2 refines the color representations of distributions, achieving perfect F1 score across all metrics.}
    \label{fig:case_study}
    \vspace{-0.08in}
\end{figure*}

\subsection{Multi-Agent System vs. Modular System}
\label{sec:result_agentic_vs_modular}

We further investigate the impact of agentic behavior of \model{} on final performance. We think self-decision-making and code execution abilities are key features that distinguish the multi-agent system from a modular system. We implement a self-revision modular system without these two key abilities,  and conduct an additional ablation study on a subset of 50 data points to examine the impact of these agentic behaviors on final performance.

The results show that, compared to \model{}, there is a 4.51\% reduction in average performance gain over direct prompting. The absence of decision-making and code execution abilities in the modular system hinders its capacity to refine generated charts effectively. Specifically, the inability to execute code for chart rendering significantly diminishes the quality of the critique, and the absence of self-decision-making ability potentially leads to error propagation that further negatively impacts the self-correction process.

This comparison underscores the critical role of the agentic approach.


\section{Conclusion}\label{sec:con}

Our work contributes empirical insights on the photorealism of AI-generated images and a taxonomy of artifacts commonly found in AI-generated images, organized into five categories: anatomical implausibilities, stylistic artifacts, functional implausibilities, violations of physics, and sociocultural implausibilities. We find that the photorealism of AI-generated images depends on the scene complexity of the image, the kind of artifacts and implausibilities, if any, detectable in an image, the duration of visual attention to an image, and the quality of human effort to select appropriate prompts and curate images. A question such as ``How photorealistic are state-of-the-art diffusion models'' may sound simple, but the answer is more complex and depends on many details, including what images are generated and selected, how photorealism is measured, what real images are included in the experiment, and how much time, skill, and effort a human participant has and willing to offer. This paper offers an initial exploration into how we can address this question and develops a practical taxonomy that offers scaffolding for building AI--literacy interventions to help people navigate the capabilities and limitations of diffusion models and whether an image is AI-generated or not. 

\begin{acks}

This material is based upon work supported by Robert Pozen, and in part with funding from the Department of Defense (DoD). Any opinions, findings, conclusions, or recommendations expressed in this material are those of the authors and do not necessarily reflect the views of the DoD or any agency or entity of the United States Government. We thank Will Thompson from Kellogg Research Support for performing a replication check.
\end{acks}
% 
\begin{figure*}[t!]
\centering
  \includegraphics[width=\linewidth]{figs/shapelib_qual_libs_v1.pdf}
   \caption{Examples of functions from the shape libraries discovered by \methodname. For each category, we show a function implementation, and a few example applications of the function. For each application, we show the full output shape, with parts corresponding to the function marked in the same color as the function name, and the function parameters. We can see that function applications are well-aligned with part semantics and that each function typically requires only a small set of parameters to represent a rich variety of part shapes.} 
  \label{fig:big_qual_libs}
\end{figure*}

\begin{figure*}[t!]
\centering
  \includegraphics[width=.85\linewidth]{figs/shapelib_qual_apps_v1.pdf}
   \caption{\methodname's abstraction functions provide a semantically aligned and interpretable interface that support downstream applications: text-based LLM editing and visual program induction from unstructured geometry. } 
  \label{fig:big_qual_apps}
\end{figure*}


\section{Additional Method Details}

\subsection{Objective Function }

When searching for programs that explain shapes, we need an objective function to guide the search. 
We take inspiration from prior approaches such as ~\cite{jones2023shapecoder}, and formulated an objective function as a weighted average of two terms.
One of these terms counts up the number of degrees of freedom in the program representation, for simplicity we treat every token in the program as a degree of freedom with the same weight (1.).
Another term ensures that the produced geometry does not deviate too far from the target structure. 
We calculate the geometric error (more on this in the next paragraph), and add that into our objective function with a weight of 10.

The geometric error function we use takes in two sets of unordered primitives. 
For every pair of primitives from the predicted to target set, we calculate the maximum minimum distance between any two corners from one primitive to the other. 
We then use a matching algorithm to assign a stable pairing between the two sets.
If any of the distances is above a threshold (0.25, where shapes are normalized to lie within the unit sphere), then we say that there is infinite geometric error.
Otherwise, the geometric error is an average of the maximum minimum corner distance (MMCD), calculated according to the best match.

\subsection{Network Design}

We implement all of our networks in PyTorch~\cite{paszke2017automatic}. 
All of our experiments are run on NVIDIA GeForce RTX 3090 graphic cards with 24GB of VRAM.
We use the Adam optimizer~\cite{Kingma2014AdamAM} with a learning rate of 1e-4.
We implement our recognition network as a Transformer decoder. 
Our network has 4 layers, 4 heads, model dim of 256, and a full feature dim of 1024.

This network has full attention over the conditioning information: each primitive in the input shape is quantized and treated as a discrete token.
We order the primitives according to their x-y-z positions, as we do not know how they should be ordered otherwise.
Programs are similarly tokenized, and our network is trained through teacher forcing. 
We use learned positional encodings, these cap the maximum sequence lengths and primitive amounts 
our network can reason over: 20 primitives and programs of up to length 64. 
We train with a batch size of 128.
For point cloud inputs, we replace the primitive token encodings with an embedding produced by a PointNet++~\cite{qi2017pointnet++} network. 
For voxel inputs, we replace the primitive token encodings with an embedding produced by a 3D-CNN. 
We train our networks for between 4-12 hours, depending on the category and task.

\subsection{Synthetic Data Sampler}

We perform two rounds of automated feedback for each `sample\_shape' function generated by the \textit{o1} LLM model.
This iterative approach aims to refine the sampler's outputs by addressing discrepancies and improving alignment with respect to seed set patterns. 
In each round of feedback, we evaluate the function by sampling a diverse set of shapes and assessing various aspects of its behavior. 
We examine whether all functions in the library were used, whether all parameter types were employed, and whether all output structures described in the function's documentation were produced. 
These checks are performed automatically.
Additionally, we analyze the structures generated by the sampled functions and determine their similarity to those observed during the validation stage. 
If significant deviations are detected, measured in the parameter space of each function, the sampler is instructed to update its logic to produce outputs closer to the expected structures.

\section{Additional Experimental Details  }

\subsection{Cost and Timing}

We provide detailed estimates for how expensive it is (from a time and API monetary expense perspective) to use our system to discover libraries of shape abstraction functions.
To produce 20 shape descriptions from images using gpt-4o: 10 cents and 1-2 minutes.
To create library interfaces from textual descriptions with o1mini: 25 cents, 2-4 minutes.
To propose function applications over (20) shapes with (1) o1mini call and (4) gpt-4o calls: \$2-3 
and 15-25 minutes.
To propose (4) implementations for each function with o1mini: \$2-4 and 15-30 minutes.
To propose a single program sampler with o1: 50 cents and 1 minute. In total, this amounts to \$5-8 and 30 minutes to 1 hour.

Notice that by default we use o1mini, but sometimes deviate based on our developmental experience. 
Making function applications without knowing function implementations is a `guess-based' exercise, so we are fine with the increased error rate that 4o produces in this step.
For the most complex tasks, like implementing a synthetic data sampler, we turn to o1 as we are able to provide enough task guidance and directives to make use of its `reasoning' capabilities.

\subsection{Data}

Collections of example shapes in the seed set are chosen by an expert user who has a design intent in mind (they also express this intent in natural language in the function descriptions).
Specifically, we have the user select 20 partNet shapes and put them in a list, and then we can automatically produce the rest of the structured data from the partNet annotations. 
Currently, we manually render associated ShapeNet meshes in MeshLab~\cite{meshlab}, but this could be easily relaxed for ease of use.

After we have selected these two shapes, we create separate `training' and 'validation' sets of shapes by randomly splitting up Partnet object instances.
We run all experiments over validation shapes, unless otherwise stated, and use the training shapes to get paired data for the visual program induction step that maps from unstructured geometry to a shape abstraction program.
The size of these train/val sets is 4000/1000 for chairs, 1216/400 for storage, 4000/1000 for tables, 434/400 for faucet, and 2625/656 for lamps.


\subsection{LLM-Direct Baseline}

The LLM-direct is an ablated version of our method that relies on only the prior of the LLM and the design intent of the expert user in the form of function descriptions.
We compare against it to validate the need for using the seed set of shapes alongside the natural language specification. 

This baseline, is equivalent to our method modulo a few critical changes. The interface creation step is exactly the same. 
After this step though, it immediately implements each function, without using any input/output guidance about how this function should constructed. 
As it has no seed set, it assumes that the LLM has perfectly implemented each function, and next advances to the synthetic sampler design stage where it prompts the LLM to produce a `sample\_shape' function from its constructed library.
Then, like the full~\methodname system, we can train a recognition network on data produced by this random sampling procedure.


\subsection{ShapeCoder}

In our comparisons against ShapeCoder we use the officially released implementation. 
The only change we make is removing the rotation operation from the base ShapeCoder language,
as we focus on structures of axis-aligned primitives in our experiments.
We develop ShapeCoder's library of abstraction over the same seed set of 20 shapes, which is much smaller than the large datasets used in the original ShapeCoder system (400 shapes).
Nevertheless, we find that ShapeCoder can generalize (in terms of compression, at least) fairly well even from these 20 shapes.

We experiment with discovering ShapeCoder libraries over a larger seed set of 400 shapes, and find that compression improves slightly on validation shapes, but not by a huge margin (Obj goes from 52.1 to 46.1, while the average library size grows from 19 to 24). 
Despite learning this library over a large collection of shapes, we still observe that this `ShapeCoder-400' variant does not find more semantically aligned function applications over validation structures.
In fact, its semantic entropy performance worsens (chair: 1.67 to 1.84, table: 1.578 to 2.16, storage: 2.07 to 2.08, lamp: 1.7 to 1.9, faucet: 2.1 to 2.3)
We view this result as lending our framing additional support: 
compression alone (even over a large dataset) is not enough to develop good shape abstraction libraries, top-down semantic guidance is also required.

%%
%% The acknowledgments section is defined using the "acks" environment
%% (and NOT an unnumbered section). This ensures the proper
%% identification of the section in the article metadata, and the
%% consistent spelling of the heading.
% \begin{acks}

% \end{acks}

%%
%% The next two lines define the bibliography style to be used, and
%% the bibliography file.
\bibliographystyle{ACM-Reference-Format}
\bibliography{bibliography}


%%
%% If your work has an appendix, this is the place to put it.
\appendix
\appendix
\beginsupplement
\clearpage
\onecolumn
\section{Further Methodological Details} \label{sec:appendix-methodol}
\FloatBarrier
\begin{figure*}[ht]
\captionsetup{justification=raggedright, singlelinecheck=false, skip=2pt}
\centering
\begin{subfigure}[t]{0.3\linewidth}
   
    \subcaption{}
    \includegraphics[width=\linewidth]{sections/images/biden.jpg}
\end{subfigure}
\hspace{1cm}
\begin{subfigure}[t]{0.31\linewidth}

   \subcaption{}
   \includegraphics[width=\linewidth]{sections/images/rock.jpg}
\end{subfigure}
\caption{\mybold{AI-Generated Images from New York Times Quiz} \normalfont{\textbf{A}. NYT's explanation for evidence pointing to this image as AI-generated is: ``Though the resemblance to President Biden is striking, he would not be wearing military fatigues as a civilian.''~\cite{nytimes2024deepfake} \textbf{B}. NYT's explanation for evidence pointing to this image as AI-generated is ``One giveaway in this image is the badge, which includes garbled text.''~\cite{nytimes2024deepfake}}}
\label{fig:nytimes}
\Description{AI-generated image of Joe Biden in a conference room and an AI-generated image of the Rock in military uniform in a mall.}
\end{figure*}
\begin{figure*}[ht]
\centering
\captionsetup{justification=raggedright, singlelinecheck=false, skip=2pt}
\begin{subfigure}[t]{0.65\textwidth}
    \caption{}
    \vtop{\vskip0pt\hbox{\includegraphics[width=\linewidth]{sections/images/refiningpipeline.jpg}}}
\end{subfigure}
\hspace{0.01\textwidth} 
\begin{minipage}[t]{0.23\textwidth}
    \begin{subfigure}[t]{\textwidth}
        \caption{}
        \vtop{\vskip0pt\hbox{\includegraphics[width=\linewidth]{sections/images/facerefine.jpg}}}
    \end{subfigure}
    \vspace{0.5cm}
    \begin{subfigure}[t]{\textwidth}
        \caption{}\hfill\vtop{\vskip0pt\hbox{\includegraphics[width=0.8\linewidth]{sections/images/handrefine.jpg}}}
    \end{subfigure}
\end{minipage}
 \caption{\textbf{Image generation process in Stable Diffusion} A. \normalfont{Four stage image generation pipeline where the image is first generated in SD1.5. The output image is then encoded as latent and upscaled to be re-generated in SDXL with ControlNets applied for pose consistency. This is passed to the face refiner \cite{comfyuiimpactpack} which detects dominant and background faces in the image via YOLOv8 \cite{yolov8} and re-generates them using an SDXL pipeline. Finally, the resulting image is passed to the hand refiner \cite{comfyuiimpactpack} which detects hands in the image via YOLOv8  and predicts the hand pose used to guide the re-generation of the hands. \textbf{B}. Faces in the image before and after the face refining process \textbf{C}. Hand refining process. The left image shows the initial generation of the hand. The center image shows a predicted skeleton for the hand that is used for a ControlNet that guides the re-generation of the hand shown in the image on the right.}}
\label{fig:refiningpipe}
\Description{4 stage image generation pipeline}
\end{figure*}
\FloatBarrier
\twocolumn
According to a New York Times (NYT) quiz, qualities that typically signify AI generation include missing fingers, misaligned eyes, repeated elements, and garbled or nonsensical details~\cite{nytimes2024deepfake}.  Examples are shown in \ref{fig:nytimes}. The NYT quiz also discusses qualities that may cause a real image to look AI-generated, such as repeated cropping and compression that often happens over social media.

A screenshot of the pipeline, along with images before and after refinement, is shown in Figure~\ref{fig:refiningpipe}.
\FloatBarrier

\FloatBarrier
Figure~\ref{fig:pose-comprehensive} displays more examples of the four pose complexities and their average accuracies.
\onecolumn
\begin{figure}[H]
\centering
\resizebox{0.85\textwidth}{!}{
% This ensures the figure fits the page
\begin{minipage}{\textwidth}
% Portraits
\begin{subfigure}{0.22\linewidth}
    \includegraphics[width=\linewidth]{sections/images/ff_portrait3_014.jpeg}
    \caption{Acc: 25\%}
\end{subfigure}
\hfill
\begin{subfigure}{0.22\linewidth}
    \includegraphics[width=\linewidth]{sections/images/sd_portrait3_003.jpg}
    \caption{Acc: 37\%}
\end{subfigure}
\hfill
\begin{subfigure}{0.22\linewidth}
    \includegraphics[width=\linewidth]{sections/images/ff_portrait1_002.jpeg}
    \caption{Acc: 66\%}
\end{subfigure}
\hfill
\begin{subfigure}{0.22\linewidth}
    \includegraphics[width=\linewidth]{sections/images/sd_portrait3_075.jpg}
    \caption{Acc: 80\%}
\end{subfigure}

\vspace{0.3cm}

% Full Body
\begin{subfigure}{0.22\linewidth}
    \includegraphics[width=\linewidth]{sections/images/mj_fullbody3_028.jpg}
    \caption{Acc: 37\%}
\end{subfigure}
\hfill
\begin{subfigure}{0.22\linewidth}
    \includegraphics[width=\linewidth]{sections/images/mj_fullbody3_012.jpg}
    \caption{Acc: 57\%}
\end{subfigure}
\hfill
\begin{subfigure}{0.22\linewidth}
    \includegraphics[width=\linewidth]{sections/images/mj_fullbody3_022.jpg}
    \caption{Acc: 66\%}
\end{subfigure}
\hfill
\begin{subfigure}{0.22\linewidth}
    \includegraphics[width=\linewidth]{sections/images/mj_fullbody3_029.jpg}
    \caption{Acc: 83\%}
\end{subfigure}

\vspace{0.3cm}

% Posed Groups
\begin{subfigure}{0.22\linewidth}
    \includegraphics[width=\linewidth]{sections/images/mj_pg3_017.jpg}
    \caption{Acc: 37\%}
\end{subfigure}
\hfill
\begin{subfigure}{0.22\linewidth}
    \includegraphics[width=\linewidth]{sections/images/mj_pg2_012.jpg}
    \caption{Acc: 57\%}
\end{subfigure}
\hfill
\begin{subfigure}{0.22\linewidth}
    \includegraphics[width=\linewidth]{sections/images/mj_pg3_003.jpg}
    \caption{Acc: 66\%}
\end{subfigure}
\hfill
\begin{subfigure}{0.22\linewidth}
    \includegraphics[width=\linewidth]{sections/images/sd_pg3_013.jpg}
    \caption{Acc: 83\%}
\end{subfigure}

\vspace{0.3cm}

% Candid Groups
\begin{subfigure}{0.22\linewidth}
    \includegraphics[width=\linewidth]{sections/images/mj_ng3_016.jpg}
    \caption{Acc: 31\%}
\end{subfigure}
\hfill
\begin{subfigure}{0.22\linewidth}
    \includegraphics[width=\linewidth]{sections/images/mj_ng2_007.jpg}
    \caption{Acc: 66\%}
\end{subfigure}
\hfill
\begin{subfigure}{0.22\linewidth}
    \includegraphics[width=\linewidth]{sections/images/mj_ng4_003.jpg}
    \caption{Acc: 75\%}
\end{subfigure}
\hfill
\begin{subfigure}{0.22\linewidth}
    \includegraphics[width=\linewidth]{sections/images/mj_ng3_005.jpg}
    \caption{Acc: 87\%}
\end{subfigure}

\end{minipage}
}
\vspace{-2mm}
\caption{\textbf{More examples of the four pose complexities and their average accuracies.} \normalfont{The first row shows Portraits, the second row Full Body images, the third row Posed Groups, and the last row Candid Groups.}}
\label{fig:pose-comprehensive}
\Description{Examples of AI-generated images in different pose complexities: Portraits, Full Body, Posed Groups, and Candid Groups, with participant accuracy percentages.}
\end{figure}
\FloatBarrier
\twocolumn
\clearpage
\subsection{Robustness Check: Dataset Comparison}

To ensure the validity of our conclusions, we conducted a robustness check comparing the results from our full dataset against a subset excluding data collected before May 10th, 2024. This comparison addresses potential biases introduced by the initial experimental design, which did not implement stratified randomization as mentioned in Section \ref{exp-design}.

Table~\ref{tab:accuracy-comparison-dataset} presents the accuracy metrics for both the full dataset and the dataset excluding pre-May 10th data. The table includes overall accuracy, as well as specific accuracy for AI-generated and real images, along with their respective 95\% confidence intervals.

\begin{table}[H]
\centering
\caption{Comparison of accuracy: Full Dataset vs. Dataset excluding data before May 10th}
\label{tab:accuracy-comparison-dataset}
\resizebox{\linewidth}{!}{
\begin{tabular}{lcccccc}
\hline
Dataset & \multicolumn{2}{c}{Overall} & \multicolumn{2}{c}{AI-generated} & \multicolumn{2}{c}{Real} \\
 & Accuracy & 95\% CI & Accuracy & 95\% CI & Accuracy & 95\% CI \\
\hline
Full Dataset & 0.75 & [0.74, 0.76] & 0.76 & [0.74, 0.77] & 0.73 & [0.71, 0.75] \\
Dataset excluding data before May 10th & 0.75 & [0.74, 0.76] & 0.76 & [0.75, 0.77] & 0.7201 & [0.70, 0.74] \\
\hline
\end{tabular}}
\Description{A robustness check by comparing accuracy in full dataset vs. dataset excluding data before May 10th}
\end{table}

Figure~\ref{fig:accuracy-comparison-dstaset} visualizes the distribution of image accuracies for both datasets. This comparison allows for direct observation of any potential shifts in accuracy distributions between the full dataset and the subset, excluding early data.
 This robustness check supports the validity of using the full dataset in our main analysis.

\begin{figure}[H]
\centering
\includegraphics[width=\linewidth]{sections/images/dataset_accuracy_distribution_comparison.jpg}
\vspace{-10mm}
\caption{Comparison of accuracies between the full dataset and the dataset excluding data before 10th data.}
\label{fig:accuracy-comparison-dstaset}
\Description{Figure compares the distribution of accuracy of images from the full dataset vs. from the dataset excluding data before May 10th}
\end{figure}

\clearpage
\onecolumn
\section{Curated and Uncurated AI-generated Images}

\begin{figure*}[!htb]
\centering
\resizebox{1.0\textwidth}{!}{ % Scale figure to 95% of text width
\begin{minipage}{\textwidth} % Ensures correct alignment
\captionsetup{justification=raggedright, singlelinecheck=false, skip=2pt}

% Row A

\begin{subfigure}[t]{0.23\linewidth}  
\subcaption{}
\vtop{\vskip0pt\hbox{\includegraphics[width=\linewidth]{sections/images/ff_portrait3_001.jpeg}}}
\end{subfigure}
\hfill
\begin{subfigure}[t]{0.23\linewidth}  
\subcaption{}
\vtop{\vskip0pt\hbox{\includegraphics[width=\linewidth]{sections/images/american_faculty1.jpg}}}
\end{subfigure}
\hfill
\begin{subfigure}[t]{0.23\linewidth}  
\subcaption{}
\vtop{\vskip0pt\hbox{\includegraphics[width=\linewidth]{sections/images/american_faculty2.jpg}}}
\end{subfigure}
\hfill
\begin{subfigure}[t]{0.23\linewidth}  
\subcaption{}
\vtop{\vskip0pt\hbox{\includegraphics[width=\linewidth]{sections/images/american_faculty3.jpg}}}
\end{subfigure}

\vspace{8pt} % Reduced spacing between rows

% Row B
\begin{subfigure}[t]{0.23\linewidth}  
\subcaption{}
\vtop{\vskip0pt\hbox{\includegraphics[width=\linewidth]{sections/images/ff_pg4_001.jpeg}}}
\end{subfigure}
\hfill
\begin{subfigure}[t]{0.23\linewidth}  
\subcaption{}
\vtop{\vskip0pt\hbox{\includegraphics[width=\linewidth]{sections/images/astronaut1.jpg}}}
\end{subfigure}
\hfill
\begin{subfigure}[t]{0.23\linewidth}  
\subcaption{}
\vtop{\vskip0pt\hbox{\includegraphics[width=\linewidth]{sections/images/astronaut2.jpg}}}
\end{subfigure}
\hfill
\begin{subfigure}[t]{0.23\linewidth}  
\subcaption{}
\vtop{\vskip0pt\hbox{\includegraphics[width=\linewidth]{sections/images/astronaut3.jpg}}}
\end{subfigure}

\end{minipage}
}
\vspace{-2mm}
\caption{\mybold{Example images generated by consistently photorealistic and consistently detectable prompts.} \normalfont{
\textbf{A.} Curated image generated with a consistently photorealistic prompt: ``American woman faculty portrait, not a close-up, blond." 
\textbf{B-D} Reprompted images generated with the same consistently photorealistic prompts. 
\textbf{E.} Curated image generated with a consistently detectable prompt: ``Persian woman astronaut in astronaut clothes, family photo with husband and two toddlers, high resolution, realistic." 
\textbf{F-H} Reprompted images of the same consistently detectable prompts.}}
\label{fig:goodandbadprompt}
\Description{Two example images where A shows a portrait image of an American woman faculty with few visible artifacts and B shows a Persian woman and her child and husband in a space suit with noticeable artifacts in all of their faces.}
\end{figure*}

\clearpage
\onecolumn
\section{Future Work on Videos}
\begin{figure*}[ht]
\centering
\resizebox{1.0\textwidth}{!}{ 
\begin{minipage}{\textwidth} 
\captionsetup{justification=raggedright, singlelinecheck=false, skip=2pt}

% Row A
\begin{subfigure}[t]{0.33\linewidth}  
\subcaption{}
\vtop{\vskip0pt\hbox{\includegraphics[width=\linewidth]{sections/images/out_7.jpg}}}
\end{subfigure}
\hfill
\begin{subfigure}[t]{0.33\linewidth}  
\subcaption{}
\vtop{\vskip0pt\hbox{\includegraphics[width=\linewidth]{sections/images/out_9.jpg}}}
\end{subfigure}
\hfill
\begin{subfigure}[t]{0.33\linewidth}  
\subcaption{}
\vtop{\vskip0pt\hbox{\includegraphics[width=\linewidth]{sections/images/out_11.jpg}}}
\end{subfigure}

\vspace{10pt} % Spacing between rows

\begin{subfigure}[t]{0.33\linewidth}  
\subcaption{}
\vtop{\vskip0pt\hbox{\includegraphics[width=\linewidth]{sections/images/out_13.jpg}}}
\end{subfigure}
\hfill
\begin{subfigure}[t]{0.33\linewidth}  
\subcaption{}
\vtop{\vskip0pt\hbox{\includegraphics[width=\linewidth]{sections/images/out_15.jpg}}}
\end{subfigure}
\hfill
\begin{subfigure}[t]{0.33\linewidth}  
\subcaption{}
\vtop{\vskip0pt\hbox{\includegraphics[width=\linewidth]{sections/images/out_17.jpg}}}
\end{subfigure}

\vspace{10pt} % Spacing between rows

\begin{subfigure}[t]{0.33\linewidth}  
\subcaption{}
\vtop{\vskip0pt\hbox{\includegraphics[width=\linewidth]{sections/images/out_19.jpg}}}
\end{subfigure}
\hfill
\begin{subfigure}[t]{0.33\linewidth}  
\subcaption{}
\vtop{\vskip0pt\hbox{\includegraphics[width=\linewidth]{sections/images/out_21.jpg}}}
\end{subfigure}
\hfill
\begin{subfigure}[t]{0.33\linewidth}  
\subcaption{}
\vtop{\vskip0pt\hbox{\includegraphics[width=\linewidth]{sections/images/out_23.jpg}}}
\end{subfigure}

\vspace{10pt} % Spacing between rows

% Row C

\begin{subfigure}[t]{0.33\linewidth}  
\subcaption{}
\vtop{\vskip0pt\hbox{\includegraphics[width=\linewidth]{sections/images/out_25.jpg}}}
\end{subfigure}
\hfill
\begin{subfigure}[t]{0.33\linewidth}  
\subcaption{}
\vtop{\vskip0pt\hbox{\includegraphics[width=\linewidth]{sections/images/out_27.jpg}}}
\end{subfigure}
\hfill
\begin{subfigure}[t]{0.33\linewidth}  
\subcaption{}
\vtop{\vskip0pt\hbox{\includegraphics[width=\linewidth]{sections/images/out_29.jpg}}}
\end{subfigure}

\end{minipage}
}
\vspace{-2mm}
\caption{\mybold{Example frames from an AI-generated video with a temporal anatomical implausibility.} \normalfont{
9 frames from a video generated by OpenAI's Sora diffusion-transformer model where the subject's right leg morphs into the left leg somewhere between E and J. Each frame is separated by 1/10 of a second. This particular artifact fits into the anatomical implausibility category of the taxonomy, but it's different from any anatomical plausibility seen in diffusion model-generated images. In particular, this implausibility has a temporal element: the transition from A to L involves the subject's right leg becoming her left in a split second, which does not fit with what we know about human anatomy.}}
\label{fig:sora}
\Description{9 frames from a video generated by OpenAI's Sora diffusion-transformer model where the subject's right morphs into the left leg somewhere between E and J. Each frame is separated by 1/10 of a second. This particular artifact fits into the anatomical implausibility category of the taxonomy, but it's different than any anatomical plausibility in diffusion model-generated images. In particular, this implausibility has a temporal element: the transition from A to L involves the subject's right leg becoming her left in a split second, which does not fit with what we know about human anatomy.}
\end{figure*}


\end{document}
\endinput
%%
%% End of file `sample-manuscript.tex'.
