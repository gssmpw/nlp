%%
%% This is file `sample-manuscript.tex',
%% generated with the docstrip utility.
%%
%% The original source files were:
%%
%% samples.dtx  (with options: `manuscript')
%% 
%% IMPORTANT NOTICE:
%% 
%% For the copyright see the source file.
%% 
%% Any modified versions of this file must be renamed
%% with new filenames distinct from sample-manuscript.tex.
%% 
%% For distribution of the original source see the terms
%% for copying and modification in the file samples.dtx.
%% 
%% This generated file may be distributed as long as the
%% original source files, as listed above, are part of the
%% same distribution. (The sources need not necessarily be
%% in the same archive or directory.)
%%
%%
%% Commands for TeXCount
%TC:macro \cite [option:text,text]
%TC:macro \citep [option:text,text]
%TC:macro \citet [option:text,text]
%TC:envir table 0 1
%TC:envir table* 0 1
%TC:envir tabular [ignore] word
%TC:envir displaymath 0 word
%TC:envir math 0 word
%TC:envir comment 0 0
%%
%%
%% The first command in your LaTeX source must be the \documentclass
%% command.
%%
%% For submission and review of your manuscript please change the
%% command to \documentclass[manuscript, screen, review]{acmart}.
%%
%% When submitting camera ready or to TAPS, please change the command
%% to \documentclass[sigconf]{acmart} or whichever template is required
%% for your publication.
%%
%%
% \documentclass[manuscript,review,anonymous]{acmart}
% \documentclass[review,anonymous,sigconf]{acmart}
\documentclass[sigconf]{acmart}

%%
%% \BibTeX command to typeset BibTeX logo in the docs
\AtBeginDocument{%
  \providecommand\BibTeX{{%
    Bib\TeX}}}

\newcommand{\beginsupplement}{%
        \setcounter{table}{0}
        \renewcommand{\thetable}{S\arabic{table}}%
        \setcounter{figure}{0}
        \renewcommand{\thefigure}{S\arabic{figure}}%
     }
\usepackage{xcolor}


\definecolor{brown}{rgb}{0.59, 0.29, 0.0}
\definecolor{darkblue}{rgb}{0.0, 0.0, 0.55}
\definecolor{darkgreen}{rgb}{0.0, 0.5, 0.0}

%% Rights management information.  This information is sent to you
%% when you complete the rights form.  These commands have SAMPLE
%% values in them; it is your responsibility as an author to replace
%% the commands and values with those provided to you when you
%% complete the rights form.
% \setcopyright{acmcopyright}
% \copyrightyear{2025}
% \acmYear{2025}
% \acmDOI{https://doi.org/10.1145/3706598.3713962}

%% These commands are for a PROCEEDINGS abstract or paper.
% \acmConference[CHI '25]{Make sure to enter the correct
%   conference title from your rights confirmation email}{April 26-May 1,
%   2025}{Yokohama, Japan}

%%
%%  Uncomment \acmBooktitle if the title of the proceedings is different
%%  from ``Proceedings of ...''!
%%
%%\acmBooktitle{Woodstock '18: ACM Symposium on Neural Gaze Detection,
%%  June 03--05, 2018, Woodstock, NY}
%\acmPrice{15.00}
%\acmISBN{978-1-4503-XXXX-X/18/06}


%%
%% Submission ID.
%% Use this when submitting an article to a sponsored event. You'll
%% receive a unique submission ID from the organizers
%% of the event, and this ID should be used as the parameter to this command.
%\acmSubmissionID{3375}

%%
%% For managing citations, it is recommended to use bibliography
%% files in BibTeX format.
%%
%% You can then either use BibTeX with the ACM-Reference-Format style,
%% or BibLaTeX with the acmnumeric or acmauthoryear sytles, that include
%% support for advanced citation of software artefact from the
%% biblatex-software package, also separately available on CTAN.
%%
%% Look at the sample-*-biblatex.tex files for templates showcasing
%% the biblatex styles.
%%
\usepackage{placeins}
\usepackage{multirow}
\usepackage{verbatim}
\usepackage{afterpage}
\usepackage{float}
\usepackage{graphicx}
\usepackage[table]{colortbl}% http://ctan.org/pkg/xcolor
\usepackage{tikz}
\usepackage{nth}
\usetikzlibrary{calc}
\usepackage{zref-savepos}
\usepackage{xcolor}
\usepackage{booktabs,caption}
\usepackage{setspace}
\usepackage{placeins}
\usepackage{xcolor}
\usepackage{listings}% http://ctan.org/pkg/listings
\lstset{
  basicstyle=\ttfamily,
  mathescape
}
% \usepackage{silence}
% \WarningsOff[images] % This suppresses warnings related to images

\usepackage{algorithm}
\usepackage{algpseudocode}
\usepackage[algo2e, ruled,vlined]{algorithm2e}
\usepackage{tikz}
\usepackage{cancel}
\usepackage{amsmath}
\let\Bbbk\relax
\usepackage{amsfonts}
\usepackage{amssymb}
\usepackage{enumitem}
\usepackage{subcaption}
\usepackage{todonotes}
\newcounter{cpModel}
\makeatletter
\newenvironment{cpModel}[1][htb]{%
  \let\c@algorithm\c@cpModel
  \renewcommand{\ALG@name}{Model}% Update algorithm name
  \begin{algorithm}[#1]%
  }{\end{algorithm}
}
\makeatother 

\definecolor{tableauOrange}{HTML}{F28E2B}
\definecolor{tableauRed}{HTML}{E15759}
\definecolor{tableauBlue}{HTML}{4D79A7}
\definecolor{tableauGreen}{HTML}{77B7B3}
\definecolor{highlightPurple}{HTML}{b17aa1}

\newcommand{\tbOrangeMono}[1]{\textcolor{tableauOrange}{\textbf{\texttt{#1}}}}
\newcommand{\tbRedMono}[1]{\textcolor{tableauRed}{\textbf{\texttt{#1}}}}
\newcommand{\tbBlueMono}[1]{\textcolor{tableauBlue}{\textbf{\texttt{#1}}}}
\newcommand{\tbGreenMono}[1]{\textcolor{tableauGreen}{\textbf{\texttt{#1}}}}
\newcommand{\footnotenum}{\arabic{footnote}}
\setlength{\intextsep}{3pt plus 1pt minus 1pt} 
\raggedbottom
% \newcounter{spmodel}
% \makeatletter
% \newenvironment{spmodel}[1][htb]{%
%   \let\c@algorithm\c@spmodel
%    \renewcommand{\thealgorithm}{} % Reset counter format locally
%   \renewcommand{\ALG@name}{Shortest Path Model}% Update algorithm name
%   \begin{algorithm}[#1]%
%   }{\end{algorithm}
% }
% \makeatother 

%%
%% The majority of ACM publications use numbered citations and
%% references.  The command \citestyle{authoryear} switches to the
%% "author year" style.
%%
%% If you are preparing content for an event
%% sponsored by ACM SIGGRAPH, you must use the "author year" style of
%% citations and references.
%% Uncommenting
%% the next command will enable that style.
%%\citestyle{acmauthoryear}


% custom commands
%!TEX root=main.tex
\newif\ifspacehack
%\spacehacktrue
\usepackage{natbib}
\hypersetup{
    colorlinks = blue,
    breaklinks,
    linkcolor = blue,
    citecolor = blue,
    urlcolor  = blue,
}
\usepackage{url} 
\usepackage{graphicx}
\usepackage{mathtools}
\usepackage{footnote}
\usepackage{float}
\usepackage{xspace}
\usepackage{multirow}
\usepackage{xcolor}
\usepackage{wrapfig}
\usepackage{framed}
\usepackage{bbm}
\usepackage[most]{tcolorbox}

\usepackage{footnote}
\usepackage{nicefrac}
\usepackage{makecell}
\usepackage[ruled,vlined]{algorithm2e}
\usepackage{amssymb}
\usepackage{bm}
\makesavenoteenv{tabular}
\makesavenoteenv{table}

\newcommand{\hl}[1]{{\color{red}[HL: #1]}}
\newcommand{\mznote}[1]{{\color{blue}[MZ: #1]}}

% macros@Peng
\newcommand\innerp[2]{\langle #1, #2 \rangle}
\renewcommand{\tilde}{\widetilde}
\renewcommand{\hat}{\widehat}


\newcommand{\TVD}[1]{\norm{#1}_\text{TV}}
\newcommand{\corral}{\textsc{Corral}\xspace}
\newcommand{\expthree}{\textsc{Exp3}\xspace}
\newcommand{\expfour}{\ensuremath{\mathsf{Exp4}}\xspace}
\newcommand{\expthreeP}{\textsc{Exp3.P}\xspace}
\newcommand{\scrible}{\textsc{SCRiBLe}\xspace}

\def \R {\mathbb{R}}
\newcommand{\eps}{\epsilon}
\newcommand{\vecc}{\mathrm{vec}}
\newcommand{\LS}{\mathrm{LS}}
\newcommand{\FG}{\mathrm{FG}}
\newcommand{\DL}{\Delta \ellhat}
\newcommand{\calA}{{\mathcal{A}}}
\newcommand{\smax}{{\mathrm{smax}}}
\newcommand{\calB}{{\mathcal{B}}}
\newcommand{\calX}{{\mathcal{X}}}
\newcommand{\calS}{{\mathcal{S}}}
\newcommand{\calF}{{\mathcal{F}}}
\newcommand{\calI}{{\mathcal{I}}}
\newcommand{\calJ}{{\mathcal{J}}}
\newcommand{\calK}{{\mathcal{K}}}
\newcommand{\calH}{{\mathcal{H}}}
\newcommand{\calD}{{\mathcal{D}}}
\newcommand{\calE}{{\mathcal{E}}}
\newcommand{\calG}{{\mathcal{G}}}
\newcommand{\calU}{{\mathcal{U}}}
\newcommand{\calR}{{\mathcal{R}}}
\newcommand{\calT}{{\mathcal{T}}}
\newcommand{\calP}{{\mathcal{P}}}
\newcommand{\calQ}{{\mathcal{Q}}}
\newcommand{\calZ}{{\mathcal{Z}}}
\newcommand{\calM}{{\mathcal{M}}}
\newcommand{\calN}{{\mathcal{N}}}
\newcommand{\calW}{{\mathcal{W}}}
\newcommand{\calY}{{\mathcal{Y}}}
\newcommand{\cD}{{\mathcal{D}_{\mathcal{X}}}}
\newcommand{\mcD}{{\mathcal{D}}}
\newcommand{\cF}{{\mathcal{F}}}
\newcommand{\cA}{{\mathcal{A}}}
\newcommand{\cX}{{\mathcal{X}}}
\newcommand{\cE}{{\mathcal{E}}}
\newcommand{\cV}{{\mathcal{V}}}
\newcommand{\cR}{{\mathcal{R}}}
\newcommand{\wcR}{\widehat{\mathcal{R}}}
\newcommand{\Reg}{{\mathrm{Reg}}}
\newcommand{\Alg}{{\mathsf{Alg}}}
\newcommand{\wReg}{\widehat{\mathrm{Reg}}}
\newcommand{\cB}{\mathcal{B}}
\newcommand{\cP}{\mathcal{P}}
\newcommand{\nctx}{\text{n-ctx}}
\newcommand{\ctx}{\text{ctx}}
\newcommand{\E}{{\mathbb{E}}}
\newcommand{\V}{\mathbb{V}}
\newcommand{\Prob}{\mathbb{P}}
\newcommand{\1}{\mathbb{I}}
\newcommand{\N}{\mathbb{N}}
\newcommand{\tup}[1]{t^{(#1)}}
\newcommand{\gup}[1]{g^{(#1)}}
\newcommand{\hatfm}{\widehat{f}_m}
\newcommand{\haty}{\widehat{y}}
\newcommand{\hatx}{\widehat{x}}
\newcommand{\yhat}{\widehat{y}}
\newcommand{\xhat}{\widehat{x}}
\newcommand{\fhat}{\widehat{f}}
\newcommand{\ghat}{\widehat{g}}

\newcommand{\inner}[1]{ \left\langle {#1} \right\rangle }
\newcommand{\ind}{\mathbb{I}}
\newcommand{\diag}{\textrm{diag}}
\newcommand{\Nout}{N_{\textrm{out}}}
\newcommand{\nout}{N_{\textrm{out}}}
\newcommand{\Nin}{{\textrm{Nin}}}
\newcommand{\nin}{{\textrm{Nin}}}
\newcommand{\order}{\mathcal{O}}


\newcommand{\Acal}{\mathcal{A}}
\newcommand{\Bcal}{\mathcal{B}}
\newcommand{\Ccal}{\mathcal{C}}
\newcommand{\Dcal}{\mathcal{D}}
\newcommand{\Ecal}{\mathcal{E}}
\newcommand{\Fcal}{\mathcal{F}}
\newcommand{\Gcal}{\mathcal{G}}
\newcommand{\Hcal}{\mathcal{H}}
\newcommand{\Ical}{\mathcal{I}}
\newcommand{\Jcal}{\mathcal{J}}
\newcommand{\Kcal}{\mathcal{K}}
\newcommand{\Lcal}{\mathcal{L}}
\newcommand{\Mcal}{\mathcal{M}}
\newcommand{\Ncal}{\mathcal{N}}
\newcommand{\Ocal}{\mathcal{O}}
\newcommand{\Pcal}{\mathcal{P}}
\newcommand{\Qcal}{\mathcal{Q}}
\newcommand{\Rcal}{\mathcal{R}}
\newcommand{\Scal}{\mathcal{S}}
\newcommand{\Tcal}{\mathcal{T}}
\newcommand{\Ucal}{\mathcal{U}}
\newcommand{\Vcal}{\mathcal{V}}
\newcommand{\Wcal}{\mathcal{W}}
\newcommand{\Xcal}{\mathcal{X}}
\newcommand{\Ycal}{\mathcal{Y}}
\newcommand{\Zcal}{\mathcal{Z}}
\newcommand{\wkdn}{d}


\newcommand{\avgR}{\wh{\cal{R}}}
%\newcommand{\ips}{\wh{r}}
\newcommand{\whpi}{\wh{\pi}}
\newcommand{\whE}{\wh{\E}}
\newcommand{\whV}{\wh{V}}

\newcommand{\whReg}{\wh{\text{\rm Reg}}}
\newcommand{\flg}{\text{\rm flag}}
\newcommand{\one}{\boldsymbol{1}}
\newcommand{\var}{\Delta}
\newcommand{\Var}{\mathrm{Var}}
\newcommand{\bvar}{\bar{\Delta}}
\newcommand{\p}{\prime}
\newcommand{\evt}{\textsc{Event}}
\newcommand{\unif}{\text{\rm Unif}}
\newcommand{\KL}{\text{\rm KL}}
\newcommand{\Lstar}{{L^\star}}
\newcommand{\istar}{{i^\star}}
\newcommand{\dynreg}{\text{Dyn-Reg}}
\newcommand{\tildedynreg}{\widetilde{\text{Dyn-Reg}}}
\newcommand{\Bstar}{{B^\star}}
\newcommand{\Ustar}{\rho}
\newcommand{\Aconst}{a}
\newcommand{\dplus}[1]{\bm{#1}}
\newcommand{\lambdamax}{\lambda_\text{\rm max}}
\newcommand{\biasone}{\textsc{Deviation}\xspace}
\newcommand{\bias}{\textsc{Bias-1}\xspace}
\newcommand{\biastwo}{\textsc{Bias-2}\xspace}
\newcommand{\biasthree}{\textsc{Bias-3}\xspace}
\newcommand{\term}[1]{\texttt{Term} ~(#1)\xspace}
\newcommand{\x}{\mathbf{x}}
\newcommand{\errorterm}{\textsc{Error}\xspace}
\newcommand{\Err}[1]{\textsc{Err-Term}(#1)\xspace}
\newcommand{\regnctx}{\textsc{Reg-NCTX}\xspace}
\newcommand{\regterm}{\textsc{Reg-Term}\xspace}
\newcommand{\LTtilde}{\wt{L}_T}
\newcommand{\Bomega}{B_{\Omega}}
\newcommand{\UOB}{UOB-REPS\xspace}
\newcommand{\Holder}{{H{\"o}lder}\xspace}
\newcommand{\dpg}{\dplus{g}}
\newcommand{\dpM}{\dplus{M}}
\newcommand{\dpf}{\dplus{f}}
\newcommand{\dpX}{\dplus{\calX}}
\newcommand{\dpw}{\dplus{w}}
\newcommand{\dpF}{\dplus{F}}
\newcommand{\dpu}{\dplus{u}}
\newcommand{\dpwtilde}{\dplus{\wtilde}}
\newcommand{\dps}{\dplus{s}}
\newcommand{\dpe}{\dplus{e}}
\newcommand{\dpx}{\dplus{x}}
\newcommand{\dpy}{\dplus{y}}
\newcommand{\dpH}{\dplus{H}}
\newcommand{\dpOmega}{\dplus{\Omega}}
\newcommand{\dpellhat}{\dplus{\ellhat}}
\newcommand{\dpell}{\dplus{\ell}}
\newcommand{\dpr}{\dplus{r}}
\newcommand{\dpxi}{\dplus{\xi}}
\newcommand{\dpv}{\dplus{v}}
\newcommand{\dpI}{\dplus{I}}
\newcommand{\dpA}{\dplus{A}}
\newcommand{\dph}{\dplus{h}}
\newcommand{\cprob}{6}
\newcommand{\sigmoid}{\ensuremath{\mathsf{Sigmoid}}\xspace}
\newcommand{\relu}{\ensuremath{\mathsf{ReLU}}\xspace}

\DeclareMathOperator*{\argmin}{argmin}
\DeclareMathOperator*{\argmax}{argmax}
\DeclareMathOperator*{\argsmax}{argsmax}
%\DeclareMathOperator*{\range}{range}
%\DeclareMathOperator*{\mydet}{det_{+}}
%\DeclarePairedDelimiter\abs{\lvert}{\rvert}
%\DeclarePairedDelimiter\bigabs{\big\lvert}{\big\rvert}
\DeclarePairedDelimiter\ceil{\lceil}{\rceil}
%\DeclarePairedDelimiter\floor{\lfloor}{\rfloor}
%\DeclarePairedDelimiter\bigceil{\big\lceil}{\big\rceil}
%\DeclarePairedDelimiter\bigfloor{\big\lfloor}{\big\rfloor}

\newcommand{\field}[1]{\mathbb{#1}}
\newcommand{\fY}{\field{Y}}
\newcommand{\fX}{\field{X}}
\newcommand{\fH}{\field{H}}
\newcommand{\fR}{\field{R}}
\newcommand{\fN}{\field{N}}
\newcommand{\fS}{\field{S}}
\newcommand{\UCB}{{\operatorname{UCB}}}
\newcommand{\LCB}{{\operatorname{LCB}}}
\newcommand{\testblock}{\textsc{EndofBlockTest}\xspace}
\newcommand{\testreplay}{\textsc{EndofReplayTest}\xspace}

\newcommand{\theset}[2]{ \left\{ {#1} \,:\, {#2} \right\} }
% \newcommand{\inner}[1]{ \langle {#1} \rangle }
\newcommand{\inn}[1]{ \langle {#1} \rangle }
\newcommand{\Ind}[1]{ \field{I}_{\{{#1}\}} }
\newcommand{\eye}[1]{ \boldsymbol{I}_{#1} }
\newcommand{\norm}[1]{\left\|{#1}\right\|}
%\newcommand{\trace}[1]{\text{tr}\left({#1}\right)}
\newcommand{\trace}[1]{\textsc{tr}({#1})}


\newcommand{\defeq}{\stackrel{\rm def}{=}}
\newcommand{\sgn}{\mbox{\sc sgn}}
\newcommand{\scI}{\mathcal{I}}
\newcommand{\scO}{\mathcal{O}}
\newcommand{\scN}{\mathcal{N}}
\newcommand{\msmwc}{\textsc{MsMwC}}

\newcommand{\dt}{\displaystyle}
\renewcommand{\ss}{\subseteq}
\newcommand{\wh}{\widehat}
\newcommand{\wt}{\widetilde}
\newcommand{\wb}{\overline}
\newcommand{\ve}{\varepsilon}
\newcommand{\hlambda}{\wh{\lambda}}

\newcommand{\Jd}{J}
\newcommand{\ellhat}{\wh{\ell}}
\newcommand{\rhat}{\wh{r}}
\newcommand{\elltilde}{\wt{\ell}}
\newcommand{\wtilde}{\wt{w}}
\newcommand{\what}{\wh{w}}

\DeclareMathOperator{\conv}{conv}
\newcommand{\ellprime}{\ellhat^\prime}

\newcommand{\upconf}{\phi}

%\newcommand{\Ltilde}{\wt{L}}

\newcommand{\hDelta}{\wh{\Delta}}
\newcommand{\hdelta}{\wh{\delta}}
\newcommand{\spin}{\{-1,+1\}}

\newcommand{\ep}[1]{\E\!\left[#1\right]}
\newcommand{\LT}{L_T}
\newcommand{\LTbar}{\overline{L}_T}
\newcommand{\LTbarep}{\mathring{L}_T}
\newcommand{\circxhat}{\mathring{\xh}}
\newcommand{\circx}{\mathring{x}}
\newcommand{\circu}{\mathring{u}}
\newcommand{\circcalX}{\mathring{\calX}}
\newcommand{\circg}{\mathring{g}}
\newcommand{\Lubar}{\overline{L}_{u}}
%\newcommand{\Lustarbar}{\overline{L}_{u^\star}}

\newcommand{\Lyr}{J}
\newcommand{\QQ}{{w}}
\newcommand{\Qt}{{\QQ_t}}
\newcommand{\Qstar}{{u}}
\newcommand{\Qpistar}{{\Qstar^{\star}}}
\newcommand{\Qhat}{\wh{\QQ}}
\newcommand{\Ut}{{\upconf_t}}
\newcommand{\intO}{\mathrm{int}(\Omega)}
\newcommand{\intK}{\mathrm{int}(K)}

\newcommand{\squareCB}{\ensuremath{\mathsf{SquareCB}}\xspace}
\newcommand{\feelgood}{\ensuremath{\mathsf{FGTS}}\xspace}
\newcommand{\graphCB}{\ensuremath{\mathsf{SquareCB.G}}\xspace}
\newcommand{\squareCBAuc}{\ensuremath{\mathsf{SquareCB.A}}\xspace}
\newcommand{\AlgSq}{\ensuremath{\mathsf{AlgSq}}\xspace}
\newcommand{\AlgLog}{\ensuremath{\mathsf{AlgLog}}\xspace}
\newcommand{\ips}{\ensuremath{\mathsf{(IPS)}}\xspace}
\newcommand{\optsq}{\ensuremath{\mathsf{(OptSq)}}\xspace}
\newcommand{\sq}{\ensuremath{\mathsf{(Sq)}}\xspace}
\newcommand{\dec}{\ensuremath{\mathsf{dec}_\gamma}\xspace}
\newcommand{\dectwo}{\ensuremath{\mathsf{dec}_{\gamma_1,\gamma_2}}\xspace}
%\newcommand{\theHalgorithm}{\arabic{algorithm}}
\newtheorem{cor}[theorem]{Corollary}
\newcommand{\context}{\text{ctx}}
\newcommand{\noncontext}{\text{n-ctx}}
%\newtheorem{remark}{Remark}
%\newtheorem{prop}{Proposition}
%\newtheorem{definition}{Definition}
%\newtheorem{assumption}{Assumption}
\newtheorem{event}{Event}
%\newtheorem*{main}{Main Result}
%\newtheorem{fact}[theorem]{Fact}

\newcommand{\paren}[1]{\left({#1}\right)}
\newcommand{\brackets}[1]{\left[{#1}\right]}
\newcommand{\braces}[1]{\left\{{#1}\right\}}

\newcommand{\normt}[1]{\norm{#1}_{t}}
\newcommand{\dualnormt}[1]{\norm{#1}_{t,*}}

\newcommand{\otil}{\ensuremath{\tilde{\mathcal{O}}}}

\newcommand{\dist}{\calP}

%%%%  brackets
\newcommand{\rbr}[1]{\left(#1\right)}
\newcommand{\sbr}[1]{\left[#1\right]}
\newcommand{\cbr}[1]{\left\{#1\right\}}
\newcommand{\nbr}[1]{\left\|#1\right\|}
\newcommand{\abr}[1]{\left|#1\right|}

\usepackage{lipsum,booktabs}
\usepackage{amsmath,mathrsfs,amssymb,amsfonts,bm,enumitem}
\usepackage{rotating}
\usepackage{pdflscape}
\usepackage{hyperref,url}
\hypersetup{
    colorlinks,
    breaklinks,
    linkcolor = blue,
    citecolor = blue,
    urlcolor  = blue,
}
\allowdisplaybreaks
\usepackage{appendix}
\usepackage{multirow,makecell}

%\usepackage{algorithmic,algorithm}
%\renewcommand{\algorithmicrequire}{ \textbf{Input:}}
%\renewcommand{\algorithmicensure}{ \textbf{Output:}}

\renewcommand{\tilde}{\widetilde}
\renewcommand{\hat}{\widehat}
\newcommand{\obs}{O}
\newcommand{\unobs}{E}
\newcommand{\unbiasSize}{c}
\newcommand{\unbias}{C}
\newcommand{\cnt}{k}

% define some macros
\def \A {\mathcal{A}}

\def \B {\mathbb{B}}
\def \B {\mathcal{B}}
\def \C {\mathcal{C}}
\def \D {\mathcal{D}}
\def \E {\mathbb{E}}
\def \F {\mathcal{F}}
\def \G {\mathcal{G}}
\def \H {\mathcal{H}}
\def \I {\mathcal{I}}
\def \J {\mathcal{J}}
\def \K {\mathcal{K}}
\def \L {\mathcal{L}}
\def \M {\mathcal{M}}
\def \N {\mathcal{N}}
\def \O {\mathcal{O}}
\def \P {\mathcal{P}}
\def \Q {\mathcal{Q}}
\def \R {\mathbb{R}}
\def \S {\mathcal{S}}
% \def \T {\mathrm{T}}
\def \T {\top}
\def \U {\mathcal{U}}
\def \V {\mathcal{V}}
\def \W {\mathcal{W}}
\def \X {\mathcal{X}}
\def \Y {\mathcal{Y}}
\def \Z {\mathcal{Z}}

\def \a {\mathbf{a}}
\def \b {\mathbf{b}}
\def \c {\mathbf{c}}
\def \d {\mathbf{d}}
\def \e {\mathbf{e}}
\def \f {\mathbf{f}}
\def \g {\mathbf{g}}
\def \h {\mathbf{h}}
\def \m {\mathbf{m}}
\def \p {\mathbf{p}}
\def \q {\mathbf{q}}
\def \u {\mathbf{u}}
\def \w {\mathbf{w}}
\def \s {\mathbf{s}}
\def \t {\mathbf{t}}
\def \v {\mathbf{v}}
\def \x {\mathbf{x}}
\def \y {y}
\def \z {\mathbf{z}}

\def \ph {\hat{p}}

\def \fh {\hat{f}}
\def \fb {\bar{f}}
\def \ft{\tilde{f}}

\def \gh {\hat{\g}}
\def \gb {\bar{\g}}
\def \gt {\tilde{g}}

\def \uh {\hat{\u}}
\def \ub {\bar{\u}}
\def \ut{\tilde{\u}}

\def \vh {\hat{\v}}
\def \vb {\bar{\v}}
\def \vt{\tilde{\v}}

\def \xh {\hat{x}}
\def \xb {\bar{\x}}
\def \xt {\tilde{\x}}

\def \zh {\hat{\z}}
\def \zb {\bar{\z}}
\def \zt {\tilde{\z}}

\def \Ecal {\mathcal{E}}
\def \Rcal {\mathcal{R}}
\def \Ot {\tilde{\O}}
\def \indicator {\mathds{1}}
\def \regret {\mbox{Regret}}
\def \proj {\mbox{Proj}}
\def \Pr {\mathsf{Pr}}
\def \ellb {\boldsymbol{\ell}}
\def \thetah {\hat{\theta}}

\newcommand{\RegSq}{\ensuremath{\mathrm{\mathbf{Reg}}_{\mathsf{Sq}}}\xspace}
\newcommand{\RegCB}{\ensuremath{\mathrm{\mathbf{Reg}}_{\mathsf{CB}}}\xspace}
\newcommand{\RegDyn}{\ensuremath{\mathrm{\mathbf{Reg}}_{\mathsf{Dyn}}}\xspace}
\usepackage{mathtools}
\let\oldnorm\norm   % <-- Store original \norm as \oldnorm
\let\norm\undefined % <-- "Undefine" \norm
\DeclarePairedDelimiter\norm{\lVert}{\rVert}
\DeclarePairedDelimiter\abs{\lvert}{\rvert}
%\newcommand\inner[2]{\langle #1, #2 \rangle}
\newcommand*\diff{\mathop{}\!\mathrm{d}}
\newcommand*\Diff[1]{\mathop{}\!\mathrm{d^#1}}

%\DeclareMathOperator*{\Reg}{Regret}
\DeclareMathOperator*{\AReg}{A-Regret}
\DeclareMathOperator*{\WAReg}{WA-Regret}
\DeclareMathOperator*{\SAReg}{SA-Regret}
\DeclareMathOperator*{\DReg}{\mbox{D-Regret}}
\DeclareMathOperator*{\poly}{poly}
%\DeclareMathOperator*{\argmax}{arg\,max}
%\DeclareMathOperator*{\argmin}{arg\,min}

% define new theorem environments
% \let\proof\relax
% \let\endproof\relax
% \newenvironment{proof}{\par\noindent{\bf Proof\ }}{\hfill\BlackBox\\[2mm]}
% \renewcommand\qedsymbol{$\blacksquare$}
\newtheorem{myThm}{Theorem}
\newtheorem{myFact}{Fact}
\newtheorem{myClaim}{Claim}
\newtheorem{myLemma}[myThm]{Lemma}
\newtheorem{myObservation}{Observation}
\newtheorem{myProp}[myThm]{Proposition}
\newtheorem{myProperty}{Property}

% Define a custom environment for prompts
\newtcolorbox{promptbox}[1][]{
  colback=blue!5!white, colframe=blue!75!black,
  fonttitle=\bfseries, title=Prompt,
  left=2mm, right=2mm, top=2mm, bottom=2mm,
  boxrule=0.5mm,  % Thickness of the frame
  coltitle=black, % Color of the title text
  colbacktitle=blue!15!white, % Background color of the title
  breakable,      % Allows the box to break across pages
  #1
}
\newtheorem{myAssum}{Assumption}
\newtheorem{myConj}{Conjecture}
\newtheorem{myCor}{Corollary}
\newtheorem{myDef}{Definition}
\newtheorem{myExample}{Example}
\newtheorem{myNote}{Note}
\newtheorem{myProblem}{Problem}

\newtheorem{myRemark}{Remark}

% add comments
\usepackage{graphicx,color} % more modern
\newcommand{\red}{\color{red}}
\newcommand{\blue}{\color{blue}}
\definecolor{wine_red}{RGB}{228,48,64}
\definecolor{DSgray}{cmyk}{0,1,0,0}
%\newcommand{\Authornote}[2]{{\small\textcolor{NavyBlue}{\sf$<<<${  #1: #2 }$>>>$}}}
% \newcommand{\Authormarginnote}[2]{\marginpar{\parbox{2cm}{\raggedright\tiny \textcolor{DSgray}{#1: #2}}}}
% \newcommand{\pnote}[1]{{\Authornote{Peng}{#1}}}
% \newcommand{\pmarginnote}[1]{{\Authormarginnote{Peng}{#1}}}

\usepackage{prettyref}
\newcommand{\pref}[1]{\prettyref{#1}}
\newcommand{\pfref}[1]{Proof of \prettyref{#1}}
\newcommand{\savehyperref}[2]{\texorpdfstring{\hyperref[#1]{#2}}{#2}}
\newrefformat{eq}{\savehyperref{#1}{Eq. \textup{(\ref*{#1})}}}
\newrefformat{eqn}{\savehyperref{#1}{Eq.~(\ref*{#1})}}
\newrefformat{lem}{\savehyperref{#1}{Lemma~\ref*{#1}}}
\newrefformat{event}{\savehyperref{#1}{Event~\ref*{#1}}}
\newrefformat{def}{\savehyperref{#1}{Definition~\ref*{#1}}}
\newrefformat{line}{\savehyperref{#1}{Line~\ref*{#1}}}
\newrefformat{thm}{\savehyperref{#1}{Theorem~\ref*{#1}}}
\newrefformat{tab}{\savehyperref{#1}{Table~\ref*{#1}}}
\newrefformat{corr}{\savehyperref{#1}{Corollary~\ref*{#1}}}
\newrefformat{cor}{\savehyperref{#1}{Corollary~\ref*{#1}}}
\newrefformat{sec}{\savehyperref{#1}{Section~\ref*{#1}}}
\newrefformat{app}{\savehyperref{#1}{Appendix~\ref*{#1}}}
\newrefformat{assum}{\savehyperref{#1}{Assumption~\ref*{#1}}}
\newrefformat{asm}{\savehyperref{#1}{Assumption~\ref*{#1}}}
\newrefformat{ex}{\savehyperref{#1}{Example~\ref*{#1}}}
\newrefformat{fig}{\savehyperref{#1}{Figure~\ref*{#1}}}
\newrefformat{alg}{\savehyperref{#1}{Algorithm~\ref*{#1}}}
\newrefformat{rem}{\savehyperref{#1}{Remark~\ref*{#1}}}
\newrefformat{conj}{\savehyperref{#1}{Conjecture~\ref*{#1}}}
\newrefformat{prop}{\savehyperref{#1}{Proposition~\ref*{#1}}}
\newrefformat{proto}{\savehyperref{#1}{Protocol~\ref*{#1}}}
\newrefformat{prob}{\savehyperref{#1}{Problem~\ref*{#1}}}
\newrefformat{claim}{\savehyperref{#1}{Claim~\ref*{#1}}}
\newrefformat{que}{\savehyperref{#1}{Question~\ref*{#1}}}
\newrefformat{op}{\savehyperref{#1}{Open Problem~\ref*{#1}}}
\newrefformat{fn}{\savehyperref{#1}{Footnote~\ref*{#1}}}

\def \p {\boldsymbol{p}}
\def \s {\boldsymbol{s}}
\def \m {\boldsymbol{m}}
\def \epsilon {\varepsilon}

% \def \base {\mathtt{base}\mbox{-}\mathtt{regret}}
% \def \meta {\mathtt{meta}\mbox{-}\mathtt{regret}}
\def \base {\textsc{base-regret}}
\def \meta {\textsc{meta-regret}}
\def \xref {\x_{\text{ref}}}
\def \fb {\bar{f}}
\def \interior {\text{int}}
\def \yh {\hat{\y}}
\def \RegLog {\Reg_{\log}^G}
\newcommand{\bra}[1]{\left[#1\right]}
\newcommand{\pa}[1]{\left(#1\right)}
\newcommand{\hhat}{\wh{h}}
\newcommand{\epsn}{\epsilon_N}
\newcommand{\rad}{\mathsf{rad}}
\newcommand{\hatr}{\wh{r}}
\newcommand{\fl}{\underline{f}^\star}

%%
%% end of the preamble, start of the body of the document source.

\newcommand{\revision}[1]{\textcolor{blue}{#1}}

\renewcommand{\figureautorefname}{Figure}
\renewcommand{\tableautorefname}{Table}
\renewcommand{\partautorefname}{Part}
\renewcommand{\appendixautorefname}{Appendix}
\renewcommand{\chapterautorefname}{Chapter}
\renewcommand{\sectionautorefname}{Section}
\renewcommand{\subsectionautorefname}{Section}
\renewcommand{\subsubsectionautorefname}{Section}

\copyrightyear{2025}
\acmYear{2025}
\setcopyright{cc}
\setcctype{by}
\acmConference[CHI '25]{CHI Conference on Human Factors in Computing Systems}{April 26-May 1, 2025}{Yokohama, Japan}
\acmBooktitle{CHI Conference on Human Factors in Computing Systems (CHI '25), April 26-May 1, 2025, Yokohama, Japan}\acmDOI{10.1145/3706598.3713962}
\acmISBN{979-8-4007-1394-1/25/04}

\begin{document}

%%
%% The "title" command has an optional parameter,
%% allowing the author to define a "short title" to be used in page headers.
% \title{An Ontological Framework for Human-Centered AI Image Analysis}
\title[Characterizing Photorealism in AI-Generated Images]{Characterizing Photorealism and Artifacts in Diffusion Model-Generated Images}

% \title[Categorizing Implausibilities in Diffusion Model-Generated Images]{Designing a Framework for Categorizing Implausibilities in Diffusion Model Generated Images}

%%
%% The "author" command and its associated commands are used to define
%% the authors and their affiliations.
%% Of note is the shared affiliation of the first two authors, and the
%% "authornote" and "authornotemark" commands
%% used to denote shared contribution to the research.
% \author{Negar Kamali}
% \email{negar.kamali@anonymous.xxx}
% \orcid{}
% \affiliation{%
%   \institution{Anonymous Institution}
%   \city{Anonymous City}
%   \state{Anonymous State}
%   \country{Anonymous Country}
% }
\author{Negar Kamali}
\email{negar.kamali@u.northwestern.edu}
\orcid{0000-0002-1086-6735}
\affiliation{%
  \institution{Northwestern University}
  \city{Evanston}
  \state{Illinois}
  \country{USA}
}

\author{Karyn Nakamura}
\email{karynnakamura68@gmail.com}
\orcid{0009-0005-4419-0701}
\affiliation{%
  \institution{Northwestern University}
  \city{Evanston}
  \state{Illinois}
  \country{USA}
}

\author{Aakriti Kumar}
\email{aakriti.kumar@kellogg.northwestern.edu}
\orcid{0000-0002-9502-013X}
\affiliation{%
  \institution{Northwestern University}
  \city{Evanston}
  \state{Illinois}
  \country{USA}
}

\author{Angelos Chatzimparmpas}
\email{a.chatzimparmpas@uu.nl}
\orcid{0000-0002-9079-2376 }
\affiliation{%
  \institution{Utrecht University}
  \city{Utrecht}
  \state{}
  \country{Netherlands}
}

\author{Jessica Hullman}
\email{jhullman@northwestern.edu}
\orcid{0000-0001-6826-3550}
\affiliation{%
  \institution{Northwestern University}
  \city{Evanston}
  \state{Illinois}
  \country{USA}
}

\author{Matthew Groh}
\email{matthew.groh@kellogg.northwestern.edu}
\orcid{0000-0002-9029-0157}
\affiliation{%
  \institution{Northwestern University}
  \city{Evanston}
  \state{Illinois}
  \country{USA}
}
%%
%% By default, the full list of authors will be used in the page
%% headers. Often, this list is too long, and will overlap
%% other information printed in the page headers. This command allows
%% the author to define a more concise list
%% of authors' names for this purpose.
\renewcommand{\shortauthors}{Kamali et al.}

% \renewcommand{\shortauthors}{Anonymous et al.}
\renewcommand{\thesubfigure}{\textbf{\Alph{subfigure}}}
\captionsetup[sub]{labelformat=simple}  % Remove parenthese
\newcommand{\mybold}[1]{\textbf{#1}}

% \received{20 February 2007}
% \received[revised]{12 March 2009}
% \received[accepted]{5 June 2009}

%%
%% This command processes the author and affiliation and title
%% information and builds the first part of the formatted document.
\begin{abstract}

During the early stages of interface design, designers need to produce multiple sketches to explore a design space.  Design tools often fail to support this critical stage, because they insist on specifying more details than necessary. Although recent advances in generative AI have raised hopes of solving this issue, in practice they fail because expressing loose ideas in a prompt is impractical. In this paper, we propose a diffusion-based approach to the low-effort generation of interface sketches. It breaks new ground by allowing flexible control of the generation process via three types of inputs: A) prompts, B) wireframes, and C) visual flows. The designer can provide any combination of these as input at any level of detail, and will get a diverse gallery of low-fidelity solutions in response. The unique benefit is that large design spaces can be explored rapidly with very little effort in input-specification. We present qualitative results for various combinations of input specifications. Additionally, we demonstrate that our model aligns more accurately with these specifications than other models. 

% OLD ABSTRACT
%When sketching Graphical User Interfaces (GUIs), designers need to explore several aspects of visual design simultaneously, such as how to guide the user’s attention to the right aspects of the design while making the intended functionality visible. Although current Large Language Models (LLMs) can generate GUIs, they do not offer the finer level of control necessary for this kind of exploration. To address this, we propose a diffusion-based model with multi-modal conditional generation. In practice, our model optionally takes semantic segmentation, prompt guidance, and flow direction to generate multiple GUIs that are aligned with the input design specifications. It produces multiple examples. We demonstrate that our approach outperforms baseline methods in producing desirable GUIs and meets the desired visual flow.

% Designing visually engaging Graphical User Interfaces (GUIs) is a challenge in HCI research. Effective GUI design must balance visual properties, like color and positioning, with user behaviors to ensure GUIs easy to comprehend and guide attention to critical elements. Modern GUIs, with their complex combinations of text, images, and interactive components, make it difficult to maintain a coherent visual flow during design.
% Although current Large Language Models (LLMs) can generate GUIs, they often lack the fine control necessary for ensuring a coherent visual flow. To address this, we propose a diffusion-based model that effectively handles multi-modal conditional generation. Our model takes semantic segmentation, optional prompt guidance, and ordered viewing elements to generate high-fidelity GUIs that are aligned with the input design specifications.
% We demonstrate that our approach outperforms baseline methods in producing desirable GUIs and meets the desired visual flow. Moreover, a user study involving XX designers indicates that our model enhances the efficiency of the GUI design ideation process and provides designers with greater control compared to existing methods.    



% %%%%%%%%%%%%%%%%%%%%%%%%%%%%%%%%%%%%%%%%%%%%%%%%%%%%%%
% % Writing Clinic Comments:
% %%%%%%%%%%%%%%%%%%%%%%%%%%%%%%%%%%%%%%%%%%%%%%%%%%%%%%
% % Define: Effective UI design
% % Motivate GANs and write in full form.
% % LLMs vs ControlNet vs GANs
% % Say something about the Figma plugin?
% % Write the work is novel or what has been done before
% % What is desirable UI and how to evalutate that?
% % Visual Flow - main theme (center around it)
% % Re-Title: use word Flow!
% % Use ControlNet++ & SPADE for abstract.
% % Write about input/output. 
% % Why better than previous work?
% %%%%%%%%%%%%%%%%%%%%%%%%%%%%%%%%%%%%%%%%%%%%%%%%%%%%%

% % v2:
% % \noindent \textcolor{red}{\textbf{NEW Abstract!} (Post Writing Clinic 1 - 25-Jun)}

% % \noindent \textcolor{red}{----------------------------------------------------------------------}

% % \noindent Designing user interfaces (UIs) is a time-consuming process, particularly for novice designers. 
% % Creating UI designs that are effective in market funneling or any other designer defined goal requires a good understanding of the visual flow to guide users' attention to UI elements in the desired order. 
% % While current Large Language Models (LLMs) can generate UIs from just prompts, they often lack finer pixel-precise control and fail to consider visual flow. 
% % In this work, we present a UI synthesis method that incorporates visual flow alongside prompts and semantic layouts. 
% % Our efficient approach uses a carefully designed Generative Adversarial Network (GAN) optimized for scenarios with limited data, making it more suitable than diffusion-based and large vision-language models.
% % We demonstrate that our method produces more "desirable" UIs according to the well-known contrast, repetition, alignment, and proximity principles of design. 
% % We further validate our method through comprehensive automatic non-reference, human-preference aligned network scoring and subjective human evaluations.
% % Finally, an evaluation with xx non-expert designers using our contributed Figma plugin shows that <method-name> improves the time-efficiency as well as the overall quality of the UI design development cycle.

% % \noindent \textcolor{red}{----------------------------------------------------------------------}


% \noindent \textcolor{blue}{\textbf{NEW Abstract!} (Pre Writing Clinic 9-July)}

% \noindent \textcolor{blue}{----------------------------------------------------------------------}

% \noindent Exploring different graphical user interface (GUI) design ideas is time-consuming, particularly for novice designers. 
% Given the segmentation masks, design requirement as prompt, and/or preferred visual flow, we aim to facilitate creative exploration for GUI design and generate different UI designs for inspiration.
% While current Vision Language Models (VLMs) can generate GUIs from just prompts, they often lack control over visual concepts and flow that are difficult to convey through language during the generation process. 
% In this work, we present FlowGenUI, a semantic map-guided GUI synthesis method that optionally incorporates visual flow information based on the user's choice alongside language prompts. 
% We demonstrate that our model not only creates more realistic GUIs but also creates "predictable" (how users pay attention to and order of looking at GUI elements) GUIs.
% Our approach uses Stable Diffusion (SD), a large paired image-text pretrained diffusion model with a rich latent space that we steer toward realistic GUIs using a trainable copy of SD's encoder for every condition (segmentation masks, prompts, and visual flow). 
% We further provide a semantic typography feature to create custom text-fonts and styles while also alleviating SD's inherent limitations in drawing coherent, meaningful and correct aspect-ratio text. 
% Finally, a subjective evaluation study of XX non-expert and expert designers demonstrates the efficiency and fidelity of our method.


% This process encourages creativity and prevents designers from falling into habitual patterns.


% ------------------------------------------------------------------
% Joongi Why is it important to create realistic GUI?
% I do not see how the Visual Flow given on the left hand side is reflected in the results on the right hand side. 
% I’d avoid making unsubstantiated claims about designers (falling into habitual patterns).
% The UIs you generate do not “align with users’ attention patterns” but rather try to control it (that’s what visual flow means)
% ------------------------------------------------------------------
% Comments - Writing Clinic - 9th July:
% Improve title. More names: FlowGen
% Figure 1: Use an inference time hand-drawn mask
% Figure 1: Show both workflows. Add a designer --> Input.
% Figure 1: Make them more diverse
% ------------------------------------------------------------------
% Designing graphical user interfaces (GUIs) requires human creativity and time. Designers often fall into habitual patterns, which can limit the exploration of new ideas. 
% To address this, we introduce FlowGenUI, a method that facilitates creative exploration and generates diverse GUI designs for inspiration. By using segmentation masks, design requirements as prompts, and/or selected visual flows, our approach enhances control over the visual concepts and flows during the generation process, which current Vision Language Models (VLMs) often lack.
% FlowGenUI uses Stable Diffusion (SD), a largely pretrained text-to-image diffusion model, and guides it to create realistic GUIs. 
% We achieve this by using a trainable copy of SD's encoder for each condition (segmentation masks, prompts, and visual flow). 
% This method enables the creation of more realistic and predictable GUIs that align with users' attention patterns and their preferred order of viewing elements.
% We also offer a semantic typography feature that creates custom text fonts and styles while addressing SD's limitations in generating coherent, meaningful, and correctly aspect-ratio text.
% Our approach's efficiency and fidelity are evaluated through a subjective user study involving XX designers. 
% The results demonstrate the effectiveness of FlowGenUI in generating high-quality GUI designs that meet user requirements and visual expectations.

% ---------------------------------------


%A critical and general issue remains while using such deep generative priors: creating coherent, meaningful and correct aspect-ratio text. 
%We tackle this issue within our framework and additionally provide a semantic typography feature to create custom text-fonts and styles. 


% %Creating UI designs that are effective in market funneling or any other designer-defined goal requires a good understanding of the visual flow to guide users' attention to UI elements in the desired order. 
% %While current largely pre-trained Vision Language Models (VLMs) can generate GUIs from just prompts, they often lack finer or pixel-precise control which can be crucial for many easy-to-understand visual concepts but difficult to convey through language. 
% % However, obtaining such pixe-level labels is an extremely expensive so we
% % For example - overlaying text on images with certain aspect ratios and two equally separated buttons 
% Additionally, all prior GUI generation work fails to consider visual flow information during the generation process. 
% We demonstrate that visual flow-informed generation not only creates more realistic and human-friendly GUIs but also creates "predictable" (how users pay attention to and order of looking at GUI elements) UIs that could be beneficial for designers for tasks like creating effective market funnels.
% In this work, we present a semantic map-guided GUI synthesis method that optionally incorporates visual flow information based on the user's choice alongside language prompts. 
% Our approach uses Stable Diffusion, a large (billions) paired image-text pretrained diffusion model with a rich latent space that we steer toward realistic GUIs using an ensemble of ControlNets. 
% % TODO: Mention it in 1 sentence:
% A critical and general issue remains while using such deep generative priors: creating coherent, meaningful and correct aspect-ratio text. 
% We tackle this issue within our framework and additionally provide a semantic typography feature to create custom text-fonts and styles. 
% To evaluate our method, we demonstrate that our method produces more "desirable" UIs according to the well-known contrast, repetition, alignment, and proximity principles of design. 
% % We further validate our method through comprehensive automatic non-reference and human-preference aligned scores. (TODO: Maybe Unskip if we get UIClip from Jason!)
% % TODO: Re-word this and only keep ideation cycles and time-efficiency.
% Finally, a subjective evaluation study of XX non-expert and expert designers demonstrates the efficiency and fidelity of our method.
% % improves the time-efficiency by quick iterations of the UI design ideation process.
% %Finally, an evaluation with xx non-expert designers using our contributed <method-name> improves the time-efficiency by quick iterations of the UI design ideation cycle.

%\noindent \textcolor{blue}{----------------------------------------------------------------------}


%In an evaluation with xx designers, we found that GenerativeLayout: 1) enhances designers' exploration by expanding the coverage of the design space, 2) reduces the time required for exploration, and 3) maintains a perceived level of control similar to that of manual exploration.



% Present-day graphical user interfaces (GUIs) exhibit diverse arrangements of text, graphics, and interactive elements such as buttons and menus, but representations of GUIs have not kept up. They do not encapsulate both semantic and visuo-spatial relationships among elements. %\color{red} 
% To seize machine learning's potential for GUIs more efficiently, \papername~ exploits graph neural networks to capture individual elements' properties and their semantic—visuo-spatial constraints in a layout. The learned representation demonstrated its effectiveness in multiple tasks, especially generating designs in a challenging GUI autocompletion task, which involved predicting the positions of remaining unplaced elements in a partially completed GUI. The new model's suggestions showed alignment and visual appeal superior to the baseline method and received higher subjective ratings for preference. 
% Furthermore, we demonstrate the practical benefits and efficiency advantages designers perceive when utilizing our model as an autocompletion plug-in.


% Overall pipeline: Maybe drop semantic typography / visual flow?
\end{abstract}

\begin{CCSXML}
<ccs2012>
   <concept>
       <concept_id>10003120.10003121.10011748</concept_id>
       <concept_desc>Human-centered computing~Empirical studies in HCI</concept_desc>
       <concept_significance>500</concept_significance>
       </concept>
   <concept>
       <concept_id>10003120.10003121</concept_id>
       <concept_desc>Human-centered computing~Human computer interaction (HCI)</concept_desc>
       <concept_significance>500</concept_significance>
       </concept>
 </ccs2012>
\end{CCSXML}

\ccsdesc[500]{Human-centered computing~Empirical studies in HCI}
\ccsdesc[500]{Human-centered computing~Human computer interaction (HCI)}

\keywords{photorealism, diffusion models, generative AI, synthetic media, deepfakes, misinformation}

\maketitle

\section{Introduction}

Tutoring has long been recognized as one of the most effective methods for enhancing human learning outcomes and addressing educational disparities~\citep{hill2005effects}. 
By providing personalized guidance to students, intelligent tutoring systems (ITS) have proven to be nearly as effective as human tutors in fostering deep understanding and skill acquisition, with research showing comparable learning gains~\citep{vanlehn2011relative,rus2013recent}.
More recently, the advancement of large language models (LLMs) has offered unprecedented opportunities to replicate these benefits in tutoring agents~\citep{dan2023educhat,jin2024teach,chen2024empowering}, unlocking the enormous potential to solve knowledge-intensive tasks such as answering complex questions or clarifying concepts.


\begin{figure}[t!]
\centering
\includegraphics[width=1.0\linewidth]{Figs/Fig.intro.pdf}
\caption{An illustration of coding tutoring, where a tutor aims to proactively guide students toward completing a target coding task while adapting to students' varying levels of background knowledge. \vspace{-5pt}}
\label{fig:example}
\end{figure}

\begin{figure}[t!]
\centering
\includegraphics[width=1.0\linewidth]{Figs/Fig.scaling.pdf}
\caption{\textsc{Traver} with the trained verifier shows inference-time scaling for coding tutoring (detailed in \S\ref{sec:scaling_analysis}). \textbf{Left}: Performance vs. sampled candidate utterances per turn. \textbf{Right}: Performance vs. total tokens consumed per tutoring session. \vspace{-15pt}}
\label{fig:scale}
\end{figure}


Previous research has extensively explored tutoring in educational fields, including language learning~\cite{swartz2012intelligent,stasaski-etal-2020-cima}, math reasoning~\cite{demszky-hill-2023-ncte,macina-etal-2023-mathdial}, and scientific concept education~\cite{yuan-etal-2024-boosting,yang2024leveraging}. 
Most aim to enhance students' understanding of target knowledge by employing pedagogical strategies such as recommending exercises~\cite{deng2023towards} or selecting teaching examples~\cite{ross-andreas-2024-toward}. 
However, these approaches fall short in broader situations requiring both understanding and practical application of specific pieces of knowledge to solve real-world, goal-driven problems. 
Such scenarios demand tutors to proactively guide people toward completing targeted tasks (e.g., coding).
Furthermore, the tutoring outcomes are challenging to assess since targeted tasks can often be completed by open-ended solutions.



To bridge this gap, we introduce \textbf{coding tutoring}, a promising yet underexplored task for LLM agents.
As illustrated in Figure~\ref{fig:example}, the tutor is provided with a target coding task and task-specific knowledge (e.g., cross-file dependencies and reference solutions), while the student is given only the coding task. The tutor does not know the student's prior knowledge about the task.
Coding tutoring requires the tutor to proactively guide the student toward completing the target task through dialogue.
This is inherently a goal-oriented process where tutors guide students using task-specific knowledge to achieve predefined objectives. 
Effective tutoring requires personalization, as tutors must adapt their guidance and communication style to students with varying levels of prior knowledge. 


Developing effective tutoring agents is challenging because off-the-shelf LLMs lack grounding to task-specific knowledge and interaction context.
Specifically, tutoring requires \textit{epistemic grounding}~\citep{tsai2016concept}, where domain expertise and assessment can vary significantly, and \textit{communicative grounding}~\citep{chai2018language}, necessary for proactively adapting communications to students' current knowledge.
To address these challenges, we propose the \textbf{Tra}ce-and-\textbf{Ver}ify (\textbf{\model}) agent workflow for building effective LLM-powered coding tutors. 
Leveraging knowledge tracing (KT)~\citep{corbett1994knowledge,scarlatos2024exploring}, \model explicitly estimates a student's knowledge state at each turn, which drives the tutor agents to adapt their language to fill the gaps in task-specific knowledge during utterance generation. 
Drawing inspiration from value-guided search mechanisms~\citep{lightman2023let,wang2024math,zhang2024rest}, \model incorporates a turn-by-turn reward model as a verifier to rank candidate utterances. 
By sampling more candidate tutor utterances during inference (see Figure~\ref{fig:scale}), \model ensures the selection of optimal utterances that prioritize goal-driven guidance and advance the tutoring progression effectively. 
Furthermore, we present \textbf{Di}alogue for \textbf{C}oding \textbf{T}utoring (\textbf{\eval}), an automatic protocol designed to assess the performance of tutoring agents. 
\eval employs code generation tests and simulated students with varying levels of programming expertise for evaluation. While human evaluation remains the gold standard for assessing tutoring agents, its reliance on time-intensive and costly processes often hinders rapid iteration during development. 
By leveraging simulated students, \eval serves as an efficient and scalable proxy, enabling reproducible assessments and accelerated agent improvement prior to final human validation. 



Through extensive experiments, we show that agents developed by \model consistently demonstrate higher success rates in guiding students to complete target coding tasks compared to baseline methods. We present detailed ablation studies, human evaluations, and an inference time scaling analysis, highlighting the transferability and scalability of our tutoring agent workflow.

\section{Background}\label{sec:relwork}

\subsection{Limitations of machine learning approaches to detect AI-generated images}

Machine learning models for detecting AI-generated images are brittle and lack robustness to simple data transformations. Corvi et al.~\cite{corvi2023intriguingpropertiessyntheticimages} compare four different machine learning approaches to deepfake detection and demonstrate that recropping and compression – simple modifications common on social media – lead to drops in accuracy such that the classifiers are nearly just as good as random guessing. Dong et al.~\cite{9879575}reveal the ease with which spectral artifacts used in the identification of GAN-generated images can be mitigated via blurring and resizing, demonstrating a noticeable decrease in accuracy under basic modifications. Cozzolino et al.~\cite{cozzolino2024raisingbaraigeneratedimage} demonstrate that post--processing images by random--cropping, resizing, and compression lead to a drop in AI-generated image detection from 90\% accuracy to 85\% accuracy. The fundamental problem is that machine learning models for deepfake detection lack robustness to context shift, out--of--distribution data and adversarial perturbations ~\cite{wang2023deepfakedetectioncomprehensivestudy, ha2024, groh2022identifying, hulzebosch2020detectingcnngeneratedfacialimages}. 

How an image is generated influences the ability of deepfake detection classifiers to accurately identify it as AI-generated. Classifiers trained to detect GAN-generated images tend to fail to detect diffusion model-generated images. For example, the approach to detecting GAN-generated images based on frequency spectra~\cite{marra2018gansleaveartificialfingerprints, yu2019attributingfakeimagesgans, 9035107, durall2020watchupconvolutioncnnbased, bi2023detectinggeneratedimagesreal, pmlr-v119-frank20a} and inconsistencies in head poses and facial landmark positions~\cite{yang2018exposingdeepfakesusing, yang2019exposinggansynthesizedfacesusing, Mundra_2023_CVPR}, do not generalize to detecting images generated by diffusion models~\cite{ojha2024universalfakeimagedetectors}. GAN-trained detection models miss these patterns because they have learned patterns for identifying GAN-generated images~\cite{wang2020cnn, ricker2024detectiondiffusionmodeldeepfakes}. Likewise, it is possible to learn the statistical regularities in diffusion model-generated images but these regularities are not invariant to image post-processing.~\cite{xi2023aigeneratedimagedetectionusing, 10334046, wang2023dirediffusiongeneratedimagedetection, ma2023exposingfakeeffectivediffusiongenerated, yang2023diffusion}. 

Moreover, machine learning models' lack of robustness for detection is exacerbated by the changing architectures of generative AI models~\cite{lin2024, Mirsky2021}. Vision transformers~\cite{radford2021learning, ojha2024universalfakeimagedetectors} and multi--architecture training~\cite{epstein2023onlinedetectionaigeneratedimages, porcile2024findingaigeneratedfaceswild, jia2024can} show promise for enhancing the detection of AI-generated images, but adversarial attacks and large architectural changes in generative models continue to affect robustness of detection.  

Figure~\ref{fig:AI-faces} highlights the increasing complexity of AI-generated images over the past decade. The changing architectures and increasing photorealism pose a challenge for both humans and machines to distinguish real from AI-generated images. However, humans and machines are fundamentally different. For example, humans can critically reason about an image's elements and its context~\cite{wang2023context}. On the other hand, machine learning classifiers for detecting AI-generated images often oversimplify image authenticity as a question of real versus fake and ignore the critical reasoning about component parts and sub--questions that an ordinary person or digital forensics expert may consider when evaluating an image's authenticity~\cite{jacobsen2024deepfakes}.


\begin{figure*}[h]
\centering
\captionsetup{justification=raggedright, singlelinecheck=false, skip=2pt}
\begin{subfigure}[t]{0.13\textwidth}
    \subcaption{}\vtop{\vskip0pt\hbox{\includegraphics[width=\linewidth]{sections/images/ganfacesgoodfellow.jpg}}}
    \caption*{\footnotesize GAN '14}
\end{subfigure}
\hfill
\begin{subfigure}[t]{0.13\textwidth}
    \subcaption{}\vtop{\vskip0pt\hbox{\includegraphics[width=\linewidth]{sections/images/dcgan.jpg}}}
    \caption*{\footnotesize DCGAN '15}
\end{subfigure}
\hfill
\begin{subfigure}[t]{0.13\textwidth}
    \subcaption{}\vtop{\vskip0pt\hbox{\includegraphics[width=\linewidth]{sections/images/PGGAN.jpg}}}
    \caption*{\footnotesize PGGAN '18}
\end{subfigure}
\hfill
\begin{subfigure}[t]{0.13\textwidth}
    \subcaption{}\vtop{\vskip0pt\hbox{\includegraphics[width=\linewidth]{sections/images/stylegan.jpg}}}
    \caption*{\footnotesize StyleGAN '19}
\end{subfigure}
\hfill
\begin{subfigure}[t]{0.13\textwidth}
    \subcaption{}\vtop{\vskip0pt\hbox{\includegraphics[width=\linewidth]{sections/images/stylegan2.jpg}}}
    \caption*{\footnotesize StyleGAN2 '20}
\end{subfigure}
\hfill
\begin{subfigure}[t]{0.13\textwidth}
    \subcaption{}\vtop{\vskip0pt\hbox{\includegraphics[width=\linewidth]{sections/images/sdxl.jpg}}}
    \caption*{\footnotesize SDXL '23}
\end{subfigure}
\hfill
\begin{subfigure}[t]{0.13\textwidth}
    \subcaption{}\vtop{\vskip0pt\hbox{\includegraphics[width=\linewidth]{sections/images/ff_portrait3_021.jpeg}}}
    \caption*{\footnotesize Firefly '24}
\end{subfigure}
\caption{\mybold{Exemplar images of photorealism across a range of generative models.} \normalfont{Examples of AI-generated images from 2014 to 2024~\cite{goodfellow2014generativeadversarialnetworks,radford2015unsupervised, faceimageforgery,karras2018progressivegrowinggansimproved,karras2019style,karras2020analyzingimprovingimagequality, podell2024sdxl,adobe_firefly}.}}
\label{fig:AI-faces}
\Description{Portrait images from various image generation models that improve in quality and complexity over the years.}
\end{figure*}


\subsection{Human perception and evaluation of AI-generated media}

In response to the increasing realism of AI-generated media, researchers have been examining the degree to which humans can distinguish between authentic and AI-generated media. For example, researchers found that GAN-generated images of faces are indistinguishable from real face portraits~\cite{nightingale2022ai, Lago_2022}. However, for video deepfakes, humans are much better than random guessing~\cite{deepfakedetectionbyhumancrowds}, which may in part be due to humans' specialized ability to process the temporal elements of faces~\cite{deepfakedetectionbyhumancrowds, sinha2006face}. Researchers found that text--to--speech voices were rated as lower in quality and clarity than human voices in 2020~\cite{cambre2020choice} but have reached the point where research participants cannot tell the difference between short 20-second recordings of AI-generated voices and authentically recorded voices~\cite{barrington2024people}.

Recent research has identified specific cues and heuristics that people use to evaluate AI-generated media. For example, cues such as recording settings in the detection of text-to-speech audio~\cite{han2024uncovering} and speaking patterns in political deepfake videos~\cite{groh2024human}. However, two studies found that participants rarely attributed their judgments to specific visual features~\cite{hameleers2024they, wohler2021towards}, and in one of these deepfake studies, researchers found that participants are noticing the artifacts but rarely linking these to manipulation~\cite{wohler2021towards}. With respect to AI-generated text, research has highlighted that people tend to use flawed heuristics when attempting to distinguish AI-generated text from human--written text, like associating grammatical errors with AI-generation~\cite{Jakesch2022HumanHF}. 

Social context also plays a significant role in both what diffusion models generate~\cite{luccioni2024stable} and how people form beliefs about AI-generated images and their content. For example, researchers have found detection ability is influenced by shared identity between the viewer and subject of the content~\cite{mink2024s}. Furthermore, researchers have found that white AI-generated faces were disproportionately judged as human more frequently than their real counterparts~\cite{miller2023ai}. GAN-generated faces in portrait images were often perceived as more trustworthy than real faces~\cite{nightingale2022ai}, and as a result, people were less likely to question their authenticity~\cite{Lago_2022}. In instances where AI-generated images are linked to misinformation, researchers find that labeling AI-generated images and the associated content as ``potentially misleading" instead of simply ``AI-generated" had a stronger influence on curtailing participants' self--reported intentions to share misinformation~\cite{epstein2023label, wittenberg2024labeling}.

Researchers have approached a number of methods for measuring photorealism perceived by humans. For example, prior research has examined photorealism with carefully worded questions such as ``Is the image photorealistic?''~\cite{liang2024rich}, ``Does the image look like a real photo or an AI-generated image?''~\cite{lee2024holistic, otani2023toward} and ``Whether this image could be taken with a camera?''\cite{yan2024sanitycheckaigeneratedimage}. These questions are useful for assessing participants' subjective opinions but do not capture the human ability to distinguish real images from fake images and can potentially suffer from demand characteristic bias. Another approach has been to characterize photorealism by examining the features that can influence realism, such as aesthetics and semantically meaningful content of an image~\cite{peng2024crafting}. A third approach involves simply defining images as photorealistic if they are rendered with computer graphics software~\cite{lyu2005realistic}. In this paper, we approach photorealism from the psychophysics perspective, examining participants' objective performance at distinguishing real images from fake images~\cite{zhou2019hype}. 

\subsection{Categorizing artifacts and implausibilities in diffusion model-generated images} \label{sec:artifimpl}

Previous research on earlier versions of diffusion models categorized the kinds of qualitative failures of diffusion model-generated images as distorted body parts, impossible geometry, physics violations, illogical relationships in a scene, and noise~\cite{borji2023qualitative}. In addition to obvious issues with hands, feet, eyes, and teeth, research at the intersection of digital foresnics and AI-generated images shows details such as corneal reflections~\cite{hu2021exposing} and irregular pupil shapes~\cite{guo2022eyes} can also be artifacts.  Likewise, violations of physics like implausible shadows, lighting, and perspective errors~\cite{farid2022perspectiveinconsistencypainttext, farid2022lighting, sarkar2024shadows} often occur in diffusion model generated images that otherwise appear photorealistic. 
\section{Synthesizing Attribution Data}

\begin{figure*}[ht]
    \centering
    \includegraphics[width=\textwidth]{img/pipeline.drawio.pdf}
    \caption{\textbf{Top:} The \synatt baseline method for synthetic attribution data generation. Given context and question-answer pairs, we prompt an LLM to identify supporting sentences, which are then used to train a smaller attribution model. However, this discriminative approach may yield noisy training data as LLMs are less suited for classification tasks (see \S\ref{sec:experiments-zero-shot}). \textbf{Bottom:} The \synqa data generation pipeline leverages LLMs' generative strengths through four steps: (1) collection of Wikipedia articles as source data; (2) extraction of context attributions by creating chains of sentences that form hops between articles; (3) generation of QA pairs by prompting an LLM with only these context attribution sentences; (4) compilation of the final training samples, each containing the generated QA pair, its context attributions, and the original articles enriched with related distractors.}
    % \caption{\textbf{Top:} The \synatt baseline. Intuitively, we can prompt an LLM for context-attribution by providing the context and question-answer pairs. Then, we train a smaller model on the obtained synthetic data. However, LLMs are less suitable for discriminative (i.e., classification) tasks, and may yield noisy training data (see \S\ref{sec:experiments-zero-shot}). \textbf{Bottom:} The \synqa data generation pipeline consists of four main steps: (1) collection of Wikipedia articles as the source data; (2) extracting the context attributions by creating chains of sentences that form hops between articles; (3) generation of QA pairs by prompting an LLM with only the context attribution sentences; (4) we obtain the resulting \synqa training sample containing three components: the generated QA pair, the context attributions, and the original articles supplemented with related distractor articles.}
    \label{fig:method}
\end{figure*}

Context attribution identifies which parts of a reference text support a given question-answer pair~\cite{rashkin2023measuring}. Formally, given a question $q$, its answer $a$, and a context text $c$ consisting of sentences ${s_1, ..., s_n}$, the task is to identify the subset of sentences $S \subseteq c$ that fully support the answer $a$ to question $q$. To train efficient attribution models without requiring expensive human annotations, we explore synthetic data generation approaches using LLMs.
% Context attribution poses the following question~\cite{rashkin2023measuring}: given a generated text $t_g$ and a context text $t_c$, is $t_g$ attributable to $t_c$? To train models to perform well on this task, we explore how to best generate synthetic attribution data using LLMs. We implement two methods: a discriminative and generative method. 
We implement two methods for synthetic data generation. Our baseline method (\synatt) is discriminative: given existing question-answer pairs and their context, an LLM identifies supporting sentences, which are then used to train a smaller attribution model. Our proposed method (\synqa) takes a generative approach: given selected context sentences, an LLM generates question-answer pairs that are fully supported by these sentences. This approach better leverages LLMs' natural strengths in text generation while ensuring clear attribution paths in the synthetic training data.

%The first method is relatively straightforward and termed \synatt. A simple way to generate synthetic data for context attribution is to ask an LLM to pick out the sentences that support a given question-answer pair. 

% \subsection{Discriminative and Generative Synthetic Data Generation}

% The first method (\synatt) is relatively straightforward: ask the LLM to pick relevant sentences from a provided context that support a given question-answer pair. However, this \textit{discriminative} approach of performing sentence classification overlooks the fact that LLMs excel at \textit{generating} text. Therefore, we design a second data generation method (\synqa) that is generative and thus capitalizes on the strength of LLMs. It involves the following pipeline steps (see also Fig.~\ref{fig:method}): context collection, question-answering generation and distractor mining, which increases the difficulty of the task, thus reflecting more realistic scenarios.

%\textbf{Attribution Synthesis.} The most straightforward approach to generating synthetic data for context attribution is discriminative: prompting an LLM to identify relevant sentences from context documents given a question-answer pair. While intuitive, this approach underutilizes LLMs' capabilities, as they excel at generative rather than discriminative tasks. LLMs are fundamentally designed to generate coherent text following instructions rather than perform binary classification of sentences. In our experiments (\S\ref{sec:experiments}) we dub this method as \synatt.

\subsection{\synqa: Generative Synthetic Data Generation Method}

\synqa consists of three parts: context selection, QA generation, and distractors mining (for an illustration of the method, see Figure~\ref{fig:method}). In what follows, we describe each part in detail.

\textbf{Context Collection.} We use Wikipedia as our data source, as each article consists of sentences about a coherent and connected topic, with two collection strategies. In the first, we select individual Wikipedia articles for dialogue-centric generation and use their sentences as context. In the second, for multi-hop reasoning, we identify sentences containing Wikipedia links and follow these links to create ``hops'' between articles, limiting to a maximum of two paths to maintain semantic coherence, while enabling more complex reasoning patterns (for more details, see Appendix~\ref{app:synthetic_data}).
% \textbf{Context Collection.}  The first step is to select a dataset where each data point is a set of sentences about a coherent and connected topic. These sentences will serve as the context in which we want to find relevant attributions later. We use Wikipedia as the data source
%To better leverage LLMs' generative capabilities, we propose \synqa, a novel and simple approach for synthesizing context attribution data (see Fig.~\ref{fig:method}). 
%We first collect Wikipedia articles that are not present in our testing datasets\footnote{We detect potential data leakage by representing each Wikipedia article as a MinHash signature. Then, for each training Wikipedia article, we retrieve candidates from the testing datasets via Locality Sensitivity Hashing and compute their Jaccard similarity \cite{dasgupta2011fast}. Pairs exceeding a tunable threshold (empirically set to 0.8) are flagged as potential leaks.}.
%For each article, 
% we implement two distinct collection strategies that differ in difficulty. First, we select individual Wikipedia articles and randomly select multiple sentences within each article. Second, we start from a randomly selected sentence containing at least one Wikipedia link
%\footnote{These are human annotated in the Wikipedia articles, or alternatively, can be obtained from entity linking methods \cite{de-cao-etal-2022-multilingual}.} 
% and follow the links to other articles, creating ``hops'' between related content. We limit the chain to a maximum of two hops (connecting up to three articles) to maintain semantic coherence while enabling the more difficult multi-hop reasoning scenarios (for more details, see Appendix~\ref{app:synthetic_data}). 
%In the second strategy, we select individual Wikipedia articles and randomly select multiple sentences within each article that can serve as evidence for generated questions.

\textbf{Question-Answer Generation.} Given the set of contexts, an LLM can now generate question-answer pairs. For single articles, we prompt the model to generate multiple question-answer pairs, each grounded in specific sentences. This creates a set of dialogue-centric samples where questions build upon the previous context. For linked articles, we prompt the model to generate questions that necessitate connecting information across the articles, encouraging multi-hop reasoning.
%\footnote{Note that multi-hop reasoning is not guranteed here; rather, the LLM has the ability to decide whether the question-answer pair involves multiple hops of reasoning. See App. for details.}. 
This yields multi-hop samples requiring integration of information across documents, as well as samples that mimic a dialogue about a specific topic given the context. We provide the full prompts used for generation in Appendix \ref{app:prompts}.

\textbf{Distractors Mining.} To make the attribution task more realistic, we augment each sample with distractor articles. With E5 \cite{wang2022text}, we embed each Wikipedia article in our collection. For each article in the training sample, we randomly select up to three distractors with the highest semantic similarity to the source articles. These distractors share thematic elements with the source articles, but lack information to answer the questions.%do not contain the information necessary to answer the generated questions.

\subsection{Advantages of \synqa}
The \synqa approach has three key advantages:
%over discriminative data generation:
% (1) it leverages LLMs' natural strength in generative tasks; (2) produces diverse multi-hop reasoning scenarios; and (3) creates coherent question-answer pairs with clear attribution paths.
(1) it leverages LLMs' strength in generation rather than classification; (2) creates diverse training samples requiring both dialogue understanding and multi-hop reasoning; and (3) ensures generated questions have clear attribution paths since they are derived from specific context sentences.
By generating both entity-centric and dialogue-centric samples, \synqa produces training data that reflects the variety of real-world QA scenarios, helping models develop robust attribution capabilities, which our experiments demonstrate to generalize across different contexts and domains.
% We formalize the problem of Context Attribution QA as follows: Given a pre-defined context $T_c=\lbrace s_1, s_2, \ldots , s_n \rbrace$---where $s_i$ is a sentence---and an answer text $t_a$ generated by an LLM, the context attribution model should provide a vector $a=(a_1, \ldots , a_n)$, where each element $a_i$ has the following possible values:
% \[
% a_i =
% \begin{cases}
%     1, & \text{if } s_i \text{ supports the generated answer } t_a\\
%     0,  & \text{otherwise} 
% \end{cases}
% \]
% In our setup, we should have at least one entry $a_i = 1$.
% \begin{itemize}
%     \item The simplest way to generate synthetic data for context-attribution is in a discriminative manner: we prompt an LLM to provide the sentence level context attributions given the context documents, question and answer. We deem this generation as discriminative as the model effectively classifies the sentences that are most relevant to the question-answer pair.
%     \item The issue with this approach is that LLM are not best suitable for discriminative tasks, but rather generative. That is, an LLM is better at generating text by following instructions, than classifing sentences/etc.
%     \item To leverage what LLMs are good for, we create a simple context attribution data generation approach where we perform the following: (1) We find wikipedia articles (which are not contained in the testing datasets)\footnote{Describe the approach for dealing with data leakage}; (2) We select a random sentence in a wikipedia article, and find the links to other wikipedia articles (the hops). We select that sentence, and hop to the other Wikipedia article (given by the link). (3) We perform the hop step for maximum of 2 times (i.e., we connect at most 3 articles, and 1 at least). We end up with 3 Wikipedia articles which constitute the hops.
%     \item We provide Llama70B with either 1 wikipedia article or the hops and ask the model to generate a multi-hop question-answer pair which ideally connects all connected articles, or as many as it can; alternatively, if we provide the model with only 1 wikipedia article, we ask the model to select as many sentences as possible in the article, and for each, generate a question-answer pair (we provide the full prompts we use in Appendix).
%     \item The output of the model is a set of question-answer pairs (or a single one), that is grounded in the evidence provided by the sentence(s). We dub the entire approach as \synqa.
%     \item In summary, we develop two settings to generate synthetic data for context attribution in question answering: one is entity-centric and yield data which might be multi-hop; and the other is dialog-centric where subsequent questions build on top of previous ones.
%     \item Finally, to all context + question + answer + context-attribution samples we add distractors: we obtain embeddings using E5 of each wikipedia page, and for each sample we select up to 3 distractors which we add to the data sample. These distractors are similar are document with similar context as the one from which the context-attributions are.
% \end{itemize}


\label{evaluation-results}
% \setlength{\tabcolsep}{4.6pt}
% \begin{table*}[t]
% \centering
% \footnotesize
% \begin{tabular}{rcccccc}
% \toprule
%                                & \multicolumn{2}{c}{\textbf{DDxPlus}} & \multicolumn{2}{c}{\textbf{iCraft-MD}} & \multicolumn{2}{c}{\textbf{RareBench}} \\ \cmidrule(lr){2-3} \cmidrule(lr){4-5} \cmidrule(lr){6-7}
%                                & \textbf{GTPA@1 $\uparrow$}          & \textbf{Avg Rank $\downarrow$}   & \textbf{GTPA@1 $\uparrow$}       & \textbf{Avg Rank $\downarrow$}       & \textbf{GTPA@1 $\uparrow$}        & \textbf{Avg Rank $\downarrow$}       \\\midrule
%                                & \multicolumn{6}{c}{\textbf{GPT-4o}}                                                                 \\\midrule
% \textcolor{cyan}{Zero-shot}                      &                &            &             &                &              &                \\
% \textcolor{cyan}{Few-shot (Standard, Dyn\_BAII)} &                &            &             &                &              &                \\
% \textcolor{cyan}{Few-shot (CoT, Dyn\_BAII)}      &                &            &             &                &              &                \\
% History Taking (\textit{n}=5)         & 0.45           & 4.13       & 0.40        & 5.58           & 0.11         & 7.84           \\
% %History Taking (\textit{n}=10)        & 0.59           & 3.16       & 0.45        & 5.35           & 0.24         & 6.67           \\
% History Taking (\textit{n}=15)        & 0.69           & 2.47       & 0.46        & 5.23           & 0.36         & 5.49           \\
% Retrieval (PubMed) \textcolor{red}{rerun/ignore?}                   & 0.69           & 2.27       & 0.68        & 3.23           & 0.45         & 3.92           \\
% MEDDxAgent (\textbf{Ours})         &                &            &             &                &              &                \\
% \textit{iter} =  1                       & 0.74           & 1.91       & 0.52        & 4.93           & 0.51         & 4.37           \\
% \textit{iter} =  2                       & 0.78           & 1.56       & \textbf{0.54}        & \textbf{4.71}           & \textbf{0.56}         & 4.10           \\
% \textit{iter} =  3                       & \textbf{0.86}           & \textbf{1.29}       & \textbf{0.54}        & 4.80           & 0.50         & \textbf{4.09}           \\\midrule
%                                & \multicolumn{6}{c}{\textbf{Llama3.1-70B}}                                                           \\ \midrule
% \textcolor{cyan}{Zero-shot}                      &                &            &             &                &              &                \\
% \textcolor{cyan}{Few-shot (Standard, Dyn\_BAII)} &                &            &             &                &              &                \\
% \textcolor{cyan}{Few-shot (CoT, Dyn\_BAII)}      &                &            &             &                &              &                \\
% History Taking (\textit{n}=5)         & 0.45           & 4.15       & 0.29        & 6.48           & 0.30         & 6.04           \\
% %History Taking (\textit{n}=10)        & 0.58           & 3.12       & 0.33        & 5.82           & 0.36         & 4.51           \\
% History Taking (\textit{n}=15)        & 0.56           & 3.50       & 0.36        & 5.36           & 0.31         & 4.80           \\
% Retrieval (PubMed)  \textcolor{red}{rerun/ignore?}                 & 0.56           & 3.42       & 0.44        & 4.72           & 0.38         & 3.96           \\
% MEDDxAgent (\textbf{Ours})         &                &            &             &                &              &                \\
% \textit{iter} =  1                       & 0.61           & 2.91       & 0.29        & 7.05           & 0.39         & 5.05           \\
% \textit{iter} =  2                       & \textbf{0.71}   & \textbf{2.20}       & 0.37        & \textbf{6.26}           & \textbf{0.48}         & 4.48           \\
% \textit{iter} =  3                       & 0.68   & 2.30       & \textbf{0.42}        & 6.31           & \textbf{0.48}         & \textbf{4.30}           \\\midrule
%                                & \multicolumn{6}{c}{\textbf{Llama3.1-8B}}                                                            \\\midrule
% \textcolor{cyan}{Zero-shot}                      &                &            &             &                &              &                \\
% \textcolor{cyan}{Few-shot (Standard, Dyn\_BAII)} &                &            &             &                &              &                \\
% \textcolor{cyan}{Few-shot (CoT, Dyn\_BAII)}     &                &            &             &                &              &                \\
% History Taking (\textit{n}=5)         & 0.23           & 6.85       & 0.10        & 8.78           & 0.05         & 8.38           \\
% %History Taking (\textit{n}=10)        & 0.35           & 5.46       & 0.12        & 8.39           & \textbf{0.13}         & 8.25           \\
% History Taking (\textit{n}=15)        & 0.40           & 5.44       & 0.11        & 8.30           & \textbf{0.11}        & 8.95           \\
% Retrieval (PubMed)  \textcolor{red}{rerun/ignore?}                 & 0.42           & 4.50       & 0.29        & 6.93           & 0.35         & 5.33           \\
% MEDDxAgent (\textbf{Ours})         &                &            &             &                &              &                \\
% \textit{iter} =  1                       & 0.34           & 5.25       & 0.11        & 9.38           & 0.08         & 8.47           \\
% \textit{iter} =  2                       & 0.56           & 3.59       & \textbf{0.14}        & 9.22           & 0.09         & \textbf{8.11}           \\
% \textit{iter} =  3                       & \textbf{0.58}           & \textbf{3.10}       & 0.12        & \textbf{9.07}           & 0.07         & 8.56        \\  
% \bottomrule
%     \end{tabular}
%     \caption{Iterative experiment performance across 3 datasets. \textcolor{red}{The \textbf{best results} are based on ignoring the Pubmed retrieval results!}}
%     \label{tab:iterative_overall}
% \end{table*}

\setlength{\tabcolsep}{3.8pt}
\begin{table*}[ht]
\centering
\scriptsize
\begin{tabular}{rccccccccc}
\toprule
                               & \multicolumn{3}{c}{\textbf{DDxPlus}} & \multicolumn{3}{c}{\textbf{iCraft-MD}} & \multicolumn{3}{c}{\textbf{RareBench}} \\ \cmidrule(lr){2-4} \cmidrule(lr){5-7} \cmidrule(lr){8-10}
                               & \textbf{GTPA@1 $\uparrow$}          & \textbf{Avg Rank $\downarrow$}   & \textbf{$\Delta$ Progress} & \textbf{GTPA@1 $\uparrow$}       & \textbf{Avg Rank $\downarrow$}     & \textbf{$\Delta$ Progress}   & \textbf{GTPA@1 $\uparrow$}        & \textbf{Avg Rank $\downarrow$}   & \textbf{$\Delta$ Progress}     \\\midrule
                               & \multicolumn{9}{c}{\textbf{GPT-4o}}                                                                 \\\midrule
%\textcolor{cyan}{Zero-shot}                      &     0.69           &    2.21        &      -      &       0.68         &     3.37         &         -       &       0.46       & 3.99            &   -              \\
%\textcolor{cyan}{Zero-shot (CoT)}                      &     0.71          &    2.10        &      -      &       0.68         &     3.35         &         -       &       0.47       & 4.02            &   -              \\
%\textcolor{cyan}{Few-shot (CoT, Dyn\_BAII)} &                &            &     -        &                &              &          -      &              &             &            -     \\
%\textcolor{cyan}{Few-shot (CoT, Dyn\_BERT/Dyn\_BAII)}      &                &            &       -      &                &              &          -      &              &             &           -      \\
%\textit{Single-Turn}      &                &            &       -      &                &              &          -      &              &             &           -      \\
KR (\textit{n}=0)                &      0.18      & 7.33  &  -  &   0.15      &    8.27     & -  &       0.07   &  9.07  &    -   \\
DS (\textit{n}=0)     &  0.27    &    6.01        &       -      &      0.18          &      7.87        &          -      &       0.11       &     8.38        &           -      \\
%SDS (\textit{n}=5)         & 0.45           & 4.13     & - & 0.40        & 5.58     &    -  & 0.11         & 7.84     &  -    \\
KR (\textit{n}=5)  &      0.52      & 3.32   &  -  &  0.49  &  5.36       & -  &     0.40   &   5.27 &    -   \\
DS (\textit{n}=5)  &    0.72       &  2.14 &  -  &  0.40 &    5.55   & -  &   0.50    &   4.94 &    -   \\\cmidrule(lr){2-10}
%History Taking (\textit{n}=10)        & 0.59           & 3.16    & -  & 0.45        & 5.35        & -  & 0.24         & 6.67      &  -   \\
%SDS (\textit{n}=15)        & 0.69           & 2.47     & - & 0.46        & 5.23      &  -   & 0.36         & 5.49      &    - \\\cmidrule(lr){2-10}

%Retrieval (Wiki) \textcolor{red}{rerun}                   &            &    &  -  &         &         & -  &          &    &    -   \\
%MEDDxAgent         &                &            &             &                &              &               &              &             &                  \\
 MEDDx (\textit{iter}=1, \textit{n}=5)                       & 0.74           & 1.91     & ~~0.00 & 0.52        & 4.93      &  ~~0.00   & 0.51         & 4.37        &   ~~0.00\\
MEDDx (\textit{iter}=2, \textit{n}=10)                       & 0.78           & 1.56    & +0.32  & \textbf{0.54}        & \textbf{4.71}    &    +0.26   & \textbf{0.56}         & 4.10   &     +0.13   \\
MEDDx (\textit{iter}=3, \textit{n}=15)                       & \textbf{0.86}           & \textbf{1.29}    & +0.32  & \textbf{0.54}        & 4.80      & +0.17    & 0.50         & \textbf{4.09}       &   +0.16 \\\midrule
                               & \multicolumn{9}{c}{\textbf{Llama3.1-70B}}                                                           \\ \midrule
%\textcolor{cyan}{Zero-shot}                      &      0.54          &     3.53       &       -      &     0.40           &       4.87       &         -      &      0.39        &    4.05         &         -         \\
%\textcolor{cyan}{Zero-shot (CoT)}                      &     0.45          &    3.69       &      -      &       0.48         &     4.50         &         -       &       0.49       & 3.91            &   -              \\
%\textcolor{cyan}{Few-shot (Standard, Dyn\_BAII)} &                &            &      -       &                &              &          -      &              &             &        -         \\
%\textcolor{cyan}{Few-shot (CoT, Dyn\_BERT/Dyn\_BAII)}      &                &            &      -       &                &              &          -    &              &             &                -   \\
KR (\textit{n}=0)           &   0.19         &  7.58  &  -  &      0.13   &   8.19      & -  &    0.09      &  9.13  &    -   \\
DS (\textit{n}=0)    &        0.17        &       7.28     &       -      &      0.11          &      8.74        &          -      &       0.20       &      6.81       &           -      \\
%History Taking (\textit{n}=5)         & 0.45           & 4.15    &  - & 0.29        & 6.48   &     -   & 0.30         & 6.04      &   -  \\
KR (\textit{n}=5)  &      0.39      & 5.03   &  -  &  0.34  &  6.86       & -  &     0.29   &   5.86 &    -   \\
DS (\textit{n}=5)  &     0.50      &  2.89 &  -  & 0.24 &    7.33   & -  &   0.23    &  5.77  &    -   \\\cmidrule(lr){2-10}
%History Taking (\textit{n}=10)        & 0.58           & 3.12     & - & 0.33        & 5.82    &    -   & 0.36         & 4.51    &    -   \\
%History Taking (\textit{n}=15)        & 0.56           & 3.50      &- & 0.36        & \textbf{5.36}       &  -  & 0.31         & 4.80    &    -   \\\cmidrule(lr){2-10}
%Retrieval (Wiki) \textcolor{red}{rerun}                   &            &    &  -  &         &         & -  &          &    &    -   \\
%MEDDxAgent         &                &            &             &                &              &            &                &              &        \\
MEDDx (\textit{iter}=1, \textit{n}=5)                       & 0.61           & 2.91    & ~~0.00  & 0.29        & 7.05       & ~~0.00   & 0.39         & 5.05   &     ~~0.00   \\
MEDDx (\textit{iter}=2, \textit{n}=10)                       & \textbf{0.71}   & \textbf{2.20}      & +0.41 & 0.37        & \textbf{6.26}      & +0.07   & \textbf{0.48}         & 4.48  &    +0.75     \\
MEDDx (\textit{iter}=3, \textit{n}=15)                       & 0.68   & 2.30    & +0.17  & \textbf{0.42}        & 6.31     &   +0.26   & \textbf{0.48}         & \textbf{4.30}      &   +0.44  \\\midrule
                               & \multicolumn{9}{c}{\textbf{Llama3.1-8B}}                                                            \\\midrule
%\textcolor{cyan}{Zero-shot}                      &    0.45            &   9.00         &   -          &    0.27             &     7.02         &      -         &    0.33           &  5.45           &             -     \\
%\textcolor{cyan}{Zero-shot (CoT)}                      &    0.45            &   4.51         &   -          &    0.27             &     7.25         &      -         &    0.24           &  5.65           &             -     \\
%\textcolor{cyan}{Few-shot (Standard, Dyn\_BAII)} &                &            &     -        &                &              &        -      &              &             &             -      \\
%\textcolor{cyan}{Few-shot (CoT, Dyn\_BERT/Dyn\_BAII)}     &                &            &       -      &                &              &         -     &              &             &           -        \\
KR (\textit{n}=0)     &     0.20       &  7.49  &  -  &   0.11      &  \textbf{8.86}       & -  &     \textbf{0.11}     &  8.58  &    -   \\
DS (\textit{n}=0)  &       0.16         &       8.45     &       -      &      0.03          &      10.37        &          -      &         0.04     &       8.52      &           -      \\
%History Taking (\textit{n}=5)         & 0.23           & 6.85    & -  & 0.10        & 8.78        &  - & 0.05         & 8.38       &   - \\
KR (\textit{n}=5)  &      0.21      & 7.42   &  -  &  0.09  &  9.48       & -  &     0.04   &   9.69 &    -   \\
DS (\textit{n}=5)  &     0.23      &  5.77  &  -  &  0.03 &   10.08    & -  &   0.06    &  8.64  &    -   \\\cmidrule(lr){2-10}
%History Taking (\textit{n}=10)        & 0.35           & 5.46    &  - & 0.12        & 8.39     &   -   & \textbf{0.13}         & 8.25   &     -   \\
%History Taking (\textit{n}=15)        & 0.40           & 5.44   &  -  & 0.11        & \textbf{8.30}       &  -  & \textbf{0.11}        & 8.95       &  -  \\\cmidrule(lr){2-10}
%Retrieval (Wiki) \textcolor{red}{rerun}                   &            &    &  -  &         &         & -  &          &    &    -   \\
%MEDDxAgent         &                &            &             &                &              &               &                &              &     \\
MEDDx (\textit{iter}=1, \textit{n}=5)                       & 0.34           & 5.25   &   ~~0.00 & 0.11        & 9.38       &  ~~0.00  & 0.08         & 8.47    &    ~~0.00   \\
MEDDx (\textit{iter}=2, \textit{n}=10)                       & 0.56           & 3.59    & +1.73  & \textbf{0.14}        & 9.22       &  +0.22  & 0.09         & \textbf{8.11}      &  +0.44   \\
MEDDx (\textit{iter}=3, \textit{n}=15)                       & \textbf{0.58}           & \textbf{3.10}    &  +1.23 & 0.12        & 9.07     & +0.17     & 0.07         & 8.56    &  +0.38  \\  
\bottomrule
    \end{tabular}
    \vspace{-0.8em}
    \caption{Interactive experiment performance across 3 datasets without \textit{full} patient profile, with KR: knowledge retrieval agent; DS: diagnosis strategy agent; $n$ is the number of turns of the simulator; MEDDx uses KR+DS.
    %We compare the single-turn (\textit{upper}) with the proposed iterative setup for MEDDxAgent (\textit{bottom}). The selection of the agents and simulator are optimized (\autoref{subsec:optimize-agents}), unless controlled by the number of questions ($n$) asked from history taking simulator.\cc{May be we just need 3 entries of single turn for best agents and simulator compared to MEDDxAgent? For the others we leave it for ablation study?}} %\cl{why we don't have the baseline with diagnosis strategy only without full patient profile (e.g., few-shot CoT, Dyn\_BAII?)}
    }
    \label{tab:interactive_overall}
    \vspace{-1.8em}
\end{table*}

We experiment on two configurations: (1) optimizing individual agents (\autoref{subsec:optimize-agents}), by determining the best settings for knowledge retrieval and diagnosis strategy agents; and (2) interactive differential diagnosis (\autoref{subsec:iterative_learning}), where the optimized agents are used to assess MEDDxAgent's performance in the interactive DDx setup.

\subsection{Optimizing Individual Agents}
\label{subsec:optimize-agents}

We first explore the optimal single-turn configuration for the knowledge retrieval and diagnosis strategy agents, before integrating them into iterative setup. For this, we provide the full patient profile as in previous work~\cite{wu2024streambench,chen2024rarebench}, and present the results in~\autoref{tab:with_patient_profile}. For the knowledge retrieval agent, PubMed performs slightly better overall than Wikipedia, especially for Rarebench, which demands more complex disease information. For the diagnosis strategy agent, the best setting varies by dataset. 
Namely, dynamic few-shot with BAII embeddings performs the best on DDxPlus and RareBench, where relevant patient examples offer reliable contextual cues to likely diseases. 
In contrast, iCraft-MD benefits more from zero-shot CoT, which enables structured reasoning through complex clinical vignettes. Few-shot learning often decreases performance for iCraft-MD because each patient vignette is distinct, so additional examples can introduce noise.
Based on the above findings, we select the following configurations for the iterative scenario:\footnote{We do not run all possible settings in the interactive environment due to cost reasons.} PubMed for knowledge retrieval agent; few-shot (dynamic BAII) for DDxPlus and RareBench, and zero-shot (CoT) for iCraft-MD for diagnosis strategy agent.

\begin{figure*}[t]
    \centering
    \begin{subfigure}{0.48\textwidth}
    \includegraphics[trim={0.2cm 0cm 0cm 0cm },clip, width=\textwidth]{img/ddxplus_history.pdf}
    \vspace{-1.8em}
    \caption{}
    \end{subfigure}
    \begin{subfigure}{0.48\textwidth}
    \includegraphics[trim={0.2cm 0cm 0cm 0cm}, clip, width=\textwidth]{img/agent_iterations_plot_ddxplus.pdf}
    \vspace{-1.8em}
    \caption{}
    \end{subfigure}
    \vspace{-0.5em}
    \caption{Results of DDxPlus compared between (a) history taking simulator, and (b) MEDDxAgent, over the number of questions and iterations. For brevity, the results of iCraft-MD and RareBench are in~\autoref{subsec:comparison_history_taking_iterative}.}
    \label{fig:ddxplus_comparison}
    \vspace{-1.8em}
\end{figure*}

\subsection{Interactive Differential Diagnosis}
\label{subsec:interactive_differential_diagnosis}
We now evaluate the more challenging task of interactive DDx, where we begin with limited patient information and the history taking simulator enables the interactive environment~(\autoref{tab:interactive_overall}).
At $n=0$, the simulator has not yet learned any patient information, and performance drops significantly from observing the full patient profile (\autoref{tab:with_patient_profile}). 
For GPT-4o in RareBench, the knowledge retrieval agent (KR)'s GTPA@1 drops from 0.45  to 0.07. Similarly, the diagnosis strategy agent (DS) drops from 0.46 (zero-shot) to 0.11. This simple baseline showcases that previous evaluations do not hold well in the interactive setup with initially limited patient information. 
Already for $n=5$, we find a large boost in performance for both KR and DS. These findings reinforce the importance of history taking for diagnostic precision. 
We illustrate the trend for changing $n$ in~\autoref{fig:ddxplus_comparison} and find that gains also plateau around \textit{n}=10-15 questions, reinforcing the optimal balance between information gathering and diagnostic efficiency \cite{ely1999analysis}.

Finally, we run MEDDxAgent, which calls KR+DS in the \textit{fixed iteration} pipeline (\autoref{subsec:iterative_learning}). MEDDxAgent exhibits clear improvements over the KR and DS baselines for $n=5$, supporting our hypothesis that all three modules are important for interactive DDx. It also improves significantly over the history taking baselines, as we illustrate in \autoref{fig:ddxplus_comparison}. MEDDxAgent is also capable of improving upon the zero-shot setting with the full patient profile (\autoref{tab:with_patient_profile}). For DDxPlus, GTPA@1 for GPT-4o and Llama3.1-70B rise from 0.56 to 0.86 and from 0.46 to 0.71, respectively. For Llama3.1-8B, the trend continues for DDxPlus but inconsistently for iCraft-MD and RareBench, highlighting the importance of model scale. Notably, MEDDxAgent improves over successive iterations, though the optimal number of iterations (2, 3) depends on the dataset and LLM. The values of $\Delta$ are consistently positive, indicating that MEDDxAgent iteratively increases the rank of the ground-truth diagnosis over time. $\Delta$ Progress also varies by dataset and model, offering explainable insight to the diagnosistic improvement of MEDDxAgent. The overall results show that MEDDxAgent can operate well in the challenging, realistic setup of interactive DDx. Additionally, MEDDxAgent logs all intermediate reasoning, action, and observations, providing critical insight into its DDx process (\autoref{fig:Example}).
\vspace{-0.5em}
\section{Discussion}\label{sec:disc}

While diffusion models can generate highly realistic images, most of the images they produce still contain visible artifacts. In particular, we find that only 17\% of diffusion model-generated images are misclassified as real at rates consistent with random guessing. Notably, this misclassification rate increases to 43\% when the viewing duration is restricted to 1 second. By curating a dataset of 599 images and conducting a large scale digital experiment, we can begin to answer fundamental questions about what drives the appearance of photorealism in diffusion model-generated images. 

First, we find that images with greater scene complexity tend to introduce more opportunities for artifacts to appear, making it easier for participants to detect AI-generated images. Our results reveal that participants were less accurate at identifying AI-generated portraits compared to more complex scenes, such as those involving multiple people in candid settings.  Based on qualitative analysis of the images, we identify three main reasons for this difference. First, portraits often feature a single person against a blurred background, which can obscure details and provide fewer cues compared to full-body or group images. Second, portraits typically involve fewer and simpler poses, focusing only on the face and torso, leaving fewer opportunities for errors or inconsistencies to be apparent. Third, the prevalence of edited and retouched portraits in real-world photography complicates the distinction between real and AI-generated portraits, addressing the question of how subject type and context (e.g., unknown people vs. public figures) influence the perceived authenticity of an image. In contrast, more complex images, like full-body or group shots, involve a greater number of elements, increasing the likelihood of noticeable errors or inconsistencies. Similar to our results on AI-generated images, we find that real images with lower scene complexity are also harder to identify as real. 

Second, we identify five high-level categories of artifacts and implausibilities and find that the easiest images to identify as diffusion model generated are the ones with anatomical implausibilities, such as unrealistic body proportions and stylistic artifacts like overly glossy or waxy features.

Third, by randomizing display time, we identify the relationship between how long an individual looks at an image and their accuracy at distinguishing between real and AI-generated images. Specifically, we find that participants' accuracy at identifying an AI-generated image upon a quick glance of 1 second is 72\% and increases by 5 percentage points with just an additional 4 seconds of viewing time and 10 percentage points when unconstrained by time. Given the nature of rapid scrolling on social media and how much time people have to see advertisements as they pass by billboards on a highway, these results reveal the importance of attentive viewing of images before making judgments about an image's veracity. 


Fourth, we find that human curation had a notable negative impact on participants' accuracy compared to uncurated images generated by the same prompts as the human-curated AI-generated images. In particular, the images curated by our research team were harder to identify as AI-generated than 84\% of the uncurated images generated using the same prompts as the curated images. This finding reveals the limitation of state-of-the-art diffusion models in producing images of consistent quality. It also suggests that human curation is a bottleneck to generating fake images at scale. The process of generating high-quality AI images is inherently iterative---users refine prompts and select outputs until they achieve their desired result. This fundamental aspect of AI image generation is evident across all applications, from advertising and marketing to education and beyond. While concerns exist about fake images being used to mislead or impersonate, many use cases exist for business and educational applications~\cite{vartiainen2023using, hartmann2023power, gvirtz2023text}. The critical role of human curation in this iterative process further emphasizes how the photorealism of images produced by diffusion models depends not only on the capabilities of the diffusion model but also on the quality of human curation, choice of prompts, and context of the scene. Given the importance of these factors beyond the generative AI model, these results reveal the importance of considering these factors in research examining human perception of AI-generated images. Without considering these elements, it is possible to produce biased findings showing AI-generated images are more or less realistic than they really appear in real-world settings. 

The taxonomy offers a practical framework on which to build AI literacy tools for the general public. We synthesized information from diverse sources such as social media posts, scientific literature, and our online behavioral study with 50,444 participants to systematically categorize artifacts in AI-generated images. Through this process, we identify five key categories: anatomical implausibilities, which involve unlikely artifacts in individual body parts or inconsistent proportions, particularly in images with multiple people;  stylistic artifacts, referring to overly glossy, waxy, or picturesque qualities of specific elements of an image; functional implausibilities, arising from a lack of understanding of real-world mechanics and leading to objects or details that appear impossible or nonsensical; violations of physics, which include inconsistencies in shadows, reflections, and perspective that defy physical logic; and sociocultural implausibilities, focusing on scenarios that violate social norms, cultural context, or historical accuracy. Our taxonomy builds upon the Borji 2023 taxonomy \cite{borji2023qualitative} and focuses on images that appear more realistic at first glance, which is useful for comparing and contrasting real photographs with diffusion model generated images for revealing the nuances of the artifacts and implausibilities~\cite{kamali2024distinguish}. Moreover, this taxonomy offers a shared language by which practitioners and researchers can communicate about artifacts commonly seen in AI-generated images and exposes the persistent challenges that can help people identify AI-generated images. 

\subsection{Future Work and Limitations}

In addition to aiding in identifying AI-generated content, the taxonomy offers insights into the open problems for producing realistic AI-generated images. Future work may explore integrating such taxonomies into model evaluation frameworks to provide iterative feedback during the development of generative models. As models advance to address the weaknesses presented in this taxonomy, new and more subtle artifacts may emerge, requiring future updates to this taxonomy. This dynamic interplay between detection and generation capabilities demonstrates why we need to maintain robust human detection abilities even as models evolve. We acknowledge the potential dual use of these insights to create more deceptive synthetic media, and we believe that transparent documentation of artifacts does more good than harm by offering detection strategies and an opportunity to develop general awareness in the public.

Large-scale digital experiments with participants who participate based on their own interests come with certain limitations. First, we did not collect demographic data from participants. Participants were not recruited for this experiment; instead, participants found the experiment organically and participated. Given the organic nature of the participation, we prioritized maximizing engagement, which involves questions unrelated to distinguishing AI-generated and real images like demographic questions. While this approach enabled substantial data collection, it limits analysis by excluding factors like age, gender, and cultural background that may influence detection. 

Second, we provided feedback on the correct answer after each participant made an observation, which has the potential to introduce learning effects. Future research could address these open questions by collecting demographic data to design more inclusive AI literacy tools and evaluating how performance changes with and without feedback. 

This research focused on images generated by state-of-the-art generative models available in 2024, and the findings are inherently tied to the state of diffusion models and generative AI technologies as of 2024. In the future, models are likely to change, and the somewhat visible errors that emerge will also likely change. Past state-of-the-art GAN models such as StyleGAN2~\cite{karras2020analyzingimprovingimagequality} and BigGAN~\cite{brock2018biggan}, often produced more noticeable artifacts in facial features, color balance, and overall photorealism, making their outputs more easily distinguishable. Nonetheless, the current taxonomy on diffusion models points out elements like anatomical implausibilities and stylistic artifacts that can be mapped to the facial feature and color balance cues. These recurring issues offer evidence of the taxonomy’s robustness to differences across model generations, but future studies should explore how the taxonomy may need to adapt to these changes, which may involve adding or removing categories or may involve further identifying nuances within these categories. As an example of how this taxonomy may be applied to AI-generated video, Figure~\ref{fig:sora} presents an example of an anatomical implausibility that we never saw in diffusion model-generated images because it involves a temporal inconsistency. Future research on the realism of AI-generated audio and video may also consider following the three-step process involved in building this taxonomy for images generated by diffusion models. Based on first surveying AI literacy resources, academic literature, and social media, second generating media with state-of-the-art models, and third collecting empirical data on the human ability to distinguish AI-generated media from authentically recorded media, researchers can build empirical insights towards characterizing realism and categorizing the artifacts in AI-generated media.  
 
The empirical insights on the photorealism of AI-generated images and the resulting taxonomy designed to help people better navigate real and synthetic images online lead to a practical research question: How can AI literacy interventions improve people's ability to distinguish real photographs and AI-generated images? Future research may address this question via randomized experiments comparing a control group with no intervention to a treatment group that receives training based on the taxonomy presented in this paper. Likewise, future research may explore this with just-in-time interventions to direct people's attention to the cues identified in the taxonomy.
\section{Conclusion and Suggestions}

Our work, including the creation of \texttt{ScholarLens} and the proposal of \texttt{LLMetrica}, provides methods for assessing LLM penetration in scholarly writing and peer review. By incorporating diverse data types and a range of evaluation techniques, we consistently observe the growing influence of LLMs across various scholarly processes, raising concerns about the credibility of academic research. As LLMs become more integrated into scholarly workflows, it is crucial to establish strategies that ensure their responsible and ethical use, addressing both content creation and the peer review process. 

Despite existing guidelines restricting LLM-generated content in scholarly writing and peer review,\footnote{\href{https://aclrollingreview.org/acguidelines\#-task-3-checking-review-quality-and-chasing-missing-reviewers}{Area Chair} \&  \href{https://aclrollingreview.org/reviewerguidelines\#q-can-i-use-generative-ai}{Reviewer} \& \href{https://www.aclweb.org/adminwiki/index.php/ACL_Policy_on_Publication_Ethics\#Guidelines_for_Generative_Assistance_in_Authorship}{Author} guidelines.} challenges still remain. 
To address these, we propose the following based on our work and findings: 
(i) \textbf{Increase transparency in LLM usage within scholarly processes} by incorporating LLM assistance into review checklists, encouraging explicit acknowledgment of LLM support in paper acknowledgments, and 
reporting LLM usage patterns across diverse demographic groups;
% reporting LLM penetration based on social demographic features;
(ii) \textbf{Adopt policies to prevent irresponsible LLM reviewers} by establishing feedback channels for authors on LLM-generated reviews and developing fine-grained LLM detection models~\cite{abassy-etal-2024-llm, cheng2024beyond, artemova2025beemobenchmarkexperteditedmachinegenerated} to distinguish acceptable LLM roles (e.g., language improvement vs. content creation);
(iii) \textbf{Promote data-driven research in scholarly processes} by supporting the collection of review data for further robust analysis~\cite{dycke-etal-2022-yes}.\footnote{\url{https://arr-data.aclweb.org/}}

% make LLM usage transparent in scholarly processes: such as incorporating LLM usage into review checklists, encouraging explicit acknowledgment of LLM assistance in paper acknowledgments, and reporting LLM penetration based on social demographic features; (ii) Adopt policies to prevent irresponsible LLM reviewers: such as providing authors feedback on LLM-assisted reviews, and developing fine-grained LLM detection models~\cite{cheng2024beyond} to distinguish acceptable LLM roles (e.g., language improvement vs. content creation); (iii) Encourage data-driven research in scholarly processes: such as supporting review data collection for further research.

 


% 
% \section{Example Appendix}
% \label{sec:appendix}

% This is an appendix.

\section{Extracting data with \spike}\label{sec:appendix-spike}

We found two patterns of statements, which can convey a clear sentiment, and built queries upon these patterns to extract statements from \spike. Examples for all types of statements are presented in Table~\ref{tab:base-sentence}.


First, are statements in which the verb in the statements is a verb with clear sentiment, that often implies the sentiment of the entire statement. E.g., `wastes', `rejects', `fails' are negative verbs, while verbs like `enjoys', `succeeds', `empowers', conveys positive statements. 

The second pattern of statements that we found suitable for conveying a clear sentiment, are statements which describe some event/action, and its consequences, where often the adjective that describes the consequences holds information whether it is positive or negative. 

Next, we needed to label and filter them due to two main issues. First, we needed to handle the cases in which negation words appear in the statement and flips the sentiment. For example, a statement like ``We did not enjoy the show'' includes a positive verb (enjoy), but the negation flips its sentiment to be a negative statement. Another issue we encountered is that there are many statements which are irrelevant to our case, even though they match the positive/negative patterns, for example ``I couldn't sympathize with the shopping aspect of the book since I hate to shop .'' does not convey any clear sentiment, despite the use of the verb `hate'.



\begin{table*}[t]
\centering
\resizebox{\textwidth}{!}{
\begin{tabular}{ll}
\toprule
\textbf{Category} & \textbf{Example Sentence} \\
\midrule
Positive Verb & ``To my surprise I did \textbf{enjoy} the book and the characters .'' \\
Negative Verb & ``This dock has done nothing but provide frustration and \textbf{waste} a great deal of my time trying to get it to work properly .'' \\
\midrule
Positive Outcome & ``This bag provides \textbf{good} protection for my snare drum at a really \textbf{good} price .'' \\
Negative Outcome & ``For me , Aspartame causes \textbf{bad} memory loss and \textbf{nasty} gastrointestinal distress .'' \\
\bottomrule
\end{tabular}
}
\caption{Examples for base statements collected using \spike. The words that inflect the sentiment are in bold.}
\label{tab:base-sentence}

\end{table*}

\subsection{SPIKE Queries}
\begin{enumerate}
    \item :something :[{pos/neg verbs}]develops
    \item:something :[{pos/neg adjectives}]badly :[{cause synonym}]causes :something
\end{enumerate}

\subsection{Word Lists}

\paragraph{Positive verbs.} achieve, admire, affirm, appreciate, aspire, awe, bless, blossom, celebrate, cherish, comfort, contribute, delight, donate, elevate, empower, enchant, encourage, energize, engage, enjoy, enrich, enthuse, excel, fervor, flourish, fortify, glisten, glow, gratitude, grow, harmonize, heal, illuminate, innovate, inspire, invigorate, laugh, learn, liberate, love, motivate, nourish, nurture, praise, prosper, radiate, rally, refresh, rejoice, renew, revel, revere, revitalize, savor, shine, smile, soar, spark, sparkle, stimulate, strengthen, succeed, support, synergize, thrive, unite, uplift, volunteer, adore, amaze, boost, captivate, win.

\paragraph{Negative verbs.} abandon, abuse, accuse, alienate, begrudge, betray, bewilder, blame, collapse, complain, condemn, confuse, contradict, criticize, decay, deceive, decline, defeat, demoralize, deny, despair, destroy, deteriorate, devalue, discourage, discriminate, dishearten, dismantle, dismiss, dissolve, doubt, exploit, fail, falter, fear, frustrate, grieve, harass, hate, hurt, ignore, inhibit, intimidate, lose, mock, overlook, overwhelm, pollute, punish, regress, reject, repress, resent, sabotage, shatter, sicken, stifle, suffer, suffocate, suppress, terrorize, torment, undermine, violate, waste, weaken, whine, withdraw, withhold, worry.

\paragraph{Positive adjectives.}
admirable, lucky, enjoyable, magnificent, enthusiastic, marvelous, euphoric, amazing, excellent, exceptional, amused, excited, amusing, extraordinary, nice, noble, outstanding, appreciative, fabulous, overjoyed, astonishing, fantastic, benevolent, fortunate, pleasant, blissful, pleasurable, brilliant, positive, glad, prominent, good, proud, charming, cheerful, reliable, gracious, grateful, clever, great, happy, superb, superior, terrific, incredible, tremendous, inspirational, delighted, delightful, joyful, joyous, uplifting, wonderful, lovely.

\paragraph{Negative adjectives.}
sad, angry, upset, disgusting, boring, disappointing, frustrating, annoying, miserable, terrible, deppressing, unhappy, melancolic, heartbreaking, Furious, iritating, emberessing, horrible, stupid, unlucky, negative, bad.

\paragraph{``Causes'' synonym.} causes, creates, generates, prompts, produces, induces, yields, affects, invokes, effectuates, results, encourages, promotes, introduces, begets, engenders, occasions, develops, starts, contributes, initiates, inaugurates, establishes, begins, cultivates, acquires, provides, launches.



\section{Adding Framing}\label{sec:framing-prompts}

\begin{table*}[]
\resizebox{\textwidth}{!}{%
\begin{tabular}{@{}lll@{}}
\toprule
\textbf{Base Sentence} &
  \textbf{Base Sentiment} &
  \textbf{Opposite Framing Sentence} \\ \midrule
``To my surprise I did enjoy the book and the characters .'' &
  Positive &
  \begin{tabular}[c]{@{}l@{}}``To my surprise I did enjoy the book and the characters, \textbf{even though}\\ \textbf{it had a disappointing ending}. ''\end{tabular} \\ \midrule
\begin{tabular}[c]{@{}l@{}}``For me , Aspartame causes bad memory loss and nasty \\ gastrointestinal distress .''\end{tabular} &
  Negative &
  \begin{tabular}[c]{@{}l@{}}``For me, Aspartame causes bad memory loss and nasty gastrointestinal \\ distress, \textbf{but this has encouraged me to seek out healthier, natural} \\ \textbf{alternatives and cultivate a balanced diet} .''\end{tabular} \\ \bottomrule
\end{tabular}%
}
\caption{Sentences after framing. Positive sentences are added with negative framing, and vice-versa. The opposite framing is in bold.}
\label{tab:after-framing}
\end{table*}

Example for statements after framing are presented in in Table~\ref{tab:after-framing}.

\subsection{Framing Prompts}

\begin{enumerate}
    \item ``Here is an example of a base statement with a negative sentiment: I failed my math test today. Here is the same statement, after adding a positive framing: I failed my math test today, however I see it as an opportunity to learn and improve in the future. Here is a negative statement: <statement> Like the example, add a positive suffix or prefix to it. Don't change the original statement.''

    \item ``Here is an example of a base statement with a positive sentiment: I got an A on my math test. Here is the same statement, after adding a negative framing: I got an A on my math test. I think I spent too much time learning to it though. Here is a positive statement: <statement>. Like the example, add a negative suffix or prefix to it. Don't change the original statement.''
\end{enumerate}

\section{Annotation Platform}\label{sec:mturk-appendix}

We select a pool of 10 qualified workers who successfully passed our qualification test, which consisted of 20 base statements (unframed), for which annotators were expected to achieve perfect accuracy. The estimated hourly wage for the entire experiment was approximately 14USD per hour.

Screenshot of the annotation platform is presented in Figure~\ref{fig:annotation-platform}.

\begin{figure*}
    \centering
    \includegraphics[width=\linewidth]{images/annotation.png}
    \caption{Screenshot of the annotation platform.}
    \label{fig:annotation-platform}
\end{figure*}

\section{Models}\label{sec:appendix-models}

We ran the open models via together-ai API.\footnote{\url{https://www.together.ai}} 
The list of models we used are:
\begin{itemize}
    \item "google/gemma-2-9b-it"
    \item "google/gemma-2-27b-it"
    \item "mistralai/Mistral-7B-Instruct-v0.3"
    \item "mistralai/Mixtral-8x7B-Instruct-v0.1"
    \item "mistralai/Mixtral-8x22B-Instruct-v0.1"
    \item "meta-llama/Llama-3-8b-chat-hf"
    \item "meta-llama/Llama-3-70b-chat-hf"
\end{itemize}

For \gpt{}, we used the OpenAI api, with "gpt-4o-2024-08-06".\footnote{\url{https://platform.openai.com/docs/overview}}

\begin{figure*}[htbp]
    \centering
    % First Subfigure
    \begin{subfigure}{0.49\textwidth} % Adjust width as needed
        \centering
        \includegraphics[width=\textwidth]{images/orig_negative_models_distribution.png} % Replace with your image path
        \caption{Sentences that are \textbf{negative} in their original form.}
        \label{fig:negative-flip}
    \end{subfigure}
    % \hfill % Adds horizontal space between subfigures
    % Second Subfigure
    \begin{subfigure}{0.49\textwidth}
        \centering
        \includegraphics[width=\textwidth]{images/orig_positive_models_distribution.png} % Replace with your image path
        \caption{Sentences that are \textbf{positive} in their original form.}
        \label{fig:positive-flip}
    \end{subfigure}
    \caption{Proportion of sentences for which LLMs flipped sentiment, became neutral, or retained the original sentiment when presented with opposite sentiment framing. For example, this measures the percentage of sentences originally labeled as positive, that were labeled as negative after applying negative framing (and vice versa).
    }
    \label{fig:flip-proportion}
\end{figure*}
\begin{figure}
    \centering
    \includegraphics[width=\linewidth]{images/pairwise_correlation_matrix.png}
    \caption{Pairwise Pearson correlation coefficients between predictions from different models, indicating the degree of similarity in their behavior under opposite sentiment framing scenarios.}
    \label{fig:heatmap-models}
\end{figure}

\begin{figure}
    \centering
    \includegraphics[width=\linewidth]{images/humans_distribution.png}
    \caption{Proportions of sentences where annotators agreed on the extent of sentiment shift after applying opposite sentiment framing. The bars represent the percentage of sentences with 0 to 5 annotators agreeing on a sentiment shift.}
    \label{fig:humans-flip}
\end{figure}



%%
%% The acknowledgments section is defined using the "acks" environment
%% (and NOT an unnumbered section). This ensures the proper
%% identification of the section in the article metadata, and the
%% consistent spelling of the heading.
% \begin{acks}

% \end{acks}

%%
%% The next two lines define the bibliography style to be used, and
%% the bibliography file.
\bibliographystyle{ACM-Reference-Format}
\bibliography{bibliography}


%%
%% If your work has an appendix, this is the place to put it.
\appendix
%chatgpt helped me with this - needs more work 
\section{Preliminaries}\label{app:prelims}

%\tbd{the basic autoencoder eq. (my colleagues didn't know what I mean exactly when I talked about this work)}

\paragraph{1- sparse SAE probes}
To evaluate how well SAE features predict a certain abstract feature, we utilize 1-sparse probes \cite{gurnee2023finding}. Specifically, we collect activations of a specific SAE feature on a contrastive dataset containing both answerable and not answerable examples, and fit a slope coefficient and intercept to predict the dataset label using linear regression. The Gemma 2 SAEs are trained using a JumpReLU activation function \cite{lieberum2024gemma}. We can sample SAE activations after the activation function (post-relu) or before (pre-relu).
Since there are more learnt features to be found in the latter setting, the main paper figures focus on that. However, we report all results for the post-relu setting in the appendix.


\paragraph{Residual stream probes}

Our residual stream probes are trained on model activations sampled from the model's residual stream. To avoid overfitting, we train the regression model using 5-fold cross validation and perform a hyperparameter optimization by sweeping over regularization parameters with 26 logarithmically spaced steps between 0.0001 and 1. To measure the variability of residual stream probes, we repeat our analysis 10 times with different randomly sampled training datasets.

\paragraph{N-sparse SAE probes}
To train SAE probes with more than 1 feature, we follow the general methodology of our 1-sparse probes. As testing all possible SAE feature combinations is computationally infeasible, we iteratively increase the number of features while testing only the most promising candidates for higher features combinations. Specifically, to find combinations of $k$ features, we use the top 50 best performing features of size $k-1$ and test all possible new combinations with the 500 best performing single SAE features. We use a constant regularization parameter of 1 for the probes, regardless of the number of features.

\paragraph{Feature similarities}

To calculate feature similarities, we use the cosine similarity of the corresponding SAE encoder weight and the slope coefficients of the linear probes trained on the residual stream. SAE features are only compared to other SAE features of the same SAE, and residual stream probes trained at the same location in the model as the SAE. To compare how similar differently sized groups of SAE features are to the residual stream probes, we calculate the mean absolute cosine sim of the top 10 best performing SAE features of a certain group size (1 to 5) with the 10 residual stream probes trained on different training subsets.

\section{Datasets}
\label{app:datasets}

\myparagraph{Full Dataset details}

\begin{itemize}[leftmargin=*,topsep=0pt,noitemsep]
    \item \textbf{SQUAD} \citep{rajpurkar2018know}:  Dataset consisting of a short context passage and a question relating to the context. We follow the training data split and prompting template provided by \citet{slobodkin2023curious}.
    \item \textbf{IDK} \citep{sulem2021we}: Dataset with questions in the style of SQUAD, containing both answerable and unanswerable examples. We specifically use the non-competitive and unanswerable subsets of the ACE-whQA dataset.
    \item \textbf{BoolQ\_3L} 
    \citep{sulem2022yes}: Yes/no questions with answerable and unanswerable subsets.
    \item \textbf{Math Equations}: Synthetic dataset contrasting solvable equations with equations containing unknown variables.
    \item \textbf{Celebrity Recognition}: Queries requiring knowledge about celebrities.
    For construction, we use a public dataset of actors and movies from IMDB\footnote{\url{https://www.kaggle.com/datasets/darinhawley/imdb-films-by-actor-for-10k-actors}}, and generate a list of the 1000 most popular actors after 1990, as measured by the total number of ratings their movies received. We construct an additional dataset of non-celebrity names by randomly generating first and last name combinations using the most common North American names from Wikipedia\footnote{\url{https://en.wikipedia.org/wiki/Lists_of_most_common_surnames_in_North_American_countries} and \url{https://en.wikipedia.org/wiki/List_of_most_popular_given_names?utm_source=chatgpt.com}}. 
\end{itemize}

\paragraph{Dataset sizes}

\begin{table}[h]
    \centering
    \begin{tabular}{lc}
        \hline
        Dataset & Size \\
        \hline
        SQUAD (train) & 2000 \\
        BoolQ (train) & 2000 \\
        SQUAD (test) & 1800 \\
        SQUAD (variations) & 1800 \\
        BoolQ (test) & 2000 \\
        IDK & 484 \\
        Equation & 2000 \\
        Celebrity & 600 \\
        \hline
    \end{tabular}
    \caption{Number of examples for each used dataset.}
    \label{table:dataset-size}
\end{table}

Table~\ref{table:dataset-size} shows the number of examples for each dataset used in our evaluation.




\section{Additional analysis}

\subsection{Answerability Detection at Different Layers} \label{app:eval-other-layers}
\begin{figure*}[t]
    \centering
    \includegraphics[width=0.8\textwidth,trim={0 1.4cm 0 3.4cm},clip]{figures/sae_feature_accuracies_layer20_post.png}
    \caption{Answerability detection accuracies for top SAE features (Layer 20, post-activation).}
    \label{fig:sae-probe_post20}
\end{figure*}

\begin{figure*}[t]
    \centering
    \includegraphics[width=0.8\textwidth,trim={0 1.4cm 0 3.4cm},clip]{figures/sae_feature_accuracies_layer20_pre.png}
    \caption{Answerability detection accuracies for top SAE features (Layer 20, pre-activation).}
    \label{fig:sae-probe_pre20}
\end{figure*}

\begin{figure*}[t]
    \centering
    \includegraphics[width=0.8\textwidth,trim={0 1.4cm 0 3.4cm},clip]{figures/sae_feature_accuracies_layer31_post.png}
    \caption{Answerability detection accuracies for top SAE features (Layer 31, post-activation).}
    \label{fig:sae-probe_post31}
\end{figure*}

\begin{figure*}[t]
    \centering
    \includegraphics[width=0.7\textwidth,trim={0 2.2cm 1cm 3.4cm},clip]{figures/all_layers_probe.png}
    \caption{Linear probe trained on Layer 20 and Layer 31 residual stream (SQuAD) and evaluated on IDK, BoolQ, Celebrity, and Equation. The plot shows the median accuracy including the first and third quartile.}
    \label{fig:res-probe-all}
\end{figure*}

We repeat our SAE feature analysis in Layer 20 of the model, as well as providing additional analysis for SAE features activations sampled after the activation function. Figure~\ref{fig:sae-probe_pre20} shows the Layer 20 results using activations sampled before the activation function, while Figures~\ref{fig:sae-probe_post20} and \ref{fig:sae-probe_post31} show analogous results when sampling SAE activations after the activation function. Sampling after the activation reduces the number of relevant features our probe finds, since many features are inactive. However, this does not change the overall results, as we still find features with good generalization performance. 

Figure~\ref{fig:res-probe-all} shows the probing accuracy for the residual stream linear probe for both Layer 20 and 31. The evaluation is repeated across 10 seeds with different training set splits. While the SAE features, as part of the pre-trained autoencoder model, do not heavily depend on the probing dataset, this is not necessarily true for the residual stream probe. The's probe performance across the out-of-distribution datasets varies strongly, indicating that the generalization performance heavily depends on the minor differences in the training data. 

\subsection{Prompt variations}

\begin{figure*}[t]
    \centering
    \includegraphics[width=0.8\textwidth,trim={0 1.4cm 0 3.4cm},clip]{figures/sae_feature_accuracies_layer31_pre_SQUAD_train_variation.png}
    \caption{Performance of top SAE features and the residual stream linear probe on variations of prompt used with the SQuAD dataset (layer 31, pre-activation).}
    \label{fig:prompt-variation}
\end{figure*}

\begin{table*}[h]
    \centering
    \begin{tabular}{lp{10cm}}
        \toprule
        Default & Given the following passage and question, answer the question:\newline Passage: \{passage\}\newline Question: \{question\} \\
        \midrule
        Variation 1 & Please read this passage and respond to the query that follows:\newline Passage: \{passage\}\newline Question: \{question\} \\
        \midrule
        Variation 2 & Based on the text below, please address the following question:\newline Text: \{passage\}\newline Question: \{question\} \\
        \midrule
        Variation 3 & Consider the following excerpt and respond to the inquiry:\newline Excerpt: \{passage\}\newline Inquiry: \{question\} \\
        \midrule
        Variation 4 & Review this content and answer the question below:\newline Content: \{passage\}\newline Question: \{question\} \\
        \midrule
        Variation 5 & Using the information provided, respond to the following:\newline Information: \{passage\}\newline Query: \{question\} \\
        \bottomrule
    \end{tabular}
    \caption{SQuAD prompt template variations.}
    \label{table:variations}
\end{table*}

% \begin{table*}[h]
%     \centering
%     \begin{tabular}{|l|p{10cm}|}
%         \hline
%         Default & Given the following passage and question, answer the question:\newline Passage: \{passage\}\newline Question: \{question\}\\
%         \hline
%         Variation 1 & Please read this passage and respond to the query that follows:\newline Passage: \{passage\}\newline Question: \{question\} \\
%         \hline
%         Variation 2 & Based on the text below, please address the following question:\newline Text: \{passage\}\newline Question: \{question\} \\
%         \hline
%         Variation 3 & Consider the following excerpt and respond to the inquiry:\newline Excerpt: \{passage\}\newline Inquiry: \{question\} \\
%         \hline
%         Variation 4 & Review this content and answer the question below:\newline Content: \{passage\}\newline Question: \{question\} \\
%         \hline
%         Variation 5 & Using the information provided, respond to the following:\newline Information: \{passage\}\newline Query: \{question\} \\
%         \hline
%     \end{tabular}
%     \caption{SQuAD prompt template variations.}
%     \label{table:variations}
% \end{table*}

We investigated if the SAE features or the residual stream probes are sensitive to small variations in the prompt. To evaluate this question, we created five variations of the prompt template used for the SQuAD training data (see Table~\ref{table:variations}). The results can be found in Figure~\ref{fig:prompt-variation}, and indicate neither the residual stream probe nor the SAE features are sensitive to this kind of variation. 

\subsection{In-domain SAE feature accuracies}

\begin{figure*}[t]
    \centering
    \includegraphics[width=0.8\textwidth,trim={0 1.4cm 0 3.4cm},clip]{figures/sae_top_feature_accuracies_in_domain.png}
    \caption{Performance of the top SAE feature's probing accuracy when training and evaluating features on each dataset individually (pre-activation).}
    \label{fig:top-in-domain}
\end{figure*}


Figure~\ref{fig:top-in-domain} shows the accuracy of 1-sparse SAE feature probes for each dataset individually, demonstrating that each of our contrastive datasets is detectable with a probing accuracy of over 80\%.

\subsection{SAE Feature Combination Analyses} \label{app:feature-combinations}
% \begin{figure*}[t]
%     \centering
%     \includegraphics[width=0.8\textwidth,trim={0 2.2cm 0 3.4cm},clip]{figures/avg_feature_k.png}
%     \caption{Average accuracy vs.\ number of features. (Potentially add the probe line, etc.)}
%     \label{fig:sae-combi-probe}
% \end{figure*}

\begin{figure*}[t]
    \centering
    \includegraphics[width=0.8\textwidth,trim={0 1.4cm 0 3.4cm},clip]{figures/top_sae_feature_group_accuracies_k_L31_pre.png}
    \caption{Performance of top feature combinations (layer 31, pre-activation).}
    \label{fig:top-combis}
\end{figure*}

Figure~\ref{fig:top-combis} shows additional probing analysis for the best performing groups of SAE features up to a group size of five. Group performance is generally dominated by the best performing features and does not majorly exceed the performance of the strongest feature. 

% \subsection{Celebrity Dataset Evaluation}
% \begin{figure*}[t]
%     \centering
%     \includegraphics[width=0.8\textwidth,trim={0 0 0 3.4cm},clip]{figures/celeb.png}
%     \caption{Comparison of top SAE features on the Celebrity dataset. \lh{remove?}}
%     \label{fig:celeb}
% \end{figure*}

\begin{figure*}[t]
    \centering
    \includegraphics[width=0.8\textwidth,trim={0 1.4cm 0 3.4cm},clip]{figures/hierarchical_sae_probe_layer20_pre.png}
    \caption{Accuracies of SAE probes trained on different numbers of SAE features (Layer 20, pre-activation).}
    \label{fig:sae-k-pre20}
\end{figure*}

Figure~\ref{fig:sae-k-pre20} shows additional analysis for SAE feature combinations in Layer~20, analogous to the results for Layer~31 given in Figure~\ref{fig:sae-k-pre31}.

\subsection{Cosine Similarities}
\begin{figure*}[t]
    \centering
    \includegraphics[width=0.8\textwidth,trim={0 0 0 3.4cm},clip]{figures/sae_probe_similarities_by_layer.png}
    \caption{Absolute cosine similarities of top 10 SAE features at different layers, compared with the residual stream probe.}
    \label{fig:similarity_k}
\end{figure*}

We conducted an additional similarity analysis for the top SAE feature groups of different sizes. The results can be found in Figure~\ref{fig:similarity_k} and show a clear trend of larger groups of features becoming more similar to the linear probes. This provides some weak evidence that by default, linear probes might learn more specialized directions that can be represented as a linear combination of more general SAE features. 

% \begin{figure*}[h!]
%     \centering %trim titles
%     \includegraphics[width=.8\textwidth,trim={0 1.4cm 0 3.4cm},clip]{figures/sae_feature_accuracies_layer20_post.png}
%     \caption{Answerability detection accuracies for top SAE features (layer 20, post-activation).}
%     \label{fig:sae-probe_post20}
% \end{figure*}

% \begin{figure*}[h!]
%     \centering %trim titles
%     \includegraphics[width=.8\textwidth,trim={0 1.4cm 0 3.4cm},clip]{figures/sae_feature_accuracies_layer20_pre.png}
%     \caption{Answerability detection accuracies for top SAE features (layer 20, pre-activation).}
%     \label{fig:sae-probe_pre20}
% \end{figure*}

% \begin{figure*}[h!]
%     \centering %trim titles
%     \includegraphics[width=.8\textwidth,trim={0 1.4cm 0 3.4cm},clip]{figures/sae_feature_accuracies_layer31_post.png}
%     \caption{Answerability detection accuracies for top SAE features (layer 31, post-activation).}
%     \label{fig:sae-probe_post31}
% \end{figure*}


% % \begin{figure*}[h!]
% %     \centering %trim titles
% %     \includegraphics[width=.7\textwidth,trim={0 2.2cm 4.2cm 3.4cm},clip]{figures/res_probe.png}
% %     \caption{\vt{can we have less space between the individual plots and make the bars thinner? best so that it fits next to fig 1. also: there's an alternative pic. image.png what's that?}}
% %     \label{fig:res-probe}
% % \end{figure*}

% \begin{figure*}[h!]
%     \centering %trim titles
%     \includegraphics[width=.7\textwidth,trim={0 2.2cm 1cm 3.4cm},clip]{figures/res_probe_20.png}
%     \caption{Linear probe trained on the Layer 20 residual stream. The probe is trained on the SQUAD dataset and then evaluated on the out-of-distribution datasets (IDK, BoolQ, Celebrity, Equation). Error bars show the standard deviation, averaged over 10 bootstrap samples.}
%     \label{fig:res-probe-20}
% \end{figure*}


% \begin{figure*}[h!]
%     \centering
%     \includegraphics[width=.8\textwidth,trim={0 2.2cm 0 3.4cm},clip]{figures/avg_feature_k.png}
%     \caption{\vt{can we add the probe in here?}}
%     \label{fig:sae-combi-probe}
% \end{figure*}

% \begin{figure*}[h!]
%     \centering %trim titles
%     \includegraphics[width=.8\textwidth,trim={0 1.4cm 0 3.4cm},clip]{figures/top_features_k.png}
%     \caption{\vt{later, better make the figures more space efficicient, borders which I cannot crop}}
%     \label{fig:top-combis}
% \end{figure*}

% \begin{figure*}[h!]
%     \centering %trim titles
%     \includegraphics[width=.8\textwidth,trim={0 0 0 3.4cm},clip]{figures/celeb.png}
%     \caption{\vt{can we add the probe directly in here? and maybe remove some of the ones of the right hand side to make it fit one column. maybe we also don't need this plot extra. let's see/discuss}}
%     \label{fig:celeb}
% \end{figure*}

% \begin{figure*}[h!]
%     \centering %trim titles
%     \includegraphics[width=.8\textwidth,trim={0 1.4cm 0 3.4cm},clip]{figures/hierarchical_sae_probe_layer20_pre.png}
%     \caption{Accuracies of SAE probes trained on different numbers of SAE features (Layer 20, pre-activation}
%     \label{fig:sae-k-pre20}
% \end{figure*}

% \begin{figure*}[h!]
%     \centering %trim titles
%     \includegraphics[width=.8\textwidth,trim={0 1.4cm 0 3.4cm},clip]{figures/hierarchical_sae_probe_layer20_post.png}
%     \caption{Accuracies of SAE probes trained on different numbers of SAE features (Layer 20, pre-activation}
%     \label{fig:sae-k-post20}
% \end{figure*}


% \begin{figure*}[h!]
%     \centering %trim titles
%     \includegraphics[width=.8\textwidth,trim={0 1.4cm 0 3.4cm},clip]{figures/hierarchical_sae_probe_layer20_pre.png}
%     \caption{Accuracies of SAE probes trained on different numbers of SAE features (Layer 31, post-activation}
%     \label{fig:sae-k-post31}
% \end{figure*}




% \begin{figure*}[h!]
%     \centering %trim titles
%     \includegraphics[width=.8\textwidth,trim={0 0 0 3.4cm},clip]{figures/sae_probe_similarities_by_layer.png}
%     \caption{Absolute cosine similarities of top 10 SAE features for different number of feature combinations, the model layer of the SAE, and activation hook points, compared with the residual stream probe.}
%     \label{fig:similarity_k}
% \end{figure*}

% \begin{figure*}[h!]
%     \centering %trim titles
%    \includegraphics[width=.4\textwidth,trim={0 2.2cm 1cm 3.4cm},clip]{figures/res_probe_31.png}
% \caption{..%Comparison between top SAE features (left), pre-activation, and linear probes (right) on layer 31.
% }
%     \label{fig:probe_pre31}
% \end{figure*}

% \begin{figure}[h!]
%     \centering %trim titles
%     \includegraphics[width=.5\textwidth,trim={0 1.4cm 0 3.4cm},clip]{figures/sae_top_feature_accuracies_in_domain.png}
%     \caption{Top SAE feature accuracies when training the 1-sparse probes on each dataset individually using a 20\% test split (pre-activation function).}
%     \label{fig:sae-in-domain}
% \end{figure}


% mention why we focus on layer 31 pre -relu





% \myparagraph{Dedicated Prompts} % prompts dedicated to the task
% QA prompt variation


%(only use 80\% of training data here for some reason, probably matching some earlier version of model probes - could rerun overnight).


\myparagraph{Other experiments}
We validated our setup by searching for bias-related features as it was done in related works.
We also experimented with (inofficial) SAEs for an instruction-tuned Llama model, but could not find SAE features with sufficient in-domain probing accuracy. Finally, we also performed analysis on Gemma 2 2B and also the base models, but performance on the answerability task was relatively low in these models (the best SAE features achieved around 70\% probing accuracy).

\end{document}
\endinput
%%
%% End of file `sample-manuscript.tex'.
