\section{LLM Prompts Used}

% \subsection{Structured Explanation Prompt}
% \label{app:prompt_structured}


% SYSTEM:

% {\footnotesize
% \begin{spacing}{1.0}
% \texttt{You are an AI assistant on an AR device that can help a user achieve their goals. 
% Given the user's current contextual information, your task is to provide the user with relevant recommendations of digital actions that they can take as a next step.}
% \end{spacing}
% }

% \noindent 
% USER:

% {\footnotesize
% \begin{spacing}{1.0}
% \texttt{Your input will be (1) a series of narrations of videos taken from a smartglasses user's perspective and (2) objects detected in the user's field of view. These two pieces of information are detected by upstream machine learning models. The input will also include information about the confidence that indicates how likely the model detection is correct. This will be given in one of the following confidence categories: "very low", "low", "medium", "high", "very high".}


% \texttt{Please think step by step:}

% \texttt{1) Estimate the [activity] the user is doing, the [object] the user is interacting with, and the [location] the user is located in. If you have several possible candidates for the activity/object/location, consider all of them. Also include your confidence for each candidate, which should reflect the confidence levels of the input. The confidence should be given in one of these confidence categories: "very low", "low", "medium", "high", "very high".
% Examples of activities include: "walking a dog", "grocery shopping", "reading", "cooking", "cleaning", "playing games", "working out". Other activities are also possible.
% Examples of objects include: "plant", "food items", "tv". Other objects are also possible.
% Examples of locations include: "dining area", "kitchen", "living room", "grocery store". Other locations are also possible.}

% \texttt{2)Infer and provide a short-term <goal> from the input information given. Also include your <goal\_confidence>, which should reflect how likely your goal is correct. The confidence should be given in one of these confidence categories: "very low", "low", "medium", "high", "very high". 
% Examples of short-term goals include: "to cook a meal", "to learn how to play a game", "to organize an outing with friends", "to remember something for later", "to remember the moment", "to go for a run". Other goals are also possible.}

% \texttt{Pick the piece of <activity> <object> <location> information you used in order to infer the goal. Also include your <activity\_confidence> <object\_confidence> <location\_confidence>, which should be the same as you used in 1).}

% \texttt{3) Considering the user's goal that you have inferred in 2), provide a <recommendation> of a digital action that the user can take as a next step. Examples of recommendations include: "open the Running playlist on the Spotify app" "search for a cooking video on the Youtube app". Other recommendations are also possible.
% Also include your <recommendation\_confidence>, which should reflect how likely your recommendation is correct. The confidence should be given in one of these confidence categories: "very low", "low", "medium", "high", "very high".} 



% \texttt{Respond in JSON format:\\
% \{\\
% "goal": str,\\
% "goal\_confidence": str,\\
% "activity": str,\\
% "activity\_confidence": str,\\
% "object": str,\\
% "object\_confidence": str,\\
% "location": str,\\
% "location\_confidence": str,\\
% "recommendation": str,\\
% "recommendation\_confidence": str\\
% \}
% }



% \texttt{\\Input:\\
% 1) Video narrations, in the order of the least to most recent: 
% ['\#C C looks around  (medium)', '\#C C turns around  (medium)', '\#c c walks in the supermarket (low)', '\#C C walks in the shop (low)', '\#C C walks in the supermarket  (medium)', '\#C C looks around  (medium)', '\#c c walks in the supermarket (high)', '\#C C looks around  (medium)', '\#C C walks in the supermarket (high)', '\#C C turns around  (low)', '\#C C turns around  (low)', '\#C C looks around (low)', '\#C C looks around  (medium)', '\#C C picks a box from the shelve  (very high)', '\#C C operates the phone (high)', '\#C C uses the phone (very low)']}

% \texttt{2) Detected objects in the user's field of view: 
% ['bowl (medium)', 'bottle (very high)', 'oven (medium)', 'potted plant (medium)', 'orange (low)', 'microwave (low)', 'refrigerator (low)', 'cup (medium)', 'book (medium)', 'sink (very low)', 'toothbrush (low)', 'cell phone (low)', 'broccoli (low)', 'person (very high)', 'scissors (very low)', 'cake (high)', 'dining table (medium)', 'knife (low)', 'couch (very low)', 'apple (very low)', 'handbag (very low)']
% }
% \end{spacing}
% }


\subsection{Unstructured Explanation Prompt}
\label{app:prompt_baseline}

% SYSTEM:

% {\footnotesize
% \begin{spacing}{1.0} % Adjust the line spacing here (e.g., 0.8 is less space)
% \texttt{You are an AI assistant on an AR device that can help a user achieve their goals. 
% Given the user's current contextual information, your task is to provide the user with relevant recommendations of digital actions that they can take as a next step.}
% \end{spacing}
% }

% \noindent
% USER:

{\footnotesize
\begin{spacing}{1.0} % Adjust the line spacing here (e.g., 0.8 is less space)
\texttt{Your input will be (1) a series of narrations of videos taken from a smartglasses user's perspective and (2) objects detected in the user's field of view. These two pieces of information are detected by upstream machine learning models.}

\texttt{Input:}

\texttt{
1) Video narrations, in the order of the least to most recent: 
['\#C C looks around ', '\#C C turns around ', '\#c c walks in the supermarket', '\#C C walks in the shop', '\#C C walks in the supermarket ', '\#C C looks around ', '\#c c walks in the supermarket', '\#C C looks around ', '\#C C walks in the supermarket', '\#C C turns around ', '\#C C turns around ', '\#C C looks around', '\#C C looks around ', '\#C C picks a box from the shelve ', '\#C C operates the phone', '\#C C uses the phone']}

\texttt{2) Detected objects in the user's field of view: 
['bowl', 'bottle', 'oven', 'potted plant', 'orange', 'microwave', 'refrigerator', 'cup', 'book', 'sink', 'toothbrush', 'cell phone', 'broccoli', 'person', 'scissors', 'cake', 'dining table', 'knife', 'couch', 'apple', 'handbag']}


\texttt{The recommendation is "open a pantry organization tutorial on the Youtube app". 
Please provide an <explanation> for this recommendation within 25 words. In the explanation, only focus on referencing information from the contextual input and please think step by step. Please do not restate the recommendation in your explanation.
}
\end{spacing}
}



%\section{Technical Evaluation Details}
%\label{app:tech_eval}

%\begin{figure*}[h]
%  \centering
%\includegraphics[width=0.6\linewidth]{figures/tech_recommendation.png}
%  \caption{Example survey question in the technical evaluation asking the correctness of the recommendation.}
%\end{figure*}


%\begin{figure*}[h]
%  \centering
%\includegraphics[width=0.6\linewidth]{figures/tech_plausibility.png}
%  \caption{Example survey question in the technical evaluation asking the plausibility of each component of the structured explanation.}
%\end{figure*}

%\begin{figure*}[h]
%  \centering
%\includegraphics[width=0.6\linewidth]{figures/tech_faithfulness.png}
%  \caption{Example survey question in the technical evaluation asking the faithfulness of each component of the structured explanation.}
%\end{figure*}


%\section{User Study Details}
%\label{app:interface}

%\begin{figure*}[h]
%  \centering
%\includegraphics[width=0.9\linewidth]{figures/interface.png}
%  \caption{User study interface for each task. Participants first watched a 30-second video, then  chose whether to accept or dismiss the AI’s recommendation on the smartwatch UI.}
%\end{figure*}

