\documentclass[conference]{IEEEtran}
%\IEEEoverridecommandlockouts
%\IEEEspecialpapernotice{2023 IEEE International Conference on Example, July 2023}

% The preceding line is only needed to identify funding in the first footnote. If that is unneeded, please comment it out.
%Template version as of 6/27/2024
\usepackage{pifont}
\usepackage{xspace}
\usepackage{colortbl}
\usepackage{xcolor}
\usepackage{array} 
\newcommand{\namex}{Janus\xspace}
\newcommand{\mypara}[1]{\textbf{\textit{#1}}}
\newcommand{\jing}[1]{\textcolor{black}{#1}}
\usepackage{subfig}
\newcommand{\qf}[1]{\textcolor{brown}{#1}}
\usepackage[ruled,vlined]{algorithm2e}
\usepackage{booktabs}
\usepackage{cite}
\usepackage{amsmath,amssymb,amsfonts}
\usepackage{algorithmic}
\usepackage{graphicx}
\usepackage{textcomp}


\def\BibTeX{{\rm B\kern-.05em{\sc i\kern-.025em b}\kern-.08em
    T\kern-.1667em\lower.7ex\hbox{E}\kern-.125emX}}

\begin{document}

\title{It Takes Two to Tango: Serverless Workflow Serving via Bilaterally Engaged Resource Adaptation
}
%\title{Efficient Serverless Workflow Serving via Bilaterally Engaged Resource Adaptation}


\author{
\IEEEauthorblockN{
Jing Wu\textsuperscript{1},
Lin Wang\textsuperscript{2}, 
Quanfeng Deng\textsuperscript{1}, 
Chen Yu\textsuperscript{1}, 
Dong Zhang\textsuperscript{3}, 
Bingheng Yan\textsuperscript{3}, 
Fangming Liu\textsuperscript{*1,4}}
\IEEEauthorblockA{\textsuperscript{1}\textit{
National Engineering Research Center for Big Data Technology and System,}\\
\textit{
Services Computing Technology and System Lab, Cluster and Grid Computing Lab,}\\
\textit{
Huazhong University of Science and Technology, China} \\
\textsuperscript{2}\textit{Paderborn University, Germany } \\
\textsuperscript{3}\textit{Inspur Data Co., Ltd., China}\\
\textsuperscript{4}\textit{Peng Cheng Laboratory, China}\\
Email: wujinghust@hust.edu.cn, lin.wang@uni-paderborn.de, quanfengdeng@foxmail.com, \\
yuchen@hust.edu.cn, \{zhangdong, yanbh\}@inspur.com, fangminghk@gmail.com}
% \IEEEcompsocitemizethanks{
% \IEEEcompsocthanksitem \textsuperscript{*§} Corresponding authors.
% }
}

% \author{
% \IEEEauthorblockN{1\textsuperscript{st} Given Name Surname}
% \IEEEauthorblockA{\textit{dept. name of organization (of Aff.)} \\
% \textit{name of organization (of Aff.)}\\
% City, Country \\
% email address or ORCID}
% \and
% \IEEEauthorblockN{2\textsuperscript{nd} Given Name Surname}
% \IEEEauthorblockA{\textit{dept. name of organization (of Aff.)} \\
% \textit{name of organization (of Aff.)}\\
% City, Country \\
% email address or ORCID}
% \and
% \IEEEauthorblockN{3\textsuperscript{rd} Given Name Surname}
% \IEEEauthorblockA{\textit{dept. name of organization (of Aff.)} \\
% \textit{name of organization (of Aff.)}\\
% City, Country \\
% email address or ORCID}
% \and
% \IEEEauthorblockN{4\textsuperscript{th} Given Name Surname}
% \IEEEauthorblockA{\textit{dept. name of organization (of Aff.)} \\
% \textit{name of organization (of Aff.)}\\
% City, Country \\
% email address or ORCID}
% \and
% \IEEEauthorblockN{5\textsuperscript{th} Given Name Surname}
% \IEEEauthorblockA{\textit{dept. name of organization (of Aff.)} \\
% \textit{name of organization (of Aff.)}\\
% City, Country \\
% email address or ORCID}

% \and
% \IEEEauthorblockN{6\textsuperscript{th} Given Name Surname}
% \IEEEauthorblockA{\textit{dept. name of organization (of Aff.)} \\
% \textit{name of organization (of Aff.)}\\
% City, Country \\
% email address or ORCID}
% }



\maketitle



\begin{abstract}
Serverless platforms typically adopt an early-binding approach for function sizing, requiring developers to specify an immutable size for each function within a workflow beforehand.
Accounting for potential runtime variability, developers must size functions for worst-case scenarios to ensure service-level objectives (SLOs), resulting in significant resource inefficiency. 
To address this issue, we propose Janus, a novel resource adaptation framework for serverless platforms. Janus employs a late-binding approach, allowing function sizes to be dynamically adapted based on runtime conditions.
The main challenge lies in the information barrier between the developer and the provider: developers lack access to runtime information, while providers lack domain knowledge about the workflow. 
To bridge this gap, Janus allows developers to provide hints containing rules and options for resource adaptation. 
Providers then follow these hints to dynamically adjust resource allocation at runtime based on real-time function execution information, ensuring compliance with SLOs. 
We implement Janus and conduct extensive experiments with real-world serverless workflows. 
Our results demonstrate that Janus enhances resource efficiency by up to 34.7\% compared to the state-of-the-art.

\end{abstract}

% \begin{IEEEkeywords}
% component, formatting, style, styling, insert.
% \end{IEEEkeywords}


The increasing reliance on LLMs for multimodal tasks across far-reaching sectors such as healthcare, finance, and manufacturing underscores the need to assess the accuracy and reliability of the information they generate. Vision-Language Models (VLM) have achieved state-of-the-art (SoTA) performance on Visual Question-Answering (VQA) benchmarks, and these models often utilize Retrieval-Augmented Generation (RAG) to maintain factual accuracy and relevance in a dynamic information environment. However, this has led to uncertainty in the information the LLM bases its answer on, as it may choose between parametric memory and retrieved sources. When models rely on memorized information instead of dynamically retrieving information, they may inadvertently propagate outdated or incorrect information, causing serious legal and ethical risks and undermining trust and reliability in AI systems \citep{huang2023survey}.
% The ability to strike a balance between generalization and specialization in AI systems is therefore crucial for ensuring the safe, reliable use of these technologies in real-world applications.

Despite these concerns, the way that Vision-Language models (VLMs) memorize and retrieve information, particularly in complex multimodal tasks, remains under-explored. Current research often focuses on either the general capabilities of large language models (LLMs) or the specialized retrieval mechanisms in retrieval augmented generation systems (RAG) \citep{incontext_rag,chen_murag_2022,liu_universal_2023}. Particularly in the context of multimodal retrieval and multihop reasoning, few studies analyze the tradeoff between finetuning for specialized tasks and zero-shot prompting for general-purpose vision-language capabilities. A lack of consensus on how to approach this tradeoff motivates the development of measures to quantify reliance on parametric memory, as well as metrics for quantifying the potential performance impact of extending LLMs with RAG systems.

To address this gap, we investigate how multimodal QA models balance accuracy with memorization on the WebQA benchmark. We compare finetuned multimodal systems against zero-shot VLMs, analyzing how retrieval performance influences QA accuracy. In particular, we focus on cases where retrieval fails, allowing us to measure reliance on parametric memory through two proposed metrics---the \ppr (\PPR) which quantifies how much model accuracy is influenced by retrieval quality, contrasting performance in best-case versus worst-case retrieval scenarios, and the \ucr (\UCR) which measures how often correct QA responses are generated when the retriever fails, providing a proxy for memorization.

To enable this analysis, we make several methodological contributions. For the finetuned QA models, we investigate Vision-Transformer (ViT) architectures, which allow for multihop reasoning over multiple sources. To investigate the impact of retrieval performance on trained LMs, we propose a variable-input Fusion-in-Decoder (FiD) model \cite{tanaka_slidevqa_2023, nlvr2}, building upon the VoLTA architecture \citep{pramanick_volta_2023}. For the zero-shot case, we build upon previous research on In-Context Retrieval \citep{incontext_rag} by demonstrating that LLMs such as GPT-4o are capable of performing the final ranking step of the retrieval process. In doing so, we find that GPT-4o, a general-purpose LLM, achieves SoTA performance on the WebQA task, outperforming existing finetuned RAG models by a significant margin (7\% higher accuracy). 

Crucially, our results reveal that while retrieval-augmented models reduce memorization, the training paradigm plays an important role. Finetuned models exhibit higher reliance on parametric memory, whereas zero-shot RAG approaches have lower memorization scores at the cost of accuracy. This suggests that while retrieval modules may mitigate the risks associated with outdated or incorrect information, SoTA performance requires that they be coupled with specialized QA models. Our memorization measures contribute to the development of transparent and reliable AI systems, particularly in applications where the sourcing of up-to-date, factual information is critical.



% We investigate the impact of question complexity on the ability of these models to integrate multiple data sources—such as images, text, and external retrievers—and produce coherent and accurate answers. We also explore whether in-context retrieval can be a viable alternative to traditional retrieval-augmented systems, offering a more streamlined approach to multimodal QA.

% To achieve this, we first compare zero-shot prompting multimodal LLMs with finetuned multimodal systems. We evaluate both types of models on the WebQA benchmark, a dataset designed for complex question answering that requires reasoning across both image and text sources. For the finetuned models, we use a Fusion-in-Decoder (FiD) architecture, which allows for multihop reasoning over multiple sources. Additionally, we introduce the concept of In-Context Retrieval Language Modeling (RLM), where the LLM itself performs retrieval tasks without the need for external retrievers. This method builds upon existing research in in-context learning  and aims to explore the viability of LLMs retrieving relevant sources and generating accurate answers directly from their context window.

% In order to investigate source utilization in finetuned multimodal models and LLMs, three lines of inquiry are established; 
% \begin{itemize}
%     \item Study 1: retrieval vs QA performance on webQA (motivating example, does QA answer correctly even with incorrect sources?)
%     \item Study 2: performance on adversarial examples where parametric knowledge would be incorrect by design
%     \item Study 3: improving performance on adversarial examples by fine-tuning (i.e model robustness)
% \end{itemize}

% Note, there is one weakness in this plan which is tying in the work we've already done. 
% If we added something from adversarial generation to the retrieval experiment (like a combination of study 1 + 3) it would be complete. So for instance we could try fine-tuning the retriever with adversarial examples (and not just the QA model)

% \begin{figure}
%     \centering
%     \includegraphics[width=0.95\linewidth]{figures/segmentation/webqa_segment_infill.png}
%     \caption{Example of the segmentation substitution pipeline from the WebQA task.}
%     % d5c76d760dba11ecb1e81171463288e9
%     \label{fig:seg_sub_pipeline}
% \end{figure}



% Retrieval augmented generation (RAG) with zero-shot prompting and fine-tuning Large Language Models (LLMs) have become the go-to methods for tasks relying on information retrieval and text generation. In many cases the LLMs parametric memory can sufficiently generalize to answer questions without being provided with retrieval mechanisms for out-of-domain knowledge. However, LLMs often hallucinate and provide wrong information in certain scenarios. This problem is amplified even further on open-domain Question Answering (QA) tasks involving multiple modalities. Grounded text generation using retrieved sources \citep{lewis2021retrievalaugmented} has been extensively studied for text-to-text QA tasks, but its application in multimodal settings has not been studied as much.


% Multimodal reasoning and question answering have gained prominence in recent research endeavors, with an increasing emphasis on handling various forms of data, particularly text and images. In this study, we address a specific gap in the existing literature by focusing on the development of a versatile multihop model capable of accommodating varying numbers of input images.

% Our motivation for this research lies in the growing complexity of answering questions using information on the web, where the challenge of navigating the open-domain setting is further complicated by the presence of multiple modalities and sometimes requires reasoning over multiple sources. WebQA is an ideal dataset on which to compare performance of finetuned RAG systems against general purpose LLMs; it is multimodal, with correct answers requiring reasoning over image and text sources. It is multihop, requiring a complex reasoning process over multiple sources. Finally, WebQA questions from different categories can be broken down into subdomains to analyze performance over domains of varying cardinality.

% Motivated by the real-world challenges of building retrieval and question answering (QA) systems, we design and finetune a closed domain, multimodal, multihop QA model, that is capable of reasoning over a varying number of sources taken as input from an external retriever module. This research contributes to the relatively underexplored domain of multihop reasoning across various input sources and modalities. Our goal is to explore the challenges posed by these scenarios and develop strategies that enable QA models to retrieve relevant information, conduct logical or numerical reasoning across diverse modalities, and generate coherent responses in natural language. To our knowledge, this is the first application of the Fusion-in-Decoder (FiD) architecture \cite{tanaka_slidevqa_2023, nlvr2} that is shown to work with a variable number of inputs, enabling multi-hop reasoning over sources.

% In-Context Learning refers to the ability of LLMs to perform any task by simply providing examples in the input prompt \citep{dong2022survey,min2022rethinking}. Inspired by this research, we propose a method to use the LLM itself as a multimodal retriever, potentially eschewing the requirement of a distinct retrieval module, thereby allowing the design of simpler retrieval-augmented QA systems. We dub this method In-Context Retrieval Language Modeling (RLM). To the best of the authors knowledge, In-Content RLM is disparate from other retrieval augmented approaches which utilize external retrieval modules \citep{incontext_rag,chen_murag_2022,liu_universal_2023}. Despite being a natural extension of In-Context learning, In-Context RLM has not yet been studied empirically.

% To expand on our contribution of In-Context Retrieval, this stems from the well-researched in-context learning of LLMs. In-context learning is the ability of a model to perform any task given a sufficient context window \citep{dong2022survey,min2022rethinking}. Such tasks could include retrieval and ranking, but typically, the go-to solution for tasks requiring retrieval has been RAG. To the best of the authors knowledge, In-Context Retrieval is distinct from In-Context Retrieval Augmented Language Modelling (RALM), and despite being a natural extension of In-Context learning, In-Context Retrieval has not yet been shown empirically.

% Finally, we explore the tradeoff between using zero-shot prompting LLMs and the fine-tuning approach. While we find that, overall, GPT-4o obtains SoTA performance on the WebQA task, outperforming the accuracy of existing finetuned RAG approaches by 7\%, finetuned approaches still perform better on more restricted subdomains\footnote{``In-Context RLM" @ \url{https://eval.ai/web/challenges/challenge-page/1255/leaderboard/3168}}. Finally, we validate that GPT-4o is relying on retrieval abilities to solve the task; we find that GPT-4o is capable of retrieving relevant sources in the presence of distractors and furthermore, when GPT-4o fails to retrieve correct sources, it answers incorrectly 75\% of the time, meaning that it is not relying on parametric memory for this task.

% \paragraph{Contributions}
% Based on our experimentation and analysis on the WebQA benchmark, we make the following contributions:
% \begin{itemize}
%     \item Propose a new architecture for multimodal multihop QA that takes variable number of input sources inspired by the Fusion-in-Decoder method.
%     \item Comparison of general purpose LLMs vs specialized models on the WebQA benchmark.
%     \item Observation of In-Context Multimodal Retrieval abilities of GPT-4o and that it does not rely on parametric memory for multimodal QA.
%     \item Analysis of relationship between retrieval and QA task performance.
%     \item Analysis of task and query complexity on the performance of retrieval and QA tasks.
% \end{itemize}
















% Throughout this paper, we will present our methodology, experiments, and findings, emphasizing our approach to multihop reasoning over varying numbers of input images. We believe that our work contributes to a deeper understanding of multimodal reasoning and has the potential to enhance the capabilities of question-answering systems in the intricate, multimodal landscape of web-based information.
\section{Background and Motivation}
\label{sec:background}

We introduce the background on serverless workload serving and motivate the use of runtime resource adaptation to address resource inefficiency in existing serverless platforms.

\subsection{Resource Inefficiency with Early Binding}
% In current serverless platforms, developers are required to specify immutable sizes for their deployed functions.
% Then, providers consider functions' runtime workloads  (e.g., concurrency)  and resource usage to scale out/in their instances.
% Moreover, due to high runtime variability, functions must size their functions for worst-case scenarios.
% This, however, incurs considerable resource inefficiency.
Current serverless workflow platforms (e.g., AWS Step Functions~\cite{aws-step-function} and Azure Durable Functions~\cite{azure-durable-function}) offer the opportunity for developers to build various applications with advanced logic like chaining, branching, and parallel execution.
These applications can be defined by JSON-based structured languages (e.g., Amazon States Language) or other programming languages.
Meanwhile, developers require to specify resource configurations, including memory size, CPU cores, and scaling options, for individual functions---an early-binding approach.
The serverless platform is responsible for monitoring the workload intensity and resource usage at runtime and scaling out/in function instances accordingly.
To account for potential runtime variability, developers must size the functions in their application workflow accounting for the worst case in order to provide SLO guarantees over the end-to-end delay of request processing, e.g., the 99th percentile (P99) of the end-to-end delay must be within a given target. 
After deployment, the function sizes become immutable. The worst case is not representative and over-shoots most of the time, leading to resource inefficiency. 


To verify this claim, we conduct a data-driven analysis with a dataset from Microsoft Azure Functions~\cite{azure-dataset} to explicitly demonstrate the resource inefficiency issue. % , deriving from the worst-case based early bind.
To quantify the inefficiency, we define a metric called \emph{slack}---the margin between the actual execution time and the SLO, which is calculated as $1-l/T$ with $l$ and $T$ representing end-to-end latency and SLO, respectively.
Under certain SLO defined with P99 latency as done by existing works (e.g., \cite{osdi22-orion,mac22-wisefuse}),  we can see from Figure \ref{fig:bg:slack} that more than 60\% function invocations have slacks over 60\%.
Particularly, we analyze slacks of the top 100 most popular functions, whose invocations account for 81.6\% of the total function invocations. % (depicted in Figure~\ref{fig:bg:popular_func}) of overall invocations.
The result shows that only 20\% of the invocations of the popular functions (blue line in Figure~\ref{fig:bg:slack}) have slacks less than 40\%.
This means the majority of requests are processed faster than necessary.
Notably, in DAG-based workloads (i.e., Azure Durable Functions), the resource inefficiency further deteriorates wherein the ratio between the 95th percentile and 50th percentile is by up to three times \cite{mac22-wisefuse}.

% \begin{figure}[t!]
% \centering
% \includegraphics[width=0.25\textwidth]{./figure/motivation/Average_P99_cdf_top=100.pdf}
% \vspace{-0.3cm}
% \caption{Sufficient function slacks in production traces.}
% \label{fig:bg:slack}
% \end{figure}

\subsection{Runtime Dynamics}
\label{sec:bg:worst-case}

The resource inefficiency caused by the large slack can be mainly attributed to the over-provisioning of resources by the developer. This is to ensure that the SLO is guaranteed even in the worst case (i.e., P99). However, normal cases deviate from the worst case significantly due to runtime dynamics. 
In particular, we observe that functions face two major dynamic factors at runtime: varying working sets and inevitable performance interference. These two factors contribute significantly to the variance of the function execution time. 
% Functions face two remarkably dynamic factors at runtime: working sets and performance interference, which lead to considerable variance of execution latency.

\begin{figure*}[!t]
	\centering
	\subfloat[]{
		\includegraphics[width=0.24\textwidth]{./figure/motivation/Average_P99_cdf_top=100.pdf}
		\label{fig:bg:slack}
	}
	\hspace{8mm}
	\subfloat[]{
		\includegraphics[width=0.25\textwidth]{./figure/motivation/function-latency-ml-analyze-varying-worksets.pdf}
		\label{fig:bg:ml-func-latency}
	}
	\hspace{8mm}
	\subfloat[]{
	\includegraphics[width=0.28\textwidth]{./figure/motivation/coresident-perf.pdf}   
	\label{fig:bg:perf-inteference}
	}
	%\vspace{-0.1cm}
	\caption{(a) slacks of function invocations in production traces, (b) function latency variance caused by varying input worksets for functions object detection (OD), question answering (QA), and and text-to-speech (TS), respectively,
 (c) performance interference attributed to co-location of homogeneous function with different dominant resource demands.}
 %\vspace{-0.4cm}
\end{figure*}

%'ml-analyze':{'text-to-speech': 'text-to-speech', 'question-answer': 'question answer',
%                      'object-detection': 'object detection'
\textbf{\textit{Varying working sets.}} 
The working set, i.e., input data like videos, audios, and texts, can have varying sizes.
Taking Microsoft Azure Function Blobs (storage service) as an example, their data size difference can be as high as nine orders of magnitude~\cite{azure-function-blob}.
Such a large difference results in substantial variance of the execution time even for the same function~\cite{socc21-faast,eurosys21-ofc}.
Specifically, we measure the execution time of three functions under different working sets (detailed in \S\ref{exp:setup}).
Figure~\ref{fig:bg:ml-func-latency} illustrates the results, where we can observe a variance of up to 3.8 times in function execution caused by varying working set sizes.

% \begin{figure}[t!]
% \centering
% \includegraphics[width=0.25\textwidth]{././figure/motivation/function-latency-ml-analyze-varying-worksets.pdf}
% \vspace{-0.3cm}
% \caption{Function latency variance caused by varying input worksets}
% \label{fig:bg:ml-func-latency}
% \end{figure}	

\textbf{\textit{Performance interference.}}
% On the other hand, function deployment, which decides when and where to deploy functions, is completely undertaken by providers.
For simplicity and security, commercial serverless platforms, such as Alibaba Function Compute, Microsoft Azure, and AWS Lambda, exclusively deploy function instances belonging to the same tenant, or even belonging to the same function, in the same virtual machine~\cite{socc22-owl,atc18-peek-bench}.
For example, the empirical study in~\cite{socc22-owl} shows that in Alibaba Function Compute 65\% of the virtual machines exclusively deploy instances of the same function.
This co-location of homogeneous function instances, however, can incur severe resource contention on the same resource dimensions, particularly for network bandwidth and memory bandwidth of virtual machines~\cite{sc21-gsight,micro19-faaSprofiler,socc22-owl,atc18-peek-bench}.
To verify this observation, we use a virtual machine to run a function increasing the number of co-located instances from one to six while measuring the execution time of four different functions with resource dominance on different dimensions namely computing, I/O, network, and memory, respectively (detailed in \S\ref{exp:setup}). 
As shown in Figure~\ref{fig:bg:perf-inteference}, the co-location of homogeneous functions leads to substantial resource contention and performance interference, prolonging the function execution time up to 8.1 times. The performance interference is often hard to model and predict.

% this co-residency results in substantial increase of execution latency by up to 8.1 times,leading to considerable variance in function execution time.
% when compared with that with concurrency as one.

%for CPU-, IO-, network- and memory-intensive functions as the concurrency rises from one to six.
%Figure shows that significant performance interference can be observed, . 
%compared with the inclusive deployment (concurrency as one), 
% this exclusive deployment (gray bar) results in substantial increase of execution latency by up to 8.1$\times$ for CPU-, IO-, network- and memory-intensive functions as the concurrency rises from one to six.

% this exclusive deployment (gray bar) results in substantial increase of execution latency by up to 8.1$\times$ for CPU-, IO-, network- and memory-intensive functions as the concurrency rises from one to six.
% As depicted in Figure~\ref{fig:bg:concurrent_latency}, with the concurrency rising  from one to six,  the exclusive deployment results in substantial increase of execution latency by up to 8.1$\times$.
% This significantly magnifies execution latency variance.

% \begin{figure}[t!]
% \centering
% \includegraphics[width=0.25\textwidth]{./figure/motivation/coresident-perf.pdf}
% \vspace{-0.3cm}
% \caption{Performance interference attributed to co-residency of homogeneous function.}
% \label{fig:bg:perf-inteference}
% \end{figure}




\subsection{Runtime Resource Adaptation}
\label{sec:bg:adaptive-allocation}
To tackle the aforementioned resource inefficiency issue, we can adopt a late-binding approach through \emph{runtime resource adaptation}, which resizes functions on the fly based on runtime information (e.g., function slacks), achieving higher resource efficiency without violating SLO. For example, given a workflow as a chain of functions, the resource allocation of the downstream functions can be adjusted when the first function finishes execution. This way, the slack from the first function can be exploited to optimize resource efficiency. 

The idea sounds straightforward and has been considered in some existing works \cite{infocom22-stepconf,middleware20-fifer,socc21-llama,socc21-kraken,middleware20-xanadu}.
However, most of these works make an unrealistic assumption that either the developer performs the adaptation decision with access to runtime information or the serverless platform provider performs the adaptation with domain knowledge of the application workflow. These assumptions render these solutions impractical to deploy in real-world serverless systems. The information barrier between the developer and the provider calls for a new solution. 

We identify the following challenges and opportunities for a full-fledged design for runtime resource adaptation. 

\textbf{\textit{Skewed function execution time distribution.}} 
Resource allocation for a serverless workflow is typically done by leveraging performance profiles of all the functions in the workflow. 
During the offline profiling, the execution time distribution for each function is first obtained by running the function with a variety of sample inputs under different resource conditions. Then, given a time budget, existing approaches typically use P99 of the function execution time as a target and calculate the corresponding resource demands. However, due to the high runtime variability, the distribution of the function execution time is highly skewed where the difference between P50 and P99 can be as high as 100 times~\cite{socc23-huawei-cloud}. This means that if only the function execution time at a single percentile (P50 or P99) is used for resource allocation, there will be significant resource under-provisioning and over-provisioning for most requests at runtime. To address this issue, our idea is to allow for the exploration of the function execution time at diverse percentiles during resource allocation. 


% It is a prerequisite to profile execution latency for adaptive resource allocation.  
% As aforementioned, owing to a variety of unexpected runtime dynamics,  execution latency demonstrates skewed distributions, by up to 100$\times$ between 99\% percentile and 50\% percentile on Huawei cloud serverless~\cite{socc23-huawei-cloud} .
% This makes the current a single statistic (e.g., mean) or 99\% percentile distribution based profiling suffer significant under- and over-estimation.
% To fix this issue, our insight is to \textit{introduce more diverse percentiles to profile execution latency}. 

\textbf{\textit{Dependencies of adaptation decisions.}}
As the function execution progresses, a sub-workflow will be generated by removing the finished function(s) from the workflow. Within each sub-workflow, the resource adaptation decisions for remaining functions are dependent on each other due to the constraint imposed by the end-to-end latency SLO. For example, under-provisioning a function will result in a reduction of the time budget for executing its downstream functions, thus calling for more resources for these downstream functions to avoid SLO violations. Meanwhile, the selection of the percentile for the execution time of each function dictates resource-latency tradeoff for that function. For example, a higher percentile means that more resources will be allocated to ensure that more requests processed by the function will finish within the given time budget. On the contrary, a lower percentile means that more requests will risk SLO violation, but at the benefits of reduced resource consumption. To address such complex dependencies, we propose the following ideas: (1) We introduce two metrics (i.e., the timeout metric and the resilience metric detailed in \S\ref{sec:profilier}) to balance the resource adaptation decisions of the head function of the current sub-workflow and those of the remaining downstream functions. These metrics help us connect the decision making across sub-workflows and avoids sub-optimal adaptation decisions in each sub-workflow. 
(2) We explore lower percentiles for the head function and a high percentile (i.e., P99) for other functions in each sub-workflow. Using lower percentiles maximizes the opportunity for resource optimization since any over-time execution of the head function can later be compensated by resource adaptation in the next round. The high percentile ensures that the resource adaptation is not too radical to cause SLO violations. 

% Each workflow generates multiple sub-workflows as the execution moves forwards. 
% Within sub-workflows, the provisioning is inter-corrected.
% For instance, under-provisioning upstream functions may directly shrink the time budget for downstream functions, which dictates more resources required by the latter against (sub-) SLO violation. 
% This makes sub-workflows generally adopted as the basic unit to make adaptation decisions~\cite{socc21-llama,rtas22-fa2}. 
%  Moreover,  due to the high variance of execution performance, runtime adaptation requires to carry out function by function, i.e.,  discrete adaptation.
%  This, however, can easily lead to a sub-optimal (analyzed in~\S~\ref{sec:synthesizer:generate}).
% Our insight is to \emph{introduce a metric (i.e., resilience detailed in \S~\ref{sec:profilier}) to quantify the inter-correlation as well as a heuristic design (i.e., heavier head explained in \S~\ref{sec:synthesizer:generate})  to calibrate the sub-optimal,  such that resource adaptation can explore higher resource efficiency without SLO guarantee}.

% In particular, latency percentiles (introduced by the profiling)  involves resource adaptation as a new knob.
% Specifically, higher percentile earns  stronger guarantees in SLOs but may be highly prone to resource over-allocation because of its latency over-estimation, impairing resource efficiency.
% In contrast, decreasing percentiles offers the opportunity to explore higher resource efficiency, but suffers the risk of timeout, i.e., execution latency beyond specified time budget, and  may thus incur  SLO violations.
% Here, our insight is to \emph{moderately explore percentiles (detailed in~\S~\ref{sec:synthesizer:generate}), where head functions of  (sub-)workflows can explore lower percentiles because this creates the opportunity to reap higher resource efficiency while possible timeout can be recovered by subsequent functions' re-adaptive allocation.
% On the other head, non head functions maintain percentiles as 99\%}.
% This can well keep the trade-off between opportunities of exploring higher resource efficiency and risks of SLO violations. 
% Additionally, it effectively shrinks the searching space, benefiting the adaptation with higher time-efficiency.


\textbf{\textit{Tight resource adaptation window.}}
Runtime resource adaptation requires to calculate a new resource allocation decision for the remaining sub-workflow immediately when a function finishes execution. Since serverless functions are typically short-lived (less than 1s on average)~\cite{atc18-peek-bench,socc22-owl,atc20-serverless-in-the-wild,socc23-huawei-cloud}, the window for resource adaptation is quite tight. Assuming the serverless platform will perform the runtime adaptation on behalf of the developer since the platform has access to full runtime information, the resource adaptation decision making should be fast without involving complex calculations and logic or exploring a large space. As discussed before, the serverless platform provider does not have domain knowledge of the serverless workflow. Hence, the developer must pass the necessary information to the serverless platform for runtime adaptation decision making. Our idea is to let the developer synthesize critical hints containing resource allocation rules and options, which the serverless platform provider utilizes to perform runtime resource adaptation. The hints should be highly condensed so the serverless platform can make adaptation decisions quickly enough. 


% Apart from highly varying execution performance, serverless functions are also short-living (less than 1s on average)~\cite{atc18-peek-bench,socc22-owl,atc20-serverless-in-the-wild,socc23-huawei-cloud}, so is the window for adaptive allocation. 
% This variance and volatility calls for a well-preparation of hints for all possible runtime situations while promising them compact and straightforward enough for providers to easily take action.

% Here, our insight is to \emph{holistically synthesize hints in an offline manner, and then utilize the discreteness of adaptive allocation in both decision-making and decision-executing (detailed in~\S~\ref{sec:synthesizer:condense}) to fully condense the hints.
% Finally, hints are warped into a simple and compact table.
% Base on that, providers can accomplish the runtime adaption promptly and properly}.

To demonstrate the potential of runtime resource adaptation incorporating all the above ideas, we take a real-world serverless workflow (explained in \S\ref{exp:setup}) as an example, and evaluate its end-to-end latency (denoted by E2E) and resource consumption (CPU cores).
As illustrated in Figure~\ref{fig:bg:size}, the late-binding (blue triangle) reduces the resource consumption by up to 42.2\% compared with existing early-binding solutions (orange circle), while ensuring SLO guarantees. This highlights the significant gains from runtime resource adaptation. 


\begin{figure}[t!]
\centering
\includegraphics[width=0.45\textwidth]{./figure/motivation/size_early_bind_vs_ours.pdf}
%\vspace{-0.1cm}
\caption{Performance comparison between early-binding (left)~\cite{eurosys19-grandslam} and late-binding (runtime resource adaptation), where the CPU consumption (right) is normalized by the optimal obtained with exhaustive search.} 
%\vspace{-0.3cm}
\label{fig:bg:size}
\end{figure}

   
	







\section{System}
\begin{figure*}[h]
    \centering
    \includegraphics[width=.85\textwidth]{fig/SYSTEM_IMAGE_TEST_FLIPPED.png}
    \caption{HumorSkills System Diagram. Given an image, the system first extracts visual details with a visual language model, then performs visual humor ideation to analyze the image and propose humorous angles. It then generates ten potential conflicts that could be used to extrapolate the image into a relatable experience. The system then generates humor with and without the narratives, for diversity. A separate instance of the LLM trained to rank gen-Z humor ranks all the captions and returns the top five.}
    \Description{HumorSkills System Diagram}
    \label{fig:system}
\end{figure*}

HumorSkills is a system that takes an input image and outputs 5 image captions. 
The architecture has three key steps that mimic human skills needed for humor. \textit{Visual Detail Extraction}, is a step that describes the image in depth in order to make non-obvious observations about it. \textit{Narrative and Conflict Extrapolation} is a step that finds narratives not in the image that could be related to it, to expand the topic of jokes to things that are not just in the image but also analogous to it.  \textit{Fine-tuning} the joke generator with examples of good Gen-Z humor helps the jokes be more relatable to the target audience by using references, slang, topics, and insecurities that resonate with this group.

% first, 
% a \textbf{divergent stage} where the image is analysed and multiple observations, angle, alternative narratives and humorous angles are generated. 
% Second, a \textbf{generation stage} where two types of captions are generated: 1) captions focusing on image content directly 2) captions that bring in an outside narrative to the image, often bringing in outside references. It generates 15 captions of each type. (Figure 1 has examples Of the Content Focused, and Narrative Expanded Captions). Finally, a \textbf{ranking stage} where a separate AI agent selects the top 5 captions

The system generates two types of captions: image-focused captions which common directly on the content in the image, and narrative-driven captions. Variety is important to humor. Humor relies on surprise, and jokes that are too similar start to become more predictable. Additionally, with an infinite set of input images with different subjects and situations, there are more strategies needed to find a humorous angle that fits the content. 

% With caption-based humor, often the humor can be focused on finding something in the image that is inherently interesting. 

% For example, the caption ”little man really thought he could escape bedtime” relies solely on information in the image. However, some images don’t have something funny in and of themselves, and it’s easier to make a joke by bringing in a new unrelated angle. For example, ``the police chasing me when I'm broke and in debt to the tune of \$100,000 for student loans''. Generally, Images with people doing interesting things lend themselves to visual humor because there are many stories one could tell about it. However, for images with only static objects, it's more difficult to tell a story on only the objects, so bringing a new story in is another way to find humor. 

\subsection{AI Humor Generation Walk Through}
Figure \ref{fig:system} contains a visual diagram and example of intermediate outputs when generating captions for an image. We describe each phase and implementation in detail.  
% The main contribution of this paper is the evaluation, rather than the system, but it is still it is important to understand the mechanism used to generate humor.
% Although the individual components of the system are not totally, the combination of features including the

\subsubsection{Visual Detail Extraction}

The first phase of the system’s workflow involves the Visual Detail Extraction component, which utilizes GPT-4o’s vision capabilities to analyze the input image. This system incorporates a prompt that asks for a detailed paragraph that explains the who/what/where of the image, distinguishing between identifying the subject of the image, the main action of the image if it exists, and the background elements of the image. This component is responsible for extracting key visual elements such as objects, human expressions, background settings, and any notable aspects that could serve as the foundation for humor.

For instance, in the demolition site example from the system diagram, the system identifies a large industrial demolition excavator and a person with a hose spraying the demolition site. 

\subsubsection{Visual Humor Ideation}

On top of the visual detail extraction, the system ideates on possible humorous elements from the visual of the image. This incorporates an additional prompt using GPT-4o that intakes the image and asks it to identify and ideate on potential humorous visual elements in the image, whether they are directly humorous elements, such as funny facial expressions, or more analogous elements. For example, for the system diagram image, the system noted the visual contrast of the excavator and person, reminiscent of a David versus Goliath scenario, which provides a foundational metaphor for generating humorous captions. 

\begin{figure*}[b]
    \centering
    \includegraphics[width=.95\textwidth]{fig/Workflow.png}
    \caption{A diagram for how narrative extrapolation works}
    % \Description{}
    \label{fig:systemLines}
\end{figure*}

\subsubsection{Narrative and Conflict Extrapolation}

In this next step, the system generates a narrative and conflict framework by drawing upon common and relatable Gen Z experiences such as work, school, family, social interactions, relationships, and more. 
The system chains together the results of the previous steps, into a new prompt sent to GPT-4o. 
% The system prompts GPT-4o to 
% GPT-4o is utilized by incorporating the text description of the image and potential humorous elements of the image, then being prompted to generate relatable scenarios applicable to the image description from a list of common Gen Z experiences. 
The prompt contains the visual details, the visual humor ideation, and a list of common Gen Z experiences,  and the instruction to "generate narratives that reflect the essence of the image that is set within the framework of the Gen Z experience."
This narrative generation adds depth to the humorous captions by applying relatable themes and conflicts to the visual elements identified earlier.

For instance, our system diagram generates narratives such as “Tackling student loans”, "Group Project Disaster", and “Relationship Issues” based on the image, both of which are common experiences among those who identify as Gen Z. These particular narratives are likely inspired by the imagery of a disaster site, referring to how the effort of paying off student loans, attempting to complete group projects during school, or addressing relationship -- all of which can feel like disaster clean up. These relatable conflicts can transform the visual of a demolition scene -- a setting that is not particularly relatable -- into a relatable scenario that has the potential for humor, thereby expanding the humorous possibilities by connecting the visual input with broader life experiences.



\subsubsection{Humorous Caption Generation}

Following the narrative and conflict extrapolation, the system generates humorous captions in the generation stage using a fine-tuned version of GPT-3.5 trained on humorous Instagram comments. This involves producing captions through two distinct strategies: one focused on the visual humor of the image, and the other by bringing in the previously generated external narratives. Caption generation is segmented into two separate prompts utilizing the fine-tuned GPT-3.5 model. For captions without generated external narratives, the prompt asks to generate 15 humorous captions in the style of Gen Z that bases the generation off the visual extraction and visual humor ideation of the input image. For captions with the external narratives, the prompt also asks to generate 15 humorous captions in the style of Gen Z that bases the generation off the visual extraction and visual humor ideation of the input image, but also asks the system to incorporate the list of generated narrative/conflict extrapolations to base the humorous captions off of.

Image-focused captions rely solely on the visual details in the image, such as “bro out here getting paid \textdollar8 an hour to spray some water on some bricks,” which references the direct visual elements in the scene in order to generate a caption. This particular caption directly references the humor of the image, poking fun at the minimal impact of the person spraying water on bricks while an excavator clearly has more impact on the demolition site. Narrative-driven captions, on the other hand, introduce external references to add humor. For instance, a caption like “The entitled bro you tried to make the group presentation with” introduces an outside, exaggerated, interpretation of the scene from earlier, "Group Project Disaster." This caption takes the group project narrative and pairs it with the visual of the image, analogizing the person spraying the hose with minimal impact on the demolition site to an entitled person who has not done much to complete the group project. 

This variety between visual humor and narrative-driven humor is crucial because jokes that are too similar become predictable, losing their element of surprise. Additionally, humor strategies need to adapt to the varying content in input images. Some images lend themselves to humor based on their inherent visual details, while others require bringing in outside references to create a joke. For instance, an image of static objects might not be inherently funny, such as the demolition image shown in the system diagram, but a caption introducing an unrelated, exaggerated narrative, such as “Eboy doin' his part to stop climate change” can inject humor and absurdity by making an unexpected connection.

\subsubsection{Caption Ranking using Gen Z Agent}

The final component of the system architecture is the Caption Ranking and Filtering Agent, a GPT-4o-based agent fine-tuned to evaluate humor from a Gen Z perspective. This agent receives the list of 30 total captions from the narrative and visual humor-based caption generations and ranks the captions generated in the previous stage based on humor, relatability, and alignment with the image and narrative.

As illustrated in our system diagram, this agent ranks captions such as “Me mopping up my last relationship” and “me pulling the emotional weight of the friend group” based on their relevance to Gen Z humor. Captions that fail to meet the humor threshold are filtered out, such as "Demolition worker really said 1v1 me bro," because although the phrase like "1v1 me bro" invokes Gen Z phrases, the content of the caption seems less relevant and relatable than a caption talking about school or relationships, ensuring that only the most effective and relatable captions are presented to the user.

\subsubsection{Fine-tuning}

To fine-tune a GPT-3.5 model, a dataset of 80 humorous comments were extracted from popular Instagram images. From three popular Instagram meme pages with over 400,000 followers, the top five comments of each image post were collected. All fit the style of Gen-Z humor. 
% The fine-tuning process ensured that the generated captions align with the humor style favored by Gen Z. 
Examples of the visual description of the images in addition to an explanation of potential humorous elements of the image were written in the fine-tuning prompts, then followed by the actual comment itself. This reflected the visual extraction and humor ideation being incorporated into the prompt of our current system.



%\section{Profiler}
\label{sec:profilier}
Here,  we discuss how to profile function execution latency.
Based on that, two metrics are proposed as a preparation for synthesizing hints.
\subsection{Diverse percentiles based profiling}
As explained in \S{\ref{sec:bg:worst-case}}, function execution latency is a distribution, making the current work, which either depends on a single statistic or a simple 99\% percentile distribution, insufficient.
To fix this issue, for any given batch sizes we introduce diverse percentile latency distributions to profile the execution latency, which is expressed as $L(p,k)$, with CPU cores and percentiles denoted as $k$ and $p$, respectively.

\subsection{Timeout and resilience.}
The diversity in percentiles enables more opportunities to explore higher resource efficiency but comes at the risk of SLO violations.
Specifically, when setting percentiles lower than 99\%, it may incur under-estimation of function execution, making functions prone to \textit{timeout}, i.e., their actual execution latency over the estimated latency.
Timeout is expressed as follows
\begin{eqnarray}
     D(p,k) = L(P_{99},k) -L(p,k).
\end{eqnarray}

Consequently, for preventing from SLO violation, \namex has to provision more CPU cores to downstream functions to absorb previous timeout.
Moreover, we propose another metric called \emph{resilience} to quantify the absorption capability, which is expressed as follows
\begin{eqnarray}
    R(p,k)= L(p,K_M)-L(p,k),
\end{eqnarray}
where $k_M$ denotes the maximum available CPU cores.
Notably, timeout must be restricted within the upper bound of resilience, such that SLO can be guaranteed.

 








\section{Synthesizer}
\label{sec:synthesizer}

We now elaborate on the workings of the synthesizer. 
The hints synthesis process consists of two steps: hints generation and hints condensing. 

% We now discuss the two steps synthesizer takes, namely hints generating and condensing.
% Profiler prepares raw material/ingredient for synthesizer to generate hints, i.e., concise and straightforward rules, to guide providers to conduct adaptive allocation at runtime.
% For maintaining high accuracy, synthesizer requires to prepare a specific hint for each given slack.
% Yet, due to the variance and volatility of slacks, \namex may generate overwhelming hints, which not only incurs extra storage resource consumption but also  hurts the time-efficiency of adaptive allocations.
% To fix this issue, \namex effectively condenses the hints.
\subsection{Hints Generation}
\label{sec:synthesizer:generate}
To generate hints tables with high hit rates and high resource efficiency, the synthesizer requires a twofold effort.
First, it must explore all potential runtime time budgets for individual sub-workflows.
Second, the synthesizer needs to balance the trade-off between higher resource efficiency and the risk of SLO violation.
To this end, we reveal the following insights.

\mypara{Insight-1: Broad time budget range.}
The time budgets are calculated based on all possibilities between the 1st and 99th percentile (P1-P99) of the function execution time under a wide range of resource allocations, aiming to achieve high hit rates.
The range of time budgets therefore are formulated as
\begin{eqnarray}
    T_{min}= \sum_{i=1}^NL_i(1,K_{max}),
    T_{max}= \sum_{i=1}^NL_i(99,K_{min}),
\end{eqnarray}
where $K_{min}$/$K_{max}$ represents the minimum/maximum available resources,
and $N$ represents the numbers of functions in the given sub-workflows. Within this range, the synthesizer explores the potential time budgets with finer granularity in milliseconds, while evaluating their corresponding resource allocation.
The synthesizer can also be configured with higher percentiles (e.g., P99.9) to meet more stringent SLO targets.

\textbf{\textit{Insight-2: Moderate percentile exploration.}}
Diverse percentiles provide more opportunities for resource optimization, but come with exponentially higher time complexity for runtime resource adaptation.
Here, our insight is to only open percentile exploration for the head function of the current sub-workflow while fixing other functions with P99.
This moderate percentile exploration benefits the synthesizer with higher resource efficiency, derived from its attempt at lower percentiles for the head function.
Meanwhile, it effectively reduces the search space for non-head functions, allowing the synthesizer to achieve high time efficiency.

\textbf{\textit{Insight-3: Resilience-aware.}}
Despite the potential of higher resource efficiency, diverse percentile exploration may put functions at the risk of timeouts, making workflows prone to SLO violations.
To address this shortcoming, the synthesizer strictly restricts the timeout within the resilience (the achievable reduction in function execution time by scaling resource up to the maximum possible).
Within this ``safety zone", the synthesizer tries its best to maximize resource efficiency.

\mypara{Insight-4: Heavier head.}
As explained in~\S\ref{sec:bg:adaptive-allocation}, facing substantial variability of execution performance, runtime resource adaptation requires to carry out (head) function by (head) function, so as to keep its high accuracy.
This, however, may lead to sub-optimal decisions due to the mismatch between the local objective and the global objective.
Specifically, the local objective is to maximize the sub-workflow's resource efficiency, while the global objective is to maximize the whole workflow's resource efficiency.
The whole efficiency is determined by that of each sub-workflow's head function, rather than that of sub-workflows.
To address this issue, the synthesizer magnifies the local objective's weight for head functions, aiming to calibrate for the mismatch.

As for how to set the weight, our insight is to increase the weight when facing loose SLOs, and vice versa.
This is because loose SLOs indicate lower resource requirements, which brings about higher resilience (depicted in Figure~\ref{fig:exp:resilience:cores}).
Increasing the weight can better utilize this higher resilience to explore lower percentiles, such that the workflow achieves higher resource efficiency with SLO guarantees.

Hints demonstrates explicit resource allocation that can ensure the sub-workflow with its maximum resource efficiency, i.e., the minimum resource consumption, under given time budgets.
This problem thus is formulated as follows:
% \begin{eqnarray}
% 	\min && W\cdot k_1+p \cdot \sum_{i=2}^{N}k_i +(1-p)\cdot (N-1)\cdot K_M \label{eq:hints:obj}\\
% 	\text{subject to} &&
%  %percentile latency
%  L(p,k_1)+\sum_{i=2}^{N}L(99,k_i) \leq T, \label{eq:hints:time-budget}\\
%  %timeouts and resilience
% && D(p,k_1) \leq \sum_{i=2}^{N}R(99,k_i), \label{eq:hints:timeout-resilience}\\
% &&  1 ~\leq p ~\leq 99,~p \in \mathbb{Z},\\
% && 0 ~< k_i ~\leq K_{max},~k_i \in \mathbb{R}, ~\forall i.
% \end{eqnarray}
\begin{eqnarray}
	\min && W k_1+p  \sum_{i=2}^{N}k_i +(1-p) (N-1) K_{max} \label{eq:hints:obj}\\
	\text{subject to} &&
 %percentile latency
 L_1(p,k_1)+\sum_{i=2}^{N}L_i(99,k_i) \leq T, \label{eq:hints:time-budget}\\
 %timeouts and resilience
&& D_1(p,k_1) \leq \sum_{i=2}^{N}R_i(99,k_i), \label{eq:hints:timeout-resilience}\\
&&  1 \leq p \leq 99,~p \in \mathbb{Z},\\
&& K_{min}\leq k_i \leq K_{max},~k_i \in \mathbb{R}, ~\forall i.
\end{eqnarray}
where $W$ is the weight for the head function (Insight-4), and $T$ and $N$ denote the time budget and the number of functions in the sub-workflow, respectively.
%$K_{max}$ represent the maximum available CPU cores.
Notably, only the head function can explore lower percentile $p$ (Insight-2).
Equation~\ref{eq:hints:obj} expresses the sub-workflow's expected resource consumption.
Specifically, $\sum_{i=2}^{N}k_i$ and $(N-1)K_{max}$ denote non-head functions' resource requirement without and with the head function's timeout, the probability of which is $p$ and $1-p$, respectively.
Equation~\ref{eq:hints:time-budget} ensures the sub-workflow's execution latency within the time budget.
Equation~\ref{eq:hints:timeout-resilience} restricts that the possible timeout of the head function can not exceed the total resilience of downstream functions (Insight-3).

\begin{algorithm}[!t]
\small
\caption{Offline hints generation\label{alg:generate}}
 	\LinesNumbered    
    \KwIn{$\mathbf{F}= \left\langle f_1,\dots,f_N \right\rangle$: (sub-)workflow}
    \KwIn{$[T_{min},T_{max}]$: time budget range}
    \KwIn{$W,\mathbf{P}$: weight and candidate percentiles  for head function $f_1$}
	%\KwIn{$\left\langle T, \mathbf{H} \right\rangle$: time budget and hints table.}
    \KwOut{$\mathbf{H}=\{\left\langle  t, \{ k_1,\dots,k_N \} \right\rangle\}$: functions' provisioned CPU cores under given time budget $t$, i.e., hints table
	}
    $\mathbf{H} \leftarrow \emptyset$, $\mathbf{P} \leftarrow \emptyset$   \\
    \ForEach{$t \in [T_{min},T_{max}]$}
    {
    $\mathbf{H} \leftarrow \mathbf{H} \cup \{ \left \langle t, ~\texttt{generate}(\mathbf{F},t,\mathbf{P}) \right \rangle \}$ \\
    $\text{return}~\mathbf{H}$
    }
	\SetKwFunction{FMain}{\texttt{generate}}
    \SetKwProg{Fn}{Function}{:}{}
    
    \Fn{\FMain{$\mathbf{F},t,\mathbf{P}$}}{
    \If{$\left| \mathbf{F} \right| = 1$} 
    {
     \textbf{return} \texttt{min\_resource}($f_1,t$)
    }
    %$r_{min} \leftarrow \infty,~X \leftarrow \emptyset $ \\
     \If{$\mathbf{P} = \emptyset$} 
    {
     $\mathbf{P}=$\texttt{explore\_percentile}($\mathbf{F},t, K_{max}$)
    }
    $s_{min} \leftarrow \infty,~\mathbf{K} \leftarrow \emptyset $\\
    \ForEach{$p \in \mathbf{P}$}
    {
        \ForEach{$k \in [K_{min},K_{max}]$}   
        {
          $\mathbf{Z} \leftarrow $~\FMain{$\mathbf{F} \setminus f_1, t-L_1(p,k), \{P_{99}\}$}\\
          \If{$\mathbf{Z} \neq \emptyset \wedge  D(p,k) \leq \sum{R(\mathbf{Z},P_{99})}$}
          {
          $s \leftarrow W  k +p \sum{\mathbf{Z}} + (1-p)  (\left| \mathbf{F} \right| -1) K_{max}$ \\
          \If{$s \leq s_{min}$}
          {
            $s_{min} \leftarrow s, \mathbf{K} \leftarrow \{k\} \cup  \mathbf{Z}  $
          }
          }
        }            
    }
    \text{return}~$\mathbf{K}$
    }
\end{algorithm}

The algorithm for generating hints is listed in Algorithm~\ref{alg:generate}.
To ensure hints tables with high hit rates, the synthesizer explores all time budgets comprehensively (lines 2--4). 
Specifically, for a given sub-workflow $\mathbf{F}$, the synthesizer first determines the percentiles $\mathbf{P}$ that can ensure $\mathbf{F}$'s execution time below the required time budget $t$, with assuming the maximum available CPU cores for each function (lines 8--9).
Then, the synthesizer explores the resource allocation for both head and non-head functions, denoted as $k$ and $\mathbf{Z}$, under given percentile $p$. Its goal is to minimize the expected resource consumption $s$, while promising timeout $ D(p,k)$ restricted within resilience $\sum{R(\mathbf{Z},P_{99})}$ (lines 12--17).
To accelerate the generation, the synthesizer explores different percentiles concurrently.

\subsection{Hints Condensing}
\label{sec:synthesizer:condense}
The synthesizer fully utilizes the discreteness in both decision-making and decision-executing to condense hints.

\mypara{Insight-5: Repeated hints.} There are various discrete variables, such as batch sizes and CPU cores, involved in resource adaptation. 
This leads to a significant number of redundant hints that share the same adaptation decisions despite having different time budgets.

\mypara{Insight-6: Unused fields.}
The dependencies of adaptation (explained in \S\ref{sec:bg:adaptive-allocation}) compels Janus to rely solely on the fields for head functions in given hints to maintain adaptation accuracy.
Consequently, removing the fields for non-head functions helps compact the hints without compromising accuracy.
 
The algorithm for condensing hints is listed in Algorithm~\ref{alg:condense}.
Specifically, the synthesizer first sorts the given hints $\mathbf{H}$ in descending order by their time budget (line 2).
Then, it gradually fuses the hints $\mathbf{H}[l]$ that share the identical size for head function $k_1$ as shown in line 4--10.
Finally, it warps hints into a table with three fields: $T_{start}$, $T_{end}$, and $k$, indicating that the head function of the target sub-workflow should be resized to $k$ when the sub-workflow's time budgets is between $T_{start}$ and $T_{end}$.
 
 In addition, the weight for head functions impacts the decision-making.
 Thus, the synthesizer maintains individual hint tables for different weights.
 We will evaluate the effectiveness of condensing algorithm in \S\ref{exp:micro:condense}, which suggests a outstanding compression ratio without hurting accuracy. 
 
\begin{algorithm}[!t]
\small
\caption{Offline hints condensing \label{alg:condense}}
 \LinesNumbered    
    \KwIn{$\mathbf{H}= \{\left\langle t,\mathbf{K} \right\rangle$\}: raw hints table}
    \KwOut{$\mathbf{U} = [\left\langle T_{start}, T_{end}, k\right\rangle ]$: condensed hints table}
	\SetKwFunction{FMain}{\texttt{condense}}
    \SetKwProg{Fn}{Function}{:}{}
    
    \Fn{\FMain{$\mathbf{H}$}}{
    $\mathbf{H} \leftarrow$  \texttt{sort}($\mathbf{H}$)\\
    $\mathbf{U} \leftarrow \emptyset$, $q, i, j \leftarrow 0$\\
    \ForEach{$l \in [0,\left| \mathbf{H} \right |]$}
    {
        $t, \left\langle k_1,\dots, k_N \right\rangle \leftarrow \mathbf{H}[l]$ \\
        \If{$q = 0 \vee k_1 = q$}
        {
        $j \leftarrow j+1$
        }
        \Else
        {
        $\mathbf{U} \leftarrow \mathbf{U} \cup \{ \left \langle \mathbf{H}[i].t, \mathbf{H}[j].t, q \right \rangle\} $ \\
        $i, j \leftarrow l$, $q \leftarrow k_1$
        }      
    }
    \text{return}~$\mathbf{U}$
    }

\end{algorithm}





















%\section{Adapter}
\label{sec:adapter}
\begin{algorithm}[!t]
    
	\caption{Adapt resource online \jing{(necessary?)}}
 	\LinesNumbered    
    \KwIn{$\mathbf{F}= \left\langle f_1,\dots,f_N \right\rangle$: (sub-)workflow}
    \KwIn{$\mathbf{U}, T$: hints table and time budget/slack}
    \KwOut{$k$: CPU cores for head function}
	\SetKwFunction{FMain}{\texttt{adapt}}
    \SetKwProg{Fn}{Function}{:}{}
    
    \Fn{\FMain{$\mathbf{F},\mathbf{U}, T$}}{
    $ i \leftarrow \mathbf{U}.$\texttt{index}($T$)\\
    \If{$i=\emptyset$}
    {
    $\mathbf{K} \leftarrow $\texttt{generate}($\mathbf{F},T, \emptyset$) \\
    $\mathbf{U} \leftarrow $\texttt{condense}($\{\left \langle T, \mathbf{K}\right\rangle\}$)\\
    }
    \Else{
    $\mathbf{K} \leftarrow \mathbf{U}[i]$
    }
    \text{return}~$\mathbf{K}[0]$
    }
\label{alg:adapt}
\end{algorithm}

Having demonstrated how to offline synthesize and condense hints, we now focus on utilizing the hints table to accomplish online adaptation.

As outlined in \S~\ref{sec:system-overview},  after each function's execution, adapter re-evaluates the available time budget for the remaining functions while adapting the head function's size accordingly.
The procedure of adapter is listed in Algorithm~\ref{alg:adapt}.
Specifically, it first accesses the sub-workflow's hints table $\mathbf{U}$ to figure out the proper hint that can well meet the required time budget $T$ with the minimum resource consumption (Line 2).
Notably, if the above searching suffers a miss hit, due to the dynamics at runtime (detailed in~\S~\ref{sec:bg:worst-case}), adapter will revoke synthesizer to generate the proper hint on spot while updating it into the table (Line 3--5).
Moreover, adapter continuously counts the above miss. 
When it exceeds a predefined threshold, adapter argues that execution latency distribution may change, when adapter will trigger a holistic update of both profiler and synthesizer, for maintaining adaptation's accuracy.

\begin{algorithm}[!t]
    \caption{Adapt resource online \label{alg:janus:adapt}}
    \LinesNumbered    
    \KwIn{$\mathbf{F}= \left\langle f_1,\dots,f_N \right\rangle$: (sub-)workflow}
    \KwIn{$\mathbf{U}, T$: hints table and time budget/slack}
    \KwOut{$k$: CPU cores for head function}
	\SetKwFunction{FMain}{\texttt{adapt}}
    \SetKwProg{Fn}{Function}{:}{}
    
    \Fn{\FMain{$\mathbf{F},\mathbf{U}, T$}}{
    $ i \leftarrow \mathbf{U}.$\texttt{index}($T$)\\
    \If{$i=\emptyset$}
    {
    $\mathbf{K} \leftarrow $\texttt{generate}($\mathbf{F},T, \emptyset$) \\
    $\mathbf{U} \leftarrow $\texttt{condense}($\{\left \langle T, \mathbf{K}\right\rangle\}$)\\
    }
    \Else{
    $\mathbf{K} \leftarrow \mathbf{U}[i]$
    }
    \text{return}~$\mathbf{K}[0]$
    }

\end{algorithm}
\vspace{-2mm}
\section{Experimental Evaluation}
\label{sec:experiments}
We present an experimental evaluation that evaluates \sysName\ effectiveness and efficiency. We aim to address the following questions:  
$\mathbf{Q1}$: How does the quality of our generated rulesets compare to that of existing methods? $\mathbf{Q2:}$ What is the efficiency of \sysName\ and how is it affected by various data and system parameters?  



\subsection{Experimental Setup}
\label{sec:exp_setup}
\sysName\ was implemented in Python, and is publicly available in~\cite{fullversion}. 
CATE values computation was performed using the DoWhy library~\cite{dowhypaper}. The generated rules were translated into natural language using \reva{simple, manually constructed templates}.
We perform experiments on CloudLab ~\cite{Duplyakin+:ATC19} xl170 machines (10-core 2.4 GHz CPU, 64 GB RAM).
% In this section, we focus solely on the variant of our problem with statistical parity group fairness and group coverage constraints, as this represents the most challenging setting. Rule coverage and individual fairness are simpler, as they primarily involve pruning rules and can be verified in Step 2 of the algorithm, thereby reducing the search space.
The datasets, protected groups, and default parameters considered are the same as those described in \cref{sec:casestudy}.





%https://www.kaggle.com/datasets/sobhanmoosavi/us-accidents





% \brit{experiments:}
% \begin{itemize}
%     \item Case study: Compare between different definitions to see the effect of different fairness and coverage constraints
%     \item Comparison to baseline algorithms - quality in term of coverage and fairness
%     \item Comparison to baselines in terms of running times
% \end{itemize}


\vspace{1mm}
\paratitle{Baselines}
We compare \sysName\ with the following baselines:
 % \textbf{Brute-Force}: The optimal solution according to \cref{def:problem}. This algorithm implements an exhaustive search over all sets of rules.\\
\textbf{1. CauSumX}:
CauSumX \cite{DBLP:journals/pacmmod/YoungmannCGR24} is designed to find a summarized causal explanation for group-by-avg SQL query results. When applied directly to the datasets, it can be viewed as a solution to our problem with only an overall coverage constraint.
\textbf{2.IDS}~\cite{lakkaraju2016interpretable} is a framework for generating Interpretable Decision Sets for prediction tasks. IDS incorporates parameters restricting the percentage of uncovered tuples and the number of rules. These parameters were assigned the same values in our system.
\textbf{3. FRL}: The authors of \cite{chen2018optimization} proposed a framework for creating Falling Rule Lists (FRLs) as a probabilistic classification model. FRLs comprise if-then rules with antecedents in the if-clauses and probabilities of the desired outcome in the then-clauses, ordered based on associated probabilities.
% \textbf{Explanation Table}: The authors of \cite{el2014interpretable} introduced an efficient method to generate \emph{explanation tables} for multi-dimensional datasets. The proposed algorithm employs an information-theoretic approach to select patterns that provide
% the most information gain about the distribution of the outcome attribute. 
% \brit{a variant with fairness?}



\smallskip
Since IDS and FRL assume a binary outcome, we binned the salary variable in SO using the average value. To address fairness considerations, we run the baseline algorithms twice (excluding Brute-Force): Once on the entire dataset to obtain a set of rules applicable to the entire population, and again solely on the tuples belonging to the protected population to generate rules specifically tailored for them. \revb{We report the number of rules generated by the baselines, their coverage, and runtime. To compare the expected utility, we proceed as follows: The rules generated by IDS and FRL are prediction rules (e.g., IF owning a house = YES, THEN credit score = 1). As such, these rules do not provide an intervention to improve outcomes. We, therefore, treat the IF clauses
in two manners: (1) IF clauses as the selected grouping patterns and then apply step 2 (\cref{subsec:treatment_patterns}) of \sysName\ to determine the intervention patterns; (2) IF clauses as the selected intervention patterns, where the grouping pattern is the entire data. }
% For the resulting set of prescription rules, we report the expected utility for both protected and non-protected groups. }

% The final solution for each baseline is considered the union of these two sets of rules.



% \vspace{1mm}
% Unless otherwise specified, the overall coverage threshold as well as the coverage threshold for the protected group are set to 0.75. The threshold of the Apriori algorithm is set to 0.1. 
% The threshold for the SP fairness constraint is set at \brit{?}, and the threshold for the BGL fairness constraint is set at \brit{?}. 
% The time cutoff is set to $3$ hours. 


% \subsection{Problem Variants Evaluation (Q1)}
% \label{exp:problem_variants}

% \brit{here we can focus on only two datasets, and show the rules with different constraints (to motivate the need for different problem variants). (fill the cells in Table \ref{tab:problem_variants})}




\subsection{Quality Evaluation (Q1)}
\label{exp:quality}
We compare the set of rules chosen by each baseline and \sysName. 

\begin{figure}[t]
    \centering
    \vspace{-3mm}\includegraphics[width=0.46\textwidth]{figs/time_barchart.pdf}
%     \begin{subfigure}[b]{0.23\textwidth}
%         \centering
% \includegraphics[width=\textwidth]{example-image-a}
%         \caption{Stack Overflow}
%         \label{fig:first}
%     \end{subfigure}
%     \hfill
%     \begin{subfigure}[b]{0.23\textwidth}
%         \centering
% \includegraphics[width=\textwidth]{example-image-b}
%         \caption{German Credit}
%         \label{fig:second}
%     \end{subfigure}
    \caption{Runtime by-step of the \sysName\ algorithm (SO)}
    \label{fig:runtime_by_step}
\end{figure}

\paratitle{Stack Overflow} As discussed in \cref{sec:casestudy}, prescription rules selected without fairness constraints, similar to the behavior of CauSumX, were significantly more advantageous for non-protected.  
The rules generated by IDS do not suggest interventions to improve outcomes. For example, one rule states that if Country = Turkey and Age = 18-24 years, then the expected salary is low (with the outcome binned). Another key distinction is that these rules are not causal, as they are based on correlations in the data. For example, one rule indicates that if the years coding = 0-2 and Sexual Orientation = Gay or Lesbian, then the expected salary is low. Similarly, rules generated by FRL do not propose interventions to improve outcomes and are not causal. For example, one rule states that if Country = US and Sexual Orientation = Straight or Heterosexual, then the expected salary is high. 
In contrast, \sysName\ generates interventions aimed at improving the outcome by leveraging causal relationships. It also allows users to impose fairness constraints, ensuring that the protected group benefits from these interventions.




% IDS generated 16 rules for the overall population and 21 rules for the protected group. Notably, these rules do not suggest interventions to improve outcomes. For example, one rule states that if Country = Turkey and Age = 18-24 years, then the expected salary is low (with the outcome binned). Another key distinction is that these rules are not causal, as they are based solely on correlations in the data. For example, one rule indicates that if the years coding = 0-2 and Sexual Orientation = Gay or Lesbian, then the expected salary is low. \brit{TODO}


% The FRL baseline generated 9 rules for the overall population and 7 for the protected group. Similar to the IDS baseline, these rules do not propose interventions to improve outcomes and are not causal. For example, one rule states that if Country = United States and Sexual Orientation = Straight or Heterosexual, then the expected salary is high. 
%   \brit{TODO}

% \brit{IDS full: 16 rules, 64 seconds, IDS protected: 21 rules, 12 seconds}
% \brit{FRL full: 9 rules 1225 seconds, FRL protected: 7 rules, 478 seconds}

\paratitle{German}
Here again, with no fairness constraint (akin to CauSumX), the selected rules were mostly beneficial for the non-protected. 
% IDS generated 12 rules for the overall population and 20 for the protected group. 
Here again, the rules generated by IDS are not causal and do not offer an intervention. For example, one of the rules suggested that single females at the age of 35-41 are unlikely to get a loan.  
% \brit{TODO}
% FRL generated 13 rules for the overall population and 11 for the protected group. 
As before, the rules generated by FRL are also not causal and do not propose ways to improve the credit risk score. For example, one rule suggests that if a person has lived in a house for 4-7 years, their credit risk score is likely to be high. Another rule states that if the purpose of the loan is to buy a used car, the credit risk score is also likely to be high. Clearly, these rules rely on correlations in the data rather than causal relationships.
In contrast, \sysName\ generated a ruleset that offers interventions to improve the credit risk score based on causal relationships. Example selected rules are shown in \cref{sec:casestudy}.



\vspace{1mm}
\revb{We report the solution size, coverage, expected utility for protected and non-protected, and the unfairness of the rulesets generated using IDS and FRL (as explained in \cref{sec:exp_setup}).
The results are presented in 
{\bf \cref{tab:problem_variants}}. Notably, the expected utility for both protected and non-protected groups across both datasets is generally lower than that achieved by \sysName. \sysName\ consistently delivers higher expected utility for both groups and a smaller difference between these values. This indicates that our approach to mining grouping and intervention patterns is more effective than relying on these algorithms for the same purpose.  However, we note that the rules in IDS and FRL had different objectives (prediction accuracy) and had to be adapted for quantitative comparison using our measures.} 

% \nativ{comment}
% \brit{IDS full: 12 rules 4 seconds, IDS protected: 20 rules, 4 seconds}
% \brit{FRL full: 13 rules 273 seconds, FRL protected: 11 rules 279 seconds}

\subsubsection{\reva{Robustness to the Causal DAG}}
\label{subsec:causal_DAG_robustness}
\reva{The quality of the generated rules may depend on the accuracy of the underlying causal DAG. To evaluate this, we examine the impact of different causal DAGs on the rules. The causal DAGs considered are as follows:
{
\textbf{(1) 1-layer Indep DAG:} A causal DAG where all attributes are independent of each other and only impact the outcome. This setting similar to the scenario where all the causal graph is ignored.
\textbf{(2) 2-layer Mutable DAG:} A simplified DAG where immutable attributes affect the mutable attributes, which impact the outcome variable. In this graph, all immutable attributes act as confounders but do not directly impact the outcome.
%Another default causal DAG where all immutable attributes point to mutable attributes, which in turn point to the outcome.  
\textbf{(3) 2-layer DAG:} A simplified DAG where all variables affect the outcome but the mutable attributes are also confounded by all immutable attributes. }
%Another 2-layer causal DAG. In this DAG, to include confounding variables, all edges in the default-2-layer DAG are present, with additional edges from the top layer to the outcome.  
\textbf{(4) PC DAG:} A causal DAG generated by the PC causal discovery algorithm~\cite{spirtes2001causation}}. 



\reva{The results are depicted in \cref{tab:causal_dag_variants}. We report the expected utility as computed over the different causal DAGs. We observe that the expected utility remains similar for the Stack overflow dataset, demonstrating robustness towards the choice of causal dag. The results show some variability in German credit. However, the PC DAG and the original causal DAG are the most accurate (as they are based on the data distribution and domain knowledge) and achieve the highest coverage and expected utility.}



\begin{table*}[h!]
\centering
\small
\caption{\reva{Metrics Comparison with different Causal DAGs. 
%In parenthesis are the expected utility values computed on the original  causal DAG.
}}
\label{tab:causal_dag_variants}
\begin{tabular}{p{40mm}ccccccc}
\toprule
\textbf{Stack Overflow (SP group fairness + group coverage)} & \textbf{\# rules} & \textbf{coverage} & \textbf{coverage pro} & \textbf{exp utility} & \textbf{exp utility non-pro}&\textbf{exp utility pro} &\textbf{unfairness} \\

\midrule 

Original causal DAG  & 11& 97.95\%& 98.85\%& 27934.76& 28144.58& 18145.23& 9999.35\\



\reva{1-Layer Indep DAG} &\reva{11}&\reva{98.38\%} & \reva{98.38\%}&\reva{28110.19}& \reva{28117}
&\reva{18117.45}
&\reva{9972} \\


% expected_utility’: 28110.19, ‘unprotected_expected_utility’: 28117.0, ‘protected_expected_utility’: 18117.45, ‘coverage_rate’: ’98.38%’, ‘protected_coverage_rate’: ’98.83%


\reva{2-Layer Mutable DAG} &\reva{10}	
&\reva{97.7\%}
 &\reva{98.4\%} &\reva{28198.59}&\reva{28193.09} &\reva{18193.23
}&\reva{9999.86} \\


\reva{2-Layer DAG} &\reva{10}	
&\reva{98.47\%}
 &\reva{98.87\%} &\reva{28106.4}&\reva{28211.17} &\reva{18211.4}&\reva{9999.77} \\

\reva{PC DAG} &\reva{10}&\reva{97.7\%} &\reva{98.4\%} &\reva{28198.59}&\reva{28193.09} &\reva{18193.23}&\reva{9999.86} \\

        
\midrule

\textbf{German Credit (BGL group fairness + group coverage)} & \textbf{\# rules} & \textbf{coverage} & \textbf{coverage pro} & \textbf{exp utility} & \textbf{exp utility non-pro}&\textbf{exp utility pro}&\textbf{unfairness}   \\
\midrule 

Original causal DAG  & 6& 100.0\%& 100.0\%& 0.36& 0.37& 0.31& 0.06 \\
\reva{1-Layer Indep DAG} &\reva{12}&\reva{100\%} &\reva{100\%} &\reva{0.31}& \reva{0.31}&\reva{0.29}&\reva{0.02} \\
\reva{2-Layer Mutable DAG}&\reva{13} &\reva{76.20\%}		
&\reva{79.35\%} & \reva{0.22}&\reva{0.22}&\reva{0.2} &\reva{0.02} \\

\reva{2-Layer DAG} &\reva{11}	
&\reva{71.20\%}
 &\reva{73.91\%} &\reva{0.26}&\reva{0.25} &\reva{0.23}&\reva{0.02} \\

\reva{PC DAG} &\reva{24}&\reva{100.00\%}	
 &\reva{100.00\%} &\reva{0.39}&\reva{0.39} &\reva{0.26}&\reva{0.13} \\
\bottomrule
\end{tabular}
%\vspace{-3mm}
\end{table*}



\subsection{Scalability Evaluation (Q2)}
\label{exp:scalability}
% In this section, we omit the results for the baselines from the presentation, as their response times are significantly slower.


% \begin{figure}[h!]
%     \centering
%     \begin{subfigure}[b]{0.23\textwidth}
%         \centering
% \includegraphics[width=\textwidth]{example-image-a}
%         \caption{Stack Overflow}
%         \label{fig:first}
%     \end{subfigure}
%     \hfill
%     \begin{subfigure}[b]{0.23\textwidth}
%         \centering
% \includegraphics[width=\textwidth]{example-image-b}
%         \caption{German Credit}
%         \label{fig:second}
%     \end{subfigure}
%     \caption{Runtime by-step of the \sysName\ algorithm}
%     \label{fig:runtime_by_step}
% \end{figure}


\begin{figure}[t]
    %\vspace{-2mm}
    \begin{subfigure}[b]{0.46\textwidth}
        \centering
        \includegraphics[width=0.6\textwidth]{figs/time_v_size.pdf}
        % \caption{Stack Overflow}
        % \label{fig:first}
    \end{subfigure}
    %\vspace{-mm}
    \caption{\revb{Runtime as a function of the dataset size (SO)}}
\label{fig:runtime_dataset_size}
\end{figure}

\begin{figure}[t]
    \vspace{-5mm}
    \centering
        \begin{subfigure}[b]{0.48\textwidth}
        \centering
        \includegraphics[width=\textwidth]{figs/time_v_num_attr_line.pdf}
        \end{subfigure}
    \caption{\revb{Runtime as a function of number of mutable and immutable attributes for SO with statistical parity}}
\label{fig:runtime_attributes}
\end{figure}



% \begin{figure}[h!]
%     \centering
%     \begin{subfigure}[b]{0.23\textwidth}
%         \centering
% \includegraphics[width=\textwidth]{figs/time_v_num_immutable_line.pdf}
%         % \caption{Stack Overflow}
%          \label{fig:immutable}
%     \end{subfigure}
%     \vspace{-1mm}
%     \caption{Runtime as a function of number of immutable attributes}
%     \label{fig:runtime_immutable_attributes}
% \end{figure}

\paratitle{Breakdown analysis by step}
Figure~\ref{fig:runtime_by_step} shows the runtime comparison of \sysName for different problem settings. Observe that using rule coverage constraint has the lowest runtime because it helps to prune rules which do not satisfy the coverage constraint. Employing rule coverage with individual fairness is the fastest among all settings, while no constraint setting takes the longest time.
The time taken by the group mining phase is less than $2$ seconds across all setups, and is therefore not visible in the plot. The intervention mining phase (Step 2) is the most inefficient phase, which takes around $6$ mins for the unconstrained setting. The running time of these components aligns with our time complexity analysis (\cref{sec:algo}). Due to space restrictions, we do not present the corresponding plot for German dataset. All conclusions remain the same but the overall running time is $\approx 10\times$ faster due to its smaller size.

\revb{The running time of \sysName and the baselines is comparable. 
FRL is an order of magnitude slower than IDS because it uses a Bayesian modeling approach to simultaneously select a subset of rules and determine their optimal order, which involves solving a computationally intensive combinatorial problem. In contrast, IDS leverages submodular optimization on an unordered set of rules, significantly reducing the size of the search.}
%
We now analyze the impact of system parameters and data size on performance. 


\smallskip
\paratitle{Data Size} \Cref{fig:runtime_dataset_size} compares the running time of \sysName\ \revb{and the baselines} for varying dataset sizes. We observe that the time taken by \sysName\  \revb{and the baselines} increases linearly for most of the settings, \revb{with \sysName\ demonstrating a runtime comparable to IDS under certain configurations}. We also observed that the quality of rules returned by sampling $25\%$ of the data points is comparable with the rules returned by using the whole dataset. Therefore, sampling-based optimizations can help to reduce the running time from $11$ min to less than $2$ min for the unconstrained setting and less than a minute with fairness constraints. 

%We analyze the impact of dataset size on runtime through random sampling of tuples. The results are shown in \cref{fig:runtime_dataset_size}. \brit{TODO}



\smallskip
\paratitle{Number of Attributes} Figure~\ref{fig:runtime_attributes} shows the runtime of \sysName\ while increasing the number of mutable and immutable attributes. 
On increasing the number of mutable attributes, the number of intervention patterns increases exponentially while on increasing immutable attributes, the number of grouping patterns increases exponentially. Therefore, both have a similar impact on runtime. \revb{IDS and FRL do not distinguish between mutable and immutable attributes and there the runtime increases slightly due to an increase in the number of attributes, as more rules are considered.}
%Comparing the two plots, we can see that the running time is dependent on the total number of attributes and not just the number of mutable attributes. However, the reasons of 
% We observe that the running time increases with increasing num
% We examine the impact of attribute quantity on runtime, by randomly excluding attributes from consideration. The results are shown in \cref{fig:runtime_attributes}. \brit{TODO}

\reva{In the following, we omit the results for the IDS and FRL baselines, as these parameters do not impact their runtime.}

\smallskip
\paratitle{Fairness Threshold} %We examine the effect of the threshold $\epsilon$ used to assess the group fairness constraint (\cref{subsec:fairness_constraint}.
\Cref{tab:fairness_variants} presents the results for varying $\epsilon$ for group and individual fairness. We observe that the unfairness of the returned solution increases with the increase in $\epsilon$.  Additionally, the overall expected utility increases but the expected utility of the protected individuals decreases. This result matches our intuition as highly unfair rules are selected for higher values of $\epsilon$. We also notice that the greedy algorithm satisfies the group fairness constraint in all scenarios (unfairness is always less than the desired threshold).

For individual fairness, the overall utility increases monotonically with $\epsilon$. However, the rate of growth for individual fairness is slower than that of group fairness.
One interesting observation about individual fairness is that when all rules have statistical parity difference less than $2500$, the overall unfairness is still around $11K$. This sudden increase in unfairness when considering multiple fair rules together is because we evaluate the upper bound of unfairness by taking the difference between max utility of unprotected and min utility of protected individuals. On manual inspection, we observed that all rules are indeed individually fair.



% In case of individual fairness,  unfair rules are chosen

% As the value of $\epsilon$ increases, the fairness of the solution may decrease. \brit{todo}

\smallskip
\paratitle{Coverage Threshold} With the change in coverage thresholds, we do not observe major difference in the overall results because the majority of the rules exhibit very high coverage (\cref{tab:problem_variants}). %We examine the effect of coverage the thresholds $\theta$ and $\theta_p$. 

\smallskip
\paratitle{Apriori Threshold} 
We observe that increasing the Apriori threshold $\tau$ leads to a reduction in the number of grouping patterns considered, and thus to a decrease in runtime. However, our findings indicate that higher $\tau$ values lead to a decrease in both utility and fairness. Based on our findings, we recommend using a default value of $0.1$, which provides satisfactory results in terms of coverage, utility, fairness and runtime.

%used to define the group coverage constraint (\cref{secsec:coverage_constraint}). As the value of $\theta$ and $\theta_p$ increase, \brit{todo}

% \brit{Here we compare the results with existing baselines in terms of running times. We then isolate each phase of the algo to investigate its effect}

% \brit{Examine the effect on running times when varying: (1) number of tuples (2) number of attributes (3) threshold of apriori (4) coverage constraint (5) fairness constraint}

% \brit{add an experiment to show how we operate with a default causal DAG (everything affects the outcome)}





\section{Related Work}
\label{sec:related-work}
%We now contextualize our work with related literature so that our contributions are highlighted. We cover FMTS, perturbations in time-series, 
% robustness testing of FMs, 
%and rating of AI systems. 

\noindent \textbf{Foundation Models Supporting Time Series} 
The use of FMs for time series forecasting has advanced significantly. 
% \cite{lu2022frozen} first demonstrated that transformers pre-trained on text data (LLMs) can effectively solve sequence modeling tasks in other modalities, paving the way for leveraging language pre-trained transformers for time series analysis. Recent studies have focused on reprogramming LLMs for time series tasks through parameter-efficient fine-tuning and suitable tokenization strategies \cite{zhou2023one, gruver2024large, jin2023time, cao2023tempo, ekambaram2024tiny}. These methods have successfully adapted transformers to the unique challenges of time series forecasting. \cite{zhou2023one} and \cite{jin2023time} further illustrate the versatility and robustness of fine-tuned language pre-trained transformers for diverse time series tasks.
\cite{lu2022frozen} showed that transformers pre-trained on text data can solve sequence modeling tasks in other modalities, enabling their application to time series analysis. Recent studies have reprogrammed LLMs for time series tasks through parameter-efficient fine-tuning and tokenization strategies \cite{zhou2023one, gruver2024large, jin2023time, cao2023tempo, ekambaram2024tiny}. 
% These methods have successfully adapted transformers to the unique challenges of time series forecasting. 
\cite{zhou2023one} and \cite{jin2023time} further illustrate the versatility and robustness of fine-tuned language pre-trained transformers for diverse time series tasks.
% Several models have contributed to the advancement of time series forecasting. \cite{ansari2024chronos} and \cite{woo2024unified} have improved forecasting accuracy and model generalization.  
% % \cite{ansari2024chronos} and \cite{woo2024unified} have pushed the boundaries of forecasting accuracy and model generalization. 
% \cite{rasul2023lag} and \cite{das2023decoder} have explored new tokenization strategies and fine-tuning methods to improve model performance. Additionally, \cite{garza2023timegpt} and \cite{ekambaram2024tiny} have focused on creating lightweight and efficient models for real-time applications. \cite{talukder2024totem} stands out with its unique approach to integrating multiple temporal patterns, enhancing forecasting precision.
% FMs trained from scratch have achieved SOTA on time series tasks. Zero-shot forecasting, exemplified by \cite{gruver2024large}, showcases the ability of these models to make accurate predictions without domain-specific training. \cite{cao2023tempo} and \cite{goswami2024moment} have introduced approaches to enhance the performance and efficiency of time series models, leveraging transformer architectures to capture temporal dependencies more effectively. In our experiments, we select Gemini-V and Phi-3 as the GP models and Chronos and MOMENT as TS models due to their SOTA performance in their respective categories.
Several models have advanced time series forecasting. \cite{ansari2024chronos} and \cite{woo2024unified} have improved forecasting accuracy and model generalization, while
% \cite{ansari2024chronos} and \cite{woo2024unified} have pushed the boundaries of forecasting accuracy and model generalization. 
\cite{rasul2023lag} and \cite{das2023decoder} have explored new tokenization strategies and fine-tuning methods. \cite{garza2023timegpt} and \cite{ekambaram2024tiny} developed lightweight models for real-time applications, and \cite{talukder2024totem} integrated multiple temporal patterns to improve precision. FMs trained from scratch, like \cite{gruver2024large}, achieved SOTA in zero-shot forecasting, with \cite{cao2023tempo} and \cite{goswami2024moment} further improving model performance. 
%In our experiments, we select Gemini-V and Phi-3 as the GP models and Chronos and MOMENT as TS models due to their SOTA performance in their respective categories.
Please see Section~\ref{sec:exp_app} for the FMTS we selected due to their SOTA performance in their respective categories.

%The use of FMs for time series forecasting has seen significant advancements in recent years. \cite{lu2022frozen} first demonstrated that transformers pre-trained on text data (LLMs) can effectively solve sequence modeling tasks in other modalities. This work opened the door to leveraging language pre-trained transformers for time series analysis. Recent studies have built on this foundation, focusing on reprogramming LLMs for time series tasks through parameter-efficient fine-tuning and suitable tokenization strategies \cite{zhou2023one, gruver2024large, jin2023time, cao2023tempo, ekambaram2024tiny}. These methods have proven successful in adapting the powerful capabilities of transformers to the unique challenges of time series forecasting. OneFitsAll \cite{zhou2023one} and Time-LLM \cite{jin2023time} further illustrate how language pre-trained transformers can be fine-tuned for diverse time series tasks, demonstrating their versatility and robustness. 
% \zhen{reason why we didn't include these models in our study, weights not available? or other justification, to prevent that naturally raised question from readers.}\kl{Good point. We need to discuss. I added 2 sentences at the bottom but they are probably not very convincing.}
%Several other models have contributed to the advancement of time series forecasting. Chronos \cite{ansari2024chronos} and Moirai \cite{woo2024unified} have pushed the boundaries of forecasting accuracy and model generalization. Lag-llama \cite{rasul2023lag} and TimesFM \cite{das2023decoder} have explored new tokenization strategies and fine-tuning methods to improve model performance. Additionally, Time-GPT1 \cite{garza2023timegpt} and Tiny-Time Mixers \cite{ekambaram2024tiny} have focused on creating lightweight and efficient models suitable for real-time applications. TOTEM \cite{talukder2024totem} stands out with its unique approach to integrating multiple temporal patterns, further enhancing forecasting precision.
%Aside from reprogramming LLMs for time series, FMs trained from scratch have achieved SOTA on times series tasks. 
%Zero-shot forecasting, exemplified by \cite{gruver2024large}, showcases the ability of these models to make accurate predictions without domain-specific training.  TEMPO \cite{cao2023tempo} and MOMENT \cite{goswami2024moment} have introduced approaches to enhance the performance and efficiency of time series models, leveraging transformer architectures to capture temporal dependencies more effectively.
% \zhen{and these are on various time series tasks including time series forecasting?}
% \zhen{These are models that are specifically trained for time series forecasting, I'd suggest mentioning them first after the LLM reprogramming, and then expanding to the models that are trained across time series tasks instead. The flow of this subsection feels a bit odd as of now.} \kl{Done.}
%In our experiments, we select Gemini-V and Phi-3 as the GP models and Chronos and MOMENT as TS models due to their SOTA performance in their respective categories. 

%\vspace{-0.3em}
\noindent \textbf{Perturbations in Time Series Data} TS data is commonly stored in spreadsheets and databases, which are prone to changes due to acts of omission (e.g., negligence, data-entry errors) or commission (e.g., adversarial attacks, sabotage). Omission errors are most common \cite{spreadsheets-errors-risks-survey}. Tools like Microsoft Excel and Google Sheets are widely used for data collection and analysis, allowing end-user programming \cite{spreadsheets-future-workshop}. However, over 90\% of spreadsheets contain errors due to issues like incorrect formulae, leading to multi-billion dollar losses \cite{spreadsheet-qa-survey}.
%\cite{spreadsheet-qa-survey,spreadsheets-errors-risks-survey}.
Adversarial attacks are also increasing in data stores and AI models for tasks like forecasting.
% \cite{papernot2016transferability} introduced a black-box attack method using a substitute model to generate adversarial examples, demonstrating transferability across tasks. \cite{baluja2017adversarial} focused on white-box attacks using gradient information. 
\cite{karim2019adversarial} adapted these concepts to time series, exploring both black-box and white-box attacks. \cite{oregi2018adversarial} revealed the vulnerability of distance-based classifiers. \cite{rathore2020untargeted} examined various adversarial attacks on time series classifiers. TSFool \cite{li2022tsfool} introduced a multi-objective black-box attack to craft imperceptible adversarial time series to fool RNN classifiers.
%Time series (TS) data is widely stored and manipulated in spreadsheets and databases. These are also the tools which see considerable changes or perturbations due to acts of omission that are unintended (e.g., negligence, data-entry errors) or commission which are deliberate (e.g., adversarial attacks, sabotage). 
%Among these, changes due to omission are most common \cite{spreadsheets-errors-risks-survey}.
%For example, a spreadsheet, implemented in tools like Microsoft Excel and Google Sheets, is a common data collection and analysis environment that also allows end-user programming \cite{spreadsheets-future-workshop}. They are used widely at the workplace and are often a door opener to more advanced scientific tools. But gaining expertise in them needs practice since a large proportion of spreadsheets ($\succ$ 90\%) are known to have errors due to issues like incorrect formulae caused by improper understanding of behavior during routine operations like copy-paste and end-user programming, which have caused losses of multi-billion dollars \cite{spreadsheet-qa-survey,spreadsheets-errors-risks-survey}.
% \zhen{do we need to relate our perturbations to these attacks? otherwise, we must manage the readers' expectations on what types of perturbations we focus on other than adversarial attacks, and motivate it properly}
%\zhen{Play down this a bit, and emphasize and justify why we focus on the type of perturbations we consider in the paper, to mimic operational errors in practice apart from adversarial attacks, citing the 2024 and 1996 papers Biplav added.} 
%Furthermore, adversarial attacks are also increasing both in data stores and in AI models created to solve tasks like forecasting.
%Foundational work by ~\cite{papernot2016transferability} introduced a black-box attack method that involved training a substitute model to generate adversarial examples capable of misleading the target model, demonstrating the transferability property across similar tasks. In contrast, research by ~\cite{baluja2017adversarial} focused on white-box attacks, using gradient information and probabilistic outputs to craft adversarial examples. Researchers~\cite{karim2019adversarial} have adapted these concepts to the time series domain, exploring both black-box and white-box attacks on time series classification models. In addition, ~\cite{oregi2018adversarial} revealed the susceptibility of distance-based time-series classifiers to adversarial examples. ~\cite{rathore2020untargeted} examined untargeted, targeted, and universal adversarial attacks on time series classifiers, demonstrating the effectiveness of these attacks across various datasets. TSFool~\cite{li2022tsfool} introduced a multi-objective black-box attack to craft highly imperceptible adversarial time series to fool RNN classifiers.
%Adversarial attacks on time-series data are initially focused on time-series classification tasks, leveraging concepts adapted from adversarial attacks in other domains.
%explored adversarial sample crafting for time series classification using elastic similarity measures,  %These works collectively underscore the ongoing efforts to understand and mitigate the risks posed by adversarial attacks on time series classification models.
% More recently, research into adversarial attacks on time series forecasting models has revealed distinct challenges and novel attack strategies. One primary challenge is targeted attacks. While targeted adversarial attacks on time series classification aim to misclassify specific instances, achieving similar precision in time series forecasting is more complex due to the sequential nature of the data. Perturbations must be designed to influence specific aspects of the forecast (e.g., directional shifts or amplitude changes) without disrupting the overall temporal dependencies, making precise control more challenging~\cite{govindarajulu2023targeted}. Another challenge is attacks on multivariate forecasting. Adversarial attacks could exploit the inter-dependences between variables. ~\cite{liu2022robust} introduced sparse and indirect cross-time-series attacks in multivariate settings, which are more effective and realistic than direct attacks in univariate cases.
% \zhen{Biplav, could we make a quick comment here as well that we focus more on data error side in practice, other than attacks? and cite the paper that you mentioned on data errors? Otherwise this section of adversarial attacks feel a bit standalone to other sections}
%These challenges underscore the need for ongoing research to develop effective adversarial attack strategies and robust defense mechanisms tailored to the unique characteristics of time series forecasting models.
% -----


\noindent \textbf{Rating AI Systems} Several works have assessed and rated AI systems for trustworthiness from a third-party perspective without access to training data. \cite{srivastava2020rating} proposed a method to rate AI systems for bias, specifically targeting gender bias in machine translators \cite{srivastava2018towards}, and used visualizations to communicate these ratings \cite{bernagozzi2021vega}. They conducted user studies on trust perception through visualizations \cite{vega-userstudy-translatorbias}, but these lacked causal interpretation. \cite{kausik2024rating} introduced a causal analysis approach to rate bias in sentiment analysis systems, extending it to assess their impact when used with translators \cite{kausik2023the}. We extend their method to rate MM-TSFM for robustness against perturbations. Causal analysis offers advantages over statistical analysis by determining accountability, aligning with humanistic values, and quantifying the direct influence of various attributes on forecasting accuracy.


% \section{Discussion}
\label{sec:discuss}
\jing{Janus maintains a warm pod pool to decrease the impact of cold start.
Despite its simplicity, this pool may inevitably compromise resource efficiency.
Additionally, Janus may exhibit sub-optimal resource adaptation for dynamic workflows with uncertain execution paths.
This uncertainty exacerbates the dependencies of adaptation decisions (as detailed in \S\ref{sec:bg:adaptive-allocation}), potentially leading to sub-optimal adaptation especially for functions shared across different paths.}







% In this work, we introduced \segsub, a Segmentation Substitution framework designed to improve the robustness of visual reasoning in VLMs. Through the application of image segmentation and inpainting techniques, we augment VQA datasets with counterfactual samples and knowledge conflicts. These samples test LLMs' abilities to recognize and respond to various types of image-based reasoning challenges. Our experiments demonstrate that while VLMs show resilience to certain perturbations such as feature modifications that lie within their training distribution, they struggle with counterfactual cases and inconsistencies across multiple image sources, especially in multi-hop scenarios.

% Our findings highlight the need for robust VQA models that can navigate diverse visual contexts. We hope our contribution to advancing visual reasoning and model resilience against counterfactual noise will encourage future research in this area. 

% The \segsub framework serves as a tool for strengthening VQA tasks, advancing the study of multimodal reasoning in real-world applications.

We introduce \segsub, a framework designed to improve the robustness of visual reasoning in VLMs. Through the application of image segmentation and inpainting techniques, we augment VQA datasets with parametric, source and counterfactual conflicts. These samples test LLMs' abilities to recognize and respond to various types of image-based reasoning challenges. While our experiments demonstrate VLM resilience to perturbations that lie within their training distribution (i.e. feature modifications that induce parametric conflicts), they struggle with counterfactual cases and conflicts across multiple image sources, especially in multi-hop scenarios. Our findings highlight the need for VQA models that are robust to knowledge conflicts and we hope that our contribution will inspire future research in advancing visual reasoning. 

% Shorter
% We introduced \segsub, a framework that enhances visual reasoning in VLMs by augmenting VQA datasets with counterfactual samples and knowledge conflicts through segmentation and inpainting. Our experiments show that while VLMs are resilient to perturbations within their training distribution, they struggle with counterfactuals and inconsistencies in multi-image contexts, particularly in multi-hop tasks. These results emphasize the need for more robust VQA models capable of handling diverse visual challenges, encouraging further research in counterfactual noise and visual reasoning


% The \segsub framework serves as a tool 

\section*{Acknowledgment}
This work was supported in part by National Science Foundation of China under grant 62232012, in part by National Key Research \&
Development (R\&D) Plan under grant 2022YFB4501703, in part by the Major Key Project of PCL under Grant PCL2024A06 and PCL2022A05, and in part by the Shenzhen Science and Technology Program under Grant RCJC20231211085918010.
% The preferred spelling of the word ``acknowledgment'' in America is without 
% an ``e'' after the ``g''. Avoid the stilted expression ``one of us (R. B. 
% G.) thanks $\ldots$''. Instead, try ``R. B. G. thanks$\ldots$''. Put sponsor 
% acknowledgments in the unnumbered footnote on the first page.

\bibliographystyle{IEEEtran}
\bibliography{ref}

\end{document}
