\documentclass[conference]{IEEEtran}
%\IEEEoverridecommandlockouts
%\IEEEspecialpapernotice{2023 IEEE International Conference on Example, July 2023}

% The preceding line is only needed to identify funding in the first footnote. If that is unneeded, please comment it out.
%Template version as of 6/27/2024
\usepackage{pifont}
\usepackage{xspace}
\usepackage{colortbl}
\usepackage{xcolor}
\usepackage{array} 
\newcommand{\namex}{Janus\xspace}
\newcommand{\mypara}[1]{\textbf{\textit{#1}}}
\newcommand{\jing}[1]{\textcolor{black}{#1}}
\usepackage{subfig}
\newcommand{\qf}[1]{\textcolor{brown}{#1}}
\usepackage[ruled,vlined]{algorithm2e}
\usepackage{booktabs}
\usepackage{cite}
\usepackage{amsmath,amssymb,amsfonts}
\usepackage{algorithmic}
\usepackage{graphicx}
\usepackage{textcomp}


\def\BibTeX{{\rm B\kern-.05em{\sc i\kern-.025em b}\kern-.08em
    T\kern-.1667em\lower.7ex\hbox{E}\kern-.125emX}}

\begin{document}

\title{It Takes Two to Tango: Serverless Workflow Serving via Bilaterally Engaged Resource Adaptation
}
%\title{Efficient Serverless Workflow Serving via Bilaterally Engaged Resource Adaptation}


\author{
\IEEEauthorblockN{
Jing Wu\textsuperscript{1},
Lin Wang\textsuperscript{2}, 
Quanfeng Deng\textsuperscript{1}, 
Chen Yu\textsuperscript{1}, 
Dong Zhang\textsuperscript{3}, 
Bingheng Yan\textsuperscript{3}, 
Fangming Liu\textsuperscript{*1,4}}
\IEEEauthorblockA{\textsuperscript{1}\textit{
National Engineering Research Center for Big Data Technology and System,}\\
\textit{
Services Computing Technology and System Lab, Cluster and Grid Computing Lab,}\\
\textit{
Huazhong University of Science and Technology, China} \\
\textsuperscript{2}\textit{Paderborn University, Germany } \\
\textsuperscript{3}\textit{Inspur Data Co., Ltd., China}\\
\textsuperscript{4}\textit{Peng Cheng Laboratory, China}\\
Email: wujinghust@hust.edu.cn, lin.wang@uni-paderborn.de, quanfengdeng@foxmail.com, \\
yuchen@hust.edu.cn, \{zhangdong, yanbh\}@inspur.com, fangminghk@gmail.com}
% \IEEEcompsocitemizethanks{
% \IEEEcompsocthanksitem \textsuperscript{*§} Corresponding authors.
% }
}

% \author{
% \IEEEauthorblockN{1\textsuperscript{st} Given Name Surname}
% \IEEEauthorblockA{\textit{dept. name of organization (of Aff.)} \\
% \textit{name of organization (of Aff.)}\\
% City, Country \\
% email address or ORCID}
% \and
% \IEEEauthorblockN{2\textsuperscript{nd} Given Name Surname}
% \IEEEauthorblockA{\textit{dept. name of organization (of Aff.)} \\
% \textit{name of organization (of Aff.)}\\
% City, Country \\
% email address or ORCID}
% \and
% \IEEEauthorblockN{3\textsuperscript{rd} Given Name Surname}
% \IEEEauthorblockA{\textit{dept. name of organization (of Aff.)} \\
% \textit{name of organization (of Aff.)}\\
% City, Country \\
% email address or ORCID}
% \and
% \IEEEauthorblockN{4\textsuperscript{th} Given Name Surname}
% \IEEEauthorblockA{\textit{dept. name of organization (of Aff.)} \\
% \textit{name of organization (of Aff.)}\\
% City, Country \\
% email address or ORCID}
% \and
% \IEEEauthorblockN{5\textsuperscript{th} Given Name Surname}
% \IEEEauthorblockA{\textit{dept. name of organization (of Aff.)} \\
% \textit{name of organization (of Aff.)}\\
% City, Country \\
% email address or ORCID}

% \and
% \IEEEauthorblockN{6\textsuperscript{th} Given Name Surname}
% \IEEEauthorblockA{\textit{dept. name of organization (of Aff.)} \\
% \textit{name of organization (of Aff.)}\\
% City, Country \\
% email address or ORCID}
% }



\maketitle



\begin{abstract}
Serverless platforms typically adopt an early-binding approach for function sizing, requiring developers to specify an immutable size for each function within a workflow beforehand.
Accounting for potential runtime variability, developers must size functions for worst-case scenarios to ensure service-level objectives (SLOs), resulting in significant resource inefficiency. 
To address this issue, we propose Janus, a novel resource adaptation framework for serverless platforms. Janus employs a late-binding approach, allowing function sizes to be dynamically adapted based on runtime conditions.
The main challenge lies in the information barrier between the developer and the provider: developers lack access to runtime information, while providers lack domain knowledge about the workflow. 
To bridge this gap, Janus allows developers to provide hints containing rules and options for resource adaptation. 
Providers then follow these hints to dynamically adjust resource allocation at runtime based on real-time function execution information, ensuring compliance with SLOs. 
We implement Janus and conduct extensive experiments with real-world serverless workflows. 
Our results demonstrate that Janus enhances resource efficiency by up to 34.7\% compared to the state-of-the-art.

\end{abstract}

% \begin{IEEEkeywords}
% component, formatting, style, styling, insert.
% \end{IEEEkeywords}


% humans are sensitive to the way information is presented.

% introduce framing as the way we address framing. say something about political views and how information is represented.

% in this paper we explore if models show similar sensitivity.

% why is it important/interesting.



% thought - it would be interesting to test it on real world data, but it would be hard to test humans because they come already biased about real world stuff, so we tested artificial.


% LLMs have recently been shown to mimic cognitive biases, typically associated with human behavior~\citep{ malberg2024comprehensive, itzhak-etal-2024-instructed}. This resemblance has significant implications for how we perceive these models and what we can expect from them in real-world interactions and decisionmaking~\citep{eigner2024determinants, echterhoff-etal-2024-cognitive}.

The \textit{framing effect} is a well-known cognitive phenomenon, where different presentations of the same underlying facts affect human perception towards them~\citep{tversky1981framing}.
For example, presenting an economic policy as only creating 50,000 new jobs, versus also reporting that it would cost 2B USD, can dramatically shift public opinion~\cite{sniderman2004structure}. 
%%%%%%%% 图1:  %%%%%%%%%%%%%%%%
\begin{figure}[t]
    \centering
    \includegraphics[width=\columnwidth]{Figs/01.pdf}
    \caption{Performance comparison (Top-1 Acc (\%)) under various open-vocabulary evaluation settings where the video learners except for CLIP are tuned on Kinetics-400~\cite{k400} with frozen text encoders. The satisfying in-context generalizability on UCF101~\cite{UCF101} (a) can be severely affected by static bias when evaluating on out-of-context SCUBA-UCF101~\cite{li2023mitigating} (b) by replacing the video background with other images.}
    \label{fig:teaser}
\end{figure}


Previous research has shown that LLMs exhibit various cognitive biases, including the framing effect~\cite{lore2024strategic,shaikh2024cbeval,malberg2024comprehensive,echterhoff-etal-2024-cognitive}. However, these either rely on synthetic datasets or evaluate LLMs on different data from what humans were tested on. In addition, comparisons between models and humans typically treat human performance as a baseline rather than comparing patterns in human behavior. 
% \gabis{looks good! what do we mean by ``most studies'' or ``rarely'' can we remove those? or we want to say that we don't know of previous work doing both at the same time?}\gili{yeah the main point is that some work has done each separated, but not all of it together. how about now?}

In this work, we evaluate LLMs on real-world data. Rather than measuring model performance in terms of accuracy, we analyze how closely their responses align with human annotations. Furthermore, while previous studies have examined the effect of framing on decision making, we extend this analysis to sentiment analysis, as sentiment perception plays a key explanatory role in decision-making \cite{lerner2015emotion}. 
%Based on this, we argue that examining sentiment shifts in response to reframing can provide deeper insights into the framing effect. \gabis{I don't understand this last claim. Maybe remove and just say we extend to sentiment analysis?}

% Understanding how LLMs respond to framing is crucial, as they are increasingly integrated into real-world applications~\citep{gan2024application, hurlin2024fairness}.
% In some applications, e.g., in virtual companions, framing can be harnessed to produce human-like behavior leading to better engagement.
% In contrast, in other applications, such as financial or legal advice, mitigating the effect of framing can lead to less biased decisions.
% In both cases, a better understanding of the framing effect on LLMs can help develop strategies to mitigate its negative impacts,
% while utilizing its positive aspects. \gabis{$\leftarrow$ reading this again, maybe this isn't the right place for this paragraph. Consider putting in the conclusion? I think that after we said that people have worked on it, we don't necessarily need this here and will shorten the long intro}


% If framing can influence their outputs, this could have significant societal effects,
% from spreading biases in automated decision-making~\citep{ghasemaghaei2024understanding} to reducing public trust in AI-generated content~\citep{afroogh2024trust}. 
% However, framing is not inherently negative -- understanding how it affects LLM outputs can offer valuable insights into both human and machine cognition.
% By systematically investigating the framing effect,


%It is therefore crucial to systematically investigate the framing effect, to better understand and mitigate its impact. \gabis{This paragraph is important - I think that right now it's saying that we don't want models to be influenced by framing (since we want to mitigate its impact, right?) When we talked I think we had a more nuanced position?}




To better understand the framing effect in LLMs in comparison to human behavior,
we introduce the \name{} dataset (Section~\ref{sec:data}), comprising 1,000 statements, constructed through a three-step process, as shown in Figure~\ref{fig:fig1}.
First, we collect a set of real-world statements that express a clear negative or positive sentiment (e.g., ``I won the highest prize'').
%as exemplified in Figure~\ref{fig:fig1} -- ``I won the highest prize'' positive base statement. (2) next,
Second, we \emph{reframe} the text by adding a prefix or suffix with an opposite sentiment (e.g., ``I won the highest prize, \emph{although I lost all my friends on the way}'').
Finally, we collect human annotations by asking different participants
if they consider the reframed statement to be overall positive or negative.
% \gabist{This allows us to quantify the extent of \textit{sentiment shifts}, which is defined as labeling the sentiment aligning with the opposite framing, rather then the base sentiment -- e.g., voting ``negative'' for the statement ``I won the highest prize, although I lost all my friends on the way'', as it aligns with the opposite framing sentiment.}
We choose to annotate Amazon reviews, where sentiment is more robust, compared to e.g., the news domain which introduces confounding variables such as prior political leaning~\cite{druckman2004political}.


%While the implications of framing on sensitive and controversial topics like politics or economics are highly relevant to real-world applications, testing these subjects in a controlled setting is challenging. Such topics can introduce confounding variables, as annotators might rely on their personal beliefs or emotions rather than focusing solely on the framing, particularly when the content is emotionally charged~\cite{druckman2004political}. To balance real-world relevance with experimental reliability, we chose to focus on statements derived from Amazon reviews. These are naturally occurring, sentiment-rich texts that are less likely to trigger strong preexisting biases or emotional reactions. For instance, a review like ``The book was engaging'' can be framed negatively without invoking specific cultural or political associations. 

 In Section~\ref{sec:results}, we evaluate eight state-of-the-art LLMs
 % including \gpt{}~\cite{openai2024gpt4osystemcard}, \llama{}~\cite{dubey2024llama}, \mistral{}~\cite{jiang2023mistral}, \mixtral{}~\cite{mistral2023mixtral}, and \gemma{}~\cite{team2024gemma}, 
on the \name{} dataset and compare them against human annotations. We find  that LLMs are influenced by framing, somewhat similar to human behavior. All models show a \emph{strong} correlation ($r>0.57$) with human behavior.
%All models show a correlation with human responses of more than $0.55$ in Pearson's $r$ \gabis{@Gili check how people report this?}.
Moreover, we find that both humans and LLMs are more influenced by positive reframing rather than negative reframing. We also find that larger models tend to be more correlated with human behavior. Interestingly, \gpt{} shows the lowest correlation with human behavior. This raises questions about how architectural or training differences might influence susceptibility to framing. 
%\gabis{this last finding about \gpt{} stands in opposition to the start of the statement, right? Even though it's probably one of the largest models, it doesn't correlate with humans? If so, better to state this explicitly}

This work contributes to understanding the parallels between LLM and human cognition, offering insights into how cognitive mechanisms such as the framing effect emerge in LLMs.\footnote{\name{} data available at \url{https://huggingface.co/datasets/gililior/WildFrame}\\Code: ~\url{https://github.com/SLAB-NLP/WildFrame-Eval}}

%\gabist{It also raises fundamental philosophical and practical questions -- should LLMs aim to emulate human-like behavior, even when such behavior is susceptible to harmful cognitive biases? or should they strive to deviate from human tendencies to avoid reproducing these pitfalls?}\gabis{$\leftarrow$ also following Itay's comment, maybe this is better in the dicsussion, since we don't address these questions in the paper.} %\gabis{This last statement brings the nuance back, so I think it contradicts the previous parapgraph where we talked about ``mitigating'' the effect of framing. Also, I think it would be nice to discuss this a bit more in depth, maybe in the discussion section.}






\section{Background on Causal Inference}
\label{sec:background-causal} 



 \newtextold{In this section, we 
 %formalize the notion of {\em Average Treatment Effect and understand the 
 review the basic concepts and key assumptions for inferring the effects of an intervention on the outcome on collected datasets without performing randomized controlled experiments. 
We use {\em Pearl's graphical causal model} for {\em observational causal analysis} \cite{pearl2009causal} to define these concepts.}


\par
\paratitle{Causal Inference and Causal DAGs} The primary goal of causal inference is to model causal dependencies between attributes and evaluate how changing one variable (referred to as intervention) would affect the other.
Pearl's Probabilistic Graphical Causal Model \cite{pearl2009causal} can be written as a tuple $(\exo, \edvar, Pr_{\exo}, \psi)$, where $\exo$ is a set of {\em exogenous} variables, $\Pr_{\exo}$ is the joint distribution of \exo, and $\edvar$ is a set of observed {\em endogenous variables}.
Here $\psi$ is a set of structural equations that encode dependencies among variables. The equation for $A \in \edvar$ takes the following form:
%that encode the dependencies among the variables.  These equations are of the form 
$$\psi_{A}: 
\dom(Pa_{\exo}(A)) {\times} \dom(Pa_{\edvar}(A)) \to \dom(A)$$
Here $Pa_{\exo}(A) {\subseteq} {\exo}$ and $Pa_{\edvar}(A) {\subseteq} \edvar \setminus \{A\}$ respectively denote the exogenous and endogenous parents of $A$. A causal relational model is associated with a directed acyclic graph ({\em causal DAG}) $G$, whose nodes are the endogenous variables $\edvar$ and there is a directed edge from $X$ to $O$ if  $X {\in} Pa_{\edvar}(O)$. The causal DAG obfuscates exogenous variables as they are unobserved. %Any given set of values for the exogenous variables completely determines the values of the endogenous variables by the structural equations (we do not need any known closed-form expressions of the structural equations in this work). 
The probability distribution $\Pr_{\exo}$ on exogenous variables $\exo$ induces a probability distribution  
on the endogenous variables $\edvar$ by the structural equations $\psi$.  A causal DAG can be constructed by a domain expert as in the above example, or using existing {\em causal discovery} algorithms~\cite{glymour2019review}. 



\begin{figure}
    \centering
    \small
    \begin{tikzpicture}[node distance=0.6cm and 1cm, every node/.style={minimum size=0.5cm}]
        \tikzset{vertex/.style = {draw, circle, align=center}}

        \node[vertex] (Ethnicity) {\bf\scriptsize{{Ethnicity}}};
        \node[vertex, right=0.3cm of Ethnicity] (Gender) {\bf{\scriptsize{Gender}}};
        \node[vertex, right=0.3cm of Gender] (Age) {\bf{\scriptsize{Age}}};
        \node[vertex, below=0.3cm of Gender] (Role) {\bf{\scriptsize{Role}}};
        \node[vertex, right=0.3cm of Role] (Education) {\bf{\small{\scriptsize{Education}}}};
        \node[vertex, below=0.3cm of Role] (Salary) {\bf{\scriptsize{Salary}}};

        \draw[->] (Ethnicity) -- (Salary);
        \draw[->] (Gender) -- (Role);
        \draw[->] (Age) -- (Role);
         \draw[->] (Education) -- (Role);
           \draw[->] (Education) -- (Salary);
             \draw[->] (Ethnicity) -- (Education);
                \draw[->] (Ethnicity) -- (Role);
             \draw[->] (Gender) -- (Education);
               \draw[->] (Age) -- (Education);
                 \draw[->] (Role) -- (Salary);
        \draw[->] (Gender) to[bend right] (Salary);
        \draw[->] (Age) -- (Salary);
    \end{tikzpicture}
    \caption{Partial causal DAG for the Stack Overflow dataset.}
    \label{fig:causal_DAG}
\end{figure}



 \begin{example}
Figure \ref{fig:causal_DAG} depicts a partial causal DAG for the SO dataset over the attributes in Table \ref{tab:data} as endogenous variables (we use a larger causal DAG with all 20 attributes in our experiments). 
  Given this causal DAG, we can observe that the role that a coder has in their company depends on their education, age gender and ethnicity.
\end{example}
\par


\par
\paratitle{Intervention} In Pearl's model, a treatment $T = t$ (on one or more variables) is considered as an {\em intervention} to a causal DAG by mechanically changing the DAG such that the values of node(s) of $T$ in $G$ are set to the value(s) in $t$, which is denoted by $\doop(T = t)$. Following this operation, the probability distribution of the nodes in the graph changes as the treatment nodes no longer depend on the values of their parents. Pearl's model gives an approach to estimate the new probability distribution by identifying the confounding factors $Z$ described earlier using conditions such as {\em d-separation} and {\em backdoor criteria} \cite{pearl2009causal}, which we do not discuss in this paper.


\par
\paratitle{Average Treatment Effect} The effects of an intervention are often measured by evaluating
% \par
% \paratitle{Causal inference, Treatment, ATE, and CATE}
% \newtextold{One of the primary goals  of {\em causal inference} is to estimate the effect of making a change in terms of a {\em treatment} $T$ (often referred to as an intervention)
% on the outcome $O$. 
% %A variable that is modified is often referred to as the treatment variable $T$ and the metric used to captures 
% The effect of treatment $T$ on outcome $O$ is measured by 
% %is known as 
{\em Conditional Average treatment effect (CATE)}, 
%a {\em treatment variable} $T$ on an outcome variable $O$ (e.g., what is the effect of higher \verb|Education| on \verb|Salary|). 
measuring the effect of an intervention on a subset of records~\cite{rubin1971use,holland1986statistics} by calculating the difference in average outcomes between the group that receives the treatment and the group that does not (called the {\em control} group), providing an estimate of how the intervention by $T$ influences an outcome $O$ for a given subpopulation. 
% Mathematically,
% \begin{equation}
%     %{\small ATE(T,O) = \mathbb{E}[O \mid \doop(T=1)] -      \mathbb{E}[O \mid \doop(T=0)]}
%     {\small ATE(T, O) = \mathbb{E}[O \mid \doop(T=1)] -  
%     \mathbb{E}[O \mid \doop(T=0)]}
% \label{eq:ate}
% \end{equation}
% In our work, where the treatment with maximum effect may vary among different subpopulations, we are interested in computing the \emph{Conditional Average Treatment Effect} (CATE), which measures the effect of a treatment on an outcome on \emph{a subset of input units}~\cite{rubin1971use,holland1986statistics}. 
Given a subset of the records defined by (a vector of) attributes $B$ and their values $b$, 
%g {\in} \Qagg(\db)$ defined by a predicate $G {=} g$ 
we can compute $CATE(T,O \mid B = b)$ as:
{
\begin{eqnarray}    
    %CATE(T,O \mid G=g) = \mathbb{E}[O \mid \doop(T=1)&, G=g] -  \mathbb{E}[O \mid \doop(T=0), G=g] 
   % CATE(T,O \mid B = b) = 
    \mathbb{E}[O \mid \doop(T=1), B = b] -  
    \mathbb{E}[O \mid \doop(T=0), B = b]\label{eq:cate}
\end{eqnarray}
}
Setting $B=\phi$ is equivalent to the ATE estimate.
The above definitions assumes that the treatment assigned to one unit does not affect the outcome of another unit (called the {Stable Unit Treatment Value Assumption (SUTVA)) \cite{rubin2005causal}}\footnote{This assumption does not hold for causal inference on multiple tables and even on a single table where tuples depend on each other.}. 


The ideal way of estimating the ATE and CATE is through {\em randomized controlled experiments}, 
where the population is randomly divided into two groups (treated and control, for binary treatments): 
%treated group that receives the treatment and control group that does not (denoted by 
%{the \em treated} group 
denoted by 
$\doop(T = 1)$ 
%for a binary treatment)  (the {\em control} group, 
and $\doop(T = 0)$ resp.)~\cite{pearl2009causal}.
%\sr{edited up to here, going to read the rest first, this section should not look like causumx}
%\par
%\par
However, randomized experiments cannot always be performed due to ethical or feasibility issues. In these scenarios, observational data is used to estimate the treatment effect, which requires the following additional assumptions. 
% {\em Observational Causal Analysis} still allows sound causal inference under additional assumptions. Randomization in controlled trials mitigates the effect of {\em confounding factors}, i.e., attributes that can affect the treatment assignment and outcome. Suppose we want to understand the causal effect of \verb|Education| on \verb|Salary| from the SO dataset.  %in Example~\ref{ex:running_example}. 
% We no longer apply Eq. (\ref{eq:ate}) since the values of \verb|Education| were not assigned at random in this data, and obtaining higher education largely depends on other attributes like \verb|Gender|, \verb|Age|, and \verb|Country|. 
% Pearl's model provides ways to account for these confounding attributes $Z$ to get an unbiased causal estimate from observational data under the following assumptions ($\independent$ denotes independence):
% \vspace{-2mm}
\newtextold{
The first assumption is called {\em unconfoundedness} or {\em strong ignorability}  \cite{rosenbaum1983central} says that the independence of outcome $O$ and treatment $T$ conditioning on a set of confounder variables  (covariates) $Z$, i.e.,
%\begin{eqnarray}
 $    O \independent T | Z {=} z$.
 %\label{eq:unconfoundedness}
%\end{eqnarray}
The second assumption called {\em overlap or positivity} says that there is a chance of observing individuals in both the treatment and control groups for every combination of covariate values, i.e., 
%\begin{eqnarray}
   $ 0 < Pr(T {=} 1 ~~|~~Z {=} z)< 1 $.
   %\label{eq:overlap}
%\end{eqnarray}
}
%\sg{Is this overlap or positivity? maybe both are the same?} \sr{yeah - same - from Google AI - The overlap assumption, also known as the positivity assumption, is a key assumption in causal inference that states that there is a chance of observing individuals in both the treatment and control groups for every combination of covariate values.}
% The above conditions are known as {\em Strong Ignorability} in Rubin's model \cite{rubin2005causal}.
The unconfoundedness assumption requires that the treatment $T$ and the outcome $O$ be independent when conditioned on a set of variables $Z$. In SO, assuming that only $Z$ =\{\verb|Gender|, \verb|Age|, \verb|Country|\} affects $T = $ \verb|Education|, if we condition on a fixed set of values of $Z$, i.e., consider people of a given gender, from a given country, and at a given age, then $T = $ \verb|Education| and $O = $ \verb|Salary| are independent. For such confounding factors $Z$,  Eq. (\ref{eq:cate}) reduces to the following form 
(positivity 
gives the feasibility of the expectation difference): 
 \vspace{-1mm}
{\small
\begin{flalign}    
% \begin{eqnarray}
   % % & ATE(T,O) = \mathbb{E}_Z \left[\mathbb{E}[O \mid T=1, Z = z] -  
   %  \mathbb{E}[O \mid T=0, Z = z] \right] \label{eq:conf-ate}\\
 & CATE(T,O {\mid} B {=} b) {=} \nonumber
    \mathbb{E}_Z \left[\mathbb{E}[O {\mid} T{=}1, B {=} b, Z {=} z] {-}  
    \mathbb{E}[O {\mid} T{=}0, B {=} b, Z {=} z]\right]\label{eq:conf-cate}
\end{flalign}
% \end{eqnarray}
}
% \vspace{-4mm}
This equation contains conditional probabilities and not $\doop(T = b)$, which can be estimated from an observed data. 
Pearl's model gives a systematic way to find such a $Z$ when a causal DAG is available. 




\section{System}
\begin{figure*}[h]
    \centering
    \includegraphics[width=.85\textwidth]{fig/SYSTEM_IMAGE_TEST_FLIPPED.png}
    \caption{HumorSkills System Diagram. Given an image, the system first extracts visual details with a visual language model, then performs visual humor ideation to analyze the image and propose humorous angles. It then generates ten potential conflicts that could be used to extrapolate the image into a relatable experience. The system then generates humor with and without the narratives, for diversity. A separate instance of the LLM trained to rank gen-Z humor ranks all the captions and returns the top five.}
    \Description{HumorSkills System Diagram}
    \label{fig:system}
\end{figure*}

HumorSkills is a system that takes an input image and outputs 5 image captions. 
The architecture has three key steps that mimic human skills needed for humor. \textit{Visual Detail Extraction}, is a step that describes the image in depth in order to make non-obvious observations about it. \textit{Narrative and Conflict Extrapolation} is a step that finds narratives not in the image that could be related to it, to expand the topic of jokes to things that are not just in the image but also analogous to it.  \textit{Fine-tuning} the joke generator with examples of good Gen-Z humor helps the jokes be more relatable to the target audience by using references, slang, topics, and insecurities that resonate with this group.

% first, 
% a \textbf{divergent stage} where the image is analysed and multiple observations, angle, alternative narratives and humorous angles are generated. 
% Second, a \textbf{generation stage} where two types of captions are generated: 1) captions focusing on image content directly 2) captions that bring in an outside narrative to the image, often bringing in outside references. It generates 15 captions of each type. (Figure 1 has examples Of the Content Focused, and Narrative Expanded Captions). Finally, a \textbf{ranking stage} where a separate AI agent selects the top 5 captions

The system generates two types of captions: image-focused captions which common directly on the content in the image, and narrative-driven captions. Variety is important to humor. Humor relies on surprise, and jokes that are too similar start to become more predictable. Additionally, with an infinite set of input images with different subjects and situations, there are more strategies needed to find a humorous angle that fits the content. 

% With caption-based humor, often the humor can be focused on finding something in the image that is inherently interesting. 

% For example, the caption ”little man really thought he could escape bedtime” relies solely on information in the image. However, some images don’t have something funny in and of themselves, and it’s easier to make a joke by bringing in a new unrelated angle. For example, ``the police chasing me when I'm broke and in debt to the tune of \$100,000 for student loans''. Generally, Images with people doing interesting things lend themselves to visual humor because there are many stories one could tell about it. However, for images with only static objects, it's more difficult to tell a story on only the objects, so bringing a new story in is another way to find humor. 

\subsection{AI Humor Generation Walk Through}
Figure \ref{fig:system} contains a visual diagram and example of intermediate outputs when generating captions for an image. We describe each phase and implementation in detail.  
% The main contribution of this paper is the evaluation, rather than the system, but it is still it is important to understand the mechanism used to generate humor.
% Although the individual components of the system are not totally, the combination of features including the

\subsubsection{Visual Detail Extraction}

The first phase of the system’s workflow involves the Visual Detail Extraction component, which utilizes GPT-4o’s vision capabilities to analyze the input image. This system incorporates a prompt that asks for a detailed paragraph that explains the who/what/where of the image, distinguishing between identifying the subject of the image, the main action of the image if it exists, and the background elements of the image. This component is responsible for extracting key visual elements such as objects, human expressions, background settings, and any notable aspects that could serve as the foundation for humor.

For instance, in the demolition site example from the system diagram, the system identifies a large industrial demolition excavator and a person with a hose spraying the demolition site. 

\subsubsection{Visual Humor Ideation}

On top of the visual detail extraction, the system ideates on possible humorous elements from the visual of the image. This incorporates an additional prompt using GPT-4o that intakes the image and asks it to identify and ideate on potential humorous visual elements in the image, whether they are directly humorous elements, such as funny facial expressions, or more analogous elements. For example, for the system diagram image, the system noted the visual contrast of the excavator and person, reminiscent of a David versus Goliath scenario, which provides a foundational metaphor for generating humorous captions. 

\begin{figure*}[b]
    \centering
    \includegraphics[width=.95\textwidth]{fig/Workflow.png}
    \caption{A diagram for how narrative extrapolation works}
    % \Description{}
    \label{fig:systemLines}
\end{figure*}

\subsubsection{Narrative and Conflict Extrapolation}

In this next step, the system generates a narrative and conflict framework by drawing upon common and relatable Gen Z experiences such as work, school, family, social interactions, relationships, and more. 
The system chains together the results of the previous steps, into a new prompt sent to GPT-4o. 
% The system prompts GPT-4o to 
% GPT-4o is utilized by incorporating the text description of the image and potential humorous elements of the image, then being prompted to generate relatable scenarios applicable to the image description from a list of common Gen Z experiences. 
The prompt contains the visual details, the visual humor ideation, and a list of common Gen Z experiences,  and the instruction to "generate narratives that reflect the essence of the image that is set within the framework of the Gen Z experience."
This narrative generation adds depth to the humorous captions by applying relatable themes and conflicts to the visual elements identified earlier.

For instance, our system diagram generates narratives such as “Tackling student loans”, "Group Project Disaster", and “Relationship Issues” based on the image, both of which are common experiences among those who identify as Gen Z. These particular narratives are likely inspired by the imagery of a disaster site, referring to how the effort of paying off student loans, attempting to complete group projects during school, or addressing relationship -- all of which can feel like disaster clean up. These relatable conflicts can transform the visual of a demolition scene -- a setting that is not particularly relatable -- into a relatable scenario that has the potential for humor, thereby expanding the humorous possibilities by connecting the visual input with broader life experiences.



\subsubsection{Humorous Caption Generation}

Following the narrative and conflict extrapolation, the system generates humorous captions in the generation stage using a fine-tuned version of GPT-3.5 trained on humorous Instagram comments. This involves producing captions through two distinct strategies: one focused on the visual humor of the image, and the other by bringing in the previously generated external narratives. Caption generation is segmented into two separate prompts utilizing the fine-tuned GPT-3.5 model. For captions without generated external narratives, the prompt asks to generate 15 humorous captions in the style of Gen Z that bases the generation off the visual extraction and visual humor ideation of the input image. For captions with the external narratives, the prompt also asks to generate 15 humorous captions in the style of Gen Z that bases the generation off the visual extraction and visual humor ideation of the input image, but also asks the system to incorporate the list of generated narrative/conflict extrapolations to base the humorous captions off of.

Image-focused captions rely solely on the visual details in the image, such as “bro out here getting paid \textdollar8 an hour to spray some water on some bricks,” which references the direct visual elements in the scene in order to generate a caption. This particular caption directly references the humor of the image, poking fun at the minimal impact of the person spraying water on bricks while an excavator clearly has more impact on the demolition site. Narrative-driven captions, on the other hand, introduce external references to add humor. For instance, a caption like “The entitled bro you tried to make the group presentation with” introduces an outside, exaggerated, interpretation of the scene from earlier, "Group Project Disaster." This caption takes the group project narrative and pairs it with the visual of the image, analogizing the person spraying the hose with minimal impact on the demolition site to an entitled person who has not done much to complete the group project. 

This variety between visual humor and narrative-driven humor is crucial because jokes that are too similar become predictable, losing their element of surprise. Additionally, humor strategies need to adapt to the varying content in input images. Some images lend themselves to humor based on their inherent visual details, while others require bringing in outside references to create a joke. For instance, an image of static objects might not be inherently funny, such as the demolition image shown in the system diagram, but a caption introducing an unrelated, exaggerated narrative, such as “Eboy doin' his part to stop climate change” can inject humor and absurdity by making an unexpected connection.

\subsubsection{Caption Ranking using Gen Z Agent}

The final component of the system architecture is the Caption Ranking and Filtering Agent, a GPT-4o-based agent fine-tuned to evaluate humor from a Gen Z perspective. This agent receives the list of 30 total captions from the narrative and visual humor-based caption generations and ranks the captions generated in the previous stage based on humor, relatability, and alignment with the image and narrative.

As illustrated in our system diagram, this agent ranks captions such as “Me mopping up my last relationship” and “me pulling the emotional weight of the friend group” based on their relevance to Gen Z humor. Captions that fail to meet the humor threshold are filtered out, such as "Demolition worker really said 1v1 me bro," because although the phrase like "1v1 me bro" invokes Gen Z phrases, the content of the caption seems less relevant and relatable than a caption talking about school or relationships, ensuring that only the most effective and relatable captions are presented to the user.

\subsubsection{Fine-tuning}

To fine-tune a GPT-3.5 model, a dataset of 80 humorous comments were extracted from popular Instagram images. From three popular Instagram meme pages with over 400,000 followers, the top five comments of each image post were collected. All fit the style of Gen-Z humor. 
% The fine-tuning process ensured that the generated captions align with the humor style favored by Gen Z. 
Examples of the visual description of the images in addition to an explanation of potential humorous elements of the image were written in the fine-tuning prompts, then followed by the actual comment itself. This reflected the visual extraction and humor ideation being incorporated into the prompt of our current system.



%\section{Profiler}
\label{sec:profilier}
Here,  we discuss how to profile function execution latency.
Based on that, two metrics are proposed as a preparation for synthesizing hints.
\subsection{Diverse percentiles based profiling}
As explained in \S{\ref{sec:bg:worst-case}}, function execution latency is a distribution, making the current work, which either depends on a single statistic or a simple 99\% percentile distribution, insufficient.
To fix this issue, for any given batch sizes we introduce diverse percentile latency distributions to profile the execution latency, which is expressed as $L(p,k)$, with CPU cores and percentiles denoted as $k$ and $p$, respectively.

\subsection{Timeout and resilience.}
The diversity in percentiles enables more opportunities to explore higher resource efficiency but comes at the risk of SLO violations.
Specifically, when setting percentiles lower than 99\%, it may incur under-estimation of function execution, making functions prone to \textit{timeout}, i.e., their actual execution latency over the estimated latency.
Timeout is expressed as follows
\begin{eqnarray}
     D(p,k) = L(P_{99},k) -L(p,k).
\end{eqnarray}

Consequently, for preventing from SLO violation, \namex has to provision more CPU cores to downstream functions to absorb previous timeout.
Moreover, we propose another metric called \emph{resilience} to quantify the absorption capability, which is expressed as follows
\begin{eqnarray}
    R(p,k)= L(p,K_M)-L(p,k),
\end{eqnarray}
where $k_M$ denotes the maximum available CPU cores.
Notably, timeout must be restricted within the upper bound of resilience, such that SLO can be guaranteed.

 








\section{Synthesizer}
\label{sec:synthesizer}

We now elaborate on the workings of the synthesizer. 
The hints synthesis process consists of two steps: hints generation and hints condensing. 

% We now discuss the two steps synthesizer takes, namely hints generating and condensing.
% Profiler prepares raw material/ingredient for synthesizer to generate hints, i.e., concise and straightforward rules, to guide providers to conduct adaptive allocation at runtime.
% For maintaining high accuracy, synthesizer requires to prepare a specific hint for each given slack.
% Yet, due to the variance and volatility of slacks, \namex may generate overwhelming hints, which not only incurs extra storage resource consumption but also  hurts the time-efficiency of adaptive allocations.
% To fix this issue, \namex effectively condenses the hints.
\subsection{Hints Generation}
\label{sec:synthesizer:generate}
To generate hints tables with high hit rates and high resource efficiency, the synthesizer requires a twofold effort.
First, it must explore all potential runtime time budgets for individual sub-workflows.
Second, the synthesizer needs to balance the trade-off between higher resource efficiency and the risk of SLO violation.
To this end, we reveal the following insights.

\mypara{Insight-1: Broad time budget range.}
The time budgets are calculated based on all possibilities between the 1st and 99th percentile (P1-P99) of the function execution time under a wide range of resource allocations, aiming to achieve high hit rates.
The range of time budgets therefore are formulated as
\begin{eqnarray}
    T_{min}= \sum_{i=1}^NL_i(1,K_{max}),
    T_{max}= \sum_{i=1}^NL_i(99,K_{min}),
\end{eqnarray}
where $K_{min}$/$K_{max}$ represents the minimum/maximum available resources,
and $N$ represents the numbers of functions in the given sub-workflows. Within this range, the synthesizer explores the potential time budgets with finer granularity in milliseconds, while evaluating their corresponding resource allocation.
The synthesizer can also be configured with higher percentiles (e.g., P99.9) to meet more stringent SLO targets.

\textbf{\textit{Insight-2: Moderate percentile exploration.}}
Diverse percentiles provide more opportunities for resource optimization, but come with exponentially higher time complexity for runtime resource adaptation.
Here, our insight is to only open percentile exploration for the head function of the current sub-workflow while fixing other functions with P99.
This moderate percentile exploration benefits the synthesizer with higher resource efficiency, derived from its attempt at lower percentiles for the head function.
Meanwhile, it effectively reduces the search space for non-head functions, allowing the synthesizer to achieve high time efficiency.

\textbf{\textit{Insight-3: Resilience-aware.}}
Despite the potential of higher resource efficiency, diverse percentile exploration may put functions at the risk of timeouts, making workflows prone to SLO violations.
To address this shortcoming, the synthesizer strictly restricts the timeout within the resilience (the achievable reduction in function execution time by scaling resource up to the maximum possible).
Within this ``safety zone", the synthesizer tries its best to maximize resource efficiency.

\mypara{Insight-4: Heavier head.}
As explained in~\S\ref{sec:bg:adaptive-allocation}, facing substantial variability of execution performance, runtime resource adaptation requires to carry out (head) function by (head) function, so as to keep its high accuracy.
This, however, may lead to sub-optimal decisions due to the mismatch between the local objective and the global objective.
Specifically, the local objective is to maximize the sub-workflow's resource efficiency, while the global objective is to maximize the whole workflow's resource efficiency.
The whole efficiency is determined by that of each sub-workflow's head function, rather than that of sub-workflows.
To address this issue, the synthesizer magnifies the local objective's weight for head functions, aiming to calibrate for the mismatch.

As for how to set the weight, our insight is to increase the weight when facing loose SLOs, and vice versa.
This is because loose SLOs indicate lower resource requirements, which brings about higher resilience (depicted in Figure~\ref{fig:exp:resilience:cores}).
Increasing the weight can better utilize this higher resilience to explore lower percentiles, such that the workflow achieves higher resource efficiency with SLO guarantees.

Hints demonstrates explicit resource allocation that can ensure the sub-workflow with its maximum resource efficiency, i.e., the minimum resource consumption, under given time budgets.
This problem thus is formulated as follows:
% \begin{eqnarray}
% 	\min && W\cdot k_1+p \cdot \sum_{i=2}^{N}k_i +(1-p)\cdot (N-1)\cdot K_M \label{eq:hints:obj}\\
% 	\text{subject to} &&
%  %percentile latency
%  L(p,k_1)+\sum_{i=2}^{N}L(99,k_i) \leq T, \label{eq:hints:time-budget}\\
%  %timeouts and resilience
% && D(p,k_1) \leq \sum_{i=2}^{N}R(99,k_i), \label{eq:hints:timeout-resilience}\\
% &&  1 ~\leq p ~\leq 99,~p \in \mathbb{Z},\\
% && 0 ~< k_i ~\leq K_{max},~k_i \in \mathbb{R}, ~\forall i.
% \end{eqnarray}
\begin{eqnarray}
	\min && W k_1+p  \sum_{i=2}^{N}k_i +(1-p) (N-1) K_{max} \label{eq:hints:obj}\\
	\text{subject to} &&
 %percentile latency
 L_1(p,k_1)+\sum_{i=2}^{N}L_i(99,k_i) \leq T, \label{eq:hints:time-budget}\\
 %timeouts and resilience
&& D_1(p,k_1) \leq \sum_{i=2}^{N}R_i(99,k_i), \label{eq:hints:timeout-resilience}\\
&&  1 \leq p \leq 99,~p \in \mathbb{Z},\\
&& K_{min}\leq k_i \leq K_{max},~k_i \in \mathbb{R}, ~\forall i.
\end{eqnarray}
where $W$ is the weight for the head function (Insight-4), and $T$ and $N$ denote the time budget and the number of functions in the sub-workflow, respectively.
%$K_{max}$ represent the maximum available CPU cores.
Notably, only the head function can explore lower percentile $p$ (Insight-2).
Equation~\ref{eq:hints:obj} expresses the sub-workflow's expected resource consumption.
Specifically, $\sum_{i=2}^{N}k_i$ and $(N-1)K_{max}$ denote non-head functions' resource requirement without and with the head function's timeout, the probability of which is $p$ and $1-p$, respectively.
Equation~\ref{eq:hints:time-budget} ensures the sub-workflow's execution latency within the time budget.
Equation~\ref{eq:hints:timeout-resilience} restricts that the possible timeout of the head function can not exceed the total resilience of downstream functions (Insight-3).

\begin{algorithm}[!t]
\small
\caption{Offline hints generation\label{alg:generate}}
 	\LinesNumbered    
    \KwIn{$\mathbf{F}= \left\langle f_1,\dots,f_N \right\rangle$: (sub-)workflow}
    \KwIn{$[T_{min},T_{max}]$: time budget range}
    \KwIn{$W,\mathbf{P}$: weight and candidate percentiles  for head function $f_1$}
	%\KwIn{$\left\langle T, \mathbf{H} \right\rangle$: time budget and hints table.}
    \KwOut{$\mathbf{H}=\{\left\langle  t, \{ k_1,\dots,k_N \} \right\rangle\}$: functions' provisioned CPU cores under given time budget $t$, i.e., hints table
	}
    $\mathbf{H} \leftarrow \emptyset$, $\mathbf{P} \leftarrow \emptyset$   \\
    \ForEach{$t \in [T_{min},T_{max}]$}
    {
    $\mathbf{H} \leftarrow \mathbf{H} \cup \{ \left \langle t, ~\texttt{generate}(\mathbf{F},t,\mathbf{P}) \right \rangle \}$ \\
    $\text{return}~\mathbf{H}$
    }
	\SetKwFunction{FMain}{\texttt{generate}}
    \SetKwProg{Fn}{Function}{:}{}
    
    \Fn{\FMain{$\mathbf{F},t,\mathbf{P}$}}{
    \If{$\left| \mathbf{F} \right| = 1$} 
    {
     \textbf{return} \texttt{min\_resource}($f_1,t$)
    }
    %$r_{min} \leftarrow \infty,~X \leftarrow \emptyset $ \\
     \If{$\mathbf{P} = \emptyset$} 
    {
     $\mathbf{P}=$\texttt{explore\_percentile}($\mathbf{F},t, K_{max}$)
    }
    $s_{min} \leftarrow \infty,~\mathbf{K} \leftarrow \emptyset $\\
    \ForEach{$p \in \mathbf{P}$}
    {
        \ForEach{$k \in [K_{min},K_{max}]$}   
        {
          $\mathbf{Z} \leftarrow $~\FMain{$\mathbf{F} \setminus f_1, t-L_1(p,k), \{P_{99}\}$}\\
          \If{$\mathbf{Z} \neq \emptyset \wedge  D(p,k) \leq \sum{R(\mathbf{Z},P_{99})}$}
          {
          $s \leftarrow W  k +p \sum{\mathbf{Z}} + (1-p)  (\left| \mathbf{F} \right| -1) K_{max}$ \\
          \If{$s \leq s_{min}$}
          {
            $s_{min} \leftarrow s, \mathbf{K} \leftarrow \{k\} \cup  \mathbf{Z}  $
          }
          }
        }            
    }
    \text{return}~$\mathbf{K}$
    }
\end{algorithm}

The algorithm for generating hints is listed in Algorithm~\ref{alg:generate}.
To ensure hints tables with high hit rates, the synthesizer explores all time budgets comprehensively (lines 2--4). 
Specifically, for a given sub-workflow $\mathbf{F}$, the synthesizer first determines the percentiles $\mathbf{P}$ that can ensure $\mathbf{F}$'s execution time below the required time budget $t$, with assuming the maximum available CPU cores for each function (lines 8--9).
Then, the synthesizer explores the resource allocation for both head and non-head functions, denoted as $k$ and $\mathbf{Z}$, under given percentile $p$. Its goal is to minimize the expected resource consumption $s$, while promising timeout $ D(p,k)$ restricted within resilience $\sum{R(\mathbf{Z},P_{99})}$ (lines 12--17).
To accelerate the generation, the synthesizer explores different percentiles concurrently.

\subsection{Hints Condensing}
\label{sec:synthesizer:condense}
The synthesizer fully utilizes the discreteness in both decision-making and decision-executing to condense hints.

\mypara{Insight-5: Repeated hints.} There are various discrete variables, such as batch sizes and CPU cores, involved in resource adaptation. 
This leads to a significant number of redundant hints that share the same adaptation decisions despite having different time budgets.

\mypara{Insight-6: Unused fields.}
The dependencies of adaptation (explained in \S\ref{sec:bg:adaptive-allocation}) compels Janus to rely solely on the fields for head functions in given hints to maintain adaptation accuracy.
Consequently, removing the fields for non-head functions helps compact the hints without compromising accuracy.
 
The algorithm for condensing hints is listed in Algorithm~\ref{alg:condense}.
Specifically, the synthesizer first sorts the given hints $\mathbf{H}$ in descending order by their time budget (line 2).
Then, it gradually fuses the hints $\mathbf{H}[l]$ that share the identical size for head function $k_1$ as shown in line 4--10.
Finally, it warps hints into a table with three fields: $T_{start}$, $T_{end}$, and $k$, indicating that the head function of the target sub-workflow should be resized to $k$ when the sub-workflow's time budgets is between $T_{start}$ and $T_{end}$.
 
 In addition, the weight for head functions impacts the decision-making.
 Thus, the synthesizer maintains individual hint tables for different weights.
 We will evaluate the effectiveness of condensing algorithm in \S\ref{exp:micro:condense}, which suggests a outstanding compression ratio without hurting accuracy. 
 
\begin{algorithm}[!t]
\small
\caption{Offline hints condensing \label{alg:condense}}
 \LinesNumbered    
    \KwIn{$\mathbf{H}= \{\left\langle t,\mathbf{K} \right\rangle$\}: raw hints table}
    \KwOut{$\mathbf{U} = [\left\langle T_{start}, T_{end}, k\right\rangle ]$: condensed hints table}
	\SetKwFunction{FMain}{\texttt{condense}}
    \SetKwProg{Fn}{Function}{:}{}
    
    \Fn{\FMain{$\mathbf{H}$}}{
    $\mathbf{H} \leftarrow$  \texttt{sort}($\mathbf{H}$)\\
    $\mathbf{U} \leftarrow \emptyset$, $q, i, j \leftarrow 0$\\
    \ForEach{$l \in [0,\left| \mathbf{H} \right |]$}
    {
        $t, \left\langle k_1,\dots, k_N \right\rangle \leftarrow \mathbf{H}[l]$ \\
        \If{$q = 0 \vee k_1 = q$}
        {
        $j \leftarrow j+1$
        }
        \Else
        {
        $\mathbf{U} \leftarrow \mathbf{U} \cup \{ \left \langle \mathbf{H}[i].t, \mathbf{H}[j].t, q \right \rangle\} $ \\
        $i, j \leftarrow l$, $q \leftarrow k_1$
        }      
    }
    \text{return}~$\mathbf{U}$
    }

\end{algorithm}





















%\section{Adapter}
\label{sec:adapter}
\begin{algorithm}[!t]
    
	\caption{Adapt resource online \jing{(necessary?)}}
 	\LinesNumbered    
    \KwIn{$\mathbf{F}= \left\langle f_1,\dots,f_N \right\rangle$: (sub-)workflow}
    \KwIn{$\mathbf{U}, T$: hints table and time budget/slack}
    \KwOut{$k$: CPU cores for head function}
	\SetKwFunction{FMain}{\texttt{adapt}}
    \SetKwProg{Fn}{Function}{:}{}
    
    \Fn{\FMain{$\mathbf{F},\mathbf{U}, T$}}{
    $ i \leftarrow \mathbf{U}.$\texttt{index}($T$)\\
    \If{$i=\emptyset$}
    {
    $\mathbf{K} \leftarrow $\texttt{generate}($\mathbf{F},T, \emptyset$) \\
    $\mathbf{U} \leftarrow $\texttt{condense}($\{\left \langle T, \mathbf{K}\right\rangle\}$)\\
    }
    \Else{
    $\mathbf{K} \leftarrow \mathbf{U}[i]$
    }
    \text{return}~$\mathbf{K}[0]$
    }
\label{alg:adapt}
\end{algorithm}

Having demonstrated how to offline synthesize and condense hints, we now focus on utilizing the hints table to accomplish online adaptation.

As outlined in \S~\ref{sec:system-overview},  after each function's execution, adapter re-evaluates the available time budget for the remaining functions while adapting the head function's size accordingly.
The procedure of adapter is listed in Algorithm~\ref{alg:adapt}.
Specifically, it first accesses the sub-workflow's hints table $\mathbf{U}$ to figure out the proper hint that can well meet the required time budget $T$ with the minimum resource consumption (Line 2).
Notably, if the above searching suffers a miss hit, due to the dynamics at runtime (detailed in~\S~\ref{sec:bg:worst-case}), adapter will revoke synthesizer to generate the proper hint on spot while updating it into the table (Line 3--5).
Moreover, adapter continuously counts the above miss. 
When it exceeds a predefined threshold, adapter argues that execution latency distribution may change, when adapter will trigger a holistic update of both profiler and synthesizer, for maintaining adaptation's accuracy.

\begin{algorithm}[!t]
    \caption{Adapt resource online \label{alg:janus:adapt}}
    \LinesNumbered    
    \KwIn{$\mathbf{F}= \left\langle f_1,\dots,f_N \right\rangle$: (sub-)workflow}
    \KwIn{$\mathbf{U}, T$: hints table and time budget/slack}
    \KwOut{$k$: CPU cores for head function}
	\SetKwFunction{FMain}{\texttt{adapt}}
    \SetKwProg{Fn}{Function}{:}{}
    
    \Fn{\FMain{$\mathbf{F},\mathbf{U}, T$}}{
    $ i \leftarrow \mathbf{U}.$\texttt{index}($T$)\\
    \If{$i=\emptyset$}
    {
    $\mathbf{K} \leftarrow $\texttt{generate}($\mathbf{F},T, \emptyset$) \\
    $\mathbf{U} \leftarrow $\texttt{condense}($\{\left \langle T, \mathbf{K}\right\rangle\}$)\\
    }
    \Else{
    $\mathbf{K} \leftarrow \mathbf{U}[i]$
    }
    \text{return}~$\mathbf{K}[0]$
    }

\end{algorithm}
\vspace{-2mm}
\section{Experimental Evaluation}
\label{sec:experiments}
We present an experimental evaluation that evaluates \sysName\ effectiveness and efficiency. We aim to address the following questions:  
$\mathbf{Q1}$: How does the quality of our generated rulesets compare to that of existing methods? $\mathbf{Q2:}$ What is the efficiency of \sysName\ and how is it affected by various data and system parameters?  



\subsection{Experimental Setup}
\label{sec:exp_setup}
\sysName\ was implemented in Python, and is publicly available in~\cite{fullversion}. 
CATE values computation was performed using the DoWhy library~\cite{dowhypaper}. The generated rules were translated into natural language using \reva{simple, manually constructed templates}.
We perform experiments on CloudLab ~\cite{Duplyakin+:ATC19} xl170 machines (10-core 2.4 GHz CPU, 64 GB RAM).
% In this section, we focus solely on the variant of our problem with statistical parity group fairness and group coverage constraints, as this represents the most challenging setting. Rule coverage and individual fairness are simpler, as they primarily involve pruning rules and can be verified in Step 2 of the algorithm, thereby reducing the search space.
The datasets, protected groups, and default parameters considered are the same as those described in \cref{sec:casestudy}.





%https://www.kaggle.com/datasets/sobhanmoosavi/us-accidents





% \brit{experiments:}
% \begin{itemize}
%     \item Case study: Compare between different definitions to see the effect of different fairness and coverage constraints
%     \item Comparison to baseline algorithms - quality in term of coverage and fairness
%     \item Comparison to baselines in terms of running times
% \end{itemize}


\vspace{1mm}
\paratitle{Baselines}
We compare \sysName\ with the following baselines:
 % \textbf{Brute-Force}: The optimal solution according to \cref{def:problem}. This algorithm implements an exhaustive search over all sets of rules.\\
\textbf{1. CauSumX}:
CauSumX \cite{DBLP:journals/pacmmod/YoungmannCGR24} is designed to find a summarized causal explanation for group-by-avg SQL query results. When applied directly to the datasets, it can be viewed as a solution to our problem with only an overall coverage constraint.
\textbf{2.IDS}~\cite{lakkaraju2016interpretable} is a framework for generating Interpretable Decision Sets for prediction tasks. IDS incorporates parameters restricting the percentage of uncovered tuples and the number of rules. These parameters were assigned the same values in our system.
\textbf{3. FRL}: The authors of \cite{chen2018optimization} proposed a framework for creating Falling Rule Lists (FRLs) as a probabilistic classification model. FRLs comprise if-then rules with antecedents in the if-clauses and probabilities of the desired outcome in the then-clauses, ordered based on associated probabilities.
% \textbf{Explanation Table}: The authors of \cite{el2014interpretable} introduced an efficient method to generate \emph{explanation tables} for multi-dimensional datasets. The proposed algorithm employs an information-theoretic approach to select patterns that provide
% the most information gain about the distribution of the outcome attribute. 
% \brit{a variant with fairness?}



\smallskip
Since IDS and FRL assume a binary outcome, we binned the salary variable in SO using the average value. To address fairness considerations, we run the baseline algorithms twice (excluding Brute-Force): Once on the entire dataset to obtain a set of rules applicable to the entire population, and again solely on the tuples belonging to the protected population to generate rules specifically tailored for them. \revb{We report the number of rules generated by the baselines, their coverage, and runtime. To compare the expected utility, we proceed as follows: The rules generated by IDS and FRL are prediction rules (e.g., IF owning a house = YES, THEN credit score = 1). As such, these rules do not provide an intervention to improve outcomes. We, therefore, treat the IF clauses
in two manners: (1) IF clauses as the selected grouping patterns and then apply step 2 (\cref{subsec:treatment_patterns}) of \sysName\ to determine the intervention patterns; (2) IF clauses as the selected intervention patterns, where the grouping pattern is the entire data. }
% For the resulting set of prescription rules, we report the expected utility for both protected and non-protected groups. }

% The final solution for each baseline is considered the union of these two sets of rules.



% \vspace{1mm}
% Unless otherwise specified, the overall coverage threshold as well as the coverage threshold for the protected group are set to 0.75. The threshold of the Apriori algorithm is set to 0.1. 
% The threshold for the SP fairness constraint is set at \brit{?}, and the threshold for the BGL fairness constraint is set at \brit{?}. 
% The time cutoff is set to $3$ hours. 


% \subsection{Problem Variants Evaluation (Q1)}
% \label{exp:problem_variants}

% \brit{here we can focus on only two datasets, and show the rules with different constraints (to motivate the need for different problem variants). (fill the cells in Table \ref{tab:problem_variants})}




\subsection{Quality Evaluation (Q1)}
\label{exp:quality}
We compare the set of rules chosen by each baseline and \sysName. 

\begin{figure}[t]
    \centering
    \vspace{-3mm}\includegraphics[width=0.46\textwidth]{figs/time_barchart.pdf}
%     \begin{subfigure}[b]{0.23\textwidth}
%         \centering
% \includegraphics[width=\textwidth]{example-image-a}
%         \caption{Stack Overflow}
%         \label{fig:first}
%     \end{subfigure}
%     \hfill
%     \begin{subfigure}[b]{0.23\textwidth}
%         \centering
% \includegraphics[width=\textwidth]{example-image-b}
%         \caption{German Credit}
%         \label{fig:second}
%     \end{subfigure}
    \caption{Runtime by-step of the \sysName\ algorithm (SO)}
    \label{fig:runtime_by_step}
\end{figure}

\paratitle{Stack Overflow} As discussed in \cref{sec:casestudy}, prescription rules selected without fairness constraints, similar to the behavior of CauSumX, were significantly more advantageous for non-protected.  
The rules generated by IDS do not suggest interventions to improve outcomes. For example, one rule states that if Country = Turkey and Age = 18-24 years, then the expected salary is low (with the outcome binned). Another key distinction is that these rules are not causal, as they are based on correlations in the data. For example, one rule indicates that if the years coding = 0-2 and Sexual Orientation = Gay or Lesbian, then the expected salary is low. Similarly, rules generated by FRL do not propose interventions to improve outcomes and are not causal. For example, one rule states that if Country = US and Sexual Orientation = Straight or Heterosexual, then the expected salary is high. 
In contrast, \sysName\ generates interventions aimed at improving the outcome by leveraging causal relationships. It also allows users to impose fairness constraints, ensuring that the protected group benefits from these interventions.




% IDS generated 16 rules for the overall population and 21 rules for the protected group. Notably, these rules do not suggest interventions to improve outcomes. For example, one rule states that if Country = Turkey and Age = 18-24 years, then the expected salary is low (with the outcome binned). Another key distinction is that these rules are not causal, as they are based solely on correlations in the data. For example, one rule indicates that if the years coding = 0-2 and Sexual Orientation = Gay or Lesbian, then the expected salary is low. \brit{TODO}


% The FRL baseline generated 9 rules for the overall population and 7 for the protected group. Similar to the IDS baseline, these rules do not propose interventions to improve outcomes and are not causal. For example, one rule states that if Country = United States and Sexual Orientation = Straight or Heterosexual, then the expected salary is high. 
%   \brit{TODO}

% \brit{IDS full: 16 rules, 64 seconds, IDS protected: 21 rules, 12 seconds}
% \brit{FRL full: 9 rules 1225 seconds, FRL protected: 7 rules, 478 seconds}

\paratitle{German}
Here again, with no fairness constraint (akin to CauSumX), the selected rules were mostly beneficial for the non-protected. 
% IDS generated 12 rules for the overall population and 20 for the protected group. 
Here again, the rules generated by IDS are not causal and do not offer an intervention. For example, one of the rules suggested that single females at the age of 35-41 are unlikely to get a loan.  
% \brit{TODO}
% FRL generated 13 rules for the overall population and 11 for the protected group. 
As before, the rules generated by FRL are also not causal and do not propose ways to improve the credit risk score. For example, one rule suggests that if a person has lived in a house for 4-7 years, their credit risk score is likely to be high. Another rule states that if the purpose of the loan is to buy a used car, the credit risk score is also likely to be high. Clearly, these rules rely on correlations in the data rather than causal relationships.
In contrast, \sysName\ generated a ruleset that offers interventions to improve the credit risk score based on causal relationships. Example selected rules are shown in \cref{sec:casestudy}.



\vspace{1mm}
\revb{We report the solution size, coverage, expected utility for protected and non-protected, and the unfairness of the rulesets generated using IDS and FRL (as explained in \cref{sec:exp_setup}).
The results are presented in 
{\bf \cref{tab:problem_variants}}. Notably, the expected utility for both protected and non-protected groups across both datasets is generally lower than that achieved by \sysName. \sysName\ consistently delivers higher expected utility for both groups and a smaller difference between these values. This indicates that our approach to mining grouping and intervention patterns is more effective than relying on these algorithms for the same purpose.  However, we note that the rules in IDS and FRL had different objectives (prediction accuracy) and had to be adapted for quantitative comparison using our measures.} 

% \nativ{comment}
% \brit{IDS full: 12 rules 4 seconds, IDS protected: 20 rules, 4 seconds}
% \brit{FRL full: 13 rules 273 seconds, FRL protected: 11 rules 279 seconds}

\subsubsection{\reva{Robustness to the Causal DAG}}
\label{subsec:causal_DAG_robustness}
\reva{The quality of the generated rules may depend on the accuracy of the underlying causal DAG. To evaluate this, we examine the impact of different causal DAGs on the rules. The causal DAGs considered are as follows:
{
\textbf{(1) 1-layer Indep DAG:} A causal DAG where all attributes are independent of each other and only impact the outcome. This setting similar to the scenario where all the causal graph is ignored.
\textbf{(2) 2-layer Mutable DAG:} A simplified DAG where immutable attributes affect the mutable attributes, which impact the outcome variable. In this graph, all immutable attributes act as confounders but do not directly impact the outcome.
%Another default causal DAG where all immutable attributes point to mutable attributes, which in turn point to the outcome.  
\textbf{(3) 2-layer DAG:} A simplified DAG where all variables affect the outcome but the mutable attributes are also confounded by all immutable attributes. }
%Another 2-layer causal DAG. In this DAG, to include confounding variables, all edges in the default-2-layer DAG are present, with additional edges from the top layer to the outcome.  
\textbf{(4) PC DAG:} A causal DAG generated by the PC causal discovery algorithm~\cite{spirtes2001causation}}. 



\reva{The results are depicted in \cref{tab:causal_dag_variants}. We report the expected utility as computed over the different causal DAGs. We observe that the expected utility remains similar for the Stack overflow dataset, demonstrating robustness towards the choice of causal dag. The results show some variability in German credit. However, the PC DAG and the original causal DAG are the most accurate (as they are based on the data distribution and domain knowledge) and achieve the highest coverage and expected utility.}



\begin{table*}[h!]
\centering
\small
\caption{\reva{Metrics Comparison with different Causal DAGs. 
%In parenthesis are the expected utility values computed on the original  causal DAG.
}}
\label{tab:causal_dag_variants}
\begin{tabular}{p{40mm}ccccccc}
\toprule
\textbf{Stack Overflow (SP group fairness + group coverage)} & \textbf{\# rules} & \textbf{coverage} & \textbf{coverage pro} & \textbf{exp utility} & \textbf{exp utility non-pro}&\textbf{exp utility pro} &\textbf{unfairness} \\

\midrule 

Original causal DAG  & 11& 97.95\%& 98.85\%& 27934.76& 28144.58& 18145.23& 9999.35\\



\reva{1-Layer Indep DAG} &\reva{11}&\reva{98.38\%} & \reva{98.38\%}&\reva{28110.19}& \reva{28117}
&\reva{18117.45}
&\reva{9972} \\


% expected_utility’: 28110.19, ‘unprotected_expected_utility’: 28117.0, ‘protected_expected_utility’: 18117.45, ‘coverage_rate’: ’98.38%’, ‘protected_coverage_rate’: ’98.83%


\reva{2-Layer Mutable DAG} &\reva{10}	
&\reva{97.7\%}
 &\reva{98.4\%} &\reva{28198.59}&\reva{28193.09} &\reva{18193.23
}&\reva{9999.86} \\


\reva{2-Layer DAG} &\reva{10}	
&\reva{98.47\%}
 &\reva{98.87\%} &\reva{28106.4}&\reva{28211.17} &\reva{18211.4}&\reva{9999.77} \\

\reva{PC DAG} &\reva{10}&\reva{97.7\%} &\reva{98.4\%} &\reva{28198.59}&\reva{28193.09} &\reva{18193.23}&\reva{9999.86} \\

        
\midrule

\textbf{German Credit (BGL group fairness + group coverage)} & \textbf{\# rules} & \textbf{coverage} & \textbf{coverage pro} & \textbf{exp utility} & \textbf{exp utility non-pro}&\textbf{exp utility pro}&\textbf{unfairness}   \\
\midrule 

Original causal DAG  & 6& 100.0\%& 100.0\%& 0.36& 0.37& 0.31& 0.06 \\
\reva{1-Layer Indep DAG} &\reva{12}&\reva{100\%} &\reva{100\%} &\reva{0.31}& \reva{0.31}&\reva{0.29}&\reva{0.02} \\
\reva{2-Layer Mutable DAG}&\reva{13} &\reva{76.20\%}		
&\reva{79.35\%} & \reva{0.22}&\reva{0.22}&\reva{0.2} &\reva{0.02} \\

\reva{2-Layer DAG} &\reva{11}	
&\reva{71.20\%}
 &\reva{73.91\%} &\reva{0.26}&\reva{0.25} &\reva{0.23}&\reva{0.02} \\

\reva{PC DAG} &\reva{24}&\reva{100.00\%}	
 &\reva{100.00\%} &\reva{0.39}&\reva{0.39} &\reva{0.26}&\reva{0.13} \\
\bottomrule
\end{tabular}
%\vspace{-3mm}
\end{table*}



\subsection{Scalability Evaluation (Q2)}
\label{exp:scalability}
% In this section, we omit the results for the baselines from the presentation, as their response times are significantly slower.


% \begin{figure}[h!]
%     \centering
%     \begin{subfigure}[b]{0.23\textwidth}
%         \centering
% \includegraphics[width=\textwidth]{example-image-a}
%         \caption{Stack Overflow}
%         \label{fig:first}
%     \end{subfigure}
%     \hfill
%     \begin{subfigure}[b]{0.23\textwidth}
%         \centering
% \includegraphics[width=\textwidth]{example-image-b}
%         \caption{German Credit}
%         \label{fig:second}
%     \end{subfigure}
%     \caption{Runtime by-step of the \sysName\ algorithm}
%     \label{fig:runtime_by_step}
% \end{figure}


\begin{figure}[t]
    %\vspace{-2mm}
    \begin{subfigure}[b]{0.46\textwidth}
        \centering
        \includegraphics[width=0.6\textwidth]{figs/time_v_size.pdf}
        % \caption{Stack Overflow}
        % \label{fig:first}
    \end{subfigure}
    %\vspace{-mm}
    \caption{\revb{Runtime as a function of the dataset size (SO)}}
\label{fig:runtime_dataset_size}
\end{figure}

\begin{figure}[t]
    \vspace{-5mm}
    \centering
        \begin{subfigure}[b]{0.48\textwidth}
        \centering
        \includegraphics[width=\textwidth]{figs/time_v_num_attr_line.pdf}
        \end{subfigure}
    \caption{\revb{Runtime as a function of number of mutable and immutable attributes for SO with statistical parity}}
\label{fig:runtime_attributes}
\end{figure}



% \begin{figure}[h!]
%     \centering
%     \begin{subfigure}[b]{0.23\textwidth}
%         \centering
% \includegraphics[width=\textwidth]{figs/time_v_num_immutable_line.pdf}
%         % \caption{Stack Overflow}
%          \label{fig:immutable}
%     \end{subfigure}
%     \vspace{-1mm}
%     \caption{Runtime as a function of number of immutable attributes}
%     \label{fig:runtime_immutable_attributes}
% \end{figure}

\paratitle{Breakdown analysis by step}
Figure~\ref{fig:runtime_by_step} shows the runtime comparison of \sysName for different problem settings. Observe that using rule coverage constraint has the lowest runtime because it helps to prune rules which do not satisfy the coverage constraint. Employing rule coverage with individual fairness is the fastest among all settings, while no constraint setting takes the longest time.
The time taken by the group mining phase is less than $2$ seconds across all setups, and is therefore not visible in the plot. The intervention mining phase (Step 2) is the most inefficient phase, which takes around $6$ mins for the unconstrained setting. The running time of these components aligns with our time complexity analysis (\cref{sec:algo}). Due to space restrictions, we do not present the corresponding plot for German dataset. All conclusions remain the same but the overall running time is $\approx 10\times$ faster due to its smaller size.

\revb{The running time of \sysName and the baselines is comparable. 
FRL is an order of magnitude slower than IDS because it uses a Bayesian modeling approach to simultaneously select a subset of rules and determine their optimal order, which involves solving a computationally intensive combinatorial problem. In contrast, IDS leverages submodular optimization on an unordered set of rules, significantly reducing the size of the search.}
%
We now analyze the impact of system parameters and data size on performance. 


\smallskip
\paratitle{Data Size} \Cref{fig:runtime_dataset_size} compares the running time of \sysName\ \revb{and the baselines} for varying dataset sizes. We observe that the time taken by \sysName\  \revb{and the baselines} increases linearly for most of the settings, \revb{with \sysName\ demonstrating a runtime comparable to IDS under certain configurations}. We also observed that the quality of rules returned by sampling $25\%$ of the data points is comparable with the rules returned by using the whole dataset. Therefore, sampling-based optimizations can help to reduce the running time from $11$ min to less than $2$ min for the unconstrained setting and less than a minute with fairness constraints. 

%We analyze the impact of dataset size on runtime through random sampling of tuples. The results are shown in \cref{fig:runtime_dataset_size}. \brit{TODO}



\smallskip
\paratitle{Number of Attributes} Figure~\ref{fig:runtime_attributes} shows the runtime of \sysName\ while increasing the number of mutable and immutable attributes. 
On increasing the number of mutable attributes, the number of intervention patterns increases exponentially while on increasing immutable attributes, the number of grouping patterns increases exponentially. Therefore, both have a similar impact on runtime. \revb{IDS and FRL do not distinguish between mutable and immutable attributes and there the runtime increases slightly due to an increase in the number of attributes, as more rules are considered.}
%Comparing the two plots, we can see that the running time is dependent on the total number of attributes and not just the number of mutable attributes. However, the reasons of 
% We observe that the running time increases with increasing num
% We examine the impact of attribute quantity on runtime, by randomly excluding attributes from consideration. The results are shown in \cref{fig:runtime_attributes}. \brit{TODO}

\reva{In the following, we omit the results for the IDS and FRL baselines, as these parameters do not impact their runtime.}

\smallskip
\paratitle{Fairness Threshold} %We examine the effect of the threshold $\epsilon$ used to assess the group fairness constraint (\cref{subsec:fairness_constraint}.
\Cref{tab:fairness_variants} presents the results for varying $\epsilon$ for group and individual fairness. We observe that the unfairness of the returned solution increases with the increase in $\epsilon$.  Additionally, the overall expected utility increases but the expected utility of the protected individuals decreases. This result matches our intuition as highly unfair rules are selected for higher values of $\epsilon$. We also notice that the greedy algorithm satisfies the group fairness constraint in all scenarios (unfairness is always less than the desired threshold).

For individual fairness, the overall utility increases monotonically with $\epsilon$. However, the rate of growth for individual fairness is slower than that of group fairness.
One interesting observation about individual fairness is that when all rules have statistical parity difference less than $2500$, the overall unfairness is still around $11K$. This sudden increase in unfairness when considering multiple fair rules together is because we evaluate the upper bound of unfairness by taking the difference between max utility of unprotected and min utility of protected individuals. On manual inspection, we observed that all rules are indeed individually fair.



% In case of individual fairness,  unfair rules are chosen

% As the value of $\epsilon$ increases, the fairness of the solution may decrease. \brit{todo}

\smallskip
\paratitle{Coverage Threshold} With the change in coverage thresholds, we do not observe major difference in the overall results because the majority of the rules exhibit very high coverage (\cref{tab:problem_variants}). %We examine the effect of coverage the thresholds $\theta$ and $\theta_p$. 

\smallskip
\paratitle{Apriori Threshold} 
We observe that increasing the Apriori threshold $\tau$ leads to a reduction in the number of grouping patterns considered, and thus to a decrease in runtime. However, our findings indicate that higher $\tau$ values lead to a decrease in both utility and fairness. Based on our findings, we recommend using a default value of $0.1$, which provides satisfactory results in terms of coverage, utility, fairness and runtime.

%used to define the group coverage constraint (\cref{secsec:coverage_constraint}). As the value of $\theta$ and $\theta_p$ increase, \brit{todo}

% \brit{Here we compare the results with existing baselines in terms of running times. We then isolate each phase of the algo to investigate its effect}

% \brit{Examine the effect on running times when varying: (1) number of tuples (2) number of attributes (3) threshold of apriori (4) coverage constraint (5) fairness constraint}

% \brit{add an experiment to show how we operate with a default causal DAG (everything affects the outcome)}






%\section{Related Work}
%\label{sec:related-work}

%\subsection{Background}

%Defect detection is critical to ensure the yield of integrated circuit manufacturing lines and reduce faults. Previous research has primarily focused on wafer map data, which engineers produce by marking faulty chips with different colors based on test results. The specific spatial distribution of defects on a wafer can provide insights into the causes, thereby helping to determine which stage of the manufacturing process is responsible for the issues. Although such research is relatively mature, the continual miniaturization of integrated circuits and the increasing complexity and density of chip components have made chip-level detection more challenging, leading to potential risks\cite{ma2023review}. Consequently, there is a need to combine this approach with magnified imaging of the wafer surface using scanning electron microscopes (SEMs) to detect, classify, and analyze specific microscopic defects, thus helping to identify the particular process steps where defects originate.

%Previously, wafer surface defect classification and detection were primarily conducted by experienced engineers. However, this method relies heavily on the engineers' expertise and involves significant time expenditure and subjectivity, lacking uniform standards. With the ongoing development of artificial intelligence, deep learning methods using multi-layer neural networks to extract and learn target features have proven highly effective for this task\cite{gao2022review}.

%In the task of defect classification, it is typical to use a model structure that initially extracts features through convolutional and pooling layers, followed by classification via fully connected layers. Researchers have recently developed numerous classification model structures tailored to specific problems. These models primarily focus on how to extract defect features effectively. For instance, Chen et al. presented a defect recognition and classification algorithm rooted in PCA and classification SVM\cite{chen2008defect}. Chang et al. utilized SVM, drawing on features like smoothness and texture intricacy, for classifying high-intensity defect images\cite{chang2013hybrid}. The classification of defect images requires the formulation of numerous classifiers tailored for myriad inspection steps and an Abundance of accurately labeled data, making data acquisition challenging. Cheon et al. proposed a single CNN model adept at feature extraction\cite{cheon2019convolutional}. They achieved a granular classification of wafer surface defects by recognizing misclassified images and employing a k-nearest neighbors (k-NN) classifier algorithm to gauge the aggregate squared distance between each image feature vector and its k-neighbors within the same category. However, when applied to new or unseen defects, such models necessitate retraining, incurring computational overheads. Moreover, with escalating CNN complexity, the computational demands surge.

%Segmentation of defects is necessary to locate defect positions and gather information such as the size of defects. Unlike classification networks, segmentation networks often use classic encoder-decoder structures such as UNet\cite{ronneberger2015u} and SegNet\cite{badrinarayanan2017segnet}, which focus on effectively leveraging both local and global feature information. Han Hui et al. proposed integrating a Region Proposal Network (RPN) with a UNet architecture to suggest defect areas before conducting defect segmentation \cite{han2020polycrystalline}. This approach enables the segmentation of various defects in wafers with only a limited set of roughly labeled images, enhancing the efficiency of training and application in environments where detailed annotations are scarce. Subhrajit Nag et al. introduced a new network structure, WaferSegClassNet, which extracts multi-scale local features in the encoder and performs classification and segmentation tasks in the decoder \cite{nag2022wafersegclassnet}. This model represents the first detection system capable of simultaneously classifying and segmenting surface defects on wafers. However, it relies on extensive data training and annotation for high accuracy and reliability. 

%Recently, Vic De Ridder et al. introduced a novel approach for defect segmentation using diffusion models\cite{de2023semi}. This approach treats the instance segmentation task as a denoising process from noise to a filter, utilizing diffusion models to predict and reconstruct instance masks for semiconductor defects. This method achieves high precision and improved defect classification and segmentation detection performance. However, the complex network structure and the computational process of the diffusion model require substantial computational resources. Moreover, the performance of this model heavily relies on high-quality and large amounts of training data. These issues make it less suitable for industrial applications. Additionally, the model has only been applied to detecting and segmenting a single type of defect(bridges) following a specific manufacturing process step, limiting its practical utility in diverse industrial scenarios.

%\subsection{Few-shot Anomaly Detection}
%Traditional anomaly detection techniques typically rely on extensive training data to train models for identifying and locating anomalies. However, these methods often face limitations in rapidly changing production environments and diverse anomaly types. Recent research has started exploring effective anomaly detection using few or zero samples to address these challenges.

%Huang et al. developed the anomaly detection method RegAD, based on image registration technology. This method pre-trains an object-agnostic registration network with various images to establish the normality of unseen objects. It identifies anomalies by aligning image features and has achieved promising results. Despite these advancements, implementing few-shot settings in anomaly detection remains an area ripe for further exploration. Recent studies show that pre-trained vision-language models such as CLIP and MiniGPT can significantly enhance performance in anomaly detection tasks.

%Dong et al. introduced the MaskCLIP framework, which employs masked self-distillation to enhance contrastive language-image pretraining\cite{zhou2022maskclip}. This approach strengthens the visual encoder's learning of local image patches and uses indirect language supervision to enhance semantic understanding. It significantly improves transferability and pretraining outcomes across various visual tasks, although it requires substantial computational resources.
%Jeong et al. crafted the WinCLIP framework by integrating state words and prompt templates to characterize normal and anomalous states more accurately\cite{Jeong_2023_CVPR}. This framework introduces a novel window-based technique for extracting and aggregating multi-scale spatial features, significantly boosting the anomaly detection performance of the pre-trained CLIP model.
%Subsequently, Li et al. have further contributed to the field by creating a new expansive multimodal model named Myriad\cite{li2023myriad}. This model, which incorporates a pre-trained Industrial Anomaly Detection (IAD) model to act as a vision expert, embeds anomaly images as tokens interpretable by the language model, thus providing both detailed descriptions and accurate anomaly detection capabilities.
%Recently, Chen et al. introduced CLIP-AD\cite{chen2023clip}, and Li et al. proposed PromptAD\cite{li2024promptad}, both employing language-guided, tiered dual-path model structures and feature manipulation strategies. These approaches effectively address issues encountered when directly calculating anomaly maps using the CLIP model, such as reversed predictions and highlighting irrelevant areas. Specifically, CLIP-AD optimizes the utilization of multi-layer features, corrects feature misalignment, and enhances model performance through additional linear layer fine-tuning. PromptAD connects normal prompts with anomaly suffixes to form anomaly prompts, enabling contrastive learning in a single-class setting.

%These studies extend the boundaries of traditional anomaly detection techniques and demonstrate how to effectively address rapidly changing and sample-scarce production environments through the synergy of few-shot learning and deep learning models. Building on this foundation, our research further explores wafer surface defect detection based on the CLIP model, especially focusing on achieving efficient and accurate anomaly detection in the highly specialized and variable semiconductor manufacturing process using a minimal amount of labeled data.

% \section{Discussion}
\label{sec:discuss}
\jing{Janus maintains a warm pod pool to decrease the impact of cold start.
Despite its simplicity, this pool may inevitably compromise resource efficiency.
Additionally, Janus may exhibit sub-optimal resource adaptation for dynamic workflows with uncertain execution paths.
This uncertainty exacerbates the dependencies of adaptation decisions (as detailed in \S\ref{sec:bg:adaptive-allocation}), potentially leading to sub-optimal adaptation especially for functions shared across different paths.}







\section{Conclusions}
In this paper, we proposed the formalism of distributionally robust DPO,  developed two novel algorithms using this framework,  and established their theoretical guarantees. We also developed efficient approximation techniques that enable scalable implementation of these algorithms as part of the existing LLM alignment pipeline. We showed extensive empirical evaluations that validate the effectiveness of our proposed algorithms in addressing preference distribution shifts in LLM alignment. In future works, we plan to extend our distributionally robust DPO algorithms to address the challenges of reward hacking. We also plan to develop distributionally robust algorithms for other RLHF approaches. 


\section*{Acknowledgment}
This work was supported in part by National Science Foundation of China under grant 62232012, in part by National Key Research \&
Development (R\&D) Plan under grant 2022YFB4501703, in part by the Major Key Project of PCL under Grant PCL2024A06 and PCL2022A05, and in part by the Shenzhen Science and Technology Program under Grant RCJC20231211085918010.
% The preferred spelling of the word ``acknowledgment'' in America is without 
% an ``e'' after the ``g''. Avoid the stilted expression ``one of us (R. B. 
% G.) thanks $\ldots$''. Instead, try ``R. B. G. thanks$\ldots$''. Put sponsor 
% acknowledgments in the unnumbered footnote on the first page.

\bibliographystyle{IEEEtran}
\bibliography{ref}

\end{document}
