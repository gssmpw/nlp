\section{Conclusion}
\label{sec:conclusion}
In this paper, we identified the resource inefficiency lying in the early-binding based resource allocation for serverless workflows, and proposed a late-binding approach to address it by promoting bilateral runtime resource adaptation engaging both the developer and the provider.
Based on this concept, we proposed \namex---a novel resource adaptation framework for serverless workflows. 
We identified the challenges in building \namex and proposed efficient algorithms for fine-grained resource allocation %with MPS%
for \namex.
Experiments based on a system prototype show that \namex achieves significant resource savings while providing latency SLO guarantee.
Future work includes adding support for more complex workflows and exploring the impact of the runtime resource adaptation on function caching strategies.



% In this paper, we identified the DNN fragment misalignment problem in inference serving for hybrid DL and proposed a new concept called re-alignment to address it by promoting request batching and sharing.
% We proposed \namex---a first-of-its-kind inference serving system for hybrid DL adopting re-alignment. 
% We identified the challenges in building \namex and proposed efficient algorithms for fine-grained resource allocation %with MPS%
% for \namex.
% Experiments based on a system prototype show that \namex achieves significant resource savings while providing latency SLO guarantee.