\documentclass[conference]{IEEEtran}
\IEEEoverridecommandlockouts
% The preceding line is only needed to identify funding in the first footnote. If that is unneeded, please comment it out.
\usepackage{cite}
\usepackage{amsmath,amssymb,amsfonts}
\usepackage{algorithm}
\usepackage[noend]{algpseudocode}
\usepackage{graphicx}
\usepackage{textcomp}
\usepackage{xcolor}
\usepackage{diagbox}
\usepackage{tikz}
\usepackage{float}
\usepackage{array}
\newcolumntype{P}[1]{>{\centering\arraybackslash}p{#1}}
\newcommand*\circled[1]{\tikz[baseline=(char.base)]{
            \node[shape=circle,draw,inner sep=2pt] (char) {#1};}}
\def\BibTeX{{\rm B\kern-.05em{\sc i\kern-.025em b}\kern-.08em
    T\kern-.1667em\lower.7ex\hbox{E}\kern-.125emX}}
\newcommand{\spaarc}{SAAP}
\begin{document}

    \title{SAAP: Spatial awareness and Association based Prefetching of Virtual Objects in Augmented Reality at the Edge}

    \author{\IEEEauthorblockN{Nikhil Sreekumar, Abhishek Chandra, Jon Weissman}
    \IEEEauthorblockA{\textit{University of Minnesota, Twin Cities} \\
    MN, USA \\
    \{sreek012, chandra, weiss039\}@umn.edu}}
    % \and
    % \IEEEauthorblockN{2\textsuperscript{nd} Given Name Surname}
    % \IEEEauthorblockA{\textit{dept. name of organization (of Aff.)} \\
    % \textit{name of organization (of Aff.)}\\
    % City, Country \\
    % email address or ORCID}
    % \and
    % \IEEEauthorblockN{3\textsuperscript{rd} Given Name Surname}
    % \IEEEauthorblockA{\textit{dept. name of organization (of Aff.)} \\
    % \textit{name of organization (of Aff.)}\\
    % City, Country \\
    % email address or ORCID}
    % \and
    % \IEEEauthorblockN{4\textsuperscript{th} Given Name Surname}
    % \IEEEauthorblockA{\textit{dept. name of organization (of Aff.)} \\
    % \textit{name of organization (of Aff.)}\\
    % City, Country \\
    % email address or ORCID}
    % \and
    % \IEEEauthorblockN{5\textsuperscript{th} Given Name Surname}
    % \IEEEauthorblockA{\textit{dept. name of organization (of Aff.)} \\
    % \textit{name of organization (of Aff.)}\\
    % City, Country \\
    % email address or ORCID}
    % \and
    % \IEEEauthorblockN{6\textsuperscript{th} Given Name Surname}
    % \IEEEauthorblockA{\textit{dept. name of organization (of Aff.)} \\
    % \textit{name of organization (of Aff.)}\\
    % City, Country \\
    % email address or ORCID}
    % }

    \maketitle

    \begin{abstract}
        Mobile Augmented Reality (MAR) applications face performance challenges due to their high computational demands and need for low-latency responses.  Traditional approaches like on-device storage or reactive data fetching from the cloud often result in limited AR experiences or unacceptable lag. Edge caching, which caches AR objects closer to the user, provides a promising solution. However, existing edge caching approaches do not consider AR-specific features such as AR object sizes, user interactions, and physical location. This paper investigates how to further optimize edge caching by employing AR-aware prefetching techniques. We present \spaarc, a Spatial Awareness and Association-based Prefetching policy specifically designed for MAR Caches.  \spaarc{} intelligently prioritizes the caching of virtual objects based on their association with other similar objects and the user's proximity to them. It also considers the recency of associations and uses a lazy fetching strategy to efficiently manage edge resources and maximize Quality of Experience (QoE).

        Through extensive evaluation using both synthetic and real-world workloads, we demonstrate that \spaarc{} significantly improves cache hit rates compared to standard caching algorithms, achieving gains ranging from 3\% to 40\% while reducing the need for on-demand data retrieval from the cloud. Further, we present an adaptive tuning algorithm that automatically tunes \spaarc{} parameters to achieve optimal performance. Our findings demonstrate the potential of \spaarc{} to substantially enhance the user experience in MAR applications by ensuring the timely availability of virtual objects.
    \end{abstract}

    \begin{IEEEkeywords}
    Augmented reality, caching, prefetching, association, proximity, support, confidence, edge computing
    \end{IEEEkeywords}
    \vspace{-3mm}
    \section{Introduction}


\begin{figure}[t]
\centering
\includegraphics[width=0.6\columnwidth]{figures/evaluation_desiderata_V5.pdf}
\vspace{-0.5cm}
\caption{\systemName is a platform for conducting realistic evaluations of code LLMs, collecting human preferences of coding models with real users, real tasks, and in realistic environments, aimed at addressing the limitations of existing evaluations.
}
\label{fig:motivation}
\end{figure}

\begin{figure*}[t]
\centering
\includegraphics[width=\textwidth]{figures/system_design_v2.png}
\caption{We introduce \systemName, a VSCode extension to collect human preferences of code directly in a developer's IDE. \systemName enables developers to use code completions from various models. The system comprises a) the interface in the user's IDE which presents paired completions to users (left), b) a sampling strategy that picks model pairs to reduce latency (right, top), and c) a prompting scheme that allows diverse LLMs to perform code completions with high fidelity.
Users can select between the top completion (green box) using \texttt{tab} or the bottom completion (blue box) using \texttt{shift+tab}.}
\label{fig:overview}
\end{figure*}

As model capabilities improve, large language models (LLMs) are increasingly integrated into user environments and workflows.
For example, software developers code with AI in integrated developer environments (IDEs)~\citep{peng2023impact}, doctors rely on notes generated through ambient listening~\citep{oberst2024science}, and lawyers consider case evidence identified by electronic discovery systems~\citep{yang2024beyond}.
Increasing deployment of models in productivity tools demands evaluation that more closely reflects real-world circumstances~\citep{hutchinson2022evaluation, saxon2024benchmarks, kapoor2024ai}.
While newer benchmarks and live platforms incorporate human feedback to capture real-world usage, they almost exclusively focus on evaluating LLMs in chat conversations~\citep{zheng2023judging,dubois2023alpacafarm,chiang2024chatbot, kirk2024the}.
Model evaluation must move beyond chat-based interactions and into specialized user environments.



 

In this work, we focus on evaluating LLM-based coding assistants. 
Despite the popularity of these tools---millions of developers use Github Copilot~\citep{Copilot}---existing
evaluations of the coding capabilities of new models exhibit multiple limitations (Figure~\ref{fig:motivation}, bottom).
Traditional ML benchmarks evaluate LLM capabilities by measuring how well a model can complete static, interview-style coding tasks~\citep{chen2021evaluating,austin2021program,jain2024livecodebench, white2024livebench} and lack \emph{real users}. 
User studies recruit real users to evaluate the effectiveness of LLMs as coding assistants, but are often limited to simple programming tasks as opposed to \emph{real tasks}~\citep{vaithilingam2022expectation,ross2023programmer, mozannar2024realhumaneval}.
Recent efforts to collect human feedback such as Chatbot Arena~\citep{chiang2024chatbot} are still removed from a \emph{realistic environment}, resulting in users and data that deviate from typical software development processes.
We introduce \systemName to address these limitations (Figure~\ref{fig:motivation}, top), and we describe our three main contributions below.


\textbf{We deploy \systemName in-the-wild to collect human preferences on code.} 
\systemName is a Visual Studio Code extension, collecting preferences directly in a developer's IDE within their actual workflow (Figure~\ref{fig:overview}).
\systemName provides developers with code completions, akin to the type of support provided by Github Copilot~\citep{Copilot}. 
Over the past 3 months, \systemName has served over~\completions suggestions from 10 state-of-the-art LLMs, 
gathering \sampleCount~votes from \userCount~users.
To collect user preferences,
\systemName presents a novel interface that shows users paired code completions from two different LLMs, which are determined based on a sampling strategy that aims to 
mitigate latency while preserving coverage across model comparisons.
Additionally, we devise a prompting scheme that allows a diverse set of models to perform code completions with high fidelity.
See Section~\ref{sec:system} and Section~\ref{sec:deployment} for details about system design and deployment respectively.



\textbf{We construct a leaderboard of user preferences and find notable differences from existing static benchmarks and human preference leaderboards.}
In general, we observe that smaller models seem to overperform in static benchmarks compared to our leaderboard, while performance among larger models is mixed (Section~\ref{sec:leaderboard_calculation}).
We attribute these differences to the fact that \systemName is exposed to users and tasks that differ drastically from code evaluations in the past. 
Our data spans 103 programming languages and 24 natural languages as well as a variety of real-world applications and code structures, while static benchmarks tend to focus on a specific programming and natural language and task (e.g. coding competition problems).
Additionally, while all of \systemName interactions contain code contexts and the majority involve infilling tasks, a much smaller fraction of Chatbot Arena's coding tasks contain code context, with infilling tasks appearing even more rarely. 
We analyze our data in depth in Section~\ref{subsec:comparison}.



\textbf{We derive new insights into user preferences of code by analyzing \systemName's diverse and distinct data distribution.}
We compare user preferences across different stratifications of input data (e.g., common versus rare languages) and observe which affect observed preferences most (Section~\ref{sec:analysis}).
For example, while user preferences stay relatively consistent across various programming languages, they differ drastically between different task categories (e.g. frontend/backend versus algorithm design).
We also observe variations in user preference due to different features related to code structure 
(e.g., context length and completion patterns).
We open-source \systemName and release a curated subset of code contexts.
Altogether, our results highlight the necessity of model evaluation in realistic and domain-specific settings.





    \section{Motivation and Background}
\label{sec:motive}
    Achieving an immersive mobile augmented reality (MAR) experience presents unique challenges \cite{bib:networkar,bib:locar}. Conventional data caching and prefetching policies, designed for general-purpose applications, might not be optimal for MAR due to its inherent characteristics\cite{bib:archaract}. Compared to other edge-native applications \cite{bib:edgenative}, MAR applications require significant computational resources and sophisticated data management strategies to handle the large volume of spatial data and real-time interactions. 

    \subsection{Motivating AR Scenarios}
    \begin{figure}
        \centering
        \includegraphics[scale=0.13]{images/motive/motiveapp.png}
        \caption{Grocery and tourist scenarios}
        \label{fig:motiveapp}
        \vspace{-6mm}
    \end{figure}
    Figure \ref{fig:motiveapp} shows two example application scenarios for AR. In a grocery store scenario\cite{bib:phara}, users interact with virtual objects of the store items to receive information about their ingredients, manufacturing details, price variations, recipes, and so on. The items are typically placed on racks in aisles based on their categories, and there is a well defined order of placement. The Field of View (FoV) of a user will have multiple items at their location in the store, from which the user might be interested in a few. 
    
    In the case of a tourist scenario, \cite{bib:dublin}, attractions are typically scattered across a region (such as a city block, campus, or neighborhood). The virtual objects may contain historical facts, 3D representations of buildings, informative audio/video, and so on. A tourist would visit the attractions based on multiple factors like user interests, transportation availability, accessibility, guide recommendations, and so on. The FoV of the user in this case would likely consist of only a few items in their vinicity, and the user will have to explore the region more to find the next attraction. 

    %\subsection{MAR challenges}
    %There are multiple MAR specific challenges when 
    %Serving users in 
    Such MAR scenarios face the following challenges:
    
    \noindent$\bullet$ {\em Limited device storage and large object sizes}:  While current virtual objects are typically under 20MB \cite{bib:carsar, bib:objaverse}, increasing interaction complexity and 3D objects will lead to larger object sizes.  Storing all objects locally on the device  becomes impractical as object size and complexity grow.
    
    \noindent$\bullet$ {\em High cloud latency and strict AR latency requirements}: Average cloud latency (around 60ms \cite{bib:clatency, bib:clt1, bib:clt2}) exceeds the sub-20ms threshold required for immersive AR experience \cite{bib:20ms}.  On-demand fetching of objects from the cloud introduces significant latency \cite{bib:delayhits}, negatively impacting user experience.  
    %Anticipating user needs and preemptively caching data at the edge enables rapid availability and efficient sharing among users.
    %Achieving such low latency necessitates the use of 
    %a combined approach utilizing 5G/6G networks and 
    %edge infrastructure.
    
    \if 0
        \noindent$\bullet$ On-demand fetch inefficiency: On-demand fetching of objects from the cloud introduces significant latency \cite{bib:delayhits}, negatively impacting user experience.  Anticipating user needs and preemptively caching data at the edge enables rapid availability and efficient sharing among users.

    
    \noindent 4. Power Consumption and Data Transfer:  On-demand fetching leads to inefficient bandwidth usage, increased data transfer latency, and higher power consumption \cite{bib:energy}.  Intelligent prefetching methods are crucial to proactively store virtual objects, minimize power consumption and improve efficiency.
    \fi

    % The key challenges in mobile AR caching for application scenarios discussed above are:
    
    % \noindent 1. Limited Device Storage and High Object Sizes:  While most current virtual objects are under 20MB \cite{bib:carsar, bib:objaverse}, as interaction complexity increases, the size will also increase. Storing all objects locally or fetching them on-demand from the cloud becomes impractical as object sizes and complexity grow.

    % \noindent 2. High Cloud Latency and Strict Latency Requirements:  Average cloud latency hovers around 60ms \cite{bib:clatency, bib:clt1, bib:clt2}, exceeding the sub-20ms latency threshold required for an immersive AR experience \cite{bib:20ms}. Only a combined approach utilizing 5G/6G networks and edge infrastructure can potentially achieve this low latency target.

    % \noindent 3. On-Demand Fetch Inefficiency:  The on-demand fetching of objects from the cloud can introduce significant latency, negatively impacting user experience. By anticipating user needs and preemptively storing data at the edge, we can ensure rapid availability and facilitate efficient sharing among users with similar interests.
    
    % \noindent 4. Power Consumption and Data Transfer:The inefficiencies inherent to on-demand fetching can result in missed opportunities to reduce bandwidth usage, subsequently leading to elevated data transfer latencies and increased power consumption \cite{bib:archaract}. To mitigate these effects, intelligent prefetching methods are required to proactively store virtual objects anticipated to be accessed in the near future, thereby minimizing power consumption.

\subsection{Edge caching}
    Edge infrastructure is well placed to meet these challenges by expanding on the limited on-device storage capacity, while reducing the user latency. The use of the edge in an MAR context necessitates novel data management policies that consider both AR-specific data characteristics and edge/cloud parameters. 
    %Achieving such low latency necessitates while meeting the storage requirements of AR objects the use of 
    %a combined approach utilizing 5G/6G networks and 
    %edge infrastructure.
    
    Recent research on MAR architecture and applications \cite{bib:edgearch, bib:carsar, bib:sear,bib:localcache,bib:slamshare} partitions data and tasks across mobile devices and edge servers to leverage resources at both locations and facilitate data sharing. Caching is crucial for improving AR application performance, and partitioning the cache across devices and edge servers has shown potential for increasing hit rates and reducing latency~\cite{bib:carsar, bib:sear}. 
    
    However, existing approaches often neglect critical AR-specific factors: (1) \textit{Increasing Object Sizes}: Caching strategies must adapt to the growing size and complexity of virtual objects. (2) \textit{Occlusion and User Proximity}: Virtual objects can be occluded by physical objects in AR environments. Additionally, users are more likely to interact with nearby objects. These spatial considerations, along with object access patterns, can inform more effective caching strategies. (3) \textit{Object Relevance and User Associations}: User interactions with virtual objects often exhibit predictable patterns (e.g., a user interacting with a milk carton may subsequently interact with nearby eggs or bread). However, caching strategies must assess the relevance of potential interactions to avoid unnecessary cache pollution.

    % These challenges necessitate novel data management policies that consider both AR-specific data characteristics and edge cloud parameters. Recent research on mobile AR architecture and applications \cite{bib:edgearch, bib:carsar, bib:sear,bib:localcache,bib:slamshare}, partitions data and tasks across mobile devices and edge servers to leverage resources at both locations and facilitate data sharing. Caching is a crucial aspect for improving AR application performance, and partitioning cache across local devices and edge servers has shown promise in increasing hit rates and reducing response latency \cite{bib:carsar, bib:sear}.  However, existing approaches often overlook AR-specific factors such as (1) \textit{Increasing Object Sizes}: Caching strategies need to adapt to the growing size and complexity of virtual objects. (2) \textit{Occlusion and User Proximity}: In AR environments, virtual objects may be obscured by physical objects positioned in front of them (occlusion). Furthermore, users are significantly more likely to interact with objects within their immediate vicinity compared to those farther away. These spatial considerations, in conjunction with object access patterns, can inform more effective caching strategies. (3) \textit{Object Relevance and User Associations}: A user interacting with a virtual milk carton in a grocery store scenario may be more likely to subsequently interact with nearby eggs, bread, or other related items. However, to avoid unnecessary cache pollution, the relevance of these potential interactions should be assessed before preemptively caching associated objects.
    
   % This paper proposes \spaarc, a prefetching policy that addresses these limitations by incorporating both the proximity of virtual objects to users and object associations to make informed decisions on edge server caching.
    %\vspace{-1.5mm}
    \subsubsection*{The need for Prefetch}
        %The increasing use of mobile devices to interact with large virtual objects presents challenges due to limited device storage. 
        Edge caching offers a promising solution by storing frequently accessed objects closer to the user, reducing cloud traffic and improving access times. However, efficient cache management is crucial given the limited and heterogeneous nature of edge resources.  Traditional cache eviction algorithms (e.g., LRU, LFU, FIFO, etc.) can be employed on the edge cache, but relying solely on reactive retrieval from the cloud upon cache misses can still introduce significant latency. Predictive prefetching based on access patterns can anticipate future requests and proactively cache necessary objects at the edge, reducing cache misses and enhancing user experience at the potential expense of increased network usage.
        
    \vspace{-1.5mm}
    \subsection{Associations and Spatial Proximity}
        Association rule mining (ARM) \cite{bib:arm}, a technique widely used in recommender systems to predict user-item interactions, offers a promising approach for AR prefetching. ARM has proven successful in various domains, including e-commerce \cite{bib:phara}, tourism \cite{bib:dublin}, fraud detection \cite{bib:fraud}, and social network analysis\cite{bib:sna}. For example, in online grocery shopping, purchasing milk and eggs often triggers recommendations for bread and jam. However, the direct application of ARM for prefetching in AR remains an underexplored area.  The rule generation process, particularly with frequent updates, can be computationally expensive, posing a challenge for latency-sensitive AR applications. This work explores adapting ARM to the specific needs of AR, aiming to achieve efficient prefetching while minimizing computational overhead. While ARM effectively identifies related virtual objects, it may not prioritize those most relevant to the user's current location. Prefetching objects far from the user's field of view (FoV) provides limited value in AR. Therefore, this work emphasizes incorporating spatial awareness into the prefetching process. By prioritizing objects near the user's FoV, we aim to optimize cache utilization and enhance the user experience.
        
        % Association rule mining (ARM) \cite{bib:arm}, a well-established technique in recommender systems for predicting user interactions with items, offers a promising approach. ARM has demonstrated success in various domains, including e-commerce \cite{bib:phara}, tourism \cite{bib:dublin}, fraud detection \cite{bib:fraud}, and social network analysis\cite{bib:sna}. For instance, in online grocery shopping, purchasing milk and eggs often leads to recommendations for bread and jam. However, directly applying ARM for prefetching/caching in AR applications is still an area to be explored. The rule generation process can be computationally expensive, especially when frequent updates are required. This becomes problematic for latency-sensitive applications like AR, where rapid response times are paramount. Therefore, this work explores methods to adapt ARM specifically for the needs of AR applications, aiming to achieve efficient prefetching while maintaining low computational overhead.While ARM can effectively identify related virtual objects, it may not prioritize those most relevant to the user's current location. Prefetching objects situated far from the user's field of view (FoV) would offer minimal value in an AR experience.  Therefore, this work emphasizes the importance of incorporating spatial awareness into the prefetching process. By prioritizing objects in close proximity to the user's FoV, we aim to optimize cache utilization and enhance the overall user experience.
    % \vspace{-1.5mm}
    % \subsection{Association rule mining}
    % \label{ssec:arm}

    %     Association rule mining (ARM) \cite{bib:armorig, bib:arm} identifies relationships between items in a transaction database by discovering frequent itemsets—groups of items that frequently co-occur in transactions. The support $(S)$ of an itemset indicates the proportion of transactions containing that itemset, with frequent itemsets exceeding a predefined minimum support threshold. Based on these frequent itemsets, ARM generates association rules of the form  $A \implies B$, where $A$ (antecedent) and $B$ (consequent) are disjoint itemsets. The confidence $(C)$ of a rule represents the conditional probability of observing the consequent in transactions containing the antecedent, controlled by a minimum confidence threshold.  The lift $(L)$ of a rule measures the strength of the association between the antecedent and consequent, quantifying how much more likely they are to co-occur than if they were statistically independent.
        
        % Association rule mining \cite{bib:armorig, bib:arm} discovers relationships between items in a transaction database by identifying frequent itemsets. An itemset represents a group of items that frequently co-occur in transactions. The support of an itemset signifies the proportion of transactions containing the itemset. Frequent itemsets are those that exceed a minimum support threshold. Based on these frequent itemsets, ARM generates a set of association rules expressed as $A \implies C$, where $A$ and $C$ are disjoint itemsets referred to as antecedents and consequents, respectively. The confidence of a rule reflects the conditional probability of encountering the consequent within transactions containing the antecedent.  A minimum confidence threshold allows control over the quality of generated rules. The lift of a rule measures how much more likely two itemsets are to occur together than if they were statistically independent. In simpler terms, it quantifies the strength of the association between the itemsets.
        % Support of an itemset $X$ ($S(X)$), confidence of a rule $A \implies B$ ($C(A \implies B)$) and lift of the rule ($L(A \implies B)$) are formally represented as follows:
        % \begin{align}
        %     S(X) & = \frac{T(X)}{T} \\
        %     C(A \implies B) & = \frac{S(A \implies B)}{S(A)}\\
        %     L(A \implies B) &= \frac{S(A \cup B)}{S(A) * S(B)}
        % \end{align}
        % where $T(X)$ and $T$ are the number of transactions containing $X$ and the total number of transactions respectively.
        % It is important to emphasize that minimum support and minimum confidence are employed as thresholds rather than fixed values in ARM. Increasing these thresholds reduces the number of generated association rules.  In extreme cases, where the desired rule quality cannot be achieved with the specified thresholds, no rules may be generated, as will be further explored in the evaluation section (Section \ref{sec:eval}).
    \section{\spaarc}
\label{sec:spaar}
We propose \spaarc, a prefetching framework that addresses the limitations of existing edge caching approaches for MAR applications. It incorporates both the proximity of virtual objects to users and object associations to make informed decisions on edge server caching. Its prefetching policy is complementary to the underlying caching algorithm employed by the edge cache, and is meant to enhance the cache performance.
%We first present the \spaarc{} architecture, followed by a discussion of its main techniques: association and proximity.

%    In this section, we discuss \spaarc{}, the spatial proximity and association based prefetching for augmented reality cache.
    % \vspace{-3mm}
    \subsection{System Architecture}
        Figure \ref{fig:marspaarc} illustrates the MAR pipeline incorporating \spaarc{} into the edge cache.  Sensor data from the user's device is transmitted to the nearest edge server, where video frames are pre-processed and object detection is performed.  Extracted features are then used for object recognition. If a corresponding virtual object exists, the system attempts to retrieve it from the edge cache.  Upon a cache miss, the \spaarc{} algorithm is invoked to identify all necessary objects for prefetching and request them from the cloud, along with the object that caused the cache miss. After receiving the virtual objects, template matching is performed to estimate their pose. This pose information, along with the virtual objects, is sent to the object tracking module on the user's device.  The tracking module identifies objects across frames, and this information, combined with data from the edge server, is used by the rendering module to overlay virtual content onto the user's display.
        
        % The MAR Pipeline for with \spaarc{} is shown in Figure \ref{fig:marspaarc}. Sensor data is transmitted from the user's device to the nearest edge server. The edge server pre-processes the video frames and applies object detection algorithms. Subsequently, features are extracted and utilized by object recognition algorithms to determine the most accurate item description. If a virtual object is linked to the identified item, the system attempts to retrieve it from the cache. In case of a cache miss, the \spaarc{} algorithm is invoked. This algorithm identifies all necessary objects, including the missed object, and sends a request to the cloud. Upon receiving the virtual objects, template matching is performed to estimate their pose. This pose information, along with the virtual objects, is transmitted to the object tracking module on the user's device. The tracking module identifies objects across frames, and this information, combined with data from the edge server, is used by the rendering module to overlay virtual content onto the user's display.

        \begin{figure}[t]
            \centering
            \includegraphics[scale=0.1]{images/spaar/marspaarc.png}
            \caption{MAR Pipeline with \spaarc{} on the edge cache}
            \label{fig:marspaarc}
            \vspace{-5mm}
        \end{figure}

        %\spaarc{} runs on top of existing cache eviction policies and 
        We next discuss the two main \spaarc{} techniques that drive its prefetching:  association and proximity.
        
        \subsection{Association}
        \label{ssec:spaarc_assoc}
        Intuitively, if a user accesses a virtual object in an MAR scenario, they are likely to soon access related objects (e.g., milk and eggs in the grocery scenario), which should be prefetched to the edge cache for faster access. \spaarc's Association technique generates association rules using {\em Association rule mining (ARM)} \cite{bib:armorig, bib:arm} to predict frequently co-accessed virtual objects based on user interaction history. 
        
        ARM identifies relationships between items in a transaction database by discovering {\em frequent itemsets}: groups of items that frequently co-occur in transactions. The {\em support} $(S)$ of an itemset indicates the proportion of transactions containing that itemset, with frequent itemsets exceeding a predefined minimum support threshold. Based on these frequent itemsets, ARM generates {\em association rules} of the form  $A \implies B$, where $A$ (antecedent) and $B$ (consequent) are disjoint itemsets. The {\em confidence} $(C)$ of a rule represents the conditional probability of observing the consequent in transactions containing the antecedent, controlled by a minimum confidence threshold.  The {\em lift} $(L)$ of a rule measures the strength of the association between the antecedent and consequent, quantifying how much more likely they are to co-occur than if they were statistically independent. Support of an itemset $X$ ($S(X)$), confidence of a rule $A \implies B$ ($C(A \implies B)$) and lift of the rule ($L(A \implies B)$) are formally represented as follows:
            \begin{align}
                S(X) & = \frac{T(X)}{T} \\
                C(A \implies B) & = \frac{S(A \implies B)}{S(A)}\\
                L(A \implies B) &= \frac{S(A \cup B)}{S(A) * S(B)}
            \end{align}
            where $T(X)$ and $T$ are the number of transactions containing $X$ and the total number of transactions respectively. We use ARM algorithms to either generate rules a priori from history user interactions or dynamically during the process.

            \begin{figure}
                \centering
                \includegraphics[scale=0.11]{images/spaar/spaarcflowchart.png}
                \caption{\spaarc{} workflow}
                \label{fig:spaarcworkflow}
                % \vspace{-5mm}
            \end{figure}

% AC: I've added the following. Please see if this makes sense
%Traditional ARM is executed on a static set of historical data, w
%As opposed to traditional use of ARM, 
\spaarc{} uses an ARM algorithm to generate a set of associated objects for prefetching whenever there is a cache miss.
However, %n a cache prefetching scenario, 
all such associated objects are not equally relevant for prefetching. Objects that have been accessed more frequently and more recently are more likely to be more useful for caching. Thus, \spaarc{} prioritizes relevant objects within an association set using an {\em association factor} value for each object. This factor reflects the combined influence of frequency and recency of association with the missing object. Objects with higher association factors are prioritized for caching due to their increased likelihood of subsequent user interaction. %\circled{3}.

We calculate the association factor ($A$) of a virtual object $vo$ as follows:
            \begin{align}
                A(vo)_{new} & = (F(vo) * \alpha) + A(vo)_{old} * (1 - \alpha)
            \end{align}
            where $\alpha = \frac{2}{1 + window}$, $window$ is the window frame of previous object interactions which are relevant, $F(vo)$ is the number of times the virtual object $vo$ is reference in the window frame. The value of $window$ could be tuned according to the recency requirement. A larger window implies longer trend and a smaller one for shorter trend. If the access patterns are not changing frequently, a larger window size would suffice. Association factor threshold value should be set such that low association objects are filtered. The higher the threshold value, higher the chance that the selected objects are accessed in the near future.
%generated by the ARM algorithm.

       

        \subsection{Proximity}
        \label{sssec:prox}
            Since relevant AR objects are based on the field of view of the user, the system only needs to prefetch objects that are in the user's physical vicinity. \spaarc{} incorporates spatial proximity to refine the prefetching list generated by the Association component. Proximity refers to the distance of virtual objects $(Prox(vo))$ from the user's location. For instance, in the grocery store scenario (Figure \ref{fig:motiveapp}), an association rule might suggest prefetching ``milk" alongside ``apples" and ``bananas". However, if the user is not yet in the aisle containing milk, immediate prefetching of the ``milk" virtual object is unnecessary. \spaarc{} prioritizes contextual relevance by employing a {\em lazy prefetching} strategy, deferring the prefetching of ``milk" until the user is in closer proximity. This ensures that prefetched objects are highly relevant to the user's immediate surroundings. A {\em proximity threshold} $(Prox(threshold))$ is used to identify objects within a specific distance from the user. This threshold is domain-specific and depends on the application use case and physical environment. %and may require expert input or a trial-and-error approach for optimal tuning \circled{4}.

            % The \spaarc{} policy leverages spatial proximity to further refine the prefetching list generated by the Association. In this context, proximity refers to the distance of relevant objects ($Prox(vo)$) from the user's location. For example (Figure \ref{fig:motiveapp}), an association rule might suggest prefetching "milk" alongside "apples" and "bananas." However, if user trajectory data indicates "milk" is located in the next aisle, immediate prefetching would be unnecessary. \spaarc{} prioritizes contextual relevance by deferring (lazy prefetching) the prefetching of "milk" until the user physically enters the aisle where it resides. This approach ensures that prefetched objects are highly relevant to the user's immediate surroundings, enhancing the overall user experience. We use a proximity threshold ($Prox(threshold)$) to identify objects within a specific distance of the user. This measure is domain specific and may need expert input ore trial and error approach.

            

             \subsection{SAAP Workflow}
            Figure \ref{fig:spaarcworkflow} illustrates the workflow of the \spaarc{} framework. Upon encountering a cache miss for virtual object $vo$ \circled{1}, \spaarc's Association module utilizes the ARM-generated rules to identify potentially associated objects based on the user's access history and the missing object \circled{2}. To refine this selection and prioritize relevant objects, it uses the association factor for each object. Objects with association factors higher than the Association factor threshold are prioritized for caching due to their increased likelihood of subsequent user interaction \circled{3}.
            Next, the Proximity module filters the selected objects further to those within the Proximity threshold of the user \circled{4}. The filtered out relevant objects are set aside for lazy fetch later when the user moves closer to them \circled{5}.
            Once all the virtual objects to be fetched are identified, the request is send to the cloud \circled{6}. On retrieval, the objects are stored in cache and the missed virtual object is returned. Note that \spaarc{} works in a complementary manner to the cache that continues to use its caching algorithms, e.g., for evicting any objects needed to make space in the cache.

            
    % \vspace{-1.5mm}
    \subsection{Adaptive Tuning}
        In ARM, selecting the minimum support and minimum confidence thresholds is often domain-dependent, requiring expert knowledge. To automate \spaarc, we focus on tuning the minimum support parameter. Minimum support is prioritized as it directly influences the generation of frequent itemsets, which are fundamental to the subsequent steps in the process. Other parameters can be tuned in a similar manner.

        \begin{table}
                \caption{Algorithm \ref{alg:minsuptune} Notations}
                \label{tab:annot}
                \begin{center}
                    % \vspace{-4mm}
                    \begin{tabular}{|P{1cm}|p{7cm}|}\hline
                        Notation & Remark\\
                        \hline\hline
                        $\delta$  & hit rate degradation threshold\\
                        \hline
                        $\beta$  &  minimum support\\
                        \hline
                        $\gamma$  & minimum confidence\\
                        \hline
                        $T$ & transactions\\
                        \hline
                        $\zeta$ & lift\\
                        \hline
                        $D$ & number of evenly spaced minimum support values to be selected in the given range that determines the granularity of the search for the optimal minimum support value\\
                        \hline
                        $\kappa$ & kurtosis, statistical measure of the "tailedness" of the distribution of the lift of generated rules\\
                        \hline
                        $\eta$ & threshold for the ratio of the number of association rules to the number of frequent itemsets that helps to control the number of the generated rules\\
                        \hline
                        $\theta$ & used to determine when the distribution of lift has changed significantly, indicating the generation of large number of rules, many of which may be irrelevant\\
                        \hline
                    \end{tabular}
                \end{center}
                % \vspace{-3mm}
            \end{table}
            
        \noindent \textbf{Minimum support}: 
            Algorithm \ref{alg:minsuptune} outlines the procedure for tuning the minimum support parameter (Notations described in Table \ref{tab:annot}). The key insight is that the quality of the parameter value (and hence the corresponding association rules) is measured by its impact on the cache performance (hit rate).
            
            The TuneMinSup function dynamically adjusts the minimum support based on cache hit rate degradation. It determines whether to generate new association rule sets, select from existing ones, or continue with the current/next set, depending on the degree of hit rate degradation relative to a predefined threshold.

            The GenARules function identifies the bounds for minimum support and generates $N$ new association rule sets, ranging from low to high support values, based on the last $n$ transactions. The value of $n$ can be adjusted to meet the application's association recency requirements.
        
            The SetARules function selects the appropriate rule set to use. Whenever new $N$ rulesets are generated in the increasing order of minimum support, the function selects the middle ruleset. Otherwise, it goes in the increasing or decreasing minimum support direction, depending on the hit rate degradation. This dynamic adjustment allows the system to adapt to changing access patterns and maintain optimal performance.
        
        \begin{algorithm}
        \footnotesize{
            \caption{Minimum support tuning and ruleset generation}
            \label{alg:minsuptune}
            \begin{algorithmic}[1]
                \Procedure{TuneMinSup}{$\delta, \gamma, T$}
                    \State $hrd \gets getDegradation()$
                    \If {$hrd > 2 * \delta$}
                        \State $GenARules(T, \gamma)$
                    \EndIf
                    \State $SetARules()$
                \EndProcedure

                \Procedure{GenARules}{$T, \gamma$}
                    \State $\{\beta_{low}, \beta_{high}\} \gets getMinSupBound(T)$
                    \State $\{\beta_{temp}\} \gets divideMinSupBound(\{\beta_{low}, \beta_{high}\}, D)$
                    \State $\kappa_{prev}, \kappa_{curr} \gets NULL$
                    \State $\{\beta_{newlow}, \beta_{newhigh}\} \gets \{\infty, -\infty\}$
                    \For{$\beta_{t} \in \{\beta_{temp}\}$} \Comment Largest to smallest minsup
                        \State $fqitemsets \gets genFreqItemsets(T, \beta_{t})$
                        \If{$fqitemsets.size > 0$}
                            \State $arules \gets genARules(fqitemsets, \gamma)$
                            \State $arules \gets arules[\zeta \ge 1]$
                            \If $\frac{arules.size}{fqitemsets.size} > \eta$
                                \State $break$
                            \Else
                                \State $\kappa_{curr} \gets kurtosis(arules[\zeta])$
                                \If {$abs(\kappa_{prev} - \kappa_{curr}) > \theta$}
                                    \State $break$
                                \Else
                                    \State $\beta_{newlow} \gets min(\beta_{newlow}, \beta_{t})$
                                    \State $\beta_{newhigh} \gets max(\beta_{newhigh}, \beta_{t})$
                                    \State $\kappa_{prev} \gets \kappa_{curr}$
                                \EndIf
                            \EndIf
                        \EndIf
                    \EndFor

                    \State $ruleset \gets genARuleSets(T, \{\beta_{newlow}, \beta_{newhigh}\}, $ $\gamma, N)$ \Comment Generate N rulesets and store
                    \State $SetARules()$
                \EndProcedure
            \end{algorithmic}
            }
        \end{algorithm}
        
        In the GenARules function, the initial lower and upper bounds for minimum support are first determined. Since popular items typically represent a small percentage of the total unique items in most datasets \cite{bib:spmf}, the lower bound is set to the average support of all items \cite{bib:minsup}.  From this initial range, $D$ evenly spaced minimum support values are selected. For each value, association rules are generated with a fixed minimum confidence. During rule generation, the process terminates if the rule count ratio exceeds a predefined threshold $(\eta)$, indicating the potential inclusion of irrelevant rules. Otherwise, the kurtosis value is recorded. If the difference in kurtosis between the current and previous rule sets exceeds another threshold $(\theta)$, the process terminates, signaling the generation of a large number of rules with relatively low lift values.  This step helps to prevent overfitting and ensures the selection of meaningful rules. If neither termination condition is met, the current minimum support range is updated. Finally, $N$ rule sets are generated from the identified minimum support bounds and utilized accordingly. This adaptive approach allows for efficient and effective rule generation tailored to the specific characteristics of the workload.
    % \input{tex/tuning.tex}
    

\section{Implementation and Evaluation}
\label{sec:evaluation}

We prototype our proposal into a tool \toolName, using approximately 5K lines of OCaml (for the program analysis) and 5K lines of Python code (for the repair). 
In particular, we employ Z3~\cite{DBLP:conf/tacas/MouraB08} as the SMT solver, clingo~\cite{DBLP:books/sp/Lifschitz19} as the ASP solver, and Souffle~\cite{scholz2016fast} as the Datalog engine. %, respectively.
To show the effectiveness, 
we design the experimental evaluation to answer the 
following research questions (RQ):
(Experiments ran on a server with an Intel® Xeon® Platinum 8468V, 504GB RAM, and 192 cores. All the dataset are publicly available from \cite{zenodo_benchmark})

\begin{itemize}[align=left, leftmargin=*,labelindent=0pt]
\item \textbf{RQ1:} How effective is \toolName in verifying CTL properties for relatively small but complex programs, compared to the state-of-the-art tool  \function~\cite{DBLP:conf/sas/UrbanU018}?


\item \textbf{RQ2:} What is the effectiveness of \toolName in detecting real-world bugs, which can be encoded using both CTL and linear temporal logic (LTL), such as non-termination gathered from GitHub \cite{DBLP:conf/sigsoft/ShiXLZCL22} and unresponsive behaviours in protocols  \cite{DBLP:conf/icse/MengDLBR22}, compared with \ultimate~\cite{DBLP:conf/cav/DietschHLP15}?

\item \textbf{RQ3:} How effective is \toolName in repairing CTL violations identified in RQ1 and RQ2? which has not been achieved by any existing tools. 


 

\end{itemize}



% \begin{itemize}[align=left, leftmargin=*,labelindent=0pt]
% \item \textbf{RQ1:} What is the effectiveness of \toolName in verifying CTL properties in a set of relatively small yet challenging programs, compared to the state-of-the-art tools, T2~\cite{DBLP:conf/fmcad/CookKP14},  \function~\cite{DBLP:conf/sas/UrbanU018}, and \ultimate~\cite{DBLP:conf/cav/DietschHLP15}?


% \item \textbf{RQ2:} What is the effectiveness of \toolName in finding  real-world bugs, which can be encoded using CTL properties, such as non-termination 
% gathered from GitHub \cite{DBLP:conf/sigsoft/ShiXLZCL22} and unresponsive behaviours in protocol implementations \cite{DBLP:conf/icse/MengDLBR22}?

% \item \textbf{RQ3:} What is the effectiveness of \toolName in repairing CTL bugs from RQ1--2?

% \end{itemize}

%The benchmark programs are from various sources. More specifically, termination bugs from real-world projects \cite{DBLP:conf/sigsoft/ShiXLZCL22} and CTL analysis \cite{DBLP:conf/fmcad/CookKP14} \cite{DBLP:conf/sas/UrbanU018}, and temporal bugs in real-world protocol implementations \cite{DBLP:conf/icse/MengDLBR22}. 



% \ly{are termination bugs ok? Do we need to add new CTL bugs?}
\subsection{RQ1: Verifying CTL Properties}

% Please add the following required packages to your document preamble:
%  \Xhline{1.5\arrayrulewidth}

\hide{\begin{figure}[!h]
\vspace{-8mm}
\begin{lstlisting}[xleftmargin=0.2em,numbersep=6pt,basicstyle=\footnotesize\ttfamily]
(*@\textcolor{mGray}{//$EF(\m{resp}{\geq}5)$}@*)
int c = *; int resp = 0;
int curr_serv = 5; 
while (curr_serv > 0){ 
 if (*) {  
   c--; 
   curr_serv--;
   resp++;} 
 else if (c<curr_serv){
   curr_serv--; }}
\end{lstlisting} 
\vspace{-2mm}
\caption{A possibly terminating loop} 
\label{fig:terminating_loop}
\vspace{-2mm}
\end{figure}}


%loses precision due to a \emph{dual widening} \cite{DBLP:conf/tacas/CourantU17}, and 

The programs listed in \tabref{tab:comparewithFuntionT2} were obtained from the evaluation benchmark of \function, which includes a total of 83 test cases across over 2,000 lines of code. We categorize these test cases into six groups, labeled according to the types of CTL properties. 
These programs are short but challenging, as they often involve complex loops or require a more precise analysis of the target properties. The \function tends to be conservative, often leading it to return ``unknown" results, resulting in an accuracy rate of 27.7\%. In contrast, \toolName demonstrates advantages with improved accuracy, particularly in \ourToolSmallBenchmark. 
%achieved by the novel loop summaries. 
The failure cases faced by \toolName highlight our limitations when loop guards are not explicitly defined or when LRFs are inadequate to prove termination. 
Although both \function and \toolName struggle to obtain meaningful invariances for infinite loops, the benefits of our loop summaries become more apparent when proving properties related to termination, such as reachability and responsiveness.  




\begin{table}[!t]
\vspace{1.5mm}
\caption{Detecting real-world CTL bugs.}
\normalsize
\label{tab:comparewithCook}
\renewcommand{\arraystretch}{0.95}
\setlength{\tabcolsep}{4pt}  
\begin{tabular}{c|l|c|cc|cc}
\Xhline{1.5\arrayrulewidth}
\multicolumn{1}{l|}{\multirow{2}{*}{\textbf{}}} & \multirow{2}{*}{\textbf{Program}}        & \multirow{2}{*}{\textbf{LoC}} & \multicolumn{2}{c|}{\textbf{\ultimateshort}}   & \multicolumn{2}{c}{\textbf{\toolName}}             \\ \cline{4-7} 
  \multicolumn{1}{l|}{}                           &                                          &                               & \multicolumn{1}{c|}{\textbf{Res.}} & \textbf{Time} & \multicolumn{1}{c|}{\textbf{Res.}} & \textbf{Time} \\ \hline
  1 \xmark                                      & \multirow{2}{*}{\makecell[l]{libvncserver\\(c311535)}}   & 25                            & \multicolumn{1}{c|}{\xmark}      & 2.845         & \multicolumn{1}{c|}{\xmark}      & 0.855         \\  
  1 \cmark                                      &                                          & 27                            & \multicolumn{1}{c|}{\cmark}      & 3.743         & \multicolumn{1}{c|}{\cmark}      & 0.476         \\ \hline
  2 \xmark                                      & \multirow{2}{*}{\makecell[l]{Ffmpeg\\(a6cba06)}}         & 40                            & \multicolumn{1}{c|}{\xmark}      & 15.254        & \multicolumn{1}{c|}{\xmark}      & 0.606         \\  
  2 \cmark                                      &                                          & 44                            & \multicolumn{1}{c|}{\cmark}      & 40.176        & \multicolumn{1}{c|}{\cmark}      & 0.397         \\ \hline
  3 \xmark                                      & \multirow{2}{*}{\makecell[l]{cmus\\(d5396e4)}}           & 87                            & \multicolumn{1}{c|}{\xmark}      & 6.904         & \multicolumn{1}{c|}{\xmark}      & 0.579         \\  
  3 \cmark                                      &                                          & 86                            & \multicolumn{1}{c|}{\cmark}      & 33.572        & \multicolumn{1}{c|}{\cmark}      & 0.986         \\ \hline
  4 \xmark                                      & \multirow{2}{*}{\makecell[l]{e2fsprogs\\(caa6003)}}      & 58                            & \multicolumn{1}{c|}{\xmark}      & 5.952         & \multicolumn{1}{c|}{\xmark}      & 0.923         \\  
  4 \cmark                                      &                                          & 63                            & \multicolumn{1}{c|}{\cmark}      & 4.533         & \multicolumn{1}{c|}{\cmark}      & 0.842         \\ \hline
  5 \xmark                                      & \multirow{2}{*}{\makecell[l]{csound-an...\\(7a611ab)}} & 43                            & \multicolumn{1}{c|}{\xmark}      & 3.654         & \multicolumn{1}{c|}{\xmark}      & 0.782         \\  
  5 \cmark                                      &                                          & 45                            & \multicolumn{1}{c|}{TO}          & -             & \multicolumn{1}{c|}{\cmark}      & 0.648         \\ \hline
  6 \xmark                                      & \multirow{2}{*}{\makecell[l]{fontconfig\\(fa741cd)}}     & 25                            & \multicolumn{1}{c|}{\xmark}      & 3.856         & \multicolumn{1}{c|}{\xmark}      & 0.769         \\  
  6 \cmark                                      &                                          & 25                            & \multicolumn{1}{c|}{Error}       & -             & \multicolumn{1}{c|}{\cmark}      & 0.651         \\ \hline
  7 \xmark                                      & \multirow{2}{*}{\makecell[l]{asterisk\\(3322180)}}       & 22                            & \multicolumn{1}{c|}{\unk}        & 12.687        & \multicolumn{1}{c|}{\unk}        & 0.196         \\  
  7 \cmark                                      &                                          & 25                            & \multicolumn{1}{c|}{\unk}        & 11.325        & \multicolumn{1}{c|}{\unk}        & 0.34          \\ \hline
  8 \xmark                                      & \multirow{2}{*}{\makecell[l]{dpdk\\(cd64eeac)}}          & 45                            & \multicolumn{1}{c|}{\xmark}      & 3.712         & \multicolumn{1}{c|}{\xmark}      & 0.447         \\  
  8 \cmark                                      &                                          & 45                            & \multicolumn{1}{c|}{\cmark}      & 2.97          & \multicolumn{1}{c|}{\unk}        & 0.481         \\ \hline
  9 \xmark                                      & \multirow{2}{*}{\makecell[l]{xorg-server\\(930b9a06)}}   & 19                            & \multicolumn{1}{c|}{\xmark}      & 3.111         & \multicolumn{1}{c|}{\xmark}      & 0.581         \\  
  9 \cmark                                      &                                          & 20                            & \multicolumn{1}{c|}{\cmark}      & 3.101         & \multicolumn{1}{c|}{\cmark}      & 0.409         \\ \hline
  10 \xmark                                      & \multirow{2}{*}{\makecell[l]{pure-ftpd\\(37ad222)}}      & 42                            & \multicolumn{1}{c|}{\cmark}      & 2.555         & \multicolumn{1}{c|}{\xmark}      & 0.933         \\  
  10 \cmark                                      &                                          & 49                            & \multicolumn{1}{c|}{\cmark}        & 2.286         & \multicolumn{1}{c|}{\cmark}      & 0.383         \\ \hline
  11 \xmark  & \multirow{2}{*}{\makecell[l]{live555$_a$\\(181126)}} & 34  & \multicolumn{1}{c|}{\cmark} &  2.715         & \multicolumn{1}{c|}{\xmark}    & 0.513   \\  
  11 \cmark  &     &   37    & \multicolumn{1}{c|}{\cmark} &  2.837         & \multicolumn{1}{c|}{\cmark}      & 0.341 \\ \hline
  12 \xmark  & \multirow{2}{*}{\makecell[l]{openssl\\(b8d2439)}} & 88  & \multicolumn{1}{c|}{\xmark} &  4.15          & \multicolumn{1}{c|}{\xmark}    & 0.78   \\
  12 \cmark  &     &  88     & \multicolumn{1}{c|}{\cmark} &  3.809         & \multicolumn{1}{c|}{\cmark}      & 0.99 \\ \hline
  13 \xmark  & \multirow{2}{*}{\makecell[l]{live555$_b$\\(131205)}} & 83  & \multicolumn{1}{c|}{\xmark} & 2.838         & \multicolumn{1}{c|}{\xmark}    & 0.602     \\  
  13 \cmark  &    &   84     & \multicolumn{1}{c|}{\cmark} &  2.393         & \multicolumn{1}{c|}{\cmark}      & 0.565 \\ \Xhline{1.5\arrayrulewidth}
                                                   & {\bf{Total}}                                  & 1249  & \multicolumn{1}{c|}{\bestBaseLineReal}          & $>$180       & \multicolumn{1}{c|}{\ourToolRealBenchmark}              & 16.01        \\ \Xhline{1.5\arrayrulewidth}
  \end{tabular}
  \end{table}

\subsection{RQ2: CTL Analysis on  Real-world Projects}




Programs in \tabref{tab:comparewithCook} are from real-world repositories, each associated with a Git commit number where developers identify and fix the bug manually. 
In particular, the property used for programs 1-9 (drawn from \cite{DBLP:conf/sigsoft/ShiXLZCL22}) is  \code{AF(Exit())}, capturing non-termination bugs. The properties used for programs 10-13 (drawn from \cite{DBLP:conf/icse/MengDLBR22}) are of the form \code{AG(\phi_1{\rightarrow}AF(\phi_2))}, capturing unresponsive behaviours from the protocol implementation. 
We extracted the main segments of these real-world bugs into smaller programs (under 100 LoC each), preserving features like data structures and pointer arithmetic. Our evaluation includes both buggy (\eg 1\,\xmark) and developer-fixed (\eg 1\,\cmark) versions.
After converting the CTL properties to LTL formulas, we compared our tool with the latest release of UltimateLTL (v0.2.4), a regular participant in SV-COMP \cite{svcomp} with competitive performance. 
Both tools demonstrate high accuracy in bug detection, while \ultimateshort often requires longer processing time. 
This experiment indicates that LRFs can effectively handle commonly seen real-world loops, and \toolName performs a more lightweight summary computation without compromising accuracy. 



%Following the convention in \cite{DBLP:conf/sigsoft/ShiXLZCL22}, t
%Prior works \cite{DBLP:conf/sigsoft/ShiXLZCL22} gathered such examples by extracting 
%\toolName successfully identifies the majority of buggy and correct programs, with the exception of programs 7 and 8. 







{
\begin{table*}[!h]
  \centering
\caption{\label{tab:repair_benchmark}
{Experimental results for repairing CTL bugs. Time spent by the ASP solver is separately recorded. 
}
}
\small
\renewcommand{\arraystretch}{0.95}
  \setlength{\tabcolsep}{9pt}
\begin{tabular}{l|c|c|c|c|c|c|c|c}
  \Xhline{1.5\arrayrulewidth}
  \multicolumn{1}{c|}{\multirow{2}{*}{\textbf{Program}}} & \multicolumn{1}{c|}{\multirow{2}{*}{\shortstack{\textbf{LoC}\\\textbf{(Datalog)}}}} & \multicolumn{3}{c|}{\textbf{Configuration}}                                 & \multicolumn{1}{c|}{\multirow{2}{*}{\textbf{Fixed}}} & \multicolumn{1}{c|}{\multirow{2}{*}{\textbf{\#Patch}}} & \multicolumn{1}{c|}{\multirow{2}{*}{\textbf{ASP(s)}}} & \multirow{2}{*}{\textbf{Total(s)}} \\ \cline{3-5}

  \multicolumn{1}{c|}{}                                  & \multicolumn{1}{c|}{}                              & \multicolumn{1}{c|}{\textbf{Symbols}} & \multicolumn{1}{c|}{\textbf{Facts}} & \multicolumn{1}{c|}{\textbf{Template}} & \multicolumn{1}{c|}{} & \multicolumn{1}{c|}{} & \multicolumn{1}{c|}{}  &                                      \\ \hline

AF\_yEQ5 (\figref{fig:first_Example})                                           & 115                           & 3+0                   & 0+1                & Add                & \cmark     & 1                   & 0.979                              & 1.593                                \\
test\_until.c                                         & 101                            & 0+3                   & 1+0                & Delete                & \cmark     & 1                   & 0.023                              & 0.498                                \\
next.c                                                & 87                            & 0+4                   & 1+0                & Delete                & \cmark     & 1                   & 0.023                              & 0.472                                \\
libvncserver                                          & 118                            & 0+6                   & 1+0                & Delete                & \cmark     & 3                   & 0.049                              & 1.081                                \\
Ffmpeg                                                & 227                           & 0+12                  & 1+0                & Delete                & \cmark     & 4                   & 13.113                              & 13.335                                \\
cmus                                                  & 145                           & 0+12                  & 1+0                & Delete                & \cmark     & 4                   & 0.098                              & 2.052                                \\
e2fsprogs                                             & 109                           & 0+8                   & 1+0                & Delete                & \cmark     & 2                   & 0.075                              & 1.515                                \\
csound-android                                        & 183                           & 0+8                   & 1+0                & Delete                & \cmark     & 4                   & 0.076                              & 1.613                                \\
fontconfig                                            & 190                           & 0+11                  & 1+0                & Delete                & \cmark     & 6                   & 0.098                              & 2.507                                \\
dpdk                                                  & 196                           & 0+12                  & 1+0                & Delete                & \cmark     & 1                   & 0.091                              & 2.006                                \\
xorg-server                                           & 118                            & 0+2                   & 1+0                & Delete                & \cmark     & 2                   & 0.026                              & 0.605                                \\
pure-ftpd                                             & 258                           & 0+21                  & 1+0                & Delete                & \cmark     & 2                   & 0.069                              & 3.590                               \\
live$_a$                                              & 112                            & 3+4                   & 1+1                & Update                & \cmark     & 1                   & 0.552                              & 0.816                                \\
openssl                                               & 315                           & 1+0                   & 0+1                & Add.                & \cmark     & 1                   & 1.188                              & 2.277                                \\
live$_b$                                              & 217                           & 1+0                   & 0+1                & Add                & \cmark     & 1                   & 0.977                              & 1.494                                 \\
  \Xhline{1.5\arrayrulewidth}
\textbf{Total}                                                 & 2491                          &                       &                    &                   &           &                     & 17.437                              & 35.454                               \\ 
  \Xhline{1.5\arrayrulewidth}           
\end{tabular}

\vspace{-2mm}
\end{table*}
}


\subsection{RQ3: Repairing CTL Property Violations} 


\tabref{tab:repair_benchmark} gathers all the program instances (from \tabref{tab:comparewithFuntionT2} and \tabref{tab:comparewithCook}) that violate their specified CTL properties and are sent to \toolName for repair.   
The \textbf{Symbols} column records the number of symbolic constants + symbolic signs, while the \textbf{Facts} column records the number of facts allowed to be removed + added. 
We gradually increase the number of symbols and the maximum number of facts that can be added or deleted. 
The \textbf{Configuration} column shows the first successful configuration that led to finding patches, and we record the total searching time till reaching such configurations. 
We configure \toolName to apply three atomic templates in a breadth-first manner with a depth limit of 1, \ie, \tabref{tab:repair_benchmark} records the patch result after one iteration of the repair. 
The templates are applied sequentially in the order: delete, update, and add. The repair process stops when a correct patch is found or when all three templates have been attempted. 
%without success. 
% Because of this configuration, \toolName only finds one patch for Program 1 (AF\_yEQ5). 
% The patch inserting \plaincode{if (i>10||x==y) \{y=5; return;\}} mentioned in \figref{fig:Patched-program} cannot be found in current configuration, as it requires deleting facts then adding new facts on the updated program.
% The `Configuration' column in \tabref{tab:repair_benchmark} shows the number of symbolic constants and signs, the number of facts allowed to be removed and added, and the template used when a patch is found.

Due to the current configuration, \toolName only finds patch (b) for Program 1 (AF\_yEQ5), while the patch (a) shown in \figref{fig:Patched-program} can be obtained by allowing two iterations of the repair: the first iteration adds the conditional than a second iteration to add a new assignment on the updated program. 
Non-termination bugs are resolved within a single iteration by adding a conditional statement that provides an earlier exit. 
For instance, \figref{fig:term-Patched-program} illustrates the main logic of 1\,\xmark, which enters an infinite loop when \code{\m{linesToRead}{\leq}0}. 
\toolName successfully 
provides a fix that prevents \code{\m{linesToRead}{\leq}0} from occurring before entering the loop. Note that such patches are more desirable which fix the non-termination bug without dropping the loops completely. 
%much like the example shown in  \figref{fig:term-Patched-program}. At the same time, 
Unresponsive bugs involve adding more function calls or assignment modifications. 
%Most repairs were completed within seconds. 

On average, the time taken to solve ASP accounts for 49.2\% (17.437/35.454) of the total repair time. We also keep track of the number of patches that successfully eliminate the CTL violations. More than one patch is available for non-termination bugs, as some patches exit the entire program without entering the loop. 
While all the patches listed are valid, those that intend to cut off the main program logic can be excluded based on the minimum change criteria. 
After a manual inspection of each buggy program shown in \tabref{tab:repair_benchmark}, we confirmed that at least one generated patch is semantically equivalent to the fix provided by the developer. 
As the first tool to achieve automated repair of CTL violations, \toolName successfully resolves all reported bugs. 



\begin{figure}[!t]
\begin{lstlisting}[xleftmargin=6em,numbersep=6pt,basicstyle=\footnotesize\ttfamily]
void main(){ //AF(Exit())
  int lines ToRead = *;
  int h = *;
  (*@\repaircode{if ( linesToRead <= 0 )  return;}@*)
  while(h>0){
    if(linesToRead>h)  
        linesToRead=h; 
    h-=linesToRead;} 
  return;}
\end{lstlisting}
\caption{Fixing a Possible Hang Found in libvncserver \cite{LibVNCClient}}
\label{fig:term-Patched-program}
\end{figure}


    % \section{Discussion}
\label{sec:discuss}
\textbf{Poisoning attacker.}
Assumption 4.2 is the core assumption of this paper, which implicitly assumes that the attacker's dataset $D_i$ is not a poisoning dataset. This assumption is based on the premise that client $i$ aims to maximize their reward and thus will not poison the grand model $\mathcal{A}(\granddataset)$, as doing so would reduce the reward derived from monetizing/utilizing $\mathcal{A}(\granddataset)$.
However, in certain scenarios, client $i$ may pursue dual objectives: both attacking the grand model and conducting data overvaluation. Addressing this dual-objective scenario requires further exploration.


\textbf{Computational efficiency.}
Similar to computing the SV, computing Truth-Shapley is time-consuming, as it requires $O(2^{N+\max_i M_i})$ times of model retraining. 
Since Truth-Shapley utilizes the SV-style approach to define both its client-level data value and block-level data value, existing techniques for accelerating SV computation can be applied to computing these two levels of data value.
Also, designing more efficient acceleration methods specifically for Truth-Shapley is a promising direction for future research.


\textbf{Extension of data overvaluation attack.}
The data overvaluation attack proposed in Definition \ref{def:overvaluation} allows client $i$ to manipulate the utility $v(\datasubset)$ of a data subset $\datasubset \subset \granddataset$ by misreporting client $i$'s data blocks $\reportedstoi$. 
Similarly, client $i$ can achieve the same objective by violating the training algorithm $\mathcal{A}$.
For example, client $i$ can decrease $v(\datasubset)$ by performing a gradient ascent attack during model training.
Truth-Shapley remains resistant to this extension of data overvaluation attack with a slight modification to Assumption 4.2: we assume that, in client $i$’s belief, following algorithm $\mathcal{A}$ maximizes the expected utility for any $\datasubset \subset \granddataset$.
    \putsec{related}{Related Work}

\noindent \textbf{Efficient Radiance Field Rendering.}
%
The introduction of Neural Radiance Fields (NeRF)~\cite{mil:sri20} has
generated significant interest in efficient 3D scene representation and
rendering for radiance fields.
%
Over the past years, there has been a large amount of research aimed at
accelerating NeRFs through algorithmic or software
optimizations~\cite{mul:eva22,fri:yu22,che:fun23,sun:sun22}, and the
development of hardware
accelerators~\cite{lee:cho23,li:li23,son:wen23,mub:kan23,fen:liu24}.
%
The state-of-the-art method, 3D Gaussian splatting~\cite{ker:kop23}, has
further fueled interest in accelerating radiance field
rendering~\cite{rad:ste24,lee:lee24,nie:stu24,lee:rho24,ham:mel24} as it
employs rasterization primitives that can be rendered much faster than NeRFs.
%
However, previous research focused on software graphics rendering on
programmable cores or building dedicated hardware accelerators. In contrast,
\name{} investigates the potential of efficient radiance field rendering while
utilizing fixed-function units in graphics hardware.
%
To our knowledge, this is the first work that assesses the performance
implications of rendering Gaussian-based radiance fields on the hardware
graphics pipeline with software and hardware optimizations.

%%%%%%%%%%%%%%%%%%%%%%%%%%%%%%%%%%%%%%%%%%%%%%%%%%%%%%%%%%%%%%%%%%%%%%%%%%
\myparagraph{Enhancing Graphics Rendering Hardware.}
%
The performance advantage of executing graphics rendering on either
programmable shader cores or fixed-function units varies depending on the
rendering methods and hardware designs.
%
Previous studies have explored the performance implication of graphics hardware
design by developing simulation infrastructures for graphics
workloads~\cite{bar:gon06,gub:aam19,tin:sax23,arn:par13}.
%
Additionally, several studies have aimed to improve the performance of
special-purpose hardware such as ray tracing units in graphics
hardware~\cite{cho:now23,liu:cha21} and proposed hardware accelerators for
graphics applications~\cite{lu:hua17,ram:gri09}.
%
In contrast to these works, which primarily evaluate traditional graphics
workloads, our work focuses on improving the performance of volume rendering
workloads, such as Gaussian splatting, which require blending a huge number of
fragments per pixel.

%%%%%%%%%%%%%%%%%%%%%%%%%%%%%%%%%%%%%%%%%%%%%%%%%%%%%%%%%%%%%%%%%%%%%%%%%%
%
In the context of multi-sample anti-aliasing, prior work proposed reducing the
amount of redundant shading by merging fragments from adjacent triangles in a
mesh at the quad granularity~\cite{fat:bou10}.
%
While both our work and quad-fragment merging (QFM)~\cite{fat:bou10} aim to
reduce operations by merging quads, our proposed technique differs from QFM in
many aspects.
%
Our method aims to blend \emph{overlapping primitives} along the depth
direction and applies to quads from any primitive. In contrast, QFM merges quad
fragments from small (e.g., pixel-sized) triangles that \emph{share} an edge
(i.e., \emph{connected}, \emph{non-overlapping} triangles).
%
As such, QFM is not applicable to the scenes consisting of a number of
unconnected transparent triangles, such as those in 3D Gaussian splatting.
%
In addition, our method computes the \emph{exact} color for each pixel by
offloading blending operations from ROPs to shader units, whereas QFM
\emph{approximates} pixel colors by using the color from one triangle when
multiple triangles are merged into a single quad.


    \section{Conclusions}

We provided deterministic distributed algorithms to efficiently simulate a round of algorithms designed for the CONGEST model on the Beeping Networks. This allowed us to improve polynomially the time complexity of several (also graph) problems on Beeping  Networks. The first simulation by the Local Broadcast algorithm is shorter by a polylogarithmic factor than the other, more general one -- yet still powerful enough to implement some algorithms, including the prominent solution to Network Decomposition~\cite{ghaffari2021improved}.
The more general one could be used for solving problems such as MIS.
We also considered efficient pipelining of messages via several layers of BN.
%We also proved that our solutions could not be substantially improved if the considered problems require content-oblivious local broadcast, by proving an almost-tight lower bound.

Two important lines of research arise from our work.
First, whether some (graph) problems do not need local broadcast to be solved deterministically, and whether their time complexity could be asymptotically below $\Delta^2$.
Second, could a lower bound on any deterministic local broadcast algorithm, better than $\Omega(\Delta\log n)$, be proved?
%our lower bound be tightened and extended to any, not necessarily content-oblivious \mam{and non-adaptive}, solutions to the Local Broadcast problem?

% \todo{Propose to develop algorithms that work in time depending on the diameter of the network}

% \todo{Discussion of noisy beeping channel.}
    % \section*{Acknowledgment}

% \section*{References}

\bibliographystyle{IEEEtran}
\bibliography{IEEEabrv, refer}

\end{document}
