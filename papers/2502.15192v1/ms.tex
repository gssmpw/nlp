\documentclass[conference]{IEEEtran}
\IEEEoverridecommandlockouts
% The preceding line is only needed to identify funding in the first footnote. If that is unneeded, please comment it out.
\usepackage{cite}
\usepackage{amsmath,amssymb,amsfonts}
\usepackage{algorithm}
\usepackage[noend]{algpseudocode}
\usepackage{graphicx}
\usepackage{textcomp}
\usepackage{xcolor}
\usepackage{diagbox}
\usepackage{tikz}
\usepackage{float}
\usepackage{array}
\newcolumntype{P}[1]{>{\centering\arraybackslash}p{#1}}
\newcommand*\circled[1]{\tikz[baseline=(char.base)]{
            \node[shape=circle,draw,inner sep=2pt] (char) {#1};}}
\def\BibTeX{{\rm B\kern-.05em{\sc i\kern-.025em b}\kern-.08em
    T\kern-.1667em\lower.7ex\hbox{E}\kern-.125emX}}
\newcommand{\spaarc}{SAAP}
\begin{document}

    \title{SAAP: Spatial awareness and Association based Prefetching of Virtual Objects in Augmented Reality at the Edge}

    \author{\IEEEauthorblockN{Nikhil Sreekumar, Abhishek Chandra, Jon Weissman}
    \IEEEauthorblockA{\textit{University of Minnesota, Twin Cities} \\
    MN, USA \\
    \{sreek012, chandra, weiss039\}@umn.edu}}
    % \and
    % \IEEEauthorblockN{2\textsuperscript{nd} Given Name Surname}
    % \IEEEauthorblockA{\textit{dept. name of organization (of Aff.)} \\
    % \textit{name of organization (of Aff.)}\\
    % City, Country \\
    % email address or ORCID}
    % \and
    % \IEEEauthorblockN{3\textsuperscript{rd} Given Name Surname}
    % \IEEEauthorblockA{\textit{dept. name of organization (of Aff.)} \\
    % \textit{name of organization (of Aff.)}\\
    % City, Country \\
    % email address or ORCID}
    % \and
    % \IEEEauthorblockN{4\textsuperscript{th} Given Name Surname}
    % \IEEEauthorblockA{\textit{dept. name of organization (of Aff.)} \\
    % \textit{name of organization (of Aff.)}\\
    % City, Country \\
    % email address or ORCID}
    % \and
    % \IEEEauthorblockN{5\textsuperscript{th} Given Name Surname}
    % \IEEEauthorblockA{\textit{dept. name of organization (of Aff.)} \\
    % \textit{name of organization (of Aff.)}\\
    % City, Country \\
    % email address or ORCID}
    % \and
    % \IEEEauthorblockN{6\textsuperscript{th} Given Name Surname}
    % \IEEEauthorblockA{\textit{dept. name of organization (of Aff.)} \\
    % \textit{name of organization (of Aff.)}\\
    % City, Country \\
    % email address or ORCID}
    % }

    \maketitle

    \begin{abstract}
        Mobile Augmented Reality (MAR) applications face performance challenges due to their high computational demands and need for low-latency responses.  Traditional approaches like on-device storage or reactive data fetching from the cloud often result in limited AR experiences or unacceptable lag. Edge caching, which caches AR objects closer to the user, provides a promising solution. However, existing edge caching approaches do not consider AR-specific features such as AR object sizes, user interactions, and physical location. This paper investigates how to further optimize edge caching by employing AR-aware prefetching techniques. We present \spaarc, a Spatial Awareness and Association-based Prefetching policy specifically designed for MAR Caches.  \spaarc{} intelligently prioritizes the caching of virtual objects based on their association with other similar objects and the user's proximity to them. It also considers the recency of associations and uses a lazy fetching strategy to efficiently manage edge resources and maximize Quality of Experience (QoE).

        Through extensive evaluation using both synthetic and real-world workloads, we demonstrate that \spaarc{} significantly improves cache hit rates compared to standard caching algorithms, achieving gains ranging from 3\% to 40\% while reducing the need for on-demand data retrieval from the cloud. Further, we present an adaptive tuning algorithm that automatically tunes \spaarc{} parameters to achieve optimal performance. Our findings demonstrate the potential of \spaarc{} to substantially enhance the user experience in MAR applications by ensuring the timely availability of virtual objects.
    \end{abstract}

    \begin{IEEEkeywords}
    Augmented reality, caching, prefetching, association, proximity, support, confidence, edge computing
    \end{IEEEkeywords}
    \vspace{-3mm}
    \section{Introduction}
\label{sec:intro}

\begin{figure*}[tb]
    \centering
    \includegraphics[width=0.848\linewidth]{figs/circuitnn.pdf} 
    \caption{Illustration of differentiable CircuitNN. CircuitNN is designed based on differentiable NAND gates. After DAS is guided by PI and PO pairs of the truth table, CircuitNN can get the precise circuit architecture logic equivalent to the truth table.}
    \label{fig:circuitnn}
\end{figure*}

% 1. Describe the importance of logic synthesis
% 2. Existing Problems
% (a) Neural Architecture Search: Unstable, Predefined Setting, etc.
% (b) Circuit Generation: Probabilistic Model, Logic Equivalence

With the rapid advancement of technology, the scale of integrated circuits (ICs) has expanded exponentially. 
This expansion has introduced significant challenges in chip manufacturing, particularly concerning power and area metrics.
A primary objective in IC design is achieving the same circuit function with fewer transistors, thereby reducing power usage and area occupancy.

Logic synthesis~\cite{hachtel2005logicsynth}, a critical step in electronic design automation (EDA), transforms behavioral-level circuit designs into optimized gate-level circuits, ultimately yielding the final IC layout. 
The primary goal of logic synthesis is to identify the physical implementation with the fewest gates for a given circuit function. 
This task constitutes a challenging NP-hard combinatorial optimization problem. 
Current logic synthesis tools~\cite{brayton2010abc, wolf2013yosys} rely on human-designed heuristics, often leading to sub-optimal outcomes.

Differentiable architecture search (DAS) techniques~\cite{liu2018darts, chu2020darts} offer novel perspectives on addressing challenges in this problem.
Circuit functions can be represented through truth tables, which map binary inputs to their corresponding outputs. 
Truth tables provide a precise representation of input-output relationships, ensuring the design of functionally equivalent circuits.
Inspired by this, researchers~\cite{deepmind2024ai4sys, wang2024tnet} have begun exploring the application of DAS to synthesize circuits directly from truth tables.
Specifically, \citet{deepmind2024ai4sys} proposed CircuitNN, a framework that learns differentiable connection structures with logic gates, enabling the automatic generation of logic circuits from truth tables.
This approach significantly reduces the complexity of traditional circuit generation. 
Building on this, \citet{wang2024tnet} introduced T-Net, a triangle-shaped variant of CircuitNN, incorporating regularization techniques to enhance the efficiency of DAS.

Despite these advancements, several challenges remain. 
The computational complexity of DAS grows quadratically with the number of gates, posing scalability issues.
Although triangle-shaped architecture~\cite{wang2024tnet} partially mitigates this problem, redundancy persists. 
%Additionally, DAS is susceptible to converging to local optima, limiting the ability to search architectures that satisfy the given truth tables~\cite{liu2018darts}. 
%Furthermore, hyperparameters (network depth and layer width) require extensive searches, introducing complexity and prolonging the synthesis process. 
Additionally, DAS is susceptible to converging to local optima~\cite{liu2018darts} and hyperparameters (network depth and layer width) require extensive searches. 
The challenges arise from the vast search space in DAS. 
% Even with predefined settings for CircuitNN, finding a configuration that meets the truth table requires extensive trial and error during the DAS process. 
Intuitively, limiting the search space through predefined parameters (network depth, gates per layer, and connection probabilities) can significantly reduce the complexity.

Recent advances~\cite{openai2023gpt4, abramson2024alphafold3, esser2024sd3, li2024mar} in conditional generative models have demonstrated remarkable performance across language, vision, and graph generation tasks. 
Motivated by these developments, we propose a novel approach to circuit generation that generates preliminary circuit structures to guide DAS in generating refined circuits matching specified truth tables. 
Firstly, we introduce CircuitVQ, a tokenizer with a discrete codebook for circuit tokenization. 
Built upon our Circuit AutoEncoder framework~\cite{hou2022graphmae,li2023maskgae,wu2025mgvga}, CircuitVQ is trained through a circuit reconstruction task. 
Specifically, the CircuitVQ encoder encodes input circuits into discrete tokens using a learnable codebook, while the decoder reconstructs the circuit adjacency matrix based on these tokens.
Subsequently, the CircuitVQ encoder serves as a circuit tokenizer for CircuitAR pretraining, which employs a masked autoregressive modeling paradigm~\cite{chang2022maskgit, li2023mage}. 
In this process, the discrete codes function as supervision signals. 
After training, CircuitAR can generate discrete tokens progressively, which can be decoded into initial circuit structures by the decoder of the CircuitVQ. 
These prior insights can guide DAS in producing refined circuits that match the target truth tables precisely.

Our key contributions can be summarized as follows:
\begin{itemize}
\item We introduce CircuitVQ, a circuit tokenizer that facilitates graph autoregressive modeling for circuit generation, based on our Circuit AutoEncoder framework;
\item Develop CircuitAR, a model trained using masked autoregressive modeling, which generates initial circuit structures conditioned on given truth tables;
\item Propose a refinement framework that integrates differentiable architecture search to produce functionally equivalent circuits guided by target truth tables;
\item Comprehensive experiments demonstrating the scalability and capability emergence of our CircuitAR and the superior performance of the proposed circuit generation approach.
\end{itemize}

% Motivation
% (a) Diffusion (Vision, Graph), Autoregressive (Language, Vision)
% (b) Circuit Generation for Predefined Setting
% (c) Neural Architecture Search for Strict Logic Equivalence

% Contribution
% (a) Circuit Tokenizer (new transformer arch, training strategy)
% (b) CircuitAR (train and gen strategies, post-ar strategy)
% (c) Extensive Evaluation including BitD (Bit Distance) for Scalability

    \section{Motivation and Background}
\label{sec:motive}
    Achieving an immersive mobile augmented reality (MAR) experience presents unique challenges \cite{bib:networkar,bib:locar}. Conventional data caching and prefetching policies, designed for general-purpose applications, might not be optimal for MAR due to its inherent characteristics\cite{bib:archaract}. Compared to other edge-native applications \cite{bib:edgenative}, MAR applications require significant computational resources and sophisticated data management strategies to handle the large volume of spatial data and real-time interactions. 

    \subsection{Motivating AR Scenarios}
    \begin{figure}
        \centering
        \includegraphics[scale=0.13]{images/motive/motiveapp.png}
        \caption{Grocery and tourist scenarios}
        \label{fig:motiveapp}
        \vspace{-6mm}
    \end{figure}
    Figure \ref{fig:motiveapp} shows two example application scenarios for AR. In a grocery store scenario\cite{bib:phara}, users interact with virtual objects of the store items to receive information about their ingredients, manufacturing details, price variations, recipes, and so on. The items are typically placed on racks in aisles based on their categories, and there is a well defined order of placement. The Field of View (FoV) of a user will have multiple items at their location in the store, from which the user might be interested in a few. 
    
    In the case of a tourist scenario, \cite{bib:dublin}, attractions are typically scattered across a region (such as a city block, campus, or neighborhood). The virtual objects may contain historical facts, 3D representations of buildings, informative audio/video, and so on. A tourist would visit the attractions based on multiple factors like user interests, transportation availability, accessibility, guide recommendations, and so on. The FoV of the user in this case would likely consist of only a few items in their vinicity, and the user will have to explore the region more to find the next attraction. 

    %\subsection{MAR challenges}
    %There are multiple MAR specific challenges when 
    %Serving users in 
    Such MAR scenarios face the following challenges:
    
    \noindent$\bullet$ {\em Limited device storage and large object sizes}:  While current virtual objects are typically under 20MB \cite{bib:carsar, bib:objaverse}, increasing interaction complexity and 3D objects will lead to larger object sizes.  Storing all objects locally on the device  becomes impractical as object size and complexity grow.
    
    \noindent$\bullet$ {\em High cloud latency and strict AR latency requirements}: Average cloud latency (around 60ms \cite{bib:clatency, bib:clt1, bib:clt2}) exceeds the sub-20ms threshold required for immersive AR experience \cite{bib:20ms}.  On-demand fetching of objects from the cloud introduces significant latency \cite{bib:delayhits}, negatively impacting user experience.  
    %Anticipating user needs and preemptively caching data at the edge enables rapid availability and efficient sharing among users.
    %Achieving such low latency necessitates the use of 
    %a combined approach utilizing 5G/6G networks and 
    %edge infrastructure.
    
    \if 0
        \noindent$\bullet$ On-demand fetch inefficiency: On-demand fetching of objects from the cloud introduces significant latency \cite{bib:delayhits}, negatively impacting user experience.  Anticipating user needs and preemptively caching data at the edge enables rapid availability and efficient sharing among users.

    
    \noindent 4. Power Consumption and Data Transfer:  On-demand fetching leads to inefficient bandwidth usage, increased data transfer latency, and higher power consumption \cite{bib:energy}.  Intelligent prefetching methods are crucial to proactively store virtual objects, minimize power consumption and improve efficiency.
    \fi

    % The key challenges in mobile AR caching for application scenarios discussed above are:
    
    % \noindent 1. Limited Device Storage and High Object Sizes:  While most current virtual objects are under 20MB \cite{bib:carsar, bib:objaverse}, as interaction complexity increases, the size will also increase. Storing all objects locally or fetching them on-demand from the cloud becomes impractical as object sizes and complexity grow.

    % \noindent 2. High Cloud Latency and Strict Latency Requirements:  Average cloud latency hovers around 60ms \cite{bib:clatency, bib:clt1, bib:clt2}, exceeding the sub-20ms latency threshold required for an immersive AR experience \cite{bib:20ms}. Only a combined approach utilizing 5G/6G networks and edge infrastructure can potentially achieve this low latency target.

    % \noindent 3. On-Demand Fetch Inefficiency:  The on-demand fetching of objects from the cloud can introduce significant latency, negatively impacting user experience. By anticipating user needs and preemptively storing data at the edge, we can ensure rapid availability and facilitate efficient sharing among users with similar interests.
    
    % \noindent 4. Power Consumption and Data Transfer:The inefficiencies inherent to on-demand fetching can result in missed opportunities to reduce bandwidth usage, subsequently leading to elevated data transfer latencies and increased power consumption \cite{bib:archaract}. To mitigate these effects, intelligent prefetching methods are required to proactively store virtual objects anticipated to be accessed in the near future, thereby minimizing power consumption.

\subsection{Edge caching}
    Edge infrastructure is well placed to meet these challenges by expanding on the limited on-device storage capacity, while reducing the user latency. The use of the edge in an MAR context necessitates novel data management policies that consider both AR-specific data characteristics and edge/cloud parameters. 
    %Achieving such low latency necessitates while meeting the storage requirements of AR objects the use of 
    %a combined approach utilizing 5G/6G networks and 
    %edge infrastructure.
    
    Recent research on MAR architecture and applications \cite{bib:edgearch, bib:carsar, bib:sear,bib:localcache,bib:slamshare} partitions data and tasks across mobile devices and edge servers to leverage resources at both locations and facilitate data sharing. Caching is crucial for improving AR application performance, and partitioning the cache across devices and edge servers has shown potential for increasing hit rates and reducing latency~\cite{bib:carsar, bib:sear}. 
    
    However, existing approaches often neglect critical AR-specific factors: (1) \textit{Increasing Object Sizes}: Caching strategies must adapt to the growing size and complexity of virtual objects. (2) \textit{Occlusion and User Proximity}: Virtual objects can be occluded by physical objects in AR environments. Additionally, users are more likely to interact with nearby objects. These spatial considerations, along with object access patterns, can inform more effective caching strategies. (3) \textit{Object Relevance and User Associations}: User interactions with virtual objects often exhibit predictable patterns (e.g., a user interacting with a milk carton may subsequently interact with nearby eggs or bread). However, caching strategies must assess the relevance of potential interactions to avoid unnecessary cache pollution.

    % These challenges necessitate novel data management policies that consider both AR-specific data characteristics and edge cloud parameters. Recent research on mobile AR architecture and applications \cite{bib:edgearch, bib:carsar, bib:sear,bib:localcache,bib:slamshare}, partitions data and tasks across mobile devices and edge servers to leverage resources at both locations and facilitate data sharing. Caching is a crucial aspect for improving AR application performance, and partitioning cache across local devices and edge servers has shown promise in increasing hit rates and reducing response latency \cite{bib:carsar, bib:sear}.  However, existing approaches often overlook AR-specific factors such as (1) \textit{Increasing Object Sizes}: Caching strategies need to adapt to the growing size and complexity of virtual objects. (2) \textit{Occlusion and User Proximity}: In AR environments, virtual objects may be obscured by physical objects positioned in front of them (occlusion). Furthermore, users are significantly more likely to interact with objects within their immediate vicinity compared to those farther away. These spatial considerations, in conjunction with object access patterns, can inform more effective caching strategies. (3) \textit{Object Relevance and User Associations}: A user interacting with a virtual milk carton in a grocery store scenario may be more likely to subsequently interact with nearby eggs, bread, or other related items. However, to avoid unnecessary cache pollution, the relevance of these potential interactions should be assessed before preemptively caching associated objects.
    
   % This paper proposes \spaarc, a prefetching policy that addresses these limitations by incorporating both the proximity of virtual objects to users and object associations to make informed decisions on edge server caching.
    %\vspace{-1.5mm}
    \subsubsection*{The need for Prefetch}
        %The increasing use of mobile devices to interact with large virtual objects presents challenges due to limited device storage. 
        Edge caching offers a promising solution by storing frequently accessed objects closer to the user, reducing cloud traffic and improving access times. However, efficient cache management is crucial given the limited and heterogeneous nature of edge resources.  Traditional cache eviction algorithms (e.g., LRU, LFU, FIFO, etc.) can be employed on the edge cache, but relying solely on reactive retrieval from the cloud upon cache misses can still introduce significant latency. Predictive prefetching based on access patterns can anticipate future requests and proactively cache necessary objects at the edge, reducing cache misses and enhancing user experience at the potential expense of increased network usage.
        
    \vspace{-1.5mm}
    \subsection{Associations and Spatial Proximity}
        Association rule mining (ARM) \cite{bib:arm}, a technique widely used in recommender systems to predict user-item interactions, offers a promising approach for AR prefetching. ARM has proven successful in various domains, including e-commerce \cite{bib:phara}, tourism \cite{bib:dublin}, fraud detection \cite{bib:fraud}, and social network analysis\cite{bib:sna}. For example, in online grocery shopping, purchasing milk and eggs often triggers recommendations for bread and jam. However, the direct application of ARM for prefetching in AR remains an underexplored area.  The rule generation process, particularly with frequent updates, can be computationally expensive, posing a challenge for latency-sensitive AR applications. This work explores adapting ARM to the specific needs of AR, aiming to achieve efficient prefetching while minimizing computational overhead. While ARM effectively identifies related virtual objects, it may not prioritize those most relevant to the user's current location. Prefetching objects far from the user's field of view (FoV) provides limited value in AR. Therefore, this work emphasizes incorporating spatial awareness into the prefetching process. By prioritizing objects near the user's FoV, we aim to optimize cache utilization and enhance the user experience.
        
        % Association rule mining (ARM) \cite{bib:arm}, a well-established technique in recommender systems for predicting user interactions with items, offers a promising approach. ARM has demonstrated success in various domains, including e-commerce \cite{bib:phara}, tourism \cite{bib:dublin}, fraud detection \cite{bib:fraud}, and social network analysis\cite{bib:sna}. For instance, in online grocery shopping, purchasing milk and eggs often leads to recommendations for bread and jam. However, directly applying ARM for prefetching/caching in AR applications is still an area to be explored. The rule generation process can be computationally expensive, especially when frequent updates are required. This becomes problematic for latency-sensitive applications like AR, where rapid response times are paramount. Therefore, this work explores methods to adapt ARM specifically for the needs of AR applications, aiming to achieve efficient prefetching while maintaining low computational overhead.While ARM can effectively identify related virtual objects, it may not prioritize those most relevant to the user's current location. Prefetching objects situated far from the user's field of view (FoV) would offer minimal value in an AR experience.  Therefore, this work emphasizes the importance of incorporating spatial awareness into the prefetching process. By prioritizing objects in close proximity to the user's FoV, we aim to optimize cache utilization and enhance the overall user experience.
    % \vspace{-1.5mm}
    % \subsection{Association rule mining}
    % \label{ssec:arm}

    %     Association rule mining (ARM) \cite{bib:armorig, bib:arm} identifies relationships between items in a transaction database by discovering frequent itemsets—groups of items that frequently co-occur in transactions. The support $(S)$ of an itemset indicates the proportion of transactions containing that itemset, with frequent itemsets exceeding a predefined minimum support threshold. Based on these frequent itemsets, ARM generates association rules of the form  $A \implies B$, where $A$ (antecedent) and $B$ (consequent) are disjoint itemsets. The confidence $(C)$ of a rule represents the conditional probability of observing the consequent in transactions containing the antecedent, controlled by a minimum confidence threshold.  The lift $(L)$ of a rule measures the strength of the association between the antecedent and consequent, quantifying how much more likely they are to co-occur than if they were statistically independent.
        
        % Association rule mining \cite{bib:armorig, bib:arm} discovers relationships between items in a transaction database by identifying frequent itemsets. An itemset represents a group of items that frequently co-occur in transactions. The support of an itemset signifies the proportion of transactions containing the itemset. Frequent itemsets are those that exceed a minimum support threshold. Based on these frequent itemsets, ARM generates a set of association rules expressed as $A \implies C$, where $A$ and $C$ are disjoint itemsets referred to as antecedents and consequents, respectively. The confidence of a rule reflects the conditional probability of encountering the consequent within transactions containing the antecedent.  A minimum confidence threshold allows control over the quality of generated rules. The lift of a rule measures how much more likely two itemsets are to occur together than if they were statistically independent. In simpler terms, it quantifies the strength of the association between the itemsets.
        % Support of an itemset $X$ ($S(X)$), confidence of a rule $A \implies B$ ($C(A \implies B)$) and lift of the rule ($L(A \implies B)$) are formally represented as follows:
        % \begin{align}
        %     S(X) & = \frac{T(X)}{T} \\
        %     C(A \implies B) & = \frac{S(A \implies B)}{S(A)}\\
        %     L(A \implies B) &= \frac{S(A \cup B)}{S(A) * S(B)}
        % \end{align}
        % where $T(X)$ and $T$ are the number of transactions containing $X$ and the total number of transactions respectively.
        % It is important to emphasize that minimum support and minimum confidence are employed as thresholds rather than fixed values in ARM. Increasing these thresholds reduces the number of generated association rules.  In extreme cases, where the desired rule quality cannot be achieved with the specified thresholds, no rules may be generated, as will be further explored in the evaluation section (Section \ref{sec:eval}).
    \section{\spaarc}
\label{sec:spaar}
We propose \spaarc, a prefetching framework that addresses the limitations of existing edge caching approaches for MAR applications. It incorporates both the proximity of virtual objects to users and object associations to make informed decisions on edge server caching. Its prefetching policy is complementary to the underlying caching algorithm employed by the edge cache, and is meant to enhance the cache performance.
%We first present the \spaarc{} architecture, followed by a discussion of its main techniques: association and proximity.

%    In this section, we discuss \spaarc{}, the spatial proximity and association based prefetching for augmented reality cache.
    % \vspace{-3mm}
    \subsection{System Architecture}
        Figure \ref{fig:marspaarc} illustrates the MAR pipeline incorporating \spaarc{} into the edge cache.  Sensor data from the user's device is transmitted to the nearest edge server, where video frames are pre-processed and object detection is performed.  Extracted features are then used for object recognition. If a corresponding virtual object exists, the system attempts to retrieve it from the edge cache.  Upon a cache miss, the \spaarc{} algorithm is invoked to identify all necessary objects for prefetching and request them from the cloud, along with the object that caused the cache miss. After receiving the virtual objects, template matching is performed to estimate their pose. This pose information, along with the virtual objects, is sent to the object tracking module on the user's device.  The tracking module identifies objects across frames, and this information, combined with data from the edge server, is used by the rendering module to overlay virtual content onto the user's display.
        
        % The MAR Pipeline for with \spaarc{} is shown in Figure \ref{fig:marspaarc}. Sensor data is transmitted from the user's device to the nearest edge server. The edge server pre-processes the video frames and applies object detection algorithms. Subsequently, features are extracted and utilized by object recognition algorithms to determine the most accurate item description. If a virtual object is linked to the identified item, the system attempts to retrieve it from the cache. In case of a cache miss, the \spaarc{} algorithm is invoked. This algorithm identifies all necessary objects, including the missed object, and sends a request to the cloud. Upon receiving the virtual objects, template matching is performed to estimate their pose. This pose information, along with the virtual objects, is transmitted to the object tracking module on the user's device. The tracking module identifies objects across frames, and this information, combined with data from the edge server, is used by the rendering module to overlay virtual content onto the user's display.

        \begin{figure}[t]
            \centering
            \includegraphics[scale=0.1]{images/spaar/marspaarc.png}
            \caption{MAR Pipeline with \spaarc{} on the edge cache}
            \label{fig:marspaarc}
            \vspace{-5mm}
        \end{figure}

        %\spaarc{} runs on top of existing cache eviction policies and 
        We next discuss the two main \spaarc{} techniques that drive its prefetching:  association and proximity.
        
        \subsection{Association}
        \label{ssec:spaarc_assoc}
        Intuitively, if a user accesses a virtual object in an MAR scenario, they are likely to soon access related objects (e.g., milk and eggs in the grocery scenario), which should be prefetched to the edge cache for faster access. \spaarc's Association technique generates association rules using {\em Association rule mining (ARM)} \cite{bib:armorig, bib:arm} to predict frequently co-accessed virtual objects based on user interaction history. 
        
        ARM identifies relationships between items in a transaction database by discovering {\em frequent itemsets}: groups of items that frequently co-occur in transactions. The {\em support} $(S)$ of an itemset indicates the proportion of transactions containing that itemset, with frequent itemsets exceeding a predefined minimum support threshold. Based on these frequent itemsets, ARM generates {\em association rules} of the form  $A \implies B$, where $A$ (antecedent) and $B$ (consequent) are disjoint itemsets. The {\em confidence} $(C)$ of a rule represents the conditional probability of observing the consequent in transactions containing the antecedent, controlled by a minimum confidence threshold.  The {\em lift} $(L)$ of a rule measures the strength of the association between the antecedent and consequent, quantifying how much more likely they are to co-occur than if they were statistically independent. Support of an itemset $X$ ($S(X)$), confidence of a rule $A \implies B$ ($C(A \implies B)$) and lift of the rule ($L(A \implies B)$) are formally represented as follows:
            \begin{align}
                S(X) & = \frac{T(X)}{T} \\
                C(A \implies B) & = \frac{S(A \implies B)}{S(A)}\\
                L(A \implies B) &= \frac{S(A \cup B)}{S(A) * S(B)}
            \end{align}
            where $T(X)$ and $T$ are the number of transactions containing $X$ and the total number of transactions respectively. We use ARM algorithms to either generate rules a priori from history user interactions or dynamically during the process.

            \begin{figure}
                \centering
                \includegraphics[scale=0.11]{images/spaar/spaarcflowchart.png}
                \caption{\spaarc{} workflow}
                \label{fig:spaarcworkflow}
                % \vspace{-5mm}
            \end{figure}

% AC: I've added the following. Please see if this makes sense
%Traditional ARM is executed on a static set of historical data, w
%As opposed to traditional use of ARM, 
\spaarc{} uses an ARM algorithm to generate a set of associated objects for prefetching whenever there is a cache miss.
However, %n a cache prefetching scenario, 
all such associated objects are not equally relevant for prefetching. Objects that have been accessed more frequently and more recently are more likely to be more useful for caching. Thus, \spaarc{} prioritizes relevant objects within an association set using an {\em association factor} value for each object. This factor reflects the combined influence of frequency and recency of association with the missing object. Objects with higher association factors are prioritized for caching due to their increased likelihood of subsequent user interaction. %\circled{3}.

We calculate the association factor ($A$) of a virtual object $vo$ as follows:
            \begin{align}
                A(vo)_{new} & = (F(vo) * \alpha) + A(vo)_{old} * (1 - \alpha)
            \end{align}
            where $\alpha = \frac{2}{1 + window}$, $window$ is the window frame of previous object interactions which are relevant, $F(vo)$ is the number of times the virtual object $vo$ is reference in the window frame. The value of $window$ could be tuned according to the recency requirement. A larger window implies longer trend and a smaller one for shorter trend. If the access patterns are not changing frequently, a larger window size would suffice. Association factor threshold value should be set such that low association objects are filtered. The higher the threshold value, higher the chance that the selected objects are accessed in the near future.
%generated by the ARM algorithm.

       

        \subsection{Proximity}
        \label{sssec:prox}
            Since relevant AR objects are based on the field of view of the user, the system only needs to prefetch objects that are in the user's physical vicinity. \spaarc{} incorporates spatial proximity to refine the prefetching list generated by the Association component. Proximity refers to the distance of virtual objects $(Prox(vo))$ from the user's location. For instance, in the grocery store scenario (Figure \ref{fig:motiveapp}), an association rule might suggest prefetching ``milk" alongside ``apples" and ``bananas". However, if the user is not yet in the aisle containing milk, immediate prefetching of the ``milk" virtual object is unnecessary. \spaarc{} prioritizes contextual relevance by employing a {\em lazy prefetching} strategy, deferring the prefetching of ``milk" until the user is in closer proximity. This ensures that prefetched objects are highly relevant to the user's immediate surroundings. A {\em proximity threshold} $(Prox(threshold))$ is used to identify objects within a specific distance from the user. This threshold is domain-specific and depends on the application use case and physical environment. %and may require expert input or a trial-and-error approach for optimal tuning \circled{4}.

            % The \spaarc{} policy leverages spatial proximity to further refine the prefetching list generated by the Association. In this context, proximity refers to the distance of relevant objects ($Prox(vo)$) from the user's location. For example (Figure \ref{fig:motiveapp}), an association rule might suggest prefetching "milk" alongside "apples" and "bananas." However, if user trajectory data indicates "milk" is located in the next aisle, immediate prefetching would be unnecessary. \spaarc{} prioritizes contextual relevance by deferring (lazy prefetching) the prefetching of "milk" until the user physically enters the aisle where it resides. This approach ensures that prefetched objects are highly relevant to the user's immediate surroundings, enhancing the overall user experience. We use a proximity threshold ($Prox(threshold)$) to identify objects within a specific distance of the user. This measure is domain specific and may need expert input ore trial and error approach.

            

             \subsection{SAAP Workflow}
            Figure \ref{fig:spaarcworkflow} illustrates the workflow of the \spaarc{} framework. Upon encountering a cache miss for virtual object $vo$ \circled{1}, \spaarc's Association module utilizes the ARM-generated rules to identify potentially associated objects based on the user's access history and the missing object \circled{2}. To refine this selection and prioritize relevant objects, it uses the association factor for each object. Objects with association factors higher than the Association factor threshold are prioritized for caching due to their increased likelihood of subsequent user interaction \circled{3}.
            Next, the Proximity module filters the selected objects further to those within the Proximity threshold of the user \circled{4}. The filtered out relevant objects are set aside for lazy fetch later when the user moves closer to them \circled{5}.
            Once all the virtual objects to be fetched are identified, the request is send to the cloud \circled{6}. On retrieval, the objects are stored in cache and the missed virtual object is returned. Note that \spaarc{} works in a complementary manner to the cache that continues to use its caching algorithms, e.g., for evicting any objects needed to make space in the cache.

            
    % \vspace{-1.5mm}
    \subsection{Adaptive Tuning}
        In ARM, selecting the minimum support and minimum confidence thresholds is often domain-dependent, requiring expert knowledge. To automate \spaarc, we focus on tuning the minimum support parameter. Minimum support is prioritized as it directly influences the generation of frequent itemsets, which are fundamental to the subsequent steps in the process. Other parameters can be tuned in a similar manner.

        \begin{table}
                \caption{Algorithm \ref{alg:minsuptune} Notations}
                \label{tab:annot}
                \begin{center}
                    % \vspace{-4mm}
                    \begin{tabular}{|P{1cm}|p{7cm}|}\hline
                        Notation & Remark\\
                        \hline\hline
                        $\delta$  & hit rate degradation threshold\\
                        \hline
                        $\beta$  &  minimum support\\
                        \hline
                        $\gamma$  & minimum confidence\\
                        \hline
                        $T$ & transactions\\
                        \hline
                        $\zeta$ & lift\\
                        \hline
                        $D$ & number of evenly spaced minimum support values to be selected in the given range that determines the granularity of the search for the optimal minimum support value\\
                        \hline
                        $\kappa$ & kurtosis, statistical measure of the "tailedness" of the distribution of the lift of generated rules\\
                        \hline
                        $\eta$ & threshold for the ratio of the number of association rules to the number of frequent itemsets that helps to control the number of the generated rules\\
                        \hline
                        $\theta$ & used to determine when the distribution of lift has changed significantly, indicating the generation of large number of rules, many of which may be irrelevant\\
                        \hline
                    \end{tabular}
                \end{center}
                % \vspace{-3mm}
            \end{table}
            
        \noindent \textbf{Minimum support}: 
            Algorithm \ref{alg:minsuptune} outlines the procedure for tuning the minimum support parameter (Notations described in Table \ref{tab:annot}). The key insight is that the quality of the parameter value (and hence the corresponding association rules) is measured by its impact on the cache performance (hit rate).
            
            The TuneMinSup function dynamically adjusts the minimum support based on cache hit rate degradation. It determines whether to generate new association rule sets, select from existing ones, or continue with the current/next set, depending on the degree of hit rate degradation relative to a predefined threshold.

            The GenARules function identifies the bounds for minimum support and generates $N$ new association rule sets, ranging from low to high support values, based on the last $n$ transactions. The value of $n$ can be adjusted to meet the application's association recency requirements.
        
            The SetARules function selects the appropriate rule set to use. Whenever new $N$ rulesets are generated in the increasing order of minimum support, the function selects the middle ruleset. Otherwise, it goes in the increasing or decreasing minimum support direction, depending on the hit rate degradation. This dynamic adjustment allows the system to adapt to changing access patterns and maintain optimal performance.
        
        \begin{algorithm}
        \footnotesize{
            \caption{Minimum support tuning and ruleset generation}
            \label{alg:minsuptune}
            \begin{algorithmic}[1]
                \Procedure{TuneMinSup}{$\delta, \gamma, T$}
                    \State $hrd \gets getDegradation()$
                    \If {$hrd > 2 * \delta$}
                        \State $GenARules(T, \gamma)$
                    \EndIf
                    \State $SetARules()$
                \EndProcedure

                \Procedure{GenARules}{$T, \gamma$}
                    \State $\{\beta_{low}, \beta_{high}\} \gets getMinSupBound(T)$
                    \State $\{\beta_{temp}\} \gets divideMinSupBound(\{\beta_{low}, \beta_{high}\}, D)$
                    \State $\kappa_{prev}, \kappa_{curr} \gets NULL$
                    \State $\{\beta_{newlow}, \beta_{newhigh}\} \gets \{\infty, -\infty\}$
                    \For{$\beta_{t} \in \{\beta_{temp}\}$} \Comment Largest to smallest minsup
                        \State $fqitemsets \gets genFreqItemsets(T, \beta_{t})$
                        \If{$fqitemsets.size > 0$}
                            \State $arules \gets genARules(fqitemsets, \gamma)$
                            \State $arules \gets arules[\zeta \ge 1]$
                            \If $\frac{arules.size}{fqitemsets.size} > \eta$
                                \State $break$
                            \Else
                                \State $\kappa_{curr} \gets kurtosis(arules[\zeta])$
                                \If {$abs(\kappa_{prev} - \kappa_{curr}) > \theta$}
                                    \State $break$
                                \Else
                                    \State $\beta_{newlow} \gets min(\beta_{newlow}, \beta_{t})$
                                    \State $\beta_{newhigh} \gets max(\beta_{newhigh}, \beta_{t})$
                                    \State $\kappa_{prev} \gets \kappa_{curr}$
                                \EndIf
                            \EndIf
                        \EndIf
                    \EndFor

                    \State $ruleset \gets genARuleSets(T, \{\beta_{newlow}, \beta_{newhigh}\}, $ $\gamma, N)$ \Comment Generate N rulesets and store
                    \State $SetARules()$
                \EndProcedure
            \end{algorithmic}
            }
        \end{algorithm}
        
        In the GenARules function, the initial lower and upper bounds for minimum support are first determined. Since popular items typically represent a small percentage of the total unique items in most datasets \cite{bib:spmf}, the lower bound is set to the average support of all items \cite{bib:minsup}.  From this initial range, $D$ evenly spaced minimum support values are selected. For each value, association rules are generated with a fixed minimum confidence. During rule generation, the process terminates if the rule count ratio exceeds a predefined threshold $(\eta)$, indicating the potential inclusion of irrelevant rules. Otherwise, the kurtosis value is recorded. If the difference in kurtosis between the current and previous rule sets exceeds another threshold $(\theta)$, the process terminates, signaling the generation of a large number of rules with relatively low lift values.  This step helps to prevent overfitting and ensures the selection of meaningful rules. If neither termination condition is met, the current minimum support range is updated. Finally, $N$ rule sets are generated from the identified minimum support bounds and utilized accordingly. This adaptive approach allows for efficient and effective rule generation tailored to the specific characteristics of the workload.
    % \input{tex/tuning.tex}
    \section{Evaluation}


\begin{table}[t]
    \centering
    % \vspace{-0.1in}
    \scalebox{0.78}{
    % \begin{small}
        \begin{tabular}{lccc}
            \toprule
            \multirow{2}*{\textbf{MoE Models}} & \textbf{Parameters} & \textbf{Experts Per Layer} & \textbf{Num. of} \\
            & \textbf{(active / total)} & \textbf{(active / total)} & \textbf{Layers} \\
            \otoprule 
            \mixtral~\cite{jiang2024mixtral} & 12.9B / 46.7B & 2 / 8 & 32 \\
            % \hline
            \qwen~\cite{yang2024qwen2} & 2.7B / 14.3B & 4 / 60 & 24 \\
            \phimoe~\cite{abdin2024phi} & 6.6B / 42B & 2 / 16 & 32 \\
            \bottomrule 
        \end{tabular}
    % \end{small}
    }
    \caption{Characteristics of three \MoE models in evaluation.}
    \vspace{-0.2in}
    \label{table:eval-moe-models}
\end{table}








\subsection{Experimental Setup}
\label{subsec:eval-setup}


% \begin{figure*}[t]
%     \centering
%     \begin{subfigure}[t]{0.48\textwidth}
%         \centering
%         \includegraphics[width=.9\linewidth]{figs/eval-overall-lmsys.pdf}
%         \caption{Serving three \MoE models with LMSYS-Chat-1M dataset.}
%     \end{subfigure}
%     \begin{subfigure}[t]{0.48\textwidth}
%         \centering
%         \includegraphics[width=.9\linewidth]{figs/eval-overall-sharegpt.pdf}
%         \caption{Serving three \MoE models with ShareGPT dataset.}
%     \end{subfigure}
%     \caption{Overall performance of prefill and decode stages for \sys and other four baselines.}
%     \label{fig:eval-overall.pdf}
% \end{figure*}


\noindent \textbf{Testbed.}
We conduct all experiments on a six-GPU testbed, where each GPU is an NVIDIA GeForce RTX 3090 with 24 GB GPU memory. 
%
All GPUs are inter-connected using pairwise NVLinks and connected to the CPU memory using PCIe 4.0 with 32GB/s bandwidth. 
%
Additionally, the testbed has a total of 32 AMD Ryzen Threadripper PRO 3955WX CPU cores and 480 GB CPU memory.


\noindent \textbf{Models.}
We employ three popular \MoE-based \LLMs in our evaluation: \mixtral~\cite{jiang2024mixtral}, \qwen~\cite{yang2024qwen2}, and \phimoe~\cite{abdin2024phi}.
Table~\ref{table:eval-moe-models} describes the parameters, number of \MoE layers, and number of experts per layer for the three models.
Following the evaluation of existing works~\cite{song2024promoe}, we profile the models to set the optimal prefetch distance $d$ to three before evaluation.
% We set $d$ of \mixtral, \qwen, and \phimoe to \todo{$xxx$}, \todo{$xxx$}, and \todo{$xxx$}, respectively.


\noindent \textbf{Datasets and traces.}
We employ two real-world prompt datasets commonly used for \LLM evaluation: LMSYS-Chat-1M~\cite{zheng2023lmsys} and ShareGPT~\cite{sharegpt}.
%
For most experiments, we split the sampled datasets in a standard 7:3 ratio, where 70\% of the prompts' context data (\ie, semantic embeddings and expert maps) are stored in \sys's Expert Map Store, and 30\% of the prompts are used for testing. 
%
For online serving experiments, we empty the Expert Map Store and use real-world \LLM inference traces~\cite{patel2024splitwise,stojkovic2025dynamollm} released by Microsoft Azure to set input and generation lengths and drive invocations.

\noindent \textbf{Baselines.}
We compare \sys against four \SOTA \MoE serving baselines:
1) \textbf{MoE-Infinity}~\cite{xue2024moe} uses coarse-grained request-level expert activation patterns and synchronous expert prediction and prefetching for \MoE serving. 
We prepare the expert activation matrix collection for MoE-Infinity before evaluation for a fair comparison.
%
% However, the open-sourced MoE-Infinity codebase~\cite{moe-infinity-code} lacks some features described in its original paper, we had to modify
%y 
2) \textbf{ProMoE}~\cite{song2024promoe} employs a stride-based speculative expert prefetching approach for \MoE serving. Since the codebase of ProMoE is not open-sourced and requires training predictors for each \MoE model, we reproduced a prototype of ProMoE on top of MoE-Infinity in our best effort.
%
3) \textbf{Mixtral-Offloading}~\cite{eliseev2023fast} combines a layer-wise speculative expert prefetching and a \LRU-based expert cache. 
%
4) \textbf{DeepSpeend-Inference} employs an expert-agnostic layer-wise parameter offloading approach, which uses pure on-demand loading and does not support prefetching. 
%
We implement the offloading logic of DeepSpeed-Inference in the MoE-Infinity codebase and add an expert cache for a fair comparison.
We enable all baselines to serve \MoE models from HuggingFace Transformer~\cite{wolf2020huggingface}. 


\noindent \textbf{Metrics.}
Following the standard evaluation methodology of existing works~\cite{song2024promoe,xue2024moe,zhong2024distserve,agrawal2024taming} on \LLM serving, we report the performance of the prefill and decode stages separately. 
We measure Time-to-First-Token (TTFT) for the prefill stage and Time-Per-Output-Token (TPOT) for the decode stage.
Additionally, we also report other system metrics, such as expert hit rate and overheads, for detailed evaluation.


% \noindent \textbf{\sys's setting.}
% The hyperparameters of \sys containing the prefetch distance $d$ for each \MoE model, Expert Map Store capacity $C$, and Expert Cache memory limit $M$.
% For most experiments, we profile the \MoE models and set the prefetch distance $d$ to their optimal values. The Expert Map Store capacity $C$ is set to \todo{$xxx$} expert maps. We configure the Expert Cache memory limit to \todo{$xxx$} GB.
% The hyperparameter sensitivity is analyzed in \S\ref{subsec:eval-sensitivity}.


\begin{figure}[t]
  \centering
  \includegraphics[width=.95\linewidth]{figs/eval-overall-arxiv.pdf}
  \vspace{-0.15in}
  \caption{Overall performance of prefill and decode stages for \sys and other four baselines.}
  \vspace{-0.2in}
  \label{fig:eval-overall}
\end{figure}


\subsection{Overall Performance}
\label{subsec:eval-overall}



We first evaluate the performance of prefill and decode stages when running \sys and other baselines with the three \MoE models, where we measure Time-To-First-Token (TTFT) and Time-Per-Output-Token (TPOT) for each stage.
Note that the inference latency with expert offloading tends to be higher than no offloading due to two reasons: 
1) During inference, an excessive amount of parameters in \MoE models are loaded and offloaded, which prolongs the inference latency.
2) All baselines and \sys are implemented on top of the MoE-Infinity codebase~\cite{moe-infinity-code}, whose inference latency is inherently impacted by MoE-Infinity's implementation.
Nevertheless, comparing \sys and baselines is fair with the same experimental setup.

Figure~\ref{fig:eval-overall} shows the \TTFT, \TPOT, and expert hit rate of \sys and other four baselines when serving three \MoE models with LMSYS-Chat-1M and ShareGPT datasets, respectively.
DeepSpeed has both the worst \TTFT and \TPOT due to expert-agnostic offloading and lacking expert prefetching.
While Mixtral-Offloading, ProMoE, and MoE-Infinity perform better than DeepSpeed-Inference, they are underperformed by \sys because of coarse-grained offloading designs.
Compared to DeepSpeed-Inference, Mixtral-Offloading, ProMoE, and MoE-Infinity, our \sys reduces the average \TTFT by 44\%, 35\%, 33\%, 30\%, and reduces the average \TPOT by 70\%, 61\%, 55\%, 48\%, across three \MoE models.
%
% Figure~\ref{fig:eval-overall} also reports the expert hit rate of \sys and each baseline. 
For expert hit rate, Mixtral-Offloading achieves a higher hit rate than the other three baselines because of its synchronous speculative prefetching with a prefetch distance of 1. However, due to synchronous prefetching, its \TTFT and \TPOT are worse than others except DeepSpeed-Inference.
\sys improves the average expert hit rate by 147\%, 11\%, 34\%, and 63\% over DeepSpeed-Inference, Mixtral-Offloading, ProMoE, and MoE-Infinity, respectively.

% \begin{figure}[t]
%   \centering
%   \includegraphics[width=.9\linewidth]{figs/eval-overall-sharegpt.pdf}
%   % \vspace{-0.15in}
%   \caption{}
%   % \vspace{-0.25in}
%   \label{fig:eval-overall-sharegpt.pdf}
% \end{figure}




\subsection{Online Serving Performance}
\label{subsec:eval-online}


Except for the offline evaluation (\ie, Expert Map Store in full capacity before serving), we also evaluate \sys against other baselines in online serving settings.
We empty the Expert Map Store of \sys and the expert activation matrix collection of MoE-Infinity for the online serving experiment.
%
The request traces are derived from Azure \LLM inference traces~\cite{patel2024splitwise,stojkovic2025dynamollm}, with 64 requests randomly sampled to drive LMSYS-Chat-1M prompts for each \MoE model serving. 
To ensure consistency, \sys and all baselines input and generate the exact number of tokens specified in the traces.
%
Figure~\ref{fig:eval-online-serve} illustrates the CDF of end-to-end request latency across three \MoE models. The results demonstrate that \sys significantly reduces overall request latency compared to other baselines in online serving scenarios.


\begin{figure}[t]
  \centering
  \includegraphics[width=.95\linewidth]{figs/eval-online-serve-arxiv.pdf}
  \vspace{-0.15in}
  \caption{CDF of request latency for \MoE online serving.}
  \vspace{-0.2in}
  \label{fig:eval-online-serve}
\end{figure}



\subsection{Impact of Expert Cache Limits}



We measure the \TPOT of \sys and other baselines by limiting the expert cache memory budget to investigate their performance in the latency-memory trade-off (\S\ref{subsec:bg-latency-memory-tradeoff}).
We mainly focus on \TPOT to show the end-to-end performance impacted by varying cache limits.
Figure~\ref{fig:eval-cache-limit.pdf} shows the \TPOT of \sys and other four baselines when serving three \MoE models under different expert cache limits.
We gradually increase the GPU memory allocated for caching experts from 6 GB to 96 GB while employing the same experimental setting in \S\ref{subsec:eval-overall}.
Similarly, DeepSpeed-Inference has the worst \TPOT due to being expert-agnostic.
\sys consistently outperforms Mixtral-Offloading, ProMoE, and MoE-Infinity under varying expert cache limits.
Especially for limited GPU memory sizes (\eg, 6GB), \sys reduces the \TPOT by 32\%, 24\%, 18\%, and 18\%, compared to DeepSpeed-Inference, Mixtral-Offloading, ProMoE, and MoE-Infinity, across three \MoE models, respectively.
With fine-grained expert offloading, \sys significantly reduces the expert on-demand loading latency while maintaining a lower GPU memory footprint, therefore achieving a better spot in the latency-memory trade-off of \MoE serving.

% \subsection{Impact of Inference Batch Size}

\subsection{Ablation Study}
\label{subsec:eval-ablation}


% \begin{figure}[t]
%   \centering
%   \includegraphics[width=.95\linewidth]{figs/eval-expert-tracking.pdf}
%   % \vspace{-0.15in}
%   \caption{Expert hit rate of different expert pattern tracking approaches.}
%   % \vspace{-0.25in}
%   \label{fig:eval-expert-tracking}
% \end{figure}



We present the ablation study of \sys's design.


\textbf{Effectiveness of expert map search.}
One of \sys's key designs is the expert map, which tracks expert selection preferences in fine granularity.
We evaluate the effectiveness of the expert map against five expert pattern-tracking approaches as follows.
%
1) \textbf{Speculate}: speculative prediction used by Mixtral-Offloading~\cite{eliseev2023fast} and ProMoE~\cite{song2024promoe}, 
%
2) \textbf{Hit count}: request-level expert hit count used by MoE-Infinity~\cite{xue2024moe}, 
%
3) \textbf{Map (T)}: expert map with only trajectory similarity search,
4) \textbf{Map (T+S)}: expert map with both trajectory and semantic similarity search,
%
and
5) \textbf{Map (T+S+$\delta$)}: expert map with full features enabled, including trajectory and semantic similarity search (\S\ref{subsec:design-similarity-match}) and dynamic expert selection (\S\ref{subsec:design-expert-prefetch}).
%
We implement the above methods in \sys's Expert Map Matcher for a fair comparison.
Figure~\ref{fig:eval-expert-tracking} shows the expert hit rate of the above expert pattern tracking methods.
%
Speculative prediction is effective due to the widespread presence of residual connections in Transformer blocks. However, its effectiveness decreases drastically as prefetch distance increases~\cite{song2024promoe}.
%
The request-level expert activation count has the worst performance due to coarse granularity.
%
As features are incrementally restored to \sys's expert map, the expert hit rate gradually increases, demonstrating its effectiveness.

% \textbf{Effectiveness of asynchronous map matching.}




\begin{figure}[t]
  \centering
  \includegraphics[width=.9\linewidth]{figs/eval-cache-limit-arxiv.pdf}
  \vspace{-0.15in}
  \caption{Performance of \sys and other four baselines under varying expert cache limits.}
  \vspace{-0.1in}
  \label{fig:eval-cache-limit.pdf}
\end{figure}

\begin{figure}[!t]
    \centering
    \begin{subfigure}[t]{0.585\linewidth}
        \centering
        \includegraphics[width=\linewidth]{figs/eval-expert-tracking.pdf}
        \caption{Expert pattern tracking approaches.}
        \label{fig:eval-expert-tracking}
    \end{subfigure}
    % \hspace{0.02in}
    \begin{subfigure}[t]{0.385\linewidth}
        \centering
        \includegraphics[width=\linewidth]{figs/eval-prefetch-and-cache-arxiv.pdf}
        \caption{Prefetch and caching.}
        \label{fig:eval-prefetch-and-cache}
    \end{subfigure}
    \vspace{-0.1in}
    \caption{Ablation study of \sys.}
    \label{fig:eval-ablation}
    \vspace{-0.2in}
\end{figure}

\textbf{Effectiveness of expert prefetching and caching.}
We evaluate \sys's expert prefetching and caching against two caching algorithms:
1) \textbf{\LRU} used by Mixtral-Offloading~\cite{eliseev2023fast}
and 
2) \textbf{\LFU} used by MoE-Infinity~\cite{xue2024moe}.
%
Figure~\ref{fig:eval-prefetch-and-cache} depicts the expert hit rate of \sys and two baselines.
The results show that \LRU performs poorly in expert offloading scenarios. Though \LFU achieves a higher hit rate than \LRU, \sys surpasses both, achieving the highest expert hit rate.

\subsection{Sensitivity Analysis}
\label{subsec:eval-sensitivity}


\begin{figure}[t]
  \centering
  \includegraphics[width=.9\linewidth]{figs/eval-prefetch-distance.pdf}
  \vspace{-0.15in}
  \caption{Performance of \sys serving \MoE models with different prefetch distances.}
  \vspace{-0.1in}
  \label{fig:eval-prefetch-distance}
\end{figure}

% \begin{figure}[t]
%   \centering
%   \includegraphics[width=.9\linewidth]{figs/eval-store-capacity.pdf}
%   % \vspace{-0.15in}
%   \caption{Semantic and trajectory similarity lower bounds in \sys's serving with different Expert Map Store capacity.}
%   % \vspace{-0.25in}
%   \label{fig:eval-store-capacity}
% \end{figure}

\begin{figure}[t]
    \centering
    \begin{subfigure}[t]{0.55\linewidth}
        \centering
        \includegraphics[width=\linewidth]{figs/eval-store-capacity.pdf}
        \caption{Expert Map Store capacity.}
        \label{fig:eval-store-capacity}
    \end{subfigure}
    % \hspace{0.02in}
    \begin{subfigure}[t]{0.435\linewidth}
        \centering
        \includegraphics[width=\linewidth]{figs/eval-batch-size-arxiv.pdf}
        \caption{Inference batch size.}
        \label{fig:eval-batch-size}
    \end{subfigure}
    \vspace{-0.1in}
    \caption{Sensitivity analysis of \sys.}
    \vspace{-0.2in}
    \label{fig:eval-sensitivity}
\end{figure}


We analyze the sensitivity of three hyperparameters: prefetch distance of \MoE models, the capacity of Expert Map Store, and inference batch size.


\textbf{Prefetch distance of \MoE models.}
Figure~\ref{fig:eval-prefetch-distance} shows the \TTFT and \TPOT of \sys when serving three \MoE models with different prefetch distances.
%
We have demonstrated that the expert hit rate decreases when gradually increasing the prefetch distance (Figure~\ref{fig:bg-hit-distance}).
%
When the prefetch distance is small ($<3$), \sys cannot perfectly hide its system delay from the inference process, such as the map matching and expert prefetching, leading to the increase of inference latency.
%
With larger prefetch distances ($>3$), \sys has worse expert hit rates that also degrade the performance. 
Therefore, we set the prefetch distance $d$ to 3 for evaluating \sys.


\textbf{Capacity of Expert Map Store.}
We measure the mean semantic and trajectory similarity scores searched in \sys's expert map matching for \MoE model serving.
%
Figure~\ref{fig:eval-store-capacity} presents the mean semantic and trajectory similarity scores of \sys with different Expert Map Store capacity sizes.
%
Both semantic and trajectory similarity scores improve as the store capacity increases.
%
While the similarity scores exhibit a significant increase with capacities below 1K, further capacity expansion yields diminishing similarity gains. 
To minimize \sys's memory overhead, we set \sys's Expert Map Store capacity to 1K in evaluation.


\textbf{Inference batch size.}
We investigate the impact of inference batch size on \sys and three baselines using \mixtral with LMSYS-Chat-1M.
%
Figure~\ref{fig:eval-batch-size} presents the performance of \sys, Mixtral-Offloading, ProMoE, and MoE-Infinity as the batch size increases from one to four. \sys achieves the lowest \TTFT and \TPOT in most cases.


% \textbf{Inference batch size.}


% \subsection{Scalability}
% \label{subsec:eval-scalability}
% From one to six GPUs


\begin{figure}[t]
  \centering
  \includegraphics[width=.92\linewidth]{figs/eval-overhead-latency.pdf}
  \vspace{-0.15in}
  \caption{Latency breakdown of \sys's one inference iteration with three \MoE models.}
  \vspace{-0.1in}
  \label{fig:eval-overhead-latency.pdf}
\end{figure}





\subsection{System Overheads}
\label{subsec:eval-overhead}


\noindent \textbf{Latency overheads of \sys's operations.}
Figure~\ref{fig:eval-overhead-latency.pdf} shows the latency breakdown of one inference iteration in \sys when serving the three \MoE models.
We report any operations of \sys in \S\ref{subsec:eval-overall} that may incur a significant latency delay, including context collection, map matching, expert on-demand loading, expert prefetching, and map update after the iteration completes.
\qwen has lower end-to-end iteration latency than \mixtral and \phimoe because of significantly fewer parameters.
Note that expert prefetching, map matching, and map update tasks are executed asynchronously, aside from the inference process. Hence, they do not contribute to the end-to-end iteration latency.
Excluding three asynchronous tasks, the total delay incurred by other operations is consistently less than 30ms (5\% of the iteration) across three \MoE models, which is negligible compared to the inference latency.


\noindent \textbf{Memory overheads of \sys's Expert Map Store.}
Figure~\ref{fig:eval-overhead-memory.pdf} shows the CPU memory footprint of \sys's Expert Map Store when varying the store capacity from 1K to 32K maps.
The memory needed to store expert maps for \qwen is more than \mixtral and \phimoe because it has more experts per layer over the other two models, which increases the map shape.
Even for the largest capacity (32K), the Expert Map Store requires less than 200MB of memory to store the maps, which is trivial since modern GPU servers usually have abundant CPU memory (\eg, p4d.24xlarge on AWS EC2~\cite{aws-ec2} has over 1100 GB of CPU memory).
In the evaluation, \sys's map store capacity with 1K maps is sufficient for maintaining performance (\S\ref{subsec:eval-sensitivity}), resulting in minimal memory overhead.



\begin{figure}[t]
  \centering
  \includegraphics[width=.85\linewidth]{figs/eval-overhead-memory.pdf}
  % \vspace{-0.1in}
  \caption{CPU memory footprint of \sys's Expert Map Store with different capacity.}
  \vspace{-0.1in}
  \label{fig:eval-overhead-memory.pdf}
\end{figure}

    % We introduced the Bidirectional Diffusion Bridge Model (BDBM), a novel
framework for bidirectional image-to-image (I2I) translation using
a single network. By leveraging the Chapman-Kolmogorov Equation, BDBM
models the shared components of forward and backward transitions,
enabling efficient bidirectional generation with minimal computational
overhead. Empirical results demonstrated that BDBM consistently outperforms
existing I2I translation methods across diverse datasets.

Despite these strengths, BDBM has so far been applied exclusively
to the image domain. Extending it to other domains, such as text,
presents an exciting direction for future research. In particular,
exploring BDBM for multimodal tasks like image$\leftrightarrow$text
generation would be a promising avenue.



    
\section{Related Work} \label{sec:related}

% \textbf{Adversarial Attack}
\textbf{Attacks on SLAM.} 
%With the rise of machine learning, 
The robustness of computer vision systems is being actively investigated. With the emergence of adversarial images in the digital domain by adding optimized noise directly to images~\cite{szegedy2013intriguing,carlini2017towards}, researchers find that such attacks also exist physically in the real world \cite{eykholt2018robust,song2018physical,zhao2019seeing}. To fill the gap between attacks in the digital and physical worlds, recent studies have demonstrated that attacks on real-world computer vision systems are practical \cite{eykholt2018robust,li2019adversarial,man2020ghostimage,sharif2016accessorize,zhao2019seeing,zhou2018invisible}. However, attacks on traditional computer vision methods such as SLAM are relatively less explored. \cite{yoshida2022adversarial} proposes an attack against the scan matching algorithm in LiDAR-based SLAM, while most SLAMs in AR/VR devices rely on different sensors like RGB/depth cameras and IMUs. \cite{ikram2022perceptual} and \cite{chen2024adversary} mislead visual SLAM by poisoning the images with special patterns, and \cite{wang2021can} causes the camera to fail using infrared light. In our work, we demonstrate attacks on Visual-Inertial SLAM (VI-SLAM) by perturbing the IMU readings, rather than cameras, and showing its impact on XR user experience. 

\textbf{Acoustic Injection Attacks.} Among various physical attacks, acoustic injection attacks are attractive due to their low cost. Son~\etal~\cite{son2015rocking} were the first to introduce acoustic attacks on MEMS gyroscopes, demonstrating how these attacks could lead to sensor denial-of-service and result in drone crashes. WALNUT~\cite{trippel2017walnut} expanded on this by developing output biasing and control attacks that enable precise manipulation of MEMS accelerometer outputs using modulated sound waves. Wang et al.~\cite{wang2017sonic} demonstrated a sonic gun, showcasing the vulnerability of various smart devices (\eg drones and self-balancing vehicles) to acoustic attacks. Tu et al. \cite{tu2018injected} designed side-swing and switching attacks to alter the outputs of MEMS gyroscopes and accelerometers. Furthermore, Ji et al. \cite{ji2021poltergeist} fool the object detectors by applying acoustic attack to the image stabilizers commonly used in modern cameras. However, none of the existing works study the relationship between the acoustic injections and SLAM outputs on recent XR devices. 

% \zijian{Do we need one session about security in AR/VR?}
% \yicheng{TODO}
%\jiasi{cite the AIVR paper (UMass Amherst?) paper is we have not already. They add IMU perturbation but w/o SLAM, iirc} \yicheng{Cited}

\textbf{XR Security and Privacy.} 
%Security and privacy concerns in XR systems have gained significant attention. 
For single-user XR systems, researchers have demonstrated various side-channel attacks to extract sensitive information (\eg keystrokes) through video feeds~\cite{ling2019know}, head movements~\cite{nair2023unique, slocum2023going}, architectural hints~\cite{zhang2023its,shang2020arspy}, power usage~\cite{li2024dangers}, and EM side-channel leakages~\cite{al2021vr}. In multi-user XR systems, Su et al.~\cite{su2024remote} use avatar motion data to infer keystrokes in shared VR environments. Slocum et al.~\cite{slocum2024doesn} reveal vulnerabilities in the shared state frameworks of multi-user AR. Similarly, Lebeck et al.~\cite{lebeck2017securing} highlight risks like deceptive virtual objects and emphasize access control for managing shared physical and virtual spaces. Ruth et al.~\cite{ruth2019secure} further propose a secure multi-user AR framework focusing on content sharing and permissions.
Chandio et al.~\cite{chandio2024stealthy} %introduced a multi-modal spatiotemporal attack that 
simultaneously manipulated visual and inertial sensors to disrupt XR pose estimation. However, their study evaluated the attack using offline datasets and assumed the attacker's capability to manipulate IMU data streams through acoustic means, without real experiments. Ours is the first to demonstrate acoustic injection attacks on recent XR devices, like the Hololens 2, in the real world.
 


    \section{Concluding Discussion}
\label{sec:conclusion}
In this paper, we examine the  trade-offs among privacy, utility, and efficiency while fine-tuning an LLM. The traditional wisdom of achieving privacy comes at the cost of computational inefficiency using dedicated methods like DP. In contrast, we demonstrate that parameter efficient fine-tuning methods like LoRA, initially designed for efficiency, achieves privacy of sensitive data without any computational overhead. Simultaneously, LoRA retains the utility of general language understanding compared to DP, or even full-fine-tuning, realizing the superiority of LoRA in optimizing all three aspects. Towards our investigation, we  establish the significance of redefining privacy and utility using a careful distinction between sensitive and non-sensitive counterparts of the fine-tuned data. Through case studies, we demonstrate how existing measures exaggerate privacy threats and undermine the utility of an LLM. Our paper calls for a joint venture of   privacy and systems communities in achieving privacy-aware efficient fine-tuning of LLMs while retaining utility.
    % \section*{Acknowledgment}

% \section*{References}

\bibliographystyle{IEEEtran}
\bibliography{IEEEabrv, refer}

\end{document}
