\section{Conclusion}
\label{sec:conclude}
This paper investigated the potential of prefetching in mobile augmented reality (MAR) applications to improve user experience by reducing latency.  To address the compute and data intensive nature of AR, we proposed \spaarc{}, a prefetching policy for edge AR caches that leverages object associations and location information to proactively fetch virtual objects from the cloud. \spaarc{} incorporates tunable parameters, including minimum support and minimum confidence, which can be adjusted to meet specific application needs. 
%We devised a strategy for optimizing the minimum support parameter based on an analysis of rule quality (lift). This approach can be extended to the minimum confidence parameter as well. 
Through extensive evaluation using both synthetic and real-world workloads, we demonstrated that \spaarc{} significantly improves cache hit rates compared to standard caching algorithms, achieving gains ranging from 3\% to 40\% while reducing the need for on-demand data retrieval from the cloud. Further, we presented an adaptive tuning algorithm that automatically tunes \spaarc{} parameters to achieve optimal performance. Our findings demonstrate the potential of \spaarc{} to substantially enhance the user experience in MAR applications by ensuring the timely availability of virtual objects.  Future research could explore further refinement of the parameter tuning process and investigate methods to mitigate the communication overhead.

%Our evaluation using synthetic and real workloads demonstrated that SPAARC significantly improves traditional cache hit rates, albeit with the trade-off of increased communication costs for prefetching.  Testing with both synthetic and real workloads confirmed SPAARC's ability to enhance hit rates after an initial warmup phase.  Future research could explore further refinement of the parameter tuning process and investigate methods to mitigate the communication overhead.