\section{Preliminaries}

\paragraph{Notation} We use boldface lowercase letters, such as $\vx, \vy, \vz$, to represent vectors, and boldface capital letters, such as $\mat{A}, \mat{C}$, for matrices. We denote by $x_i$ the $i$th coordinate of a vector $\vx \in \R^n$. We use the shorthand notation $[n] \defeq \{1, 2, \dots, n\}$. $\Delta^n \defeq \{ \vx \in \R^n_{\geq 0} : \sum_{i=1}^n x_i = 1 \}$ is the probability simplex on $\R^n$. For $i \in [n]$, $\vec{e}_i \in \Delta^n$ is the $i$th unit vector. $\langle \cdot, \cdot \rangle$ denotes the inner product. For a vector $\vx \in \R^n$, $\|\vx\|_2 = \sqrt{\langle \vx, \vx \rangle}$ is its Euclidean norm. For $m \leq n$, $\vx_{[1 \cdots m]} \in \R^m$ is the vector containing the first $m$ coordinates of $\vx$. We sometimes use the $O(\cdot), \Theta(\cdot), \Omega(\cdot)$ notation to suppress absolute constants. We say that a continuously differentiable function $f$ is \emph{$L$-smooth} if its gradient is $L$-Lipschitz continuous with respect to $\|\cdot\|_2$; that is, $\| \nabla f(\vx) - \nabla f(\vx')  \|_2 \leq L \| \vx - \vx' \|_2$ for all $\vx, \vx'$ in the domain.

\paragraph{Two-player games} In a two-player game, represented in normal-form game, each player has a finite set, let $[n]$, of actions. Under a pair of actions $(i, j) \in [n] \times [n]$, the utility of the \emph{row} player is given by $\mat{R}_{i, j} $, where $\mat{R} \in \mathbb{Q}^{n \times n}$ is the payoff matrix of the row player. Further, we let $\mat{C} \in \mathbb{Q}^{n \times n}$ be the payoff matrix of the column player. Players are allowed to randomize by selecting mixed strategies---points in $\Delta^n$. Under a pair of mixed strategies $(\vx, \vy) \in \Delta^n \times \Delta^n$, the expected utility of the players is given by $\langle \vx, \mat{R} \vy \rangle$ and $\langle \vx, \mat{C} \vy \rangle$, respectively. The canonical solution concept in such games is the \emph{Nash equilibrium}~\citep{nash1951non}, which is recalled below.

\begin{definition}
    \label{def:NE}
    A pair of strategies $(\vx^*, \vy^*)$ is an \emph{$\epsilon$-Nash equilibrium} of $(\mat{R}, \mat{C})$ if
    \[
        \langle \vx^*, \mat{R} \vy^* \rangle \geq \langle \vx, \mat{R} \vy^* \rangle - \epsilon \quad \text{and} \quad \langle \vx^*, \mat{C} \vy^* \rangle \geq \langle \vx^*, \mat{C} \vy \rangle - \epsilon \quad \forall (\vx, \vy) \in \Delta^n \times \Delta^n.
    \]
\end{definition}

\paragraph{Symmetric two-player games} One of our reductions is based on \emph{symmetric} two-player games, meaning that $\mat{R} = \mat{C}^\top$. A basic fact, which goes back to~\citet{nash1951non}, is that any symmetric game admits a \emph{symmetric} Nash equilibrium $(\vx^*, \vx^*)$. Further, computing a Nash equilibrium in a general game can be readily reduced to computing a symmetric Nash equilibrium in a symmetric game~\citep[Theorem 2.4]{Nisan07:Algorithmic}. In conjunction with the hardness result of~\citet{Chen09:Settling}, we state the following immediate consequence.

\begin{theorem}[\citealp{Chen09:Settling}] \label{theorem:PPAD_for _symmetric}
    Computing a symmetric $\nicefrac{1}{n^c}$-Nash equilibrium in a symmetric two-player game is \PPAD-hard for any constant $c > 0$.
\end{theorem}

\paragraph{Team zero-sum games} A (two-)team zero-sum game is a multi-player game---represented in normal form for the purposes of this paper---in which the players' utilities have a certain structure; namely, we can partition the players into two (disjoint) subsets, such that each player within the same team shares the same utility, whereas players in different teams have opposite utilities---under any possible combination of actions. An \emph{adversarial} team game is a specific type of team zero-sum game wherein one team consists of a single player. As in~\Cref{def:NE} for two-player games, an \emph{$\epsilon$-Nash equilibrium} is a tuple of strategies such that no unilateral deviation yields more than an $\epsilon$ additive improvement in the utility of the deviator.