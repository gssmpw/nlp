\section{Omitted proofs}
\label{sec:proofs}

This section contains the proofs omitted from the main body.

\subsection{Proofs from Section~\ref{sec:cls}}
\label{sec:proofs1}

We begin with~\Cref{lemma:mass_on_small_reward}.

\begin{proof}[Proof of~\Cref{lemma:mass_on_small_reward}]
    For the sake of contradiction, suppose that $x^*_i(a_k) > \frac{\epsilon^2}{c}$ for some $k \in [n]$ such that $u_i(a_k, \vx^*_{-i}) \leq u_i(a_j, \vx^*_{-i}) - c$. Consider the strategy $\Delta^n \ni \vx_i' = \vx_i^* + x_i^*(a_k) \ve_{j} - x_i^*(a_k) \ve_{k}$. Then, we have
    \begin{align*}
        u_i(\vx_i', \vx^*_{-i}) - u_i(\vx_i^*, \vx^*_{-i}) &= x_i^*(a_k) u_i(a_j, \vx^*_{-i}) - x_i^*(a_k) u_i(a_k, \vx^*_{-i}) \\
        &\geq c x_i^*(a_k) \\
        &> \epsilon^2.
    \end{align*}
    That is, deviating to $\vx_i'$ yields a utility benefit strictly larger than $\epsilon^2$, which contradicts the assumption that $(\vx_i^*, \vx^*_{-i})$ is an $\epsilon^2$-Nash Equilibrium.
\end{proof}

We continue with~\Cref{thm:teamadvtosymme}, which combines~\Cref{lemma:small_z,lemma:closeness} to complete the $\CLS$-hardness reduction of~\Cref{sec:cls}.

\teamadvtosymme*

\begin{proof}
    Since $(\vx^*, \vy^*, \vz^*)$ is an $\epsilon^2$-Nash equilibrium, we have that for any for any deviation $\vy' \in \Delta^n$ of Player $\vy$,
    \begin{align}
        \langle \vx^*, \mat{A} \vy^* \rangle & \leq \langle \vx^*, \mat{A} \vy' \rangle  + \frac{|\Amin|}{\epsilon} \left(\sum_{i = 1}^n z_i (x^*_i - y'_i) + z_{n + i} (y'_i - x^*_i)\right) + \epsilon^2. \label{eq:nash1}
        % & \leq {\vx^*}^T A \vy' + \frac{|A_{min}|}{\epsilon} \cdot 2\epsilon^2 + \epsilon^2 \label{eq:theom2} \\
        % & = {\vx^*}^T A \vy' + 2|A_{min}| \cdot \epsilon + \epsilon^2. \label{eq:theom_final1}
    \end{align}
    Moreover, by considering a deviation of Player $\vx$ again to $\vy'$,
    \begin{align}
        \langle \vx^*, \mat{A} \vy^* \rangle  & \leq \langle \vy', \mat{A} \vy^* \rangle  + \frac{|\Amin|}{\epsilon} \left(\sum_{i = 1}^n z_i (y'_i - y^*_i) + z_{n + i} (y^*_i - y'
        _i)\right) + \epsilon^2 \label{eq:nash2}
    \end{align}
    Adding~\eqref{eq:nash1} and~\eqref{eq:nash2}, and using the fact that $\mat{A}$ is a symmetric matrix,
    \begin{align}
        2 \langle \vx^*, \mat{A} \vy^* \rangle  & \leq \langle \vy', \mat{A} (\vx^* + \vy^*) \rangle + \frac{|\Amin|}{\epsilon} \left(\sum_{i = 1}^n z_i (x^*_i - y^*_i) + z_{n + i} (y^*_i - x^*
        _i) \right) + 2 \epsilon^2 \notag \\
        & \leq \langle \vy', \mat{A} (2 \vy^*) \rangle  + 2\epsilon n |\Amin| + \frac{|\Amin|}{\epsilon} ( 2n\cdot 9 \epsilon \cdot 2\epsilon ) + 2\epsilon^2 \label{eq:bounderror1} \\
        &= 2 \langle \vy', \mat{A} \vy^* \rangle + (38 n + 2) |\Amin| \epsilon, \label{eq:final1}
    \end{align}
    where in~\eqref{eq:bounderror1} we use \Cref{lemma:closeness,lemma:small_z}. Also,
    \begin{align}
        \langle \vy^*, \mat{A} \vy^* \rangle & = \langle \vx^*, \mat{A} \vy^* \rangle + \langle \vy^* - \vx^*, \mat{A} \vy^* \rangle \notag \\
        & \leq \langle \vx^*, \mat{A} \vy^* \rangle + 2 \epsilon n |\Amin|. \label{eq:final2}
    \end{align}
    Finally, combining~\eqref{eq:final1} and~\eqref{eq:final2}, we conclude that for any $\vy' \in \Delta^n$,
    \begin{align*}
        \langle \vy^*, \mat{A} \vy^* \rangle \leq \langle \vy', \mat{A} \vy^* \rangle  + (21 n + 1) |\Amin|\epsilon.
    \end{align*}
    This concludes the proof.
\end{proof}

\subsection{Proofs from Section~\ref{sec:symmetric}}
\label{sec:proofs2}

We continue by applying Brouwer's fixed point theorem to show that symmetric min-max optimization problems always admit a symmetric equilibrium.

\begin{lemma}\label{lem:exists}
Let $\calX$ be a convex and compact set. Then, any $L$-smooth, antisymmetric function (\Cref{def:symmetric}) $f:\calX\times \calX \to \R$ admits a symmetric first-order Nash equilibrium $(\vx^*,\vx^*)$.
\end{lemma}

\begin{proof}
We define the function $M: \calX \to \calX$ to be
\begin{equation}
M(\vx') := \prod_{\calX} \left[\vx' - \nabla _{\vx}  f(\vx,\vy)\Big|_{(\vx,\vy)=(\vx',\vx')} \right].
\end{equation}
Given that $f$ is $L$-smooth, we conclude that $M(\vx')$ is $(L+1)$-Lipschitz, hence continuous. Therefore, from Brouwer's fixed point theorem, there exists an $\vx^*$ so that $M(\vx^*) = \vx^*.$ Moreover, the symmetry of $f$ implies that $\nabla_{\vy}f(\vx,\vy) \Big|_{(\vx,\vy)=(\vw,\vw)} =-\nabla_{\vx}f(\vx,\vy) \Big|_{(\vx,\vy)=(\vw,\vw)} $ for all $\vw\in\calX$; as a result,
\begin{equation*}
    \begin{array}{cc}
    \vx^* &= \left[\vx^* - \nabla _{\vx}  f(\vx,\vy)\Big|_{(\vx,\vy)=(\vx^*,\vx^*)} \right] \\
    &= \left[\vx^* + \nabla _{\vy}  f(\vx,\vy)\Big|_{(\vx,\vy)=(\vx^*,\vx^*)} \right].
    \end{array} 
\end{equation*}
Therefore, $(\vx^*,\vx^*)$ is a first-order Nash equilibrium of the symmetric min-max problem with function $f$. 
\end{proof}

We next state a standard lemma that connects first-order optimality with the fixed-point gap of gradient ascent.

\begin{lemma}[\citep{ghadimi2016accelerated}, Lemma 3 for $c=1$]\label{lem:approxsmooth} Let $f(\vx)$ be a $L$-smooth function in $\vx \in \Delta^n$. Define the gradient mapping 
\[G(\vx) := \prod_{\Delta^n}
\left\{\vx+\nabla_{\vx}f(\vx)\right\}-\vx.\]
If $\norm{G(\vx^*)}_2 \leq \epsilon$, that is, $\vx^*$ is an $\epsilon$-approximate fixed point of gradient ascent with stepsize equal to one, then
\[
\max_{\vx^*+\bm{\delta} \in \Delta^n, \norm{\bm{\delta}}_2\leq 1} \bm{\delta}^{\top}\nabla_{\vx} f(\vx^*) \leq \epsilon(L+1).
\]
\end{lemma}

Similarly, the next lemma makes such a connection for min-max optimization problems with coupled constraints; it is mostly extracted from~\citet[Section B.2]{DSZ21}.

\begin{lemma}\label{lem:safe} Let $f(\vx,\vy)$ be a $G$-Lipschitz, $L$-smooth function defined in some polytope domain $\mathcal{D} \subseteq \Delta^n \times \Delta^n$ of diameter $D$. Define the mapping 
\[G(\vx,\vy) := \prod_{\mathcal{D}}
\left\{\vx-\nabla_{\vx}f(\vx,\vy),\vy+\nabla_{\vy}f(\vx,\vy)\right\}-(\vx,\vy).\]
If $\norm{G(\vx^*,\vy^*)}_2 \leq \epsilon$, that is, $(\vx^*,\vy^*)$ is an $\epsilon$-approximate fixed point of (the safe version) of GDA with stepsize equal to one, then 
\[
\langle \vx-\vx^*,\nabla_{\vx}f(\vx^*,\vy^*)\rangle
\geq  -\sqrt{\epsilon}K \textrm { for }\vx \in \mathcal{D}(\vy^*) \textrm{ and }\langle \vy-\vy^*,\nabla_{\vy}f(\vx^*,\vy^*)\rangle
\leq  \sqrt{\epsilon}K \textrm { for }\vy \in \mathcal{D}(\vx^*),
\]
where $\mathcal{D}(\vx^*) = \{\vy:(\vx^*,\vy)\in\mathcal{D}\}$, 
$\mathcal{D}(\vy^*) = \{\vx:(\vx,\vy^*)\in\mathcal{D}\}$ and $K = (L+1)\sqrt{(G+4\sqrt{2})}.$
\end{lemma}
\begin{proof}
Let 
$(\vx_{\Delta},\vy_{\Delta}) = (\vx^*-\nabla_{\vx}f(\vx^*,\vy^*),\vy^*+\nabla_{\vy}f(\vx^*,\vy^*)).$ 
In~\citet[Claim B.2]{DSZ21}, it was shown that for all $(\vx,\vy)\in \mathcal{D}$, we have
\[\langle(\vx_{\Delta},\vy_{\Delta})-(\vx^*,\vy^*),(\vx,\vy)-(\vx^*,\vy^*)\rangle\leq (G+2D)\epsilon.\]
Using the above inequality, it was concluded that  $(\vx^*,\vy^*)$ is an approximate fixed point of the ``unsafe'' version of GDA; specifically,
\[
\left\|\vx^* - \prod_{\mathcal{D}(\vy^*)}\{\vx^* - \nabla_{\vx}f(\vx^*,\vy^*)\}\right\|
\leq \sqrt{(G+2D)\epsilon}
\]
and
\[
\left\|\vy^* - \prod_{\mathcal{D}(\vx^*)}\{\vy^* + \nabla_{\vy}f(\vx^*,\vy^*)\}\right\|
\leq \sqrt{(G+2D)\epsilon}.
\]
We now use~\Cref{lem:approxsmooth} for both inequalities above, together the fact that $D = 2\sqrt{2}$, to conclude that
\[
\langle \vx-\vx^*,\nabla_{\vx}f(\vx^*,\vy^*)\rangle
\geq  -\sqrt{(G+4\sqrt{2})\epsilon}(L+1) \textrm { for }\vx \in \mathcal{D}(\vy^*),\]  and \[\langle \vy-\vy^*,\nabla_{\vy}f(\vx^*,\vy^*)\rangle
\leq  \sqrt{(G+4\sqrt{2})\epsilon}(L+1) \textrm { for }\vy \in \mathcal{D}(\vx^*).
\]
\end{proof}

\subsection{Proofs from Section~\ref{sec:nonsymmetric}}
\label{sec:proofs3}

In this subsection, our goal is to establish~\Cref{theorem:symmetric-new} that was claimed earlier in the main body, which forms the basis for~\Cref{theorem:uniqueATG,theorem:non-symmetric}.

We consider the symmetric game $(\lineA, \lineA^{\top})$, where $\lineA \in \mathbb{R}^{n \times n}$ is a symmetric matrix. In particular, since $\lineA = \lineA^{\top}$, the two players in the game share the same payoff matrix. The payoff matrix $\lineA$ is constructed based on an underlying graph $G = ([n], E) $ and a parameter $\delta \in (0, 1)$ as follows:
\begin{align*}
    \lineA_{i, j} = 
    \begin{cases}
        \delta & \text{if $i = j$}, \\
        1 & \text{if $(i, j) \in E$}, \\
        0 & \text{otherwise.}
    \end{cases}
\end{align*}

\begin{lemma} \label{lemma:value_of_approx_nash_on_clique}
    Let $(\hat{\vx}, \hat{\vx})$ be an $\epsilon$-well-supported NE of the game where $\hat{\vx}$ is supported on a max clique of size $k$, denoted as $C_k$, then the value $u(\hat{\vx}, \hat{\vx})$ is at least $ 1 - \frac{1}{k} + \frac{\delta}{k} - (\frac{k - \delta}{1 - \delta})  \epsilon$ and $\norm{\hat{\vx} - \vx^*}_{\infty} \leq \frac{k - \delta}{1 - \delta} \epsilon$ where $\vx^* = \frac{1}{k} \sum_{i \in C_k} \ve_{i}$.
\end{lemma}

\begin{proof}
    Since $\hat{\vx}$ is supported on $C_k$, let $i$ be a coordinate that $\hat{\vx}^*$ puts the least probability mass on; that is, $i \in \argmin_{j \in C_k} \hat{x}_j.$ Considering the utility of playing action $a_i$, we have
    \begin{align*}
        u(\ve_i, \hat{\vx}) = \hat{x}_i \cdot \delta + (1 - \hat{x}_i) \cdot 1 = 1 - \hat{x}_i + \hat{x}_i \delta.
    \end{align*}
    Moreover, let $j \in C_k$ be a coordinate such that $\hat{\vx}_j \geq \frac{1}{k}$. It should hold that 
    \begin{align*}
    u(\ve_j, \hat{\vx}) = 1 - (1-\delta)\hat{x}_j \leq 1 - \frac{1}{k} + \frac{\delta}{k}.
    \end{align*}
    Since $(\hat{\vx}, \hat{\vx})$ is an $\epsilon$-well-supported NE, we have
    \begin{align*}
    & u(\ve_j, \hat{\vx}) \geq u(\ve_i, \hat{\vx}) - \epsilon \\
    \Rightarrow  \quad  &1 - \frac{1}{k} + \frac{\delta}{k} + \epsilon \geq 1 - \hat{x}_i + \hat{x}_i \delta \\
        \Rightarrow \quad &\hat{x}_i \geq \frac{1}{k} - \frac{\epsilon}{1 - \delta}.
    \end{align*}
    Thus, for all coordinates $i\in C_k$ , we have $\frac{1}{k} -\frac{\epsilon}{1 - \delta}\leq \hat{x}_i \leq \frac{1}{k} + \frac{\epsilon}{1 - \delta} \cdot (k - 1).$
    It then holds that
    \begin{align*}
        u(\hat{\vx}, \hat{\vx}) & \geq u(\ve_i, \hat{\vx}) - \epsilon \\
        & \geq \left(\frac{1}{k} + \frac{\epsilon}{1 - \delta} \cdot (k - 1)\right) \cdot \delta + \left(1 - \frac{1}{k} - \frac{\epsilon}{1 - \delta} \cdot (k - 1)\right) - \epsilon \\
        & \geq 1 - \frac{1}{k} + \frac{\delta}{k} - \left(\frac{k - \delta}{1 - \delta}\right) \epsilon.
    \end{align*}
\end{proof}

\begin{assumption} \label{assump:parameter}
    For the rest of this subsection, we set the parameters as follows:
    \begin{itemize}
        \item $n \geq k \geq 10;$
        \item $\epsilon < \delta(1 - \delta)/6n^7;$
        \item $\delta \defeq 1/2.$
    \end{itemize}
\end{assumption}

(Using the symbolic value of $\delta$ is more convenient in our derivations below.)

\begin{proof}[Proof of~\Cref{lemma:well_supported_nash_value}]
    We let $\vx^* = \frac{1}{k} \sum_{i \in C_k} \ve_{i}$. The proof considers the following three cases:
    \begin{itemize}
        \item The symmetric $\epsilon$-well-supported NE $(\hat{\vx}, \hat{\vx})$ has support size less than $k$.
        \item The symmetric $\epsilon$-well-supported NE $(\hat{\vx}, \hat{\vx})$ has support size greater than $k$.
        \item The symmetric $\epsilon$-well-supported NE $(\hat{\vx}, \hat{\vx})$ has support size equal to $k$ but is not supported on a clique.
    \end{itemize}
    We proceed to show that for any of the three cases above, we would not be able to have a symmetric $\epsilon$-well-supported NE that achieves value greater than $1 - \frac{1}{k} + \frac{\delta}{k} - \frac{2\delta}{ n^2k^4} + 2 \epsilon$, which is a contradiction.
    \begin{itemize}
        \item For the first case, since the support has size less than $k$, we can find a coordinate $i \in [n]$ such that $\hat{\vx}_i \geq \frac{1}{k - 1}.$ Therefore, the value of playing that action is 
        \begin{align*}
            u(\ve_i, \hat{\vx}) \leq 1 - \frac{1}{k - 1} + \frac{\delta}{k - 1}.
        \end{align*}  
        Since $(\hat{\vx}, \hat{\vx})$ is a symmetric $\epsilon$-well-supported NE, we have
        \begin{align*}
            u(\hat{\vx}, \hat{\vx}) & \leq u(\ve_i, \hat{\vx}) + \epsilon \\
            & \leq 1 - \frac{1}{k - 1} + \frac{\delta}{k - 1} + \epsilon \\
            & \leq 1 - \frac{1}{k} +\frac{\delta}{k} - \frac{2\delta}{n^2k^4} + 2\epsilon,
        \end{align*}
        where in the last step we use~\Cref{assump:parameter}.
        \item For the second case, suppose $|\text{supp}(\hat{\vx})| = m > k$. By \Cref{lemma:not_connected_set}, since the maximum clique size is $k$, we can find a set $\mathcal{S} \subseteq \text{supp}(\hat{\vx})$ with at least $m - k + 1$ elements such that for each coordinate $i \in \mathcal{S}$, we can find a coordinate $j \in \mathcal{S}$ such that $\lineA_{i, j} = \lineA_{j, i} = 0.$ Now we consider the utility of playing action $a_i$ and $a_j$, we have
    \begin{align*}
        & u(\ve_i, \hat{\vx}) = \sum_{l \in \text{supp}(\hat{\vx}) - \{i, j\}} \hat{x}_l \cdot \lineA_{i, l} + \hat{x}_i \cdot \delta + \hat{x}_j  \cdot \lineA_{i, j}, \\
        &u(\ve_j, \hat{\vx}) = \sum_{l \in \text{supp}(\hat{\vx}) - \{i, j\}} \hat{x}_l \cdot \lineA_{j, l} + \hat{x}_j \cdot \delta + \hat{x}_i  \cdot \lineA_{i, j}.
    \end{align*}
    Since $(\hat{\vx}, \hat{\vx})$ is a symmetric $\epsilon$-well-supported NE, we have
    \begin{align*}
        & \sum_{l \in \text{supp}(\hat{\vx}) - \{i, j\}} \hat{x}_l \cdot \lineA_{i, l} + \hat{x}_i \cdot \delta + \hat{x}_j  \cdot\lineA_{i, j} \geq \sum_{l \in \text{supp}(\hat{\vx}) - \{i, j\}} \hat{x}_l \cdot \lineA_{j, l} + \hat{x}_j \cdot \delta + \hat{x}_i  \cdot \lineA_{i, j} - \epsilon\\
        \Rightarrow \quad & \sum_{l \in \text{supp}(\hat{\vx}) - \{i, j\}} \hat{x}_l \cdot \lineA_{i, l}  - \sum_{l \in \text{supp}(\hat{\vx}) - \{i, j\}} \hat{x}_l \cdot \lineA_{j, l}   \geq \hat{x}_j \cdot \delta - \hat{x}_i \cdot \delta - \epsilon.
    \end{align*}
    Now, by moving all the probability mass from action $j$ to action $i$, we form a new strategy $\vx' = \hat{\vx} + \hat{x}_j \cdot \ve_i - \hat{x}_j \cdot \ve_j $ such that
    \begin{align}
        u(\vx', \vx') - u(\hat{\vx}, \hat{\vx}) &=  2 \hat{x}_j\left(\sum_{l \in \text{supp}(\hat{\vx}) - \{i ,j\}} \hat{x}_l \cdot \lineA_{i, l} - \sum_{l \in \text{supp}(\hat{\vx}) - \{i ,j\}} \hat{x}_ml\cdot \lineA_{j, l}\right) + 2 \hat{x}_i \cdot \hat{x}_j \cdot \delta \notag \\
        & \geq  2 \hat{x}_j \cdot ((\hat{x}_j - \hat{x}_i) \delta - \epsilon)+ 2 \hat{x}_i \cdot\hat{x}_j \cdot \delta \notag \\
        & \geq 2 {\hat{x}_j}^2 \delta - 2 \epsilon \label{eq:marginal_gain_when_one_move}. 
    \end{align}
        Suppose there is a coordinate $i \in \mathcal{S}$ such that $\hat{x}_i \geq \frac{1}{nk^2}$ then from \eqref{eq:marginal_gain_when_one_move}, we have 
        \begin{align*}
            u(\hat{\vx}, \hat{\vx}) &\leq u(\vx^*, \vx^*) - 2\cdot(\frac{1}{nk^2})^2 \cdot \delta + 2 \epsilon\\
            & = 1 - \frac{1}{k} + \frac{\delta}{k} - \frac{2\delta}{ n^2k^4} + 2 \epsilon.
        \end{align*}
        If $\hat{x}_i < \frac{1}{nk^2}$ for any $i \in \mathcal{S}$, then there exists a coordinate $l \not \in \mathcal{S}$ such that $\hat{x}_l > \frac{1 - (m - k + 1)\cdot\frac{1}{nk^2}}{m - (m - k + 1)} \geq \frac{1}{k} + \frac{1}{k^2}.$ Then, considering the utility when playing action $a_l$,
        \begin{align}
            u(\ve_l, \hat{\vx}) &\leq 1 \cdot (1 - \frac{1}{k} - \frac{1}{k^2}) + \delta \cdot (\frac{1}{k} + \frac{1}{k^2}) \notag \\
            & = 1 - \frac{1}{k} + \frac{\delta}{k} - \frac{1}{k^2} + \frac{\delta}{k^2} \notag \\
            &< 1 - \frac{1}{k} + \frac{\delta}{k} - \frac{2\delta}{n^2k^4}, \label{eq:case_two_vaue_of_nash}
        \end{align}
        where in~\eqref{eq:case_two_vaue_of_nash} we used~\Cref{assump:parameter}.
        
        Since the $l$th action is played with positive probability and $(\hat{\vx}, \hat{\vx})$ is an $\epsilon$-well-supported NE, we have $u(\hat{\vx}, \hat{\vx}) \leq u(\ve_l, \hat{\vx}) + \epsilon < 1 - \frac{1}{k} + \frac{\delta}{k} - \frac{2\delta}{n^2k^4} + 2\epsilon.$ 
        \item For the third case, since the support is not on a clique, the exists at least coordinates $i, j$ such that $\hat{x}_i >0, \hat{x}_j>0,$ and $\lineA_{i, j} = \lineA_{j, i} = 0.$ Similarly as case two, if $\hat{x}_i \geq \frac{1}{nk^2}$ or $\hat{x}_j \geq \frac{1}{nk^2}$, then we have  
        \begin{align*}
            u(\hat{\vx}, \hat{\vx}) &\leq u(\vx^*, \vx^*) - 2\cdot(\frac{1}{nk^2})^2 \cdot \delta + 2 \epsilon\\
            & = 1 - \frac{1}{k} + \frac{\delta}{k} - \frac{2\delta}{ n^2k^4} + 2\epsilon.
        \end{align*}
        If $\hat{x}_i < \frac{1}{nk^2}$ and $\hat{x}_j < \frac{1}{nk^2}$, then there exists an coordinate $l$ such that $\hat{x}_l \geq \frac{1 - 2 \cdot \frac{1}{nk^2}}{k - 2} > \frac{1}{k} + \frac{1}{k^2}.$ Same as \eqref{eq:case_two_vaue_of_nash}, we conclude that $u(\hat{\vx}, \hat{\vx})$ is at most $1 - \frac{1}{k} + \frac{\delta}{k} - \frac{2\delta}{n^2k^4} + 2\epsilon.$
    \end{itemize}
    The proof is complete.
\end{proof}

We now construct a new symmetric identical payoff game $(\mat{B}, \mat{B})$, where $\mat{B}$ is defined as
    \begin{align}
        \mat{B} = \begin{bmatrix}
         \lineA_{1, 1} & \cdots & \lineA_{1, n}  & r\\
         \vdots & \ddots & \vdots & \vdots \\
         \lineA_{n, 1} & \cdots & \lineA_{n,n} & r\\
         r & \cdots & r  & V\\
    \end{bmatrix}. \label{eq:unique_NP_matrix}
    \end{align}
Above, $V \defeq 1 - \frac{1}{k} + \frac{\delta}{k}$ and $r \defeq 1 - \frac{1}{k} + \frac{\delta}{k} - \frac{\delta}{n^2k^4} + 3\epsilon$. (We caution that the values of $V$ and $r$ have been set differently compared to~\Cref{sec:nonsymmetric}.) Similarly to our derivation in~\cref{eq:three_exact_NE}, it follows that the symmetric (exact) Nash equilibria of this game can only be in one of the following forms:
\begin{enumerate}
        \item $(\vx^*, \vx^*)$ with $\vx^* \defeq \ve_{n+ 1}$; \label{eq:NE_case1}
        \item $(\vx^*, \vx^*)$ with $\vx^* \defeq \frac{1}{k} \sum_{i \in C_k} \ve_i$, where $C_k \subseteq [n]$ is a clique in $G$ of size $k$ \label{eq:NE_case2};
        \item $(\vx^*, \vx^*)$ with $\vx^* \defeq \frac{1}{2} \ve_{n + 1} + \frac{1}{2k} \sum_{i \in C_k} \ve_i$, where $C_k \subseteq [n]$ is a clique in $G$ of size $k$ \label{eq:NE_case3}.
\end{enumerate}

We now show the following lemma.

\begin{lemma} \label{lemma:NP_completeness_for_two_symmetric_well_supported}
    For any $\epsilon$-well-supported NE $(\hat{\vx}, \hat{\vx})$ in game $(\mat{B}, \mat{B})$, it holds that $\norm{\hat{\vx} - \vx^*}_{\infty} \leq 2n^6\epsilon$, where $(\vx^*, \vx^*)$ is an exact NE in one of the three cases above (\ref{eq:NE_case1},\ref{eq:NE_case2},\ref{eq:NE_case3}).
\end{lemma}

\begin{proof}
    First, we observe that since $V > r$, clearly $(\ve_{n +1}, \ve_{n + 1})$ is a $\epsilon$-well-supported NE; in this case, $\norm{\vx' - \vx^*}_{\infty} = 0.$ Furthermore, since $r = 1 - \frac{1}{k} + \frac{\delta}{k} - \frac{\delta}{n^2k^4} + 3 \epsilon$, the game does not attain any $\epsilon$-NE with value less than $1 - \frac{1}{k} + \frac{\delta}{k} - \frac{\delta}{n^2k^4} + 2 \epsilon.$ Suppose the game admits an symmetric $\epsilon$-well-supported NE $(\hat{\vx}, \hat{\vx})$ where $\hat{\vx}$ is supported only on the first $n$ actions. Since $u(\hat{\vx}, \hat{\vx}) \geq 1 - \frac{1}{k} + \frac{\delta}{k} - \frac{\delta}{n^2k^4} + 2 \epsilon > 1 - \frac{1}{k} + \frac{\delta}{k} - \frac{2\delta}{n^2k^4} + 2\epsilon,$ taking  $\vx^*$ as in \eqref{eq:NE_case2}, we conclude that $\norm{\hat{\vx} - \vx^*}_{\infty} \leq \frac{k - 1}{1 - \delta} \epsilon < 2n^6\epsilon$ from \cref{lemma:well_supported_nash_value}. 
    
     We proceed to the case where there is a mixed symmetric $\epsilon$-well-supported Nash $(\hat{\vx}, \hat{\vx})$ between the last action and the rest of actions such that $0 < \hat{x}_{n+1} < 1.$ Denote $\hat{x}_{n + 1} = \alpha$ and $\eta(\cdot)$ to denote the renormalization operation. Since $\left(\eta(\hat{\vx}_{[1 \cdots n]}), \eta(\hat{\vx}_{[1 \cdots n]})\right)$ is a symmetric strategy profile, from~\citet[Proposition 4]{MCLENNAN2010683}, we conclude that there is at least one coordinate $i \in \text{supp}(\hat{\vx}) - \{n+1\}$ such that
     \begin{align*}
         u_A\left(\ve_i, \eta(\hat{\vx}_{[1 \cdots n]})\right) \leq 1 - \frac{1}{k} + \frac{\delta}{k} = V,
     \end{align*}
     where $u_A$ is the utility when the payoff matrix is $\lineA.$ Since $\left(\eta(\hat{\vx}_{[1 \cdots n]}), \eta(\hat{\vx}_{[1 \cdots n]})\right)$ is an $\epsilon$-well-supported NE, we have
     \begin{align}
         & u_B(\ve_i, \hat{\vx}) \geq u_B(\ve_{n + 1}, \hat{\vx}) - \epsilon \notag \\
         \Rightarrow \quad & (1 - \alpha)\cdot V + \alpha \cdot r \geq (1 - \alpha)r + \alpha V - \epsilon \notag \\
         \Rightarrow \quad & \alpha \leq \frac{1}{2} + \frac{\epsilon}{2(V - r)}, \label{eq:lower_bound_on_alpha}
     \end{align}
     where $u_B$ is the utility function when the payoff matrix is $\mat{B}$. Plugging in the value of $V$ and $r$ and using~\Cref{assump:parameter}, we find that $\alpha \leq \frac{1}{2} + 2n^6\epsilon \leq \frac{2}{3}$.
     
     Now, we observe that for any action in the support other than the last action $a_i$, the utility of playing such action $u_B(a_i, \hat{\vx}) = u_A(a_i, \hat{\vx}_{[1 \cdots n]}) + r \alpha$. Since $(\hat{\vx}, \hat{\vx})$ is an $\epsilon$-well-supported Nash Equilibrium, we have $u_B(a_i, \hat{\vx}) \geq \max_{j} u_B(a_j, \hat{\vx}) - \epsilon$ for all pairs $(i, j) \in \text{supp}(\hat{\vx}).$ Since $\alpha \leq \frac{2}{3},$ it follows that $u_A\left(a_i, \eta(\hat{\vx}_{[1 \cdots n]})\right) \geq \max_j u_A\left(a_j, \eta(\hat{\vx}_{[1 \cdots n]})\right) - 3\epsilon$ for any pairs $(i, j) \in \text{supp}(\hat{\vx}) - \{n + 1\}.$ Thus, we conclude that $\left(\eta(\hat{\vx}_{[1 \cdots n]}), \eta(\hat{\vx}_{[1 \cdots n]})\right)$ forms a symmetric $3\epsilon$-well-supported Nash Equilibrium in game $(\lineA, \lineA)$. Further, the value of playing the last action is $(1 - \alpha)r + \alpha V > r$, and so the only situation where there is a mixed Nash between the last action and the rest actions is when $u_A\left(\left(\eta(\hat{\vx}_{[1\cdots n]}), \eta(\hat{\vx}_{[1\cdots n]})\right)\right) \geq r - \epsilon$. Therefore, by \Cref{lemma:value_of_approx_nash_on_clique} and \Cref{lemma:well_supported_nash_value}, we conclude that $\hat{\vx}_{[1 \cdots n]}$ is supported on a clique of size $k.$
     % we conclude that for any coordinate $i \in \text{supp}(\hat{\vx}), i \neq n +1,$ $\frac{1}{k} - \frac{3\epsilon}{1 - \delta}\leq \eta(\hat{\vx})_i \leq \frac{1}{k} + \frac{3k - 3}{1 - \delta} \epsilon.$ 
     There exits at least one coordinate $i \in [n]$, with $0 < \hat{x}_i$ and $ \hat{x}_i \geq \frac{1}{k},$ such that
    \begin{align*}
        u_A(\ve_{i}, \eta(\hat{\vx}_{[1 \cdots n]})) \geq 1 - \frac{1}{k} + \frac{\delta}{k} = V.
    \end{align*}
    Since $(\hat{\vx}, \hat{\vx})$ is an $\epsilon$-well-supported Nash, we have $u(\ve_i, \hat{\vx}) \leq u(\ve_{n +1}, 
    \hat{\vx}) + \epsilon$, and so this gives
    \begin{align}
        % & \begin{cases}
        %     (1 - \alpha) \cdot \left(\delta\left(\frac{1}{k} - \frac{\epsilon}{1 - \delta}\right) + \left(1 - \frac{1}{k} + \frac{\epsilon}{1 - \delta}\right)\right) + \alpha \cdot r \geq (1 - \alpha) \cdot r + \alpha \cdot V - \epsilon \\
        %     (1 - \alpha) \cdot \left(\delta(\frac{1}{k} + \frac{k - 1}{1 - \delta}\epsilon) + \left(1 - \frac{1}{k} - \frac{k-1}{1-\delta} \epsilon \right)\right) + \alpha \cdot r \leq (1 - \alpha) \cdot r + \alpha \cdot V + \epsilon 
        % \end{cases} 
        & (1 - \alpha) \cdot V + \alpha \cdot r \leq (1 - \alpha) \cdot r + \alpha \cdot V + \epsilon \label{eq:alpha_upper_lower_bound}\\
        \Rightarrow \quad &  \alpha \geq \frac{1}{2} - \frac{\epsilon}{2(V - r)}. \label{eq:range_alpha}
        % \Rightarrow \quad & \alpha \geq \frac{1}{2} - \frac{3n^2k^5\epsilon}{4 \delta (1 - \delta)} \label{eq:range_alpha}.
    \end{align}
    Using~\Cref{assump:parameter} and 
    % \begin{itemize}
    %     \item in \eqref{eq:alpha_upper_lower_bound} we use the fact that $\eta(\hat{\vx})_i \leq \frac{1}{k} + \frac{3k - 3}{1 - \delta} \epsilon$ for any $i \in \text{supp}(\hat{\vx}), i \neq n +1;$
    %     \item in \eqref{eq:range_alpha} we plug in the value of $V$ and $r$.
    % \end{itemize}
    combining with~\eqref{eq:lower_bound_on_alpha},
    \begin{align*}
        \frac{1}{2} - 2n^6\epsilon \leq \alpha \leq \frac{1}{2} + 2n^6\epsilon.
    \end{align*}
    By taking $\vx^*$ as in \eqref{eq:NE_case3}, we conclude that $\norm{\hat{\vx} - \vx^*}_{\infty} \leq 2n^6 \epsilon$.
    % Suppose there are two symmetric $\epsilon$-well supported Nash $(\hat{\vx}, \hat{\vx})$ and $(\hat{\vx}', \hat{\vx}')$ such that $\norm{\hat{\vx} - \hat{\vx}'}_{\infty} \geq \frac{1}{3}.$ Since for all $\epsilon$-well-supported Nash, then for any $i \in [n]$, it holds that $\hat{\vx}_i \leq \frac{1}{k} + \frac{k - 1}{1 - \delta} \epsilon \leq \frac{1}{10} + 18 \epsilon < \frac{1}{3}$ and similarly $\hat{\vx}' < \frac{1}{4}.$ Then $\norm{\hat{\vx} - \hat{\vx}'}_{\infty} \geq \frac{1}{3}$ would imply $|\hat{\vx}_{n + 1} - \hat{\vx}'_{n + 1}| \geq \frac{1}{3},$ which further implies that there is a clique of size $k$ in graph $G$. The proof is complete. 
\end{proof}

\begin{theorem}\label{thm:betadefined}
    % It is \NP-complete to answer the following question: Given a two player identical payoff game, can we find two symmetric $\epsilon$-NE that are $O(1)$ away from each other?
    For any $\epsilon$-NE $(\vx, \vx)$ in game $(\mat{B}, \mat{B})$, it holds that $\norm{\vx - \vx^*}_{\infty} \leq n^6 \sqrt{\epsilon}$, where $(\vx^*, \vx^*)$ is an exact NE in one of the forms specified above (\ref{eq:NE_case1},\ref{eq:NE_case2},\ref{eq:NE_case3}).
\end{theorem}

\begin{proof}
    \citet[Lemma 3.2]{Chen09:Settling} showed that from any $\epsilon^2/8$-NE $(\vx, \vy)$ in any two player bimatrix game, one can construct (in polynomial time) an $\epsilon$-well-supported NE $(\vx', \vy')$ such that $\norm{\vx - \vx'}_{\infty} \leq \frac{\epsilon}{4}$ and $\norm{\vy - \vy'}_{\infty} \leq \frac{\epsilon}{4}$. By setting $\epsilon' \defeq \frac{\epsilon^2}{8}$ for the $\epsilon$ defined in ~\Cref{lemma:NP_completeness_for_two_symmetric_well_supported} the proof follows.
    % In game $(\mat{B}, \mat{B})$, if we can find two symmetric $\epsilon'$-NE $(\vx, \vx)$ and $(\vx', \vx')$ such that $\norm{\vx - \vx'}_{\infty} \geq \frac{1}{2} > \frac{1}{3} + \frac{\epsilon}{2}$, this would imply that in polynomial time we can get two symmetric $\epsilon$-well-supported NE $(\hat{\vx}, \hat{\vx})$ and $(\hat{\vx}', \hat{\vx}')$ such that $\norm{\vx - \vx'}_{\infty} \geq \frac{1}{3}.$ The complexity result follows from \cref{lemma:NP_completeness_for_two_symmetric_well_supported}.
\end{proof}

The proof of \Cref{theorem:symmetric-new} follows directly by observing that having two symmetric $\epsilon$-NE $(\vx, \vx)$ and $(\vy, \vy)$ such that $\norm{\vx - \vy}_{\infty} > 2n^6 \sqrt{\epsilon}$ would imply that the game $(\mat{B}, \mat{B})$ has two distinct exact NE, which in turn implies that there is a clique of size $k$ in the graph.

We finally conclude this subsection by stating and proving an auxiliary lemma that was used earlier.

\begin{lemma} \label{lemma:not_connected_set}
    For any graph $G = (V, E)$ with $n$ vertices, if the maximum clique has size $k$, then we can form a set $\mathcal{S} \subseteq V$ of size at least $n - k + 1$ such that for any vertex $i \in \mathcal{S}$, there exists a vertex $j \in \mathcal{S}$ such that $i$ and $j$ are not connected.
\end{lemma}

\begin{proof}
    Suppose the largest set $\mathcal{S}$ we can form has cardinality $|\mathcal{S}| < n - k + 1$, this implies there is a set $\mathcal{S}' = V - \mathcal{S}$ with at least $n - (n - k ) = k$ vertices such that each vertex in $\mathcal{S'}$ is connected to all other vertices in $G$. However if this is the case, there is at least one vertex $v \not \in \mathcal{S'}$ that are connected to all vertices in $\mathcal{S'}$. This contradicts the fact that maximum clique has size $k$.
\end{proof}



\subsection{Proofs from Section~\ref{sec:teamzero}}
\label{sec:proofs4}

We conclude with the missing proofs from~\Cref{sec:teamzero}.

\begin{proof}[Proof of~\Cref{lemma:close-teams}]
    For the sake of contradiction, suppose that $\|\vx^* - \vy^*\|_\infty > 2 \epsilon$. Without loss of generality, let us further assume that there is some coordinate $i$ such that $x_i^* - y_i^* > 2 \epsilon$. The payoff difference for Player $\hvz$ when playing the $i$th action compared to action $a_{2n + 1}$ reads
    \begin{align*}
        u(\vx^*, \vy^*, \vz^*, \hat{\vx}^*, \hat{\vy}^*, a_i) - u(\vx^*, \vy^*, \vz^*, \hat{\vx}^*, \hat{\vy}^*, a_{2n + 1}) & = \frac{|\Amin|}{\epsilon} \cdot (x^*_i - y^*_i) - |\Amin| \\
        & > |\Amin|.
    \end{align*}
    By~\Cref{lemma:mass_on_small_reward}, it follows that $\hvz_{2n+1}^* < \frac{\epsilon^2}{|\mat{A}_{\min}|} \leq \epsilon^2$. Moreover, given that $(\vx^*, \vy^*, \vz^*, \hvx^*, \hvy^*, \hvz^*)$ is an $\epsilon^2$-Nash equilibrium, we have
\begin{align*}
        u(\vx^*, \vy^*, \vz^*, \hat{\vx}^*, \hat{\vy}^*, \hat{\vz}^*) & \geq u(\vx^*, \vy^*, \vz^*, \hat{\vx}^*, \hat{\vy}^*, a_i) - \epsilon^2 \\
        & = \langle \vx^*, \mat{A} \vy^* \rangle  - \langle \hvx^*, \mat{A} \hvy^* \rangle + \langle \vx^*, \mat{C} \hvx^* \rangle  + \frac{|\mat{A}_{\min}|}{\epsilon}(x^*_i - y^*_i) - \delta(\hvx, \hvy, \vz) - \epsilon^2 \\
        & \geq \Amin - \langle \hvx^*, \mat{A} \hvy^* \rangle + \langle \vx^*, \mat{C} \hvx^* \rangle + 2 |\mat{A}_{\min}| - \delta(\hvx, \hvy, \vz) - \epsilon^2 \\
        & = |\mat{A}_{\min}| - \langle \hvx^*, \mat{A} \hvy^* \rangle  + \langle \vx^*, \mat{C} \hvx^* \rangle   - \delta(\hvx, \hvy, \vz) - \epsilon^2.
\end{align*}
Now, considering the deviation of Player $\vy$ to $\vy' \defeq \vx^*$,
\begin{align*}
        u(\vx^*, \vy', \vz^*, \hvx^*, \hvy^*, \hvz^*) - u(\vx^*, \vy^*, \vz^*, \hvx^*, \hvy^*, \hvz^*) & \leq \langle \vx^*, \mat{A} \vx^* \rangle + \epsilon^2 |\Amin| - |\Amin| + \epsilon^2 \\
        &\leq -2 + 2 \epsilon^2 \\
        &< - \epsilon^2,
    \end{align*}
    which contradicts the fact that $(\vx^*, \vy^*, \vz^*, \hat{\vx}^*, \hat{\vy}^*, \hat{\vz}^*)$ is an $\epsilon^2$-Nash equilibrium. We conclude that $\norm{\vx^* - \vy^*}_{\infty} \leq 2 \epsilon$; the proof for the fact that $\norm{\hat{\vx}^* - \hat{\vy}^*}_{\infty} \leq 2\epsilon$ follows similarly.
\end{proof}

\begin{proof}[Proof of~\Cref{lemma:smallzteam}]
    We will prove that $\vz_j \leq 9 \epsilon$ for all $j \in [2n]$; the corresponding claim for Player $\hvz$ follows similarly. Fix $i \in [n]$. \Cref{lemma:close-teams} shows that $|y_i^* - x_i^* | \leq 2 \epsilon$. We shall consider two cases.

    First, suppose that $|y_i^* - x_i^*| \leq \nicefrac{\epsilon}{2}$. Then,
    \begin{align*}
        u(\vx^*, \vy^*, \vz^*, \hat{\vx}^*, \hat{\vy}^*, a_{2n + 1}) - u(\vx^*, \vy^*, \vz^*, \hat{\vx}^*, \hat{\vy}^*, a_i) &\geq |\mat{A}_{\min}| - \frac{|\mat{A}_{\min}|}{\epsilon} \cdot (x^*_i - y^*_i)\\
        & \geq \frac{|\mat{A}_{\min}|}{2} \geq \frac{1}{2}
    \end{align*}
    By~\Cref{lemma:mass_on_small_reward}, it follows that $\hat{z}_i \leq 2 \epsilon^2$, and similar reasoning yields $\hat{z}_{n + i} \leq 2 \epsilon^2$. On the other hand, suppose that $|y_i^* - x_i^*| > \nicefrac{\epsilon}{2}$. Without loss of generality, we can assume that $y_i^* - x_i^* \geq 0$; the contrary case is symmetric. Since $\vx^* \in \Delta^n$ and $\vy^* \in \Delta^n$, there is some coordinate $j \in [n]$ such that $y^*_j - x^*_j < 0$. As before, by \Cref{lemma:mass_on_small_reward}, it follows that $\hat{z}^*_i \leq \epsilon^2$ and $\hat{z}^*_{n + j} \leq \epsilon^2$. Now, we consider the deviation
    \begin{align*}
        \Delta^n \ni \vy' = \vy^* + (x_i^* - y_i^*) \ve_i + (y_i^* - x^*_i) \ve_j.
    \end{align*}
    Then, we have
    \begin{align*}
        &u(\vx^*, \vy', \vz^*, \hat{\vx}^*, \hat{\vy}^*, \hat{\vz}^*) - u(\vx^*, \vy^*, \vz^*, \hat{\vx}^*, \hat{\vy}^*, \hat{\vz}^*)  \\ 
        &=  \langle \vx^*, \mat{A} (\vy' - \vy^*) \rangle + \frac{|\Amin|}{\epsilon} \left(\hat{z}^*_i (x^*_i - y'_i - (x^*_i - y^*_i)) + \hat{z}^*_j(x^*_j - y'_j - (x^*_j - y^*_j)) \right) \\
        &\phantom{===============} + \frac{|\mat{A}_{\min}|}{\epsilon} \left(\hat{z}^*_{n+i} ( y'_{i} -x^*_{i} - (y^*_{i} - x^*_{i})) + \hat{z}^*_{n+j}(y'_{j} - x^*_{j} - (y^*_{j} - x^*_{j})) \right) \\
        &= \langle \vx^*, \mat{A} (\vy' - \vy^*) \rangle + \frac{|\Amin|}{\epsilon} \left(\hat{z}^*_i (y_i^* - y'_i) + \hat{z}^*_j (y^*_j - y'_j) + \hat{z}^*_{n + i} (y'_{i} - y^*_{i})  + \hat{z}^*_{n + j} (y'_{ j} - y^*_{j})\right) \\
        &\leq 4\epsilon |\Amin| + \frac{|\Amin|}{\epsilon}\left(\epsilon^2 \cdot 2\epsilon + \hat{z}^*_{n + i} \cdot \left(- \frac{\epsilon}{2} \right) + \epsilon^2 \cdot 2\epsilon \right) \\
        &\leq -\left(\frac{1}{2}\hat{z}^*_{n +i} - 4 \epsilon^2 - 4\epsilon\right) |\Amin|.
    \end{align*}
    At the same time, since $(\vx^*, \vy^*, \vz^*, \hat{\vx}^*, \hat{\vy}^*, \hat{\vz}^*)$ is an $\epsilon^2$-Nash equilibrium, we have
    \begin{align*}
        u(\vx^*, \vy', \vz^*, \hat{\vx}^*, \hat{\vy}^*, \hat{\vz}^*) - u(\vx^*, \vy^*, \vz^*, \hat{\vx}^*, \hat{\vy}^*, \hat{\vz}^*) \geq - \epsilon^2.
    \end{align*}
    Thus,
    \begin{equation*}
        \left(\frac{1}{2}\hat{z}^*_{n +i} - 4 \epsilon^2 -4\epsilon\right) |\mat{A}_{\min}| \leq \epsilon^2 \implies \hat{z}^*_{n +i} \leq 9 \epsilon.
    \end{equation*}
    We conclude that $\hat{z}^*_i \leq \epsilon^2$ and $\hat{z}^*_{n + i} \leq 9\epsilon$. The case where $x_i^* - y_i^* \geq 0$ can be treated similarly.
\end{proof}

\begin{proof}[Proof of~\Cref{theorem:team-hard}]
    Suppose that $(\vx^*, \vy^*, \vz^*, \hat{\vx}^*, \hat{\vy}^*, \hat{\vz}^*)$ is an $\epsilon^2$-Nash equilibrium. We have that for any $\vy' \in \Delta^n$,
    \begin{equation}
        \label{eq:dev1}
        \langle \vx^*, \mat{A} \vy^* \rangle \leq \langle \vx^*, \mat{A} \vy' \rangle + \frac{|\Amin|}{\epsilon} \left(\sum_{i = 1}^n \hat{z}^*_i (y^*_i - y'_i) + \hat{z}^*_{n + i} (y'_i - y^*_i)\right) + \epsilon^2.
    \end{equation}
    Moreover, considering a deviation from $\vx^*$ to $\vy'$,
    \begin{equation}
        \label{eq:dev2}
        \langle \vx^*, \mat{A} \vy^* \rangle + \langle \vx^*, \mat{C} \hvx^* \rangle \leq \langle \vy', \mat{A} \vy^* \rangle + \langle \vy', \mat{C} \hvx^* \rangle + \frac{|\Amin|}{\epsilon} \left(\sum_{i = 1}^n \hat{z}^*_i (y'_i - x^*_i) + \hat{z}^*_{n + i} (x^*_i - y'
        _i)\right) + \epsilon^2.
    \end{equation}
    Summing~\eqref{eq:dev1} and~\eqref{eq:dev2},
    \begin{align}
        2 \langle \vx^*, \mat{A} \vy^* \rangle + \langle \vx^*, \mat{C} \hvx^* \rangle & \leq \langle \vy', \mat{A} (\vx^* + \vy^*) \rangle + \langle \vy', \mat{C} \hvx^* \rangle + \frac{|\Amin|}{\epsilon} \left(\sum_{i = 1}^n \hat{z}^*_i (y^*_i - x^*_i) + \hat{z}^*_{n + i} (x^*_i - y^*
        _i)\right) + 2\epsilon^2 \notag \\
        & \leq 2 \langle \vx^*, \mat{A} \vy' \rangle + \langle \vy', \mat{C} \hvx^* \rangle + 2 \epsilon n |\Amin| + \frac{|\Amin|}{\epsilon} ( 2n \cdot 9 \epsilon \cdot 2\epsilon) + 2\epsilon^2 \notag \\
        & \leq 2 \langle \vx^*, \mat{A} \vy' \rangle + \langle \vy', \mat{C} \hvx^* \rangle + (38n + 2) |\Amin| \epsilon \label{eq:team_game_VI_sum}.
    \end{align}
    Moreover, 
    \begin{align}
        \langle \vx^*, \mat{A} \vx^* \rangle =  \langle \vx^*, \mat{A} \vy^* \rangle + \langle \vx^*, \mat{A} (\vx^* - \vy^*) \rangle \leq \langle \vx^*, \mat{A} \vy^* \rangle + 2 |\Amin| n \epsilon \label{eq:team_game_show_symmetric_nash}.
    \end{align}
    Combining~\eqref{eq:team_game_VI_sum} and~\eqref{eq:team_game_show_symmetric_nash}, we get that for all $\vy' \in \Delta^n$,
    \begin{align}
        \langle \vx^*, \mat{A} \vx^* \rangle + \frac{1}{2} \langle \vx^*, \mat{C} \hat{\vx}^* \rangle \leq \langle \vy', \mat{A} \vx^* \rangle + \frac{1}{2} \langle \vy', \mat{C} \hat{\vx}^* \rangle + (21n + 1 ) |\mat{A}_{\min}|\epsilon. \label{eq:team_game_VI1}
    \end{align}
    Similarly, we can show that for all $\hat{\vx}' \in \Delta^n$,
    \begin{align}
        \langle -\hat{\vx}^*, \mat{A} \hat{\vx}^* \rangle + \frac{1}{2} \langle \vx^*, \mat{C} \hat{\vx}^* \rangle \geq - \langle \hat{\vx}', \mat{A} \hat{\vx}^* \rangle + \frac{1}{2} \langle \vx^*, \mat{C} \hat{\vx}' \rangle - (21n + 1) |\Amin| \epsilon. \label{eq:team_game_VI2}
    \end{align}
    Taking $\vy' = \hat{\vx}'$ in~\eqref{eq:team_game_VI1} and summing with~\eqref{eq:team_game_VI2}, we get that for all $\hat{\vx}' \in \Delta^n$,
\begin{equation*}
        \langle \hat{\vx}', \mat{A} \vx^* \rangle + 
\langle \hat{\vx}', \mat{A} \hat{\vx}^* \rangle + \frac{1}{2} (\langle \hat{\vx}', \mat{C} \hat{\vx}^* \rangle +  \langle \hat{\vx}', \mat{C} \vx^* \rangle) + (42n + 2) |\mat{A}_{\min}| \epsilon \geq \langle \vx^*, \mat{A} \vx^* \rangle + \langle \hat{\vx}^*, \mat{A} \hat{\vx}^* \rangle,
\end{equation*}
    where we used the fact that $\mat{C}$ is skew-symmetric. Now, using the fact that the Nash equilibrium is symmetric, so that $\vx^* = \hvx^*$, we have
\begin{align}
        \langle \hat{\vx}', \mat{A} \vx^* \rangle + \frac{1}{2} \langle \hat{\vx}', \mat{C} \vx^* \rangle + (21n + 1) |\Amin| \epsilon &\geq \langle \vx^*, \mat{A} \vx^* \rangle \notag \\
        &\geq \langle \vx^*, \mat{A} \vx^* \rangle + \frac{1}{2} \langle \vx^*, \mat{C} \vx^* \rangle. \label{eq:team_game_final_eq3}
    \end{align}
    Setting $\mat{A} \defeq - \frac{1}{2} (\mat{R} + \mat{R}^\top)$ and $\mat{C} \defeq \mat{R}^\top - \mat{R}$, \eqref{eq:team_game_final_eq3} shows that 
    \begin{equation*}
        \langle \hvx', \mat{R} \vx^* \rangle \leq \langle \vx^*, \mat{R} \vx^* \rangle + (21n + 1) |\Amin| \epsilon
    \end{equation*}
    for any $\hvx' \in \Delta^n$; \emph{ergo}, $(\vx^*, \vx^*)$ is a symmetric $(21n + 1)  |\Amin| \epsilon$-Nash equilibrium of the symmetric (two-player) game $(\mat{R}, \mat{R}^\top)$, and the proof follows from~\Cref{theorem:PPAD_for _symmetric}.
\end{proof}

% \begin{lemma} \label{lemma:not_connected_set}
%     For any graph $G = (V, E)$ with $n$ vertices, if the maximum clique has size $k$, then we can form a set $\mathcal{S} \subseteq V$ of size at least $n - k + 1$ such that for any vertex $i \in \mathcal{S}$, there exists a vertex $j \in \mathcal{S}$ such that $i$ and $j$ are not connected.
% \end{lemma}

% \begin{proof}
%     Suppose the largest set $\mathcal{S}$ we can form has cardinality $|\mathcal{S}| < n - k + 1$, this implies there is a set $\mathcal{S}' = V - \mathcal{S}$ with at least $n - (n - k ) = k$ vertices such that each vertex in $\mathcal{S'}$ is connected to all other vertices in $G$. However if this is the case, there is at least one vertex $v \not \in \mathcal{S'}$ that are connected to all vertices in $\mathcal{S'}$. This contradicts the fact that maximum clique has size $k$.
% \end{proof}

