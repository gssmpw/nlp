\section{Complexity of adversarial team games}

We begin by examining equilibrium computation in adversarial team games.

\subsection{$\CLS$-completeness for 3-player games}
\label{sec:cls}

Computing $\epsilon$-Nash equilibria in adversarial team games was placed in $\CLS$ by~\citet{Anagnostides23:Algorithms}, but whether $\CLS$ tightly characterizes the complexity of that problem remained open---that was only known when the number of players is large, so that the hardness result of~\citet{Babichenko21:Settling} can kick in. Our reduction here answers this question in the affirmative. 

We rely on a recent hardness result of~\citet{ghosh2024complexitysymmetricbimatrixgames} concerning symmetric, two-player games with identical payoffs. We summarize their main result below.

\begin{theorem}[\citealp{ghosh2024complexitysymmetricbimatrixgames}]
    \label{theorem:hard-2cls}
    Computing an $\epsilon$-Nash equilibrium in a symmetric, identical-payoffs, two-player game is~$\CLS$-complete.
\end{theorem}

Now, let $\mat{A} \in \mathbb{Q}^{n \times n}$ be the common payoff matrix of a two-player game, which satisfies $\mat{A} = \mat{A}^\top$ so that the game is symmetric. Without loss of generality, we will assume that $\mat{A}_{i, j} \leq -1$ for all $i, j \in [n]$. We denote by $\Amin$ and $\Amax$ the minimum and maximum entry of $\mat{A}$, respectively (which satisfy $\Amax, \Amin \leq -1$). The basic idea of our proof is to suitably use the adversarial player so as to force the other two players to play roughly the same strategy (\Cref{lemma:closeness}), while (approximately) maintaining the structure of the game (\Cref{lemma:small_z}).

\paragraph{Definition of the adversarial team game} Based on $\mat{A}$, we construct a $3$-player adversarial team game as follows. The utility function of the adversary reads
\begin{equation}
    \label{eq:util-atg}
    u(\vx, \vy, \vz) \defeq  \langle \vx, \mat{A} \vy \rangle  + \frac{|\Amin|}{\epsilon} \sum_{i = 1}^n \left(z_i(x_i - y_i) + z_{n + i} (y_i - x_i)\right) + z_{2n+1} |\Amin|.
\end{equation}
The adversary selects a strategy $ \vz \in \Delta^{2n + 1}$, while the team players, who endeavor to minimize~\eqref{eq:util-atg}, select strategies $\vx \in \Delta^n$ and $\vy \in \Delta^n$, respectively. (While the range of the utilities in~\eqref{eq:util-atg} grows with $\nicefrac{1}{\epsilon}$, normalizing to $[-1, 1]$ maintains all of the consequences by suitably adjusting the approximation.)

We begin by stating a simple auxiliary lemma (\Cref{sec:proofs1} contains the proof).

\begin{restatable}{lemma}{wellsuppo}
    \label{lemma:mass_on_small_reward}
    Let $(\vx_i^*, \vx^*_{-i})$ be an $\epsilon^2$-Nash equilibrium of a normal-form game, and $a_j$ any action of player $i$. If $u_i(a_k, \vx^*_{-i}) \leq u_i(a_j, \vx^*_{-i}) - c$ for some $c>0$ and $k \in [n]$, then $x^*_i(a_k) \leq \nicefrac{\epsilon^2}{c}.$
\end{restatable}

In a so-called \emph{well-supported} Nash equilibrium, every strategy $k$ that satisfies the precondition of~\Cref{lemma:mass_on_small_reward} would be played with probability $0$, but it will be easy to account for that slack.

The first important lemma establishes that, in equilibrium, $\vx \approx \vy$. The basic argument proceeds as follows. By construction of the adversarial team game~\eqref{eq:util-atg}, the adversary would be able to secure a large payoff whenever there is a coordinate $i \in [n]$ such that $|x_i - y_i| \gg 0$---by virtue of the second term in~\eqref{eq:util-atg}. But that cannot happen in equilibrium, for Player $\vx$ (or symmetrically Player $\vy$) can simply neutralize that term in the adversary's utility by playing $\vx = \vy$. We provide the precise, quantitative argument below.

\begin{lemma}[Equilibrium forces symmetry] \label{lemma:closeness}
    Consider an $\epsilon^2$-Nash equilibrium $(\vx^*, \vy^*, \vz^*)$ of the adversarial team game~\eqref{eq:util-atg} with $\epsilon^2 \leq \nicefrac{1}{2}$. Then, $\norm{\vx^* - \vy^*}_{\infty} \leq 2\epsilon$.
\end{lemma}

\begin{proof}
    For the sake of contradiction, suppose that $x_i^* - y_i^* > 2 \epsilon$ for some $i \in [n]$ (the case where $y_i^* - x_i^* > 2 \epsilon$ is symmetric, and can be treated analogously). Player $\vz$ could then choose action $a_i$ (with probability $1$), which secures a utility of
    \begin{align*}
        u(\vx^*, \vy^*, a_i) 
        & = \langle \vx^*, \mat{A} \vy^* \rangle  + \frac{|\Amin|}{\epsilon} \cdot (x^*_i - y^*_i) > \langle \vx^*, \mat{A} \vy^* \rangle + 2 |\Amin|,
    \end{align*}
    since $x_i^* - y_i^* > 2 \epsilon$. At the same time, Player $\vz$ could choose action $a_{2n+1}$, which secures a utility of $u(\vx^*, \vy^*, a_{2n+1}) = \langle \vx^*, \mat{A} \vy^* \rangle + |\Amin|$. So,
    \begin{equation*}
        u(\vx^*, \vy^*, a_i) - u(\vx^*, \vy^*, a_{2n+1}) \geq |\Amin|.
    \end{equation*}
    Applying~\Cref{lemma:mass_on_small_reward},
    \begin{align}
        \label{ineq:z}
    z^*_{2n+1} \leq \frac{\epsilon^2}{|\Amin|} \leq \epsilon^2.
    \end{align}
    Also, using the fact that $(\vx^*, \vy^*, \vz^*)$ is an $\epsilon^2$-Nash equilibrium,
    \begin{align}
    u(\vx^*,\vy^*,\vz^*) &\geq u(\vx^*,\vy^*,a_i) - \epsilon^2 \notag \\
    &\geq \langle \vx^*, \mat{A} \vy^* \rangle + 2|\Amin| -\epsilon^2 \notag \\
    &\geq \Amin + 2|\Amin|-\epsilon^2
    \notag \\
    &= |\Amin| -\epsilon^2, \label{ineq:utility}
    \end{align}
    since we have assumed that $\Amin < 0$. Now, consider the deviation of Player $\vx$ (from strategy $\vx^*$) to $\vx' \defeq \vy^*$. Then, $u(\vx', \vy^*, \vz^*) = \langle \vy^*, \mat{A} \vy^* \rangle + z^*_{2n+1} \cdot |\Amin|$. Thus, combining with~\eqref{ineq:utility} and~\eqref{ineq:z},
    \begin{align}
        u(\vx', \vy^*, \vz^*) - u(\vx^*, \vy^*, \vz^*) & \leq \langle \vy^*, \mat{A} \vy^* \rangle  + z_{2n+1}^* \cdot |\Amin| - |\Amin| + \epsilon^2 \notag \\
        & \leq \langle \vy^*, \mat{A} \vy^* \rangle + \epsilon^2 \cdot |\Amin| -  |\Amin | + \epsilon^2 \notag \\
        & \leq \Amax - |\Amin| + \epsilon^2(|\Amin|+1) \notag \\&\leq -1 < - \epsilon^2,  \label{eq:contract}
    \end{align}
    where we used that $\Amax, \Amin \leq -1$ and $\epsilon^2 \leq \nicefrac{1}{2}$. But~\eqref{eq:contract} contradicts the fact that $(\vx^*, \vy^*, \vz^*)$ is an $\epsilon^2$-Nash equilibrium since deviating to $\vx'$ yields a utility improvement (equivalently, decrease in cost) strictly larger than $\epsilon^2$. This completes the proof.
\end{proof}

Having established that $\vx \approx \vy$, the next step is to make sure that the adversarial player does not distort the original game by much. In particular, we need to make sure that the effect of the second term in~\eqref{eq:util-atg} is negligible. We do so by showing that $\vz_{2n+1} \approx 1$ (\Cref{lemma:small_z}).

The argument here is more subtle; roughly speaking, it goes as follows. Suppose that $z_{i} \gg 0$ or $z_{n+i} \gg 0$ for some $i \in [n]$. Since Player $\vz$ is approximately best responding, it would then follow that $|y_i^* - x_i^*| \gg 0$---otherwise Player $\vz$ would prefer to switch to action $2n+1$. But, if $|y_i^* - x^*_i| \gg 0$, Player $\vx$ could profitably deviate by reallocating probability mass by either removing or adding to $i$ (depending on whether $y_i^* - x_i^* > 0$), which leads to a contradiction.

\begin{lemma}[Most probability mass in $a_{2n+1}$] \label{lemma:small_z}
    Given any $\epsilon^2$-Nash equilibrium $(\vx^*, \vy^*, \vz^*)$ of the adversarial team game~\eqref{eq:util-atg} with $\epsilon \leq \nicefrac{1}{10}$, $z_j \leq 9 \epsilon$ for all $j \in [2n]$. In particular, $z_{2n+1} \geq 1 - 18 n \epsilon$.
\end{lemma}

\begin{proof}
    \Cref{lemma:closeness} shows that $\norm{\vx^* - \vy^*}_{\infty} \leq 2 \epsilon$. Let $i \in [n]$. We assume that $i$ is such that $x_i^* - y_i^* \geq 0$; the contrary case is symmetric. We consider two cases. First, suppose that $| x_i^* - y_i^*| \leq \nicefrac{\epsilon}{2}$. Then, we have
    \begin{align*}
        u(\vx^*, \vy^*, a_{2n+1}) - u(\vx^*, \vy^*, a_i) & \geq |\Amin| - \frac{|\Amin|}{\epsilon} ( x^*_i - y^*_i ) \\
        & \geq \frac{1}{2} |\Amin| \geq \frac{1}{2}.
    \end{align*}
    Thus, by~\cref{lemma:mass_on_small_reward}, we conclude that $z_i \leq 2 \epsilon^2$. Similarly,
    \begin{equation*}
        u(\vx^*, \vy^*, a_{2n+1}) - u(\vx^*, \vy^*, a_{n+i}) \geq |\Amin| - \frac{|\Amin|}{\epsilon} (  y^*_i - x_i^* ) \geq |\Amin| \geq 1,
    \end{equation*}
    since $x_i^* - y_i^* \geq 0$. Again, \Cref{lemma:mass_on_small_reward} implies that $z_{n+i} \leq \epsilon^2$.
    
    It thus suffices to treat the case where $| x_i^* - y_i^* | > \nicefrac{\epsilon}{2}$ (assuming that $x_i^* - y_i^* \geq 0$). It follows that there exists $j \in [n]$ such that $x^*_j - y^*_j < 0$. In addition, we observe that $u(\vx^*, \vy^*, a_{2n+1}) = \langle \vx^*, \mat{A} \vy^* \rangle + |\Amin| \geq \langle \vx^*, \mat{A} \vy^* \rangle + 1$, whereas $u(\vx^*, \vy^*, a_j) < \langle \vx^*, \mat{A} \vy^* \rangle$ and $u(\vx^*, \vy^*, a_{n+i}) < \langle \vx^*, \mat{A} \vy^* \rangle$. As a result, \Cref{lemma:mass_on_small_reward} implies that $z^*_{n+i} \leq \epsilon^2$ and $z^*_{j} \leq \epsilon^2$. 
    
    Now, consider deviation
    \begin{align*}
        \Delta^n \ni \vx' = \vx^* + (y^*_i - x^*_i) \ve_i + (x^*_i - y^*_i) \ve_j;
    \end{align*}
    that is, $\vx'$ is the strategy that results from $\vx$ by reallocating $(x^*_i - y^*_i)$ probability mass from action $a_i$ to action $a_j$. The difference $u(\vx', \vy^*, \vz^*) - u(\vx^*, \vy^*, \vz^*)$ can be expressed as
    \begin{align}
        &\langle \vx' - \vx^*, \mat{A} \vy^* \rangle + \frac{|\Amin|}{\epsilon} \left(z^*_i (x_i' - x_i^*) + z^*_j (x'_j - x^*_j) + z^*_{n + i}(x^*_i - x_i') + z^*_{n + j}(x^*_j - x'_j)\right) \notag \\
        &= \langle \vx' - \vx^*, \mat{A} \vy^* \rangle + \frac{|\Amin|}{\epsilon} \left(z^*_i (y_i^* - x_i^*) + z^*_j (x_i^* - y_i^*) + z^*_{n + i}(x^*_i - y_i^*) + z^*_{n + j}(y_i^* - x_i^* )\right) \notag \\
        &\leq 4 \epsilon |\Amin| + \frac{|\Amin|}{\epsilon} \left(z^*_i \cdot \left( - \frac{\epsilon}{2} \right) + \epsilon^2 \cdot 2 \epsilon + \epsilon^2 \cdot 2 \epsilon\right) \label{eq:cancel1}\\
        &\leq - \left( \frac{1}{2}z^*_i - 4 \epsilon^2-4 \epsilon \right) |\Amin|, \label{eq:cancel2}
    \end{align}
    where~\eqref{eq:cancel1} uses the following:
    \begin{itemize}
        \item $ y_i^* - x_i^* \leq - \nicefrac{\epsilon}{2}$; 
        \item $x_i^* - y_i^* \leq 2 \epsilon$ (\Cref{lemma:closeness});
        \item $z^*_{n + i} \leq \epsilon^2$ and $z_j^* \leq \epsilon^2$; and 
        \item $\langle \vx' - \vx^*, \mat{A} \vy^* \rangle \leq \| \vx' - \vx^* \|_1 \|\mat{A} \vy^* \|_\infty \leq 2 | x_i^* - y_i^* | |\Amin| \leq 4 \epsilon |\Amin| $ (since $\mat{A}$ has negative entries); 
    \end{itemize}
    Moreover, given that $(\vx^*, \vy^*, \vz^*)$ is assumed to be an $\epsilon^2$-Nash equilibrium, we have 
    \begin{align}
        \label{eq:Nashcond}
        - u(\vx', \vy^*, \vz^*) + u(\vx^*, \vy^*, \vz^*) \leq \epsilon^2.
    \end{align}
    (The utility of Player $\vx$ is given by $-u$.) Combining~\eqref{eq:Nashcond} and~\eqref{eq:cancel2},
    \begin{align}
         &\left(\frac{1}{2} z^*_i - 4 \epsilon^2-4 \epsilon\right) |\Amin| \leq  \epsilon^2 \\
        \Rightarrow \quad  z^*_i & \leq 8 \epsilon + 10\epsilon^2 \leq  9 \epsilon \textrm{ for }\epsilon \leq \frac{1}{10}.
    \end{align}
    In summary, when $x_i^* - y_i^* \geq 0$, we have shown that $z_i^* \leq 9 \epsilon$ and $z_{n+i}^* \leq \epsilon^2$. The case where $y_i^* - x_i^* \geq 0$ can be treated similarly.
\end{proof}

We next combine~\Cref{lemma:closeness,lemma:small_z} to complete the reduction from symmetric two-player games with common payoffs to $3$-player adversarial team games.

\begin{restatable}{theorem}{teamadvtosymme}
    \label{thm:teamadvtosymme}
    Given any $\epsilon^2$-Nash equilibrium $(\vx^*, \vy^*, \vz^*)$ in the adversarial team game~\eqref{eq:util-atg}, with $\epsilon \leq \nicefrac{1}{10}$, $(\vy^* , \vy^*)$ is a symmetric $(21 n + 1) |\Amin| \epsilon$-Nash equilibrium of the symmetric, two-player game $(\mat{A}, \mat{A})$ (that is, $\mat{A} = \mat{A}^\top$).
\end{restatable}

The proof is straightforward and is included in~\Cref{sec:proofs1}. \Cref{theorem:cls-completeness} restates the main complexity implication of~\Cref{theorem:hard-2cls}.

\maincls*

\begin{proof}
    $\CLS$-hardness follows directly from~\Cref{thm:teamadvtosymme} and~\Cref{theorem:hard-2cls} (due to~\citet{ghosh2024complexitysymmetricbimatrixgames}). The inclusion was shown by~\citet{Anagnostides23:Algorithms}.
\end{proof}

\subsection{A 3-player adversarial team game with a unique irrational Nash}

So far, in~\Cref{sec:cls}, we have focused on the complexity of computing $\epsilon$-Nash equilibria in adversarial team games; this type of guarantee is referred to as a \emph{weak} approximation, as it does not in general guarantee closeness---in a geometric sense---to an exact Nash equilibrium~\citep{Etessami10:On}. To complement our results in~\Cref{sec:cls}, here we come up with an adversarial team game (with rational utilities) in which the only exact Nash equilibrium is irrational, thereby precluding even representing an exact Nash equilibria in such games; this is a well-known fact in general $3$-player games~\citep{Etessami10:On}, but it was hitherto unknown whether this holds under the utility constraints imposed in an adversarial team game.

We consider a $3$-player adversarial team game in which the utility function of the adversary $u : \{1, 2\} \times \{1, 2\} \times \{1, 2\} : (\vx, \vy, \vz) \mapsto \R$ reads

\begin{center}
\renewcommand{\arraystretch}{2}
\begin{tabular}{ c|cccc } 
 \diagbox[height = 1.2cm]{$\vz$}{$(\vx, \vy)$}& (1, 1) & (1, 2) & (2, 1) & (2, 2) \\ [0.5ex]
 \toprule
 1 & $1$ & $3$ & $\frac{99}{100}$ & $- \frac{1}{100}$ \\ [0.5ex]
 2 & $\frac{9}{10}$ & $-\frac{1}{10}$ & 1 & $3$ \\ [0.5ex]
 \bottomrule
\end{tabular}
\end{center}

\begin{theorem}
    \label{theorem:irrational}
    The above adversarial team game has a unique Nash equilibrium that is supported on irrationals.
\end{theorem}

The proof of this result makes use of a characterization of Nash equilibria in $2 \times 2$ two-player zero-sum games, stated below; for the proof, we refer to, for example, \citet[Theorem 1.2]{sun2022propertiesnashequilibrium2}.

\begin{lemma}
    \label{lemma:2x2}
    Let $\mat{A} \in \R^{2 \times 2}$ such that
    \begin{equation*}
        ( \mat{A}_{1, 1} - \mat{A}_{1, 2} )( \mat{A}_{2, 2} - \mat{A}_{2, 1}) > 0 \text{ and } ( \mat{A}_{1, 1} - \mat{A}_{2, 1} )( \mat{A}_{2, 2} - \mat{A}_{1, 2}) > 0.
    \end{equation*}
    Then, the two-player zero-sum game $\min_{\vx \in \Delta^2} \max_{\vz \in \Delta^2} \langle \vx, \mat{A} \vz \rangle$ admits a unique (exact) Nash equilibrium with value
    \begin{equation*}
        v \defeq \frac{ \mat{A}_{1, 1} \mat{A}_{2, 2} - \mat{A}_{1, 2} \mat{A}_{2, 1}}{ \mat{A}_{1, 1} - \mat{A}_{1, 2} - \mat{A}_{2, 1} + \mat{A}_{2, 2}}.
    \end{equation*}
    Furthermore, the unique Nash equilibrium $(\vx^*, \vz^*)$ satisfies
    \begin{align*}
        & \vx^* = \left(\frac{\mat{A}_{2, 2} - \mat{A}_{2, 1}}{\mat{A}_{1, 1} - \mat{A}_{1, 2} -\mat{A}_{2, 1} + \mat{A}_{2, 2}}, \frac{\mat{A}_{1, 1} - \mat{A}_{1, 2}}{\mat{A}_{1, 1} - \mat{A}_{1, 2} -\mat{A}_{2, 1} + \mat{A}_{2, 2}}\right) \\
        & \vz^* = \left(\frac{\mat{A}_{2, 2} - \mat{A}_{1,2}}{\mat{A}_{1, 1} - \mat{A}_{1, 2} -\mat{A}_{2, 1} + \mat{A}_{2, 2}}, \frac{\mat{A}_{1, 1} - \mat{A}_{2, 1}}{\mat{A}_{1, 1} - \mat{A}_{1, 2} -\mat{A}_{2, 1} + \mat{A}_{2, 2}}\right).
    \end{align*}
\end{lemma}

\begin{proof}[Proof of~\Cref{theorem:irrational}]
    By construction of the adversarial team game, the mixed extension of the utility can be expressed as
    \begin{equation*}
        x_1 z_1 \left( y_1 + 3 y_2 \right) + x_1 z_2 \left( \frac{9}{10} y_1 - \frac{1}{10} y_2 \right) + x_2 z_1 \left( \frac{99}{100} y_1 - \frac{1}{100} y_2 \right) + x_2 z_2 \left( 3y_2 + y_1 \right).
    \end{equation*}
    Suppose that we fix $\vy \in \Delta^2$. Then, Players $\vx$ and $\vy$ are engaged in a (two-player) zero-sum game with payoff matrix
    \begin{equation}
        \label{eq:game-y}
    \mat{A}(\vy) \defeq 
        \begin{bmatrix}
        1 + 2y_2 & \frac{9}{10} - y_2 \\
        \frac{99}{100} - y_2 & 1 + 2 y_2
    \end{bmatrix}.
    \end{equation}
    We now invoke~\Cref{lemma:2x2}. Indeed, we have
    \begin{equation}
        \label{eq:diag-pos1}
        (\mat{A}(\vy)_{1,1} - \mat{A}(\vy)_{1, 2}) (\mat{A}(\vy)_{2, 2} - \mat{A}(\vy)_{2, 1}) = \left( 3 y_2 + \frac{1}{10} \right) \left( 3 y_2 + \frac{1}{100} \right) > 0
    \end{equation}
    and
    \begin{equation}
        \label{eq:diag-pos2}
        (\mat{A}(\vy)_{1,1} - \mat{A}(\vy)_{2, 1}) (\mat{A}(\vy)_{2, 2} - \mat{A}(\vy)_{1, 2}) = \left( 3 y_2 + \frac{1}{100} \right) \left( 3 y_2 + \frac{1}{10} \right) > 0;
    \end{equation}
    that is, the precondition of~\Cref{lemma:2x2} is satisfied, and so the value of~\eqref{eq:game-y} reads
    \begin{equation}
        \label{eq:value}
        v(\vy) = \min_{\vx \in \Delta^2} \max_{\vz \in \Delta^2} \langle \vx, \mat{A}(\vy) \vz \rangle = \frac{109 + 5890 y_2 + 3000 y_2^2}{110 + 6000y_2}.
    \end{equation}
    It is easy to verify that $v$ is a strictly convex function in $[0, 1]$, and admits a unique minimum corresponding to
    $
        \vy^* = 
        \left(\frac{611 - 9 \sqrt{3}}{600},
            \frac{9\sqrt{3} - 11}{600}\right),%^{\top}
    $
    which is irrational. Now, suppose that $(\vx^*, \vy^*, \vz^*)$ is a Nash equilibrium of the adversarial team game. We will first argue that $(\vx^*, \vz^*)$ is the unique Nash equilibrium of $\mat{A}(\vy^*)$. Indeed, suppose that there exists $\vx' \in \Delta^2$ such that $\langle \vx', \mat{A}(\vy^*) \vz^* \rangle < \langle \vx^*, \mat{A}(\vy^*) \vz^* \rangle$, or equivalently, $u(\vx', \vy^*, \vz^*) < u(\vx^*, \vy^*, \vz^*)$; this is a contradiction since $(\vx^*, \vy^*, \vz^*)$ is assumed to be a Nash equilibrium. Similar reasoning applies with respect to Player $\vz$. Thus, $(\vx^*, \vz^*)$ is a Nash equilibrium of $\mat{A}(\vy^*)$, and thereby uniquely determined by $\vy^*$---by~\Cref{lemma:2x2} coupled with \eqref{eq:diag-pos1} and~\eqref{eq:diag-pos2}. Furthermore, given the value of $\vy^*$, we get that $\vx^* = \left(\frac{3 - \sqrt{3}}{6}, \frac{3 + \sqrt{3}}{6}\right)$ and $\vz^* = \left(\frac{3 + \sqrt{3}}{6}, \frac{3 -
    \sqrt{3}}{6}\right)$. Now, consider the utility of Player $\vy$ when playing the first action $a_1$ or the second action $a_2$; plugging in the value of $\vx^*$ and $\vz^*$, we have $u(\vx^*, \ve_1, \vz^*) = \frac{578 + 9 \sqrt{3}}{600}$ and $u(\vx^*, \ve_2, \vz^*) = \frac{578 + 9 \sqrt{3}}{600}.$ Since $u(\vx^*, \ve_1, \vz^*) = u(\vx^*, \ve_2, \vz^*)$, $(\vx^*, \vy^*, \vz^*)$ is a Nash equilibrium.
    
    Moreover, suppose there exists another NE $(\vx', \vy' ,\vz')$ that is different from $(\vx^*, \vy^*, \vz^*)$. As shown above, $(\vx', \vz')$ is the unique NE of the zero-sum game induced by $\vy'$. Thus, if we have two different Nash equilibria, it implies that $\vy' \neq \vy^*$. We consider the following three cases:
    \begin{itemize}
        \item First, let $\vy'$ be a (fully) mixed strategy. Since $\vx'$ and $\vz'$ forms the unique NE in of $\mat{A}(\vy')$, we have
        \begin{align*}
            &\vx' = \left(\frac{1 + 300  y'_2}{11 + 600  y'_2}, \frac{10 + 300 y'_2}{11 + 600  y'_2}\right),\\
            &\vz' = \left(\frac{10 + 300  y'_2}{11 + 600  y'_2}, \frac{1+ 300 y'_2}{11 + 600  y'_2}\right).
        \end{align*}
        Further, for Player $\vy$,
        \begin{align*}
            &u(\vx', \ve_1, \vz') = \frac{1199 + 130800 y_2 + 3501000 y_2^2}{10(11 + 600 y_2)^2}, \\
            &u(\vx', \ve_2, \vz') = \frac{589 + 196800 y_2 + 5301000 y_2^2}{10(11 + 600 y_2)^2}.
        \end{align*}
        Since $\vy'$ is a mixed strategy, we have $u(\vx', \ve_1, \vz') = u(\vx', \ve_2, \vz')$; solving the equality we get $
        \vy' = 
        \left(\frac{611 - 9 \sqrt{3}}{600},
            \frac{9\sqrt{3} - 11}{600}\right),
    $ which contradicts the assumption that $\vy' \neq \vy^*$.
    \item If $\vy' = (1, 0)$, we have $u(\vx', \ve_1, \vz') = \frac{1199}{1210}$ and $u(\vx', \ve_2, \vz') = \frac{589}{1210}$. Thus, it follows that by unilaterally deviating to play $(0, 1)$, Player $\vy$ can decrease the utility of the adversary, contradicting the fact that $(\vx', \vy', \vz')$ is a Nash equilibrium.
    \item Finally, suppose that $\vy' = (0, 1)$. Similarly to the second case, we get $u(\vx', \ve_1, \vz') < u(\vx', \ve_2, \vz')$, which is a contradiction.
    \end{itemize}
    Thus, we conclude that $(\vx^*, \vy^*, \vz^*)$ is the unique Nash equilibrium of the 3-player adversarial team game defined above, completing the proof.
    
    % But given that the value function~\eqref{eq:value} admits a unique minimum, any strategy $\Delta^2 \ni \vy' \neq \vy^*$ would result in
    % \begin{equation}
    %  v(\vy') > v(\vy^*) \iff \langle \vx^*, \mat{A}(\vy') \vz^* \rangle > \langle \vx^*, \mat{A}(\vy^*) \vz^* \rangle \iff u(\vx^*, \vy', \vz^*) > u(\vx^*, \vy^*, \vz^*).
    % \end{equation}
    % This completes the proof.
%    \ya{The proof has a small bug, found a way to fix it.}
\end{proof}

\subsection{The complexity of determining uniqueness}

A natural question arising from~\Cref{theorem:irrational} concerns the complexity of determining whether an adversarial team game admits a unique Nash equilibrium. Our next theorem establishes \NP-hardness for a version of that problem that accounts for approximate Nash equilibria.

\begin{theorem}
    \label{theorem:uniqueATG}
    For polymatrix, $3$-player adversarial team games, constants $c_1, c_2 > 0$, and $\epsilon = n^{-c_1}$, it is \NP-hard to distinguish between the following two cases:
    \begin{itemize}[noitemsep,topsep=0pt]
        \item any two $\epsilon$-Nash equilibria have $\ell_1$-distance at most $n^{-c_2}$, and
        \item there are two $\epsilon$-Nash equilibria that have $\ell_1$-distance $\Omega(1)$.
    \end{itemize}
\end{theorem}

We will see the proof of this theorem later in~\Cref{sec:nonsymmetric} when we examine the complexity of computing non-symmetric equilibria in symmetric min-max optimization problems.