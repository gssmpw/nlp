In this subsection, our goal is to establish~\Cref{theorem:symmetric-new} that was claimed earlier in the main body, which forms the basis for~\Cref{theorem:uniqueATG,theorem:non-symmetric}.

We consider the symmetric game $(\lineA, \lineA^{\top})$, where $\lineA \in \mathbb{R}^{n \times n}$ is a symmetric matrix. In particular, since $\lineA = \lineA^{\top}$, the two players in the game share the same payoff matrix. The payoff matrix $\lineA$ is constructed based on an underlying graph $G = ([n], E) $ and a parameter $\delta \in (0, 1)$ as follows:
\begin{align*}
    \lineA_{i, j} = 
    \begin{cases}
        \delta & \text{if $i = j$}, \\
        1 & \text{if $(i, j) \in E$}, \\
        0 & \text{otherwise.}
    \end{cases}
\end{align*}

\begin{lemma} \label{lemma:value_of_approx_nash_on_clique}
    Let $(\hat{\vx}, \hat{\vx})$ be an $\epsilon$-well-supported NE of the game where $\hat{\vx}$ is supported on a max clique of size $k$, denoted as $C_k$, then the value $u(\hat{\vx}, \hat{\vx})$ is at least $ 1 - \frac{1}{k} + \frac{\delta}{k} - (\frac{k - \delta}{1 - \delta})  \epsilon$ and $\norm{\hat{\vx} - \vx^*}_{\infty} \leq \frac{k - \delta}{1 - \delta} \epsilon$ where $\vx^* = \frac{1}{k} \sum_{i \in C_k} \ve_{i}$.
\end{lemma}

\begin{proof}
    Since $\hat{\vx}$ is supported on $C_k$, let $i$ be a coordinate that $\hat{\vx}^*$ puts the least probability mass on; that is, $i \in \argmin_{j \in C_k} \hat{x}_j.$ Considering the utility of playing action $a_i$, we have
    \begin{align*}
        u(\ve_i, \hat{\vx}) = \hat{x}_i \cdot \delta + (1 - \hat{x}_i) \cdot 1 = 1 - \hat{x}_i + \hat{x}_i \delta.
    \end{align*}
    Moreover, let $j \in C_k$ be a coordinate such that $\hat{\vx}_j \geq \frac{1}{k}$. It should hold that 
    \begin{align*}
    u(\ve_j, \hat{\vx}) = 1 - (1-\delta)\hat{x}_j \leq 1 - \frac{1}{k} + \frac{\delta}{k}.
    \end{align*}
    Since $(\hat{\vx}, \hat{\vx})$ is an $\epsilon$-well-supported NE, we have
    \begin{align*}
    & u(\ve_j, \hat{\vx}) \geq u(\ve_i, \hat{\vx}) - \epsilon \\
    \Rightarrow  \quad  &1 - \frac{1}{k} + \frac{\delta}{k} + \epsilon \geq 1 - \hat{x}_i + \hat{x}_i \delta \\
        \Rightarrow \quad &\hat{x}_i \geq \frac{1}{k} - \frac{\epsilon}{1 - \delta}.
    \end{align*}
    Thus, for all coordinates $i\in C_k$ , we have $\frac{1}{k} -\frac{\epsilon}{1 - \delta}\leq \hat{x}_i \leq \frac{1}{k} + \frac{\epsilon}{1 - \delta} \cdot (k - 1).$
    It then holds that
    \begin{align*}
        u(\hat{\vx}, \hat{\vx}) & \geq u(\ve_i, \hat{\vx}) - \epsilon \\
        & \geq \left(\frac{1}{k} + \frac{\epsilon}{1 - \delta} \cdot (k - 1)\right) \cdot \delta + \left(1 - \frac{1}{k} - \frac{\epsilon}{1 - \delta} \cdot (k - 1)\right) - \epsilon \\
        & \geq 1 - \frac{1}{k} + \frac{\delta}{k} - \left(\frac{k - \delta}{1 - \delta}\right) \epsilon.
    \end{align*}
\end{proof}

\begin{assumption} \label{assump:parameter}
    For the rest of this subsection, we set the parameters as follows:
    \begin{itemize}
        \item $n \geq k \geq 10;$
        \item $\epsilon < \delta(1 - \delta)/6n^7;$
        \item $\delta \defeq 1/2.$
    \end{itemize}
\end{assumption}

(Using the symbolic value of $\delta$ is more convenient in our derivations below.)

\begin{proof}[Proof of~\Cref{lemma:well_supported_nash_value}]
    We let $\vx^* = \frac{1}{k} \sum_{i \in C_k} \ve_{i}$. The proof considers the following three cases:
    \begin{itemize}
        \item The symmetric $\epsilon$-well-supported NE $(\hat{\vx}, \hat{\vx})$ has support size less than $k$.
        \item The symmetric $\epsilon$-well-supported NE $(\hat{\vx}, \hat{\vx})$ has support size greater than $k$.
        \item The symmetric $\epsilon$-well-supported NE $(\hat{\vx}, \hat{\vx})$ has support size equal to $k$ but is not supported on a clique.
    \end{itemize}
    We proceed to show that for any of the three cases above, we would not be able to have a symmetric $\epsilon$-well-supported NE that achieves value greater than $1 - \frac{1}{k} + \frac{\delta}{k} - \frac{2\delta}{ n^2k^4} + 2 \epsilon$, which is a contradiction.
    \begin{itemize}
        \item For the first case, since the support has size less than $k$, we can find a coordinate $i \in [n]$ such that $\hat{\vx}_i \geq \frac{1}{k - 1}.$ Therefore, the value of playing that action is 
        \begin{align*}
            u(\ve_i, \hat{\vx}) \leq 1 - \frac{1}{k - 1} + \frac{\delta}{k - 1}.
        \end{align*}  
        Since $(\hat{\vx}, \hat{\vx})$ is a symmetric $\epsilon$-well-supported NE, we have
        \begin{align*}
            u(\hat{\vx}, \hat{\vx}) & \leq u(\ve_i, \hat{\vx}) + \epsilon \\
            & \leq 1 - \frac{1}{k - 1} + \frac{\delta}{k - 1} + \epsilon \\
            & \leq 1 - \frac{1}{k} +\frac{\delta}{k} - \frac{2\delta}{n^2k^4} + 2\epsilon,
        \end{align*}
        where in the last step we use~\Cref{assump:parameter}.
        \item For the second case, suppose $|\text{supp}(\hat{\vx})| = m > k$. By \Cref{lemma:not_connected_set}, since the maximum clique size is $k$, we can find a set $\mathcal{S} \subseteq \text{supp}(\hat{\vx})$ with at least $m - k + 1$ elements such that for each coordinate $i \in \mathcal{S}$, we can find a coordinate $j \in \mathcal{S}$ such that $\lineA_{i, j} = \lineA_{j, i} = 0.$ Now we consider the utility of playing action $a_i$ and $a_j$, we have
    \begin{align*}
        & u(\ve_i, \hat{\vx}) = \sum_{l \in \text{supp}(\hat{\vx}) - \{i, j\}} \hat{x}_l \cdot \lineA_{i, l} + \hat{x}_i \cdot \delta + \hat{x}_j  \cdot \lineA_{i, j}, \\
        &u(\ve_j, \hat{\vx}) = \sum_{l \in \text{supp}(\hat{\vx}) - \{i, j\}} \hat{x}_l \cdot \lineA_{j, l} + \hat{x}_j \cdot \delta + \hat{x}_i  \cdot \lineA_{i, j}.
    \end{align*}
    Since $(\hat{\vx}, \hat{\vx})$ is a symmetric $\epsilon$-well-supported NE, we have
    \begin{align*}
        & \sum_{l \in \text{supp}(\hat{\vx}) - \{i, j\}} \hat{x}_l \cdot \lineA_{i, l} + \hat{x}_i \cdot \delta + \hat{x}_j  \cdot\lineA_{i, j} \geq \sum_{l \in \text{supp}(\hat{\vx}) - \{i, j\}} \hat{x}_l \cdot \lineA_{j, l} + \hat{x}_j \cdot \delta + \hat{x}_i  \cdot \lineA_{i, j} - \epsilon\\
        \Rightarrow \quad & \sum_{l \in \text{supp}(\hat{\vx}) - \{i, j\}} \hat{x}_l \cdot \lineA_{i, l}  - \sum_{l \in \text{supp}(\hat{\vx}) - \{i, j\}} \hat{x}_l \cdot \lineA_{j, l}   \geq \hat{x}_j \cdot \delta - \hat{x}_i \cdot \delta - \epsilon.
    \end{align*}
    Now, by moving all the probability mass from action $j$ to action $i$, we form a new strategy $\vx' = \hat{\vx} + \hat{x}_j \cdot \ve_i - \hat{x}_j \cdot \ve_j $ such that
    \begin{align}
        u(\vx', \vx') - u(\hat{\vx}, \hat{\vx}) &=  2 \hat{x}_j\left(\sum_{l \in \text{supp}(\hat{\vx}) - \{i ,j\}} \hat{x}_l \cdot \lineA_{i, l} - \sum_{l \in \text{supp}(\hat{\vx}) - \{i ,j\}} \hat{x}_ml\cdot \lineA_{j, l}\right) + 2 \hat{x}_i \cdot \hat{x}_j \cdot \delta \notag \\
        & \geq  2 \hat{x}_j \cdot ((\hat{x}_j - \hat{x}_i) \delta - \epsilon)+ 2 \hat{x}_i \cdot\hat{x}_j \cdot \delta \notag \\
        & \geq 2 {\hat{x}_j}^2 \delta - 2 \epsilon \label{eq:marginal_gain_when_one_move}. 
    \end{align}
        Suppose there is a coordinate $i \in \mathcal{S}$ such that $\hat{x}_i \geq \frac{1}{nk^2}$ then from \eqref{eq:marginal_gain_when_one_move}, we have 
        \begin{align*}
            u(\hat{\vx}, \hat{\vx}) &\leq u(\vx^*, \vx^*) - 2\cdot(\frac{1}{nk^2})^2 \cdot \delta + 2 \epsilon\\
            & = 1 - \frac{1}{k} + \frac{\delta}{k} - \frac{2\delta}{ n^2k^4} + 2 \epsilon.
        \end{align*}
        If $\hat{x}_i < \frac{1}{nk^2}$ for any $i \in \mathcal{S}$, then there exists a coordinate $l \not \in \mathcal{S}$ such that $\hat{x}_l > \frac{1 - (m - k + 1)\cdot\frac{1}{nk^2}}{m - (m - k + 1)} \geq \frac{1}{k} + \frac{1}{k^2}.$ Then, considering the utility when playing action $a_l$,
        \begin{align}
            u(\ve_l, \hat{\vx}) &\leq 1 \cdot (1 - \frac{1}{k} - \frac{1}{k^2}) + \delta \cdot (\frac{1}{k} + \frac{1}{k^2}) \notag \\
            & = 1 - \frac{1}{k} + \frac{\delta}{k} - \frac{1}{k^2} + \frac{\delta}{k^2} \notag \\
            &< 1 - \frac{1}{k} + \frac{\delta}{k} - \frac{2\delta}{n^2k^4}, \label{eq:case_two_vaue_of_nash}
        \end{align}
        where in~\eqref{eq:case_two_vaue_of_nash} we used~\Cref{assump:parameter}.
        
        Since the $l$th action is played with positive probability and $(\hat{\vx}, \hat{\vx})$ is an $\epsilon$-well-supported NE, we have $u(\hat{\vx}, \hat{\vx}) \leq u(\ve_l, \hat{\vx}) + \epsilon < 1 - \frac{1}{k} + \frac{\delta}{k} - \frac{2\delta}{n^2k^4} + 2\epsilon.$ 
        \item For the third case, since the support is not on a clique, the exists at least coordinates $i, j$ such that $\hat{x}_i >0, \hat{x}_j>0,$ and $\lineA_{i, j} = \lineA_{j, i} = 0.$ Similarly as case two, if $\hat{x}_i \geq \frac{1}{nk^2}$ or $\hat{x}_j \geq \frac{1}{nk^2}$, then we have  
        \begin{align*}
            u(\hat{\vx}, \hat{\vx}) &\leq u(\vx^*, \vx^*) - 2\cdot(\frac{1}{nk^2})^2 \cdot \delta + 2 \epsilon\\
            & = 1 - \frac{1}{k} + \frac{\delta}{k} - \frac{2\delta}{ n^2k^4} + 2\epsilon.
        \end{align*}
        If $\hat{x}_i < \frac{1}{nk^2}$ and $\hat{x}_j < \frac{1}{nk^2}$, then there exists an coordinate $l$ such that $\hat{x}_l \geq \frac{1 - 2 \cdot \frac{1}{nk^2}}{k - 2} > \frac{1}{k} + \frac{1}{k^2}.$ Same as \eqref{eq:case_two_vaue_of_nash}, we conclude that $u(\hat{\vx}, \hat{\vx})$ is at most $1 - \frac{1}{k} + \frac{\delta}{k} - \frac{2\delta}{n^2k^4} + 2\epsilon.$
    \end{itemize}
    The proof is complete.
\end{proof}

We now construct a new symmetric identical payoff game $(\mat{B}, \mat{B})$, where $\mat{B}$ is defined as
    \begin{align}
        \mat{B} = \begin{bmatrix}
         \lineA_{1, 1} & \cdots & \lineA_{1, n}  & r\\
         \vdots & \ddots & \vdots & \vdots \\
         \lineA_{n, 1} & \cdots & \lineA_{n,n} & r\\
         r & \cdots & r  & V\\
    \end{bmatrix}. \label{eq:unique_NP_matrix}
    \end{align}
Above, $V \defeq 1 - \frac{1}{k} + \frac{\delta}{k}$ and $r \defeq 1 - \frac{1}{k} + \frac{\delta}{k} - \frac{\delta}{n^2k^4} + 3\epsilon$. (We caution that the values of $V$ and $r$ have been set differently compared to~\Cref{sec:nonsymmetric}.) Similarly to our derivation in~\cref{eq:three_exact_NE}, it follows that the symmetric (exact) Nash equilibria of this game can only be in one of the following forms:
\begin{enumerate}
        \item $(\vx^*, \vx^*)$ with $\vx^* \defeq \ve_{n+ 1}$; \label{eq:NE_case1}
        \item $(\vx^*, \vx^*)$ with $\vx^* \defeq \frac{1}{k} \sum_{i \in C_k} \ve_i$, where $C_k \subseteq [n]$ is a clique in $G$ of size $k$ \label{eq:NE_case2};
        \item $(\vx^*, \vx^*)$ with $\vx^* \defeq \frac{1}{2} \ve_{n + 1} + \frac{1}{2k} \sum_{i \in C_k} \ve_i$, where $C_k \subseteq [n]$ is a clique in $G$ of size $k$ \label{eq:NE_case3}.
\end{enumerate}

We now show the following lemma.

\begin{lemma} \label{lemma:NP_completeness_for_two_symmetric_well_supported}
    For any $\epsilon$-well-supported NE $(\hat{\vx}, \hat{\vx})$ in game $(\mat{B}, \mat{B})$, it holds that $\norm{\hat{\vx} - \vx^*}_{\infty} \leq 2n^6\epsilon$, where $(\vx^*, \vx^*)$ is an exact NE in one of the three cases above (\ref{eq:NE_case1},\ref{eq:NE_case2},\ref{eq:NE_case3}).
\end{lemma}

\begin{proof}
    First, we observe that since $V > r$, clearly $(\ve_{n +1}, \ve_{n + 1})$ is a $\epsilon$-well-supported NE; in this case, $\norm{\vx' - \vx^*}_{\infty} = 0.$ Furthermore, since $r = 1 - \frac{1}{k} + \frac{\delta}{k} - \frac{\delta}{n^2k^4} + 3 \epsilon$, the game does not attain any $\epsilon$-NE with value less than $1 - \frac{1}{k} + \frac{\delta}{k} - \frac{\delta}{n^2k^4} + 2 \epsilon.$ Suppose the game admits an symmetric $\epsilon$-well-supported NE $(\hat{\vx}, \hat{\vx})$ where $\hat{\vx}$ is supported only on the first $n$ actions. Since $u(\hat{\vx}, \hat{\vx}) \geq 1 - \frac{1}{k} + \frac{\delta}{k} - \frac{\delta}{n^2k^4} + 2 \epsilon > 1 - \frac{1}{k} + \frac{\delta}{k} - \frac{2\delta}{n^2k^4} + 2\epsilon,$ taking  $\vx^*$ as in \eqref{eq:NE_case2}, we conclude that $\norm{\hat{\vx} - \vx^*}_{\infty} \leq \frac{k - 1}{1 - \delta} \epsilon < 2n^6\epsilon$ from \cref{lemma:well_supported_nash_value}. 
    
     We proceed to the case where there is a mixed symmetric $\epsilon$-well-supported Nash $(\hat{\vx}, \hat{\vx})$ between the last action and the rest of actions such that $0 < \hat{x}_{n+1} < 1.$ Denote $\hat{x}_{n + 1} = \alpha$ and $\eta(\cdot)$ to denote the renormalization operation. Since $\left(\eta(\hat{\vx}_{[1 \cdots n]}), \eta(\hat{\vx}_{[1 \cdots n]})\right)$ is a symmetric strategy profile, from~\citet[Proposition 4]{MCLENNAN2010683}, we conclude that there is at least one coordinate $i \in \text{supp}(\hat{\vx}) - \{n+1\}$ such that
     \begin{align*}
         u_A\left(\ve_i, \eta(\hat{\vx}_{[1 \cdots n]})\right) \leq 1 - \frac{1}{k} + \frac{\delta}{k} = V,
     \end{align*}
     where $u_A$ is the utility when the payoff matrix is $\lineA.$ Since $\left(\eta(\hat{\vx}_{[1 \cdots n]}), \eta(\hat{\vx}_{[1 \cdots n]})\right)$ is an $\epsilon$-well-supported NE, we have
     \begin{align}
         & u_B(\ve_i, \hat{\vx}) \geq u_B(\ve_{n + 1}, \hat{\vx}) - \epsilon \notag \\
         \Rightarrow \quad & (1 - \alpha)\cdot V + \alpha \cdot r \geq (1 - \alpha)r + \alpha V - \epsilon \notag \\
         \Rightarrow \quad & \alpha \leq \frac{1}{2} + \frac{\epsilon}{2(V - r)}, \label{eq:lower_bound_on_alpha}
     \end{align}
     where $u_B$ is the utility function when the payoff matrix is $\mat{B}$. Plugging in the value of $V$ and $r$ and using~\Cref{assump:parameter}, we find that $\alpha \leq \frac{1}{2} + 2n^6\epsilon \leq \frac{2}{3}$.
     
     Now, we observe that for any action in the support other than the last action $a_i$, the utility of playing such action $u_B(a_i, \hat{\vx}) = u_A(a_i, \hat{\vx}_{[1 \cdots n]}) + r \alpha$. Since $(\hat{\vx}, \hat{\vx})$ is an $\epsilon$-well-supported Nash Equilibrium, we have $u_B(a_i, \hat{\vx}) \geq \max_{j} u_B(a_j, \hat{\vx}) - \epsilon$ for all pairs $(i, j) \in \text{supp}(\hat{\vx}).$ Since $\alpha \leq \frac{2}{3},$ it follows that $u_A\left(a_i, \eta(\hat{\vx}_{[1 \cdots n]})\right) \geq \max_j u_A\left(a_j, \eta(\hat{\vx}_{[1 \cdots n]})\right) - 3\epsilon$ for any pairs $(i, j) \in \text{supp}(\hat{\vx}) - \{n + 1\}.$ Thus, we conclude that $\left(\eta(\hat{\vx}_{[1 \cdots n]}), \eta(\hat{\vx}_{[1 \cdots n]})\right)$ forms a symmetric $3\epsilon$-well-supported Nash Equilibrium in game $(\lineA, \lineA)$. Further, the value of playing the last action is $(1 - \alpha)r + \alpha V > r$, and so the only situation where there is a mixed Nash between the last action and the rest actions is when $u_A\left(\left(\eta(\hat{\vx}_{[1\cdots n]}), \eta(\hat{\vx}_{[1\cdots n]})\right)\right) \geq r - \epsilon$. Therefore, by \Cref{lemma:value_of_approx_nash_on_clique} and \Cref{lemma:well_supported_nash_value}, we conclude that $\hat{\vx}_{[1 \cdots n]}$ is supported on a clique of size $k.$
     % we conclude that for any coordinate $i \in \text{supp}(\hat{\vx}), i \neq n +1,$ $\frac{1}{k} - \frac{3\epsilon}{1 - \delta}\leq \eta(\hat{\vx})_i \leq \frac{1}{k} + \frac{3k - 3}{1 - \delta} \epsilon.$ 
     There exits at least one coordinate $i \in [n]$, with $0 < \hat{x}_i$ and $ \hat{x}_i \geq \frac{1}{k},$ such that
    \begin{align*}
        u_A(\ve_{i}, \eta(\hat{\vx}_{[1 \cdots n]})) \geq 1 - \frac{1}{k} + \frac{\delta}{k} = V.
    \end{align*}
    Since $(\hat{\vx}, \hat{\vx})$ is an $\epsilon$-well-supported Nash, we have $u(\ve_i, \hat{\vx}) \leq u(\ve_{n +1}, 
    \hat{\vx}) + \epsilon$, and so this gives
    \begin{align}
        % & \begin{cases}
        %     (1 - \alpha) \cdot \left(\delta\left(\frac{1}{k} - \frac{\epsilon}{1 - \delta}\right) + \left(1 - \frac{1}{k} + \frac{\epsilon}{1 - \delta}\right)\right) + \alpha \cdot r \geq (1 - \alpha) \cdot r + \alpha \cdot V - \epsilon \\
        %     (1 - \alpha) \cdot \left(\delta(\frac{1}{k} + \frac{k - 1}{1 - \delta}\epsilon) + \left(1 - \frac{1}{k} - \frac{k-1}{1-\delta} \epsilon \right)\right) + \alpha \cdot r \leq (1 - \alpha) \cdot r + \alpha \cdot V + \epsilon 
        % \end{cases} 
        & (1 - \alpha) \cdot V + \alpha \cdot r \leq (1 - \alpha) \cdot r + \alpha \cdot V + \epsilon \label{eq:alpha_upper_lower_bound}\\
        \Rightarrow \quad &  \alpha \geq \frac{1}{2} - \frac{\epsilon}{2(V - r)}. \label{eq:range_alpha}
        % \Rightarrow \quad & \alpha \geq \frac{1}{2} - \frac{3n^2k^5\epsilon}{4 \delta (1 - \delta)} \label{eq:range_alpha}.
    \end{align}
    Using~\Cref{assump:parameter} and 
    % \begin{itemize}
    %     \item in \eqref{eq:alpha_upper_lower_bound} we use the fact that $\eta(\hat{\vx})_i \leq \frac{1}{k} + \frac{3k - 3}{1 - \delta} \epsilon$ for any $i \in \text{supp}(\hat{\vx}), i \neq n +1;$
    %     \item in \eqref{eq:range_alpha} we plug in the value of $V$ and $r$.
    % \end{itemize}
    combining with~\eqref{eq:lower_bound_on_alpha},
    \begin{align*}
        \frac{1}{2} - 2n^6\epsilon \leq \alpha \leq \frac{1}{2} + 2n^6\epsilon.
    \end{align*}
    By taking $\vx^*$ as in \eqref{eq:NE_case3}, we conclude that $\norm{\hat{\vx} - \vx^*}_{\infty} \leq 2n^6 \epsilon$.
    % Suppose there are two symmetric $\epsilon$-well supported Nash $(\hat{\vx}, \hat{\vx})$ and $(\hat{\vx}', \hat{\vx}')$ such that $\norm{\hat{\vx} - \hat{\vx}'}_{\infty} \geq \frac{1}{3}.$ Since for all $\epsilon$-well-supported Nash, then for any $i \in [n]$, it holds that $\hat{\vx}_i \leq \frac{1}{k} + \frac{k - 1}{1 - \delta} \epsilon \leq \frac{1}{10} + 18 \epsilon < \frac{1}{3}$ and similarly $\hat{\vx}' < \frac{1}{4}.$ Then $\norm{\hat{\vx} - \hat{\vx}'}_{\infty} \geq \frac{1}{3}$ would imply $|\hat{\vx}_{n + 1} - \hat{\vx}'_{n + 1}| \geq \frac{1}{3},$ which further implies that there is a clique of size $k$ in graph $G$. The proof is complete. 
\end{proof}

\begin{theorem}\label{thm:betadefined}
    % It is \NP-complete to answer the following question: Given a two player identical payoff game, can we find two symmetric $\epsilon$-NE that are $O(1)$ away from each other?
    For any $\epsilon$-NE $(\vx, \vx)$ in game $(\mat{B}, \mat{B})$, it holds that $\norm{\vx - \vx^*}_{\infty} \leq n^6 \sqrt{\epsilon}$, where $(\vx^*, \vx^*)$ is an exact NE in one of the forms specified above (\ref{eq:NE_case1},\ref{eq:NE_case2},\ref{eq:NE_case3}).
\end{theorem}

\begin{proof}
    \citet[Lemma 3.2]{Chen09:Settling} showed that from any $\epsilon^2/8$-NE $(\vx, \vy)$ in any two player bimatrix game, one can construct (in polynomial time) an $\epsilon$-well-supported NE $(\vx', \vy')$ such that $\norm{\vx - \vx'}_{\infty} \leq \frac{\epsilon}{4}$ and $\norm{\vy - \vy'}_{\infty} \leq \frac{\epsilon}{4}$. By setting $\epsilon' \defeq \frac{\epsilon^2}{8}$ for the $\epsilon$ defined in ~\Cref{lemma:NP_completeness_for_two_symmetric_well_supported} the proof follows.
    % In game $(\mat{B}, \mat{B})$, if we can find two symmetric $\epsilon'$-NE $(\vx, \vx)$ and $(\vx', \vx')$ such that $\norm{\vx - \vx'}_{\infty} \geq \frac{1}{2} > \frac{1}{3} + \frac{\epsilon}{2}$, this would imply that in polynomial time we can get two symmetric $\epsilon$-well-supported NE $(\hat{\vx}, \hat{\vx})$ and $(\hat{\vx}', \hat{\vx}')$ such that $\norm{\vx - \vx'}_{\infty} \geq \frac{1}{3}.$ The complexity result follows from \cref{lemma:NP_completeness_for_two_symmetric_well_supported}.
\end{proof}

The proof of \Cref{theorem:symmetric-new} follows directly by observing that having two symmetric $\epsilon$-NE $(\vx, \vx)$ and $(\vy, \vy)$ such that $\norm{\vx - \vy}_{\infty} > 2n^6 \sqrt{\epsilon}$ would imply that the game $(\mat{B}, \mat{B})$ has two distinct exact NE, which in turn implies that there is a clique of size $k$ in the graph.

We finally conclude this subsection by stating and proving an auxiliary lemma that was used earlier.

\begin{lemma} \label{lemma:not_connected_set}
    For any graph $G = (V, E)$ with $n$ vertices, if the maximum clique has size $k$, then we can form a set $\mathcal{S} \subseteq V$ of size at least $n - k + 1$ such that for any vertex $i \in \mathcal{S}$, there exists a vertex $j \in \mathcal{S}$ such that $i$ and $j$ are not connected.
\end{lemma}

\begin{proof}
    Suppose the largest set $\mathcal{S}$ we can form has cardinality $|\mathcal{S}| < n - k + 1$, this implies there is a set $\mathcal{S}' = V - \mathcal{S}$ with at least $n - (n - k ) = k$ vertices such that each vertex in $\mathcal{S'}$ is connected to all other vertices in $G$. However if this is the case, there is at least one vertex $v \not \in \mathcal{S'}$ that are connected to all vertices in $\mathcal{S'}$. This contradicts the fact that maximum clique has size $k$.
\end{proof}

