\documentclass[11pt]{article}
\usepackage{amssymb}

\usepackage{booktabs} % For formal tables
\usepackage{multirow}
\usepackage[ruled]{algorithm2e} % For algorithms
\renewcommand{\algorithmcfname}{ALGORITHM}
\SetAlFnt{\small}
\SetAlCapFnt{\small}
\SetAlCapNameFnt{\small}
\SetAlCapHSkip{0pt}
\IncMargin{-\parindent}

\usepackage[table]{xcolor}

\usepackage{bm}
\newcommand{\ya}[1]{\textcolor{red}{[* Yannis: #1 *]}}

\newcommand{\defeq}{\coloneqq}
%\usepackage{soul}

\usepackage{complexity}
\newcommand{\CLS}{\mathsf{CLS}}
\newcommand{\NPP}{\mathsf{NP}}
\let\R\relax
\let\E\relax

\newcommand{\hvx}{\hat{\vx}}
\newcommand{\hvy}{\hat{\vy}}
\newcommand{\hvz}{\hat{\vz}}

\newcommand{\lineA}{\overline{\mat{A}}}

\usepackage[hidelinks, colorlinks = true, linkcolor=blue, citecolor=green]{hyperref}

\newcommand{\declarecolor}[2]{\definecolor{#1}{RGB}{#2}\expandafter\newcommand\csname #1\endcsname[1]{\textcolor{#1}{##1}}}
\declarecolor{White}{255, 255, 255}
\declarecolor{Black}{0, 0, 0}
\declarecolor{Maroon}{128, 0, 0}
\declarecolor{Coral}{255, 127, 80}
\declarecolor{Red}{182, 21, 21}
\declarecolor{LimeGreen}{50, 205, 50}
\declarecolor{DarkGreen}{0, 80, 0}
\declarecolor{Purple}{146, 42, 158}
\declarecolor{Navy}{0, 0, 128}
\declarecolor{LightBlue}{84, 101, 202}
\definecolor{mydarkblue}{rgb}{0,0.08,0.45}
\hypersetup{ %
    pdftitle={},
    pdfkeywords={},
    pdfborder=0 0 0,
    pdfpagemode=UseNone,
    colorlinks=true,
    linkcolor=Navy,
    citecolor=DarkGreen,
    filecolor=Purple,
    urlcolor=Black,
}

\usepackage{fullpage}
\usepackage{authblk}

\usepackage{MnSymbol}

%%%%% NEW MATH DEFINITIONS %%%%%

% \usepackage{amsmath,amsfonts,bm}
\usepackage{multirow,array}
\usepackage{diagbox}

% Highlight a newly defined term
\newcommand{\newterm}[1]{{\bf #1}}

\newcommand{\matA}{\mat{A}}
\newcommand{\vecb}{\vec{b}}


% Random variables
\def\reta{{\textnormal{$\eta$}}}
\def\ra{{\textnormal{a}}}
\def\rb{{\textnormal{b}}}
\def\rc{{\textnormal{c}}}
\def\rd{{\textnormal{d}}}
\def\re{{\textnormal{e}}}
\def\rf{{\textnormal{f}}}
\def\rg{{\textnormal{g}}}
\def\rh{{\textnormal{h}}}
\def\ri{{\textnormal{i}}}
\def\rj{{\textnormal{j}}}
\def\rk{{\textnormal{k}}}
\def\rl{{\textnormal{l}}}
% rm is already a command, just don't name any random variables m
\def\rn{{\textnormal{n}}}
\def\ro{{\textnormal{o}}}
\def\rp{{\textnormal{p}}}
\def\rq{{\textnormal{q}}}
\def\rr{{\textnormal{r}}}
\def\rs{{\textnormal{s}}}
\def\rt{{\textnormal{t}}}
\def\ru{{\textnormal{u}}}
\def\rv{{\textnormal{v}}}
\def\rw{{\textnormal{w}}}
\def\rx{{\textnormal{x}}}
\def\ry{{\textnormal{y}}}
\def\rz{{\textnormal{z}}}

% Random vectors
\def\rvepsilon{{\mathbf{\epsilon}}}
\def\rvtheta{{\mathbf{\theta}}}
\def\rva{{\mathbf{a}}}
\def\rvb{{\mathbf{b}}}
\def\rvc{{\mathbf{c}}}
\def\rvd{{\mathbf{d}}}
\def\rve{{\mathbf{e}}}
\def\rvf{{\mathbf{f}}}
\def\rvg{{\mathbf{g}}}
\def\rvh{{\mathbf{h}}}
\def\rvu{{\mathbf{i}}}
\def\rvj{{\mathbf{j}}}
\def\rvk{{\mathbf{k}}}
\def\rvl{{\mathbf{l}}}
\def\rvm{{\mathbf{m}}}
\def\rvn{{\mathbf{n}}}
\def\rvo{{\mathbf{o}}}
\def\rvp{{\mathbf{p}}}
\def\rvq{{\mathbf{q}}}
\def\rvr{{\mathbf{r}}}
\def\rvs{{\mathbf{s}}}
\def\rvt{{\mathbf{t}}}
\def\rvu{{\mathbf{u}}}
\def\rvv{{\mathbf{v}}}
\def\rvw{{\mathbf{w}}}
\def\rvx{{\mathbf{x}}}
\def\rvy{{\mathbf{y}}}
\def\rvz{{\mathbf{z}}}

% Elements of random vectors
\def\erva{{\textnormal{a}}}
\def\ervb{{\textnormal{b}}}
\def\ervc{{\textnormal{c}}}
\def\ervd{{\textnormal{d}}}
\def\erve{{\textnormal{e}}}
\def\ervf{{\textnormal{f}}}
\def\ervg{{\textnormal{g}}}
\def\ervh{{\textnormal{h}}}
\def\ervi{{\textnormal{i}}}
\def\ervj{{\textnormal{j}}}
\def\ervk{{\textnormal{k}}}
\def\ervl{{\textnormal{l}}}
\def\ervm{{\textnormal{m}}}
\def\ervn{{\textnormal{n}}}
\def\ervo{{\textnormal{o}}}
\def\ervp{{\textnormal{p}}}
\def\ervq{{\textnormal{q}}}
\def\ervr{{\textnormal{r}}}
\def\ervs{{\textnormal{s}}}
\def\ervt{{\textnormal{t}}}
\def\ervu{{\textnormal{u}}}
\def\ervv{{\textnormal{v}}}
\def\ervw{{\textnormal{w}}}
\def\ervx{{\textnormal{x}}}
\def\ervy{{\textnormal{y}}}
\def\ervz{{\textnormal{z}}}

% Random matrices
\def\rmA{{\mathbf{A}}}
\def\rmB{{\mathbf{B}}}
\def\rmC{{\mathbf{C}}}
\def\rmD{{\mathbf{D}}}
\def\rmE{{\mathbf{E}}}
\def\rmF{{\mathbf{F}}}
\def\rmG{{\mathbf{G}}}
\def\rmH{{\mathbf{H}}}
\def\rmI{{\mathbf{I}}}
\def\rmJ{{\mathbf{J}}}
\def\rmK{{\mathbf{K}}}
\def\rmL{{\mathbf{L}}}
\def\rmM{{\mathbf{M}}}
\def\rmN{{\mathbf{N}}}
\def\rmO{{\mathbf{O}}}
\def\rmP{{\mathbf{P}}}
\def\rmQ{{\mathbf{Q}}}
\def\rmR{{\mathbf{R}}}
\def\rmS{{\mathbf{S}}}
\def\rmT{{\mathbf{T}}}
\def\rmU{{\mathbf{U}}}
\def\rmV{{\mathbf{V}}}
\def\rmW{{\mathbf{W}}}
\def\rmX{{\mathbf{X}}}
\def\rmY{{\mathbf{Y}}}
\def\rmZ{{\mathbf{Z}}}

% Elements of random matrices
\def\ermA{{\textnormal{A}}}
\def\ermB{{\textnormal{B}}}
\def\ermC{{\textnormal{C}}}
\def\ermD{{\textnormal{D}}}
\def\ermE{{\textnormal{E}}}
\def\ermF{{\textnormal{F}}}
\def\ermG{{\textnormal{G}}}
\def\ermH{{\textnormal{H}}}
\def\ermI{{\textnormal{I}}}
\def\ermJ{{\textnormal{J}}}
\def\ermK{{\textnormal{K}}}
\def\ermL{{\textnormal{L}}}
\def\ermM{{\textnormal{M}}}
\def\ermN{{\textnormal{N}}}
\def\ermO{{\textnormal{O}}}
\def\ermP{{\textnormal{P}}}
\def\ermQ{{\textnormal{Q}}}
\def\ermR{{\textnormal{R}}}
\def\ermS{{\textnormal{S}}}
\def\ermT{{\textnormal{T}}}
\def\ermU{{\textnormal{U}}}
\def\ermV{{\textnormal{V}}}
\def\ermW{{\textnormal{W}}}
\def\ermX{{\textnormal{X}}}
\def\ermY{{\textnormal{Y}}}
\def\ermZ{{\textnormal{Z}}}

% Vectors
\def\vzero{{\bm{0}}}
\def\vone{{\bm{1}}}
% \def\vmu{{\bm{\mu}}}
% \def\vtheta{{\bm{\theta}}}
\def\va{{\bm{a}}}
\def\vb{{\bm{b}}}
\def\vc{{\bm{c}}}
\def\vd{{\bm{d}}}
\def\ve{{\bm{e}}}
\def\vf{{\bm{f}}}
\def\vg{{\bm{g}}}
\def\vh{{\bm{h}}}
\def\vi{{\bm{i}}}
\def\vj{{\bm{j}}}
\def\vk{{\bm{k}}}
\def\vl{{\bm{l}}}
\def\vm{{\bm{m}}}
\def\vn{{\bm{n}}}
\def\vo{{\bm{o}}}
\def\vp{{\bm{p}}}
\def\vq{{\bm{q}}}
\def\vr{{\bm{r}}}
\def\vs{{\bm{s}}}
\def\vt{{\bm{t}}}
\def\vu{{\bm{u}}}
\def\vv{{\bm{v}}}
\def\vw{{\bm{w}}}
\def\vx{{\bm{x}}}
\def\vy{{\bm{y}}}
\def\vz{{\bm{z}}}

% Elements of vectors
\def\evalpha{{\alpha}}
\def\evbeta{{\beta}}
\def\evepsilon{{\epsilon}}
\def\evlambda{{\lambda}}
\def\evomega{{\omega}}
\def\evmu{{\mu}}
\def\evpsi{{\psi}}
\def\evsigma{{\sigma}}
\def\evtheta{{\theta}}
\def\eva{{a}}
\def\evb{{b}}
\def\evc{{c}}
\def\evd{{d}}
\def\eve{{e}}
\def\evf{{f}}
\def\evg{{g}}
\def\evh{{h}}
\def\evi{{i}}
\def\evj{{j}}
\def\evk{{k}}
\def\evl{{l}}
\def\evm{{m}}
\def\evn{{n}}
\def\evo{{o}}
\def\evp{{p}}
\def\evq{{q}}
\def\evr{{r}}
\def\evs{{s}}
\def\evt{{t}}
\def\evu{{u}}
\def\evv{{v}}
\def\evw{{w}}
\def\evx{{x}}
\def\evy{{y}}
\def\evz{{z}}

% Matrix
% \def\mA{{\bm{A}}}
% \def\mB{{\bm{B}}}
% \def\mC{{\bm{C}}}
% \def\mD{{\bm{D}}}
% \def\mE{{\bm{E}}}
% \def\mF{{\bm{F}}}
% \def\mG{{\bm{G}}}
% \def\mH{{\bm{H}}}
% \def\mI{{\bm{I}}}
% \def\mJ{{\bm{J}}}
% \def\mK{{\bm{K}}}
% \def\mL{{\bm{L}}}
% \def\mM{{\bm{M}}}
% \def\mN{{\bm{N}}}
% \def\mO{{\bm{O}}}
% \def\mP{{\bm{P}}}
% \def\mQ{{\bm{Q}}}
% \def\mR{{\bm{R}}}
% \def\mS{{\bm{S}}}
% \def\mT{{\bm{T}}}
% \def\mU{{\bm{U}}}
% \def\mV{{\bm{V}}}
% \def\mW{{\bm{W}}}
% \def\mX{{\bm{X}}}
% \def\mY{{\bm{Y}}}
% \def\mZ{{\bm{Z}}}
% \def\mBeta{{\bm{\beta}}}
% \def\mPhi{{\bm{\Phi}}}
% \def\mLambda{{\bm{\Lambda}}}
% \def\mSigma{{\bm{\Sigma}}}

% Tensor
\DeclareMathAlphabet{\mathsfit}{\encodingdefault}{\sfdefault}{m}{sl}
\SetMathAlphabet{\mathsfit}{bold}{\encodingdefault}{\sfdefault}{bx}{n}
\newcommand{\tens}[1]{\bm{\mathsfit{#1}}}
\def\tA{{\tens{A}}}
\def\tB{{\tens{B}}}
\def\tC{{\tens{C}}}
\def\tD{{\tens{D}}}
\def\tE{{\tens{E}}}
\def\tF{{\tens{F}}}
\def\tG{{\tens{G}}}
\def\tH{{\tens{H}}}
\def\tI{{\tens{I}}}
\def\tJ{{\tens{J}}}
\def\tK{{\tens{K}}}
\def\tL{{\tens{L}}}
\def\tM{{\tens{M}}}
\def\tN{{\tens{N}}}
\def\tO{{\tens{O}}}
\def\tP{{\tens{P}}}
\def\tQ{{\tens{Q}}}
\def\tR{{\tens{R}}}
\def\tS{{\tens{S}}}
\def\tT{{\tens{T}}}
\def\tU{{\tens{U}}}
\def\tV{{\tens{V}}}
\def\tW{{\tens{W}}}
\def\tX{{\tens{X}}}
\def\tY{{\tens{Y}}}
\def\tZ{{\tens{Z}}}


% Graph
\def\gA{{\mathcal{A}}}
\def\gB{{\mathcal{B}}}
\def\gC{{\mathcal{C}}}
\def\gD{{\mathcal{D}}}
\def\gE{{\mathcal{E}}}
\def\gF{{\mathcal{F}}}
\def\gG{{\mathcal{G}}}
\def\gH{{\mathcal{H}}}
\def\gI{{\mathcal{I}}}
\def\gJ{{\mathcal{J}}}
\def\gK{{\mathcal{K}}}
\def\gL{{\mathcal{L}}}
\def\gM{{\mathcal{M}}}
\def\gN{{\mathcal{N}}}
\def\gO{{\mathcal{O}}}
\def\gP{{\mathcal{P}}}
\def\gQ{{\mathcal{Q}}}
\def\gR{{\mathcal{R}}}
\def\gS{{\mathcal{S}}}
\def\gT{{\mathcal{T}}}
\def\gU{{\mathcal{U}}}
\def\gV{{\mathcal{V}}}
\def\gW{{\mathcal{W}}}
\def\gX{{\mathcal{X}}}
\def\gY{{\mathcal{Y}}}
\def\gZ{{\mathcal{Z}}}

% Sets
\def\sA{{\mathbb{A}}}
\def\sB{{\mathbb{B}}}
\def\sC{{\mathbb{C}}}
\def\sD{{\mathbb{D}}}
% Don't use a set called E, because this would be the same as our symbol
% for expectation.
\def\sF{{\mathbb{F}}}
\def\sG{{\mathbb{G}}}
\def\sH{{\mathbb{H}}}
\def\sI{{\mathbb{I}}}
\def\sJ{{\mathbb{J}}}
\def\sK{{\mathbb{K}}}
\def\sL{{\mathbb{L}}}
\def\sM{{\mathbb{M}}}
\def\sN{{\mathbb{N}}}
\def\sO{{\mathbb{O}}}
\def\sP{{\mathbb{P}}}
\def\sQ{{\mathbb{Q}}}
\def\sR{{\mathbb{R}}}
\def\sS{{\mathbb{S}}}
\def\sT{{\mathbb{T}}}
\def\sU{{\mathbb{U}}}
\def\sV{{\mathbb{V}}}
\def\sW{{\mathbb{W}}}
\def\sX{{\mathbb{X}}}
\def\sY{{\mathbb{Y}}}
\def\sZ{{\mathbb{Z}}}

% Entries of a matrix
\def\emLambda{{\Lambda}}
\def\emA{{A}}
\def\emB{{B}}
\def\emC{{C}}
\def\emD{{D}}
\def\emE{{E}}
\def\emF{{F}}
\def\emG{{G}}
\def\emH{{H}}
\def\emI{{I}}
\def\emJ{{J}}
\def\emK{{K}}
\def\emL{{L}}
\def\emM{{M}}
\def\emN{{N}}
\def\emO{{O}}
\def\emP{{P}}
\def\emQ{{Q}}
\def\emR{{R}}
\def\emS{{S}}
\def\emT{{T}}
\def\emU{{U}}
\def\emV{{V}}
\def\emW{{W}}
\def\emX{{X}}
\def\emY{{Y}}
\def\emZ{{Z}}
\def\emSigma{{\Sigma}}

% entries of a tensor
% Same font as tensor, without \bm wrapper
\newcommand{\etens}[1]{\mathsfit{#1}}
\def\etLambda{{\etens{\Lambda}}}
\def\etA{{\etens{A}}}
\def\etB{{\etens{B}}}
\def\etC{{\etens{C}}}
\def\etD{{\etens{D}}}
\def\etE{{\etens{E}}}
\def\etF{{\etens{F}}}
\def\etG{{\etens{G}}}
\def\etH{{\etens{H}}}
\def\etI{{\etens{I}}}
\def\etJ{{\etens{J}}}
\def\etK{{\etens{K}}}
\def\etL{{\etens{L}}}
\def\etM{{\etens{M}}}
\def\etN{{\etens{N}}}
\def\etO{{\etens{O}}}
\def\etP{{\etens{P}}}
\def\etQ{{\etens{Q}}}
\def\etR{{\etens{R}}}
\def\etS{{\etens{S}}}
\def\etT{{\etens{T}}}
\def\etU{{\etens{U}}}
\def\etV{{\etens{V}}}
\def\etW{{\etens{W}}}
\def\etX{{\etens{X}}}
\def\etY{{\etens{Y}}}
\def\etZ{{\etens{Z}}}


\newcommand{\calA}{\ensuremath{\mathcal{A}}}
\newcommand{\calB}{\ensuremath{\mathcal{B}}}
\newcommand{\calC}{\ensuremath{\mathcal{C}}}
\newcommand{\calD}{\ensuremath{\mathcal{D}}}
\newcommand{\calE}{\ensuremath{\mathcal{E}}}
\newcommand{\calF}{\ensuremath{\mathcal{F}}}
\newcommand{\calG}{\ensuremath{\mathcal{G}}}
\newcommand{\calH}{\ensuremath{\mathcal{H}}}
\newcommand{\calI}{\ensuremath{\mathcal{I}}}
\newcommand{\calJ}{\ensuremath{\mathcal{J}}}
\newcommand{\calK}{\ensuremath{\mathcal{K}}}
\newcommand{\calL}{\ensuremath{\mathcal{L}}}
\newcommand{\calM}{\ensuremath{\mathcal{M}}}
\newcommand{\calN}{\ensuremath{\mathcal{N}}}
\newcommand{\calO}{\ensuremath{\mathcal{O}}}
\newcommand{\calP}{\ensuremath{\mathcal{P}}}
\newcommand{\calQ}{\ensuremath{\mathcal{Q}}}
\newcommand{\calR}{\ensuremath{\mathcal{R}}}
\newcommand{\calS}{\ensuremath{\mathcal{S}}}
\newcommand{\calT}{\ensuremath{\mathcal{T}}}
\newcommand{\calU}{\ensuremath{\mathcal{U}}}
\newcommand{\calV}{\ensuremath{\mathcal{V}}}
\newcommand{\calW}{\ensuremath{\mathcal{W}}}
\newcommand{\calX}{\ensuremath{\mathcal{X}}}
\newcommand{\calY}{\ensuremath{\mathcal{Y}}}
\newcommand{\calZ}{\ensuremath{\mathcal{Z}}}



\newcommand{\vxstar}{\vx^*}
\newcommand{\vystar}{\vy^*}

\newcommand{\Amax}{\mat{A}_{\max}}
\newcommand{\Amin}{\mat{A}_{\min}}

\newcommand{\E}{\mathbb{E}}
\newcommand{\Ls}{\mathcal{L}}
%\newcommand{\Linv}{L_{\lambda_{\mathrm{inv}}}}
\newcommand{\R}{\mathbb{R}}
%\newcommand{\emp}{\tilde{p}}
%\newcommand{\lr}{\alpha}
%\newcommand{\reg}{\lambda}
%\newcommand{\rect}{\mathrm{rectifier}}
%\newcommand{\softmax}{\mathrm{softmax}}
%\newcommand{\sigmoid}{\sigma}
%\newcommand{\softplus}{\zeta}
%\newcommand{\KL}{D_{\mathrm{KL}}}
%\newcommand{\Var}{\mathrm{Var}}
%\newcommand{\standarderror}{\mathrm{SE}}
%\newcommand{\Cov}{\mathrm{Cov}}
% Wolfram Mathworld says $L^2$ is for function spaces and $\ell^2$ is for vectors
% But then they seem to use $L^2$ for vectors throughout the site, and so does
% wikipedia.
\newcommand{\normlzero}{L^0}
\newcommand{\normlone}{L^1}
\newcommand{\normltwo}{L^2}
\newcommand{\normlp}{L^p}
\newcommand{\normmax}{L^\infty}

\newcommand{\parents}{Pa} % See usage in notation.tex. Chosen to match Daphne's book.

% \DeclareMathOperator*{\argmax}{arg\,max}
% \DeclareMathOperator*{\argmin}{arg\,min}

% \DeclareMathOperator{\sign}{sign}
% \DeclareMathOperator{\Tr}{Tr}
% \let\ab\allowbreak

% \newtheorem{claim}[theorem]{Claim}
% \newtheorem*{nclaim}{Claim}

%%%%%% Paper specific commands
% \renewcommand{\diam}{\mathrm{diam}}
\newcommand{\metric}{\mathrm{d}}
\newcommand{\hausdorff}{\mathrm{d}_{\rmH}}
\newcommand{\size}{\mathrm{size}}
% \newcommand{\nm}[1]{\b{#1} - 1}
% \renewcommand{\sign}{\mathrm{sign}}
\renewcommand{\bar}[1]{\overline{#1}}


\newcommand{\tstar}{t^{\star}}

\newcommand{\autakia}[1]{``#1''}

\newcommand{\step}[1]{^{(#1)} }
\newcommand{\projoper}{\Pi_{\Delta}}
\newcommand{\projoperX}[1]{{\Pi_{\calX_{#1}}}}
\newcommand{\projoperY}[1]{{\Pi_{\calY_{#1}}}}
\newcommand{\projoperZ}[1]{{\Pi_{\calZ_{#1}}}}


% \renewcommand{\vv}[1]{\mathbf{#1}}
% \AtBeginDocument{\renewcommand{\vec}[1]{\mathbf{#1}}}

%%%%% New commands
% % \renewcommand{\algorithmicrequire}{\textbf{Input:}}
% \renewcommand{\algorithmicensure}{\textbf{Output:}}
% \newcommand{\algrule}[1][.2pt]{\par\vskip.5\baselineskip\hrule height #1\par\vskip.5\baselineskip}

\usepackage{subfiles} % Best loaded last in the preamble
\usepackage{amsthm}
\usepackage[shortlabels]{enumitem}

\usepackage[thinc]{esdiff}

\newcommand{\Ex}{\mathbb{E}}
\newcommand{\E}{\mathrm{E}}
\newcommand{\I}{\mathrm{I}}
\newcommand{\W}{\mathbf{W}}
\newcommand{\U}{\mathbf{U}}
\newcommand{\h}{\mathbf{h}}
\newcommand{\x}{\mathbf{x}}
\newcommand{\X}{\mathbf{X}}
\newcommand{\F}{\mathbf{F}}
\newcommand{\Y}{\mathbf{Y}}
\newcommand{\g}{\mathbf{g}}
\newcommand{\z}{\mathbf{z}}
\newcommand{\R}{\mathbb{R}}
\newcommand{\N}{\mathbb{N}}
\newcommand{\Var}{\mathrm{Var}}

\newcommand\ddfrac[2]{\frac{\displaystyle #1}{\displaystyle #2}}


\newcommand{\vones}{\mathbbm{1}}
\newcommand{\p}{\partial}
\newcommand\w{{\mathbf{w}}}
\newcommand{\Cov}{\text{Cov}}
\newcommand{\be}{\begin{equation}}
\newcommand{\ee}{\end{equation}}
\newcommand{\s}{\subseteq}
\newcommand{\bbmat}{\begin{bmatrix}}
\newcommand{\ebmat}{\end{bmatrix}}

\def\bea#1\eea{\begin{align}#1\end{align}}

\newcommand{\Target}{{\mathbf y}}
\newcommand{\normX}{{\chi}}
\newcommand{\noise}{{\xi}}
\newcommand{\actFunc}{{f}}
\newcommand{\gain}{{g}}
\newcommand{\Gain}{{\mathbf g}}
%\newcommand{\bias}{{b}}
\newcommand{\Bias}{{\mathbf b}}
\newcommand{\Weights}{{W}}
\newcommand{\weights}{{w}}
\newcommand{\hidden}{{h}}
\newcommand{\Hidden}{{\mathbf h}}
\newcommand{\act}{{a}}
\newcommand{\Act}{{\mathbf a}}
\newcommand{\fisher}{{F}}
\newcommand{\data}{{\mathbf x}}
\newcommand{\loss}{\mathcal{L}}
\newcommand{\expectation}{\mathop{\mathbb{E}}}
%\newcommand{\kldiv}{\mathrm{D}_{\rm KL}}
%\newcommand{\klBars}{\,\|\,}
\newcommand{\entropy}{\mathcal{H}}
\newcommand{\partitionFn}{\mathcal{Z}}
\newcommand{\given}{\,|\,}

\newcommand{\kldiv}{\mathrm{D}_{\rm KL}}
\newcommand{\klBars}{\,\|\,}

\newcommand{\bdelta}{\boldsymbol{\delta}}
\newcommand{\btheta}{\boldsymbol{\theta}}
\newcommand{\balpha}{\boldsymbol{\alpha}}
\newcommand{\y}{\mathbf{y}}
\newcommand{\ba}{\mathbf{a}}
\newcommand{\bias}{{\mathbf{b}}}
\newcommand{\xxTmatrix}{\left[ 
			\begin{array}{cc}  
				\mathbf{x}\mathbf{x}^T 	& \mathbf{x}\\ 
				\mathbf{x}^T 			& 1 
			\end{array}
		\right]}

\newcommand{\titleappendix}[1]{
    \vbox{%
        \hsize\textwidth\linewidth\hsize
        \vskip 0.1in \vskip -\parskip \hrule height 1pt \vskip 0.09in \vskip 0.2in
        \centering{\LARGE\sc #1\par}
        \vskip 0.29in \vskip -\parskip \hrule height 1pt \vskip 0.09in \vskip 0.2in
    }
}

\newtheorem*{theorem*}{Theorem}


% \usepackage[urlcolor=blue]{hyperref}
% \usepackage{enumerate}
% \usepackage{amsmath}
% \usepackage{amsthm}
%\usepackage[short,nocomma]{optidef}
% \newtheorem{theorem}{Theorem}
% \newtheorem{lemma}{Lemma}

% \usepackage[lmargin=2cm, rmargin=2cm, tmargin=2cm, bmargin=2.5cm]{geometry}
% \usepackage{natbib}
% \bibliographystyle{unsrtnat}

%%%%%%%%%%%%%%%%%%%%Paper specific
%%%%%%%%%%%%%%%%%%%%%%%%%%%%%%%%%%%%%%%%

\newcommand{\px}{\bm{x}}
\newcommand{\pxhat}{\hat{\bm{x}}}
\newcommand{\pxtilde}{\tilde{\bm{x}}}

\newcommand{\py}{\bm{y}}
\newcommand{\pyhat}{\hat{\bm{y}}}
\newcommand{\pytilde}{\tilde{\bm{y}}}

\newcommand{\allagents}{i \in \calN}
\newcommand{\policy}[1]{\boldsymbol{\pi}_{{#1}}}
\newcommand{\adv}{\text{adv}}
\newcommand{\sign}{\mathrm{sign}}
\newcommand{\diag}{\mathrm{diag}}


\newcommand{\norm}[1]{\left\| #1 \right\|}
%%%%%%%%%%%%%%%%%%%%%%%%%%%%%%%%%%%%%%%%
\newcommand{\lpref}[1]{\oldref{#1}}
%%%%%%%%%%%%%%%%%%%%%%%%%%%%%%%%%%%%%%%%
\newcommand*\vell{\ensuremath{\boldsymbol\ell}}

\newcommand{\valpha}{\boldsymbol{\alpha} }
\newcommand{\vbeta}{\boldsymbol{\beta} }
\newcommand{\vgamma}{\boldsymbol{\gamma} }
\newcommand{\vdelta}{\boldsymbol{\delta} }
\newcommand{\vepsilon}{\boldsymbol{\epsilon} }
\newcommand{\vzeta}{\boldsymbol{\zeta} }
\newcommand{\veta}{\boldsymbol{\eta} }
\newcommand{\vtheta}{\boldsymbol{\theta} }
\newcommand{\viota}{\boldsymbol{\iota} }
\newcommand{\vkappa}{\boldsymbol{\kappa} }
\newcommand{\vlambda}{\boldsymbol{\lambda} }
\newcommand{\vmu}{\boldsymbol{\mu} }
\newcommand{\vnu}{\boldsymbol{\nu} }
\newcommand{\vxi}{\boldsymbol{\xi} }
\newcommand{\vomikron}{\boldsymbol{\omikron} }
\newcommand{\vpi}{\boldsymbol{\pi} }
\newcommand{\vrho}{\boldsymbol{\rho} }
\newcommand{\vsigma}{\boldsymbol{\sigma} }
\newcommand{\vtau}{\boldsymbol{\tau} }
\newcommand{\vypsilon}{\boldsymbol{\ypsilon} }
\newcommand{\vphi}{\boldsymbol{\phi} }
\newcommand{\vchi}{\boldsymbol{\chi} }
\newcommand{\vpsi}{\boldsymbol{\psi} }
\newcommand{\vomega}{\boldsymbol{\omega} }
\DeclareMathOperator*{\argmax}{argmax}
\DeclareMathOperator*{\argmin}{argmin}

\DeclareMathOperator{\dom}{dom}
\DeclareMathOperator{\dist}{dist}
\newcommand{\prox}{\mathrm{prox}}
\newcommand{\paren}[1]{\left(#1\right)}

\newcommand{\inprod}[2]{\left\langle#1,#2\right\rangle}
\newcommand{\mybrace}[1]{\left\{ #1 \right\}}

\newcommand{\mybracket}[1]{\left[ #1 \right]}

\usepackage{amsthm}
\usepackage{mathtools}
\usepackage{amsmath}
\usepackage{bbm}
\usepackage{amsfonts}


%\theoremstyle{definition}  %
% \newtheorem{exercise}{Exercise}
% \newtheorem{claim}{Claim}
% \newtheorem{lemma}{Lemma}
% \newtheorem{conjecture}{Conjecture}
% \newtheorem{corollary}{Corollary}
% \newtheorem{proposition}{Proposition}
% \newtheorem{fact}{Fact}
% \newtheorem{assumption}{Assumption}
% \newtheorem{problem}{Problem}
% \newtheorem{question}{Question}
% \newtheorem{model}{Model}
% \theoremstyle{plain}
% \newtheorem{remark}{Remark}
% \newtheorem{example}{Example}
% \newtheorem{theorem}{Theorem}
% \newtheorem{observation}{Observation}

% \newtheorem{definition}{Definition}
% \newtheorem{openproblem}{Open Problem}

%\numberwithin{observation}{section}


%\numberwithin{claim}{section}
%\numberwithin{fact}{section}
%\numberwithin{lemma}{section}
%\numberwithin{proposition}{section}
%\numberwithin{theorem}{section}
%\numberwithin{corollary}{section}
%\numberwithin{definition}{section}
%\numberwithin{assumption}{section}

\usepackage{nicefrac}

\usepackage{tikz}
\usetikzlibrary{positioning}
\usetikzlibrary{shapes,arrows}

%%%%%%%%%%%%%%%%%%%%%%%%%%%%%%%%
% \input{macros/nlp-env}
% \usepackage{lpform}
%%%%%%%%%%%%%%%%%%%%%%%%%%%%%%%%
\usepackage{subfigure}
%%%%%%%%%%%%%%%%%%%%%%%%%%%%%%%

\usepackage[%
linewidth=2pt,
linecolor=gray,
middlelinecolor= black,
middlelinewidth=0.4pt,
roundcorner=1pt,
topline = false,
rightline = false,
bottomline = false,
rightmargin=0pt,
skipabove=0pt,
skipbelow=0pt,
leftmargin=0pt,
innerleftmargin=4pt,
innerrightmargin=0pt,
innertopmargin=0pt,
innerbottommargin=0pt,
]{mdframed}

\usepackage[capitalize,noabbrev,nameinlink]{cleveref}

\usepackage{enumitem}


\usepackage[natbib=true,maxcitenames=3,maxnames=3,style=alphabetic]{biblatex}
\usepackage{biblatex2bibitem}


\renewcommand{\vec}[1]{\bm{#1}}
\newcommand{\mat}[1]{\mathbf{#1}}
\newcommand{\io}[1]{\textcolor{olive}{[* Ioannis: #1 *]}}
\newcommand{\jy}[1]{\textcolor{blue}{[* Jingming: #1 *]}}
\newcommand{\todo}[1]{\textcolor{red}{[#1]}}

\newenvironment{nproblem}[1][\unskip]{%
  \medskip
  \begin{mdframed}
  \noindent
  \textbf{\underline{$#1$ Problem.}} \\
  \noindent
}{%
  \end{mdframed}
  \medskip % Add this line to create vertical space after the environment
}

\newcommand{\mismatch}{D_{\mathrm{m}}}
\newcommand{\gdaFixed}{\textsc{GDAFixedPoint}}
\newcommand{\symgdaFixed}{\textsc{SymGDAFixedPoint}}
\newcommand{\nsymgdaFixed}{\textsc{NonSymGDAFixedPoint}}

%\usepackage{autonum}

\usepackage{thm-restate}

\theoremstyle{plain}
\newtheorem{theorem}{Theorem}[section]
\newtheorem{lemma}[theorem]{Lemma}
\newtheorem{corollary}[theorem]{Corollary}
\newtheorem{proposition}[theorem]{Proposition}
\newtheorem{fact}[theorem]{Fact}
\newtheorem{conjecture}[theorem]{Conjecture}
\newtheorem{observation}[theorem]{Observation}
\newtheorem{claim}[theorem]{Claim}
\newtheorem{assumption}[theorem]{Assumption}
\newtheorem{property}[theorem]{Property}
\newtheorem{condition}[theorem]{Condition}

\theoremstyle{definition}
\newtheorem{definition}[theorem]{Definition}

\theoremstyle{remark}
\newtheorem{remark}[theorem]{Remark}
\newtheorem{example}[theorem]{Example}

\title{The Complexity of Symmetric Equilibria in Min-Max Optimization and Team Zero-Sum Games\thanks{The authors are ordered alphabetically.}}

\author[1]{Ioannis Anagnostides}
\author[2]{Ioannis Panageas}
\author[3]{Tuomas Sandholm}
\author[4]{Jingming Yan}

\affil[1,3]{Carnegie Mellon University}
\affil[2,4]{University of California, Irvine}
\affil[3]{Strategy Robot, Inc.}
\affil[3]{Strategic Machine, Inc.}
\affil[3]{Optimized Markets, Inc.}
\affil[ ]{\texttt{\{ianagnos,sandholm\}}\texttt{@cs.cmu.edu}, \texttt{\{ipanagea,jingmy1\}}\texttt{@uci.edu}}

\bibliography{main}

\begin{document}

\maketitle

\begin{abstract}
    We consider the problem of computing stationary points in min-max optimization, with a particular focus on the special case of computing Nash equilibria in (two-)team zero-sum games.

    We first show that computing $\epsilon$-Nash equilibria in $3$-player \emph{adversarial} team games---wherein a team of $2$ players competes against a \emph{single} adversary---is \textsf{CLS}-complete, resolving the complexity of Nash equilibria in such settings. Our proof proceeds by reducing from \emph{symmetric} $\epsilon$-Nash equilibria in \emph{symmetric}, identical-payoff, two-player games, by suitably leveraging the adversarial player so as to enforce symmetry---without disturbing the structure of the game. In particular, the class of instances we construct comprises solely polymatrix games, thereby also settling a question left open by Hollender, Maystre, and Nagarajan (2024). We also provide some further results concerning equilibrium computation in adversarial team games.

    Moreover, we establish that computing \emph{symmetric} (first-order) equilibria in \emph{symmetric} min-max optimization is \textsf{PPAD}-complete, even for quadratic functions. Building on this reduction, we further show that computing symmetric $\epsilon$-Nash equilibria in symmetric, $6$-player ($3$ vs. $3$) team zero-sum games is also \textsf{PPAD}-complete, even for $\epsilon = \text{poly}(1/n)$. As an immediate corollary, this precludes the existence of symmetric dynamics---which includes many of the algorithms considered in the literature---converging to stationary points. Finally, we prove that computing a \emph{non-symmetric} $\text{poly}(1/n)$-equilibrium in symmetric min-max optimization is \FNP-hard.
\end{abstract}

\pagenumbering{gobble}

% Optionally include a table of contents
\clearpage
\tableofcontents

\clearpage

\pagenumbering{arabic}

\section{Introduction}


\begin{figure}[t]
\centering
\includegraphics[width=0.6\columnwidth]{figures/evaluation_desiderata_V5.pdf}
\vspace{-0.5cm}
\caption{\systemName is a platform for conducting realistic evaluations of code LLMs, collecting human preferences of coding models with real users, real tasks, and in realistic environments, aimed at addressing the limitations of existing evaluations.
}
\label{fig:motivation}
\end{figure}

\begin{figure*}[t]
\centering
\includegraphics[width=\textwidth]{figures/system_design_v2.png}
\caption{We introduce \systemName, a VSCode extension to collect human preferences of code directly in a developer's IDE. \systemName enables developers to use code completions from various models. The system comprises a) the interface in the user's IDE which presents paired completions to users (left), b) a sampling strategy that picks model pairs to reduce latency (right, top), and c) a prompting scheme that allows diverse LLMs to perform code completions with high fidelity.
Users can select between the top completion (green box) using \texttt{tab} or the bottom completion (blue box) using \texttt{shift+tab}.}
\label{fig:overview}
\end{figure*}

As model capabilities improve, large language models (LLMs) are increasingly integrated into user environments and workflows.
For example, software developers code with AI in integrated developer environments (IDEs)~\citep{peng2023impact}, doctors rely on notes generated through ambient listening~\citep{oberst2024science}, and lawyers consider case evidence identified by electronic discovery systems~\citep{yang2024beyond}.
Increasing deployment of models in productivity tools demands evaluation that more closely reflects real-world circumstances~\citep{hutchinson2022evaluation, saxon2024benchmarks, kapoor2024ai}.
While newer benchmarks and live platforms incorporate human feedback to capture real-world usage, they almost exclusively focus on evaluating LLMs in chat conversations~\citep{zheng2023judging,dubois2023alpacafarm,chiang2024chatbot, kirk2024the}.
Model evaluation must move beyond chat-based interactions and into specialized user environments.



 

In this work, we focus on evaluating LLM-based coding assistants. 
Despite the popularity of these tools---millions of developers use Github Copilot~\citep{Copilot}---existing
evaluations of the coding capabilities of new models exhibit multiple limitations (Figure~\ref{fig:motivation}, bottom).
Traditional ML benchmarks evaluate LLM capabilities by measuring how well a model can complete static, interview-style coding tasks~\citep{chen2021evaluating,austin2021program,jain2024livecodebench, white2024livebench} and lack \emph{real users}. 
User studies recruit real users to evaluate the effectiveness of LLMs as coding assistants, but are often limited to simple programming tasks as opposed to \emph{real tasks}~\citep{vaithilingam2022expectation,ross2023programmer, mozannar2024realhumaneval}.
Recent efforts to collect human feedback such as Chatbot Arena~\citep{chiang2024chatbot} are still removed from a \emph{realistic environment}, resulting in users and data that deviate from typical software development processes.
We introduce \systemName to address these limitations (Figure~\ref{fig:motivation}, top), and we describe our three main contributions below.


\textbf{We deploy \systemName in-the-wild to collect human preferences on code.} 
\systemName is a Visual Studio Code extension, collecting preferences directly in a developer's IDE within their actual workflow (Figure~\ref{fig:overview}).
\systemName provides developers with code completions, akin to the type of support provided by Github Copilot~\citep{Copilot}. 
Over the past 3 months, \systemName has served over~\completions suggestions from 10 state-of-the-art LLMs, 
gathering \sampleCount~votes from \userCount~users.
To collect user preferences,
\systemName presents a novel interface that shows users paired code completions from two different LLMs, which are determined based on a sampling strategy that aims to 
mitigate latency while preserving coverage across model comparisons.
Additionally, we devise a prompting scheme that allows a diverse set of models to perform code completions with high fidelity.
See Section~\ref{sec:system} and Section~\ref{sec:deployment} for details about system design and deployment respectively.



\textbf{We construct a leaderboard of user preferences and find notable differences from existing static benchmarks and human preference leaderboards.}
In general, we observe that smaller models seem to overperform in static benchmarks compared to our leaderboard, while performance among larger models is mixed (Section~\ref{sec:leaderboard_calculation}).
We attribute these differences to the fact that \systemName is exposed to users and tasks that differ drastically from code evaluations in the past. 
Our data spans 103 programming languages and 24 natural languages as well as a variety of real-world applications and code structures, while static benchmarks tend to focus on a specific programming and natural language and task (e.g. coding competition problems).
Additionally, while all of \systemName interactions contain code contexts and the majority involve infilling tasks, a much smaller fraction of Chatbot Arena's coding tasks contain code context, with infilling tasks appearing even more rarely. 
We analyze our data in depth in Section~\ref{subsec:comparison}.



\textbf{We derive new insights into user preferences of code by analyzing \systemName's diverse and distinct data distribution.}
We compare user preferences across different stratifications of input data (e.g., common versus rare languages) and observe which affect observed preferences most (Section~\ref{sec:analysis}).
For example, while user preferences stay relatively consistent across various programming languages, they differ drastically between different task categories (e.g. frontend/backend versus algorithm design).
We also observe variations in user preference due to different features related to code structure 
(e.g., context length and completion patterns).
We open-source \systemName and release a curated subset of code contexts.
Altogether, our results highlight the necessity of model evaluation in realistic and domain-specific settings.





\section{Background}\label{sec:backgrnd}

\subsection{Cold Start Latency and Mitigation Techniques}

Traditional FaaS platforms mitigate cold starts through snapshotting, lightweight virtualization, and warm-state management. Snapshot-based methods like \textbf{REAP} and \textbf{Catalyzer} reduce initialization time by preloading or restoring container states but require significant memory and I/O resources, limiting scalability~\cite{dong_catalyzer_2020, ustiugov_benchmarking_2021}. Lightweight virtualization solutions, such as \textbf{Firecracker} microVMs, achieve fast startup times with strong isolation but depend on robust infrastructure, making them less adaptable to fluctuating workloads~\cite{agache_firecracker_2020}. Warm-state management techniques like \textbf{Faa\$T}~\cite{romero_faa_2021} and \textbf{Kraken}~\cite{vivek_kraken_2021} keep frequently invoked containers ready, balancing readiness and cost efficiency under predictable workloads but incurring overhead when demand is erratic~\cite{romero_faa_2021, vivek_kraken_2021}. While these methods perform well in resource-rich cloud environments, their resource intensity challenges applicability in edge settings.

\subsubsection{Edge FaaS Perspective}

In edge environments, cold start mitigation emphasizes lightweight designs, resource sharing, and hybrid task distribution. Lightweight execution environments like unikernels~\cite{edward_sock_2018} and \textbf{Firecracker}~\cite{agache_firecracker_2020}, as used by \textbf{TinyFaaS}~\cite{pfandzelter_tinyfaas_2020}, minimize resource usage and initialization delays but require careful orchestration to avoid resource contention. Function co-location, demonstrated by \textbf{Photons}~\cite{v_dukic_photons_2020}, reduces redundant initializations by sharing runtime resources among related functions, though this complicates isolation in multi-tenant setups~\cite{v_dukic_photons_2020}. Hybrid offloading frameworks like \textbf{GeoFaaS}~\cite{malekabbasi_geofaas_2024} balance edge-cloud workloads by offloading latency-tolerant tasks to the cloud and reserving edge resources for real-time operations, requiring reliable connectivity and efficient task management. These edge-specific strategies address cold starts effectively but introduce challenges in scalability and orchestration.

\subsection{Predictive Scaling and Caching Techniques}

Efficient resource allocation is vital for maintaining low latency and high availability in serverless platforms. Predictive scaling and caching techniques dynamically provision resources and reduce cold start latency by leveraging workload prediction and state retention.
Traditional FaaS platforms use predictive scaling and caching to optimize resources, employing techniques (OFC, FaasCache) to reduce cold starts. However, these methods rely on centralized orchestration and workload predictability, limiting their effectiveness in dynamic, resource-constrained edge environments.



\subsubsection{Edge FaaS Perspective}

Edge FaaS platforms adapt predictive scaling and caching techniques to constrain resources and heterogeneous environments. \textbf{EDGE-Cache}~\cite{kim_delay-aware_2022} uses traffic profiling to selectively retain high-priority functions, reducing memory overhead while maintaining readiness for frequent requests. Hybrid frameworks like \textbf{GeoFaaS}~\cite{malekabbasi_geofaas_2024} implement distributed caching to balance resources between edge and cloud nodes, enabling low-latency processing for critical tasks while offloading less critical workloads. Machine learning methods, such as clustering-based workload predictors~\cite{gao_machine_2020} and GRU-based models~\cite{guo_applying_2018}, enhance resource provisioning in edge systems by efficiently forecasting workload spikes. These innovations effectively address cold start challenges in edge environments, though their dependency on accurate predictions and robust orchestration poses scalability challenges.

\subsection{Decentralized Orchestration, Function Placement, and Scheduling}

Efficient orchestration in serverless platforms involves workload distribution, resource optimization, and performance assurance. While traditional FaaS platforms rely on centralized control, edge environments require decentralized and adaptive strategies to address unique challenges such as resource constraints and heterogeneous hardware.



\subsubsection{Edge FaaS Perspective}

Edge FaaS platforms adopt decentralized and adaptive orchestration frameworks to meet the demands of resource-constrained environments. Systems like \textbf{Wukong} distribute scheduling across edge nodes, enhancing data locality and scalability while reducing network latency. Lightweight frameworks such as \textbf{OpenWhisk Lite}~\cite{kravchenko_kpavelopenwhisk-light_2024} optimize resource allocation by decentralizing scheduling policies, minimizing cold starts and latency in edge setups~\cite{benjamin_wukong_2020}. Hybrid solutions like \textbf{OpenFaaS}~\cite{noauthor_openfaasfaas_2024} and \textbf{EdgeMatrix}~\cite{shen_edgematrix_2023} combine edge-cloud orchestration to balance resource utilization, retaining latency-sensitive functions at the edge while offloading non-critical workloads to the cloud. While these approaches improve flexibility, they face challenges in maintaining coordination and ensuring consistent performance across distributed nodes.


\begin{table}[ht!]
\centering
\caption{\textbf{Classification Performance Results on the Berlin BCI Competition III, Dataset V test set.} We remove 16 (\textit{scale=2)} and 24 (\textit{scale=4)} of the dataset's 32 EEG channels and use our proposed WGAN spatial upsampling method to recreate the missing channels using the remaining channels. We then train and evaluate a classifier on the upsampled channels and show minimal loss in performance compared to using the original dataset channels.}
\label{tab:cls}
\resizebox{0.9\linewidth}{!}{%
\begin{tabular}{@{}ccccc@{}}
\toprule
\textbf{Scale} &
\textbf{Metric} &
\textbf{Class} &
\textbf{HR} &
\textbf{WGAN (Ours)} \\
\toprule
\multirow{7}{*}{2} & Accuracy                   & - & 87.75 & 83.88 \\ \cmidrule(l){2-5} 
                   & \multirow{3}{*}{Precision} & 2 & 86.65 & 82.24 \\
                   &                            & 3 & 87.54 & 82.84 \\
                   &                            & 7 & 88.77 & 86.12 \\ \cmidrule(l){2-5} 
                   & \multirow{3}{*}{Recall}    & 2 & 84.33 & 80.21 \\
                   &                            & 3 & 88.57 & 84.37 \\
                   &                            & 7 & 89.62 & 86.26 \\ \midrule
\multirow{7}{*}{4} & Accuracy                   & - & 87.75 & 82.00 \\ \cmidrule(l){2-5} 
                   & \multirow{3}{*}{Precision} & 2 & 86.65 & 80.21 \\
                   &                            & 3 & 87.54 & 82.03 \\
                   &                            & 7 & 88.77 & 83.28 \\ \cmidrule(l){2-5} 
                   & \multirow{3}{*}{Recall}    & 2 & 84.33 & 77.73 \\
                   &                            & 3 & 88.57 & 81.34 \\
                   &                            & 7 & 89.62 & 85.94 \\ \bottomrule
\end{tabular}%
}
\end{table}
\section{Complexity of equilibria in symmetric min-max optimization}


This section characterizes the complexity of computing symmetric first-order Nash equilibria\\ (\Cref{def:FONE}) in symmetric min-max optimization problems in the sense of~\Cref{def:symmetric}; namely, when $f(\vx, \vy) = - f(\vy, \vx)$ for all $(\vx, \vy) \in \calX \times \calY$ and $\calX = \calY$. 


\subsection{Problem definitions and hardness results for symmetric equilibria}
\label{sec:symmetric}

Given a continuously differentiable function $f : \mathcal{D} \to \R$, we set $F_{\textrm{GDA}}:\mathcal{D} \to \mathcal{D}$ to be
$$F_{\textrm{GDA}}(\vx,\vy) \defeq \prod_{\mathcal{D}} \left[\vx - \nabla_{\vx}f(\vx,\vy),\vy + \nabla_{\vy}f(\vx,\vy)\right] \textrm{ for }(\vx,\vy)\in \mathcal{D},$$
the norm of which measures the fixed-point gap and corresponds to the update rule of GDA with stepsize equal to one; we recall that Player $\vx$ is the minimizer, while Player $\vy$ is the maximizer. The domain $\mathcal{D}$ is a compact subset of $\R^d$ for some $d \in \mathbb{N}$. Moreover, the projection operator $\prod$ is applied jointly on $\mathcal{D}$.\footnote{This is referred to as the ``safe'' version of GDA because it ensures that the mapping always lies in $\mathcal{D}$. One could also project independently on $\mathcal{D}(\vy)=\{\vx': (\vx',\vy)\in\mathcal{D}\}$ and $\mathcal{D}(\vx)=\{\vy': (\vx,\vy')\in\mathcal{D}\}$; see \citet{DSZ21} for further details and the polynomial equivalence for finding fixed points for both versions.} When $\mathcal{D}$ can be expressed as a \emph{Cartesian} product $\calX\times\calY$, the domain set is called \emph{uncoupled} (and the projection can happen independently), otherwise it is called \emph{coupled/joint}.

We begin by introducing the problem of computing fixed points of gradient descent/ascent (GDA) for domains expressed as the Cartesian product of polytopes, modifying the computational problem $\gdaFixed$ introduced by~\citet{DSZ21}.

\begin{nproblem}[\gdaFixed]
  \textsc{Input:} 
  \begin{itemize}
  \item Precision parameter $\epsilon > 0$ and smoothness parameter $L$,
  \item Polynomial-time Turing
  machine $\calC_f$ evaluating a $L$-smooth function $f : \mathcal{X} \times \mathcal{Y} \to \R$ and its gradient
  $\nabla f: \calX\times \calY \to \R^{d}$, where
   $\mathcal{X} = \{\vx:\matA_x \vx \leq \vecb_x\}$ and $\mathcal{Y} = \{\vy:\matA_y \vy \leq \vecb_y\}$ are nonempty, bounded polytopes described by input matrices
  $\matA_x \in \R^{m_x \times d_x}$, $\matA_y \in \R^{m_y \times d_y}$ and vectors $\vecb_x \in \R^{m_x}, \vecb_y \in \R^{m_y}$, with $d \defeq d_x + d_y$.
\end{itemize}
  \noindent \textsc{Output:} A point
  $(\vxstar,\vystar)\in \calX\times\calY$ such that
  $\norm{(\vx^*, \vy^*) - F_{GDA}(\vx^*,\vy^*)}_2 \leq \epsilon$.
\end{nproblem}

Based on $\gdaFixed$, we introduce the problem $\symgdaFixed$, which captures the problem of computing \emph{symmetric} (approximate) fixed points of GDA for symmetric min-max optimization problems. We would like to note that we define our computational problems as promise problems.
  
\begin{nproblem}[\symgdaFixed]
 \textsc{Input:}    \begin{itemize}
  \item Precision parameter $\epsilon > 0$ and smoothness parameter $L$, 
  \item Polynomial-time Turing
  machine $\calC_f$ evaluating a $L$-smooth, antisymmetric function $f : \mathcal{X} \times \mathcal{X} \to \R$ and its gradient
  $\nabla f: \calX\times \calX \to \R^{2d}$, where
  $\mathcal{X} = \{\vx:\matA \vx \leq \vecb\}$ is a nonempty, bounded polytope described by an input matrix
  $\matA \in \R^{m \times d}$ and vector $\vecb \in \R^{m}$.
\end{itemize}

  \noindent \textsc{Output:} A point
  $(\vxstar,\vxstar)\in \calX\times\calX$ such that
  $\norm{(\vx^*, \vx^*) - F_{GDA}(\vx^*,\vx^*)}_2 \leq \epsilon$.
\end{nproblem}

We start by showing that the problem $\symgdaFixed$ also lies in $\PPAD$; the fact that $\gdaFixed$ is in $\PPAD$---even under coupled domains---was shown to be the case by~\citet{DSZ21}.

\begin{lemma}\label{lem:membership}
  $\symgdaFixed$ is a total
search problem and lies in \PPAD.
  \end{lemma}
\begin{proof}  
We first define the function (as in~\Cref{lem:exists}) $M: \calX \to \calX$ as
\begin{equation*}
M(\vx') \defeq \prod_{\calX} \left[\vx' - \nabla _{\vx}  f(\vx,\vy)\Big|_{(\vx,\vy)=(\vx',\vx')} \right],
\end{equation*}
where we recall that $\Pi$ is the projection operator on $\calX.$ Assuming that the input function $f$ is $L$-smooth, it follows that $M(\vx')$ is $(L+1)$-Lipschitz. Furthermore, projecting on the polytope $\calX$ takes polynomial time, and so $M$ is polynomial-time computable. As a result, we can use~\citet[Proposition 2, part 2]{Etessami10:On} (see also~\citet[Proposition D.1]{Fearnley23:Complexity}), where it was shown that finding an $\epsilon$-approximate fixed point of a Brouwer function that is efficiently computable and continuous, when the domain is a bounded polytope, lies in \PPAD.
\end{proof}

Having established that $\symgdaFixed$ belongs in \PPAD, we now prove the first main hardness result of this section.

\begin{theorem}[Complexity for symmetric equilibrium]\label{thm:symmetricminmax} 
$\symgdaFixed$ is \PPAD-complete, even for quadratic functions.
\end{theorem}
\begin{proof}
We $P$-time reduce the problem of finding approximate symmetric NE in two-player symmetric games to $\symgdaFixed$.  
Given any two-player symmetric game with payoff matrices $(\mat{R},\mat{R}^{\top})$ of size $n\times n$, we set 
\begin{equation}\label{eq:matrices}
\mat{A} \defeq \frac{1}{2} \left(\mat{R}+\mat{R}^{\top}\right) \textrm{ (symmetric matrix) and } 
\mat{C} \defeq \frac{1}{2}\left(\mat{R}-\mat{R}^{\top}\right) \textrm{(skew-symmetric matrix)}.
\end{equation}

\noindent We define the \emph{quadratic}, antisymmetric function
\begin{equation}
    \label{eq:hard-quad}
 f(\vx,\vy) \defeq \frac{1}{2} \langle \vy, \mat{A}\vy \rangle - \frac{1}{2} \langle \vx, \mat{A}\vx \rangle + \langle \vy, \mat{C}\vx \rangle   
\end{equation}
with domain $\Delta^n \times \Delta^n$. Indeed, to see that $f$ is antisymmetric, one can observe that $$f(\vy,\vx) = \frac{1}{2} \langle \vx, \mat{A} \vx \rangle - \frac{1}{2} \langle \vy, \mat{A} \vy \rangle + \langle \vx, \mat{C} \vy \rangle = \frac{1}{2} \langle \vx, \mat{A}\vx \rangle -\frac{1}{2} \langle \vy, \mat{A} \vy \rangle - \langle \vy, \mat{C}^{\top} \vx \rangle = -f(\vx,\vy).$$
Assuming that all entries of $\mat{R}$ lie in $[-1,1],$ it follows that the singular values of $\mat{A}$ and $\mat{C}$ are bounded by $n.$ As a result $f$ and $\nabla_{\vx} f = -\mat{A}\vx-\mat{C}\vy, \nabla_{\vy} f = \mat{A}\vy +\mat{C}\vx$ are polynomial time computable and continuous, and $\nabla_{\vx} f, \nabla_{\vy}f$ are $L$-Lipschitz for $L \leq 2n,$ thus $f$ is $4n$-smooth.

We assume $\vx$ is the minimizer and $\vy$ is the maximizer, and let $(\vx^*,\vx^*)$ be an $\epsilon$-approximate fixed point of GDA. We shall show that $(\vx^*,\vx^*)$ is an $4n\epsilon$-approximate NE of the symmetric two-player game $(\mat{R},\mat{R}^{\top})$. Since $(\vx^*,\vx^*)$ is an $\epsilon$-approximate fixed point of GDA, we can use~\Cref{lem:approxsmooth} (\Cref{sec:proofs2}) and obtain the following variational inequalities:
\begin{equation*}
\max_{\vx^*+\bm{\delta} \in \Delta^n, \norm{\bm{\delta}}_2\leq 1} \bm{\delta}^{\top} (\mat{A}\vx^*+\mat{C}\vx^*)\leq \epsilon\left(2n+1\right),
\end{equation*}
implying that (since the diameter of $\Delta^n$ is $\sqrt{2}$ in $\ell_2$)
\begin{equation}
\tag{VI for NE}\label{eq:VIforNE}
\langle \vx - \vx^*,(\mat{A}+\mat{C})\vx^*  \rangle \leq \sqrt{2}\epsilon\left(2n+1\right) \textrm{ for any }\vx\in\Delta^n. 
\end{equation}

Now, we observe that \eqref{eq:VIforNE} implies that $(\vx^*,\vx^*)$ is a $\sqrt{2}\epsilon\left(2n+1\right)$-approximate symmetric NE in the two-player symmetric game with payoff matrices $(\mat{A}+\mat{C},\mat{A}-\mat{C})$ (recall Definition \eqref{def:NE}). Since $\sqrt{2}\epsilon\left(2n+1\right) \leq 4n\epsilon$ for $n\geq 2$, our claim follows.
%\textrm{ or equivalently } \langle \vx - \vx^*,  \rangle \leq \]

By~\Cref{theorem:PPAD_for _symmetric} and~\Cref{lem:membership}, we conclude that $\symgdaFixed$ is \PPAD-complete, even for quadratic functions that are $O(n)$-smooth, $O(n)$-Lipschitz and $\epsilon \leq \nicefrac{1}{n^{1+c}}$, for any $c>0$.
\end{proof}

For symmetric first-order Nash equilibria, the same argument establishes \PPAD-hardness for any $\epsilon \leq \nicefrac{1}{n^c}$, where $c > 0$ (as claimed in~\Cref{theorem:main}). Moreover, leveraging the hardness result of~\citet{Rubinstein16:Settling}, we can also immediately obtain constant inapproximability under the so-called \emph{exponential-time hypothesis (ETH)} for \PPAD---which postulates than any algorithm for solving \textsc{EndOfALine}, the prototypical \PPAD-complete problem, requires $2^{\tilde{\Omega}(n)}$ time.

\begin{corollary}
    \label{cor:constant}
    Computing an $\Theta(1)$-approximate first-order Nash equilibrium in symmetric $n$-dimensional min-max optimization requires $n^{\tilde{\Omega}(\log n)}$ time, assuming ETH for \PPAD.
\end{corollary}

\begin{remark}[Comparison with \citet{DSZ21}] The argument of~\Cref{thm:symmetricminmax} can be slightly modified to imply one of the main results of~\citet{DSZ21}---with simplex constraints instead of box constraints. We provide a simple proof of this fact below (\Cref{thm:simple}). The main idea is to introduce coupled constraints in order to \emph{force symmetry}, that is, constraints of the form $-\delta \leq x_i - y_i \leq \delta$ for all $i \in [n]$, where, if $\epsilon$ is the approximation accuracy, $\delta$ is of order $\Theta\left(\epsilon^{1/4}\right)$. Our result pertaining to symmetric equilibria is stronger in that it accounts for deviations in the whole domain, not merely on the coupled feasibility set. 
\end{remark}

\begin{theorem}[\PPAD-hardness for coupled domains]\label{thm:simple} The problem $\gdaFixed$ is \PPAD-hard when the domain is a joint polytope, even for quadratic functions.
\end{theorem}
\begin{proof}
The proof follows similar steps with \Cref{thm:symmetricminmax}, namely, we $P$-time reduce the problem of finding approximate symmetric NE in two-player symmetric games to $\gdaFixed$ with coupled domains.  
Given a two-player symmetric game with payoff matrices $(\mat{R},\mat{R}^{\top})$ of size $n\times n$, we set 
$\mat{A} \defeq \frac{1}{2} \left(\mat{R}+\mat{R}^{\top}\right)$, 
$\mat{C} \defeq \frac{1}{2}\left(\mat{R}-\mat{R}^{\top}\right)$ and define again the quadratic, antisymmetric function
\begin{equation*}
    %\label{eq:hard-quad}
 f(\vx,\vy) \defeq \frac{1}{2} \langle \vy, \mat{A}\vy \rangle - \frac{1}{2} \langle \vx, \mat{A}\vx \rangle + \langle \vy, \mat{C}\vx \rangle.   
\end{equation*}
Moreover, given a parameter $\delta > 0$ (to be specified shortly), we define the joint  domain of $f$ to be
\begin{equation}\label{eq:polytope}\tag{joint Domain}
\mathcal{D} := \left\{(\vx,\vy) \in \Delta^n \times \Delta^n: -\delta\leq x_i - y_i \leq \delta \textrm{ for all }i\in[n]\right\}.
\end{equation}
Let $(\vx^*,\vy^*)$ be an $\epsilon$-approximate fixed point of GDA. We will show that $\left(\frac{\vx^*+\vy^*}{2},\frac{\vx^*+\vy^*}{2}\right)$ is an $O(\epsilon^{1/4})$-approximate (symmetric) NE of the game $(\mat{R},\mat{R}^{\top})$ for an appropriate choice of $\delta.$ 

\smallskip

We set $\mathcal{D}(\vx^*) = \{\vy:(\vx^*,\vy)\in\mathcal{D}\}$ and 
$\mathcal{D}(\vy^*) = \{\vx:(\vx,\vy^*)\in\mathcal{D}\}.$
In words, $\mathcal{D}(\vx^*)$ and $\mathcal{D}(\vy^*)$ capture the allowed deviations for $\vy$ and $\vx$ respectively. It also holds that $f$ is $G$-Lipschitz continuous with $G=4n$ and also $4n$-smooth (using the same reasoning as in Theorem~\ref{thm:symmetricminmax}). 

\smallskip
\noindent Since $(\vx^*,\vy^*)$ is an $\epsilon$-approximate fixed point of GDA, using~\Cref{lem:safe} (\Cref{sec:proofs2}), the following VIs must hold for some positive constant $K<10$ and $n$ sufficiently large:

\begin{equation}
\label{eq:VIforcoupled}
\begin{array}{cc}
\langle \vx - \vx^*,-\mat{A}\vx^*+\mat{C}^{\top}\vy^*  \rangle \geq -Kn^{3/2}\sqrt{\epsilon} \textrm{ for any }\vx \in \mathcal{D}(\vy^*) \textrm{ and }\\ 
\langle \vy - \vy^*,\mat{A}\vy^*+\mat{C}\vx^*  \rangle \leq Kn^{3/2}\sqrt{\epsilon} \textrm{ for any }\vy \in \mathcal{D}(\vx^*).
\end{array}
\end{equation}

\noindent Let $\overline{\mathcal{D}} = \left\{\vz \in \Delta^n: \left\|\vz-\frac{\vx^*+\vy^*}{2}\right\| _{\infty}\leq \frac{\delta}{2}\right\}.$ By triangle inequality, it follows that 
$\overline{\mathcal{D}} \subseteq \mathcal{D}(\vy^*)\cap\mathcal{D}(\vx^*).$
We express the VIs of \eqref{eq:VIforcoupled} using a single variable $\vz$ and common deviation domain:
\begin{equation*}
\label{eq:VIforcoupledz}
\begin{array}{cc}
\langle \vz - \vx^*,-\mat{A}\vx^*+\mat{C}^{\top}\vy^*  \rangle \geq -Kn^{3/2}\sqrt{\epsilon} \textrm{ and } 
\langle \vz - \vy^*,\mat{A}\vy^*+\mat{C}\vx^*  \rangle \leq Kn^{3/2}\sqrt{\epsilon} \textrm{ for any }\vz \in \overline{\mathcal{D}}.
\end{array}
\end{equation*}
Multiplying the first inequality by $-1/2$ and the second with $1/2$ and adding them up gives
\begin{equation}
\label{eq:touse}
\left\langle \vz - \frac{\vx^*+\vy^*}{2}, (\mat{A}+\mat{C})\frac{\vx^*+\vy^*}{2} \right\rangle \leq \frac{1}{4}\left\langle \vx^*-\vy^*,\mat{A}(\vx^*-\vy^*)\right\rangle + Kn^{3/2}\sqrt{\epsilon}. 
\end{equation}
Since $\vx^*,\vy^* \in \mathcal{D}$, it follows that $\left\langle \vx^*-\vy^*,\mat{A}(\vx^*-\vy^*)\right\rangle \leq n \|\vx^*-\vy^*\|^2_2 \leq n^2 \delta^2$. 
Combining this fact with \eqref{eq:touse}, we conclude that
\begin{equation}
\tag{VImedian}
\label{eq:lastVINE}
\left\langle \vz - \frac{\vx^*+\vy^*}{2}, (\mat{A}+\mat{C})\frac{\vx^*+\vy^*}{2} \right\rangle \leq n^2 \delta^2 + Kn^{3/2}\sqrt{\epsilon} \textrm{ for any }\vz \in \overline{\mathcal{D}}. 
\end{equation}

\eqref{eq:lastVINE} shows that by deviating from $\left(\frac{\vx^*+\vy^*}{2},\frac{\vx^*+\vy^*}{2}\right)$ to some $\vz$ in $\overline{\mathcal{D}}$, the payoff cannot increase by more than $\left(n^2\delta^2 + Kn^{3/2}\sqrt{\epsilon}\right)$ in the two-player symmetric game with matrices $(\mat{R},\mat{R}^{\top})$.

\noindent We consider any pure strategy $\bm{e}_j$ for $j\in[n]$. If $\left\|\bm{e}_j - \frac{\vx^*+\vy^*}{2}\right\|_{\infty} \leq \frac{\delta}{2}$ then $\bm{e}_j \in \overline{\mathcal{D}}$ 
and it is captured by \eqref{eq:lastVINE}. Suppose that 
$\left\|\bm{e}_j - \frac{\vx^*+\vy^*}{2}\right\|_{\infty} > \frac{\delta}{2}$
and consider the point $\vz' \in \overline{\mathcal{D}}$ on the line segment between $\bm{e}_j$ and $\frac{\vx^*+\vy^*}{2}$ that intersects the boundary of $\overline{\mathcal{D}}.$ It holds that $\bm{e}_j - \frac{\vx^*+\vy^*}{2} = c\left(\vz' - \frac{\vx^*+\vy^*}{2} \right)$ for some positive $c \leq \frac{2}{\delta}$ (it cannot be larger because otherwise the infinity norm of the difference between $\bm{e}_j$ and $\frac{\vx^*+\vy^*}{2}$ would exceed one, which is impossible as they both belong to $\Delta^n$). Therefore, 
\begin{equation}
\label{eq:lastlastVINE}
\left\langle \bm{e}_j - \frac{\vx^*+\vy^*}{2}, (\mat{A}+\mat{C})\frac{\vx^*+\vy^*}{2} \right\rangle \leq 2n^2 \delta + \frac{2Kn^{3/2}\sqrt{\epsilon}}{\delta} \textrm{ for any pure strategy }j. 
\end{equation}

From \eqref{eq:lastlastVINE}, we conclude that $\frac{\vx^*+\vy^*}{2}$ is $\left(2n^{2}\delta + \frac{2Kn^{3/2}\sqrt{\epsilon}}{\delta}\right)$-approximate NE of the symmetric two-player game $(\mat{R},\mat{R}^{\top}).$ We choose $\delta = \epsilon^{1/4} n^{-1/4}$ and we get that $\frac{\vx^*+\vy^*}{2}$ is an $O(n^{7/4}\epsilon^{1/4})$-approximate NE for $(\mat{R},\mat{R}^{\top})$, and thus the hardness result holds for $\epsilon$ of order  $O\left(\frac{1}{n^{7+c}}\right)$ for any constant $c>0.$ We note that if instead of an $\epsilon$-approximate fixed point of GDA, we were given an $\epsilon$-approximate First-order NE, then the hardness result would hold for any $\epsilon$ of order $\frac{1}{n^{c}}$ with $c>0.$
\end{proof}

\paragraph{Hardness results for symmetric dynamics}

Another interesting consequence of~\Cref{thm:symmetricminmax} is that it immediately precludes convergence under a broad class of iterative algorithms in general min-max optimization problems. 

\begin{definition}[Symmetric learning algorithms for min-max]
    \label{def:sym-dynamics}
    Let $T \in \mathbb{N}$. A deterministic, polynomial-time learning algorithm $\calA$ proceeds as follows for any time $t \in [T]$. It outputs a strategy as a function of the history $\mathcal{H}^{(t)}$ it has observed so far (where $\mathcal{H}^{(1)} \defeq \emptyset$ ), and then receives as feedback $\vec{g}^{(t)}$. It then updates $\mathcal{H}^{(t+1)} \defeq ( \mathcal{H}^{(t)}, \vec{g}^{(t)})$. 
    
    A \emph{symmetric} learning algorithm in min-max optimization consists of Player $\vx$ employing algorithm $\calA$ with history $\mathcal{H}_x^{(t)} \defeq (\nabla_{\vx} f(\vx^{(t)}, \vy^{(t)}) )_{t=1}^T$, and Player $\vy$ employing the \emph{same} algorithm with history $\mathcal{H}_y^{(t)} \defeq (- \nabla_{\vy} f(\vx^{(t)}, \vy^{(t)}) )_{t=1}^T$.
\end{definition}

(A consequence of the above definition is that both players initialize from the same strategy.) Many natural and well-studied algorithms in min-max optimization adhere to~\Cref{def:sym-dynamics}. Besides the obvious example of gradient descent/ascent, we mention extragradient descent(/ascent), optimistic gradient descent(/ascent), and optimistic multiplicative weights---all assumed to be executed simultaneously. A simple non-example is \emph{alternating} gradient descent(/ascent)~\citep{Wibisono22:Alternating,Bailey20:Finite}, wherein players do not update their strategies simultaneously.

\begin{theorem}
    \label{theorem:sym-dyn}
    No symmetric learning algorithm (per~\Cref{def:sym-dynamics}) can converge to $\epsilon$-first-order Nash equilibria in min-max optimization in polynomial time when $\epsilon = \nicefrac{1}{n^c}$, unless $\PPAD = \P$.
\end{theorem}

Indeed, this is a direct consequence of our argument in~\Cref{thm:symmetricminmax}: under~\Cref{def:sym-dynamics} and the min-max optimization problem~\eqref{eq:hard-quad}, it follows inductively that $\vx^{(t)} = \vy^{(t)}$ and $\mathcal{H}_x^{(t)} = \mathcal{H}_y^{(t)}$ for all $t \in [T]$. But~\Cref{thm:symmetricminmax} implies that computing a symmetric first-order Nash equilibrium is $\PPAD$-hard when $\epsilon = \nicefrac{1}{n^c}$.

Assuming that $\P \neq \PPAD$, \Cref{theorem:sym-dyn}, and in particular its instantiation in team zero-sum games (\Cref{theorem:team-hard}), recovers and significantly generalizes some impossibility results shown by~\citet{kalogiannis2021teamwork} concerning lack of convergence for certain algorithms, such as optimistic gradient descent(/ascent)---our hardness result goes much further, precluding any algorithm subject to~\Cref{def:sym-dynamics}, albeit being conditional.

\subsection{The complexity of non-symmetric fixed points}
\label{sec:nonsymmetric}

An immediate question raised by~\Cref{thm:symmetricminmax} concerns the computational complexity of finding \emph{non-symmetric} fixed points of GDA for symmetric min-max optimization problems. Since totality is not guaranteed, unlike $\symgdaFixed$, we cannot hope to prove membership in \PPAD. In fact, we show that finding a non-symmetric fixed point of GDA is \FNP-hard (\Cref{theorem:non-symmetric}). To do so, we first define formally the computational problem of interest.

\begin{nproblem}[\nsymgdaFixed]
 \textsc{Input:}  \begin{itemize}
   \item Parameters $\epsilon,\delta>0$ and Lipschitz constant $L$ and
  \item Polynomial-time Turing
  machine $\calC_f$ evaluating a $L$-smooth antisymmetric function $f : \mathcal{X} \times \mathcal{X} \to \R$ and its gradient
  $\nabla f: \calX\times \calX \to \R^{2d}$, where
  $\mathcal{X} = \{\vx:\matA \vx \leq \vecb\}$ is a nonempty, bounded polytope described by a matrix
  $\matA \in \R^{m \times d}$ and vector $\vecb \in \R^{m}$.
\end{itemize}

  \noindent \textsc{Output:} A point
  $(\vxstar,\vystar)\in \calX\times\calX$ such that $\norm{\vxstar-\vystar}_2 \geq \delta$ and
  $\norm{(\vx^*, \vy^*) - F_{GDA}(\vx^*,\vy^*)}_2 \leq \epsilon$ if it exists, otherwise return \textsf{NO}.
\end{nproblem}

\noindent We establish that $\nsymgdaFixed$ is \FNP-hard. Our main result is restated below.

\nonsymmetric*

Our reduction builds on the hardness result of~\citet{MCLENNAN2010683}---in turn based on earlier work by~\citet{gilboa1989nash,Conitzer08:New}---which we significantly refine in order to account for $\poly(1/n)$-Nash equilibria. We begin by describing their basic approach. Let $G = ([n], E)$ be an $n$-node, undirected, unweighted graph, and construct
\begin{equation}
    \label{eq:matA}
    \mat{A}_{i, j} = 
    \begin{cases}
        -1 & \text{if $i = j$}, \\
        0 & \text{if $\{i, j\} \in E$}, \\
        -2 & \text{otherwise.}
    \end{cases}
\end{equation}
(\Cref{fig:graph} depicts an illustrative example.) Based on this matrix, \citet{MCLENNAN2010683} consider the symmetric, identical-payoff, two-player game $(\mat{A}, \mat{A})$---by construction, $\mat{A} = \mat{A}^\top$, and so this game is indeed symmetric. They were able to show the following key property.
\begin{tikzpicture}[scale=0.6]
\pgfdeclarelayer{bg}
\pgfsetlayers{bg,main}
\node[vertex, fill=white, fill opacity=0.8, text opacity=1] (0) at (3.56, 4.1) {\scriptsize 0};
\node[vertex, fill=white, fill opacity=0.8, text opacity=1] (1) at (1.5, 3.91) {\scriptsize 1};
\node[vertex, fill=white, fill opacity=0.8, text opacity=1] (2) at (0.5, 3.37) {\scriptsize 2};
\node[vertex, fill=white, fill opacity=0.8, text opacity=1] (4) at (2.8, 1.21) {\scriptsize 4};
\node[vertex, fill=white, fill opacity=0.8, text opacity=1] (6) at (3.56, 2.29) {\scriptsize 6};
\node[vertex, fill=white, fill opacity=0.8, text opacity=1] (7) at (4.5, 0.2) {\scriptsize 7};
\node[vertex, fill=white, fill opacity=0.8, text opacity=1] (9) at (0.67, 0.2) {\scriptsize 9};
\node[vertex, fill=white, fill opacity=0.8, text opacity=1] (10) at (2.6, 3.5) {\scriptsize 10};
\node[vertex, fill=white, fill opacity=0.8, text opacity=1] (11) at (2.2, 2.6) {\scriptsize 11};
\node[vertex, fill=white, fill opacity=0.8, text opacity=1] (12) at (1.7, 1.3) {\scriptsize 12};
\node[vertex, fill=white, fill opacity=0.8, text opacity=1] (13) at (4.04, 2.83) {\scriptsize 13};
\node[vertex, fill=white, fill opacity=0.8, text opacity=1] (3) at (4.04, 1.75) {\scriptsize 3};
\node[vertex, fill=white, fill opacity=0.8, text opacity=1] (5) at (1.34, 0.67) {\scriptsize 5};
\node[vertex, fill=white, fill opacity=0.8, text opacity=1] (8) at (5.00, 1.21) {\scriptsize 8};
\begin{pgfonlayer}{bg}
\draw[-latex] (0) edge[bend left=10] (1);
\draw[-latex] (0) edge[bend left=10] (2);
\draw[-latex] (0) edge[bend left=10] (4);
\draw[-latex] (0) edge[bend left=10] (6);
\draw[-latex] (0) edge[bend left=10] (7);
\draw[-latex] (0) edge[bend left=10] (9);
\draw[-latex] (0) edge[bend left=10] (10);
\draw[-latex] (0) edge[bend left=10] (11);
\draw[-latex] (0) edge[bend left=10] (12);
\draw[-latex] (1) edge[bend left=10] (2);
\draw[-latex] (1) edge[bend left=10] (4);
\draw[-latex] (1) edge[bend left=10] (13);
\draw[-latex] (2) edge[bend left=10] (0);
\draw[-latex] (2) edge[bend left=10] (1);
\draw[-latex] (2) edge[bend left=10] (3);
\draw[-latex] (2) edge[bend left=10] (4);
\draw[-latex] (2) edge[bend left=10] (5);
\draw[-latex] (2) edge[bend left=10] (6);
\draw[-latex] (2) edge[bend left=10] (7);
\draw[-latex] (2) edge[bend left=10] (8);
\draw[-latex] (2) edge[bend left=10] (9);
\draw[-latex] (4) edge[bend left=10] (0);
\draw[-latex] (4) edge[bend left=10] (1);
\draw[-latex] (4) edge[bend left=10] (2);
\draw[-latex] (4) edge[bend left=10] (5);
\draw[-latex] (4) edge[bend left=10] (6);
\draw[-latex] (4) edge[bend left=10] (8);
\draw[-latex] (4) edge[bend left=10] (9);
\draw[-latex] (4) edge[bend left=10] (10);
\draw[-latex] (4) edge[bend left=10] (11);
\draw[-latex] (6) edge[bend left=10] (2);
\draw[-latex] (6) edge[bend left=10] (4);
\draw[-latex] (6) edge[bend left=10] (7);
\draw[-latex] (7) edge[bend left=10] (2);
\draw[-latex] (7) edge[bend left=10] (12);
\draw[-latex] (9) edge[bend left=10] (0);
\draw[-latex] (9) edge[bend left=10] (2);
\draw[-latex] (9) edge[bend left=10] (4);
\draw[-latex] (9) edge[bend left=10] (8);
\draw[-latex] (10) edge[bend left=10] (0);
\draw[-latex] (10) edge[bend left=10] (4);
\draw[-latex] (10) edge[bend left=10] (11);
\draw[-latex] (11) edge[bend left=10] (10);
\draw[-latex] (3) edge[bend left=10] (2);
\draw[-latex] (5) edge[bend left=10] (2);
\draw[-latex] (5) edge[bend left=10] (4);
\draw[-latex] (5) edge[bend left=10] (7);
\draw[-latex] (8) edge[bend right=10] (0);
\draw[-latex] (8) edge[bend left=10] (2);
\draw[-latex] (8) edge[bend left=10] (4);
\draw[-latex] (8) edge[bend left=10] (9);
\end{pgfonlayer}
\end{tikzpicture}


\begin{lemma}[\citep{MCLENNAN2010683}]
    \label{lemma:Nashgap}
    Let $C_k \subseteq [n]$ be a maximum clique of $G$ with size $k$ and $\vx^* = \frac{1}{k} \sum_{i \in C_k} \ve_{i}$. Then, $(\vx^*, \vx^*)$ is a Nash equilibrium of $(\mat{A}, \mat{A})$ that attains value $-\nicefrac{1}{k}$. Furthermore, any symmetric Nash equilibrium not in the form described above has value at most $- \nicefrac{1}{k - 1}$.
\end{lemma}
The idea now is to construct a new symmetric, identical-payoff game $(\mat{B}, \mat{B})$, for
    \begin{align}
        \mat{B} \defeq \begin{bmatrix}
         \mat{A}_{1, 1} & \cdots & \mat{A}_{1, n}  & r\\
         \vdots & \ddots & \vdots & \vdots \\
         \mat{A}_{n, 1} & \cdots & \mat{A}_{n,n} & r\\
         r & \cdots & r  & V\\
    \end{bmatrix}, \label{eq:unique_NP}
    \end{align}
    where $V \defeq - \frac{1}{k}$ and $r = \frac{1}{2} (- \frac{1}{k} - \frac{1}{k - 1}) = -\frac{2k-1}{2(k - 1)k}$. Coupled with~\Cref{lemma:Nashgap}, this new game yields the following \NP-hardness result.

\begin{theorem}[\citep{MCLENNAN2010683}]
    \label{theorem:knownNP}
    It is \NP-hard to determine whether a symmetric, identical-payoff, two-player game has a unique symmetric Nash equilibrium.
\end{theorem}

Our goal here is to prove a stronger result, \Cref{theorem:symmetric-new}, that characterizes the set of $\epsilon$-Nash equilibria even for $\epsilon = \nicefrac{1}{n^c}$; this will form the basis for our hardness result in min-max optimization and adversarial team games. To do so, we first derive some basic properties of game~\eqref{eq:unique_NP}.%, which will be useful for us as well.% (even though we will have to rely on a slightly different matrix $\mat{A}$).

Game~\eqref{eq:unique_NP} always admits the trivial (symmetric) Nash equilibrium $(\vec{e}_{n+1}, \vec{e}_{n+1})$. Now, consider any symmetric Nash equilibrium $(\vx^*, \vx^*)$ with $x^*_{n+1} \neq 1$. If $x^*_{n+1} = 0$, it follows that $(\vx^*_{[i \cdots n] }, \vx^*_{[i \cdots n]})$ is a Nash equilibrium of $(\mat{A}, \mat{A})$, which in turn implies that $G$ admits a clique of size $k$; this follows from~\Cref{lemma:Nashgap}, together with the fact that $- \nicefrac{1}{k - 1} < r < -\nicefrac{1}{k}$.

We now analyze the case where $x^*_{n+1} \in (0, 1)$. It then follows that $(\nicefrac{\vx^*_{[1 \cdots n]}}{1 - x^*_{n+1}}, \nicefrac{\vx^*_{[1 \cdots n]}}{1 - x^*_{n+1}})$ is a (symmetric) Nash equilibrium of $(\mat{A}, \mat{A})$. Furthermore, the utility of playing action $a_{n+1}$ is $(1 - x^*_{n+1}) r + x^*_{n+1} V > r $. By~\Cref{lemma:Nashgap}, it follows that $(\nicefrac{\vx^*_{[1 \cdots n] }}{1 - x^*_{n+1}}, \nicefrac{\vx^*_{[1 \cdots n]}}{1 - x^*_{n+1}})$ has a value of $V$ and $G$ admits a clique of size $k$. As a result, the utility of playing any action $a_i$, with $i \in \text{supp}(\vx^*)$ and $i \neq n+1$, is $( 1 - x^{*}_{n+1}) V + x^*_{n+1} r$. At the same time, the utility of playing action $a_{2n+1}$ reads $(1 - x_{n+1}^*) r + x_{n+1}^* V$. Equating those two quantities, it follows that $x^*_{n+1} = \nicefrac{1}{2}$.

In summary, $G$ contains a clique of size $k$ if and only if game~\eqref{eq:unique_NP} admits a unique symmetric Nash equilibrium, which implies~\Cref{theorem:knownNP}. What is more, we have shown a stronger property. Namely, any symmetric Nash equilibrium of $(\mat{B}, \mat{B})$ has to be in one of the following forms:
\begin{enumerate} \label{eq:three_exact_NE}
        \item $(\vx^*, \vx^*)$ with $\vx^* \defeq \ve_{n+ 1}$;\label{item:case1}
        \item $(\vx^*, \vx^*)$ with $\vx^* \defeq \frac{1}{k} \sum_{i \in C_k} \ve_i$, where $C_k \subseteq [n]$ is a clique in $G$ of size $k$;\label{item:case2}
        \item $(\vx^*, \vx^*)$ with $\vx^* \defeq \frac{1}{2} \ve_{n + 1} + \frac{1}{2k} \sum_{i \in C_k} \ve_i$, where $C_k \subseteq [n]$ is a clique in $G$ of size $k$.\label{item:case3}
\end{enumerate}
In particular, the equilibria in Items~\ref{item:case2} or~\ref{item:case3}---which exist iff $G$ contains a clique of size $k$---are always far from the one in Item~\ref{item:case1}. However, this characterization only applies to exact Nash equilibria. In two-player games, when $\epsilon$ is sufficiently small with $\log(1/\epsilon) \leq \poly(|\Gamma|)$, \citet{Etessami10:On} have shown that any $\epsilon$-Nash equilibrium is within $\ell_1$-distance $\delta$ from an exact one, and so the above characterization can be applied; unfortunately, this does not apply (for general games) in the regime we are interested, namely $\epsilon = \poly(1/n)$.

We address this challenge by refining the result of~\citet{MCLENNAN2010683}. Our main result, which forms the basis for~\Cref{theorem:non-symmetric} and~\Cref{theorem:uniqueATG}, is summarized below.

\begin{theorem}
    \label{theorem:symmetric-new}
    For symmetric, identical-interest, two-player games, constants $c_1, c_2 > 0$, and $\epsilon = n^{-c_1}$, it is \NP-hard to distinguish between the following two cases:
    \begin{itemize}[noitemsep,topsep=0pt]
        \item any two symmetric $\epsilon$-Nash equilibria have $\ell_1$-distance at most $n^{-c_2}$, and
        \item there are two symmetric $\epsilon$-Nash equilibria that have $\ell_1$-distance $\Omega(1)$.
    \end{itemize}
\end{theorem}

Our reduction proceeds similarly, but defines $\lineA$ to be the adjacency matrix of $G$ with $\delta \in (0, 1)$ in each diagonal entry. Using $\lineA$, we show that we can refine~\Cref{lemma:Nashgap} of~\citet{MCLENNAN2010683}. Before we state the key property we prove in \Cref{lemma:well_supported_nash_value}, we recall the following definition.

\begin{definition}[Well-supported NE]
    A symmetric strategy profile $(\vx, \vx)$ is an \emph{$\epsilon$-well-supported} Nash equilibrium of the symmetric, identical-payoff game $(\lineA, \lineA)$ if for all $i \in [n]$,
    \begin{align*}
        x_i > 0 \implies (\lineA \vx)_i \geq \max_{j \in [n]} (\lineA \vx)_j - \epsilon.
    \end{align*}
\end{definition}

\begin{restatable}{lemma}{wellsupclique}
\label{lemma:well_supported_nash_value}
    Suppose that the maximum clique in $G$ is of size $k$. For any symmetric $\epsilon$-well-supported NE $(\hat{\vx}, \hat{\vx})$ of $(\lineA, \lineA)$ not supported on a clique of size $k$, we have $u(\hat{\vx}, \hat{\vx}) \leq 1 - \frac{1}{k} + \frac{\delta}{k} - \frac{2\delta}{ n^2k^4} + 2 \epsilon$.
\end{restatable}

Equipped with this property, we show (in~\Cref{sec:proofs3}) that a similar argument to the one described earlier concerning game~\eqref{eq:unique_NP} establishes~\Cref{theorem:symmetric-new}. We proceed now with~\Cref{theorem:non-symmetric}.
\begin{proof}[Proof of~\Cref{theorem:non-symmetric}]
It suffices to consider the antisymmetric function $f(\vx,\vy) \defeq \vy^{\top} \mat{B}\vy - \vx^{\top} \mat{B}\vx$, where symmetric matrix $\mat{B}$ is defined as in \eqref{eq:unique_NP}, using our new matrix $\lineA$ instead of $\mat{A}$ (see \eqref{eq:unique_NP_matrix}). Any $\epsilon$-first-order Nash equilibrium $(\vx^*, \vy^*)$ of this (separable) min-max optimization problem induces, two symmetric $\epsilon$-Nash equilibria---namely, $(\vx^*, \vx^*)$ and $(\vy^*, \vy^*)$---in the symmetric, identical-interest, game $(\mat{B}, \mat{B})$. Using~\Cref{theorem:symmetric-new}, the claim follows.
\end{proof}

Finally, the proof of~\Cref{theorem:uniqueATG} that was claimed earlier follows immediately by combining~\Cref{theorem:symmetric-new} with the reduction of~\Cref{sec:cls}, and in particular, \Cref{lemma:closeness,lemma:small_z}.

\iffalse

\begin{corollary}\label{cor:either} Given antisymmetric, continuous, $L$-smooth functions $f,g$, the problem of finding either an approximate symmetric fixed point of GDA for $f$ or an approximate non-symmetric fixed point of GDA for $g$ is \PPAD-complete.
\end{corollary}
\begin{proof} We consider the problem $\mathcal{A}$ to be $\symgdaFixed$ and $\mathcal{B}$ $\nsymgdaFixed.$ Using the same reasoning as by~\citet{Daskalakis11:Continuous}, we conclude that the problem $\textrm{EITHER}(\mathcal{A},\mathcal{B})$ is $(\PPAD \cap \NPP)$-complete, thus \PPAD-complete.
\end{proof}
\begin{remark} If one can show Corollary \ref{cor:either} for a single function, this would imply \PPAD-completeness of finding fixed points of GDA for Cartesian product domains.
\end{remark}
\fi

\subsection{Team zero-sum games}
\label{sec:teamzero}

Our previous hardness result concerning symmetric min-max optimization problems does not have any immediate implications for (normal-form) team zero-sum games since the class of hard instances we constructed earlier contains a quadratic term. Our next result provides such a hardness result by combining the basic gadget we introduced in~\Cref{sec:cls} in the context of adversarial team games; the basic pieces of the argument are similar to the ones we described in~\Cref{sec:cls}, and so the proof is deferred to~\Cref{sec:proofs4}. Our goal is to prove the following.

\teamhard*

We begin by describing the class of $3$ vs. $3$ team zero-sum games upon which our hardness result is based on. To do so, based on~\eqref{eq:util-atg}, let us define the auxiliary function
\begin{equation*}
    \delta: \Delta^n \times \Delta^n \times \Delta^{2n} \ni (\vx, \vy, \vz) \mapsto \frac{|\Amin|}{\epsilon} \sum_{i=1}^n ( z_i (x_i - y_i) + z_{n+i} (y_i - x_i)) + |\Amin| z_{2n+1}.
\end{equation*}
In what follows, the $3$ players of the one team will be identified with $(\vx, \vy, \vz)$, while the $3$ players of the other team with $(\hvx, \hvy, \hvz)$. Now, we define the utility of the latter team to be
\begin{equation}
    \label{eq:team-sym}
        u(\vx, \vy, \vz , \hat{\vx}, \hat{\vy}, \hat{\vz})  = \langle \vx, \mat{A} \vy \rangle - \langle \hvx, \mat{A} \hvy \rangle + \langle \vx, \mat{C} \hvx \rangle  + \delta(\vx, \vy, \hvz) - \delta(\hvx, \hvy, \vz),
\end{equation}
where again $\mat{A}$ is a symmetric matrix and $\mat{C}$ is skew-symmetric. As in~\Cref{sec:cls}, it is assumed for convenience that $\mat{A}_{i, j} \leq -1$ for all $i, j \in [n]$; we denote by $\Amin$ the minimum entry of $\mat{A}$. That the game defined above is symmetric is clear: $u(\vx, \vy, \vz, \hvx, \hvy, \hvz) = - u(\hvx, \hvy, \hvz, \vx, \vy, \vz)$ for all joint strategies (since $\mat{C} = - \mat{C}^\top$). It is also evident that~\eqref{eq:team-sym} is a polymatrix game, as promised.

The first key lemma, which mirrors~\Cref{lemma:closeness}, shows that, in an approximate Nash equilibrium of~\eqref{eq:team-sym}, $\vx \approx \vy$ and $\hvx \approx \hvy$. This is crucial as it allows us to construct---up to some small error---quadratic terms in the utility function, as in our hardness result for symmetric min-max optimization.

\begin{restatable}{lemma}{closeteams}
    \label{lemma:close-teams}
    Let $(\vx^*, \vy^*, \vz^*, \hvx^*, \hvy^*, \hvz^*)$ be an $\epsilon^2$-Nash equilibrium of~\eqref{eq:team-sym} with $\epsilon^2 \leq \nicefrac{1}{2}$. Then, $\| \vx^* - \vy^* \|_\infty \leq 2 \epsilon$ and $\| \hvx^* - \hvy^* \|_\infty \leq 2 \epsilon$.
\end{restatable}

Next, following the argument of~\Cref{lemma:small_z}, we show that, in equilibrium, Players $\vz$ and $\hvz$ place most of their probability mass on action $a_{2n + 1}$, thereby having only a small effect on the game between Players $\vx$ and $\vy$ vs. $\hvx$ and $\hvy$.

\begin{restatable}{lemma}{smallz}
    \label{lemma:smallzteam}
    Let $(\vx^*, \vy^*, \vz^*, \hvx^*, \hvy^*, \hvz^*)$ be an $\epsilon^2$-Nash equilibrium of~\eqref{eq:team-sym} with $\epsilon \leq \nicefrac{1}{10}$. Then, $z_j, \hat{z}_j \leq 9 \epsilon$ for all $j \in [2n]$.
\end{restatable}

Armed with those two basic lemmas, we complete the proof of~\Cref{theorem:team-hard} in~\Cref{sec:proofs4}.


\vspace{-0.2cm}
\section{Impact: Why Free Scientific Knowledge?}
\vspace{-0.1cm}

Historically, making knowledge widely available has driven transformative progress. Gutenberg’s printing press broke medieval monopolies on information, increasing literacy and contributing to the Renaissance and Scientific Revolution. In today's world, open source projects such as GNU/Linux and Wikipedia show that freely accessible and modifiable knowledge fosters innovation while ensuring creators are credited through copyleft licenses. These examples highlight a key idea: \textit{access to essential knowledge supports overall advancement.} 

This aligns with the arguments made by Prabhakaran et al. \cite{humanrightsbasedapproachresponsible}, who specifically highlight the \textbf{ human right to participate in scientific advancement} as enshrined in the Universal Declaration of Human Rights. They emphasize that this right underscores the importance of \textit{ equal access to the benefits of scientific progress for all}, a principle directly supported by our proposal for Knowledge Units. The UN Special Rapporteur on Cultural Rights further reinforces this, advocating for the expansion of copyright exceptions to broaden access to scientific knowledge as a crucial component of the right to science and culture \cite{scienceright}. 

However, current intellectual property regimes often create ``patently unfair" barriers to this knowledge, preventing innovation and access, especially in areas critical to human rights, as Hale compellingly argues \cite{patentlyunfair}. Finding a solution requires carefully balancing the imperative of open access with the legitimate rights of authors. As Austin and Ginsburg remind us, authors' rights are also human rights, necessitating robust protection \cite{authorhumanrights}. Shareable knowledge entities like Knowledge Units offer a potential mechanism to achieve this delicate balance in the scientific domain, enabling wider dissemination of research findings while respecting authors' fundamental rights.

\vspace{-0.2cm}
\subsection{Impact Across Sectors}

\textbf{Researchers:} Collaboration across different fields becomes easier when knowledge is shared openly. For instance, combining machine learning with biology or applying quantum principles to cryptography can lead to important breakthroughs. Removing copyright restrictions allows researchers to freely use data and methods, speeding up discoveries while respecting original contributions.

\textbf{Practitioners:} Professionals, especially in healthcare, benefit from immediate access to the latest research. Quick access to newer insights on the effectiveness of drugs, and alternative treatments speeds up adoption and awareness, potentially saving lives. Additionally, open knowledge helps developing countries gain access to health innovations.

\textbf{Education:} Education becomes more accessible when teachers use the latest research to create up-to-date curricula without prohibitive costs. Students can access high-quality research materials and use LM assistance to better understand complex topics, enhancing their learning experience and making high-quality education more accessible.

\textbf{Public Trust:} When information is transparent and accessible, the public can better understand and trust decision-making processes. Open access to government policies and industry practices allows people to review and verify information, helping to reduce misinformation. This transparency encourages critical thinking and builds trust in scientific and governmental institutions.

Overall, making scientific knowledge accessible supports global fairness. By viewing knowledge as a common resource rather than a product to be sold, we can speed up innovation, encourage critical thinking, and empower communities to address important challenges.

\vspace{-0.2cm}
\section{Open Problems}
\vspace{-0.1cm}

Moving forward, we identify key research directions to further exploit the potential of converting original texts into shareable knowledge entities such as demonstrated by the conversion into Knowledge Units in this work:


\textbf{1. Enhancing Factual Accuracy and Reliability:}  Refining KUs through cross-referencing with source texts and incorporating community-driven correction mechanisms, similar to Wikipedia, can minimize hallucinations and ensure the long-term accuracy of knowledge-based datasets at scale.

\textbf{2. Developing Applications for Education and Research:}  Using KU-based conversion for datasets to be employed in practical tools, such as search interfaces and learning platforms, can ensure rapid dissemination of any new knowledge into shareable downstream resources, significantly improving the accessibility, spread, and impact of KUs.

\textbf{3. Establishing Standards for Knowledge Interoperability and Reuse:}  Future research should focus on defining standardized formats for entities like KU and knowledge graph layouts \citep{lenat1990cyc}. These standards are essential to unlock seamless interoperability, facilitate reuse across diverse platforms, and foster a vibrant ecosystem of open scientific knowledge. 

\textbf{4. Interconnecting Shareable Knowledge for Scientific Workflow Assistance and Automation:} There might be further potential in constructing a semantic web that interconnects publicly shared knowledge, together with mechanisms that continually update and validate all shareable knowledge units. This can be starting point for a platform that uses all collected knowledge to assist scientific workflows, for instance by feeding such a semantic web into recently developed reasoning models equipped with retrieval augmented generation. Such assistance could assemble knowledge across multiple scientific papers, guiding scientists more efficiently through vast research landscapes. Given further progress in model capabilities, validation, self-repair and evolving new knowledge from already existing vast collection in the semantic web can lead to automation of scientific discovery, assuming that knowledge data in the semantic web can be freely shared.

We open-source our code and encourage collaboration to improve extraction pipelines, enhance Knowledge Unit capabilities, and expand coverage to additional fields.

\vspace{-0.2cm}
\section{Conclusion}
\vspace{-0.1cm}

In this paper, we highlight the potential of systematically separating factual scientific knowledge from protected artistic or stylistic expression. By representing scientific insights as structured facts and relationships, prototypes like Knowledge Units (KUs) offer a pathway to broaden access to scientific knowledge without infringing copyright, aligning with legal principles like German \S 24(1) UrhG and U.S. fair use standards. Extensive testing across a range of domains and models shows evidence that Knowledge Units (KUs) can feasibly retain core information. These findings offer a promising way forward for openly disseminating scientific information while respecting copyright constraints.

\section*{Author Contributions}

Christoph conceived the project and led organization. Christoph and Gollam led all the experiments. Nick and Huu led the legal aspects. Tawsif led the data collection. Ameya and Andreas led the manuscript writing. Ludwig, Sören, Robert, Jenia and Matthias provided feedback. advice and scientific supervision throughout the project. 

\section*{Acknowledgements}

The authors would like to thank (in alphabetical order): Sebastian Dziadzio, Kristof Meding, Tea Mustać, Shantanu Prabhat for insightful feedback and suggestions. Special thanks to Andrej Radonjic for help in scaling up data collection. GR and SA acknowledge financial support by the German Research Foundation (DFG) for the NFDI4DataScience Initiative (project number 460234259). AP and MB acknowledge financial support by the Federal Ministry of Education and Research (BMBF), FKZ: 011524085B and Open Philanthropy Foundation funded by the Good Ventures Foundation. AH acknowledges financial support by the Federal Ministry of Education and Research (BMBF), FKZ: 01IS24079A and the Carl Zeiss Foundation through the project "Certification and Foundations of Safe ML Systems" as well as the support from the International Max Planck Research School for Intelligent Systems (IMPRS-IS). JJ acknowledges funding by the Federal Ministry of Education and Research of Germany (BMBF) under grant no. 01IS22094B (WestAI - AI Service Center West), under grant no. 01IS24085C (OPENHAFM) and under the grant DE002571 (MINERVA), as well as co-funding by EU from EuroHPC Joint Undertaking programm under grant no. 101182737 (MINERVA) and from Digital Europe Programme under grant no. 101195233 (openEuroLLM) 

\section*{Acknowledgments}
I.P. would like to acknowledge ICS research award and a start-up grant from UCI. Part of this work was done while I.P. and J.Y. were visiting Archimedes Research Unit. 
T.S. is supported by the Vannevar Bush Faculty Fellowship ONR N00014-23-1-2876, National Science Foundation grants RI-2312342 and RI-1901403, ARO award W911NF2210266, and NIH award A240108S001. We are grateful to Alexandros Hollender for many valuable discussions.

%\bibliographystyle{plain}
%\bibliography{main}

\printbibliography

\appendix
\subsection{Lloyd-Max Algorithm}
\label{subsec:Lloyd-Max}
For a given quantization bitwidth $B$ and an operand $\bm{X}$, the Lloyd-Max algorithm finds $2^B$ quantization levels $\{\hat{x}_i\}_{i=1}^{2^B}$ such that quantizing $\bm{X}$ by rounding each scalar in $\bm{X}$ to the nearest quantization level minimizes the quantization MSE. 

The algorithm starts with an initial guess of quantization levels and then iteratively computes quantization thresholds $\{\tau_i\}_{i=1}^{2^B-1}$ and updates quantization levels $\{\hat{x}_i\}_{i=1}^{2^B}$. Specifically, at iteration $n$, thresholds are set to the midpoints of the previous iteration's levels:
\begin{align*}
    \tau_i^{(n)}=\frac{\hat{x}_i^{(n-1)}+\hat{x}_{i+1}^{(n-1)}}2 \text{ for } i=1\ldots 2^B-1
\end{align*}
Subsequently, the quantization levels are re-computed as conditional means of the data regions defined by the new thresholds:
\begin{align*}
    \hat{x}_i^{(n)}=\mathbb{E}\left[ \bm{X} \big| \bm{X}\in [\tau_{i-1}^{(n)},\tau_i^{(n)}] \right] \text{ for } i=1\ldots 2^B
\end{align*}
where to satisfy boundary conditions we have $\tau_0=-\infty$ and $\tau_{2^B}=\infty$. The algorithm iterates the above steps until convergence.

Figure \ref{fig:lm_quant} compares the quantization levels of a $7$-bit floating point (E3M3) quantizer (left) to a $7$-bit Lloyd-Max quantizer (right) when quantizing a layer of weights from the GPT3-126M model at a per-tensor granularity. As shown, the Lloyd-Max quantizer achieves substantially lower quantization MSE. Further, Table \ref{tab:FP7_vs_LM7} shows the superior perplexity achieved by Lloyd-Max quantizers for bitwidths of $7$, $6$ and $5$. The difference between the quantizers is clear at 5 bits, where per-tensor FP quantization incurs a drastic and unacceptable increase in perplexity, while Lloyd-Max quantization incurs a much smaller increase. Nevertheless, we note that even the optimal Lloyd-Max quantizer incurs a notable ($\sim 1.5$) increase in perplexity due to the coarse granularity of quantization. 

\begin{figure}[h]
  \centering
  \includegraphics[width=0.7\linewidth]{sections/figures/LM7_FP7.pdf}
  \caption{\small Quantization levels and the corresponding quantization MSE of Floating Point (left) vs Lloyd-Max (right) Quantizers for a layer of weights in the GPT3-126M model.}
  \label{fig:lm_quant}
\end{figure}

\begin{table}[h]\scriptsize
\begin{center}
\caption{\label{tab:FP7_vs_LM7} \small Comparing perplexity (lower is better) achieved by floating point quantizers and Lloyd-Max quantizers on a GPT3-126M model for the Wikitext-103 dataset.}
\begin{tabular}{c|cc|c}
\hline
 \multirow{2}{*}{\textbf{Bitwidth}} & \multicolumn{2}{|c|}{\textbf{Floating-Point Quantizer}} & \textbf{Lloyd-Max Quantizer} \\
 & Best Format & Wikitext-103 Perplexity & Wikitext-103 Perplexity \\
\hline
7 & E3M3 & 18.32 & 18.27 \\
6 & E3M2 & 19.07 & 18.51 \\
5 & E4M0 & 43.89 & 19.71 \\
\hline
\end{tabular}
\end{center}
\end{table}

\subsection{Proof of Local Optimality of LO-BCQ}
\label{subsec:lobcq_opt_proof}
For a given block $\bm{b}_j$, the quantization MSE during LO-BCQ can be empirically evaluated as $\frac{1}{L_b}\lVert \bm{b}_j- \bm{\hat{b}}_j\rVert^2_2$ where $\bm{\hat{b}}_j$ is computed from equation (\ref{eq:clustered_quantization_definition}) as $C_{f(\bm{b}_j)}(\bm{b}_j)$. Further, for a given block cluster $\mathcal{B}_i$, we compute the quantization MSE as $\frac{1}{|\mathcal{B}_{i}|}\sum_{\bm{b} \in \mathcal{B}_{i}} \frac{1}{L_b}\lVert \bm{b}- C_i^{(n)}(\bm{b})\rVert^2_2$. Therefore, at the end of iteration $n$, we evaluate the overall quantization MSE $J^{(n)}$ for a given operand $\bm{X}$ composed of $N_c$ block clusters as:
\begin{align*}
    \label{eq:mse_iter_n}
    J^{(n)} = \frac{1}{N_c} \sum_{i=1}^{N_c} \frac{1}{|\mathcal{B}_{i}^{(n)}|}\sum_{\bm{v} \in \mathcal{B}_{i}^{(n)}} \frac{1}{L_b}\lVert \bm{b}- B_i^{(n)}(\bm{b})\rVert^2_2
\end{align*}

At the end of iteration $n$, the codebooks are updated from $\mathcal{C}^{(n-1)}$ to $\mathcal{C}^{(n)}$. However, the mapping of a given vector $\bm{b}_j$ to quantizers $\mathcal{C}^{(n)}$ remains as  $f^{(n)}(\bm{b}_j)$. At the next iteration, during the vector clustering step, $f^{(n+1)}(\bm{b}_j)$ finds new mapping of $\bm{b}_j$ to updated codebooks $\mathcal{C}^{(n)}$ such that the quantization MSE over the candidate codebooks is minimized. Therefore, we obtain the following result for $\bm{b}_j$:
\begin{align*}
\frac{1}{L_b}\lVert \bm{b}_j - C_{f^{(n+1)}(\bm{b}_j)}^{(n)}(\bm{b}_j)\rVert^2_2 \le \frac{1}{L_b}\lVert \bm{b}_j - C_{f^{(n)}(\bm{b}_j)}^{(n)}(\bm{b}_j)\rVert^2_2
\end{align*}

That is, quantizing $\bm{b}_j$ at the end of the block clustering step of iteration $n+1$ results in lower quantization MSE compared to quantizing at the end of iteration $n$. Since this is true for all $\bm{b} \in \bm{X}$, we assert the following:
\begin{equation}
\begin{split}
\label{eq:mse_ineq_1}
    \tilde{J}^{(n+1)} &= \frac{1}{N_c} \sum_{i=1}^{N_c} \frac{1}{|\mathcal{B}_{i}^{(n+1)}|}\sum_{\bm{b} \in \mathcal{B}_{i}^{(n+1)}} \frac{1}{L_b}\lVert \bm{b} - C_i^{(n)}(b)\rVert^2_2 \le J^{(n)}
\end{split}
\end{equation}
where $\tilde{J}^{(n+1)}$ is the the quantization MSE after the vector clustering step at iteration $n+1$.

Next, during the codebook update step (\ref{eq:quantizers_update}) at iteration $n+1$, the per-cluster codebooks $\mathcal{C}^{(n)}$ are updated to $\mathcal{C}^{(n+1)}$ by invoking the Lloyd-Max algorithm \citep{Lloyd}. We know that for any given value distribution, the Lloyd-Max algorithm minimizes the quantization MSE. Therefore, for a given vector cluster $\mathcal{B}_i$ we obtain the following result:

\begin{equation}
    \frac{1}{|\mathcal{B}_{i}^{(n+1)}|}\sum_{\bm{b} \in \mathcal{B}_{i}^{(n+1)}} \frac{1}{L_b}\lVert \bm{b}- C_i^{(n+1)}(\bm{b})\rVert^2_2 \le \frac{1}{|\mathcal{B}_{i}^{(n+1)}|}\sum_{\bm{b} \in \mathcal{B}_{i}^{(n+1)}} \frac{1}{L_b}\lVert \bm{b}- C_i^{(n)}(\bm{b})\rVert^2_2
\end{equation}

The above equation states that quantizing the given block cluster $\mathcal{B}_i$ after updating the associated codebook from $C_i^{(n)}$ to $C_i^{(n+1)}$ results in lower quantization MSE. Since this is true for all the block clusters, we derive the following result: 
\begin{equation}
\begin{split}
\label{eq:mse_ineq_2}
     J^{(n+1)} &= \frac{1}{N_c} \sum_{i=1}^{N_c} \frac{1}{|\mathcal{B}_{i}^{(n+1)}|}\sum_{\bm{b} \in \mathcal{B}_{i}^{(n+1)}} \frac{1}{L_b}\lVert \bm{b}- C_i^{(n+1)}(\bm{b})\rVert^2_2  \le \tilde{J}^{(n+1)}   
\end{split}
\end{equation}

Following (\ref{eq:mse_ineq_1}) and (\ref{eq:mse_ineq_2}), we find that the quantization MSE is non-increasing for each iteration, that is, $J^{(1)} \ge J^{(2)} \ge J^{(3)} \ge \ldots \ge J^{(M)}$ where $M$ is the maximum number of iterations. 
%Therefore, we can say that if the algorithm converges, then it must be that it has converged to a local minimum. 
\hfill $\blacksquare$


\begin{figure}
    \begin{center}
    \includegraphics[width=0.5\textwidth]{sections//figures/mse_vs_iter.pdf}
    \end{center}
    \caption{\small NMSE vs iterations during LO-BCQ compared to other block quantization proposals}
    \label{fig:nmse_vs_iter}
\end{figure}

Figure \ref{fig:nmse_vs_iter} shows the empirical convergence of LO-BCQ across several block lengths and number of codebooks. Also, the MSE achieved by LO-BCQ is compared to baselines such as MXFP and VSQ. As shown, LO-BCQ converges to a lower MSE than the baselines. Further, we achieve better convergence for larger number of codebooks ($N_c$) and for a smaller block length ($L_b$), both of which increase the bitwidth of BCQ (see Eq \ref{eq:bitwidth_bcq}).


\subsection{Additional Accuracy Results}
%Table \ref{tab:lobcq_config} lists the various LOBCQ configurations and their corresponding bitwidths.
\begin{table}
\setlength{\tabcolsep}{4.75pt}
\begin{center}
\caption{\label{tab:lobcq_config} Various LO-BCQ configurations and their bitwidths.}
\begin{tabular}{|c||c|c|c|c||c|c||c|} 
\hline
 & \multicolumn{4}{|c||}{$L_b=8$} & \multicolumn{2}{|c||}{$L_b=4$} & $L_b=2$ \\
 \hline
 \backslashbox{$L_A$\kern-1em}{\kern-1em$N_c$} & 2 & 4 & 8 & 16 & 2 & 4 & 2 \\
 \hline
 64 & 4.25 & 4.375 & 4.5 & 4.625 & 4.375 & 4.625 & 4.625\\
 \hline
 32 & 4.375 & 4.5 & 4.625& 4.75 & 4.5 & 4.75 & 4.75 \\
 \hline
 16 & 4.625 & 4.75& 4.875 & 5 & 4.75 & 5 & 5 \\
 \hline
\end{tabular}
\end{center}
\end{table}

%\subsection{Perplexity achieved by various LO-BCQ configurations on Wikitext-103 dataset}

\begin{table} \centering
\begin{tabular}{|c||c|c|c|c||c|c||c|} 
\hline
 $L_b \rightarrow$& \multicolumn{4}{c||}{8} & \multicolumn{2}{c||}{4} & 2\\
 \hline
 \backslashbox{$L_A$\kern-1em}{\kern-1em$N_c$} & 2 & 4 & 8 & 16 & 2 & 4 & 2  \\
 %$N_c \rightarrow$ & 2 & 4 & 8 & 16 & 2 & 4 & 2 \\
 \hline
 \hline
 \multicolumn{8}{c}{GPT3-1.3B (FP32 PPL = 9.98)} \\ 
 \hline
 \hline
 64 & 10.40 & 10.23 & 10.17 & 10.15 &  10.28 & 10.18 & 10.19 \\
 \hline
 32 & 10.25 & 10.20 & 10.15 & 10.12 &  10.23 & 10.17 & 10.17 \\
 \hline
 16 & 10.22 & 10.16 & 10.10 & 10.09 &  10.21 & 10.14 & 10.16 \\
 \hline
  \hline
 \multicolumn{8}{c}{GPT3-8B (FP32 PPL = 7.38)} \\ 
 \hline
 \hline
 64 & 7.61 & 7.52 & 7.48 &  7.47 &  7.55 &  7.49 & 7.50 \\
 \hline
 32 & 7.52 & 7.50 & 7.46 &  7.45 &  7.52 &  7.48 & 7.48  \\
 \hline
 16 & 7.51 & 7.48 & 7.44 &  7.44 &  7.51 &  7.49 & 7.47  \\
 \hline
\end{tabular}
\caption{\label{tab:ppl_gpt3_abalation} Wikitext-103 perplexity across GPT3-1.3B and 8B models.}
\end{table}

\begin{table} \centering
\begin{tabular}{|c||c|c|c|c||} 
\hline
 $L_b \rightarrow$& \multicolumn{4}{c||}{8}\\
 \hline
 \backslashbox{$L_A$\kern-1em}{\kern-1em$N_c$} & 2 & 4 & 8 & 16 \\
 %$N_c \rightarrow$ & 2 & 4 & 8 & 16 & 2 & 4 & 2 \\
 \hline
 \hline
 \multicolumn{5}{|c|}{Llama2-7B (FP32 PPL = 5.06)} \\ 
 \hline
 \hline
 64 & 5.31 & 5.26 & 5.19 & 5.18  \\
 \hline
 32 & 5.23 & 5.25 & 5.18 & 5.15  \\
 \hline
 16 & 5.23 & 5.19 & 5.16 & 5.14  \\
 \hline
 \multicolumn{5}{|c|}{Nemotron4-15B (FP32 PPL = 5.87)} \\ 
 \hline
 \hline
 64  & 6.3 & 6.20 & 6.13 & 6.08  \\
 \hline
 32  & 6.24 & 6.12 & 6.07 & 6.03  \\
 \hline
 16  & 6.12 & 6.14 & 6.04 & 6.02  \\
 \hline
 \multicolumn{5}{|c|}{Nemotron4-340B (FP32 PPL = 3.48)} \\ 
 \hline
 \hline
 64 & 3.67 & 3.62 & 3.60 & 3.59 \\
 \hline
 32 & 3.63 & 3.61 & 3.59 & 3.56 \\
 \hline
 16 & 3.61 & 3.58 & 3.57 & 3.55 \\
 \hline
\end{tabular}
\caption{\label{tab:ppl_llama7B_nemo15B} Wikitext-103 perplexity compared to FP32 baseline in Llama2-7B and Nemotron4-15B, 340B models}
\end{table}

%\subsection{Perplexity achieved by various LO-BCQ configurations on MMLU dataset}


\begin{table} \centering
\begin{tabular}{|c||c|c|c|c||c|c|c|c|} 
\hline
 $L_b \rightarrow$& \multicolumn{4}{c||}{8} & \multicolumn{4}{c||}{8}\\
 \hline
 \backslashbox{$L_A$\kern-1em}{\kern-1em$N_c$} & 2 & 4 & 8 & 16 & 2 & 4 & 8 & 16  \\
 %$N_c \rightarrow$ & 2 & 4 & 8 & 16 & 2 & 4 & 2 \\
 \hline
 \hline
 \multicolumn{5}{|c|}{Llama2-7B (FP32 Accuracy = 45.8\%)} & \multicolumn{4}{|c|}{Llama2-70B (FP32 Accuracy = 69.12\%)} \\ 
 \hline
 \hline
 64 & 43.9 & 43.4 & 43.9 & 44.9 & 68.07 & 68.27 & 68.17 & 68.75 \\
 \hline
 32 & 44.5 & 43.8 & 44.9 & 44.5 & 68.37 & 68.51 & 68.35 & 68.27  \\
 \hline
 16 & 43.9 & 42.7 & 44.9 & 45 & 68.12 & 68.77 & 68.31 & 68.59  \\
 \hline
 \hline
 \multicolumn{5}{|c|}{GPT3-22B (FP32 Accuracy = 38.75\%)} & \multicolumn{4}{|c|}{Nemotron4-15B (FP32 Accuracy = 64.3\%)} \\ 
 \hline
 \hline
 64 & 36.71 & 38.85 & 38.13 & 38.92 & 63.17 & 62.36 & 63.72 & 64.09 \\
 \hline
 32 & 37.95 & 38.69 & 39.45 & 38.34 & 64.05 & 62.30 & 63.8 & 64.33  \\
 \hline
 16 & 38.88 & 38.80 & 38.31 & 38.92 & 63.22 & 63.51 & 63.93 & 64.43  \\
 \hline
\end{tabular}
\caption{\label{tab:mmlu_abalation} Accuracy on MMLU dataset across GPT3-22B, Llama2-7B, 70B and Nemotron4-15B models.}
\end{table}


%\subsection{Perplexity achieved by various LO-BCQ configurations on LM evaluation harness}

\begin{table} \centering
\begin{tabular}{|c||c|c|c|c||c|c|c|c|} 
\hline
 $L_b \rightarrow$& \multicolumn{4}{c||}{8} & \multicolumn{4}{c||}{8}\\
 \hline
 \backslashbox{$L_A$\kern-1em}{\kern-1em$N_c$} & 2 & 4 & 8 & 16 & 2 & 4 & 8 & 16  \\
 %$N_c \rightarrow$ & 2 & 4 & 8 & 16 & 2 & 4 & 2 \\
 \hline
 \hline
 \multicolumn{5}{|c|}{Race (FP32 Accuracy = 37.51\%)} & \multicolumn{4}{|c|}{Boolq (FP32 Accuracy = 64.62\%)} \\ 
 \hline
 \hline
 64 & 36.94 & 37.13 & 36.27 & 37.13 & 63.73 & 62.26 & 63.49 & 63.36 \\
 \hline
 32 & 37.03 & 36.36 & 36.08 & 37.03 & 62.54 & 63.51 & 63.49 & 63.55  \\
 \hline
 16 & 37.03 & 37.03 & 36.46 & 37.03 & 61.1 & 63.79 & 63.58 & 63.33  \\
 \hline
 \hline
 \multicolumn{5}{|c|}{Winogrande (FP32 Accuracy = 58.01\%)} & \multicolumn{4}{|c|}{Piqa (FP32 Accuracy = 74.21\%)} \\ 
 \hline
 \hline
 64 & 58.17 & 57.22 & 57.85 & 58.33 & 73.01 & 73.07 & 73.07 & 72.80 \\
 \hline
 32 & 59.12 & 58.09 & 57.85 & 58.41 & 73.01 & 73.94 & 72.74 & 73.18  \\
 \hline
 16 & 57.93 & 58.88 & 57.93 & 58.56 & 73.94 & 72.80 & 73.01 & 73.94  \\
 \hline
\end{tabular}
\caption{\label{tab:mmlu_abalation} Accuracy on LM evaluation harness tasks on GPT3-1.3B model.}
\end{table}

\begin{table} \centering
\begin{tabular}{|c||c|c|c|c||c|c|c|c|} 
\hline
 $L_b \rightarrow$& \multicolumn{4}{c||}{8} & \multicolumn{4}{c||}{8}\\
 \hline
 \backslashbox{$L_A$\kern-1em}{\kern-1em$N_c$} & 2 & 4 & 8 & 16 & 2 & 4 & 8 & 16  \\
 %$N_c \rightarrow$ & 2 & 4 & 8 & 16 & 2 & 4 & 2 \\
 \hline
 \hline
 \multicolumn{5}{|c|}{Race (FP32 Accuracy = 41.34\%)} & \multicolumn{4}{|c|}{Boolq (FP32 Accuracy = 68.32\%)} \\ 
 \hline
 \hline
 64 & 40.48 & 40.10 & 39.43 & 39.90 & 69.20 & 68.41 & 69.45 & 68.56 \\
 \hline
 32 & 39.52 & 39.52 & 40.77 & 39.62 & 68.32 & 67.43 & 68.17 & 69.30  \\
 \hline
 16 & 39.81 & 39.71 & 39.90 & 40.38 & 68.10 & 66.33 & 69.51 & 69.42  \\
 \hline
 \hline
 \multicolumn{5}{|c|}{Winogrande (FP32 Accuracy = 67.88\%)} & \multicolumn{4}{|c|}{Piqa (FP32 Accuracy = 78.78\%)} \\ 
 \hline
 \hline
 64 & 66.85 & 66.61 & 67.72 & 67.88 & 77.31 & 77.42 & 77.75 & 77.64 \\
 \hline
 32 & 67.25 & 67.72 & 67.72 & 67.00 & 77.31 & 77.04 & 77.80 & 77.37  \\
 \hline
 16 & 68.11 & 68.90 & 67.88 & 67.48 & 77.37 & 78.13 & 78.13 & 77.69  \\
 \hline
\end{tabular}
\caption{\label{tab:mmlu_abalation} Accuracy on LM evaluation harness tasks on GPT3-8B model.}
\end{table}

\begin{table} \centering
\begin{tabular}{|c||c|c|c|c||c|c|c|c|} 
\hline
 $L_b \rightarrow$& \multicolumn{4}{c||}{8} & \multicolumn{4}{c||}{8}\\
 \hline
 \backslashbox{$L_A$\kern-1em}{\kern-1em$N_c$} & 2 & 4 & 8 & 16 & 2 & 4 & 8 & 16  \\
 %$N_c \rightarrow$ & 2 & 4 & 8 & 16 & 2 & 4 & 2 \\
 \hline
 \hline
 \multicolumn{5}{|c|}{Race (FP32 Accuracy = 40.67\%)} & \multicolumn{4}{|c|}{Boolq (FP32 Accuracy = 76.54\%)} \\ 
 \hline
 \hline
 64 & 40.48 & 40.10 & 39.43 & 39.90 & 75.41 & 75.11 & 77.09 & 75.66 \\
 \hline
 32 & 39.52 & 39.52 & 40.77 & 39.62 & 76.02 & 76.02 & 75.96 & 75.35  \\
 \hline
 16 & 39.81 & 39.71 & 39.90 & 40.38 & 75.05 & 73.82 & 75.72 & 76.09  \\
 \hline
 \hline
 \multicolumn{5}{|c|}{Winogrande (FP32 Accuracy = 70.64\%)} & \multicolumn{4}{|c|}{Piqa (FP32 Accuracy = 79.16\%)} \\ 
 \hline
 \hline
 64 & 69.14 & 70.17 & 70.17 & 70.56 & 78.24 & 79.00 & 78.62 & 78.73 \\
 \hline
 32 & 70.96 & 69.69 & 71.27 & 69.30 & 78.56 & 79.49 & 79.16 & 78.89  \\
 \hline
 16 & 71.03 & 69.53 & 69.69 & 70.40 & 78.13 & 79.16 & 79.00 & 79.00  \\
 \hline
\end{tabular}
\caption{\label{tab:mmlu_abalation} Accuracy on LM evaluation harness tasks on GPT3-22B model.}
\end{table}

\begin{table} \centering
\begin{tabular}{|c||c|c|c|c||c|c|c|c|} 
\hline
 $L_b \rightarrow$& \multicolumn{4}{c||}{8} & \multicolumn{4}{c||}{8}\\
 \hline
 \backslashbox{$L_A$\kern-1em}{\kern-1em$N_c$} & 2 & 4 & 8 & 16 & 2 & 4 & 8 & 16  \\
 %$N_c \rightarrow$ & 2 & 4 & 8 & 16 & 2 & 4 & 2 \\
 \hline
 \hline
 \multicolumn{5}{|c|}{Race (FP32 Accuracy = 44.4\%)} & \multicolumn{4}{|c|}{Boolq (FP32 Accuracy = 79.29\%)} \\ 
 \hline
 \hline
 64 & 42.49 & 42.51 & 42.58 & 43.45 & 77.58 & 77.37 & 77.43 & 78.1 \\
 \hline
 32 & 43.35 & 42.49 & 43.64 & 43.73 & 77.86 & 75.32 & 77.28 & 77.86  \\
 \hline
 16 & 44.21 & 44.21 & 43.64 & 42.97 & 78.65 & 77 & 76.94 & 77.98  \\
 \hline
 \hline
 \multicolumn{5}{|c|}{Winogrande (FP32 Accuracy = 69.38\%)} & \multicolumn{4}{|c|}{Piqa (FP32 Accuracy = 78.07\%)} \\ 
 \hline
 \hline
 64 & 68.9 & 68.43 & 69.77 & 68.19 & 77.09 & 76.82 & 77.09 & 77.86 \\
 \hline
 32 & 69.38 & 68.51 & 68.82 & 68.90 & 78.07 & 76.71 & 78.07 & 77.86  \\
 \hline
 16 & 69.53 & 67.09 & 69.38 & 68.90 & 77.37 & 77.8 & 77.91 & 77.69  \\
 \hline
\end{tabular}
\caption{\label{tab:mmlu_abalation} Accuracy on LM evaluation harness tasks on Llama2-7B model.}
\end{table}

\begin{table} \centering
\begin{tabular}{|c||c|c|c|c||c|c|c|c|} 
\hline
 $L_b \rightarrow$& \multicolumn{4}{c||}{8} & \multicolumn{4}{c||}{8}\\
 \hline
 \backslashbox{$L_A$\kern-1em}{\kern-1em$N_c$} & 2 & 4 & 8 & 16 & 2 & 4 & 8 & 16  \\
 %$N_c \rightarrow$ & 2 & 4 & 8 & 16 & 2 & 4 & 2 \\
 \hline
 \hline
 \multicolumn{5}{|c|}{Race (FP32 Accuracy = 48.8\%)} & \multicolumn{4}{|c|}{Boolq (FP32 Accuracy = 85.23\%)} \\ 
 \hline
 \hline
 64 & 49.00 & 49.00 & 49.28 & 48.71 & 82.82 & 84.28 & 84.03 & 84.25 \\
 \hline
 32 & 49.57 & 48.52 & 48.33 & 49.28 & 83.85 & 84.46 & 84.31 & 84.93  \\
 \hline
 16 & 49.85 & 49.09 & 49.28 & 48.99 & 85.11 & 84.46 & 84.61 & 83.94  \\
 \hline
 \hline
 \multicolumn{5}{|c|}{Winogrande (FP32 Accuracy = 79.95\%)} & \multicolumn{4}{|c|}{Piqa (FP32 Accuracy = 81.56\%)} \\ 
 \hline
 \hline
 64 & 78.77 & 78.45 & 78.37 & 79.16 & 81.45 & 80.69 & 81.45 & 81.5 \\
 \hline
 32 & 78.45 & 79.01 & 78.69 & 80.66 & 81.56 & 80.58 & 81.18 & 81.34  \\
 \hline
 16 & 79.95 & 79.56 & 79.79 & 79.72 & 81.28 & 81.66 & 81.28 & 80.96  \\
 \hline
\end{tabular}
\caption{\label{tab:mmlu_abalation} Accuracy on LM evaluation harness tasks on Llama2-70B model.}
\end{table}

%\section{MSE Studies}
%\textcolor{red}{TODO}


\subsection{Number Formats and Quantization Method}
\label{subsec:numFormats_quantMethod}
\subsubsection{Integer Format}
An $n$-bit signed integer (INT) is typically represented with a 2s-complement format \citep{yao2022zeroquant,xiao2023smoothquant,dai2021vsq}, where the most significant bit denotes the sign.

\subsubsection{Floating Point Format}
An $n$-bit signed floating point (FP) number $x$ comprises of a 1-bit sign ($x_{\mathrm{sign}}$), $B_m$-bit mantissa ($x_{\mathrm{mant}}$) and $B_e$-bit exponent ($x_{\mathrm{exp}}$) such that $B_m+B_e=n-1$. The associated constant exponent bias ($E_{\mathrm{bias}}$) is computed as $(2^{{B_e}-1}-1)$. We denote this format as $E_{B_e}M_{B_m}$.  

\subsubsection{Quantization Scheme}
\label{subsec:quant_method}
A quantization scheme dictates how a given unquantized tensor is converted to its quantized representation. We consider FP formats for the purpose of illustration. Given an unquantized tensor $\bm{X}$ and an FP format $E_{B_e}M_{B_m}$, we first, we compute the quantization scale factor $s_X$ that maps the maximum absolute value of $\bm{X}$ to the maximum quantization level of the $E_{B_e}M_{B_m}$ format as follows:
\begin{align}
\label{eq:sf}
    s_X = \frac{\mathrm{max}(|\bm{X}|)}{\mathrm{max}(E_{B_e}M_{B_m})}
\end{align}
In the above equation, $|\cdot|$ denotes the absolute value function.

Next, we scale $\bm{X}$ by $s_X$ and quantize it to $\hat{\bm{X}}$ by rounding it to the nearest quantization level of $E_{B_e}M_{B_m}$ as:

\begin{align}
\label{eq:tensor_quant}
    \hat{\bm{X}} = \text{round-to-nearest}\left(\frac{\bm{X}}{s_X}, E_{B_e}M_{B_m}\right)
\end{align}

We perform dynamic max-scaled quantization \citep{wu2020integer}, where the scale factor $s$ for activations is dynamically computed during runtime.

\subsection{Vector Scaled Quantization}
\begin{wrapfigure}{r}{0.35\linewidth}
  \centering
  \includegraphics[width=\linewidth]{sections/figures/vsquant.jpg}
  \caption{\small Vectorwise decomposition for per-vector scaled quantization (VSQ \citep{dai2021vsq}).}
  \label{fig:vsquant}
\end{wrapfigure}
During VSQ \citep{dai2021vsq}, the operand tensors are decomposed into 1D vectors in a hardware friendly manner as shown in Figure \ref{fig:vsquant}. Since the decomposed tensors are used as operands in matrix multiplications during inference, it is beneficial to perform this decomposition along the reduction dimension of the multiplication. The vectorwise quantization is performed similar to tensorwise quantization described in Equations \ref{eq:sf} and \ref{eq:tensor_quant}, where a scale factor $s_v$ is required for each vector $\bm{v}$ that maps the maximum absolute value of that vector to the maximum quantization level. While smaller vector lengths can lead to larger accuracy gains, the associated memory and computational overheads due to the per-vector scale factors increases. To alleviate these overheads, VSQ \citep{dai2021vsq} proposed a second level quantization of the per-vector scale factors to unsigned integers, while MX \citep{rouhani2023shared} quantizes them to integer powers of 2 (denoted as $2^{INT}$).

\subsubsection{MX Format}
The MX format proposed in \citep{rouhani2023microscaling} introduces the concept of sub-block shifting. For every two scalar elements of $b$-bits each, there is a shared exponent bit. The value of this exponent bit is determined through an empirical analysis that targets minimizing quantization MSE. We note that the FP format $E_{1}M_{b}$ is strictly better than MX from an accuracy perspective since it allocates a dedicated exponent bit to each scalar as opposed to sharing it across two scalars. Therefore, we conservatively bound the accuracy of a $b+2$-bit signed MX format with that of a $E_{1}M_{b}$ format in our comparisons. For instance, we use E1M2 format as a proxy for MX4.

\begin{figure}
    \centering
    \includegraphics[width=1\linewidth]{sections//figures/BlockFormats.pdf}
    \caption{\small Comparing LO-BCQ to MX format.}
    \label{fig:block_formats}
\end{figure}

Figure \ref{fig:block_formats} compares our $4$-bit LO-BCQ block format to MX \citep{rouhani2023microscaling}. As shown, both LO-BCQ and MX decompose a given operand tensor into block arrays and each block array into blocks. Similar to MX, we find that per-block quantization ($L_b < L_A$) leads to better accuracy due to increased flexibility. While MX achieves this through per-block $1$-bit micro-scales, we associate a dedicated codebook to each block through a per-block codebook selector. Further, MX quantizes the per-block array scale-factor to E8M0 format without per-tensor scaling. In contrast during LO-BCQ, we find that per-tensor scaling combined with quantization of per-block array scale-factor to E4M3 format results in superior inference accuracy across models. 

\end{document}