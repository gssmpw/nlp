\section{Related Work}
There have been many other works that incorporate symbolic reasoning into deep learning, computer vision, and autonomous vehicle models. Suchan et al. explore commonsense reasoning-based approaches such as an integrated neurosymbolic online abduction vision and semantics-based approach for autonomous driving \cite{suchan2021commonsense, suchan2020driven}. These techniques are primarily focused on integrating with the perception model using answer set programming (ASP), a nonmonotonic reasoning system using stable models \cite{gelfond2014knowledge,lifschitz2019answer}. While their framework is similar, this approach is more decoupled from the vision models, allowing us to show improvements on existing AV models.

Neurosymbolic AI, AIs that integrate symbolic and neural network-based approaches \cite{hitzler2022neuro}s, have been applied towards autonomous driving as well. For safety-critical systems, such as autonomous driving, neurosymbolic techniques can improve compliance with guidelines and safety constraints \cite{sheth2023neurosymbolic}. Anderson et al. propose a neurosymbolic framework that incorporates symbolic policies with a deep reinforcement learning model \cite{anderson2020neurosymbolic}. They assert that this approach can improve the safety of reinforcement learning approaches in safety-critical domains, including autonomous vehicles. These systems are related to this research in the sense that both are using symbolic methods to improve existing deep learning-based systems. However, this research is different in that it is using commonsense reasoning as the proposed symbolic model and that, while it is being used to improve on a deep learning model, it is a different layer that is generated separately. While autonomous vehicles and computer vision technologies are primarily deep learning-based, this approach could be used to improve upon reinforcement learning-based, other non-neural machine learning-based, or even search-based vehicles.

A framework created earlier, AUTO-DISCERN \cite{kothawade2021auto}, proposes a goal-directed commonsense reasoning ASP system that makes driving decisions based on the observations of the environment. This research is an extension of this approach by creating a commonsense reasoning model that makes safe decisions and reasons over a road scenario. This experiment pushes it farther by incorporating the model with an AV system and using the commonsense model to improve aspects of autonomous driving.