\section{Introduction}


%% Experiment Plans
% 1. Compare [bubble-view, Importance Annot, Grid] in terms of variance
% 2. Compare [no definition, my definition] in terms of coverage (guess correct?)
% 3. (Optional) Compare Adaptive Grid in terms of grid config
% 4. Compare multiple approaches in terms of IoU(intersection of Union), SSIM score

Where do people look in visualizations under tasks?
Understanding salient parts of visualizations is crucial for designing compelling visualizations that optimally support analytic tasks~\cite{graphicDesignImportance}. 
However, modeling saliency is challenging as where people visually focus is inherently task-dependent. 
For example, during free viewing, participants may primarily engage in bottom-up processes driven by visually salient elements, such as a bright red patch~\cite{kinchla1979ordervisualprocesse}.
However, when given an analytic task, participants are more likely to engage in top-down processing, directing their attention toward task-relevant visualization regions. For example, in a ``find the extreme'' task, they may focus on the tip of the highest bar~\cite{gilbert2013topdown}.
In this context, task-dependent saliency is strongly related to importance, which involves actively filtering areas with sufficient information for task-solving.
Existing work has defined the image's ``importance'' as regions where individual annotators believe as important~\cite{importAnnot}. 
We supplement this definition by taking a task-dependent approach to propose a new alternate definition for ``task-specific importance'' as ``\textit{the minimum area in a visualization required for a user to complete a task successfully}.'' 
% \will{related to previous comment. now this definition feels like it is interjected in a discussion of your toolkit. Could consider hoisting it to an earlier paragraph}

% \will{I'm not sure what the last sentence means. Actually I think this could be a good place to clarify the similarities and differences bw saliency and importance, as they are core concepts in the paper but may appear confusing.}

However, existing mouse-tracking-based saliency collection methods rely on free-viewing~\cite{turkeyes, importAnnot, bubbleView}, as they are designed to encourage users to explore and describe an image. However, because visualizations are often used for analytic tasks~\cite{amar2005low}, we posit that an effective model for predicting where people look should consider task relevance when identifying regions of importance.

To address this limitation, we contribute \textit{Grid Labeling}a toolkit to enhance existing saliency prediction models with task-specific annotation. 
A key advantage of our tool is that it is more resource-efficient than traditional methods, such as eye-tracking or mouse-tracking~\cite{graphicDesignImportance, importAnnot, bubbleView, turkeyes, salchartQA}.

Grid Labeling segments visualizations into Adaptive Grids that dynamically adjust based on existing graphical elements, making it easily adapted to various visualization sizes and designs. 
This approach enables participants to identify critical areas simply by clicking relevant grids, eliminating the need for cumbersome mouse interactions, such as free-form drawing to annotate regions~\cite{importAnnot} or clicking the same regions multiple times~\cite{bubbleView}.
Moreover, Grid Labeling streamlines data collection, reducing the number of participants required to converge to a stable importance map.
In a human-subject experiment, we demonstrate that, compared to the two popular approaches ImportAnnots~\cite{importAnnot} and BubbleView~\cite{bubbleView}, Grid Labeling produces less noisy data with higher levels of agreement between participant responses.
Additionally, participants reported lower perceived effort when using our method. We also explore the key distinction between saliency and importance, contributing to differences in annotation duration.
% \andreas{no need to bold all/any instances of "Grid Labeling". if you want to highlight the first occurrence, you can make that italic.}

The specific contributions of our work are three-fold:
% \will{I recommend keeping each contribution in a bullet point on its own line. This is somewhere reviewers will likely come back to multiple times, so it's probably worth making space at other places.}
%\vspace{2mm}
%\noindent \textbf{Contributions:} 
\begin{itemize}
    \item We introduce the Grid Labeling method for capturing ``task-specific importance'' in visualizations, which enhances task-specific saliency modeling.

    \item Through a human-subject study, we illustrate the importance of considering task-specific ``areas of importance'' in visualizations.

    \item We quantitatively demonstrate that our Grid Labeling method outperforms traditional crowd-sourcing methods for collecting task-specific importance data in visualizations. Participants in our study could identify important areas with less effort and with higher inter-participant agreement.
\end{itemize}

