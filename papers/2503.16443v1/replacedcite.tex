\section{Related Works}
Research on pedestrian-distracted walking safety typically focuses on several themes, including types of distractions, simulation methods, safety measures, and counteractive interventions. A summary of the literature along these lines is presented below, along with a critical evaluation of each study’s contributions and limitations, highlighting gaps this study seeks to address.

\subsection{Distraction Types and Behavior Analysis}
Pedestrian behavior can vary based on several factors, including Age, Gender, Group size, and other environmental factors. In the study, ____ examined the pedestrian safety behavior at urban signalized intersections using Logistic regression the author highlighted factors influencing risky behaviors like crossing on red lights, emphasizing the role of intersection design, pedestrian characteristics, and wait times in safety outcomes. Using the observational data collection method from 24 intersections (12 per city) in Montreal and Quebec City the study found that the most common distractions include using headphones and walking with companions, while significant head movement differences were observed between cities. 
____ employed Ordinary Least Squares (OLS) regression to model pedestrian walking speeds and binary logit models to analyze pedestrian distraction and violations to investigate pedestrian behaviors at signalized crosswalks, focusing on distracted walking, pedestrian signal violations, and walking speeds. Utilizing field-recorded observational data from one intersection in New York and three in Flagstaff, Arizona, the researchers analyzed data from 3,038 pedestrian crossings through video-based observation and found that specific site and demographic variables significantly influenced pedestrian distraction, signal violations, and walking speeds. On the other hand, ____ investigated the influence of social and technological distractions on pedestrian crossing behavior, focusing on cautionary actions and crossing durations. Using a prospective observational approach, data were collected at 20 high-risk intersections in Seattle, recording demographic, behavioral, and distraction data of 1102 pedestrians. Key data collected were distraction types such as texting, talking, or listening to music. Results showed that nearly 30\% of pedestrians engaged in distractions while crossing, with text messaging notably increasing crossing times and the likelihood of risky behaviors, like failing to obey traffic signals.
 Furthermore, ____ observed pedestrian behaviors in real environments, focusing on socio-demographic influences like age and gender on waiting times before crossing. While this study provided real-world insights, its observational nature could not repeatedly simulate high-risk scenarios for detailed analysis. ____ examined college students crossing a virtual street while engaging in distracting activities (e.g., texting, talking, listening to music) and found that distracted participants were more likely to be "hit" in the virtual simulation. 

\subsection{Methodological Approaches: Simulation Environments}
A common approach for studying distracted walking is using simulated environments to observe pedestrian responses under various scenarios.  ____ used immersive virtual environments to assess children and college students crossing streets while on cell phones, highlighting significant declines in attention and riskier behaviors. However, while immersive, these studies did not examine multiple types of distractions or compare different safety measures within the same simulation, limiting their applicability to broader distracted pedestrian contexts. Virtual Immersive Reality Environments (VIREs) are emerging as tools for testing diverse distraction scenarios, allowing researchers to simulate complex, controlled environments safely ____. 
____ employed a descriptive statistical and ANOVA analysis methods in assessing crossing behaviors under varying traffic conditions (easy vs. hard) for participants characterized by high or low sensation-seeking tendencies. The study, used virtual reality (VR) environment, focusing on children (10-13 years), adolescents (14-18 years), and young adults (20-24 years), Data collection included 209 participants who completed virtual crossings under both traffic scenarios. Key findings revealed that adolescents, particularly those high in sensation-seeking, exhibited the most dangerous crossings. Age and sensation-seeking were significant predictors of risky pedestrian behavior, with traffic conditions significantly influencing crossing behaviors across groups.
Furthermore, ____ evaluated the efficacy of using virtual reality (VR) technology for pedestrian safety research by developing a pedestrian simulator using the HTC VIVE headset and Unity software. The study involved, tracking participants' head movements as they navigated a virtual intersection with varying traffic scenarios, measuring their responses, and gathering subjective data on user experience and simulation sickness. Data from 21 participants, excluding four due to simulator sickness, were analyzed for crossing behaviors and collision rates under different traffic conditions. Key findings showed that VR environments replicated realistic pedestrian behaviors, with participants adjusting their speed and actions based on traffic conditions, confirming VR's potential as a valid tool for studying pedestrian safety behaviors in controlled yet realistic settings.

\subsection{Safety Performance Metrics and Surrogate Measures}
Many studies evaluate pedestrian safety through measures like Time to Collision (TTC), Post-Encroachment Time (PET), and maximum acceleration or deceleration. ____ explored the effects of smartphone use on pedestrian wait times and decision-making, finding that distracted individuals tend to delay crossing, potentially affecting traffic flow. ____ and ____ utilized conflict indicators like Time to Collision (TTC) and Post-Encroachment Time (PET) to assess pedestrian safety at signalized intersections, using observational data collection method, Support Vector Machine (SVM) was used in the analysis of the data, and the key findings were that signalization generally reduces pedestrian risk, with greater efficacy at high-speed intersections and with limited turning traffic, while factors such as crosswalk proximity, refuge presence, and red-light behavior influenced safety outcomes and that conflict indicators are good for better accuracy in severity estimation, suggesting improvements in pedestrian safety analysis and targeted traffic behavior education.
However, these studies approach lacked the ability to simulate real-time pedestrian-vehicle interactions, limiting its relevance for high-stakes scenarios. The research suggests a need for more robust experimental designs to validate these measures in diverse conditions.

\subsection{Counteractive Measures and Intervention Effectiveness}
Countermeasures such as LED flashlights and crossing reminders have been proposed to mitigate distracted walking risks. One study ____ tested LED crosswalk lights as a safety measure, noting a slight increase in crossing success rates but no significant reduction in overall risk, as the lights may have contributed to distraction. ____ studied distracted walking behaviors at high-risk intersections in Manhattan, observing that nearly 30\% of pedestrians were distracted by mobile devices regardless of traffic signals. While the findings highlighted the prevalence of distraction, the study’s observational nature limited the exploration of practical interventions to reduce it.


The aforementioned literature indicates that distractions significantly reduce pedestrian safety, yet effective and scalable interventions remain limited. Few studies have examined countermeasures in controlled virtual environments. To address these gaps, this study employs a Virtual Immersive Reality to simulate pedestrian behavior across three distinct scenarios: Undistracted crossing, crossing while using a handheld device, and crossing with an implemented safety measure. These simulations provide a versatile framework for developing and evaluating interventions, offering practical insights for real-world applications.