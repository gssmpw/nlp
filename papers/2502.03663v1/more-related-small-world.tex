\section{Related Prior Work} \label{sec:related-work}

Prior to Milgram's experiment, Pool and Kochen \cite{de1978contacts} as well as others \cite{kochensmallworld} found that if social connections were selected uniformly from the population, then the diameter of the graph is small with high probability. 
While interesting, these early results did not provide a compelling reason for the small-world phenomenon. 
In 1998, Watts and Strogatz suggested that individuals should have both ``local'' connections to adjacent nodes as well as ``long-range'' contacts like the ones studied prior \cite{watts1998collective}. 
This captured a sense of locality to social networks, and was able to model several natural and manmade phenomena. 
Kleinberg \cite{kleinberg2000small} generalized and improved their model, altering the criteria for choosing long-range contacts such that the graph enabled greedy routing between any two nodes to succeed with high probability in $\mathcal{O}(\log^2 n)$ time. 
Martel and Nguyen \cite{martel} later found that greedy routing in Kleinberg's model runs in $\Theta(\log^2 n)$ time yet the diameter of the graph is $\Theta(\log n)$, meaning that Kleinberg's algorithm fails to find the true shortest paths available in the graph by a $\log n$ factor. 
Other works have implemented Kleinberg's analyses in other topologies and types of graphs \cite{fraigniaudgreedy,kleinbergdynamics,barriereefficient,duchon2006could}. 

Another small-world model is the \emph{preferential attachment} model of Barabási and Albert \cite{barabasipreferential}. 
Here, nodes are added to the graph one at a time.
New nodes connect to randomly selected other nodes, with probability proportional to the other nodes' degree (a ``rich-get-richer'' process). Dommers, Hofstad, and Hooghiemstra \cite{dommers2010diameters} showed that the diameter of this model is low, yet unlike Kleinberg's model, no greedy algorithm exists to find such short paths between nodes. 
Since then, there have been efforts to combine the two models in order to get the best of both worlds.
Bringmann, Keusch, Lengler, Maus, and Molla propose such a model with an average $\mathcal{O}(\log\log n)$ greedy routing time, yet their model requires nodes of the network to be randomly distributed in some geometric space. 
Goodrich and Ozel take another approach, inspired by the U.S. highway system (which is ultimately what carried the messages shared in Milgram's experiment), called the \emph{neighborhood preferential attachment} model. 
Nodes are again added to the graph one at a time, but the probability of connecting to some other node $v$ is based both on the degree of $v$ as well as the distance between the new node and $v$. 
While their intuition was corroborated by their strong empiricial results as well as by Abraham, Fiat, Goldberg, and Werneck's work on the \emph{highway dimension} of graphs \cite{abrahamhighway}, Goodrich and Ozel were not able to prove any theoretical bounds for their model \cite{goodrich2022modeling}. 
In 2023, Gila, Ozel, and Goodrich \cite{gila2023highway} adapted the model of Goodrich and Ozel back to a lattice, their \emph{randomized highway} model, and were able to show that it expects to perform greedy routing in $\mathcal{O}(\log n)$ time. 