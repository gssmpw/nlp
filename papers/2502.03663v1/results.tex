\section{Our Results in the Fixed-Growth Model}

In this section, we present several results for highway graphs built from more
general underlying graphs.
As discovered by \cite{gila2023highway}, we obtain optimal results when $k =
\Theta(\log n)$, and use this as the canonical value for our key results.
% We also provide supplemental results for general values of $k$.
% For simplicity, we are very loose with our constants, using big-$\mathcal{O}$
% notation liberally.

\subsection{Preliminary Results}

In this section we go over results which are essential for 
our bounds on greedy routing and diameter.

\subsubsection{Balls.}
%
One of the key difficulties overcome in the original randomized Kleinberg
highway model was dealing with dependent probabilities.
% \vinesh{what is $t$ here? is it any node or specifically the destination of somethign?}
For example, if there are only a few highway nodes close to the destination $t$, the probability that
a highway node along the greedy routing path has a long-range contact near $t$ will
be smaller for every highway node visited.
Gila, Ozel, and Goodrich solved this in two key ways---assuming all edges are
directed\footnote{Using undirected edges would be a strict improvement.}, and by
subdividing the graph into separate, disjoint diamonds (balls) \cite{gila2023highway}.
Each ball is large enough such that the number of highway nodes it contains can
be bounded with high probability.
Here we generalize these results for $\alpha$-dimensional fixed-growth graphs:
%
\begin{lemma} \label{lem:balls}
    Results for any randomized highway graph $\mathcal{G}$ with FG dimensionality $\alpha$ and highway constant $k$:
    \begin{enumerate}
        \item There exist balls of radius $\Omega(\sqrt[\alpha]{k \log{n}}) \leq
        \ell \leq \sqrt[\alpha]{n}$, centered around any of the $n$ nodes,
        containing $\Theta(\ell^\alpha / k)$ highway nodes with high probability in
        $n$. \label{lem:balls:always}

        % Commented out cause redundant, but still perhaps nice
        % \item All balls of radius $\Theta(\sqrt[\alpha]{k \log{n}})$, centered around any
        % of the $n$ nodes in the graph, contain $\Theta(\log{n})$ highway nodes
        % w.h.p. in $n$.

        \item With high probability in
        $\log{n}$, the balls of radius $\Theta(\sqrt[\alpha]{k
        \log{\log{n}}})$ centered around any $\Theta(\log^2{n})$ nodes each contain
        $\Theta(\log{\log{n}})$ highway nodes. \label{lem:balls:log}
        
        \item Any arbitrary ball of radius $\Theta(\sqrt[\alpha]{k})$ contains
        no highway nodes with constant probability.
        This result is not a high probability bound, and is only independent for
        balls centered around nodes with distance at least $c \sqrt[\alpha]{k}$,
        for some constant $c$ hidden in the big-$\mathcal{O}$ notation.
        \label{lem:balls:constant}
    \end{enumerate}
\end{lemma}

\begin{proof}
    Since $\mathcal{G}$ is an $\alpha$-dimensional fixed-growth graph, we know
    that there are $\Theta(\ell^\alpha)$ nodes in any ball of radius $\ell \in
    \mathcal{O}(\sqrt[\alpha]{n})$.
    There exists some constant $c_1$ such that, for any node $u$ in the graph,
    $|B_\ell(u)| \geq c_1 \ell^\alpha$.
    Letting $\ell = c_2 \sqrt[\alpha]{k \log{n}}$ for some constant $c_2$, there
    are at least $c_1 c_2^\alpha k \log{n}$ nodes in $B_\ell(u)$.
    Because the probability of any node being a highway node is $1/k$, the
    expected number of highway nodes in $B_\ell(u)$ is at least $c_1
    c_2^\alpha \log{n}$.
    Letting $X$ be the number of highway nodes in $B_\ell(u)$, we can apply the
    simplified Chernoff bounds of~\cite{DILLENCOURT2023106397} to bound the number
    of highway nodes in the ball:
    % 
    \begin{equation*}
        \Pr(X < (1 - \delta) \mu) < 2^{-\mu} \leq 2^{-c_1 c_2^\alpha \log{n}} = n^{-c_1 c_2^\alpha}
    \end{equation*}
    Note that this holds for $\delta > 0.91$.
    Applying a union bound over all $n$ nodes in the graph, we can bound the
    probability that any such ball contains fewer nodes by $n^{1 - c_1
    c_2^\alpha}$.
    The constant factor in the ball radius, $c_2$, can be chosen to be large
    enough to make this a high probability bound.
    For an upper bound, we apply the same reasoning, assuming that $|B_\ell(u)|
    \le c_3 \ell^\alpha$ for some constant $c_3 > c_1$.
    For our aforementioned balls, we expect to have at most $c_3 c_2^\alpha
    \log{n}$ highway nodes.
    Using a similar Chernoff bound from~\cite{DILLENCOURT2023106397}, for
    $\delta > 2.21$, we have:
    %
    \begin{equation*}
        \Pr(X > (1 + \delta) \mu) < 2^{-\delta \mu} \leq 2^{-\delta c_3 c_2^\alpha \log{n}} = n^{-\delta c_3 c_2^\alpha}
    \end{equation*}

    In this case, the probability any such ball contains more nodes can be
    bounded by $n^{1 - \delta c_3 c_2^\alpha}$.
    $\delta$ can now be chosen to be sufficiently large such that this is a high
    probability bound.
    Combining the lower and upper bounds, we see that balls of radius $\ell =
    c_2 \sqrt[\alpha]{k \log{n}}$ contain $\Theta(\log{n}) = \Theta(\ell^\alpha
    / k)$ highway nodes with high probability in $n$.
    Since this holds for some $\ell$ in $\Theta(\sqrt[\alpha]{k \log{n}})$, it
    trivially holds for all larger balls 
    (observe that increasing $\ell$ would only make the exponent 
    derived from the Chernoff bounds smaller).
    This proves the first part of the lemma.
    The remaining parts of the lemma can be proven similarly.
    \qed
\end{proof}

\vspace*{-\bigskipamount}

\subsubsection{Shells.}

In the original Kleinberg model and its variants, the fact that the underlying
graph $\mathcal{L}$ was a perfect lattice was heavily utilized in the analyses 
\cite{kleinberg2000small,martel,gila2023highway}.
Unfortunately, we are unable to make such strict assumptions about the
underlying geometry of our fixed-growth graph $\mathcal{F}$.
One of the places this assumption breaks down is in the ability to neatly tile
independent balls the graph as is done in the nested lattice construction of
\cite{gila2023highway}.
In the fixed-growth setting, we have have no guarantees that, 
if we were to do this, no two balls would overlap.
This would violate their independence, making us unable to analyze how long-range
contacts are created.
We instead utilize the concept of a \emph{shell} when handling fixed-growth graphs.
Let $\mathcal{S}_b^{(w)}(u) = \mathcal{B}_{(b + 1)w_{b + 1}}(u) - \mathcal{B}_{bw_b}(u)$
be the shell at shell-distance $b$ and shell-width $w$ around node $u$.
Notice that this system of shells partitions the plane, 
even in the fixed-growth setting.
See \Cref{fig:balls-and-shells}.

\begin{lemma} \label{lem:shells}
    There exists a sequence of shell widths $w_b \in \Theta(\sqrt[\alpha]{k
    \log{n}})$, such that the number of highway nodes in a shell at
    shell-distance $b$ from any node is $\Theta(b^{\alpha - 1} \log{n})$ w.h.p.
    for any $1 \leq b \leq \Theta(\sqrt[\alpha]{n / (k \log{n})})$.
    For simplicity, we refer to any shell width $w_i$ as $w \in
    \Theta(\sqrt[\alpha]{k \log{n}})$, affecting at most constant factors.
\end{lemma}

\begin{proof}
    The number of nodes in this shell $|\mathcal{S}_b^{(w)}(u)|$ is the difference
    in the number of nodes between balls of radii $(b + 1)w_{b + 1}$ and $bw_b$, i.e.,
    $\Theta(((b + 1)w_{b + 1})^\alpha) - \Theta((bw_b)^\alpha)$.
    All $w_i$ are chosen from $\Theta(\sqrt[\alpha]{k \log{n}})$ such that
    the constants in this theta notation are the same, making the difference
    $\Theta(((b + 1)^\alpha - b^\alpha)w^\alpha)$.
    Through a simple polynomial expansion, we see that the total number of nodes in
    this ball is $\Theta(b^{\alpha - 1} w^\alpha)$ for any $b \ge 1$.
    Since each node has a $1/k$ chance of being a highway node, the expected number
    of highway nodes in this shell is $\Theta(b^{\alpha - 1} w^\alpha / k)$.
    As in the proof of \Cref{lem:balls}, in order to obtain a high probability
    bound over all $n$ nodes, we require at least $\Theta(\log{n})$ highway
    nodes in each shell.
    This is achieved by choosing $w = c_2 \sqrt[\alpha]{k \log{n}}$ for a
    sufficiently large constant $c_2$.
    \qed
\end{proof}

\subsubsection{Normalization Constant.}
%
A fundamental piece of information we need to know about any given
highway node $u$ is the probability that it is connected to another specific
highway node $v$.
In our fixed-growth model, in a graph $\mathcal{G}$ with dimensionality
$\alpha$, the probability is inversely proportional to the distance to the
power of $\alpha$, i.e., $\Pr(u \to v) \propto d(u, v)^{-\alpha}$.
To get the exact probability, we need to normalize this constant, i.e.,
$\Pr(u \to v) = d(u, v)^{-\alpha} / \sum_{h \in \mathcal H}(d(u, h)^{-\alpha})$, where
$\mathcal H$ is the subgraph of highway nodes in $\mathcal G$.
That denominator is the normalization constant, $z$. 
To lower bound
the probability, we must upper bound this constant.

\begin{lemma} \label{lem:z}
    Results for any randomized highway graph $\mathcal{G}$ with fixed-growth
    dimensionality $\alpha$ and highway constant $k$:
    %
    \begin{enumerate}
        \item The normalization constant $z(u)$ is
        $\mathcal{O}(\frac{\log{n}}{k} + \log{\log{n}})$ for any highway node
        $u$ with high probability in $n$. \label{lem:z:always}

        \item The normalization constant $z(u)$ is
        $\mathcal{O}(\frac{\log{n}}{k} + \log{\log{\log{n}}})$ for
        $\mathcal{O}(\log^2{n})$ arbitrary highway nodes w.h.p. in $\log{n}$. \label{lem:z:log}

        \item The normalization constant $z(u)$ is
        $\mathcal{O}(\frac{\log{n}}{k})$ for an arbitrary highway node $u$
        with constant probability. \label{lem:z:constant}

        \item The normalization constant $z(u)$ is $\Omega(\frac{\log{n}}{k})$
        for any highway node $u$ with high probability in $n$.
        \label{lem:z:lower-always}
    \end{enumerate}
\end{lemma}

\begin{proof}
    Previous papers~\cite{kleinberg2000small,martel,gila2023highway} made
    extensive use of the rigid underlying geometry of the lattice to bound the
    normalization constant, specifically to determine the number of nodes at any
    distance from a given node.
    In our fixed-growth model, since we cannot make such strict assumptions, we
    instead sum over various shells with specific widths at different radii,
    using the results of \Cref{lem:shells}.
    First, let's consider the contribution to the normalization constant of an
    arbitrary highway node $u$ from all highway nodes at shell distance $b \geq
    1$, $z(u)_{b \geq 1}$. 
    Intuitively, we get the largest value of $z(u)_{b \geq 1}$ when all highway nodes
    in each shell are clustered as close to $u$ as they can be.
    %
    \begin{align*}
        z(u)_{b \ge 1} &< \sum_{b = 1}^m{\frac{\text{max \# highway nodes in } \mathcal{S}_b^{(w)}(u)}{(\text{min distance to node in } \mathcal{S}_b^{(w)}(u))^\alpha}} \\
        &= \sum_{b = 1}^m{\frac{\Theta(b^{\alpha - 1} \log{n})}{(bw)^\alpha}} = \sum_{b = 1}^m{\frac{\Theta(b^{\alpha - 1} \log{n})}{b^\alpha k \log{n}}} \\
        &= \Theta\pars*{\frac{1}{k}} \sum_{b = 1}^m{\frac{1}{b}} = \Theta\pars*{\frac{\log{n}}{k}}
    \end{align*}
    %
    This contribution is added to all the normalization constants w.h.p.
    A small adjustment to the above proofs provides results for the lower bound
    in part 4 of this lemma.
    Now let us consider the contribution of highway nodes within distance $w$ of
    $u$.
    From \Crefpart{lem:balls}{lem:balls:always} we know that there are at most
    $\Theta(\log{n})$ highway nodes within distance $w$ of $u$.
    By applying a derivative and as shown in~\cite{stefanExpansion}, there can be at
    most $\Theta(j^{\alpha - 1})$ nodes at distance $j$ from $u$.
    For our worst case analysis, we assume that all $\Theta(\log{n})$ nodes are
    packed tightly around $u$:
    %
    \begin{equation*}
        z(u)_{b = 0} \leq \sum_{j = 1}^{\Theta(\sqrt[\alpha]{\log{n}})}{\frac{\Theta(j^{\alpha - 1})}{j^\alpha}} = \Theta(\log{\log{n}})
    \end{equation*}
    %
    This concludes the proof for the first part of the lemma.
    For a slightly tighter bound, we limit how many nodes are packed around $u$
    using \Crefpart{lem:balls}{lem:balls:log}, where we learned that there are
    only $\Theta(\log{\log{n}})$ highway nodes within some inner distance
    $\Theta(\sqrt[\alpha]{k \log{\log{n}}})$ of $u$:
    %
    \begin{equation*}
        z(u)_{b = 0, \text{ inner}} \leq \sum_{j = 1}^{\Theta(\log{\log{n}})}{\frac{\Theta(j^{\alpha - 1})}{j^\alpha}} = \Theta(\log{\log{\log{n}}})
    \end{equation*}
    %
    Recall that we still have $\Theta(\log{n})$ unaccounted for highway nodes
    within distance $w$ of $u$.
    Again assuming the worst case possible, they are all at at the edge of the
    inner ball:
    %
    \begin{equation*}
        z(u)_{b = 0, \text{ outer}} < \frac{\Theta(\log{n})}{\Theta(\sqrt[\alpha]{k \log{\log{n}}})^\alpha} = \Theta\pars*{\frac{\log{n}}{k \log{\log{n}}}}
    \end{equation*}
    %
    This concludes the proof for the second part of the lemma.
    Finally, we prove the third part of this lemma where there are no highway
    nodes within some distance $\Theta(\sqrt[\alpha]{k})$ of $u$ which we know
    occurs with constant probability from
    \Crefpart{lem:balls}{lem:balls:constant}.
    Assuming all $\log{n}$ highway are all at distance
    $\Theta(\sqrt[\alpha]{k})$ from $u$, their total contribution would be at
    most $\Theta(\log{n} / k)$.
    \qed
\end{proof}

\subsubsection{Distance to the Highway.}

An important step of both the greedy routing algorithm and the diameter in
general is the ability to reach the highway.
From \Crefpart{lem:balls}{lem:balls:always}, we know that the maximum distance
between any arbitrary node and the nearest highway node is at most
$\mathcal{O}(\sqrt[\alpha]{k \log{n}})$.
In this section, we prove that this is a tight bound.
%
\begin{lemma} \label{lem:distance-to-highway}
    In any randomized highway graph $\mathcal{G}$ of FG dimensionality $\alpha$ with $k > 1 + \epsilon$, the maximum
    distance between any node and the nearest highway node is
    $\Theta(\sqrt[\alpha]{k \log{n}})$ w.h.p.
\end{lemma}
%
\begin{proof}
    To prove this tight bound, we will prove that for any ball of radius $\ell \in
    \mathcal{O}(\sqrt[\alpha]{k \log{n}})$, for sufficiently large $n$, there will exist
    some ball of radius $\ell$ that contains no highway nodes.
    The probability of any node being a highway node is $1/k$, and the
    probability it is not a highway node is $1 - 1/k$.
    Due to the definition of the fixed-growth model, there exists some constant
    $c_1$ such that for any node $u$ in the graph, $|B_\ell(u)| \leq c_1
    \ell^\alpha$.
    For any ball of radius $\ell$, the probability that it contains no highway
    nodes is at least $(1 - 1/k)^{c_1 \ell^\alpha}$.
    Due to the fact that for any $x < 1$, $1 - x \geq e^{-\frac{x}{1 - x}}$,
    this probability is at least $e^{-\frac{c_1 \ell^\alpha}{k - 1}}$, and
    consequently the probability that there is at least one highway node in the
    ball is at most $1 - e^{-\frac{c_1 \ell^\alpha}{k - 1}}$.
    Since $1 - x \leq e^{-x}$, this probability is at most $e^{-e^{-\frac{c_1
    \ell^\alpha}{k - 1}}}$.
    From \Cref{lem:indep-balls}, we know that there are $\Theta(n /
    \ell^\alpha)$ independent balls of radius $\ell$, which is at least $c_2 n /
    \ell^\alpha$ for some constant $c_2$.
    The probability that all balls contain at least one highway node is
    at most $1 - e^{-c_2 \frac{n}{\ell^\alpha} e^{-\frac{c_1
    \ell^\alpha}{k - 1}}}$ by a union bound.

    To show that at least one of these balls will be empty, we must show that
    this probability will be 0 for sufficiently large $n$, i.e., that
    $\lim_{n \to \infty} e^{-c_2 \frac{n}{\ell^\alpha} e^{-\frac{c_1
    \ell^\alpha}{k - 1}}} = 0$.
    We can see that this limit is 0 iff $\lim_{n \to \infty} c_2
    \frac{n}{\ell^\alpha} e^{-\frac{c_1 \ell^\alpha}{k - 1}} = \infty$.
    Taking the logarithm of both sides, we see that this is equivalent to
    $\lim_{n \to \infty} \log{c_2} + \log{n} - \alpha \log{\ell} - \frac{c_1'
    \ell^\alpha}{k - 1} = \infty$, where $c_1' = \frac{c_1}{\ln{2}}$.
    This limit showing that at least one ball will be empty will be satisfied if
    $\log{\ell} + \frac{c_1' \ell^\alpha}{k - 1} = \mathcal{O}(\log{n})$.
    Now let us finally assume that $\ell \in \mathcal{O}(\sqrt[\alpha]{k \log{n}})$.
    In this case, $\log{\ell} = \mathcal{O}(\log(k \log{n})) = \mathcal{O}(\log{k} +
    \log{\log{n}})$.
    This is $\mathcal{O}(\log{n})$ for any $k \in \mathcal{O}(n)$, which is indeed any
    reasonable value of $k$.
    For the second clause, we have that $\frac{c_1' \ell^\alpha}{k - 1} =
    c_1' \frac{k}{k - 1} \mathcal{O}(\log{n})$, which is $\mathcal{O}(\log{n})$ for any $k > 1 +
    \epsilon$.
    In other words, for sufficiently large $n$, there will exist at least one
    ball of radius $\ell \in \mathcal{O}(\sqrt[\alpha]{k \log{n}})$ that contains no
    highway nodes, so the maximum distance to a highway node must be at least
    $\Omega(\sqrt[\alpha]{k \log{n}})$.
    \qed
\end{proof}

\subsection{Greedy Routing in Expectation}
%
The key result of the small-world models is the ability to effectively
route between any two nodes of the graph using only local information.
In the original Kleinberg model the greedy routing time was shown to be
$\mathcal{O}(\log^2{n})$~\cite{kleinberg2000small}, a result later
proven tight~\cite{martel}.
Later papers have shown that highway variants achieve better results, either
empirically~\cite{goodrich2022modeling} or theoretically~\cite{gila2023highway}.
In the latter, the authors prove that greedy routing time can be reduced to 
$\mathcal{O}(\log{n})$ for two dimensional lattices by picking $k \in \Theta(\log{n})$.
In this section we generalize these results to fixed-growth graphs that are not
restricted to lattices, over any dimensionality $\alpha \ge 1$ including
non-integral values, and prove that these bounds are tight.
%
\begin{theorem} \label{thm:greedy-routing-exp} In any fixed-growth graph
    $\mathcal{G}$ with FG dimensionality $\alpha$ and highway constant $k \in
    \Theta(\log{n})$, greedy routing between two arbitrary nodes $s$ and $t$ succeeds in $\Theta(\log{n})$ hops with constant probability, if $d(s, t) =
    \Theta(\sqrt[\alpha]{n})$.
\end{theorem}

% There are also two slight variants of $\mathcal{G}$, one where each non-highway
% node has no knowledge of the highway network, and another where each non-highway
% node stores the location of the nearest highway node.
% We prove tight bounds for both of these variants and in two parts---the upper
% and lower bounds---using three steps:
\begin{proof}
    We prove tight bounds for greedy routing considering three steps:
    %
    \begin{enumerate}
        \item Reaching the highway using local contacts. \label{greedy-routing:step-1}

        \item Navigating along the highway using long-range contacts. \label{greedy-routing:step-2}
        
        \item Reaching the destination from the highway using local contacts. \label{greedy-routing:step-3}
    \end{enumerate}
    %
    \crefname{enumi}{step}{steps}
    \Crefname{enumi}{Step}{Steps}
    %
    % For the upper bound, let us assume that non-highway nodes have no knowledge of
    % the highway network.
    For \cref{greedy-routing:step-1}, by greedily walking using local contacts
    towards $t$ and because $k \in \Theta(\log n)$, we expect to 
    reach the highway in $\Theta(k)$ hops.
    Note that without any knowledge of the highway network, there is always a
    constant probability that we are unable to reach the highway in only $\Theta(k)$
    hops, serving as a trivial lower bound for this form of greedy routing.

    Once we reach the highway, we use long-range contacts to navigate the highway
    towards $t$ in \cref{greedy-routing:step-2}.
    From \Cref{lem:prob-halving}, we know that the probability that a long-range
    contact of a highway node $u$ improves its distance to $t$ by a
    factor of $c$ is at least proportional to $[(c + 1)^\alpha z(u)]^{-1}$ for some
    large enough constant $c$.
    We can improve our distance by a factor of $c$ at most $\log_c{d(s, t)} =
    \mathcal{O}(\log_c{\sqrt[\alpha]{n}}) = \mathcal{O}(\log{n})$ times
    before reaching $t$.
    If we assume that we are always able to use our tightest normalization
    constant bound from \Crefpart{lem:z}{lem:z:constant}, we expect $z(u)$ and
    consequently the probability of a factor $c$ improvement to be at most
    constant, and consequently expect to make $\mathcal{O}(\log{n})$ hops along
    the highway.
    This assumption is unreasonable, however, since it only holds with constant
    probability.
    Instead, we show in \Cref{lem:closer-contacts} that even when there is no factor of $c$
    improvement, we expect there to be long-range contacts with
    improvements in distance that allow us to visit enough fresh nodes to reach
    distance $\mathcal{O}(\sqrt[\alpha]{k})$ from $t$.
    Doing so incurs only a constant factor increase in the number of hops.
    Note that it is important to find a \textit{long-range contact} which
    improves our distance, as this guarantees that we are able to remain on the
    highway.
    Finally, we trivially complete \cref{greedy-routing:step-3} in
    $\mathcal{O}(\sqrt[\alpha]{k})$ time.
    \qed
\end{proof}

\subsection{Greedy Routing with High Probability}
%
In this section
, we go one step beyond several of the previous
analyses~\cite{kleinberg2000small,martel,gila2023highway} and 
prove high
probability bounds for greedy routing in fixed-growth graphs.
In order to do this, we make a small adjustment to the model, adding a constant
number of information to each non-highway node.
Specifically, each non-highway node knows which of its neighbors is the closest
to the highway, allowing it to reach the highway using the fewest number of
hops.
%
\begin{theorem} \label{thm:greedy-routing-hp}
    Let $\mathcal{G}$ be a randomized highway graph with FG dimensionality
    $\alpha$ and highway constant $k \in \Theta(\log{n})$.
    If $d(s, t) =
    \Theta(\sqrt[\alpha]{n})$, then
    greedy routing between any two nodes $s$ and $t$ succeeds in $\Theta(\log{n})$ hops with high
    probability in $\log{n}$ if $\alpha \geq 2$, and in
    $\Theta(\sqrt[\alpha]{\log^2{n}})$ hops if $\alpha \leq 2$.
    % Furthermore, 
\end{theorem}

\begin{proof}
    \crefname{enumi}{step}{steps}
    \Crefname{enumi}{Step}{Steps}
    %
    The upper bound of this proof is very similar to the proof in expectation,
    with two main differences.
    First, in order to reach the highway for \cref{greedy-routing:step-1}, we
    make use of the information stored at each non-highway node, and due to the
    results of \Cref{lem:distance-to-highway}, we know that we can reach the
    highway in $\mathcal{O}(\sqrt[\alpha]{k \log{n}})$ hops with high
    probability.
    Second, while navigating the highway in \cref{greedy-routing:step-2}, we use
    the high probability result of \Cref{lem:closer-contacts} reaching only
    within distance $\mathcal{O}(\sqrt[\alpha]{k \log{n}})$ of $t$ rather than
    within distance $\mathcal{O}(\sqrt[\alpha]{k})$ as used in the expectation
    proof.

    For $\alpha \leq 2$, the $\Theta(\sqrt[\alpha]{k \log{n}})$ cost to reach
    the highway is $\Omega(\log{n})$ and so the bound is trivially tight.
    For larger values of $\alpha$, however, we must show that the time spent
    navigating the highway is greater than the time spent reaching the highway,
    i.e., that the time spent navigating the highway is $\Omega(\log{n})$.

    Consider a sequence of $f$ greedy hops along the highway, where $d_0$ and $d_f$
    are the distances from the first (resp. last) highway nodes on the path to
    $t$.
    % As such, $f$ is the number of hops
    % it took to route $s$ to $t$.
    In the best case, there always exists a long-range contact which improves
    our distance until we reach $d_f = \Theta(\sqrt[\alpha]{k \log{n}})$.
    Each hop improves our distance to $t$ by some factor.
    Let $x_i$ refer to the factor of improvement for the $i$-th hop (in greedy routing, it is always true that $x_i > 1$). Working
    backwards, we have that $d_f \times x_f \times x_{f - 1} \times \cdots
    \times x_0 = d_0$.
    Rearranging, we have that $\prod_i{x_i} = d_0 / d_f$.
    Taking the log of both sides, we have that $\sum_i{\log{x_i}} =
    \log{d_0} - \log{d_f}$.
    Since the first highway node is within distance $\mathcal{O}(\sqrt[\alpha]{k
    \log{n}})$ of $s$ and $d(s, t) = \Theta(\sqrt[\alpha]{n})$, then $\log{d_0}
    - \log{d_f} = \Theta(\log{n})$, and consequently $\sum_i{\log{x_i}} =
    \Theta(\log{n})$.
    If we can upper bound the expectation of $\log{x}$, $\mathbb{E}[\log{x}]$,
    then by linearity of expectation, we have that
    $\mathbb{E}[\sum_i{\log{x_i}}] = f \mathbb{E}[\log{x}]$.
    What is left to show is that $\mathbb{E}[\log{x}] = \mathcal{O}(1)$, such
    that the expected number of hops $\mu$ on the highway is $\Omega(\log{n})$.
    It is then possible to use a Chernoff bound to show that the probability of
    the number of hops being significantly less than $\mu$ is small.

    Now fix some constant $c_0 \geq 2$. 
    Due to the results of \Cref{cor:prob-halving-upper}, we know that the
    probability that some $x \geq c$ is at most proportional to $(c -
    1)^{-\alpha}$ for $c \geq c_0$ for some constant $c_0$.
    Note that we assume generously that $z(u)$ is always in $\Theta(1)$.
    When $x \leq c_0$, since $x$ is bounded by a constant, the contribution of
    $\log{x}$ to the sum is also at most constant.
    Now we consider $x > c_0$.
    Due to the definition of expectation, $\mathbb{E}_{x >
    c_0}[\log{x}] \leq \int_{c_0}^{\infty}{\log{c} \cdot \Pr(x = c) \text{ } dc}$, loosely
    summing as $c \to \infty$ even though $c$ can be more tightly bounded by
    $d_0 / d_f$.
    $\Pr(x = c)$ is certainly no greater than $\Pr(x \geq c) =
    \mathcal{O}(c^{-\alpha}) \leq \mathcal{O}(c^{-2})$ for all $\alpha \geq 2$.
    Finally, our results follow from the fact that
    $\int_{c_0}^{\infty}{\frac{\log{c}}{c^2} \text{ } dc} =
    \mathcal{O}(1)$.
    \qed
\end{proof}

\subsection{Diameter}

In this section we prove tight bounds on the diameter of randomized highway
graphs with fixed growth.
%
\begin{theorem} \label{thm:diameter}
    Let $\mathcal{G}$ be a randomized highway graph with FG dimensionality
    $\alpha$ and highway constant $k \in \Theta(\log{n})$.
    The diameter of $\mathcal{G}$ is $\Theta(\frac{\log{n}}{\log{\log{n}}})$ if
    $\alpha > 2$, and $\Theta(\sqrt[\alpha]{\log^2{n}})$ if $\alpha \leq 2$.
\end{theorem}

\begin{proof}
    The lower bound for the diameter is straightforward---from
    \Cref{lem:distance-to-highway}, there exists some node that is
    $\Omega(\sqrt[\alpha]{k \log{n}})$ hops from the highway, which is
    sufficient for $\alpha \leq 2$.
    For larger $\alpha$, we can reach the highway quickly, and the bottleneck
    rather becomes the time spent navigating the highway, $\mathcal{H}$.
    Let us consider the diameter of $\mathcal{H}$, starting from an arbitrary
    highway node $u$. 
    For a lower bound, we consider the most advantageous configuration.
    $u$ has $\Theta(k)$ long-range contacts, all to highway nodes, and
    $\Theta(1)$ local contacts, which we generously assume are
    also to highway nodes.
    Also, let us assume that, impossibly, all the $\Theta(k)$ contacts of
    all already-visited highway nodes are to new highway nodes at every step.
    Ignoring constant factors, there is at most a factor of $k$ increase in the
    number of nodes visited at each step.
    Since there are $\Theta(n/k)$ highway nodes, we can lower bound the amount
    of hops $f$ spent navigating the highway by:
    %
    \begin{equation*}
        k^f \geq \frac{n}{k} \implies f \geq \frac{\log{n} - \log{k}}{\log{k}}
    \end{equation*}
    %
    which is $\Omega(\frac{\log{n}}{\log{\log{n}}})$ when $k \in
    \Theta(\log{n})$.

    What remains to show is that it is indeed possible, with high
    probability, to navigate the highway $\mathcal{H}$ in
    $\mathcal{O}(\frac{\log{n}}{\log{\log{n}}})$ hops.
    In the standard approach for upper bounding the diameter~\cite{martel}, we
    build two large-enough sets $S$ and $T$ surrounding the source and
    destination nodes, respectively, and show that there exists an edge from $S$
    to $T$ with high probability.
    $S$ is a set of nodes that can reach be reached from $s$
    in at most
    $\ell$ hops (including taking long-range contacts), while $T$ can be thought of as its inverse, a set
    of nodes that can reach $t$ in at most $\ell$ hops.
    See \Cref{fig:diameter}.
    In our case, we are interested in the highway subgraph $\mathcal{H}$, and
    therefore construct sets $S$ and $T$ using only highway nodes and
    long-range contacts.

    Let us consider the growth of set $S$, referring to all the ``fresh'' nodes
    added at step $i$ as $\phi_i$, with $\phi_0 = \Theta(\log{n})$ consisting of the
    highway nodes within distance $\Theta(\sqrt[\alpha]{k \log{n}})$ of $s$.
    In the worst case, all the nodes in $\phi_i$ are as far away from other highway
    nodes as possible.
    Let $|S|$ be the final size of set $S$.
    If they were all bunched up together as close as possible, their maximum
    radius in the underlying graph $\mathcal{F}$ would be
    $\Theta(\sqrt[\alpha]{k |S|})$ as determined from
    \Crefpart{lem:balls}{lem:balls:always}.
    In order to reach $\Theta(n_\mathcal{H}^\eta)$ highway nodes, where
    $n_\mathcal{H} = \Theta(n / k)$ is the total number of highway nodes and
    $\eta < 1$ is some constant, we must only consider distances from $\phi$ up
    to $\Theta(\sqrt[\alpha]{k n_\mathcal{H}^\eta}) =
    \mathcal{O}(\sqrt[\alpha]{n^\theta})$ for some other constant $\theta < 1$.
    Using \Cref{lem:contacts-out} we state that the probability
    of a long-range contact $v$ of a highway node $u$ in $\phi_i$ to be fresh
    is at least proportional to $\frac{\log{n}}{k z(u)}$.
    Using the pessimistic normalization constant bound of
    \Crefpart{lem:z}{lem:z:always}, and the fact that $k \in \Omega(\log{n})$,
    the probability that $v$ is fresh is at least proportional to
    $\frac{\log{n}}{k \log{\log{n}}}$.
    The expected number of fresh nodes added at step $i$, $|\phi_{i + 1}|$,
    considering that each node in $\phi_i$ has $\Theta(k)$ long-range
    contacts, is at least proportional to $|\phi_i|
    \frac{\log{n}}{\log{\log{n}}}$.
    Recalling that the number of nodes in $|\phi_i|$ is always
    $\Omega(\log{n})$, the probability that there are asymptotically fewer fresh
    highway nodes is at most $n^{-\frac{\log{n}}{\log{\log{n}}}}$ using the
    simplified Chernoff bounds of~\cite{DILLENCOURT2023106397}.
    Equipped with a growth factor of $\frac{\log{n}}{\log{\log{n}}}$, we can
    apply similar reasoning to the diameter lower bound to state that we reach
    $n_\mathcal{H}^\eta$ total highway nodes in at most
    $\mathcal{O}(\frac{\log{n}}{\log{\log{n}} - \log{\log{\log{n}}}}) =
    \mathcal{O}(\frac{\log{n}}{\log{\log{n}}})$ hops.
    A similar argument along with the results of \Cref{cor:contacts-in} can
    grow set $T$ to similar magnitude.

    \begin{figure}[t]
        \centering
        \includegraphics[width=0.8\linewidth]{figures/diameter.png}
        \caption{
            A visualization of the two sets $S$ and $T$.
            $S$ is the set of highway nodes that can be efficiently reached from
            the source $s$, while $T$ is the set of highway nodes that can
            efficiently reach the destination $t$.
            We show that these two sets intersect with high probability, and
            thus $s$ can route to $t$ efficiently.
        }
        \label{fig:diameter}
        \vspace*{-\medskipamount}
    \end{figure}

    If $S$ and $T$ intersect, a path exists between
    $\phi_0$ and $\psi_0$ in $\mathcal{H}$ in at most
    $\mathcal{O}(\frac{\log{n}}{\log{\log{n}}})$ hops, where $\psi_i$ is the
    set of fresh nodes added at step $i$ to $T$, completing our proof.
    We conclude by showing that if $S$ does not intersect with $T$ by
    step $f$, it almost surely will intersect with $T$ at the next step.
    It is easy to see that $|\phi_{f_1}|$ and $|\psi_{f_2}|$
    are both in $\Theta(n_\mathcal{H}^\eta)$.
    The maximum distance between any two nodes in $\mathcal{F}$ is
    $\Theta(\sqrt[\alpha]{n})$, and so the probability that an arbitrary
    long-range contact $v$ of a highway node $u$ in $\phi_f$ is a
    member of $\psi_f$ is at least proportional to $n_\mathcal{H}^\eta/(n
    z(u)) \geq n^{\eta - 1}/(k z(u))$.
    The probability that $v$ is not a member of $\psi_f$ is therefore at most
    $1 - n^{\eta - 1}/(k z(u)) \leq e^{-\frac{n^{\eta - 1}}{k z(u)}}$.
    The $\Theta(n_\mathcal{H}^\eta)$ members of $\phi_f$ each have $\Theta(k)$
    long-range contacts, and so the probability that none are in
    $\psi_f$ is at most $e^{-\frac{n^{2 \eta - 1}}{k z(u)}} \leq
    e^{-\frac{n^{2 \eta - 1}}{\log{n} \log{\log{n}}}}$ 
    for $k \in
    \Theta(\log{n})$.
    By picking $\eta > 0.75$, $n^{2 \eta - 1} \geq \sqrt{n}\log^3{n}$,
    the probability of failure is at most $e^{-\sqrt{n} \ln{n}} =
    n^{-\sqrt{n}}$, and of success is at least $1 -
    n^{-\sqrt{n}}$ which is certainly with high probability.
    \qed
\end{proof}
