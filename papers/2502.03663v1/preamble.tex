\usepackage{amsmath, amssymb} 

\usepackage[T1]{fontenc}
\usepackage{tikz}
\usepackage{xcolor}
\usepackage{cite}
\usetikzlibrary{trees,positioning,decorations.pathreplacing,calc,patterns,fit}
\usepackage{color}
\usepackage{graphicx}
\usepackage{calc}
%\usepackage{asmthm}
\usepackage{amsfonts}
\usepackage{xstring}
\usepackage[pdfencoding=auto]{hyperref}
\usepackage{cleveref}
\usepackage{caption}
\usepackage{subcaption}
\usepackage{ifthen}
\usepackage[misc]{ifsym}
\usepackage{pifont}
\usepackage{mathtools}
\usepackage{bbding} % \Envelope

%comment these two out to work with submission.tex
% \newtheorem{theorem}{Theorem}
% \newtheorem{lemma}{Lemma}

\newcommand{\R}{\mathbb{R}}
\newcommand{\N}{\mathbb{N}}

%
% The following is a hack to save space.
%
% \renewcommand{\subsection}[1]{\textbf{#1}.}
% \renewcommand{\subsubsection}[1]{\textbf{#1}.}

\hypersetup{
    colorlinks=true,
    linkcolor=blue,
    urlcolor=blue,
    linktoc=all,
    citecolor=blue
}

\definecolor{okabe1}{HTML}{000000}
\definecolor{okabe2}{HTML}{E69F00}
\definecolor{okabe3}{HTML}{56B4E9}
\definecolor{okabe4}{HTML}{009E73}
\definecolor{okabe5}{HTML}{F0E442}
\definecolor{okabe6}{HTML}{0072B2}
\definecolor{okabe7}{HTML}{D55E00}
\definecolor{okabe8}{HTML}{CC79A7}

\newcommand{\cmark}{\textcolor{okabe4}{\ding{51}}}%
\newcommand{\xmark}{\textcolor{okabe7}{\ding{55}}}%

\usepackage[]{todonotes} %disable option
\newcommand{\mike}[2][inline]{\todo[color=okabe2!50,#1]{\sf \textbf{Mike:} #2}}
\newcommand{\evrim}[2][inline]{\todo[color=okabe3!50,#1]{\sf \textbf{Evrim:} #2}}
\newcommand{\ofek}[2][inline]{\todo[color=okabe4!50,#1]{\sf \textbf{Ofek:} #2}}
\newcommand{\vinesh}[2][inline]{\todo[color=okabe5!50,#1]{\sf \textbf{Vinesh:} #2}}
\newcommand{\abraham}[2][inline]{\todo[color=okabe6!50,#1]{\sf \textbf{Abraham:} #2}}

\newcommand{\crefpart}[2]{%
	\namecref{#1}~\hyperref[#2]{\labelcref*{#1}.\ref*{#2}}%
}

\newcommand{\Crefpart}[2]{%
	\nameCref{#1}~\hyperref[#2]{\labelcref*{#1}.\ref*{#2}}%
}
\renewcommand{\emph}[1]{\textit{\textbf{#1}}}
\clubpenalty=1000
\widowpenalty=1000
\hyphenpenalty=2000
\tolerance=1000

\let\epsilon\varepsilon

\newcommand{\polylog}{\operatorname{polylog}}

\DeclarePairedDelimiter\pars{\lparen}{\rparen}

% \pgfplotsset{compat=1.18}