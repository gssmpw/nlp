\section{More Experiments}\label{sec:more-exp}

In this section we support our assumption that the estimated dimensionality
$\alpha$ is a better indicator of the optimal clustering exponent than
network size.
Our experimental framework is very similar to that of the recent papers
analyzing U.S. road networks as small-world
graphs~\cite{goodrich2022modeling,gila2023highway}.
We use the same data set and methodology to represent the
graphs~\cite{routingkit,geisberger2012exact}.
The full code for our experiments can be found at
\url{https://github.com/ofekih/Fast-Geographic-Routing-in-Fixed-Growth-Graphs}.

\subsubsection{Estimating Dimensionality and Clustering Coefficient.}
In a fixed-growth graph with constant dimensionality $\alpha$,
$|\mathcal{B}_\ell(u)| \in \Theta(\ell^\alpha)$ for any $u$ and reasonable
values of $\ell > 0$.
When $\ell = 0$, however, there is only one node in the ball, so we can correct
our estimate by adding 1 to the ball size, i.e., $|\mathcal{B}_\ell(u)| = 1 +
\Theta(\ell^\alpha)$, or equivalently, $|\mathcal{B}_\ell(u)| - 1 =
\Theta(\ell^\alpha)$.
Removing the asymptotic notation, we have $c_1 \ell^\alpha \leq
|\mathcal{B}_\ell(u)| - 1 \leq c_2 \ell^\alpha$ for some constants $c_1 < c_2$.
For any finite graph, we can use the distance between $c_1$ and $c_2$ as some
measure of how well the graph fits a fixed-growth model with dimensionality
$\alpha$.

For simplicity, let us momentarily imagine a lattice $\mathcal{L}$.
A more accurate estimate for the ball size of a lattice is actually has an
additional factor of $2\alpha$, i.e., $2\alpha \ell^\alpha$, since nodes
that are not on the edge have $2 \alpha$ local contacts.
Furthermore, if the lattice does not have wrap-around edges, a node in a corner of a
2D lattice will only have one fourth the growth rate of a node in the center, or
$2^{-\alpha}$ times the growth rate for general $\alpha$.
Both of these factors that depend on $\alpha$ would be hidden from the
big-$\mathcal{O}$ notation but would affect the constants $c_1$ and $c_2$.
To minimize their influence, we use the \textit{ratio} $c_2 / c_1$ as our measure of how
well the graph fits a fixed-growth model with dimensionality $\alpha$.
It is then just a matter of finding the $\alpha$ that minimizes this ratio over all nodes with balls of all radii.
Since we dealt with rather large graphs, minimizing the ratio was
computationally expensive, so we instead sampled balls of all sizes for
thousands of random nodes, determined the best $\alpha$ for each node, and
used the median of these $\alpha$ values as our estimate for the overall graph
dimensionality.

In order to estimate the optimal clustering coefficient $s$, we tried a range of
different values for each state, hundreds of thousands of randomized greedy
routing experiments for each.
We used a highway constant $k = \log{n}$, but anecdotally reached very similar
results using $k = 1$ and even by using a power law distribution as was done
in~\cite{goodrich2022modeling}.
Running these experiments was very time consuming, and there was often a range
of clustering exponents that achieved very similar performance
(see~\cite{goodrich2022modeling}), so our results may contain errors of up to
$\pm 0.05$, which may explain some of the variance but should not significantly
affect our conclusions.

\subsubsection{Dimensionality vs. Network Size.}
%
In a previous analysis of U.S. road networks~\cite{goodrich2022modeling}, it was
conjectured that the optimal clustering exponent should increase as the size of
the state increases, tending towards 2.
From our method of estimating dimensionality from the previous section, it is
indeed true for lattices that as the side length increases, the estimated
dimensionality increases until it approaches the true dimensionality.
The question remains, however, whether it is the network size or the estimated
dimensionality which more directly predicts the optimal clustering exponent.
While it is hard to determine a clear causal relationship, it is clear from our
results in \Cref{fig:indicator_comparison} that our estimated dimensionality
$\alpha$ is a much better indicator than the network size.
These results further support and motivate our fixed-growth model and the
theoretical findings in this paper.

\begin{figure}[h!]
	\centering
	\includegraphics[height=.3\textheight, clip]{figures/clustering_exponent_vs_dimensionality.png}
	\hfill
	\includegraphics[height=.3\textheight,trim={1cm 0 0 0}, clip]{figures/clustering_exponent_vs_size.png}
	\vspace*{-\medskipamount}
	\caption{
		A comparison between the dimensionality $\alpha$ (left) and
		the number of nodes $n$ (right) as indicators for the optimal clustering
		coefficient $s$.
		It is clear that the dimensionality $\alpha$ is a much better indicator
		than the network size.
		\label{fig:indicator_comparison}
	}
\end{figure}
