\section{Definitions}

In general, small-world models take an existing undirected connected graph $G$,
such as a lattice $\mathcal{L}$, and augment it into a small-world graph
$\mathcal G$ by adding \textit{directed} long-range edges.
The original edges of $G$ are referred to as \textit{local contacts}, and the
new edges are referred to as \textit{long-range contacts}.
% Let $n$ be the number of vertices of $G$.
% We call the original edges of $G$ local contacts. 
% Here, we consider the edges of $G$ as bidirectional, but the long-distance edges as directed.
% We define the distance $d(u,v)$ as the number of local contacts it takes to walk from nodes $u$ to $v$ in some small-world graph. 
% We say that the chance 
The probability
% that some node $u$ has a long-range contact $v$
that a long-range contact added to node $u$ is another node $v$
% in $\mathcal G$ 
is proportional to $d(u,v)^{-s}$, where $s$ is called the \textit{clustering
coefficient}, and $d(u,v)$ denotes the distance between $u$ and $v$ using only local
contacts in the original graph $G$.
A greater clustering coefficient biases long-range contacts to nodes that are
nearer to $u$.
% We call $s$ the clustering coefficient and its choice impacts the speed at which greedy routing takes place. 
A ``ball'' of radius $\ell$ centered around a node $u$, denoted as
% Let ball 
$\mathcal B_\ell(u)$%
, is
% of radius $\ell$ at $u$ to be 
the set of all nodes within distance $\ell$ of $u$.

% moved down

% Consider two balls $\mathcal B_{\ell_1}$ and $\mathcal B_{\ell_2}$ such that $\ell_2 \geq \ell_1$. Let $w = \ell_2 - \ell_1$ and let shell $S^{\ell_2}_{\ell_1} = \mathcal B_{\ell_2}(u) - \mathcal B_{\ell_1}(u)$, the set of all vertices of distance greater than $\ell_1$ and less than or equal to $\ell_2$ of $u$. 
% We call $w$ the width of $S$.
% We denote $\mathcal S^{(w)}(u)$ as the set of all disjoint shells centered at $u$, each of width $w$ (the first shell equals the ball $\mathcal B_w(u)$). 

We define a graph family $\mathcal{F}$ as having \emph{fixed-growth} (FG)
dimensionality $\alpha$ if, for all possible graph sizes $n$,
for all nodes $u \in \mathcal{F}(n)$, and for all radii
$1 \leq \ell \leq \Theta(\sqrt[\alpha]{n})$, $|\mathcal B_\ell(u)| =
\Theta(\ell^\alpha)$, where the ball of some radius $\ell \in
\Theta(\sqrt[\alpha]{n})$ encompasses the entire graph.\footnote{
Below, we loosely use the term ``graph'' to denote entire graph families.}
Note that the constants hidden in the $\Theta$ notation are defined over the
entire graph family $\mathcal{F}$ and do not depend on the graph size $n$.
Also observe that all fixed-growth graphs
have constant degree, since the number of nodes at distance 1 from any node,
$|\mathcal B_1(u)|$, is $\Theta(1)$.
Nodes in $\alpha$-dimensional lattices have degree $2 \alpha$, which is indeed
constant with respect to $\alpha$.
We compare the growth of balls in fixed-growth graphs to that of lattices in
\Cref{fig:balls-and-shells}.

Throughout the paper, we denote certain events as having high probability in
$x$, where $x$ is a variable such as $n$ or a function of a variable such as
$\log n$. An event succeeds with high probability in $x$ if it succeeds with
probability at least $1 - x^{-c}$ for some constant $c > 1$. 
% As shown in \Cref{fig:weird-shell}, fixed-growth graphs may have unusual
% underlying geometry.
% Regardless, our definitions of balls and shells still
% apply.

% \begin{figure}
%     \centering
%     \includegraphics[width=0.7\linewidth]{figures/Balls_and_Shells.png}
%     \caption{Here we have an example of a ball and a shell centered at the blue node $u$. The lightly shaded region is the ball $\mathcal B_2(u)$ and the darkly shaded region is the shell $S^4_2(u)$. This shell's width $w$ is 2. Note that, by definition, $S^4_2(u)$ does not contain the nodes on the boundary of ball $\mathcal B_2(u)$. The reason for this is to prevent overlap when defining a system of shells $\mathcal S$.}
%     \label{fig:balls-shells}
% \end{figure}

\begin{figure}[t]
	\centering
	\includegraphics[height=.28\textheight]{figures/Balls_and_Shells_v2.png}
	\hfill
	\includegraphics[height=.28\textheight]{figures/weird_shell.png}
	% \vspace*{-\medskipamount}
	\caption{
		In these figures we show balls of various widths surrounding some
		central node $u$ in a lattice graph (left) and a fixed-growth graph
        (right).
		Note how balls grow far more predictably in lattices than they do in
		general fixed-growth graphs.
		We refer to the shaded regions in between two balls as shells, and the
		difference in radius between two balls as the shell width $w$.
        \label{fig:balls-and-shells}
    }
	\vspace*{-\medskipamount}
\end{figure}

% \begin{figure}[t]
%     \centering
%     \includegraphics[width=0.6\linewidth]{figures/weird_shell.png}
%     \caption{Here we have a representation system of shells $\mathcal S$ surrounding some node $u$ in an arbitrary fixed-growth graph. We make no assumptions of geometric structure, so concentrations of nodes in space may vary. Thus, shell shapes may have arbitrary structure, as shown here. Each shell is nonetheless required to have width $w$. The innermost darkly shaded ball only contains nodes that can be reached from $u$ by walking along at most $w$ edges. Likewise, the first shell contains all nodes that can only be reached by walking more than $w$ edges but at most $2w$ edges from $u$.
%     }
%     \label{fig:weird-shell}
% \end{figure}

\subsection{Randomized Highway Model}
%
The \emph{randomized highway model} introduced by 
Gila, Ozel, and Goodrich
% 's randomized highway model
takes a
2-dimensional lattice $\mathcal L$ and augments it into a small-world graph
$\mathcal K$ with some highway constant $k$~\cite{gila2023highway},
by adding long-range contacts with a clustering coefficient of 2.
% They introduce the highway constant $k \in [1, n]$.
However, unlike in previous models, not all nodes have long-range
contacts.
Rather,
each node
in $\mathcal L$ independently has $1/k$ chance of becoming a {\it highway node},
where only highway nodes have long-range contacts.
Because there are fewer highway nodes, each is able to have more long-range
contacts while maintaining a constant average degree.
Specifically,
to create the small-world graph $\mathcal K$, they add $q \times k$
long-range contacts to each highway node for some constant $q$.
As a result, the average degree of $\mathcal{K}$ is $q$.
Further, these
long-range contacts exclusively point to other highway nodes.
% The authors choose two as their optimal clustering coefficient, i.e., the chance
% that a long distance connection is drawn between two highway nodes $u$ and $v$
% is proportional to $d(u,v)^{-2}$. 
% Let $\mathcal H$ denote the subgraph of highway nodes of $\mathcal K$.
% If $H$ is
% the set of highway nodes, then the probability that, for $u,v \in H$, $u$ has a
% long-distance connection to $v$ is $d(u,v)^{-d} / \sum_{v \neq u \in H}
% d(u,v)^{d}$. 

% included in results sections, not necessary to be repeated for definitions

% Because, unlike in Kleinberg's original model, not all nodes have long-distance connections, greedy routing from
% arbitrary nodes $s$ to $t$ in $\mathcal K$ is split into three phases. First,
% we navigate from $s$ to the closest highway node using local contacts. We
% take the highway until we reach a highway node close to $t$. Then we leave the
% highway and take local contacts to $t$. Gila, Ozel, and Goodrich show that the
% optimal value of $k$ is $\Theta(\log n)$. With this value of $k$, they show that
% greedy routing completes in $\Theta(\log n)$ hops with high probability between
% any two nodes of the graph.
% For general $k$, greedy routing in $\mathcal K'$
% completes in $\mathcal{O}(\log^2 n / k)$ hops w.h.p.

% \subsection{Normalization Constant}
% \ofek{Should be moved to the preliminary results, added a section}
% An important notion in the study of small-world models is the normalization constant, a bound on the influence of possible destinations when we are drawing long-distance connections. For example, in Kleinberg's original model, the probability that $u$ has a long-distance connection to $v$ in a two-dimensional lattice is $d(u,v)^{-2}/\sum_{v \neq u} d(u,v)^{2}$. Here, we sum over all nodes $v \neq u$ because every node could be a recipient of $u$'s long-distance connection. The normalization constant $z$ bounds $\sum_{v \neq u} d(u,v)^{2}$, and Kleinberg showed that $z = \mathcal{O}(\log n)$ \cite{kleinberg2000small}. 

% Likewise, in the Randomized Kleinberg Highway Model, the normalization constant bounds $\sum_{v \neq u \in H} d(u,v)^{d}$, and was found to be $\mathcal{O}(\log n / k + \log n / k\log\log n + \log\log\log n )$. However, they also show that a tighter bound $z' = \mathcal{O}(\log n / k)$ applies with constant probability \cite{gila2023highway}.

% The normalization constant is found by fixing some node $u$ and, as the initial summation suggests, bounding the influence that other nodes have on it regarding its long-distance connections. It is applied when we want to bound the probability that a given node's long-distance connections fall within some specific set of nodes we are interested in.

% \subsection{Randomized Fixed-Growth Highway Model}

In our paper we extend the randomized highway model to a more general setting.
Consider an arbitrary fixed-growth graph $\mathcal F$ with dimensionality
$\alpha$.
We turn it into a randomized highway graph $\mathcal{G}$ by adding long-range
contacts with a clustering coefficient of $\alpha$ rather than 2.
We also allow each highway node to add $\Theta(k)$ long-range contacts rather
than strictly $q \times k$, where the constant bounds are the same across all
nodes.
% In this paper, we prove tight bounds for highway constant $k \in
% \Theta(\log{n})$, which was shown to be optimal by Gila, Ozel, and
% Goodrich~\cite{gila2023highway}.
% In our \emph{randomized fixed-growth highway} model, we again choose a
% highway constant $k$. Each node in $\mathcal F$ has a $1/k$ chance of becoming a
% highway node. As in the randomized Kleinberg highway model, highway nodes have
% $q\times k$ long-distance connections to each other. We choose our clustering
% coefficient to equal the fixed-growth dimensionality of $\mathcal F$. Thus, for
% nodes $u,v$ in $\mathcal H$, $u$ has a long-distance connection to $v$ with
% probability proportional to $d(u,v)^{-\alpha}
% % / \sum_{v \neq u \in \mathcal H} d(u,v)^{-\alpha}
% $
% . In this manner, we convert fixed-growth graph $\mathcal F$ into a small-world graph $\mathcal G$, which we denote as a \emph{fixed-growth highway graph}.

% In the following sections, we will prove several results about greedy routing
% and diameter in small-world graphs of the randomized fixed-growth highway model. 


% Needed for non-constant alpha I think. -vinesh
% Add back in as needed.
% We say that a family of graphs $\{G_i\}_{i\in \N}$ has dimensionality $\alpha$ if there exist $l_0\in \mathbb{N}, c_0,c_1,c_2\in \R^+$ such that for any vertex $u$ in $V(G_i)$, for all $l$ such that $l_0\leq l\leq c_0\sqrt[\alpha]{|V(G)|}$, $c_1l^\alpha\leq|B(u,l)|\leq c_2l^\alpha$

% We say that a family of graphs $\{G_i\}_{i\in \N}$ has dimensionality $\alpha:\N\mapsto\R^+$ if there exist $l_0\in \N, c_0,c_1,c_2\in \R^+$ such that for any vertex $u$ in $V(G_i)$, for all $l$ such that $l_0^{\alpha(l_0)}\leq l^{\alpha(l)}\leq c_0{|V(G)|}$, $c_1l^{\alpha(l)}\leq|B(u,l)|\leq c_2l^{\alpha(l)}$

% We say that a family of graphs $\{G_i\}_{i\in \N}$ has dimensionality $\alpha:\N\mapsto\R^+$ if there exist $l_0\in \N, c_0,c_1,c_2\in \R^+$ such that for any vertex $u$ in $V(G_i)$, for all $l$ such that $l_0^{\alpha(l_0)}\leq l^{\alpha(l)}\leq c_0{|V(G)|}$, $(c_1l)^{\alpha(l)}\leq|B(u,l)|\leq c_2l^{\alpha(l)}$