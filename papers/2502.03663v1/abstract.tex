\begin{abstract}
    In the 1960s, the social scientist Stanley Milgram performed his famous
    ``small-world'' experiments
    % . He
    where he 
    found that people in the US who are far
    apart geographically are nevertheless connected by remarkably short chains
    of acquaintances.
    Since then, there has been considerable 
        work to design networks that
    accurately model the phenomenon that Milgram observed.
    One well-known approach
    was Barab{\'a}si and Albert's \emph{preferential attachment} model,
    which has small diameter yet lacks an algorithm that can efficiently find
    those short connections between nodes.
    Jon Kleinberg, in contrast,
    proposed a small-world graph formed from an $n \times n$ lattice that
    guarantees that greedy routing can navigate between any 
    two nodes in $\mathcal{O}(\log^2 n)$ time
    with high probability.
    Further work by Goodrich and Ozel and by Gila, Goodrich, and Ozel 
        present a hybrid technique that combines elements from these previous
    approaches to improve 
        greedy routing time to $\mathcal{O}(\log n)$ hops.
    These are
    important theoretical results, but we believe that their reliance on the
        square lattice limits their application in the real world.
    Indeed, a lattice
    enforces a fixed integral \emph{growth rate}, i.e. the rate of increase in
    the number of nodes at some distance $\ell$ reachable from some node $u$. 
    % We are motivated by an earlier finding of Goodrich and Ozel, which 
    % experimentally found that routing in graphs that model the US road network
    % perform best when this growth rate parameter is near 1.5.
    % Thus, models that assume integral growth rates are not able to take advantage
    % of the underlying network topology.
    In
    this work, we generalize the model of Gila, Ozel, and Goodrich to any class
    of what we call \emph{fixed-growth} graphs of dimensionality $\alpha$, a subset of \textit{bounded-growth} graphs introduced in several
    prior papers.
    We prove tight bounds for greedy routing and diameter in these graphs, both
    in expectation and with high probability.
    We then apply our model to the U.S. road network to show that by modeling the
    network as a fixed-growth graph rather than as a lattice, we are able to
    improve greedy routing performance over all 50 states.
    We also show empirically that the optimal clustering exponent for the U.S.
    road network is much better modeled by the dimensionality of the network
    $\alpha$ than by the network's size, as was conjectured in a previous work.
    % We to take advantage
    % of the underlying network topology.
    % prove that our model also admits $\mathcal{O}(\log n)$ greedy routing in addition to
    % other bounds related to greedy routing and diameter. 
    % We believe that our result will help researchers model diverse phenomena
    % more accurately as small worlds. 
\end{abstract}
