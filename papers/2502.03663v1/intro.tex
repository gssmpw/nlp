\section{Introduction}

In a series of experiments in the 1960s, Stanley Milgram explored what is now
known as the \emph{small-world phenomenon}
\cite{milgram1967small,travers1977experimental}.
Milgram asked random volunteers across the United States to send messages to
each other by mail.
If a participant did not know the designated final recipient, they were
instructed to send the message to a friend or family member they judged more
likely to know the intended recipient.
Remarkably, Milgram found that it took a median of only six hops for the
messages to reach their final destination despite the vast geographic distances
between participants. 

Milgram's groundbreaking work spurred a field of research that seeks to model
the small-world phenomenon, finding networks that capture the following
observations that he made.
First, the network must be tightly connected, linking arbitrary pairs of nodes
with a short chain of connections.
Second, the network must admit an efficient greedy algorithm that allows nodes
to find such short chains with just their local knowledge.
Third, links should be formed in a way that models how social connections are made
in the real world.

\subsection{Kleinberg's Model and Extensions}

The most well-known model of the small-world phenomenon was described by
Kleinberg \cite{kleinberg2000small}.
Similar to the earlier Watts and Strogatz model \cite{watts1998collective},
Kleinberg envisions the social network as a two-dimensional lattice, in which
individuals have local connections to the nodes adjacent to them in the lattice
as well as ``long-range'' contacts to randomly chosen other nodes.
Kleinberg showed that the graph enables a greedy routing protocol in expected
$\mathcal{O}(\log^2 n)$ hops (where the graph is an $n \times n$ lattice).
These greedy routing results were impressive, but the model itself was not
realistic---each node had the same degree. In contrast, we observe in 
real-world data that node degree generally follows a power law distribution. 
As such, the perfect lattice structure was not
correlated with real population density, and the $\mathcal{O}(\log^2{n})$ hop bound
of Kleinberg's model, which was 
later shown to be tight~\cite{martel}, is not small enough to explain Milgram's
results.

A recent paper by Goodrich and Ozel~\cite{goodrich2022modeling} sought to
address these limitations by creating a model where node degrees are drawn from
a power law distribution while the average degree remains constant.
They tested their model on road networks of all 50 U.S.~states as a 
better proxy for population density.
While they found that their model performed
well enough \textit{empirically} to explain Milgram's experiments, 
they did not provide any theoretical results.

Even more recently, Gila, Ozel, and Goodrich provided some theoretical
motivation for these empirical results, showing how a simpler model comprised of
two types of nodes, normal nodes and \emph{highway nodes}, could perform greedy
routing using only $\mathcal{O}(\log n)$ hops in
expectation~\cite{gila2023highway}.
Highway nodes contain long-range contacts to other highway nodes, 
forming a highway subgraph, $\mathcal{H}$. 
They have higher degrees than normal nodes, but were sparse enough 
such that the graph's average degree was still a constant.
They showed how results for their simpler model can be used to show good,
namely $\mathcal{O}(\log^{1 + \epsilon} n)$, greedy routing time for similar
graphs with a power law degree distribution.
However, they only proved their results for perfect two-dimensional lattices, and
showed that there was a gap between their expected greedy routing time to the
best possible lower bound.
Further, they did not provide high probability bounds on their greedy routing
time, nor did they provide bounds on the diameter.

% Later work expanded on Kleinberg's results.
% In particular, we are interested in the recent small-world model created by
% Gila, Goodrich, and Ozel \cite{gila2023highway}.
% They use the fundamental metaphor of a road
% network. Instead of all nodes having long-distance connections, Gila {\it et
% al.} randomly designate certain nodes of the lattice as \emph{highway nodes},
% which have a greater concentration of long distance connections to each other.
% They improved routing to just $\mathcal{O}(\log n)$ time while still having the
% same average degree per node as Kleinberg's network.

% One limitation of
% %this line of work 
% these recent results is that these models
% % are variants of a lattice and thus 
% assume that the underlying network behaves like a two dimensional lattice, and thus that the
% optimal \emph{clustering coefficient}, a
% parameter that influences choice in long-distance connections, is two.

\subsubsection{Other Small-World Models.}

There is a rich body of related work on small-world models, including
work in other topologies, preferential attachment models, and other classical
results.
We review this material in \Cref{sec:related-work}.

\subsection{Dimensionality in Graphs}

Most theoretical results proved for Kleinberg's model and its extensions assume
that the underlying network, before adding long-range contacts, is a lattice
$\mathcal{L}$~\cite{kleinberg2000small,martel,gila2023highway}.
Some work even considers lattices of different dimensions, showing that the
optimal \emph{clustering coefficient}, a
parameter that influences choice in long-range contacts, is
the dimension of the lattice~\cite{martel}.
One limitation of the recent results with power law degree distributions is that
since they assume the underlying network behaves like a two dimensional lattice,
they use a value of two for the clustering
coefficient~\cite{goodrich2022modeling,gila2023highway}.
The experimental paper by Goodrich and Ozel \cite{goodrich2022modeling} tested
different coefficients for two U.S.~states 
and found that the best value for each of
them was different, and both much closer to 1.5. 
They conjectured that the difference is
related to the size of the state.
Inspired by their results, we are interested in this
paper on a notion of ``dimensionality'' that
we could apply to more general graphs than lattices, 
including road networks, to give
more insight into their behavior as small worlds.

One of the earliest notions of graph dimensionality was developed by 
Erd{\"o}s,
Harary, and Tutte~\cite{Erdos_Harary_Tutte_1965}.
They define a dimension of a graph $\mathcal{G}$ geometrically: the smallest
$d$ such that $G$ can be embedded in $d$-dimensional Euclidean space such that
all edges are of unit distance.
Another notion popular in the study of networks is the \emph{doubling dimension}
\cite{gupta2003bounded,abraham2006routing,cole2006searching,konjevod2008dynamic,chan2016hierarchical,konjevod2007compact}.
A metric space is doubling if there is some constant $\lambda$ such that any
ball of size $r$ can be covered by the union of $\lambda$ balls of size $r/2$.
Abraham, Fiat, Goldberg, and Werneck introduced the \emph{highway dimension} of
a graph, providing a unified framework for understanding shortest-path
algorithms \cite{abrahamhighway}.
A graph has small highway dimension if, for some set of {\it access nodes} $S_r$
such that all shortest paths of length $> r$ include a vertex of $r$, every ball
of size $\mathcal{O}(r)$ contains some elements of $S_r$.
Work has also been done to translate the study of \emph{fractal dimensions} of
geometric objects, such as the Hausdorff or box counting dimensions, to studying
complex networks.
Fractal dimensions model graphs of ``non-integral''
growth. See \cite{rosenbergsurvey} for a survey of fractal dimensions and their
application to networks.

We are more interested, however, in the body of work that studies graphs with
some limitation on the rate of ``growth'' of the graph,
% Several works also study graphs with the property that the growth of the graph,
the rate of increase in the number of nodes in a ball of radius $\ell$ around any
node $u$, is bounded by some function of
$\ell$~\cite{stefanExpansion,johannesExpansion,kargerExpansion,ittai2005name,kuhn2005locality,krauthgamer2003intrinsic,BARTAL2005192,gfeller2007randomized}.
For example, some enforce a \emph{bounded-growth} requirement, that
the size of a ball of radius $\ell$ around any node contain at most
$\mathcal{O}(\ell^\alpha)$ nodes for some constant $\alpha$.
% \vinesh{changed half a sentence:}
Our work considers a subset of such bounded-growth graphs with a slightly stricter requirement.
We consider the family of \emph{fixed-growth} graphs 
$\mathcal{F}$, where the number of
nodes in balls of radius $\ell$ around any node is both lower and upper bounded
by values in  $\Theta(\ell^\alpha)$ for reasonable values of $\ell$, where
$\alpha$ can be considered the dimensionality of the graph. 
Lattices of any dimension $\alpha$ are examples of fixed-growth graphs.
% Lattices of any dimension are examples of fixed-growth graphs, where $\alpha$ is
% the dimension of the lattice.

A paper by Duchon, Hanusse, Lebhar, and Schabanel extends Kleinberg's results to
more general graphs following a similar requirement~\cite{duchon2006could}.
In that paper, however, they were only able to show that greedy routing runs in
some polylogarithmic time, without determining the specific exponent, achieving
worse results than Kleinberg's let alone those of more recent papers.

% Prior work most similar to ours is a paper authored by Duchon, Hanusse, Lebhar,
% and Schabanel \cite{duchon2006could}.
% They apply Kleinberg's results to a more general set of graphs, but this means
% that the best they can prove is that greedy routing runs in $\mathcal{O}(\log^2
% n)$ time.
% For many types of graphs, they are only able to show that greedy routing runs in
% polylogarithmic time, without determining the specific exponent.
% Furthermore, they only focus on greedy routing time whereas we also study the
% diameter of the graphs in our generalization.


% With the exception of highway dimension, all of these notions seek to bound how
% fast a graph grows or the complexity of a graph's representation. Indeed, the
% bounded-growth-style definitions are almost identical to our fixed-growth model.
% Because of these broad similarities, we believe that our small-world models will
% be broadly applicable in future network research that considers graphs with
% these related definitions of dimensionality.  

\subsection{Our Results}

In this paper, we extend on the theoretical work 
of Gila, Ozel, and Goodrich~\cite{gila2023highway},
providing tight bounds on greedy routing time and diameter.
Further, in addition to greedy routing results \textit{in expectation}, we
provide results for greedy routing with high probability.
We prove all these results for a more general class of graphs, fixed-growth
graphs.

We motivate our use of the fixed-growth model with empirical results on U.S.
road networks, showing how by modeling U.S. road networks as fixed-growth graphs
we are able to pick clustering coefficients that perform much better than by
just assuming a lattice structure.
Further, we show how our notion of fixed-growth dimensionality is a much better
indicator of the optimal clustering coefficient than the size of the network, as
was conjectured in~\cite{goodrich2022modeling}.

% In this paper, we apply the heuristic developed by Goodrich and Ozel
% \cite{goodrich2022modeling} to graphs representing the road networks of all 50
% US states.
% They run their algorithm assuming several values of this clustering coefficient
% and identify which one minimizes the cost of greedy routing.
% We do the same and find that these clustering coefficient values are notably
% smaller than two and are correlated with the dimensionality of each road network
% graph, which we also measure. This suggests that prior lattice-based methods
% overestimate the growth rate of these networks and make greedy routing slower
% than it could be.

% Prior work most similar to ours is a paper authored by Duchon, Hanusse, Lebhar,
% and Schabanel \cite{duchon2006could}.
% They apply Kleinberg's results to a more general set of graphs, but this means
% that the best they can prove is that greedy routing runs in $\mathcal{O}(\log^2
% n)$ time.
% For many types of graphs, they are only able to show that greedy routing runs in
% polylogarithmic time, without determining the specific exponent.
% Furthermore, they only focus on greedy routing time whereas we also study the
% diameter of the graphs in our generalization.

% We pose a generalization of the recent small-world model of Gila {\it et al.}
% \cite{gila2023highway} to a family of what we call \emph{fixed-growth graphs},
% i.e. any graph such that the number of nodes of distance $\ell$ surrounding some
% node $u$ is $\Theta(\ell^\alpha)$ for some constant $\alpha \geq 1$.
% We show that, regardless of $\alpha$, greedy routing occurs in $\mathcal{O}(\log
% n)$ time. 
We believe that by extending the results of Gila {\it et al.} to fixed-growth graphs, our work enables more accurate analyses of phenomena
modeled using small-world graphs such as disease spread
\cite{BARTHELEMY20111,warren2001firewallsdisorderpercolationepidemics,SANDER20031,LI2021111294,costaanalyzing2011},
decentralized peer-to-peer communication
\cite{dahliaviceroy2002,zhangusing2002,qindongsecure2022,shinsmall2011,li2005searching,singla2012jellyfish,hui2004small},
and gossip protocols \cite{kempe2004spatial,kempe2002protocols}.

% \subsection{Related Work}

% \subsubsection{Small-World Models.}

% Prior to Milgram's experiment, Pool and Kochen \cite{de1978contacts} as well as others \cite{kochensmallworld} found that if social connections were selected uniformly from the population, then the diameter of the graph is small with high probability. 
% While interesting, these early results did not provide a compelling reason for the small-world phenomenon. 
% In 1998, Watts and Strogatz suggested that individuals should have both ``local'' connections to adjacent nodes as well as ``long-range'' connections like the ones studied prior \cite{watts1998collective}. 
% This captured a sense of locality to social networks, and was able to model several natural and manmade phenomena. 
% Kleinberg \cite{kleinberg2000small} generalized and improved their model, altering the criteria for choosing long-range connections such that the graph enabled greedy routing between any two nodes to succeed with high probability in $\mathcal{O}(\log^2 n)$ time. 
% Martel and Nguyen \cite{martel} later found that greedy routing in Kleinberg's model runs in $\Theta(\log^2 n)$ time yet the diameter of the graph is $\Theta(\log n)$, meaning that Kleinberg's algorithm fails to find the true shortest paths available in the graph by a $\log n$ factor. 
% Other works have implemented Kleinberg's analyses in other topologies and types of graphs \cite{fraigniaudgreedy,kleinbergdynamics,barriereefficient,duchon2006could}. 

% Another small-world model is the \emph{preferential attachment} model of Barabási and Albert \cite{barabasipreferential}. 
% Here, nodes are added to the graph one at a time.
% New nodes connect to randomly selected other nodes, with probability proportional to the other nodes' degree (a ``rich-get-richer'' process). Dommers, Hofstad, and Hooghiemstra \cite{dommers2010diameters} showed that the diameter of this model is low, yet unlike Kleinberg's model, no greedy algorithm exists to find such short paths between nodes. 
% Since then, there have been efforts to combine the two models in order to get the best of both worlds.
% Bringmann, Keusch, Lengler, Maus, and Molla propose such a model with an average $\mathcal{O}(\log\log n)$ greedy routing time, yet their model requires nodes of the network to be randomly distributed in some geometric space. 
% Goodrich and Ozel take another approach, inspired by the US highway system (which is ultimately what carried the messages shared in Milgram's experiment), called the \emph{neighborhood preferential attachment} model. 
% Nodes are again added to the graph one at a time, but the probability of connecting to some other node $v$ is based both on the degree of $v$ as well as the distance between the new node and $v$. 
% While their intuition was corroborated by their strong empiricial results as well as by Abraham, Fiat, Goldberg, and Werneck's work on the \emph{highway dimension} of graphs \cite{abrahamhighway}, Goodrich and Ozel were not able to prove any theoretical bounds for their model \cite{goodrich2022modeling}. 
% In 2023, Gila, Ozel, and Goodrich \cite{gila2023highway} adapted the model of Goodrich and Ozel back to a lattice, their \emph{randomized highway} model, and were able to show that it performs greedy routing in $\mathcal{O}(\log n)$ time. 

% moved up

% \subsubsection{Dimensionality in Graphs.}

% In this section, we will place our fixed-growth model of dimensionality in the context of other definitions of graph dimensionality that have been developed over years.  
% One of the earliest notions of dimensionality in graphs was developed by Erdős, Harary, and Tutte \cite{Erdos_Harary_Tutte_1965}. 
% They define dimension of $G$ in a geometrical way: the smallest $d$ such that $G$ can be embedded in $d$-dimensional Euclidean space such that all edges are of unit distance. 
% Another notion popular in the study of networks is the \emph{doubling dimension} \cite{gupta2003bounded,abraham2006routing,cole2006searching,konjevod2008dynamic,chan2016hierarchical,konjevod2007compact}.  
% A metric space is doubling if there is some constant $\lambda$ such that any ball of size $r$ can be covered by the union of $\lambda$ balls of size $r/2$. 
% The highway dimension of a graph was introduced by Abraham, Fiat, Goldberg, and Werneck to provide a unified framework for understanding shortest-path algorithms \cite{abrahamhighway}. 
% A graph has small highway dimension if, for some set of {\it access nodes} $S_r$ such that all shortest paths of length $> r$ include a vertex of $r$, every ball of size $\mathcal{O}(r)$ contains some elements of $S_r$. 

% Several works also study graphs with the property that the growth of the graph, the rate of increase in the number of nodes in a ball of radius $r$ around any node $u$, is bounded by some function of $r$ \cite{stefanExpansion,johannesExpansion,kargerExpansion,ittai2005name,kuhn2005locality,krauthgamer2003intrinsic,BARTAL2005192,gfeller2007randomized}. Several of these works require that the size of a ball of radius $r$ contain $\mathcal{O}(r^\alpha)$ nodes for some constant $\alpha$.
% Work has also been done to translate the study of \emph{fractal dimensions} of geometric objects, such as the Hausdorff or box counting dimensions, to studying complex networks. These fractal dimensions are useful for modeling graphs of ``non-integral'' growth. See \cite{rosenbergsurvey} for a survey of fractal dimensions and their application to networks.

% With the exception of highway dimension, all of these notions seek to bound how fast a graph grows or the complexity of a graph's representation. Indeed, the bounded-growth-style definitions are almost identical to our fixed-growth model. Because of these broad similarities, we believe that our small-world models will be broadly applicable in future network research that considers graphs with these related definitions of dimensionality. 

% moved up

% \subsection{Our Contributions}

% In this work, we generalize the randomized Kleinberg highway model of Gila {\it
% et al.} \cite{gila2023highway} into our randomized fixed-growth highway model.
% We show that we can perform greedy routing using the model of Gila {\it et al.}
% for any $\alpha$ in expected $\mathcal{O}(\log n)$ time.
% We also prove theoretical results not found in their work, such as a lower bound
% on our greedy routing and a bound on the diameter of the graph for any $\alpha$.
% We find that greedy routing in our model is optimal for $\alpha < 2$ and is
% non-optimal by only a $\log\log n$ factor for $\alpha \geq 2$, unlike
% Kleinberg's original model, which was shown to have a $\log n$-factor gap
% between greedy routing and graph diameter \cite{martel}.

% moved up

% Ultimately, our work seeks to provide a theoretical underpinning to the
% empirical results found by Goodrich and Ozel \cite{goodrich2022modeling} in
% their study of the neighborhood preferential attachment model.
% Intriguingly, when analyzing the road networks of US states converted into
% graphs, they found that greedy routing performed best with parameters that
% suggested the graphs had fractional growth rates between one and two.
% Goodrich and Ozel only measured data for the highway networks of two states,
% Hawaii and California \cite{goodrich2022modeling}.
% In this work, we will validate their empirical results and extend them to all 50
% US states under our new randomized fixed-growth model.
% We observe that the parameters for optimal greedy routing are correlated with
% the intrinsic growth rate of the given graph.
% This suggests that our model is better able to explain Milgram's results than
% prior work that assumed all networks are similar to
% lattices~\cite{kleinberg2000small,gila2023highway}.

% moved up

% Recently, Gila, Ozel, and Goodrich sought to provide a theoretical explanation
% for the results found by Goodrich and Ozel
% \cite{gila2023highway,goodrich2022modeling}. However, to do so they had to embed
% the original neighborhood preferential model into a two-dimensional lattice to
% prove any results, which restricts them to modeling graphs of integral growth
% rate. By generalizing Gila, Ozel, and Goodrich's results to any fixed-growth
% graph, in particular ones with non-integral $\alpha$, we finally have a
% small-world model that both has strong theoretical results and also explains the
% empirical results of Goodrich and Ozel.
