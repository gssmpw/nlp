\section{Experiments}

One of the key assumptions made in previous papers is that the underlying
network is a lattice, and consequently for 2-dimensional data, the optimal
clustering exponent should be 2.
This assumption was even made in two recent papers that analyzed road networks
of the 50 U.S. states~\cite{goodrich2022modeling,gila2023highway} and DC, despite the
fact that these networks are certainly not laid out like perfect square grids.
Our work rejects the notion that the road networks are best modeled as a
lattice, and instead propose modeling them as fixed-growth graphs with some
dimensionality $\alpha$ that is not necessarily an integer.
In this section we present experiments to support this claim, showing how by
modeling the road networks as fixed-growth graphs, and by picking a clustering
coefficient as determined by the underlying dimensionality $\alpha$ of the graph
rather than assuming it is 2, we can achieve better routing performance for
\textit{every} U.S. state.
See \Cref{fig:greedy_routing_comp}.

It is important to note, however, that our definition of ``fixed-growth'' graphs
is not directly meaningful for finite graphs or graph families as the constants
hidden in the fixed-growth $\Theta$ can be picked to make any dimensionality
$\alpha$ work.
Nevertheless, we have found a heuristic to estimate
which dimensionality $\alpha$ is most appropriate for a given graph: 
minimizing the discrepancy in size between balls of the same radius. 
We also employ the same method as \cite{goodrich2022modeling} to 
empirically determine the best clustering coefficient%
% for a graph
.
We go into more detail in \Cref{sec:more-exp}.

\begin{figure}[t]
	% \vspace*{-\bigskipamount}
	\centering
	\includegraphics[width=\linewidth]{figures/greedy_routing_comp.png}
	\vspace*{-\bigskipamount}
	\vspace*{-\medskipamount}
	\caption{
		Here we compare the average greedy routing performance of randomized
		highway graphs constructed from the road networks of all 50 U.S. states
		and DC using different clustering coefficients $s$.
		Specifically, we compare the performance for when $s$ is set to 2
		(blue), which is the value used by previous papers who assume that the
		road networks behave like a 2-dimensional lattice, to when $s$ is set to
		$\alpha$ (orange), the estimated dimensionality of the graph assuming
		that the road networks behave like fixed-growth graphs.
		Modeling them as fixed-growth graphs \textit{always} provides better
		greedy routing performance, as evident by their difference (green)
		always being positive.
		\label{fig:greedy_routing_comp}
	}
	\vspace*{-\medskipamount}
\end{figure}

\begin{figure}[t]
	\centering
	\includegraphics[width=\linewidth]{figures/clustering_exponent_vs_dimensionality_exact.png}
	\vspace*{-\bigskipamount}
	\vspace*{-\medskipamount}
	\caption{
    % In this experiment, we have utilized road network data from all 50 U.S. states to compute estimates of the optimal clustering coefficient $s$ (used to draw long-range edges) and of the FG dimensionality $\alpha$ of each road network.
    % We compute each value of $s$ as in \cite{goodrich2022modeling}, by repeatedly performing randomized instances of our greedy routing algorithm on each graph with different choices of $s$ until we find an $s$ that minimizes the number of hops. 
    % To compute the dimensionality of each graph, we use a heuristic: finding an $\alpha$ that minimizes the discrepancy in size between balls of the same radius. 
    Here we compare our estimated dimensionality $\alpha$, as used in \Cref{fig:greedy_routing_comp},
	to the empirically
    determined optimal clustering exponent $s$ for each state.
	There is a clear correlation between our values of $s$ and $\alpha$, and
    importantly, none of the optimal clustering coefficients are close to two,
    the assumed dimensionality in previous work
    \cite{gila2023highway,goodrich2022modeling,kleinberg2000small}.
    This suggests that our fixed-growth model is able to take advantage of the
    intrinsic growth rate of these graphs in a way that abstracting them to a
    lattice does not.
    \label{fig:clustering_exponent_vs_dimensionality}}
	\vspace*{-\medskipamount}
\end{figure}

We used the same U.S. road network data as the previous papers to provide an
estimate for the approximate dimensionality $\alpha$ and plotted it against the
empirically determined optimal clustering exponent $s$ in
\Cref{fig:clustering_exponent_vs_dimensionality}.
The first thing to note is that the estimated dimensionality $\alpha$ for U.S.
states is some fraction between 1 and 2.
Most previous small world analyses only apply to graphs with integer
dimensionality.
We observe that the correlation, while not perfect, is still clearly there.
Road networks that have much lower dimensionality such as Alaska
have a much smaller optimal clustering exponent than those with higher
dimensionality like that of Washington, DC.
And for all networks, the approximate dimensionality is a much better choice for
the clustering exponent than 2
as is further confirmed by \Cref{fig:greedy_routing_comp}. 
% As a result, we believe that
% routing in our fixed-growth model 
% better explains Milgram's initial findings than previous models which
% model networks in a lattice. 

A previous paper empirically analyzing the U.S. road networks in
depth~\cite{goodrich2022modeling} conjectured that as the size of the state
increases, the optimal clustering exponent should increase tending towards 2,
based on a comparison of two U.S. states: Hawaii and California. 
In \Cref{sec:more-exp}, we present data from all 50 states showing
a much stronger correlation between the optimal clustering exponent and the
approximate dimensionality of the graph than to the state size.
