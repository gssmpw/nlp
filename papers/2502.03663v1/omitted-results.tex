\section{Omitted Results and Proofs} \label{sec:misc-proofs}

\vspace*{-\medskipamount}

Here we provide additional lemmas and proofs omitted from
the main text.
\vspace*{-\medskipamount}

\subsubsection{The Number of Independent Balls.}
%
The following result tightly bounds the number of independent balls present
in the graph.
This result is important for lower bounding the distance to the highway, as is
done in the proof of \Cref{lem:distance-to-highway}.
%
\vspace*{-\medskipamount}
\vspace*{-\medskipamount}
\begin{lemma} \label{lem:indep-balls}
	A fixed-growth graph $\mathcal{G}$ with dimensionality $\alpha$ contains
	$\Theta\pars*{\frac{n}{\ell^\alpha}}$ independent balls of radius $\ell$.
\end{lemma}
%
\begin{proof}
	Consider the following procedure:
	\begin{enumerate}
		\item Consider all nodes in $\mathcal{G}$ as ``available''.
		
		\item Pick an arbitrary available node $u$ and mark all nodes in
		$\mathcal{B}_{2\ell}(u)$ as ``unavailable''.

		\item Add $u$ to the set of ball centers.
		
		\item Repeat steps 2 and 3 until all nodes are marked as unavailable.
	\end{enumerate}

	In step 2 we mark $\Theta(\ell^\alpha)$ nodes as unavailable,
	allowing our procedure to repeat $\Theta\pars*{\frac{n}{\ell^\alpha}}$
	times, picking a new ball center each time.
	Balls of radius $\ell$ centered around each center must be
	independent since they are at least $2\ell$ apart.
	\qed
\end{proof}

\vspace*{-\medskipamount}
\vspace*{-\medskipamount}

\subsubsection{The Probability of Distance Improving.} \label{sec:prob-halving}
%
In this section we lower bound the probability that a random long-range
contact $v$ of node $u$ improves its distance to another node $t$ by a constant
factor.
% This result is used when proving tight bounds for greedy routing.
%
\begin{lemma} \label{lem:prob-halving}
	The probability that any of the long-range contacts of a highway node $u$
	improves its distance to the destination $t$ by a factor of $c$ is at least
	proportional to $[(c + 1)^\alpha z(u)]^{-1}$ for $c \geq c_0$ for some
	constant $c_0$.
\end{lemma}
%
\begin{corollary} \label{cor:prob-halving-upper}
	The probability that any of the long-range contacts of a highway node $u$
	improves its distance to the destination $t$ by a factor of $c$ is at most
	proportional to $[(c - 1)^\alpha z(u)]^{-1}$ for $c \geq c_0$ for some
	constant $c_0$.
\end{corollary}
%
\begin{proof}
	Assume we are at highway node $u$ which is at distance $d > c \text{ }
	\Theta(\sqrt[\alpha]{k \log{n}})$ from the destination $t$ for some
	arbitrary constant $c > 1$.
	From \Crefpart{lem:balls}{lem:balls:always}, we know that there are
	$\Theta(\frac{d^\alpha}{c^\alpha k})$ highway nodes in
	$\mathcal{B}_{d/c}(u)$ w.h.p.
	% , referring to the exact number of highway
	% nodes as $\frac{c_2 d^\alpha}{c^\alpha k}$ for some $c_2$ that is both
	% lower and upper bounded by constants for reasons that will become clear
	% later.
	The maximum distance between $u$ and any node in $\mathcal{B}_{d/c}(u)$ is
	$d + d/c$.
	Using this, and letting $v$ be an arbitrary long-range connection of $u$,
	we obtain:
	%
	\begin{equation*}
		\Pr(v \in \mathcal{B}_{d/c}(u)) \geq \frac{\Theta\pars*{\frac{ d^\alpha}{c^\alpha k}}}{d^\alpha (1 + 1/c)^\alpha z(u)} = \Theta([(c + 1)^\alpha k z(u) ]^{-1})% = \Theta([k z(u)]^{-1})
	\end{equation*}
	%
	% Since we assume that $u$ has the improved normalization constant bounded by
	% \Crefpart{lem:z}{lem:z:constant}, the probability is at least $c_2 [(c +
	% 1)^\alpha (\log{n} + k) ]^{-1}$.
	The probability of $v$ not being in $\mathcal{B}_{d/c}(u)$ is
	$1 - \Pr(v \in \mathcal{B}_{d/c}(u)) \leq e^{-\Pr(v \in
	\mathcal{B}_{d/c}(u))}$, using the fact that $1 + x \leq e^x$.
	Recalling that each highway node has $\Theta(k)$ long-range contacts,
	% referring to the exact number of long-range contacts of $u$ as $c_3 k$
	% for some $c_3$ that is both lower and upper bounded by constants,
	the probability that none of them are in $\mathcal{B}_{d/c}(u)$ is at most
	$e^{-k \Pr(v \in \mathcal{B}_{d/c}(u))} = e^{-\Theta([(c + 1)^\alpha
	z(u)]^{-1})}$.
	Since $z(u)$ is at least a constant, $c$ can be picked large enough such
	that the exponent is small, such that the inequality $e^{-x} \leq 1 - x/2$
	holds, and the probability that none of $u$'s long-range contacts are
	in the ball is at most $1 - \Theta([(c + 1)^\alpha
	z(u)]^{-1})$.
	Consequently, the probability that any of $u$'s long-range contacts
	are in the ball is at least $\Theta([(c + 1)^\alpha z(u)]^{-1})$.
	The corollary follows from a very similar analysis by assuming that all the
	highway nodes are as close to $u$ as possible, i.e., at distance $d - d/c$
	from $u$.
	\qed
\end{proof}
\vspace*{-\medskipamount}
\vspace*{-\medskipamount}
\subsubsection{The Existence of Closer long-range Contacts.}
\label{sec:closer-contacts}
%
While greedily routing on a fixed-growth graph it is important to stay on the
highway for as long as possible, since there may be significant cost associated
with returning to the highway, e.g., $\mathcal{O}(k)$ hops.
Note that there exist variants of our fixed-growth graphs where this time may be
significantly reduced, or where there is always a closer highway node by
construction,\footnote{As long as the destination is a sufficient distance
away.} but we include this result for the most general and loosest case.
%
\vspace*{-\medskipamount}
\vspace*{-\medskipamount}
\begin{lemma} \label{lem:closer-contacts}
	For $k \in \Omega(\log{n})$, the greedy routing process stays on
	the highway $\mathcal{H}$ long enough to reach within distance
	$\mathcal{O}(\sqrt[\alpha]{k \log{n}})$ of the destination $t$ w.h.p.
\end{lemma}
%
\begin{corollary} \label{cor:closer-contacts-exp}
	For $k \in \Omega(\log{n})$, the greedy routing process can expect to stay
	on $\mathcal{H}$ long enough to reach within distance
	$\mathcal{O}(\sqrt[\alpha]{k})$ of an arbitrary destination $t$.
\end{corollary}
%
\begin{proof}
	When \cite{gila2023highway} prove a similar result for their lattice-based
	randomized highway model, they make extensive use of the underlying lattice
	geometry, specifically the fact where they know exactly how many balls of
	some radius and some distance from $u$ are closer to $t$.
	While we do not have this luxury, we can still achieve similar results in
	this more general setting by using a bit more care.
	
	Specifically, let us consider an arbitrary, optimal (i.e., $d(u, t)$-length) path from $u$ to
	$t$.
	Let $c_b$ be the $1 + (b \times c)$-th node on this path.\footnote{In order to
	achieve independence for using the improved normalization constant from
	\Crefpart{lem:z}{lem:z:constant}, we need to consider nodes at least a distance
	$\Omega(\sqrt[\alpha]{k})$ apart. This can be achieved by letting $c_b$
	refer to the $a \sqrt[\alpha]{k} + b \times w$-th node on the path,
	for some sufficiently large constant $a$, without affecting the results.}
	All nodes within radius $b w$ of $c_b$ are closer to $t$ than to $u$.
	And any node within radius $b w$ of $c_b$ which is not within radius $(b -
	1)w$ of $c_{b - 1}$ is a previously unseen, ``fresh'' node.
	Setting our width $w$ to $\Theta(\sqrt[\alpha]{k \log{n}})$ our results from
	\Cref{lem:shells} directly apply, and we expect to see $\Theta(b^{\alpha -
	1} \log{n})$ fresh nodes within radius $b w$ of $c_b$.
	The maximum distance between $u$ and any such node is $1 + 2 bw =
	\mathcal{O}(bw)$.
	Similarly, there are $\Theta(d(u, t)/w)$ such ball centers $b$ along the
	path.
	Let us consider the probability that one particular long-range connection 
	$v$ of $u$ is closer to $t$.
	%
	\begin{align*}
		\Pr(v \text{ is closer to } t) &\geq \sum_{b = 1}^{\Theta(d(u, t)/w)}{\frac{\text{min \# fresh highway nodes}}{z(u) (\text{max distance to fresh node})^\alpha}} \\
		&\geq z(u)^{-1} \sum_{b = 1}^{\Theta(d(u, t)/w)}{\frac{\Theta(b^{\alpha - 1} \log{n})}{\Theta(b w)^\alpha}} \\
		&= \Theta([k z(u)]^{-1}) \sum_{b = 1}^{\Theta(d(u, t)/w)}{\frac{1}{b}} = \Theta\pars*{\frac{1}{k z(u)} \log{\frac{d(u, t)}{w}}}
	\end{align*}
	%
	Assuming that $d(u, t) \in \Omega(\sqrt[\alpha]{k \log{n}}) = \Omega(w)$, we
	see that the probability that $v$ is closer to $t$ is at least proportional
	to $\log(d(u, t))/(k z(u))$.
	Recalling that $u$ has $\Theta(k)$ long-range contacts, the
	probability that none of them are closer to $t$ is at most
	$d(u, t)^{-\Theta(1/z(u))}$.
	% For $k \in \Omega(\log{n})$, $z(u)$ is bounded by a constant, so the
	% probability is at most $d^{-\Theta(1)} \leq d^{-c_1}$ for some large enough
	% constant $c_1$.
	This probability is identical to that used in~\cite{gila2023highway}, and
	the rest of the proof follows for values of $k \in \Omega(\log{n})$.
	The corollary follows by assuming with constant probability that there exist
	highway nodes within distance $\mathcal{O}(\sqrt[\alpha]{k})$ of $t$, along
	with a more optimistic choice of $w$.
	Note that the corollary is in expectation rather than w.h.p.
	\qed
\end{proof}

\subsubsection{Contacts Into and Out Of a Ball.}
%
When upper bounding the diameter of a randomized highway graph with fixed
growth, it is important to know how quickly the highway nodes can be navigated.
In the worst case, all the already-visited highway nodes are as far as possible
from other highway nodes.
For example, we consider the case where a highway node $u$ has no available
long-range constants within some radius $\ell$.
%
\begin{lemma} \label{lem:contacts-out}
	The probability that a long-range contact $v$ of a highway node $u$ lies
	outside of $\mathcal{B}_\ell(u)$, for $\ell \leq \sqrt[\alpha]{n^\theta}$
	and $\theta < 1$ is at least proportional to $\frac{\log{n}}{k z(u)}$.
\end{lemma}
%
\begin{corollary} \label{cor:contacts-in}
	The probability that highway node $u$ is the $i$-th long-range contact of
	any highway node outside of $\mathcal{B}_\ell(u)$ is at least proportional
	to $\frac{\log{n}}{k z}$, where $z$ is the weakest normalization constant bound
	of any highway node from \Crefpart{lem:z}{lem:z:always}.
\end{corollary}

\begin{proof}
	Without loss of generality we can assume that $\ell \in
	\omega(\sqrt[\alpha]{k \log{n}})$ since any smaller value of $\ell$ would
	only make the probability larger.
	Doing this, we can lower bound the probability of $v$ being outside of
	$\mathcal{B}_\ell(u)$ as follows.
	%
	\begin{align*}
		\Pr(v \notin \mathcal{B}_\ell(u)) &\geq \sum_{b = \ell/w}^{\Theta(\sqrt[\alpha]{n}/w)}{\frac{\text{min \# highway nodes in } \mathcal{S}_b^{(w)}}{(\text{max distance to node in } \mathcal{S}_b^{(w)})^\alpha z(u)}} \\
		&= z(u)^{-1} \sum_{b = \ell/w}^{\Theta(\sqrt[\alpha]{n}/w)}{\frac{\Theta(b^{\alpha - 1} \log{n})}{(b w)^\alpha}} \\
		&= \frac{1}{\Theta(k z(u))} (\log{\frac{\sqrt[\alpha]{n}}{w}} - \log{\frac{\ell}{w}}) = \frac{1}{\Theta(k z(u))} (\log{\sqrt[\alpha]{n}} - \log{\ell}) \\
		&\geq \frac{1}{\Theta(k z(u))} (\log{\sqrt[\alpha]{n}} - \log{\sqrt[\alpha]{n^\theta}}) = \frac{\frac{1}{\alpha} \log{n}}{\Theta(k z(u))} (1 - \theta) \\
		&= \Theta\pars*{\frac{\log{n}}{k z(u)}}
	\end{align*}

	The corollary follows by a similar summation in the reverse direction,
	only considering one long-range contact per highway node, and assuming the
	worst case scenario for $z$.
	\qed
\end{proof}
