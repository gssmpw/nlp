\section{Related Work}
% \subsubsection{Small-World Models.}

% Prior to Milgram's experiment, Pool and Kochen \cite{de1978contacts} as well as others \cite{kochensmallworld} found that if social connections were selected uniformly from the population, then the diameter of the graph is small with high probability. 
% While interesting, these early results did not provide a compelling reason for the small-world phenomenon. 
% In 1998, Watts and Strogatz suggested that individuals should have both ``local'' connections to adjacent nodes as well as ``long-range'' connections like the ones studied prior \cite{watts1998collective}. 
% This captured a sense of locality to social networks, and was able to model several natural and manmade phenomena. 
% Kleinberg \cite{kleinberg2000small} generalized and improved their model, altering the criteria for choosing long-range connections such that the graph enabled greedy routing between any two nodes to succeed with high probability in $\mathcal{O}(\log^2 n)$ time. 
% Martel and Nguyen \cite{martel} later found that greedy routing in Kleinberg's model runs in $\Theta(\log^2 n)$ time yet the diameter of the graph is $\Theta(\log n)$, meaning that Kleinberg's algorithm fails to find the true shortest paths available in the graph by a $\log n$ factor. 
% Other works have implemented Kleinberg's analyses in other topologies and types of graphs \cite{fraigniaudgreedy,kleinbergdynamics,barriereefficient,duchon2006could}. 

% Another small-world model is the \emph{preferential attachment} model of Barabási and Albert \cite{barabasipreferential}. 
% Here, nodes are added to the graph one at a time.
% New nodes connect to randomly selected other nodes, with probability proportional to the other nodes' degree (a ``rich-get-richer'' process). Dommers, Hofstad, and Hooghiemstra \cite{dommers2010diameters} showed that the diameter of this model is low, yet unlike Kleinberg's model, no greedy algorithm exists to find such short paths between nodes. 
% Since then, there have been efforts to combine the two models in order to get the best of both worlds.
% Bringmann, Keusch, Lengler, Maus, and Molla propose such a model with an average $\mathcal{O}(\log\log n)$ greedy routing time, yet their model requires nodes of the network to be randomly distributed in some geometric space. 
% Goodrich and Ozel take another approach, inspired by the US highway system (which is ultimately what carried the messages shared in Milgram's experiment), called the \emph{neighborhood preferential attachment} model. 
% Nodes are again added to the graph one at a time, but the probability of connecting to some other node $v$ is based both on the degree of $v$ as well as the distance between the new node and $v$. 
% While their intuition was corroborated by their strong empiricial results as well as by Abraham, Fiat, Goldberg, and Werneck's work on the \emph{highway dimension} of graphs \cite{abrahamhighway}, Goodrich and Ozel were not able to prove any theoretical bounds for their model \cite{goodrich2022modeling}. 
% In 2023, Gila, Ozel, and Goodrich \cite{gila2023highway} adapted the model of Goodrich and Ozel back to a lattice, their \emph{randomized highway} model, and were able to show that it performs greedy routing in $\mathcal{O}(\log n)$ time. 

% moved up

% \subsubsection{Dimensionality in Graphs.}

% In this section, we will place our fixed-growth model of dimensionality in the context of other definitions of graph dimensionality that have been developed over years.  
% One of the earliest notions of dimensionality in graphs was developed by Erdős, Harary, and Tutte \cite{Erdos_Harary_Tutte_1965}. 
% They define dimension of $G$ in a geometrical way: the smallest $d$ such that $G$ can be embedded in $d$-dimensional Euclidean space such that all edges are of unit distance. 
% Another notion popular in the study of networks is the \emph{doubling dimension} \cite{gupta2003bounded,abraham2006routing,cole2006searching,konjevod2008dynamic,chan2016hierarchical,konjevod2007compact}.  
% A metric space is doubling if there is some constant $\lambda$ such that any ball of size $r$ can be covered by the union of $\lambda$ balls of size $r/2$. 
% The highway dimension of a graph was introduced by Abraham, Fiat, Goldberg, and Werneck to provide a unified framework for understanding shortest-path algorithms \cite{abrahamhighway}. 
% A graph has small highway dimension if, for some set of {\it access nodes} $S_r$ such that all shortest paths of length $> r$ include a vertex of $r$, every ball of size $\mathcal{O}(r)$ contains some elements of $S_r$. 

% Several works also study graphs with the property that the growth of the graph, the rate of increase in the number of nodes in a ball of radius $r$ around any node $u$, is bounded by some function of $r$ \cite{stefanExpansion,johannesExpansion,kargerExpansion,ittai2005name,kuhn2005locality,krauthgamer2003intrinsic,BARTAL2005192,gfeller2007randomized}. Several of these works require that the size of a ball of radius $r$ contain $\mathcal{O}(r^\alpha)$ nodes for some constant $\alpha$.
% Work has also been done to translate the study of \emph{fractal dimensions} of geometric objects, such as the Hausdorff or box counting dimensions, to studying complex networks. These fractal dimensions are useful for modeling graphs of ``non-integral'' growth. See \cite{rosenbergsurvey} for a survey of fractal dimensions and their application to networks.

% With the exception of highway dimension, all of these notions seek to bound how fast a graph grows or the complexity of a graph's representation. Indeed, the bounded-growth-style definitions are almost identical to our fixed-growth model. Because of these broad similarities, we believe that our small-world models will be broadly applicable in future network research that considers graphs with these related definitions of dimensionality. 

% moved up

%