\section{Related Work}
\paragraph{Literary social network extraction.}

Significant previous research has addressed extracting social networks from literary texts. One traditional approach involves creating networks by hand ____,
but manual annotations are time intensive and do not scale to large datasets.
Alternative approaches look for character co-occurrences in windowed units like  sentences or chapters ____.
Identifying co-occurrences is computationally lightweight, but their dependency on surface-level features limits their accuracy and applicability.
Neural networks have also been used more widely in recent years for this task ____. 
Specifically, ____ and ____ both use generative models to extract literary social networks, but their approaches are semi-supervised and thus not easily scaled, limiting their studies to datasets in the low hundreds of volumes.

\paragraph{Literary social networks in use.}

Literary social networks are often used to study particular character or character-relationship traits such as prominence ____, cooperativeness ____, relationship trajectory ____, and relationship valence ____.
Some studies also use social networks to ground characters in particular locations ____.
Social networks are likewise useful for studying aspects of plot, including conflict ____, narrative trajectory ____, textual genre ____, and text veracity ____.
They also provide data for studies comparing differences within a corpus ____, over time ____, and between different social theories ____.
However, these studies make use of relatively small corpora, limiting the statistical significance of their results.