We find that character networks and relationships differ significantly between fictional and nonfictional works along a variety of axes.

\paragraph{Network size.}

The mean node and edge counts are significantly larger for nonfiction networks ($p\approx0$ for both). The distributions of these values have thicker right tails for nonfiction networks, and there appears to be a subtype of nonfiction network which is much larger than the average ($\sim$130 nodes and edges instead of $\sim$43 nodes and edges) that does not exist for fictional texts (Figure --).

However, the average number of nodes and edges in nonfiction networks are negatively correlated with decade (nodes -- Pearson's R: -0.52, $p < 0.05$, edges -- Pearson's R: -0.50, $p < 0.05$) and the average number of edges is negatively correlated with decade in fiction networks (Pearson's R: -0.68, $p < 5\times10^{-3}$). This indicates that the size of nonfiction networks are shrinking on average with time, and that the connectivity of fiction networks is decreasing. Examining these trends in detail, we see that the average number of nodes and edges in nonfiction networks fell specifically from the 1820s to the 1950s, but began increasing again in the 1960s (Figure --). Similar, but smaller, trends exist for the node and edge counts for fiction networks, although the average size of these networks begins increasing about a decade later, in the 1970s.

\paragraph{Community structure.}

Differences in several network metrics provide evidence that fiction networks tend to be more interconnected and clustered and focus less on a single character than nonfiction networks. However, if we examine trends in these metrics over time, we find evidence that fictional networks became less connected by some metrics from the 1800s to the 1990s whereas nonfictional networks become more connected by others.

We find that most fiction and nonfiction networks have a single component (Figure ---). However, the distribution of component counts for nonfiction networks has a longer and thicker right tail, demonstrating that a greater proportion of these networks have multiple components. This leads to the mean number of components in nonfiction networks being significantly larger than in fiction networks ($p\approx0$). However, there is a positive correlation between the average number of components and decade for fiction networks (Pearson's R: 0.65, $p < 5\times10^{-3}$); the average number of components increases from 1.35 in the 1800s to 2.97 in the 1990s, suggesting that fiction networks may be becoming less connected over time.

We also look at the average betweenness centrality and eigenvector centrality of the networks. The mean of both values is significantly higher for fiction networks ($p < 1\times10^{-251}$ and $p < 1\times10^{-170}$ respectively), demonstrating that fictional networks are more connected than nonfiction networks. Specifically, these results show that on average nodes serve more frequently as a bridge between other nodes and are more likely to be connected to many other significant nodes in fictional networks. The distribution of both these properties for nonfiction graphs is very right skewed (Figure ---). This means that a small proportion of nonfiction networks are more highly clustered, but the majority are not. The average betweenness centrality of nonfiction networks is positively correlated with decade (Pearson's R: 0.48, $p < 0.05$), meaning characters are on average more likely to serve as important social links in more recent nonfictional works.


The average degree of nodes in a network also indicates how closely connected the networks are. From this metric, we again find evidence that fiction networks are more densely connected; the mean average degree is significantly larger for fiction networks than nonfiction ($p\approx0$) and the distribution of average degree for fiction networks has a greater spread (Figure --). Despite this, the means are relatively similar: 2.13 for nonfiction and 2.66 for fiction. Interestingly, over time the average degree of these networks is growing closer; there is a positive correlation between decade and this metric for nonfiction networks (Pearson's R: 0.54, $p < 0.05$) and a negative correlation between this metric and decade for fiction (Pearson's R: -0.48, $p < 0.05$). However, although they are significant, the changes in these values are small (Figure --).

Another way of measuring network clustering is transitivity, which counts the proportion of possible triangles that are closed in a graph. Intuitively, this metric measures the probability that two characters have a relationship with each other if they both have a relationship with the same third figure. Again, the mean transitivity is significantly higher for fiction networks ($p\approx0$, 0.25 $>$ 0.13).

\begin{itemize}
    \item Community structure
    \begin{itemize}
        \item A manual observation of the networks show that many bear a resemblance to a star graph
        \item To study how similar our networks are to star graphs overall, we look at the edit distance from each graph to a star graph. Specifically, we erase all edges that are not connected to the node with the max degree ($|E|-max\_degree$) and then create an edge from all nodes not already connected to the node of max degree ($|V|-1-max\_degree$): $edit\_distance = |V| + |E| -2(max\_degree) - 1$. We then normalize this value by the number of nodes in the graph in order to compare across graphs of different sizes.
        \item Networks from both fiction and nonfiction texts both require about one edge edit per node on average to create star graphs. However, the mean normalized edit distance from networks to star graphs is significantly higher for fictional networks (1.09 to 0.93, $p < 1\times10^{-111}$) and, observing the distributions, we see that many more nonfiction networks are already star graphs or are very close to star graphs (have a normalized edit distance of 0 or nearly 0). This suggest that there exists a kind of nonfiction text that focuses on only one character and their relations to others, giving very little detail about other relationships between characters. In contrast, this kind of network structure is much less common in fictional texts
        \item If we look at the change in this edit distance by decade, we see that, although there is a negative correlation between decade and star edit distance for fictional texts (Pearson's R: -0.52, $p < 0.05$), only very small changes in the average star edit distance occur by decade
        \item We next look at how closely nodes tend to cluster together. The transitivity of a graph measures what proportion of possible triangles in a graph are closed; that is, if one character has a relationship with two others, how likely is it that they are also related.
        \item Again, we find that the average transitivity is significantly greater for fiction networks ($p=0$). This again suggests that fictional networks are more tightly clustered than nonfiction networks. Examining the distributions, we find that transitivity is much more evenly distributed for fiction networks; they appear to vary more along this axis. In contrast, the distribution for nonfiction networks is very right skewed; this may be because so many nonfiction networks resemble star graphs, where none of the triangles are closed.
        \item However, the transitivity of nonfiction networks is significantly correlated with decade (Pearson's R: 0.70, $p < 1\times10^{-3}$), suggesting that these networks are becoming increasingly clustered with time. However, the changes are again small
        \item We find increasing evidence that fictional networks focus more on the relationships among a community rather than the relations of one character to the world
    \end{itemize}
    \item Literary metrics
    \begin{itemize}
        \item We then compare the difference between these metrics using two literary metrics defined by Mark Algee-Hewitt in 2017 \cite{algee2017distributed}.
        \item Specifically, we look at protagonism, which he defines to measure ``the tendency of the [text] to concentrate the function of the protagonist in a single character,'' and mediatedness, which he defines as ``the relative importance of bridging characters.'' For a more in depth description of these metrics please see \citet{algee2017distributed}.
        \item Interestingly, we find that both the mean protagonism and mediatedness are significantly lower for nonfiction texts ($p<1\times10^{-6}$, $p=0$). The distribution of protagonism values for nonfiction networks is much more evenly spread, suggesting that these texts differ more along this front. Similarly, although both mediatedness distributions are right skewed, that for nonfiction texts has a thicker tail. Overall, these results suggests that on average, nonfiction texts tend to concentrate more on a single character and make more use of a single bridging character (maybe again related to similarity to star graphs), but that nonfiction works also differ more than fiction works on how much they tend to focus on a single character.
        \item There are no significant correlations between protagonism and decade, but there are signficiant negative correlations between mediatedness and decade for both fiction and nonfiction networks (Fiction -- Pearson's R: -0.60, $p < 5\times10^{-3}$, Nonfiction -- Pearson's R: -0.53, $p < 0.05$), demonstrating that both kinds of texts are relying less on single mediating characters on average over time.
    \end{itemize}
    \item Relationship types
    \begin{itemize}
        \item We use our relationship labels to further study what kinds of relationships exist in these networks
        \item First, we look at the affinity of the relationships depicted in these networks.
        \item We find that fiction networks are more likely to have polarized relationships than nonfiction networks; the average proportion of positive ($p < 1\times10^{-11}$) and negative ($p < 1\times10^{-59}$) edges is significantly higher in fiction networks. Conversely, the average proportion of neutral edges is significantly higher in nonfiction networks ($p < 1\times10^{-90}$). Thus, we find that the relationships in fiction networks are overall more likely to be associated with a particular emotion, although these differences may be small. Interestingly, by looking at the distributions we see that a higher proportion of nonfiction networks have entirely positive or entirely negative edges; fiction networks more frequently contain a mix of affinities.
        \item Average proportion of positive edges is negatively correlated with decade for fiction (Pearson's R: -0.61, $p < 1\times10^{-2}$), meaning that whereas the average proportion of neutral edges is increasing (Pearson's R: 0.87, $p < 1\times10^{-6}$). The average prop positive edges drops by about 8.4\%, from 73.99\% to 65.63\%, whereas average prop neutral goes up by about 9.45\%, from 10.83\% to 20.28\%. Say something interesting about analysis here
        \item The kinds of relationships also differ between fictional and nonfiction networks. 
    \end{itemize}
\end{itemize}