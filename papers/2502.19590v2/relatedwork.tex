\section{Related Work}
\paragraph{Literary social network extraction.}

Significant previous research has addressed extracting social networks from literary texts. One traditional approach involves creating networks by hand \cite{moretti2011network, smeets2021modeling, sugishita2023social},
but manual annotations are time intensive and do not scale to large datasets.
Alternative approaches look for character co-occurrences in windowed units like  sentences or chapters \cite{way2018framework, evalyn2018analyzing, fischer2021social}.
Identifying co-occurrences is computationally lightweight, but their dependency on surface-level features limits their accuracy and applicability.
Neural networks have also been used more widely in recent years for this task \cite{nijila2018extraction, kim2019frowning, chen-etal-2020-mpdd, mellace2020temporal}. 
Specifically, \citet{piper-etal-2024-social} and \citet{zhao2024large} both use generative models to extract literary social networks, but their approaches are semi-supervised and thus not easily scaled, limiting their studies to datasets in the low hundreds of volumes.

\paragraph{Literary social networks in use.}

Literary social networks are often used to study particular character or character-relationship traits such as prominence \cite{masias2017exploring, sudhahar2013automated}, cooperativeness \cite{Chaturvedi_Srivastava_Daume_III_Dyer_2016}, relationship trajectory \cite{chaturvedi2017unsupervised, mellace2020temporal}, and relationship valence \cite{nijila2018extraction, kim2019frowning, piper-etal-2024-social}.
Some studies also use social networks to ground characters in particular locations \cite{lee2017shakespeare, lee2012extracting}.
Social networks are likewise useful for studying aspects of plot, including conflict \cite{smeets2021modeling}, narrative trajectory \cite{min2016network, moretti2011network}, textual genre \cite{agarwal2021genre, evalyn2018analyzing}, and text veracity \cite{sugishita2023social, volker2020imagined}.
They also provide data for studies comparing differences within a corpus \cite{fischer2021social}, over time \cite{algee2017distributed}, and between different social theories \cite{elson2010extracting, falk2016making, bonato2016mining, stiller2005weak, stiller2003small}.
However, these studies make use of relatively small corpora, limiting the statistical significance of their results.